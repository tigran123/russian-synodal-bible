\documentclass{book}
\usepackage[nofnpara,keepfnmark,vssup,nomarnvs,chapbookm,lettrine,nogeometry]{bible}
\usepackage{xltxtra,polyglossia,fancybox,multicol}
%\multicoltolerance=3900
\tolerance=9990
\let\rsbpar\par
\setdefaultlanguage{russian}

\DeclareTextAccent{\'}{\encodingdefault}{180}

% US Letter, two columns
%\tunemarkuptag{pgletter}

% 6" (9x12cm) Kindle ebook reader
\tunemarkuptag{pghanlin}

\tunemarkup{pghanlin}{\RequirePackage[dvips,headheight=14pt,headsep=0pt,vmargin={0.2in},hmargin={0.1in},marginparsep=0pt,twoside=false,papersize={90mm,120mm}]{geometry}}
\tunemarkup{pgletter}{\RequirePackage[dvips,headheight=20pt,headsep=8pt,vmargin={0.6in,0.5in},hmargin={0.5in},twoside=true,papersize={8.25in,10.75in}]{geometry}}

\makeatletter
% fix page dimentions in xelatex - can be dropped when geometry.sty
% will become xetex-aware and/or geometry.cfg in texlive will be fixed
\@ifundefined{XeTeXversion}{}{\def\Gm@checkdrivers{\Gm@setdriver{pdftex}}}
\makeatother

%\tunemarkup{pghanlin}{\setmainfont[Mapping=tex-text,BoldFont=Garamond Premier Pro Bold]{Garamond Premier Pro}}
%\newfontfamily\cyrillicfont[Mapping=tex-text]{Garamond Premier Pro}
\tunemarkup{pgletter}{\setmainfont[Mapping=tex-text]{Minion Pro}}

%\tunemarkup{pghanlin}{\setmainfont[Mapping=tex-text,Script=Cyrillic]{Octava}}
%\newfontfamily\cyrillicfont[Mapping=tex-text,Script=Cyrillic]{Octava}

\tunemarkup{pghanlin}{\setmainfont[Mapping=tex-text,Script=Cyrillic]{Academy}}
\newfontfamily\cyrillicfont[Mapping=tex-text,Script=Cyrillic]{Academy}

\newfontfamily\engfont[Mapping=tex-text]{Linux Libertine O}
\newfontfamily\armfont[Mapping=tex-text]{ArTarumianGrqiNor_U}
\newcommand{\latintext}{\normalfont}

\bibmetadata
{Russian Bible with Apocrypha}
{Russian Synodal Bible (Bibles.org.uk 7th Edition)}
{russian,synodal,1876,bible}
{20050408174820}
{Russian}

\def\mytoday{{\ddmmyyyydate\today}}
\columnseprule=0.0pt

\tunemarkup{pgletter}{%
\definecolor{ubdarkblue}{rgb}{0.0, 0.0, 0.5}
\renewcommand{\bibtocheadfont}{\fontsize{14}{14}\selectfont}
\renewcommand{\bibtocfont}{\fontsize{12}{14}\selectfont}
\renewcommand{\bibheadfont}{\Large\addfontfeature{Letters=SmallCaps}}
\renewcommand{\bibheadchapsize}{\bfseries\large}
\renewcommand{\bibheadversesize}{\bfseries\normalsize}
\renewcommand{\bibheadpagesize}{\bfseries\large}
}

\tunemarkup{pghanlin}{%
\definecolor{ubdarkblue}{rgb}{0.0, 0.0, 0.0}
\renewcommand{\bibtocheadfont}{\fontsize{11}{11}\scshape}
\renewcommand{\bibtocfont}{\fontsize{8}{10}\selectfont}
\renewcommand{\bibheadfont}{\normalsize\scshape\addfontfeature{Letters=SmallCaps,LetterSpace=2.0}}
\renewcommand{\bibheadchapsize}{\bfseries\normalsize}
\renewcommand{\bibheadversesize}{\bfseries\small}
\renewcommand{\bibheadpagesize}{\bfseries\small}
%\renewcommand{\LARGE}{\fontsize{26}{18}\selectfont}
%\renewcommand{\Huge}{\fontsize{30}{24}\selectfont}
}

\renewcommand{\bibdropcapscolor}{ubdarkblue}

\renewcommand{\bibmarnchfont}{\normalsize\bfseries\upshape}
\renewcommand{\bibbookend}{\fontsize{15}{17}\selectfont}
\newcommand{\bibmainfontsize}{15/17}
\renewcommand{\footnotesize}{\fontsize{13}{15}\selectfont}
\renewcommand{\bibpage}{\tiny Стр.}
\renewcommand{\bibpsalmname}{Псалом}
\renewcommand{\bibchapname}{Глава}
%\renewcommand{\bibpsalmname}{Psalm}
%\renewcommand{\bibchapname}{Chapter}
%\renewcommand{\bibcontname}{Оглавление}
\renewcommand{\bibcontname}{Contents}
\renewcommand{\bibtitlefont}{\large\scshape}

%\renewcommand{\LARGE}{\large}
%\renewcommand{\Huge}{\Large}

\setlength{\fboxsep}{1.5pt}
\makeatletter
\def\bibchap{\if@firstvs@bk\else\noindent{\doublebox{\normalfont\Large\bfseries\thechnum}}\nobreakspace\fi}
%%\def\bibchap{\if@firstvs@bk\else\printvssup{\curr@vs}\fi}
\protected\def\print@vssup#1{\@textsuperscript{\fontsize{6}{6}\selectfont#1}\kern0.1em\relax}
\makeatother

% to start a book on a new page, as an exception
%\newcommand{\newbookpage}{\newpage\thispagestyle{empty}}
\newcommand{\newbookpage}{}

\input select-book

\begin{hyphenrules}{russian}
\lccode`\==`\- % for setting exceptions for compound words with hyphen
\hyphenation{
не-у-же-ли
Шу-шан=-Эдуф
ос-мотр
не-ис-тов-ство
Иу-дей-ские
Аа-ро-на
Ер-аст
Ев-вул
ос-корб-ля-ли
Иу-дею
бе-зум-ным
уб-ранст-ва
оз-ло-бил-ся
не-о-би-та-е-мой
бла-го-де-тельст-во-вал
пра-во-те
Фас-ги
не-прав-ды
Иу-дою
Ио-а-са
Ио-а-фа-ма
Ио-а-ким
оп-рес-но-ков
Иу-дей-ско-го
ус-трой
Ецион=-Га-ве-ре
Ио-си-фа
Не-бес-ное
Не-во
Ус-лышь
Ио-аки-мо-ва
Ио-сии
Ок-руг-ле-ние
ус-во-ить
не-о-би-та-е-мая
не-пра-вед-ные
оск-вер-ни-ли
тетра-драх-мой
}
\end{hyphenrules}

\begin{document}
\pagenumbering{arabic}
\newcommand{\tux}{{\engfont ^^^^e000}}
\newcommand{\wheel}{\raisebox{0.3mm}{\fontsize{11}{12}\armfont ^^^^00a1}}
\newcommand{\titlesep}{\begin{center}\usefont{U}{webo}{xl}{n}\huge ahb\end{center}}
\newcommand{\copyrsepline}{\begin{center}\usefont{U}{webo}{xl}{n}\fontsize{5}{6}\selectfont IJLKIJLKIJLKIJLKIJ\end{center}}
\pagestyle{empty}

\begin{titlepage}
\centering
%\selectlanguage{russian}
\vspace*{\stretch{1}}
{
\addfontfeature{LetterSpace=2.0}
{\fontsize{30}{36}\bfseries БИБЛИЯ}\\*[6ex]
\fontsize{14}{20}\selectfont
КНИГИ СВЯЩЕННОГО ПИСАНИЯ\\*[1ex]
ВЕТХОГО И НОВОГО ЗАВЕТА\\*[2ex]
\fontsize{11}{20}\selectfont
СИНОДАЛЬНЫЙ ПЕРЕВОД\\
{\itshape (с книгами апокрифальными)}\\
}
\vspace*{\stretch{2}}
\titlesep
\vspace*{\stretch{1}}

\wheel~{\Large\bfseries\upshape Bibles.org.uk}~\wheel\\
\vspace*{\stretch{0.3}}
\end{titlepage}

\tunemarkup{pgletter}{\onecolumn}
%\selectlanguage{russian}
\vspace*{\stretch{1}}
\begin{center}
% uncomment the next group for the gift edition.
%{
%\newfontfamily\coventry{CoventryCyrillic}
%\coventry\fontsize{20}{26}\selectfont
%\vspace*{\stretch{1}}
%XXXXX\\
%от редактора.\\[1cm]
%Тигран Айвазян,\\
%Лондон, 19 июля 2011 г.\\
%}
\vspace*{\stretch{1}}
{\Large\itshape\bfseries Седьмое издание}
\vspace*{\stretch{4}}
\end{center}

\begin{center}
\fontsize{9}{12}\selectfont
%\selectlanguage{russian}
Все права защищены. \copyright\ 2002--2021 Bibles.org.uk\\
Издание подготовили Тигран Айвазян и Владимир Волович на основе материалов, представленных на сайте Издательства Московской Патриархии.\\
\tux\ Книга набрана в \XeLaTeX\ в системе Linux\\
\vspace*{1mm}

\copyrsepline

\vspace*{1mm}
All rights reserved. Copyright \copyright\ 2002--2021 Bibles.org.uk.\\
This Edition was prepared by Tigran Aivazian and Vladimir Volovich, based on the sources provided by The Publishing House of the Moscow Patriarchate.\\
Text set in \textbf{\itshape Academy} at \bibmainfontsize pt\\
\tux\ PDF typeset with \XeLaTeX\ under Linux (\mytoday)\\
\end{center}
\vspace*{\stretch{0.3}}
\tunemarkup{pgletter}{\twocolumn}
\newpage
%\selectlanguage{russian}
\tunemarkup{pgletter}{\onecolumn\bibtableofcontents{twocol}\newpage\null\twocolumn}
%\tunemarkup{pghanlin}{\bibtableofcontents{twocol}\newpage}
\pagestyle{fancy}
\thispagestyle{empty}
\bibmark{book}{ВВЕДЕНИЕ}
\bibpdfbookmark{Введение}{intro}
\begin{center}
\Large\bfseries ВВЕДЕНИЕ\\
\end{center}
\fontsize{12}{15}\selectfont

%\begin{multicols}{2}
В 1804 году было основано \bibemph{Британское и Иностранное Библейское Общество} (BFBS),
полу-автономным филиалом которого стало \bibemph{Российское Библейское Общество} (РБО),
основанное 6 декабря 1812 года.
В своей работе РБО опиралось на поддержку царя Александра~I, а председателем Общества
был избран князь Александр Голицын (1773--1844),
который тогда был обер-прокурором Святейшего Правительствующего Синода
Русской Православной Церкви, а позже Министром Религии и Народного Образования ---
так называемого <<сугубого министерства>>.
Общество было открыто под именем Санкт-Петербургского, а в сентябре 1814 года
переименовано на Российское.

О русском библейском переводе впервые открыто заговорили в 1816 году.
Князь Голицын, как председатель РБО, получил Высочайшее изустное повеление,
<<дабы предложить Святейшему Синоду искреннее и точное желание Его Величества
доставить и россиянам способ читать Слово Божие на природном своем российском
языке, яко вразумительнейшем для них славянского наречия, на коем книги
Священного Писания у нас издаются>>.
Предполагалось при этом, что новый перевод будет издаваться со славянским
текстом совокупно, как еще раньше уже было выпущено послание к Римлянам,
с дозволения Синода (имелась в виду книга архиепископа Мефодия Смирнова,
перевод и толкование; первое издание в 1794 г., третье в 1815 г.).

Голицын в оправдание предложенного перевода на современный русский язык
ссылался на то, что греческой патриаршей грамотой одобрено народу
чтение священного писания Нового Завета на новейшем греческом наречии
вместо древнего (сама грамота патр. Кирилла была припечатана в отчете
РБО за 1814 год).

Синод не принял на себя руководства библейским переводом и не взял за
него ответственности на себя.
Перевод был отдан в ведение Комиссии духовных училищ, которой надлежало
избрать надежных переводчиков в местной Духовной Академии. 

Перевод был поставлен под охрану Высочайшего имени.
Замысел сей принадлежал самому Государю, или был ему приписан:
<<Не токмо одобряет все споспешествующее сему спасительному делу, но и
одушевляет деятельность Общества внушениями собственного сердца.
Он сам снимает печать невразумительного наречия, заграждавшую доныне от
многих из Россиян евангелие Иисусово, и открывает сию книгу для самых
младенцев народа, от которых не ея назначение, но единственно мрак
времен закрыл оную.>>
Невразумительное наречие закрывало Библию не столько
от народа, сколько именно от высшего круга, от самого императора, прежде
всего, он сам привык читать Новый Завет по-французски (в известном
переводе Де-Саси), и не изменил этой привычке и с изданием российского
перевода.

Ведение перевода от Комиссии духовных училищ было поручено Филарету,
тогда архимандриту и ректору Санкт-Пе\-тер\-бургс\-кой Академии, и он имел
избрать переводчиков по своему усмотрению.
Считалось, что перевод производится при Академии.
Филарет сам взял на себя Евангелие от Иоанна.
От Матфея переводил Павский, от Марка архим. Поликарп (Гайтанников),
тогда ректор Санкт-Петербургской семинарии, а вскоре и Московской
Академии, и от Луки архим. Моисей (Антипов-Платонов), ректор Киевской
семинарии, а потом и Академии, бывший перед тем бакалавром в
Санкт-Петербурге, впоследствии Экзарх Грузии.
Работа отдельных сотрудников пересматривалась и сверялась в особом
комитете из членов Библейского общества.
В нем участвовали: митр. Михаил (Десницкий), впоследствии митрополит
Санкт-Петербургский;
Серафим (Глаголевский), тоже будущий митрополит;
Филарет;
Лабзин;
В.~М.~Попов, директор департамента в двойном министерстве и секретарь
Библейского общества --- человек крайних мистических взглядов, переводчик
Линдля и Госнера, член кружка Татариновой, окончивший жизнь свою в
Зилантовом монастыре в Казани, как заточенный, кроткий изувер, как
его остроумно называет Вигель.

Правила для перевода были составлены Филаретом, это сразу чувствуется
уже в их стиле.
Переводить надлежало с греческого, как первоначального, преимущественно
перед славянским, с тем, чтобы в переводе удерживать или употреблять
слова славянские, если они ближе русских подходят к греческим, не
производя в речи темноты или нестройности, или если соответственные
русские не принадлежат к чистому книжному языку.
В переводе всего важнее точность, затем ясность, наконец, чистота.
Очень характерны некоторые стилистические директивы.
Величие Священного Писания состоит в силе, а не в блеске слов; из сего
следует, что не должно слишком привязываться к славянским словам и выражениям,
ради мнимой их важности.
Еще важнее другое замечание.
Тщательно наблюдать должно дух речи, дабы разговор перелагать слогом
разговорным, повествование повествовательным, и так далее.

Эти положения литературным архаистам показались дурной стилистической
ересью, и это был один из решающих моментов взволнованного восстания
или интриги против русской Библии в 20-х годах.

В этот период были переведены на русский язык и в сотрудничестве с BFBS
опубликованы: Евангелие (1819), Новый Завет (1820) и Псалтирь (1822).
В то же время началась работа над Пятикнижием.
Филарет в своих Записках на книгу Бытия (первое издание уже в 1816 г.)
всюду дает библейский текст в русском переводе, с еврейского.
К переводческим работам были привлечены и вновь открытые Академии:
Московская и Киевская, также и некоторые семинарии.
Сразу же встал трудный и сложный вопрос о соотношении еврейского и
греческого текстов, о достоинстве и достоинствах перевода Семидесяти, о
значении Массоретских чтений, и эти вопросы обострялись тем, что
всякое отступление от Семидесяти означало практически и расхождение
со славянской Библией, остававшейся в богослужебном употреблении,
а потому нуждалось в нарочитых оправданиях и оговорках.
Для начала вопрос был решен просто.
В основу был положен еврейский (Массоретский) текст, как подлинный,
а в объяснение расхождений со славянской Библией было составлено
особое предисловие, убедительное и для незнающих древних языков.
Составил его Филарет, и подписано оно было митр. Михаилом, митр.
Серафимом, тогда еще Московским, и самим Филаретом, тогда
архиепископом Ярославским.

Окончательная корректура Пятикнижия была поручена Герасиму Павскому.
Печатание было закончено в 1825 году, но по изменившимся
обстоятельствам  издание не только не было выпущено в свет, но было
арестовано и вскоре сожжено.
Само библейское дело было остановлено и Библейское общество закрыто
и запрещено 12 апреля 1826 года, в основном благодаря интриграм архимандрита
Фотия, адмирала Шишкова и Аракчеева.

В 1840-х годах профессор Павский впервые перевел на русский язык весь
Ветхий Завет непосредственно с еврейских оригиналов, за что был отдан
под суд и результаты его стараний были уничтожены.

С приходом к власти царя Александра~II работа РБО была возобновлена
под руководством Митрополита Московкого Филарета.
В декабре 1857 года библейское дело получило официальное движение.
Синодальное определение состоялось 20 марта 1858 года, а Высочайшее
повеление о возобновлении русского перевода было опубликовано в мае.

Перевод был возобновлен с Нового Завета, к участию в работах снова были
привлечены все академии, а редактирование поручено петербургскому
профессору Е.~И.~Ловягину.
Высшее наблюдение и последний просмотр были доверены Филарету.
Несмотря на свой преклонный возраст, он очень деятельно участвовал в
работе, со вниманием перечитывая и проверяя весь материал.

В 1860 году было издано русское Четвероевангелие, а в 1862 и полный
Новый Завет.

Перевод Ветхого Завета потребовал больше времени. Уже с самого начала
60-х годов в различных духовных журналах стали появляться частные опыты
перевода отдельных книг.
И, прежде всего, были опубликованы эти так незадолго перед тем запретные
переводы Павского (в журнале Дух Христианина за 1862 и 1863 годы) и
арх. Макария (в Православном Обозрении с 1860-го по 1867-ой, особым
приложением).
Это был очень живой и яркий симптом сдвига и поворота.
Было признано полезным и нужным предать гласности эти опыты, чтобы через
свободное обсуждение в печати подготовить окончательное издание.
С этой целью было предложено и профессорам академии заняться переводами
отдельных книг, с тем чтобы эти новые опыты были в свое время использованы
Синодальной комиссией.
Нечто подобное предлагал в свое время о. Макарий Глухарев,
издавать при Петербургской Академии особый журнал: Опыты в переводе с
еврейского и греческого, и рассылать по академиям и семинариям, с
примечаниями и сносками, потом этот материал пригодится.

В академических изданиях, в Христианском Чтении и в Трудах Киевской
духовной академии в эти годы появляется перевод многих книг.
В Киеве особенно потрудился проф. М.~С.~Гуляев, а в Петербурге проф.
М.~А.~Голубев в сотрудничестве с П.~И.~Савваитовым, Д.~А.~Хвольсоном и др.
Появились и отдельные издания.
Издавал свои библейские переводы с греческого Порфирий Успенский, тогда епископ
Чигиринский.
Это был полный разрыв с режимом предыдущего царствования.

Но встречались и трудности.
Не сразу удалось решить вопрос о принципах перевода.
Было заявлено мнение, что и Ветхий Завет переводить нужно с греческого,
к этому мнению удалось склонить и митр. Григория.
Филарет Московский настоял, чтобы перевод делался по сличению обоих
текстов, и расхождение в важнейших местах было отмечаемо под чертой.
Сперва предложено было начать с Псалмов; над исправлением перевода
Псалмов Филарет работал в свои последние годы.
Но затем он сам предложил издавать в порядке обычного текста,
т.~к. Пятикнижие легче Псалмов по языку.
Синодальный перевод начал выходить с 1868 года отдельными томами, а
всё издание закончилось в 1875 со включением и книг неканонических.

Особенно резким противником еврейского текста был епископ Феофан
Говоров, тогда уже Вышенский затворник.
Новый русский перевод Ветхого Завета он называл Синодальным сочинением,
совсем как Афанасий, и мечтал, что эту Библию новомодную доведет до
сожжения на Исаакиевской площади.
Употребление еврейского текста, никогда не бывшего в церковном
употреблении, означало в его понимании прямое отступничество.
Еврейская библия к нам нейдет, потому что никогда не было ее в Церкви и
в церковном употреблении.
Поэтому принимать ее значит отступать от того, что всегда было в Церкви,
т.~е. сдвигаться с коренного основания православия.
Феофан вполне признавал нужду в русском переводе, он возражал только
против еврейского образца.
И синодальный перевод считал поэтому соблазнительным и вредным.
Церковь Божия не знала другого Слова Божия, кроме 70-ти толковников, и
когда говорила, что Писание богодухновенно, разумела Писание именно в
этом переводе.
Об этом он очень резко писал в Душеполезном Чтении (1875 и 1876),
ему отвечал в Православном Обозрении проф. П. И. Горский-Платонов с
неменьшей резкостью.
Но Феофан не ограничивался критикой.
Он предлагал заняться изданием общедоступных толкований Библии по
славянскому тексту (и особенно книг учительных и пророческих),
чтобы приучить именно к этому тексту, т.~е. к Семидесяти.
Выйдет, что, несмотря на существование Библии в переводе с
еврейского, знать ее и понимать и читать все будут по
Семидесяти, по причине сего толкования.
Проект этот не был осуществлен, сам Феофан издал только
толкование на Псалом Сто Осмьнадцатый (сто восемнадцатый).
Возникла у него и мысль сесть за перевод всей Библии с греческого,
с замечаниями в оправдание греческого текста и в осуждение
еврейского.
Это намерение осталось тоже без исполнения.
Уже только много позже некоторые книги Ветхого Завета были переведены с
греческого казанским профессором П.~А. Юнгеровым (пророки, Псалтирь,
Притчи, Бытие, книги неканонические).

В процессе работы над переводом Ветхого Завета снова и снова
открывалось, что соотношение Массоретской редакции и Семидесяти слишком
сложно, чтобы можно было ставить вопрос о выборе между ними в общем
виде.
Можно спрашивать только о предпочтительном или надежном чтении
отдельных отрывков или стихов, и приходится выбирать иногда еврейскую
истину, иногда же греческое чтение.
Филологически лучшим будет именно сводный текст.
Богословскому заключению о догматическом достоинстве определенного
текста, во всяком случае, должно предшествовать подробное исследование
отдельных книг.
Примером такой работы в те годы была диссертация И.~С.~Якимова о книге
пророка Иеремии (1874).
Следует упомянуть и работы Д.~А.~Хвольсона и И.~А.~Олесницкого.  

Обнаруживалась и другая трудность.
Оказывалось, что и Славянскую Библию не приходится в целом приравнивать
к Семидесяти, что и сам славянский текст есть уже сводный, в известном
смысле и пределах.
В этом и была принципиальная важность описания библейских рукописей Горским
и Невоструевым в Московской Синодальной библиотеке.
Начинается историческое изучение Славянской Библии.
И уже нельзя так упрощенно спрашивать о выборе между славянским и русским.  

Оживает интерес и к вопросам библейской критики.
Большинство русских исследователей придерживались умеренных
взглядов, но и у них влияние западной критической
литературы сказывалось очень заметно.
Достаточно назвать работы архим. Филарета Филаретова (ректора
Киевской академии, впоследствии епископа Рижского, 1824-1882).
В его диссертации о Происхождении книги Иова (1872) он не только
принимал позднюю послепленную датировку книги, но и разбирал ее скорее,
как памятник литературы, нежели как книгу священного канона.
К тому же всё исследование было проведено по еврейскому тексту,
безо всякого внимания к славянским чтениям.
Митр. Арсений Киевский нашел сам тон диссертации
несоответствующим богодухновенному характеру библейской книги, и
публичная защита диссертации была запрещена Святейшим Синодом.
А в следующем (1873) году в Трудах Киевской Академии были
напечатаны устаревшие лекции по введению в священные книги
Ветхого Завета, читанные самим митр. Арсением в Петербургской
академии еще в 1823--1825 годах.
Впрочем, в кратком предисловии от редакции было оговорено, что
читатель сам сможет судить, насколько вперед подвинулись у нас
библиологическая наука с того времени до настоящего.

Переводы, выполненные в 1810--25~гг. и отредактированные
в 1860--70~гг., составили книгу именуемую \bibemph{Русской Синодальной Библией}.
Однако не все отнеслись благосклонно к появлению Библии на русском языке,
предпочитая старо-славянский перевод используемый и по сей день в церковном служении.
Даже Святейший Синод благословил Библию 1876 года \bibemph{исключительно}
для приватного употребления, для чтения дома, но не для церковного служения.

Впоследствии, текст Русской Синодальной Библии был существенно изменен
с целью распространения <<протестантизма>>.
А именно, слова и целые фразы соответствующие текстам греческой
Септуагинты и латинской Вульгаты были удалены, хотя и не полностью и
с многочисленными ошибками, чтобы поддержать \bibemph{миф} о том, что,
якобы, Бог чудодейственным образом <<сохранил>> Свое
Слово в одном единственном варианте и естественно выбор такого
<<идеального>> варианта пал на Массоретский Текст.
В данном издании мы не делаем идола из Слова Божиего и, посему,
приводим текст Синодального издания в его изначальной форме (за исключением
использования архаичного правописания и букв), включая неканонические
книги в том порядке и виде, в каком они были приведены в Библии 1876 года.

Тексты Книг Священного Писания Ветхого и Нового Завета и приложения,
использованные в данном издании, взяты с сайта Издательства Московской
Патриархии,
и соответствуют Си\-но\-даль\-но\-му переводу издания Московской Патриархии
кроме 70-и стихов находящихся между 35 и 36 стихами 7-й главы
3-й Книги Ездры, взятых нами из ``Толковой Библии'' А.~П.~Лопухина (Петербург, 1904)
и имеющихся также в Брюссельской Библии (Брюссель, 1973).
Электронные тексты были переведены в формат типографской системы,
используемой для всех Библий, издаваемых Bibles.org.uk, основанной
на \XeLaTeX\ в системе Linux.
Выражаем благодарность Самуэлю Ким за найденные опечатки.
Мы будем очень признательны, если найденные Вами в этом издании
опечатки, будут отправлены по электронной почте по адресу
{\makeatletter aivazian.tigran@gmail.com\makeatother}.

Оформление текста Библии в данном издании имеет следующие особенности:
\begin{itemize}
\item Для облегчения ссылок и чтения нумерация стихов выведена на поля.
\item Слова, напечатанные \bibemph{курсивом}, приведены для ясности
      и отсутствуют в оригиналах.
\item В тексте Ветхого Завета в квадратные скобки заключены слова,
      заимствованные из греческого перевода 70-ти толковников (III в.~до Р.~Х.)
      --- Септуагинты.
\end{itemize}

Несмотря на <<до-Пушкинский>> язык Русской Синодальной Библии, она
продолжает успешно служить миллионам людей на планете как самый
достоверный и читаемый перевод Священного Писания на русский язык.

Да благословит Господь Бог ваше изучение Его Слова, дабы подчинить
Сыну своему Иисусу Христу Господу нашему всякую мысль вашего сердца
и всякое слово, исходящее из уст ваших. Аминь.
%\end{multicols}

\begingroup
\vfill
\noindent
\itshape
\parbox{4cm}{
Владимир Волович,\\
Воронеж, Россия.
}
\hfill
\parbox{4cm}{
Тигран Айвазян,\\
Лондон, Англия.
}
\vfill
\endgroup
\newpage
%\begin{center}
\Large\bfseries
БИБЛЕЙСКИЙ КАЛЕНДАРЬ
\end{center}

\makeatletter
\def\bibstrut{%
  \vrule
    \@height\dimexpr\f@size pt*13/10\relax
    \@depth.6\baselineskip
    \@width\z@
}
\makeatother

\newsavebox{\nazvanija}
\newsavebox{\mesjatsi}
\sbox{\nazvanija}{\parbox{9cm}{\centering Названия месяцев {\bfseries ДРЕВНИЕ} и \bibemph{АССИРО-ВАВИЛОНСКИЕ},\\ число дней в месяце и особые дни}}
\sbox{\mesjatsi}{\parbox{2cm}{\centering Соответствует месяцам современного календаря}}

\hspace*{-5mm}
\begin{tabular}{|>{\raggedleft}p{1cm}|>{\raggedleft}p{1cm}|p{9cm}|c|}
\hline
\multicolumn{2}{|c|}{\bibstrut Счет месяцев} & & \\
\cline{1-2}
\multicolumn{1}{|p{1cm}|}{\footnotesize в священном году} & 
\multicolumn{1}{p{1cm}|}{\footnotesize в гражданском году} &
\usebox{\nazvanija} & \usebox{\mesjatsi} \tabularnewline
\hline
\bibstrut I & \bibstrut 7 &
\bibstrut {\bfseries АВИВ}, \bibemph{НИСАН}. 30 дней.
14. Пасха (\bibref[Исх. гл. 12]{Exo 12:1}; \bibref{Lev 23:5}; \bibref{Num 28:16}).
16. Принесение первого снопа жатвы ячменя (\bibref[Лев 23:10--14]{Lev 23:10}).&
\bibstrut март --- апрель\tabularnewline
\hline
\bibstrut II & \bibstrut 8 &
\bibstrut {\bfseries ЗИФ}, \bibemph{ИЯР}. 29 дней.
14. Вторая пасха --- для тех, кто не мог совершить первую (\bibref[Чис 9:10--12]{Num 9:10}).&
\bibstrut апрель --- май\tabularnewline
\hline
\bibstrut III & \bibstrut 9 &
\bibstrut \bibemph{СИВАН}. 30 дней.
6. Пятидесятница (\bibref{Lev 23:16}) или праздник седмиц (\bibref[Втор 16:9--10]{Deu 16:9}).
Принесение начатков жатвы пшеницы (\bibref[Лев 23:15--21]{Lev 23:15}) и начатков всех плодов земли
(\bibref{Num 28:26}; \bibref[Втор 26:2,10]{Deu 26:2}).&
\bibstrut май --- июнь\tabularnewline
\hline
\bibstrut IV & \bibstrut 10 &
\bibstrut \bibemph{ФАММУЗ}. 29 дней.
17. Пост. Взятие Иерусалима (\bibref{Zec 8:19}).&
\bibstrut июнь --- июль\tabularnewline
\hline
\bibstrut V & \bibstrut 11 &
\bibstrut \bibemph{АВ}. 30 дней.
9. Пост. Разрушение храма иерусалимского (\bibref{Zec 8:19}).&
\bibstrut июль --- август\tabularnewline
\hline
\bibstrut VI & \bibstrut 12 &
\bibstrut \bibemph{ЭЛУЛ}. 29 дней.&
\bibstrut август --- сентябрь\tabularnewline
\hline
\bibstrut VII & \bibstrut 1 &
\bibstrut {\bfseries АФАНИМ}, \bibemph{ТИШРИ}. 30 дней.
1. Праздник труб (\bibref{Num 29:1}). Новый год.
10. День очищения (\bibref{Lev 16:29}; \bibref[25:9]{Lev 25:9}).
15-22. Праздник кущей (\bibref[Лев 23:34--36]{Lev 23:34}; \bibref[Чис 29:12--35]{Num 29:12}).&
\bibstrut сентябрь --- октябрь\tabularnewline
\hline
\bibstrut VIII & \bibstrut 2 &
\bibstrut {\bfseries БУЛ}, \bibemph{МАРХЕШВАН}. 29 дней.&
\bibstrut октябрь --- ноябрь\tabularnewline
\hline
\bibstrut IX & \bibstrut 3 &
\bibstrut \bibemph{КИСЛЕВ}. 30 дней.
25. Праздник обновления (\bibref[1~Мак 4:52--59]{1Ma 4:52}; \bibref[Ин 10:22]{Joh 10:22}).&
\bibstrut ноябрь --- декабрь\tabularnewline
\hline
\bibstrut X & \bibstrut 4 &
\bibstrut \bibemph{ТЕБЕФ}. 29 дней.&
\bibstrut декабрь --- январь\tabularnewline
\hline
\bibstrut XI & \bibstrut 5 &
\bibstrut \bibemph{ШЕВАТ}. 30 дней.&
\bibstrut январь --- февраль\tabularnewline
\hline
\bibstrut XII & \bibstrut 6 &
\bibstrut \bibemph{АДАР}. 29 дней.
11. Пост Есфири (\bibref{Est 4:16}).
14--15. Праздник Пурим (\bibref[Есф 9:17--32]{Est 9:17}).&
\bibstrut февраль --- март\tabularnewline
\hline
\end{tabular}

\vspace*{6mm}

%\begin{multicols}{2}
В Библии священный год со времени исхода из Египта начинается с весны, с
месяца авив, что значит месяц зрелого колоса
(\bibref{Exo 13:4}; \bibref[12:2]{Exo 12:2}).
Это был месяц весеннего равноденствия и время созревания ячменя
(\bibref[Лев 23:10--14]{Lev 23:10}).
Позже он стал называться нисаном.
В 14-й день этого месяца, который приходится в полнолуние, праздновали
Пасху (\bibref[Исх. гл. 12]{Exo 12:1}).
Другие месяцы названий не имели, о них говорили: второй месяц, десятый месяц
и т.~д.
Лишь в рассказе о постройке храма Соломона при участии финикийцев три месяца
названы особо:
зиф (месяц цветения) --- \bibref{1Ki 6:1},
афаним (месяц бурных ветров) --- \bibref{1Ki 8:2},
и бул (месяц произрастания) --- \bibref{1Ki 6:38}; это финикийские названия.
После плена вавилонского появились ассиро-вавилонские названия месяцев:
нисан (\bibref{Neh 2:1}),
ияр, сиван (\bibref{Est 8:9}), фаммуз или таммуз, ав, элул (\bibref{Neh 6:15}),
тишри, мархешван, кислев или хаслев (\bibref{Neh 1:1}; \bibref{Zec 7:1}),
тебеф (\bibref{Est 2:16}), шеват (\bibref{Zec 1:7})
и адар (\bibref{Est 3:7}).
Те из них, которые не встречаются в Библии, известны по сочинениям Иосифа Флавия
(I век по Р.~Х.) и другим древним источникам.

Гражданский год начинался и кончался осенью
(ср. \bibref{Exo 23:16}; \bibref[34:22]{Exo 34:22}), после уборки урожая
(в месяце тишри).
В Библии встречается счет месяцев и по священному и по гражданскому году.

Начало месяца определялось по появлению видимого серпа новой луны; этот
день, новомесячие, был праздничным (\bibref{Num 10:10}; \bibref[28:11]{Num 28:11}).
От одного новолуния до другого проходит 29 1/2 суток, поэтому месяцы имели
продолжительность в 29 и 30 дней попеременно.
12 лунных месяцев составляют год в 354 дня, что на 11 дней меньше солнечного
года.
За три года разница между луннным и солнечным годом составит целый месяц,
поэтому примерно раз в три года добавлялся 13-й месяц и получался год
продолжительностью в 384 дня.
Это делалось для того, чтобы авив оставался весенним месяцем.

В прилагаемой таблице указаны названия месяцев в гражданском и священном 
году (т.~е. первый месяц, второй и т.~д.), а также древние (ханаанские и
финикийские) и послепленные (ассиро-вавилонские) названия в том виде, в
каком они приведены в русской Библии. Обозначено количество дней в
месяце, перечислены библейские праздники и посты и показано
приблизительное соответствие библейских месяцев современным.

Дни недели, кроме субботы (шаб\'ат), особых наименований не имели, если не
считать существовавшего в эллинистическую эпоху греческого названия дня
перед субботой --- параскев\'и, что значит <<приготовление>>
(к дню покоя --- субботе).
Неделя завершалась субботой, поэтому <<день первый>> (после субботы, см.
\bibref[Мф 28:1]{Mat 28:1}) соответствует нашему воскресенью,
<<день второй>> --- понедельнику и т.~д.

День (в смысле суток) начинался с захода солнца, т.~е. с позднего
вечера. В древности как ночь, так и день делились на три части: ночь на
первую, вторую и третью стражи (\bibref{Jdg 7:19}), а
день --- на утро, полдень и вечер (см. \bibref{Psa 54:18}).
Позднее, со времен римского владычества, ночь делилась на четыре стражи
(\bibref[Лк 12:38]{Luk 12:38}; \bibref[Мф 14:25]{Mat 14:25}) и вошло в употребление понятие <<час>> ---
двенадцатая часть дня или ночи (\bibref[Мф 20:1--8]{Mat 20:1}; \bibref[Деян 23:23]{Act 23:23}).

%\end{multicols}

\thispagestyle{empty}
\begin{center}
\normalsize\bfseries
О КНИГАХ КАНОНИЧЕСКИХ И НЕКАНОНИЧЕСКИХ
\end{center}

%\begin{multicols}{2}
Христианская Библия состоит из двух частей: Ветхого Завета и Нового Завета,
Книги Ветхого Завета писались на протяжении более тысячи лет до Рождества
Христова на древнееврейском языке, книги Нового Завета написаны на греческом
языке в I в. по Р.~Х.

В Ветхом Завете есть книги канонические и неканонические.
Основное различие между ними в том, что книги канонические более древние,
написаны в XV--V вв.~до Р.~Х., а книги неканонические, т.~е. не вошедшие в
канон, в собрание священных книг, написаны позже, в IV--I вв.~до Р.~Х.
Ветхозаветный канон создавался постепенно.
Первым собирателем священных книг воедино считают Ездру (V в.~до Р.~Х.).
В III в.~до Р.~Х. -- I в. по Р.~Х. ветхозаветный канон приобрел тот вид,
который существует в современной еврейской, так называемой массоретской,
Библии (она содержит лишь Ветхий Завет; массореты, хранители предания,
закончили работу над ней в VIII в. по Р.~Х.).
В ней 39 книг, которые разделены на три отдела: \bibemph{закон},
\bibemph{пророки} и \bibemph{писания} (этими словами в древности называли
Ветхий Завет,-- см. \bibref{Mat 7:12}; \bibref{Luk 24:44}).
\bibemph{Закон} (по-еврейски тор\'а) содержит Пятикнижие Моисея: Бытие, Исход, Левит,
Числа и Второзаконие.
\bibemph{Пророки} (неби\'им) делятся на первых или старших, которым принадлежат книги
Иисуса Навина, Судей, две книги Самуила (в нашей Библии это 1 и 2 Царств) и
две книги Царей (наши 3 и 4 Царств; в христианской Церкви книги старших
пророков, а также Руфь, Есфирь, Ездры, Неемии и Паралипоменон принято
считать историческими книгами), и на последних или младших, которые в
свою очередь подразделяются на великих пророков и малых.
Книги трёх великих пророков: Исайя, Иеремия, Иезекииль;
двенадцати малых: Осия.  Иоиль, Амос, Авдий, Иона, Михей, Наум, Аввакум,
Софония, Аггей, Захария и Малахия.
\bibemph{Писания} (кетуб\'им) составляют: Псалмы, Притчи, Иов, Песнь Песней, Руфь,
Плач Иеремии, Екклезиаст, Есфирь, Даниил. Ездра, Неемия и Летописи
($=$ 1 и 2 Паралипоменон).

После возвращения евреев из плена вавилонского, т.~е. после V в.~до Р.~Х.,
было составлено и написано еще несколько книг на еврейском и греческом языках.
В канон еврейских священных книг их уже не включили, но они вошли, как
полезные и назидательные, в Септуагинту, т.~е. греческий перевод Библии.  

Этот перевод был сделан в III--II вв.~до Р.~Х. для александрийских
евреев-эллинистов и иудеев рассеяния, т.~е. живущих вне Палестины, которые
уже забывали родной язык и говорили по-гречески
(см. предисловие к Книге Иисуса сына Сирахова).
Древнее предание говорит о 70-ти (или 72-х) толковниках, т.~е. переводчиках,
которые перевели священные книги с еврейского языка на греческий, поэтому и
перевод этот называется <<переводом семидесяти>> или по-гречески <<Септуагинта>>.
Он в некоторых деталях отличается от масоретского текста, так как массореты и
переводчики на греческий пользовались разными списками (рукописями) древнего текста.

Библейские книги на латинском языке были известны уже в конце II века по Р.~Х.
Блаж. Иероним перевел их заново в конце IV -- начале V в., и этот перевод, известный
под названием <<Вульгата>>, получил широкое распространение в Католической Церкви.

Перевод книг Священного Писания на славянский язык начат был святыми равноапостольными Кириллом
и Мефодием в IX в.
Современная славянская Библия используемая в церковном служении представляет собой
перепечатку Елизаветинского издания 1751--1756~гг., в котором текст Ветхого Завета
был выверен по греческой Библии.

На русский язык Библия переведена в середине XIX века.
Канонические книги переводились с еврейского массоретского текста с
дополнениями и вариантами из Септуагинты, а неканонические --- с греческого,
за исключением Третьей книги Ездры, переведенной с латинского, так как этой
книги нет ни в еврейской, ни в греческой Библии.
Русская православная Библия, как и славянская, содержит все 39 канонических и
11 неканонических книг Ветхого Завета.

Что касается деления книг на главы, то оно было введено уже в XIII веке
двумя западными исследователями Библии --- кардиналом Стефаном Лангтоном и
доминиканцем Гуго де Сен-Шира.

Деление глав на стихи ввел в середине XVI века парижский типограф
Робертус Стефанус. \bibemph{(Новая толковая Библия, 1990, Ленинград)}
%\end{multicols}
\newpage
%\makeatletter

\define@key{bibevnt}{n}{\gdef\bibevnt@n{#1}}
\define@key{bibevnt}{event}{\gdef\bibevnt@event{#1}}
\define@key{bibevnt}{Mat}{\gdef\bibevnt@Mat{#1}}
\define@key{bibevnt}{Mar}{\gdef\bibevnt@Mar{#1}}
\define@key{bibevnt}{Luk}{\gdef\bibevnt@Luk{#1}}
\define@key{bibevnt}{Joh}{\gdef\bibevnt@Joh{#1}}

\def\DescribeEvent#1{%
  \global\let\bibevnt@n\@empty
  \global\let\bibevnt@event\@empty
  \global\let\bibevnt@Mat\@empty
  \global\let\bibevnt@Mar\@empty
  \global\let\bibevnt@Luk\@empty
  \global\let\bibevnt@Joh\@empty
  \setkeys{bibevnt}{#1}%
  \noindent\hangindent=\evntw\hangafter=1
  \hb@xt@\evntw{\hfill\footnotesize\bibevnt@n\kern3pt}%
  \footnotesize\bibevnt@event
  &\footnotesize\bibevnt@Mat
  &\footnotesize\bibevnt@Mar
  &\footnotesize\bibevnt@Luk
  &\footnotesize\bibevnt@Joh
  \tabularnewline
}

\def\EventHeaderCap#1{\multicolumn{5}{c}{%
  \fontsize{10}{12}\bfseries\scshape\bibstrut#1}\tabularnewline}
\def\EventHeader#1{\multicolumn{5}{c}{%
  \footnotesize\bfseries\bibstrut#1}\tabularnewline}

\newlength\maxrangew
\setbox0=\hbox{\footnotesize 88:88-88:88,}
\maxrangew=\wd0
\advance\maxrangew by 2pt

\newlength\evntw
\setbox0=\hbox{\footnotesize 888}
\evntw=\wd0
\advance\evntw by 3pt

\def\bibstrut{%
  \vrule
    \@height\dimexpr\f@size pt*11/10\relax
    \@depth.3\baselineskip
    \@width\z@
}

\makeatother

\bibmark{book}{последовательность евангельских событий}
\bibpdfbookmark{Последовательность Евангельских событий}{ntevents}

\begin{landscape}

\begin{center}
\Large\bfseries
ПОСЛЕДОВАТЕЛЬНОСТЬ ЕВАНГЕЛЬСКИХ СОБЫТИЙ ПО ЧЕТЫРЕМ ЕВАНГЕЛИСТАМ
\end{center}

\begin{longtable}{|>{\raggedright}p{0.5\linewidth}|p{\maxrangew}|p{\maxrangew}|p{\maxrangew}|p{\maxrangew}|}
\hline
\multicolumn{1}{|c|}{\multirow{2}*{Евангельское повествование}}&\multicolumn{4}{c|}{\bibstrut Главы и стихи евангелистов}\tabularnewline
\cline{2-5}
&\multicolumn{1}{c|}{\bibstrut Матфея}&\multicolumn{1}{c|}{Марка}&\multicolumn{1}{c|}{Луки}&\multicolumn{1}{c|}{Иоанна}\tabularnewline
\hline
\endhead
\hline
\endfoot
\DescribeEvent{
  n={1},
  event={Пролог},
  Mat={\bibref[1:1]{Mat 1:1}},
  Mar={\bibref[1:1--3]{Mar 1:1}},
  Luk={\bibref[1:1--4]{Luk 1:1}},
  Joh={\bibref[1:1--18]{Joh 1:1}}
}
\DescribeEvent{
  n={2},
  event={Родословие Иисуса Христа},
  Mat={\bibref[1:1--17]{Mat 1:1}},
  Mar={},
  Luk={\bibref[3:23--38]{Luk 3:23}},
  Joh={}
}
\DescribeEvent{
  n={3},
  event={Благовестие Захарии о рождении Иоанна Предтечи},
  Mat={},
  Mar={},
  Luk={\bibref[1:5--25]{Luk 1:5}},
  Joh={}
}
\DescribeEvent{
  n={4},
  event={Благовещение Марии},
  Mat={},
  Mar={},
  Luk={\bibref[1:26--38]{Luk 1:26}},
  Joh={}
}
\DescribeEvent{
  n={},
  event={Мария в доме Елисаветы},
  Mat={},
  Mar={},
  Luk={\bibref[1:39--56]{Luk 1:39}},
  Joh={}
}
\DescribeEvent{
  n={5},
  event={Рождение Иоанна Предтечи},
  Mat={},
  Mar={},
  Luk={\bibref[1:57--80]{Luk 1:57}},
  Joh={}
}
\DescribeEvent{
  n={6},
  event={Откровение Иосифу Праведному о Боговоплощении},
  Mat={\bibref[1:18--25]{Mat 1:18}},
  Mar={},
  Luk={},
  Joh={}
}
\DescribeEvent{
  n={7},
  event={Рождество Иисуса Христа},
  Mat={\bibref[2:1]{Mat 2:1}},
  Mar={},
  Luk={\bibref[2:1--7]{Luk 2:1}},
  Joh={}
}
\DescribeEvent{
  n={8},
  event={Поклонение пастырей},
  Mat={},
  Mar={},
  Luk={\bibref[2:8--20]{Luk 2:8}},
  Joh={}
}
\DescribeEvent{
  n={9},
  event={Обрезание и наречение имени Иисус},
  Mat={\bibref[1:25]{Mat 1:25}},
  Mar={},
  Luk={\bibref[2:21]{Luk 2:21}},
  Joh={}
}
\DescribeEvent{
  n={10},
  event={Сретение Господа в храме},
  Mat={},
  Mar={},
  Luk={\bibref[2:22--33]{Luk 2:22}},
  Joh={}
}
\DescribeEvent{
  n={11},
  event={Поклонение волхвов},
  Mat={\bibref[2:1--12]{Mat 2:1}},
  Mar={},
  Luk={},
  Joh={}
}
\DescribeEvent{
  n={12},
  event={Бегство в Египет},
  Mat={\bibref[2:13--15]{Mat 2:13}},
  Mar={},
  Luk={},
  Joh={}
}
\DescribeEvent{
  n={13},
  event={Избиение младенцев в Вифлееме},
  Mat={\bibref[2:16--18]{Mat 2:16}},
  Mar={},
  Luk={},
  Joh={}
}
\DescribeEvent{
  n={14},
  event={Возвращение из Египта, поселение в Назарете; отрок Иисус в храме},
  Mat={\bibref[2:19--23]{Mat 2:19}},
  Mar={},
  Luk={\bibref[2:39--52]{Luk 2:39}},
  Joh={}
}
\DescribeEvent{
  n={15},
  event={Проповедь Иоанна Предтечи в пустыне; свидетельство об Иисусе Христе},
  Mat={\bibref[3:1--12]{Mat 3:1}},
  Mar={\bibref[1:1--8]{Mar 1:1}},
  Luk={\bibref[3:1--18]{Luk 3:1}},
  Joh={\bibref[1:19--28]{Joh 1:19}}
}
\DescribeEvent{
  n={16},
  event={Крещение Господне},
  Mat={\bibref[13:13--17]{Mat 13:13}},
  Mar={\bibref[1:9--11]{Mar 1:9}},
  Luk={\bibref[3:21--22]{Luk 3:21}},
  Joh={}
}
\DescribeEvent{
  n={17},
  event={Искушение Иисуса Христа в пустыне},
  Mat={\bibref[4:1--11]{Mat 4:1}},
  Mar={\bibref[1:12--13]{Mar 1:12}},
  Luk={\bibref[4:1--13]{Luk 4:1}},
  Joh={}
}
\hline
\EventHeader{Дела Господа Иисуса Христа в Иудее после искушения Его в пустыне до первой Пасхи}
\hline
\DescribeEvent{
  n={18},
  event={Последующие свидетельства Иоанна Предтечи об Иисусе Христе},
  Mat={},
  Mar={},
  Luk={},
  Joh={\bibref[1:19--36]{Joh 1:19}}
}
\DescribeEvent{
  n={},
  event={Первые ученики Господа},
  Mat={},
  Mar={},
  Luk={},
  Joh={\bibref[1:37--51]{Joh 1:37}}
}
\hline
\EventHeader{Дела Господа Иисуса Христа в Галилее по возвращении из Иудеи}
\hline
\DescribeEvent{
  n={19},
  event={Чудо в Кане галилейской},
  Mat={},
  Mar={},
  Luk={},
  Joh={\bibref[2:1--11]{Joh 2:1}}
}
\DescribeEvent{
  n={20},
  event={Посещение Капернаума},
  Mat={},
  Mar={},
  Luk={},
  Joh={\bibref[2:12]{Joh 2:12}}
}
\hline
\EventHeaderCap{Служение Господа Иисуса Христа от первой пасхи до второй}
\EventHeader{События в Иудее}
\hline
\DescribeEvent{
  n={21},
  event={Изгнание торгующих из храма. Свидетельство Иисуса Христа о Своем Богосыновстве},
  Mat={},
  Mar={},
  Luk={},
  Joh={\bibref[2:13--25]{Joh 2:13}}
}
\DescribeEvent{
  n={22},
  event={Беседа Иисуса Христа с Никодимом},
  Mat={},
  Mar={},
  Luk={},
  Joh={\bibref[3:1--21]{Joh 3:1}}
}
\DescribeEvent{
  n={23},
  event={Иисус с учениками в Иудее},
  Mat={},
  Mar={},
  Luk={},
  Joh={\bibref[3:22]{Joh 3:22}}
}
\DescribeEvent{
  n={},
  event={Последнее свидетельство Иоанна Крестителя об Иисусе Христе пред учениками},
  Mat={},
  Mar={},
  Luk={},
  Joh={\bibref[3:23--36]{Joh 3:23}}
}
\DescribeEvent{
  n={24},
  event={Заключение Иоанна Предтечи в темницу},
  Mat={\bibref[14:3--5]{Mat 14:3}},
  Mar={\bibref[6:17--20]{Mar 6:17}},
  Luk={\bibref[3:19--20]{Luk 3:19}},
  Joh={}
}
\hline
\EventHeader{События по пути из Иудеи в Галилею}
\hline
\DescribeEvent{
  n={25},
  event={Возвращение в Галилею},
  Mat={\bibref[4:12--17]{Mat 4:12}},
  Mar={},
  Luk={},
  Joh={\bibref[4:1--3]{Joh 4:1}}
}
\DescribeEvent{
  n={},
  event={Беседа с самарянкой},
  Mat={},
  Mar={},
  Luk={},
  Joh={\bibref[4:4--42]{Joh 4:4}}
}
\hline
\EventHeader{Служение Иисуса Христа в Галилее}
\hline
\DescribeEvent{
  n={26},
  event={Приход Иисуса Христа в Кану галилейскую; исцеление сына капернаумского царедворца},
  Mat={},
  Mar={},
  Luk={},
  Joh={\bibref[4:43--54]{Joh 4:43}}
}
\DescribeEvent{
  n={27},
  event={Начало Евангельской проповеди},
  Mat={},
  Mar={\bibref[1:14--15]{Mar 1:14}},
  Luk={\bibref[4:14--15]{Luk 4:14}},
  Joh={}
}
\DescribeEvent{
  n={28},
  event={Проповедь Иисуса Христа в назаретской синагоге},
  Mat={},
  Mar={},
  Luk={\bibref[4:16--30]{Luk 4:16}},
  Joh={}
}
\DescribeEvent{
  n={29},
  event={Поселение и проповедь в Капернауме},
  Mat={\bibref[4:13--16]{Mat 4:13}},
  Mar={\bibref[1:21]{Mar 1:21}},
  Luk={\bibref[4:31--32]{Luk 4:31}},
  Joh={}
}
\DescribeEvent{
  n={30},
  event={Призвание к апостольству Петра, Андрея, Иакова и Иоанна},
  Mat={\bibref[4:18--22]{Mat 4:18}},
  Mar={\bibref[1:16--20]{Mar 1:16}},
  Luk={\bibref[5:1--11]{Luk 5:1}},
  Joh={}
}
\DescribeEvent{
  n={31},
  event={Исцеление бесноватого в капернаумской синагоге},
  Mat={},
  Mar={\bibref[1:21--28]{Mar 1:21}},
  Luk={\bibref[4:31--37]{Luk 4:31}},
  Joh={}
}
\DescribeEvent{
  n={32},
  event={Исцеление тещи Симона},
  Mat={},
  Mar={\bibref[1:29--31]{Mar 1:29}},
  Luk={\bibref[4:38--39]{Luk 4:38}},
  Joh={}
}
\DescribeEvent{
  n={33},
  event={Исцеление многих больных},
  Mat={\bibref[8:16--17]{Mat 8:16}},
  Mar={\bibref[1:32--34]{Mar 1:32}},
  Luk={\bibref[4:40--41]{Luk 4:40}},
  Joh={}
}
\DescribeEvent{
  n={34},
  event={Благовестие в Галилее},
  Mat={\bibref[4:23--25]{Mat 4:23}},
  Mar={\bibref[1:35--39]{Mar 1:35}},
  Luk={\bibref[4:42--44]{Luk 4:42}},
  Joh={}
}
\DescribeEvent{
  n={35},
  event={Исцеление прокаженного},
  Mat={\bibref[8:2--4]{Mat 8:2}},
  Mar={\bibref[1:40--45]{Mar 1:40}},
  Luk={\bibref[5:12--16]{Luk 5:12}},
  Joh={}
}
\DescribeEvent{
  n={36},
  event={Исцеление расслабленного в Капернауме},
  Mat={\bibref[9:1--8]{Mat 9:1}},
  Mar={\bibref[2:1--12]{Mar 2:1}},
  Luk={\bibref[5:17--26]{Luk 5:17}},
  Joh={}
}
\DescribeEvent{
  n={37},
  event={Призвание Левия-Матфея к апостольству},
  Mat={\bibref[9:9--13]{Mat 9:9}},
  Mar={\bibref[2:13--17]{Mar 2:13}},
  Luk={\bibref[5:27--32]{Luk 5:27}},
  Joh={}
}
\DescribeEvent{
  n={38},
  event={Ответ ученикам Иоанна о посте},
  Mat={\bibref[9:14--17]{Mat 9:14}},
  Mar={\bibref[2:18--22]{Mar 2:18}},
  Luk={\bibref[5:33--39]{Luk 5:33}},
  Joh={}
}
\hline
\EventHeaderCap{Служение Господа Иисуса Христа от второй пасхи до третьей}
\EventHeader{События в Иудее}
\hline
\DescribeEvent{
  n={39},
  event={Иисус Христос в Иерусалиме на второй Пасхе. Исцеление расслабленного при Овчей купели},
  Mat={},
  Mar={},
  Luk={},
  Joh={\bibref[5:1--17]{Joh 5:1}}
}
\DescribeEvent{
  n={40},
  event={Откровение Иисуса Христа о Своем Богосыновстве},
  Mat={},
  Mar={},
  Luk={},
  Joh={\bibref[5:17--47]{Joh 5:17}}
}
\hline
\EventHeader{Проповедь и чудеса Иисуса Христа в Галилее}
\hline
\DescribeEvent{
  n={41},
  event={О значении субботы; срывание колосьев},
  Mat={\bibref[12:1--8]{Mat 12:1}},
  Mar={\bibref[2:23--28]{Mar 2:23}},
  Luk={\bibref[6:1--5]{Luk 6:1}},
  Joh={}
}
\DescribeEvent{
  n={42},
  event={Исцеление сухорукого},
  Mat={\bibref[12:9--13]{Mat 12:9}},
  Mar={\bibref[3:1--5]{Mar 3:1}},
  Luk={\bibref[6:6--11]{Luk 6:6}},
  Joh={}
}
\DescribeEvent{
  n={43},
  event={Злоба фарисеев и стремление народа к Иисусу},
  Mat={\bibref[12:14--21]{Mat 12:14}},
  Mar={\bibref[3:6--12]{Mar 3:6}},
  Luk={\bibref[6:11]{Luk 6:11}, \bibref[17--19]{Luk 6:17}},
  Joh={}
}
\DescribeEvent{
  n={44},
  event={Избрание 12 апостолов},
  Mat={\bibref[10:1--4]{Mat 10:1}},
  Mar={\bibref[3:13--19]{Mar 3:13}},
  Luk={\bibref[6:12--16]{Luk 6:12}},
  Joh={}
}
\DescribeEvent{
  n={45},
  event={Нагорная проповедь},
  Mat={\bibref[5:1--7:29]{Mat 5:1}},
  Mar={},
  Luk={\bibref[6:17--49]{Luk 6:17}},
  Joh={}
}
\DescribeEvent{
  n={46},
  event={Исцеление слуги капернаумского сотника},
  Mat={\bibref[8:5--13]{Mat 8:5}},
  Mar={},
  Luk={\bibref[7:1--10]{Luk 7:1}},
  Joh={}
}
\DescribeEvent{
  n={47},
  event={Воскрешение сына вдовы в Наине},
  Mat={},
  Mar={},
  Luk={\bibref[7:11--17]{Luk 7:11}},
  Joh={}
}
\DescribeEvent{
  n={48},
  event={Свидетельство Иисуса о Себе и об Иоанне Крестителе пред Иоанновыми учениками},
  Mat={\bibref[11:1--19]{Mat 11:1}},
  Mar={},
  Luk={\bibref[7:18--35]{Luk 7:18}},
  Joh={}
}
\DescribeEvent{
  n={49},
  event={Призыв труждающихся и обремененных},
  Mat={\bibref[11:27--30]{Mat 11:27}},
  Mar={},
  Luk={},
  Joh={}
}
\DescribeEvent{
  n={50},
  event={Прощение грешницы в доме фарисея Симона},
  Mat={},
  Mar={},
  Luk={\bibref[7:36--50]{Luk 7:36}},
  Joh={}
}
\DescribeEvent{
  n={51},
  event={Исцеление бесноватого глухонемого слепца},
  Mat={\bibref[12:22--23]{Mat 12:22}},
  Mar={},
  Luk={\bibref[11:14]{Luk 11:14}},
  Joh={}
}
\DescribeEvent{
  n={52},
  event={Изобличение хулы на Духа Святого},
  Mat={\bibref[12:24--37]{Mat 12:24}},
  Mar={\bibref[3:20--30]{Mar 3:20}},
  Luk={\bibref[11:15--26]{Luk 11:15}},
  Joh={}
}
\DescribeEvent{
  n={53},
  event={О требовании знамения},
  Mat={\bibref[12:38--45]{Mat 12:38}},
  Mar={},
  Luk={\bibref[11:16]{Luk 11:16}, \bibref[29--32]{Luk 11:29}},
  Joh={}
}
\DescribeEvent{
  n={54},
  event={О внутреннем свете},
  Mat={},
  Mar={},
  Luk={\bibref[11:33--36]{Luk 11:33}},
  Joh={}
}
\DescribeEvent{
  n={55},
  event={Похвала слушающим слово Божие},
  Mat={\bibref[12:46--50]{Mat 12:46}},
  Mar={\bibref[3:31--35]{Mar 3:31}},
  Luk={\bibref[8:19--21]{Luk 8:19}, \bibref[11:27--28]{Luk 11:27}},
  Joh={}
}
\DescribeEvent{
  n={56},
  event={Изобличение внешней праведности},
  Mat={},
  Mar={},
  Luk={\bibref[11:37--54]{Luk 11:37}},
  Joh={}
}
\DescribeEvent{
  n={57},
  event={Учение при море притчами о Царствии Божием},
  Mat={\bibref[13:1--58]{Mat 13:1}},
  Mar={\bibref[4:1--34]{Mar 4:1}},
  Luk={\bibref[12:1--59]{Luk 12:1}, \bibref[8:4--18]{Luk 8:4}, \bibref[13:18--21]{Luk 13:18}},
  Joh={}
}
\DescribeEvent{
  n={58},
  event={О последовании Иисусу},
  Mat={\bibref[8:18--22]{Mat 8:18}},
  Mar={},
  Luk={},
  Joh={}
}
\DescribeEvent{
  n={59},
  event={Укрощение бури на пути через Геннисаретское озеро в Гадаринскую страну},
  Mat={\bibref[8:23--27]{Mat 8:23}},
  Mar={\bibref[4:35--41]{Mar 4:35}},
  Luk={\bibref[8:22--25]{Luk 8:22}},
  Joh={}
}
\DescribeEvent{
  n={60},
  event={Исцеление бесноватых в Гадаринской стране},
  Mat={\bibref[8:28--31]{Mat 8:28}},
  Mar={\bibref[5:1--20]{Mar 5:1}},
  Luk={\bibref[8:26--39]{Luk 8:26}},
  Joh={}
}
\DescribeEvent{
  n={61},
  event={Воскрешение дочери Иаира, исцеление кровоточивой больной},
  Mat={},
  Mar={\bibref[5:22--43]{Mar 5:22}},
  Luk={\bibref[8:40--56]{Luk 8:40}},
  Joh={}
}
\DescribeEvent{
  n={62},
  event={Исцеление двух слепцов},
  Mat={\bibref[9:27--31]{Mat 9:27}},
  Mar={},
  Luk={},
  Joh={}
}
\DescribeEvent{
  n={63},
  event={Исцеление бесноватого немого},
  Mat={\bibref[9:32--34]{Mat 9:32}},
  Mar={},
  Luk={},
  Joh={}
}
\DescribeEvent{
  n={64},
  event={Проповедь в назаретской синагоге},
  Mat={\bibref[13:54--58]{Mat 13:54}},
  Mar={\bibref[6:1--6]{Mar 6:1}},
  Luk={},
  Joh={}
}
\DescribeEvent{
  n={65},
  event={Проповедь в окрестных городах и селениях},
  Mat={\bibref[9:35--38]{Mat 9:35}},
  Mar={\bibref[6:6]{Mar 6:6}},
  Luk={},
  Joh={}
}
\DescribeEvent{
  n={66},
  event={Наставления 12 при послании их на проповедь},
  Mat={\bibref[10:1--42]{Mat 10:1}},
  Mar={\bibref[6:7--13]{Mar 6:7}},
  Luk={\bibref[9:1--6]{Luk 9:1}},
  Joh={}
}
\DescribeEvent{
  n={67},
  event={Смерть Иоанна Крестителя},
  Mat={\bibref[14:6--12]{Mat 14:6}},
  Mar={\bibref[6:17--29]{Mar 6:17}},
  Luk={},
  Joh={}
}
\DescribeEvent{
  n={68},
  event={Молва об Иисусе Христе, смятение Ирода},
  Mat={\bibref[14:1--2]{Mat 14:1}},
  Mar={\bibref[6:14--16]{Mar 6:14}},
  Luk={\bibref[9:7--9]{Luk 9:7}},
  Joh={}
}
\DescribeEvent{
  n={69},
  event={Возвращение 12 с проповеди},
  Mat={},
  Mar={\bibref[6:30]{Mar 6:30}},
  Luk={\bibref[9:10]{Luk 9:10}},
  Joh={}
}
\DescribeEvent{
  n={70},
  event={Насыщение 5000 народа пятью хлебами и двумя рыбами},
  Mat={\bibref[14:13--21]{Mat 14:13}},
  Mar={\bibref[6:31--44]{Mar 6:31}},
  Luk={\bibref[9:11--17]{Luk 9:11}},
  Joh={\bibref[6:1--14]{Joh 6:1}}
}
\DescribeEvent{
  n={71},
  event={Шествие Иисуса Христа к ученикам по воде},
  Mat={\bibref[14:22--34]{Mat 14:22}},
  Mar={\bibref[6:45--53]{Mar 6:45}},
  Luk={},
  Joh={\bibref[6:15--21]{Joh 6:15}}
}
\DescribeEvent{
  n={72},
  event={Исцеление больных в Геннисаретской стране},
  Mat={\bibref[14:35--36]{Mat 14:35}},
  Mar={\bibref[6:54--56]{Mar 6:54}},
  Luk={},
  Joh={}
}
\DescribeEvent{
  n={73},
  event={Беседа Иисуса Христа о небесном хлебе},
  Mat={},
  Mar={},
  Luk={},
  Joh={\bibref[6:22--71]{Joh 6:22}}
}
\hline
\EventHeaderCap{События от третьей пасхи до четвертой --- пасхи страданий}
\hline
\DescribeEvent{
  n={74},
  event={Обличение иудеев в лицемерном исполнении заповедей},
  Mat={},
  Mar={\bibref[7:1--23]{Mar 7:1}},
  Luk={},
  Joh={}
}
\DescribeEvent{
  n={75},
  event={Путешествие Иисуса Христа в пределы Тира и Сидона, исцеление дочери хананеянки},
  Mat={\bibref[15:21--28]{Mat 15:21}},
  Mar={\bibref[7:24--30]{Mar 7:24}},
  Luk={},
  Joh={}
}
\DescribeEvent{
  n={76},
  event={Исцеление глухонемого},
  Mat={},
  Mar={\bibref[7:31--37]{Mar 7:31}},
  Luk={},
  Joh={}
}
\DescribeEvent{
  n={77},
  event={Исцеление множества народа у Геннисаретского озера},
  Mat={\bibref[15:29--31]{Mat 15:29}},
  Mar={},
  Luk={},
  Joh={}
}
\DescribeEvent{
  n={78},
  event={Насыщение 4000 семью хлебами и несколькими рыбами},
  Mat={\bibref[15:32--38]{Mat 15:32}},
  Mar={\bibref[8:1--9]{Mar 8:1}},
  Luk={},
  Joh={}
}
\DescribeEvent{
  n={79},
  event={Прибытие в страну Магдалинскую. Ответ фарисеям просившим знамения с неба},
  Mat={\bibref[15:39--16:4]{Mat 15:39}},
  Mar={\bibref[8:10--13]{Mar 8:10}},
  Luk={},
  Joh={}
}
\DescribeEvent{
  n={80},
  event={Предостережение ученикам от закваски фарисейской},
  Mat={\bibref[15:1--12]{Mat 15:1}},
  Mar={\bibref[8:14--21]{Mar 8:14}},
  Luk={},
  Joh={}
}
\DescribeEvent{
  n={81},
  event={Исцеление слепого в Вифсаиде},
  Mat={},
  Mar={\bibref[8:22--26]{Mar 8:22}},
  Luk={},
  Joh={}
}
\DescribeEvent{
  n={82},
  event={Исповедание Петра у Кесарии Филипповой},
  Mat={\bibref[16:13--20]{Mat 16:13}},
  Mar={\bibref[8:27--30]{Mar 8:27}},
  Luk={\bibref[9:18--21]{Luk 9:18}},
  Joh={}
}
\DescribeEvent{
  n={83},
  event={Первое предсказание Иисуса Христа о крестных страданиях; наставление о несении креста},
  Mat={\bibref[16:21--28]{Mat 16:21}},
  Mar={\bibref[8:31--9:1]{Mar 8:31}},
  Luk={\bibref[9:22--27]{Luk 9:22}},
  Joh={}
}
\DescribeEvent{
  n={84},
  event={Преображение Господне},
  Mat={\bibref[17:1--13]{Mat 17:1}},
  Mar={\bibref[9:1--9]{Mar 9:1}},
  Luk={\bibref[9:28--36]{Luk 9:28}},
  Joh={}
}
\DescribeEvent{
  n={85},
  event={Второе предсказание Иисуса Христа о крестных страданиях},
  Mat={},
  Mar={\bibref[9:10--13]{Mar 9:10}},
  Luk={},
  Joh={}
}
\DescribeEvent{
  n={86},
  event={Исцеление бесноватого лунатика},
  Mat={\bibref[17:14--21]{Mat 17:14}},
  Mar={\bibref[9:14--29]{Mar 9:14}},
  Luk={\bibref[9:37--43]{Luk 9:37}},
  Joh={}
}
\DescribeEvent{
  n={87},
  event={Третье предсказание Иисуса Христа о крестных страданиях},
  Mat={\bibref[17:22--23]{Mat 17:22}},
  Mar={\bibref[9:30--32]{Mar 9:30}},
  Luk={\bibref[9:43--45]{Luk 9:43}},
  Joh={}
}
\DescribeEvent{
  n={88},
  event={Последнее пребывание Господа Иисуса Христа в Капернауме; ответ Иисуса Христа о подати на храм},
  Mat={\bibref[17:24--27]{Mat 17:24}},
  Mar={},
  Luk={},
  Joh={}
}
\DescribeEvent{
  n={89},
  event={Наставление о смирении},
  Mat={\bibref[18:1--6]{Mat 18:1}},
  Mar={\bibref[9:33--37]{Mar 9:33}},
  Luk={\bibref[9:46--50]{Luk 9:46}},
  Joh={}
}
\DescribeEvent{
  n={90},
  event={Речь о спасении погибающих, притча о пропавшей овце},
  Mat={\bibref[18:7--17]{Mat 18:7}},
  Mar={},
  Luk={},
  Joh={}
}
\DescribeEvent{
  n={91},
  event={Учение о прощении грехов ближнего},
  Mat={\bibref[18:18--22]{Mat 18:18}, \bibref[18:23--35]{Mat 18:23}},
  Mar={},
  Luk={},
  Joh={}
}
\hline
\EventHeader{Господь Иисус Христос в Галилее пред праздником Кущей}
\hline
\DescribeEvent{
  n={92},
  event={Отказ Иисуса Христа идти на праздник в Иерусалим},
  Mat={},
  Mar={},
  Luk={},
  Joh={\bibref[7:2--9]{Joh 7:2}}
}
\DescribeEvent{
  n={93},
  event={Путешествие Иисуса Христа в Иерусалим},
  Mat={\bibref[19:1]{Mat 19:1}},
  Mar={\bibref[10:1]{Mar 10:1}},
  Luk={\bibref[9:51]{Luk 9:51}},
  Joh={}
}
\DescribeEvent{
  n={94},
  event={Неприязненное отношение к Иисусу Христу самарийцев},
  Mat={},
  Mar={},
  Luk={\bibref[9:52--56]{Luk 9:52}},
  Joh={}
}
\DescribeEvent{
  n={95},
  event={Ответ Господа Иисуса Христа пожелавшим следовать за Ним},
  Mat={},
  Mar={},
  Luk={\bibref[9:57--62]{Luk 9:57}},
  Joh={}
}
\DescribeEvent{
  n={96},
  event={Послание 70 на проповедь; укор городам не принявшим проповеди},
  Mat={\bibref[11:20--24]{Mat 11:20}},
  Mar={},
  Luk={\bibref[10:1--24]{Luk 10:1}},
  Joh={}
}
\DescribeEvent{
  n={97},
  event={Призыв труждающихся и обремененных},
  Mat={\bibref[11:27--30]{Mat 11:27}},
  Mar={},
  Luk={},
  Joh={}
}
\DescribeEvent{
  n={98},
  event={Вопрос законника о вечной жизни и о ближнем и притча о милосердном самарянине},
  Mat={},
  Mar={},
  Luk={\bibref[10:25--37]{Luk 10:25}},
  Joh={}
}
\DescribeEvent{
  n={99},
  event={Иисус Христос в доме Марфы и Марии},
  Mat={},
  Mar={},
  Luk={\bibref[10:38--42]{Luk 10:38}},
  Joh={}
}
\DescribeEvent{
  n={100},
  event={Учение Иисуса Христа о молитве},
  Mat={},
  Mar={},
  Luk={\bibref[11:1--13]{Luk 11:1}},
  Joh={}
}
\DescribeEvent{
  n={101},
  event={Иисус на обеде у фарисея: обличение законников},
  Mat={},
  Mar={},
  Luk={\bibref[11:37--54]{Luk 11:37}},
  Joh={}
}
\DescribeEvent{
  n={102},
  event={Наставления о правилах жизни Христовым последователям},
  Mat={},
  Mar={},
  Luk={\bibref[12:1--59]{Luk 12:1}},
  Joh={}
}
\hline
\EventHeader{Иисус Христос в Иерусалиме на празднике Кущей}
\hline
\DescribeEvent{
  n={103},
  event={Спор в народе о Христе},
  Mat={},
  Mar={},
  Luk={},
  Joh={\bibref[7:10--53]{Joh 7:10}}
}
\DescribeEvent{
  n={104},
  event={О женщине взятой в прелюбодеянии},
  Mat={},
  Mar={},
  Luk={},
  Joh={\bibref[8:1--11]{Joh 8:1}}
}
\DescribeEvent{
  n={105},
  event={Обличение иудеев при сокровищнице храма},
  Mat={},
  Mar={},
  Luk={},
  Joh={\bibref[8:12--59]{Joh 8:12}}
}
\DescribeEvent{
  n={106},
  event={Исцеление слепорожденного},
  Mat={},
  Mar={},
  Luk={},
  Joh={\bibref[9:1--41]{Joh 9:1}}
}
\DescribeEvent{
  n={107},
  event={<<Пастырь добрый>>},
  Mat={},
  Mar={},
  Luk={},
  Joh={\bibref[10:1--21]{Joh 10:1}}
}
\DescribeEvent{
  n={108},
  event={Известие об избитых Пилатом галилеянах, наставление о покаянии},
  Mat={},
  Mar={},
  Luk={\bibref[13:1--5]{Luk 13:1}},
  Joh={}
}
\DescribeEvent{
  n={109},
  event={Притча о бесплодной смоковнице},
  Mat={},
  Mar={},
  Luk={\bibref[13:6--9]{Luk 13:6}},
  Joh={}
}
\DescribeEvent{
  n={110},
  event={Исцеление в синагоге в субботу женщины согбенной 18 лет},
  Mat={},
  Mar={},
  Luk={\bibref[13:10--17]{Luk 13:10}},
  Joh={}
}
\DescribeEvent{
  n={111},
  event={Учение Иисуса Христа о Царствии Божием в притчах},
  Mat={},
  Mar={},
  Luk={\bibref[13:18--21]{Luk 13:18}},
  Joh={}
}
\DescribeEvent{
  n={112},
  event={Беседа Иисуса Христа в праздник Обновления в притворе Соломоновом},
  Mat={},
  Mar={},
  Luk={},
  Joh={\bibref[10:22--39]{Joh 10:22}}
}
\DescribeEvent{
  n={113},
  event={Удаление Господа Иисуса Христа из Иерусалима в заиорданскую страну Перею},
  Mat={},
  Mar={},
  Luk={},
  Joh={\bibref[10:40--42]{Joh 10:40}}
}
\hline
\EventHeader{Учение Господа Иисуса Христа на обратном пути из заиорданской страны в Иерусалим}
\hline
\DescribeEvent{
  n={114},
  event={Речь Иисуса Христа о числе спасающихся},
  Mat={},
  Mar={},
  Luk={\bibref[18:22--30]{Luk 18:22}},
  Joh={}
}
\DescribeEvent{
  n={115},
  event={Предсказание о страданиях в Иерусалиме},
  Mat={},
  Mar={},
  Luk={\bibref[13:31--35]{Luk 13:31}},
  Joh={}
}
\DescribeEvent{
  n={116},
  event={Иисус Христос на обеде у фарисея; исцеление больного водянкой; притча о званых на вечерю},
  Mat={},
  Mar={},
  Luk={\bibref[14:1--24]{Luk 14:1}},
  Joh={}
}
\DescribeEvent{
  n={117},
  event={Самоотвержение последователей Христа},
  Mat={},
  Mar={},
  Luk={\bibref[14:25--35]{Luk 14:25}},
  Joh={}
}
\DescribeEvent{
  n={118},
  event={Притчи о погибшей овце, потерянной драхме и блудном сыне},
  Mat={},
  Mar={},
  Luk={\bibref[15:1--32]{Luk 15:1}},
  Joh={}
}
\DescribeEvent{
  n={119},
  event={О неверном домоправителе},
  Mat={},
  Mar={},
  Luk={\bibref[16:1--13]{Luk 16:1}},
  Joh={}
}
\DescribeEvent{
  n={120},
  event={Обличение фарисеев; притча о богаче и Лазаре},
  Mat={},
  Mar={},
  Luk={\bibref[16:14--31]{Luk 16:14}},
  Joh={}
}
\DescribeEvent{
  n={121},
  event={Наставление ученикам о исполнении долга},
  Mat={},
  Mar={},
  Luk={\bibref[17:1--10]{Luk 17:1}},
  Joh={}
}
\DescribeEvent{
  n={122},
  event={Исцеление десяти прокаженных},
  Mat={},
  Mar={},
  Luk={\bibref[17:11--19]{Luk 17:11}},
  Joh={}
}
\DescribeEvent{
  n={123},
  event={Ответ о втором пришествии},
  Mat={},
  Mar={},
  Luk={\bibref[17:20--37]{Luk 17:20}},
  Joh={}
}
\DescribeEvent{
  n={124},
  event={Притча о несправедливом судье},
  Mat={},
  Mar={},
  Luk={\bibref[18:1--8]{Luk 18:1}},
  Joh={}
}
\DescribeEvent{
  n={125},
  event={Притча о мытаре и фарисее},
  Mat={},
  Mar={},
  Luk={\bibref[18:9--14]{Luk 18:9}},
  Joh={}
}
\DescribeEvent{
  n={126},
  event={Учение о браке},
  Mat={\bibref[19:1--12]{Mat 19:1}},
  Mar={\bibref[10:1--12]{Mar 10:1}},
  Luk={\bibref[16:18]{Luk 16:18}},
  Joh={}
}
\DescribeEvent{
  n={127},
  event={Благословение детей},
  Mat={\bibref[19:13--15]{Mat 19:13}},
  Mar={\bibref[10:13--16]{Mar 10:13}},
  Luk={\bibref[18:15--17]{Luk 18:15}},
  Joh={}
}
\DescribeEvent{
  n={128},
  event={Ответ богатому юноше},
  Mat={\bibref[19:16--26]{Mat 19:16}},
  Mar={\bibref[10:17--27]{Mar 10:17}},
  Luk={\bibref[18:18--27]{Luk 18:18}},
  Joh={}
}
\DescribeEvent{
  n={129},
  event={Вопрос Петра о награде последователям Иисуса Христа},
  Mat={\bibref[19:27--30]{Mat 19:27}},
  Mar={\bibref[10:28--31]{Mar 10:28}},
  Luk={\bibref[18:28--30]{Luk 18:28}},
  Joh={}
}
\DescribeEvent{
  n={130},
  event={Притча о нанятых в виноградник работниках},
  Mat={\bibref[20:1--16]{Mat 20:1}},
  Mar={},
  Luk={},
  Joh={}
}
\DescribeEvent{
  n={131},
  event={Воскрешение Лазаря},
  Mat={},
  Mar={},
  Luk={},
  Joh={\bibref[11:1--46]{Joh 11:1}}
}
\DescribeEvent{
  n={132},
  event={Заговор начальников иудейских против Иисуса Христа},
  Mat={},
  Mar={},
  Luk={},
  Joh={\bibref[11:47--57]{Joh 11:47}}
}
\DescribeEvent{
  n={133},
  event={Предсказание Иисуса Христа о крестных страданиях на пути к Иерусалиму},
  Mat={\bibref[20:17--19]{Mat 20:17}},
  Mar={\bibref[10:32--34]{Mar 10:32}},
  Luk={\bibref[18:31--34]{Luk 18:31}},
  Joh={}
}
\DescribeEvent{
  n={134},
  event={Просьба сынов Зеведеевых о месте в Царствии Иисуса Христа},
  Mat={\bibref[20:20--28]{Mat 20:20}},
  Mar={\bibref[10:35--45]{Mar 10:35}},
  Luk={},
  Joh={}
}
\DescribeEvent{
  n={135},
  event={Иисус Христос в Иерихоне; исцеление слепцов},
  Mat={\bibref[20:29--34]{Mat 20:29}},
  Mar={\bibref[10:46--52]{Mar 10:46}},
  Luk={\bibref[18:35--43]{Luk 18:35}},
  Joh={}
}
\DescribeEvent{
  n={136},
  event={Обращение Закхея},
  Mat={},
  Mar={},
  Luk={\bibref[19:1--10]{Luk 19:1}},
  Joh={}
}
\DescribeEvent{
  n={137},
  event={Притча об ушедшем на войну и об умноживших таланты},
  Mat={\bibref[25:13--30]{Mat 25:13}},
  Mar={},
  Luk={\bibref[19:11--28]{Luk 19:11}},
  Joh={}
}
\DescribeEvent{
  n={138},
  event={Иисус Христос в Вифании},
  Mat={\bibref[26:6--13]{Mat 26:6}},
  Mar={\bibref[14:3--9]{Mar 14:3}},
  Luk={},
  Joh={\bibref[11:55--12:11]{Joh 11:55}}
}
\DescribeEvent{
  n={139},
  event={Вход Господень в Иерусалим; исцеление больных в Иерусалиме},
  Mat={\bibref[21:1--11]{Mat 21:1}, \bibref[14--17]{Mat 21:14}},
  Mar={\bibref[11:1--11]{Mar 11:1}},
  Luk={\bibref[19:29--44]{Luk 19:29}},
  Joh={\bibref[12:12--19]{Joh 12:12}}
}
\hline
\EventHeader{Великий понедельник}
\hline
\DescribeEvent{
  n={140},
  event={Бесплодная смоковница},
  Mat={\bibref[21:18--22]{Mat 21:18}},
  Mar={\bibref[11:12--14]{Mar 11:12}, \bibref[20--26]{Mar 11:20}},
  Luk={},
  Joh={}
}
\DescribeEvent{
  n={141},
  event={Изгнание торгующих из храма},
  Mat={\bibref[21:12--13]{Mat 21:12}},
  Mar={\bibref[11:15--19]{Mar 11:15}},
  Luk={\bibref[19:45--46]{Luk 19:45}},
  Joh={}
}
\hline
\EventHeader{Великий вторник}
\hline
\DescribeEvent{
  n={142},
  event={Обличение начальников иудейских и поучения Господа в храме},
  Mat={\bibref[21:23--23:39]{Mat 21:23}},
  Mar={\bibref[11:27--12:40]{Mar 11:27}},
  Luk={\bibref[19:47--20:47]{Luk 19:47}},
  Joh={}
}
\DescribeEvent{
  n={143},
  event={О жертве бедной вдовицы},
  Mat={},
  Mar={\bibref[12:41--44]{Mar 12:41}},
  Luk={\bibref[21:1--4]{Luk 21:1}},
  Joh={}
}
\DescribeEvent{
  n={144},
  event={Речь Иисуса Христа пред эллинами и иудеями},
  Mat={},
  Mar={},
  Luk={},
  Joh={\bibref[12:20--50]{Joh 12:20}}
}
\DescribeEvent{
  n={145},
  event={Пророчества и притчи Иисуса Христа о Иерусалиме и о втором пришествии},
  Mat={\bibref[24:1--25:46]{Mat 24:1}},
  Mar={\bibref[13:1--37]{Mar 13:1}},
  Luk={\bibref[21:5--38]{Luk 21:5}},
  Joh={}
}
\hline
\EventHeader{Великая среда}
\hline
\DescribeEvent{
  n={146},
  event={Иисус Христос в Вифании},
  Mat={\bibref[26:1--2]{Mat 26:1}, \bibref[6--13]{Mat 26:6}},
  Mar={\bibref[14:3--9]{Mar 14:3}},
  Luk={},
  Joh={}
}
\DescribeEvent{
  n={147},
  event={Заговор иудеев, предательство Иуды},
  Mat={\bibref[26:3--5]{Mat 26:3}, \bibref[14--16]{Mat 26:14}},
  Mar={\bibref[14:1--2]{Mar 14:1}, \bibref[10--11]{Mar 14:10}},
  Luk={\bibref[22:1--6]{Luk 22:1}},
  Joh={}
}
\hline
\EventHeader{Великий четверг}
\hline
\DescribeEvent{
  n={148},
  event={Пасхальная вечеря},
  Mat={\bibref[26:17--35]{Mat 26:17}},
  Mar={\bibref[14:12--31]{Mar 14:12}},
  Luk={\bibref[22:7--38]{Luk 22:7}},
  Joh={\bibref[13:1--17]{Joh 13:1}, \bibref[26]{Joh 13:26}}
}
\DescribeEvent{
  n={149},
  event={Моление о чаше},
  Mat={\bibref[26:36--46]{Mat 26:36}},
  Mar={\bibref[14:32--42]{Mar 14:32}},
  Luk={\bibref[22:39--46]{Luk 22:39}},
  Joh={\bibref[18:1]{Joh 18:1}}
}
\hline
\EventHeader{Великая пятница}
\hline
\DescribeEvent{
  n={150},
  event={Взятие Иисуса Христа под стражу},
  Mat={\bibref[26:47--56]{Mat 26:47}},
  Mar={\bibref[14:43--52]{Mar 14:43}},
  Luk={\bibref[22:47--53]{Luk 22:47}},
  Joh={\bibref[18:2--12]{Joh 18:2}}
}
\DescribeEvent{
  n={151},
  event={Допрос Иисуса Христа у Анны},
  Mat={},
  Mar={},
  Luk={},
  Joh={\bibref[18:13]{Joh 18:13}, \bibref[19--24]{Joh 18:19}}
}
\DescribeEvent{
  n={152},
  event={Суд Синедриона над Иисусом Христом у Каиафы},
  Mat={\bibref[26:57--68]{Mat 26:57}},
  Mar={\bibref[14:53--65]{Mar 14:53}},
  Luk={\bibref[22:54]{Luk 22:54}, \bibref[63--65]{Luk 22:63}},
  Joh={\bibref[18:14]{Joh 18:14}}
}
\DescribeEvent{
  n={153},
  event={Отречение Петра},
  Mat={\bibref[26:58]{Mat 26:58}, \bibref[69--75]{Mat 26:69}},
  Mar={\bibref[14:54]{Mar 14:54}, \bibref[66--72]{Mar 14:66}},
  Luk={\bibref[22:54--62]{Luk 22:54}},
  Joh={\bibref[18:15--18]{Joh 18:15}, \bibref[25--27]{Joh 18:25}}
}
\DescribeEvent{
  n={154},
  event={Приговор Синедриона},
  Mat={\bibref[27:1]{Mat 27:1}},
  Mar={\bibref[15:1]{Mar 15:1}},
  Luk={\bibref[22:66--71]{Luk 22:66}},
  Joh={}
}
\DescribeEvent{
  n={155},
  event={Конец Иуды},
  Mat={\bibref[27:3--10]{Mat 27:3}},
  Mar={},
  Luk={},
  Joh={}
}
\DescribeEvent{
  n={156},
  event={Иисус Христос у Пилата},
  Mat={\bibref[27:2--31]{Mat 27:2}},
  Mar={\bibref[15:1--20]{Mar 15:1}},
  Luk={\bibref[23:1--25]{Luk 23:1}},
  Joh={\bibref[18:28--19:16]{Joh 18:28}}
}
\DescribeEvent{
  n={157},
  event={Крестный путь и Голгофа},
  Mat={\bibref[27:31--56]{Mat 27:31}},
  Mar={\bibref[15:20--41]{Mar 15:20}},
  Luk={\bibref[23:26--49]{Luk 23:26}},
  Joh={\bibref[19:16--37]{Joh 19:16}}
}
\DescribeEvent{
  n={158},
  event={Погребение Иисуса Христа},
  Mat={\bibref[27:57--66]{Mat 27:57}},
  Mar={\bibref[15:42--47]{Mar 15:42}},
  Luk={\bibref[23:50--56]{Luk 23:50}},
  Joh={\bibref[19:38--42]{Joh 19:38}}
}
\hline
\EventHeaderCap{Воскресение Иисуса Христа}
\EventHeader{Утро Воскресения}
\hline
\DescribeEvent{
  n={159},
  event={Поздно в субботу Мария Магдалина с другой Марией идут смотреть гроб},
  Mat={\bibref[28:1]{Mat 28:1}},
  Mar={},
  Luk={},
  Joh={}
}
\DescribeEvent{
  n={160},
  event={Мария Магдалина и другие женщины покупают ароматы чтобы утром помазать Иисуса},
  Mat={},
  Mar={\bibref[16:1]{Mar 16:1}},
  Luk={},
  Joh={}
}
\DescribeEvent{
  n={161},
  event={Землетрясение, ангел отваливает камень от пещеры},
  Mat={\bibref[28:2--4]{Mat 28:2}},
  Mar={},
  Luk={},
  Joh={}
}
\DescribeEvent{
  n={162},
  event={Мария Магдалина спешит ко гробу. Увидев, что он пуст, она возвращается к Петру и Иоанну},
  Mat={},
  Mar={},
  Luk={},
  Joh={\bibref[20:1--3]{Joh 20:1}}
}
\DescribeEvent{
  n={163},
  event={Приход ко гробу до восхода солнца группы галилейских жен-мироносиц; явление им ангелов},
  Mat={},
  Mar={},
  Luk={\bibref[24:1--9]{Luk 24:1}},
  Joh={}
}
\DescribeEvent{
  n={164},
  event={Петр и Иоанн с Марией Магдалиной прибегают к пустому гробу},
  Mat={},
  Mar={},
  Luk={},
  Joh={\bibref[20:4--10]{Joh 20:4}}
}
\DescribeEvent{
  n={165},
  event={Явление Воскресшего Христа Марии Магдалине},
  Mat={},
  Mar={},
  Luk={},
  Joh={\bibref[20:11--18]{Joh 20:11}}
}
\DescribeEvent{
  n={166},
  event={Приход на гроб при восходе солнца другой группы жен-мироносиц; явление им ангела},
  Mat={\bibref[28:5--8]{Mat 28:5}},
  Mar={\bibref[16:1--8]{Mar 16:1}},
  Luk={},
  Joh={}
}
\DescribeEvent{
  n={167},
  event={Уход мироносиц от гроба; явление им Воскресшего Христа},
  Mat={\bibref[28:9--10]{Mat 28:9}},
  Mar={},
  Luk={},
  Joh={}
}
\DescribeEvent{
  n={168},
  event={Извещение учеников о Воскресении Господа группой галилейских жен-мироносиц},
  Mat={},
  Mar={\bibref[16:8]{Mar 16:8}},
  Luk={\bibref[24:9--12]{Luk 24:9}},
  Joh={}
}
\DescribeEvent{
  n={169},
  event={Извещение учеников о Воскресении Господа Марией Магдалиной},
  Mat={},
  Mar={\bibref[16:9--11]{Mar 16:9}},
  Luk={},
  Joh={}
}
\DescribeEvent{
  n={170},
  event={Явление Иисуса Христа Эммаусским путникам},
  Mat={},
  Mar={\bibref[16:12--13]{Mar 16:12}},
  Luk={\bibref[24:13--35]{Luk 24:13}},
  Joh={}
}
\DescribeEvent{
  n={171},
  event={Явление Иисуса Христа в первый день недели ученикам без Фомы},
  Mat={},
  Mar={\bibref[16:14]{Mar 16:14}},
  Luk={\bibref[24:36--49]{Luk 24:36}},
  Joh={\bibref[20:19--25]{Joh 20:19}}
}
\DescribeEvent{
  n={172},
  event={Явление Иисуса Христа по прошествии восьми дней},
  Mat={},
  Mar={},
  Luk={},
  Joh={\bibref[20:26--29]{Joh 20:26}}
}
\DescribeEvent{
  n={173},
  event={Явление Иисуса Христа при море Тивериадском},
  Mat={},
  Mar={},
  Luk={},
  Joh={\bibref[21:1--25]{Joh 21:1}}
}
\DescribeEvent{
  n={174},
  event={Явление Иисуса Христа на горе},
  Mat={\bibref[28:16--20]{Mat 28:16}},
  Mar={\bibref[16:15--18]{Mar 16:15}},
  Luk={},
  Joh={}
}
\DescribeEvent{
  n={175},
  event={Вознесение Господа},
  Mat={},
  Mar={\bibref[16:19--20]{Mar 16:19}},
  Luk={\bibref[24:50--53]{Luk 24:50}},
  Joh={}
}
\end{longtable}
\end{landscape}
\newpage
%\bibpdfbookmark{Денежные Единицы}{ntmoney}
\bibmark{book}{ДЕНЕЖНЫЕ ЕДИНИЦЫ}
\thispagestyle{empty}
\pagestyle{fancy}

\begin{center}
\normalsize\bfseries
ДЕНЕЖНЫЕ ЕДИНИЦЫ В НОВОМ ЗАВЕТЕ
\end{center}

%\begin{multicols}{2}
В новозаветное время в Палестине в основном имели хождение монеты
греческие и римские. В книгах Нового Завета упоминается десять видов
монет: одна монета иудейской чеканки, пять --- греческой и четыре ---
римской.

Ходячей \bibemph{иудейской монетой} был \textbf{сребреник}, остаток
маккавейской чеканки. Он равнялся сиклю и считался национальной
монетой, употреблявшейся предпочтительно пред всеми другими при
храме. За тридцать таких сребреников Иуда предал Христа (\bibref{Mat
26:15}; \bibref[27:3--6,9]{Mat 27:3}). По тогдашним ценам это была
достаточная сумма, чтобы купить небольшой участок земли даже в
окрестностях Иерусалима.

\bibemph{Греческие монеты}. Основная денежная единица \textbf{драхма}
(\bibref[\bk{Luk}~15:8,~9]{Luk 15:8}) --- серебряная монета, равная
римскому динарию. Одна драхма составляла 6000-ю часть аттического таланта,
100-ю часть мины и разделялась на 6 оволов (оболов). В зависимости от
места чеканки драхма имела разный вес: аттическая драхма ---
4,37~\bibemph{г} серебра (т.~е.\ в два раза меньше нашего серебряного
полтинника чеканки 1922 и 1924 года), эгинская --- 6,3~\bibemph{г}. В
разное время вес монеты и ее цена тоже колебались, поэтому вообще
сравнивать покупную способность древних денег с современными можно
только приблизительно. Серебряная монета достоинством в две драхмы
называлась \textbf{дидрахма} (\bibref{Mat 17:24}); внешне дидрахма
могла походить на серебряный полтинник. Дидрахма приравнивалась к
полусиклю, так что принималась вместо последнего в уплату храмовой
подати. Четыре драхмы составляли \textbf{статир} (\bibref{Mat 17:27}) ---
серебряную монету, называвшуюся также \textbf{тетрадрахмой} (он мог
быть вроде серебряного рубля чеканки 1924 года). Статир приравнивался
к полному священному сиклю или сребренику. Такой статир был найден
ап. Петром в пойманной им рыбе и отдан в уплату храмовой подати за
Иисуса Христа и за себя. Сто драхм или 25 статиров составляли
\textbf{мину} (\bibref{Luk 19:13}). Высшей денежной единицей был
\textbf{талант}, золотой или серебряный (\bibref{Mat 18:24};
\bibref[25:15]{Mat 25:15}; \bibref{Rev 16:21}). Золотой талант был
равен десяти серебряным. Аттический талант равнялся 60 минам или 6000
драхм, а коринфский талант --- 100 минам. Последний более подходит к
ценности собственно еврейского (ветхозаветного) серебряного таланта,
но к I~веку по Р.~Х.\ вес и стоимость таланта понизились.


\bibemph{Римские монеты}. \textbf{Динарий} (denarius) --- серебряная
монета, часто упоминаемая в евангелиях (\bibref{Mat 18:28};
\bibref[20:2]{Mat 20:2}; \bibref{Mar 6:37}; \bibref[12:15]{Mar 12:15};
\bibref{Luk 7:41}; \bibref[20:24]{Luk 20:24}; \bibref{Joh 6:7};
\bibref[12:5]{Joh 12:5}, также \bibref{Rev 6:6}). По весу и ценности
динарий приравнивался к греческой драхме или 1/4 сикля, но во время
земной жизни Спасителя он имел меньшую ценность. На лицевой стороне
монеты изображался царствующий император
(\bibref[\bk{Mat}~22:19--21]{Mat 22:19}).  Динарий составлял
ежедневную плату римскому воину, как драхма --- ежедневную плату
афинским воинам. Он же составлял обычную поденную плату рабочим
(\bibref{Mat 20:2}). Динарию же равнялась поголовная подать, которую
иудеи обязаны были платить римлянам (\bibref{Mat 22:19}). Динарий
разделялся на десять, а позднее --- на шестнадцать \textbf{ассариев}
или \textbf{асов} (\bibref{Mat 10:29}; \bibref{Luk 12:6}). Это была
медная монета. Четвертую часть ассария составлял \textbf{кодрант}
(quadrans) (\bibref{Mat 5:26}; \bibref{Mar 12:42}). На этих монетах
тоже изображался император. Половину кодранта составляла минута
(minutum) или \textbf{лепта} --- в русском переводе <<полушка>>
(\bibref{Luk 12:59}; \bibref{Mar 12:42}) --- самая мелкая медная
монета. Две такие монеты и положила в сокровищницу храма бедная
вдовица (\bibref[\bk{Mar}~12:41--44]{Mar 12:41}).
%\end{multicols}

\pagestyle{fancy}
\bibpart{Книги Ветхого Завета\\[5pt]\normalfont\protect\small(Знаком * отмечены книги неканонические)}{Ветхий Завет}{OT}
\thispagestyle{empty}
\bibbookdescr{Gen}{
  inline={\LARGE Первая книга Моисеева\\\Huge Бытие},
  toc={Бытие},
  bookmark={Бытие},
  header={Бытие},
  %headerleft={},
  %headerright={},
  abbr={Быт}
}
\vs Gen 1:1 В начале сотворил Бог небо и землю.
\vs Gen 1:2 Земля же была безвидна и пуста, и тьма над бездною, и Дух Божий носился над водою.
\rsbpar\vs Gen 1:3 И сказал Бог: да будет свет. И стал свет.
\vs Gen 1:4 И увидел Бог свет, что он хорош, и отделил Бог свет от тьмы.
\vs Gen 1:5 И назвал Бог свет днем, а тьму ночью. И был вечер, и было утро: день один.
\rsbpar\vs Gen 1:6 И сказал Бог: да будет твердь посреди воды, и да отделяет она воду от воды. [И стало так.]
\vs Gen 1:7 И создал Бог твердь, и отделил воду, которая под твердью, от воды, которая над твердью. И стало так.
\vs Gen 1:8 И назвал Бог твердь небом. [И увидел Бог, что \bibemph{это} хорошо.] И был вечер, и было утро: день второй.
\rsbpar\vs Gen 1:9 И сказал Бог: да соберется вода, которая под небом, в одно место, и да явится суша. И стало так. [И собралась вода под небом в свои места, и явилась суша.]
\vs Gen 1:10 И назвал Бог сушу землею, а собрание вод назвал морями. И увидел Бог, что \bibemph{это} хорошо.
\vs Gen 1:11 И сказал Бог: да произрастит земля зелень, траву, сеющую семя [по роду и по подобию \bibemph{ее}, и] дерево плодовитое, приносящее по роду своему плод, в котором семя его на земле. И стало так.
\vs Gen 1:12 И произвела земля зелень, траву, сеющую семя по роду [и по подобию] ее, и дерево [плодовитое], приносящее плод, в котором семя его по роду его [на земле]. И увидел Бог, что \bibemph{это} хорошо.
\vs Gen 1:13 И был вечер, и было утро: день третий.
\rsbpar\vs Gen 1:14 И сказал Бог: да будут светила на тверди небесной [для освещения земли и] для отделения дня от ночи, и для знамений, и времен, и дней, и годов;
\vs Gen 1:15 и да будут они светильниками на тверди небесной, чтобы светить на землю. И стало так.
\vs Gen 1:16 И создал Бог два светила великие: светило большее, для управления днем, и светило меньшее, для управления ночью, и звезды;
\vs Gen 1:17 и поставил их Бог на тверди небесной, чтобы светить на землю,
\vs Gen 1:18 и управлять днем и ночью, и отделять свет от тьмы. И увидел Бог, что \bibemph{это} хорошо.
\vs Gen 1:19 И был вечер, и было утро: день четвёртый.
\rsbpar\vs Gen 1:20 И сказал Бог: да произведет вода пресмыкающихся, душу живую; и птицы да полетят над землею, по тверди небесной. [И стало так.]
\vs Gen 1:21 И сотворил Бог рыб больших и всякую душу животных пресмыкающихся, которых произвела вода, по роду их, и всякую птицу пернатую по роду ее. И увидел Бог, что \bibemph{это} хорошо.
\vs Gen 1:22 И благословил их Бог, говоря: плодитесь и размножайтесь, и наполняйте воды в морях, и птицы да размножаются на земле.
\vs Gen 1:23 И был вечер, и было утро: день пятый.
\rsbpar\vs Gen 1:24 И сказал Бог: да произведет земля душу живую по роду ее, скотов, и гадов, и зверей земных по роду их. И стало так.
\vs Gen 1:25 И создал Бог зверей земных по роду их, и скот по роду его, и всех гадов земных по роду их. И увидел Бог, что \bibemph{это} хорошо.
\rsbpar\vs Gen 1:26 И сказал Бог: сотворим человека по образу Нашему [и] по подобию Нашему, и да владычествуют они над рыбами морскими, и над птицами небесными, [и над зверями,] и над скотом, и над всею землею, и над всеми гадами, пресмыкающимися по земле.
\vs Gen 1:27 И сотворил Бог человека по образу Своему, по образу Божию сотворил его; мужчину и женщину сотворил их.
\vs Gen 1:28 И благословил их Бог, и сказал им Бог: плодитесь и размножайтесь, и наполняйте землю, и обладайте ею, и владычествуйте над рыбами морскими [и над зверями,] и над птицами небесными, [и над всяким скотом, и над всею землею,] и над всяким животным, пресмыкающимся по земле.
\vs Gen 1:29 И сказал Бог: вот, Я дал вам всякую траву, сеющую семя, какая есть на всей земле, и всякое дерево, у которого плод древесный, сеющий семя;~--- вам \bibemph{сие} будет в пищу;
\vs Gen 1:30 а всем зверям земным, и всем птицам небесным, и всякому [гаду,] пресмыкающемуся по земле, в котором душа живая, \bibemph{дал} Я всю зелень травную в пищу. И стало так.
\vs Gen 1:31 И увидел Бог все, что Он создал, и вот, хорошо весьма. И был вечер, и было утро: день шестой.
\vs Gen 2:1 Так совершены небо и земля и все воинство их.
\vs Gen 2:2 И совершил Бог к седьмому дню дела Свои, которые Он делал, и почил в день седьмый от всех дел Своих, которые делал.
\vs Gen 2:3 И благословил Бог седьмой день, и освятил его, ибо в оный почил от всех дел Своих, которые Бог творил и созидал.
\rsbpar\vs Gen 2:4 Вот происхождение неба и земли, при сотворении их, в то время, когда Господь Бог создал землю и небо,
\vs Gen 2:5 и всякий полевой кустарник, которого еще не было на земле, и всякую полевую траву, которая еще не росла, ибо Господь Бог не посылал дождя на землю, и не было человека для возделывания земли,
\vs Gen 2:6 но пар поднимался с земли и орошал все лице земли.
\vs Gen 2:7 И создал Господь Бог человека из праха земного, и вдунул в лице его дыхание жизни, и стал человек душею живою.
\vs Gen 2:8 И насадил Господь Бог рай в Едеме на востоке, и поместил там человека, которого создал.
\vs Gen 2:9 И произрастил Господь Бог из земли всякое дерево, приятное на вид и хорошее для пищи, и дерево жизни посреди рая, и дерево познания добра и зла.
\vs Gen 2:10 Из Едема выходила река для орошения рая; и потом разделялась на четыре реки.
\vs Gen 2:11 Имя одной Фисон: она обтекает всю землю Хавила, ту, где золото;
\vs Gen 2:12 и золото той земли хорошее; там бдолах и камень оникс.
\vs Gen 2:13 Имя второй реки Гихон [Геон]: она обтекает всю землю Куш.
\vs Gen 2:14 Имя третьей реки Хиддекель [Тигр]: она протекает пред Ассириею. Четвертая река Евфрат.
\vs Gen 2:15 И взял Господь Бог человека, [которого создал,] и поселил его в саду Едемском, чтобы возделывать его и хранить его.
\vs Gen 2:16 И заповедал Господь Бог человеку, говоря: от всякого дерева в саду ты будешь есть,
\vs Gen 2:17 а от дерева познания добра и зла не ешь от него, ибо в день, в который ты вкусишь от него, смертью умрешь.
\vs Gen 2:18 И сказал Господь Бог: не хорошо быть человеку одному; сотворим ему помощника, соответственного ему.
\vs Gen 2:19 Господь Бог образовал из земли всех животных полевых и всех птиц небесных, и привел [их] к человеку, чтобы видеть, как он назовет их, и чтобы, как наречет человек всякую душу живую, так и было имя ей.
\vs Gen 2:20 И нарек человек имена всем скотам и птицам небесным и всем зверям полевым; но для человека не нашлось помощника, подобного ему.
\vs Gen 2:21 И навел Господь Бог на человека крепкий сон; и, когда он уснул, взял одно из ребр его, и закрыл то место плотию.
\vs Gen 2:22 И создал Господь Бог из ребра, взятого у человека, жену, и привел ее к человеку.
\vs Gen 2:23 И сказал человек: вот, это кость от костей моих и плоть от плоти моей; она будет называться женою, ибо взята от мужа [своего].
\vs Gen 2:24 Потому оставит человек отца своего и мать свою и прилепится к жене своей; и будут [два] одна плоть.
\vs Gen 2:25 И были оба наги, Адам и жена его, и не стыдились.
\vs Gen 3:1 Змей был хитрее всех зверей полевых, которых создал Господь Бог. И сказал змей жене: подлинно ли сказал Бог: не ешьте ни от какого дерева в раю?
\vs Gen 3:2 И сказала жена змею: плоды с дерев мы можем есть,
\vs Gen 3:3 только плодов дерева, которое среди рая, сказал Бог, не ешьте их и не прикасайтесь к ним, чтобы вам не умереть.
\vs Gen 3:4 И сказал змей жене: нет, не умрете,
\vs Gen 3:5 но знает Бог, что в день, в который вы вкусите их, откроются глаза ваши, и вы будете, как боги, знающие добро и зло.
\vs Gen 3:6 И увидела жена, что дерево хорошо для пищи, и что оно приятно для глаз и вожделенно, потому что дает знание; и взяла плодов его и ела; и дала также мужу своему, и он ел.
\vs Gen 3:7 И открылись глаза у них обоих, и узнали они, что наги, и сшили смоковные листья, и сделали себе опоясания.
\rsbpar\vs Gen 3:8 И услышали голос Господа Бога, ходящего в раю во время прохлады дня; и скрылся Адам и жена его от лица Господа Бога между деревьями рая.
\vs Gen 3:9 И воззвал Господь Бог к Адаму и сказал ему: [Адам,] где ты?
\vs Gen 3:10 Он сказал: голос Твой я услышал в раю, и убоялся, потому что я наг, и скрылся.
\vs Gen 3:11 И сказал [Бог]: кто сказал тебе, что ты наг? не ел ли ты от дерева, с которого Я запретил тебе есть?
\vs Gen 3:12 Адам сказал: жена, которую Ты мне дал, она дала мне от дерева, и я ел.
\vs Gen 3:13 И сказал Господь Бог жене: что ты это сделала? Жена сказала: змей обольстил меня, и я ела.
\rsbpar\vs Gen 3:14 И сказал Господь Бог змею: за то, что ты сделал это, проклят ты пред всеми скотами и пред всеми зверями полевыми; ты будешь ходить на чреве твоем, и будешь есть прах во все дни жизни твоей;
\vs Gen 3:15 и вражду положу между тобою и между женою, и между семенем твоим и между семенем ее; оно будет поражать тебя в голову, а ты будешь жалить его в пяту\fns{По другому чтению: и между Семенем ее; Он будет поражать тебя в голову, а ты будешь жалить Его в пяту.}.
\rsbpar\vs Gen 3:16 Жене сказал: умножая умножу скорбь твою в беременности твоей; в болезни будешь рождать детей; и к мужу твоему влечение твое, и он будет господствовать над тобою.
\vs Gen 3:17 Адаму же сказал: за то, что ты послушал голоса жены твоей и ел от дерева, о котором Я заповедал тебе, сказав: не ешь от него, проклята земля за тебя; со скорбью будешь питаться от нее во все дни жизни твоей;
\vs Gen 3:18 терния и волчцы произрастит она тебе; и будешь питаться полевою травою;
\vs Gen 3:19 в поте лица твоего будешь есть хлеб, доколе не возвратишься в землю, из которой ты взят, ибо прах ты и в прах возвратишься.
\vs Gen 3:20 И нарек Адам имя жене своей: Ева\fns{Жизнь.}, ибо она стала матерью всех живущих.
\rsbpar\vs Gen 3:21 И сделал Господь Бог Адаму и жене его одежды кожаные и одел их.
\rsbpar\vs Gen 3:22 И сказал Господь Бог: вот, Адам стал как один из Нас, зная добро и зло; и теперь как бы не простер он руки своей, и не взял также от дерева жизни, и не вкусил, и не стал жить вечно.
\vs Gen 3:23 И выслал его Господь Бог из сада Едемского, чтобы возделывать землю, из которой он взят.
\vs Gen 3:24 И изгнал Адама, и поставил на востоке у сада Едемского Херувима и пламенный меч обращающийся, чтобы охранять путь к дереву жизни.
\vs Gen 4:1 Адам познал Еву, жену свою; и она зачала, и родила Каина, и сказала: приобрела я человека от Господа.
\vs Gen 4:2 И еще родила брата его, Авеля. И был Авель пастырь овец, а Каин был земледелец.
\rsbpar\vs Gen 4:3 Спустя несколько времени, Каин принес от плодов земли дар Господу,
\vs Gen 4:4 и Авель также принес от первородных стада своего и от тука их. И призрел Господь на Авеля и на дар его,
\vs Gen 4:5 а на Каина и на дар его не призрел. Каин сильно огорчился, и поникло лице его.
\vs Gen 4:6 И сказал Господь [Бог] Каину: почему ты огорчился? и отчего поникло лице твое?
\vs Gen 4:7 если делаешь доброе, то не поднимаешь ли лица? а если не делаешь доброго, то у дверей грех лежит; он влечет тебя к себе, но ты господствуй над ним.
\vs Gen 4:8 И сказал Каин Авелю, брату своему: [пойдем в поле]. И когда они были в поле, восстал Каин на Авеля, брата своего, и убил его.
\rsbpar\vs Gen 4:9 И сказал Господь [Бог] Каину: где Авель, брат твой? Он сказал: не знаю; разве я сторож брату моему?
\vs Gen 4:10 И сказал [Господь]: что ты сделал? голос крови брата твоего вопиет ко Мне от земли;
\vs Gen 4:11 и ныне проклят ты от земли, которая отверзла уста свои принять кровь брата твоего от руки твоей;
\vs Gen 4:12 когда ты будешь возделывать землю, она не станет более давать силы своей для тебя; ты будешь изгнанником и скитальцем на земле.
\vs Gen 4:13 И сказал Каин Господу [Богу]: наказание мое больше, нежели снести можно;
\vs Gen 4:14 вот, Ты теперь сгоняешь меня с лица земли, и от лица Твоего я скроюсь, и буду изгнанником и скитальцем на земле; и всякий, кто встретится со мною, убьет меня.
\vs Gen 4:15 И сказал ему Господь [Бог]: за то всякому, кто убьет Каина, отмстится всемеро. И сделал Господь [Бог] Каину знамение, чтобы никто, встретившись с ним, не убил его.
\vs Gen 4:16 И пошел Каин от лица Господня и поселился в земле Нод, на восток от Едема.
\vs Gen 4:17 И познал Каин жену свою; и она зачала и родила Еноха. И построил он город; и назвал город по имени сына своего: Енох.
\vs Gen 4:18 У Еноха родился Ирад [Гаидад]; Ирад родил Мехиаеля [Малелеила]; Мехиаель родил Мафусала; Мафусал родил Ламеха.
\vs Gen 4:19 И взял себе Ламех две жены: имя одной: Ада, и имя второй: Цилла [Селла].
\vs Gen 4:20 Ада родила Иавала: он был отец живущих в шатрах со стадами.
\vs Gen 4:21 Имя брату его Иувал: он был отец всех играющих на гуслях и свирели.
\vs Gen 4:22 Цилла также родила Тувалкаина [Фовела], который был ковачом всех орудий из меди и железа. И сестра Тувалкаина Ноема.
\vs Gen 4:23 И сказал Ламех женам своим: Ада и Цилла! послушайте голоса моего; жены Ламеховы! внимайте словам моим: я убил мужа в язву мне и отрока в рану мне;
\vs Gen 4:24 если за Каина отмстится всемеро, то за Ламеха в семьдесят раз всемеро.
\rsbpar\vs Gen 4:25 И познал Адам еще [Еву,] жену свою, и она родила сына, и нарекла ему имя: Сиф, потому что, [говорила она,] Бог положил мне другое семя, вместо Авеля, которого убил Каин.
\vs Gen 4:26 У Сифа также родился сын, и он нарек ему имя: Енос; тогда начали призывать имя Господа [Бога].
\vs Gen 5:1 Вот родословие Адама: когда Бог сотворил человека, по подобию Божию создал его,
\vs Gen 5:2 мужчину и женщину сотворил их, и благословил их, и нарек им имя: человек, в день сотворения их.
\vs Gen 5:3 Адам жил сто тридцать [230] лет и родил [сына] по подобию своему [и] по образу своему, и нарек ему имя: Сиф.
\vs Gen 5:4 Дней Адама по рождении им Сифа было восемьсот [700] лет, и родил он сынов и дочерей.
\vs Gen 5:5 Всех же дней жизни Адамовой было девятьсот тридцать лет; и он умер.
\rsbpar\vs Gen 5:6 Сиф жил сто пять [205] лет и родил Еноса.
\vs Gen 5:7 По рождении Еноса Сиф жил восемьсот семь [707] лет и родил сынов и дочерей.
\vs Gen 5:8 Всех же дней Сифовых было девятьсот двенадцать лет; и он умер.
\rsbpar\vs Gen 5:9 Енос жил девяносто [190] лет и родил Каинана.
\vs Gen 5:10 По рождении Каинана Енос жил восемьсот пятнадцать [715] лет и родил сынов и дочерей.
\vs Gen 5:11 Всех же дней Еноса было девятьсот пять лет; и он умер.
\rsbpar\vs Gen 5:12 Каинан жил семьдесят [170] лет и родил Малелеила.
\vs Gen 5:13 По рождении Малелеила Каинан жил восемьсот сорок [740] лет и родил сынов и дочерей.
\vs Gen 5:14 Всех же дней Каинана было девятьсот десять лет; и он умер.
\rsbpar\vs Gen 5:15 Малелеил жил шестьдесят пять [165] лет и родил Иареда.
\vs Gen 5:16 По рождении Иареда Малелеил жил восемьсот тридцать [730] лет и родил сынов и дочерей.
\vs Gen 5:17 Всех же дней Малелеила было восемьсот девяносто пять лет; и он умер.
\rsbpar\vs Gen 5:18 Иаред жил сто шестьдесят два года и родил Еноха.
\vs Gen 5:19 По рождении Еноха Иаред жил восемьсот лет и родил сынов и дочерей.
\vs Gen 5:20 Всех же дней Иареда было девятьсот шестьдесят два года; и он умер.
\rsbpar\vs Gen 5:21 Енох жил шестьдесят пять [165] лет и родил Мафусала.
\vs Gen 5:22 И ходил Енох пред Богом, по рождении Мафусала, триста [200] лет и родил сынов и дочерей.
\vs Gen 5:23 Всех же дней Еноха было триста шестьдесят пять лет.
\vs Gen 5:24 И ходил Енох пред Богом; и не стало его, потому что Бог взял его.
\rsbpar\vs Gen 5:25 Мафусал жил сто восемьдесят семь лет и родил Ламеха.
\vs Gen 5:26 По рождении Ламеха Мафусал жил семьсот восемьдесят два года и родил сынов и дочерей.
\vs Gen 5:27 Всех же дней Мафусала было девятьсот шестьдесят девять лет; и он умер.
\rsbpar\vs Gen 5:28 Ламех жил сто восемьдесят два [188] года и родил сына,
\vs Gen 5:29 и нарек ему имя: Ной, сказав: он утешит нас в работе нашей и в трудах рук наших при \bibemph{возделывании} земли, которую проклял Господь [Бог].
\vs Gen 5:30 И жил Ламех по рождении Ноя пятьсот девяносто пять [565] лет и родил сынов и дочерей.
\vs Gen 5:31 Всех же дней Ламеха было семьсот семьдесят семь [753] лет; и он умер.
\rsbpar\vs Gen 5:32 Ною было пятьсот лет и родил Ной [трех сынов]: Сима, Хама и Иафета.
\vs Gen 6:1 Когда люди начали умножаться на земле и родились у них дочери,
\vs Gen 6:2 тогда сыны Божии увидели дочерей человеческих, что они красивы, и брали \bibemph{их} себе в жены, какую кто избрал.
\vs Gen 6:3 И сказал Господь [Бог]: не вечно Духу Моему быть пренебрегаемым человеками [сими], потому что они плоть; пусть будут дни их сто двадцать лет.
\vs Gen 6:4 В то время были на земле исполины, особенно же с того времени, как сыны Божии стали входить к дочерям человеческим, и они стали рождать им: это сильные, издревле славные люди.
\rsbpar\vs Gen 6:5 И увидел Господь [Бог], что велико развращение человеков на земле, и что все мысли и помышления сердца их были зло во всякое время;
\vs Gen 6:6 и раскаялся Господь, что создал человека на земле, и восскорбел в сердце Своем.
\vs Gen 6:7 И сказал Господь: истреблю с лица земли человеков, которых Я сотворил, от человека до скотов, и гадов и птиц небесных истреблю, ибо Я раскаялся, что создал их.
\rsbpar\vs Gen 6:8 Ной же обрел благодать пред очами Господа [Бога].
\rsbpar\vs Gen 6:9 Вот житие Ноя: Ной был человек праведный и непорочный в роде своем; Ной ходил пред Богом.
\vs Gen 6:10 Ной родил трех сынов: Сима, Хама и Иафета.
\vs Gen 6:11 Но земля растлилась пред лицем Божиим, и наполнилась земля злодеяниями.
\vs Gen 6:12 И воззрел [Господь] Бог на землю, и вот, она растленна, ибо всякая плоть извратила путь свой на земле.
\vs Gen 6:13 И сказал [Господь] Бог Ною: конец всякой плоти пришел пред лице Мое, ибо земля наполнилась от них злодеяниями; и вот, Я истреблю их с земли.
\vs Gen 6:14 Сделай себе ковчег из дерева гофер; отделения сделай в ковчеге и осмоли его смолою внутри и снаружи.
\vs Gen 6:15 И сделай его так: длина ковчега триста локтей; ширина его пятьдесят локтей, а высота его тридцать локтей.
\vs Gen 6:16 И сделай отверстие в ковчеге, и в локоть сведи его вверху, и дверь в ковчег сделай с боку его; устрой в нем нижнее, второе и третье [жилье].
\vs Gen 6:17 И вот, Я наведу на землю потоп водный, чтоб истребить всякую плоть, в которой есть дух жизни, под небесами; все, что есть на земле, лишится жизни.
\vs Gen 6:18 Но с тобою Я поставлю завет Мой, и войдешь в ковчег ты, и сыновья твои, и жена твоя, и жены сынов твоих с тобою.
\vs Gen 6:19 Введи также в ковчег [из всякого скота, и из всех гадов, и] из всех животных, и от всякой плоти по паре, чтоб они остались с тобою в живых; мужеского пола и женского пусть они будут.
\vs Gen 6:20 Из [всех] птиц по роду их, и из [всех] скотов по роду их, и из всех пресмыкающихся по земле по роду их, из всех по паре войдут к тебе, чтобы остались в живых [с тобою, мужеского пола и женского].
\vs Gen 6:21 Ты же возьми себе всякой пищи, какою питаются, и собери к себе; и будет она для тебя и для них пищею.
\vs Gen 6:22 И сделал Ной всё: как повелел ему [Господь] Бог, так он и сделал.
\vs Gen 7:1 И сказал Господь [Бог] Ною: войди ты и все семейство твое в ковчег, ибо тебя увидел Я праведным предо Мною в роде сем;
\vs Gen 7:2 и всякого скота чистого возьми по семи, мужеского пола и женского, а из скота нечистого по два, мужеского пола и женского;
\vs Gen 7:3 также и из птиц небесных [чистых] по семи, мужеского пола и женского, [и из всех птиц нечистых по две, мужеского пола и женского,] чтобы сохранить племя для всей земли,
\vs Gen 7:4 ибо чрез семь дней Я буду изливать дождь на землю сорок дней и сорок ночей; и истреблю все существующее, что Я создал, с лица земли.
\vs Gen 7:5 Ной сделал все, что Господь [Бог] повелел ему.
\rsbpar\vs Gen 7:6 Ной же был шестисот лет, как потоп водный пришел на землю.
\vs Gen 7:7 И вошел Ной и сыновья его, и жена его, и жены сынов его с ним в ковчег от вод потопа.
\vs Gen 7:8 И [из птиц чистых и из птиц нечистых, и] из скотов чистых и из скотов нечистых, [и из зверей] и из всех пресмыкающихся по земле
\vs Gen 7:9 по паре, мужеского пола и женского, вошли к Ною в ковчег, как [Господь] Бог повелел Ною.
\rsbpar\vs Gen 7:10 Чрез семь дней воды потопа пришли на землю.
\vs Gen 7:11 В шестисотый год жизни Ноевой, во второй месяц, в семнадцатый [27] день месяца, в сей день разверзлись все источники великой бездны, и окна небесные отворились;
\vs Gen 7:12 и лился на землю дождь сорок дней и сорок ночей.
\vs Gen 7:13 В сей самый день вошел в ковчег Ной, и Сим, Хам и Иафет, сыновья Ноевы, и жена Ноева, и три жены сынов его с ними.
\vs Gen 7:14 Они, и все звери [земли] по роду их, и всякий скот по роду его, и все гады, пресмыкающиеся по земле, по роду их, и все летающие по роду их, все птицы, все крылатые,
\vs Gen 7:15 и вошли к Ною в ковчег по паре [мужеского пола и женского] от всякой плоти, в которой есть дух жизни;
\vs Gen 7:16 и вошедшие [к Ною в ковчег] мужеский и женский пол всякой плоти вошли, как повелел ему [Господь] Бог. И затворил Господь [Бог] за ним [ковчег].
\vs Gen 7:17 И продолжалось на земле наводнение сорок дней [и сорок ночей], и умножилась вода, и подняла ковчег, и он возвысился над землею;
\vs Gen 7:18 вода же усиливалась и весьма умножалась на земле, и ковчег плавал по поверхности вод.
\vs Gen 7:19 И усилилась вода на земле чрезвычайно, так что покрылись все высокие горы, какие есть под всем небом;
\vs Gen 7:20 на пятнадцать локтей поднялась над ними вода, и покрылись [все высокие] горы.
\vs Gen 7:21 И лишилась жизни всякая плоть, движущаяся по земле, и птицы, и скоты, и звери, и все гады, ползающие по земле, и все люди;
\vs Gen 7:22 все, что имело дыхание духа жизни в ноздрях своих на суше, умерло.
\vs Gen 7:23 Истребилось всякое существо, которое было на поверхности [всей] земли; от человека до скота, и гадов, и птиц небесных,~--- все истребилось с земли, остался только Ной и что \bibemph{было} с ним в ковчеге.
\vs Gen 7:24 Вода же усиливалась на земле сто пятьдесят дней.
\vs Gen 8:1 И вспомнил Бог о Ное, и о всех зверях, и о всех скотах, [и о всех птицах, и о всех гадах пресмыкающихся,] бывших с ним в ковчеге; и навел Бог ветер на землю, и воды остановились.
\vs Gen 8:2 И закрылись источники бездны и окна небесные, и перестал дождь с неба.
\vs Gen 8:3 Вода же постепенно возвращалась с земли, и стала убывать вода по окончании ста пятидесяти дней.
\vs Gen 8:4 И остановился ковчег в седьмом месяце, в семнадцатый день месяца, на горах Араратских.
\vs Gen 8:5 Вода постоянно убывала до десятого месяца; в первый день десятого месяца показались верхи гор.
\rsbpar\vs Gen 8:6 По прошествии сорока дней Ной открыл сделанное им окно ковчега
\vs Gen 8:7 и выпустил ворона, [чтобы видеть, убыла ли вода с земли,] который, вылетев, отлетал и прилетал, пока осушилась земля от воды.
\vs Gen 8:8 Потом выпустил от себя голубя, чтобы видеть, сошла ли вода с лица земли,
\vs Gen 8:9 но голубь не нашел места покоя для ног своих и возвратился к нему в ковчег, ибо вода была еще на поверхности всей земли; и он простер руку свою, и взял его, и принял к себе в ковчег.
\vs Gen 8:10 И помедлил еще семь дней других и опять выпустил голубя из ковчега.
\vs Gen 8:11 Голубь возвратился к нему в вечернее время, и вот, свежий масличный лист во рту у него, и Ной узнал, что вода сошла с земли.
\vs Gen 8:12 Он помедлил еще семь дней других и [опять] выпустил голубя; и он уже не возвратился к нему.
\rsbpar\vs Gen 8:13 Шестьсот первого года [жизни Ноевой] к первому [дню] первого месяца иссякла вода на земле; и открыл Ной кровлю ковчега и посмотрел, и вот, обсохла поверхность земли.
\vs Gen 8:14 И во втором месяце, к двадцать седьмому дню месяца, земля высохла.
\rsbpar\vs Gen 8:15 И сказал [Господь] Бог Ною:
\vs Gen 8:16 выйди из ковчега ты и жена твоя, и сыновья твои, и жены сынов твоих с тобою;
\vs Gen 8:17 выведи с собою всех животных, которые с тобою, от всякой плоти, из птиц, и скотов, и всех гадов, пресмыкающихся по земле: пусть разойдутся они по земле, и пусть плодятся и размножаются на земле.
\vs Gen 8:18 И вышел Ной и сыновья его, и жена его, и жены сынов его с ним;
\vs Gen 8:19 все звери, и [весь скот, и] все гады, и все птицы, все движущееся по земле, по родам своим, вышли из ковчега.
\rsbpar\vs Gen 8:20 И устроил Ной жертвенник Господу; и взял из всякого скота чистого и из всех птиц чистых и принес во всесожжение на жертвеннике.
\vs Gen 8:21 И обонял Господь приятное благоухание, и сказал Господь [Бог] в сердце Своем: не буду больше проклинать землю за человека, потому что помышление сердца человеческого~--- зло от юности его; и не буду больше поражать всего живущего, как Я сделал:
\vs Gen 8:22 впредь во все дни земли сеяние и жатва, холод и зной, лето и зима, день и ночь не прекратятся.
\vs Gen 9:1 И благословил Бог Ноя и сынов его и сказал им: плодитесь и размножайтесь, и наполняйте землю [и обладайте ею];
\vs Gen 9:2 да страшатся и да трепещут вас все звери земные, [и весь скот земной,] и все птицы небесные, все, что движется на земле, и все рыбы морские: в ваши руки отданы они;
\vs Gen 9:3 все движущееся, что живет, будет вам в пищу; как зелень травную даю вам все;
\vs Gen 9:4 только плоти с душею ее, с кровью ее, не ешьте;
\vs Gen 9:5 Я взыщу и вашу кровь, \bibemph{в которой} жизнь ваша, взыщу ее от всякого зверя, взыщу также душу человека от руки человека, от руки брата его;
\vs Gen 9:6 кто прольет кровь человеческую, того кровь прольется рукою человека: ибо человек создан по образу Божию;
\vs Gen 9:7 вы же плодитесь и размножайтесь, и распространяйтесь по земле, и умножайтесь на ней.
\rsbpar\vs Gen 9:8 И сказал Бог Ною и сынам его с ним:
\vs Gen 9:9 вот, Я поставляю завет Мой с вами и с потомством вашим после вас,
\vs Gen 9:10 и со всякою душею живою, которая с вами, с птицами и со скотами, и со всеми зверями земными, которые у вас, со всеми вышедшими из ковчега, со всеми животными земными;
\vs Gen 9:11 поставляю завет Мой с вами, что не будет более истреблена всякая плоть водами потопа, и не будет уже потопа на опустошение земли.
\vs Gen 9:12 И сказал [Господь] Бог: вот знамение завета, который Я поставляю между Мною и между вами и между всякою душею живою, которая с вами, в роды навсегда:
\vs Gen 9:13 Я полагаю радугу Мою в облаке, чтоб она была знамением [вечного] завета между Мною и между землею.
\vs Gen 9:14 И будет, когда Я наведу облако на землю, то явится радуга [Моя] в облаке;
\vs Gen 9:15 и Я вспомню завет Мой, который между Мною и между вами и между всякою душею живою во всякой плоти; и не будет более вода потопом на истребление всякой плоти.
\vs Gen 9:16 И будет радуга [Моя] в облаке, и Я увижу ее, и вспомню завет вечный между Богом [и между землею] и между всякою душею живою во всякой плоти, которая на земле.
\vs Gen 9:17 И сказал Бог Ною: вот знамение завета, который Я поставил между Мною и между всякою плотью, которая на земле.
\rsbpar\vs Gen 9:18 Сыновья Ноя, вышедшие из ковчега, были: Сим, Хам и Иафет. Хам же был отец Ханаана.
\vs Gen 9:19 Сии трое были сыновья Ноевы, и от них населилась вся земля.
\rsbpar\vs Gen 9:20 Ной начал возделывать землю и насадил виноградник;
\vs Gen 9:21 и выпил он вина, и опьянел, и \bibemph{лежал} обнаженным в шатре своем.
\vs Gen 9:22 И увидел Хам, отец Ханаана, наготу отца своего, и выйдя рассказал двум братьям своим.
\vs Gen 9:23 Сим же и Иафет взяли одежду и, положив ее на плечи свои, пошли задом и покрыли наготу отца своего; лица их были обращены назад, и они не видали наготы отца своего.
\vs Gen 9:24 Ной проспался от вина своего и узнал, что сделал над ним меньший сын его,
\vs Gen 9:25 и сказал: проклят Ханаан; раб рабов будет он у братьев своих.
\vs Gen 9:26 Потом сказал: благословен Господь Бог Симов; Ханаан же будет рабом ему;
\vs Gen 9:27 да распространит Бог Иафета, и да вселится он в шатрах Симовых; Ханаан же будет рабом ему.
\rsbpar\vs Gen 9:28 И жил Ной после потопа триста пятьдесят лет.
\vs Gen 9:29 Всех же дней Ноевых было девятьсот пятьдесят лет, и он умер.
\vs Gen 10:1 Вот родословие сынов Ноевых: Сима, Хама и Иафета. После потопа родились у них дети.
\rsbpar\vs Gen 10:2 Сыны Иафета: Гомер, Магог, Мадай, Иаван, [Елиса,] Фувал, Мешех и Фирас.
\vs Gen 10:3 Сыны Гомера: Аскеназ, Рифат и Фогарма.
\vs Gen 10:4 Сыны Иавана: Елиса, Фарсис, Киттим и Доданим.
\vs Gen 10:5 От сих населились острова народов в землях их, каждый по языку своему, по племенам своим, в народах своих.
\rsbpar\vs Gen 10:6 Сыны Хама: Хуш, Мицраим, Фут и Ханаан.
\vs Gen 10:7 Сыны Хуша: Сева, Хавила, Савта, Раама и Савтеха. Сыны Раамы: Шева и Дедан.
\vs Gen 10:8 Хуш родил также Нимрода; сей начал быть силен на земле;
\vs Gen 10:9 он был сильный зверолов пред Господом [Богом], потому и говорится: сильный зверолов, как Нимрод, пред Господом [Богом].
\vs Gen 10:10 Царство его вначале \bibemph{составляли}: Вавилон, Эрех, Аккад и Халне в земле Сеннаар.
\vs Gen 10:11 Из сей земли вышел Ассур и построил Ниневию, Реховоф-ир, Калах
\vs Gen 10:12 и Ресен между Ниневиею и между Калахом; это город великий.
\vs Gen 10:13 От Мицраима произошли Лудим, Анамим, Легавим, Нафтухим,
\vs Gen 10:14 Патрусим, Каслухим, откуда вышли Филистимляне, и Кафторим.
\vs Gen 10:15 От Ханаана родились: Сидон, первенец его, Хет,
\vs Gen 10:16 Иевусей, Аморрей, Гергесей,
\vs Gen 10:17 Евей, Аркей, Синей,
\vs Gen 10:18 Арвадей, Цемарей и Химафей. Впоследствии племена Ханаанские рассеялись,
\vs Gen 10:19 и были пределы Хананеев от Сидона к Герару до Газы, отсюда к Содому, Гоморре, Адме и Цевоиму до Лаши.
\vs Gen 10:20 Это сыны Хамовы, по племенам их, по языкам их, в землях их, в народах их.
\rsbpar\vs Gen 10:21 Были дети и у Сима, отца всех сынов Еверовых, старшего брата Иафетова.
\vs Gen 10:22 Сыны Сима: Елам, Ассур, Арфаксад, Луд, Арам [и Каинан].
\vs Gen 10:23 Сыны Арама: Уц, Хул, Гефер и Маш.
\vs Gen 10:24 Арфаксад родил [Каинана, Каинан родил] Салу, Сала родил Евера.
\vs Gen 10:25 У Евера родились два сына; имя одному: Фалек, потому что во дни его земля разделена; имя брату его: Иоктан.
\vs Gen 10:26 Иоктан родил Алмодада, Шалефа, Хацармавефа, Иераха,
\vs Gen 10:27 Гадорама, Узала, Диклу,
\vs Gen 10:28 Овала, Авимаила, Шеву,
\vs Gen 10:29 Офира, Хавилу и Иовава. Все эти сыновья Иоктана.
\vs Gen 10:30 Поселения их были от Меши до Сефара, горы восточной.
\vs Gen 10:31 Это сыновья Симовы по племенам их, по языкам их, в землях их, по народам их.
\vs Gen 10:32 Вот племена сынов Ноевых, по родословию их, в народах их. От них распространились народы на земле после потопа.
\vs Gen 11:1 На всей земле был один язык и одно наречие.
\vs Gen 11:2 Двинувшись с востока, они нашли в земле Сеннаар равнину и поселились там.
\vs Gen 11:3 И сказали друг другу: наделаем кирпичей и обожжем огнем. И стали у них кирпичи вместо камней, а земляная смола вместо извести.
\vs Gen 11:4 И сказали они: построим себе город и башню, высотою до небес, и сделаем себе имя, прежде нежели рассеемся по лицу всей земли.
\vs Gen 11:5 И сошел Господь посмотреть город и башню, которые строили сыны человеческие.
\vs Gen 11:6 И сказал Господь: вот, один народ, и один у всех язык; и вот что начали они делать, и не отстанут они от того, что задумали делать;
\vs Gen 11:7 сойдем же и смешаем там язык их, так чтобы один не понимал речи другого.
\vs Gen 11:8 И рассеял их Господь оттуда по всей земле; и они перестали строить город [и башню].
\vs Gen 11:9 Посему дано ему имя: Вавилон, ибо там смешал Господь язык всей земли, и оттуда рассеял их Господь по всей земле.
\rsbpar\vs Gen 11:10 Вот родословие Сима: Сим был ста лет и родил Арфаксада, чрез два года после потопа;
\vs Gen 11:11 по рождении Арфаксада Сим жил пятьсот лет и родил сынов и дочерей [и умер].
\vs Gen 11:12 Арфаксад жил тридцать пять [135] лет и родил [Каинана. По рождении Каинана Арфаксад жил триста тридцать лет и родил сынов и дочерей и умер. Каинан жил сто тридцать лет, и родил] Салу.
\vs Gen 11:13 По рождении Салы Арфаксад [Каинан] жил четыреста три [330] года и родил сынов и дочерей [и умер].
\vs Gen 11:14 Сала жил тридцать [130] лет и родил Евера.
\vs Gen 11:15 По рождении Евера Сала жил четыреста три [330] года и родил сынов и дочерей [и умер].
\vs Gen 11:16 Евер жил тридцать четыре [134] года и родил Фалека.
\vs Gen 11:17 По рождении Фалека Евер жил четыреста тридцать [370] лет и родил сынов и дочерей [и умер].
\vs Gen 11:18 Фалек жил тридцать [130] лет и родил Рагава.
\vs Gen 11:19 По рождении Рагава Фалек жил двести девять лет и родил сынов и дочерей [и умер].
\vs Gen 11:20 Рагав жил тридцать два [132] года и родил Серуха.
\vs Gen 11:21 По рождении Серуха Рагав жил двести семь лет и родил сынов и дочерей [и умер].
\vs Gen 11:22 Серух жил тридцать [130] лет и родил Нахора.
\vs Gen 11:23 По рождении Нахора Серух жил двести лет и родил сынов и дочерей [и умер].
\vs Gen 11:24 Нахор жил двадцать девять [79] лет и родил Фарру.
\vs Gen 11:25 По рождении Фарры Нахор жил сто девятнадцать [129] лет и родил сынов и дочерей [и умер].
\vs Gen 11:26 Фарра жил семьдесят лет и родил Аврама, Нахора и Арана.
\rsbpar\vs Gen 11:27 Вот родословие Фарры: Фарра родил Аврама, Нахора и Арана. Аран родил Лота.
\vs Gen 11:28 И умер Аран при Фарре, отце своем, в земле рождения своего, в Уре Халдейском.
\vs Gen 11:29 Аврам и Нахор взяли себе жен; имя жены Аврамовой: Сара; имя жены Нахоровой: Милка, дочь Арана, отца Милки и отца Иски.
\vs Gen 11:30 И Сара была неплодна и бездетна.
\vs Gen 11:31 И взял Фарра Аврама, сына своего, и Лота, сына Аранова, внука своего, и Сару, невестку свою, жену Аврама, сына своего, и вышел с ними из Ура Халдейского, чтобы идти в землю Ханаанскую; но, дойдя до Харрана, они остановились там.
\vs Gen 11:32 И было дней \bibemph{жизни} Фарры [в Харранской земле] двести пять лет, и умер Фарра в Харране.
\vs Gen 12:1 И сказал Господь Авраму: пойди из земли твоей, от родства твоего и из дома отца твоего [и иди] в землю, которую Я укажу тебе;
\vs Gen 12:2 и Я произведу от тебя великий народ, и благословлю тебя, и возвеличу имя твое, и будешь ты в благословение;
\vs Gen 12:3 Я благословлю благословляющих тебя, и злословящих тебя прокляну; и благословятся в тебе все племена земные.
\rsbpar\vs Gen 12:4 И пошел Аврам, как сказал ему Господь; и с ним пошел Лот. Аврам был семидесяти пяти лет, когда вышел из Харрана.
\vs Gen 12:5 И взял Аврам с собою Сару, жену свою, Лота, сына брата своего, и все имение, которое они приобрели, и всех людей, которых они имели в Харране; и вышли, чтобы идти в землю Ханаанскую; и пришли в землю Ханаанскую.
\vs Gen 12:6 И прошел Аврам по земле сей [по длине ее] до места Сихема, до дубравы Мор\acc{е}. В этой земле тогда [жили] Хананеи.
\vs Gen 12:7 И явился Господь Авраму и сказал [ему]: потомству твоему отдам Я землю сию. И создал там [Аврам] жертвенник Господу, Который явился ему.
\vs Gen 12:8 Оттуда двинулся он к горе, на восток от Вефиля; и поставил шатер свой \bibemph{так, что от него} Вефиль \bibemph{был} на запад, а Гай на восток; и создал там жертвенник Господу и призвал имя Господа [явившегося ему].
\vs Gen 12:9 И поднялся Аврам и продолжал идти к югу.
\rsbpar\vs Gen 12:10 И был голод в той земле. И сошел Аврам в Египет, пожить там, потому что усилился голод в земле той.
\vs Gen 12:11 Когда же он приближался к Египту, то сказал Саре, жене своей: вот, я знаю, что ты женщина, прекрасная видом;
\vs Gen 12:12 и когда Египтяне увидят тебя, то скажут: это жена его; и убьют меня, а тебя оставят в живых;
\vs Gen 12:13 скажи же, что ты мне сестра, дабы мне хорошо было ради тебя, и дабы жива была душа моя чрез тебя.
\vs Gen 12:14 И было, когда пришел Аврам в Египет, Египтяне увидели, что она женщина весьма красивая;
\vs Gen 12:15 увидели ее и вельможи фараоновы и похвалили ее фараону; и взята была она в дом фараонов.
\vs Gen 12:16 И Авраму хорошо было ради ее; и был у него мелкий и крупный скот и ослы, и рабы и рабыни, и лошаки и верблюды.
\vs Gen 12:17 Но Господь поразил тяжкими ударами фараона и дом его за Сару, жену Аврамову.
\vs Gen 12:18 И призвал фараон Аврама и сказал: что ты это сделал со мною? для чего не сказал мне, что она жена твоя?
\vs Gen 12:19 для чего ты сказал: она сестра моя? и я взял было ее себе в жену. И теперь вот жена твоя; возьми [ее] и пойди.
\vs Gen 12:20 И дал о нем фараон повеление людям, и проводили его, и жену его, и все, что у него было, [и Лота с ним].
\vs Gen 13:1 И поднялся Аврам из Египта, сам и жена его, и всё, что у него было, и Лот с ним, на юг.
\vs Gen 13:2 И был Аврам очень богат скотом, и серебром, и золотом.
\vs Gen 13:3 И продолжал он переходы свои от юга до Вефиля, до места, где прежде был шатер его между Вефилем и между Гаем,
\vs Gen 13:4 до места жертвенника, который он сделал там вначале; и там призвал Аврам имя Господа.
\rsbpar\vs Gen 13:5 И у Лота, который ходил с Аврамом, также был мелкий и крупный скот и шатры.
\vs Gen 13:6 И непоместительна была земля для них, чтобы жить вместе, ибо имущество их было так велико, что они не могли жить вместе.
\vs Gen 13:7 И был спор между пастухами скота Аврамова и между пастухами скота Лотова; и Хананеи и Ферезеи жили тогда в той земле.
\vs Gen 13:8 И сказал Аврам Лоту: да не будет раздора между мною и тобою, и между пастухами моими и пастухами твоими, ибо мы родственники;
\vs Gen 13:9 не вся ли земля пред тобою? отделись же от меня: если ты налево, то я направо; а если ты направо, то я налево.
\vs Gen 13:10 Лот возвел очи свои и увидел всю окрестность Иорданскую, что она, прежде нежели истребил Господь Содом и Гоморру, вся до Сигора орошалась водою, как сад Господень, как земля Египетская;
\vs Gen 13:11 и избрал себе Лот всю окрестность Иорданскую; и двинулся Лот к востоку. И отделились они друг от друга.
\vs Gen 13:12 Аврам стал жить на земле Ханаанской; а Лот стал жить в городах окрестности и раскинул шатры до Содома.
\vs Gen 13:13 Жители же Содомские были злы и весьма грешны пред Господом.
\rsbpar\vs Gen 13:14 И сказал Господь Авраму, после того как Лот отделился от него: возведи очи твои и с места, на котором ты теперь, посмотри к северу и к югу, и к востоку и к западу;
\vs Gen 13:15 ибо всю землю, которую ты видишь, тебе дам Я и потомству твоему навеки,
\vs Gen 13:16 и сделаю потомство твое, как песок земной; если кто может сосчитать песок земной, то и потомство твое сочтено будет;
\vs Gen 13:17 встань, пройди по земле сей в долготу и в широту ее, ибо Я тебе дам ее [и потомству твоему навсегда].
\vs Gen 13:18 И двинул Аврам шатер, и пошел, и поселился у дубравы Мамре, что в Хевроне; и создал там жертвенник Господу.
\vs Gen 14:1 И было во дни Амрафела, царя Сеннаарского, Ариоха, царя Елласарского, Кедорлаомера, царя Еламского, и Фидала, царя Гоимского,
\vs Gen 14:2 пошли они войною против Беры, царя Содомского, против Бирши, царя Гоморрского, Шинава, царя Адмы, Шемевера, царя Севоимского, и против царя Белы, которая есть Сигор.
\vs Gen 14:3 Все сии соединились в долине Сиддим, где \bibemph{ныне} море Соленое.
\vs Gen 14:4 Двенадцать лет были они в порабощении у Кедорлаомера, а в тринадцатом году возмутились.
\vs Gen 14:5 В четырнадцатом году пришел Кедорлаомер и цари, которые с ним, и поразили Рефаимов в Аштероф-Карнаиме, Зузимов в Гаме, Эмимов в Шаве-Кириафаиме,
\vs Gen 14:6 и Хорреев в горе их Сеире, до Эл-Фарана, что при пустыне.
\vs Gen 14:7 И возвратившись оттуда, они пришли к источнику Мишпат, который есть Кадес, и поразили всю страну Амаликитян, и также Аморреев, живущих в Хацацон-Фамаре.
\vs Gen 14:8 И вышли царь Содомский, царь Гоморрский, царь Адмы, царь Севоимский и царь Белы, которая есть Сигор; и вступили в сражение с ними в долине Сиддим,
\vs Gen 14:9 с Кедорлаомером, царем Еламским, Фидалом, царем Гоимским, Амрафелом, царем Сеннаарским, Ариохом, царем Елласарским,~--- четыре царя против пяти.
\vs Gen 14:10 В долине же Сиддим было много смоляных ям. И цари Содомский и Гоморрский, обратившись в бегство, упали в них, а остальные убежали в горы.
\vs Gen 14:11 \bibemph{Победители} взяли все имущество Содома и Гоморры и весь запас их и ушли.
\vs Gen 14:12 И взяли Лота, племянника Аврамова, жившего в Содоме, и имущество его и ушли.
\rsbpar\vs Gen 14:13 И пришел один из уцелевших и известил Аврама Еврея, жившего тогда у дубравы Мамре, Аморреянина, брата Эшколу и брата Анеру, которые были союзники Аврамовы.
\vs Gen 14:14 Аврам, услышав, что [Лот] сродник его взят в плен, вооружил рабов своих, рожденных в доме его, триста восемнадцать, и преследовал \bibemph{неприятелей} до Дана;
\vs Gen 14:15 и, разделившись, \bibemph{напал} на них ночью, сам и рабы его, и поразил их, и преследовал их до Ховы, что по левую сторону Дамаска;
\vs Gen 14:16 и возвратил все имущество и Лота, сродника своего, и имущество его возвратил, также и женщин и народ.
\rsbpar\vs Gen 14:17 Когда он возвращался после поражения Кедорлаомера и царей, бывших с ним, царь Содомский вышел ему навстречу в долину Шаве, что \bibemph{ныне} долина царская;
\vs Gen 14:18 и Мелхиседек, царь Салимский, вынес хлеб и вино,~--- он был священник Бога Всевышнего,~---
\vs Gen 14:19 и благословил его, и сказал: благословен Аврам от Бога Всевышнего, Владыки неба и земли;
\vs Gen 14:20 и благословен Бог Всевышний, Который предал врагов твоих в руки твои. [Аврам] дал ему десятую часть из всего.
\vs Gen 14:21 И сказал царь Содомский Авраму: отдай мне людей, а имение возьми себе.
\vs Gen 14:22 Но Аврам сказал царю Содомскому: поднимаю руку мою к Господу Богу Всевышнему, Владыке неба и земли,
\vs Gen 14:23 что даже нитки и ремня от обуви не возьму из всего твоего, чтобы ты не сказал: я обогатил Аврама;
\vs Gen 14:24 кроме того, что съели отроки, и кроме доли, принадлежащей людям, которые ходили со мною; Анер, Эшкол и Мамрий пусть возьмут свою долю.
\vs Gen 15:1 После сих происшествий было слово Господа к Авраму в видении [ночью], и сказано: не бойся, Аврам; Я твой щит; награда твоя [будет] весьма велика.
\vs Gen 15:2 Аврам сказал: Владыка Господи! что Ты дашь мне? я остаюсь бездетным; распорядитель в доме моем этот Елиезер из Дамаска.
\vs Gen 15:3 И сказал Аврам: вот, Ты не дал мне потомства, и вот, домочадец мой наследник мой.
\vs Gen 15:4 И было слово Господа к нему, и сказано: не будет он твоим наследником, но тот, кто произойдет из чресл твоих, будет твоим наследником.
\vs Gen 15:5 И вывел его вон и сказал [ему]: посмотри на небо и сосчитай звезды, если ты можешь счесть их. И сказал ему: столько будет у тебя потомков.
\vs Gen 15:6 Аврам поверил Господу, и Он вменил ему это в праведность.
\rsbpar\vs Gen 15:7 И сказал ему: Я Господь, Который вывел тебя из Ура Халдейского, чтобы дать тебе землю сию во владение.
\vs Gen 15:8 Он сказал: Владыка Господи! по чему мне узнать, что я буду владеть ею?
\vs Gen 15:9 \bibemph{Господь} сказал ему: возьми Мне трехлетнюю телицу, трехлетнюю козу, трехлетнего овна, горлицу и молодого голубя.
\vs Gen 15:10 Он взял всех их, рассек их пополам и положил одну часть против другой; только птиц не рассек.
\vs Gen 15:11 И налетели на трупы хищные птицы; но Аврам отгонял их.
\rsbpar\vs Gen 15:12 При захождении солнца крепкий сон напал на Аврама, и вот, напал на него ужас и мрак великий.
\vs Gen 15:13 И сказал \bibemph{Господь} Авраму: знай, что потомки твои будут пришельцами в земле не своей, и поработят их, и будут угнетать их четыреста лет,
\vs Gen 15:14 но Я произведу суд над народом, у которого они будут в порабощении; после сего они выйдут [сюда] с большим имуществом,
\vs Gen 15:15 а ты отойдешь к отцам твоим в мире \bibemph{и} будешь погребен в старости доброй;
\vs Gen 15:16 в четвертом роде возвратятся они сюда: ибо \bibemph{мера} беззаконий Аморреев доселе еще не наполнилась.
\vs Gen 15:17 Когда зашло солнце и наступила тьма, вот, дым \bibemph{как бы из} печи и пламя огня прошли между рассеченными \bibemph{животными}.
\vs Gen 15:18 В этот день заключил Господь завет с Аврамом, сказав: потомству твоему даю Я землю сию, от реки Египетской до великой реки, реки Евфрата:
\vs Gen 15:19 Кенеев, Кенезеев, Кедмонеев,
\vs Gen 15:20 Хеттеев, Ферезеев, Рефаимов,
\vs Gen 15:21 Аморреев, Хананеев, [Евеев,] Гергесеев и Иевусеев.
\vs Gen 16:1 Но Сара, жена Аврамова, не рождала ему. У ней была служанка Египтянка, именем Агарь.
\vs Gen 16:2 И сказала Сара Авраму: вот, Господь заключил чрево мое, чтобы мне не рождать; войди же к служанке моей: может быть, я буду иметь детей от нее. Аврам послушался слов Сары.
\vs Gen 16:3 И взяла Сара, жена Аврамова, служанку свою, Египтянку Агарь, по истечении десяти лет пребывания Аврамова в земле Ханаанской, и дала ее Авраму, мужу своему, в жену.
\vs Gen 16:4 Он вошел к Агари, и она зачала. Увидев же, что зачала, она стала презирать госпожу свою.
\vs Gen 16:5 И сказала Сара Авраму: в обиде моей ты виновен; я отдала служанку мою в недро твое; а она, увидев, что зачала, стала презирать меня; Господь пусть будет судьею между мною и между тобою.
\vs Gen 16:6 Аврам сказал Саре: вот, служанка твоя в твоих руках; делай с нею, что тебе угодно. И Сара стала притеснять ее, и она убежала от нее.
\vs Gen 16:7 И нашел ее Ангел Господень у источника воды в пустыне, у источника на дороге к Суру.
\vs Gen 16:8 И сказал [ей Ангел Господень]: Агарь, служанка Сарина! откуда ты пришла и куда идешь? Она сказала: я бегу от лица Сары, госпожи моей.
\vs Gen 16:9 Ангел Господень сказал ей: возвратись к госпоже своей и покорись ей.
\vs Gen 16:10 И сказал ей Ангел Господень: умножая умножу потомство твое, так что нельзя будет и счесть его от множества.
\vs Gen 16:11 И еще сказал ей Ангел Господень: вот, ты беременна, и родишь сына, и наречешь ему имя Измаил, ибо услышал Господь страдание твое;
\vs Gen 16:12 он будет \bibemph{между} людьми, \bibemph{как} дикий осел; руки его на всех, и руки всех на него; жить будет он пред лицем всех братьев своих.
\vs Gen 16:13 И нарекла [Агарь] Господа, Который говорил к ней, \bibemph{сим} именем: Ты Бог видящий меня. Ибо сказала она: точно я видела здесь в след видящего меня.
\vs Gen 16:14 Посему источник \bibemph{тот} называется: Беэр-лахай-рои\fns{Источник Живаго, видящего меня.}. Он находится между Кадесом и между Баредом.
\rsbpar\vs Gen 16:15 Агарь родила Авраму сына; и нарек [Аврам] имя сыну своему, рожденному от Агари: Измаил.
\vs Gen 16:16 Аврам был восьмидесяти шести лет, когда Агарь родила Авраму Измаила.
\vs Gen 17:1 Аврам был девяноста девяти лет, и Господь явился Авраму и сказал ему: Я Бог Всемогущий; ходи предо Мною и будь непорочен;
\vs Gen 17:2 и поставлю завет Мой между Мною и тобою, и весьма, весьма размножу тебя.
\vs Gen 17:3 И пал Аврам на лице свое. Бог продолжал говорить с ним и сказал:
\vs Gen 17:4 Я~--- вот завет Мой с тобою: ты будешь отцом множества народов,
\vs Gen 17:5 и не будешь ты больше называться Аврамом, но будет тебе имя: Авраам, ибо Я сделаю тебя отцом множества народов;
\vs Gen 17:6 и весьма, весьма распложу тебя, и произведу от тебя народы, и цари произойдут от тебя;
\vs Gen 17:7 и поставлю завет Мой между Мною и тобою и между потомками твоими после тебя в роды их, завет вечный в том, что Я буду Богом твоим и потомков твоих после тебя;
\vs Gen 17:8 и дам тебе и потомкам твоим после тебя землю, по которой ты странствуешь, всю землю Ханаанскую, во владение вечное; и буду им Богом.
\vs Gen 17:9 И сказал Бог Аврааму: ты же соблюди завет Мой, ты и потомки твои после тебя в роды их.
\vs Gen 17:10 Сей есть завет Мой, который вы \bibemph{должны} соблюдать между Мною и между вами и между потомками твоими после тебя [в роды их]: да будет у вас обрезан весь мужеский пол;
\vs Gen 17:11 обрезывайте крайнюю плоть вашу: и сие будет знамением завета между Мною и вами.
\vs Gen 17:12 Восьми дней от рождения да будет обрезан у вас в роды ваши всякий \bibemph{младенец} мужеского пола, рожденный в доме и купленный за серебро у какого-нибудь иноплеменника, который не от твоего семени.
\vs Gen 17:13 Непременно да будет обрезан рожденный в доме твоем и купленный за серебро твое, и будет завет Мой на теле вашем заветом вечным.
\vs Gen 17:14 Необрезанный же мужеского пола, который не обрежет крайней плоти своей [в восьмой день], истребится душа та из народа своего, \bibemph{ибо} он нарушил завет Мой.
\vs Gen 17:15 И сказал Бог Аврааму: Сару, жену твою, не называй Сарою, но да будет имя ей: Сарра;
\vs Gen 17:16 Я благословлю ее и дам тебе от нее сына; благословлю ее, и произойдут от нее народы, и цари народов произойдут от нее.
\vs Gen 17:17 И пал Авраам на лице свое, и рассмеялся, и сказал сам в себе: неужели от столетнего будет сын? и Сарра, девяностолетняя, неужели родит?
\vs Gen 17:18 И сказал Авраам Богу: о, хотя бы Измаил был жив пред лицем Твоим!
\vs Gen 17:19 Бог же сказал [Аврааму]: именно Сарра, жена твоя, родит тебе сына, и ты наречешь ему имя: Исаак; и поставлю завет Мой с ним заветом вечным [в том, что Я буду Богом ему и] потомству его после него.
\vs Gen 17:20 И о Измаиле Я услышал тебя: вот, Я благословлю его, и возращу его, и весьма, весьма размножу; двенадцать князей родятся от него; и Я произведу от него великий народ.
\vs Gen 17:21 Но завет Мой поставлю с Исааком, которого родит тебе Сарра в сие самое время на другой год.
\vs Gen 17:22 И Бог перестал говорить с Авраамом и восшел от него.
\rsbpar\vs Gen 17:23 И взял Авраам Измаила, сына своего, и всех рожденных в доме своем и всех купленных за серебро свое, весь мужеский пол людей дома Авраамова; и обрезал крайнюю плоть их в тот самый день, как сказал ему Бог.
\vs Gen 17:24 Авраам был девяноста девяти лет, когда была обрезана крайняя плоть его.
\vs Gen 17:25 А Измаил, сын его, был тринадцати лет, когда была обрезана крайняя плоть его.
\vs Gen 17:26 В тот же самый день обрезаны были Авраам и Измаил, сын его,
\vs Gen 17:27 и с ним обрезан был весь мужеский пол дома его, рожденные в доме и купленные за серебро у иноплеменников.
\vs Gen 18:1 И явился ему Господь у дубравы Мамре, когда он сидел при входе в шатер [свой], во время зноя дневного.
\vs Gen 18:2 Он возвел очи свои и взглянул, и вот, три мужа стоят против него. Увидев, он побежал навстречу им от входа в шатер [свой] и поклонился до земли,
\vs Gen 18:3 и сказал: Владыка! если я обрел благоволение пред очами Твоими, не пройди мимо раба Твоего;
\vs Gen 18:4 и принесут немного воды, и омоют ноги ваши; и отдохните под сим деревом,
\vs Gen 18:5 а я принесу хлеба, и вы подкрепите сердца ваши; потом пойдите [в путь свой]; так как вы идете мимо раба вашего. Они сказали: сделай так, как говоришь.
\vs Gen 18:6 И поспешил Авраам в шатер к Сарре и сказал [ей]: поскорее замеси три саты лучшей муки и сделай пресные хлебы.
\vs Gen 18:7 И побежал Авраам к стаду, и взял теленка нежного и хорошего, и дал отроку, и тот поспешил приготовить его.
\vs Gen 18:8 И взял масла и молока и теленка приготовленного, и поставил перед ними, а сам стоял подле них под деревом. И они ели.
\vs Gen 18:9 И сказали ему: где Сарра, жена твоя? Он отвечал: здесь, в шатре.
\vs Gen 18:10 И сказал \bibemph{один из них}: Я опять буду у тебя в это же время [в следующем году], и будет сын у Сарры, жены твоей. А Сарра слушала у входа в шатер, сзади его.
\vs Gen 18:11 Авраам же и Сарра были стары и в летах преклонных, и обыкновенное у женщин у Сарры прекратилось.
\vs Gen 18:12 Сарра внутренно рассмеялась, сказав: мне ли, когда я состарилась, иметь сие утешение? и господин мой стар.
\vs Gen 18:13 И сказал Господь Аврааму: отчего это [сама в себе] рассмеялась Сарра, сказав: <<неужели я действительно могу родить, когда я состарилась>>?
\vs Gen 18:14 Есть ли что трудное для Господа? В назначенный срок буду Я у тебя в следующем году, и [будет] у Сарры сын.
\vs Gen 18:15 Сарра же не призналась, а сказала: я не смеялась. Ибо она испугалась. Но Он сказал [ей]: нет, ты рассмеялась.
\vs Gen 18:16 И встали те мужи и оттуда отправились к Содому [и Гоморре]; Авраам же пошел с ними, проводить их.
\rsbpar\vs Gen 18:17 И сказал Господь: утаю ли Я от Авраама [раба Моего], что хочу делать!
\vs Gen 18:18 От Авраама точно произойдет народ великий и сильный, и благословятся в нем все народы земли,
\vs Gen 18:19 ибо Я избрал его для того, чтобы он заповедал сынам своим и дому своему после себя, ходить путем Господним, творя правду и суд; и исполнит Господь над Авраамом [все], что сказал о нем.
\vs Gen 18:20 И сказал Господь: вопль Содомский и Гоморрский, велик он, и грех их, тяжел он весьма;
\vs Gen 18:21 сойду и посмотрю, точно ли они поступают так, каков вопль на них, восходящий ко Мне, или нет; узнаю.
\vs Gen 18:22 И обратились мужи оттуда и пошли в Содом; Авраам же еще стоял пред лицем Господа.
\vs Gen 18:23 И подошел Авраам и сказал: неужели Ты погубишь праведного с нечестивым [и с праведником будет то же, что с нечестивым]?
\vs Gen 18:24 может быть, есть в этом городе пятьдесят праведников? неужели Ты погубишь, и не пощадишь [всего] места сего ради пятидесяти праведников, [если они находятся] в нем?
\vs Gen 18:25 не может быть, чтобы Ты поступил так, чтобы Ты погубил праведного с нечестивым, чтобы то же было с праведником, что с нечестивым; не может быть от Тебя! Судия всей земли поступит ли неправосудно?
\vs Gen 18:26 Господь сказал: если Я найду в городе Содоме пятьдесят праведников, то Я ради них пощажу [весь город и] все место сие.
\vs Gen 18:27 Авраам сказал в ответ: вот, я решился говорить Владыке, я, прах и пепел:
\vs Gen 18:28 может быть, до пятидесяти праведников недостанет пяти, неужели за \bibemph{недостатком} пяти Ты истребишь весь город? Он сказал: не истреблю, если найду там сорок пять.
\vs Gen 18:29 \bibemph{Авраам} продолжал говорить с Ним и сказал: может быть, найдется там сорок? Он сказал: не сделаю \bibemph{того} и ради сорока.
\vs Gen 18:30 И сказал \bibemph{Авраам}: да не прогневается Владыка, что я буду говорить: может быть, найдется там тридцать? Он сказал: не сделаю, если найдется там тридцать.
\vs Gen 18:31 \bibemph{Авраам} сказал: вот, я решился говорить Владыке: может быть, найдется там двадцать? Он сказал: не истреблю ради двадцати.
\vs Gen 18:32 \bibemph{Авраам} сказал: да не прогневается Владыка, что я скажу еще однажды: может быть, найдется там десять? Он сказал: не истреблю ради десяти.
\vs Gen 18:33 И пошел Господь, перестав говорить с Авраамом; Авраам же возвратился в свое место.
\vs Gen 19:1 И пришли те два Ангела в Содом вечером, когда Лот сидел у ворот Содома. Лот увидел, и встал, чтобы встретить их, и поклонился лицем до земли
\vs Gen 19:2 и сказал: государи мои! зайдите в дом раба вашего и ночуйте, и умойте ноги ваши, и встаньте поутру и пойдете в путь свой. Но они сказали: нет, мы ночуем на улице.
\vs Gen 19:3 Он же сильно упрашивал их; и они пошли к нему и пришли в дом его. Он сделал им угощение и испек пресные хлебы, и они ели.
\vs Gen 19:4 Еще не легли они спать, как городские жители, Содомляне, от молодого до старого, весь народ со \bibemph{всех} концов \bibemph{города}, окружили дом
\vs Gen 19:5 и вызвали Лота и говорили ему: где люди, пришедшие к тебе на ночь? выведи их к нам; мы позн\acc{а}ем их.
\vs Gen 19:6 Лот вышел к ним ко входу, и запер за собою дверь,
\vs Gen 19:7 и сказал [им]: братья мои, не делайте зла;
\vs Gen 19:8 вот у меня две дочери, которые не познали мужа; лучше я выведу их к вам, делайте с ними, что вам угодно, только людям сим не делайте ничего, так как они пришли под кров дома моего.
\vs Gen 19:9 Но они сказали [ему]: пойди сюда. И сказали: вот пришлец, и хочет судить? теперь мы хуже поступим с тобою, нежели с ними. И очень приступали к человеку сему, к Лоту, и подошли, чтобы выломать дверь.
\vs Gen 19:10 Тогда мужи те простерли руки свои и ввели Лота к себе в дом, и дверь [дома] заперли;
\vs Gen 19:11 а людей, бывших при входе в дом, поразили слепотою, от малого до большого, так что они измучились, искав входа.
\vs Gen 19:12 Сказали мужи те Лоту: кто у тебя есть еще здесь? зять ли, сыновья ли твои, дочери ли твои, и кто бы ни был у тебя в городе, всех выведи из сего места,
\vs Gen 19:13 ибо мы истребим сие место, потому что велик вопль на жителей его к Господу, и Господь послал нас истребить его.
\vs Gen 19:14 И вышел Лот, и говорил с зятьями своими, которые брали за себя дочерей его, и сказал: встаньте, выйдите из сего места, ибо Господь истребит сей город. Но зятьям его показалось, что он шутит.
\rsbpar\vs Gen 19:15 Когда взошла заря, Ангелы начали торопить Лота, говоря: встань, возьми жену твою и двух дочерей твоих, которые у тебя, чтобы не погибнуть тебе за беззакония города.
\vs Gen 19:16 И как он медлил, то мужи те [Ангелы], по милости к нему Господней, взяли за руку его и жену его, и двух дочерей его, и вывели его и поставили его вне города.
\vs Gen 19:17 Когда же вывели их вон, \bibemph{то один из них} сказал: спасай душу свою; не оглядывайся назад и нигде не останавливайся в окрестности сей; спасайся на гору, чтобы тебе не погибнуть.
\vs Gen 19:18 Но Лот сказал им: нет, Владыка!
\vs Gen 19:19 вот, раб Твой обрел благоволение пред очами Твоими, и велика милость Твоя, которую Ты сделал со мною, что спас жизнь мою; но я не могу спасаться на гору, чтоб не застигла меня беда и мне не умереть;
\vs Gen 19:20 вот, ближе бежать в сей город, он же мал; побегу я туда,~--- он же мал; и сохранится жизнь моя [ради Тебя].
\vs Gen 19:21 И сказал ему: вот, в угодность тебе Я сделаю и это: не ниспровергну города, о котором ты говоришь;
\vs Gen 19:22 поспешай, спасайся туда, ибо Я не могу сделать дела, доколе ты не придешь туда. Потому и назван город сей: Сигор.
\vs Gen 19:23 Солнце взошло над землею, и Лот пришел в Сигор.
\rsbpar\vs Gen 19:24 И пролил Господь на Содом и Гоморру дождем серу и огонь от Господа с неба,
\vs Gen 19:25 и ниспроверг города сии, и всю окрестность сию, и всех жителей городов сих, и [все] произрастания земли.
\vs Gen 19:26 Жена же \bibemph{Лотова} оглянулась позади его, и стала соляным столпом.
\rsbpar\vs Gen 19:27 И встал Авраам рано утром [и пошел] на место, где стоял пред лицем Господа,
\vs Gen 19:28 и посмотрел к Содому и Гоморре и на все пространство окрестности и увидел: вот, дым поднимается с земли, как дым из печи.
\vs Gen 19:29 И было, когда Бог истреблял [все] города окрестности сей, вспомнил Бог об Аврааме и выслал Лота из среды истребления, когда ниспровергал города, в которых жил Лот.
\rsbpar\vs Gen 19:30 И вышел Лот из Сигора и стал жить в гор\acc{е}, и с ним две дочери его, ибо он боялся жить в Сигоре. И жил в пещере, и с ним две дочери его.
\vs Gen 19:31 И сказала старшая младшей: отец наш стар, и нет человека на земле, который вошел бы к нам по обычаю всей земли;
\vs Gen 19:32 итак напоим отца нашего вином, и переспим с ним, и восставим от отца нашего племя.
\vs Gen 19:33 И напоили отца своего вином в ту ночь; и вошла старшая и спала с отцом своим [в ту ночь]; а он не знал, когда она легла и когда встала.
\vs Gen 19:34 На другой день старшая сказала младшей: вот, я спала вчера с отцом моим; напоим его вином и в эту ночь; и ты войди, спи с ним, и восставим от отца нашего племя.
\vs Gen 19:35 И напоили отца своего вином и в эту ночь; и вошла младшая и спала с ним; и он не знал, когда она легла и когда встала.
\vs Gen 19:36 И сделались обе дочери Лотовы беременными от отца своего,
\vs Gen 19:37 и родила старшая сына, и нарекла ему имя: Моав [говоря: \bibemph{он} от отца моего]. Он отец Моавитян доныне.
\vs Gen 19:38 И младшая также родила сына, и нарекла ему имя: Бен-Амми [говоря: \bibemph{он} сын рода моего]. Он отец Аммонитян доныне.
\vs Gen 20:1 Авраам поднялся оттуда к югу и поселился между Кадесом и между Суром; и был на время в Гераре.
\vs Gen 20:2 И сказал Авраам о Сарре, жене своей: она сестра моя. [Ибо он боялся сказать, что это жена его, чтобы жители города того не убили его за нее.] И послал Авимелех, царь Герарский, и взял Сарру.
\vs Gen 20:3 И пришел Бог к Авимелеху ночью во сне и сказал ему: вот, ты умрешь за женщину, которую ты взял, ибо она имеет мужа.
\vs Gen 20:4 Авимелех же не прикасался к ней и сказал: Владыка! неужели Ты погубишь [не знавший \bibemph{сего}] и невинный народ?
\vs Gen 20:5 Не сам ли он сказал мне: она сестра моя? И она сама сказала: он брат мой. Я сделал это в простоте сердца моего и в чистоте рук моих.
\vs Gen 20:6 И сказал ему Бог во сне: и Я знаю, что ты сделал сие в простоте сердца твоего, и удержал тебя от греха предо Мною, потому и не допустил тебя прикоснуться к ней;
\vs Gen 20:7 теперь же возврати жену мужу, ибо он пророк и помолится о тебе, и ты будешь жив; а если не возвратишь, то знай, что непременно умрешь ты и все твои.
\vs Gen 20:8 И встал Авимелех утром рано, и призвал всех рабов своих, и пересказал все слова сии в уши их; и люди сии [все] весьма испугались.
\vs Gen 20:9 И призвал Авимелех Авраама и сказал ему: что ты с нами сделал? чем согрешил я против тебя, что ты навел было на меня и на царство мое великий грех? Ты сделал со мною дела, каких не делают.
\vs Gen 20:10 И сказал Авимелех Аврааму: что ты имел в виду, когда делал это дело?
\vs Gen 20:11 Авраам сказал: я подумал, что нет на месте сем страха Божия, и убьют меня за жену мою;
\vs Gen 20:12 да она и подлинно сестра мне: она дочь отца моего, только не дочь матери моей; и сделалась моею женою;
\vs Gen 20:13 когда Бог повел меня странствовать из дома отца моего, то я сказал ей: сделай со мною сию милость, в какое ни придем мы место, везде говори обо мне: это брат мой.
\vs Gen 20:14 И взял Авимелех [серебра тысячу сиклей и] мелкого и крупного скота, и рабов и рабынь, и дал Аврааму; и возвратил ему Сарру, жену его.
\vs Gen 20:15 И сказал Авимелех [Аврааму]: вот, земля моя пред тобою; живи, где тебе угодно.
\vs Gen 20:16 И Сарре сказал: вот, я дал брату твоему тысячу \bibemph{сиклей} серебра; вот, это тебе покрывало для очей пред всеми, которые с тобою, и пред всеми ты оправдана.
\vs Gen 20:17 И помолился Авраам Богу, и исцелил Бог Авимелеха, и жену его, и рабынь его, и они стали рождать;
\vs Gen 20:18 ибо заключил Господь всякое чрево в доме Авимелеха за Сарру, жену Авраамову.
\vs Gen 21:1 И призрел Господь на Сарру, как сказал; и сделал Господь Сарре, как говорил.
\vs Gen 21:2 Сарра зачала и родила Аврааму сына в старости его во время, о котором говорил ему Бог;
\vs Gen 21:3 и нарек Авраам имя сыну своему, родившемуся у него, которого родила ему Сарра, Исаак;
\vs Gen 21:4 и обрезал Авраам Исаака, сына своего, в восьмой день, как заповедал ему Бог.
\vs Gen 21:5 Авраам был ста лет, когда родился у него Исаак, сын его.
\vs Gen 21:6 И сказала Сарра: смех сделал мне Бог; кто ни услышит обо мне, рассмеется.
\vs Gen 21:7 И сказала: кто сказал бы Аврааму: Сарра будет кормить детей грудью? ибо в старости его я родила сына.
\vs Gen 21:8 Дитя выросло и отнято от груди; и Авраам сделал большой пир в тот день, когда Исаак [сын его] отнят был от груди.
\rsbpar\vs Gen 21:9 И увидела Сарра, что сын Агари Египтянки, которого она родила Аврааму, насмехается [над ее сыном, Исааком],
\vs Gen 21:10 и сказала Аврааму: выгони эту рабыню и сына ее, ибо не наследует сын рабыни сей с сыном моим Исааком.
\vs Gen 21:11 И показалось это Аврааму весьма неприятным ради сына его [Измаила].
\vs Gen 21:12 Но Бог сказал Аврааму: не огорчайся ради отрока и рабыни твоей; во всем, что скажет тебе Сарра, слушайся голоса ее, ибо в Исааке наречется тебе семя;
\vs Gen 21:13 и от сына рабыни Я произведу [великий] народ, потому что он семя твое.
\vs Gen 21:14 Авраам встал рано утром, и взял хлеба и мех воды, и дал Агари, положив ей на плечи, и отрока, и отпустил ее. Она пошла, и заблудилась в пустыне Вирсавии;
\vs Gen 21:15 и не стало воды в мехе, и она оставила отрока под одним кустом
\vs Gen 21:16 и пошла, села вдали, в расстоянии на \bibemph{один} выстрел из лука. Ибо она сказала: не \bibemph{хочу} видеть смерти отрока. И она села [поодаль] против [него], и подняла вопль, и плакала;
\vs Gen 21:17 и услышал Бог голос отрока [оттуда, где он был]; и Ангел Божий с неба воззвал к Агари и сказал ей: что с тобою, Агарь? не бойся; Бог услышал голос отрока оттуда, где он находится;
\vs Gen 21:18 встань, подними отрока и возьми его за руку, ибо Я произведу от него великий народ.
\vs Gen 21:19 И Бог открыл глаза ее, и она увидела колодезь с водою [живою], и пошла, наполнила мех водою и напоила отрока.
\vs Gen 21:20 И Бог был с отроком; и он вырос, и стал жить в пустыне, и сделался стрелком из лука.
\vs Gen 21:21 Он жил в пустыне Фаран; и мать его взяла ему жену из земли Египетской.
\rsbpar\vs Gen 21:22 И было в то время, Авимелех с [Ахузафом невестоводителем и] Фихолом, военачальником своим, сказал Аврааму: с тобою Бог во всем, что ты ни делаешь;
\vs Gen 21:23 и теперь поклянись мне здесь Богом, что ты не обидишь ни меня, ни сына моего, ни внука моего; и как я хорошо поступал с тобою, так и ты будешь поступать со мною и землею, в которой ты гостишь.
\vs Gen 21:24 И сказал Авраам: я клянусь.
\vs Gen 21:25 И Авраам упрекал Авимелеха за колодезь с водою, который отняли рабы Авимелеховы.
\vs Gen 21:26 Авимелех же сказал [ему]: не знаю, кто это сделал, и ты не сказал мне; я даже и не слыхал \bibemph{о том} доныне.
\vs Gen 21:27 И взял Авраам мелкого и крупного скота и дал Авимелеху, и они оба заключили союз.
\vs Gen 21:28 И поставил Авраам семь агниц из \bibemph{стада} мелкого скота особо.
\vs Gen 21:29 Авимелех же сказал Аврааму: на что здесь сии семь агниц [\bibemph{из стада} овец], которых ты поставил особо?
\vs Gen 21:30 [Авраам] сказал: семь агниц сих возьми от руки моей, чтобы они были мне свидетельством, что я выкопал этот колодезь.
\vs Gen 21:31 Потому и назвал он сие место: Вирсавия, ибо тут оба они клялись
\vs Gen 21:32 и заключили союз в Вирсавии. И встал Авимелех, и [Ахузаф, невестоводитель его, и] Фихол, военачальник его, и возвратились в землю Филистимскую.
\vs Gen 21:33 И насадил [Авраам] при Вирсавии рощу и призвал там имя Господа, Бога вечного.
\vs Gen 21:34 И жил Авраам в земле Филистимской, как странник, дни многие.
\vs Gen 22:1 И было, после сих происшествий Бог искушал Авраама и сказал ему: Авраам! Он сказал: вот я.
\vs Gen 22:2 \bibemph{Бог} сказал: возьми сына твоего, единственного твоего, которого ты любишь, Исаака; и пойди в землю Мориа и там принеси его во всесожжение на одной из гор, о которой Я скажу тебе.
\vs Gen 22:3 Авраам встал рано утром, оседлал осла своего, взял с собою двоих из отроков своих и Исаака, сына своего; наколол дров для всесожжения, и встав пошел на место, о котором сказал ему Бог.
\vs Gen 22:4 На третий день Авраам возвел очи свои, и увидел то место издалека.
\vs Gen 22:5 И сказал Авраам отрокам своим: останьтесь вы здесь с ослом, а я и сын пойдем туда и поклонимся, и возвратимся к вам.
\vs Gen 22:6 И взял Авраам дрова для всесожжения, и возложил на Исаака, сына своего; взял в руки огонь и нож, и пошли оба вместе.
\vs Gen 22:7 И начал Исаак говорить Аврааму, отцу своему, и сказал: отец мой! Он отвечал: вот я, сын мой. Он сказал: вот огонь и дрова, где же агнец для всесожжения?
\vs Gen 22:8 Авраам сказал: Бог усмотрит Себе агнца для всесожжения, сын мой. И шли \bibemph{далее} оба вместе.
\rsbpar\vs Gen 22:9 И пришли на место, о котором сказал ему Бог; и устроил там Авраам жертвенник, разложил дрова и, связав сына своего Исаака, положил его на жертвенник поверх дров.
\vs Gen 22:10 И простер Авраам руку свою и взял нож, чтобы заколоть сына своего.
\vs Gen 22:11 Но Ангел Господень воззвал к нему с неба и сказал: Авраам! Авраам! Он сказал: вот я.
\vs Gen 22:12 \bibemph{Ангел} сказал: не поднимай руки твоей на отрока и не делай над ним ничего, ибо теперь Я знаю, что боишься ты Бога и не пожалел сына твоего, единственного твоего, для Меня.
\vs Gen 22:13 И возвел Авраам очи свои и увидел: и вот, позади овен, запутавшийся в чаще рогами своими. Авраам пошел, взял овна и принес его во всесожжение вместо [Исаака], сына своего.
\vs Gen 22:14 И нарек Авраам имя месту тому: Иегова-ире\fns{Господь усмотрит.}. Посему \bibemph{и} ныне говорится: на горе Иеговы усмотрится.
\vs Gen 22:15 И вторично воззвал к Аврааму Ангел Господень с неба
\vs Gen 22:16 и сказал: Мною клянусь, говорит Господь, что, так как ты сделал сие дело, и не пожалел сына твоего, единственного твоего, [для Меня,]
\vs Gen 22:17 то Я благословляя благословлю тебя и умножая умножу семя твое, как звезды небесные и как песок на берегу моря; и овладеет семя твое городами врагов своих;
\vs Gen 22:18 и благословятся в семени твоем все народы земли за то, что ты послушался гласа Моего.
\vs Gen 22:19 И возвратился Авраам к отрокам своим, и встали и пошли вместе в Вирсавию; и жил Авраам в Вирсавии.
\rsbpar\vs Gen 22:20 После сих происшествий Аврааму возвестили, сказав: вот, и Милка родила Нахору, брату твоему, сынов:
\vs Gen 22:21 Уца, первенца его, Вуза, брата сему, Кемуила, отца Арамова,
\vs Gen 22:22 Кеседа, Хазо, Пилдаша, Идлафа и Вафуила;
\vs Gen 22:23 от Вафуила родилась Ревекка. Восьмерых сих [сынов] родила Милка Нахору, брату Авраамову;
\vs Gen 22:24 и наложница его, именем Реума, также родила Теваха, Гахама, Тахаша и Мааху.
\vs Gen 23:1 Жизни Сарриной было сто двадцать семь лет: \bibemph{вот} лета жизни Сарриной;
\vs Gen 23:2 и умерла Сарра в Кириаф-Арбе, [который на долине,] что \bibemph{ныне} Хеврон, в земле Ханаанской. И пришел Авраам рыдать по Сарре и оплакивать ее.
\vs Gen 23:3 И отошел Авраам от умершей своей, и говорил сынам Хетовым, и сказал:
\vs Gen 23:4 я у вас пришлец и поселенец; дайте мне в собственность \bibemph{место для} гроба между вами, чтобы мне умершую мою схоронить от глаз моих.
\vs Gen 23:5 Сыны Хета отвечали Аврааму и сказали ему:
\vs Gen 23:6 послушай нас, господин наш; ты князь Божий посреди нас; в лучшем из наших погребальных мест похорони умершую твою; никто из нас не откажет тебе в погребальном месте, для погребения [на нем] умершей твоей.
\vs Gen 23:7 Авраам встал и поклонился народу земли той, сынам Хетовым;
\vs Gen 23:8 и говорил им [Авраам] и сказал: если вы согласны, чтобы я похоронил умершую мою, то послушайте меня, попросите за меня Ефрона, сына Цохарова,
\vs Gen 23:9 чтобы он отдал мне пещеру Махпелу, которая у него на конце поля его, чтобы за довольную цену отдал ее мне посреди вас, в собственность для погребения.
\vs Gen 23:10 Ефрон же сидел посреди сынов Хетовых; и отвечал Ефрон Хеттеянин Аврааму вслух сынов Хета, всех входящих во врата города его, и сказал:
\vs Gen 23:11 нет, господин мой, послушай меня: я даю тебе поле и пещеру, которая на нем, даю тебе, пред очами сынов народа моего дарю тебе ее, похорони умершую твою.
\vs Gen 23:12 Авраам поклонился пред народом земли той
\vs Gen 23:13 и говорил Ефрону вслух [всего] народа земли той и сказал: если послушаешь, я даю тебе за поле серебро; возьми у меня, и я похороню там умершую мою.
\vs Gen 23:14 Ефрон отвечал Аврааму и сказал ему:
\vs Gen 23:15 господин мой! послушай меня: земля \bibemph{стоит} четыреста сиклей серебра; для меня и для тебя что это? похорони умершую твою.
\vs Gen 23:16 Авраам выслушал Ефрона; и отвесил Авраам Ефрону серебра, сколько он объявил вслух сынов Хетовых, четыреста сиклей серебра, какое ходит у купцов.
\vs Gen 23:17 И стало поле Ефроново, которое при Махпеле, против Мамре, поле и пещера, которая на нем, и все деревья, которые на поле, во всех пределах его вокруг,
\vs Gen 23:18 владением Авраамовым пред очами сынов Хета, всех входящих во врата города его.
\rsbpar\vs Gen 23:19 После сего Авраам похоронил Сарру, жену свою, в пещере поля в Махпеле, против Мамре, что \bibemph{ныне} Хеврон, в земле Ханаанской.
\vs Gen 23:20 Так достались Аврааму от сынов Хетовых поле и пещера, которая на нем, в собственность для погребения.
\vs Gen 24:1 Авраам был уже стар и в летах преклонных. Господь благословил Авраама всем.
\vs Gen 24:2 И сказал Авраам рабу своему, старшему в доме его, управлявшему всем, что у него было: положи руку твою под стегно мое
\vs Gen 24:3 и клянись мне Господом, Богом неба и Богом земли, что ты не возьмешь сыну моему [Исааку] жены из дочерей Хананеев, среди которых я живу,
\vs Gen 24:4 но пойдешь в землю мою, на родину мою [и к племени моему], и возьмешь [оттуда] жену сыну моему Исааку.
\vs Gen 24:5 Раб сказал ему: может быть, не захочет женщина идти со мною в эту землю, должен ли я возвратить сына твоего в землю, из которой ты вышел?
\vs Gen 24:6 Авраам сказал ему: берегись, не возвращай сына моего туда;
\vs Gen 24:7 Господь, Бог неба [и Бог земли], Который взял меня из дома отца моего и из земли рождения моего, Который говорил мне и Который клялся мне, говоря: [тебе и] потомству твоему дам сию землю,~--- Он пошлет Ангела Своего пред тобою, и ты возьмешь жену сыну моему [Исааку] оттуда;
\vs Gen 24:8 если же не захочет женщина идти с тобою [в землю сию], ты будешь свободен от сей клятвы моей; только сына моего не возвращай туда.
\vs Gen 24:9 И положил раб руку свою под стегно Авраама, господина своего, и клялся ему в сем.
\vs Gen 24:10 И взял раб из верблюдов господина своего десять верблюдов и пошел. В руках у него были также всякие сокровища господина его. Он встал и пошел в Месопотамию, в город Нахора,
\vs Gen 24:11 и остановил верблюдов вне города, у колодезя воды, под вечер, в то время, когда выходят женщины черпать [воду],
\vs Gen 24:12 и сказал: Господи, Боже господина моего Авраама! пошли \bibemph{ее} сегодня навстречу мне и сотвори милость с господином моим Авраамом;
\vs Gen 24:13 вот, я стою у источника воды, и дочери жителей города выходят черпать воду;
\vs Gen 24:14 и девица, которой я скажу: наклони кувшин твой, я напьюсь, и которая скажет [мне]: пей, я и верблюдам твоим дам пить, [пока не напьются,]~--- вот та, которую Ты назначил рабу Твоему Исааку; и по сему узн\acc{а}ю я, что Ты творишь милость с господином моим [Авраамом].
\vs Gen 24:15 Еще не перестал он говорить [в уме своем], и вот, вышла Ревекка, которая родилась от Вафуила, сына Милки, жены Нахора, брата Авраамова, и кувшин ее на плече ее;
\vs Gen 24:16 девица \bibemph{была} прекрасна видом, дева, которой не познал муж. Она сошла к источнику, наполнила кувшин свой и пошла вверх.
\vs Gen 24:17 И побежал раб навстречу ей и сказал: дай мне испить немного воды из кувшина твоего.
\vs Gen 24:18 Она сказала: пей, господин мой. И тотчас спустила кувшин свой на руку свою и напоила его.
\vs Gen 24:19 И, когда напоила его, сказала: я стану черпать и для верблюдов твоих, пока не напьются [все].
\vs Gen 24:20 И тотчас вылила воду из кувшина своего в поило и побежала опять к колодезю почерпнуть [воды], и начерпала для всех верблюдов его.
\vs Gen 24:21 Человек тот смотрел на нее с изумлением в молчании, желая уразуметь, благословил ли Господь путь его, или нет.
\vs Gen 24:22 Когда верблюды перестали пить, тогда человек тот взял золотую серьгу, весом полсикля, и два запястья на руки ей, весом в десять \bibemph{сиклей} золота;
\vs Gen 24:23 [и спросил ее] и сказал: чья ты дочь? скажи мне, есть ли в доме отца твоего место нам ночевать?
\vs Gen 24:24 Она сказала ему: я дочь Вафуила, сына Милки, которого она родила Нахору.
\vs Gen 24:25 И еще сказала ему: у нас много соломы и корму, и \bibemph{есть} место для ночлега.
\vs Gen 24:26 И преклонился человек тот и поклонился Господу,
\vs Gen 24:27 и сказал: благословен Господь Бог господина моего Авраама, Который не оставил господина моего милостью Своею и истиною Своею! Господь прямым путем привел меня к дому брата господина моего.
\vs Gen 24:28 Девица побежала и рассказала об этом в доме матери своей.
\vs Gen 24:29 У Ревекки был брат, именем Лаван. Лаван выбежал к тому человеку, к источнику.
\vs Gen 24:30 И когда он увидел серьгу и запястья на руках у сестры своей и услышал слова Ревекки, сестры своей, которая говорила: так говорил со мною этот человек,~--- то пришел к человеку, и вот, он стоит при верблюдах у источника;
\vs Gen 24:31 и сказал [ему]: войди, благословенный Господом; зачем ты стоишь вне? я приготовил дом и место для верблюдов.
\vs Gen 24:32 И вошел человек. \bibemph{Лаван} расседлал верблюдов и дал соломы и корму верблюдам, и воды умыть ноги ему и людям, которые были с ним;
\vs Gen 24:33 и предложена была ему пища; но он сказал: не стану есть, доколе не скажу дела своего. И сказали: говори.
\vs Gen 24:34 Он сказал: я раб Авраамов;
\vs Gen 24:35 Господь весьма благословил господина моего, и он сделался великим: Он дал ему овец и волов, серебро и золото, рабов и рабынь, верблюдов и ослов;
\vs Gen 24:36 Сарра, жена господина моего, уже состарившись, родила господину моему [одного] сына, которому он отдал все, что у него;
\vs Gen 24:37 и взял с меня клятву господин мой, сказав: не бери жены сыну моему из дочерей Хананеев, в земле которых я живу,
\vs Gen 24:38 а пойди в дом отца моего и к родственникам моим, и возьмешь [оттуда] жену сыну моему.
\vs Gen 24:39 Я сказал господину моему: может быть, не пойдет женщина со мною.
\vs Gen 24:40 Он сказал мне: Господь [Бог], пред лицем Которого я хожу, пошлет с тобою Ангела Своего и благоустроит путь твой, и возьмешь жену сыну моему из родных моих и из дома отца моего;
\vs Gen 24:41 тогда будешь ты свободен от клятвы моей, когда сходишь к родственникам моим; и если они не дадут тебе, то будешь свободен от клятвы моей.
\vs Gen 24:42 И пришел я ныне к источнику, и сказал: Господи, Боже господина моего Авраама! Если Ты благоустроишь путь, который я совершаю,
\vs Gen 24:43 то вот, я стою у источника воды, [и дочери жителей города выходят черпать воду,] и девица, которая выйдет почерпать, и которой я скажу: дай мне испить немного из кувшина твоего,
\vs Gen 24:44 и которая скажет мне: и ты пей, и верблюдам твоим я начерпаю,~--- вот жена, которую Господь назначил сыну господина моего [рабу Своему Исааку; и по сему узнаю я, что Ты творишь милость с господином моим Авраамом].
\vs Gen 24:45 Еще не перестал я говорить в уме моем, и вот вышла Ревекка, и кувшин ее на плече ее, и сошла к источнику и почерпнула [воды]; и я сказал ей: напой меня.
\vs Gen 24:46 Она тотчас спустила с себя кувшин свой [на руку свою] и сказала: пей, и верблюдов твоих я напою. И я пил, и верблюдов [моих] она напоила.
\vs Gen 24:47 Я спросил ее и сказал: чья ты дочь? [скажи мне]. Она сказала: дочь Вафуила, сына Нахорова, которого родила ему Милка. И дал я серьги ей и запястья на руки ее.
\vs Gen 24:48 И преклонился я и поклонился Господу, и благословил Господа, Бога господина моего Авраама, Который прямым путем привел меня, чтобы взять дочь брата господина моего за сына его.
\vs Gen 24:49 И ныне скажите мне: намерены ли вы оказать милость и правду господину моему или нет? скажите мне, и я обращусь направо, или налево.
\vs Gen 24:50 И отвечали Лаван и Вафуил и сказали: от Господа пришло это дело; мы не можем сказать тебе вопреки ни худого, ни доброго;
\vs Gen 24:51 вот Ревекка пред тобою; возьми [ее] и пойди; пусть будет она женою сыну господина твоего, как сказал Господь.
\vs Gen 24:52 Когда раб Авраамов услышал слова их, то поклонился Господу до земли.
\vs Gen 24:53 И вынул раб серебряные вещи и золотые вещи и одежды и дал Ревекке; также и брату ее и матери ее дал богатые подарки.
\vs Gen 24:54 И ели и пили он и люди, бывшие с ним, и переночевали. Когда же встали поутру, то он сказал: отпустите меня [и я пойду] к господину моему.
\vs Gen 24:55 Но брат ее и мать ее сказали: пусть побудет с нами девица дней хотя десять, потом пойдешь.
\vs Gen 24:56 Он сказал им: не удерживайте меня, ибо Господь благоустроил путь мой; отпустите меня, и я пойду к господину моему.
\vs Gen 24:57 Они сказали: призовем девицу и спросим, что она скажет.
\vs Gen 24:58 И призвали Ревекку и сказали ей: пойдешь ли с этим человеком? Она сказала: пойду.
\vs Gen 24:59 И отпустили Ревекку, сестру свою, и кормилицу ее, и раба Авраамова, и людей его.
\vs Gen 24:60 И благословили Ревекку и сказали ей: сестра наша! да родятся от тебя тысячи тысяч, и да владеет потомство твое жилищами врагов твоих!
\vs Gen 24:61 И встала Ревекка и служанки ее, и сели на верблюдов, и поехали за тем человеком. И раб взял Ревекку и пошел.
\vs Gen 24:62 А Исаак пришел из Беэр-лахай-рои, ибо жил он в земле полуденной.
\vs Gen 24:63 При наступлении вечера Исаак вышел в поле поразмыслить, и возвел очи свои, и увидел: вот, идут верблюды.
\vs Gen 24:64 Ревекка взглянула, и увидела Исаака, и спустилась с верблюда.
\vs Gen 24:65 И сказала рабу: кто этот человек, который идет по полю навстречу нам? Раб сказал: это господин мой. И она взяла покрывало и покрылась.
\vs Gen 24:66 Раб же сказал Исааку все, что сделал.
\vs Gen 24:67 И ввел ее Исаак в шатер Сарры, матери своей, и взял Ревекку, и она сделалась ему женою, и он возлюбил ее; и утешился Исаак в \bibemph{печали} по [Сарре,] матери своей.
\vs Gen 25:1 И взял Авраам еще жену, именем Хеттуру.
\vs Gen 25:2 Она родила ему Зимрана, Иокшана, Медана, Мадиана, Ишбака и Шуаха.
\vs Gen 25:3 Иокшан родил Шеву, [Фемана] и Дедана. Сыны Дедана были: [Рагуил, Навдеил,] Ашурим, Летушим и Леюмим.
\vs Gen 25:4 Сыны Мадиана: Ефа, Ефер, Ханох, Авида и Елдага. Все сии сыны Хеттуры.
\vs Gen 25:5 И отдал Авраам все, что было у него, Исааку [сыну своему],
\vs Gen 25:6 а сынам наложниц, которые были у Авраама, дал Авраам подарки и отослал их от Исаака, сына своего, еще при жизни своей, на восток, в землю восточную.
\vs Gen 25:7 Дней жизни Авраамовой, которые он прожил, было сто семьдесят пять лет;
\vs Gen 25:8 и скончался Авраам, и умер в старости доброй, престарелый и насыщенный [жизнью], и приложился к народу своему.
\vs Gen 25:9 И погребли его Исаак и Измаил, сыновья его, в пещере Махпеле, на поле Ефрона, сына Цохара, Хеттеянина, которое против Мамре,
\vs Gen 25:10 на поле [и в пещере], которые Авраам приобрел от сынов Хетовых. Там погребены Авраам и Сарра, жена его.
\vs Gen 25:11 По смерти Авраама Бог благословил Исаака, сына его. Исаак жил при Беэр-лахай-рои.
\vs Gen 25:12 Вот родословие Измаила, сына Авраамова, которого родила Аврааму Агарь Египтянка, служанка Саррина;
\vs Gen 25:13 и вот имена сынов Измаиловых, имена их по родословию их: первенец Измаилов Наваиоф, \bibemph{за ним} Кедар, Адбеел, Мивсам,
\vs Gen 25:14 Мишма, Дума, Масса,
\vs Gen 25:15 Хадад, Фема, Иетур, Нафиш и Кедма.
\vs Gen 25:16 Сии суть сыны Измаиловы, и сии имена их, в селениях их, в кочевьях их. \bibemph{Это} двенадцать князей племен их.
\vs Gen 25:17 Лет же жизни Измаиловой было сто тридцать семь лет; и скончался он, и умер, и приложился к народу своему.
\vs Gen 25:18 Они жили от Хавилы до Сура, что пред Египтом, как идешь к Ассирии. Они поселились пред лицем всех братьев своих.
\rsbpar\vs Gen 25:19 Вот родословие Исаака, сына Авраамова. Авраам родил Исаака.
\vs Gen 25:20 Исаак был сорока лет, когда он взял себе в жену Ревекку, дочь Вафуила Арамеянина из Месопотамии, сестру Лавана Арамеянина.
\vs Gen 25:21 И молился Исаак Господу о [Ревекке] жене своей, потому что она была неплодна; и Господь услышал его, и зачала Ревекка, жена его.
\vs Gen 25:22 Сыновья в утробе ее стали биться, и она сказала: если так будет, то для чего мне это? И пошла вопросить Господа.
\vs Gen 25:23 Господь сказал ей: два племени во чреве твоем, и два различных народа произойдут из утробы твоей; один народ сделается сильнее другого, и больший будет служить меньшему.
\vs Gen 25:24 И настало время родить ей: и вот близнецы в утробе ее.
\vs Gen 25:25 Первый вышел красный, весь, как кожа, косматый; и нарекли ему имя Исав.
\vs Gen 25:26 Потом вышел брат его, держась рукою своею за пяту Исава; и наречено ему имя Иаков. Исаак же был шестидесяти лет, когда они родились [от Ревекки].
\rsbpar\vs Gen 25:27 Дети выросли, и стал Исав человеком искусным в звероловстве, человеком полей; а Иаков человеком кротким, живущим в шатрах.
\vs Gen 25:28 Исаак любил Исава, потому что дичь его была по вкусу его, а Ревекка любила Иакова.
\vs Gen 25:29 И сварил Иаков кушанье; а Исав пришел с поля усталый.
\vs Gen 25:30 И сказал Исав Иакову: дай мне поесть красного, красного этого, ибо я устал. От сего дано ему прозвание: Едом.
\vs Gen 25:31 Но Иаков сказал [Исаву]: продай мне теперь же свое первородство.
\vs Gen 25:32 Исав сказал: вот, я умираю, что мне в этом первородстве?
\vs Gen 25:33 Иаков сказал [ему]: поклянись мне теперь же. Он поклялся ему, и продал [Исав] первородство свое Иакову.
\vs Gen 25:34 И дал Иаков Исаву хлеба и кушанья из чечевицы; и он ел и пил, и встал и пошел; и пренебрег Исав первородство.
\vs Gen 26:1 Был голод в земле, сверх прежнего голода, который был во дни Авраама; и пошел Исаак к Авимелеху, царю Филистимскому, в Герар.
\vs Gen 26:2 Господь явился ему и сказал: не ходи в Египет; живи в земле, о которой Я скажу тебе,
\vs Gen 26:3 странствуй по сей земле, и Я буду с тобою и благословлю тебя, ибо тебе и потомству твоему дам все земли сии и исполню клятву [Мою], которою Я клялся Аврааму, отцу твоему;
\vs Gen 26:4 умножу потомство твое, как звезды небесные, и дам потомству твоему все земли сии; благословятся в семени твоем все народы земные,
\vs Gen 26:5 за то, что Авраам [отец твой] послушался гласа Моего и соблюдал, что Мною \bibemph{заповедано} было соблюдать: повеления Мои, уставы Мои и законы Мои.
\vs Gen 26:6 Исаак поселился в Гераре.
\vs Gen 26:7 Жители места того спросили о [Ревекке] жене его, и он сказал: это сестра моя; потому что боялся сказать: жена моя, чтобы не убили меня, \bibemph{думал он}, жители места сего за Ревекку, потому что она прекрасна видом.
\vs Gen 26:8 Но когда уже много времени он там прожил, Авимелех, царь Филистимский, посмотрев в окно, увидел, что Исаак играет с Ревеккою, женою своею.
\vs Gen 26:9 И призвал Авимелех Исаака и сказал: вот, это жена твоя; как же ты сказал: она сестра моя? Исаак сказал ему: потому что я думал, не умереть бы мне ради ее.
\vs Gen 26:10 Но Авимелех сказал [ему]: что это ты сделал с нами? едва один из народа [моего] не совокупился с женою твоею, и ты ввел бы нас в грех.
\vs Gen 26:11 И дал Авимелех повеление всему народу, сказав: кто прикоснется к сему человеку и к жене его, тот предан будет смерти.
\vs Gen 26:12 И сеял Исаак в земле той и получил в тот год ячменя во сто крат: так благословил его Господь.
\vs Gen 26:13 И стал великим человек сей и возвеличивался больше и больше до того, что стал весьма великим.
\vs Gen 26:14 У него были стада мелкого и стада крупного скота и множество пахотных полей, и Филистимляне стали завидовать ему.
\vs Gen 26:15 И все колодези, которые выкопали рабы отца его при жизни отца его Авраама, Филистимляне завалили и засыпали землею.
\vs Gen 26:16 И Авимелех сказал Исааку: удались от нас, ибо ты сделался гораздо сильнее нас.
\vs Gen 26:17 И Исаак удалился оттуда, и расположился шатрами в долине Герарской, и поселился там.
\vs Gen 26:18 И вновь выкопал Исаак колодези воды, которые выкопаны были во дни Авраама, отца его, и которые завалили Филистимляне по смерти Авраама [отца его]; и назвал их теми же именами, которыми назвал их [Авраам,] отец его.
\vs Gen 26:19 И копали рабы Исааковы в долине [Герарской] и нашли там колодезь воды живой.
\vs Gen 26:20 И спорили пастухи Герарские с пастухами Исаака, говоря: наша вода. И он нарек колодезю имя: Есек, потому что спорили с ним.
\vs Gen 26:21 [Когда двинулся оттуда Исаак,] выкопали другой колодезь; спорили также и о нем; и он нарек ему имя: Ситна.
\vs Gen 26:22 И он двинулся отсюда и выкопал иной колодезь, о котором уже не спорили, и нарек ему имя: Реховоф, ибо, сказал он, теперь Господь дал нам пространное место, и мы размножимся на земле.
\rsbpar\vs Gen 26:23 Оттуда перешел он в Вирсавию.
\vs Gen 26:24 И в ту ночь явился ему Господь и сказал: Я Бог Авраама, отца твоего; не бойся, ибо Я с тобою; и благословлю тебя и умножу потомство твое, ради [отца твоего] Авраама, раба Моего.
\vs Gen 26:25 И он устроил там жертвенник и призвал имя Господа. И раскинул там шатер свой, и выкопали там рабы Исааковы колодезь, [в долине Герарской].
\vs Gen 26:26 Пришел к нему из Герара Авимелех и Ахузаф, друг его, и Фихол, военачальник его.
\vs Gen 26:27 Исаак сказал им: для чего вы пришли ко мне, когда вы возненавидели меня и выслали меня от себя?
\vs Gen 26:28 Они сказали: мы ясно увидели, что Господь с тобою, и потому мы сказали: поставим между нами и тобою клятву и заключим с тобою союз,
\vs Gen 26:29 чтобы ты не делал нам зла, как и мы не коснулись до тебя, а делали тебе одно доброе и отпустили тебя с миром; теперь ты благословен Господом.
\vs Gen 26:30 Он сделал им пиршество, и они ели и пили.
\vs Gen 26:31 И встав рано утром, поклялись друг другу; и отпустил их Исаак, и они пошли от него с миром.
\vs Gen 26:32 В тот же день пришли рабы Исааковы и известили его о колодезе, который копали они, и сказали ему: мы нашли воду.
\vs Gen 26:33 И он назвал его: Шива. Посему имя городу тому Беэршива [Вирсавия] до сего дня.
\vs Gen 26:34 И был Исав сорока лет, и взял себе в жены Иегудифу, дочь Беэра Хеттеянина, и Васемафу, дочь Елона Хеттеянина;
\vs Gen 26:35 и они были в тягость Исааку и Ревекке.
\vs Gen 27:1 Когда Исаак состарился и притупилось зрение глаз его, он призвал старшего сына своего Исава и сказал ему: сын мой! Тот сказал ему: вот я.
\vs Gen 27:2 [Исаак] сказал: вот, я состарился; не знаю дня смерти моей;
\vs Gen 27:3 возьми теперь орудия твои, колчан твой и лук твой, пойди в поле, и налови мне дичи,
\vs Gen 27:4 и приготовь мне кушанье, какое я люблю, и принеси мне есть, чтобы благословила тебя душа моя, прежде нежели я умру.
\vs Gen 27:5 Ревекка слышала, когда Исаак говорил сыну своему Исаву. И пошел Исав в поле достать и принести дичи;
\vs Gen 27:6 а Ревекка сказала [меньшему] сыну своему Иакову: вот, я слышала, как отец твой говорил брату твоему Исаву:
\vs Gen 27:7 принеси мне дичи и приготовь мне кушанье; я поем и благословлю тебя пред лицем Господним, пред смертью моею.
\vs Gen 27:8 Теперь, сын мой, послушайся слов моих в том, что я прикажу тебе:
\vs Gen 27:9 пойди в \bibemph{стадо} и возьми мне оттуда два козленка [молодых] хороших, и я приготовлю из них отцу твоему кушанье, какое он любит,
\vs Gen 27:10 а ты принесешь отцу твоему, и он поест, чтобы благословить тебя пред смертью своею.
\vs Gen 27:11 Иаков сказал Ревекке, матери своей: Исав, брат мой, человек косматый, а я человек гладкий;
\vs Gen 27:12 может статься, ощупает меня отец мой, и я буду в глазах его обманщиком и наведу на себя проклятие, а не благословение.
\vs Gen 27:13 Мать его сказала ему: на мне пусть будет проклятие твое, сын мой, только послушайся слов моих и пойди, принеси мне.
\vs Gen 27:14 Он пошел, и взял, и принес матери своей; и мать его сделала кушанье, какое любил отец его.
\vs Gen 27:15 И взяла Ревекка богатую одежду старшего сына своего Исава, бывшую у ней в доме, и одела [в нее] младшего сына своего Иакова;
\vs Gen 27:16 а руки его и гладкую шею его обложила кожею козлят;
\vs Gen 27:17 и дала кушанье и хлеб, которые она приготовила, в руки Иакову, сыну своему.
\vs Gen 27:18 Он вошел к отцу своему и сказал: отец мой! Тот сказал: вот я; кто ты, сын мой?
\vs Gen 27:19 Иаков сказал отцу своему: я Исав, первенец твой; я сделал, как ты сказал мне; встань, сядь и поешь дичи моей, чтобы благословила меня душа твоя.
\vs Gen 27:20 И сказал Исаак сыну своему: что так скоро нашел ты, сын мой? Он сказал: потому что Господь Бог твой послал мне навстречу.
\vs Gen 27:21 И сказал Исаак Иакову: подойди [ко мне], я ощупаю тебя, сын мой, ты ли сын мой Исав, или нет?
\vs Gen 27:22 Иаков подошел к Исааку, отцу своему, и он ощупал его и сказал: голос, голос Иакова; а руки, руки Исавовы.
\vs Gen 27:23 И не узнал его, потому что руки его были, как руки Исава, брата его, косматые; и благословил его
\vs Gen 27:24 и сказал: ты ли сын мой Исав? Он отвечал: я.
\vs Gen 27:25 \bibemph{Исаак} сказал: подай мне, я поем дичи сына моего, чтобы благословила тебя душа моя. \bibemph{Иаков} подал ему, и он ел; принес ему и вина, и он пил.
\vs Gen 27:26 Исаак, отец его, сказал ему: подойди [ко мне], поцелуй меня, сын мой.
\vs Gen 27:27 Он подошел и поцеловал его. И ощутил \bibemph{Исаак} запах от одежды его и благословил его и сказал: вот, запах от сына моего, как запах от поля [полного], которое благословил Господь;
\vs Gen 27:28 да даст тебе Бог от росы небесной и от тука земли, и множество хлеба и вина;
\vs Gen 27:29 да послужат тебе народы, и да поклонятся тебе племена; будь господином над братьями твоими, и да поклонятся тебе сыны матери твоей; проклинающие тебя~--- прокляты; благословляющие тебя~--- благословенны!
\rsbpar\vs Gen 27:30 Как скоро совершил Исаак благословение над Иаковом [сыном своим], и как только вышел Иаков от лица Исаака, отца своего, Исав, брат его, пришел с ловли своей.
\vs Gen 27:31 Приготовил и он кушанье, и принес отцу своему, и сказал отцу своему: встань, отец мой, и поешь дичи сына твоего, чтобы благословила меня душа твоя.
\vs Gen 27:32 Исаак же, отец его, сказал ему: кто ты? Он сказал: я сын твой, первенец твой, Исав.
\vs Gen 27:33 И вострепетал Исаак весьма великим трепетом, и сказал: кто ж это, который достал [мне] дичи и принес мне, и я ел от всего, прежде нежели ты пришел, и я благословил его? он и будет благословен.
\vs Gen 27:34 Исав, выслушав слова отца своего [Исаака], поднял громкий и весьма горький вопль и сказал отцу своему: отец мой! благослови и меня.
\vs Gen 27:35 Но он сказал [ему]: брат твой пришел с хитростью и взял благословение твое.
\vs Gen 27:36 И сказал [Исав]: не потому ли дано ему имя: Иаков, что он запнул меня уже два раза? Он взял первородство мое, и вот, теперь взял благословение мое. И \bibemph{еще} сказал [Исав отцу своему]: неужели ты не оставил [и] мне благословения?
\vs Gen 27:37 Исаак отвечал Исаву: вот, я поставил его господином над тобою и всех братьев его отдал ему в рабы; одарил его хлебом и вином; что же я сделаю для тебя, сын мой?
\vs Gen 27:38 Но Исав сказал отцу своему: неужели, отец мой, одно у тебя благословение? благослови и меня, отец мой! И [как Исаак молчал,] возвысил Исав голос свой и заплакал.
\vs Gen 27:39 И отвечал Исаак, отец его, и сказал ему: вот, от тука земли будет обитание твое и от росы небесной свыше;
\vs Gen 27:40 и ты будешь жить мечом твоим и будешь служить брату твоему; будет же \bibemph{время}, когда воспротивишься и свергнешь иго его с выи твоей.
\vs Gen 27:41 И возненавидел Исав Иакова за благословение, которым благословил его отец его; и сказал Исав в сердце своем: приближаются дни плача по отце моем, и я убью Иакова, брата моего.
\vs Gen 27:42 И пересказаны были Ревекке слова Исава, старшего сына ее; и она послала, и призвала младшего сына своего Иакова, и сказала ему: вот, Исав, брат твой, грозит убить тебя;
\vs Gen 27:43 и теперь, сын мой, послушайся слов моих, встань, беги [в Месопотамию] к Лавану, брату моему, в Харран,
\vs Gen 27:44 и поживи у него несколько времени, пока утолится ярость брата твоего,
\vs Gen 27:45 пока утолится гнев брата твоего на тебя, и он позабудет, что ты сделал ему: тогда я пошлю и возьму тебя оттуда; для чего мне в один день лишиться обоих вас?
\vs Gen 27:46 И сказала Ревекка Исааку: я жизни не рада от дочерей Хеттейских; если Иаков возьмет жену из дочерей Хеттейских, каковы эти, из дочерей этой земли, то к чему мне и жизнь?
\vs Gen 28:1 И призвал Исаак Иакова и благословил его, и заповедал ему и сказал: не бери себе жены из дочерей Ханаанских;
\vs Gen 28:2 встань, пойди в Месопотамию, в дом Вафуила, отца матери твоей, и возьми себе жену оттуда, из дочерей Лавана, брата матери твоей;
\vs Gen 28:3 Бог же Всемогущий да благословит тебя, да расплодит тебя и да размножит тебя, и да будет от тебя множество народов,
\vs Gen 28:4 и да даст тебе благословение Авраама [отца моего], тебе и потомству твоему с тобою, чтобы тебе наследовать землю странствования твоего, которую Бог дал Аврааму!
\vs Gen 28:5 И отпустил Исаак Иакова, и он пошел в Месопотамию к Лавану, сыну Вафуила Арамеянина, к брату Ревекки, матери Иакова и Исава.
\rsbpar\vs Gen 28:6 Исав увидел, что Исаак благословил Иакова и благословляя послал его в Месопотамию, взять себе жену оттуда, и заповедал ему, сказав: не бери жены из дочерей Ханаанских;
\vs Gen 28:7 и что Иаков послушался отца своего и матери своей и пошел в Месопотамию.
\vs Gen 28:8 И увидел Исав, что дочери Ханаанские не угодны Исааку, отцу его;
\vs Gen 28:9 и пошел Исав к Измаилу и взял себе жену Махалафу, дочь Измаила, сына Авраамова, сестру Наваиофову, сверх \bibemph{других} жен своих.
\rsbpar\vs Gen 28:10 Иаков же вышел из Вирсавии и пошел в Харран,
\vs Gen 28:11 и пришел на \bibemph{одно} место, и \bibemph{остался} там ночевать, потому что зашло солнце. И взял \bibemph{один} из камней того места, и положил себе изголовьем, и лег на том месте.
\vs Gen 28:12 И увидел во сне: вот, лестница стоит на земле, а верх ее касается неба; и вот, Ангелы Божии восходят и нисходят по ней.
\vs Gen 28:13 И вот, Господь стоит на ней и говорит: Я Господь, Бог Авраама, отца твоего, и Бог Исаака; [не бойся]. Землю, на которой ты лежишь, Я дам тебе и потомству твоему;
\vs Gen 28:14 и будет потомство твое, как песок земной; и распространишься к морю и к востоку, и к северу и к полудню; и благословятся в тебе и в семени твоем все племена земные;
\vs Gen 28:15 и вот Я с тобою, и сохраню тебя везде, куда ты ни пойдешь; и возвращу тебя в сию землю, ибо Я не оставлю тебя, доколе не исполню того, что Я сказал тебе.
\vs Gen 28:16 Иаков пробудился от сна своего и сказал: истинно Господь присутствует на месте сем; а я не знал!
\vs Gen 28:17 И убоялся и сказал: как страшно сие место! это не иное что, как дом Божий, это врата небесные.
\vs Gen 28:18 И встал Иаков рано утром, и взял камень, который он положил себе изголовьем, и поставил его памятником, и возлил елей на верх его.
\vs Gen 28:19 И нарек [Иаков] имя месту тому: Вефиль\fns{Дом Божий.}, а прежнее имя того города было: Луз.
\vs Gen 28:20 И положил Иаков обет, сказав: если [Господь] Бог будет со мною и сохранит меня в пути сем, в который я иду, и даст мне хлеб есть и одежду одеться,
\vs Gen 28:21 и я в мире возвращусь в дом отца моего, и будет Господь моим Богом,~---
\vs Gen 28:22 то этот камень, который я поставил памятником, будет [у меня] домом Божиим; и из всего, что Ты, \bibemph{Боже}, даруешь мне, я дам Тебе десятую часть.
\vs Gen 29:1 И встал Иаков и пошел в землю сынов востока [к Лавану, сыну Вафуила Арамеянина, к брату Ревекки, матери Иакова и Исава].
\vs Gen 29:2 И увидел: вот, на поле колодезь, и там три стада мелкого скота, лежавшие около него, потому что из того колодезя поили стада. Над устьем колодезя был большой камень.
\vs Gen 29:3 Когда собирались туда все стада, отваливали камень от устья колодезя и поили овец; потом опять клали камень на свое место, на устье колодезя.
\vs Gen 29:4 Иаков сказал им [пастухам]: братья мои! откуда вы? Они сказали: мы из Харрана.
\vs Gen 29:5 Он сказал им: знаете ли вы Лавана, сына Нахорова? Они сказали: знаем.
\vs Gen 29:6 Он еще сказал им: здравствует ли он? Они сказали: здравствует; и вот, Рахиль, дочь его, идет с овцами.
\vs Gen 29:7 И сказал [Иаков]: вот, дня еще много; не время собирать скот; напойте овец и пойдите, пасите.
\vs Gen 29:8 Они сказали: не можем, пока не соберутся все стада, и не отвалят камня от устья колодезя; тогда будем мы поить овец.
\vs Gen 29:9 Еще он говорил с ними, как пришла Рахиль [дочь Лавана] с мелким скотом отца своего, потому что она пасла [мелкий скот отца своего].
\vs Gen 29:10 Когда Иаков увидел Рахиль, дочь Лавана, брата матери своей, и овец Лавана, брата матери своей, то подошел Иаков, отвалил камень от устья колодезя и напоил овец Лавана, брата матери своей.
\vs Gen 29:11 И поцеловал Иаков Рахиль и возвысил голос свой и заплакал.
\vs Gen 29:12 И сказал Иаков Рахили, что он родственник отцу ее и что он сын Ревеккин. А она побежала и сказала отцу своему [всё сие].
\vs Gen 29:13 Лаван, услышав о Иакове, сыне сестры своей, выбежал ему навстречу, обнял его и поцеловал его, и ввел его в дом свой; и он рассказал Лавану всё сие.
\vs Gen 29:14 Лаван же сказал ему: подлинно ты кость моя и плоть моя. И жил у него \bibemph{Иаков} целый месяц.
\rsbpar\vs Gen 29:15 И Лаван сказал Иакову: неужели ты даром будешь служить мне, потому что ты родственник? скажи мне, что заплатить тебе?
\vs Gen 29:16 У Лавана же было две дочери; имя старшей: Лия; имя младшей: Рахиль.
\vs Gen 29:17 Лия была слаба глазами, а Рахиль была красива станом и красива лицем.
\vs Gen 29:18 Иаков полюбил Рахиль и сказал: я буду служить тебе семь лет за Рахиль, младшую дочь твою.
\vs Gen 29:19 Лаван сказал [ему]: лучше отдать мне ее за тебя, нежели отдать ее за другого кого; живи у меня.
\vs Gen 29:20 И служил Иаков за Рахиль семь лет; и они показались ему за несколько дней, потому что он любил ее.
\vs Gen 29:21 И сказал Иаков Лавану: дай жену мою, потому что мне уже исполнилось время, чтобы войти к ней.
\vs Gen 29:22 Лаван созвал всех людей того места и сделал пир.
\vs Gen 29:23 Вечером же взял [Лаван] дочь свою Лию и ввел ее к нему; и вошел к ней [Иаков].
\vs Gen 29:24 И дал Лаван служанку свою Зелфу в служанки дочери своей Лии.
\vs Gen 29:25 Утром же оказалось, что это Лия. И [Иаков] сказал Лавану: что это сделал ты со мною? не за Рахиль ли я служил у тебя? зачем ты обманул меня?
\vs Gen 29:26 Лаван сказал: в нашем месте так не делают, чтобы младшую выдать прежде старшей;
\vs Gen 29:27 окончи неделю этой, потом дадим тебе и ту за службу, которую ты будешь служить у меня еще семь лет других.
\vs Gen 29:28 Иаков так и сделал и окончил неделю этой. И [Лаван] дал Рахиль, дочь свою, ему в жену.
\vs Gen 29:29 И дал Лаван служанку свою Валлу в служанки дочери своей Рахили.
\vs Gen 29:30 [Иаков] вошел и к Рахили, и любил Рахиль больше, нежели Лию; и служил у него еще семь лет других.
\rsbpar\vs Gen 29:31 Господь [Бог] узрел, что Лия была нелюбима, и отверз утробу ее, а Рахиль была неплодна.
\vs Gen 29:32 Лия зачала и родила [Иакову] сына, и нарекла ему имя: Рувим, потому что сказала она: Господь призрел на мое бедствие [и дал мне сына], ибо теперь будет любить меня муж мой.
\vs Gen 29:33 И зачала [Лия] опять и родила [Иакову второго] сына, и сказала: Господь услышал, что я нелюбима, и дал мне и сего. И нарекла ему имя: Симеон.
\vs Gen 29:34 И зачала еще и родила сына, и сказала: теперь-то прилепится ко мне муж мой, ибо я родила ему трех сынов. От сего наречено ему имя: Левий.
\vs Gen 29:35 И еще зачала и родила сына, и сказала: теперь-то я восхвалю Господа. Посему нарекла ему имя Иуда. И перестала рождать.
\vs Gen 30:1 И увидела Рахиль, что она не рождает детей Иакову, и позавидовала Рахиль сестре своей, и сказала Иакову: дай мне детей, а если не так, я умираю.
\vs Gen 30:2 Иаков разгневался на Рахиль и сказал [ей]: разве я Бог, Который не дал тебе плода чрева?
\vs Gen 30:3 Она сказала: вот служанка моя Валла; войди к ней; пусть она родит на колени мои, чтобы и я имела детей от нее.
\vs Gen 30:4 И дала она Валлу, служанку свою, в жену ему; и вошел к ней Иаков.
\vs Gen 30:5 Валла [служанка Рахилина] зачала и родила Иакову сына.
\vs Gen 30:6 И сказала Рахиль: судил мне Бог, и услышал голос мой, и дал мне сына. Посему нарекла ему имя: Дан.
\vs Gen 30:7 И еще зачала и родила Валла, служанка Рахилина, другого сына Иакову.
\vs Gen 30:8 И сказала Рахиль: борьбою сильною боролась я с сестрою моею и превозмогла. И нарекла ему имя: Неффалим.
\vs Gen 30:9 Лия увидела, что перестала рождать, и взяла служанку свою Зелфу, и дала ее Иакову в жену, [и он вошел к ней].
\vs Gen 30:10 И Зелфа, служанка Лиина, [зачала и] родила Иакову сына.
\vs Gen 30:11 И сказала Лия: прибавилось. И нарекла ему имя: Гад.
\vs Gen 30:12 И [еще зачала] Зелфа, служанка Лии, [и] родила другого сына Иакову.
\vs Gen 30:13 И сказала Лия: к благу моему, ибо блаженною будут называть меня женщины. И нарекла ему имя: Асир.
\vs Gen 30:14 Рувим пошел во время жатвы пшеницы, и нашел мандрагоровые яблоки в поле, и принес их Лии, матери своей. И Рахиль сказала Лии [сестре своей]: дай мне мандрагоров сына твоего.
\vs Gen 30:15 Но [Лия] сказала ей: неужели мало тебе завладеть мужем моим, что ты домогаешься и мандрагоров сына моего? Рахиль сказала: так пусть он ляжет с тобою эту ночь, за мандрагоры сына твоего.
\vs Gen 30:16 Иаков пришел с поля вечером, и Лия вышла ему навстречу и сказала: войди ко мне [сегодня], ибо я купила тебя за мандрагоры сына моего. И лег он с нею в ту ночь.
\vs Gen 30:17 И услышал Бог Лию, и она зачала и родила Иакову пятого сына.
\vs Gen 30:18 И сказала Лия: Бог дал возмездие мне за то, что я отдала служанку мою мужу моему. И нарекла ему имя: Иссахар [что значит возмездие].
\vs Gen 30:19 И еще зачала Лия и родила Иакову шестого сына.
\vs Gen 30:20 И сказала Лия: Бог дал мне прекрасный дар; теперь будет жить у меня муж мой, ибо я родила ему шесть сынов. И нарекла ему имя: Завулон.
\vs Gen 30:21 Потом родила дочь и нарекла ей имя: Дина.
\vs Gen 30:22 И вспомнил Бог о Рахили, и услышал ее Бог, и отверз утробу ее.
\vs Gen 30:23 Она зачала и родила [Иакову] сына, и сказала [Рахиль]: снял Бог позор мой.
\vs Gen 30:24 И нарекла ему имя: Иосиф, сказав: Господь даст мне и другого сына.
\rsbpar\vs Gen 30:25 После того, как Рахиль родила Иосифа, Иаков сказал Лавану: отпусти меня, и пойду я в свое место, и в свою землю;
\vs Gen 30:26 отдай [мне] жен моих и детей моих, за которых я служил тебе, и я пойду, ибо ты знаешь службу мою, какую я служил тебе.
\vs Gen 30:27 И сказал ему Лаван: о, если бы я нашел благоволение пред очами твоими! я примечаю, что за тебя Господь благословил меня.
\vs Gen 30:28 И сказал: назначь себе награду от меня, и я дам [тебе].
\vs Gen 30:29 И сказал ему [Иаков]: ты знаешь, как я служил тебе, и каков стал скот твой при мне;
\vs Gen 30:30 ибо мало было у тебя до меня, а стало много; Господь благословил тебя с приходом моим; когда же я буду работать для своего дома?
\vs Gen 30:31 И сказал [ему Лаван]: что дать тебе? Иаков сказал [ему]: не давай мне ничего. Если только сделаешь мне, чт\acc{о} я скажу, то я опять буду пасти и стеречь овец твоих.
\vs Gen 30:32 Я пройду сегодня по всему \bibemph{стаду} овец твоих; отдели из него всякий скот с крапинами и с пятнами, всякую скотину черную из овец, также с пятнами и с крапинами из коз. \bibemph{Такой скот} будет наградою мне [и будет мой].
\vs Gen 30:33 И будет говорить за меня пред тобою справедливость моя в следующее время, когда придешь посмотреть награду мою. Всякая из коз не с крапинами и не с пятнами, и из овец не черная, краденое это у меня.
\vs Gen 30:34 Лаван сказал [ему]: хорошо, пусть будет по твоему слову.
\vs Gen 30:35 И отделил в тот день козлов пестрых и с пятнами, и всех коз с крапинами и с пятнами, всех, на которых было \bibemph{несколько} белого, и всех черных овец, и отдал на руки сыновьям своим;
\vs Gen 30:36 и назначил расстояние между собою и между Иаковом на три дня пути. Иаков же пас остальной мелкий скот Лаванов.
\vs Gen 30:37 И взял Иаков свежих прутьев тополевых, миндальных и яворовых, и вырезал на них [Иаков] белые полосы, сняв кору до белизны, которая на прутьях,
\vs Gen 30:38 и положил прутья с нарезкою перед скотом в водопойных корытах, куда скот приходил пить, и где, приходя пить, зачинал пред прутьями.
\vs Gen 30:39 И зачинал скот пред прутьями, и рождался скот пестрый, и с крапинами, и с пятнами.
\vs Gen 30:40 И отделял Иаков ягнят и ставил скот лицем к пестрому и всему черному скоту Лаванову; и держал свои стада особо и не ставил их вместе со скотом Лавана.
\vs Gen 30:41 Каждый раз, когда зачинал скот крепкий, Иаков клал прутья в корытах пред глазами скота, чтобы он зачинал пред прутьями.
\vs Gen 30:42 А когда зачинал скот слабый, тогда он не клал. И доставался слабый \bibemph{скот} Лавану, а крепкий Иакову.
\vs Gen 30:43 И сделался этот человек весьма, весьма богатым, и было у него множество мелкого скота [и крупного скота], и рабынь, и рабов, и верблюдов, и ослов.
\vs Gen 31:1 И услышал [Иаков] слова сынов Лавановых, которые говорили: Иаков завладел всем, что было у отца нашего, и из имения отца нашего составил все богатство сие.
\vs Gen 31:2 И увидел Иаков лице Лавана, и вот, оно не таково к нему, как было вчера и третьего дня.
\vs Gen 31:3 И сказал Господь Иакову: возвратись в землю отцов твоих и на родину твою; и Я буду с тобою.
\vs Gen 31:4 И послал Иаков, и призвал Рахиль и Лию в поле, к \bibemph{стаду} мелкого скота своего,
\vs Gen 31:5 и сказал им: я вижу лице отца вашего, что оно ко мне не таково, как было вчера и третьего дня; но Бог отца моего был со мною;
\vs Gen 31:6 вы сами знаете, что я всеми силами служил отцу вашему,
\vs Gen 31:7 а отец ваш обманывал меня и раз десять переменял награду мою; но Бог не попустил ему сделать мне зло.
\vs Gen 31:8 Когда сказал он, что \bibemph{скот} с крапинами будет тебе в награду, то скот весь родил с крапинами. А когда он сказал: пестрые будут тебе в награду, то скот весь и родил пестрых.
\vs Gen 31:9 И отнял Бог [весь] скот у отца вашего и дал [его] мне.
\vs Gen 31:10 Однажды в такое время, когда скот зачинает, я взглянул и увидел во сне, и вот козлы [и овны], поднявшиеся на скот [на коз и овец] пестрые, с крапинами и пятнами.
\vs Gen 31:11 Ангел Божий сказал мне во сне: Иаков! Я сказал: вот я.
\vs Gen 31:12 Он сказал: возведи очи твои и посмотри: все козлы [и овны], поднявшиеся на скот [на коз и овец], пестрые, с крапинами и с пятнами, ибо Я вижу все, что Лаван делает с тобою;
\vs Gen 31:13 Я Бог [явившийся тебе] в Вефиле, где ты возлил елей на памятник и где ты дал Мне обет; теперь встань, выйди из земли сей и возвратись в землю родины твоей [и Я буду с тобою].
\vs Gen 31:14 Рахиль и Лия сказали ему в ответ: есть ли еще нам доля и наследство в доме отца нашего?
\vs Gen 31:15 не за чужих ли он нас почитает? ибо он продал нас и съел даже серебро наше;
\vs Gen 31:16 посему все [имение и] богатство, которое Бог отнял у отца нашего, есть наше и детей наших; итак делай все, что Бог сказал тебе.
\rsbpar\vs Gen 31:17 И встал Иаков, и посадил детей своих и жен своих на верблюдов,
\vs Gen 31:18 и взял с собою весь скот свой и все богатство свое, которое приобрел, скот собственный его, который он приобрел в Месопотамии, [и все свое,] чтобы идти к Исааку, отцу своему, в землю Ханаанскую.
\vs Gen 31:19 И как Лаван пошел стричь скот свой, то Рахиль похитила идолов, которые были у отца ее.
\vs Gen 31:20 Иаков же похитил сердце у Лавана Арамеянина, потому что не известил его, что удаляется.
\vs Gen 31:21 И ушел со всем, что у него было; и, встав, перешел реку и направился к горе Галаад.
\vs Gen 31:22 На третий день сказали Лавану [Арамеянину], что Иаков ушел.
\vs Gen 31:23 Тогда он взял с собою [сынов и] родственников своих, и гнался за ним семь дней, и догнал его на горе Галаад.
\vs Gen 31:24 И пришел Бог к Лавану Арамеянину ночью во сне и сказал ему: берегись, не говори Иакову ни доброго, ни худого.
\vs Gen 31:25 И догнал Лаван Иакова; Иаков же поставил шатер свой на горе, и Лаван со сродниками своими поставил на горе Галаад.
\vs Gen 31:26 И сказал Лаван Иакову: что ты сделал? для чего ты обманул меня, и увел дочерей моих, как плененных оружием?
\vs Gen 31:27 зачем ты убежал тайно, и укрылся от меня, и не сказал мне? я отпустил бы тебя с веселием и с песнями, с тимпаном и с гуслями;
\vs Gen 31:28 ты не позволил мне даже поцеловать внуков моих и дочерей моих; безрассудно ты сделал.
\vs Gen 31:29 Есть в руке моей сила сделать вам зло; но Бог отца вашего вчера говорил ко мне и сказал: берегись, не говори Иакову ни хорошего, ни худого.
\vs Gen 31:30 Но пусть бы ты ушел, потому что ты нетерпеливо захотел быть в доме отца твоего,~--- зачем ты украл богов моих?
\vs Gen 31:31 Иаков отвечал Лавану и сказал: \bibemph{я} боялся, ибо я думал, не отнял бы ты у меня дочерей своих [и всего моего].
\vs Gen 31:32 [И сказал Иаков:] у кого найдешь богов твоих, тот не будет жив; при родственниках наших узнавай, что [есть твоего] у меня, и возьми себе. [Но он ничего у него не узнал.] Иаков не знал, что Рахиль [жена его] украла их.
\vs Gen 31:33 И ходил Лаван в шатер Иакова, и в шатер Лии, и в шатер двух рабынь, [и обыскивал,] но не нашел. И, выйдя из шатра Лии, вошел в шатер Рахили.
\vs Gen 31:34 Рахиль же взяла идолов, и положила их под верблюжье седло и села на них. И обыскал Лаван весь шатер; но не нашел.
\vs Gen 31:35 Она же сказала отцу своему: да не прогневается господин мой, что я не могу встать пред тобою, ибо у меня обыкновенное женское. И [Лаван] искал [во всем шатре], но не нашел идолов.
\vs Gen 31:36 Иаков рассердился и вступил в спор с Лаваном. И начал Иаков говорить и сказал Лавану: какая вина моя, какой грех мой, что ты преследуешь меня?
\vs Gen 31:37 ты осмотрел у меня все вещи [в доме моем], что нашел ты из всех вещей твоего дома? покажи здесь пред родственниками моими и пред родственниками твоими; пусть они рассудят между нами обоими.
\vs Gen 31:38 Вот, двадцать лет я \bibemph{был} у тебя; овцы твои и козы твои не выкидывали; овнов стада твоего я не ел;
\vs Gen 31:39 растерзанного зверем я не приносил к тебе, это был мой убыток; ты с меня взыскивал, днем ли что пропадало, ночью ли пропадало;
\vs Gen 31:40 я томился днем от жара, а ночью от стужи, и сон мой убегал от глаз моих.
\vs Gen 31:41 Таковы мои двадцать лет в доме твоем. Я служил тебе четырнадцать лет за двух дочерей твоих и шесть лет за скот твой, а ты десять раз переменял награду мою.
\vs Gen 31:42 Если бы не был со мною Бог отца моего, Бог Авраама и страх Исаака, ты бы теперь отпустил меня ни с чем. Бог увидел бедствие мое и труд рук моих и вступился \bibemph{за меня} вчера.
\vs Gen 31:43 И отвечал Лаван и сказал Иакову: дочери~--- мои дочери; дети~--- мои дети; скот~--- мой скот, и все, что ты видишь, это мое: могу ли я что сделать теперь с дочерями моими и с детьми их, которые рождены ими?
\vs Gen 31:44 Теперь заключим союз я и ты, и это будет свидетельством между мною и тобою. [При сем Иаков сказал ему: вот, с нами нет никого; смотри, Бог свидетель между мною и тобою.]
\rsbpar\vs Gen 31:45 И взял Иаков камень и поставил его памятником.
\vs Gen 31:46 И сказал Иаков родственникам своим: наберите камней. Они взяли камни, и сделали холм, и ели [и пили] там на холме. [И сказал ему Лаван: холм сей свидетель сегодня между мною и тобою.]
\vs Gen 31:47 И назвал его Лаван: Иегар-Сагадуфа; а Иаков назвал его Галаадом.
\vs Gen 31:48 И сказал Лаван [Иакову]: сегодня этот холм [и памятник, который я поставил,] между мною и тобою свидетель. Посему и наречено ему имя: Галаад,
\vs Gen 31:49 \bibemph{также}: Мицпа, оттого, что Лаван сказал: да надзирает Господь надо мною и над тобою, когда мы скроемся друг от друга;
\vs Gen 31:50 если ты будешь худо поступать с дочерями моими, или если возьмешь жен сверх дочерей моих, то, хотя нет человека между нами, [который бы видел,] но смотри, Бог свидетель между мною и между тобою.
\vs Gen 31:51 И сказал Лаван Иакову: вот холм сей и вот памятник, который я поставил между мною и тобою;
\vs Gen 31:52 этот холм свидетель, и этот памятник свидетель, что ни я не перейду к тебе за этот холм, ни ты не перейдешь ко мне за этот холм и за этот памятник, для зла;
\vs Gen 31:53 Бог Авраамов и Бог Нахоров да судит между нами, Бог отца их. Иаков поклялся страхом отца своего Исаака.
\vs Gen 31:54 И заколол Иаков жертву на горе и позвал родственников своих есть хлеб; и они ели хлеб [и пили] и ночевали на горе.
\vs Gen 31:55 И встал Лаван рано утром и поцеловал внуков своих и дочерей своих, и благословил их. И пошел и возвратился Лаван в свое место.
\vs Gen 32:1 А Иаков пошел путем своим. [И, взглянув, увидел ополчение Божие ополчившееся.] И встретили его Ангелы Божии.
\vs Gen 32:2 Иаков, увидев их, сказал: это ополчение Божие. И нарек имя месту тому: Маханаим.
\vs Gen 32:3 И послал Иаков пред собою вестников к брату своему Исаву в землю Сеир, в область Едом,
\vs Gen 32:4 и приказал им, сказав: так скажите господину моему Исаву: вот что говорит раб твой Иаков: я жил у Лавана и прожил доныне;
\vs Gen 32:5 и есть у меня волы и ослы и мелкий скот, и рабы и рабыни; и я послал известить \bibemph{о себе} господина моего [Исава], дабы приобрести [рабу твоему] благоволение пред очами твоими.
\vs Gen 32:6 И возвратились вестники к Иакову и сказали: мы ходили к брату твоему Исаву; он идет навстречу тебе, и с ним четыреста человек.
\vs Gen 32:7 Иаков очень испугался и смутился; и разделил людей, бывших с ним, и скот мелкий и крупный и верблюдов на два стана.
\vs Gen 32:8 И сказал [Иаков]: если Исав нападет на один стан и побьет его, то остальной стан может спастись.
\vs Gen 32:9 И сказал Иаков: Боже отца моего Авраама и Боже отца моего Исаака, Господи [Боже], сказавший мне: возвратись в землю твою, на родину твою, и Я буду благотворить тебе!
\vs Gen 32:10 Недостоин я всех милостей и всех благодеяний, которые Ты сотворил рабу Твоему, ибо я с посохом моим перешел этот Иордан, а теперь у меня два стана.
\vs Gen 32:11 Избавь меня от руки брата моего, от руки Исава, ибо я боюсь его, чтобы он, придя, не убил меня [и] матери с детьми.
\vs Gen 32:12 Ты сказал: Я буду благотворить тебе и сделаю потомство твое, как песок морской, которого не исчислить от множества.
\vs Gen 32:13 И ночевал там \bibemph{Иаков} в ту ночь. И взял из того, что у него было, [и послал] в подарок Исаву, брату своему:
\vs Gen 32:14 двести коз, двадцать козлов, двести овец, двадцать овнов,
\vs Gen 32:15 тридцать верблюдиц дойных с жеребятами их, сорок коров, десять волов, двадцать ослиц, десять ослов.
\vs Gen 32:16 И дал в руки рабам своим каждое стадо особо и сказал рабам своим: пойдите предо мною и оставляйте расстояние от стада до стада.
\vs Gen 32:17 И приказал первому, сказав: когда брат мой Исав встретится тебе и спросит тебя, говоря: чей ты? и куда идешь? и чье это \bibemph{стадо} [идет] пред тобою?
\vs Gen 32:18 то скажи: раба твоего Иакова; это подарок, посланный господину моему Исаву; вот, и сам он за нами [идет].
\vs Gen 32:19 То же [что первому] приказал он и второму, и третьему, и всем, которые шли за стадами, говоря: так скажите Исаву, когда встретите его;
\vs Gen 32:20 и скажите: вот, и раб твой Иаков [идет] за нами. Ибо он сказал \bibemph{сам в себе}: умилостивлю его дарами, которые идут предо мною, и потом увижу лице его; может быть, и примет меня.
\vs Gen 32:21 И пошли дары пред ним, а он ту ночь ночевал в стане.
\vs Gen 32:22 И встал в ту ночь, и, взяв двух жен своих и двух рабынь своих, и одиннадцать сынов своих, перешел через Иавок вброд;
\vs Gen 32:23 и, взяв их, перевел через поток, и перевел все, что у него \bibemph{было}.
\rsbpar\vs Gen 32:24 И остался Иаков один. И боролся Некто с ним до появления зари;
\vs Gen 32:25 и, увидев, что не одолевает его, коснулся состава бедра его и повредил состав бедра у Иакова, когда он боролся с Ним.
\vs Gen 32:26 И сказал [ему]: отпусти Меня, ибо взошла заря. Иаков сказал: не отпущу Тебя, пока не благословишь меня.
\vs Gen 32:27 И сказал: как имя твое? Он сказал: Иаков.
\vs Gen 32:28 И сказал [ему]: отныне имя тебе будет не Иаков, а Израиль, ибо ты боролся с Богом, и человеков одолевать будешь.
\vs Gen 32:29 Спросил и Иаков, говоря: скажи [мне] имя Твое. И Он сказал: на что ты спрашиваешь о имени Моем? [оно чудно.] И благословил его там.
\vs Gen 32:30 И нарек Иаков имя месту тому: Пенуэл; ибо, \bibemph{говорил он}, я видел Бога лицем к лицу, и сохранилась душа моя.
\vs Gen 32:31 И взошло солнце, когда он проходил Пенуэл; и хромал он на бедро свое.
\vs Gen 32:32 Поэтому и доныне сыны Израилевы не едят жилы, которая на составе бедра, потому что \bibemph{Боровшийся} коснулся жилы на составе бедра Иакова.
\vs Gen 33:1 Взглянул Иаков и увидел, и вот, идет Исав, [брат его,] и с ним четыреста человек. И разделил [Иаков] детей Лии, Рахили и двух служанок.
\vs Gen 33:2 И поставил [двух] служанок и детей их впереди, Лию и детей ее за ними, а Рахиль и Иосифа позади.
\vs Gen 33:3 А сам пошел пред ними и поклонился до земли семь раз, подходя к брату своему.
\vs Gen 33:4 И побежал Исав к нему навстречу и обнял его, и пал на шею его и целовал его, и плакали [оба].
\vs Gen 33:5 И взглянул [Исав] и увидел жен и детей и сказал: кто это у тебя? \bibemph{Иаков} сказал: дети, которых Бог даровал рабу твоему.
\vs Gen 33:6 И подошли служанки и дети их и поклонились;
\vs Gen 33:7 подошла и Лия и дети ее и поклонились; наконец подошли Иосиф и Рахиль и поклонились.
\vs Gen 33:8 И сказал Исав: для чего у тебя это множество, которое я встретил? И сказал Иаков: дабы [рабу твоему] приобрести благоволение в очах господина моего.
\vs Gen 33:9 Исав сказал: у меня много, брат мой; пусть будет твое у тебя.
\vs Gen 33:10 Иаков сказал: нет, если я приобрел благоволение в очах твоих, прими дар мой от руки моей, ибо я увидел лице твое, как бы кто увидел лице Божие, и ты был благосклонен ко мне;
\vs Gen 33:11 прими благословение мое, которое я принес тебе, потому что Бог даровал мне, и есть у меня всё. И упросил его, и тот взял
\vs Gen 33:12 и сказал: поднимемся и пойдем; и я пойду пред тобою.
\vs Gen 33:13 Иаков сказал ему: господин мой знает, что дети нежны, а мелкий и крупный скот у меня дойный: если погнать его один день, то помрет весь скот;
\vs Gen 33:14 пусть господин мой пойдет впереди раба своего, а я пойду медленно, как пойдет скот, который предо мною, и как пойдут дети, и приду к господину моему в Сеир.
\vs Gen 33:15 Исав сказал: оставлю я с тобою \bibemph{несколько} из людей, которые при мне. Иаков сказал: к чему это? только бы мне приобрести благоволение в очах господина моего!
\vs Gen 33:16 И возвратился Исав в тот же день путем своим в Сеир.
\vs Gen 33:17 А Иаков двинулся в Сокхоф, и построил себе дом, и для скота своего сделал шалаши. От сего он нарек имя месту: Сокхоф.
\vs Gen 33:18 Иаков, возвратившись из Месопотамии, благополучно пришел в город Сихем, который в земле Ханаанской, и расположился пред городом.
\vs Gen 33:19 И купил часть поля, на котором раскинул шатер свой, у сынов Еммора, отца Сихемова, за сто монет.
\vs Gen 33:20 И поставил там жертвенник, и призвал имя Господа Бога Израилева.
\vs Gen 34:1 Дина, дочь Лии, которую она родила Иакову, вышла посмотреть на дочерей земли той.
\vs Gen 34:2 И увидел ее Сихем, сын Еммора Евеянина, князя земли той, и взял ее, и спал с нею, и сделал ей насилие.
\vs Gen 34:3 И прилепилась душа его к Дине, дочери Иакова, и он полюбил девицу и говорил по сердцу девицы.
\vs Gen 34:4 И сказал Сихем Еммору, отцу своему, говоря: возьми мне эту девицу в жену.
\vs Gen 34:5 Иаков слышал, что [сын Емморов] обесчестил Дину, дочь его, но как сыновья его были со скотом его в поле, то Иаков молчал, пока не пришли они.
\vs Gen 34:6 И вышел Еммор, отец Сихемов, к Иакову, поговорить с ним.
\vs Gen 34:7 Сыновья же Иакова пришли с поля, и когда услышали, то огорчились мужи те и воспылали гневом, потому что бесчестие сделал он Израилю, переспав с дочерью Иакова, а так не надлежало делать.
\vs Gen 34:8 Еммор стал говорить им, и сказал: Сихем, сын мой, прилепился душею к дочери вашей; дайте же ее в жену ему;
\vs Gen 34:9 породнитесь с нами; отдавайте за нас дочерей ваших, а наших дочерей берите себе [за сыновей ваших];
\vs Gen 34:10 и живите с нами; земля сия [пространна] пред вами, живите и промышляйте на ней и приобретайте ее во владение.
\vs Gen 34:11 Сихем же сказал отцу ее и братьям ее: только бы мне найти благоволение в очах ваших, я дам, что ни скажете мне;
\vs Gen 34:12 назначьте самое большое вено и дары; я дам, что ни скажете мне, только отдайте мне девицу в жену.
\vs Gen 34:13 И отвечали сыновья Иакова Сихему и Еммору, отцу его, с лукавством; а говорили так потому, что он обесчестил Дину, сестру их;
\vs Gen 34:14 и сказали им [Симеон и Левий, братья Дины, сыновья Лиины]: не можем этого сделать, выдать сестру нашу за человека, который необрезан, ибо это бесчестно для нас;
\vs Gen 34:15 только на том условии мы согласимся с вами [и поселимся у вас], если вы будете как мы, чтобы и у вас весь мужеский пол был обрезан;
\vs Gen 34:16 и будем отдавать за вас дочерей наших и брать за себя ваших дочерей, и будем жить с вами, и составим один народ;
\vs Gen 34:17 а если не послушаетесь нас в том, чтобы обрезаться, то мы возьмем дочь нашу и удалимся.
\vs Gen 34:18 И понравились слова сии Еммору и Сихему, сыну Емморову.
\vs Gen 34:19 Юноша не умедлил исполнить это, потому что любил дочь Иакова. А он более всех уважаем был из дома отца своего.
\vs Gen 34:20 И пришел Еммор и Сихем, сын его, к воротам города своего, и стали говорить жителям города своего и сказали:
\vs Gen 34:21 сии люди мирны с нами; пусть они селятся на земле и промышляют на ней; земля же вот пространна пред ними. Станем брать дочерей их себе в жены и наших дочерей выдавать за них.
\vs Gen 34:22 Только на том условии сии люди соглашаются жить с нами и быть одним народом, чтобы и у нас обрезан был весь мужеский пол, как они обрезаны.
\vs Gen 34:23 Не для нас ли стада их, и имение их, и весь скот их? Только [в том] согласимся с ними, и будут жить с нами.
\vs Gen 34:24 И послушались Еммора и Сихема, сына его, все выходящие из ворот города его: и обрезан был весь мужеский пол,~--- все выходящие из ворот города его.
\vs Gen 34:25 На третий день, когда они были в болезни, два сына Иакова, Симеон и Левий, братья Динины, взяли каждый свой меч, и смело напали на город, и умертвили весь мужеский пол;
\vs Gen 34:26 и самого Еммора и Сихема, сына его, убили мечом; и взяли Дину из дома Сихемова и вышли.
\vs Gen 34:27 Сыновья Иакова пришли к убитым и разграбили город за то, что обесчестили [Дину] сестру их.
\vs Gen 34:28 Они взяли мелкий и крупный скот их, и ослов их, и что ни было в городе, и что ни было в поле;
\vs Gen 34:29 и все богатство их, и всех детей их, и жен их взяли в плен, и разграбили всё, что было в [городе, и всё, что было в] домах.
\vs Gen 34:30 И сказал Иаков Симеону и Левию: вы возмутили меня, сделав меня ненавистным для [всех] жителей сей земли, для Хананеев и Ферезеев. У меня людей мало; соберутся против меня, поразят меня, и истреблен буду я и дом мой.
\vs Gen 34:31 Они же сказали: а разве можно поступать с сестрою нашею, как с блудницею!
\vs Gen 35:1 Бог сказал Иакову: встань, пойди в Вефиль и живи там, и устрой там жертвенник Богу, явившемуся тебе, когда ты бежал от лица Исава, брата твоего.
\vs Gen 35:2 И сказал Иаков дому своему и всем бывшим с ним: бросьте богов чужих, находящихся у вас, и очиститесь, и перемените одежды ваши;
\vs Gen 35:3 встанем и пойдем в Вефиль; там устрою я жертвенник Богу, Который услышал меня в день бедствия моего и был со мною [и хранил меня] в пути, которым я ходил.
\vs Gen 35:4 И отдали Иакову всех богов чужих, бывших в руках их, и серьги, бывшие в ушах у них, и закопал их Иаков под дубом, который близ Сихема. [И оставил их безвестными даже до нынешнего дня.]
\vs Gen 35:5 И отправились они [от Сихема]. И был ужас Божий на окрестных городах, и не преследовали сынов Иаковлевых.
\vs Gen 35:6 И пришел Иаков в Луз, что в земле Ханаанской, то есть в Вефиль, сам и все люди, бывшие с ним,
\vs Gen 35:7 и устроил там жертвенник, и назвал сие место: Эл-Вефиль, ибо тут явился ему Бог, когда он бежал от лица [Исава] брата своего.
\vs Gen 35:8 И умерла Девора, кормилица Ревеккина, и погребена ниже Вефиля под дубом, который и назвал \bibemph{Иаков} дубом плача.
\rsbpar\vs Gen 35:9 И явился Бог Иакову [в Лузе] по возвращении его из Месопотамии, и благословил его,
\vs Gen 35:10 и сказал ему Бог: имя твое Иаков; отныне ты не будешь называться Иаковом, но будет имя тебе: Израиль. И нарек ему имя: Израиль.
\vs Gen 35:11 И сказал ему Бог: Я Бог Всемогущий; плодись и умножайся; народ и множество народов будет от тебя, и цари произойдут из чресл твоих;
\vs Gen 35:12 землю, которую Я дал Аврааму и Исааку, Я дам тебе, и потомству твоему по тебе дам землю сию.
\vs Gen 35:13 И восшел от него Бог с места, на котором говорил ему.
\vs Gen 35:14 И поставил Иаков памятник на месте, на котором говорил ему [Бог], памятник каменный, и возлил на него возлияние, и возлил на него елей;
\vs Gen 35:15 и нарек Иаков имя месту, на котором Бог говорил ему: Вефиль.
\rsbpar\vs Gen 35:16 И отправились из Вефиля. [И раскинул он шатер свой за башнею Гадер.] И когда еще оставалось некоторое расстояние земли до Ефрафы, Рахиль родила, и роды ее были трудны.
\vs Gen 35:17 Когда же она страдала в родах, повивальная бабка сказала ей: не бойся, ибо и это тебе сын.
\vs Gen 35:18 И когда выходила из нее душа, ибо она умирала, то нарекла ему имя: Бенони. Но отец его назвал его Вениамином.
\vs Gen 35:19 И умерла Рахиль, и погребена на дороге в Ефрафу, то есть Вифлеем.
\vs Gen 35:20 Иаков поставил над гробом ее памятник. Это надгробный памятник Рахили до сего дня.
\vs Gen 35:21 И отправился [оттуда] Израиль и раскинул шатер свой за башнею Гадер.
\vs Gen 35:22 Во время пребывания Израиля в той стране, Рувим пошел и переспал с Валлою, наложницею отца своего [Иакова]. И услышал Израиль [и принял то с огорчением].\rsbpar Сынов же у Иакова было двенадцать.
\vs Gen 35:23 Сыновья Лии: первенец Иакова Рувим, \bibemph{по нем} Симеон, Левий, Иуда, Иссахар и Завулон.
\vs Gen 35:24 Сыновья Рахили: Иосиф и Вениамин.
\vs Gen 35:25 Сыновья Валлы, служанки Рахилиной: Дан и Неффалим.
\vs Gen 35:26 Сыновья Зелфы, служанки Лииной: Гад и Асир. Сии сыновья Иакова, родившиеся ему в Месопотамии.
\rsbpar\vs Gen 35:27 И пришел Иаков к Исааку, отцу своему, [ибо он был еще жив,] в Мамре, в Кириаф-Арбу, то есть Хеврон [в земле Ханаанской,] где странствовал Авраам и Исаак.
\vs Gen 35:28 И было дней [жизни] Исааковой сто восемьдесят лет.
\vs Gen 35:29 И испустил Исаак дух и умер, и приложился к народу своему, будучи стар и насыщен жизнью; и погребли его Исав и Иаков, сыновья его.
\vs Gen 36:1 Вот родословие Исава, он же Едом.
\vs Gen 36:2 Исав взял себе жен из дочерей Ханаанских: Аду, дочь Елона Хеттеянина, и Оливему, дочь Аны, сына Цивеона Евеянина,
\vs Gen 36:3 и Васемафу, дочь Измаила, сестру Наваиофа.
\vs Gen 36:4 Ада родила Исаву Елифаза, Васемафа родила Рагуила,
\vs Gen 36:5 Оливема родила Иеуса, Иеглома и Корея. Это сыновья Исава, родившиеся ему в земле Ханаанской.
\vs Gen 36:6 И взял Исав жен своих и сыновей своих, и дочерей своих, и всех людей дома своего, и [все] стада свои, и весь скот свой, и всё имение свое, которое он приобрел в земле Ханаанской, и пошел [Исав] в \bibemph{другую} землю от лица Иакова, брата своего,
\vs Gen 36:7 ибо имение их было так велико, что они не могли жить вместе, и земля странствования их не вмещала их, по множеству стад их.
\vs Gen 36:8 И поселился Исав на горе Сеир, Исав, он же Едом.
\rsbpar\vs Gen 36:9 И вот родословие Исава, отца Идумеев, на горе Сеир.
\vs Gen 36:10 Вот имена сынов Исава: Елифаз, сын Ады, жены Исавовой, и Рагуил, сын Васемафы, жены Исавовой.
\vs Gen 36:11 У Елифаза были сыновья: Феман, Омар, Цефо, Гафам и Кеназ.
\vs Gen 36:12 Фамна же была наложница Елифаза, сына Исавова, и родила Елифазу Амалика. Вот сыновья Ады, жены Исавовой.
\vs Gen 36:13 И вот сыновья Рагуила: Нахаф и Зерах, Шамма и Миза. Это сыновья Васемафы, жены Исавовой.
\vs Gen 36:14 И сии были сыновья Оливемы, дочери Аны, сына Цивеонова, жены Исавовой: она родила Исаву Иеуса, Иеглома и Корея.
\vs Gen 36:15 Вот старейшины сынов Исавовых. Сыновья Елифаза, первенца Исавова: старейшина Феман, старейшина Омар, старейшина Цефо, старейшина Кеназ,
\vs Gen 36:16 старейшина Корей, старейшина Гафам, старейшина Амалик. Сии старейшины Елифазовы в земле Едома; сии сыновья Ады.
\vs Gen 36:17 Сии сыновья Рагуила, сына Исавова: старейшина Нахаф, старейшина Зерах, старейшина Шамма, старейшина Миза. Сии старейшины Рагуиловы в земле Едома; сии сыновья Васемафы, жены Исавовой.
\vs Gen 36:18 Сии сыновья Оливемы, жены Исавовой: старейшина Иеус, старейшина Иеглом, старейшина Корей. Сии старейшины Оливемы, дочери Аны, жены Исавовой.
\vs Gen 36:19 Вот сыновья Исава, и вот старейшины их. Это Едом.
\rsbpar\vs Gen 36:20 Сии сыновья Сеира Хорреянина, жившие в земле той: Лотан, Шовал, Цивеон, Ана,
\vs Gen 36:21 Дишон, Эцер и Дишан. Сии старейшины Хорреев, сынов Сеира, в земле Едома.
\vs Gen 36:22 Сыновья Лотана были: Хори и Геман; а сестра у Лотана: Фамна.
\vs Gen 36:23 Сии сыновья Шовала: Алван, Манахаф, Эвал, Шефо и Онам.
\vs Gen 36:24 Сии сыновья Цивеона: Аиа и Ана. Это тот Ана, который нашел теплые воды в пустыне, когда пас ослов Цивеона, отца своего.
\vs Gen 36:25 Сии дети Аны: Дишон и Оливема, дочь Аны.
\vs Gen 36:26 Сии сыновья Дишона: Хемдан, Эшбан, Ифран и Херан.
\vs Gen 36:27 Сии сыновья Эцера: Билган, Зааван, [Укам] и Акан.
\vs Gen 36:28 Сии сыновья Дишана: Уц и Аран.
\vs Gen 36:29 Сии старейшины Хорреев: старейшина Лотан, старейшина Шовал, старейшина Цивеон, старейшина Ана,
\vs Gen 36:30 старейшина Дишон, старейшина Эцер, старейшина Дишан. Вот старейшины Хорреев, по старшинствам их в земле Сеир.
\rsbpar\vs Gen 36:31 Вот цари, царствовавшие в земле Едома, прежде царствования царей у сынов Израилевых:
\vs Gen 36:32 царствовал в Едоме Бела, сын Веоров, а имя городу его Дингава.
\vs Gen 36:33 И умер Бела, и воцарился по нем Иовав, сын Зераха, из Восоры.
\vs Gen 36:34 Умер Иовав, и воцарился по нем Хушам, из земли Феманитян.
\vs Gen 36:35 И умер Хушам, и воцарился по нем Гадад, сын Бедадов, который поразил Мадианитян на поле Моава; имя городу его Авиф.
\vs Gen 36:36 И умер Гадад, и воцарился по нем Самла из Масреки.
\vs Gen 36:37 И умер Самла, и воцарился по нем Саул из Реховофа, что при реке.
\vs Gen 36:38 И умер Саул, и воцарился по нем Баал-Ханан, сын Ахбора.
\vs Gen 36:39 И умер Баал-Ханан, сын Ахбора, и воцарился по нем Гадар [сын Варадов]; имя городу его Пау; имя жене его Мегетавеель, дочь Матреды, сына Мезагава.
\rsbpar\vs Gen 36:40 Сии имена старейшин Исавовых, по племенам их, по местам их, по именам их, [по народам их]: старейшина Фимна, старейшина Алва, старейшина Иетеф,
\vs Gen 36:41 старейшина Оливема, старейшина Эла, старейшина Пинон,
\vs Gen 36:42 старейшина Кеназ, старейшина Феман, старейшина Мивцар,
\vs Gen 36:43 старейшина Магдиил, старейшина Ирам. Вот старейшины Идумейские, по их селениям, в земле обладания их. Вот Исав, отец Идумеев.
\vs Gen 37:1 Иаков жил в земле странствования отца своего [Исаака], в земле Ханаанской.
\vs Gen 37:2 Вот житие Иакова. Иосиф, семнадцати лет, пас скот [отца своего] вместе с братьями своими, будучи отроком, с сыновьями Валлы и с сыновьями Зелфы, жен отца своего. И доводил Иосиф худые о них слухи до [Израиля] отца их.
\vs Gen 37:3 Израиль любил Иосифа более всех сыновей своих, потому что он был сын старости его,~--- и сделал ему разноцветную одежду.
\vs Gen 37:4 И увидели братья его, что отец их любит его более всех братьев его; и возненавидели его и не могли говорить с ним дружелюбно.
\vs Gen 37:5 И видел Иосиф сон, и рассказал [его] братьям своим: и они возненавидели его еще более.
\vs Gen 37:6 Он сказал им: выслушайте сон, который я видел:
\vs Gen 37:7 вот, мы вяжем снопы посреди поля; и вот, мой сноп встал и стал прямо; и вот, ваши снопы стали кругом и поклонились моему снопу.
\vs Gen 37:8 И сказали ему братья его: неужели ты будешь царствовать над нами? неужели будешь владеть нами? И возненавидели его еще более за сны его и за слова его.
\vs Gen 37:9 И видел он еще другой сон и рассказал его [отцу своему и] братьям своим, говоря: вот, я видел еще сон: вот, солнце и луна и одиннадцать звезд поклоняются мне.
\vs Gen 37:10 И он рассказал отцу своему и братьям своим; и побранил его отец его и сказал ему: что это за сон, который ты видел? неужели я и твоя мать, и твои братья придем поклониться тебе до земли?
\vs Gen 37:11 Братья его досадовали на него, а отец его заметил это слово.
\rsbpar\vs Gen 37:12 Братья его пошли пасти скот отца своего в Сихем.
\vs Gen 37:13 И сказал Израиль Иосифу: братья твои не пасут ли в Сихеме? пойди, я пошлю тебя к ним. Он отвечал ему: вот я.
\vs Gen 37:14 [Израиль] сказал ему: пойди, посмотри, здоровы ли братья твои и цел ли скот, и принеси мне ответ. И послал его из долины Хевронской; и он пришел в Сихем.
\vs Gen 37:15 И нашел его некто блуждающим в поле, и спросил его тот человек, говоря: чего ты ищешь?
\vs Gen 37:16 Он сказал: я ищу братьев моих; скажи мне, где они пасут?
\vs Gen 37:17 И сказал тот человек: они ушли отсюда, ибо я слышал, как они говорили: пойдем в Дофан. И пошел Иосиф за братьями своими и нашел их в Дофане.
\vs Gen 37:18 И увидели они его издали, и прежде нежели он приблизился к ним, стали умышлять против него, чтобы убить его.
\vs Gen 37:19 И сказали друг другу: вот, идет сновидец;
\vs Gen 37:20 пойдем теперь, и убьем его, и бросим его в какой-нибудь ров, и скажем, что хищный зверь съел его; и увидим, что будет из его снов.
\vs Gen 37:21 И услышал \bibemph{сие} Рувим и избавил его от рук их, сказав: не убьем его.
\vs Gen 37:22 И сказал им Рувим: не проливайте крови; бросьте его в ров, который в пустыне, а руки не налагайте на него. \bibemph{Сие говорил он} [с тем намерением], чтобы избавить его от рук их и возвратить его к отцу его.
\vs Gen 37:23 Когда Иосиф пришел к братьям своим, они сняли с Иосифа одежду его, одежду разноцветную, которая была на нем,
\vs Gen 37:24 и взяли его и бросили его в ров; ров же тот был пуст; воды в нем не было.
\vs Gen 37:25 И сели они есть хлеб, и, взглянув, увидели, вот, идет из Галаада караван Измаильтян, и верблюды их несут стираксу, бальзам и ладан: идут они отвезти это в Египет.
\vs Gen 37:26 И сказал Иуда братьям своим: что пользы, если мы убьем брата нашего и скроем кровь его?
\vs Gen 37:27 Пойдем, продадим его Измаильтянам, а руки наши да не будут на нем, ибо он брат наш, плоть наша. Братья его послушались
\vs Gen 37:28 и, когда проходили купцы Мадиамские, вытащили Иосифа изо рва и продали Иосифа Измаильтянам за двадцать сребреников; а они отвели Иосифа в Египет.
\vs Gen 37:29 Рувим же пришел опять ко рву; и вот, нет Иосифа во рве. И разодрал он одежды свои,
\vs Gen 37:30 и возвратился к братьям своим, и сказал: отрока нет, а я, куда я денусь?
\vs Gen 37:31 И взяли одежду Иосифа, и закололи козла, и вымарали одежду кровью;
\vs Gen 37:32 и послали разноцветную одежду, и доставили к отцу своему, и сказали: мы это нашли; посмотри, сына ли твоего эта одежда, или нет.
\vs Gen 37:33 Он узнал ее и сказал: \bibemph{это} одежда сына моего; хищный зверь съел его; верно, растерзан Иосиф.
\vs Gen 37:34 И разодрал Иаков одежды свои, и возложил вретище на чресла свои, и оплакивал сына своего многие дни.
\vs Gen 37:35 И собрались все сыновья его и все дочери его, чтобы утешить его; но он не хотел утешиться и сказал: с печалью сойду к сыну моему в преисподнюю. Так оплакивал его отец его.
\vs Gen 37:36 Мадианитяне же продали его в Египте Потифару, царедворцу фараонову, начальнику телохранителей.
\vs Gen 38:1 В то время Иуда отошел от братьев своих и поселился близ одного Одолламитянина, которому имя: Хира.
\vs Gen 38:2 И увидел там Иуда дочь одного Хананеянина, которому имя: Шуа; и взял ее и вошел к ней.
\vs Gen 38:3 Она зачала и родила сына; и он нарек ему имя: Ир.
\vs Gen 38:4 И зачала опять, и родила сына, и нарекла ему имя: Онан.
\vs Gen 38:5 И еще родила сына [третьего] и нарекла ему имя: Шела. Иуда был в Хезиве, когда она родила его.
\vs Gen 38:6 И взял Иуда жену Иру, первенцу своему; имя ей Фамарь.
\vs Gen 38:7 Ир, первенец Иудин, был неугоден пред очами Господа, и умертвил его Господь.
\vs Gen 38:8 И сказал Иуда Онану: войди к жене брата твоего, женись на ней, как деверь, и восстанови семя брату твоему.
\vs Gen 38:9 Онан знал, что семя будет не ему, и потому, когда входил к жене брата своего, изливал [семя] на землю, чтобы не дать семени брату своему.
\vs Gen 38:10 Зло было пред очами Господа то, что он делал; и Он умертвил и его.
\vs Gen 38:11 И сказал Иуда Фамари, невестке своей [по смерти двух сыновей своих]: живи вдовою в доме отца твоего, пока подрастет Шела, сын мой. Ибо он сказал [в уме своем]: не умер бы и он подобно братьям его. Фамарь пошла и стала жить в доме отца своего.
\rsbpar\vs Gen 38:12 Прошло много времени, и умерла дочь Шуи, жена Иудина. Иуда, утешившись, пошел в Фамну к стригущим скот его, сам и Хира, друг его, Одолламитянин.
\vs Gen 38:13 И уведомили Фамарь, говоря: вот, свекор твой идет в Фамну стричь скот свой.
\vs Gen 38:14 И сняла она с себя одежду вдовства своего, покрыла себя покрывалом и, закрывшись, села у ворот Енаима, что на дороге в Фамну. Ибо видела, что Шела вырос, и она не дана ему в жену.
\vs Gen 38:15 И увидел ее Иуда и почел ее за блудницу, потому что она закрыла лице свое. [И не узнал ее.]
\vs Gen 38:16 Он поворотил к ней и сказал: войду я к тебе. Ибо не знал, что это невестка его. Она сказала: что ты дашь мне, если войдешь ко мне?
\vs Gen 38:17 Он сказал: я пришлю тебе козленка из стада [моего]. Она сказала: дашь ли ты мне залог, пока пришлешь?
\vs Gen 38:18 Он сказал: какой дать тебе залог? Она сказала: печать твою, и перевязь твою, и трость твою, которая в руке твоей. И дал он ей и вошел к ней; и она зачала от него.
\vs Gen 38:19 И, встав, пошла, сняла с себя покрывало свое и оделась в одежду вдовства своего.
\vs Gen 38:20 Иуда же послал козленка чрез друга своего Одолламитянина, чтобы взять залог из руки женщины, но он не нашел ее.
\vs Gen 38:21 И спросил жителей того места, говоря: где блудница, \bibemph{которая была} в Енаиме при дороге? Но они сказали: здесь не было блудницы.
\vs Gen 38:22 И возвратился он к Иуде и сказал: я не нашел ее; да и жители места того сказали: здесь не было блудницы.
\vs Gen 38:23 Иуда сказал: пусть она возьмет себе, чтобы только не стали над нами смеяться; вот, я посылал этого козленка, но ты не нашел ее.
\vs Gen 38:24 Прошло около трех месяцев, и сказали Иуде, говоря: Фамарь, невестка твоя, впала в блуд, и вот, она беременна от блуда. Иуда сказал: выведите ее, и пусть она будет сожжена.
\vs Gen 38:25 Но когда повели ее, она послала сказать свекру своему: я беременна от того, чьи эти вещи. И сказала: узнавай, чья эта печать и перевязь и трость.
\vs Gen 38:26 Иуда узнал и сказал: она правее меня, потому что я не дал ее Шеле, сыну моему. И не познавал ее более.
\vs Gen 38:27 Во время родов ее оказалось, что близнецы в утробе ее.
\vs Gen 38:28 И во время родов ее показалась рука [одного]; и взяла повивальная бабка и навязала ему на руку красную нить, сказав: этот вышел первый.
\vs Gen 38:29 Но он возвратил руку свою; и вот, вышел брат его. И она сказала: как ты расторг себе преграду? И наречено ему имя: Фарес.
\vs Gen 38:30 Потом вышел брат его с красной нитью на руке. И наречено ему имя: Зара.
\vs Gen 39:1 Иосиф же отведен был в Египет, и купил его из рук Измаильтян, приведших его туда, Египтянин Потифар, царедворец фараонов, начальник телохранителей.
\vs Gen 39:2 И был Господь с Иосифом: он был успешен в делах и жил в доме господина своего, Египтянина.
\vs Gen 39:3 И увидел господин его, что Господь с ним и что всему, что он делает, Господь в руках его дает успех.
\vs Gen 39:4 И снискал Иосиф благоволение в очах его и служил ему. И он поставил его над домом своим, и все, что имел, отдал на руки его.
\vs Gen 39:5 И с того времени, как он поставил его над домом своим и над всем, что имел, Господь благословил дом Египтянина ради Иосифа, и было благословение Господне на всем, что имел он в доме и в поле [его].
\vs Gen 39:6 И оставил он все, что имел, в руках Иосифа и не знал при нем ничего, кроме хлеба, который он ел.\rsbpar Иосиф же был красив станом и красив лицем.
\vs Gen 39:7 И обратила взоры на Иосифа жена господина его и сказала: спи со мною.
\vs Gen 39:8 Но он отказался и сказал жене господина своего: вот, господин мой не знает при мне ничего в доме, и все, что имеет, отдал в мои руки;
\vs Gen 39:9 нет больше меня в доме сем; и он не запретил мне ничего, кроме тебя, потому что ты жена ему; как же сделаю я сие великое зло и согрешу пред Богом?
\vs Gen 39:10 Когда так она ежедневно говорила Иосифу, а он не слушался ее, чтобы спать с нею и быть с нею,
\vs Gen 39:11 случилось в один день, что он вошел в дом делать дело свое, а никого из домашних тут в доме не было;
\vs Gen 39:12 она схватила его за одежду его и сказала: ложись со мной. Но он, оставив одежду свою в руках ее, побежал и выбежал вон.
\vs Gen 39:13 Она же, увидев, что он оставил одежду свою в руках ее и побежал вон,
\vs Gen 39:14 кликнула домашних своих и сказала им так: посмотрите, он привел к нам Еврея ругаться над нами. Он пришел ко мне, чтобы лечь со мною, но я закричала громким голосом,
\vs Gen 39:15 и он, услышав, что я подняла вопль и закричала, оставил у меня одежду свою, и побежал, и выбежал вон.
\vs Gen 39:16 И оставила одежду его у себя до прихода господина его в дом свой.
\vs Gen 39:17 И пересказала ему те же слова, говоря: раб Еврей, которого ты привел к нам, приходил ко мне ругаться надо мною [и говорил мне: лягу я с тобою],
\vs Gen 39:18 но, когда [услышал, что] я подняла вопль и закричала, он оставил у меня одежду свою и убежал вон.
\vs Gen 39:19 Когда господин его услышал слова жены своей, которые она сказала ему, говоря: так поступил со мною раб твой, то воспылал гневом;
\vs Gen 39:20 и взял Иосифа господин его и отдал его в темницу, где заключены узники царя. И был он там в темнице.
\rsbpar\vs Gen 39:21 И Господь был с Иосифом, и простер к нему милость, и даровал ему благоволение в очах начальника темницы.
\vs Gen 39:22 И отдал начальник темницы в руки Иосифу всех узников, находившихся в темнице, и во всем, что они там ни делали, он был распорядителем.
\vs Gen 39:23 Начальник темницы и не смотрел ни за чем, что было у него в руках, потому что Господь был с \bibemph{Иосифом}, и во всем, что он делал, Господь давал успех.
\vs Gen 40:1 После сего виночерпий царя Египетского и хлебодар провинились пред господином своим, царем Египетским.
\vs Gen 40:2 И прогневался фараон на двух царедворцев своих, на главного виночерпия и на главного хлебодара,
\vs Gen 40:3 и отдал их под стражу в дом начальника телохранителей, в темницу, в место, где заключен был Иосиф.
\vs Gen 40:4 Начальник телохранителей приставил к ним Иосифа, и он служил им. И пробыли они под стражею несколько времени.
\rsbpar\vs Gen 40:5 Однажды виночерпию и хлебодару царя Египетского, заключенным в темнице, виделись сны, каждому свой сон, обоим в одну ночь, каждому сон особенного значения.
\vs Gen 40:6 И пришел к ним Иосиф поутру, увидел их, и вот, они в смущении.
\vs Gen 40:7 И спросил он царедворцев фараоновых, находившихся с ним в доме господина его под стражею, говоря: отчего у вас сегодня печальные лица?
\vs Gen 40:8 Они сказали ему: нам виделись сны; а истолковать их некому. Иосиф сказал им: не от Бога ли истолкования? расскажите мне.
\vs Gen 40:9 И рассказал главный виночерпий Иосифу сон свой и сказал ему: мне снилось, вот виноградная лоза предо мною;
\vs Gen 40:10 на лозе три ветви; она развилась, показался на ней цвет, выросли и созрели на ней ягоды;
\vs Gen 40:11 и чаша фараонова в руке у меня; я взял ягод, выжал их в чашу фараонову и подал чашу в руку фараону.
\vs Gen 40:12 И сказал ему Иосиф: вот истолкование его: три ветви~--- это три дня;
\vs Gen 40:13 через три дня фараон вознесет главу твою и возвратит тебя на место твое, и ты подашь чашу фараонову в руку его, по прежнему обыкновению, когда ты был у него виночерпием;
\vs Gen 40:14 вспомни же меня, когда хорошо тебе будет, и сделай мне благодеяние, и упомяни обо мне фараону, и выведи меня из этого дома,
\vs Gen 40:15 ибо я украден из земли Евреев; а также и здесь ничего не сделал, за что бы бросить меня в темницу.
\vs Gen 40:16 Главный хлебодар увидел, что истолковал он хорошо, и сказал Иосифу: мне также снилось: вот на голове у меня три корзины решетчатых;
\vs Gen 40:17 в верхней корзине всякая пища фараонова, изделие пекаря, и птицы [небесные] клевали ее из корзины на голове моей.
\vs Gen 40:18 И отвечал Иосиф и сказал [ему]: вот истолкование его: три корзины~--- это три дня;
\vs Gen 40:19 через три дня фараон снимет с тебя голову твою и повесит тебя на дереве, и птицы [небесные] будут клевать плоть твою с тебя.
\vs Gen 40:20 На третий день, день рождения фараонова, сделал он пир для всех слуг своих и вспомнил о главном виночерпии и главном хлебодаре среди слуг своих;
\vs Gen 40:21 и возвратил главного виночерпия на прежнее место, и он подал чашу в руку фараону,
\vs Gen 40:22 а главного хлебодара повесил [на дереве], как истолковал им Иосиф.
\vs Gen 40:23 И не вспомнил главный виночерпий об Иосифе, но забыл его.
\vs Gen 41:1 По прошествии двух лет фараону снилось: вот, он стоит у реки;
\vs Gen 41:2 и вот, вышли из реки семь коров, хороших видом и тучных плотью, и паслись в тростнике;
\vs Gen 41:3 но вот, после них вышли из реки семь коров других, худых видом и тощих плотью, и стали подле тех коров, на берегу реки;
\vs Gen 41:4 и съели коровы худые видом и тощие плотью семь коров хороших видом и тучных. И проснулся фараон,
\vs Gen 41:5 и заснул опять, и снилось ему в другой раз: вот, на одном стебле поднялось семь колосьев тучных и хороших;
\vs Gen 41:6 но вот, после них выросло семь колосьев тощих и иссушенных восточным ветром;
\vs Gen 41:7 и пожрали тощие колосья семь колосьев тучных и полных. И проснулся фараон и \bibemph{понял, что} это сон.
\vs Gen 41:8 Утром смутился дух его, и послал он, и призвал всех волхвов Египта и всех мудрецов его, и рассказал им фараон сон свой; но не было никого, кто бы истолковал его фараону.
\vs Gen 41:9 И стал говорить главный виночерпий фараону и сказал: грехи мои вспоминаю я ныне;
\vs Gen 41:10 фараон прогневался на рабов своих и отдал меня и главного хлебодара под стражу в дом начальника телохранителей;
\vs Gen 41:11 и снился нам сон в одну ночь, мне и ему, каждому снился сон особенного значения;
\vs Gen 41:12 там же был с нами молодой Еврей, раб начальника телохранителей; мы рассказали ему сны наши, и он истолковал нам каждому соответственно с его сновидением;
\vs Gen 41:13 и как он истолковал нам, так и сбылось: я возвращен на место мое, а тот повешен.
\vs Gen 41:14 И послал фараон и позвал Иосифа. И поспешно вывели его из темницы. Он остригся и переменил одежду свою и пришел к фараону.
\vs Gen 41:15 Фараон сказал Иосифу: мне снился сон, и нет никого, кто бы истолковал его, а о тебе я слышал, что ты умеешь толковать сны.
\vs Gen 41:16 И отвечал Иосиф фараону, говоря: это не мое; Бог даст ответ во благо фараону.
\vs Gen 41:17 И сказал фараон Иосифу: мне снилось: вот, стою я на берегу реки;
\vs Gen 41:18 и вот, вышли из реки семь коров тучных плотью и хороших видом и паслись в тростнике;
\vs Gen 41:19 но вот, после них вышли семь коров других, худых, очень дурных видом и тощих плотью: я не видывал во всей земле Египетской таких худых, как они;
\vs Gen 41:20 и съели тощие и худые коровы прежних семь коров тучных;
\vs Gen 41:21 и вошли \bibemph{тучные} в утробу их, но не приметно было, что они вошли в утробу их: они были так же худы видом, как и сначала. И я проснулся.
\vs Gen 41:22 \bibemph{Потом} снилось мне: вот, на одном стебле поднялись семь колосьев полных и хороших;
\vs Gen 41:23 но вот, после них выросло семь колосьев тонких, тощих и иссушенных восточным ветром;
\vs Gen 41:24 и пожрали тощие колосья семь колосьев хороших. Я рассказал это волхвам, но никто не изъяснил мне.
\vs Gen 41:25 И сказал Иосиф фараону: сон фараонов один: чт\acc{о} Бог сделает, т\acc{о} Он возвестил фараону.
\vs Gen 41:26 Семь коров хороших, это семь лет; и семь колосьев хороших, это семь лет: сон один;
\vs Gen 41:27 и семь коров тощих и худых, вышедших после тех, это семь лет, также и семь колосьев тощих и иссушенных восточным ветром, это семь лет голода.
\vs Gen 41:28 Вот почему сказал я фараону: чт\acc{о} Бог сделает, т\acc{о} Он показал фараону.
\vs Gen 41:29 Вот, наступает семь лет великого изобилия во всей земле Египетской;
\vs Gen 41:30 после них настанут семь лет голода, и забудется все то изобилие в земле Египетской, и истощит голод землю,
\vs Gen 41:31 и неприметно будет прежнее изобилие на земле, по причине голода, который последует, ибо он будет очень тяжел.
\vs Gen 41:32 А что сон повторился фараону дважды, \bibemph{это значит}, что сие истинно слово Божие, и что вскоре Бог исполнит сие.
\vs Gen 41:33 И ныне да усмотрит фараон мужа разумного и мудрого и да поставит его над землею Египетскою.
\vs Gen 41:34 Да повелит фараон поставить над землею надзирателей и собирать в семь лет изобилия пятую часть [всех произведений] земли Египетской;
\vs Gen 41:35 пусть они берут всякий хлеб этих наступающих хороших годов и соберут в городах хлеб под ведение фараона в пищу, и пусть берегут;
\vs Gen 41:36 и будет сия пища в запас для земли на семь лет голода, которые будут в земле Египетской, дабы земля не погибла от голода.
\rsbpar\vs Gen 41:37 Сие понравилось фараону и всем слугам его.
\vs Gen 41:38 И сказал фараон слугам своим: найдем ли мы такого, как он, человека, в котором был бы Дух Божий?
\vs Gen 41:39 И сказал фараон Иосифу: так как Бог открыл тебе все сие, то нет столь разумного и мудрого, как ты;
\vs Gen 41:40 ты будешь над домом моим, и твоего слова держаться будет весь народ мой; только престолом я буду больше тебя.
\vs Gen 41:41 И сказал фараон Иосифу: вот, я поставляю тебя над всею землею Египетскою.
\vs Gen 41:42 И снял фараон перстень свой с руки своей и надел его на руку Иосифа; одел его в виссонные одежды, возложил золотую цепь на шею ему;
\vs Gen 41:43 велел везти его на второй из своих колесниц и провозглашать пред ним: преклоняйтесь! И поставил его над всею землею Египетскою.
\vs Gen 41:44 И сказал фараон Иосифу: я фараон; без тебя никто не двинет ни руки своей, ни ноги своей во всей земле Египетской.
\vs Gen 41:45 И нарек фараон Иосифу имя: Цафнаф-панеах, и дал ему в жену Асенефу, дочь Потифера, жреца Илиопольского. И пошел Иосиф по земле Египетской.
\vs Gen 41:46 Иосифу было тридцать лет от рождения, когда он предстал пред лице фараона, царя Египетского. И вышел Иосиф от лица фараонова и прошел по всей земле Египетской.
\vs Gen 41:47 Земля же в семь лет изобилия приносила \bibemph{из зерна} по горсти.
\vs Gen 41:48 И собрал он всякий хлеб семи лет, которые были [плодородны] в земле Египетской, и положил хлеб в городах; в \bibemph{каждом} городе положил хлеб полей, окружающих его.
\vs Gen 41:49 И скопил Иосиф хлеба весьма много, как песку морского, так что перестал и считать, ибо не стало счета.
\rsbpar\vs Gen 41:50 До наступления годов голода, у Иосифа родились два сына, которых родила ему Асенефа, дочь Потифера, жреца Илиопольского.
\vs Gen 41:51 И нарек Иосиф имя первенцу: Манассия, потому что [говорил он] Бог дал мне забыть все несчастья мои и весь дом отца моего.
\vs Gen 41:52 А другому нарек имя: Ефрем, потому что [говорил он] Бог сделал меня плодовитым в земле страдания моего.
\rsbpar\vs Gen 41:53 И прошли семь лет изобилия, которое было в земле Египетской,
\vs Gen 41:54 и наступили семь лет голода, как сказал Иосиф. И был голод во всех землях, а во всей земле Египетской был хлеб.
\vs Gen 41:55 Но когда и вся земля Египетская начала терпеть голод, то народ начал вопиять к фараону о хлебе. И сказал фараон всем Египтянам: пойдите к Иосифу и делайте, что он вам скажет.
\vs Gen 41:56 И был голод по всей земле; и отворил Иосиф все житницы, и стал продавать хлеб Египтянам. Голод же усиливался в земле Египетской.
\vs Gen 41:57 И из всех стран приходили в Египет покупать хлеб у Иосифа, ибо голод усилился по всей земле.
\vs Gen 42:1 И узнал Иаков, что в Египте есть хлеб, и сказал Иаков сыновьям своим: что вы смотрите?
\vs Gen 42:2 И сказал: вот, я слышал, что есть хлеб в Египте; пойдите туда и купите нам оттуда хлеба, чтобы нам жить и не умереть.
\vs Gen 42:3 Десять братьев Иосифовых пошли купить хлеба в Египте,
\vs Gen 42:4 а Вениамина, брата Иосифова, не послал Иаков с братьями его, ибо сказал: не случилось бы с ним беды.
\vs Gen 42:5 И пришли сыны Израилевы покупать хлеб, вместе с другими пришедшими, ибо в земле Ханаанской был голод.
\vs Gen 42:6 Иосиф же был начальником в земле той; он и продавал хлеб всему народу земли. Братья Иосифа пришли и поклонились ему лицем до земли.
\vs Gen 42:7 И увидел Иосиф братьев своих и узнал их; но показал, будто не знает их, и говорил с ними сурово и сказал им: откуда вы пришли? Они сказали: из земли Ханаанской, купить пищи.
\vs Gen 42:8 Иосиф узнал братьев своих, но они не узнали его.
\vs Gen 42:9 И вспомнил Иосиф сны, которые снились ему о них; и сказал им: вы соглядатаи, вы пришли высмотреть наготу\fns{Слабые места.} земли сей.
\vs Gen 42:10 Они сказали ему: нет, господин наш; рабы твои пришли купить пищи;
\vs Gen 42:11 мы все дети одного человека; мы люди честные; рабы твои не бывали соглядатаями.
\vs Gen 42:12 Он сказал им: нет, вы пришли высмотреть наготу земли сей.
\vs Gen 42:13 Они сказали: нас, рабов твоих, двенадцать братьев; мы сыновья одного человека в земле Ханаанской, и вот, меньший теперь с отцом нашим, а одного не стало.
\vs Gen 42:14 И сказал им Иосиф: это самое я и говорил вам, сказав: вы соглядатаи;
\vs Gen 42:15 вот как вы будете испытаны: \bibemph{клянусь} жизнью фараона, вы не выйдете отсюда, если не придет сюда меньший брат ваш;
\vs Gen 42:16 пошлите одного из вас, и пусть он приведет брата вашего, а вы будете задержаны; и откроется, правда ли у вас; и если нет, \bibemph{то клянусь} жизнью фараона, что вы соглядатаи.
\vs Gen 42:17 И отдал их под стражу на три дня.
\vs Gen 42:18 И сказал им Иосиф в третий день: вот что сделайте, и останетесь живы, ибо я боюсь Бога:
\vs Gen 42:19 если вы люди честные, то один брат из вас пусть содержится в доме, где вы заключены; а вы пойдите, отвезите хлеб, ради голода семейств ваших;
\vs Gen 42:20 брата же вашего меньшого приведите ко мне, чтобы оправдались слова ваши и чтобы не умереть вам. Так они и сделали.
\vs Gen 42:21 И говорили они друг другу: точно мы наказываемся за грех против брата нашего; мы видели страдание души его, когда он умолял нас, но не послушали [его]; за то и постигло нас горе сие.
\vs Gen 42:22 Рувим отвечал им и сказал: не говорил ли я вам: не грешите против отрока? но вы не послушались; вот, кровь его взыскивается.
\vs Gen 42:23 А того не знали они, что Иосиф понимает; ибо между ними был переводчик.
\vs Gen 42:24 И отошел от них [Иосиф] и заплакал. И возвратился к ним, и говорил с ними, и, взяв из них Симеона, связал его пред глазами их.
\vs Gen 42:25 И приказал Иосиф наполнить мешки их хлебом, а серебро их возвратить каждому в мешок его, и дать им запасов на дорогу. Так и сделано с ними.
\vs Gen 42:26 Они положили хлеб свой на ослов своих, и пошли оттуда.
\vs Gen 42:27 И открыл один \bibemph{из них} мешок свой, чтобы дать корму ослу своему на ночлеге, и увидел серебро свое в отверстии мешка его,
\vs Gen 42:28 и сказал своим братьям: серебро мое возвращено; вот оно в мешке у меня. И смутилось сердце их, и они с трепетом друг другу говорили: что это Бог сделал с нами?
\vs Gen 42:29 И пришли к Иакову, отцу своему, в землю Ханаанскую и рассказали ему всё случившееся с ними, говоря:
\vs Gen 42:30 начальствующий над тою землею говорил с нами сурово и принял нас за соглядатаев земли той.
\vs Gen 42:31 И сказали мы ему: мы люди честные; мы не бывали соглядатаями;
\vs Gen 42:32 нас двенадцать братьев, сыновей у отца нашего; одного не стало, а меньший теперь с отцом нашим в земле Ханаанской.
\vs Gen 42:33 И сказал нам начальствующий над тою землею: вот как узнаю я, честные ли вы люди: оставьте у меня одного брата из вас, а вы возьмите хлеб ради голода семейств ваших и пойдите,
\vs Gen 42:34 и приведите ко мне меньшого брата вашего; и узнаю я, что вы не соглядатаи, но люди честные; отдам вам брата вашего, и вы можете промышлять в этой земле.
\vs Gen 42:35 Когда же они опорожняли мешки свои, вот, у каждого узел серебра его в мешке его. И увидели они узлы серебра своего, они и отец их, и испугались.
\vs Gen 42:36 И сказал им Иаков, отец их: вы лишили меня детей: Иосифа нет, и Симеона нет, и Вениамина взять хотите,~--- все это на меня!
\vs Gen 42:37 И сказал Рувим отцу своему, говоря: убей двух моих сыновей, если я не приведу его к тебе; отдай его на мои руки; я возвращу его тебе.
\vs Gen 42:38 Он сказал: не пойдет сын мой с вами; потому что брат его умер, и он один остался; если случится с ним несчастье на пути, в который вы пойдете, то сведете вы седину мою с печалью во гроб.
\vs Gen 43:1 Голод усилился на земле.
\vs Gen 43:2 И когда они съели хлеб, который привезли из Египта, тогда отец их сказал им: пойдите опять, купите нам немного пищи.
\vs Gen 43:3 И сказал ему Иуда, говоря: тот человек решительно объявил нам, сказав: не являйтесь ко мне на лице, если брата вашего не будет с вами.
\vs Gen 43:4 Если пошлешь с нами брата нашего, то пойдем и купим тебе пищи,
\vs Gen 43:5 а если не пошлешь, то не пойдем, ибо тот человек сказал нам: не являйтесь ко мне на лице, если брата вашего не будет с вами.
\vs Gen 43:6 Израиль сказал: для чего вы сделали мне такое зло, сказав тому человеку, что у вас есть еще брат?
\vs Gen 43:7 Они сказали: расспрашивал тот человек о нас и о родстве нашем, говоря: жив ли еще отец ваш? есть ли у вас брат? Мы и рассказали ему по этим расспросам. Могли ли мы знать, что он скажет: приведите брата вашего?
\vs Gen 43:8 Иуда же сказал Израилю, отцу своему: отпусти отрока со мною, и мы встанем и пойдем, и живы будем и не умрем и мы, и ты, и дети наши;
\vs Gen 43:9 я отвечаю за него, из моих рук потребуешь его; если я не приведу его к тебе и не поставлю его пред лицем твоим, то останусь я виновным пред тобою во все дни жизни;
\vs Gen 43:10 если бы мы не медлили, то уже сходили бы два раза.
\vs Gen 43:11 Израиль, отец их, сказал им: если так, то вот что сделайте: возьмите с собою плодов земли сей и отнесите в дар тому человеку несколько бальзама и несколько меду, стираксы и ладану, фисташков и миндальных орехов;
\vs Gen 43:12 возьмите и другое серебро в руки ваши; а серебро, обратно положенное в отверстие мешков ваших, возвратите руками вашими: может быть, это недосмотр;
\vs Gen 43:13 и брата вашего возьмите и, встав, пойдите опять к человеку тому;
\vs Gen 43:14 Бог же Всемогущий да даст вам найти милость у человека того, чтобы он отпустил вам и другого брата вашего и Вениамина, а мне если уже быть бездетным, то пусть буду бездетным.
\vs Gen 43:15 И взяли те люди дары эти, и серебра вдвое взяли в руки свои, и Вениамина, и встали, пошли в Египет и предстали пред лице Иосифа.
\vs Gen 43:16 Иосиф, увидев между ними Вениамина [брата своего, сына матери своей], сказал начальнику дома своего: введи сих людей в дом и заколи что-нибудь из скота, и приготовь, потому что со мною будут есть эти люди в полдень.
\vs Gen 43:17 И сделал человек тот, как сказал Иосиф, и ввел человек тот людей сих в дом Иосифов.
\vs Gen 43:18 И испугались люди эти, что ввели их в дом Иосифов, и сказали: это за серебро, возвращенное прежде в мешки наши, ввели нас, чтобы придраться к нам и напасть на нас, и взять нас в рабство, и ослов наших.
\vs Gen 43:19 И подошли они к начальнику дома Иосифова, и стали говорить ему у дверей дома,
\vs Gen 43:20 и сказали: послушай, господин наш, мы приходили уже прежде покупать пищи,
\vs Gen 43:21 и случилось, что, когда пришли мы на ночлег и открыли мешки наши,~--- вот серебро каждого в отверстии мешка его, серебро наше по весу его, и мы возвращаем его своими руками;
\vs Gen 43:22 а для покупки пищи мы принесли другое серебро в руках наших, мы не знаем, кто положил серебро наше в мешки наши.
\vs Gen 43:23 Он сказал: будьте спокойны, не бойтесь; Бог ваш и Бог отца вашего дал вам клад в мешках ваших; серебро ваше дошло до меня. И привел к ним Симеона.
\vs Gen 43:24 И ввел тот человек людей сих в дом Иосифов и дал воды, и они омыли ноги свои; и дал корму ослам их.
\vs Gen 43:25 И они приготовили дары к приходу Иосифа в полдень, ибо слышали, что там будут есть хлеб.
\vs Gen 43:26 И пришел Иосиф домой; и они принесли ему в дом дары, которые были на руках их, и поклонились ему до земли.
\vs Gen 43:27 Он спросил их о здоровье и сказал: здоров ли отец ваш старец, о котором вы говорили? жив ли еще он?
\vs Gen 43:28 Они сказали: здоров раб твой, отец наш; еще жив. [Он сказал: благословен человек сей от Бога.] И преклонились они и поклонились.
\vs Gen 43:29 И поднял глаза свои [Иосиф], и увидел Вениамина, брата своего, сына матери своей, и сказал: это брат ваш меньший, о котором вы сказывали мне? И сказал: да будет милость Божия с тобою, сын мой!
\vs Gen 43:30 И поспешно удалился Иосиф, потому что воскипела любовь к брату его, и он готов был заплакать, и вошел он во внутреннюю комнату и плакал там.
\vs Gen 43:31 И умыв лице свое, вышел, и скрепился и сказал: подавайте кушанье.
\vs Gen 43:32 И подали ему особо, и им особо, и Египтянам, обедавшим с ним, особо, ибо Египтяне не могут есть с Евреями, потому что это мерзость для Египтян.
\vs Gen 43:33 И сели они пред ним, первородный по первородству его, и младший по молодости его, и дивились эти люди друг пред другом.
\vs Gen 43:34 И посылались им кушанья от него, и доля Вениамина была впятеро больше долей каждого из них. И пили, и довольно пили они с ним.
\vs Gen 44:1 И приказал [Иосиф] начальнику дома своего, говоря: наполни мешки этих людей пищею, сколько они могут нести, и серебро каждого положи в отверстие мешка его,
\vs Gen 44:2 а чашу мою, чашу серебряную, положи в отверстие мешка к младшему вместе с серебром за купленный им хлеб. И сделал тот по слову Иосифа, которое сказал он.
\vs Gen 44:3 Утром, когда рассвело, эти люди были отпущены, они и ослы их.
\vs Gen 44:4 Еще не далеко отошли они от города, как Иосиф сказал начальнику дома своего: ступай, догоняй этих людей и, когда догонишь, скажи им: для чего вы заплатили злом за добро? [для чего украли у меня серебряную чашу?]
\vs Gen 44:5 Не та ли это, из которой пьет господин мой и он гадает на ней? Худо это вы сделали.
\vs Gen 44:6 Он догнал их и сказал им эти слова.
\vs Gen 44:7 Они сказали ему: для чего господин наш говорит такие слова? Нет, рабы твои не сделают такого дела.
\vs Gen 44:8 Вот, серебро, найденное нами в отверстии мешков наших, мы обратно принесли тебе из земли Ханаанской: как же нам украсть из дома господина твоего серебро или золото?
\vs Gen 44:9 У кого из рабов твоих найдется [чаша], тому смерть, и мы будем рабами господину нашему.
\vs Gen 44:10 Он сказал: хорошо; как вы сказали, так пусть и будет: у кого найдется [чаша], тот будет мне рабом, а вы будете не виноваты.
\vs Gen 44:11 Они поспешно спустили каждый свой мешок на землю и открыли каждый свой мешок.
\vs Gen 44:12 Он обыскал, начал со старшего и окончил младшим; и нашлась чаша в мешке Вениаминовом.
\vs Gen 44:13 И разодрали они одежды свои, и, возложив каждый на осла своего ношу, возвратились в город.
\vs Gen 44:14 И пришли Иуда и братья его в дом Иосифа, который был еще дома, и пали пред ним на землю.
\vs Gen 44:15 Иосиф сказал им: что это вы сделали? разве вы не знали, что такой человек, как я, конечно угадает?
\vs Gen 44:16 Иуда сказал: что нам сказать господину нашему? что говорить? чем оправдываться? Бог нашел неправду рабов твоих; вот, мы рабы господину нашему, и мы, и тот, в чьих руках нашлась чаша.
\vs Gen 44:17 Но [Иосиф] сказал: нет, я этого не сделаю; тот, в чьих руках нашлась чаша, будет мне рабом, а вы пойдите с миром к отцу вашему.
\vs Gen 44:18 И подошел Иуда к нему и сказал: господин мой, позволь рабу твоему сказать слово в уши господина моего, и не прогневайся на раба твоего, ибо ты то же, что фараон.
\vs Gen 44:19 Господин мой спрашивал рабов своих, говоря: есть ли у вас отец или брат?
\vs Gen 44:20 Мы сказали господину нашему, что у нас есть отец престарелый, и [у него] младший сын, сын старости, которого брат умер, а он остался один \bibemph{от} матери своей, и отец любит его.
\vs Gen 44:21 Ты же сказал рабам твоим: приведите его ко мне, чтобы мне взглянуть на него.
\vs Gen 44:22 Мы сказали господину нашему: отрок не может оставить отца своего, и если он оставит отца своего, то сей умрет.
\vs Gen 44:23 Но ты сказал рабам твоим: если не придет с вами меньший брат ваш, то вы более не являйтесь ко мне на лице.
\vs Gen 44:24 Когда мы пришли к рабу твоему, отцу нашему, то пересказали ему слова господина моего.
\vs Gen 44:25 И сказал отец наш: пойдите опять, купите нам немного пищи.
\vs Gen 44:26 Мы сказали: нельзя нам идти; а если будет с нами меньший брат наш, то пойдем; потому что нельзя нам видеть лица того человека, если не будет с нами меньшого брата нашего.
\vs Gen 44:27 И сказал нам раб твой, отец наш: вы знаете, что жена моя родила мне двух \bibemph{сынов};
\vs Gen 44:28 один пошел от меня, и я сказал: верно он растерзан; и я не видал его доныне;
\vs Gen 44:29 если и сего возьмете от глаз моих, и случится с ним несчастье, то сведете вы седину мою с горестью во гроб.
\vs Gen 44:30 Теперь если я приду к рабу твоему, отцу нашему, и не будет с нами отрока, с душею которого связана душа его,
\vs Gen 44:31 то он, увидев, что нет отрока, умрет; и сведут рабы твои седину раба твоего, отца нашего, с печалью во гроб.
\vs Gen 44:32 Притом я, раб твой, взялся отвечать за отрока отцу моему, сказав: если не приведу его к тебе [и не поставлю его пред тобою], то останусь я виновным пред отцом моим во все дни жизни.
\vs Gen 44:33 Итак пусть я, раб твой, вместо отрока останусь рабом у господина моего, а отрок пусть идет с братьями своими:
\vs Gen 44:34 ибо как пойду я к отцу моему, когда отрока не будет со мною? я увидел бы бедствие, которое постигло бы отца моего.
\vs Gen 45:1 Иосиф не мог более удерживаться при всех стоявших около него и закричал: удалите от меня всех. И не оставалось при Иосифе никого, когда он открылся братьям своим.
\vs Gen 45:2 И громко зарыдал он, и услышали Египтяне, и услышал дом фараонов.
\vs Gen 45:3 И сказал Иосиф братьям своим: я~--- Иосиф, жив ли еще отец мой? Но братья его не могли отвечать ему, потому что они смутились пред ним.
\vs Gen 45:4 И сказал Иосиф братьям своим: подойдите ко мне. Они подошли. Он сказал: я~--- Иосиф, брат ваш, которого вы продали в Египет;
\vs Gen 45:5 но теперь не печальтесь и не жалейте о том, что вы продали меня сюда, потому что Бог послал меня перед вами для сохранения вашей жизни;
\vs Gen 45:6 ибо теперь два года голода на земле: [остается] еще пять лет, в которые ни орать, ни жать не будут;
\vs Gen 45:7 Бог послал меня перед вами, чтобы оставить вас на земле и сохранить вашу жизнь великим избавлением.
\vs Gen 45:8 Итак не вы послали меня сюда, но Бог, Который и поставил меня отцом фараону и господином во всем доме его и владыкою во всей земле Египетской.
\vs Gen 45:9 Идите скорее к отцу моему и скажите ему: так говорит сын твой Иосиф: Бог поставил меня господином над всем Египтом; приди ко мне, не медли;
\vs Gen 45:10 ты будешь жить в земле Гесем; и будешь близ меня, ты, и сыны твои, и сыны сынов твоих, и мелкий и крупный скот твой, и все твое;
\vs Gen 45:11 и прокормлю тебя там, ибо голод будет еще пять лет, чтобы не обнищал ты и дом твой и все твое.
\vs Gen 45:12 И вот, очи ваши и очи брата моего Вениамина видят, что это мои уста говорят с вами;
\vs Gen 45:13 скажите же отцу моему о всей славе моей в Египте и о всем, что вы видели, и приведите скорее отца моего сюда.
\vs Gen 45:14 И пал он на шею Вениамину, брату своему, и плакал; и Вениамин плакал на шее его.
\vs Gen 45:15 И целовал всех братьев своих и плакал, обнимая их. Потом говорили с ним братья его.
\rsbpar\vs Gen 45:16 Дошел в дом фараона слух, что пришли братья Иосифа; и приятно было фараону и рабам его.
\vs Gen 45:17 И сказал фараон Иосифу: скажи братьям твоим: вот что сделайте: навьючьте скот ваш [хлебом] и ступайте в землю Ханаанскую;
\vs Gen 45:18 и возьмите отца вашего и семейства ваши и придите ко мне; я дам вам лучшее [место] в земле Египетской, и вы будете есть тук земли.
\vs Gen 45:19 Тебе же повелеваю сказать им: сделайте сие: возьмите себе из земли Египетской колесниц для детей ваших и для жен ваших, и привезите отца вашего и придите;
\vs Gen 45:20 и не жалейте вещей ваших, ибо лучшее из всей земли Египетской \bibemph{дам} вам.
\rsbpar\vs Gen 45:21 Так и сделали сыны Израилевы. И дал им Иосиф колесницы по приказанию фараона, и дал им путевой запас,
\vs Gen 45:22 каждому из них он дал перемену одежд, а Вениамину дал триста сребреников и пять перемен одежд;
\vs Gen 45:23 также и отцу своему послал десять ослов, навьюченных лучшими \bibemph{произведениями} Египетскими, и десять ослиц, навьюченных зерном, хлебом и припасами отцу своему на путь.
\vs Gen 45:24 И отпустил братьев своих, и они пошли. И сказал им: не ссорьтесь на дороге.
\vs Gen 45:25 И пошли они из Египта, и пришли в землю Ханаанскую к Иакову, отцу своему,
\vs Gen 45:26 и известили его, сказав: Иосиф [сын твой] жив и теперь владычествует над всею землею Египетскою. Но сердце его смутилось, ибо он не верил им.
\vs Gen 45:27 Когда же они пересказали ему все слова Иосифа, которые он говорил им, и когда увидел колесницы, которые прислал Иосиф, чтобы везти его, тогда ожил дух Иакова, отца их,
\vs Gen 45:28 и сказал Израиль: довольно [сего для меня], еще жив сын мой Иосиф; пойду и увижу его, пока не умру.
\vs Gen 46:1 И отправился Израиль со всем, что у него было, и пришел в Вирсавию, и принес жертвы Богу отца своего Исаака.
\vs Gen 46:2 И сказал Бог Израилю в видении ночном: Иаков! Иаков! Он сказал: вот я.
\vs Gen 46:3 \bibemph{Бог} сказал: Я Бог, Бог отца твоего; не бойся идти в Египет, ибо там произведу от тебя народ великий;
\vs Gen 46:4 Я пойду с тобою в Египет, Я и выведу тебя обратно. Иосиф своею рукою закроет глаза твои.
\rsbpar\vs Gen 46:5 Иаков отправился из Вирсавии; и повезли сыны Израилевы Иакова отца своего, и детей своих, и жен своих на колесницах, которые послал фараон, чтобы привезти его.
\vs Gen 46:6 И взяли они скот свой и имущество свое, которое приобрели в земле Ханаанской, и пришли в Египет,~--- Иаков и весь род его с ним.
\vs Gen 46:7 Сынов своих и внуков своих с собою, дочерей своих и внучек своих и весь род свой привел он с собою в Египет.
\rsbpar\vs Gen 46:8 Вот имена сынов Израилевых, пришедших в Египет: Иаков и сыновья его. Первенец Иакова Рувим.
\vs Gen 46:9 Сыны Рувима: Ханох и Фаллу, Хецрон и Харми.
\vs Gen 46:10 Сыны Симеона: Иемуил и Иамин, и Огад, и Иахин, и Цохар, и Саул, сын Хананеянки.
\vs Gen 46:11 Сыны Левия: Гирсон, Кааф и Мерари.
\vs Gen 46:12 Сыны Иуды: Ир и Онан, и Шела, и Фарес, и Зара; но Ир и Онан умерли в земле Ханаанской. Сыны Фареса были: Есром и Хамул.
\vs Gen 46:13 Сыны Иссахара: Фола и Фува, Иов и Шимрон.
\vs Gen 46:14 Сыны Завулона: Серед и Елон, и Иахлеил.
\vs Gen 46:15 Это сыны Лии, которых она родила Иакову в Месопотамии, и Дину, дочь его. Всех душ сынов его и дочерей его~--- тридцать три.
\vs Gen 46:16 Сыны Гада: Цифион и Хагги, Шуни и Эцбон, Ери и Ароди и Арели.
\vs Gen 46:17 Сыны Асира: Имна и Ишва, и Ишви, и Бриа, и Серах, сестра их. Сыны Брии: Хевер и Малхиил.
\vs Gen 46:18 Это сыны Зелфы, которую Лаван дал Лии, дочери своей; она родила их Иакову шестнадцать душ.
\vs Gen 46:19 Сыны Рахили, жены Иакова: Иосиф и Вениамин.
\vs Gen 46:20 И родились у Иосифа в земле Египетской Манассия и Ефрем, которых родила ему Асенефа, дочь Потифера, жреца Илиопольского.
\vs Gen 46:21 Сыны Вениамина: Бела и Бехер и Ашбел; [сыны Белы были:] Гера и Нааман, Эхи и Рош, Муппим и Хуппим и Ард.
\vs Gen 46:22 Это сыны Рахили, которые родились у Иакова, всего четырнадцать душ.
\vs Gen 46:23 Сын Дана: Хушим.
\vs Gen 46:24 Сыны Неффалима: Иахцеил и Гуни, и Иецер, и Шиллем.
\vs Gen 46:25 Это сыны Валлы, которую дал Лаван дочери своей Рахили; она родила их Иакову всего семь душ.
\vs Gen 46:26 Всех душ, пришедших с Иаковом в Египет, которые произошли из чресл его, кроме жен сынов Иаковлевых, всего шестьдесят шесть душ.
\vs Gen 46:27 Сынов Иосифа, которые родились у него в Египте, две души. Всех душ дома Иаковлева, перешедших [с Иаковом] в Египет, семьдесят [пять].
\vs Gen 46:28 Иуду послал он пред собою к Иосифу, чтобы он указал \bibemph{путь} в Гесем. И пришли в землю Гесем.
\vs Gen 46:29 Иосиф запряг колесницу свою и выехал навстречу Израилю, отцу своему, в Гесем, и, увидев его, пал на шею его, и долго плакал на шее его.
\vs Gen 46:30 И сказал Израиль Иосифу: умру я теперь, увидев лице твое, ибо ты еще жив.
\vs Gen 46:31 И сказал Иосиф братьям своим и дому отца своего: я пойду, извещу фараона и скажу ему: братья мои и дом отца моего, которые были в земле Ханаанской, пришли ко мне;
\vs Gen 46:32 эти люди пастухи овец, ибо скотоводы они; и мелкий и крупный скот свой, и все, что у них, привели они.
\vs Gen 46:33 Если фараон призовет вас и скажет: какое занятие ваше?
\vs Gen 46:34 то вы скажите: \bibemph{мы}, рабы твои, скотоводами были от юности нашей доныне, и мы и отцы наши, чтобы вас поселили в земле Гесем. Ибо мерзость для Египтян всякий пастух овец.
\vs Gen 47:1 И пришел Иосиф и известил фараона и сказал: отец мой и братья мои, с мелким и крупным скотом своим и со всем, что у них, пришли из земли Ханаанской; и вот, они в земле Гесем.
\vs Gen 47:2 И из братьев своих он взял пять человек и представил их фараону.
\vs Gen 47:3 И сказал фараон братьям его: какое ваше занятие? Они сказали фараону: пастухи овец рабы твои, и мы и отцы наши.
\vs Gen 47:4 И сказали они фараону: мы пришли пожить в этой земле, потому что нет пажити для скота рабов твоих, ибо в земле Ханаанской сильный голод; итак позволь поселиться рабам твоим в земле Гесем.
\vs Gen 47:5 И сказал фараон Иосифу: отец твой и братья твои пришли к тебе;
\vs Gen 47:6 земля Египетская пред тобою; на лучшем месте земли посели отца твоего и братьев твоих; пусть живут они в земле Гесем; и если знаешь, что между ними есть способные люди, поставь их смотрителями над моим скотом.
\vs Gen 47:7 И привел Иосиф Иакова, отца своего, и представил его фараону; и благословил Иаков фараона.
\vs Gen 47:8 Фараон сказал Иакову: сколько лет жизни твоей?
\vs Gen 47:9 Иаков сказал фараону: дней странствования моего сто тридцать лет; малы и несчастны дни жизни моей и не достигли до лет жизни отцов моих во днях странствования их.
\vs Gen 47:10 И благословил фараона Иаков и вышел от фараона.
\vs Gen 47:11 И поселил Иосиф отца своего и братьев своих, и дал им владение в земле Египетской, в лучшей части земли, в земле Раамсес, как повелел фараон.
\vs Gen 47:12 И снабжал Иосиф отца своего и братьев своих и весь дом отца своего хлебом, по потребностям каждого семейства.
\rsbpar\vs Gen 47:13 И не было хлеба по всей земле, потому что голод весьма усилился, и изнурены были от голода земля Египетская и земля Ханаанская.
\vs Gen 47:14 Иосиф собрал все серебро, какое было в земле Египетской и в земле Ханаанской, за хлеб, который покупали, и внес Иосиф серебро в дом фараонов.
\vs Gen 47:15 И серебро истощилось в земле Египетской и в земле Ханаанской. Все Египтяне пришли к Иосифу и говорили: дай нам хлеба; зачем нам умирать пред тобою, потому что серебро вышло у нас?
\vs Gen 47:16 Иосиф сказал: пригоняйте скот ваш, и я буду давать вам [хлеб] за скот ваш, если серебро вышло у вас.
\vs Gen 47:17 И пригоняли они к Иосифу скот свой; и давал им Иосиф хлеб за лошадей, и за стада мелкого скота, и за стада крупного скота, и за ослов; и снабжал их хлебом в тот год за весь скот их.
\vs Gen 47:18 И прошел этот год; и пришли к нему на другой год и сказали ему: не скроем от господина нашего, что серебро истощилось и стада скота нашего у господина нашего; ничего не осталось у нас пред господином нашим, кроме тел наших и земель наших;
\vs Gen 47:19 для чего нам погибать в глазах твоих, и нам и землям нашим? купи нас и земли наши за хлеб, и мы с землями нашими будем рабами фараону, а ты дай нам семян, чтобы нам быть живыми и не умереть, и чтобы не опустела земля.
\vs Gen 47:20 И купил Иосиф всю землю Египетскую для фараона, потому что продали Египтяне каждый свое поле, ибо голод одолевал их. И досталась земля фараону.
\vs Gen 47:21 И народ сделал он рабами от одного конца Египта до другого.
\vs Gen 47:22 Только земли жрецов не купил [Иосиф], ибо жрецам от фараона положен был участок, и они питались своим участком, который дал им фараон; посему и не продали земли своей.
\vs Gen 47:23 И сказал Иосиф народу: вот, я купил теперь для фараона вас и землю вашу; вот вам семена, и засевайте землю;
\vs Gen 47:24 когда будет жатва, давайте пятую часть фараону, а четыре части останутся вам на засеяние полей, на пропитание вам и тем, кто в домах ваших, и на пропитание детям вашим.
\vs Gen 47:25 Они сказали: ты спас нам жизнь; да обретем милость в очах господина нашего и да будем рабами фараону.
\vs Gen 47:26 И поставил Иосиф в закон земле Египетской, даже до сего дня: пятую часть давать фараону, исключая только землю жрецов, которая не принадлежала фараону.
\rsbpar\vs Gen 47:27 И жил Израиль в земле Египетской, в земле Гесем, и владели они ею, и плодились, и весьма умножились.
\vs Gen 47:28 И жил Иаков в земле Египетской семнадцать лет; и было дней Иакова, годов жизни его, сто сорок семь лет.
\vs Gen 47:29 И пришло время Израилю умереть, и призвал он сына своего Иосифа и сказал ему: если я нашел благоволение в очах твоих, положи руку твою под стегно мое и \bibemph{клянись}, что ты окажешь мне милость и правду, не похоронишь меня в Египте,
\vs Gen 47:30 дабы мне лечь с отцами моими; вынесешь меня из Египта и похоронишь меня в их гробнице. \bibemph{Иосиф} сказал: сделаю по слову твоему.
\vs Gen 47:31 И сказал: клянись мне. И клялся ему. И поклонился Израиль на возглавие постели\fns{По переводу 70-ти: на верх жезла его.}.
\vs Gen 48:1 После того Иосифу сказали: вот, отец твой болен. И он взял с собою двух сынов своих, Манассию и Ефрема [и пошел к Иакову].
\vs Gen 48:2 Иакова известили и сказали: вот, сын твой Иосиф идет к тебе. Израиль собрал силы свои и сел на постели.
\vs Gen 48:3 И сказал Иаков Иосифу: Бог Всемогущий явился мне в Лузе, в земле Ханаанской, и благословил меня,
\vs Gen 48:4 и сказал мне: вот, Я распложу тебя, и размножу тебя, и произведу от тебя множество народов, и дам землю сию потомству твоему после тебя, в вечное владение.
\vs Gen 48:5 И ныне два сына твои, родившиеся тебе в земле Египетской, до моего прибытия к тебе в Египет, мои они; Ефрем и Манассия, как Рувим и Симеон, будут мои;
\vs Gen 48:6 дети же твои, которые родятся от тебя после них, будут твои; они под именем братьев своих будут именоваться в их уделе.
\vs Gen 48:7 Когда я шел из Месопотамии, умерла у меня Рахиль [мать твоя] в земле Ханаанской, по дороге, не доходя несколько до Ефрафы, и я похоронил ее там на дороге к Ефрафе, что \bibemph{ныне} Вифлеем.
\vs Gen 48:8 И увидел Израиль сыновей Иосифа и сказал: кто это?
\vs Gen 48:9 И сказал Иосиф отцу своему: это сыновья мои, которых Бог дал мне здесь. [Иаков] сказал: подведи их ко мне, и я благословлю их.
\vs Gen 48:10 Глаза же Израилевы притупились от старости; не мог он видеть \bibemph{ясно. Иосиф} подвел их к нему, и он поцеловал их и обнял их.
\vs Gen 48:11 И сказал Израиль Иосифу: не надеялся я видеть твое лице; но вот, Бог показал мне и детей твоих.
\vs Gen 48:12 И отвел их Иосиф от колен его и поклонился ему лицем своим до земли.
\vs Gen 48:13 И взял Иосиф обоих [сыновей своих], Ефрема в правую свою руку против левой Израиля, а Манассию в левую против правой Израиля, и подвел к нему.
\vs Gen 48:14 Но Израиль простер правую руку свою и положил на голову Ефрему, хотя сей был меньший, а левую на голову Манассии. С намерением положил он так руки свои, хотя Манассия был первенец.
\vs Gen 48:15 И благословил Иосифа и сказал: Бог, пред Которым ходили отцы мои Авраам и Исаак, Бог, пасущий меня с тех пор, как я существую, до сего дня,
\vs Gen 48:16 Ангел, избавляющий меня от всякого зла, да благословит отроков сих; да будет на них наречено имя мое и имя отцов моих Авраама и Исаака, и да возрастут они во множество посреди земли.
\vs Gen 48:17 И увидел Иосиф, что отец его положил правую руку свою на голову Ефрема; и прискорбно было ему это. И взял он руку отца своего, чтобы переложить ее с головы Ефрема на голову Манассии,
\vs Gen 48:18 и сказал Иосиф отцу своему: не так, отец мой, ибо это~--- первенец; положи на его голову правую руку твою.
\vs Gen 48:19 Но отец его не согласился и сказал: знаю, сын мой, знаю; и от него произойдет народ, и он будет велик; но меньший его брат будет больше его, и от семени его произойдет многочисленный народ.
\vs Gen 48:20 И благословил их в тот день, говоря: тобою будет благословлять Израиль, говоря: Бог да сотворит тебе, как Ефрему и Манассии. И поставил Ефрема выше Манассии.
\vs Gen 48:21 И сказал Израиль Иосифу: вот, я умираю; и Бог будет с вами и возвратит вас в землю отцов ваших;
\vs Gen 48:22 я даю тебе, преимущественно пред братьями твоими, один участок, который я взял из рук Аморреев мечом моим и луком моим.
\vs Gen 49:1 И призвал Иаков сыновей своих и сказал: соберитесь, и я возвещу вам, чт\acc{о} будет с вами в грядущие дни;
\vs Gen 49:2 сойдитесь и послушайте, сыны Иакова, послушайте Израиля, отца вашего.
\rsbpar\vs Gen 49:3 Рувим, первенец мой! ты~--- крепость моя и начаток силы моей, верх достоинства и верх могущества;
\vs Gen 49:4 но ты бушевал, как вода,~--- не будешь преимуществовать, ибо ты взошел на ложе отца твоего, ты осквернил постель мою, [на которую] взошел.
\rsbpar\vs Gen 49:5 Симеон и Левий братья, орудия жестокости мечи их;
\vs Gen 49:6 в совет их да не внидет душа моя, и к собранию их да не приобщится слава моя, ибо они во гневе своем убили мужа и по прихоти своей перерезали жилы тельца;
\vs Gen 49:7 проклят гнев их, ибо жесток, и ярость их, ибо свирепа; разделю их в Иакове и рассею их в Израиле.
\rsbpar\vs Gen 49:8 Иуда! тебя восхвалят братья твои. Рука твоя на хребте врагов твоих; поклонятся тебе сыны отца твоего.
\vs Gen 49:9 Молодой лев Иуда, с добычи, сын мой, поднимается. Преклонился он, лег, как лев и как львица: кто поднимет его?
\vs Gen 49:10 Не отойдет скипетр от Иуды и законодатель от чресл его, доколе не приидет Примиритель, и Ему покорность народов.
\vs Gen 49:11 Он привязывает к виноградной лозе осленка своего и к лозе лучшего винограда сына ослицы своей; моет в вине одежду свою и в крови гроздов одеяние свое;
\vs Gen 49:12 блестящи очи [его] от вина, и белы зубы [его] от молока.
\rsbpar\vs Gen 49:13 Завулон при береге морском будет жить и у пристани корабельной, и предел его до Сидона.
\rsbpar\vs Gen 49:14 Иссахар осел крепкий, лежащий между протоками вод;
\vs Gen 49:15 и увидел он, что покой хорош, и что земля приятна: и преклонил плечи свои для ношения бремени и стал работать в уплату дани.
\rsbpar\vs Gen 49:16 Дан будет судить народ свой, как одно из колен Израиля;
\vs Gen 49:17 Дан будет змеем на дороге, аспидом на пути, уязвляющим ногу коня, так что всадник его упадет назад.
\vs Gen 49:18 На помощь твою надеюсь, Господи!
\rsbpar\vs Gen 49:19 Гад,~--- толпа будет теснить его, но он оттеснит ее по пятам.
\rsbpar\vs Gen 49:20 Для Асира~--- слишком тучен хлеб его, и он будет доставлять царские яства.
\rsbpar\vs Gen 49:21 Неффалим~--- теревинф рослый, распускающий прекрасные ветви\fns{По другому чтению: Неффалим~--- серна стройная; он говорит прекрасные изречения.}.
\rsbpar\vs Gen 49:22 Иосиф~--- отрасль плодоносного \bibemph{дерева}, отрасль плодоносного \bibemph{дерева} над источником; ветви его простираются над стеною;
\vs Gen 49:23 огорчали его, и стреляли и враждовали на него стрельцы,
\vs Gen 49:24 но тверд остался лук его, и крепки мышцы рук его, от рук мощного \bibemph{Бога} Иаковлева. Оттуда Пастырь и твердыня Израилева,
\vs Gen 49:25 от Бога отца твоего, \bibemph{Который} и да поможет тебе, и от Всемогущего, Который и да благословит тебя благословениями небесными свыше, благословениями бездны, лежащей долу, благословениями сосцов и утробы,
\vs Gen 49:26 благословениями отца твоего, которые превышают благословения гор древних и приятности холмов вечных; да будут они на голове Иосифа и на темени избранного между братьями своими.
\rsbpar\vs Gen 49:27 Вениамин, хищный волк, утром будет есть ловитву и вечером будет делить добычу.
\rsbpar\vs Gen 49:28 Вот все двенадцать колен Израилевых; и вот что сказал им отец их; и благословил их, и дал им благословение, каждому свое.
\vs Gen 49:29 И заповедал он им и сказал им: я прилагаюсь к народу моему; похороните меня с отцами моими в пещере, которая на поле Ефрона Хеттеянина,
\vs Gen 49:30 в пещере, которая на поле Махпела, что пред Мамре, в земле Ханаанской, которую [пещеру] купил Авраам с полем у Ефрона Хеттеянина в собственность для погребения;
\vs Gen 49:31 там похоронили Авраама и Сарру, жену его; там похоронили Исаака и Ревекку, жену его; и там похоронил я Лию;
\vs Gen 49:32 это поле и пещера, которая на нем, куплена у сынов Хеттеевых.
\vs Gen 49:33 И окончил Иаков завещание сыновьям своим, и положил ноги свои на постель, и скончался, и приложился к народу своему.
\vs Gen 50:1 Иосиф пал на лице отца своего, и плакал над ним, и целовал его.
\vs Gen 50:2 И повелел Иосиф слугам своим~--- врачам, бальзамировать отца его; и врачи набальзамировали Израиля.
\vs Gen 50:3 И исполнилось ему сорок дней, ибо столько дней употребляется на бальзамирование, и оплакивали его Египтяне семьдесят дней.
\vs Gen 50:4 Когда же прошли дни плача по нем, Иосиф сказал придворным фараона, говоря: если я обрел благоволение в очах ваших, то скажите фараону так:
\vs Gen 50:5 отец мой заклял меня, сказав: вот, я умираю; во гробе моем, который я выкопал себе в земле Ханаанской, там похорони меня. И теперь хотел бы я пойти и похоронить отца моего и возвратиться. [Слова Иосифа пересказали фараону.]
\vs Gen 50:6 И сказал фараон: пойди и похорони отца твоего, как он заклял тебя.
\vs Gen 50:7 И пошел Иосиф хоронить отца своего. И пошли с ним все слуги фараона, старейшины дома его и все старейшины земли Египетской,
\vs Gen 50:8 и весь дом Иосифа, и братья его, и дом отца его. Только детей своих и мелкий и крупный скот свой оставили в земле Гесем.
\vs Gen 50:9 С ним отправились также колесницы и всадники, так что сонм был весьма велик.
\vs Gen 50:10 И дошли они до Горен-гаатада при Иордане и плакали там плачем великим и весьма сильным; и сделал \bibemph{Иосиф} плач по отце своем семь дней.
\vs Gen 50:11 И видели жители земли той, Хананеи, плач в Горен-гаатаде, и сказали: велик плач этот у Египтян! Посему наречено имя [месту] тому: плач Египтян, что при Иордане.
\vs Gen 50:12 И сделали сыновья \bibemph{Иакова} с ним, как он заповедал им;
\vs Gen 50:13 и отнесли его сыновья его в землю Ханаанскую и похоронили его в пещере на поле Махпела, которую купил Авраам с полем в собственность для погребения у Ефрона Хеттеянина, пред Мамре.
\rsbpar\vs Gen 50:14 И возвратился Иосиф в Египет, сам и братья его и все ходившие с ним хоронить отца его, после погребения им отца своего.
\vs Gen 50:15 И увидели братья Иосифовы, что умер отец их, и сказали: что, если Иосиф возненавидит нас и захочет отмстить нам за всё зло, которое мы ему сделали?
\vs Gen 50:16 И послали они сказать Иосифу: отец твой пред смертью своею завещал, говоря:
\vs Gen 50:17 так скажите Иосифу: прости братьям твоим вину и грех их, так как они сделали тебе зло. И ныне прости вины рабов Бога отца твоего. Иосиф плакал, когда ему говорили это.
\vs Gen 50:18 Пришли и сами братья его, и пали пред лицем его, и сказали: вот, мы рабы тебе.
\vs Gen 50:19 И сказал Иосиф: не бойтесь, ибо я боюсь Бога;
\vs Gen 50:20 вот, вы умышляли против меня зло; но Бог обратил это в добро, чтобы сделать то, что теперь есть: сохранить жизнь великому числу людей;
\vs Gen 50:21 итак не бойтесь: я буду питать вас и детей ваших. И успокоил их и говорил по сердцу их.
\rsbpar\vs Gen 50:22 И жил Иосиф в Египте сам и дом отца его; жил же Иосиф всего сто десять лет.
\vs Gen 50:23 И видел Иосиф детей у Ефрема до третьего рода, также и сыновья Махира, сына Манассиина, родились на колени Иосифа.
\vs Gen 50:24 И сказал Иосиф братьям своим: я умираю, но Бог посетит вас и выведет вас из земли сей в землю, о которой клялся Аврааму, Исааку и Иакову.
\vs Gen 50:25 И заклял Иосиф сынов Израилевых, говоря: Бог посетит вас, и вынесите кости мои отсюда.
\vs Gen 50:26 И умер Иосиф ста десяти лет. И набальзамировали его и положили в ковчег в Египте.

\bibbookdescr{Exo}{
  inline={\LARGE Вторая книга Моисеева\\\Huge Исход},
  toc={Исход},
  bookmark={Исход},
  header={Исход},
  %headerleft={},
  %headerright={},
  abbr={Исх}
}
\vs Exo 1:1 Вот имена сынов Израилевых, которые вошли в Египет с Иаковом [отцом их], вошли каждый со [всем] домом своим:
\vs Exo 1:2 Рувим, Симеон, Левий и Иуда,
\vs Exo 1:3 Иссахар, Завулон и Вениамин,
\vs Exo 1:4 Дан и Неффалим, Гад и Асир.
\vs Exo 1:5 Всех же душ, происшедших от чресл Иакова, было семьдесят [пять], а Иосиф был \bibemph{уже} в Египте.
\vs Exo 1:6 И умер Иосиф и все братья его и весь род их;
\vs Exo 1:7 а сыны Израилевы расплодились и размножились, и возросли и усилились чрезвычайно, и наполнилась ими земля та.
\rsbpar\vs Exo 1:8 И восстал в Египте новый царь, который не знал Иосифа,
\vs Exo 1:9 и сказал народу своему: вот, народ сынов Израилевых многочислен и сильнее нас;
\vs Exo 1:10 перехитрим же его, чтобы он не размножался; иначе, когда случится война, соединится и он с нашими неприятелями, и вооружится против нас, и выйдет из земли [нашей].
\vs Exo 1:11 И поставили над ним начальников работ, чтобы изнуряли его тяжкими работами. И он построил фараону Пифом и Раамсес, города для запасов, [и Он, иначе Илиополь].
\vs Exo 1:12 Но чем более изнуряли его, тем более он умножался и тем более возрастал, так что [Египтяне] опасались сынов Израилевых.
\vs Exo 1:13 И потому Египтяне с жестокостью принуждали сынов Израилевых к работам
\vs Exo 1:14 и делали жизнь их горькою от тяжкой работы над глиною и кирпичами и от всякой работы полевой, от всякой работы, к которой принуждали их с жестокостью.
\rsbpar\vs Exo 1:15 Царь Египетский повелел повивальным бабкам Евреянок, из коих одной имя Шифра, а другой Фуа,
\vs Exo 1:16 и сказал [им]: когда вы будете повивать у Евреянок, то наблюдайте при родах: если будет сын, то умерщвляйте его, а если дочь, то пусть живет.
\vs Exo 1:17 Но повивальные бабки боялись Бога и не делали так, как говорил им царь Египетский, и оставляли детей в живых.
\vs Exo 1:18 Царь Египетский призвал повивальных бабок и сказал им: для чего вы делаете такое дело, что оставляете детей в живых?
\vs Exo 1:19 Повивальные бабки сказали фараону: Еврейские женщины не так, как Египетские; они здоровы, ибо прежде нежели придет к ним повивальная бабка, они уже рождают.
\vs Exo 1:20 За сие Бог делал добро повивальным бабкам, а народ умножался и весьма усиливался.
\vs Exo 1:21 И так как повивальные бабки боялись Бога, то Он устроял домы их.
\vs Exo 1:22 Тогда фараон всему народу своему повелел, говоря: всякого новорожденного [у Евреев] сына бросайте в реку, а всякую дочь оставляйте в живых.
\vs Exo 2:1 Некто из племени Левиина пошел и взял себе жену из того же племени.
\vs Exo 2:2 Жена зачала и родила сына и, видя, что он очень красив, скрывала его три месяца;
\vs Exo 2:3 но не могши долее скрывать его, взяла корзинку из тростника и осмолила ее асфальтом и смолою и, положив в нее младенца, поставила в тростнике у берега реки,
\vs Exo 2:4 а сестра его стала вдали наблюдать, что с ним будет.
\vs Exo 2:5 И вышла дочь фараонова на реку мыться, а прислужницы ее ходили по берегу реки. Она увидела корзинку среди тростника и послала рабыню свою взять ее.
\vs Exo 2:6 Открыла и увидела младенца; и вот, дитя плачет [в корзинке]; и сжалилась над ним [дочь фараонова] и сказала: это из Еврейских детей.
\vs Exo 2:7 И сказала сестра его дочери фараоновой: не сходить ли мне и не позвать ли к тебе кормилицу из Евреянок, чтоб она вскормила тебе младенца?
\vs Exo 2:8 Дочь фараонова сказала ей: сходи. Девица пошла и призвала мать младенца.
\vs Exo 2:9 Дочь фараонова сказала ей: возьми младенца сего и вскорми его мне; я дам тебе плату. Женщина взяла младенца и кормила его.
\vs Exo 2:10 И вырос младенец, и она привела его к дочери фараоновой, и он был у нее вместо сына, и нарекла имя ему: Моисей, потому что, говорила она, я из воды вынула его.
\rsbpar\vs Exo 2:11 Спустя много времени, когда Моисей вырос, случилось, что он вышел к братьям своим [сынам Израилевым] и увидел тяжкие работы их; и увидел, что Египтянин бьет одного Еврея из братьев его, [сынов Израилевых].
\vs Exo 2:12 Посмотрев туда и сюда и видя, что нет никого, он убил Египтянина и скрыл его в песке.
\vs Exo 2:13 И вышел он на другой день, и вот, два Еврея ссорятся; и сказал он обижающему: зачем ты бьешь ближнего твоего?
\vs Exo 2:14 А тот сказал: кто поставил тебя начальником и судьею над нами? не думаешь ли убить меня, как убил [вчера] Египтянина? Моисей испугался и сказал: верно, узнали об этом деле.
\vs Exo 2:15 И услышал фараон об этом деле и хотел убить Моисея; но Моисей убежал от фараона и остановился в земле Мадиамской, и [придя в землю Мадиамскую] сел у колодезя.
\vs Exo 2:16 У священника Мадиамского [было] семь дочерей, [которые пасли овец отца своего Иофора]. Они пришли, начерпали \bibemph{воды} и наполнили корыта, чтобы напоить овец отца своего [Иофора].
\vs Exo 2:17 И пришли пастухи и отогнали их. Тогда встал Моисей и защитил их, [и начерпал им воды] и напоил овец их.
\vs Exo 2:18 И пришли они к Рагуилу, отцу своему, и он сказал [им]: что вы так скоро пришли сегодня?
\vs Exo 2:19 Они сказали: какой-то Египтянин защитил нас от пастухов, и даже начерпал нам воды и напоил овец [наших].
\vs Exo 2:20 Он сказал дочерям своим: где же он? зачем вы его оставили? позовите его, и пусть он ест хлеб.
\vs Exo 2:21 Моисею понравилось жить у сего человека; и он выдал за Моисея дочь свою Сепфору.
\vs Exo 2:22 Она [зачала и] родила сына, и [Моисей] нарек ему имя: Гирсам, потому что, говорил он, я стал пришельцем в чужой земле. [И зачав еще, родила другого сына, и он нарек ему имя: Елиезер, сказав: Бог отца моего был мне помощником и избавил меня от руки фараона.]
\rsbpar\vs Exo 2:23 Спустя долгое время, умер царь Египетский. И стенали сыны Израилевы от работы и вопияли, и вопль их от работы восшел к Богу.
\vs Exo 2:24 И услышал Бог стенание их, и вспомнил Бог завет Свой с Авраамом, Исааком и Иаковом.
\vs Exo 2:25 И увидел Бог сынов Израилевых, и призрел их Бог.
\vs Exo 3:1 Моисей пас овец у Иофора, тестя своего, священника Мадиамского. Однажды провел он стадо далеко в пустыню и пришел к горе Божией, Хориву.
\vs Exo 3:2 И явился ему Ангел Господень в пламени огня из среды тернового куста. И увидел он, что терновый куст горит огнем, но куст не сгорает.
\vs Exo 3:3 Моисей сказал: пойду и посмотрю на сие великое явление, отчего куст не сгорает.
\vs Exo 3:4 Господь увидел, что он идет смотреть, и воззвал к нему Бог из среды куста, и сказал: Моисей! Моисей! Он сказал: вот я, [Господи]!
\vs Exo 3:5 И сказал Бог: не подходи сюда; сними обувь твою с ног твоих, ибо место, на котором ты стоишь, есть земля святая.
\vs Exo 3:6 И сказал [ему]: Я Бог отца твоего, Бог Авраама, Бог Исаака и Бог Иакова. Моисей закрыл лице свое, потому что боялся воззреть на Бога.
\vs Exo 3:7 И сказал Господь [Моисею]: Я увидел страдание народа Моего в Египте и услышал вопль его от приставников его; Я знаю скорби его
\vs Exo 3:8 и иду избавить его от руки Египтян и вывести его из земли сей [и ввести его] в землю хорошую и пространную, где течет молоко и мед, в землю Хананеев, Хеттеев, Аморреев, Ферезеев, [Гергесеев,] Евеев и Иевусеев.
\vs Exo 3:9 И вот, уже вопль сынов Израилевых дошел до Меня, и Я вижу угнетение, каким угнетают их Египтяне.
\vs Exo 3:10 Итак пойди: Я пошлю тебя к фараону [царю Египетскому]; и выведи из Египта народ Мой, сынов Израилевых.
\vs Exo 3:11 Моисей сказал Богу: кто я, чтобы мне идти к фараону [царю Египетскому] и вывести из Египта сынов Израилевых?
\vs Exo 3:12 И сказал [Бог]: Я буду с тобою, и вот тебе знамение, что Я послал тебя: когда ты выведешь народ [Мой] из Египта, вы совершите служение Богу на этой горе.
\vs Exo 3:13 И сказал Моисей Богу: вот, я приду к сынам Израилевым и скажу им: Бог отцов ваших послал меня к вам. А они скажут мне: как Ему имя? Что сказать мне им?
\vs Exo 3:14 Бог сказал Моисею: Я есмь Сущий. И сказал: так скажи сынам Израилевым: Сущий [Иегова] послал меня к вам.
\vs Exo 3:15 И сказал еще Бог Моисею: так скажи сынам Израилевым: Господь, Бог отцов ваших, Бог Авраама, Бог Исаака и Бог Иакова послал меня к вам. Вот имя Мое на веки, и памятование о Мне из рода в род.
\vs Exo 3:16 Пойди, собери старейшин [сынов] Израилевых и скажи им: Господь, Бог отцов ваших, явился мне, Бог Авраама, [Бог] Исаака и [Бог] Иакова, и сказал: Я посетил вас и \bibemph{увидел}, что делается с вами в Египте.
\vs Exo 3:17 И сказал: Я выведу вас от угнетения Египетского в землю Хананеев, Хеттеев, Аморреев, Ферезеев, [Гергесеев,] Евеев и Иевусеев, в землю, где течет молоко и мед.
\vs Exo 3:18 И они послушают голоса твоего, и пойдешь ты и старейшины Израилевы к [фараону] царю Египетскому, и скажете ему: Господь, Бог Евреев, призвал нас; итак отпусти нас в пустыню, на три дня пути, чтобы принести жертву Господу, Богу нашему.
\vs Exo 3:19 Но Я знаю, что [фараон] царь Египетский не позволит вам идти, если \bibemph{не принудить его} рукою крепкою;
\vs Exo 3:20 и простру руку Мою и поражу Египет всеми чудесами Моими, которые сделаю среди его; и после того он отпустит вас.
\vs Exo 3:21 И дам народу сему милость в глазах Египтян; и когда пойдете, то пойдете не с пустыми руками:
\vs Exo 3:22 каждая женщина выпросит у соседки своей и у живущей в доме ее вещей серебряных и вещей золотых, и одежд, и вы нарядите ими и сыновей ваших и дочерей ваших, и оберете Египтян.
\vs Exo 4:1 И отвечал Моисей и сказал: а если они не поверят мне и не послушают голоса моего и скажут: не явился тебе Господь? [что сказать им?]
\vs Exo 4:2 И сказал ему Господь: что это в руке у тебя? Он отвечал: жезл.
\vs Exo 4:3 \bibemph{Господь} сказал: брось его на землю. Он бросил его на землю, и жезл превратился в змея, и Моисей побежал от него.
\vs Exo 4:4 И сказал Господь Моисею: простри руку твою и возьми его за хвост. Он простер руку свою, и взял его [за хвост]; и он стал жезлом в руке его.
\vs Exo 4:5 Это для того, чтобы поверили [тебе], что явился тебе Господь, Бог отцов их, Бог Авраама, Бог Исаака и Бог Иакова.
\vs Exo 4:6 Еще сказал ему Господь: положи руку твою к себе в пазуху. И он положил руку свою к себе в пазуху, вынул ее [из пазухи своей], и вот, рука его побелела от проказы, как снег.
\vs Exo 4:7 [Еще] сказал [ему Господь]: положи опять руку твою к себе в пазуху. И он положил руку свою к себе в пазуху; и вынул ее из пазухи своей, и вот, она опять стала такою же, как тело его.
\vs Exo 4:8 Если они не поверят тебе и не послушают голоса первого знамения, то поверят голосу знамения другого;
\vs Exo 4:9 если же не поверят и двум сим знамениям и не послушают голоса твоего, то возьми воды \bibemph{из} реки и вылей на сушу; и вода, взятая из реки, сделается кровью на суше.
\rsbpar\vs Exo 4:10 И сказал Моисей Господу: о, Господи! человек я не речистый, \bibemph{и таков был} и вчера и третьего дня, и когда Ты начал говорить с рабом Твоим: я тяжело говорю и косноязычен.
\vs Exo 4:11 Господь сказал [Моисею]: кто дал уста человеку? кто делает немым, или глухим, или зрячим, или слепым? не Я ли Господь [Бог]?
\vs Exo 4:12 итак пойди, и Я буду при устах твоих и научу тебя, что тебе говорить.
\vs Exo 4:13 [Моисей] сказал: Господи! пошли другого, кого можешь послать.
\vs Exo 4:14 И возгорелся гнев Господень на Моисея, и Он сказал: разве нет у тебя Аарона брата, Левитянина? Я знаю, что он может говорить [вместо тебя], и вот, он выйдет навстречу тебе, и, увидев тебя, возрадуется в сердце своем;
\vs Exo 4:15 ты будешь ему говорить и влагать слова [Мои] в уста его, а Я буду при устах твоих и при устах его и буду учить вас, что вам делать;
\vs Exo 4:16 и будет говорить он вместо тебя к народу; итак он будет твоими устами, а ты будешь ему вместо Бога;
\vs Exo 4:17 и жезл сей [который был обращен в змея] возьми в руку твою: им ты будешь творить знамения.
\rsbpar\vs Exo 4:18 И пошел Моисей, и возвратился к Иофору, тестю своему, и сказал ему: пойду я, и возвращусь к братьям моим, которые в Египте, и посмотрю, живы ли еще они? И сказал Иофор Моисею: иди с миром. [Спустя много времени умер царь Египетский.]
\rsbpar\vs Exo 4:19 И сказал Господь Моисею в [земле] Мадиамской: пойди, возвратись в Египет, ибо умерли все, искавшие души твоей.
\vs Exo 4:20 И взял Моисей жену свою и сыновей своих, посадил их на осла и отправился в землю Египетскую. И жезл Божий Моисей взял в руку свою.
\vs Exo 4:21 И сказал Господь Моисею: когда пойдешь и возвратишься в Египет, смотри, все чудеса, которые Я поручил тебе, сделай пред лицем фараона, а Я ожесточу сердце его, и он не отпустит народа.
\vs Exo 4:22 И скажи фараону: так говорит Господь [Бог Еврейский]: Израиль \bibemph{есть} сын Мой, первенец Мой;
\vs Exo 4:23 Я говорю тебе: отпусти сына Моего, чтобы он совершил Мне служение; а если не отпустишь его, то вот, Я убью сына твоего, первенца твоего.
\rsbpar\vs Exo 4:24 Дорогою на ночлеге случилось, что встретил его Господь и хотел умертвить его.
\vs Exo 4:25 Тогда Сепфора, взяв каменный нож, обрезала крайнюю плоть сына своего и, бросив к ногам его, сказала: ты жених крови у меня.
\vs Exo 4:26 И отошел от него \bibemph{Господь}. Тогда сказала она: жених крови~--- по обрезанию.
\rsbpar\vs Exo 4:27 И Господь сказал Аарону: пойди навстречу Моисею в пустыню. И он пошел, и встретился с ним при горе Божией, и поцеловал его.
\vs Exo 4:28 И пересказал Моисей Аарону все слова Господа, Который его послал, и все знамения, которые Он заповедал.
\vs Exo 4:29 И пошел Моисей с Аароном, и собрали они всех старейшин сынов Израилевых,
\vs Exo 4:30 и пересказал [им] Аарон все слова, которые говорил Господь Моисею; и сделал \bibemph{Моисей} знамения пред глазами народа,
\vs Exo 4:31 и поверил народ; и услышали, что Господь посетил сынов Израилевых и увидел страдание их, и преклонились они и поклонились.
\vs Exo 5:1 После сего Моисей и Аарон пришли к фараону и сказали [ему]: так говорит Господь, Бог Израилев: отпусти народ Мой, чтоб он совершил Мне праздник в пустыне.
\vs Exo 5:2 Но фараон сказал: кто такой Господь, чтоб я послушался голоса Его \bibemph{и} отпустил [сынов] Израиля? я не знаю Господа и Израиля не отпущу.
\vs Exo 5:3 Они сказали [ему]: Бог Евреев призвал нас; отпусти нас в пустыню на три дня пути принести жертву Господу, Богу нашему, чтобы Он не поразил нас язвою, или мечом.
\vs Exo 5:4 И сказал им царь Египетский: для чего вы, Моисей и Аарон, отвлекаете народ [мой] от дел его? ступайте [каждый из вас] на свою работу.
\vs Exo 5:5 И сказал фараон: вот, народ в земле сей многочислен, и вы отвлекаете его от работ его.
\rsbpar\vs Exo 5:6 И в тот же день фараон дал повеление приставникам над народом и надзирателям, говоря:
\vs Exo 5:7 не давайте впредь народу соломы для делания кирпича, как вчера и третьего дня, пусть они сами ходят и собирают себе солому,
\vs Exo 5:8 а кирпичей наложите на них то же урочное число, какое они делали вчера и третьего дня, и не убавляйте; они праздны, потому и кричат: пойдем, принесем жертву Богу нашему;
\vs Exo 5:9 дать им больше работы, чтоб они работали и не занимались пустыми речами.
\vs Exo 5:10 И вышли приставники народа и надзиратели его и сказали народу: так говорит фараон: не даю вам соломы;
\vs Exo 5:11 сами пойдите, берите себе солому, где найдете, а от работы вашей ничего не убавляется.
\vs Exo 5:12 И рассеялся народ по всей земле Египетской собирать жниво вместо соломы.
\vs Exo 5:13 Приставники же понуждали [их], говоря: выполняйте [урочную] работу свою каждый день, как и тогда, когда была \bibemph{у вас} солома.
\vs Exo 5:14 А надзирателей из сынов Израилевых, которых поставили над ними приставники фараоновы, били, говоря: почему вы вчера и сегодня не изготовляете урочного числа кирпичей, как было до сих пор?
\vs Exo 5:15 И пришли надзиратели сынов Израилевых и возопили к фараону, говоря: для чего ты так поступаешь с рабами твоими?
\vs Exo 5:16 соломы не дают рабам твоим, а кирпичи, говорят нам, делайте. И вот, рабов твоих бьют; грех народу твоему.
\vs Exo 5:17 Но он сказал [им]: праздны вы, праздны, поэтому и говорите: пойдем, принесем жертву Господу.
\vs Exo 5:18 Пойдите же, работайте; соломы не дадут вам, а положенное число кирпичей давайте.
\vs Exo 5:19 И увидели надзиратели сынов Израилевых беду свою в словах: не убавляйте числа кирпичей, какое [положено] на каждый день.
\vs Exo 5:20 И когда они вышли от фараона, то встретились с Моисеем и Аароном, которые стояли, ожидая их,
\vs Exo 5:21 и сказали им: да видит и судит вам Господь за то, что вы сделали нас ненавистными в глазах фараона и рабов его и дали им меч в руки, чтобы убить нас.
\vs Exo 5:22 И обратился Моисей к Господу и сказал: Господи! для чего Ты подвергнул такому бедствию народ сей, [и] для чего послал меня?
\vs Exo 5:23 ибо с того времени, как я пришел к фараону и стал говорить именем Твоим, он начал хуже поступать с народом сим; избавить же,~--- Ты не избавил народа Твоего.
\vs Exo 6:1 И сказал Господь Моисею: теперь увидишь ты, что Я сделаю с фараоном; по действию руки крепкой он отпустит их; по действию руки крепкой даже выгонит их из земли своей.
\vs Exo 6:2 И говорил Бог Моисею и сказал ему: Я Господь.
\vs Exo 6:3 Являлся Я Аврааму, Исааку и Иакову с \bibemph{именем} <<Бог Всемогущий>>, а с именем \bibemph{Моим} <<Господь>>\fns{Иегова.} не открылся им;
\vs Exo 6:4 и Я поставил завет Мой с ними, чтобы дать им землю Ханаанскую, землю странствования их, в которой они странствовали.
\vs Exo 6:5 И Я услышал стенание сынов Израилевых о том, что Египтяне держат их в рабстве, и вспомнил завет Мой.
\vs Exo 6:6 Итак скажи сынам Израилевым: Я Господь, и выведу вас из-под ига Египтян, и избавлю вас от рабства их, и спасу вас мышцею простертою и судами великими;
\vs Exo 6:7 и приму вас Себе в народ и буду вам Богом, и вы узнаете, что Я Господь, Бог ваш, изведший вас [из земли Египетской] из-под ига Египетского;
\vs Exo 6:8 и введу вас в ту землю, о которой Я, подняв руку Мою, \bibemph{клялся} дать ее Аврааму, Исааку и Иакову, и дам вам ее в наследие. Я Господь.
\vs Exo 6:9 Моисей пересказал это сынам Израилевым; но они не послушали Моисея по малодушию и тяжести работ.
\vs Exo 6:10 И сказал Господь Моисею, говоря:
\vs Exo 6:11 войди, скажи фараону, царю Египетскому, чтобы он отпустил сынов Израилевых из земли своей.
\vs Exo 6:12 И сказал Моисей пред Господом, говоря: вот, сыны Израилевы не слушают меня; как же послушает меня фараон? а я не словесен.
\vs Exo 6:13 И говорил Господь Моисею и Аарону, и давал им повеления к сынам Израилевым и к фараону, царю Египетскому, чтобы вывести сынов Израилевых из земли Египетской.
\rsbpar\vs Exo 6:14 Вот начальники поколений их: сыны Рувима, первенца Израилева: Ханох и Фаллу, Хецрон и Харми: это семейства Рувимовы.
\vs Exo 6:15 Сыны Симеона: Иемуил и Иамин, и Огад, и Иахин, и Цохар, и Саул, сын Хананеянки: это семейства Симеона.
\vs Exo 6:16 Вот имена сынов Левия по родам их: Гирсон и Кааф и Мерари. А лет жизни Левия было сто тридцать семь.
\vs Exo 6:17 Сыны Гирсона: Ливни и Шимеи с семействами их.
\vs Exo 6:18 Сыны Каафовы: Амрам и Ицгар, и Хеврон, и Узиил. А лет жизни Каафа было сто тридцать три года.
\vs Exo 6:19 Сыны Мерари: Махли и Муши. Это семейства Левия по родам их.
\vs Exo 6:20 Амрам взял Иохаведу, тетку свою, себе в жену, и она родила ему Аарона и Моисея [и Мариам, сестру их]. А лет жизни Амрама было сто тридцать семь.
\vs Exo 6:21 Сыны Ицгаровы: Корей и Нефег и Зихри.
\vs Exo 6:22 Сыны Узииловы: Мисаил и Елцафан и Сифри.
\vs Exo 6:23 Аарон взял себе в жену Елисавету, дочь Аминадава, сестру Наассона, и она родила ему Надава и Авиуда, Елеазара и Ифамара.
\vs Exo 6:24 Сыны Корея: Асир, Елкана и Авиасаф: это семейства Кореевы.
\vs Exo 6:25 Елеазар, сын Аарона, взял себе в жену \bibemph{одну} из дочерей Футииловых, и она родила ему Финееса. Вот начальники поколений левитских по семействам их.
\vs Exo 6:26 Аарон и Моисей, это~--- те, которым сказал Господь: выведите сынов Израилевых из земли Египетской по ополчениям их.
\vs Exo 6:27 Они-то говорили фараону, царю Египетскому, чтобы вывести сынов Израилевых из Египта; это~--- Моисей и Аарон.
\vs Exo 6:28 Итак в то время, когда Господь говорил Моисею в земле Египетской,
\vs Exo 6:29 Господь сказал Моисею, говоря: Я Господь! скажи фараону, царю Египетскому, всё, что Я говорю тебе.
\vs Exo 6:30 Моисей же сказал пред Господом: вот, я несловесен: как же послушает меня фараон?
\vs Exo 7:1 Но Господь сказал Моисею: смотри, Я поставил тебя Богом фараону, а Аарон, брат твой, будет твоим пророком:
\vs Exo 7:2 ты будешь говорить [ему] все, что Я повелю тебе, а Аарон, брат твой, будет говорить фараону, чтобы он отпустил сынов Израилевых из земли своей;
\vs Exo 7:3 но Я ожесточу сердце фараоново, и явлю множество знамений Моих и чудес Моих в земле Египетской;
\vs Exo 7:4 фараон не послушает вас, и Я наложу руку Мою на Египет и выведу воинство Мое, народ Мой, сынов Израилевых, из земли Египетской~--- судами великими;
\vs Exo 7:5 тогда узнают [все] Египтяне, что Я Господь, когда простру руку Мою на Египет и выведу сынов Израилевых из среды их.
\vs Exo 7:6 И сделали Моисей и Аарон, как повелел им Господь, так они и сделали.
\vs Exo 7:7 Моисей \bibemph{был} восьмидесяти, а Аарон [брат его] восьмидесяти трех лет, когда стали говорить они к фараону.
\rsbpar\vs Exo 7:8 И сказал Господь Моисею и Аарону, говоря:
\vs Exo 7:9 если фараон скажет вам: сделайте [знамение или] чудо, то ты скажи Аарону [брату твоему]: возьми жезл твой и брось [на землю] пред фараоном [и пред рабами его],~--- он сделается змеем.
\vs Exo 7:10 Моисей и Аарон пришли к фараону [и к рабам его] и сделали так, как повелел [им] Господь. И бросил Аарон жезл свой пред фараоном и пред рабами его, и он сделался змеем.
\vs Exo 7:11 И призвал фараон мудрецов [Египетских] и чародеев; и эти волхвы Египетские сделали то же своими чарами:
\vs Exo 7:12 каждый из них бросил свой жезл, и они сделались змеями, но жезл Ааронов поглотил их жезлы.
\vs Exo 7:13 Сердце фараоново ожесточилось, и он не послушал их, как и говорил [им] Господь.
\rsbpar\vs Exo 7:14 И сказал Господь Моисею: упорно сердце фараоново: он не хочет отпустить народ.
\vs Exo 7:15 Пойди к фараону завтра: вот, он выйдет к воде, ты стань на пути его, на берегу реки, и жезл, который превращался в змея, возьми в руку твою
\vs Exo 7:16 и скажи ему: Господь, Бог Евреев, послал меня сказать тебе: отпусти народ Мой, чтобы он совершил Мне служение в пустыне; но вот, ты доселе не послушался.
\vs Exo 7:17 Так говорит Господь: из сего узнаешь, что Я Господь: вот этим жезлом, который в руке моей, я ударю по воде, которая в реке, и она превратится в кровь,
\vs Exo 7:18 и рыба в реке умрет, и река воссмердит, и Египтянам омерзительно будет пить воду из реки.
\vs Exo 7:19 И сказал Господь Моисею: скажи Аарону [брату твоему]: возьми жезл твой [в руку твою] и простри руку твою на воды Египтян: на реки их, на потоки их, на озера их и на всякое вместилище вод их,~--- и превратятся в кровь, и будет кровь по всей земле Египетской и в деревянных и в каменных сосудах.
\vs Exo 7:20 И сделали Моисей и Аарон, как повелел [им] Господь. И поднял [Аарон] жезл [свой] и ударил по воде речной пред глазами фараона и пред глазами рабов его, и вся вода в реке превратилась в кровь,
\vs Exo 7:21 и рыба в реке вымерла, и река воссмердела, и Египтяне не могли пить воды из реки; и была кровь по всей земле Египетской.
\vs Exo 7:22 И волхвы Египетские чарами своими сделали то же. И ожесточилось сердце фараона, и не послушал их, как и говорил Господь.
\vs Exo 7:23 И оборотился фараон, и пошел в дом свой; и сердце его не тронулось и сим.
\vs Exo 7:24 И стали копать все Египтяне около реки \bibemph{чтобы найти} воду для питья, потому что не могли пить воды из реки.
\vs Exo 7:25 И исполнилось семь дней после того, как Господь поразил реку.
\vs Exo 8:1 И сказал Господь Моисею: пойди к фараону и скажи ему: так говорит Господь: отпусти народ Мой, чтобы он совершил Мне служение;
\vs Exo 8:2 если же ты не согласишься отпустить, то вот, Я поражаю всю область твою жабами;
\vs Exo 8:3 и воскишит река жабами, и они выйдут и войдут в дом твой, и в спальню твою, и на постель твою, и в домы рабов твоих и народа твоего, и в печи твои, и в квашни твои,
\vs Exo 8:4 и на тебя, и на народ твой, и на всех рабов твоих взойдут жабы.
\vs Exo 8:5 И сказал Господь Моисею: скажи Аарону [брату твоему]: простри руку твою с жезлом твоим на реки, на потоки и на озера и выведи жаб на землю Египетскую.
\vs Exo 8:6 Аарон простер руку свою на воды Египетские [и вывел жаб]; и вышли жабы и покрыли землю Египетскую.
\vs Exo 8:7 То же сделали и волхвы [Египетские] чарами своими и вывели жаб на землю Египетскую.
\vs Exo 8:8 И призвал фараон Моисея и Аарона и сказал: помолитесь [обо мне] Господу, чтоб Он удалил жаб от меня и от народа моего, и я отпущу народ \bibemph{Израильский} принести жертву Господу.
\vs Exo 8:9 Моисей сказал фараону: назначь мне сам, когда помолиться за тебя, за рабов твоих и за народ твой, чтобы жабы исчезли у тебя, [у народа твоего,] в домах твоих, и остались только в реке.
\vs Exo 8:10 Он сказал: завтра. \bibemph{Моисей} отвечал: \bibemph{будет} по слову твоему, дабы ты узнал, что нет никого, как Господь Бог наш;
\vs Exo 8:11 и удалятся жабы от тебя, от домов твоих [и с полей], и от рабов твоих и от твоего народа; только в реке они останутся.
\vs Exo 8:12 Моисей и Аарон вышли от фараона, и Моисей воззвал к Господу о жабах, которых Он навел на фараона.
\vs Exo 8:13 И сделал Господь по слову Моисея: жабы вымерли в домах, на дворах и на полях [их];
\vs Exo 8:14 и собрали их в груды, и воссмердела земля.
\vs Exo 8:15 И увидел фараон, что сделалось облегчение, и ожесточил сердце свое, и не послушал их, как и говорил Господь.
\rsbpar\vs Exo 8:16 И сказал Господь Моисею: скажи Аарону: простри [рукою] жезл твой и ударь в персть земную, и [будут мошки на людях и на скоте и на фараоне, и в доме его и на рабах его, вся персть земная] сделается мошками по всей земле Египетской.
\vs Exo 8:17 Так они и сделали: Аарон простер руку свою с жезлом своим и ударил в персть земную, и явились мошки на людях и на скоте. Вся персть земная сделалась мошками по всей земле Египетской.
\vs Exo 8:18 Старались также и волхвы чарами своими произвести мошек, но не могли. И были мошки на людях и на скоте.
\vs Exo 8:19 И сказали волхвы фараону: это перст Божий. Но сердце фараоново ожесточилось, и он не послушал их, как и говорил Господь.
\rsbpar\vs Exo 8:20 И сказал Господь Моисею: завтра встань рано и явись пред лице фараона. Вот, он пойдет к воде, и ты скажи ему: так говорит Господь: отпусти народ Мой, чтобы он совершил Мне служение [в пустыне];
\vs Exo 8:21 а если не отпустишь народа Моего, то вот, Я пошлю на тебя и на рабов твоих, и на народ твой, и в домы твои песьих мух, и наполнятся домы Египтян песьими мухами и самая земля, на которой они \bibemph{живут};
\vs Exo 8:22 и отделю в тот день землю Гесем, на которой пребывает народ Мой, и там не будет песьих мух, дабы ты знал, что Я Господь [Бог] среди [всей] земли;
\vs Exo 8:23 Я сделаю разделение между народом Моим и между народом твоим. Завтра будет сие знамение [на земле].
\vs Exo 8:24 Так и сделал Господь: налетело множество песьих мух в дом фараонов, и в домы рабов его, и на всю землю Египетскую: погибала земля от песьих мух.
\vs Exo 8:25 И призвал фараон Моисея и Аарона и сказал: пойдите, принесите жертву [Господу] Богу вашему в сей земле.
\vs Exo 8:26 Но Моисей сказал: нельзя сего сделать, ибо отвратительно для Египтян жертвоприношение наше Господу, Богу нашему: если мы отвратительную для Египтян жертву станем приносить в глазах их, то не побьют ли они нас камнями?
\vs Exo 8:27 мы пойдем в пустыню, на три дня пути, и принесем жертву Господу, Богу нашему, как скажет нам [Господь].
\vs Exo 8:28 И сказал фараон: я отпущу вас принести жертву Господу Богу вашему в пустыне, только не уходите далеко; помолитесь обо мне [Господу].
\vs Exo 8:29 Моисей сказал: вот, я выхожу от тебя и помолюсь Господу [Богу], и удалятся песьи мухи от фараона, и от рабов его, и от народа его завтра, только фараон пусть перестанет обманывать, не отпуская народа принести жертву Господу.
\vs Exo 8:30 И вышел Моисей от фараона и помолился Господу.
\vs Exo 8:31 И сделал Господь по слову Моисея и удалил песьих мух от фараона, от рабов его и от народа его: не осталось ни одной.
\vs Exo 8:32 Но фараон ожесточил сердце свое и на этот раз и не отпустил народа.
\vs Exo 9:1 И сказал Господь Моисею: пойди к фараону и скажи ему: так говорит Господь, Бог Евреев: отпусти народ Мой, чтобы он совершил Мне служение;
\vs Exo 9:2 ибо если ты не захочешь отпустить [народ Мой] и еще будешь удерживать его,
\vs Exo 9:3 то вот, рука Господня будет на скоте твоем, который в поле, на конях, на ослах, на верблюдах, на волах и овцах: будет моровая язва весьма тяжкая;
\vs Exo 9:4 и разделит Господь [в то время] между скотом Израильским и скотом Египетским, и из всего [скота] сынов Израилевых не умрет ничего.
\vs Exo 9:5 И назначил Господь время, сказав: завтра сделает это Господь в земле сей.
\vs Exo 9:6 И сделал это Господь на другой день, и вымер весь скот Египетский; из скота же сынов Израилевых не умерло ничего.
\vs Exo 9:7 Фараон послал \bibemph{узнать}, и вот, из [всего] скота [сынов] Израилевых не умерло ничего. Но сердце фараоново ожесточилось, и он не отпустил народа.
\rsbpar\vs Exo 9:8 И сказал Господь Моисею и Аарону: возьмите по полной горсти пепла из печи, и пусть бросит его Моисей к небу в глазах фараона [и рабов его];
\vs Exo 9:9 и поднимется пыль по всей земле Египетской, и будет на людях и на скоте воспаление с нарывами, во всей земле Египетской.
\vs Exo 9:10 Они взяли пепла из печи и предстали пред лице фараона. Моисей бросил его к небу, и сделалось воспаление с нарывами на людях и на скоте.
\vs Exo 9:11 И не могли волхвы устоять пред Моисеем по причине воспаления, потому что воспаление было на волхвах и на всех Египтянах.
\vs Exo 9:12 Но Господь ожесточил сердце фараона, и он не послушал их, как и говорил Господь Моисею.
\rsbpar\vs Exo 9:13 И сказал Господь Моисею: завтра встань рано и явись пред лице фараона, и скажи ему: так говорит Господь, Бог Евреев: отпусти народ Мой, чтобы он совершил Мне служение;
\vs Exo 9:14 ибо в этот раз Я пошлю все язвы Мои в сердце твое, и на рабов твоих, и на народ твой, дабы ты узнал, что нет подобного Мне на всей земле;
\vs Exo 9:15 так как Я простер руку Мою, то поразил бы тебя и народ твой язвою, и ты истреблен был бы с земли:
\vs Exo 9:16 но для того Я сохранил тебя, чтобы показать на тебе силу Мою, и чтобы возвещено было имя Мое по всей земле;
\vs Exo 9:17 ты еще противостоишь народу Моему, чтобы не отпускать его,~---
\vs Exo 9:18 вот, Я пошлю завтра, в это самое время, град весьма сильный, которому подобного не было в Египте со дня основания его доныне;
\vs Exo 9:19 итак пошли собрать стада твои и все, что есть у тебя в поле: на всех людей и скот, которые останутся в поле и не соберутся в домы, падет град, и они умрут.
\vs Exo 9:20 Те из рабов фараоновых, которые убоялись слова Господня, поспешно собрали рабов своих и стада свои в домы;
\vs Exo 9:21 а кто не обратил сердца своего к слову Господню, тот оставил рабов своих и стада свои в поле.
\vs Exo 9:22 И сказал Господь Моисею: простри руку твою к небу, и падет град на всю землю Египетскую, на людей, на скот и на всю траву полевую в земле Египетской.
\vs Exo 9:23 И простер Моисей жезл свой к небу, и Господь произвел гром и град, и огонь разливался по земле; и послал Господь град на [всю] землю Египетскую;
\vs Exo 9:24 и был град и огонь между градом, [град] весьма сильный, какого не было во всей земле Египетской со времени населения ее.
\vs Exo 9:25 И побил град по всей земле Египетской все, что было в поле, от человека до скота, и всю траву полевую побил град, и все деревья в поле поломал [град];
\vs Exo 9:26 только в земле Гесем, где жили сыны Израилевы, не было града.
\vs Exo 9:27 И послал фараон, и призвал Моисея и Аарона, и сказал им: на этот раз я согрешил; Господь праведен, а я и народ мой виновны;
\vs Exo 9:28 помолитесь [обо мне] Господу: пусть перестанут громы Божии и град [и огонь на земле], и отпущу вас и не буду более удерживать.
\vs Exo 9:29 Моисей сказал ему: как скоро я выйду из города, простру руки мои к Господу [на небо], громы перестанут, и града [и дождя] более не будет, дабы ты узнал, что Господня земля;
\vs Exo 9:30 но я знаю, что ты и рабы твои еще не убоитесь Господа Бога.
\vs Exo 9:31 Лен и ячмень были побиты, потому что ячмень выколосился, а лен осеменился;
\vs Exo 9:32 а пшеница и полба не побиты, потому что они были поздние.
\vs Exo 9:33 И вышел Моисей от фараона из города и простер руки свои к Господу, и прекратились гром и град, и дождь перестал литься на землю.
\vs Exo 9:34 И увидел фараон, что перестал дождь и град и гром, и продолжал грешить, и отягчил сердце свое сам и рабы его.
\vs Exo 9:35 И ожесточилось сердце фараона [и рабов его], и он не отпустил сынов Израилевых, как и говорил Господь чрез Моисея.
\vs Exo 10:1 И сказал Господь Моисею: войди к фараону, ибо Я отягчил сердце его и сердце рабов его, чтобы явить между ними сии знамения Мои,
\vs Exo 10:2 и чтобы ты рассказывал сыну твоему и сыну сына твоего о том, что Я сделал в Египте, и о знамениях Моих, которые Я показал в нем, и чтобы вы знали, что Я Господь.
\vs Exo 10:3 Моисей и Аарон пришли к фараону и сказали ему: так говорит Господь, Бог Евреев: долго ли ты не смиришься предо Мною? отпусти народ Мой, чтобы он совершил Мне служение;
\vs Exo 10:4 а если ты не отпустишь народа Моего, то вот, завтра [в это самое время] Я наведу саранчу на [всю] твою область:
\vs Exo 10:5 она покроет лице земли так, что нельзя будет видеть земли, и поест у вас [все] оставшееся [на земле], уцелевшее от града; объест также все дерева, растущие у вас в поле,
\vs Exo 10:6 и наполнит домы твои, домы всех рабов твоих и [все] домы всех Египтян, чего не видели отцы твои, ни отцы отцов твоих со дня, как живут на земле, даже до сего дня. [Моисей] обратился и вышел от фараона.
\vs Exo 10:7 Тогда рабы фараоновы сказали ему: долго ли он будет мучить нас? отпусти сих людей, пусть они совершат служение Господу, Богу своему; неужели ты еще не видишь, что Египет гибнет?
\vs Exo 10:8 И возвратили Моисея и Аарона к фараону, и [фараон] сказал им: пойдите, совершите служение Господу, Богу вашему; кто же и кто пойдет?
\vs Exo 10:9 И сказал Моисей: пойдем с малолетними нашими и стариками нашими, с сыновьями нашими и дочерями нашими, и с овцами нашими и с волами нашими пойдем, ибо у нас праздник Господу [Богу нашему].
\vs Exo 10:10 [Фараон] сказал им: пусть будет так, Господь с вами! я готов отпустить вас: но зачем с детьми? видите, у вас худое намерение!
\vs Exo 10:11 нет: пойдите \bibemph{одни} мужчины и совершите служение Господу, так как вы сего просили. И выгнали их от фараона.
\rsbpar\vs Exo 10:12 Тогда Господь сказал Моисею: простри руку твою на землю Египетскую, и пусть нападет саранча на землю Египетскую и поест всю траву земную [и все плоды древесные], всё, что уцелело от града.
\vs Exo 10:13 И простер Моисей жезл свой на землю Египетскую, и Господь навел на сию землю восточный ветер, \bibemph{продолжавшийся} весь тот день и всю ночь. Настало утро, и восточный ветер нанес саранчу.
\vs Exo 10:14 И напала саранча на всю землю Египетскую и легла по всей стране Египетской в великом множестве: прежде не бывало такой саранчи, и после сего не будет такой;
\vs Exo 10:15 она покрыла лице всей земли, так что земли не было видно, и поела всю траву земную и все плоды древесные, уцелевшие от града, и не осталось никакой зелени ни на деревах, ни на траве полевой во всей земле Египетской.
\vs Exo 10:16 Фараон поспешно призвал Моисея и Аарона и сказал: согрешил я пред Господом, Богом вашим, и пред вами;
\vs Exo 10:17 теперь простите грех мой еще раз и помолитесь Господу Богу вашему, чтобы Он только отвратил от меня сию смерть.
\vs Exo 10:18 [Моисей] вышел от фараона и помолился Господу.
\vs Exo 10:19 И воздвигнул Господь с противной стороны западный весьма сильный ветер, и он понес саранчу и бросил ее в Чермное море: не осталось ни одной саранчи во всей стране Египетской.
\vs Exo 10:20 Но Господь ожесточил сердце фараона, и он не отпустил сынов Израилевых.
\rsbpar\vs Exo 10:21 И сказал Господь Моисею: простри руку твою к небу, и будет тьма на земле Египетской, осязаемая тьма.
\vs Exo 10:22 Моисей простер руку свою к небу, и была густая тьма по всей земле Египетской три дня;
\vs Exo 10:23 не видели друг друга, и никто не вставал с места своего три дня; у всех же сынов Израилевых был свет в жилищах их.
\vs Exo 10:24 Фараон призвал Моисея [и Аарона] и сказал: пойдите, совершите служение Господу [Богу вашему], пусть только останется мелкий и крупный скот ваш, а дети ваши пусть идут с вами.
\vs Exo 10:25 Но Моисей сказал: [нет,] дай также в руки наши жертвы и всесожжения, чтобы принести Господу Богу нашему;
\vs Exo 10:26 пусть пойдут и стада наши с нами, не останется ни копыта; ибо из них мы возьмем на жертву Господу, Богу нашему; но доколе не придем туда, мы не знаем, что принести в жертву Господу [Богу нашему].
\vs Exo 10:27 И ожесточил Господь сердце фараона, и он не захотел отпустить их.
\vs Exo 10:28 И сказал ему фараон: пойди от меня; берегись, не являйся более пред лице мое; в тот день, когда ты увидишь лице мое, умрешь.
\vs Exo 10:29 И сказал Моисей: как сказал ты, так и будет; я не увижу более лица твоего.
\vs Exo 11:1 И сказал Господь Моисею: еще одну казнь Я наведу на фараона и на Египтян; после того он отпустит вас отсюда; когда же он будет отпускать [вас], с поспешностью будет гнать вас отсюда;
\vs Exo 11:2 внуши народу [тайно], чтобы каждый у ближнего своего и каждая женщина у ближней своей выпросили вещей серебряных и вещей золотых [и одежд].
\vs Exo 11:3 И дал Господь милость народу [Своему] в глазах Египтян, [и они давали ему;] да и Моисей был весьма велик в земле Египетской, в глазах [фараона и] рабов фараоновых и в глазах [всего] народа.
\vs Exo 11:4 И сказал Моисей: так говорит Господь: в полночь Я пройду посреди Египта,
\vs Exo 11:5 и умрет всякий первенец в земле Египетской от первенца фараона, который сидит на престоле своем, до первенца рабыни, которая при жерновах, и всё первородное из скота;
\vs Exo 11:6 и будет вопль великий по всей земле Египетской, какого не бывало и какого не будет более;
\vs Exo 11:7 у всех же сынов Израилевых ни на человека, ни на скот не пошевелит пес языком своим, дабы вы знали, какое различие делает Господь между Египтянами и между Израильтянами.
\vs Exo 11:8 И придут все рабы твои сии ко мне и поклонятся мне, говоря: выйди ты и весь народ [твой], которым ты предводительствуешь. После сего я и выйду. И вышел [Моисей] от фараона с гневом.
\rsbpar\vs Exo 11:9 И сказал Господь Моисею: не послушал вас фараон, чтобы умножились [знамения Мои и] чудеса Мои в земле Египетской.
\vs Exo 11:10 Моисей и Аарон сделали все сии [знамения и] чудеса пред фараоном; но Господь ожесточил сердце фараона, и он не отпустил сынов Израилевых из земли своей.
\vs Exo 12:1 И сказал Господь Моисею и Аарону в земле Египетской, говоря:
\vs Exo 12:2 месяц сей \bibemph{да будет} у вас началом месяцев, первым \bibemph{да будет} он у вас между месяцами года.
\vs Exo 12:3 Скажите всему обществу [сынов] Израилевых: в десятый \bibemph{день} сего месяца пусть возьмут себе каждый одного агнца по семействам, по агнцу на семейство;
\vs Exo 12:4 а если семейство так мало, что не \bibemph{съест} агнца, то пусть возьмет с соседом своим, ближайшим к дому своему, по числу душ: по той мере, сколько каждый съест, расчислитесь на агнца.
\vs Exo 12:5 Агнец у вас должен быть без порока, мужеского пола, однолетний; возьмите его от овец, или от коз,
\vs Exo 12:6 и пусть он хранится у вас до четырнадцатого дня сего месяца: тогда пусть заколет его все собрание общества Израильского вечером,
\vs Exo 12:7 и пусть возьмут от крови \bibemph{его} и помажут на обоих косяках и на перекладине дверей в домах, где будут есть его;
\vs Exo 12:8 пусть съедят мясо его в сию самую ночь, испеченное на огне; с пресным хлебом и с горькими \bibemph{травами} пусть съедят его;
\vs Exo 12:9 не ешьте от него недопеченного, или сваренного в воде, но ешьте испеченное на огне, голову с ногами и внутренностями;
\vs Exo 12:10 не оставляйте от него до утра [и кости его не сокрушайте], но оставшееся от него до утра сожгите на огне.
\vs Exo 12:11 Ешьте же его так: пусть будут чресла ваши препоясаны, обувь ваша на ногах ваших и посохи ваши в руках ваших, и ешьте его с поспешностью: это~--- Пасха Господня.
\vs Exo 12:12 А Я в сию самую ночь пройду по земле Египетской и поражу всякого первенца в земле Египетской, от человека до скота, и над всеми богами Египетскими произведу суд. Я Господь.
\vs Exo 12:13 И будет у вас кровь знамением на домах, где вы находитесь, и увижу кровь и пройду мимо вас, и не будет между вами язвы губительной, когда буду поражать землю Египетскую.
\vs Exo 12:14 И да будет вам день сей памятен, и празднуйте в оный праздник Господу во [все] роды ваши; \bibemph{как} установление вечное празднуйте его.
\vs Exo 12:15 Семь дней ешьте пресный хлеб; с самого первого дня уничтожьте квасное в домах ваших, ибо кто будет есть квасное с первого дня до седьмого дня, душа та истреблена будет из среды Израиля.
\vs Exo 12:16 И в первый день да будет у вас священное собрание, и в седьмой день священное собрание: никакой работы не должно делать в них; только чт\acc{о} есть каждому, одно т\acc{о} можно делать вам.
\vs Exo 12:17 Наблюдайте опресноки, ибо в сей самый день Я вывел ополчения ваши из земли Египетской, и наблюдайте день сей в роды ваши, как установление вечное.
\vs Exo 12:18 С четырнадцатого дня первого месяца, с вечера ешьте пресный хлеб до вечера двадцать первого дня того же месяца;
\vs Exo 12:19 семь дней не должно быть закваски в домах ваших, ибо кто будет есть квасное, душа та истреблена будет из общества [сынов] Израилевых, пришлец ли то, или природный житель земли той.
\vs Exo 12:20 Ничего квасного не ешьте; во всяком местопребывании вашем ешьте пресный хлеб.
\rsbpar\vs Exo 12:21 И созвал Моисей всех старейшин [сынов] Израилевых и сказал им: выберите и возьмите себе агнцев по семействам вашим и заколите пасху;
\vs Exo 12:22 и возьмите пучок иссопа, и обмочите в кровь, которая в сосуде, и помажьте перекладину и оба косяка дверей кровью, которая в сосуде; а вы никто не выходите за двери дома своего до утра.
\vs Exo 12:23 И пойдет Господь поражать Египет, и увидит кровь на перекладине и на обоих косяках, и пройдет Господь мимо дверей, и не попустит губителю войти в домы ваши для поражения.
\vs Exo 12:24 Храните сие, как закон для себя и для сынов своих на веки.
\vs Exo 12:25 Когда войдете в землю, которую Господь даст вам, как Он говорил, соблюдайте сие служение.
\vs Exo 12:26 И когда скажут вам дети ваши: что это за служение?
\vs Exo 12:27 скажите [им]: это пасхальная жертва Господу, Который прошел мимо домов сынов Израилевых в Египте, когда поражал Египтян, и домы наши избавил. И преклонился народ и поклонился.
\vs Exo 12:28 И пошли сыны Израилевы и сделали: как повелел Господь Моисею и Аарону, так и сделали.
\rsbpar\vs Exo 12:29 В полночь Господь поразил всех первенцев в земле Египетской, от первенца фараона, сидевшего на престоле своем, до первенца узника, находившегося в темнице, и все первородное из скота.
\vs Exo 12:30 И встал фараон ночью сам и все рабы его и весь Египет; и сделался великий вопль [во всей земле] Египетской, ибо не было дома, где не было бы мертвеца.
\vs Exo 12:31 И призвал [фараон] Моисея и Аарона ночью и сказал [им]: встаньте, выйдите из среды народа моего, как вы, так и сыны Израилевы, и пойдите, совершите служение Господу [Богу вашему], как говорили вы;
\vs Exo 12:32 и мелкий и крупный скот ваш возьмите, как вы говорили; и пойдите и благословите меня.
\vs Exo 12:33 И понуждали Египтяне народ, чтобы скорее выслать его из земли той; ибо говорили они: мы все помрем.
\rsbpar\vs Exo 12:34 И понес народ тесто свое, прежде нежели оно вскисло; квашни их, завязанные в одеждах их, были на плечах их.
\vs Exo 12:35 И сделали сыны Израилевы по слову Моисея и просили у Египтян вещей серебряных и вещей золотых и одежд.
\vs Exo 12:36 Господь же дал милость народу [Своему] в глазах Египтян: и они давали ему, и обобрал он Египтян.
\vs Exo 12:37 И отправились сыны Израилевы из Раамсеса в Сокхоф до шестисот тысяч пеших мужчин, кроме детей;
\vs Exo 12:38 и множество разноплеменных людей вышли с ними, и мелкий и крупный скот, стадо весьма большое.
\vs Exo 12:39 И испекли они из теста, которое вынесли из Египта, пресные лепешки, ибо оно еще не вскисло, потому что они выгнаны были из Египта и не могли медлить, и даже пищи не приготовили себе на дорогу.
\rsbpar\vs Exo 12:40 Времени же, в которое сыны Израилевы [и отцы их] обитали в Египте [и в земле Ханаанской], было четыреста тридцать лет.
\vs Exo 12:41 По прошествии четырехсот тридцати лет, в этот самый день вышло все ополчение Господне из земли Египетской ночью.
\vs Exo 12:42 Это~--- ночь бдения Господу за изведение их из земли Египетской; эта самая ночь~--- бдение Господу у всех сынов Израилевых в роды их.
\rsbpar\vs Exo 12:43 И сказал Господь Моисею и Аарону: вот устав Пасхи: никакой иноплеменник не должен есть ее;
\vs Exo 12:44 а всякий раб, купленный за серебро, когда обрежешь его, может есть ее;
\vs Exo 12:45 поселенец и наемник не должен есть ее.
\vs Exo 12:46 В одном доме должно есть ее, [не оставляйте от нее до утра,] не выносите мяса вон из дома и костей ее не сокрушайте.
\vs Exo 12:47 Все общество [сынов] Израиля должно совершать ее.
\vs Exo 12:48 Если же поселится у тебя пришлец и захочет совершить Пасху Господу, то обрежь у него всех мужеского пола, и тогда пусть он приступит к совершению ее и будет как природный житель земли; а никакой необрезанный не должен есть ее;
\vs Exo 12:49 один закон да будет и для природного жителя и для пришельца, поселившегося между вами.
\rsbpar\vs Exo 12:50 И сделали все сыны Израилевы: как повелел Господь Моисею и Аарону, так и сделали.
\vs Exo 12:51 В этот самый день Господь вывел сынов Израилевых из земли Египетской по ополчениям их.
\vs Exo 13:1 И сказал Господь Моисею, говоря:
\vs Exo 13:2 освяти Мне каждого первенца, разверзающего всякие ложесна между сынами Израилевыми, от человека до скота, [потому что] Мои они.
\vs Exo 13:3 И сказал Моисей народу: помните сей день, в который вышли вы из Египта, из дома рабства, ибо рукою крепкою вывел вас Господь оттоле, и не ешьте квасного:
\vs Exo 13:4 сегодня вых\acc{о}дите вы, в месяце Авиве\fns{В месяце колосьев.}.
\vs Exo 13:5 И когда введет тебя Господь [Бог твой] в землю Хананеев и Хеттеев, и Аморреев, и Евеев, и Иевусеев, [Гергесеев, и Ферезеев,] о которой клялся Он отцам твоим, что даст тебе землю, где течет молоко и мед, то совершай сие служение в сем месяце;
\vs Exo 13:6 семь дней ешь пресный хлеб, и в седьмой день~--- праздник Господу;
\vs Exo 13:7 пресный хлеб д\acc{о}лжно есть семь дней, и не должн\acc{о} находиться у тебя квасного хлеба, и не должн\acc{о} находиться у тебя квасного во всех пределах твоих.
\vs Exo 13:8 И объяви в день тот сыну твоему, говоря: это ради того, что Господь [Бог] сделал со мною, когда я вышел из Египта.
\vs Exo 13:9 И да будет тебе это знаком на руке твоей и памятником пред глазами твоими, дабы закон Господень был в устах твоих, ибо рукою крепкою вывел тебя Господь [Бог] из Египта.
\vs Exo 13:10 Исполняй же устав сей в назначенное время, из года в год.
\vs Exo 13:11 И когда введет тебя Господь [Бог твой] в землю Ханаанскую, как Он клялся тебе и отцам твоим, и даст ее тебе,~---
\vs Exo 13:12 отделяй Господу все [мужеского пола] разверзающее ложесна; и все первородное из скота, какой у тебя будет, мужеского пола, [посвящай] Господу,
\vs Exo 13:13 а всякого из ослов, разверзающего [утробу], заменяй агнцем; а если не заменишь, выкупи его; и каждого первенца человеческого из сынов твоих выкуп\acc{а}й.
\vs Exo 13:14 И когда после спросит тебя сын твой, говоря: что это? то скажи ему: рукою крепкою вывел нас Господь из Египта, из дома рабства;
\vs Exo 13:15 ибо когда фараон упорствовал отпустить нас, Господь умертвил всех первенцев в земле Египетской, от первенца человеческого до первенца из скота,~--- посему я приношу в жертву Господу всё, разверзающее ложесна, мужеского пола, а всякого первенца \bibemph{из} сынов моих выкуп\acc{а}ю;
\vs Exo 13:16 и да будет это знаком на руке твоей и вместо повязки над глазами твоими, ибо рукою крепкою Господь вывел нас из Египта.
\rsbpar\vs Exo 13:17 Когда же фараон отпустил народ, Бог не повел \bibemph{его} по дороге земли Филистимской, потому что она близка; ибо сказал Бог: чтобы не раскаялся народ, увидев войну, и не возвратился в Египет.
\vs Exo 13:18 И обвел Бог народ дорогою пустынною к Чермному морю. И вышли сыны Израилевы вооруженные из земли Египетской.
\vs Exo 13:19 И взял Моисей с собою кости Иосифа, ибо [Иосиф] клятвою заклял сынов Израилевых, сказав: посетит вас Бог, и вы с собою вынесите кости мои отсюда.
\vs Exo 13:20 И двинулись [сыны Израилевы] из Сокхофа и расположились станом в Ефаме, в конце пустыни.
\vs Exo 13:21 Господь же шел пред ними днем в столпе облачном, показывая им путь, а ночью в столпе огненном, светя им, дабы идти им и днем и ночью.
\vs Exo 13:22 Не отлучался столп облачный днем и столп огненный ночью от лица [всего] народа.
\vs Exo 14:1 И сказал Господь Моисею, говоря:
\vs Exo 14:2 скажи сынам Израилевым, чтобы они обратились и расположились станом пред Пи-Гахирофом, между Мигдолом и между морем, пред Ваал-Цефоном; напротив его поставьте стан у моря.
\vs Exo 14:3 И скажет фараон [народу своему] о сынах Израилевых: они заблудились в земле сей, заперла их пустыня.
\vs Exo 14:4 А Я ожесточу сердце фараона, и он погонится за ними, и покажу славу Мою на фараоне и на всем войске его; и познают [все] Египтяне, что Я Господь. И сделали так.
\rsbpar\vs Exo 14:5 И возвещено было царю Египетскому, что народ бежал; и обратилось сердце фараона и рабов его против народа сего, и они сказали: что это мы сделали? зачем отпустили Израильтян, чтобы они не работали нам?
\vs Exo 14:6 [Фараон] запряг колесницу свою и народ свой взял с собою;
\vs Exo 14:7 и взял шестьсот колесниц отборных и все колесницы Египетские, и начальников над всеми ими.
\vs Exo 14:8 И ожесточил Господь сердце фараона, царя Египетского [и рабов его], и он погнался за сынами Израилевыми; сыны же Израилевы шли под рукою высокою.
\vs Exo 14:9 И погнались за ними Египтяне, и все кони с колесницами фараона, и всадники, и всё войско его, и настигли их расположившихся у моря, при Пи-Гахирофе пред Ваал-Цефоном.
\vs Exo 14:10 Фараон приблизился, и сыны Израилевы оглянулись, и вот, Египтяне идут за ними: и весьма устрашились и возопили сыны Израилевы к Господу,
\vs Exo 14:11 и сказали Моисею: разве нет гробов в Египте, что ты привел нас умирать в пустыне? чт\acc{о} это ты сделал с нами, выведя нас из Египта?
\vs Exo 14:12 Не это ли самое говорили мы тебе в Египте, сказав: оставь нас, пусть мы работаем Египтянам? Ибо лучше быть нам в рабстве у Египтян, нежели умереть в пустыне.
\vs Exo 14:13 Но Моисей сказал народу: не бойтесь, стойте~--- и увидите спасение Господне, которое Он соделает вам ныне, ибо Египтян, которых видите вы ныне, более не увидите во веки;
\vs Exo 14:14 Господь будет поборать за вас, а вы будьте спокойны.
\rsbpar\vs Exo 14:15 И сказал Господь Моисею: что ты вопиешь ко Мне? скажи сынам Израилевым, чтоб они шли,
\vs Exo 14:16 а ты подними жезл твой и простри руку твою на море, и раздели его, и пройдут сыны Израилевы среди моря по суше;
\vs Exo 14:17 Я же ожесточу сердце [фараона и всех] Египтян, и они пойдут вслед за ними; и покажу славу Мою на фараоне и на всем войске его, на колесницах его и на всадниках его;
\vs Exo 14:18 и узнают [все] Египтяне, что Я Господь, когда покажу славу Мою на фараоне, на колесницах его и на всадниках его.
\rsbpar\vs Exo 14:19 И двинулся Ангел Божий, шедший пред станом [сынов] Израилевых, и пошел позади их; двинулся и столп облачный от лица их и стал позади их;
\vs Exo 14:20 и вошел в средину между станом Египетским и между станом [сынов] Израилевых, и был облаком и мраком \bibemph{для одних} и освещал ночь \bibemph{для других}, и не сблизились одни с другими во всю ночь.
\vs Exo 14:21 И простер Моисей руку свою на море, и гнал Господь море сильным восточным ветром всю ночь и сделал море сушею, и расступились в\acc{о}ды.
\vs Exo 14:22 И пошли сыны Израилевы среди моря по суше: в\acc{о}ды же были им стеною по правую и по левую сторону.
\vs Exo 14:23 Погнались Египтяне, и вошли за ними в средину моря все кони фараона, колесницы его и всадники его.
\rsbpar\vs Exo 14:24 И в утреннюю стражу воззрел Господь на стан Египтян из столпа огненного и облачного и привел в замешательство стан Египтян;
\vs Exo 14:25 и отнял колеса у колесниц их, так что они влекли их с трудом. И сказали Египтяне: побежим от Израильтян, потому что Господь поборает за них против Египтян.
\vs Exo 14:26 И сказал Господь Моисею: простри руку твою на море, и да обратятся воды на Египтян, на колесницы их и на всадников их.
\vs Exo 14:27 И простер Моисей руку свою на море, и к утру вода возвратилась в свое место; а Египтяне бежали навстречу [воде]. Так потопил Господь Египтян среди моря.
\vs Exo 14:28 И вода возвратилась и покрыла колесницы и всадников всего войска фараонова, вошедших за ними в море; не осталось ни одного из них.
\vs Exo 14:29 А сыны Израилевы прошли по суше среди моря: воды [были] им стеною по правую и [стеною] по левую сторону.
\vs Exo 14:30 И избавил Господь в день тот Израильтян из рук Египтян, и увидели [сыны] Израилевы Египтян мертвыми на берегу моря.
\vs Exo 14:31 И увидели Израильтяне руку великую, которую явил Господь над Египтянами, и убоялся народ Господа и поверил Господу и Моисею, рабу Его. Тогда Моисей и сыны Израилевы воспели Господу песнь сию и говорили:
\vs Exo 15:1 Пою Господу, ибо Он высоко превознесся; коня и всадника его ввергнул в море.
\vs Exo 15:2 Господь крепость моя и слава моя, Он был мне спасением. Он Бог мой, и прославлю Его; Бог отца моего, и превознесу Его.
\vs Exo 15:3 Господь муж брани, Иегова имя Ему.
\vs Exo 15:4 Колесницы фараона и войско его ввергнул Он в море, и избранные военачальники его потонули в Чермном море.
\vs Exo 15:5 Пучины покрыли их: они пошли в глубину, как камень.
\vs Exo 15:6 Десница Твоя, Господи, прославилась силою; десница Твоя, Господи, сразила врага.
\vs Exo 15:7 Величием славы Твоей Ты низложил восставших против Тебя. Ты послал гнев Твой, и он попалил их, как солому.
\vs Exo 15:8 От дуновения Твоего расступились воды, влага стала, как стена, огустели пучины в сердце моря.
\vs Exo 15:9 Враг сказал: погонюсь, настигну, разделю добычу; насытится ими душа моя, обнажу меч мой, истребит их рука моя.
\vs Exo 15:10 Ты дунул духом Твоим, и покрыло их море: они погрузились, как свинец, в великих водах.
\vs Exo 15:11 Кто, как Ты, Господи, между богами? Кто, как Ты, величествен святостью, досточтим хвалами, Творец чудес?
\vs Exo 15:12 Ты простер десницу Твою: поглотила их земля.
\vs Exo 15:13 Ты ведешь милостью Твоею народ сей, который Ты избавил,~--- сопровождаешь силою Твоею в жилище святыни Твоей.
\vs Exo 15:14 Услышали народы и трепещут: ужас объял жителей Филистимских;
\vs Exo 15:15 тогда смутились князья Едомовы, трепет объял вождей Моавитских, уныли все жители Ханаана.
\vs Exo 15:16 Да нападет на них страх и ужас; от величия мышцы Твоей да онемеют они, как камень, доколе проходит народ Твой, Господи, доколе проходит сей народ, который Ты приобрел.
\vs Exo 15:17 Введи его и насади его на горе достояния Твоего, на месте, которое Ты соделал жилищем Себе, Господи, во святилище, \bibemph{которое} создали руки Твои, Владыка!
\vs Exo 15:18 Господь будет царствовать во веки и в вечность.
\vs Exo 15:19 Когда вошли кони фараона с колесницами его и с всадниками его в море, то Господь обратил на них в\acc{о}ды морские, а сыны Израилевы прошли по суше среди моря.
\rsbpar\vs Exo 15:20 И взяла Мариам пророчица, сестра Ааронова, в руку свою тимпан, и вышли за нею все женщины с тимпанами и ликованием.
\vs Exo 15:21 И воспела Мариам пред ними: пойте Господу, ибо высоко превознесся Он, коня и всадника его ввергнул в море.
\rsbpar\vs Exo 15:22 И повел Моисей Израильтян от Чермного моря, и они вступили в пустыню Сур; и шли они три дня по пустыне и не находили воды.
\vs Exo 15:23 Пришли в Мерру~--- и не могли пить воды в Мерре, ибо она была горька, почему и наречено тому [месту] имя: Мерра\fns{Горечь.}.
\vs Exo 15:24 И возроптал народ на Моисея, говоря: что нам пить?
\vs Exo 15:25 [Моисей] возопил к Господу, и Господь показал ему дерево, и он бросил его в воду, и вода сделалась сладкою. Там \bibemph{Бог} дал \bibemph{народу} устав и закон и там испытывал его.
\vs Exo 15:26 И сказал: если ты будешь слушаться гласа Господа, Бога твоего, и делать угодное пред очами Его, и внимать заповедям Его, и соблюдать все уставы Его, то не наведу на тебя ни одной из болезней, которые навел Я на Египет, ибо Я Господь [Бог твой], целитель твой.
\rsbpar\vs Exo 15:27 И пришли в Елим; там \bibemph{было} двенадцать источников воды и семьдесят финиковых дерев, и расположились там станом при водах.
\vs Exo 16:1 И двинулись из Елима, и пришло всё общество сынов Израилевых в пустыню Син, что между Елимом и между Синаем, в пятнадцатый день второго месяца по выходе их из земли Египетской.
\vs Exo 16:2 И возроптало все общество сынов Израилевых на Моисея и Аарона в пустыне,
\vs Exo 16:3 и сказали им сыны Израилевы: о, если бы мы умерли от руки Господней в земле Египетской, когда мы сидели у котлов с мясом, когда мы ели хлеб досыта! ибо вывели вы нас в эту пустыню, чтобы всё собрание это уморить голодом.
\vs Exo 16:4 И сказал Господь Моисею: вот, Я одождю вам хлеб с неба, и пусть народ выходит и собирает ежедневно, сколько нужно на день, чтобы Мне испытать его, будет ли он поступать по закону Моему, или нет;
\vs Exo 16:5 а в шестой день пусть заготовляют, что принесут, и будет вдвое против того, по скольку собирают в прочие дни.
\vs Exo 16:6 И сказали Моисей и Аарон всему [обществу] сынов Израилевых: вечером узн\acc{а}ете вы, что Господь вывел вас из земли Египетской,
\vs Exo 16:7 и утром увидите славу Господню, ибо услышал Он ропот ваш на Господа: а мы чт\acc{о} такое, что ропщете на нас?
\vs Exo 16:8 И сказал Моисей: \bibemph{узнаете}, когда Господь вечером даст вам мяса в пищу, а утром хлеба досыта, ибо Господь услышал ропот ваш, который вы подняли против Него: а мы что? не на нас ропот ваш, но на Господа.
\vs Exo 16:9 И сказал Моисей Аарону: скажи всему обществу сынов Израилевых: предстаньте пред лице Господа, ибо Он услышал ропот ваш.
\vs Exo 16:10 И когда говорил Аарон ко всему обществу сынов Израилевых, то они оглянулись к пустыне, и вот, слава Господня явилась в облаке.
\rsbpar\vs Exo 16:11 И сказал Господь Моисею, говоря:
\vs Exo 16:12 Я услышал ропот сынов Израилевых; скажи им: вечером будете есть мясо, а поутру насытитесь хлебом~--- и узнаете, что Я Господь, Бог ваш.
\rsbpar\vs Exo 16:13 Вечером налетели перепелы и покрыли стан, а поутру лежала роса около стана;
\vs Exo 16:14 роса поднялась, и вот, на поверхности пустыни \bibemph{нечто} мелкое, круповидное, мелкое, как иней на земле.
\vs Exo 16:15 И увидели сыны Израилевы и говорили друг другу: что это? Ибо не знали, что это. И Моисей сказал им: это хлеб, который Господь дал вам в пищу;
\vs Exo 16:16 вот что повелел Господь: собирайте его каждый по стольку, сколько ему съесть; по гомору на человека, по числу душ, сколько у кого в шатре, собирайте.
\vs Exo 16:17 И сделали так сыны Израилевы и собрали, кто много, кто мало;
\vs Exo 16:18 и меряли гомором, и у того, кто собрал много, не было лишнего, и у того, кто мало, не было недостатка: каждый собрал, сколько ему съесть.
\vs Exo 16:19 И сказал им Моисей: никто не оставляй сего до утра.
\vs Exo 16:20 Но не послушали они Моисея, и оставили от сего некоторые до утра,~--- и завелись черви, и оно воссмердело. И разгневался на них Моисей.
\vs Exo 16:21 И собирали его рано поутру, каждый сколько ему съесть; когда же обогревало солнце, оно таяло.
\vs Exo 16:22 В шестой же день собрали хлеба вдвое, по два гомора на каждого. И пришли все начальники общества и донесли Моисею.
\vs Exo 16:23 И [Моисей] сказал им: вот что сказал Господь: завтра покой, святая суббота Господня; что надобно печь, пеките, и что надобно варить, варите \bibemph{сегодня}, а что останется, отложите и сберегите до утра.
\vs Exo 16:24 И отложили то до утра, как повелел [им] Моисей, и оно не воссмердело, и червей не было в нем.
\vs Exo 16:25 И сказал Моисей: ешьте его сегодня, ибо сегодня суббота Господня; сегодня не найдете его на поле;
\vs Exo 16:26 шесть дней собирайте его, а в седьмой день~--- суббота: не будет его в \bibemph{этот день}.
\vs Exo 16:27 \bibemph{Но некоторые} из народа вышли в седьмой день собирать~--- и не нашли.
\rsbpar\vs Exo 16:28 И сказал Господь Моисею: долго ли будете вы уклоняться от соблюдения заповедей Моих и законов Моих?
\vs Exo 16:29 смотрите, Господь дал вам субботу, посему Он и дает в шестой день хлеба на два дня: оставайтесь каждый у себя [в доме своем], никто не выходи от места своего в седьмой день.
\vs Exo 16:30 И покоился народ в седьмой день.
\vs Exo 16:31 И нарек дом Израилев \bibemph{хлебу} тому имя: манна; она была, как кориандровое семя, белая, вкусом же как лепешка с медом.
\vs Exo 16:32 И сказал Моисей: вот что повелел Господь: наполните [манною] гомор для хранения в роды ваши, дабы видели хлеб, которым Я питал вас в пустыне, когда вывел вас из земли Египетской.
\vs Exo 16:33 И сказал Моисей Аарону: возьми один сосуд [золотой], и положи в него полный гомор манны, и поставь его пред Господом, для хранения в роды ваши.
\vs Exo 16:34 И поставил его Аарон пред ковчегом свидетельства для хранения, как повелел Господь Моисею.
\rsbpar\vs Exo 16:35 Сыны Израилевы ели манну сорок лет, доколе не пришли в землю обитаемую; манну ели они, доколе не пришли к пределам земли Ханаанской.
\vs Exo 16:36 А гомор есть десятая часть ефы.
\vs Exo 17:1 И двинулось всё общество сынов Израилевых из пустыни Син в путь свой, по повелению Господню, и расположилось станом в Рефидиме, и не было воды пить народу.
\vs Exo 17:2 И укорял народ Моисея, и говорили: дайте нам воды пить. И сказал им Моисей: что вы укоряете меня? что искушаете Господа?
\vs Exo 17:3 И жаждал там народ воды, и роптал народ на Моисея, говоря: зачем ты вывел нас из Египта, уморить жаждою нас и детей наших и стада наши?
\vs Exo 17:4 Моисей возопил к Господу и сказал: что мне делать с народом сим? еще немного, и побьют меня камнями.
\vs Exo 17:5 И сказал Господь Моисею: пройди перед народом, и возьми с собою \bibemph{некоторых} из старейшин Израильских, и жезл твой, которым ты ударил по воде, возьми в руку твою, и пойди;
\vs Exo 17:6 вот, Я стану пред тобою там на скале в Хориве, и ты ударишь в скалу, и пойдет из нее вода, и будет пить народ. И сделал так Моисей в глазах старейшин Израильских.
\vs Exo 17:7 И нарек месту тому имя: Масса и Мерива\fns{Искушение и укорение.}, по причине укорения сынов Израилевых и потому, что они искушали Господа, говоря: есть ли Господь среди нас, или нет?
\rsbpar\vs Exo 17:8 И пришли Амаликитяне и воевали с Израильтянами в Рефидиме.
\vs Exo 17:9 Моисей сказал Иисусу: выбери нам мужей [сильных] и пойди, сразись с Амаликитянами; завтра я стану на вершине холма, и жезл Божий будет в руке моей.
\vs Exo 17:10 И сделал Иисус, как сказал ему Моисей, и [пошел] сразиться с Амаликитянами; а Моисей и Аарон и Ор взошли на вершину холма.
\vs Exo 17:11 И когда Моисей поднимал руки свои, одолевал Израиль, а когда опускал руки свои, одолевал Амалик;
\vs Exo 17:12 но руки Моисеевы отяжелели, и тогда взяли камень и подложили под него, и он сел на нем, Аарон же и Ор поддерживали руки его, один с одной, а другой с другой \bibemph{стороны}. И были руки его подняты до захождения солнца.
\vs Exo 17:13 И низложил Иисус Амалика и народ его острием меча.
\vs Exo 17:14 И сказал Господь Моисею: напиши сие для памяти в книгу и внуши Иисусу, что Я совершенно изглажу память Амаликитян из поднебесной.
\vs Exo 17:15 И устроил Моисей жертвенник [Господу] и нарек ему имя: Иегова Нисси\fns{Господь знамя мое.}.
\vs Exo 17:16 Ибо, сказал он, рука на престоле Господа: брань у Господа против Амалика из рода в род.
\vs Exo 18:1 И услышал Иофор, священник Мадиамский, тесть Моисеев, о всем, что сделал Бог для Моисея и для Израиля, народа Своего, когда вывел Господь Израиля из Египта,
\vs Exo 18:2 и взял Иофор, тесть Моисеев, Сепфору, жену Моисееву, пред тем возвращенную,
\vs Exo 18:3 и двух сынов ее, из которых одному имя Гирсам, потому что говорил \bibemph{Моисей}: я пришлец в земле чужой;
\vs Exo 18:4 а другому имя Елиезер, потому что [говорил он] Бог отца моего был мне помощником и избавил меня от меча фараонова.
\vs Exo 18:5 И пришел Иофор, тесть Моисея, с сыновьями его и женою его к Моисею в пустыню, где он расположился станом у горы Божией,
\vs Exo 18:6 и дал знать Моисею: я, тесть твой Иофор, иду к тебе, и жена твоя, и два сына ее с нею.
\vs Exo 18:7 Моисей вышел навстречу тестю своему, и поклонился [ему], и целовал его, и после взаимного приветствия они вошли в шатер.
\vs Exo 18:8 И рассказал Моисей тестю своему о всем, что сделал Господь с фараоном и со [всеми] Египтянами за Израиля, и о всех трудностях, какие встретили их на пути, и как избавил их Господь [из руки фараона и из руки Египтян].
\vs Exo 18:9 Иофор радовался о всех благодеяниях, которые Господь явил Израилю, когда избавил его из руки Египтян [и из руки фараона],
\vs Exo 18:10 и сказал Иофор: благословен Господь, Который избавил вас из руки Египтян и из руки фараоновой, Который избавил народ сей из-под власти Египтян;
\vs Exo 18:11 ныне узнал я, что Господь велик паче всех богов, в том самом, чем они превозносились над \bibemph{Израильтянами}.
\vs Exo 18:12 И принес Иофор, тесть Моисеев, всесожжение и жертвы Богу; и пришел Аарон и все старейшины Израилевы есть хлеба с тестем Моисеевым пред Богом.
\rsbpar\vs Exo 18:13 На другой день сел Моисей судить народ, и стоял народ пред Моисеем с утра до вечера.
\vs Exo 18:14 И видел [Иофор,] тесть Моисеев, всё, что он делает с народом, и сказал: что это такое делаешь ты с народом? для чего ты сидишь один, а весь народ стоит пред тобою с утра до вечера?
\vs Exo 18:15 И сказал Моисей тестю своему: народ приходит ко мне просить суда у Бога;
\vs Exo 18:16 когда случается у них какое дело, они приходят ко мне, и я сужу между тем и другим и объявляю [им] уставы Божии и законы Его.
\vs Exo 18:17 Но тесть Моисеев сказал ему: не хорошо это ты делаешь:
\vs Exo 18:18 ты измучишь и себя и народ сей, который с тобою, ибо слишком тяжело для тебя это дело: ты один не можешь исправлять его;
\vs Exo 18:19 итак послушай слов моих; я дам тебе совет, и будет Бог с тобою: будь ты для народа посредником пред Богом и представляй Богу дела [его];
\vs Exo 18:20 научай их уставам [Божиим] и законам [Его], указывай им путь [Его], по которому они должны идти, и дела, которые они должны делать;
\vs Exo 18:21 ты же усмотри [себе] из всего народа людей способных, боящихся Бога, людей правдивых, ненавидящих корысть, и поставь [их] над ним тысяченачальниками, стоначальниками, пятидесятиначальниками и десятиначальниками [и письмоводителями];
\vs Exo 18:22 пусть они судят народ во всякое время и о всяком важном деле доносят тебе, а все малые дела судят сами: и будет тебе легче, и они понесут с тобою \bibemph{бремя};
\vs Exo 18:23 если ты сделаешь это, и Бог повелит тебе, то ты можешь устоять, и весь народ сей будет отходить в свое место с миром.
\vs Exo 18:24 И послушал Моисей слов тестя своего и сделал все, что он говорил [ему];
\vs Exo 18:25 и выбрал Моисей из всего Израиля способных людей и поставил их начальниками народа, тысяченачальниками, стоначальниками, пятидесятиначальниками и десятиначальниками [и письмоводителями],
\vs Exo 18:26 и судили они народ во всякое время; о [всех] делах важных доносили Моисею, а все малые дела судили сами.
\vs Exo 18:27 И отпустил Моисей тестя своего, и он пошел в землю свою.
\vs Exo 19:1 В третий месяц по исходе сынов Израиля из земли Египетской, в самый день новолуния, пришли они в пустыню Синайскую.
\vs Exo 19:2 И двинулись они из Рефидима, и пришли в пустыню Синайскую, и расположились там станом в пустыне; и расположился там Израиль станом против горы.
\rsbpar\vs Exo 19:3 Моисей взошел к Богу [на гору], и воззвал к нему Господь с горы, говоря: так скажи дому Иаковлеву и возвести сынам Израилевым:
\vs Exo 19:4 вы видели, что Я сделал Египтянам, и как Я носил вас [как бы] на орлиных крыльях, и принес вас к Себе;
\vs Exo 19:5 итак, если вы будете слушаться гласа Моего и соблюдать завет Мой, то будете Моим уделом из всех народов, ибо Моя вся земля,
\vs Exo 19:6 а вы будете у Меня царством священников и народом святым; вот слова, которые ты скажешь сынам Израилевым.
\vs Exo 19:7 И пришел Моисей и созвал старейшин народа и предложил им все сии слова, которые заповедал ему Господь.
\vs Exo 19:8 И весь народ отвечал единогласно, говоря: всё, что сказал Господь, исполним [и будем послушны]. И донес Моисей слова народа Господу.
\rsbpar\vs Exo 19:9 И сказал Господь Моисею: вот, Я приду к тебе в густом облаке, дабы слышал народ, как Я буду говорить с тобою, и поверил тебе навсегда. И Моисей объявил слова народа Господу.
\vs Exo 19:10 И сказал Господь Моисею: пойди к народу, [объяви] и освяти его сегодня и завтра; пусть вымоют одежды свои,
\vs Exo 19:11 чтоб быть готовыми к третьему дню: ибо в третий день сойдет Господь пред глазами всего народа на гору Синай;
\vs Exo 19:12 и проведи для народа черту со всех сторон и скажи: берегитесь восходить на гору и прикасаться к подошве ее; всякий, кто прикоснется к горе, предан будет смерти;
\vs Exo 19:13 рука да не прикоснется к нему, а пусть побьют его камнями, или застрелят стрелою; скот ли то, или человек, да не останется в живых; во время протяжного трубного звука, [когда облако отойдет от горы,] могут они взойти на гору.
\vs Exo 19:14 И сошел Моисей с горы к народу и освятил народ, и они вымыли одежду свою.
\vs Exo 19:15 И сказал народу: будьте готовы к третьему дню; не прикасайтесь к женам.
\rsbpar\vs Exo 19:16 На третий день, при наступлении утра, были громы и молнии, и густое облако над горою [Синайскою], и трубный звук весьма сильный; и вострепетал весь народ, бывший в стане.
\vs Exo 19:17 И вывел Моисей народ из стана в сретение Богу, и стали у подошвы горы.
\vs Exo 19:18 Гора же Синай вся дымилась оттого, что Господь сошел на нее в огне; и восходил от нее дым, как дым из печи, и вся гора сильно колебалась;
\vs Exo 19:19 и звук трубный становился сильнее и сильнее. Моисей говорил, и Бог отвечал ему голосом.
\vs Exo 19:20 И сошел Господь на гору Синай, на вершину горы, и призвал Господь Моисея на вершину горы, и взошел Моисей.
\vs Exo 19:21 И сказал Господь Моисею: сойди и подтверди народу, чтобы он не порывался к Господу видеть \bibemph{Его}, и чтобы не пали многие из него;
\vs Exo 19:22 священники же, приближающиеся к Господу [Богу], должны освятить себя, чтобы не поразил их Господь.
\vs Exo 19:23 И сказал Моисей Господу: не может народ взойти на гору Синай, потому что Ты предостерег нас, сказав: проведи черту вокруг горы и освяти ее.
\vs Exo 19:24 И Господь сказал ему: пойди, сойди, потом взойди ты и с тобою Аарон; а священники и народ да не порываются восходить к Господу, чтобы [Господь] не поразил их.
\vs Exo 19:25 И сошел Моисей к народу и пересказал ему.
\vs Exo 20:1 И изрек Бог [к Моисею] все слова сии, говоря:
\rsbpar\vs Exo 20:2 Я Господь, Бог твой, Который вывел тебя из земли Египетской, из дома рабства;
\vs Exo 20:3 да не будет у тебя других богов пред лицем Моим.
\rsbpar\vs Exo 20:4 Не делай себе кумира и никакого изображения того, что на небе вверху, и что на земле внизу, и что в воде ниже земли;
\vs Exo 20:5 не поклоняйся им и не служи им, ибо Я Господь, Бог твой, Бог ревнитель, наказывающий детей за вину отцов до третьего и четвертого \bibemph{рода}, ненавидящих Меня,
\vs Exo 20:6 и творящий милость до тысячи родов любящим Меня и соблюдающим заповеди Мои.
\rsbpar\vs Exo 20:7 Не произноси имени Господа, Бога твоего, напрасно, ибо Господь не оставит без наказания того, кто произносит имя Его напрасно.
\rsbpar\vs Exo 20:8 Помни день субботний, чтобы святить его;
\vs Exo 20:9 шесть дней работай и делай [в них] всякие дела твои,
\vs Exo 20:10 а день седьмой~--- суббота Господу, Богу твоему: не делай в оный никакого дела ни ты, ни сын твой, ни дочь твоя, ни раб твой, ни рабыня твоя, ни [вол твой, ни осел твой, ни всякий] скот твой, ни пришлец, который в жилищах твоих;
\vs Exo 20:11 ибо в шесть дней создал Господь небо и землю, море и все, что в них, а в день седьмой почил; посему благословил Господь день субботний и освятил его.
\rsbpar\vs Exo 20:12 Почитай отца твоего и мать твою, [чтобы тебе было хорошо и] чтобы продлились дни твои на земле, которую Господь, Бог твой, дает тебе.
\rsbpar\vs Exo 20:13 Не убивай.
\rsbpar\vs Exo 20:14 Не прелюбодействуй.
\rsbpar\vs Exo 20:15 Не кради.
\rsbpar\vs Exo 20:16 Не произноси ложного свидетельства на ближнего твоего.
\rsbpar\vs Exo 20:17 Не желай д\acc{о}ма ближнего твоего; не желай жены ближнего твоего, [ни поля его,] ни раба его, ни рабыни его, ни вола его, ни осла его, [ни всякого скота его,] ничего, что у ближнего твоего.
\rsbpar\vs Exo 20:18 Весь народ видел громы и пламя, и звук трубный, и гору дымящуюся; и увидев \bibemph{то}, [весь] народ отступил и стал вдали.
\vs Exo 20:19 И сказали Моисею: говори ты с нами, и мы будем слушать, но чтобы не говорил с нами Бог, дабы нам не умереть.
\vs Exo 20:20 И сказал Моисей народу: не бойтесь; Бог [к вам] пришел, чтобы испытать вас и чтобы страх Его был пред лицем вашим, дабы вы не грешили.
\vs Exo 20:21 И стоял [весь] народ вдали, а Моисей вступил во мрак, где Бог.
\rsbpar\vs Exo 20:22 И сказал Господь Моисею: так скажи [дому Иаковлеву и возвести] сынам Израилевым: вы видели, как Я с неба говорил вам;
\vs Exo 20:23 не делайте предо Мною богов серебряных, или богов золотых, не делайте себе:
\vs Exo 20:24 сделай Мне жертвенник из земли и приноси на нем всесожжения твои и мирные жертвы твои, овец твоих и волов твоих; на всяком месте, где Я положу память имени Моего, Я приду к тебе и благословлю тебя;
\vs Exo 20:25 если же будешь делать Мне жертвенник из камней, то не сооружай его из тесаных, ибо, как скоро наложишь на них тесло твое, то осквернишь их;
\vs Exo 20:26 и не всходи по ступеням к жертвеннику Моему, дабы не открылась при нем нагота твоя.
\vs Exo 21:1 И вот законы, которые ты объявишь им:
\vs Exo 21:2 если купишь раба Еврея, пусть он работает [тебе] шесть лет, а в седьмой [год] пусть выйдет на волю даром;
\vs Exo 21:3 если он пришел один, пусть один и выйдет; а если он женатый, пусть выйдет с ним и жена его;
\vs Exo 21:4 если же господин его дал ему жену и она родила ему сынов, или дочерей, то жена и дети ее пусть останутся у господина ее, а он выйдет один;
\vs Exo 21:5 но если раб скажет: люблю господина моего, жену мою и детей моих, не пойду на волю,~---
\vs Exo 21:6 то пусть господин его приведет его пред богов\fns{Т. е. пред судей. Пс 81:1, 2, 6.} и поставит его к двери, или к косяку, и проколет ему господин его ухо шилом, и он останется рабом его вечно.
\rsbpar\vs Exo 21:7 Если кто продаст дочь свою в рабыни, то она не может выйти, как выходят рабы;
\vs Exo 21:8 если она не угодна господину своему и он не обручит ее, пусть позволит выкупить ее; а чужому народу продать ее [господин] не властен, когда сам пренебрег ее;
\vs Exo 21:9 если он обручит ее сыну своему, пусть поступит с нею по праву дочерей;
\vs Exo 21:10 если же другую возьмет за него, то она не должна лишаться пищи, одежды и супружеского сожития;
\vs Exo 21:11 а если он сих трех \bibemph{вещей} не сделает для нее, пусть она отойдет даром, без выкупа.
\rsbpar\vs Exo 21:12 Кто ударит человека так, что он умрет, да будет предан смерти;
\vs Exo 21:13 но если кто не злоумышлял, а Бог попустил ему попасть под руки его, то Я назначу у тебя место, куда убежать [убийце];
\vs Exo 21:14 а если кто с намерением умертвит ближнего коварно [и прибежит к жертвеннику], то \bibemph{и} от жертвенника Моего бери его на смерть.
\rsbpar\vs Exo 21:15 Кто ударит отца своего, или свою мать, того должно предать смерти.
\rsbpar\vs Exo 21:16 Кто украдет человека [из сынов Израилевых] и [поработив его] продаст его, или найдется он в руках у него, то должно предать его смерти.
\rsbpar\vs Exo 21:17 Кто злословит отца своего, или свою мать, того должно предать смерти.
\rsbpar\vs Exo 21:18 Когда ссорятся [двое], и один человек ударит другого камнем, или кулаком, и тот не умрет, но сляжет в постель,
\vs Exo 21:19 то, если он встанет и будет выходить из дома с помощью палки, ударивший [его] не будет повинен \bibemph{смерти}; только пусть заплатит за остановку в его работе и даст на лечение его.
\vs Exo 21:20 А если кто ударит раба своего, или служанку свою палкою, и они умрут под рукою его, то он должен быть наказан;
\vs Exo 21:21 но если они день или два дня переживут, то не должно наказывать его, ибо это его серебро.
\rsbpar\vs Exo 21:22 Когда дерутся люди, и ударят беременную женщину, и она выкинет, но не будет \bibemph{другого} вреда, то взять с \bibemph{виновного} пеню, какую наложит на него муж той женщины, и он должен заплатить оную при посредниках;
\vs Exo 21:23 а если будет вред, то отдай душу за душу,
\vs Exo 21:24 глаз за глаз, зуб за зуб, руку за руку, ногу за ногу,
\vs Exo 21:25 обожжение за обожжение, рану за рану, ушиб за ушиб.
\rsbpar\vs Exo 21:26 Если кто раба своего ударит в глаз, или служанку свою в глаз, и повредит его, пусть отпустит их на волю за глаз;
\vs Exo 21:27 и если выбьет зуб рабу своему, или рабе своей, пусть отпустит их на волю за зуб.
\rsbpar\vs Exo 21:28 Если вол забодает мужчину или женщину до смерти, то вола побить камнями и мяса его не есть; а хозяин вола не виноват;
\vs Exo 21:29 но если вол бодлив был и вчера и третьего дня, и хозяин его, быв извещен о сем, не стерег его, а он убил мужчину или женщину, то вола побить камнями, и хозяина его предать смерти;
\vs Exo 21:30 если на него наложен будет выкуп, пусть даст выкуп за душу свою, какой наложен будет на него.
\vs Exo 21:31 Сына ли забодает, дочь ли забодает,~--- по сему же закону поступать с ним.
\vs Exo 21:32 Если вол забодает раба или рабу, то господину их заплатить тридцать сиклей серебра, а вола побить камнями.
\rsbpar\vs Exo 21:33 Если кто раскроет яму, или если выкопает яму и не покроет ее, и упадет в нее вол или осел,
\vs Exo 21:34 то хозяин ямы должен заплатить, отдать серебро хозяину их, а труп будет его.
\rsbpar\vs Exo 21:35 Если чей-нибудь вол забодает до смерти вола у соседа его, пусть продадут живого вола и разделят пополам цену его; также и убитого пусть разделят пополам;
\vs Exo 21:36 а если известно было, что вол бодлив был и вчера и третьего дня, но хозяин его [быв извещен о сем] не стерег его, то должен он заплатить вола за вола, а убитый будет его.
\vs Exo 22:1 Если кто украдет вола или овцу и заколет или продаст, то пять волов заплатит за вола и четыре овцы за овцу.
\rsbpar\vs Exo 22:2 Если \bibemph{кто} застанет вора подкапывающего и ударит его, так что он умрет, то кровь не \bibemph{вменится} ему;
\vs Exo 22:3 но если взошло над ним солнце, то \bibemph{вменится} ему кровь. \bibemph{Укравший} должен заплатить; а если нечем, то пусть продадут его \bibemph{для уплаты} за украденное им;
\vs Exo 22:4 если [он пойман будет и] украденное найдется у него в руках живым, вол ли то, или осел, или овца, пусть заплатит [за них] вдвое.
\rsbpar\vs Exo 22:5 Если кто потравит поле, или виноградник, пустив скот свой травить чужое поле, [смотря по плодам его пусть заплатит со своего поля; а если потравит всё поле,] пусть вознаградит лучшим из поля своего и лучшим из виноградника своего.
\rsbpar\vs Exo 22:6 Если появится огонь и охватит терн и выжжет копны, или жатву, или поле, то должен заплатить, кто произвел сей пожар.
\rsbpar\vs Exo 22:7 Если кто отдаст ближнему на сохранение серебро или вещи, и они украдены будут из дома его, то, если найдется вор, пусть он заплатит вдвое;
\vs Exo 22:8 а если не найдется вор, пусть хозяин дома придет пред судей [и поклянется], что не простер руки своей на собственность ближнего своего.
\vs Exo 22:9 О всякой вещи спорной, о воле, об осле, об овце, об одежде, о всякой вещи потерянной, о которой кто-нибудь скажет, что она его, дело обоих должно быть доведено до судей: кого обвинят судьи, тот заплатит ближнему своему вдвое.
\rsbpar\vs Exo 22:10 Если кто отдаст ближнему своему осла, или вола, или овцу, или какой другой скот на сбережение, а он умрет, или будет поврежден, или уведен, так что никто сего не увидит,~---
\vs Exo 22:11 клятва пред Господом да будет между обоими в том, что \bibemph{взявший} не простер руки своей на собственность ближнего своего; и хозяин должен принять, а \bibemph{тот} не будет платить;
\vs Exo 22:12 а если украден будет у него, то должен заплатить хозяину его;
\vs Exo 22:13 если же будет \bibemph{зверем} растерзан, то пусть в доказательство представит растерзанное: за растерзанное он не платит.
\rsbpar\vs Exo 22:14 Если кто займет у ближнего своего скот, и он будет поврежден, или умрет, а хозяина его не было при нем, то должен заплатить;
\vs Exo 22:15 если же хозяин его был при нем, то не должен платить; если он взят был в наймы за деньги, то пусть и пойдет за ту цену.
\rsbpar\vs Exo 22:16 Если обольстит кто девицу необрученную и переспит с нею, пусть даст ей вено [и возьмет ее] себе в жену;
\vs Exo 22:17 а если отец не согласится [и не захочет] выдать ее за него, пусть заплатит [отцу] \bibemph{столько} серебра, сколько \bibemph{полагается} на вено девицам.
\rsbpar\vs Exo 22:18 Ворожеи не оставляй в живых.
\rsbpar\vs Exo 22:19 Всякий скотоложник да будет предан смерти.
\rsbpar\vs Exo 22:20 Приносящий жертву богам, кроме одного Господа, да будет истреблен.
\rsbpar\vs Exo 22:21 Пришельца не притесняй и не угнетай его, ибо вы сами были пришельцами в земле Египетской.
\rsbpar\vs Exo 22:22 Ни вдовы, ни сироты не притесняйте;
\vs Exo 22:23 если же ты притеснишь их, то, когда они возопиют ко Мне, Я услышу вопль их,
\vs Exo 22:24 и воспламенится гнев Мой, и убью вас мечом, и будут жены ваши вдовами и дети ваши сиротами.
\rsbpar\vs Exo 22:25 Если дашь деньги взаймы бедному из народа Моего, то не притесняй его и не налагай на него роста.
\vs Exo 22:26 Если возьмешь в залог одежду ближнего твоего, до захождения солнца возврати ее,
\vs Exo 22:27 ибо она есть единственный покров у него, она~--- одеяние тела его: в чем будет он спать? итак, когда он возопиет ко Мне, Я услышу, ибо Я милосерд.
\rsbpar\vs Exo 22:28 Судей не злословь и начальника в народе твоем не поноси.
\rsbpar\vs Exo 22:29 Не медли [приносить Мне] начатки от гумна твоего и от точила твоего; отдавай Мне первенца из сынов твоих;
\vs Exo 22:30 то же делай с волом твоим и с овцою твоею [и с ослом твоим]: семь дней пусть они будут при матери своей, а в восьмой день отдавай их Мне.
\vs Exo 22:31 И будете у Меня людьми святыми; и мяса, растерзанного зверем в поле, не ешьте, псам бросайте его.
\vs Exo 23:1 Не внимай пустому слуху, не давай руки твоей нечестивому, чтоб быть свидетелем неправды.
\rsbpar\vs Exo 23:2 Не следуй за большинством на зло, и не решай тяжбы, отступая по большинству от правды;
\vs Exo 23:3 и бедному не потворствуй в тяжбе его.
\rsbpar\vs Exo 23:4 Если найдешь вола врага твоего, или осла его заблудившегося, приведи его к нему;
\vs Exo 23:5 если увидишь осла врага твоего упавшим под ношею своею, то не оставляй его; развьючь вместе с ним.
\rsbpar\vs Exo 23:6 Не суди превратно тяжбы бедного твоего.
\vs Exo 23:7 Удаляйся от неправды и не умерщвляй невинного и правого, ибо Я не оправдаю беззаконника.
\vs Exo 23:8 Даров не принимай, ибо дары слепыми делают зрячих и превращают дело правых.
\rsbpar\vs Exo 23:9 Пришельца не обижай [и не притесняй его]: вы знаете душу пришельца, потому что сами были пришельцами в земле Египетской.
\rsbpar\vs Exo 23:10 Шесть лет засевай землю твою и собирай произведения ее,
\vs Exo 23:11 а в седьмой оставляй ее в покое, чтобы питались убогие из твоего народа, а остатками после них питались звери полевые; так же поступай с виноградником твоим и с маслиною твоею.
\vs Exo 23:12 Шесть дней делай дела твои, а в седьмой день покойся, чтобы отдохнул вол твой и осел твой и успокоился сын рабы твоей и пришлец.
\rsbpar\vs Exo 23:13 Соблюдайте всё, что Я сказал вам, и имени других богов не упоминайте; да не слышится оно из уст твоих.
\rsbpar\vs Exo 23:14 Три раза в году празднуй Мне:
\vs Exo 23:15 наблюдай праздник опресноков: семь дней ешь пресный хлеб, как Я повелел тебе, в назначенное время месяца Авива, ибо в оном ты вышел из Египта; и пусть не являются пред лице Мое с пустыми \bibemph{руками};
\vs Exo 23:16 \bibemph{наблюдай} и праздник жатвы первых плодов труда твоего, какие ты сеял на поле, и праздник собирания плодов в конце года, когда уберешь с поля работу твою.
\vs Exo 23:17 Три раза в году должен являться весь мужеский пол твой пред лице Владыки, Господа [твоего].
\rsbpar\vs Exo 23:18 [Когда изгоню язычников от лица твоего и распространю пределы твои], не изливай крови жертвы Моей на квасное, и тук от праздничной жертвы Моей не должен оставаться до утра.
\rsbpar\vs Exo 23:19 Начатки плодов земли твоей приноси в дом Господа, Бога твоего. Не вари козленка в молоке матери его.
\rsbpar\vs Exo 23:20 Вот, Я посылаю пред тобою Ангела [Моего] хранить тебя на пути и ввести тебя в то место, которое Я приготовил [тебе];
\vs Exo 23:21 блюди себя пред лицем Его и слушай гласа Его; не упорствуй против Него, потому что Он не простит греха вашего, ибо имя Мое в Нем.
\vs Exo 23:22 [Если будешь слушать гласа Моего, и будешь исполнять все, что скажу тебе, и сохранишь завет Мой, то вы будете у Меня народом избранным из всех племен, ибо вся земля Моя; вы будете у Меня царственным священством и народом святым. Сии слова скажи сынам Израилевым.] Если ты будешь слушать гласа Его и исполнять все, что скажу [тебе], то врагом буду врагов твоих и противником противников твоих.
\vs Exo 23:23 Когда пойдет пред тобою Ангел Мой и поведет тебя к Аморреям, Хеттеям, Ферезеям, Хананеям, [Гергесеям,] Евеям и Иевусеям, и истреблю их [от лица вашего],
\vs Exo 23:24 то не поклоняйся богам их, и не служи им, и не подражай делам их, но сокруши их и разрушь столбы их:
\rsbpar\vs Exo 23:25 служите Господу, Богу вашему, и Он благословит хлеб твой [и вино твое] и воду твою; и отвращу от вас болезни.
\vs Exo 23:26 Не будет преждевременно рождающих и бесплодных в земле твоей; число дней твоих сделаю полным.
\vs Exo 23:27 Ужас Мой пошлю пред тобою, и в смущение приведу всякий народ, к которому ты придешь, и буду обращать к тебе тыл всех врагов твоих;
\vs Exo 23:28 пошлю пред тобою шершней, и они погонят от лица твоего [Аморреев,] Евеев, [Иевусеев,] Хананеев и Хеттеев;
\vs Exo 23:29 не выгоню их от лица твоего в один год, чтобы земля не сделалась пуста и не умножились против тебя звери полевые:
\vs Exo 23:30 мало-помалу буду прогонять их от тебя, доколе ты не размножишься и не возьмешь во владение земли сей.
\vs Exo 23:31 Проведу пределы твои от моря Чермного до моря Филистимского и от пустыни до реки [великой Евфрата], ибо предам в руки ваши жителей сей земли, и прогонишь их от лица твоего;
\vs Exo 23:32 [не смешивайся и] не заключай союза ни с ними, ни с богами их;
\vs Exo 23:33 не должны они жить в земле твоей, чтобы они не ввели тебя в грех против Меня; ибо если ты будешь служить богам их, то это будет тебе сетью.
\vs Exo 24:1 И Моисею сказал Он: взойди к Господу ты и Аарон, Надав и Авиуд и семьдесят из старейшин Израилевых, и поклонитесь [Господу] издали;
\vs Exo 24:2 Моисей один пусть приблизится к Господу, а они пусть не приближаются, и народ пусть не восходит с ним.
\rsbpar\vs Exo 24:3 И пришел Моисей и пересказал народу все слова Господни и все законы. И отвечал весь народ в один голос, и сказали: все, что сказал Господь, сделаем [и будем послушны].
\rsbpar\vs Exo 24:4 И написал Моисей все слова Господни и, встав рано поутру, поставил под горою жертвенник и двенадцать камней, по \bibemph{числу} двенадцати колен Израилевых;
\vs Exo 24:5 и послал юношей из сынов Израилевых, и принесли они всесожжения, и заклали тельцов в мирную жертву Господу [Богу].
\vs Exo 24:6 Моисей, взяв половину крови, влил в чаши, а \bibemph{другою} половиною окропил жертвенник;
\vs Exo 24:7 и взял книгу завета и прочитал вслух народу, и сказали они: всё, что сказал Господь, сделаем и будем послушны.
\vs Exo 24:8 И взял Моисей крови и окропил народ, говоря: вот кровь завета, который Господь заключил с вами о всех словах сих.
\rsbpar\vs Exo 24:9 Потом взошел Моисей и Аарон, Надав и Авиуд и семьдесят из старейшин Израилевых,
\vs Exo 24:10 и видели [место стояния] Бога Израилева; и под ногами Его нечто подобное работе из чистого сапфира и, как самое небо, ясное.
\vs Exo 24:11 И Он не простер руки Своей на избранных из сынов Израилевых: они видели [место] Бога, и ели и пили.
\rsbpar\vs Exo 24:12 И сказал Господь Моисею: взойди ко Мне на гору и будь там; и дам тебе скрижали каменные, и закон и заповеди, которые Я написал для научения их.
\vs Exo 24:13 И встал Моисей с Иисусом, служителем своим, и пошел Моисей на гору Божию,
\vs Exo 24:14 а старейшинам сказал: оставайтесь здесь, доколе мы не возвратимся к вам; вот Аарон и Ор с вами; кто будет иметь дело, пусть приходит к ним.
\vs Exo 24:15 И взошел Моисей на гору, и покрыло облако гору,
\vs Exo 24:16 и слава Господня осенила гору Синай; и покрывало ее облако шесть дней, а в седьмой день [Господь] воззвал к Моисею из среды облака.
\vs Exo 24:17 Вид же славы Господней на вершине горы был пред глазами сынов Израилевых, как огонь поядающий.
\vs Exo 24:18 Моисей вступил в средину облака и взошел на гору; и был Моисей на горе сорок дней и сорок ночей.
\vs Exo 25:1 И сказал Господь Моисею, говоря:
\vs Exo 25:2 скажи сынам Израилевым, чтобы они сделали Мне приношения; от всякого человека, у которого будет усердие, принимайте приношения Мне.
\rsbpar\vs Exo 25:3 Вот приношения, которые вы должны принимать от них: золото и серебро и медь,
\vs Exo 25:4 и \bibemph{шерсть} голубую, пурпуровую и червленую, и виссон, и козью [шерсть],
\vs Exo 25:5 и кожи бараньи красные, и кожи синие, и дерев\acc{а} ситтим,
\vs Exo 25:6 елей для светильника, ароматы для елея помазания и для благовонного курения,
\vs Exo 25:7 камень оникс и камни вставные для ефода\fns{Верхняя короткая одежда.} и для наперсника.
\vs Exo 25:8 И устроят они Мне святилище, и буду обитать посреди их;
\vs Exo 25:9 всё [сделайте], как Я показываю тебе, и образец скинии и образец всех сосудов ее; так и сделайте.
\rsbpar\vs Exo 25:10 Сделайте ковчег из дерева ситтим: длина ему два локтя с половиною, и ширина ему полтора локтя, и высота ему полтора локтя;
\vs Exo 25:11 и обложи его чистым золотом, изнутри и снаружи покрой его; и сделай наверху вокруг его золотой венец [витый];
\vs Exo 25:12 и вылей для него четыре кольца золотых и утверди на четырех нижних углах его: два кольца на одной стороне его, два кольца на другой стороне его.
\vs Exo 25:13 Сделай из дерева ситтим шесты и обложи их [чистым] золотом;
\vs Exo 25:14 и вложи шесты в кольца, по сторонам ковчега, чтобы посредством их носить ковчег;
\vs Exo 25:15 в кольцах ковчега должны быть шесты и не должны отниматься от него.
\vs Exo 25:16 И положи в ковчег откровение, которое Я дам тебе.
\vs Exo 25:17 Сделай также крышку из чистого золота: длина ее два локтя с половиною, а ширина ее полтора локтя;
\vs Exo 25:18 и сделай из золота двух херувимов: чеканной работы сделай их на обоих концах крышки;
\vs Exo 25:19 сделай одного херувима с одного края, а другого херувима с другого края; \bibemph{выдавшимися} из крышки сделайте херувимов на обоих краях ее;
\vs Exo 25:20 и будут херувимы с распростертыми вверх крыльями, покрывая крыльями своими крышку, а лицами своими \bibemph{будут} друг к другу: к крышке будут лица херувимов.
\vs Exo 25:21 И положи крышку на ковчег сверху, в ковчег же положи откровение, которое Я дам тебе;
\vs Exo 25:22 там Я буду открываться тебе и говорить с тобою над крышкою, посреди двух херувимов, которые над ковчегом откровения, о всем, что ни буду заповедовать чрез тебя сынам Израилевым.
\rsbpar\vs Exo 25:23 И сделай стол из дерева ситтим, длиною в два локтя, шириною в локоть, и вышиною в полтора локтя,
\vs Exo 25:24 и обложи его золотом чистым, и сделай вокруг него золотой венец [витый];
\vs Exo 25:25 и сделай вокруг него стенки в ладонь и у стенок его сделай золотой венец вокруг;
\vs Exo 25:26 и сделай для него четыре кольца золотых и утверди кольца на четырех углах у четырех ножек его;
\vs Exo 25:27 при стенках должны быть кольца, чтобы влагать шесты, для ношения на них стола;
\vs Exo 25:28 а шесты сделай из дерева ситтим и обложи их [чистым] золотом, и будут носить на них сей стол;
\vs Exo 25:29 сделай также для него блюдо, кадильницы, чаши и кружки, чтобы возливать ими: из золота чистого сделай их;
\vs Exo 25:30 и полагай на стол хлебы предложения пред лицем Моим постоянно.
\rsbpar\vs Exo 25:31 И сделай светильник из золота чистого; чеканный должен быть сей светильник; стебель его, ветви его, чашечки его, яблоки его и цветы его должны выходить из него;
\vs Exo 25:32 шесть ветвей должны выходить из боков его: три ветви светильника из одного бока его и три ветви светильника из другого бока его;
\vs Exo 25:33 три чашечки наподобие миндального цветка, с яблоком и цветами, должны быть на одной ветви, и три чашечки наподобие миндального цветка на другой ветви, с яблоком и цветами: так на \bibemph{всех} шести ветвях, выходящих из светильника;
\vs Exo 25:34 а на \bibemph{стебле} светильника должны быть четыре чашечки наподобие миндального цветка с яблоками и цветами;
\vs Exo 25:35 у шести ветвей, выходящих из \bibemph{стебля} светильника, яблоко под двумя ветвями его, и яблоко под другими двумя ветвями, и яблоко под \bibemph{третьими} двумя ветвями его [и на светильнике четыре чашечки, наподобие миндального цветка];
\vs Exo 25:36 яблоки и ветви их из него должны выходить: он весь \bibemph{должен быть} чеканный, цельный, из чистого золота.
\vs Exo 25:37 И сделай к нему семь лампад и поставь на него лампады его, чтобы светили на переднюю сторону его;
\vs Exo 25:38 и щипцы к нему и лотки к нему [сделай] из чистого золота;
\vs Exo 25:39 из таланта золота чистого пусть сделают его со всеми сими принадлежностями.
\vs Exo 25:40 Смотри, сделай их по тому образцу, какой показан тебе на горе.
\vs Exo 26:1 Скинию же сделай из десяти покрывал крученого виссона и из голубой, пурпуровой и червленой \bibemph{шерсти}, и херувимов сделай на них искусною работою;
\vs Exo 26:2 длина каждого покрывала двадцать восемь локтей, а ширина каждого покрывала четыре локтя: мера одна всем покрывалам.
\vs Exo 26:3 Пять покрывал пусть будут соединены одно с другим, и \bibemph{другие} пять покрывал соединены одно с другим.
\vs Exo 26:4 Сделай [к ним] петли голубого \bibemph{цвета} на краю первого покрывала, в конце соединяющего обе половины; так сделай и на краю последнего покрывала, соединяющего обе половины;
\vs Exo 26:5 пятьдесят петлей сделай у одного покрывала и пятьдесят петлей сделай на краю покрывала, которое соединяется с другим; петли \bibemph{должны} соответствовать одна другой;
\vs Exo 26:6 и сделай пятьдесят крючков золотых и крючками соедини покрывала одно с другим, и будет скиния одно \bibemph{целое}.
\vs Exo 26:7 И сделай покрывала на козьей \bibemph{шерсти}, чтобы покрывать скинию; одиннадцать покрывал сделай таких;
\vs Exo 26:8 длина одного покрывала тридцать локтей, а ширина четыре локтя; \bibemph{это} одно покрывало: одиннадцати покрывалам одна мера.
\vs Exo 26:9 И соедини пять покрывал особо и шесть покрывал особо; шестое покрывало сделай двойное с передней стороны скинии.
\vs Exo 26:10 Сделай пятьдесят петлей на краю крайнего покрывала, для соединения его \bibemph{с другим}, и пятьдесят петлей [сделай] на краю другого покрывала, для соединения с ним;
\vs Exo 26:11 сделай пятьдесят крючков медных, и вложи крючки в петли, и соедини покров, чтобы он составлял одно.
\vs Exo 26:12 А излишек, остающийся от покрывал скиний,~--- половина излишнего покрывала пусть будет свешена на задней стороне скинии;
\vs Exo 26:13 а излишек от длины покрывал скинии, на локоть с одной и на локоть с другой стороны, пусть будет свешен по бокам скинии с той и с другой стороны, для покрытия ее.
\vs Exo 26:14 И сделай покрышку для покрова из кож бараньих красных и еще покров верхний из кож синих.
\rsbpar\vs Exo 26:15 И сделай брусья для скинии из дерева ситтим, чтобы они стояли:
\vs Exo 26:16 длиною в десять локтей [сделай] брус, и полтора локтя каждому брусу ширина;
\vs Exo 26:17 у каждого бруса по два шипа [на концах], один против другого: так сделай у всех брусьев скинии.
\vs Exo 26:18 Так сделай брусья для скинии: двадцать брусьев для полуденной стороны к югу,
\vs Exo 26:19 и под двадцать брусьев сделай сорок серебряных подножий: два подножия под один брус для двух шипов его, и два подножия под другой брус для двух шипов его;
\vs Exo 26:20 и двадцать брусьев для другой стороны скинии к северу,
\vs Exo 26:21 и для них сорок подножий серебряных: два подножия [для двух шипов его] под один брус, и два подножия под другой брус [для двух шипов его];
\vs Exo 26:22 для задней же стороны скинии к западу сделай шесть брусьев
\vs Exo 26:23 и два бруса сделай для углов скинии на заднюю сторону;
\vs Exo 26:24 они должны быть соединены внизу и соединены вверху к одному кольцу: так должно быть с ними обоими; для обоих углов пусть они будут;
\vs Exo 26:25 и так будет восемь брусьев, и для них серебряных подножий шестнадцать: два подножия под один брус, и два подножия под другой брус [для двух шипов его].
\vs Exo 26:26 И сделай шесты из дерева ситтим, пять [шестов] для брусьев одной стороны скинии,
\vs Exo 26:27 и пять шестов для брусьев другой стороны скинии, и пять шестов для брусьев задней стороны сзади скинии, к западу;
\vs Exo 26:28 а внутренний шест будет проходить по средине брусьев от одного конца до другого;
\vs Exo 26:29 брусья же обложи золотом, и кольца, для вкладывания шестов, сделай из золота, и шесты обложи золотом.
\vs Exo 26:30 И поставь скинию по образцу, который показан тебе на горе.
\rsbpar\vs Exo 26:31 И сделай завесу из голубой, пурпуровой и червленой шерсти и крученого виссона; искусною работою должны быть сделаны на ней херувимы;
\vs Exo 26:32 и повесь ее на четырех столбах из ситтим, обложенных золотом, с золотыми крючками, на четырех подножиях серебряных;
\vs Exo 26:33 и повесь завесу на крючках и внеси туда за завесу ковчег откровения; и будет завеса отделять вам святилище от Святаго Святых.
\vs Exo 26:34 И положи крышку на ковчег откровения во Святом Святых.
\vs Exo 26:35 И поставь стол вне завесы и светильник против стола на стороне скинии к югу; стол же поставь на северной стороне [скинии].
\vs Exo 26:36 И сделай завесу для входа в скинию из голубой и пурпуровой и червленой \bibemph{шерсти} и из крученого виссона узорчатой работы;
\vs Exo 26:37 и сделай для завесы пять столбов из ситтим и обложи их золотом; крючки к ним золотые; и вылей для них пять подножий медных.
\vs Exo 27:1 И сделай жертвенник из дерева ситтим длиною пяти локтей и шириною пяти локтей, так чтобы он был четыреугольный, и вышиною трех локтей.
\vs Exo 27:2 И сделай роги на четырех углах его, так чтобы роги выходили из него; и обложи его медью.
\vs Exo 27:3 Сделай к нему горшки для высыпания в них пепла, и лопатки, и чаши, и вилки, и \acc{у}гольницы; все принадлежности сделай из меди.
\vs Exo 27:4 Сделай к нему решетку, род сетки, из меди, и сделай на сетке, на четырех углах ее, четыре кольца медных;
\vs Exo 27:5 и положи ее по окраине жертвенника внизу, так чтобы сетка была до половины жертвенника.
\vs Exo 27:6 И сделай шесты для жертвенника, шесты из дерева ситтим, и обложи их медью;
\vs Exo 27:7 и вкладывай шесты его в кольца, так чтобы шесты были по обоим бокам жертвенника, когда нести его.
\vs Exo 27:8 Сделай его пустой внутри, дощатый: как показано тебе на горе, так пусть сделают [его].
\rsbpar\vs Exo 27:9 Сделай двор скинии: с полуденной стороны к югу завесы для двора должны быть из крученого виссона, длиною во сто локтей по одной стороне;
\vs Exo 27:10 столбов для них двадцать, и подножий для них двадцать медных; крючки у столбов и связи на них из серебра.
\vs Exo 27:11 Также и вдоль по северной стороне~--- завесы ста локтей длиною; столбов для них двадцать, и подножий для них двадцать медных; крючки у столбов и связи на них [и подножия их] из серебра.
\vs Exo 27:12 В ширину же двора с западной стороны~--- завесы пятидесяти локтей; столбов для них десять, и подножий к ним десять.
\vs Exo 27:13 И в ширину двора с передней стороны к востоку~--- [завесы] пятидесяти локтей; [столбов для них десять, и подножий для них десять].
\vs Exo 27:14 К одной стороне~--- завесы в пятнадцать локтей [вышиною], столбов для них три, и подножий для них три;
\vs Exo 27:15 и к другой стороне~--- завесы в пятнадцать [локтей вышиною], столбов для них три, и подножий для них три.
\vs Exo 27:16 А для ворот двора завеса в двадцать локтей [вышиною] из голубой и пурпуровой и червленой шерсти и из крученого виссона узорчатой работы; столбов для нее четыре, и подножий к ним четыре.
\vs Exo 27:17 Все столбы вокруг двора должны быть соединены связями из серебра; крючки у них из серебра, а подножия к ним из меди.
\vs Exo 27:18 Длина двора сто локтей, а ширина по всему протяжению пятьдесят, высота пять локтей; \bibemph{завесы} из крученого виссона, а подножия \bibemph{у столбов} из меди.
\vs Exo 27:19 Все принадлежности скинии для всякого употребления в ней, и все колья ее, и все колья двора~--- из меди.
\rsbpar\vs Exo 27:20 И вели сынам Израилевым, чтобы они приносили тебе елей чистый, выбитый из маслин, для освещения, чтобы горел светильник во всякое время;
\vs Exo 27:21 в скинии собрания вне завесы, которая пред \bibemph{ковчегом} откровения, будет зажигать его Аарон и сыновья его, от вечера до утра, пред лицем Господним. \bibemph{Это} устав вечный для поколений их от сынов Израилевых.
\vs Exo 28:1 И возьми к себе Аарона, брата твоего, и сынов его с ним, от среды сынов Израилевых, чтоб он был священником Мне, Аарона и Надава, Авиуда, Елеазара и Ифамара, сынов Аароновых.
\vs Exo 28:2 И сделай священные одежды Аарону, брату твоему, для славы и благолепия.
\vs Exo 28:3 И скажи всем мудрым сердцем, которых Я исполнил духа премудрости [и смышления], чтобы они сделали Аарону [священные] одежды для посвящения его, чтобы он был священником Мне.
\rsbpar\vs Exo 28:4 Вот одежды, которые должны они сделать: наперсник, ефод\fns{Короткая одежда.}, верхняя риза, хитон\fns{Долгая нижняя одежда.} стяжной, кидар\fns{Головное украшение.} и пояс. Пусть сделают священные одежды Аарону, брату твоему, и сынам его, чтобы он был священником Мне.
\vs Exo 28:5 Пусть они возьмут золота, голубой и пурпуровой и червленой шерсти и виссона,
\vs Exo 28:6 и сделают ефод из золота, из голубой, пурпуровой и червленой \bibemph{шерсти}, и из крученого виссона, искусною работою.
\vs Exo 28:7 У него должны быть на обоих концах его два связывающие нарамника, чтобы он был связан.
\vs Exo 28:8 И пояс ефода, который поверх его, должен быть одинаковой с ним работы, из [чистого] золота, из голубой, пурпуровой и червленой \bibemph{шерсти} и из крученого виссона.
\vs Exo 28:9 И возьми два камня оникса и вырежь на них имена сынов Израилевых:
\vs Exo 28:10 шесть имен их на одном камне и шесть имен остальных на другом камне, по \bibemph{порядку} рождения их;
\vs Exo 28:11 чрез резчика на камне, который вырезывает печати, вырежь на двух камнях имена сынов Израилевых; и вставь их в золотые гнезда
\vs Exo 28:12 и положи два камня сии на нарамники ефода: \bibemph{это} камни на память сынам Израилевым; и будет Аарон носить имена их пред Господом на обоих раменах своих для памяти.
\vs Exo 28:13 И сделай гнезда из [чистого] золота;
\vs Exo 28:14 и [сделай] две цепочки из чистого золота, витыми сделай их работою плетеною, и прикрепи витые цепочки к гнездам [на нарамниках их спереди].
\rsbpar\vs Exo 28:15 Сделай наперсник судный искусною работою; сделай его такою же работою, как ефод: из золота, из голубой, пурпуровой и червленой \bibemph{шерсти} и из крученого виссона сделай его;
\vs Exo 28:16 он должен быть четыреугольный, двойной, в пядень длиною и в пядень шириною;
\vs Exo 28:17 и вставь в него оправленные камни в четыре ряда; рядом: рубин, топаз, изумруд,~--- это один ряд;
\vs Exo 28:18 второй ряд: карбункул, сапфир и алмаз;
\vs Exo 28:19 третий ряд: яхонт, агат и аметист;
\vs Exo 28:20 четвертый ряд: хризолит, оникс и яспис; в золотых гнездах должны быть вставлены они.
\vs Exo 28:21 Сих камней должно быть двенадцать, по \bibemph{числу} [двенадцати имен] сынов Израилевых [на двух раменах его], по именам их [и по рождению их]; на каждом, как на печати, должно быть вырезано по одному имени из числа двенадцати колен.
\vs Exo 28:22 К наперснику сделай цепочки витые плетеною работою из чистого золота;
\vs Exo 28:23 и сделай к наперснику два кольца из золота и прикрепи два [золотых] кольца к двум концам наперсника;
\vs Exo 28:24 и вдень две плетеные цепочки из золота в оба кольца по [обоим] концам наперсника,
\vs Exo 28:25 а два конца двух цепочек прикрепи к двум гнездам и прикрепи к нарамникам ефода с лицевой стороны его;
\vs Exo 28:26 еще сделай два кольца золотых и прикрепи их к двум \bibemph{другим} концам наперсника, на той стороне, которая лежит к ефоду внутрь;
\vs Exo 28:27 также сделай два кольца золотых и прикрепи их к двум нарамникам ефода снизу, с лицевой стороны его, у соединения его, над поясом ефода;
\vs Exo 28:28 и прикрепят наперсник кольцами его к кольцам ефода шнуром из голубой шерсти, чтобы он был над поясом ефода, и чтоб не спадал наперсник с ефода.
\vs Exo 28:29 И будет носить Аарон имена сынов Израилевых на наперснике судном у сердца своего, когда будет входить во святилище, для постоянной памяти пред Господом. [И положи на наперсник судный витые цепочки, положи на оба конца наперсника, и положи оба гнезда на обоих плечах на нарамнике с лица.]
\vs Exo 28:30 На наперсник судный возложи урим и туммим, и они будут у сердца Ааронова, когда будет он входить [во святилище] пред лице Господне; и будет Аарон всегда носить суд сынов Израилевых у сердца своего пред лицем Господним.
\rsbpar\vs Exo 28:31 И сделай верхнюю ризу к ефоду всю голубого \bibemph{цвета};
\vs Exo 28:32 среди ее должно быть отверстие для головы; у отверстия ее вокруг должна быть обшивка тканая, подобно как у отверстия брони, чтобы не дралось;
\vs Exo 28:33 по подолу ее сделай яблоки из \bibemph{нитей} голубого, яхонтового, пурпурового и червленого \bibemph{цвета} [и из крученого виссона], вокруг по подолу ее; [такого вида яблоки и] позвонки золотые между ними кругом:
\vs Exo 28:34 золотой позвонок и яблоко, золотой позвонок и яблоко, по подолу верхней ризы кругом;
\vs Exo 28:35 она будет на Аароне в служении, дабы слышен был от него звук, когда он будет входить во святилище пред лице Господне и когда будет выходить, чтобы ему не умереть.
\rsbpar\vs Exo 28:36 И сделай полированную дощечку из чистого золота, и вырежь на ней, как вырезывают на печати: <<Святыня Господня>>,
\vs Exo 28:37 и прикрепи ее шнуром голубого цвета к кидару, так чтобы она была на передней стороне кидара;
\vs Exo 28:38 и будет она на челе Аароновом, и понесет на себе Аарон недостатки приношений, посвящаемых от сынов Израилевых, и всех даров, ими приносимых; и будет она непрестанно на челе его, для благоволения Господня к ним.
\rsbpar\vs Exo 28:39 И сделай хитон из виссона и кидар из виссона и сделай пояс узорчатой работы;
\vs Exo 28:40 сделай и сынам Аароновым хитоны, сделай им поясы, и головные повязки сделай им для славы и благолепия,
\vs Exo 28:41 и облеки в них Аарона, брата твоего, и сынов его с ним, и помажь их, и наполни руки их, и посвяти их, и они будут священниками Мне.
\vs Exo 28:42 И сделай им нижнее платье льняное, для прикрытия телесной наготы от чресл до голеней,
\vs Exo 28:43 и да будут они на Аароне и на сынах его, когда будут они входить в скинию собрания, или приступать к жертвеннику для служения во святилище, чтобы им не навести [на себя] греха и не умереть. \bibemph{Это} устав вечный, [да будет] для него и для потомков его по нем.
\vs Exo 29:1 Вот что должен ты совершить над ними, чтобы посвятить их во священники Мне: возьми одного тельца из волов, и двух овнов без порока,
\vs Exo 29:2 и хлебов пресных, и опресноков, смешанных с елеем, и лепешек пресных, помазанных елеем: из муки пшеничной сделай их,
\vs Exo 29:3 и положи их в одну корзину, и принеси их в корзине, и вместе тельца и двух овнов.
\vs Exo 29:4 Аарона же и сынов его приведи ко входу в скинию собрания и омой их водою.
\vs Exo 29:5 И возьми [священные] одежды, и облеки Аарона в хитон и в верхнюю ризу, в ефод и в наперсник, и опояшь его по ефоду;
\vs Exo 29:6 и возложи ему на голову кидар и укрепи диадиму святыни на кидаре;
\vs Exo 29:7 и возьми елей помазания, и возлей ему на голову, и помажь его.
\vs Exo 29:8 И приведи также сынов его и облеки их в хитоны;
\vs Exo 29:9 и опояшь их поясом, Аарона и сынов его, и возложи на них повязки, и будет им принадлежать священство по уставу на веки; и наполни руки Аарона и сынов его.
\vs Exo 29:10 И приведи тельца пред скинию собрания, и возложат Аарон и сыны его руки свои на голову тельца [пред Господом у дверей скинии собрания];
\vs Exo 29:11 и заколи тельца пред лицем Господним при входе в скинию собрания;
\vs Exo 29:12 возьми крови тельца и возложи перстом твоим на роги жертвенника, а всю [остальную] кровь вылей у основания жертвенника;
\vs Exo 29:13 возьми весь тук, покрывающий внутренности, и сальник с печени, и обе почки и тук, который на них, и воскури на жертвеннике;
\vs Exo 29:14 а мясо тельца и кожу его и нечистоты его сожги на огне вне стана: это~--- \bibemph{жертва} за грех.
\vs Exo 29:15 И возьми одного овна, и возложат Аарон и сыны его руки свои на голову овна;
\vs Exo 29:16 и заколи овна, и возьми крови его, и покропи на жертвенник со всех сторон;
\vs Exo 29:17 рассеки овна на части, вымой [в воде] внутренности его и голени его, и положи \bibemph{их} на рассеченные части его и на голову его;
\vs Exo 29:18 и сожги всего овна на жертвеннике. Это всесожжение Господу, благоухание приятное, жертва Господу.
\rsbpar\vs Exo 29:19 Возьми и другого овна, и возложат Аарон и сыны его руки свои на голову овна;
\vs Exo 29:20 и заколи овна, и возьми крови его, и возложи на край правого уха Ааронова и на край правого уха сынов его, и на большой палец правой руки их, и на большой палец правой ноги их; и покропи кровью на жертвенник со всех сторон;
\vs Exo 29:21 и возьми крови, которая на жертвеннике, и елея помазания, и покропи на Аарона и на одежды его, и на сынов его, и на одежды сынов его с ним,~--- и будут освящены, он и одежды его, и сыны его и одежды их с ним.
\vs Exo 29:22 И возьми от овна тук и курдюк, и тук, покрывающий внутренности, и сальник с печени, и обе почки и тук, который на них, правое плечо [потому что это овен вручения священства],
\vs Exo 29:23 и один круглый хлеб, одну лепешку на елее и один опреснок из корзины, которая пред Господом,
\vs Exo 29:24 и положи всё на руки Аарону и на руки сынам его, и принеси это, потрясая пред лицем Господним;
\vs Exo 29:25 и возьми это с рук их и сожги на жертвеннике со всесожжением, в благоухание пред Господом: это жертва Господу.
\vs Exo 29:26 И возьми грудь от овна вручения, который для Аарона, и принеси ее, потрясая пред лицем Господним,~--- и это будет твоя доля;
\vs Exo 29:27 и освяти грудь приношения, которая потрясаема была, и плечо возношения, которое было возносимо, от овна вручения, который для Аарона и для сынов его,~---
\vs Exo 29:28 и будет \bibemph{это} Аарону и сынам его в участок вечный от сынов Израилевых, ибо это~--- возношение; возношение должно быть от сынов Израилевых при мирных жертвах [сынов Израилевых], возношение их Господу.
\vs Exo 29:29 А священные одежды, которые для Аарона, перейдут после него к сынам его, чтобы в них помазывать их и вручать им \bibemph{священство};
\vs Exo 29:30 семь дней должен облачаться в них [великий] священник из сынов его, заступающий его место, который будет входить в скинию собрания для служения во святилище.
\vs Exo 29:31 Овна же вручения возьми и свари мясо его на месте святом;
\vs Exo 29:32 и пусть съедят Аарон и сыны его мясо овна сего из корзины, у дверей скинии собрания,
\vs Exo 29:33 ибо чрез это совершено очищение для вручения им священства и для посвящения их; посторонний не должен есть \bibemph{сего}, ибо это святыня;
\vs Exo 29:34 если останется от мяса вручения и от хлеба до утра, то сожги остаток на огне: не должно есть его, ибо это святыня.
\vs Exo 29:35 И поступи с Аароном и с сынами его во всем так, как Я повелел тебе; в семь дней наполняй руки их.
\vs Exo 29:36 И тельца за грех приноси каждый день для очищения, и жертву за грех совершай на жертвеннике для очищения его, и помажь его для освящения его;
\vs Exo 29:37 семь дней очищай жертвенник, и освяти его, и будет жертвенник святыня великая: все, прикасающееся к жертвеннику, освятится.
\rsbpar\vs Exo 29:38 Вот что будешь ты приносить на жертвеннике: двух агнцев однолетних [без порока] каждый день постоянно [в жертву всегдашнюю];
\vs Exo 29:39 одного агнца приноси поутру, а другого агнца приноси вечером,
\vs Exo 29:40 и десятую \bibemph{часть ефы} пшеничной муки, смешанной с четвертью гина битого елея, а для возлияния четверть гина вина, для одного агнца;
\vs Exo 29:41 другого агнца приноси вечером: с мучным даром, подобным утреннему, и с таким же возлиянием приноси его в благоухание приятное, в жертву Господу.
\vs Exo 29:42 Это~--- всесожжение постоянное в роды ваши пред дверями скинии собрания пред Господом, где буду открываться вам, чтобы говорить с тобою;
\vs Exo 29:43 там буду открываться сынам Израилевым, и освятится \bibemph{место сие} славою Моею.
\vs Exo 29:44 И освящу скинию собрания и жертвенник; и Аарона и сынов его освящу, чтобы они священнодействовали Мне;
\vs Exo 29:45 и буду обитать среди сынов Израилевых, и буду им Богом,
\vs Exo 29:46 и узнают, что Я Господь, Бог их, Который вывел их из земли Египетской, чтобы Мне обитать среди них. Я Господь, Бог их.
\vs Exo 30:1 И сделай жертвенник для приношения курений, из дерева ситтим сделай его:
\vs Exo 30:2 длина ему локоть, и ширина ему локоть; он должен быть четыреугольный; а вышина ему два локтя; из него \bibemph{должны выходить} роги его;
\vs Exo 30:3 обложи его чистым золотом, верх его и бока его кругом, и роги его; и сделай к нему золотой венец вокруг;
\vs Exo 30:4 под венцом его на двух углах его сделай два кольца из [чистого] золота; сделай их с двух сторон его; и будут они влагалищем для шестов, чтобы носить его на них;
\vs Exo 30:5 шесты сделай из дерева ситтим и обложи их золотом.
\vs Exo 30:6 И поставь его пред завесою, которая пред ковчегом откровения, против крышки, которая на \bibemph{ковчеге} откровения, где Я буду открываться тебе.
\rsbpar\vs Exo 30:7 На нем Аарон будет курить благовонным курением; каждое утро, когда он приготовляет лампады, будет курить им;
\vs Exo 30:8 и когда Аарон зажигает лампады вечером, он будет курить им: \bibemph{это}~--- всегдашнее курение пред Господом в роды ваши.
\vs Exo 30:9 Не приносите на нем никакого иного курения, ни всесожжения, ни приношения хлебного, и возлияния не возливайте на него.
\vs Exo 30:10 И будет совершать Аарон очищение над рогами его однажды в год; кровью очистительной \bibemph{жертвы} за грех он будет очищать его однажды в год в роды ваши. Это святыня великая у Господа.
\rsbpar\vs Exo 30:11 И сказал Господь Моисею, говоря:
\vs Exo 30:12 когда будешь делать исчисление сынов Израилевых при пересмотре их, то пусть каждый даст выкуп за душу свою Господу при исчислении их, и не будет между ними язвы губительной при исчислении их;
\vs Exo 30:13 всякий, поступающий в исчисление, должен давать половину сикля, сикля священного; в сикле двадцать гер: полсикля приношение Господу;
\vs Exo 30:14 всякий, поступающий в исчисление от двадцати лет и выше, должен давать приношение Господу;
\vs Exo 30:15 богатый не больше и бедный не меньше полсикля должны давать в приношение Господу, для выкупа душ ваших;
\vs Exo 30:16 и возьми серебро выкупа от сынов Израилевых и употребляй его на служение скинии собрания; и будет это для сынов Израилевых в память пред Господом, для искупления душ ваших.
\rsbpar\vs Exo 30:17 И сказал Господь Моисею, говоря:
\vs Exo 30:18 сделай умывальник медный для омовения и подножие его медное, и поставь его между скиниею собрания и между жертвенником, и налей в него воды;
\vs Exo 30:19 и пусть Аарон и сыны его омывают из него руки свои и ноги свои;
\vs Exo 30:20 когда они должны входить в скинию собрания, пусть они омываются водою, чтобы им не умереть; или когда должны приступать к жертвеннику для служения, для жертвоприношения Господу,
\vs Exo 30:21 пусть они омывают руки свои и ноги свои водою, чтобы им не умереть; и будет им это уставом вечным, ему и потомкам его в роды их.
\rsbpar\vs Exo 30:22 И сказал Господь Моисею, говоря:
\vs Exo 30:23 возьми себе самых лучших благовонных веществ: смирны самоточной пятьсот [сиклей], корицы благовонной половину против того, двести пятьдесят, тростника благовонного двести пятьдесят,
\vs Exo 30:24 касии пятьсот \bibemph{сиклей}, по сиклю священному, и масла оливкового гин;
\vs Exo 30:25 и сделай из сего миро для священного помазания, масть составную, искусством составляющего масти: это будет миро для священного помазания;
\vs Exo 30:26 и помажь им скинию собрания и ковчег [скинии] откровения,
\vs Exo 30:27 и стол и все принадлежности его, и светильник и все принадлежности его, и жертвенник курения,
\vs Exo 30:28 и жертвенник всесожжения и все принадлежности его, и умывальник и подножие его;
\vs Exo 30:29 и освяти их, и будет святыня великая: все, прикасающееся к ним, освятится;
\vs Exo 30:30 помажь и Аарона и сынов его и посвяти их, чтобы они были священниками Мне.
\vs Exo 30:31 А сынам Израилевым скажи: это будет у Меня миро священного помазания в роды ваши;
\vs Exo 30:32 тела прочих людей не должно помазывать им, и по составу его не делайте [сами себе] подобного ему; оно~--- святыня: святынею должно быть для вас;
\vs Exo 30:33 кто составит подобное ему или кто помажет им постороннего, тот истребится из народа своего.
\rsbpar\vs Exo 30:34 И сказал Господь Моисею: возьми себе благовонных веществ: стакти, ониха, халвана душистого и чистого ливана, всего половину,
\vs Exo 30:35 и сделай из них искусством составляющего масти курительный состав, стертый, чистый, святый,
\vs Exo 30:36 и истолки его мелко, и полагай его пред \bibemph{ковчегом} откровения в скинии собрания, где Я буду открываться тебе: это будет святыня великая для вас;
\vs Exo 30:37 курения, сделанного по сему составу, не делайте себе: святынею да будет оно у тебя для Господа;
\vs Exo 30:38 кто сделает подобное, чтобы курить им, [душа та] истребится из народа своего.
\vs Exo 31:1 И сказал Господь Моисею, говоря:
\vs Exo 31:2 смотри, Я назначаю именно Веселеила, сына Уриева, сына Орова, из колена Иудина;
\vs Exo 31:3 и Я исполнил его Духом Божиим, мудростью, разумением, ведением и всяким искусством,
\vs Exo 31:4 работать из золота, серебра и меди, [из голубой, пурпуровой и червленой \bibemph{шерсти} и из крученого виссона],
\vs Exo 31:5 резать камни для вставливания и резать дерево для всякого дела;
\vs Exo 31:6 и вот, Я даю ему помощником Аголиава, сына Ахисамахова, из колена Данова, и в сердце всякого мудрого вложу мудрость, дабы они сделали всё, что Я повелел тебе:
\vs Exo 31:7 скинию собрания и ковчег откровения и крышку на него, и все принадлежности скинии,
\vs Exo 31:8 и стол и [все] принадлежности его, и светильник из чистого золота и все принадлежности его, и жертвенник курения,
\vs Exo 31:9 и жертвенник всесожжения и все принадлежности его, и умывальник и подножие его,
\vs Exo 31:10 и одежды служебные и одежды священные Аарону священнику, и одежды сынам его, для священнослужения,
\vs Exo 31:11 и елей помазания и курение благовонное для святилища: всё так, как Я повелел тебе, они сделают.
\rsbpar\vs Exo 31:12 И сказал Господь Моисею, говоря:
\vs Exo 31:13 скажи сынам Израилевым так: субботы Мои соблюдайте, ибо это~--- знамение между Мною и вами в роды ваши, дабы вы знали, что Я Господь, освящающий вас;
\vs Exo 31:14 и соблюдайте субботу, ибо она свята для вас: кто осквернит ее, тот да будет предан смерти; кто станет в оную делать дело, та душа должна быть истреблена из среды народа своего;
\vs Exo 31:15 шесть дней пусть делают дела, а в седьмой~--- суббота покоя, посвященная Господу: всякий, кто делает дело в день субботний, да будет предан смерти;
\vs Exo 31:16 и пусть хранят сыны Израилевы субботу, празднуя субботу в роды свои, как завет вечный;
\vs Exo 31:17 это~--- знамение между Мною и сынами Израилевыми на веки, потому что в шесть дней сотворил Господь небо и землю, а в день седьмой почил и покоился.
\rsbpar\vs Exo 31:18 И когда [Бог] перестал говорить с Моисеем на горе Синае, дал ему две скрижали откровения, скрижали каменные, на которых написано было перстом Божиим.
\vs Exo 32:1 Когда народ увидел, что Моисей долго не сходит с горы, то собрался к Аарону и сказал ему: встань и сделай нам бога, который бы шел перед нами, ибо с этим человеком, с Моисеем, который вывел нас из земли Египетской, не знаем, что сделалось.
\vs Exo 32:2 И сказал им Аарон: выньте золотые серьги, которые в ушах ваших жен, ваших сыновей и ваших дочерей, и принесите ко мне.
\vs Exo 32:3 И весь народ вынул золотые серьги из ушей своих и принесли к Аарону.
\vs Exo 32:4 Он взял их из рук их, и сделал из них литого тельца, и обделал его резцом. И сказали они: вот бог твой, Израиль, который вывел тебя из земли Египетской!
\vs Exo 32:5 Увидев \bibemph{сие}, Аарон поставил пред ним жертвенник, и провозгласил Аарон, говоря: завтра праздник Господу.
\vs Exo 32:6 На другой день они встали рано и принесли всесожжения и привели жертвы мирные: и сел народ есть и пить, а после встал играть.
\rsbpar\vs Exo 32:7 И сказал Господь Моисею: поспеши сойти [отсюда], ибо развратился народ твой, который ты вывел из земли Египетской;
\vs Exo 32:8 скоро уклонились они от пути, который Я заповедал им: сделали себе литого тельца и поклонились ему, и принесли ему жертвы и сказали: вот бог твой, Израиль, который вывел тебя из земли Египетской!
\vs Exo 32:9 И сказал Господь Моисею: Я вижу народ сей, и вот, народ он~--- жестоковыйный;
\vs Exo 32:10 итак оставь Меня, да воспламенится гнев Мой на них, и истреблю их, и произведу многочисленный народ от тебя.
\vs Exo 32:11 Но Моисей стал умолять Господа, Бога своего, и сказал: да не воспламеняется, Господи, гнев Твой на народ Твой, который Ты вывел из земли Египетской силою великою и рукою крепкою,
\vs Exo 32:12 чтобы Египтяне не говорили: на погибель Он вывел их, чтобы убить их в горах и истребить их с лица земли; отврати пламенный гнев Твой и отмени погубление народа Твоего;
\vs Exo 32:13 вспомни Авраама, Исаака и Израиля [Иакова], рабов Твоих, которым клялся Ты Собою, говоря: умножая умножу семя ваше, как звезды небесные, и всю землю сию, о которой Я сказал, дам семени вашему, и будут владеть [ею] вечно.
\vs Exo 32:14 И отменил Господь зло, о котором сказал, что наведет его на народ Свой.
\rsbpar\vs Exo 32:15 И обратился и сошел Моисей с горы; в руке его \bibemph{были} две скрижали откровения [каменные], на которых написано было с обеих сторон: и на той и на другой стороне написано было;
\vs Exo 32:16 скрижали были дело Божие, и письмена, начертанные на скрижалях, были письмена Божии.
\vs Exo 32:17 И услышал Иисус голос народа шумящего и сказал Моисею: военный крик в стане.
\vs Exo 32:18 Но [Моисей] сказал: это не крик побеждающих и не вопль поражаемых; я слышу голос поющих.
\vs Exo 32:19 Когда же он приблизился к стану и увидел тельца и пляски, тогда он воспламенился гневом и бросил из рук своих скрижали и разбил их под горою;
\vs Exo 32:20 и взял тельца, которого они сделали, и сжег его в огне, и стер в прах, и рассыпал по воде, и дал ее пить сынам Израилевым.
\vs Exo 32:21 И сказал Моисей Аарону: что сделал тебе народ сей, что ты ввел его в грех великий?
\vs Exo 32:22 Но Аарон сказал [Моисею]: да не возгорается гнев господина моего; ты знаешь этот народ, что он буйный.
\vs Exo 32:23 Они сказали мне: сделай нам бога, который шел бы перед нами; ибо с Моисеем, с этим человеком, который вывел нас из земли Египетской, не знаем, что сделалось.
\vs Exo 32:24 И я сказал им: у кого есть золото, снимите с себя. [Они сняли] и отдали мне; я бросил его в огонь, и вышел этот телец.
\rsbpar\vs Exo 32:25 Моисей увидел, что это народ необузданный, ибо Аарон допустил его до необузданности, к посрамлению пред врагами его.
\vs Exo 32:26 И стал Моисей в воротах стана и сказал: кто Господень, [иди] ко мне! И собрались к нему все сыны Левиины.
\vs Exo 32:27 И он сказал им: так говорит Господь Бог Израилев: возложите каждый свой меч на бедро свое, пройдите по стану от ворот до ворот и обратно, и убивайте каждый брата своего, каждый друга своего, каждый ближнего своего.
\vs Exo 32:28 И сделали сыны Левиины по слову Моисея: и пало в тот день из народа около трех тысяч человек.
\vs Exo 32:29 Ибо Моисей сказал [им]: сегодня посвятите руки ваши Господу, каждый в сыне своем и брате своем, да ниспошлет Он вам сегодня благословение.
\rsbpar\vs Exo 32:30 На другой день сказал Моисей народу: вы сделали великий грех; итак я взойду к Господу, не заглажу ли греха вашего.
\vs Exo 32:31 И возвратился Моисей к Господу и сказал: о, [Господи!] народ сей сделал великий грех: сделал себе золотого бога;
\vs Exo 32:32 прости им грех их, а если нет, то изгладь и меня из книги Твоей, в которую Ты вписал.
\vs Exo 32:33 Господь сказал Моисею: того, кто согрешил предо Мною, изглажу из книги Моей;
\vs Exo 32:34 итак, иди, [сойди,] веди народ сей, куда Я сказал тебе; вот Ангел Мой пойдет пред тобою, и в день посещения Моего Я посещу их за грех их.
\vs Exo 32:35 И поразил Господь народ за сделанного тельца, которого сделал Аарон.
\vs Exo 33:1 И сказал Господь Моисею: пойди, иди отсюда ты и народ, который ты вывел из земли Египетской, в землю, о которой Я клялся Аврааму, Исааку и Иакову, говоря: потомству твоему дам ее;
\vs Exo 33:2 и пошлю пред тобою Ангела [Моего], и прогоню Хананеев, Аморреев, Хеттеев, Ферезеев, [Гергесеев,] Евеев и Иевусеев,
\vs Exo 33:3 [и введет он вас] в землю, где течет молоко и мед; ибо Сам не пойду среди вас, чтобы не погубить Мне вас на пути, потому что вы народ жестоковыйный.
\vs Exo 33:4 Народ, услышав грозное слово сие, возрыдал, и никто не возложил на себя украшений своих.
\vs Exo 33:5 Ибо Господь сказал Моисею: скажи сынам Израилевым: вы народ жестоковыйный; если Я пойду среди вас, то в одну минуту истреблю вас; итак снимите с себя украшения свои; Я посмотрю, что Мне делать с вами.
\vs Exo 33:6 Сыны Израилевы сняли с себя украшения свои у горы Хорива.
\rsbpar\vs Exo 33:7 Моисей же взял и поставил себе шатер вне стана, вдали от стана, и назвал его скиниею собрания; и каждый, ищущий Господа, приходил в скинию собрания, находившуюся вне стана.
\vs Exo 33:8 И когда Моисей выходил к скинии, весь народ вставал, и становился каждый у входа в свой шатер и смотрел вслед Моисею, доколе он не входил в скинию.
\vs Exo 33:9 Когда же Моисей входил в скинию, тогда спускался столп облачный и становился у входа в скинию, и [Господь] говорил с Моисеем.
\vs Exo 33:10 И видел весь народ столп облачный, стоявший у входа в скинию; и вставал весь народ, и поклонялся каждый у входа в шатер свой.
\vs Exo 33:11 И говорил Господь с Моисеем лицем к лицу, как бы говорил кто с другом своим; и он возвращался в стан; а служитель его Иисус, сын Навин, юноша, не отлучался от скинии.
\rsbpar\vs Exo 33:12 Моисей сказал Господу: вот, Ты говоришь мне: веди народ сей, а не открыл мне, кого пошлешь со мною, хотя Ты сказал: <<Я знаю тебя по имени, и ты приобрел благоволение в очах Моих>>;
\vs Exo 33:13 итак, если я приобрел благоволение в очах Твоих, то молю: открой мне путь Твой, дабы я познал Тебя, чтобы приобрести благоволение в очах Твоих; и помысли, что сии люди Твой народ.
\vs Exo 33:14 [Господь] сказал [ему]: Сам Я пойду [пред тобою] и введу тебя в покой.
\vs Exo 33:15 [Моисей] сказал Ему: если не пойдешь Ты Сам [с нами], то и не выводи нас отсюда,
\vs Exo 33:16 ибо по чему узнать, что я и народ Твой обрели благоволение в очах Твоих? не по тому ли, когда Ты пойдешь с нами? тогда я и народ Твой будем славнее всякого народа на земле.
\vs Exo 33:17 И сказал Господь Моисею: и то, о чем ты говорил, Я сделаю, потому что ты приобрел благоволение в очах Моих, и Я знаю тебя по имени.
\vs Exo 33:18 [Моисей] сказал: покажи мне славу Твою.
\vs Exo 33:19 И сказал [Господь Моисею]: Я проведу пред тобою всю славу Мою и провозглашу имя Иеговы пред тобою, и кого помиловать~--- помилую, кого пожалеть~--- пожалею.
\vs Exo 33:20 И потом сказал Он: лица Моего не можно тебе увидеть, потому что человек не может увидеть Меня и остаться в живых.
\vs Exo 33:21 И сказал Господь: вот место у Меня, стань на этой скале;
\vs Exo 33:22 когда же будет проходить слава Моя, Я поставлю тебя в расселине скалы и покрою тебя рукою Моею, доколе не пройду;
\vs Exo 33:23 и когда сниму руку Мою, ты увидишь Меня сзади, а лице Мое не будет видимо [тебе].
\vs Exo 34:1 И сказал Господь Моисею: вытеши себе две скрижали каменные, подобные прежним, [и взойди ко Мне на гору,] и Я напишу на сих скрижалях слова, какие были на прежних скрижалях, которые ты разбил;
\vs Exo 34:2 и будь готов к утру, и взойди утром на гору Синай, и предстань предо Мною там на вершине горы;
\vs Exo 34:3 но никто не должен восходить с тобою, и никто не должен показываться на всей горе; даже скот, мелкий и крупный, не должен пастись близ горы сей.
\rsbpar\vs Exo 34:4 И вытесал Моисей две скрижали каменные, подобные прежним, и, встав рано поутру, взошел на гору Синай, как повелел ему Господь; и взял в руки свои две скрижали каменные.
\vs Exo 34:5 И сошел Господь в облаке, и остановился там близ него, и провозгласил имя Иеговы.
\vs Exo 34:6 И прошел Господь пред лицем его и возгласил: Господь, Господь, Бог человеколюбивый и милосердый, долготерпеливый и многомилостивый и истинный,
\vs Exo 34:7 сохраняющий [правду и являющий] милость в тысячи \bibemph{родов}, прощающий вину и преступление и грех, но не оставляющий без наказания, наказывающий вину отцов в детях и в детях детей до третьего и четвертого рода.
\vs Exo 34:8 Моисей тотчас пал на землю и поклонился [Богу]
\vs Exo 34:9 и сказал: если я приобрел благоволение в очах Твоих, Владыка, то да пойдет Владыка посреди нас; ибо народ сей жестоковыен; прости беззакония наши и грехи наши и сделай нас наследием Твоим.
\vs Exo 34:10 И сказал [Господь Моисею]: вот, Я заключаю завет: пред всем народом твоим соделаю чудеса, каких не было по всей земле и ни у каких народов; и увидит весь народ, среди которого ты находишься, дело Господа; ибо страшно будет то, что Я сделаю для тебя;
\vs Exo 34:11 сохрани то, что повелеваю тебе ныне: вот, Я изгоняю от лица твоего Аморреев, Хананеев, Хеттеев, Ферезеев, Евеев, [Гергесеев] и Иевусеев;
\vs Exo 34:12 смотри, не вступай в союз с жителями той земли, в которую ты войдешь, дабы они не сделались сетью среди вас.
\vs Exo 34:13 Жертвенники их разрушьте, столбы их сокрушите, вырубите \bibemph{священные} рощи их, [и изваяния богов их сожгите огнем],
\vs Exo 34:14 ибо ты не должен поклоняться богу иному, кроме Господа [Бога], потому что имя Его~--- ревнитель; Он Бог ревнитель.
\vs Exo 34:15 Не вступай в союз с жителями той земли, чтобы, когда они будут блудодействовать вслед богов своих и приносить жертвы богам своим, не пригласили и тебя, и ты не вкусил бы жертвы их;
\vs Exo 34:16 и не бери из дочерей их жен сынам своим [и дочерей своих не давай в замужество за сыновей их], дабы дочери их, блудодействуя вслед богов своих, не ввели и сынов твоих в блужение вслед богов своих.
\vs Exo 34:17 Не делай себе богов литых.
\vs Exo 34:18 Праздник опресноков соблюдай: семь дней ешь пресный хлеб, как Я повелел тебе, в назначенное время месяца Авива, ибо в месяце Авиве вышел ты из Египта.
\vs Exo 34:19 Все, разверзающее ложесна, Мне, как и весь скот твой мужеского пола, разверзающий ложесна, из волов и овец;
\vs Exo 34:20 первородное из ослов заменяй агнцем, а если не заменишь, то выкупи его; всех первенцев из сынов твоих выкуп\acc{а}й; пусть не являются пред лице Мое с пустыми руками.
\vs Exo 34:21 Шесть дней работай, а в седьмой день покойся; покойся и во время посева и жатвы.
\vs Exo 34:22 И праздник седмиц совершай, праздник начатков жатвы пшеницы и праздник собирания \bibemph{плодов} в конце года;
\vs Exo 34:23 три раза в году должен являться весь мужеский пол твой пред лице Владыки, Господа Бога Израилева,
\vs Exo 34:24 ибо Я прогоню народы от лица твоего и распространю пределы твои, и никто не пожелает земли твоей, если ты будешь являться пред лице Господа Бога твоего три раза в году.
\vs Exo 34:25 Не изливай крови жертвы Моей на квасное, и жертва праздника Пасхи не должна переночевать до утра.
\vs Exo 34:26 Самые первые плоды земли твоей принеси в дом Господа Бога твоего. Не вари козленка в молоке матери его.
\vs Exo 34:27 И сказал Господь Моисею: напиши себе слова сии, ибо в сих словах Я заключаю завет с тобою и с Израилем.
\vs Exo 34:28 И пробыл там [Моисей] у Господа сорок дней и сорок ночей, хлеба не ел и воды не пил; и написал [Моисей] на скрижалях слова завета, десятословие.
\rsbpar\vs Exo 34:29 Когда сходил Моисей с горы Синая, и две скрижали откровения были в руке у Моисея при сошествии его с горы, то Моисей не знал, что лице его стало сиять лучами оттого, что \bibemph{Бог} говорил с ним.
\vs Exo 34:30 И увидел Моисея Аарон и все сыны Израилевы, и вот, лице его сияет, и боялись подойти к нему.
\vs Exo 34:31 И призвал их Моисей, и пришли к нему Аарон и все начальники общества, и разговаривал Моисей с ними.
\vs Exo 34:32 После сего приблизились [к нему] все сыны Израилевы, и он заповедал им все, что говорил ему Господь на горе Синае.
\vs Exo 34:33 И когда Моисей перестал разговаривать с ними, то положил на лице свое покрывало.
\vs Exo 34:34 Когда же входил Моисей пред лице Господа, чтобы говорить с Ним, тогда снимал покрывало, доколе не выходил; а выйдя пересказывал сынам Израилевым все, что заповедано было [ему от Господа].
\vs Exo 34:35 И видели сыны Израилевы, что сияет лице Моисеево, и Моисей опять полагал покрывало на лице свое, доколе не входил говорить с Ним.
\vs Exo 35:1 И собрал Моисей все общество сынов Израилевых и сказал им: вот что заповедал Господь делать:
\vs Exo 35:2 шесть дней делайте дела, а день седьмой должен быть у вас святым, суббота покоя Господу: всякий, кто будет делать в нее дело, предан будет смерти;
\vs Exo 35:3 не зажигайте огня во всех жилищах ваших в день субботы. [Я Господь.]
\vs Exo 35:4 И сказал Моисей всему обществу сынов Израилевых: вот что заповедал Господь:
\vs Exo 35:5 сделайте от себя приношения Господу: каждый по усердию пусть принесет приношение Господу, золото, серебро, медь,
\vs Exo 35:6 \bibemph{шерсть} голубого, пурпурового и червленого \bibemph{цвета}, и виссон [крученый], и козью шерсть,
\vs Exo 35:7 кожи бараньи красные, и кожи синие, и дерево ситтим,
\vs Exo 35:8 и елей для светильника, и ароматы для елея помазания и для благовонных курений,
\vs Exo 35:9 камень оникс и камни вставные для ефода и наперсника.
\vs Exo 35:10 И всякий из вас мудрый сердцем пусть придет и сделает все, что повелел Господь:
\vs Exo 35:11 скинию и покров ее и \bibemph{верхнюю} покрышку ее, крючки и брусья ее, шесты ее, столбы ее и подножия ее,
\vs Exo 35:12 ковчег и шесты его, крышку и завесу для преграды, [и завесы двора и столбы его, и камни смарагдовые и фимиам и елей помазания,]
\vs Exo 35:13 стол и шесты его и все принадлежности его, и хлебы предложения,
\vs Exo 35:14 и светильник для освещения со [всеми] принадлежностями его, и лампады его и елей для освещения,
\vs Exo 35:15 и жертвенник для курений и шесты его, и елей помазания, и благовонные курения, и завесу ко входу скинии,
\vs Exo 35:16 жертвенник всесожжения и решетку медную для него, и шесты его и все принадлежности его, умывальник и подножие его,
\vs Exo 35:17 завесы двора, столбы его и подножия их, и завесу у входа во двор,
\vs Exo 35:18 колья скинии, и колья двора и веревки их,
\vs Exo 35:19 одежды служебные для служения во святилище, и священные одежды Аарону священнику и одежды сынам его для священнодействия.
\rsbpar\vs Exo 35:20 И пошло все общество сынов Израилевых от Моисея.
\vs Exo 35:21 И приходили все, которых влекло к тому сердце, и все, которых располагал дух, и приносили приношения Господу для устроения скинии собрания и для всех потребностей ее и для [всех] священных одежд;
\vs Exo 35:22 и приходили мужья с женами, и все по расположению сердца приносили кольца, серьги, перстни и привески, всякие золотые вещи, каждый, кто только хотел приносить золото Господу;
\vs Exo 35:23 и каждый, у кого была \bibemph{шерсть} голубого, пурпурового и червленого \bibemph{цвета}, виссон и козья шерсть, кожи бараньи красные и кожи синие, приносил их;
\vs Exo 35:24 и каждый, кто жертвовал серебро или медь, приносил сие в дар Господу; и каждый, у кого было дерево ситтим, приносил сие на всякую потребность \bibemph{для скинии};
\vs Exo 35:25 и все женщины, мудрые сердцем, пряли своими руками и приносили пряжу голубого, пурпурового и червленого \bibemph{цвета} и виссон;
\vs Exo 35:26 и все женщины, которых влекло сердце, умевшие прясть, пряли козью шерсть;
\vs Exo 35:27 князья же приносили камень оникс и камни вставные для ефода и наперсника,
\vs Exo 35:28 также и благовония, и елей для светильника и для \bibemph{составления} елея помазания и для благовонных курений;
\vs Exo 35:29 и все мужья и жены из сынов Израилевых, которых влекло сердце принести на всякое дело, какое Господь чрез Моисея повелел сделать, приносили добровольный дар Господу.
\rsbpar\vs Exo 35:30 И сказал Моисей сынам Израилевым: смотрите, Господь назначил именно Веселеила, сына Урии, сына Ора, из колена Иудина,
\vs Exo 35:31 и исполнил его Духом Божиим, мудростью, разумением, в\acc{е}дением и всяким искусством,
\vs Exo 35:32 составлять искусные ткани, работать из золота, серебра и меди,
\vs Exo 35:33 и резать камни для вставливания, и резать дерево, и делать всякую художественную работу;
\vs Exo 35:34 и способность учить \bibemph{других} вложил в сердце его, его и Аголиава, сына Ахисамахова, из колена Данова;
\vs Exo 35:35 Он исполнил сердце их мудростью, чтобы делать всякую работу [для святилища] резчика и искусного ткача, и вышивателя по голубой, пурпуровой, червленой и виссонной ткани, и ткачей, делающих всякую работу и составляющих искусные ткани.
\vs Exo 36:1 И стал работать Веселеил и Аголиав и все мудрые сердцем, которым Господь дал мудрость и разумение, чтоб уметь сделать всякую работу, потребную для святилища, как повелел Господь.
\vs Exo 36:2 И призвал Моисей Веселеила и Аголиава и всех мудрых сердцем, которым Господь дал мудрость, и всех, коих влекло сердце приступить к работе и работать.
\vs Exo 36:3 И взяли они от Моисея все приношения, которые принесли сыны Израилевы, на [все] потребности святилища, чтобы работать. Между тем еще продолжали приносить к нему добровольные дары каждое утро.
\vs Exo 36:4 Тогда пришли все мудрые сердцем, производившие всякие работы святилища, каждый от своей работы, какою кто занимался,
\vs Exo 36:5 и сказали Моисею, говоря: народ много приносит, более нежели потребно для работ, какие повелел Господь сделать.
\vs Exo 36:6 И приказал Моисей, и объявлено было в стане, чтобы ни мужчина, ни женщина не делали уже ничего для приношения во святилище; и перестал народ приносить.
\vs Exo 36:7 Запаса было достаточно на всякие работы, какие надлежало делать, и даже осталось.
\rsbpar\vs Exo 36:8 И сделали все мудрые сердцем, занимавшиеся работою скинии: десять покрывал из крученого виссона и из голубой, пурпуровой и червленой \bibemph{шерсти}; и херувимов сделали на них искусною работою;
\vs Exo 36:9 длина каждого покрывала двадцать восемь локтей, и ширина каждого покрывала четыре локтя: всем покрывалам одна мера.
\vs Exo 36:10 И соединил он пять покрывал одно с другим, и \bibemph{другие} пять покрывал соединил одно с другим.
\vs Exo 36:11 И сделал петли голубого \bibemph{цвета} на краю одного покрывала, где оно соединяется с другим; так же сделал он и на краю последнего покрывала, для соединения его с другим;
\vs Exo 36:12 пятьдесят петлей сделал он у одного покрывала, и пятьдесят петлей сделал в конце покрывала, где оно соединяется с другим; петли сии соответствовали одна другой;
\vs Exo 36:13 и сделал пятьдесят крючков золотых, и крючками соединил одно покрывало с другим, и стала скиния одно \bibemph{целое}.
\rsbpar\vs Exo 36:14 Потом сделал покрывала из козьей шерсти для покрытия скинии: одиннадцать покрывал сделал таких;
\vs Exo 36:15 длиною покрывало тридцать локтей, и шириною покрывало четыре локтя: одиннадцати покрывалам мера одна.
\vs Exo 36:16 И соединил он пять покрывал особо и шесть покрывал особо.
\vs Exo 36:17 И сделал пятьдесят петлей на краю покрывала крайнего, где оно соединяется с другим, и пятьдесят петлей сделал на краю покрывала, соединяющегося с другим;
\vs Exo 36:18 и сделал пятьдесят медных крючков для соединения покрова, чтоб составилось одно \bibemph{целое}.
\vs Exo 36:19 И сделал для скинии покров из красных бараньих кож и покрышку сверху из кож синих.
\rsbpar\vs Exo 36:20 И сделал брусья для скинии из дерева ситтим прямостоящие:
\vs Exo 36:21 десять локтей длина бруса, и полтора локтя ширина каждого бруса;
\vs Exo 36:22 у каждого бруса по два шипа, один против другого: так сделал он все брусья скинии.
\vs Exo 36:23 И сделал для скинии двадцать таких брусьев для полуденной стороны,
\vs Exo 36:24 и сорок серебряных подножий сделал под двадцать брусьев: два подножия под один брус для двух шипов его, и два подножия под другой брус для двух шипов его;
\vs Exo 36:25 и для другой стороны скинии, к северу, сделал двадцать брусьев
\vs Exo 36:26 и сорок серебряных подножий: два подножия под один брус, и два подножия под другой брус;
\vs Exo 36:27 а для задней стороны скинии, к западу, сделал шесть брусьев,
\vs Exo 36:28 и два бруса сделал для угла в скинии на заднюю сторону;
\vs Exo 36:29 и были они соединены внизу и соединены вверху к одному кольцу: так сделал он с ними обоими на обоих углах;
\vs Exo 36:30 и было восемь брусьев и серебряных подножий шестнадцать, по два подножия под каждый брус.
\rsbpar\vs Exo 36:31 И сделал шесты из дерева ситтим, пять для брусьев одной стороны скинии,
\vs Exo 36:32 и пять шестов для брусьев другой стороны скинии, и пять шестов для брусьев задней стороны скинии;
\vs Exo 36:33 и сделал внутренний шест, который проходил бы по средине брусьев от одного конца до другого;
\vs Exo 36:34 брусья обложил золотом, и кольца, в которые вкладываются шесты, сделал из золота, и шесты обложил золотом.
\rsbpar\vs Exo 36:35 И сделал завесу из голубой, пурпуровой и червленой \bibemph{шерсти} и из крученого виссона, и искусною работою сделал на ней херувимов;
\vs Exo 36:36 и сделал для нее четыре столба из ситтим и обложил их золотом, с золотыми крючками, и вылил для них четыре серебряных подножия.
\vs Exo 36:37 И сделал завесу ко входу скинии из голубой, пурпуровой и червленой \bibemph{шерсти} и из крученого виссона, узорчатой работы,
\vs Exo 36:38 и пять столбов для нее с крючками; и обложил верхи их и связи их золотом, и \bibemph{вылил} пять медных подножий.
\vs Exo 37:1 И сделал Веселеил ковчег из дерева ситтим; длина его два локтя с половиною, ширина его полтора локтя и высота его полтора локтя;
\vs Exo 37:2 и обложил его чистым золотом внутри и снаружи и сделал вокруг него золотой венец;
\vs Exo 37:3 и вылил для него четыре кольца золотых, на четырех нижних углах его: два кольца на одной стороне его и два кольца на другой стороне его.
\vs Exo 37:4 И сделал шесты из дерева ситтим и обложил их золотом;
\vs Exo 37:5 и вложил шесты в кольца, по сторонам ковчега, чтобы носить ковчег.
\vs Exo 37:6 И сделал крышку из чистого золота: длина ее два локтя с половиною, а ширина полтора локтя.
\vs Exo 37:7 И сделал двух херувимов из золота: чеканной работы сделал их на обоих концах крышки,
\vs Exo 37:8 одного херувима с одного конца, а другого херувима с другого конца: выдавшимися из крышки сделал херувимов с обоих концов ее;
\vs Exo 37:9 и были херувимы с распростертыми вверх крыльями и покрывали крыльями своими крышку, а лицами своими были \bibemph{обращены} друг к другу; к крышке \bibemph{были} лица херувимов.
\rsbpar\vs Exo 37:10 И сделал стол из дерева ситтим длиною в два локтя, шириною в локоть и вышиною в полтора локтя,
\vs Exo 37:11 и обложил его золотом чистым, и сделал вокруг него золотой венец;
\vs Exo 37:12 и сделал вокруг него стенки в ладонь и сделал золотой венец у стенок его;
\vs Exo 37:13 и вылил для него четыре кольца золотых и утвердил кольца на четырех углах, у четырех ножек его;
\vs Exo 37:14 при стенках были кольца, чтобы влагать шесты для ношения стола;
\vs Exo 37:15 и сделал шесты из дерева ситтим и обложил их золотом для ношения стола.
\vs Exo 37:16 Потом сделал сосуды, принадлежащие к столу: блюда, кадильницы, кружки и чаши, чтобы возливать ими, из чистого золота.
\rsbpar\vs Exo 37:17 И сделал светильник из золота чистого, чеканный сделал светильник; стебель его, ветви его, чашечки его, яблоки его и цветы его \bibemph{выходили} из него;
\vs Exo 37:18 шесть ветвей выходило из боков его: три ветви светильника из одного бока его и три ветви светильника из другого бока его;
\vs Exo 37:19 три чашечки были наподобие миндального цветка, яблоко и цветы на одной ветви, и три чашечки наподобие миндального цветка, яблоко и цветы на другой ветви: так на \bibemph{всех} шести ветвях, выходящих из светильника;
\vs Exo 37:20 а на \bibemph{стебле} светильника было четыре чашечки наподобие миндального цветка с яблоками и цветами;
\vs Exo 37:21 у шести ветвей, выходящих из него, яблоко под первыми двумя ветвями, и яблоко под \bibemph{вторыми} двумя ветвями, и яблоко под \bibemph{третьими} двумя ветвями;
\vs Exo 37:22 яблоки и ветви их выходили из него; весь он \bibemph{был} чеканный, цельный, из чистого золота.
\vs Exo 37:23 И сделал к нему семь лампад, и щипцы к нему и лотки к нему, из чистого золота;
\vs Exo 37:24 из таланта чистого золота сделал его со всеми принадлежностями его.
\rsbpar\vs Exo 37:25 И сделал жертвенник курения из дерева ситтим: длина его локоть и ширина его локоть, четыреугольный, вышина его два локтя; из него выходили роги его;
\vs Exo 37:26 и обложил его чистым золотом, верх его и стороны его кругом, и роги его, и сделал к нему золотой венец вокруг;
\vs Exo 37:27 под венцом его на двух углах его сделал два кольца золотых; с двух сторон его сделал их, чтобы вкладывать в них шесты для ношения его;
\vs Exo 37:28 шесты сделал из дерева ситтим и обложил их золотом.
\vs Exo 37:29 И сделал миро для священного помазания и курение благовонное, чистое, искусством составляющего масти.
\vs Exo 38:1 И сделал жертвенник всесожжения из дерева ситтим длиною в пять локтей и шириною в пять локтей, четыреугольный, вышиною в три локтя;
\vs Exo 38:2 и сделал роги на четырех углах его, так что из него выходили роги, и обложил его медью.
\vs Exo 38:3 И сделал все принадлежности жертвенника: горшки, лопатки, чаши, вилки и \acc{у}гольницы; все принадлежности его сделал из меди.
\vs Exo 38:4 И сделал для жертвенника решетку, род сетки, из меди, по окраине его внизу до половины его;
\vs Exo 38:5 и сделал четыре кольца на четырех углах медной решетки для вкладывания шестов.
\vs Exo 38:6 И сделал шесты из дерева ситтим, и обложил их медью,
\vs Exo 38:7 и вложил шесты в кольца на боках жертвенника, чтобы носить его посредством их; пустой внутри из досок сделал его.
\rsbpar\vs Exo 38:8 И сделал умывальник из меди и подножие его из меди с изящными изображениями, украшающими вход скинии собрания.
\rsbpar\vs Exo 38:9 И сделал двор: с полуденной стороны, к югу, завесы из крученого виссона, длиною во сто локтей;
\vs Exo 38:10 столбов для них двадцать и подножий к ним двадцать медных; крючки у столбов и связи их из серебра.
\vs Exo 38:11 И по северной стороне~--- \bibemph{завесы} во сто локтей; столбов для них двадцать и подножий к ним двадцать медных; крючки у столбов и связи их из серебра.
\vs Exo 38:12 И с западной стороны~--- завесы в пятьдесят локтей, столбов для них десять и подножий к ним десять; крючки у столбов и связи их из серебра.
\vs Exo 38:13 И с передней стороны к востоку~--- \bibemph{завесы} в пятьдесят локтей.
\vs Exo 38:14 Для одной стороны \bibemph{ворот двора}~--- завесы в пятнадцать локтей, столбов для них три и подножий к ним три;
\vs Exo 38:15 и для другой стороны [по обеим сторонам ворот двора]~--- завесы в пятнадцать локтей, столбов для них три и подножий к ним три.
\vs Exo 38:16 Все завесы во все стороны двора из крученого виссона,
\vs Exo 38:17 а подножия у столбов из меди, крючки у столбов и связи их из серебра; верхи же у них обложены серебром, и все столбы двора соединены связями серебряными.
\vs Exo 38:18 Завеса же для ворот двора узорчатой работы из голубой, пурпуровой и червленой \bibemph{шерсти} и из крученого виссона, длиною в двадцать локтей, вышиною в пять локтей, по всему протяжению, подобно завесам двора;
\vs Exo 38:19 и столбов для нее четыре, и подножий к ним четыре медных; крючки у них серебряные, а верхи их обложены серебром, и связи их серебряные.
\vs Exo 38:20 Все колья вокруг скинии и двора медные.
\rsbpar\vs Exo 38:21 Вот исчисление того, что употреблено для скинии откровения, сделанное по повелению Моисея, посредством левитов под надзором Ифамара, сына Ааронова, священника.
\vs Exo 38:22 Делал же все, что повелел Господь Моисею, Веселеил, сын Урии, сына Ора, из колена Иудина,
\vs Exo 38:23 и с ним Аголиав, сын Ахисамахов, из колена Данова, резчик и искусный ткач и вышиватель по голубой, пурпуровой, червленой и виссонной \bibemph{ткани}.
\vs Exo 38:24 Всего золота, употребленного в дело на все принадлежности святилища, золота, принесенного в дар, было двадцать девять талантов и семьсот тридцать сиклей, сиклей священных;
\vs Exo 38:25 серебра же от исчисленных \bibemph{лиц} общества сто талантов и тысяча семьсот семьдесят пять сиклей, сиклей священных;
\vs Exo 38:26 с шестисот трех тысяч пятисот пятидесяти человек, с каждого поступившего в исчисление, от двадцати лет и выше, по полсиклю с человека, считая на сикль священный.
\vs Exo 38:27 Сто талантов серебра употреблено на вылитие подножий святилища и подножий у завесы; сто подножий из ста талантов, по таланту на подножие;
\vs Exo 38:28 а из тысячи семисот семидесяти пяти \bibemph{сиклей} сделал он крючки у столбов и покрыл верхи их и сделал связи для них.
\vs Exo 38:29 Меди же, принесенной в дар, было семьдесят талантов и две тысячи четыреста сиклей;
\vs Exo 38:30 из нее сделал он подножия \bibemph{для столбов} у входа в скинию свидетельства, и жертвенник медный, и решетку медную для него, и все сосуды жертвенника,
\vs Exo 38:31 и подножия \bibemph{для столбов} всего двора, и подножия \bibemph{для столбов} ворот двора, и все колья скинии и все колья вокруг двора.
\vs Exo 39:1 Из голубой же, пурпуровой и червленой \bibemph{шерсти} сделали они служебные одежды, для служения во святилище; также сделали священные одежды Аарону, как повелел Господь Моисею.
\vs Exo 39:2 И сделал ефод из золота, из голубой, пурпуровой и червленой \bibemph{шерсти} и из крученого виссона;
\vs Exo 39:3 и разбили они золото в листы и вытянули нити, чтобы воткать их между голубыми, пурпуровыми, червлеными и виссонными \bibemph{нитями}, искусною работою.
\vs Exo 39:4 И сделали у него нарамники связывающие; на обоих концах своих он был связан.
\vs Exo 39:5 И пояс ефода, который поверх его, одинаковой с ним работы, \bibemph{сделан был} из золота, из голубой, пурпуровой и червленой \bibemph{шерсти} и крученого виссона, как повелел Господь Моисею.
\vs Exo 39:6 И обделали камни ониксовые, вставив их в золотые гнезда и вырезав на них имена сынов Израилевых, как вырезывают на печати;
\vs Exo 39:7 и положил он их на нарамники ефода, в память сынов Израилевых, как повелел Господь Моисею.
\rsbpar\vs Exo 39:8 И сделал наперсник искусною работою, такою же работою, как ефод, из золота, из голубой, пурпуровой и червленой \bibemph{шерсти} и из крученого виссона;
\vs Exo 39:9 он был четыреугольный; двойной сделали они наперсник в пядень длиною и в пядень шириною, двойной он был;
\vs Exo 39:10 и вставили в него в четыре ряда камни. Рядом: рубин, топаз, изумруд,~--- это первый ряд;
\vs Exo 39:11 во втором ряду: карбункул, сапфир и алмаз;
\vs Exo 39:12 в третьем ряду: яхонт, агат и аметист;
\vs Exo 39:13 в четвертом ряду: хризолит, оникс и яспис; и вставлены они в золотых гнездах.
\vs Exo 39:14 Камней было по числу имен сынов Израилевых: двенадцать было их, по числу имен их, и на каждом из них вырезано было, \bibemph{как} на печати, по одному имени, для двенадцати колен.
\vs Exo 39:15 К наперснику сделали толстые цепочки витою работою из чистого золота;
\vs Exo 39:16 и сделали два золотых гнезда и два золотых кольца и прикрепили два кольца к двум концам наперсника;
\vs Exo 39:17 и вдели обе плетеные цепочки из золота в два кольца по концам наперсника,
\vs Exo 39:18 а два конца двух цепочек прикрепили к двум гнездам и прикрепили их к нарамникам ефода с лицевой стороны его;
\vs Exo 39:19 еще сделали два кольца золотых и прикрепили к двум \bibemph{другим} концам наперсника, на той стороне, которая находится к ефоду внутрь;
\vs Exo 39:20 и еще сделали два кольца золотых и прикрепили их к двум нарамникам ефода снизу, с лицевой стороны его, у соединения его над поясом ефода;
\vs Exo 39:21 и прикрепили наперсник кольцами его к кольцам ефода посредством шнура из голубой \bibemph{шерсти}, чтобы он был над поясом ефода, и чтобы не отставал наперсник от ефода, как повелел Господь Моисею.
\rsbpar\vs Exo 39:22 И сделал верхнюю ризу к ефоду, тканую, всю из голубой \bibemph{шерсти},
\vs Exo 39:23 и среди верхней ризы отверстие, как отверстие у брони, и вокруг него обшивку, чтобы не дралось;
\vs Exo 39:24 по подолу верхней ризы сделали они яблоки из голубой, пурпуровой и червленой \bibemph{шерсти};
\vs Exo 39:25 и сделали позвонки из чистого золота и повесили позвонки между яблоками по подолу верхней ризы кругом;
\vs Exo 39:26 позвонок и яблоко, позвонок и яблоко, по подолу верхней ризы кругом для служения, как повелел Господь Моисею.
\rsbpar\vs Exo 39:27 И сделали для Аарона и для сыновей его хитоны из виссона, тканые,
\vs Exo 39:28 и кидар из виссона, и головные повязки из виссона, и нижнее льняное платье из крученого виссона,
\vs Exo 39:29 и пояс из крученого виссона и из голубой, пурпуровой и червленой \bibemph{шерсти}, узорчатой работы, как повелел Господь Моисею.
\rsbpar\vs Exo 39:30 И сделали полированную дощечку, диадиму святыни, из чистого золота, и начертали на ней письмена, как вырезывают на печати: Святыня Господня;
\vs Exo 39:31 и прикрепили к ней шнур из голубой \bibemph{шерсти}, чтобы привязать ее к кидару сверху, как повелел Господь Моисею.
\rsbpar\vs Exo 39:32 Так кончена была вся работа для скинии собрания; и сделали сыны Израилевы всё: как повелел Господь Моисею, так и сделали.
\vs Exo 39:33 И принесли к Моисею скинию, покров и все принадлежности ее, крючки ее, брусья ее, шесты ее, столбы ее и подножия ее,
\vs Exo 39:34 покров из кож бараньих красных и покров из кож синих и завесу закрывающую,
\vs Exo 39:35 ковчег откровения и шесты его, и крышку,
\vs Exo 39:36 стол со всеми принадлежностями его и хлебы предложения,
\vs Exo 39:37 светильник из чистого золота, лампады его, лампады расставленные на нем и все принадлежности его, и елей для освещения,
\vs Exo 39:38 золотой жертвенник и елей помазания, и благовония для курения, и завесу ко входу в скинию,
\vs Exo 39:39 жертвенник медный и медную решетку к нему, шесты его и все принадлежности его, умывальник и подножие его,
\vs Exo 39:40 завесы двора, столбы и подножия, завесу к воротам двора, веревки и колья и все вещи, принадлежащие к служению в скинии собрания,
\vs Exo 39:41 одежды служебные для служения во святилище, священные одежды Аарону священнику и одежды сыновьям его для священнодействия.
\vs Exo 39:42 Как повелел Господь Моисею, так и сделали сыны Израилевы все сии работы.
\vs Exo 39:43 И увидел Моисей всю работу, и вот они сделали ее: как повелел Господь, так и сделали. И благословил их Моисей.
\vs Exo 40:1 И сказал Господь Моисею, говоря:
\vs Exo 40:2 в первый месяц, в первый день месяца поставь скинию собрания,
\vs Exo 40:3 и поставь в ней ковчег откровения, и закрой ковчег завесою;
\vs Exo 40:4 и внеси стол и расставь на нем все вещи его, и внеси светильник и поставь на нем лампады его;
\vs Exo 40:5 и поставь золотой жертвенник для курения пред ковчегом откровения и повесь завесу у входа в скинию [собрания];
\vs Exo 40:6 и поставь жертвенник всесожжения пред входом в скинию собрания;
\vs Exo 40:7 и поставь умывальник между скиниею собрания и между жертвенником и влей в него воды;
\vs Exo 40:8 и поставь двор кругом и повесь завесу в воротах двора.
\vs Exo 40:9 И возьми елея помазания, и помажь скинию и все, что в ней, и освяти ее и все принадлежности ее, и будет свята;
\vs Exo 40:10 помажь жертвенник всесожжения и все принадлежности его, и освяти жертвенник, и будет жертвенник святыня великая;
\vs Exo 40:11 и помажь умывальник и подножие его и освяти его.
\vs Exo 40:12 И приведи Аарона и сынов его ко входу в скинию собрания и омой их водою,
\vs Exo 40:13 и облеки Аарона в священные одежды, и помажь его, и освяти его, чтобы он был священником Мне.
\vs Exo 40:14 И сынов его приведи, и одень их в хитоны,
\vs Exo 40:15 и помажь их, как помазал ты отца их, чтобы они были священниками Мне, и помазание их посвятит их в вечное священство в роды их.
\rsbpar\vs Exo 40:16 И сделал Моисей все, как повелел ему Господь, так и сделал.
\vs Exo 40:17 В первый месяц второго года [по исшествии их из Египта], в первый \bibemph{день} месяца поставлена скиния.
\vs Exo 40:18 И поставил Моисей скинию, положил подножия ее, поставил брусья ее, положил шесты и поставил столбы ее,
\vs Exo 40:19 распростер покров над скиниею, и положил покрышку поверх сего покрова, как повелел Господь Моисею.
\vs Exo 40:20 И взял и положил откровение в ковчег, и вложил шесты в \bibemph{кольца} ковчега, и положил крышку на ковчег сверху;
\vs Exo 40:21 и внес ковчег в скинию, и повесил завесу, и закрыл ковчег откровения, как повелел Господь Моисею.
\vs Exo 40:22 И поставил стол в скинии собрания, на северной стороне скинии, вне завесы,
\vs Exo 40:23 и разложил на нем ряд хлебов пред Господом, как повелел Господь Моисею.
\vs Exo 40:24 И поставил светильник в скинии собрания против стола, на южной стороне скинии,
\vs Exo 40:25 и поставил лампады [его] пред Господом, как повелел Господь Моисею.
\vs Exo 40:26 И поставил золотой жертвенник в скинии собрания пред завесою
\vs Exo 40:27 и воскурил на нем благовонное курение, как повелел Господь Моисею.
\vs Exo 40:28 И повесил завесу при входе в скинию;
\vs Exo 40:29 и жертвенник всесожжения поставил у входа в скинию собрания и принес на нем всесожжения и приношение хлебное, как повелел Господь Моисею.
\vs Exo 40:30 И поставил умывальник между скиниею собрания и жертвенником и налил в него воды для омовения,
\vs Exo 40:31 и омывали из него Моисей и Аарон и сыны его руки свои и ноги свои:
\vs Exo 40:32 когда они входили в скинию собрания и подходили к жертвеннику [служить], тогда омывались [из него], как повелел Господь Моисею.
\vs Exo 40:33 И поставил двор вокруг скинии и жертвенника и повесил завесу в воротах двора.\rsbpar И так окончил Моисей дело.
\vs Exo 40:34 И покрыло облако скинию собрания, и слава Господня наполнила скинию;
\vs Exo 40:35 и не мог Моисей войти в скинию собрания, потому что осеняло ее облако, и слава Господня наполняла скинию.
\vs Exo 40:36 Когда поднималось облако от скинии, тогда отправлялись в путь сыны Израилевы во все путешествие свое;
\vs Exo 40:37 если же не поднималось облако, то и они не отправлялись в путь, доколе оно не поднималось,
\vs Exo 40:38 ибо облако Господне стояло над скиниею днем, и огонь был ночью в ней пред глазами всего дома Израилева во все путешествие их.

\bibbookdescr{Lev}{
  inline={\LARGE Третья книга Моисеева\\\Huge Левит},
  toc={Левит},
  bookmark={Левит},
  header={Левит},
  %headerleft={},
  %headerright={},
  abbr={Лев}
}
\vs Lev 1:1 И воззвал Господь к Моисею и сказал ему из скинии собрания, говоря:
\vs Lev 1:2 объяви сынам Израилевым и скажи им: когда кто из вас хочет принести жертву Господу, то, если из скота, приносите жертву вашу из скота крупного и мелкого.
\vs Lev 1:3 Если жертва его есть всесожжение из крупного скота, пусть принесет ее мужеского пола, без порока; пусть приведет ее к дверям скинии собрания, чтобы приобрести ему благоволение пред Господом;
\vs Lev 1:4 и возложит руку свою на голову \bibemph{жертвы} всесожжения~--- и приобретет он благоволение, во очищение грехов его;
\vs Lev 1:5 и заколет тельца пред Господом; сыны же Аароновы, священники, принесут кровь и покропят кровью со всех сторон на жертвенник, который у входа скинии собрания;
\vs Lev 1:6 и снимет кожу с \bibemph{жертвы} всесожжения и рассечет ее на части;
\vs Lev 1:7 сыны же Аароновы, священники, положат на жертвенник огонь и на огне разложат дрова;
\vs Lev 1:8 и разложат сыны Аароновы, священники, части, голову и тук на дровах, которые на огне, на жертвеннике;
\vs Lev 1:9 а внутренности \bibemph{жертвы} и ноги ее вымоет он водою, и сожжет священник все на жертвеннике: \bibemph{это} всесожжение, жертва, благоухание, приятное Господу.
\rsbpar\vs Lev 1:10 Если жертва всесожжения его [Господу] из мелкого скота, из овец, или из коз, пусть принесет ее мужеского пола, без порока, [и пусть возложит руку на голову ее,]
\vs Lev 1:11 и заколет ее пред Господом на северной стороне жертвенника, и сыны Аароновы, священники, покропят кровью ее на жертвенник со всех сторон;
\vs Lev 1:12 и рассекут ее на части, \bibemph{отделив} голову ее и тук ее, и разложит их священник на дровах, которые на огне, на жертвеннике,
\vs Lev 1:13 а внутренности и ноги вымоет водою, и принесет священник всё и сожжет на жертвеннике: \bibemph{это} всесожжение, жертва, благоухание, приятное Господу.
\rsbpar\vs Lev 1:14 Если же из птиц приносит он Господу всесожжение, пусть принесет жертву свою из горлиц, или из молодых голубей;
\vs Lev 1:15 священник принесет ее к жертвеннику, и свернет ей голову, и сожжет на жертвеннике, а кровь выцедит к стене жертвенника;
\vs Lev 1:16 зоб ее с перьями ее отнимет и бросит его подле жертвенника на восточную сторону, где пепел;
\vs Lev 1:17 и надломит ее в крыльях ее, не отделяя их, и сожжет ее священник на жертвеннике, на дровах, которые на огне: это всесожжение, жертва, благоухание, приятное Господу.
\vs Lev 2:1 Если какая душа хочет принести Господу жертву приношения хлебного, пусть принесет пшеничной муки, и вольет на нее елея, и положит на нее ливана,
\vs Lev 2:2 и принесет ее к сынам Аароновым, священникам, и возьмет полную горсть муки с елеем и со всем ливаном, и сожжет сие священник в память на жертвеннике; \bibemph{это} жертва, благоухание, приятное Господу;
\vs Lev 2:3 а остатки от приношения хлебного Аарону и сынам его: \bibemph{это} великая святыня из жертв Господних.
\rsbpar\vs Lev 2:4 Если же приносишь жертву приношения хлебного из печеного в печи, \bibemph{то приноси} пшеничные хлебы пресные, смешанные с елеем, и лепешки пресные, помазанные елеем.
\vs Lev 2:5 Если жертва твоя приношение хлебное со сковороды, то это должна быть пшеничная мука, смешанная с елеем, пресная;
\vs Lev 2:6 разломи ее на куски и влей на нее елея: это приношение хлебное [Господу].
\vs Lev 2:7 Если жертва твоя приношение хлебное из горшка, то должно сделать оное из пшеничной муки с елеем,
\vs Lev 2:8 и принеси приношение, которое из сего составлено, Господу; представь оное священнику, а он принесет его к жертвеннику;
\vs Lev 2:9 и возьмет священник из сей жертвы часть в память и сожжет на жертвеннике: \bibemph{это} жертва, благоухание, приятное Господу;
\vs Lev 2:10 а остатки приношения хлебного Аарону и сынам его: \bibemph{это} великая святыня из жертв Господних.
\rsbpar\vs Lev 2:11 Никакого приношения хлебного, которое прин\acc{о}сите Господу, не делайте квасного, ибо ни квасного, ни меду не должны вы сожигать в жертву Господу;
\vs Lev 2:12 как приношение начатков принос\acc{и}те их Господу, а на жертвенник не должно возносить их в приятное благоухание.
\vs Lev 2:13 Всякое приношение твое хлебное сол\acc{и} солью, и не оставляй жертвы твоей без соли завета Бога твоего: при всяком приношении твоем приноси [Господу Богу твоему] соль.
\rsbpar\vs Lev 2:14 Если приносишь Господу приношение хлебное из первых плодов, приноси в дар от первых плодов твоих из колосьев, высушенных на огне, растолченные зерна,
\vs Lev 2:15 и влей на них елея, и положи на них ливана: это приношение хлебное;
\vs Lev 2:16 и сожжет священник в память часть зерен и елея со всем ливаном: \bibemph{это} жертва Господу.
\vs Lev 3:1 Если жертва его жертва мирная, и если он приносит из крупного скота, мужеского или женского пола, пусть принесет ее Господу, не имеющую порока,
\vs Lev 3:2 и возложит руку свою на голову жертвы своей, и заколет ее у дверей скинии собрания; сыны же Аароновы, священники, покропят кровью на жертвенник со всех сторон;
\vs Lev 3:3 и принесет он из мирной жертвы в жертву Господу тук, покрывающий внутренности, и весь тук, который на внутренностях,
\vs Lev 3:4 и обе почки и тук, который на них, который на стегнах, и сальник, который на печени; с почками он отделит это;
\vs Lev 3:5 и сыны Аароновы сожгут это на жертвеннике вместе со всесожжением, которое на дровах, на огне: \bibemph{это} жертва, благоухание, приятное Господу.
\rsbpar\vs Lev 3:6 А если из мелкого скота приносит он мирную жертву Господу, мужеского или женского пола, пусть принесет ее, не имеющую порока.
\vs Lev 3:7 Если из овец приносит он жертву свою, пусть представит ее пред Господа,
\vs Lev 3:8 и возложит руку свою на голову жертвы своей, и заколет ее пред скиниею собрания, и сыны Аароновы покропят кровью ее на жертвенник со всех сторон;
\vs Lev 3:9 и пусть принесет из мирной жертвы в жертву Господу тук ее, весь курдюк, отрезав его по самую хребтовую кость, и тук, покрывающий внутренности, и весь тук, который на внутренностях,
\vs Lev 3:10 и обе почки и тук, который на них, который на стегнах, и сальник, который на печени; с почками он отделит это;
\vs Lev 3:11 священник сожжет это на жертвеннике; \bibemph{это} пища огня~--- жертва Господу.
\rsbpar\vs Lev 3:12 А если он приносит жертву из коз, пусть представит ее пред Господа,
\vs Lev 3:13 и возложит руку свою на голову ее, и заколет ее перед скиниею собрания, и покропят сыны Аароновы кровью ее на жертвенник со всех сторон;
\vs Lev 3:14 и принесет из нее в приношение, в жертву Господу тук, покрывающий внутренности, и весь тук, который на внутренностях,
\vs Lev 3:15 и обе почки и тук, который на них, который на стегнах, и сальник, который на печени; с почками он отделит это;
\vs Lev 3:16 и сожжет их священник на жертвеннике: \bibemph{это} пища огня~--- приятное благоухание [Господу]; весь тук Господу.
\vs Lev 3:17 Это постановление вечное в роды ваши, во всех жилищах ваших; никакого тука и никакой крови не ешьте.
\vs Lev 4:1 И сказал Господь Моисею, говоря:
\vs Lev 4:2 скажи сынам Израилевым: если какая душа согрешит по ошибке против каких-либо заповедей Господних и сделает что-нибудь, чего не должно делать;
\vs Lev 4:3 если священник помазанный согрешит и сделает виновным народ,~--- то за грех свой, которым согрешил, пусть представит из крупного скота тельца, без порока, Господу в жертву о грехе,
\vs Lev 4:4 и приведет тельца к дверям скинии собрания пред Господа, и возложит руки свои на голову тельца, и заколет тельца пред Господом;
\vs Lev 4:5 и возьмет священник помазанный, [посвященный совершенным посвящением,] крови тельца и внесет ее в скинию собрания,
\vs Lev 4:6 и омочит священник перст свой в кровь и покропит кровью семь раз пред Господом пред завесою святилища;
\vs Lev 4:7 и возложит священник крови [тельца] пред Господом на роги жертвенника благовонных курений, который в скинии собрания, а остальную кровь тельца выльет к подножию жертвенника всесожжений, который у входа скинии собрания;
\vs Lev 4:8 и вынет из тельца за грех весь тук его, тук, покрывающий внутренности, и весь тук, который на внутренностях,
\vs Lev 4:9 и обе почки и тук, который на них, который на стегнах, и сальник на печени; с почками отделит он это,
\vs Lev 4:10 как отделяется из тельца жертвы мирной; и сожжет их священник на жертвеннике всесожжения;
\vs Lev 4:11 а кожу тельца и все мясо его с головою и с ногами его, и внутренности его и нечистоту его,
\vs Lev 4:12 всего тельца пусть вынесет вне стана на чистое место, где высыпается пепел, и сожжет его огнем на дровах; где высыпается пепел, там пусть сожжен будет.
\rsbpar\vs Lev 4:13 Если же все общество Израилево согрешит по ошибке и скрыто будет дело от глаз собрания, и сделает что-нибудь против заповедей Господних, чего не надлежало делать, и будет виновно,
\vs Lev 4:14 то, когда узнан будет грех, которым они согрешили, пусть от всего общества представят они из крупного скота тельца в жертву за грех и приведут его пред скинию собрания;
\vs Lev 4:15 и возложат старейшины общества руки свои на голову тельца пред Господом и заколют тельца пред Господом.
\vs Lev 4:16 И внесет священник помазанный крови тельца в скинию собрания,
\vs Lev 4:17 и омочит священник перст свой в кровь [тельца] и покропит семь раз пред Господом пред завесою [святилища],
\vs Lev 4:18 и возложит крови на роги жертвенника [благовонных курений], который пред лицем Господним в скинии собрания, а остальную кровь выльет к подножию жертвенника всесожжений, который у входа скинии собрания;
\vs Lev 4:19 и весь тук его вынет из него и сожжет на жертвеннике;
\vs Lev 4:20 и сделает с тельцом то, что делается с тельцом за грех; так должен сделать с ним, и так очистит их священник, и прощено будет им;
\vs Lev 4:21 и вынесет тельца вне стана, и сожжет его так, как сожег прежнего тельца. Это жертва за грех общества.
\rsbpar\vs Lev 4:22 А если согрешит начальник, и сделает по ошибке что-нибудь против заповедей Господа, Бога своего, чего не надлежало делать, и будет виновен,
\vs Lev 4:23 то, когда узнан будет им грех, которым он согрешил, пусть приведет он в жертву козла без порока,
\vs Lev 4:24 и возложит руку свою на голову козла, и заколет его на месте, где заколаются всесожжения пред Господом: это жертва за грех;
\vs Lev 4:25 и возьмет священник перстом своим крови от жертвы за грех и возложит на роги жертвенника всесожжения, а остальную кровь его выльет к подножию жертвенника всесожжения;
\vs Lev 4:26 и весь тук его сожжет на жертвеннике, подобно как тук жертвы мирной, и так очистит его священник от греха его, и прощено будет ему.
\rsbpar\vs Lev 4:27 Если же кто из народа земли согрешит по ошибке и сделает что-нибудь против заповедей Господних, чего не надлежало делать, и виновен будет,
\vs Lev 4:28 то, когда узнан будет им грех, которым он согрешил, пусть приведет он в жертву козу без порока за грех свой, которым он согрешил,
\vs Lev 4:29 и возложит руку свою на голову жертвы за грех, и заколют [козу] в жертву за грех на месте, [где заколают] жертву всесожжения;
\vs Lev 4:30 и возьмет священник крови ее перстом своим, и возложит на роги жертвенника всесожжения, а остальную кровь ее выльет к подножию жертвенника;
\vs Lev 4:31 и весь тук ее отделит, подобно как отделяется тук из жертвы мирной, и сожжет \bibemph{его} священник на жертвеннике в приятное благоухание Господу; и так очистит его священник, и прощено будет ему.
\vs Lev 4:32 А если из стада овец захочет он принести жертву за грех, пусть принесет женского пола, без порока,
\vs Lev 4:33 и возложит руку свою на голову жертвы за грех, и заколет ее в жертву за грех на том месте, где заколают жертву всесожжения;
\vs Lev 4:34 и возьмет священник перстом своим крови от сей жертвы за грех и возложит на роги жертвенника всесожжения, а остальную кровь ее выльет к подножию жертвенника;
\vs Lev 4:35 и весь тук ее отделит, как отделяется тук овцы из жертвы мирной, и сожжет сие священник на жертвеннике в жертву Господу; и так очистит его священник от греха, которым он согрешил, и прощено будет ему.
\vs Lev 5:1 Если кто согрешит тем, что слышал голос проклятия и был свидетелем, или видел, или знал, но не объявил, то он понесет на себе грех.
\vs Lev 5:2 Или если прикоснется к чему-нибудь нечистому, или к трупу зверя нечистого, или к трупу скота нечистого, или к трупу гада нечистого, но не знал того, то он нечист и виновен.
\vs Lev 5:3 Или если прикоснется к нечистоте человеческой, какая бы то ни была нечистота, от которой оскверняются, и он не знал того, но после узнает, то он виновен.
\vs Lev 5:4 Или если кто безрассудно устами своими поклянется сделать что-нибудь худое или доброе, какое бы то ни было дело, в котором люди безрассудно клянутся, и он не знал того, но после узнает, то он виновен в том.
\vs Lev 5:5 Если он виновен в чем-нибудь из сих, и исповедается, в чем он согрешил,
\vs Lev 5:6 то пусть принесет Господу за грех свой, которым он согрешил, жертву повинности из мелкого скота, овцу или козу, за грех, и очистит его священник от греха его.
\vs Lev 5:7 Если же он не в состоянии принести овцы, то в повинность за грех свой пусть принесет Господу двух горлиц или двух молодых голубей, одного в жертву за грех, а другого во всесожжение;
\vs Lev 5:8 пусть принесет их к священнику, и [священник] представит прежде ту \bibemph{из сих птиц}, которая за грех, и надломит голову ее от шеи ее, но не отделит;
\vs Lev 5:9 и покропит кровью сей жертвы за грех на стену жертвенника, а остальную кровь выцедит к подножию жертвенника: это жертва за грех;
\vs Lev 5:10 а другую употребит во всесожжение по установлению; и так очистит его священник от греха его, которым он согрешил, и прощено будет ему.
\vs Lev 5:11 Если же он не в состоянии принести двух горлиц или двух молодых голубей, пусть принесет за то, что согрешил, десятую часть ефы пшеничной муки в жертву за грех; пусть не льет на нее елея, и ливана пусть не кладет на нее, ибо это жертва за грех;
\vs Lev 5:12 и принесет ее к священнику, а священник возьмет из нее полную горсть в память и сожжет на жертвеннике в жертву Господу: это жертва за грех;
\vs Lev 5:13 и так очистит его священник от греха его, которым он согрешил в котором-нибудь из оных \bibemph{случаев}, и прощено будет ему; [остаток] же принадлежит священнику, как приношение хлебное.
\rsbpar\vs Lev 5:14 И сказал Господь Моисею, говоря:
\vs Lev 5:15 если кто сделает преступление и по ошибке согрешит против посвященного Господу, пусть за вину свою принесет Господу из стада овец овна без порока, по твоей оценке, серебряными сиклями по сиклю священному, в жертву повинности;
\vs Lev 5:16 за ту святыню, против которой он согрешил, пусть воздаст и прибавит к тому пятую долю, и отдаст сие священнику, и священник очистит его овном жертвы повинности, и прощено будет ему.
\vs Lev 5:17 Если кто согрешит и сделает что-нибудь против заповедей Господних, чего не надлежало делать, и по неведению сделается виновным и понесет на себе грех,
\vs Lev 5:18 пусть принесет к священнику в жертву повинности овна без порока, по оценке твоей, и загладит священник проступок его, в чем он преступил по неведению, и прощено будет ему.
\vs Lev 5:19 Это жертва повинности, \bibemph{которою} он провинился пред Господом.
\vs Lev 6:1 И сказал Господь Моисею, говоря:
\vs Lev 6:2 если кто согрешит и сделает преступление пред Господом и запрется пред ближним своим в том, что ему поручено, или у него положено, или им похищено, или обманет ближнего своего,
\vs Lev 6:3 или найдет потерянное и запрется в том, и поклянется ложно в чем-нибудь, что люди делают и тем грешат,~---
\vs Lev 6:4 то, согрешив и сделавшись виновным, он должен возвратить похищенное, что похитил, или отнятое, что отнял, или порученное, что ему поручено, или потерянное, что он нашел;
\vs Lev 6:5 или если он в чем поклялся ложно, то должен отдать сполна, и приложить к тому пятую долю и отдать тому, кому принадлежит, в день приношения жертвы повинности;
\vs Lev 6:6 и за вину свою пусть принесет Господу к священнику в жертву повинности из стада овец овна без порока, по оценке твоей;
\vs Lev 6:7 и очистит его священник пред Господом, и прощено будет ему, что бы он ни сделал, все, в чем он сделался виновным.
\rsbpar\vs Lev 6:8 И сказал Господь Моисею, говоря:
\vs Lev 6:9 заповедай Аарону и сынам его: вот закон всесожжения: всесожжение пусть остается на месте сожигания на жертвеннике всю ночь до утра, и огонь жертвенника пусть горит на нем [и не угасает];
\vs Lev 6:10 и пусть священник оденется в льняную одежду свою, и наденет на тело свое льняное нижнее платье, и снимет пепел от всесожжения, которое сжег огонь на жертвеннике, и положит его подле жертвенника;
\vs Lev 6:11 и пусть снимет с себя одежды свои, и наденет другие одежды, и вынесет пепел вне стана на чистое место;
\vs Lev 6:12 а огонь на жертвеннике пусть горит [и] не угасает; и пусть священник зажигает на нем дрова каждое утро, и раскладывает на нем всесожжение, и сожигает на нем тук мирной жертвы;
\vs Lev 6:13 огонь непрестанно пусть горит на жертвеннике \bibemph{и} не угасает.
\rsbpar\vs Lev 6:14 Вот закон о приношении хлебном: [священники] сыны Аароновы должны приносить его пред Господа к жертвеннику;
\vs Lev 6:15 и пусть возьмет [священник] горстью своею из приношения хлебного и пшеничной муки и елея и весь ливан, который на жертве, и сожжет на жертвеннике: \bibemph{это} приятное благоухание, в память пред Господом;
\vs Lev 6:16 а остальное из него пусть едят Аарон и сыны его; пресным должно есть его на святом месте, на дворе скинии собрания пусть едят его;
\vs Lev 6:17 не должно печь его квасным. Сие даю Я им в долю из жертв Моих. Это великая святыня, подобно как жертва за грех и жертва повинности.
\vs Lev 6:18 Все потомки Аароновы мужеского пола могут есть ее. Это вечный участок в роды ваши из жертв Господних. Все, прикасающееся к ним, освятится.
\rsbpar\vs Lev 6:19 И сказал Господь Моисею, говоря:
\vs Lev 6:20 вот приношение от Аарона и сынов его, которое принесут они Господу в день помазания его: десятая часть ефы пшеничной муки в жертву постоянную, половина сего для утра и половина для вечера;
\vs Lev 6:21 на сковороде в елее она должна быть приготовлена; напитанную \bibemph{елеем} приноси ее в кусках, как разламывается в куски приношение хлебное; приноси ее в приятное благоухание Господу;
\vs Lev 6:22 и священник, помазанный на место его из сынов его, должен совершать сие: это вечный устав Господа. Вся она должна быть сожжена;
\vs Lev 6:23 и всякое хлебное приношение от священника все да будет сожигаемо, а не съедаемо.
\rsbpar\vs Lev 6:24 И сказал Господь Моисею, говоря:
\vs Lev 6:25 скажи Аарону и сынам его: вот закон о жертве за грех: жертва за грех должна быть заколаема пред Господом на том месте, где заколается всесожжение; это великая святыня;
\vs Lev 6:26 священник, совершающий жертву за грех, должен есть ее; она должна быть съедаема на святом месте, на дворе скинии собрания;
\vs Lev 6:27 все, что прикоснется к мясу ее, освятится; и если кровью ее обрызгана будет одежда, то обрызганное омой на святом месте;
\vs Lev 6:28 глиняный сосуд, в котором она варилась, должно разбить; если же она варилась в медном сосуде, то должно его вычистить и вымыть водою;
\vs Lev 6:29 весь мужеский пол священнического рода может есть ее: это великая святыня [у Господа];
\vs Lev 6:30 а всякая жертва за грех, от которой кровь вносится в скинию собрания для очищения во святилище, не должна быть съедаема; ее должно сожигать на огне.
\vs Lev 7:1 Вот закон о жертве повинности: это великая святыня;
\vs Lev 7:2 жертву повинности должно заколать на том месте, где заколается всесожжение, и кровью ее кропить на жертвенник со всех сторон;
\vs Lev 7:3 \bibemph{приносящий} должен представить из нее весь тук, курдюк и тук, покрывающий внутренности,
\vs Lev 7:4 и обе почки и тук, который на них, который на стегнах, и сальник, который на печени; с почками пусть он отделит сие;
\vs Lev 7:5 и сожжет сие священник на жертвеннике в жертву Господу: это жертва повинности.
\vs Lev 7:6 Весь мужеский пол священнического рода может есть ее; на святом месте должно есть ее: это великая святыня.
\vs Lev 7:7 Как о жертве за грех, так и о жертве повинности закон один: она принадлежит священнику, который очищает посредством ее.
\vs Lev 7:8 И когда священник приносит чью-нибудь жертву всесожжения, кожа от \bibemph{жертвы} всесожжения, которое он приносит, принадлежит священнику;
\vs Lev 7:9 и всякое приношение хлебное, которое печено в печи, и всякое приготовленное в горшке или на сковороде, принадлежит священнику, приносящему его;
\vs Lev 7:10 и всякое приношение хлебное, смешанное с елеем и сухое, принадлежит всем сынам Аароновым, как одному, так и другому.
\rsbpar\vs Lev 7:11 Вот закон о жертве мирной, которую приносят Господу:
\vs Lev 7:12 если кто в благодарность приносит ее, то при жертве благодарности он должен принести пресные хлебы, смешанные с елеем, и пресные лепешки, помазанные елеем, и пшеничную муку, напитанную \bibemph{елеем}, хлебы, смешанные с елеем;
\vs Lev 7:13 кроме лепешек пусть он приносит в приношение свое квасный хлеб, при мирной жертве благодарной;
\vs Lev 7:14 одно что-нибудь из всего приношения своего пусть принесет он в возношение Господу: это принадлежит священнику, кропящему кровью мирной жертвы;
\vs Lev 7:15 мясо мирной жертвы благодарности должно съесть в день приношения ее, не должно оставлять от него до утра.
\vs Lev 7:16 Если же кто приносит жертву по обету, или от усердия, то жертву его должно есть в день приношения, и на другой день оставшееся от нее есть можно,
\vs Lev 7:17 а оставшееся от жертвенного мяса к третьему дню должно сжечь на огне;
\vs Lev 7:18 если же будут есть мясо мирной жертвы на третий день, то она не будет благоприятна; кто ее принесет, тому ни во что не вменится: это осквернение, и кто будет есть ее, тот понесет на себе грех;
\vs Lev 7:19 мяса сего, если оно прикоснется к чему-либо нечистому, не должно есть, но должно сжечь его на огне; а мясо чистое может есть всякий чистый;
\vs Lev 7:20 если же какая душа, имея на себе нечистоту, будет есть мясо мирной жертвы Господней, то истребится душа та из народа своего;
\vs Lev 7:21 и если какая душа, прикоснувшись к чему-нибудь нечистому, к нечистоте человеческой, или к нечистому скоту, или какому-нибудь нечистому гаду, будет есть мясо мирной жертвы Господней, то истребится душа та из народа своего.
\rsbpar\vs Lev 7:22 И сказал Господь Моисею, говоря:
\vs Lev 7:23 скажи сынам Израилевым: никакого тука ни из вола, ни из овцы, ни из козла не ешьте.
\vs Lev 7:24 Тук из мертвого и тук из растерзанного зверем можно употреблять на всякое дело; а есть не ешьте его;
\vs Lev 7:25 ибо, кто будет есть тук из скота, который приносится в жертву Господу, истребится душа та из народа своего;
\vs Lev 7:26 и никакой крови не ешьте во всех жилищах ваших ни из птиц, ни из скота;
\vs Lev 7:27 а кто будет есть какую-нибудь кровь, истребится душа та из народа своего.
\rsbpar\vs Lev 7:28 И сказал Господь Моисею, говоря:
\vs Lev 7:29 скажи сынам Израилевым: кто представляет мирную жертву свою Господу, тот из мирной жертвы часть должен принести в приношение Господу;
\vs Lev 7:30 своими руками должен он принести в жертву Господу: тук с грудью должен он принести [и сальник на печени], потрясая грудь пред лицем Господним;
\vs Lev 7:31 тук сожжет священник на жертвеннике, а грудь принадлежит Аарону и сынам его;
\vs Lev 7:32 и правое плечо, как возношение, из мирных жертв ваших отдавайте священнику:
\vs Lev 7:33 кто из сынов Аароновых приносит кровь из мирной жертвы и тук, тому и правое плечо на долю;
\vs Lev 7:34 ибо Я беру от сынов Израилевых из мирных жертв их грудь потрясания и плечо возношения, и отдаю их Аарону священнику и сынам его в вечный участок от сынов Израилевых.
\vs Lev 7:35 Вот участок Аарону и участок сынам его из жертв Господних со дня, когда они предстанут пред Господа для священнодействия,
\vs Lev 7:36 который повелел Господь давать им со дня помазания их от сынов Израилевых. \bibemph{Это} вечное постановление в роды их.~---
\vs Lev 7:37 Вот закон о всесожжении, о приношении хлебном, о жертве за грех, о жертве повинности, о жертве посвящения и о жертве мирной,
\vs Lev 7:38 который дал Господь Моисею на горе Синае, когда повелел сынам Израилевым, в пустыне Синайской, приносить Господу приношения их.
\vs Lev 8:1 И сказал Господь Моисею, говоря:
\vs Lev 8:2 возьми Аарона и сынов его с ним, и одежды и елей помазания, и тельца для жертвы за грех и двух овнов, и корзину опресноков,
\vs Lev 8:3 и собери все общество ко входу скинии собрания.
\rsbpar\vs Lev 8:4 Моисей сделал так, как повелел ему Господь, и собралось общество ко входу скинии собрания.
\vs Lev 8:5 И сказал Моисей к обществу: вот что повелел Господь сделать.
\vs Lev 8:6 И привел Моисей Аарона и сынов его и омыл их водою;
\vs Lev 8:7 и возложил на него хитон, и опоясал его поясом, и надел на него верхнюю ризу, и возложил на него ефод, и опоясал его поясом ефода и прикрепил им ефод на нем,
\vs Lev 8:8 и возложил на него наперсник, и на наперсник положил урим и туммим,
\vs Lev 8:9 и возложил на голову его кидар, а на кидар с передней стороны его возложил полированную дощечку, диадиму святыни, как повелел Господь Моисею.
\vs Lev 8:10 И взял Моисей елей помазания, и помазал скинию и все, что в ней, и освятил это;
\vs Lev 8:11 и покропил им на жертвенник семь раз, и помазал жертвенник и все принадлежности его и умывальницу и подножие ее, чтобы освятить их;
\vs Lev 8:12 и возлил [Моисей] елей помазания на голову Аарона и помазал его, чтоб освятить его.
\vs Lev 8:13 И привел Моисей сынов Аароновых, и одел их в хитоны, и опоясал их поясом, и возложил на них кидары, как повелел Господь Моисею.
\rsbpar\vs Lev 8:14 И привел [Моисей] тельца для жертвы за грех, и Аарон и сыны его возложили руки свои на голову тельца за грех;
\vs Lev 8:15 и заколол \bibemph{его} [Моисей] и взял крови, и перстом своим возложил на роги жертвенника со всех сторон, и очистил жертвенник, а \bibemph{остальную} кровь вылил к подножию жертвенника, и освятил его, чтобы сделать его чистым.
\vs Lev 8:16 И взял [Моисей] весь тук, который на внутренностях, и сальник на печени, и обе почки и тук их, и сжег Моисей на жертвеннике;
\vs Lev 8:17 а тельца и кожу его, и мясо его, и нечистоту его сжег на огне вне стана, как повелел Господь Моисею.
\vs Lev 8:18 И привел [Моисей] овна для всесожжения, и возложили Аарон и сыны его руки свои на голову овна;
\vs Lev 8:19 и заколол \bibemph{его} Моисей и покропил кровью на жертвенник со всех сторон;
\vs Lev 8:20 и рассек овна на части, и сжег Моисей голову и части и тук,
\vs Lev 8:21 а внутренности и ноги вымыл водою, и сжег Моисей всего овна на жертвеннике: это всесожжение в приятное благоухание, это жертва Господу, как повелел Господь Моисею.
\vs Lev 8:22 И привел [Моисей] другого овна, овна посвящения, и возложили Аарон и сыны его руки свои на голову овна;
\vs Lev 8:23 и заколол \bibemph{его} Моисей, и взял крови его, и возложил на край правого уха Ааронова и на большой палец правой руки его и на большой палец правой ноги его.
\vs Lev 8:24 И привел Моисей сынов Аароновых, и возложил крови на край правого уха их и на большой палец правой руки их и на большой палец правой ноги их, и покропил Моисей кровью на жертвенник со всех сторон.
\vs Lev 8:25 И взял [Моисей] тук и курдюк и весь тук, который на внутренностях, и сальник на печени, и обе почки и тук их и правое плечо;
\vs Lev 8:26 и из корзины с опресноками, которая пред Господом, взял один опреснок и один хлеб с елеем и одну лепешку, и возложил на тук и на правое плечо;
\vs Lev 8:27 и положил все это на руки Аарону и на руки сынам его, и принес это, потрясая пред лицем Господним;
\vs Lev 8:28 и взял это Моисей с рук их и сжег на жертвеннике со всесожжением: это жертва посвящения в приятное благоухание, это жертва Господу.
\vs Lev 8:29 И взял Моисей грудь и принес ее, потрясая пред лицем Господним: это была доля Моисеева от овна посвящения, как повелел Господь Моисею.
\vs Lev 8:30 И взял Моисей елея помазания и крови, которая на жертвеннике, и покропил Аарона и одежды его, и сынов его и одежды сынов его с ним; и так освятил Аарона и одежды его, и сынов его и одежды сынов его с ним.
\rsbpar\vs Lev 8:31 И сказал Моисей Аарону и сынам его: сварите мясо у входа скинии собрания и там ешьте его с хлебом, который в корзине посвящения, как мне повелено и сказано: Аарон и сыны его должны есть его;
\vs Lev 8:32 а остатки мяса и хлеба сожгите на огне.
\vs Lev 8:33 Семь дней не отходите от дверей скинии собрания, пока не исполнятся дни посвящения вашего, ибо семь дней должно совершаться посвящение ваше;
\vs Lev 8:34 как сегодня было сделано, так повелел Господь делать для очищения вас;
\vs Lev 8:35 у входа скинии собрания будьте день и ночь в продолжение семи дней и будьте на страже у Господа, чтобы не умереть, ибо так мне повелено [от Господа Бога].
\vs Lev 8:36 И исполнил Аарон и сыны его все, что повелел Господь чрез Моисея.
\vs Lev 9:1 В восьмой день призвал Моисей Аарона и сынов его и старейшин Израилевых
\vs Lev 9:2 и сказал Аарону: возьми себе из волов тельца в жертву за грех и овна во всесожжение, обоих без порока, и представь пред лице Господне;
\vs Lev 9:3 и сынам Израилевым скажи: возьмите козла в жертву за грех, [и овна,] и тельца, и агнца, однолетних, без порока, во всесожжение,
\vs Lev 9:4 и вола и овна в жертву мирную, чтобы совершить жертвоприношение пред лицем Господним, и приношение хлебное, смешанное с елеем, ибо сегодня Господь явится вам.
\vs Lev 9:5 И принесли то, что приказал Моисей, пред скинию собрания, и пришло все общество и стало пред лицем Господним.
\vs Lev 9:6 И сказал Моисей: вот что повелел Господь сделать, и явится вам слава Господня.
\vs Lev 9:7 И сказал Моисей Аарону: приступи к жертвеннику и соверши жертву твою о грехе и всесожжение твое, и очисти себя и народ, и сделай приношение от народа, и очисти их, как повелел Господь.
\rsbpar\vs Lev 9:8 И приступил Аарон к жертвеннику и заколол тельца, который за него, в жертву за грех:
\vs Lev 9:9 сыны Аарона поднесли ему кровь, и он омочил перст свой в крови и возложил на роги жертвенника, а \bibemph{остальную} кровь вылил к подножию жертвенника,
\vs Lev 9:10 а тук и почки и сальник на печени от жертвы за грех сжег на жертвеннике, как повелел Господь Моисею;
\vs Lev 9:11 мясо же и кожу сжег на огне вне стана.
\vs Lev 9:12 И заколол всесожжение, и сыны Аарона поднесли ему кровь; он покропил ею на жертвенник со всех сторон;
\vs Lev 9:13 и принесли ему всесожжение в кусках и голову, и он сжег на жертвеннике,
\vs Lev 9:14 а внутренности и ноги омыл и сжег со всесожжением на жертвеннике.
\vs Lev 9:15 И принес приношение от народа, и взял от народа козла за грех, и заколол его, и принес его в жертву за грех, как и прежнего.
\vs Lev 9:16 И принес всесожжение и совершил его по уставу.
\vs Lev 9:17 И принес приношение хлебное, и наполнил им руки свои, и сжег на жертвеннике сверх утреннего всесожжения.
\vs Lev 9:18 И заколол вола и овна, которые от народа, в жертву мирную; и сыны Аарона поднесли ему кровь, и он покропил ею на жертвенник со всех сторон;
\vs Lev 9:19 \bibemph{поднесли} и тук из вола, и из овна курдюк, и [тук] покрывающий [внутренности], почки и сальник на печени,
\vs Lev 9:20 и положили тук на грудь, и он сжег тук на жертвеннике;
\vs Lev 9:21 грудь же и правое плечо принес Аарон, потрясая пред лицем Господним, как повелел Моисей.
\vs Lev 9:22 И поднял Аарон руки свои, \bibemph{обратившись} к народу, и благословил его, и сошел, совершив жертву за грех, всесожжение и жертву мирную.
\vs Lev 9:23 И вошли Моисей и Аарон в скинию собрания, и вышли, и благословили народ. И явилась слава Господня всему народу:
\vs Lev 9:24 и вышел огонь от Господа и сжег на жертвеннике всесожжение и тук; и видел весь народ, и воскликнул от радости, и пал на лице свое.
\vs Lev 10:1 Надав и Авиуд, сыны Аароновы, взяли каждый свою кадильницу, и положили в них огня, и вложили в него курений, и принесли пред Господа огонь чуждый, которого Он не велел им;
\vs Lev 10:2 и вышел огонь от Господа и сжег их, и умерли они пред лицем Господним.
\vs Lev 10:3 И сказал Моисей Аарону: вот о чем говорил Господь, когда сказал: в приближающихся ко Мне освящусь и пред всем народом прославлюсь. Аарон молчал.
\vs Lev 10:4 И позвал Моисей Мисаила и Елцафана, сынов Узиила, дяди Ааронова, и сказал им: пойдите, вынесите братьев ваших из святилища за стан.
\vs Lev 10:5 И пошли и вынесли их в хитонах их за стан, как сказал Моисей.
\vs Lev 10:6 Аарону же и Елеазару и Ифамару, сынам его, Моисей сказал: голов ваших не обнажайте и одежд ваших не раздирайте, чтобы вам не умереть и не навести гнева на все общество; но братья ваши, весь дом Израилев, могут плакать о сожженных, которых сожег Господь,
\vs Lev 10:7 и из дверей скинии собрания не выходите, чтобы не умереть вам, ибо на вас елей помазания Господня. И сделали по слову Моисея.
\rsbpar\vs Lev 10:8 И сказал Господь Аарону, говоря:
\vs Lev 10:9 вина и крепких напитков не пей ты и сыны твои с тобою, когда входите в скинию собрания, [или приступаете к жертвеннику,] чтобы не умереть. \bibemph{Это} вечное постановление в роды ваши,
\vs Lev 10:10 чтобы вы могли отличать священное от несвященного и нечистое от чистого,
\vs Lev 10:11 и научать сынов Израилевых всем уставам, которые изрек им Господь чрез Моисея.
\rsbpar\vs Lev 10:12 И сказал Моисей Аарону и Елеазару и Ифамару, оставшимся сынам его: возьмите приношение хлебное, оставшееся от жертв Господних, и ешьте его пресное у жертвенника, ибо это великая святыня;
\vs Lev 10:13 и ешьте его на святом месте, ибо это участок твой и участок сынов твоих из жертв Господних: так мне повелено [от Господа];
\vs Lev 10:14 и грудь потрясания и плечо возношения ешьте на чистом месте, ты и сыновья твои и дочери твои с тобою, ибо это дано в участок тебе и в участок сынам твоим из мирных жертв сынов Израилевых;
\vs Lev 10:15 плечо возношения и грудь потрясания должны они приносить с жертвами тука, потрясая пред лицем Господним, и да будет это вечным участком тебе и сыновьям твоим [и дочерям твоим] с тобою, как повелел Господь [Моисею].
\vs Lev 10:16 И козла жертвы за грех искал Моисей, и вот, он сожжен. И разгневался [Моисей] на Елеазара и Ифамара, оставшихся сынов Аароновых, и сказал:
\vs Lev 10:17 почему вы не ели жертвы за грех на святом месте? ибо она святыня великая, и она дана вам, чтобы снимать грехи с общества и очищать их пред Господом;
\vs Lev 10:18 вот, кровь ее не внесена внутрь святилища, а вы должны были есть ее на святом месте, как повелено мне.
\vs Lev 10:19 Аарон сказал Моисею: вот, сегодня принесли они жертву свою за грех и всесожжение свое пред Господом, и это случилось со мною; если я сегодня съем жертву за грех, будет ли это угодно Господу?
\vs Lev 10:20 И услышал Моисей и одобрил.
\vs Lev 11:1 И сказал Господь Моисею и Аарону, говоря им:
\vs Lev 11:2 скажите сынам Израилевым: вот животные, которые можно вам есть из всего скота на земле:
\vs Lev 11:3 всякий скот, у которого раздвоены копыта и на копытах глубокий разрез, и который жует жвачку, ешьте;
\vs Lev 11:4 только сих не ешьте из жующих жвачку и имеющих раздвоенные копыта: верблюда, потому что он жует жвачку, но копыта у него не раздвоены, нечист он для вас;
\vs Lev 11:5 и тушканчика, потому что он жует жвачку, но копыта у него не раздвоены, нечист он для вас,
\vs Lev 11:6 и зайца, потому что он жует жвачку, но копыта у него не раздвоены, нечист он для вас;
\vs Lev 11:7 и свиньи, потому что копыта у нее раздвоены и на копытах разрез глубокий, но она не жует жвачки, нечиста она для вас;
\vs Lev 11:8 мяса их не ешьте и к трупам их не прикасайтесь; нечисты они для вас.
\vs Lev 11:9 Из всех \bibemph{животных}, которые в воде, ешьте сих: у которых есть перья и чешуя в воде, в морях ли, или реках, тех ешьте;
\vs Lev 11:10 а все те, у которых нет перьев и чешуи, в морях ли, или реках, из всех плавающих в водах и из всего живущего в водах, скверны для вас;
\vs Lev 11:11 они должны быть скверны для вас: мяса их не ешьте и трупов их гнушайтесь;
\vs Lev 11:12 все \bibemph{животные}, у которых нет перьев и чешуи в воде, скверны для вас.
\vs Lev 11:13 Из птиц же гнушайтесь сих [не должно их есть, скверны они]: орла, грифа и морского орла,
\vs Lev 11:14 коршуна и сокола с породою его,
\vs Lev 11:15 всякого ворона с породою его,
\vs Lev 11:16 страуса, совы, чайки и ястреба с породою его,
\vs Lev 11:17 филина, рыболова и ибиса,
\vs Lev 11:18 лебедя, пеликана и сипа,
\vs Lev 11:19 цапли, зуя с породою его, удода и нетопыря.
\vs Lev 11:20 Все \bibemph{животные} пресмыкающиеся, крылатые, ходящие на четырех \bibemph{ногах}, скверны для вас;
\vs Lev 11:21 из всех пресмыкающихся, крылатых, ходящих на четырех \bibemph{ногах}, тех только ешьте, у которых есть голени выше ног, чтобы скакать ими по земле;
\vs Lev 11:22 сих ешьте из них: саранчу с ее породою, солам с ее породою, харгол с ее породою и хагаб с ее породою.
\vs Lev 11:23 Всякое \bibemph{другое} пресмыкающееся, крылатое, у которого четыре ноги, скверно для вас;
\vs Lev 11:24 от них вы будете нечисты: всякий, кто прикоснется к трупу их, нечист будет до вечера;
\vs Lev 11:25 и всякий, кто возьмет труп их, должен омыть одежду свою и нечист будет до вечера.
\vs Lev 11:26 Всякий скот, у которого копыта раздвоены, но нет глубокого разреза, и который не жует жвачки, нечист для вас: всякий, кто прикоснется к нему, будет нечист [до вечера].
\vs Lev 11:27 Из всех зверей четвероногих те, которые ходят на лапах, нечисты для вас: всякий, кто прикоснется к трупу их, нечист будет до вечера;
\vs Lev 11:28 кто возьмет труп их, тот должен омыть одежды свои и нечист будет до вечера: нечисты они для вас.
\rsbpar\vs Lev 11:29 Вот что нечисто для вас из животных, пресмыкающихся по земле: крот, мышь, ящерица с ее породою,
\vs Lev 11:30 анака, хамелеон, летаа, хомет и тиншемет,~---
\vs Lev 11:31 сии нечисты для вас из всех пресмыкающихся: всякий, кто прикоснется к ним мертвым, нечист будет до вечера.
\vs Lev 11:32 И всё, на что упадет которое-нибудь из них мертвое, всякий деревянный сосуд, или одежда, или кожа, или мешок, и всякая вещь, которая употребляется на дело, будут нечисты: в воду должно положить их, и нечисты будут до вечера, потом будут чисты;
\vs Lev 11:33 если же которое-нибудь из них упадет в какой-нибудь глиняный сосуд, то находящееся в нем будет нечисто, и самый [сосуд] разбейте.
\vs Lev 11:34 Всякая пища, которую едят, на которой была вода \bibemph{из такого сосуда}, нечиста будет [для вас], и всякое питье, которое пьют, во всяком \bibemph{таком} сосуде нечисто будет.
\vs Lev 11:35 Всё, на что упадет что-нибудь от трупа их, нечисто будет: печь и очаг должно разломать, они нечисты; и они должны быть нечисты для вас;
\vs Lev 11:36 только источник и колодезь, вмещающий воду, остаются чистыми; а кто прикоснется к трупу их, тот нечист.
\vs Lev 11:37 И если что-нибудь от трупа их упадет на какое-либо семя, которое сеют, то оно чисто;
\vs Lev 11:38 если же тогда, как вода налита на семя, упадет на него что-нибудь от трупа их, то оно нечисто для вас.
\vs Lev 11:39 И когда умрет какой-либо скот, который употребляется вами в пищу, то прикоснувшийся к трупу его нечист будет до вечера;
\vs Lev 11:40 и тот, кто будет есть мертвечину его, должен омыть одежды свои и нечист будет до вечера; и тот, кто понесет труп его, должен омыть одежды свои и нечист будет до вечера.
\vs Lev 11:41 Всякое животное, пресмыкающееся по земле, скверно для вас, не должно есть \bibemph{его};
\vs Lev 11:42 всего ползающего на чреве и всего ходящего на четырех ногах, и многоножных из животных пресмыкающихся по земле, не ешьте, ибо они скверны;
\vs Lev 11:43 не оскверняйте душ ваших каким-либо животным пресмыкающимся и не делайте себя чрез них нечистыми, чтоб быть чрез них нечистыми,
\vs Lev 11:44 ибо Я~--- Господь Бог ваш: освящайтесь и будьте святы, ибо Я [Господь, Бог ваш] свят; и не оскверняйте душ ваших каким-либо животным, ползающим по земле,
\vs Lev 11:45 ибо Я~--- Господь, выведший вас из земли Египетской, чтобы быть вашим Богом. Итак будьте святы, потому что Я свят.
\rsbpar\vs Lev 11:46 Вот закон о скоте, о птицах, о всех животных, живущих в водах, и о всех животных, пресмыкающихся по земле,
\vs Lev 11:47 чтобы отличать нечистое от чистого, и животных, которых можно есть, от животных, которых есть не должно.
\vs Lev 12:1 И сказал Господь Моисею, говоря:
\vs Lev 12:2 скажи сынам Израилевым: если женщина зачнет и родит \bibemph{младенца} мужеского пола, то она нечиста будет семь дней; как во дни страдания ее очищением, она будет нечиста;
\vs Lev 12:3 в восьмой же день обрежется у него крайняя плоть его;
\vs Lev 12:4 и тридцать три дня должна она сидеть, очищаясь от кровей своих; ни к чему священному не должна прикасаться и к святилищу не должна приходить, пока не исполнятся дни очищения ее.
\vs Lev 12:5 Если же она родит \bibemph{младенца} женского пола, то во время очищения своего она будет нечиста две недели, и шестьдесят шесть дней должна сидеть, очищаясь от кровей своих.
\vs Lev 12:6 По окончании дней очищения своего за сына или за дочь она должна принести однолетнего агнца во всесожжение и молодого голубя или горлицу в жертву за грех, ко входу скинии собрания к священнику;
\vs Lev 12:7 он принесет это пред Господа и очистит ее, и она будет чиста от течения кровей ее. Вот закон о родившей \bibemph{младенца} мужеского или женского пола.
\vs Lev 12:8 Если же она не в состоянии принести агнца, то пусть возьмет двух горлиц или двух молодых голубей, одного во всесожжение, а другого в жертву за грех, и очистит ее священник, и она будет чиста.
\vs Lev 13:1 И сказал Господь Моисею и Аарону, говоря:
\vs Lev 13:2 когда у кого появится на коже тела его опухоль, или лишаи, или пятно, и на коже тела его сделается как бы язва проказы, то должно привести его к Аарону священнику, или к одному из сынов его, священников;
\vs Lev 13:3 священник осмотрит язву на коже тела, и если волосы на язве изменились в белые, и язва оказывается углубленною в кожу тела его, то это язва проказы; священник, осмотрев его, объявит его нечистым.
\vs Lev 13:4 А если на коже тела его пятно белое, но оно не окажется углубленным в кожу, и волосы на нем не изменились в белые, то священник \bibemph{имеющего} язву должен заключить на семь дней;
\vs Lev 13:5 в седьмой день священник осмотрит его, и если язва остается в своем виде и не распространяется язва по коже, то священник должен заключить его на другие семь дней;
\vs Lev 13:6 в седьмой день опять священник осмотрит его, и если язва менее приметна и не распространилась язва по коже, то священник должен объявить его чистым: это лишаи, и пусть он омоет одежды свои, и будет чист.
\vs Lev 13:7 Если же лишаи станут распространяться по коже, после того как он являлся к священнику для очищения, то он вторично должен явиться к священнику;
\vs Lev 13:8 священник, увидев, что лишаи распространяются по коже, объявит его нечистым: это проказа.
\rsbpar\vs Lev 13:9 Если будет на ком язва проказы, то должно привести его к священнику;
\vs Lev 13:10 священник осмотрит, и если опухоль на коже бела, и волос изменился в белый, и на опухоли живое мясо,
\vs Lev 13:11 то это застарелая проказа на коже тела его; и священник объявит его нечистым и заключит его, ибо он нечист.
\vs Lev 13:12 Если же проказа расцветет на коже, и покроет проказа всю кожу больного от головы его до ног, сколько могут видеть глаза священника,
\vs Lev 13:13 и увидит священник, что проказа покрыла все тело его, то он объявит больного чистым, потому что все превратилось в белое: он чист.
\vs Lev 13:14 Когда же окажется на нем живое мясо, то он нечист;
\vs Lev 13:15 священник, увидев живое мясо, объявит его нечистым; живое мясо нечисто: это проказа.
\vs Lev 13:16 Если же живое мясо изменится и обратится в белое, пусть он придет к священнику;
\vs Lev 13:17 священник осмотрит его, и если язва обратилась в белое, священник объявит больного чистым; он чист.
\rsbpar\vs Lev 13:18 Если у кого на коже тела был нарыв и зажил,
\vs Lev 13:19 и на месте нарыва появилась белая опухоль, или пятно белое или красноватое, то он должен явиться к священнику;
\vs Lev 13:20 священник осмотрит его, и если оно окажется ниже кожи, и волос его изменился в белый, то священник объявит его нечистым: это язва проказы, она расцвела на нарыве;
\vs Lev 13:21 если же священник увидит, что волос на ней не бел, и она не ниже кожи, и притом мало приметна, то священник заключит его на семь дней;
\vs Lev 13:22 если она станет очень распространяться по коже, то священник объявит его нечистым: это язва;
\vs Lev 13:23 если же пятно остается на своем месте и не распространяется, то это воспаление нарыва, и священник объявит его чистым.
\vs Lev 13:24 Или если у кого на коже тела будет ожог, и на зажившем ожоге окажется красноватое или белое пятно,
\vs Lev 13:25 и священник увидит, что волос на пятне изменился в белый, и оно окажется углубленным в коже, то это проказа, она расцвела на ожоге; и священник объявит его нечистым: это язва проказы;
\vs Lev 13:26 если же священник увидит, что волос на пятне не бел, и оно не ниже кожи, и притом мало приметно, то священник заключит его на семь дней;
\vs Lev 13:27 в седьмой день священник осмотрит его, и если оно очень распространяется по коже, то священник объявит его нечистым: это язва проказы;
\vs Lev 13:28 если же пятно остается на своем месте и не распространяется по коже, и притом мало приметно, то это опухоль от ожога; священник объявит его чистым, ибо это воспаление от ожога.
\rsbpar\vs Lev 13:29 Если у мужчины или у женщины будет язва на голове или на бороде,
\vs Lev 13:30 и осмотрит священник язву, и она окажется углубленною в коже, и волос на ней желтоватый тонкий, то священник объявит их нечистыми: это паршивость, это проказа на голове или на бороде;
\vs Lev 13:31 если же священник осмотрит язву паршивости и она не окажется углубленною в коже, и волос на ней не черный, то священник \bibemph{имеющего} язву паршивости заключит на семь дней;
\vs Lev 13:32 в седьмой день священник осмотрит язву, и если паршивость не распространяется, и нет на ней желтоватого волоса, и паршивость не окажется углубленною в коже,
\vs Lev 13:33 то \bibemph{больного} должно остричь, но паршивого места не остригать, и священник должен паршивого вторично заключить на семь дней;
\vs Lev 13:34 в седьмой день священник осмотрит паршивость, и если паршивость не распространяется по коже и не окажется углубленною в коже, то священник объявит его чистым; пусть он омоет одежды свои, и будет чист.
\vs Lev 13:35 Если же после очищения его будет очень распространяться паршивость по коже,
\vs Lev 13:36 и священник увидит, что паршивость распространяется по коже, то священник пусть не ищет желтоватого волоса: он нечист.
\vs Lev 13:37 Если же паршивость остается в своем виде, и показывается на ней волос черный, то паршивость прошла, он чист; священник объявит его чистым.
\vs Lev 13:38 Если у мужчины или у женщины на коже тела их будут пятна, пятна белые,
\vs Lev 13:39 и священник увидит, что на коже тела их пятна бледно-белые, то это лишай, расцветший на коже: он чист.
\rsbpar\vs Lev 13:40 Если у кого на голове вылезли \bibemph{волосы}, то это плешивый: он чист;
\vs Lev 13:41 а если на передней стороне головы вылезли \bibemph{волосы}, то это лысый: он чист.
\vs Lev 13:42 Если же на плеши или на лысине будет белое или красноватое пятно, то на плеши его или на лысине его расцвела проказа;
\vs Lev 13:43 священник осмотрит его, и если увидит, что опухоль язвы бела \bibemph{или} красновата на плеши его или на лысине его, видом похожа на проказу кожи тела,
\vs Lev 13:44 то он прокаженный, нечист он; священник должен объявить его нечистым, у него на голове язва.
\vs Lev 13:45 У прокаженного, на котором эта язва, должна быть разодрана одежда, и голова его должна быть не покрыта, и до уст он должен быть закрыт и кричать: нечист! нечист!
\vs Lev 13:46 Во все дни, доколе на нем язва, он должен быть нечист, нечист он; он должен жить отдельно, вне стана жилище его.
\rsbpar\vs Lev 13:47 Если язва проказы будет на одежде, на одежде шерстяной, или на одежде льняной,
\vs Lev 13:48 или на основе, или на утоке из льна или шерсти, или на коже, или на каком-нибудь изделии кожаном,
\vs Lev 13:49 и пятно будет зеленоватое или красноватое на одежде, или на коже, или на основе, или на утоке, или на какой-нибудь кожаной вещи,~--- то это язва проказы: должно показать ее священнику;
\vs Lev 13:50 священник осмотрит язву и заключит зараженное язвою на семь дней;
\vs Lev 13:51 в седьмой день осмотрит священник зараженное, и если язва распространилась по одежде, или по основе, или по утоку, или по коже, или по какому-либо изделию, сделанному из кожи, то это проказа едкая, язва нечистая;
\vs Lev 13:52 он должен сжечь одежду, или основу, или уток шерстяной или льняной, или какую бы то ни было кожаную вещь, на которой будет язва, ибо это проказа едкая: должно сжечь на огне.
\vs Lev 13:53 Если же священник увидит, что язва не распространилась по одежде, или по основе, или по утоку, или по какой бы то ни было кожаной вещи,
\vs Lev 13:54 то священник прикажет омыть то, на чем язва, и вторично заключит на семь дней;
\vs Lev 13:55 если по омытии зараженной \bibemph{вещи} священник увидит, что язва не изменила вида своего и не распространилась язва, то она нечиста, сожги ее на огне; это выеденная ямина на лицевой стороне или на изнанке;
\vs Lev 13:56 если же священник увидит, что язва по омытии ее сделалась менее приметна, то священник пусть оторвет ее от одежды, или от кожи, или от основы, или от утока.
\vs Lev 13:57 Если же она опять покажется на одежде, или на основе, или на утоке, или на какой-нибудь кожаной вещи, то это расцветающая язва: сожги на огне то, на чем язва.
\vs Lev 13:58 Если же одежду, или основу, или уток, или какую-нибудь кожаную вещь вымоешь, и сойдет с них язва, то должно вымыть их вторично, и они будут чисты.
\vs Lev 13:59 Вот закон о язве проказы на одежде шерстяной или льняной, или на основе и на утоке, или на какой-нибудь кожаной вещи, как объявлять ее чистою или нечистою.
\vs Lev 14:1 И сказал Господь Моисею, говоря:
\vs Lev 14:2 вот закон о прокаженном, когда надобно его очистить: приведут его к священнику;
\vs Lev 14:3 священник выйдет вон из стана, и если священник увидит, что прокаженный исцелился от болезни прокажения,
\vs Lev 14:4 то священник прикажет взять для очищаемого двух птиц живых чистых, кедрового дерева, червленую нить и иссопа,
\vs Lev 14:5 и прикажет священник заколоть одну птицу над глиняным сосудом, над живою водою;
\vs Lev 14:6 а сам он возьмет живую птицу, кедровое дерево, червленую нить и иссоп, и омочит их и живую птицу в крови птицы заколотой над живою водою,
\vs Lev 14:7 и покропит на очищаемого от проказы семь раз, и объявит его чистым, и пустит живую птицу в поле.
\vs Lev 14:8 Очищаемый омоет одежды свои, острижет все волосы свои, омоется водою, и будет чист; потом войдет в стан и пробудет семь дней вне шатра своего;
\vs Lev 14:9 в седьмой день обреет все волосы свои, голову свою, бороду свою, брови глаз своих, все волосы свои обреет, и омоет одежды свои, и омоет тело свое водою, и будет чист;
\vs Lev 14:10 в восьмой день возьмет он двух овнов [однолетних] без порока, и одну овцу однолетнюю без порока, и три десятых части ефы пшеничной муки, смешанной с елеем, в приношение хлебное, и один лог елея;
\vs Lev 14:11 священник очищающий поставит очищаемого человека с ними пред Господом у входа скинии собрания;
\vs Lev 14:12 и возьмет священник одного овна, и представит его в жертву повинности, и лог елея, и принесет это, потрясая пред Господом;
\vs Lev 14:13 и заколет овна на том месте, где заколают жертву за грех и всесожжение, на месте святом, ибо сия жертва повинности, подобно жертве за грех, принадлежит священнику: это великая святыня;
\vs Lev 14:14 и возьмет священник крови жертвы повинности, и возложит священник на край правого уха очищаемого и на большой палец правой руки его и на большой палец правой ноги его;
\vs Lev 14:15 и возьмет священник из лога елея и польет на левую свою ладонь;
\vs Lev 14:16 и омочит священник правый перст свой в елей, который на левой ладони его, и покропит елеем с перста своего семь раз пред лицем Господа;
\vs Lev 14:17 оставшийся же елей, который на ладони его, возложит священник на край правого уха очищаемого, на большой палец правой руки его и на большой палец правой ноги его, на \bibemph{места, где} кровь жертвы повинности;
\vs Lev 14:18 а остальной елей, который на ладони священника, возложит он на голову очищаемого, и очистит его священник пред лицем Господа.
\vs Lev 14:19 И совершит священник жертву за грех, и очистит очищаемого от нечистоты его; после того заколет \bibemph{жертву} всесожжения;
\vs Lev 14:20 и возложит священник всесожжение и приношение хлебное на жертвенник; и очистит его священник, и он будет чист.
\vs Lev 14:21 Если же он беден и не имеет достатка, то пусть возьмет одного овна в жертву повинности для потрясания, чтоб очистить себя, и одну десятую часть \bibemph{ефы} пшеничной муки, смешанной с елеем, в приношение хлебное, и лог елея,
\vs Lev 14:22 и двух горлиц или двух молодых голубей, что достанет рука его, одну \bibemph{из птиц} в жертву за грех, а другую во всесожжение;
\vs Lev 14:23 и принесет их в восьмой день очищения своего к священнику ко входу скинии собрания, пред лице Господа;
\vs Lev 14:24 священник возьмет овна жертвы повинности и лог елея, и принесет это священник, потрясая пред Господом;
\vs Lev 14:25 и заколет овна в жертву повинности, и возьмет священник крови жертвы повинности, и возложит на край правого уха очищаемого и на большой палец правой руки его и на большой палец правой ноги его;
\vs Lev 14:26 и нальет священник елея на левую свою ладонь,
\vs Lev 14:27 и елеем, который на левой ладони его, покропит священник с правого перста своего семь раз пред лицем Господним;
\vs Lev 14:28 и возложит священник елея, который на ладони его, на край правого уха очищаемого, на большой палец правой руки его и на большой палец правой ноги его, на места, \bibemph{где} кровь жертвы повинности;
\vs Lev 14:29 а остальной елей, который на ладони священника, возложит он на голову очищаемого, чтоб очистить его пред лицем Господа;
\vs Lev 14:30 и принесет одну из горлиц или одного из молодых голубей, что достанет рука \bibemph{очищаемого},
\vs Lev 14:31 из того, что достанет рука его, одну \bibemph{птицу} в жертву за грех, а другую во всесожжение, вместе с приношением хлебным; и очистит священник очищаемого пред лицем Господа.
\vs Lev 14:32 Вот закон о прокаженном, который во время очищения своего не имеет достатка.
\rsbpar\vs Lev 14:33 И сказал Господь Моисею и Аарону, говоря:
\vs Lev 14:34 когда войдете в землю Ханаанскую, которую Я даю вам во владение, и Я наведу язву проказы на домы в земле владения вашего,
\vs Lev 14:35 тогда тот, чей дом, должен пойти и сказать священнику: у меня на доме показалась как бы язва.
\vs Lev 14:36 Священник прикажет опорожнить дом, прежде нежели войдет священник осматривать язву, чтобы не сделалось нечистым все, что в доме; после сего придет священник осматривать дом.
\vs Lev 14:37 Если он, осмотрев язву, увидит, что язва на стенах дома состоит из зеленоватых или красноватых ямин, которые окажутся углубленными в стене,
\vs Lev 14:38 то священник выйдет из дома к дверям дома и запрет дом на семь дней.
\vs Lev 14:39 В седьмой день опять придет священник, и если увидит, что язва распространилась по стенам дома,
\vs Lev 14:40 то священник прикажет выломать камни, на которых язва, и бросить их вне города на место нечистое;
\vs Lev 14:41 а дом внутри пусть весь оскоблят, и обмазку, которую отскоблят, высыпят вне города на место нечистое;
\vs Lev 14:42 и возьмут другие камни, и вставят вместо тех камней, и возьмут другую обмазку, и обмажут дом.
\vs Lev 14:43 Если язва опять появится и будет цвести на доме после того, как выломали камни и оскоблили дом и обмазали,
\vs Lev 14:44 то священник придет и осмотрит, и если язва на доме распространилась, то это едкая проказа на доме, нечист он;
\vs Lev 14:45 должно разломать сей дом, и камни его и дерево его и всю обмазку дома вынести вне города на место нечистое;
\vs Lev 14:46 кто входит в дом во все время, когда он заперт, тот нечист до вечера;
\vs Lev 14:47 и кто спит в доме том, тот должен вымыть одежды свои [и нечист будет до вечера]; и кто ест в доме том, тот должен вымыть одежды свои [и нечист будет до вечера].
\vs Lev 14:48 Если же священник придет и увидит, что язва на доме не распространилась после того, как обмазали дом, то священник объявит дом чистым, потому что язва прошла.
\vs Lev 14:49 И чтобы очистить дом, возьмет он две птицы, кедрового дерева, червленую нить и иссопа,
\vs Lev 14:50 и заколет одну птицу над глиняным сосудом, над живою водою;
\vs Lev 14:51 и возьмет кедровое дерево и иссоп, и червленую нить и живую птицу, и омочит их в крови птицы заколотой и в живой воде, и покропит дом семь раз;
\vs Lev 14:52 и очистит дом кровью птицы и живою водою, и живою птицею и кедровым деревом, и иссопом и червленою нитью;
\vs Lev 14:53 и пустит живую птицу вне города в поле и очистит дом, и будет чист.
\rsbpar\vs Lev 14:54 Вот закон о всякой язве проказы и о паршивости,
\vs Lev 14:55 и о проказе на одежде и на доме, и об опухоли, и о лишаях, и о пятнах,~---
\vs Lev 14:56 чтобы указать, когда это нечисто и когда чисто. Вот закон о проказе.
\vs Lev 15:1 И сказал Господь Моисею и Аарону, говоря:
\vs Lev 15:2 объявите сынам Израилевым и скажите им: если у кого будет истечение из тела его, то от истечения своего он нечист.
\vs Lev 15:3 И вот [закон] о нечистоте его от истечения его: когда течет из тела его истечение его, и когда задерживается в теле его истечение его, это нечистота его;
\vs Lev 15:4 всякая постель, на которой ляжет имеющий истечение, нечиста, и всякая вещь, на которую сядет [имеющий истечение семени], нечиста;
\vs Lev 15:5 и кто прикоснется к постели его, тот должен вымыть одежды свои и омыться водою и нечист будет до вечера;
\vs Lev 15:6 кто сядет на какую-либо вещь, на которой сидел имеющий истечение, тот должен вымыть одежды свои и омыться водою и нечист будет до вечера;
\vs Lev 15:7 и кто прикоснется к телу имеющего истечение, тот должен вымыть одежды свои и омыться водою и нечист будет до вечера;
\vs Lev 15:8 если имеющий истечение плюнет на чистого, то сей должен вымыть одежды свои и омыться водою, и нечист будет до вечера;
\vs Lev 15:9 и всякая повозка, в которой ехал имеющий истечение, нечиста [будет до вечера];
\vs Lev 15:10 и всякий, кто прикоснется к чему-нибудь, что было под ним, нечист будет до вечера; и кто понесет это, должен вымыть одежды свои и омыться водою, и нечист будет до вечера;
\vs Lev 15:11 и всякий, к кому прикоснется имеющий истечение, не омыв рук своих водою, должен вымыть одежды свои и омыться водою, и нечист будет до вечера;
\vs Lev 15:12 глиняный сосуд, к которому прикоснется имеющий истечение, должно разбить, а всякий деревянный сосуд должно вымыть водою [и будет чист].
\vs Lev 15:13 А когда имеющий истечение освободится от истечения своего, тогда должен он отсчитать себе семь дней для очищения своего и вымыть одежды свои, и омыть тело свое живою водою, и будет чист;
\vs Lev 15:14 и в восьмой день возьмет он себе двух горлиц или двух молодых голубей, и придет пред лице Господне ко входу скинии собрания, и отдаст их священнику;
\vs Lev 15:15 и принесет священник из сих \bibemph{птиц} одну в жертву за грех, а другую во всесожжение, и очистит его священник пред Господом от истечения его.
\vs Lev 15:16 Если у кого случится излияние семени, то он должен омыть водою все тело свое, и нечист будет до вечера;
\vs Lev 15:17 и всякая одежда и всякая кожа, на которую попадет семя, должна быть вымыта водою, и нечиста будет до вечера;
\vs Lev 15:18 если мужчина ляжет с женщиной и \bibemph{будет} у него излияние семени, то они должны омыться водою, и нечисты будут до вечера.
\vs Lev 15:19 Если женщина имеет истечение крови, текущей из тела ее, то она должна сидеть семь дней во время очищения своего, и всякий, кто прикоснется к ней, нечист будет до вечера;
\vs Lev 15:20 и всё, на чем она ляжет в продолжение очищения своего, нечисто; и всё, на чем сядет, нечисто;
\vs Lev 15:21 и всякий, кто прикоснется к постели ее, должен вымыть одежды свои и омыться водою и нечист будет до вечера;
\vs Lev 15:22 и всякий, кто прикоснется к какой-нибудь вещи, на которой она сидела, должен вымыть одежды свои и омыться водою, и нечист будет до вечера;
\vs Lev 15:23 и если кто прикоснется к чему-нибудь на постели или на той вещи, на которой она сидела, нечист будет до вечера;
\vs Lev 15:24 если переспит с нею муж, то нечистота ее будет на нем; он нечист будет семь дней, и всякая постель, на которой он ляжет, будет нечиста.
\vs Lev 15:25 Если у женщины течет кровь многие дни не во время очищения ее, или если она имеет истечение долее \bibemph{обыкновенного} очищения ее, то во все время истечения нечистоты ее, подобно как в продолжение очищения своего, она нечиста;
\vs Lev 15:26 всякая постель, на которой она ляжет во все время истечения своего, будет \bibemph{нечиста}, подобно как постель в продолжение очищения ее; и всякая вещь, на которую она сядет, будет нечиста, как нечисто это во время очищения ее;
\vs Lev 15:27 и всякий, кто прикоснется к ним, будет нечист, и должен вымыть одежды свои и омыться водою, и нечист будет до вечера.
\vs Lev 15:28 А когда она освободится от истечения своего, тогда должна отсчитать себе семь дней, и потом будет чиста;
\vs Lev 15:29 в восьмой день возьмет она себе двух горлиц или двух молодых голубей и принесет их к священнику ко входу скинии собрания;
\vs Lev 15:30 и принесет священник одну \bibemph{из птиц} в жертву за грех, а другую во всесожжение, и очистит ее священник пред Господом от истечения нечистоты ее.
\rsbpar\vs Lev 15:31 Так предохраняйте сынов Израилевых от нечистоты их, чтоб они не умерли в нечистоте своей, оскверняя жилище Мое, которое среди них:
\vs Lev 15:32 вот закон об имеющем истечение и о том, у кого случится излияние семени, делающее его нечистым,
\vs Lev 15:33 и о страдающей очищением своим, и о имеющих истечение, мужчине или женщине, и о муже, который переспит с нечистою.
\vs Lev 16:1 И говорил Господь Моисею по смерти двух сынов Аароновых, когда они, приступив [с чуждым огнем] пред лице Господне, умерли,
\vs Lev 16:2 и сказал Господь Моисею: скажи Аарону, брату твоему, чтоб он не во всякое время входил во святилище за завесу пред крышку [очистилище], что на ковчеге [откровения], дабы ему не умереть, ибо над крышкою Я буду являться в облаке.
\vs Lev 16:3 Вот с чем должен входить Аарон во святилище: с тельцом в жертву за грех и с овном во всесожжение;
\vs Lev 16:4 священный льняной хитон должен надевать он, нижнее платье льняное да будет на теле его, и льняным поясом пусть опоясывается, и льняной кидар надевает: это священные одежды; и пусть омывает он тело свое водою и надевает их;
\vs Lev 16:5 и от общества сынов Израилевых пусть возьмет [из стада коз] двух козлов в жертву за грех и одного овна во всесожжение.
\vs Lev 16:6 И принесет Аарон тельца в жертву за грех за себя и очистит себя и дом свой.
\vs Lev 16:7 И возьмет двух козлов и поставит их пред лицем Господним у входа скинии собрания;
\vs Lev 16:8 и бросит Аарон об обоих козлах жребии: один жребий для Господа, а другой жребий для отпущения;
\vs Lev 16:9 и приведет Аарон козла, на которого вышел жребий для Господа, и принесет его в жертву за грех,
\vs Lev 16:10 а козла, на которого вышел жребий для отпущения, поставит живого пред Господом, чтобы совершить над ним очищение и отослать его в пустыню для отпущения [и чтоб он понес на себе их беззакония в землю непроходимую].
\vs Lev 16:11 И приведет Аарон тельца в жертву за грех за себя, и очистит себя и дом свой, и заколет тельца в жертву за грех за себя;
\vs Lev 16:12 и возьмет горящих угольев полную кадильницу с жертвенника, который пред лицем Господним, и благовонного мелко истолченного курения полные горсти, и внесет за завесу;
\vs Lev 16:13 и положит курение на огонь пред лицем Господним, и облако курения покроет крышку, которая над \bibemph{ковчегом} откровения, дабы ему не умереть;
\vs Lev 16:14 и возьмет крови тельца и покропит перстом своим на крышку спереди и пред крышкою, семь раз покропит кровью с перста своего.
\vs Lev 16:15 И заколет козла в жертву за грех за народ, и внесет кровь его за завесу, и сделает с кровью его то же, что делал с кровью тельца, и покропит ею на крышку и пред крышкою,~---
\vs Lev 16:16 и очистит святилище от нечистот сынов Израилевых и от преступлений их, во всех грехах их. Так должен поступить он и со скиниею собрания, находящеюся у них, среди нечистот их.
\vs Lev 16:17 Ни один человек не должен быть в скинии собрания, когда входит он для очищения святилища, до самого выхода его. И так очистит он себя, дом свой и все общество Израилево.
\vs Lev 16:18 И выйдет он к жертвеннику, который пред лицем Господним, и очистит его, и возьмет крови тельца и крови козла, и возложит на роги жертвенника со всех сторон,
\vs Lev 16:19 и покропит на него кровью с перста своего семь раз, и очистит его, и освятит его от нечистот сынов Израилевых.
\vs Lev 16:20 И совершив очищение святилища, скинии собрания и жертвенника [и очистив священников], приведет он живого козла,
\vs Lev 16:21 и возложит Аарон обе руки свои на голову живого козла, и исповедает над ним все беззакония сынов Израилевых и все преступления их и все грехи их, и возложит их на голову козла, и отошлет с нарочным человеком в пустыню:
\vs Lev 16:22 и понесет козел на себе все беззакония их в землю непроходимую, и пустит он козла в пустыню.
\vs Lev 16:23 И войдет Аарон в скинию собрания, и снимет льняные одежды, которые надевал, входя во святилище, и оставит их там,
\vs Lev 16:24 и омоет тело свое водою на святом месте, и наденет одежды свои, и выйдет и совершит всесожжение за себя и всесожжение за народ, и очистит себя, [дом свой] и народ [и священников];
\vs Lev 16:25 а тук жертвы за грех воскурит на жертвеннике.
\vs Lev 16:26 И тот, кто отводил козла для отпущения, должен вымыть одежды свои, омыть тело свое водою, и потом может войти в стан.
\vs Lev 16:27 А тельца за грех и козла за грех, которых кровь внесена была для очищения святилища, пусть вынесут вон из стана и сожгут на огне кожи их и мясо их и нечистоту их;
\vs Lev 16:28 кто сожжет их, тот должен вымыть одежды свои и омыть тело свое водою, и после того может войти в стан.
\rsbpar\vs Lev 16:29 И да будет сие для вас вечным постановлением: в седьмой месяц, в десятый [день] месяца смиряйте души ваши и никакого дела не делайте, ни туземец, ни пришлец, поселившийся между вами,
\vs Lev 16:30 ибо в сей день очищают вас, чтобы сделать вас чистыми от всех грехов ваших, чтобы вы были чисты пред лицем Господним;
\vs Lev 16:31 это суббота покоя для вас, смиряйте души ваши: это постановление вечное.
\vs Lev 16:32 Очищать же должен священник, который помазан и который посвящен, чтобы священнодействовать ему вместо отца своего: и наденет он льняные одежды, одежды священные,
\vs Lev 16:33 и очистит Святое Святых и скинию собрания, и жертвенник очистит, и священников и весь народ общества очистит.
\vs Lev 16:34 И да будет сие для вас вечным постановлением: очищать сынов Израилевых от всех грехов их однажды в году. И сделал он так, как повелел Господь Моисею.
\vs Lev 17:1 И сказал Господь Моисею, говоря:
\vs Lev 17:2 объяви Аарону и сынам его и всем сынам Израилевым и скажи им: вот что повелевает Господь:
\vs Lev 17:3 если кто из дома Израилева [или из пришельцев, присоединившихся к вам] заколет тельца или овцу или козу в стане, или если кто заколет вне стана
\vs Lev 17:4 и не приведет ко входу скинии собрания, [чтобы принести во всесожжение или в жертву о спасении, угодную Господу, в приятное благоухание, и если кто заколет вне \bibemph{стана} и ко входу скинии собрания не принесет,] чтобы представить в жертву Господу пред жилищем Господним, то человеку тому вменена будет кровь: он пролил кровь, и истребится человек тот из народа своего;
\vs Lev 17:5 \bibemph{это} для того, чтобы приводили сыны Израилевы жертвы свои, которые они заколают на поле, чтобы приводили их пред Господа ко входу скинии собрания, к священнику, и заколали их Господу в жертвы мирные;
\vs Lev 17:6 и покропит священник кровью на жертвенник Господень у входа скинии собрания и воскурит тук в приятное благоухание Господу,
\vs Lev 17:7 чтоб они впредь не приносили жертв своих идолам, за которыми блудно ходят они. Сие да будет для них постановлением вечным в роды их.
\rsbpar\vs Lev 17:8 \bibemph{Еще} скажи им: если кто из дома Израилева и из пришельцев, которые живут между вами, приносит всесожжение или жертву
\vs Lev 17:9 и не приведет ко входу скинии собрания, чтобы совершить ее Господу, то истребится человек тот из народа своего.
\vs Lev 17:10 Если кто из дома Израилева и из пришельцев, которые живут между вами, будет есть какую-нибудь кровь, то обращу лице Мое на душу того, кто будет есть кровь, и истреблю ее из народа ее,
\vs Lev 17:11 потому что душа тела в крови, и Я назначил ее вам для жертвенника, чтобы очищать души ваши, ибо кровь сия душу очищает;
\vs Lev 17:12 потому Я и сказал сынам Израилевым: ни одна душа из вас не должна есть крови, и пришлец, живущий между вами, не должен есть крови.
\vs Lev 17:13 Если кто из сынов Израилевых и из пришельцев, живущих между вами, на ловле поймает зверя или птицу, которую можно есть, то он должен дать вытечь крови ее и покрыть ее землею,
\vs Lev 17:14 ибо душа всякого тела \bibemph{есть} кровь его, она душа его; потому Я сказал сынам Израилевым: не ешьте крови ни из какого тела, потому что душа всякого тела есть кровь его: всякий, кто будет есть ее, истребится.
\vs Lev 17:15 И всякий, кто будет есть мертвечину или растерзанное зверем, туземец или пришлец, должен вымыть одежды свои и омыться водою, и нечист будет до вечера, а \bibemph{потом} будет чист;
\vs Lev 17:16 если же не вымоет [одежд своих] и не омоет тела своего, то понесет на себе беззаконие свое.
\vs Lev 18:1 И сказал Господь Моисею, говоря:
\vs Lev 18:2 объяви сынам Израилевым и скажи им: Я Господь, Бог ваш.
\vs Lev 18:3 По делам земли Египетской, в которой вы жили, не поступайте, и по делам земли Ханаанской, в которую Я веду вас, не поступайте, и по установлениям их не ходите:
\vs Lev 18:4 Мои законы исполняйте и Мои постановления соблюдайте, поступая по ним. Я Господь, Бог ваш.
\vs Lev 18:5 Соблюдайте постановления Мои и законы Мои, которые исполняя, человек будет жив. Я Господь [Бог ваш].
\rsbpar\vs Lev 18:6 Никто ни к какой родственнице по плоти не должен приближаться с тем, чтобы открыть наготу. Я Господь.
\vs Lev 18:7 Наготы отца твоего и наготы матери твоей не открывай: она мать твоя, не открывай наготы ее.
\vs Lev 18:8 Наготы жены отца твоего не открывай: это нагота отца твоего.
\vs Lev 18:9 Наготы сестры твоей, дочери отца твоего или дочери матери твоей, родившейся в доме или вне дома, не открывай наготы их.
\vs Lev 18:10 Наготы дочери сына твоего или дочери дочери твоей, не открывай наготы их, ибо они твоя нагота.
\vs Lev 18:11 Наготы дочери жены отца твоего, родившейся от отца твоего, она сестра твоя [по отцу], не открывай наготы ее.
\vs Lev 18:12 Наготы сестры отца твоего не открывай, она единокровная отцу твоему.
\vs Lev 18:13 Наготы сестры матери твоей не открывай, ибо она единокровная матери твоей.
\vs Lev 18:14 Наготы брата отца твоего не открывай и к жене его не приближайся: она тетка твоя.
\vs Lev 18:15 Наготы невестки твоей не открывай: она жена сына твоего, не открывай наготы ее.
\vs Lev 18:16 Наготы жены брата твоего не открывай, это нагота брата твоего.
\vs Lev 18:17 Наготы жены и дочери ее не открывай; дочери сына ее и дочери дочери ее не бери, чтоб открыть наготу их, они единокровные ее; это беззаконие.
\vs Lev 18:18 Не бери жены вместе с сестрою ее, чтобы сделать ее соперницею, чтоб открыть наготу ее при ней, при жизни ее.
\vs Lev 18:19 И к жене во время очищения нечистот ее не приближайся, чтоб открыть наготу ее.
\vs Lev 18:20 И с женою ближнего твоего не ложись, чтобы излить семя и оскверниться с нею.
\vs Lev 18:21 Из детей твоих не отдавай на служение Молоху и не бесчести имени Бога твоего. Я Господь.
\vs Lev 18:22 Не ложись с мужчиною, как с женщиною: это мерзость.
\vs Lev 18:23 И ни с каким скотом не ложись, чтоб излить [семя] и оскверниться от него; и женщина не должна становиться пред скотом для совокупления с ним: это гнусно.
\rsbpar\vs Lev 18:24 Не оскверняйте себя ничем этим, ибо всем этим осквернили себя народы, которых Я прогоняю от вас:
\vs Lev 18:25 и осквернилась земля, и Я воззрел на беззаконие ее, и свергнула с себя земля живущих на ней.
\vs Lev 18:26 А вы соблюдайте постановления Мои и законы Мои и не делайте всех этих мерзостей, ни туземец, ни пришлец, живущий между вами,
\vs Lev 18:27 ибо все эти мерзости делали люди сей земли, что пред вами, и осквернилась земля;
\vs Lev 18:28 чтоб и вас не свергнула с себя земля, когда вы станете осквернять ее, как она свергнула народы, бывшие прежде вас;
\vs Lev 18:29 ибо если кто будет делать все эти мерзости, то души делающих это истреблены будут из народа своего.
\vs Lev 18:30 Итак соблюдайте повеления Мои, чтобы не поступать по гнусным обычаям, по которым поступали прежде вас, и чтобы не оскверняться ими. Я Господь, Бог ваш.
\vs Lev 19:1 И сказал Господь Моисею, говоря:
\vs Lev 19:2 объяви всему обществу сынов Израилевых и скажи им: святы будьте, ибо свят Я Господь, Бог ваш.
\vs Lev 19:3 Бойтесь каждый матери своей и отца своего и субботы Мои храните. Я Господь, Бог ваш.
\vs Lev 19:4 Не обращайтесь к идолам и богов литых не делайте себе. Я Господь, Бог ваш.
\vs Lev 19:5 Когда будете приносить Господу жертву мирную, то приносите ее, чтобы приобрести себе благоволение:
\vs Lev 19:6 в день жертвоприношения вашего и на другой день должно есть ее, а оставшееся к третьему дню должно сжечь на огне;
\vs Lev 19:7 если же кто станет есть ее на третий день, это гнусно, это не будет благоприятно;
\vs Lev 19:8 кто станет есть ее, тот понесет на себе грех, ибо он осквернил святыню Господню, и истребится душа та из народа своего.
\vs Lev 19:9 Когда будете жать жатву на земле вашей, не дожинай до края поля твоего, и оставшегося от жатвы твоей не подбирай,
\vs Lev 19:10 и виноградника твоего не обирай дочиста, и поп\acc{а}давших ягод в винограднике не подбирай; оставь это бедному и пришельцу. Я Господь, Бог ваш.
\vs Lev 19:11 Не крадите, не лгите и не обманывайте друг друга.
\vs Lev 19:12 Не клянитесь именем Моим во лжи, и не бесчести имени Бога твоего. Я Господь [Бог ваш].
\vs Lev 19:13 Не обижай ближнего твоего и не грабительствуй. Плата наемнику не должна оставаться у тебя до утра.
\vs Lev 19:14 Не злословь глухого и пред слепым не клади ничего, чтобы преткнуться ему; бойся [Господа] Бога твоего. Я Господь [Бог ваш].
\vs Lev 19:15 Не делайте неправды на суде; не будь лицеприятен к нищему и не угождай лицу великого; по правде суди ближнего твоего.
\vs Lev 19:16 Не ходи переносчиком в народе твоем и не восставай на жизнь ближнего твоего. Я Господь [Бог ваш].
\vs Lev 19:17 Не враждуй на брата твоего в сердце твоем; обличи ближнего твоего, и не понесешь за него греха.
\vs Lev 19:18 Не мсти и не имей злобы на сынов народа твоего, но люби ближнего твоего, как самого себя. Я Господь [Бог ваш].
\vs Lev 19:19 Уставы Мои соблюдайте; скота твоего не своди с иною породою; поля твоего не засевай двумя родами \bibemph{семян}; в одежду из разнородных нитей, из шерсти и льна, не одевайся.
\vs Lev 19:20 Если кто переспит с женщиною, а она раба, обрученная мужу, но еще не выкупленная, или свобода еще не дана ей, то должно наказать их, но не смертью, потому что она несвободная:
\vs Lev 19:21 пусть приведет он Господу ко входу скинии собрания жертву повинности, овна в жертву повинности своей;
\vs Lev 19:22 и очистит его священник овном повинности пред Господом от греха, которым он согрешил, и прощен будет ему грех, которым он согрешил.
\vs Lev 19:23 Когда придете в землю, [которую Господь Бог даст вам,] и посадите какое-либо плодовое дерево, то плоды его почитайте за необрезанные: три года должно почитать их за необрезанные, не должно есть их;
\vs Lev 19:24 а в четвертый год все плоды его должны быть посвящены для празднеств Господних;
\vs Lev 19:25 в пятый же год вы можете есть плоды его и собирать себе все произведения его. Я Господь, Бог ваш.
\vs Lev 19:26 Не ешьте с кровью; не ворожите и не гадайте.
\vs Lev 19:27 Не стригите головы вашей кругом, и не порти края бороды твоей.
\vs Lev 19:28 Ради умершего не делайте нарезов на теле вашем и не накалывайте на себе письмен. Я Господь [Бог ваш].
\vs Lev 19:29 Не оскверняй дочери твоей, допуская ее до блуда, чтобы не блудодействовала земля и не наполнилась земля развратом.
\vs Lev 19:30 Субботы Мои храните и святилище Мое чтите. Я Господь.
\vs Lev 19:31 Не обращайтесь к вызывающим мертвых, и к волшебникам не ходите, и не доводите себя до осквернения от них. Я Господь, Бог ваш.
\vs Lev 19:32 Пред лицем седого вставай и почитай лице старца, и бойся [Господа] Бога твоего. Я Господь [Бог ваш].
\vs Lev 19:33 Когда поселится пришлец в земле вашей, не притесняйте его:
\vs Lev 19:34 пришлец, поселившийся у вас, да будет для вас то же, что туземец ваш; люби его, как себя; ибо и вы были пришельцами в земле Египетской. Я Господь, Бог ваш.
\vs Lev 19:35 Не делайте неправды в суде, в мере, в весе и в измерении:
\vs Lev 19:36 да будут у вас весы верные, гири верные, ефа верная и гин верный. Я Господь, Бог ваш, Который вывел вас из земли Египетской.
\vs Lev 19:37 Соблюдайте все уставы Мои и все законы Мои и исполняйте их. Я Господь [Бог ваш].
\vs Lev 20:1 И сказал Господь Моисею, говоря:
\vs Lev 20:2 скажи сие сынам Израилевым: кто из сынов Израилевых и из пришельцев, живущих между Израильтянами, даст из детей своих Молоху, тот да будет предан смерти: народ земли да побьет его камнями;
\vs Lev 20:3 и Я обращу лице Мое на человека того и истреблю его из народа его за то, что он дал из детей своих Молоху, чтоб осквернить святилище Мое и обесчестить святое имя Мое;
\vs Lev 20:4 и если народ земли не обратит очей своих на человека того, когда он даст из детей своих Молоху, и не умертвит его,
\vs Lev 20:5 то Я обращу лице Мое на человека того и на род его и истреблю его из народа его, и всех блудящих по следам его, чтобы блудно ходить вслед Молоха.
\vs Lev 20:6 И если какая душа обратится к вызывающим мертвых и к волшебникам, чтобы блудно ходить вслед их, то Я обращу лице Мое на ту душу и истреблю ее из народа ее.
\vs Lev 20:7 Освящайте себя и будьте святы, ибо Я Господь, Бог ваш, [свят].
\vs Lev 20:8 Соблюдайте постановления Мои и исполняйте их, ибо Я Господь, освящающий вас.
\vs Lev 20:9 Кто будет злословить отца своего или мать свою, тот да будет предан смерти; отца своего и мать свою он злословил: кровь его на нем.
\vs Lev 20:10 Если кто будет прелюбодействовать с женой замужнею, если кто будет прелюбодействовать с женою ближнего своего,~--- да будут преданы смерти и прелюбодей и прелюбодейка.
\vs Lev 20:11 Кто ляжет с женою отца своего, тот открыл наготу отца своего: оба они да будут преданы смерти, кровь их на них.
\vs Lev 20:12 Если кто ляжет с невесткою своею, то оба они да будут преданы смерти: мерзость сделали они, кровь их на них.
\vs Lev 20:13 Если кто ляжет с мужчиною, как с женщиною, то оба они сделали мерзость: да будут преданы смерти, кровь их на них.
\vs Lev 20:14 Если кто возьмет себе жену и мать ее: это беззаконие; на огне должно сжечь его и их, чтобы не было беззакония между вами.
\vs Lev 20:15 Кто смесится со скотиною, того предать смерти, и скотину убейте.
\vs Lev 20:16 Если женщина пойдет к какой-нибудь скотине, чтобы совокупиться с нею, то убей женщину и скотину: да будут они преданы смерти, кровь их на них.
\vs Lev 20:17 Если кто возьмет сестру свою, дочь отца своего или дочь матери своей, и увидит наготу ее, и она увидит наготу его: это срам, да будут они истреблены пред глазами сынов народа своего; он открыл наготу сестры своей: грех свой понесет он.
\vs Lev 20:18 Если кто ляжет с женою во время болезни \bibemph{кровоочищения} и откроет наготу ее, то он обнажил истечения ее, и она открыла течение кровей своих: оба они да будут истреблены из народа своего.
\vs Lev 20:19 Наготы сестры матери твоей и сестры отца твоего не открывай, ибо таковой обнажает плоть свою: грех свой понесут они.
\vs Lev 20:20 Кто ляжет с теткою своею, тот открыл наготу дяди своего; грех свой понесут они, бездетными умрут.
\vs Lev 20:21 Если кто возьмет жену брата своего: это гнусно; он открыл наготу брата своего, бездетны будут они.
\rsbpar\vs Lev 20:22 Соблюдайте все уставы Мои и все законы Мои и исполняйте их,~--- и не свергнет вас с себя земля, в которую Я веду вас жить.
\vs Lev 20:23 Не поступайте по обычаям народа, который Я прогоняю от вас; ибо они всё это делали, и Я вознегодовал на них,
\vs Lev 20:24 и сказал Я вам: вы владейте землею их, и вам отдаю в наследие землю, в которой течет молоко и мед. Я Господь, Бог ваш, Который отделил вас от всех народов.
\vs Lev 20:25 Отличайте скот чистый от нечистого и птицу чистую от нечистой и не оскверняйте душ ваших скотом и птицею и всем пресмыкающимся по земле, что отличил Я, как нечистое.
\vs Lev 20:26 Будьте предо Мною святы, ибо Я свят Господь [Бог ваш], и Я отделил вас от народов, чтобы вы были Мои.
\vs Lev 20:27 Мужчина ли или женщина, если будут они вызывать мертвых или волхвовать, да будут преданы смерти: камнями должно побить их, кровь их на них.
\vs Lev 21:1 И сказал Господь Моисею: объяви священникам, сынам Аароновым, и скажи им: да не оскверняют себя \bibemph{прикосновением} к умершему из народа своего;
\vs Lev 21:2 только к ближнему родственнику своему, к матери своей и к отцу своему, к сыну своему и дочери своей, к брату своему
\vs Lev 21:3 и к сестре своей, девице, живущей при нем и не бывшей замужем, можно ему \bibemph{прикасаться}, не оскверняя себя;
\vs Lev 21:4 и \bibemph{прикосновением} к кому бы то ни было в народе своем не должен он осквернять себя, чтобы не сделаться нечистым.
\vs Lev 21:5 Они не должны брить головы своей и подстригать края бороды своей и делать нарезы на теле своем.
\vs Lev 21:6 Они должны быть святы Богу своему и не должны бесчестить имени Бога своего, ибо они приносят жертвы Господу, хлеб Богу своему, и потому должны быть святы.
\vs Lev 21:7 Они не должны брать за себя блудницу и опороченную, не должны брать и жену, отверженную мужем своим, ибо они святы [Господу] Богу своему.
\vs Lev 21:8 Святи его, ибо он приносит хлеб [Господу] Богу твоему: да будет он у тебя свят, ибо свят Я Господь, освящающий вас.
\vs Lev 21:9 Если дочь священника осквернит себя блудодеянием, то она бесчестит отца своего; огнем должно сжечь ее.
\rsbpar\vs Lev 21:10 Великий же священник из братьев своих, на голову которого возлит елей помазания, и который освящен, чтобы облачаться в \bibemph{священные} одежды, не должен обнажать головы своей и раздирать одежд своих;
\vs Lev 21:11 и ни к какому умершему не должен он приступать: даже \bibemph{прикосновением к умершему} отцу своему и матери своей он не должен осквернять себя.
\vs Lev 21:12 И от святилища он не должен отходить и бесчестить святилище Бога своего, ибо освящение елеем помазания Бога его на нем. Я Господь.
\vs Lev 21:13 В жену он должен брать девицу [из народа своего]:
\vs Lev 21:14 вдову, или отверженную, или опороченную, [или] блудницу, не должен он брать, но девицу из народа своего должен он брать в жену;
\vs Lev 21:15 он не должен порочить семени своего в народе своем, ибо Я Господь [Бог], освящающий его.
\rsbpar\vs Lev 21:16 И сказал Господь Моисею, говоря:
\vs Lev 21:17 скажи Аарону: никто из семени твоего во \bibemph{все} роды их, у которого \bibemph{на теле} будет недостаток, не должен приступать, чтобы приносить хлеб Богу своему;
\vs Lev 21:18 никто, у кого на теле есть недостаток, не должен приступать, ни слепой, ни хромой, ни уродливый,
\vs Lev 21:19 ни такой, у которого переломлена нога или переломлена рука,
\vs Lev 21:20 ни горбатый, ни с сухим членом, ни с бельмом на глазу, ни коростовый, ни паршивый, ни с поврежденными ятрами;
\vs Lev 21:21 ни один человек из семени Аарона священника, у которого \bibemph{на теле} есть недостаток, не должен приступать, чтобы приносить жертвы Господу; недостаток \bibemph{на нем}, поэтому не должен он приступать, чтобы приносить хлеб Богу своему;
\vs Lev 21:22 хлеб Бога своего из великих святынь и из святынь он может есть;
\vs Lev 21:23 но к завесе не должен он приходить и к жертвеннику не должен приступать, потому что недостаток на нем: не должен он бесчестить святилища Моего, ибо Я Господь, освящающий их.
\vs Lev 21:24 И объявил \bibemph{это} Моисей Аарону и сынам его и всем сынам Израилевым.
\vs Lev 22:1 И сказал Господь Моисею, говоря:
\vs Lev 22:2 скажи Аарону и сынам его, чтоб они осторожно поступали со святынями сынов Израилевых и не бесчестили святаго имени Моего в том, что они посвящают Мне. Я Господь.
\vs Lev 22:3 Скажи им: если кто из всего потомства вашего в роды ваши, имея на себе нечистоту, приступит к святыням, которые посвящают сыны Израилевы Господу, то истребится душа та от лица Моего. Я Господь [Бог ваш].
\vs Lev 22:4 Кто из семени Ааронова прокажен, или имеет истечение, тот не должен есть святынь, пока не очистится; и кто прикоснется к чему-нибудь нечистому от мертвого, или у кого случится излияние семени,
\vs Lev 22:5 или кто прикоснется к какому-нибудь гаду, от которого он сделается нечист, или к человеку, от которого он сделается нечист какою бы то ни было нечистотою,~---
\vs Lev 22:6 тот, прикоснувшийся к сему, нечист будет до вечера и не должен есть святынь, прежде нежели омоет тело свое водою;
\vs Lev 22:7 но когда зайдет солнце и он очистится, тогда может он есть святыни, ибо это его пища.
\vs Lev 22:8 Мертвечины и звероядины он не должен есть, чтобы не оскверниться этим. Я Господь.
\vs Lev 22:9 Да соблюдают они повеления Мои, чтобы не понести на себе греха и не умереть в нем, когда нарушат сие. Я Господь [Бог], освящающий их.
\vs Lev 22:10 Никто посторонний не должен есть святыни; поселившийся у священника и наемник не должен есть святыни;
\vs Lev 22:11 если же священник купит себе человека за серебро, то сей может есть оную; также и домочадцы его могут есть хлеб его.
\vs Lev 22:12 Если дочь священника выйдет в замужество за постороннего, то она не должна есть приносимых святынь;
\vs Lev 22:13 когда же дочь священника будет вдова, или разведенная, и детей нет у нее, и возвратится в дом отца своего, как \bibemph{была} в юности своей, тогда она может есть хлеб отца своего; а посторонний никто не должен есть его.
\vs Lev 22:14 Кто по ошибке съест \bibemph{что-нибудь} из святыни, тот должен отдать священнику святыню и приложить к ней пятую ее долю.
\vs Lev 22:15 \bibemph{Священники} сами не должны порочить святыни сынов Израилевых, которые они приносят Господу,
\vs Lev 22:16 и не должны навлекать на себя вину в преступлении, когда будут есть святыни свои, ибо Я Господь, освящающий их.
\rsbpar\vs Lev 22:17 И сказал Господь Моисею, говоря:
\vs Lev 22:18 объяви Аарону и сынам его и всем сынам Израилевым и скажи им: если кто из дома Израилева, или из пришельцев, [поселившихся] между Израильтянами, по обету ли какому, или по усердию приносит жертву свою, которую приносят Господу во всесожжение,
\vs Lev 22:19 то, чтобы сим приобрести благоволение \bibemph{от Бога, жертва должна быть} без порока, мужеского пола, из крупного скота, из овец и из коз;
\vs Lev 22:20 никакого \bibemph{животного}, на котором есть порок, не приносите [Господу], ибо это не приобретет вам благоволения.
\vs Lev 22:21 И если кто приносит мирную жертву Господу, исполняя обет, или по усердию, [или в праздники ваши,] из крупного скота или из мелкого, то \bibemph{жертва должна быть} без порока, чтоб быть угодною \bibemph{Богу}: никакого порока не должно быть на ней;
\vs Lev 22:22 \bibemph{животного} слепого, или поврежденного, или уродливого, или больного, или коростового, или паршивого, таких не приносите Господу и в жертву не давайте их на жертвенник Господень;
\vs Lev 22:23 тельца и агнца с членами, несоразмерно длинными или короткими, в жертву усердия принести можешь; а если по обету, то это не угодно будет \bibemph{Богу};
\vs Lev 22:24 \bibemph{животного}, у которого ятра раздавлены, разбиты, оторваны или вырезаны, не приносите Господу и в земле вашей не делайте \bibemph{сего};
\vs Lev 22:25 и из рук иноземцев не приносите всех таковых \bibemph{животных} в дар Богу вашему, потому что на них повреждение, порок на них: не приобретут они вам благоволения.
\rsbpar\vs Lev 22:26 И сказал Господь Моисею, говоря:
\vs Lev 22:27 когда родится теленок, или ягненок, или козленок, то семь дней он должен пробыть при матери своей, а от восьмого дня и далее будет благоугоден для приношения в жертву Господу;
\vs Lev 22:28 но ни коровы, ни овцы не заколайте в один день с порождением ее.
\vs Lev 22:29 Если прин\acc{о}сите Господу жертву благодарения, то принос\acc{и}те ее так, чтоб она приобрела вам благоволение;
\vs Lev 22:30 в тот же день должно съесть ее, не оставляйте от нее до утра. Я Господь.
\rsbpar\vs Lev 22:31 И соблюдайте заповеди Мои и исполняйте их. Я Господь.
\vs Lev 22:32 Не бесчестите святого имени Моего, чтоб Я был святим среди сынов Израилевых. Я Господь, освящающий вас,
\vs Lev 22:33 Который вывел вас из земли Египетской, чтоб быть вашим Богом. Я Господь.
\vs Lev 23:1 И сказал Господь Моисею, говоря:
\vs Lev 23:2 объяви сынам Израилевым и скажи им о праздниках Господних, в которые должно созывать священные собрания. Вот праздники Мои:
\vs Lev 23:3 шесть дней можно делать дела, а в седьмой день суббота покоя, священное собрание; никакого дела не делайте; это суббота Господня во всех жилищах ваших.
\rsbpar\vs Lev 23:4 Вот праздники Господни, священные собрания, которые вы должны созывать в свое время:
\vs Lev 23:5 в первый месяц, в четырнадцатый [день] месяца вечером Пасха Господня;
\vs Lev 23:6 и в пятнадцатый день того же месяца праздник опресноков Господу; семь дней ешьте опресноки;
\vs Lev 23:7 в первый день да будет у вас священное собрание; никакой работы не работайте;
\vs Lev 23:8 и в течение семи дней приносите жертвы Господу; в седьмой день также священное собрание; никакой работы не работайте.
\rsbpar\vs Lev 23:9 И сказал Господь Моисею, говоря:
\vs Lev 23:10 объяви сынам Израилевым и скажи им: когда придете в землю, которую Я даю вам, и будете жать на ней жатву, то принесите первый сноп жатвы вашей к священнику;
\vs Lev 23:11 он вознесет этот сноп пред Господом, чтобы вам приобрести благоволение; на другой день праздника вознесет его священник;
\vs Lev 23:12 и в день возношения снопа принесите во всесожжение Господу агнца однолетнего, без порока,
\vs Lev 23:13 и с ним хлебного приношения две десятых части \bibemph{ефы} пшеничной муки, смешанной с елеем, в жертву Господу, в приятное благоухание, и возлияния к нему четверть гина вина;
\vs Lev 23:14 никакого \bibemph{нового} хлеба, ни сушеных зерен, ни зерен сырых не ешьте до того дня, в который принесете приношения Богу вашему: это вечное постановление в роды ваши во всех жилищах ваших.
\vs Lev 23:15 Отсчитайте себе от первого дня после праздника, от того дня, в который приносите сноп потрясания, семь полных недель,
\vs Lev 23:16 до первого дня после седьмой недели отсчитайте пятьдесят дней, \bibemph{и тогда} принесите новое хлебное приношение Господу:
\vs Lev 23:17 от жилищ ваших приносите два хлеба возношения, которые должны состоять из двух десятых частей \bibemph{ефы} пшеничной муки и должны быть испечены кислые, \bibemph{как} первый плод Господу;
\vs Lev 23:18 вместе с хлебами представьте семь агнцев без порока, однолетних, и из крупного скота одного тельца и двух овнов [без порока]; да будет это во всесожжение Господу, и хлебное приношение и возлияние к ним, в жертву, в приятное благоухание Господу.
\vs Lev 23:19 Приготовьте также из \bibemph{стада} коз одного козла в жертву за грех и двух однолетних агнцев в жертву мирную [вместе с хлебом первого плода];
\vs Lev 23:20 священник должен принести это, потрясая пред Господом, вместе с потрясаемыми хлебами первого плода и с двумя агнцами, и это будет святынею Господу; священнику, [который приносит, это принадлежит];
\vs Lev 23:21 и созывайте \bibemph{народ} в сей день, священное собрание да будет у вас, никакой работы не работайте: это постановление вечное во всех жилищах ваших в роды ваши.
\vs Lev 23:22 Когда будете жать жатву на земле вашей, не дожинай до края поля твоего, когда жнешь, и оставшегося от жатвы твоей не подбирай; бедному и пришельцу оставь это. Я Господь, Бог ваш.
\rsbpar\vs Lev 23:23 И сказал Господь Моисею, говоря:
\vs Lev 23:24 скажи сынам Израилевым: в седьмой месяц, в первый [день] месяца да будет у вас покой, праздник труб, священное собрание [да будет у вас];
\vs Lev 23:25 никакой работы не работайте и приносите жертву Господу.
\rsbpar\vs Lev 23:26 И сказал Господь Моисею, говоря:
\vs Lev 23:27 также в девятый [день] седьмого месяца сего, день очищения, да будет у вас священное собрание; смиряйте души ваши и приносите жертву Господу;
\vs Lev 23:28 никакого дела не делайте в день сей, ибо это день очищения, дабы очистить вас пред лицем Господа, Бога вашего;
\vs Lev 23:29 а всякая душа, которая не смирит себя в этот день, истребится из народа своего;
\vs Lev 23:30 и если какая душа будет делать какое-нибудь дело в день сей, Я истреблю ту душу из народа ее;
\vs Lev 23:31 никакого дела не делайте: это постановление вечное в роды ваши, во всех жилищах ваших;
\vs Lev 23:32 это для вас суббота покоя, и смиряйте души ваши, с вечера девятого [дня] месяца; от вечера до вечера [десятого дня месяца] празднуйте субботу вашу.
\rsbpar\vs Lev 23:33 И сказал Господь Моисею, говоря:
\vs Lev 23:34 скажи сынам Израилевым: с пятнадцатого дня того же седьмого месяца праздник кущей, семь дней Господу;
\vs Lev 23:35 в первый день священное собрание, никакой работы не работайте;
\vs Lev 23:36 в \bibemph{течение} семи дней приносите жертву Господу; в восьмой день священное собрание да будет у вас, и приносите жертву Господу: это отдание праздника, никакой работы не работайте.
\vs Lev 23:37 Вот праздники Господни, в которые должно созывать священные собрания, чтобы приносить в жертву Господу всесожжение, хлебное приношение, заколаемые жертвы и возлияния, каждое в свой день,
\vs Lev 23:38 кроме суббот Господних и кроме даров ваших, и кроме всех обетов ваших и кроме всего \bibemph{приносимого} по усердию вашему, что вы даете Господу.
\vs Lev 23:39 А в пятнадцатый день седьмого месяца, когда вы собираете произведения земли, празднуйте праздник Господень семь дней: в первый день покой и в восьмой день покой;
\vs Lev 23:40 в первый день возьмите себе ветви красивых дерев, ветви пальмовые и ветви дерев широколиственных и верб речных, и веселитесь пред Господом Богом вашим семь дней;
\vs Lev 23:41 и празднуйте этот праздник Господень семь дней в году: это постановление вечное в роды ваши; в седьмой месяц празднуйте его;
\vs Lev 23:42 в кущах живите семь дней; всякий туземец Израильтянин должен жить в кущах,
\vs Lev 23:43 чтобы знали роды ваши, что в кущах поселил Я сынов Израилевых, когда вывел их из земли Египетской. Я Господь, Бог ваш.
\vs Lev 23:44 И объявил Моисей сынам Израилевым о праздниках Господних.
\vs Lev 24:1 И сказал Господь Моисею, говоря:
\vs Lev 24:2 прикажи сынам Израилевым, чтоб они принесли тебе елея чистого, выбитого, для освещения, чтобы непрестанно горел светильник;
\vs Lev 24:3 вне завесы \bibemph{ковчега} откровения в скинии собрания Аарон [и сыны его] должны ставить оный пред Господом от вечера до утра всегда: это вечное постановление в роды ваши;
\vs Lev 24:4 на подсвечнике чистом должны они ставить светильник пред Господом всегда.
\vs Lev 24:5 И возьми пшеничной муки и испеки из нее двенадцать хлебов; в каждом хлебе должны быть две десятых \bibemph{ефы};
\vs Lev 24:6 и положи их в два ряда, по шести в ряд, на чистом столе пред Господом;
\vs Lev 24:7 и положи на [каждый] ряд чистого ливана [и соли], и будет это при хлебе, в память, в жертву Господу;
\vs Lev 24:8 в каждый день субботы постоянно должно полагать их пред Господом от сынов Израилевых: это завет вечный;
\vs Lev 24:9 они будут принадлежать Аарону и сынам его, которые будут есть их на святом месте, ибо это великая святыня для них из жертв Господних: \bibemph{это} постановление вечное.
\rsbpar\vs Lev 24:10 И вышел сын одной Израильтянки, родившейся от Египтянина, к сынам Израилевым, и поссорился в стане сын Израильтянки с Израильтянином;
\vs Lev 24:11 хулил сын Израильтянки имя [Господне] и злословил. И привели его к Моисею [имя же матери его Саломиф, дочь Давриина, из племени Данова];
\vs Lev 24:12 и посадили его под стражу, доколе не будет объявлена им воля Господня.
\rsbpar\vs Lev 24:13 И сказал Господь Моисею, говоря:
\vs Lev 24:14 выведи злословившего вон из стана, и все слышавшие пусть положат руки свои на голову его, и все общество побьет его камнями;
\vs Lev 24:15 и сынам Израилевым скажи: кто будет злословить Бога своего, тот понесет грех свой;
\vs Lev 24:16 и хулитель имени Господня должен умереть, камнями побьет его все общество: пришлец ли, туземец ли станет хулить имя [Господне], предан будет смерти.
\vs Lev 24:17 Кто убьет какого-либо человека, тот предан будет смерти.
\vs Lev 24:18 Кто убьет скотину, должен заплатить за нее, скотину за скотину.
\vs Lev 24:19 Кто сделает повреждение на теле ближнего своего, тому должно сделать то же, что он сделал:
\vs Lev 24:20 перелом за перелом, око за око, зуб за зуб; как он сделал повреждение на \bibemph{теле} человека, так и ему должно сделать.
\vs Lev 24:21 Кто убьет скотину, должен заплатить за нее; а кто убьет человека, того должно предать смерти.
\vs Lev 24:22 Один суд должен быть у вас, как для пришельца, так и для туземца; ибо Я Господь, Бог ваш.
\vs Lev 24:23 И сказал Моисей сынам Израилевым; и вывели злословившего вон из стана, и побили его камнями, и сделали сыны Израилевы, как повелел Господь Моисею.
\vs Lev 25:1 И сказал Господь Моисею на горе Синае, говоря:
\vs Lev 25:2 объяви сынам Израилевым и скажи им: когда придете в землю, которую Я даю вам, тогда земля должна покоиться в субботу Господню;
\vs Lev 25:3 шесть лет засевай поле твое и шесть лет обрезывай виноградник твой, и собирай произведения их,
\vs Lev 25:4 а в седьмой год да будет суббота покоя земли, суббота Господня: поля твоего не засевай и виноградника твоего не обрезывай;
\vs Lev 25:5 что само вырастет на жатве твоей, не сжинай, и гроздов с необрезанных лоз твоих не снимай; да будет это год покоя земли;
\vs Lev 25:6 и будет это в продолжение субботы земли \bibemph{всем} вам в пищу, тебе и рабу твоему, и рабе твоей, и наемнику твоему, и поселенцу твоему, поселившемуся у тебя;
\vs Lev 25:7 и скоту твоему и зверям, которые на земле твоей, да будут все произведения ее в пищу.
\rsbpar\vs Lev 25:8 И насчитай себе семь субботних лет, семь раз по семи лет, чтоб было у тебя в семи субботних годах сорок девять лет;
\vs Lev 25:9 и воструби трубою в седьмой месяц, в десятый [день] месяца, в день очищения вострубите трубою по всей земле вашей;
\vs Lev 25:10 и освятите пятидесятый год и объявите свободу на земле всем жителям ее: да будет это у вас юбилей; и возвратитесь каждый во владение свое, и каждый возвратитесь в свое племя.
\vs Lev 25:11 Пятидесятый год да будет у вас юбилей: не сейте и не жните, что само вырастет на земле, и не снимайте ягод с необрезанных \bibemph{лоз} ее,
\vs Lev 25:12 ибо это юбилей: священным да будет он для вас; с поля ешьте произведения ее.
\vs Lev 25:13 В юбилейный год возвратитесь каждый во владение свое.
\vs Lev 25:14 Если будешь продавать что ближнему твоему, или будешь покупать что у ближнего твоего, не обижайте друг друга;
\vs Lev 25:15 по расчислению лет после юбилея ты должен покупать у ближнего твоего, и по расчислению лет дохода он должен продавать тебе;
\vs Lev 25:16 если много \bibemph{остается} лет, умножь цену; а если мало лет \bibemph{остается}, уменьши цену, ибо известное число \bibemph{лет} жатв он продает тебе.
\vs Lev 25:17 Не обижайте один другого; бойся Бога твоего, ибо Я Господь, Бог ваш.
\vs Lev 25:18 Исполняйте постановления Мои, и храните законы Мои и исполняйте их, и будете жить спокойно на земле;
\vs Lev 25:19 и будет земля давать плод свой, и будете есть досыта, и будете жить спокойно на ней.
\vs Lev 25:20 Если скажете: что же нам есть в седьмой год, когда мы не будем ни сеять, ни собирать произведений наших?
\vs Lev 25:21 Я пошлю благословение Мое на вас в шестой год, и он принесет произведений на три года;
\vs Lev 25:22 и будете сеять в восьмой год, но есть будете произведения старые до девятого года; доколе не поспеют произведения его, будете есть старое.
\rsbpar\vs Lev 25:23 Землю не должно продавать навсегда, ибо Моя земля: вы пришельцы и поселенцы у Меня;
\vs Lev 25:24 по всей земле владения вашего дозволяйте выкуп земли.
\vs Lev 25:25 Если брат твой обеднеет и продаст от владения своего, то придет близкий его родственник и выкупит проданное братом его;
\vs Lev 25:26 если же некому за него выкупить, но сам он будет иметь достаток и найдет, сколько нужно на выкуп,
\vs Lev 25:27 то пусть он расчислит годы продажи своей и возвратит остальное тому, кому он продал, и вступит опять во владение свое;
\vs Lev 25:28 если же не найдет рука его, сколько нужно возвратить ему, то проданное им останется в руках покупщика до юбилейного года, а в юбилейный год отойдет оно, и он опять вступит во владение свое.
\vs Lev 25:29 Если кто продаст жилой дом в городе, \bibemph{огражденном} стеною, то выкупить его можно до истечения года от продажи его: в течение года выкупить его можно;
\vs Lev 25:30 если же не будет он выкуплен до истечения целого года, то дом, который в городе, имеющем стену, останется навсегда у купившего его в роды его, и в юбилей не отойдет \bibemph{от него}.
\vs Lev 25:31 А домы в селениях, вокруг которых нет стены, должно считать наравне с полем земли: выкупать их [всегда] можно, и в юбилей они отходят.
\vs Lev 25:32 А города левитов, домы в городах владения их, левитам всегда можно выкупать;
\vs Lev 25:33 а кто из левитов не выкупит, то проданный дом в городе владения их в юбилей отойдет, потому что домы в городах левитских составляют их владение среди сынов Израилевых;
\vs Lev 25:34 и полей вокруг городов их продавать нельзя, потому что это вечное владение их.
\vs Lev 25:35 Если брат твой обеднеет и придет в упадок у тебя, то поддержи его, пришлец ли он, или поселенец, чтоб он жил с тобою;
\vs Lev 25:36 не бери от него роста и прибыли и бойся Бога твоего; [Я Господь,] чтоб жил брат твой с тобою;
\vs Lev 25:37 серебра твоего не отдавай ему в рост и хлеба твоего не отдавай ему для \bibemph{получения} прибыли.
\vs Lev 25:38 Я Господь, Бог ваш, Который вывел вас из земли Египетской, чтобы дать вам землю Ханаанскую, чтоб быть вашим Богом.
\vs Lev 25:39 Когда обеднеет у тебя брат твой и продан будет тебе, то не налагай на него работы рабской:
\vs Lev 25:40 он должен быть у тебя как наемник, как поселенец; до юбилейного года пусть работает у тебя,
\vs Lev 25:41 а \bibemph{тогда} пусть отойдет он от тебя, сам и дети его с ним, и возвратится в племя свое, и вступит опять во владение отцов своих,
\vs Lev 25:42 потому что они~--- Мои рабы, которых Я вывел из земли Египетской: не должно продавать их, как продают рабов;
\vs Lev 25:43 не господствуй над ним с жестокостью и бойся Бога твоего.
\vs Lev 25:44 А чтобы раб твой и рабыня твоя были у тебя, то покупайте себе раба и рабыню у народов, которые вокруг вас;
\vs Lev 25:45 также и из детей поселенцев, поселившихся у вас, можете покупать, и из племени их, которое у вас, которое у них родилось в земле вашей, и они могут быть вашей собственностью;
\vs Lev 25:46 можете передавать их в наследство и сынам вашим по себе, как имение; вечно владейте ими, как рабами. А над братьями вашими, сынами Израилевыми, друг над другом, не господствуйте с жестокостью.
\vs Lev 25:47 Если пришлец или поселенец твой будет иметь достаток, а брат твой пред ним обеднеет и продастся пришельцу, поселившемуся у тебя, или кому-нибудь из племени пришельца,
\vs Lev 25:48 то после продажи можно выкупить его; кто-нибудь из братьев его должен выкупить его,
\vs Lev 25:49 или дядя его, или сын дяди его должен выкупить его, или кто-нибудь из родства его, из племени его, должен выкупить его; или если будет иметь достаток, сам выкупится.
\vs Lev 25:50 И он должен рассчитаться с купившим его, \bibemph{начиная} от того года, когда он продал себя, до года юбилейного, и серебро, за которое он продал себя, должно отдать ему по числу лет; как временный наемник он должен быть у него;
\vs Lev 25:51 и если еще много \bibemph{остается} лет, то по мере их он должен отдать в выкуп за себя серебро, за которое он куплен;
\vs Lev 25:52 если же мало остается лет до юбилейного года, то он должен сосчитать и по мере лет отдать за себя выкуп.
\vs Lev 25:53 Он должен быть у него, как наемник, во все годы; он не должен господствовать над ним с жестокостью в глазах твоих.
\vs Lev 25:54 Если же он не выкупится таким образом, то в юбилейный год отойдет сам и дети его с ним,
\vs Lev 25:55 потому что сыны Израилевы Мои рабы; они Мои рабы, которых Я вывел из земли Египетской. Я Господь, Бог ваш.
\vs Lev 26:1 Не делайте себе кумиров и изваяний, и столбов не ставьте у себя, и камней с изображениями не кладите в земле вашей, чтобы кланяться пред ними, ибо Я Господь Бог ваш.
\vs Lev 26:2 Субботы Мои соблюдайте и святилище Мое чтите: Я Господь.
\vs Lev 26:3 Если вы будете поступать по уставам Моим и заповеди Мои будете хранить и исполнять их,
\vs Lev 26:4 то Я дам вам дожди в свое время, и земля даст произрастения свои, и дерева полевые дадут плод свой;
\vs Lev 26:5 и молотьба \bibemph{хлеба} будет достигать у вас собирания винограда, собирание винограда будет достигать посева, и будете есть хлеб свой досыта, и будете жить на земле [вашей] безопасно;
\vs Lev 26:6 пошлю мир на землю [вашу], ляжете, и никто вас не обеспокоит, сгоню лютых зверей с земли [вашей], и меч не пройдет по земле вашей;
\vs Lev 26:7 и будете прогонять врагов ваших, и падут они пред вами от меча;
\vs Lev 26:8 пятеро из вас прогонят сто, и сто из вас прогонят тьму, и падут враги ваши пред вами от меча;
\vs Lev 26:9 призрю на вас [и благословлю вас], и плодородными сделаю вас, и размножу вас, и буду тверд в завете Моем с вами;
\vs Lev 26:10 и будете есть старое прошлогоднее, и выбросите старое ради нового;
\vs Lev 26:11 и поставлю жилище Мое среди вас, и душа Моя не возгнушается вами;
\vs Lev 26:12 и буду ходить среди вас и буду вашим Богом, а вы будете Моим народом.
\vs Lev 26:13 Я Господь Бог ваш, Который вывел вас из земли Египетской, чтоб вы не были там рабами, и сокрушил узы ярма вашего, и повел вас с поднятою головою.
\vs Lev 26:14 Если же не послушаете Меня и не будете исполнять всех заповедей сих,
\vs Lev 26:15 и если презрите Мои постановления, и если душа ваша возгнушается Моими законами, так что вы не будете исполнять всех заповедей Моих, нарушив завет Мой,~---
\vs Lev 26:16 то и Я поступлю с вами так: пошлю на вас ужас, чахлость и горячку, от которых истомятся глаза и измучится душа, и будете сеять семена ваши напрасно, и враги ваши съедят их;
\vs Lev 26:17 обращу лице Мое на вас, и падете пред врагами вашими, и будут господствовать над вами неприятели ваши, и побежите, когда никто не гонится за вами.
\vs Lev 26:18 Если и при всем том не послушаете Меня, то Я всемеро увеличу наказание за грехи ваши,
\vs Lev 26:19 и сломлю гордое упорство ваше, и небо ваше сделаю, как железо, и землю вашу, как медь;
\vs Lev 26:20 и напрасно будет истощаться сила ваша, и земля ваша не даст произрастений своих, и дерева земли [вашей] не дадут плодов своих.
\vs Lev 26:21 Если же [после сего] пойдете против Меня и не захотите слушать Меня, то Я прибавлю вам ударов всемеро за грехи ваши:
\vs Lev 26:22 пошлю на вас зверей полевых, которые лишат вас детей, истребят скот ваш и вас уменьшат, так что опустеют дороги ваши.
\vs Lev 26:23 Если и после сего не исправитесь и пойдете против Меня,
\vs Lev 26:24 то и Я [в ярости] пойду против вас и поражу вас всемеро за грехи ваши,
\vs Lev 26:25 и наведу на вас мстительный меч в отмщение за завет; если же вы укроетесь в города ваши, то пошлю на вас язву, и преданы будете в руки врага;
\vs Lev 26:26 хлеб, подкрепляющий \bibemph{человека}, истреблю у вас; десять женщин будут печь хлеб ваш в одной печи и будут отдавать хлеб ваш весом; вы будете есть и не будете сыты.
\vs Lev 26:27 Если же и после сего не послушаете Меня и пойдете против Меня,
\vs Lev 26:28 то и Я в ярости пойду против вас и накажу вас всемеро за грехи ваши,
\vs Lev 26:29 и будете есть плоть сынов ваших, и плоть дочерей ваших будете есть;
\vs Lev 26:30 разорю высоты ваши и разрушу столбы ваши, и повергну трупы ваши на обломки идолов ваших, и возгнушается душа Моя вами;
\vs Lev 26:31 города ваши сделаю пустынею, и опустошу святилища ваши, и не буду обонять приятного благоухания [жертв] ваших;
\vs Lev 26:32 опустошу землю [вашу], так что изумятся о ней враги ваши, поселившиеся на ней;
\vs Lev 26:33 а вас рассею между народами и обнажу вслед вас меч, и будет земля ваша пуста и города ваши разрушены.
\vs Lev 26:34 Тогда удовлетворит себя земля за субботы свои во все дни запустения [своего]; когда вы будете в земле врагов ваших, тогда будет покоиться земля и удовлетворит себя за субботы свои;
\vs Lev 26:35 во все дни запустения [своего] будет она покоиться, сколько не покоилась в субботы ваши, когда вы жили на ней.
\vs Lev 26:36 Оставшимся из вас пошлю в сердца робость в земле врагов их, и шум колеблющегося листа погонит их, и побегут, как от меча, и падут, когда никто не преследует,
\vs Lev 26:37 и споткнутся друг на друга, как от меча, между тем как никто не преследует, и не будет у вас силы противостоять врагам вашим;
\vs Lev 26:38 и погибнете между народами, и пожрет вас земля врагов ваших;
\vs Lev 26:39 а оставшиеся из вас исчахнут за свои беззакония в землях врагов ваших и за беззакония отцов своих исчахнут;
\vs Lev 26:40 тогда признаются они в беззаконии своем и в беззаконии отцов своих, как они совершали преступления против Меня и шли против Меня,
\vs Lev 26:41 \bibemph{за что} и Я [в ярости] шел против них и ввел их в землю врагов их; тогда покорится необрезанное сердце их, и тогда потерпят они за беззакония свои.
\vs Lev 26:42 И Я вспомню завет Мой с Иаковом и завет Мой с Исааком, и завет Мой с Авраамом вспомню, и землю вспомню;
\vs Lev 26:43 тогда как земля оставлена будет ими и будет удовлетворять себя за субботы свои, опустев от них, и они будут терпеть за свое беззаконие, за то, что презирали законы Мои и душа их гнушалась постановлениями Моими,
\vs Lev 26:44 и тогда как они будут в земле врагов их,~--- Я не презрю их и не возгнушаюсь ими до того, чтоб истребить их, чтоб разрушить завет Мой с ними, ибо Я Господь, Бог их;
\vs Lev 26:45 вспомню для них завет с предками, которых вывел Я из земли Египетской пред глазами народов, чтоб быть их Богом. Я Господь.
\rsbpar\vs Lev 26:46 Вот постановления и определения и законы, которые постановил Господь между Собою и между сынами Израилевыми на горе Синае, чрез Моисея.
\vs Lev 27:1 И сказал Господь Моисею, говоря:
\vs Lev 27:2 объяви сынам Израилевым и скажи им: если кто дает обет посвятить душу Господу по оценке твоей,
\vs Lev 27:3 то оценка твоя мужчине от двадцати лет до шестидесяти должна быть пятьдесят сиклей серебряных, по сиклю священному;
\vs Lev 27:4 если же это женщина, то оценка твоя должна быть тридцать сиклей;
\vs Lev 27:5 от пяти лет до двадцати оценка твоя мужчине должна быть двадцать сиклей, а женщине десять сиклей;
\vs Lev 27:6 а от месяца до пяти лет оценка твоя мужчине должна быть пять сиклей серебра, а женщине оценка твоя три сикля серебра;
\vs Lev 27:7 от шестидесяти лет и выше мужчине оценка твоя должна быть пятнадцать сиклей серебра, а женщине десять сиклей.
\vs Lev 27:8 Если же он беден и не в силах \bibemph{отдать} по оценке твоей, то пусть представят его священнику, и священник пусть оценит его: соразмерно с состоянием давшего обет пусть оценит его священник.
\vs Lev 27:9 Если же то будет скот, который приносят в жертву Господу, то все, что дано Господу, должно быть свято:
\vs Lev 27:10 не должно выменивать его и заменять хорошее худым, или худое хорошим; если же станет кто заменять скотину скотиною, то и она и замен ее будет святынею.
\vs Lev 27:11 Если же то будет какая-нибудь скотина нечистая, которую не приносят в жертву Господу, то должно представить скотину священнику,
\vs Lev 27:12 и священник оценит ее, хороша ли она, или худа, и как оценит священник, так и должно быть;
\vs Lev 27:13 если же кто хочет выкупить ее, то пусть прибавит пятую долю к оценке твоей.
\vs Lev 27:14 Если кто посвящает дом свой в святыню Господу, то священник должен оценить его, хорош ли он, или худ, и как оценит его священник, так и состоится;
\vs Lev 27:15 если же посвятивший захочет выкупить дом свой, то пусть прибавит пятую часть серебра оценки твоей, и \bibemph{тогда} будет его.
\vs Lev 27:16 Если поле из своего владения посвятит кто Господу, то оценка твоя должна быть по мере посева: за посев хомера ячменя пятьдесят сиклей серебра;
\vs Lev 27:17 если от юбилейного года посвящает кто поле свое,~--- должно состояться по оценке твоей;
\vs Lev 27:18 если же после юбилея посвящает кто поле свое, то священник должен рассчитать серебро по мере лет, оставшихся до юбилейного года, и должно убавить из оценки твоей;
\vs Lev 27:19 если же захочет выкупить поле посвятивший его, то пусть он прибавит пятую часть серебра оценки твоей, и оно останется за ним;
\vs Lev 27:20 если же он не выкупит поля, и будет продано поле другому человеку, то уже нельзя выкупить:
\vs Lev 27:21 поле то, когда оно в юбилей отойдет, будет святынею Господу, как бы поле заклятое; священнику достанется оно во владение.
\vs Lev 27:22 А если кто посвятит Господу поле купленное, которое не из полей его владения,
\vs Lev 27:23 то священник должен рассчитать ему количество оценки до юбилейного года, и должен он отдать по расчету в тот же день, \bibemph{как} святыню Господню;
\vs Lev 27:24 поле же в юбилейный год перейдет опять к тому, у кого куплено, кому принадлежит владение той земли.
\vs Lev 27:25 Всякая оценка твоя должна быть по сиклю священному, двадцать гер должно быть в сикле.
\vs Lev 27:26 Только первенцев из скота, которые по первенству принадлежат Господу, не должен никто посвящать: вол ли то, или мелкий скот,~--- Господни они.
\vs Lev 27:27 Если же скот нечистый, то должно выкупить по оценке твоей и приложить к тому пятую часть; если не выкупят, то должно продать по оценке твоей.
\vs Lev 27:28 Только все заклятое, что под заклятием отдает человек Господу из своей собственности,~--- человека ли, скотину ли, поле ли своего владения,~--- не продается и не выкупается: все заклятое есть великая святыня Господня;
\vs Lev 27:29 все заклятое, что заклято от людей, не выкупается: оно должно быть предано смерти.
\vs Lev 27:30 И всякая десятина на земле из семян земли и из плодов дерева принадлежит Господу: это святыня Господня;
\vs Lev 27:31 если же кто захочет выкупить десятину свою, то пусть приложит к \bibemph{цене} ее пятую долю.
\vs Lev 27:32 И всякую десятину из крупного и мелкого скота, из всего, что проходит под жезлом десятое, должно посвящать Господу;
\vs Lev 27:33 не должно разбирать, хорошее ли то, или худое, и не должно заменять его; если же кто заменит его, то и само оно и замен его будет святынею и не может быть выкуплено.
\rsbpar\vs Lev 27:34 Вот заповеди, которые заповедал Господь Моисею для сынов Израилевых на горе Синае.

\bibbookdescr{Num}{
  inline={\LARGE Четвертая книга Моисеева\\\Huge Числа},
  toc={Числа},
  bookmark={Числа},
  header={Числа},
  %headerleft={},
  %headerright={},
  abbr={Чис}
}
\vs Num 1:1 И сказал Господь Моисею в пустыне Синайской, в скинии собрания, в первый [день] второго месяца, во второй год по выходе их из земли Египетской, говоря:
\vs Num 1:2 исчислите все общество сынов Израилевых по родам их, по семействам их, по числу имен, всех мужеского пола поголовно:
\vs Num 1:3 от двадцати лет и выше, всех годных для войны у Израиля, по ополчениям их исчислите их~--- ты и Аарон;
\vs Num 1:4 с вами должны быть из каждого колена по одному человеку, который в роде своем есть главный.
\vs Num 1:5 И вот имена мужей, которые будут с вами: от Рувима Елицур, сын Шедеура;
\vs Num 1:6 от Симеона Шелумиил, сын Цуришаддая;
\vs Num 1:7 от Иуды Наассон, сын Аминадава;
\vs Num 1:8 от Иссахара Нафанаил, сын Цуара;
\vs Num 1:9 от Завулона Елиав, сын Хелона;
\vs Num 1:10 от сынов Иосифа: от Ефрема Елишама, сын Аммиуда; от Манассии Гамалиил, сын Педацура;
\vs Num 1:11 от Вениамина Авидан, сын Гидеония;
\vs Num 1:12 от Дана Ахиезер, сын Аммишаддая;
\vs Num 1:13 от Асира Пагиил, сын Охрана;
\vs Num 1:14 от Гада Елиасаф, сын Регуила;
\vs Num 1:15 от Неффалима Ахира, сын Енана.
\vs Num 1:16 Это~--- избранные мужи общества, начальники колен отцов своих, главы тысяч Израилевых.
\rsbpar\vs Num 1:17 И взял Моисей и Аарон мужей сих, которые названы поименно,
\vs Num 1:18 и собрали они все общество в первый [день] второго месяца. И объявили они родословия свои, по родам их, по семействам их, по числу имен, от двадцати лет и выше, поголовно,
\vs Num 1:19 как повелел Господь Моисею. И сделал он счисление им в пустыне Синайской.
\vs Num 1:20 И было сынов Рувима, первенца Израилева, по родам их, по племенам их, по семействам их, по числу имен, поголовно, всех мужеского пола, от двадцати лет и выше, всех годных для войны,
\vs Num 1:21 исчислено в колене Рувимовом сорок шесть тысяч пятьсот.
\vs Num 1:22 Сынов Симеона по родам их, по племенам их, по семействам их, по числу имен, поголовно, всех мужеского пола, от двадцати лет и выше, всех годных для войны,
\vs Num 1:23 исчислено в колене Симеоновом пятьдесят девять тысяч триста.
\vs Num 1:24 Сынов Гада по родам их, по племенам их, по семействам их, по числу имен [их, поголовно, всех мужеского пола], от двадцати лет и выше, всех годных для войны,
\vs Num 1:25 исчислено в колене Гадовом сорок пять тысяч шестьсот пятьдесят.
\vs Num 1:26 Сынов Иуды по родам их, по племенам их, по семействам их, по числу имен [их, поголовно, всех мужеского пола], от двадцати лет и выше, всех годных для войны,
\vs Num 1:27 исчислено в колене Иудином семьдесят четыре тысячи шестьсот.
\vs Num 1:28 Сынов Иссахара по родам их, по племенам их, по семействам их, по числу имен [их, поголовно, всех мужеского пола], от двадцати лет и выше, всех годных для войны,
\vs Num 1:29 исчислено в колене Иссахаровом пятьдесят четыре тысячи четыреста.
\vs Num 1:30 Сынов Завулона по родам их, по племенам их, по семействам их, по числу имен [их, поголовно, всех мужеского пола], от двадцати лет и выше, всех годных для войны,
\vs Num 1:31 исчислено в колене Завулоновом пятьдесят семь тысяч четыреста.
\vs Num 1:32 Сынов Иосифа, сынов Ефрема по родам их, по племенам их, по семействам их, по числу имен [их, поголовно, всех мужеского пола], от двадцати лет и выше, всех годных для войны,
\vs Num 1:33 исчислено в колене Ефремовом сорок тысяч пятьсот.
\vs Num 1:34 Сынов Манассии по родам их, по племенам их, по семействам их, по числу имен [их, поголовно, всех мужеского пола], от двадцати лет и выше, всех годных для войны,
\vs Num 1:35 исчислено в колене Манассиином тридцать две тысячи двести.
\vs Num 1:36 Сынов Вениамина по родам их, по племенам их, по семействам их, по числу имен [их, поголовно, всех мужеского пола], от двадцати лет и выше, всех годных для войны,
\vs Num 1:37 исчислено в колене Вениаминовом тридцать пять тысяч четыреста.
\vs Num 1:38 Сынов Дана по родам их, по племенам их, по семействам их, по числу имен [их, поголовно, всех мужеского пола], от двадцати лет и выше, всех годных для войны,
\vs Num 1:39 исчислено в колене Дановом шестьдесят две тысячи семьсот.
\vs Num 1:40 Сынов Асира по родам их, по племенам их, по семействам их, по числу имен [их, поголовно, всех мужеского пола], от двадцати лет и выше, всех годных для войны,
\vs Num 1:41 исчислено в колене Асировом сорок одна тысяча пятьсот.
\vs Num 1:42 Сынов Неффалима по родам их, по племенам их, по семействам их, по числу имен [их, поголовно, всех мужеского пола], от двадцати лет и выше, всех годных для войны,
\vs Num 1:43 исчислено в колене Неффалимовом пятьдесят три тысячи четыреста.
\vs Num 1:44 Вот вошедшие в исчисление, которых исчислил Моисей и Аарон и начальники Израиля~--- двенадцать человек, по одному человеку из каждого племени.
\vs Num 1:45 И было всех, вошедших в исчисление, сынов Израилевых, по семействам их, от двадцати лет и выше, всех годных для войны у Израиля,
\vs Num 1:46 и было всех вошедших в исчисление шестьсот три тысячи пятьсот пятьдесят.
\vs Num 1:47 А левиты по поколениям отцов их не были исчислены между ними.
\rsbpar\vs Num 1:48 И сказал Господь Моисею, говоря:
\vs Num 1:49 только колена Левиина не вноси в перепись, и не исчисляй их вместе с сынами Израиля;
\vs Num 1:50 но поручи левитам скинию откровения и все принадлежности ее и всё, что при ней; пусть они носят скинию и все принадлежности ее, и служат при ней, и около скинии пусть ставят стан свой;
\vs Num 1:51 и когда надобно переносить скинию, пусть поднимают ее левиты, и когда надобно остановиться скинии, пусть ставят ее левиты; а если приступит кто посторонний, предан будет смерти.
\vs Num 1:52 Сыны Израилевы должны становиться каждый в стане своем и каждый при своем знамени, по ополчениям своим;
\vs Num 1:53 а левиты должны ставить стан около скинии откровения, чтобы не было гнева на общество сынов Израилевых, и будут левиты стоять на страже у скинии откровения.
\vs Num 1:54 И сделали сыны Израилевы; как повелел Господь Моисею, так они и сделали.
\vs Num 2:1 И сказал Господь Моисею и Аарону, говоря:
\vs Num 2:2 сыны Израилевы должны каждый ставить стан свой при знамени своем, при знаках семейств своих; пред скиниею собрания вокруг должны ставить стан свой.
\vs Num 2:3 С передней стороны к востоку ставят стан: знамя стана Иудина по ополчениям их, и начальник сынов Иуды Наассон, сын Аминадава,
\vs Num 2:4 и воинства его, вошедших в исчисление его, семьдесят четыре тысячи шестьсот;
\vs Num 2:5 после него ставит стан колено Иссахарово, и начальник сынов Иссахара Нафанаил, сын Цуара,
\vs Num 2:6 и воинства его, вошедших в исчисление его, пятьдесят четыре тысячи четыреста;
\vs Num 2:7 [далее ставит стан] колено Завулона, и начальник сынов Завулона Елиав, сын Хелона,
\vs Num 2:8 и воинства его, вошедших в исчисление его, пятьдесят семь тысяч четыреста;
\vs Num 2:9 всех вошедших в исчисление к стану Иуды сто восемьдесят шесть тысяч четыреста, по ополчениям их; первыми они должны отправляться.
\vs Num 2:10 Знамя стана Рувимова к югу, по ополчениям их, и начальник сынов Рувимовых Елицур, сын Шедеура,
\vs Num 2:11 и воинства его, вошедших в исчисление его, сорок шесть тысяч пятьсот;
\vs Num 2:12 подле него ставит стан колено Симеоново, и начальник сынов Симеона Шелумиил, сын Цуришаддая,
\vs Num 2:13 и воинства его, вошедших в исчисление его, пятьдесят девять тысяч триста;
\vs Num 2:14 потом колено Гада, и начальник сынов Гада Елиасаф, сын Регуила,
\vs Num 2:15 и воинства его, вошедших в исчисление его, сорок пять тысяч шестьсот пятьдесят;
\vs Num 2:16 всех вошедших в исчисление к стану Рувима сто пятьдесят одна тысяча четыреста пятьдесят, по ополчениям их; вторыми они должны отправляться.
\vs Num 2:17 Когда пойдет скиния собрания, стан левитов будет в середине станов. Как стоят, так и должны идти, каждый на своем месте при знаменах своих.
\vs Num 2:18 Знамя стана Ефремова по ополчениям их к западу, и начальник сынов Ефрема Елишама, сын Аммиуда,
\vs Num 2:19 и воинства его, вошедших в исчисление его, сорок тысяч пятьсот;
\vs Num 2:20 подле него колено Манассиино, и начальник сынов Манассии Гамалиил, сын Педацура,
\vs Num 2:21 и воинства его, вошедших в исчисление его, тридцать две тысячи двести;
\vs Num 2:22 потом колено Вениамина, и начальник сынов Вениамина Авидан, сын Гидеония,
\vs Num 2:23 и воинства его, вошедших в исчисление его, тридцать пять тысяч четыреста;
\vs Num 2:24 всех вошедших в исчисление к стану Ефрема сто восемь тысяч сто, по ополчениям их; третьими они должны отправляться.
\vs Num 2:25 Знамя стана Данова к северу, по ополчениям их, и начальник сынов Дана Ахиезер, сын Аммишаддая,
\vs Num 2:26 и воинства его, вошедших в исчисление его, шестьдесят две тысячи семьсот;
\vs Num 2:27 подле него ставит стан колено Асирово, и начальник сынов Асира Пагиил, сын Охрана,
\vs Num 2:28 и воинства его, вошедших в исчисление его, сорок одна тысяча пятьсот;
\vs Num 2:29 далее [ставит стан] колено Неффалима, и начальник сынов Неффалима Ахира, сын Енана,
\vs Num 2:30 и воинства его, вошедших в исчисление его, пятьдесят три тысячи четыреста;
\vs Num 2:31 всех вошедших в исчисление к стану Дана сто пятьдесят семь тысяч шестьсот; они должны идти последними при знаменах своих.
\vs Num 2:32 Вот вошедшие в исчисление сыны Израиля по семействам их. Всех вошедших в исчисление в станах, по ополчениям их, шестьсот три тысячи пятьсот пятьдесят.
\vs Num 2:33 А левиты не вошли в исчисление вместе с сынами Израиля, как повелел Господь Моисею.
\vs Num 2:34 И сделали сыны Израилевы всё, что повелел Господь Моисею: так становились станами при знаменах своих и так шли каждый по племенам своим, по семействам своим.
\vs Num 3:1 Вот родословие Аарона и Моисея, когда говорил Господь Моисею на горе Синае,
\vs Num 3:2 и вот имена сынов Аарона: первенец Надав, Авиуд, Елеазар и Ифамар;
\vs Num 3:3 это имена сынов Аарона, священников, помазанных, которых он посвятил, чтобы священнодействовать;
\vs Num 3:4 но Надав и Авиуд умерли пред лицем Господа, когда они принесли огонь чуждый пред лице Господа в пустыне Синайской, детей же у них не было; и остались священниками Елеазар и Ифамар при Аароне, отце своем.
\rsbpar\vs Num 3:5 И сказал Господь Моисею, говоря:
\vs Num 3:6 приведи колено Левиино, и поставь его пред Аароном священником, чтоб они служили ему;
\vs Num 3:7 и пусть они будут на страже за него и на страже за все общество при скинии собрания, чтобы отправлять службы при скинии;
\vs Num 3:8 и пусть хранят все вещи скинии собрания, и будут на страже за сынов Израилевых, чтобы отправлять службы при скинии;
\vs Num 3:9 отдай левитов Аарону [брату твоему] и сынам его [священникам] \bibemph{в распоряжение}: да будут они отданы ему из сынов Израилевых;
\vs Num 3:10 Аарону же и сынам его поручи [скинию откровения], чтобы они наблюдали священническую должность свою [и все, что при жертвеннике и за завесою]; а если приступит кто посторонний, предан будет смерти.
\rsbpar\vs Num 3:11 И сказал Господь Моисею, говоря:
\vs Num 3:12 вот, Я взял левитов из сынов Израилевых вместо всех первенцев, разверзающих ложесна из сынов Израилевых [они будут взамен их]; левиты должны быть Мои,
\vs Num 3:13 ибо все первенцы~--- Мои; в тот день, когда поразил Я всех первенцев в земле Египетской, освятил Я Себе всех первенцев Израилевых от человека до скота; они должны быть Мои. Я Господь.
\rsbpar\vs Num 3:14 И сказал Господь Моисею в пустыне Синайской, говоря:
\vs Num 3:15 исчисли сынов Левииных по семействам их, по родам их; всех мужеского пола от одного месяца и выше исчисли.
\vs Num 3:16 И исчислил их Моисей [и Аарон] по слову Господню, как повелено.
\vs Num 3:17 И вот сыны Левиины по именам их: Гирсон, Кааф и Мерари.
\vs Num 3:18 И вот имена сынов Гирсоновых по родам их: Ливни и Шимей.
\vs Num 3:19 И сыны Каафа по родам их: Амрам и Ицгар, Хеврон и Узиил.
\vs Num 3:20 И сыны Мерари по родам их: Махли и Муши. Вот роды Левиины по семействам их.
\vs Num 3:21 От Гирсона род Ливни и род Шимея: это роды Гирсоновы.
\vs Num 3:22 Исчисленных было всех мужеского пола, от одного месяца и выше, семь тысяч пятьсот.
\vs Num 3:23 Роды Гирсоновы должны становиться станом позади скинии на запад;
\vs Num 3:24 начальник поколения сынов Гирсоновых Елиасаф, сын Лаелов;
\vs Num 3:25 хранению сынов Гирсоновых в скинии собрания поручается скиния и покров ее, и завеса входа скинии собрания,
\vs Num 3:26 и завесы двора и завеса входа двора, который вокруг скинии и жертвенника, и веревки ее, со всеми их принадлежностями.
\vs Num 3:27 От Каафа род Амрама и род Ицгара, и род Хеврона, и род Узиила: это роды Каафа.
\vs Num 3:28 По счету всех мужеского пола, от одного месяца и выше, восемь тысяч шестьсот, которые охраняли святилище.
\vs Num 3:29 Роды сынов Каафовых должны ставить стан свой на южной стороне скинии;
\vs Num 3:30 начальник же поколения родов Каафовых Елцафан, сын Узиила;
\vs Num 3:31 в хранении у них ковчег, стол, светильник, жертвенники, священные сосуды, которые употребляются при служении, и завеса со всеми принадлежностями ее.
\vs Num 3:32 Начальник над начальниками левитов Елеазар, сын Аарона священника; под его надзором те, которым вверено хранение святилища.
\vs Num 3:33 От Мерари род Махли и род Муши: это роды Мерари;
\vs Num 3:34 исчисленных по числу всех мужеского пола, от одного месяца и выше~--- шесть тысяч двести;
\vs Num 3:35 начальник поколения родов Мерари Цуриил, сын Авихаила; они должны ставить стан свой на северной стороне скинии;
\vs Num 3:36 хранению сынов Мерари поручаются брусья скинии и шесты ее, и столбы ее, и подножия ее и все вещи ее, со всем устройством их,
\vs Num 3:37 и столбы двора со всех сторон и подножия их и колья их и веревки их.
\vs Num 3:38 А с передней стороны скинии, к востоку пред скиниею собрания, должны ставить стан Моисей и Аарон и сыны его, которым вверено хранение святилища за сынов Израилевых; а если приступит кто посторонний, предан будет смерти.
\vs Num 3:39 Всех исчисленных левитов, которых исчислил Моисей и Аарон по повелению Господню, по родам их, всех мужеского пола, от одного месяца и выше, двадцать две тысячи.
\rsbpar\vs Num 3:40 И сказал Господь Моисею: исчисли всех первенцев мужеского пола из сынов Израилевых, от одного месяца и выше, и пересчитай их поименно;
\vs Num 3:41 и возьми левитов для Меня,~--- Я Господь,~--- вместо всех первенцев из сынов Израиля, а скот левитов вместо всего первородного скота сынов Израилевых.
\vs Num 3:42 И исчислил Моисей, как повелел ему Господь, всех первенцев из сынов Израилевых
\vs Num 3:43 и было всех первенцев мужеского пола, по числу имен, от одного месяца и выше, двадцать две тысячи двести семьдесят три.
\rsbpar\vs Num 3:44 И сказал Господь Моисею, говоря:
\vs Num 3:45 возьми левитов вместо всех первенцев из сынов Израиля и скот левитов вместо скота их; пусть левиты будут Мои. Я Господь.
\vs Num 3:46 А в выкуп двухсот семидесяти трех, которые лишние против \bibemph{числа} левитов, из первенцев Израильских,
\vs Num 3:47 возьми по пяти сиклей за человека, по сиклю священному возьми, двадцать гер в сикле,
\vs Num 3:48 и отдай серебро сие Аарону и сынам его, в выкуп за излишних против \bibemph{числа} их.
\vs Num 3:49 И взял Моисей серебро выкупа за излишних против \bibemph{числа} замененных левитами,
\vs Num 3:50 от первенцев Израилевых взял серебра тысячу триста шестьдесят пять [сиклей], по сиклю священному,
\vs Num 3:51 и отдал Моисей серебро выкупа [за излишних] Аарону и сынам его по слову Господню, как повелел Господь Моисею.
\vs Num 4:1 И сказал Господь Моисею и Аарону, говоря:
\vs Num 4:2 исчисли сынов Каафовых из сынов Левия по родам их, по семействам их,
\vs Num 4:3 от тридцати лет и выше до пятидесяти лет, всех способных к службе, чтобы отправлять работы в скинии собрания.
\vs Num 4:4 Вот служение сынов Каафовых [левитов по родам их, по семействам их,] в скинии собрания: \bibemph{носить} Святое Святых.
\vs Num 4:5 Когда стану надобно подняться в путь, Аарон и сыны его войдут, и снимут завесу закрывающую, и покроют ею ковчег откровения;
\vs Num 4:6 и положат на нее покров из кож синего цвета, и сверх его накинут покрывало всё из голубой \bibemph{шерсти}, и вложат шесты его;
\vs Num 4:7 и стол \bibemph{хлебов} предложения накроют одеждою из голубой \bibemph{шерсти}, и поставят на нем блюда, тарелки, чаши и кружки для возлияния, и хлеб \bibemph{его} всегдашний должен быть на нем;
\vs Num 4:8 и возложат на них одежду багряную, и покроют ее покровом из кожи синего цвета, и вложат шесты его;
\vs Num 4:9 и возьмут одежду из голубой \bibemph{шерсти}, и покроют светильник и лампады его, и щипцы его, и лотки его, и все сосуды для елея, которые употребляют при нем,
\vs Num 4:10 и покроют его и все принадлежности его покровом из кож синих, и положат на носилки;
\vs Num 4:11 и на золотой жертвенник возложат одежду из голубой \bibemph{шерсти}, и покроют его покровом из кож синих, и вложат шесты его.
\vs Num 4:12 И возьмут все вещи служебные, которые употребляются для служения во святилище, и положат в одежду из голубой \bibemph{шерсти}, и покроют их покровом из кож синих, и положат на носилки.
\vs Num 4:13 И очистят жертвенник от пепла и накроют его одеждою пурпуровою;
\vs Num 4:14 и положат на него все сосуды его, которые употребляются для служения при нем~--- \acc{у}гольницы, вилки, лопатки и чаши, все сосуды жертвенника~--- и покроют его покровом из кож синих, и вложат шесты его. [И возьмут пурпуровую одежду, и покроют умывальник и подножия его, и положат на них кожаное синее покрывало, и поставят на носилки.]
\vs Num 4:15 Когда, при отправлении в путь стана, Аарон и сыны его покроют всё святилище и все вещи святилища, тогда сыны Каафа подойдут, чтобы нести; но не должны они касаться святилища, чтобы не умереть. Сии \bibemph{части} скинии собрания должны носить сыны Каафовы.
\vs Num 4:16 Елеазару, сыну Аарона священника, поручается елей для светильника и благовонное курение, и всегдашнее хлебное приношение и елей помазания,~--- поручается вся скиния и все, что в ней, святилище и принадлежности его.
\rsbpar\vs Num 4:17 И сказал Господь Моисею и Аарону, говоря:
\vs Num 4:18 не погубите колена племен Каафовых из среды левитов,
\vs Num 4:19 но вот что сделайте им, чтобы они были живы и не умерли, когда приступают к Святому Святых: Аарон и сыны его пусть придут и поставят их каждого в служении его и у ноши его;
\vs Num 4:20 но сами они не должны подходить смотреть святыню, когда покрывают ее, чтобы не умереть.
\rsbpar\vs Num 4:21 И сказал Господь Моисею, говоря:
\vs Num 4:22 исчисли и сынов Гирсона по семействам их, по родам их,
\vs Num 4:23 от тридцати лет и выше до пятидесяти лет, исчисли их всех способных к службе, чтобы отправлять работы при скинии собрания.
\vs Num 4:24 Вот работы семейств Гирсоновых, при их служении и ношении тяжестей:
\vs Num 4:25 они должны носить покровы скинии и скинию собрания, и покров ее, и покров кожаный синий, который поверх его, и завесу входа скинии собрания,
\vs Num 4:26 и завесы двора, и завесу входа во двор, который вокруг скинии и жертвенника, и веревки их, и все вещи, принадлежащие к ним; и все, что делается при них, они должны работать;
\vs Num 4:27 по повелению Аарона и сынов его должны производиться все службы сынов Гирсоновых при всяком ношении тяжестей и всякой работе их, и поручите их хранению все, что они носят;
\vs Num 4:28 вот службы родов сынов Гирсоновых в скинии собрания, и вот что поручается их хранению под надзором Ифамара, сына Аарона, священника.
\vs Num 4:29 Сынов Мерариных по родам их, по семействам их исчисли,
\vs Num 4:30 от тридцати лет и выше до пятидесяти лет, исчисли всех способных на службу, чтобы отправлять работы при скинии собрания.
\vs Num 4:31 Вот что они должны носить, по службе их при скинии собрания: брусья скинии и шесты ее, и столбы ее и подножия ее,
\vs Num 4:32 и столбы двора со всех сторон и подножия их, и колья их, и веревки их, и все вещи при них и все принадлежности их; и поименно сосчитайте вещи, которые они обязаны носить;
\vs Num 4:33 вот работы родов сынов Мерариных, по службе их при скинии собрания, под надзором Ифамара, сына Аарона, священника.
\vs Num 4:34 И исчислили Моисей и Аарон и начальники общества сынов Каафовых по родам их и по семействам их,
\vs Num 4:35 от тридцати лет и выше до пятидесяти лет, всех способных к службе, для работ в скинии собрания;
\vs Num 4:36 и было исчислено, по родам их, две тысячи семьсот пятьдесят:
\vs Num 4:37 это~--- исчисленные из родов Каафовых, все служащие при скинии собрания, которых исчислил Моисей и Аарон по повелению Господню, \bibemph{данному} чрез Моисея.
\vs Num 4:38 И исчислены сыны Гирсона по родам их и по семействам их,
\vs Num 4:39 от тридцати лет и выше до пятидесяти лет, все способные к службе, для работ в скинии собрания;
\vs Num 4:40 и было исчислено по родам их, по семействам их, две тысячи шестьсот тридцать:
\vs Num 4:41 это~--- исчисленные из родов сынов Гирсона, все служащие при скинии собрания, которых исчислил Моисей и Аарон, по повелению Господню.
\vs Num 4:42 И исчислены роды сынов Мерариных по родам их, по семействам их,
\vs Num 4:43 от тридцати лет и выше до пятидесяти лет, все способные к службе, для работ при скинии собрания;
\vs Num 4:44 и было исчислено по родам их, [по семействам их,] три тысячи двести:
\vs Num 4:45 это~--- исчисленные из родов сынов Мерариных, которых исчислил Моисей и Аарон по повелению Господню, \bibemph{данному} чрез Моисея.
\vs Num 4:46 И исчислены все левиты, которых исчислил Моисей и Аарон и начальники Израиля по родам их и по семействам их,
\vs Num 4:47 от тридцати лет и выше до пятидесяти лет, все способные \bibemph{к службе} для работ и ношения в скинии собрания;
\vs Num 4:48 и было исчислено их восемь тысяч пятьсот восемьдесят;
\vs Num 4:49 по повелению Господню чрез Моисея определены они каждый к своей работе и ношению, и исчислены, как повелел Господь Моисею.
\vs Num 5:1 И сказал Господь Моисею, говоря:
\vs Num 5:2 повели сынам Израилевым выслать из стана всех прокаженных, и всех имеющих истечение, и всех осквернившихся от мертвого,
\vs Num 5:3 и мужчин и женщин вышлите, за стан вышлите их, чтобы не оскверняли они станов своих, среди которых Я живу.
\vs Num 5:4 И сделали так сыны Израилевы, и выслали их вон из стана; как говорил Господь Моисею, так и сделали сыны Израилевы.
\rsbpar\vs Num 5:5 И сказал Господь Моисею, говоря:
\vs Num 5:6 скажи сынам Израилевым: если мужчина или женщина сделает какой-либо грех против человека, и чрез это сделает преступление против Господа, и виновна будет душа та,
\vs Num 5:7 то пусть исповедаются во грехе своем, который они сделали, и возвратят сполна то, в чем виновны, и прибавят к тому пятую часть и отдадут тому, против кого согрешили;
\vs Num 5:8 если же у него нет наследника, которому следовало бы возвратить за вину: то посвятить это Господу; пусть будет это священнику, сверх овна очищения, которым он очистит его;
\vs Num 5:9 и всякое возношение из всех святынь сынов Израилевых, которые они приносят к священнику, ему принадлежит,
\vs Num 5:10 и посвященное кем-либо ему принадлежит; все, что даст кто священнику, ему принадлежит.
\rsbpar\vs Num 5:11 И сказал Господь Моисею, говоря:
\vs Num 5:12 объяви сынам Израилевым и скажи им: если изменит кому жена, и нарушит верность к нему,
\vs Num 5:13 и переспит кто с ней и излиет семя, и это будет скрыто от глаз мужа ее, и она осквернится тайно, и не будет на нее свидетеля, и не будет уличена,
\vs Num 5:14 и найдет на него дух ревности, и будет ревновать жену свою, когда она осквернена, или найдет на него дух ревности, и он будет ревновать жену свою, когда она не осквернена,~---
\vs Num 5:15 пусть приведет муж жену свою к священнику и принесет за нее в жертву десятую часть ефы ячменной муки, но не возливает на нее елея и не кладет ливана, потому что это приношение ревнования, приношение воспоминания, напоминающее о беззаконии;
\vs Num 5:16 а священник пусть приведет и поставит ее пред лице Господне,
\vs Num 5:17 и возьмет священник святой воды в глиняный сосуд, и возьмет священник земли с полу скинии и положит в воду;
\vs Num 5:18 и поставит священник жену пред лице Господне, и обнажит голову жены, и даст ей в руки приношение воспоминания,~--- это приношение ревнования, в руке же у священника будет горькая вода, наводящая проклятие.
\vs Num 5:19 И заклянет ее священник и скажет жене: если никто не переспал с тобою, и ты не осквернилась и не изменила мужу своему, то невредима будешь от сей горькой воды, наводящей проклятие;
\vs Num 5:20 но если ты изменила мужу твоему и осквернилась, и если кто переспал с тобою кроме мужа твоего,~---
\vs Num 5:21 тогда священник пусть заклянет жену клятвою проклятия и скажет священник жене: да предаст тебя Господь проклятию и клятве в народе твоем, и да соделает Господь лоно твое опавшим и живот твой опухшим;
\vs Num 5:22 и да пройдет вода сия, наводящая проклятие, во внутренность твою, чтобы опух живот [твой] и опало лоно [твое]. И скажет жена: аминь, аминь.
\vs Num 5:23 И напишет священник заклинания сии на свитке, и смоет их в горькую воду;
\vs Num 5:24 и даст жене выпить горькую воду, наводящую проклятие, и войдет в нее вода, наводящая проклятие, ко вреду ее.
\vs Num 5:25 И возьмет священник из рук жены хлебное приношение ревнования, и вознесет сие приношение пред Господом, и отнесет его к жертвеннику;
\vs Num 5:26 и возьмет священник горстью из хлебного приношения часть в память, и сожжет на жертвеннике, и потом даст жене выпить воды;
\vs Num 5:27 и когда напоит ее водою, тогда, если она нечиста и сделала преступление против мужа своего, горькая вода, наводящая проклятие, войдет в нее, ко вреду ее, и опухнет чрево ее и опадет лоно ее, и будет эта жена проклятою в народе своем;
\vs Num 5:28 если же жена не осквернилась и была чиста, то останется невредимою и будет оплодотворяема семенем.
\vs Num 5:29 Вот закон о ревновании, когда жена изменит мужу своему и осквернится,
\vs Num 5:30 или когда на мужа найдет дух ревности, и он будет ревновать жену свою, тогда пусть он поставит жену пред лицем Господа, и сделает с нею священник все по сему закону,~---
\vs Num 5:31 и будет муж чист от греха, а жена понесет на себе грех свой.
\vs Num 6:1 И сказал Господь Моисею, говоря:
\vs Num 6:2 объяви сынам Израилевым и скажи им: если мужчина или женщина решится дать обет назорейства, чтобы посвятить себя в назореи Господу,
\vs Num 6:3 то он должен воздержаться от вина и \bibemph{крепкого} напитка, и не должен употреблять ни уксусу из вина, ни уксусу из напитка, и ничего приготовленного из винограда не должен пить, и не должен есть ни сырых, ни сушеных виноградных ягод;
\vs Num 6:4 во все дни назорейства своего не должен он есть [и пить] ничего, что делается из винограда, от зерен до кожи.
\vs Num 6:5 Во все дни обета назорейства его бритва не должна касаться головы его; до исполнения дней, на которые он посвятил себя в назореи Господу, свят он: должен растить волосы на голове своей.
\vs Num 6:6 Во все дни, на которые он посвятил себя в назореи Господу, не должен он подходить к мертвому телу:
\vs Num 6:7 \bibemph{прикосновением} к отцу своему, и матери своей, и брату своему, и сестре своей, не должен он оскверняться, когда они умрут, потому что посвящение Богу его на главе его;
\vs Num 6:8 во все дни назорейства своего свят он Господу.
\vs Num 6:9 Если же умрет при нем кто-нибудь вдруг, нечаянно, и он осквернит тем голову назорейства своего: то он должен остричь голову свою в день очищения его, в седьмой день должен остричь ее,
\vs Num 6:10 и в восьмой день должен принести двух горлиц, или двух молодых голубей, к священнику, ко входу скинии собрания;
\vs Num 6:11 священник одну \bibemph{из птиц} принесет в жертву за грех, а другую во всесожжение, и очистит его от осквернения мертвым телом, и освятит голову его в тот день;
\vs Num 6:12 и должен он снова начать посвященные Господу дни назорейства своего и принести однолетнего агнца в жертву повинности; прежние же дни пропали, потому что назорейство его осквернено.
\vs Num 6:13 И вот закон о назорее, когда исполнятся дни назорейства его: должно привести его ко входу скинии собрания,
\vs Num 6:14 и он принесет в жертву Господу одного однолетнего агнца без порока во всесожжение, и одну однолетнюю агницу без порока в жертву за грех, и одного овна без порока в жертву мирную,
\vs Num 6:15 и корзину опресноков из пшеничной муки, хлебов, испеченных с елеем, и пресных лепешек, помазанных елеем, и при них хлебное приношение и возлияние;
\vs Num 6:16 и представит \bibemph{сие} священник пред Господа, и принесет жертву его за грех и всесожжение его;
\vs Num 6:17 овна принесет в жертву мирную Господу с корзиною опресноков, также совершит священник хлебное приношение его и возлияние его;
\vs Num 6:18 и острижет назорей у входа скинии собрания голову назорейства своего, и возьмет волосы головы назорейства своего, и положит на огонь, который под мирною жертвою;
\vs Num 6:19 и возьмет священник сваренное плечо овна и один пресный пирог из корзины и одну пресную лепешку, и положит на руки назорею, после того, как острижет он голову назорейства своего;
\vs Num 6:20 и вознесет сие священник, потрясая пред Господом: эта святыня~--- для священника, сверх груди потрясания и сверх плеча возношения. После сего назорей может пить вино.
\vs Num 6:21 Вот закон о назорее, который дал обет, и жертва его Господу за назорейство свое, кроме того, что позволит ему достаток его; по обету своему, какой он даст, так и должен он делать, сверх узаконенного о назорействе его.
\rsbpar\vs Num 6:22 И сказал Господь Моисею, говоря:
\vs Num 6:23 скажи Аарону и сынам его: так благословляйте сынов Израилевых, говоря им:
\vs Num 6:24 да благословит тебя Господь и сохранит тебя!
\vs Num 6:25 да призрит на тебя Господь светлым лицем Своим и помилует тебя!
\vs Num 6:26 да обратит Господь лице Свое на тебя и даст тебе мир!
\vs Num 6:27 Так пусть призывают имя Мое на сынов Израилевых, и Я [Господь] благословлю их.
\vs Num 7:1 Когда Моисей поставил скинию, и помазал ее, и освятил ее и все принадлежности ее, и жертвенник и все принадлежности его, и помазал их и освятил их,
\vs Num 7:2 тогда пришли [двенадцать] начальников Израилевых, гл\acc{а}вы семейств их, начальники колен, заведовавшие исчислением,
\vs Num 7:3 и представили приношение свое пред Господа, шесть крытых повозок и двенадцать волов, по одной повозке от двух начальников и по одному волу от каждого, и представили сие пред скинию.
\rsbpar\vs Num 7:4 И сказал Господь Моисею, говоря:
\vs Num 7:5 возьми от них; это будет для отправления работ при скинии собрания; и отдай это левитам, смотря по роду службы их.
\vs Num 7:6 И взял Моисей повозки и волов, и отдал их левитам:
\vs Num 7:7 две повозки и четырех волов отдал сынам Гирсоновым, по роду служб их:
\vs Num 7:8 и четыре повозки и восемь волов отдал сынам Мерариным, по роду служб их, под надзором Ифамара, сына Аарона, священника;
\vs Num 7:9 а сынам Каафовым не дал, потому что служба их~--- \bibemph{носить} святилище; на плечах они должны носить.
\vs Num 7:10 И принесли начальники жертвы освящения жертвенника в день помазания его, и представили начальники приношение свое пред жертвенник.
\rsbpar\vs Num 7:11 И сказал Господь Моисею: по одному начальнику в день пусть приносят приношение свое для освящения жертвенника.
\vs Num 7:12 В первый день принес приношение свое Наассон, сын Аминадавов, от колена Иудина;
\vs Num 7:13 приношение его было: одно серебряное блюдо, весом в сто тридцать \bibemph{сиклей}, одна серебряная чаша в семьдесят сиклей, по сиклю священному, наполненные пшеничною мукою, смешанною с елеем, в приношение хлебное,
\vs Num 7:14 одна золотая кадильница в десять \bibemph{сиклей}, наполненная курением,
\vs Num 7:15 один телец, один овен, один однолетний агнец, во всесожжение,
\vs Num 7:16 один козел в жертву за грех,
\vs Num 7:17 и в жертву мирную два вола, пять овнов, пять козлов, пять однолетних агнцев; вот приношение Наассона, сына Аминадавова.
\vs Num 7:18 Во второй день принес Нафанаил, сын Цуара, начальник [колена] Иссахарова;
\vs Num 7:19 он принес от себя приношение: одно серебряное блюдо, весом в сто тридцать \bibemph{сиклей}, одну серебряную чашу в семьдесят сиклей, по сиклю священному, наполненные пшеничною мукою, смешанною с елеем, в приношение хлебное,
\vs Num 7:20 одну золотую кадильницу в десять \bibemph{сиклей}, наполненную курением,
\vs Num 7:21 одного тельца, одного овна, одного однолетнего агнца, во всесожжение,
\vs Num 7:22 одного козла в жертву за грех,
\vs Num 7:23 и в жертву мирную двух волов, пять овнов, пять козлов, пять однолетних агнцев; вот приношение Нафанаила, сына Цуарова.
\vs Num 7:24 В третий день начальник сынов Завулоновых Елиав, сын Хелона;
\vs Num 7:25 приношение его: одно серебряное блюдо, весом в сто тридцать \bibemph{сиклей}, одна серебряная чаша в семьдесят сиклей, по сиклю священному, наполненные пшеничною мукою, смешанною с елеем, в приношение хлебное,
\vs Num 7:26 одна золотая кадильница в десять \bibemph{сиклей}, наполненная курением,
\vs Num 7:27 один телец, один овен, один однолетний агнец, во всесожжение,
\vs Num 7:28 один козел в жертву за грех,
\vs Num 7:29 и в жертву мирную два вола, пять овнов, пять козлов, пять однолетних агнцев; вот приношение Елиава, сына Хелонова.
\vs Num 7:30 В четвертый день начальник сынов Рувимовых Елицур, сын Шедеуров;
\vs Num 7:31 приношение его: одно серебряное блюдо, весом в сто тридцать \bibemph{сиклей}, одна серебряная чаша в семьдесят сиклей, по сиклю священному, наполненные пшеничною мукою, смешанною с елеем, в приношение хлебное,
\vs Num 7:32 одна золотая кадильница в десять \bibemph{сиклей}, наполненная курением,
\vs Num 7:33 один телец, один овен, один однолетний агнец, во всесожжение,
\vs Num 7:34 один козел в жертву за грех,
\vs Num 7:35 и в жертву мирную два вола, пять овнов, пять козлов и пять однолетних агнцев; вот приношение Елицура, сына Шедеурова.
\vs Num 7:36 В пятый день начальник сынов Симеоновых Шелумиил, сын Цуришаддая;
\vs Num 7:37 приношение его: одно серебряное блюдо, весом в сто тридцать \bibemph{сиклей}, одна серебряная чаша в семьдесят сиклей, по сиклю священному, наполненные пшеничною мукою, смешанною с елеем, в приношение хлебное,
\vs Num 7:38 одна золотая кадильница в десять \bibemph{сиклей}, наполненная курением,
\vs Num 7:39 один телец, один овен, один однолетний агнец, во всесожжение,
\vs Num 7:40 один козел в жертву за грех,
\vs Num 7:41 и в жертву мирную два вола, пять овнов, пять козлов и пять однолетних агнцев; вот приношение Шелумиила, сына Цуришаддаева.
\vs Num 7:42 В шестой день начальник сынов Гадовых Елиасаф, сын Регуила;
\vs Num 7:43 приношение его: одно серебряное блюдо, весом в сто тридцать \bibemph{сиклей}, одна серебряная чаша в семьдесят сиклей, по сиклю священному, наполненные пшеничною мукою, смешанною с елеем, в приношение хлебное,
\vs Num 7:44 одна золотая кадильница в десять \bibemph{сиклей}, наполненная курением,
\vs Num 7:45 один телец, один овен, один однолетний агнец, во всесожжение,
\vs Num 7:46 один козел в жертву за грех,
\vs Num 7:47 и в жертву мирную два вола, пять овнов, пять козлов и пять однолетних агнцев; вот приношение Елиасафа, сына Регуилова.
\vs Num 7:48 В седьмой день начальник сынов Ефремовых Елишама, сын Аммиуда.
\vs Num 7:49 Приношение его: одно серебряное блюдо, весом в сто тридцать \bibemph{сиклей}, одна серебряная чаша в семьдесят сиклей, по сиклю священному, наполненные пшеничною мукою, смешанною с елеем, в приношение хлебное,
\vs Num 7:50 одна золотая кадильница в десять \bibemph{сиклей}, наполненная курением,
\vs Num 7:51 один телец, один овен, один однолетний агнец, во всесожжение,
\vs Num 7:52 один козел в жертву за грех,
\vs Num 7:53 и в жертву мирную два вола, пять овнов, пять козлов, пять однолетних агнцев; вот приношение Елишамы, сына Аммиудова.
\vs Num 7:54 В восьмой день начальник сынов Манассииных Гамалиил, сын Педацура.
\vs Num 7:55 Приношение его: одно серебряное блюдо, весом в сто тридцать \bibemph{сиклей}, одна серебряная чаша в семьдесят сиклей, по сиклю священному, наполненные пшеничною мукою, смешанною с елеем, в приношение хлебное,
\vs Num 7:56 одна золотая кадильница в десять \bibemph{сиклей}, наполненная курением,
\vs Num 7:57 один телец, один овен, один однолетний агнец, во всесожжение,
\vs Num 7:58 один козел в жертву за грех,
\vs Num 7:59 и в жертву мирную два вола, пять овнов, пять козлов, пять однолетних агнцев; вот приношение Гамалиила, сына Педацурова.
\vs Num 7:60 В девятый день начальник сынов Вениаминовых Авидан, сын Гидеония;
\vs Num 7:61 приношение его: одно серебряное блюдо, весом в сто тридцать \bibemph{сиклей}, одна серебряная чаша в семьдесят сиклей, по сиклю священному, наполненные пшеничною мукою, смешанною с елеем, в приношение хлебное,
\vs Num 7:62 одна золотая кадильница в десять \bibemph{сиклей}, наполненная курением,
\vs Num 7:63 один телец, один овен, один однолетний агнец, во всесожжение,
\vs Num 7:64 один козел в жертву за грех,
\vs Num 7:65 и в жертву мирную два вола, пять овнов, пять козлов, пять однолетних агнцев; вот приношение Авидана, сына Гидеониева.
\vs Num 7:66 В десятый день начальник сынов Дановых Ахиезер, сын Аммишаддая;
\vs Num 7:67 приношение его: одно серебряное блюдо, весом в сто тридцать \bibemph{сиклей}, одна серебряная чаша в семьдесят сиклей, по сиклю священному, наполненные пшеничною мукою, смешанною с елеем, в приношение хлебное,
\vs Num 7:68 одна золотая кадильница в десять \bibemph{сиклей}, наполненная курением,
\vs Num 7:69 один телец, один овен, один однолетний агнец, во всесожжение,
\vs Num 7:70 один козел в жертву за грех,
\vs Num 7:71 и в жертву мирную два вола, пять овнов, пять козлов, пять однолетних агнцев; вот приношение Ахиезера, сына Аммишаддаева.
\vs Num 7:72 В одиннадцатый день начальник сынов Асировых Пагиил, сын Охрана;
\vs Num 7:73 приношение его: одно серебряное блюдо, весом в сто тридцать \bibemph{сиклей}, одна серебряная чаша в семьдесят сиклей, по сиклю священному, наполненные пшеничною мукою, смешанною с елеем, в приношение хлебное,
\vs Num 7:74 одна золотая кадильница в десять \bibemph{сиклей}, наполненная курением,
\vs Num 7:75 один телец, один овен, один однолетний агнец, во всесожжение,
\vs Num 7:76 один козел в жертву за грех,
\vs Num 7:77 и в жертву мирную два вола, пять овнов, пять козлов, пять однолетних агнцев; вот приношение Пагиила, сына Охранова.
\vs Num 7:78 В двенадцатый день начальник сынов Неффалимовых Ахира, сын Енана;
\vs Num 7:79 приношение его: одно серебряное блюдо, весом в сто тридцать \bibemph{сиклей}, одна серебряная чаша в семьдесят сиклей, по сиклю священному, наполненные пшеничною мукою, смешанною с елеем, в приношение хлебное,
\vs Num 7:80 одна золотая кадильница в десять \bibemph{сиклей}, наполненная курением,
\vs Num 7:81 один телец, один овен, один однолетний агнец, во всесожжение,
\vs Num 7:82 один козел в жертву за грех,
\vs Num 7:83 и в жертву мирную два вола, пять овнов, пять козлов, пять однолетних агнцев; вот приношение Ахиры, сына Енанова.
\rsbpar\vs Num 7:84 Вот \bibemph{приношения} от начальников Израилевых при освящении жертвенника в день помазания его: двенадцать серебряных блюд, двенадцать серебряных чаш, двенадцать золотых кадильниц;
\vs Num 7:85 по сто тридцати \bibemph{сиклей} серебра в каждом блюде и по семидесяти в каждой чаше: итак всего серебра в сих сосудах две тысячи четыреста \bibemph{сиклей}, по сиклю священному;
\vs Num 7:86 золотых кадильниц, наполненных курением, двенадцать, в каждой кадильнице по десяти \bibemph{сиклей}, по сиклю священному: всего золота в кадильницах сто двадцать \bibemph{сиклей};
\vs Num 7:87 во всесожжение всего двенадцать тельцов из скота крупного, двенадцать овнов, двенадцать однолетних агнцев и при них хлебное приношение, и в жертву за грех двенадцать козлов,
\vs Num 7:88 и в жертву мирную всего из крупного скота двадцать четыре тельца, шестьдесят овнов, шестьдесят [однолетних] козлов, шестьдесят однолетних агнцев [без порока]; вот приношения при освящении жертвенника после помазания его.
\vs Num 7:89 Когда Моисей входил в скинию собрания, чтобы говорить с Господом, слышал голос, говорящий ему с крышки, которая над ковчегом откровения между двух херувимов, и он говорил ему.
\vs Num 8:1 И сказал Господь Моисею, говоря:
\vs Num 8:2 объяви Аарону и скажи ему: когда ты будешь зажигать лампады, то на передней стороне светильника должны гореть семь лампад.
\vs Num 8:3 Аарон так и сделал: на передней стороне светильника зажег лампады его, как повелел Господь Моисею.
\vs Num 8:4 И вот устройство светильника: чеканный он из золота, от стебля его и до цветов чеканный; по образу, который показал Господь Моисею, он сделал светильник.
\rsbpar\vs Num 8:5 И сказал Господь Моисею, говоря:
\vs Num 8:6 возьми левитов из среды сынов Израилевых и очисти их;
\vs Num 8:7 а чтобы очистить их, поступи с ними так: окропи их очистительною водою, и пусть они обреют бритвою все тело свое и вымоют одежды свои, и будут чисты;
\vs Num 8:8 и пусть возьмут тельца и хлебное приношение к нему, пшеничной муки, смешанной с елеем, и другого тельца возьми в жертву за грех;
\vs Num 8:9 и приведи левитов пред скинию собрания; и собери все общество сынов Израилевых
\vs Num 8:10 и приведи левитов их пред Господа, и пусть возложат сыны Израилевы руки свои на левитов;
\vs Num 8:11 Аарон же пусть совершит над левитами посвящение их пред Господом от сынов Израилевых, чтобы отправляли они служение Господу;
\vs Num 8:12 а левиты пусть возложат руки свои на голову тельцов, и принеси одного в жертву за грех, а другого во всесожжение Господу, для очищения левитов;
\vs Num 8:13 и поставь левитов пред Аароном и пред сынами его, и соверши над ними посвящение их Господу;
\vs Num 8:14 и так отдели левитов от сынов Израилевых, чтобы левиты были Моими.
\vs Num 8:15 После сего войдут левиты служить скинии собрания, когда ты очистишь их и совершишь над ними посвящение их; ибо они отданы Мне из сынов Израилевых:
\vs Num 8:16 вместо всех первенцев из сынов Израилевых, разверзающих всякие ложесна, Я беру их Себе;
\vs Num 8:17 ибо Мои все первенцы у сынов Израилевых, от человека до скота: в тот день, когда Я поразил всех первенцев в земле Египетской, Я освятил их Себе
\vs Num 8:18 и взял левитов вместо всех первенцев у сынов Израилевых;
\vs Num 8:19 и отдал левитов Аарону и сынам его из среды сынов Израилевых, чтобы они отправляли службы за сынов Израилевых при скинии собрания и служили охранением для сынов Израилевых, чтобы не постигло сынов Израилевых поражение, когда бы сыны Израилевы приступили к святилищу.
\vs Num 8:20 И сделали так Моисей и Аарон и все общество сынов Израилевых с левитами: как повелел Господь Моисею о левитах, так и сделали с ними сыны Израилевы.
\vs Num 8:21 И очистились левиты и омыли одежды свои, и совершил над ними Аарон посвящение их пред Господом, и очистил их Аарон, чтобы сделать их чистыми;
\vs Num 8:22 после сего вошли левиты отправлять службы свои в скинии собрания пред Аароном и пред сынами его. Как повелел Господь Моисею о левитах, так и сделали они с ними.
\rsbpar\vs Num 8:23 И сказал Господь Моисею, говоря:
\vs Num 8:24 вот \bibemph{закон} о левитах: от двадцати пяти лет и выше должны вступать они в службу для работ при скинии собрания,
\vs Num 8:25 а в пятьдесят лет должны прекращать отправление работ и более не работать:
\vs Num 8:26 тогда пусть помогают они братьям своим содержать стражу при скинии собрания, работать же~--- пусть не работают; так поступай с левитами касательно служения их.
\vs Num 9:1 И сказал Господь Моисею в пустыне Синайской во второй год по исшествии их из земли Египетской, в первый месяц, говоря:
\vs Num 9:2 пусть сыны Израилевы совершат Пасху в назначенное для нее время:
\vs Num 9:3 в четырнадцатый день сего месяца вечером совершите ее в назначенное для нее время, по всем постановлениям и по всем обрядам ее совершите ее.
\vs Num 9:4 И сказал Моисей сынам Израилевым, чтобы совершили Пасху.
\vs Num 9:5 И совершили они Пасху в первый \bibemph{месяц}, в четырнадцатый день месяца вечером, в пустыне Синайской: во всем, как повелел Господь Моисею, так и поступили сыны Израилевы.
\rsbpar\vs Num 9:6 Были люди, которые были нечисты от \bibemph{прикосновения} к мертвым телам человеческим, и не могли совершить Пасхи в тот день; и пришли они к Моисею и Аарону в тот день,
\vs Num 9:7 и сказали ему те люди: мы нечисты от \bibemph{прикосновения} к мертвым телам человеческим; для чего нас лишать того, чтобы мы принесли приношение Господу в назначенное время среди сынов Израилевых?
\vs Num 9:8 И сказал им Моисей: постойте, я послушаю, что повелит о вас Господь.
\rsbpar\vs Num 9:9 И сказал Господь Моисею, говоря:
\vs Num 9:10 скажи сынам Израилевым: если кто из вас или из потомков ваших будет нечист от \bibemph{прикосновения} к мертвому телу, или будет в дальней дороге, то и он должен совершить Пасху Господню;
\vs Num 9:11 в четырнадцатый день второго месяца вечером пусть таковые совершат ее и с опресноками и горькими травами пусть едят ее;
\vs Num 9:12 и пусть не оставляют от нее до утра и костей ее не сокрушают; пусть совершат ее по всем уставам о Пасхе;
\vs Num 9:13 а кто чист и не находится в [дальней] дороге и не совершит Пасхи,~--- истребится душа та из народа своего, ибо он не принес приношения Господу в свое время: понесет на себе грех человек тот;
\vs Num 9:14 если будет жить у вас пришелец, то и он должен совершать Пасху Господню: по уставу о Пасхе и по обряду ее он должен совершить ее; один устав пусть будет у вас и для пришельца и для туземца.
\rsbpar\vs Num 9:15 В тот день, когда поставлена была скиния, облако покрыло скинию откровения, и с вечера над скиниею как бы огонь виден был до самого утра.
\vs Num 9:16 Так было и всегда: облако покрывало ее [днем] и подобие огня ночью.
\vs Num 9:17 И когда облако поднималось от скинии, тогда сыны Израилевы отправлялись в путь, и на месте, где останавливалось облако, там останавливались станом сыны Израилевы.
\vs Num 9:18 По повелению Господню отправлялись сыны Израилевы в путь, и по повелению Господню останавливались: во все то время, когда облако стояло над скиниею, и они стояли;
\vs Num 9:19 и если облако долгое время было над скиниею, то и сыны Израилевы следовали этому указанию Господа и не отправлялись;
\vs Num 9:20 иногда же облако немного времени было над скиниею: они по указанию Господню останавливались, и по указанию Господню отправлялись в путь;
\vs Num 9:21 иногда облако стояло \bibemph{только} от вечера до утра, и поутру поднималось облако, тогда и они отправлялись; или день и ночь стояло облако, и когда поднималось, и они тогда отправлялись;
\vs Num 9:22 или, если два дня, или месяц, или несколько дней стояло облако над скиниею, то и сыны Израилевы стояли и не отправлялись в путь; а когда оно поднималось, тогда отправлялись;
\vs Num 9:23 по указанию Господню останавливались, и по указанию Господню отправлялись в путь: следовали указанию Господню по повелению Господню, \bibemph{данному} чрез Моисея.
\vs Num 10:1 И сказал Господь Моисею, говоря:
\vs Num 10:2 сделай себе две серебряные трубы, чеканные сделай их, чтобы они служили тебе для созывания общества и для снятия станов;
\vs Num 10:3 когда затрубят ими, соберется к тебе все общество ко входу скинии собрания;
\vs Num 10:4 когда одною трубою затрубят, соберутся к тебе князья и тысяченачальники Израилевы;
\vs Num 10:5 когда затрубите тревогу, поднимутся станы, становящиеся к востоку;
\vs Num 10:6 когда во второй раз затрубите тревогу, поднимутся станы, становящиеся к югу; [когда затрубите в третий раз тревогу, поднимутся станы, становящиеся к морю; когда в четвертый раз затрубите тревогу, поднимутся станы, становящиеся к северу;] тревогу пусть трубят при отправлении их в путь;
\vs Num 10:7 а когда надобно собрать собрание, трубите, но не тревогу;
\vs Num 10:8 сыны Аароновы, священники, должны трубить трубами: это будет вам постановлением вечным в роды ваши;
\vs Num 10:9 и когда пойдете на войну в земле вашей против врага, наступающего на вас, трубите тревогу трубами,~--- и будете воспомянуты пред Господом, Богом вашим, и спасены будете от врагов ваших;
\vs Num 10:10 и в день веселия вашего, и в праздники ваши, и в новомесячия ваши трубите трубами при всесожжениях ваших и при мирных жертвах ваших,~--- и это будет напоминанием о вас пред Богом вашим. Я Господь, Бог ваш.
\rsbpar\vs Num 10:11 Во второй год, во второй месяц, в двадцатый \bibemph{день} месяца поднялось облако от скинии откровения;
\vs Num 10:12 и отправились сыны Израилевы по станам своим из пустыни Синайской, и остановилось облако в пустыне Фаран.
\vs Num 10:13 И поднялись они в первый раз, по повелению Господню, \bibemph{данному} чрез Моисея.
\vs Num 10:14 Поднято было во-первых знамя стана сынов Иудиных по ополчениям их; над ополчением их Наассон, сын Аминадава;
\vs Num 10:15 и над ополчением колена сынов Иссахаровых Нафанаил, сын Цуара;
\vs Num 10:16 и над ополчением колена сынов Завулоновых Елиав, сын Хелона.
\vs Num 10:17 И снята была скиния, и пошли сыны Гирсоновы и сыны Мерарины, носящие скинию.
\vs Num 10:18 И поднято было знамя стана Рувимова по ополчениям их; и над ополчением его Елицур, сын Шедеура;
\vs Num 10:19 и над ополчением колена сынов Симеоновых Шелумиил, сын Цуришаддая;
\vs Num 10:20 и над ополчением колена сынов Гадовых Елиасаф, сын Регуила.
\vs Num 10:21 Потом пошли сыны Каафовы, носящие святилище; скиния же была поставляема до прихода их.
\vs Num 10:22 И поднято было знамя стана сынов Ефремовых по ополчениям их; и над ополчением их Елишама, сын Аммиуда;
\vs Num 10:23 и над ополчением колена сынов Манассииных Гамалиил, сын Педацура;
\vs Num 10:24 и над ополчением колена сынов Вениаминовых Авидан, сын Гидеония.
\vs Num 10:25 Последним из всех станов поднято было знамя стана сынов Дановых с ополчениями их; и над ополчением их Ахиезер, сын Аммишаддая;
\vs Num 10:26 и над ополчением колена сынов Асировых Пагиил, сын Охрана;
\vs Num 10:27 и над ополчением колена сынов Неффалимовых Ахира, сын Енана.
\vs Num 10:28 Вот \bibemph{порядок} шествия сынов Израилевых по ополчениям их. И отправились они.
\rsbpar\vs Num 10:29 И сказал Моисей Ховаву, сыну Рагуилову, Мадианитянину, родственнику Моисееву: мы отправляемся в то место, о котором Господь сказал: вам отдам его; иди с нами, мы сделаем тебе добро, ибо Господь доброе изрек об Израиле.
\vs Num 10:30 Но он сказал ему: не пойду; я пойду в свою землю и на свою родину.
\vs Num 10:31 \bibemph{Моисей} же сказал: не оставляй нас, потому что ты знаешь, как располагаемся мы станом в пустыне, и будешь для нас глазом;
\vs Num 10:32 если пойдешь с нами, то добро, которое Господь сделает нам, мы сделаем тебе.
\vs Num 10:33 И отправились они от горы Господней на три дня пути, и ковчег завета Господня шел пред ними три дня пути, чтоб усмотреть им место, где остановиться.
\vs Num 10:34 И облако Господне осеняло их днем, когда они отправлялись из стана.
\vs Num 10:35 Когда поднимался ковчег в путь, Моисей говорил: восстань, Господи, и рассыплются враги Твои, и побегут от лица Твоего ненавидящие Тебя!
\vs Num 10:36 А когда останавливался ковчег, он говорил: возвратись, Господи, к тысячам и тьмам Израилевым!
\vs Num 11:1 Народ стал роптать вслух Господа; и Господь услышал, и воспламенился гнев Его, и возгорелся у них огонь Господень, и начал истреблять край стана.
\vs Num 11:2 И возопил народ к Моисею; и помолился Моисей Господу, и утих огонь.
\vs Num 11:3 И нарекли имя месту сему: Тавера\fns{Горение.}, потому что возгорелся у них огонь Господень.
\vs Num 11:4 Пришельцы между ними стали обнаруживать прихоти; а с ними и сыны Израилевы сидели и плакали и говорили: кто накормит нас мясом?
\vs Num 11:5 Мы помним рыбу, которую в Египте мы ели даром, огурцы и дыни, и лук, и репчатый лук и чеснок;
\vs Num 11:6 а ныне душа наша изнывает; ничего нет, только манна в глазах наших.
\vs Num 11:7 Манна же была подобна кориандровому семени, видом, как бдолах;
\vs Num 11:8 народ ходил и собирал ее, и молол в жерновах или толок в ступе, и варил в котле, и делал из нее лепешки; вкус же ее подобен был вкусу лепешек с елеем.
\vs Num 11:9 И когда роса сходила на стан ночью, тогда сходила на него и манна.
\vs Num 11:10 Моисей слышал, что народ плачет в семействах своих, каждый у дверей шатра своего; и сильно воспламенился гнев Господень, и прискорбно было для Моисея.
\vs Num 11:11 И сказал Моисей Господу: для чего Ты мучишь раба Твоего? и почему я не нашел милости пред очами Твоими, что Ты возложил на меня бремя всего народа сего?
\vs Num 11:12 разве я носил во чреве весь народ сей, и разве я родил его, что Ты говоришь мне: неси его на руках твоих, как нянька носит ребенка, в землю, которую Ты с клятвою обещал отцам его?
\vs Num 11:13 откуда мне \bibemph{взять} мяса, чтобы дать всему народу сему? ибо они плачут предо мною и говорят: дай нам есть мяса.
\vs Num 11:14 Я один не могу нести всего народа сего, потому что он тяжел для меня;
\vs Num 11:15 когда Ты так поступаешь со мною, то \bibemph{лучше} умертви меня, если я нашел милость пред очами Твоими, чтобы мне не видеть бедствия моего.
\rsbpar\vs Num 11:16 И сказал Господь Моисею: собери Мне семьдесят мужей из старейшин Израилевых, которых ты знаешь, что они старейшины и надзиратели его, и возьми их к скинии собрания, чтобы они стали там с тобою;
\vs Num 11:17 Я сойду, и буду говорить там с тобою, и возьму от Духа, Который на тебе, и возложу на них, чтобы они несли с тобою бремя народа, а не один ты носил.
\vs Num 11:18 Народу же скажи: очиститесь к завтрашнему дню, и будете есть мясо; так как вы плакали вслух Господа и говорили: кто накормит нас мясом? хорошо нам было в Египте,~--- то и даст вам Господь мясо, и будете есть [мясо]:
\vs Num 11:19 не один день будете есть, не два дня, не пять дней, не десять дней и не двадцать дней,
\vs Num 11:20 но целый месяц [будете есть], пока не пойдет оно из ноздрей ваших и не сделается для вас отвратительным, за то, что вы презрели Господа, Который среди вас, и плакали пред Ним, говоря: для чего было нам выходить из Египта?
\vs Num 11:21 И сказал Моисей: шестьсот тысяч пеших в народе сем, среди которого я \bibemph{нахожусь}; а Ты говоришь: Я дам им мясо, и будут есть целый месяц!
\vs Num 11:22 заколоть ли всех овец и волов, чтобы им было довольно? или вся рыба морская соберется, чтобы удовлетворить их?
\vs Num 11:23 И сказал Господь Моисею: разве рука Господня коротка? ныне ты увидишь, сбудется ли слово Мое тебе, или нет?
\rsbpar\vs Num 11:24 Моисей вышел и сказал народу слова Господни, и собрал семьдесят мужей из старейшин народа и поставил их около скинии.
\vs Num 11:25 И сошел Господь в облаке, и говорил с ним, и взял от Духа, Который на нем, и дал семидесяти мужам старейшинам. И когда почил на них Дух, они стали пророчествовать, но потом перестали.
\vs Num 11:26 Двое из мужей оставались в стане, одному имя Елдад, а другому имя Модад; но и на них почил Дух [они были из числа записанных, только не выходили к скинии], и они пророчествовали в стане.
\vs Num 11:27 И прибежал отрок, и донес Моисею, и сказал: Елдад и Модад пророчествуют в стане.
\vs Num 11:28 В ответ на это Иисус, сын Навин, служитель Моисея, один из избранных его, сказал: господин мой Моисей! запрети им.
\vs Num 11:29 Но Моисей сказал ему: не ревнуешь ли ты за меня? о, если бы все в народе Господнем были пророками, когда бы Господь послал Духа Своего на них!
\vs Num 11:30 И возвратился Моисей в стан, он и старейшины Израилевы.
\vs Num 11:31 И поднялся ветер от Господа, и принес от моря перепелов, и набросал их около стана, на путь дня по одну сторону и на путь дня по другую сторону около стана, на два почти локтя от земли.
\vs Num 11:32 И встал народ, и весь тот день, и всю ночь, и весь следующий день собирали перепелов; и кто мало собирал, тот собрал десять хомеров; и разложили их для себя вокруг стана.
\vs Num 11:33 Мясо еще было в зубах их и не было еще съедено, как гнев Господень возгорелся на народ, и поразил Господь народ весьма великою язвою.
\vs Num 11:34 И нарекли имя месту сему: Киброт-Гаттаава\fns{Гробы прихоти.}, ибо там похоронили прихотливый народ.
\vs Num 11:35 От Киброт-Гаттаавы двинулся народ в Асироф, и остановился в Асирофе.
\vs Num 12:1 И упрекали Мариам и Аарон Моисея за жену Ефиоплянку, которую он взял,~--- ибо он взял \bibemph{за себя} Ефиоплянку,~---
\vs Num 12:2 и сказали: одному ли Моисею говорил Господь? не говорил ли Он и нам? И услышал \bibemph{сие} Господь.
\vs Num 12:3 Моисей же был человек кротчайший из всех людей на земле.
\rsbpar\vs Num 12:4 И сказал Господь внезапно Моисею и Аарону и Мариами: выйдите вы трое к скинии собрания. И вышли все трое.
\vs Num 12:5 И сошел Господь в облачном столпе, и стал у входа скинии, и позвал Аарона и Мариам, и вышли они оба.
\vs Num 12:6 И сказал: слушайте слова Мои: если бывает у вас пророк Господень, то Я открываюсь ему в видении, во сне говорю с ним;
\vs Num 12:7 но не так с рабом Моим Моисеем,~--- он верен во всем дом\acc{у} Моем:
\vs Num 12:8 устами к устам говорю Я с ним, и явно, а не в гаданиях, и образ Господа он видит; как же вы не убоялись упрекать раба Моего, Моисея?
\vs Num 12:9 И воспламенился гнев Господа на них, и Он отошел.
\vs Num 12:10 И облако отошло от скинии, и вот, Мариам покрылась проказою, как снегом. Аарон взглянул на Мариам, и вот, она в проказе.
\vs Num 12:11 И сказал Аарон Моисею: господин мой! не поставь нам в грех, что мы поступили глупо и согрешили;
\vs Num 12:12 не попусти, чтоб она была, как мертворожденный \bibemph{младенец}, у которого, когда он выходит из чрева матери своей, истлела уже половина тела.
\vs Num 12:13 И возопил Моисей к Господу, говоря: Боже, исцели ее!
\vs Num 12:14 И сказал Господь Моисею: если бы отец ее плюнул ей в лице, то не должна ли была бы она стыдиться семь дней? итак пусть будет она в заключении семь дней вне стана, а после опять возвратится.
\vs Num 12:15 И пробыла Мариам в заключении вне стана семь дней, и народ не отправлялся в путь, доколе не возвратилась Мариам.
\vs Num 13:1 После сего народ двинулся из Асирофа, и остановился в пустыне Фаран.
\vs Num 13:2 И сказал Господь Моисею, говоря:
\vs Num 13:3 пошли от себя людей, чтобы они высмотрели землю Ханаанскую, которую Я даю сынам Израилевым; по одному человеку от колена отцов их пошлите, главных из них.
\vs Num 13:4 И послал их Моисей из пустыни Фаран, по повелению Господню, и все они мужи главные у сынов Израилевых.
\vs Num 13:5 Вот имена их: из колена Рувимова Саммуа, сын Закхуров,
\vs Num 13:6 из колена Симеонова Сафат, сын Хориев,
\vs Num 13:7 из колена Иудина Халев, сын Иефонниин,
\vs Num 13:8 из колена Иссахарова Игал, сын Иосифов,
\vs Num 13:9 из колена Ефремова Осия, сын Навин,
\vs Num 13:10 из колена Вениаминова Фалтий, сын Рафуев,
\vs Num 13:11 из колена Завулонова Гаддиил, сын Содиев,
\vs Num 13:12 из колена Иосифова от Манассии Гаддий, сын Сусиев,
\vs Num 13:13 из колена Данова Аммиил, сын Гемаллиев,
\vs Num 13:14 из колена Асирова Сефур, сын Михаилев,
\vs Num 13:15 из колена Неффалимова Нахбий, сын Вофсиев,
\vs Num 13:16 из колена Гадова Геуил, сын Махиев.
\vs Num 13:17 Вот имена мужей, которых посылал Моисей высмотреть землю. И назвал Моисей Осию, сына Навина, Иисусом.
\rsbpar\vs Num 13:18 И послал их Моисей [из пустыни Фаран] высмотреть землю Ханаанскую и сказал им: пойдите в эту южную страну, и взойдите на гору,
\vs Num 13:19 и осмотрите землю, какова она, и народ живущий на ней, силен ли он или слаб, малочислен ли он или многочислен?
\vs Num 13:20 и какова земля, на которой он живет, хороша ли она или худа? и каковы города, в которых он живет, в шатрах ли он живет или в укреплениях?
\vs Num 13:21 и какова земля, тучна ли она или тоща? есть ли на \bibemph{ней} дерева или нет? будьте смелы, и возьмите от плодов земли. Было же это ко времени созревания винограда.
\vs Num 13:22 Они пошли и высмотрели землю от пустыни Син даже до Рехова, близ Емафа;
\vs Num 13:23 и пошли в южную страну, и дошли до Хеврона, где жили Ахиман, Сесай и Фалмай, дети Енаковы: Хеврон же построен был семью годами прежде Цоана, [города] Египетского;
\vs Num 13:24 и пришли к долине Есхол, [и осмотрели ее,] и срезали там виноградную ветвь с одною кистью ягод, и понесли ее на шесте двое; \bibemph{взяли} также гранатовых яблок и смокв;
\vs Num 13:25 место сие назвали долиною Есхол\fns{Виноградная кисть.}, по причине виноградной кисти, которую срезали там сыны Израилевы.
\vs Num 13:26 И высмотрев землю, возвратились они через сорок дней.
\vs Num 13:27 И пошли и пришли к Моисею и Аарону и ко всему обществу сынов Израилевых в пустыню Фаран, в Кадес, и принесли им и всему обществу ответ, и показали им плоды земли;
\vs Num 13:28 и рассказывали ему и говорили: мы ходили в землю, в которую ты посылал нас; в ней подлинно течет молоко и мед, и вот плоды ее;
\vs Num 13:29 но народ, живущий на земле той, силен, и города укрепленные, весьма большие, и сынов Енаковых мы видели там;
\vs Num 13:30 Амалик живет на южной части земли, Хеттеи, [Евеи,] Иевусеи и Аморреи живут на горе, Хананеи же живут при море и на берегу Иордана.
\vs Num 13:31 Но Халев успокаивал народ пред Моисеем, говоря: пойдем и завладеем ею, потому что мы можем одолеть ее.
\vs Num 13:32 А те, которые ходили с ним, говорили: не можем мы идти против народа сего, ибо он сильнее нас.
\vs Num 13:33 И распускали худую молву о земле, которую они осматривали, между сынами Израилевыми, говоря: земля, которую проходили мы для осмотра, есть земля, поедающая живущих на ней, и весь народ, который видели мы среди ее, люди великорослые;
\vs Num 13:34 там видели мы и исполинов, сынов Енаковых, от исполинского рода; и мы были в глазах наших \bibemph{пред ними}, как саранча, такими же были мы и в глазах их.
\vs Num 14:1 И подняло все общество вопль, и плакал народ во [всю] ту ночь;
\vs Num 14:2 и роптали на Моисея и Аарона все сыны Израилевы, и все общество сказало им: о, если бы мы умерли в земле Египетской, или умерли бы в пустыне сей!
\vs Num 14:3 и для чего Господь ведет нас в землю сию, чтобы мы пали от меча? жены наши и дети наши достанутся в добычу \bibemph{врагам}; не лучше ли нам возвратиться в Египет?
\vs Num 14:4 И сказали друг другу: поставим себе начальника и возвратимся в Египет.
\vs Num 14:5 И пали Моисей и Аарон на лица свои пред всем собранием общества сынов Израилевых.
\vs Num 14:6 И Иисус, сын Навин, и Халев, сын Иефонниин, из осматривавших землю, разодрали одежды свои
\vs Num 14:7 и сказали всему обществу сынов Израилевых: земля, которую мы проходили для осмотра, очень, очень хороша;
\vs Num 14:8 если Господь милостив к нам, то введет нас в землю сию и даст нам ее~--- эту землю, в которой течет молоко и мед;
\vs Num 14:9 только против Господа не восставайте и не бойтесь народа земли сей; ибо он достанется нам на съедение: защиты у них не стало, а с нами Господь; не бойтесь их.
\vs Num 14:10 И сказало все общество: побить их камнями! Но слава Господня явилась [в облаке] в скинии собрания всем сынам Израилевым.
\rsbpar\vs Num 14:11 И сказал Господь Моисею: доколе будет раздражать Меня народ сей? и доколе будет он не верить Мне при всех знамениях, которые делал Я среди его?
\vs Num 14:12 поражу его язвою и истреблю его и произведу от тебя [и от дома отца твоего] народ многочисленнее и сильнее его.
\vs Num 14:13 Но Моисей сказал Господу: услышат Египтяне, из среды которых Ты силою Твоею вывел народ сей,
\vs Num 14:14 и скажут жителям земли сей, которые слышали, что Ты, Господь, находишься среди народа сего, и что Ты, Господь, даешь им видеть Себя лицем к лицу, и облако Твое стоит над ними, и Ты идешь пред ними днем в столпе облачном, а ночью в столпе огненном;
\vs Num 14:15 и если Ты истребишь народ сей, как одного человека, то народы, которые слышали славу Твою, скажут:
\vs Num 14:16 Господь не мог ввести народ сей в землю, которую Он с клятвою обещал ему, а потому и погубил его в пустыне.
\vs Num 14:17 Итак да возвеличится сила Господня, как Ты сказал, говоря:
\vs Num 14:18 Господь долготерпелив и многомилостив [и истинен], прощающий беззакония и преступления [и грехи], и не оставляющий без наказания, но наказывающий беззаконие отцов в детях до третьего и четвертого рода.
\vs Num 14:19 Прости грех народу сему по великой милости Твоей, как Ты прощал народ сей от Египта доселе.
\rsbpar\vs Num 14:20 И сказал Господь [Моисею]: прощаю по слову твоему;
\vs Num 14:21 но жив Я, [и всегда живет имя Мое,] и славы Господней полна вся земля:
\vs Num 14:22 все, которые видели славу Мою и знамения Мои, сделанные Мною в Египте и в пустыне, и искушали Меня уже десять раз, и не слушали гласа Моего,
\vs Num 14:23 не увидят земли, которую Я с клятвою обещал отцам их; [только детям их, которые здесь со Мною, которые не знают, что добро, что зло, всем малолетним, ничего не смыслящим, им дам землю, а] все, раздражавшие Меня, не увидят ее;
\vs Num 14:24 но раба Моего, Халева, за то, что в нем был иной дух, и он совершенно повиновался Мне, введу в землю, в которую он ходил, и семя его наследует ее;
\vs Num 14:25 Амаликитяне и Хананеи живут в долине; завтра обратитесь и идите в пустыню к Чермному морю.
\rsbpar\vs Num 14:26 И сказал Господь Моисею и Аарону, говоря:
\vs Num 14:27 доколе злому обществу сему роптать на Меня? ропот сынов Израилевых, которым они ропщут на Меня, Я слышу.
\vs Num 14:28 Скажи им: живу Я, говорит Господь: как говорили вы вслух Мне, так и сделаю вам;
\vs Num 14:29 в пустыне сей падут тела ваши, и все вы исчисленные, сколько вас числом, от двадцати лет и выше, которые роптали на Меня,
\vs Num 14:30 не войдете в землю, на которой Я, подъемля руку Мою, \bibemph{клялся} поселить вас, кроме Халева, сына Иефонниина, и Иисуса, сына Навина;
\vs Num 14:31 детей ваших, о которых вы говорили, что они достанутся в добычу \bibemph{врагам}, Я введу \bibemph{туда}, и они узнают землю, которую вы презрели,
\vs Num 14:32 а ваши трупы падут в пустыне сей;
\vs Num 14:33 а сыны ваши будут кочевать в пустыне сорок лет, и будут нести \bibemph{наказание} за блудодейство ваше, доколе не погибнут все тела ваши в пустыне;
\vs Num 14:34 по числу сорока дней, в которые вы осматривали землю, вы понесете наказание за грехи ваши сорок лет, год за день, дабы вы познали, \bibemph{что значит} быть оставленным Мною.
\vs Num 14:35 Я, Господь, говорю, и так и сделаю со всем сим злым обществом, восставшим против Меня: в пустыне сей все они погибнут и перемрут.
\vs Num 14:36 И те, которых посылал Моисей для осмотрения земли, и которые, возвратившись, возмутили против него все сие общество, распуская худую молву о земле,
\vs Num 14:37 сии, распустившие худую молву о земле, умерли, быв поражены пред Господом;
\vs Num 14:38 только Иисус, сын Навин, и Халев, сын Иефонниин, остались живы из тех мужей, которые ходили осматривать землю.
\rsbpar\vs Num 14:39 И сказал Моисей слова сии пред всеми сынами Израилевыми, и народ сильно опечалился.
\vs Num 14:40 И, встав рано поутру, пошли на вершину горы, говоря: вот, мы пойдем на то место, о котором сказал Господь, ибо мы согрешили.
\vs Num 14:41 Моисей сказал: для чего вы преступаете повеление Господне? это будет безуспешно;
\vs Num 14:42 не ходите, ибо нет среди вас Господа, чтобы не поразили вас враги ваши;
\vs Num 14:43 ибо Амаликитяне и Хананеи там пред вами, и вы падете от меча, потому что вы отступили от Господа, и не будет с вами Господа.
\vs Num 14:44 Но они дерзнули подняться на вершину горы; ковчег же завета Господня и Моисей не оставляли стана.
\vs Num 14:45 И сошли Амаликитяне и Хананеи, живущие на горе той, и разбили их, и гнали их до Хормы, [и возвратились в стан.]
\vs Num 15:1 И сказал Господь Моисею, говоря:
\vs Num 15:2 объяви сынам Израилевым и скажи им: когда вы войдете в землю вашего жительства, которую Я даю вам,
\vs Num 15:3 и будете приносить жертву Господу, всесожжение, или жертву заколаемую, от волов и овец, во исполнение обета, или по усердию, или в праздники ваши, дабы сделать приятное благоухание Господу,~---
\vs Num 15:4 тогда приносящий жертву свою Господу должен принести в приношение от хлеба десятую часть [ефы] пшеничной муки, смешанной с четвертою частью гина елея;
\vs Num 15:5 и вина для возлияния приноси четвертую часть гина при всесожжении, или при заколаемой жертве, на каждого агнца [в приятное благоухание Господу].
\vs Num 15:6 А принося овна, приноси в приношение хлебное две десятых части \bibemph{ефы} пшеничной муки, смешанной с третьею частью гина елея;
\vs Num 15:7 и вина для возлияния приноси третью часть гина в приятное благоухание Господу.
\vs Num 15:8 Если молодого вола приносишь во всесожжение или жертву заколаемую, во исполнение обета или в мирную жертву Господу,
\vs Num 15:9 то вместе с волом должно принести приношения хлебного три десятых части \bibemph{ефы} пшеничной муки, смешанной с половиною гина елея;
\vs Num 15:10 и вина для возлияния приноси полгина в жертву, в приятное благоухание Господу.
\vs Num 15:11 Так делай при каждом приношении вола и овна и агнца из овец, или коз;
\vs Num 15:12 по числу \bibemph{жертв}, которые вы приносите, так делайте при каждой, по числу их.
\vs Num 15:13 Всякий туземец так должен делать это, принося жертву в приятное благоухание Господу;
\vs Num 15:14 и если будет между вами жить пришелец, или кто бы ни был среди вас в роды ваши, и принесет жертву в приятное благоухание Господу, то и он должен делать так, как вы делаете;
\vs Num 15:15 для вас, общество [Господне], и для пришельца, живущего [у вас], устав один, устав вечный в роды ваши: что вы, то и пришелец да будет пред Господом;
\vs Num 15:16 закон один и одни права да будут для вас и для пришельца, живущего у вас.
\rsbpar\vs Num 15:17 И сказал Господь Моисею, говоря:
\vs Num 15:18 объяви сынам Израилевым и скажи им: когда вы войдете в землю, в которую Я веду вас,
\vs Num 15:19 и будете есть хлеб той земли, то возносите возношение Господу;
\vs Num 15:20 от начатков теста вашего лепешку возносите в возношение; возносите ее так, как возношение с гумна;
\vs Num 15:21 от начатков теста вашего отдавайте в возношение Господу в роды ваши.
\vs Num 15:22 Если же преступите по неведению и не исполните всех сих заповедей, которые изрек Господь Моисею,
\vs Num 15:23 всего, что заповедал вам Господь [Бог] чрез Моисея, от того дня, в который Господь заповедал вам, и впредь в роды ваши,~---
\vs Num 15:24 то, если по недосмотру общества сделана ошибка, пусть все общество принесет одного молодого вола [без порока] во всесожжение, в приятное благоухание Господу, с хлебным приношением и возлиянием его, по уставу, и одного козла в жертву за грех;
\vs Num 15:25 и очистит священник все общество сынов Израилевых, и будет прощено им, ибо это была ошибка, и они принесли приношение свое в жертву Господу, и жертву за грех свой пред Господом, за свою ошибку;
\vs Num 15:26 и будет прощено всему обществу сынов Израилевых и пришельцу, живущему между ними, потому что весь народ сделал это по ошибке.
\vs Num 15:27 Если же один кто согрешит по неведению, то пусть принесет козу однолетнюю в жертву за грех;
\vs Num 15:28 и очистит священник душу, сделавшую по ошибке грех пред Господом, и очищена будет, и прощено будет ей;
\vs Num 15:29 один закон да будет для вас, как для природного жителя из сынов Израилевых, так и для пришельца, живущего у вас, если кто сделает что по ошибке.
\vs Num 15:30 Если же кто из туземцев, или из пришельцев, сделает что дерзкою рукою, то он хулит Господа: истребится душа та из народа своего,
\vs Num 15:31 ибо слово Господне он презрел и заповедь Его нарушил; истребится душа та; грех ее на ней.
\rsbpar\vs Num 15:32 Когда сыны Израилевы были в пустыне, нашли человека, собиравшего дрова в день субботы;
\vs Num 15:33 и привели его нашедшие его собирающим дрова [в день субботы] к Моисею и Аарону и ко всему обществу [сынов Израилевых];
\vs Num 15:34 и посадили его под стражу, потому что не было еще определено, что должно с ним сделать.
\vs Num 15:35 И сказал Господь Моисею: должен умереть человек сей; пусть побьет его камнями все общество вне стана.
\vs Num 15:36 И вывело его все общество вон из стана, и побили его камнями, и он умер, как повелел Господь Моисею.
\rsbpar\vs Num 15:37 И сказал Господь Моисею, говоря:
\vs Num 15:38 объяви сынам Израилевым и скажи им, чтоб они делали себе кисти на краях одежд своих в роды их, и в кисти, которые на краях, вставляли нити из голубой шерсти;
\vs Num 15:39 и будут они в кистях у вас для того, чтобы вы, смотря на них, вспоминали все заповеди Господни, и исполняли их, и не ходили вслед сердца вашего и очей ваших, которые влекут вас к блудодейству,
\vs Num 15:40 чтобы вы помнили и исполняли все заповеди Мои и были святы пред Богом вашим.
\vs Num 15:41 Я Господь, Бог ваш, Который вывел вас из земли Египетской, чтоб быть вашим Богом: Я Господь, Бог ваш.
\vs Num 16:1 Корей, сын Ицгара, сын Каафов, сын Левиин, и Дафан и Авирон, сыны Елиава, и Авнан, сын Фалефа, сыны Рувимовы,
\vs Num 16:2 восстали на Моисея, и \bibemph{с ними} из сынов Израилевых двести пятьдесят мужей, начальники общества, призываемые на собрания, люди именитые.
\vs Num 16:3 И собрались против Моисея и Аарона и сказали им: полно вам; все общество, все святы, и среди их Господь! почему же вы ставите себя выше народа Господня?
\vs Num 16:4 Моисей, услышав это, пал на лице свое
\vs Num 16:5 и сказал Корею и всем сообщникам его, говоря: завтра покажет Господь, кто Его, и кто свят, чтобы приблизить его к Себе; и кого Он изберет, того и приблизит к Себе;
\vs Num 16:6 вот что сделайте: Корей и все сообщники его возьмите себе кадильницы
\vs Num 16:7 и завтра положите в них огня и всыпьте в них курения пред Господом; и кого изберет Господь, тот и будет свят. Полно вам, сыны Левиины!
\vs Num 16:8 И сказал Моисей Корею: послушайте, сыны Левия!
\vs Num 16:9 неужели мало вам того, что Бог Израилев отделил вас от общества Израильского и приблизил вас к Себе, чтобы вы исполняли службы при скинии Господней и стояли пред обществом [Господним], служа для них?
\vs Num 16:10 Он приблизил тебя и с тобою всех братьев твоих, сынов Левия, и вы домогаетесь еще и священства.
\vs Num 16:11 Итак ты и все твое общество собрались против Господа. Что Аарон, что вы ропщете на него?
\vs Num 16:12 И послал Моисей позвать Дафана и Авирона, сынов Елиава. Но они сказали: не пойдем!
\vs Num 16:13 разве мало того, что ты вывел нас из земли, в которой течет молоко и мед, чтобы погубить нас в пустыне? и ты еще хочешь властвовать над нами!
\vs Num 16:14 привел ли ты нас в землю, где течет молоко и мед, и дал ли нам во владение поля и виноградники? глаза людей сих ты хочешь ослепить? не пойдем!
\vs Num 16:15 Моисей весьма огорчился и сказал Господу: не обращай взора Твоего на приношение их; я не взял ни у одного из них осла и не сделал зла ни одному из них.
\vs Num 16:16 И сказал Моисей Корею: завтра ты и все общество твое будьте пред лицем Господа, ты, они и Аарон;
\vs Num 16:17 и возьмите каждый свою кадильницу, и положите в них курения, и принесите пред лице Господне каждый свою кадильницу, двести пятьдесят кадильниц; ты и Аарон, каждый свою кадильницу.
\vs Num 16:18 И взял каждый свою кадильницу, и положили в них огня, и всыпали в них курения, и стали при входе в скинию собрания; также и Моисей и Аарон.
\vs Num 16:19 И собрал против них Корей все общество ко входу скинии собрания. И явилась слава Господня всему обществу.
\rsbpar\vs Num 16:20 И сказал Господь Моисею и Аарону, говоря:
\vs Num 16:21 отделитесь от общества сего, и Я истреблю их во мгновение.
\vs Num 16:22 Они же пали на лица свои и сказали: Боже, Боже духов всякой плоти! один человек согрешил, и Ты гневаешься на все общество?
\vs Num 16:23 И сказал Господь Моисею, говоря:
\vs Num 16:24 скажи обществу: отступите со всех сторон от жилища Корея, Дафана и Авирона.
\vs Num 16:25 И встал Моисей, и пошел к Дафану и Авирону, и за ним пошли старейшины Израилевы.
\vs Num 16:26 И сказал обществу: отойдите от шатров нечестивых людей сих, и не прикасайтесь ни к чему, что принадлежит им, чтобы не погибнуть вам [вместе] во всех грехах их.
\vs Num 16:27 И отошли они со всех сторон от жилища Корея, Дафана и Авирона; а Дафан и Авирон вышли и стояли у дверей шатров своих с женами своими и сыновьями своими и с малыми детьми своими.
\vs Num 16:28 И сказал Моисей: из сего узнаете, что Господь послал меня делать все дела сии, а не по своему произволу [я делаю сие]:
\vs Num 16:29 если они умрут, как умирают все люди, и постигнет их такое наказание, какое \bibemph{постигает} всех людей, то не Господь послал меня;
\vs Num 16:30 а если Господь сотворит необычайное, и земля разверзет уста свои и поглотит их [и домы их и шатры их] и все, что у них, и они живые сойдут в преисподнюю, то знайте, что люди сии презрели Господа.
\vs Num 16:31 Лишь только он сказал слова сии, расселась земля под ними;
\vs Num 16:32 и разверзла земля уста свои, и поглотила их и домы их, и всех людей Кореевых и все имущество;
\vs Num 16:33 и сошли они со всем, что принадлежало им, живые в преисподнюю, и покрыла их земля, и погибли они из среды общества.
\vs Num 16:34 И все Израильтяне, которые были вокруг них, побежали при их вопле, дабы, говорили они, и нас не поглотила земля.
\vs Num 16:35 И вышел огонь от Господа и пожрал тех двести пятьдесят мужей, которые принесли курение.
\rsbpar\vs Num 16:36 И сказал Господь Моисею, говоря:
\vs Num 16:37 скажи Елеазару, сыну Аарона, священнику, пусть он соберет [медные] кадильницы сожженных и огонь выбросит вон; ибо освятились
\vs Num 16:38 кадильницы грешников сих смертью их, и пусть разобьют их в листы для покрытия жертвенника, ибо они принесли их пред лице Господа, и они сделались освященными; и будут они знамением для сынов Израилевых.
\vs Num 16:39 И взял Елеазар священник медные кадильницы, которые принесли сожженные, и разбили их в листы для покрытия жертвенника,
\vs Num 16:40 в память сынам Израилевым, чтобы никто посторонний, который не от семени Аарона, не приступал приносить курение пред лице Господне, и не было с ним, что с Кореем и сообщниками его, как говорил ему Господь чрез Моисея.
\rsbpar\vs Num 16:41 На другой день все общество сынов Израилевых возроптало на Моисея и Аарона и говорило: вы умертвили народ Господень.
\vs Num 16:42 И когда собралось общество против Моисея и Аарона, они обратились к скинии собрания, и вот, облако покрыло ее, и явилась слава Господня.
\vs Num 16:43 И пришел Моисей и Аарон к скинии собрания.
\rsbpar\vs Num 16:44 И сказал Господь Моисею [и Аарону], говоря:
\vs Num 16:45 отсторонитесь от общества сего, и Я погублю их во мгновение. Но они пали на лица свои.
\vs Num 16:46 И сказал Моисей Аарону: возьми кадильницу и положи в нее огня с жертвенника и всыпь курения, и неси скорее к обществу и заступи их, ибо вышел гнев от Господа, [и] началось поражение.
\vs Num 16:47 И взял Аарон, как сказал Моисей, и побежал в среду общества, и вот, уже началось поражение в народе. И он положил курения и заступил народ;
\vs Num 16:48 стал он между мертвыми и живыми, и поражение прекратилось.
\vs Num 16:49 И умерло от поражения четырнадцать тысяч семьсот человек, кроме умерших по делу Корееву.
\vs Num 16:50 И возвратился Аарон к Моисею, ко входу скинии собрания, после того как поражение прекратилось.
\vs Num 17:1 И сказал Господь Моисею, говоря:
\vs Num 17:2 скажи сынам Израилевым и возьми у них по жезлу от колена, от всех начальников их по коленам, двенадцать жезлов, и каждого имя напиши на жезле его;
\vs Num 17:3 имя Аарона напиши на жезле Левиином, ибо один жезл от начальника колена их [должны они дать];
\vs Num 17:4 и положи их в скинии собрания, пред \bibemph{ковчегом} откровения, где являюсь Я вам;
\vs Num 17:5 и кого Я изберу, того жезл расцветет; и так Я успокою ропот сынов Израилевых, которым они ропщут на вас.
\rsbpar\vs Num 17:6 И сказал Моисей сынам Израилевым, и дали ему все начальники их, от каждого начальника по жезлу, по коленам их двенадцать жезлов, и жезл Ааронов был среди жезлов их.
\vs Num 17:7 И положил Моисей жезлы пред лицем Господа в скинии откровения.
\vs Num 17:8 На другой день вошел Моисей [и Аарон] в скинию откровения, и вот, жезл Ааронов, от дома Левиина, расцвел, пустил почки, дал цвет и принес миндали.
\vs Num 17:9 И вынес Моисей все жезлы от лица Господня ко всем сынам Израилевым. И увидели они это и взяли каждый свой жезл.
\rsbpar\vs Num 17:10 И сказал Господь Моисею: положи опять жезл Ааронов пред \bibemph{ковчегом} откровения на сохранение, в знамение для непокорных, чтобы прекратился ропот их на Меня, и они не умирали.
\vs Num 17:11 Моисей сделал это; как повелел ему Господь, так он и сделал.
\vs Num 17:12 И сказали сыны Израилевы Моисею: вот, мы умираем, погибаем, все погибаем!
\vs Num 17:13 всякий, приближающийся к скинии Господней, умирает: не придется ли всем нам умереть?
\vs Num 18:1 И сказал Господь Аарону: ты и сыны твои и дом отца твоего с тобою понесете на себе грех за \bibemph{небрежность во} святилище; и ты и сыны твои с тобою понесете на себе грех за \bibemph{неисправность} в священстве вашем.
\vs Num 18:2 Также и братьев твоих, колено Левиино, племя отца твоего, возьми себе: пусть они будут при тебе и служат тебе, а ты и сыны твои с тобою \bibemph{будете} при скинии откровения;
\vs Num 18:3 пусть они отправляют службу тебе и службу во всей скинии; только чтобы не приступали к вещам святилища и к жертвеннику, дабы не умереть и им и вам.
\vs Num 18:4 Пусть они будут при тебе и отправляют службу в скинии собрания, все работы по скинии; а посторонний не должен приближаться к вам.
\vs Num 18:5 Так отправляйте службу во святилище и при жертвеннике, дабы не было впредь гнева на сынов Израилевых;
\vs Num 18:6 ибо братьев ваших, левитов, Я взял от сынов Израилевых и дал их вам, в дар Господу, для отправления службы при скинии собрания;
\vs Num 18:7 и ты и сыны твои с тобою наблюдайте священство ваше во всем, что принадлежит жертвеннику и что внутри за завесою, и служите; вам даю Я в дар службу священства, а посторонний, приступивший, предан будет смерти.
\rsbpar\vs Num 18:8 И сказал Господь Аарону: вот, Я поручаю тебе наблюдать за возношениями Мне; от всего, посвящаемого сынами Израилевыми, Я дал тебе и сынам твоим, ради священства вашего, уставом вечным;
\vs Num 18:9 вот, что принадлежит тебе из святынь великих, от сожигаемого: всякое приношение их хлебное, и всякая жертва их за грех, и всякая жертва их повинности, что они принесут Мне; это великая святыня тебе и сынам твоим.
\vs Num 18:10 На святейшем месте ешьте это; все мужеского пола могут есть, [ты и сыны твои]; это святынею да будет для тебя.
\vs Num 18:11 И вот, что тебе из возношений даров их: все возношения сынов Израилевых Я дал тебе и сынам твоим и дочерям твоим с тобою, уставом вечным; всякий чистый в доме твоем может есть это.
\vs Num 18:12 Все лучшее из елея и все лучшее из винограда и хлеба, начатки их, которые они дают Господу, Я отдал тебе;
\vs Num 18:13 все первые произведения земли их, которые они принесут Господу, да будут твоими; всякий чистый в доме твоем может есть это.
\vs Num 18:14 Все заклятое в земле Израилевой да будет твоим.
\vs Num 18:15 Все, разверзающее ложесна у всякой плоти, которую приносят Господу, из людей и из скота, да будет твоим; только первенец из людей должен быть выкуплен, и первородное из скота нечистого должно быть выкуплено;
\vs Num 18:16 а выкуп за них: начиная от одного месяца, по оценке твоей, бери выкуп пять сиклей серебра, по сиклю священному, который в двадцать гер;
\vs Num 18:17 но за первородное из волов, и за первородное из овец, и за первородное из коз, не бери выкупа: они святыня; кровью их окропляй жертвенник, и тук их сожигай в жертву, в приятное благоухание Господу;
\vs Num 18:18 мясо же их тебе принадлежит, равно как грудь возношения и правое плечо тебе принадлежит.
\vs Num 18:19 Все возносимые святыни, которые возносят сыны Израилевы Господу, отдаю тебе и сынам твоим и дочерям твоим с тобою, уставом вечным; это завет соли вечный пред Господом, данный для тебя и потомства твоего с тобою.
\rsbpar\vs Num 18:20 И сказал Господь Аарону: в земле их не будешь иметь удела и части не будет тебе между ними; Я часть твоя и удел твой среди сынов Израилевых;
\vs Num 18:21 а сынам Левия, вот, Я дал в удел десятину из всего, что у Израиля, за службу их, за то, что они отправляют службы в скинии собрания;
\vs Num 18:22 и сыны Израилевы не должны впредь приступать к скинии собрания, чтобы не понести греха и не умереть:
\vs Num 18:23 пусть левиты исправляют службы в скинии собрания и несут на себе грех их. Это устав вечный в роды ваши; среди же сынов Израилевых они не получат удела;
\vs Num 18:24 так как десятину сынов Израилевых, которую они приносят в возношение Господу, Я отдаю левитам в удел, потому и сказал Я им: между сынами Израилевыми они не получат удела.
\rsbpar\vs Num 18:25 И сказал Господь Моисею, говоря:
\vs Num 18:26 объяви левитам и скажи им: когда вы будете брать от сынов Израилевых десятину, которую Я дал вам от них в удел, то возносите из нее возношение Господу, десятину из десятины,~---
\vs Num 18:27 и вменено будет вам это возношение ваше, как хлеб с гумна и как взятое от точила;
\vs Num 18:28 так и вы будете возносить возношение Господу из всех десятин ваших, которые будете брать от сынов Израилевых, и будете давать из них возношение Господне Аарону священнику;
\vs Num 18:29 из всего, даруемого вам, возносите возношение Господу, из всего лучшего освящаемого.
\vs Num 18:30 И скажи им: когда вы принесете из сего лучшее, то это вменено будет левитам, как получаемое с гумна и получаемое от точила;
\vs Num 18:31 вы можете есть это на всяком месте, вы и [сыны ваши и] семейства ваши, ибо это вам плата за работы ваши в скинии собрания;
\vs Num 18:32 и не понесете за это греха, когда принесете лучшее из сего; и посвящаемого сынами Израилевыми не оскверните, и не умрете.
\vs Num 19:1 И сказал Господь Моисею и Аарону, говоря:
\vs Num 19:2 вот устав закона, который заповедал Господь, говоря: скажи сынам Израилевым, пусть приведут тебе рыжую телицу без порока, у которой нет недостатка, [и] на которой не было ярма;
\vs Num 19:3 и отдайте ее Елеазару священнику, и выведет ее вон из стана [на место чистое], и заколют ее при нем;
\vs Num 19:4 и пусть возьмет Елеазар священник перстом своим крови ее и кровью покропит к передней стороне скинии собрания семь раз;
\vs Num 19:5 и сожгут телицу при его глазах: кожу ее и мясо ее и кровь ее с нечистотою ее пусть сожгут;
\vs Num 19:6 и пусть возьмет священник кедрового дерева и иссопа и нить из червленой шерсти и бросит на сожигаемую телицу;
\vs Num 19:7 и пусть вымоет священник одежды свои, и омоет тело свое водою, и потом войдет в стан, и нечист будет священник до вечера.
\vs Num 19:8 И сожигавший ее пусть вымоет одежды свои водою, и омоет тело свое водою, и нечист будет до вечера;
\vs Num 19:9 и кто-нибудь чистый пусть соберет пепел телицы и положит вне стана на чистом месте, и будет он сохраняться для общества сынов Израилевых, для воды очистительной: это жертва за грех;
\vs Num 19:10 и собиравший пепел телицы пусть вымоет одежды свои, и нечист будет до вечера. Это для сынов Израилевых и для пришельцев, живущих у них, да будет уставом вечным.
\vs Num 19:11 Кто прикоснется к мертвому телу какого-либо человека, нечист будет семь дней:
\vs Num 19:12 он должен очистить себя сею [водою] в третий день и в седьмой день, и будет чист; если же он не очистит себя в третий и седьмой день, то не будет чист;
\vs Num 19:13 всякий, прикоснувшийся к мертвому телу какого-либо человека умершего и не очистивший себя, осквернит жилище Господа: истребится человек тот из среды Израиля, ибо он не окроплен очистительною водою, он нечист, еще нечистота его на нем.
\vs Num 19:14 Вот закон: если человек умрет в шатре, то всякий, кто придет в шатер, и все, что в шатре, нечисто будет семь дней;
\vs Num 19:15 всякий открытый сосуд, который не обвязан и не покрыт, нечист.
\vs Num 19:16 Всякий, кто прикоснется на поле к убитому мечом, или к умершему, или к кости человеческой, или ко гробу, нечист будет семь дней.
\vs Num 19:17 Для нечистого пусть возьмут пепла той сожженной жертвы за грех и нальют на него живой воды в сосуд;
\vs Num 19:18 и пусть кто-нибудь чистый возьмет иссоп, и омочит его в воде, и окропит шатер и все сосуды и людей, которые находятся в нем, и прикоснувшегося к кости [человеческой], или к убитому, или к умершему, или ко гробу;
\vs Num 19:19 и пусть окропит чистый нечистого в третий и седьмой день, и очистит его в седьмой день; и вымоет он одежды свои, и омоет [тело свое] водою, и к вечеру будет чист.
\vs Num 19:20 Если же кто будет нечист и не очистит себя, то истребится человек тот из среды народа, ибо он осквернил святилище Господа; очистительною водою он не окроплен, он нечист.
\vs Num 19:21 И да будет это для них уставом вечным. И кропивший очистительною водою пусть вымоет одежды свои; и прикоснувшийся к очистительной воде нечист будет до вечера.
\vs Num 19:22 И все, к чему прикоснется нечистый, будет нечисто; и прикоснувшийся человек нечист будет до вечера.
\vs Num 20:1 И пришли сыны Израилевы, все общество, в пустыню Син в первый месяц, и остановился народ в Кадесе, и умерла там Мариам и погребена там.
\vs Num 20:2 И не было воды для общества, и собрались они против Моисея и Аарона;
\vs Num 20:3 и возроптал народ на Моисея и сказал: о, если бы умерли тогда и мы, когда умерли братья наши пред Господом!
\vs Num 20:4 зачем вы привели общество Господне в эту пустыню, чтобы умереть здесь нам и скоту нашему?
\vs Num 20:5 и для чего вывели вы нас из Египта, чтобы привести нас на это негодное место, где нельзя сеять, нет ни смоковниц, ни винограда, ни гранатовых яблок, ни даже воды для питья?
\vs Num 20:6 И пошел Моисей и Аарон от народа ко входу скинии собрания, и пали на лица свои, и явилась им слава Господня.
\rsbpar\vs Num 20:7 И сказал Господь Моисею, говоря:
\vs Num 20:8 Возьми жезл и собери общество, ты и Аарон, брат твой, и скажите в глазах их скале, и она даст из себя воду: и так ты изведешь им воду из скалы, и напоишь общество и скот его.
\vs Num 20:9 И взял Моисей жезл от лица Господа, как Он повелел ему.
\vs Num 20:10 И собрали Моисей и Аарон народ к скале, и сказал он им: послушайте, непокорные, разве нам из этой скалы извести для вас воду?
\vs Num 20:11 И поднял Моисей руку свою и ударил в скалу жезлом своим дважды, и потекло много воды, и пило общество и скот его.
\rsbpar\vs Num 20:12 И сказал Господь Моисею и Аарону: за то, что вы не поверили Мне, чтоб явить святость Мою пред очами сынов Израилевых, не введете вы народа сего в землю, которую Я даю ему.
\vs Num 20:13 Это вода Меривы, у которой вошли в распрю сыны Израилевы с Господом, и Он явил им святость Свою.
\rsbpar\vs Num 20:14 И послал Моисей из Кадеса послов к царю Едомскому [сказать]: так говорит брат твой Израиль: ты знаешь все трудности, которые постигли нас;
\vs Num 20:15 отцы наши перешли в Египет, и мы жили в Египте много времени, и худо поступали Египтяне с нами и отцами нашими;
\vs Num 20:16 и воззвали мы к Господу, и услышал Он голос наш, и послал Ангела, и вывел нас из Египта; и вот, мы в Кадесе, городе у самого предела твоего;
\vs Num 20:17 позволь нам пройти землею твоею: мы не пойдем по полям и по виноградникам и не будем пить воды из колодезей [твоих]; но пойдем дорогою царскою, не своротим ни направо ни налево, доколе не перейдем пределов твоих.
\vs Num 20:18 Но Едом сказал ему: не проходи через меня, иначе я с мечом выступлю против тебя.
\vs Num 20:19 И сказали ему сыны Израилевы: мы пойдем большою дорогою, и если будем пить твою воду, я и скот мой, то буду платить за нее; только ногами моими пройду, что ничего не ст\acc{о}ит.
\vs Num 20:20 Но он сказал: не проходи [через меня]. И выступил против него Едом с многочисленным народом и с сильною рукою.
\vs Num 20:21 Итак не согласился Едом позволить Израилю пройти чрез его пределы, и Израиль пошел в сторону от него.
\vs Num 20:22 И отправились сыны Израилевы из Кадеса, и пришло все общество к горе Ор.
\rsbpar\vs Num 20:23 И сказал Господь Моисею и Аарону на горе Ор, у пределов земли Едомской, говоря:
\vs Num 20:24 пусть приложится Аарон к народу своему; ибо он не войдет в землю, которую Я даю сынам Израилевым, за то, что вы непокорны были повелению Моему у вод Меривы;
\vs Num 20:25 и возьми Аарона [брата твоего] и Елеазара, сына его, и возведи их на гору Ор [пред всем обществом];
\vs Num 20:26 и сними с Аарона одежды его, и облеки в них Елеазара, сына его, и пусть Аарон отойдет и умрет там.
\vs Num 20:27 И сделал Моисей так, как повелел Господь. Пошли они на гору Ор в глазах всего общества,
\vs Num 20:28 и снял Моисей с Аарона одежды его, и облек в них Елеазара, сына его; и умер там Аарон на вершине горы. А Моисей и Елеазар сошли с горы.
\vs Num 20:29 И увидело все общество, что Аарон умер, и оплакивал Аарона весь дом Израилев тридцать дней.
\vs Num 21:1 Ханаанский царь Арада, живущий к югу, услышав, что Израиль идет дорогою от Афарима, вступил в сражение с Израильтянами и несколько из них взял в плен.
\vs Num 21:2 И дал Израиль обет Господу, и сказал: если предашь народ сей в руки мои, то положу заклятие [на них и] на города их.
\vs Num 21:3 Господь услышал голос Израиля и предал Хананеев в руки ему, и он положил заклятие на них и на города их и нарек имя месту тому: Хорма\fns{Заклятие.}.
\rsbpar\vs Num 21:4 От горы Ор отправились они путем Чермного моря, чтобы миновать землю Едома. И стал малодушествовать народ на пути,
\vs Num 21:5 и говорил народ против Бога и против Моисея: зачем вывели вы нас из Египта, чтоб умереть [нам] в пустыне, ибо \bibemph{здесь} нет ни хлеба, ни воды, и душе нашей опротивела эта негодная пища.
\vs Num 21:6 И послал Господь на народ ядовитых змеев, которые жалили народ, и умерло множество народа из [сынов] Израилевых.
\vs Num 21:7 И пришел народ к Моисею и сказал: согрешили мы, что говорили против Господа и против тебя; помолись Господу, чтоб Он удалил от нас змеев. И помолился Моисей [Господу] о народе.
\rsbpar\vs Num 21:8 И сказал Господь Моисею: сделай себе [медного] змея и выставь его на знамя, и [если ужалит змей какого-либо человека], ужаленный, взглянув на него, останется жив.
\vs Num 21:9 И сделал Моисей медного змея и выставил его на знамя, и когда змей ужалил человека, он, взглянув на медного змея, оставался жив.
\rsbpar\vs Num 21:10 И отправились сыны Израилевы и остановились в Овофе;
\vs Num 21:11 и отправились из Овофа и остановились в Ийе-Авариме, в пустыне, что против Моава, к восходу солнца;
\vs Num 21:12 оттуда отправились, и остановились на долине Заред;
\vs Num 21:13 отправившись отсюда, остановились у той части Арнона в пустыне, которая течет вне пределов Аморрея, ибо Арнон граница Моава, между Моавом и Аморреем.
\vs Num 21:14 Потому и сказано в книге браней Господних:
\vs Num 21:15 Вагеб в Суфе и потоки Арнона, и верховье потоков, которое склоняется к Шебет-Ару и прилегает к пределам Моава.
\vs Num 21:16 Отсюда [отправились] к Беэр\fns{Колодец.}; это тот колодезь, о котором Господь сказал Моисею: собери народ, и дам им воды.
\vs Num 21:17 Тогда воспел Израиль песнь сию: наполняйся, колодезь, пойте ему;
\vs Num 21:18 колодезь, который выкопали князья, вырыли вожди народа с законодателем жезлами своими. Из пустыни [отправились] в Матанну,
\vs Num 21:19 из Матанны в Нагалиил, из Нагалиила в Вамоф,
\vs Num 21:20 из Вамофа в Гай, который в земле Моава, на вершине \bibemph{горы} Фасги, обращенной лицем к пустыне.
\vs Num 21:21 И послал Израиль послов к Сигону, царю Аморрейскому, [с предложением мирным,] чтобы сказать:
\vs Num 21:22 позволь мне пройти землею твоею; [мы пойдем дорогою,] не будем заходить в поля и виноградники, не будем пить воды из колодезей [твоих], а пойдем путем царским, доколе не перейдем пределов твоих.
\vs Num 21:23 Но Сигон не позволил Израилю идти через свои пределы; и собрал Сигон весь народ свой и выступил против Израиля в пустыню, и дошел до Иаацы, и сразился с Израилем.
\vs Num 21:24 И поразил его Израиль мечом и взял во владение землю его от Арнона до Иавока, до \bibemph{пределов} Аммонитских, ибо крепок был предел Аммонитян;
\vs Num 21:25 и взял Израиль все города сии, и жил Израиль во всех городах Аморрейских, в Есевоне и во всех зависящих от него;
\vs Num 21:26 ибо Есевон был город Сигона, царя Аморрейского, и он воевал с прежним царем Моавитским и взял из руки его всю землю его до Арнона.
\vs Num 21:27 Потому говорят пр\acc{и}точники: идите в Есевон, да устроят и утвердят город Сигона;
\vs Num 21:28 ибо огонь вышел из Есевона, пламень из города Сигонова, и пожрал Ар-Моав и владеющих высотами Арнона.
\vs Num 21:29 Горе тебе, Моав! погиб ты, народ Хамоса! Разбежались сыновья его, и дочери его сделались пленницами Аморрейского царя Сигона;
\vs Num 21:30 мы поразили их стрелами; погиб Есевон до Дивона, мы опустошили их до Нофы, которая близ Медевы.
\vs Num 21:31 И жил Израиль в земле Аморрейской.
\vs Num 21:32 И послал Моисей высмотреть Иазер, и взяли [его и] селения, зависящие от него, и прогнали Аморреев, которые в них были.
\vs Num 21:33 И поворотили и пошли к Васану. И выступил против них Ог, царь Васанский, сам и весь народ его, на сражение к Едреи.
\rsbpar\vs Num 21:34 И сказал Господь Моисею: не бойся его, ибо Я предам его и весь народ его и всю землю его в руки твои, и поступишь с ним, как поступил с Сигоном, царем Аморрейским, который жил в Есевоне.
\vs Num 21:35 И поразили они его и сынов его и весь народ его, так что ни одного не осталось [живого], и овладели землею его.
\vs Num 22:1 И отправились сыны Израилевы, и остановились на равнинах Моава, при Иордане, против Иерихона.
\rsbpar\vs Num 22:2 И видел Валак, сын Сепфоров, все, что сделал Израиль Аморреям;
\vs Num 22:3 и весьма боялись Моавитяне народа сего, потому что он был многочислен; и устрашились Моавитяне сынов Израилевых.
\vs Num 22:4 И сказали Моавитяне старейшинам Мадиамским: этот народ поедает теперь все вокруг нас, как вол поедает траву полевую. Валак же, сын Сепфоров, был царем Моавитян в то время.
\vs Num 22:5 И послал он послов к Валааму, сыну Веорову, в Пефор, который на реке \bibemph{Евфрате}, в земле сынов народа его, чтобы позвать его \bibemph{и} сказать: вот, народ вышел из Египта и покрыл лице земли, и живет он подле меня;
\vs Num 22:6 итак приди, прокляни мне народ сей, ибо он сильнее меня: может быть, я тогда буду в состоянии поразить его и выгнать его из земли; я знаю, что кого ты благословишь, тот благословен, и кого ты проклянешь, тот проклят.
\vs Num 22:7 И пошли старейшины Моавитские и старейшины Мадиамские, с подарками в руках за волхвование, и пришли к Валааму, и пересказали ему слова Валаковы.
\vs Num 22:8 И сказал он им: переночуйте здесь ночь, и дам вам ответ, как скажет мне Господь. И остались старейшины Моавитские у Валаама.
\vs Num 22:9 И пришел Бог к Валааму и сказал: какие это люди у тебя?
\vs Num 22:10 Валаам сказал Богу: Валак, сын Сепфоров, царь Моавитский, прислал [их] ко мне [сказать]:
\vs Num 22:11 вот, народ вышел из Египта и покрыл лице земли, [и живет подле меня]; итак приди, прокляни мне его; может быть я тогда буду в состоянии сразиться с ним и выгнать его [из земли].
\vs Num 22:12 И сказал Бог Валааму: не ходи с ними, не проклинай народа сего, ибо он благословен.
\vs Num 22:13 И встал Валаам поутру и сказал князьям Валаковым: пойдите в землю вашу, ибо не хочет Господь позволить мне идти с вами.
\vs Num 22:14 И встали князья Моавитские, и пришли к Валаку, и сказали [ему]: не согласился Валаам идти с нами.
\rsbpar\vs Num 22:15 Валак послал еще князей, более и знаменитее тех.
\vs Num 22:16 И пришли они к Валааму и сказали ему: так говорит Валак, сын Сепфоров: не откажись прийти ко мне;
\vs Num 22:17 я окажу тебе великую почесть и сделаю [тебе] все, что ни скажешь мне; приди же, прокляни мне народ сей.
\vs Num 22:18 И отвечал Валаам и сказал рабам Валаковым: хотя бы Валак давал мне полный свой дом серебра и золота, не могу преступить повеления Господа, Бога моего, и сделать что-либо малое или великое [по своему произволу];
\vs Num 22:19 впрочем, останьтесь здесь и вы на ночь, и я узнаю, что еще скажет мне Господь.
\vs Num 22:20 И пришел Бог к Валааму ночью и сказал ему: если люди сии пришли звать тебя, встань, пойди с ними; но только делай то, что Я буду говорить тебе.
\vs Num 22:21 Валаам встал поутру, оседлал ослицу свою и пошел с князьями Моавитскими.
\vs Num 22:22 И воспылал гнев Божий за то, что он пошел, и стал Ангел Господень на дороге, чтобы воспрепятствовать ему. Он ехал на ослице своей и с ним двое слуг его.
\vs Num 22:23 И увидела ослица Ангела Господня, стоящего на дороге с обнаженным мечом в руке, и своротила ослица с дороги, и пошла на поле; а Валаам стал бить ослицу, чтобы возвратить ее на дорогу.
\vs Num 22:24 И стал Ангел Господень на узкой дороге, между виноградниками, \bibemph{где} с одной стороны стена и с другой стороны стена.
\vs Num 22:25 Ослица, увидев Ангела Господня, прижалась к стене и прижала ногу Валаамову к стене; и он опять стал бить ее.
\vs Num 22:26 Ангел Господень опять перешел и стал в тесном месте, где некуда своротить, ни направо, ни налево.
\vs Num 22:27 Ослица, увидев Ангела Господня, легла под Валаамом. И воспылал гнев Валаама, и стал он бить ослицу палкою.
\vs Num 22:28 И отверз Господь уста ослицы, и она сказала Валааму: что я тебе сделала, что ты бьешь меня вот уже третий раз?
\vs Num 22:29 Валаам сказал ослице: за то, что ты поругалась надо мною; если бы у меня в руке был меч, то я теперь же убил бы тебя.
\vs Num 22:30 Ослица же сказала Валааму: не я ли твоя ослица, на которой ты ездил сначала до сего дня? имела ли я привычку так поступать с тобою? Он сказал: нет.
\vs Num 22:31 И открыл Господь глаза Валааму, и увидел он Ангела Господня, стоящего на дороге с обнаженным мечом в руке, и преклонился, и пал на лице свое.
\vs Num 22:32 И сказал ему Ангел Господень: за что ты бил ослицу твою вот уже три раза? Я вышел, чтобы воспрепятствовать [тебе], потому что путь [твой] не прав предо Мною;
\vs Num 22:33 и ослица, видев Меня, своротила от Меня вот уже три раза; если бы она не своротила от Меня, то Я убил бы тебя, а ее оставил бы живою.
\vs Num 22:34 И сказал Валаам Ангелу Господню: согрешил я, ибо не знал, что Ты стоишь против меня на дороге; итак, если это неприятно в очах Твоих, то я возвращусь.
\vs Num 22:35 И сказал Ангел Господень Валааму: пойди с людьми сими, только говори то, что Я буду говорить тебе. И пошел Валаам с князьями Валаковыми.
\rsbpar\vs Num 22:36 Валак, услышав, что идет Валаам, вышел навстречу ему в город Моавитский, который на границе при Арноне, что у самого предела.
\vs Num 22:37 И сказал Валак Валааму: не посылал ли я к тебе, звать тебя? почему ты не шел ко мне? неужели я в самом деле не могу почтить тебя?
\vs Num 22:38 И сказал Валаам Валаку: вот, я и пришел к тебе, но могу ли я что \bibemph{от себя} сказать? что вложит Бог в уста мои, то и буду говорить.
\vs Num 22:39 И пошел Валаам с Валаком и пришли в Кириаф-Хуцоф.
\vs Num 22:40 И заколол Валак волов и овец, и послал к Валааму и князьям, которые были с ним.
\vs Num 22:41 На другой день утром Валак взял Валаама и возвел его на высоты Вааловы, чтобы он увидел оттуда часть народа.
\vs Num 23:1 И сказал Валаам Валаку: построй мне здесь семь жертвенников и приготовь мне семь тельцов и семь овнов.
\vs Num 23:2 Валак сделал так, как говорил Валаам, и вознесли Валак и Валаам по тельцу и по овну на каждом жертвеннике.
\vs Num 23:3 И сказал Валаам Валаку: постой у всесожжения твоего, а я пойду; может быть, Господь выйдет мне навстречу, и что Он откроет мне, я объявлю тебе. И [остался Валак у всесожжения своего, а Валаам] пошел на возвышенное место [вопросить Бога].
\vs Num 23:4 И встретился Бог с Валаамом, и сказал Ему [Валаам]: семь жертвенников устроил я и вознес по тельцу и по овну на каждом жертвеннике.
\vs Num 23:5 И вложил Господь слово в уста Валаамовы и сказал: возвратись к Валаку и так говори.
\vs Num 23:6 И возвратился к нему, и вот, он стоит у всесожжения своего, он и все князья Моавитские. [И был на нем Дух Божий.]
\vs Num 23:7 И произнес притчу свою и сказал: из Месопотамии привел меня Валак, царь Моава, от гор восточных: приди, прокляни мне Иакова, приди, изреки зло на Израиля!
\vs Num 23:8 Как прокляну я? Бог не проклинает его. Как изреку зло? Господь не изрекает [на него] зла.
\vs Num 23:9 С вершины скал вижу я его, и с холмов смотрю на него: вот, народ живет отдельно и между народами не числится.
\vs Num 23:10 Кто исчислит песок Иакова и число четвертой части Израиля? Да умрет душа моя смертью праведников, и да будет кончина моя, как их!
\vs Num 23:11 И сказал Валак Валааму: что ты со мною делаешь? я взял тебя, чтобы проклясть врагов моих, а ты, вот, благословляешь?
\vs Num 23:12 И отвечал он и сказал: не должен ли я в точности сказать то, что влагает Господь в уста мои?
\vs Num 23:13 И сказал ему Валак: пойди со мною на другое место, с которого ты увидишь его, но только часть его увидишь, а всего его не увидишь; и прокляни мне его оттуда.
\vs Num 23:14 И взял его на место стражей, на вершину \bibemph{горы} Фасги, и построил семь жертвенников, и вознес по тельцу и по овну на каждом жертвеннике.
\vs Num 23:15 И сказал [Валаам] Валаку: постой здесь у всесожжения твоего, а я [пойду] туда навстречу [Богу].
\vs Num 23:16 И встретился Господь с Валаамом, и вложил слово в уста его, и сказал: возвратись к Валаку и так говори.
\vs Num 23:17 И пришел к нему, и вот, он стоит у всесожжения своего, и с ним [все] князья Моавитские. И сказал ему Валак: что говорил Господь?
\vs Num 23:18 Он произнес притчу свою и сказал: встань, Валак, и послушай, внимай мне, сын Сепфоров.
\vs Num 23:19 Бог не человек, чтоб Ему лгать, и не сын человеческий, чтоб Ему изменяться. Он ли скажет и не сделает? будет говорить и не исполнит?
\vs Num 23:20 Вот, благословлять начал я, ибо Он благословил, и я не могу изменить сего.
\vs Num 23:21 Не видно бедствия в Иакове, и не заметно несчастья в Израиле; Господь, Бог его, с ним, и трубный царский звук у него;
\vs Num 23:22 Бог вывел их из Египта, быстрота единорога у него;
\vs Num 23:23 нет волшебства в Иакове и нет ворожбы в Израиле. В \bibemph{свое} время скажут об Иакове и об Израиле: вот что творит Бог!
\vs Num 23:24 Вот, народ как львица встает и как лев поднимается; не ляжет, пока не съест добычи и не напьется крови убитых.
\vs Num 23:25 И сказал Валак Валааму: ни клясть не кляни его, ни благословлять не благословляй его.
\vs Num 23:26 И отвечал Валаам и сказал Валаку: не говорил ли я тебе, что я буду делать все то, что скажет мне Господь?
\vs Num 23:27 И сказал Валак Валааму: пойди, я возьму тебя на другое место; может быть, угодно будет Богу, и оттуда проклянешь мне его.
\vs Num 23:28 И взял Валак Валаама на верх Фегора, обращенного к пустыне.
\vs Num 23:29 И сказал Валаам Валаку: построй мне здесь семь жертвенников и приготовь мне здесь семь тельцов и семь овнов.
\vs Num 23:30 И сделал Валак, как сказал Валаам, и вознес по тельцу и овну на каждом жертвеннике.
\vs Num 24:1 Валаам увидел, что Господу угодно благословлять Израиля, и не пошел, как прежде, для волхвования, но обратился лицем своим к пустыне.
\vs Num 24:2 И взглянул Валаам и увидел Израиля, стоявшего по коленам своим, и был на нем Дух Божий.
\vs Num 24:3 И произнес он притчу свою и сказал: говорит Валаам, сын Веоров, говорит муж с открытым оком,
\vs Num 24:4 говорит слышащий слова Божии, который видит видения Всемогущего; падает, но открыты глаза его:
\vs Num 24:5 как прекрасны шатры твои, Иаков, жилища твои, Израиль!
\vs Num 24:6 расстилаются они как долины, как сады при реке, как алойные дерева, насажденные Господом, как кедры при водах;
\vs Num 24:7 польется вода из ведр его, и семя его \bibemph{будет} как великие воды, превзойдет Агага царь его и возвысится царство его.
\vs Num 24:8 Бог вывел его из Египта, быстрота единорога у него, пожирает народы, враждебные ему, раздробляет кости их и стрелами своими разит [врага].
\vs Num 24:9 Преклонился, лежит как лев и как львица, кто поднимет его? Благословляющий тебя благословен, и проклинающий тебя проклят!
\rsbpar\vs Num 24:10 И воспламенился гнев Валака на Валаама, и всплеснул он руками своими, и сказал Валак Валааму: я призвал тебя проклясть врагов моих, а ты благословляешь их вот уже третий раз;
\vs Num 24:11 итак, беги в свое место; я хотел почтить тебя, но вот, Господь лишает тебя чести.
\vs Num 24:12 И сказал Валаам Валаку: не говорил ли я послам твоим, которых ты присылал ко мне:
\vs Num 24:13 <<хотя бы давал мне Валак полный свой дом серебра и золота, не могу преступить повеления Господня, чтобы сделать что-либо доброе или худое по своему произволу: что скажет Господь, то и буду говорить>>?
\vs Num 24:14 Итак, вот, я иду к народу своему; пойди, я возвещу тебе, что сделает народ сей с народом твоим в последствие времени.
\vs Num 24:15 И произнес притчу свою и сказал: говорит Валаам, сын Веоров, говорит муж с открытым оком,
\vs Num 24:16 говорит слышащий слова Божии, имеющий ведение от Всевышнего, который видит видения Всемогущего, падает, но открыты очи его.
\vs Num 24:17 Вижу Его, но ныне еще нет; зрю Его, но не близко. Восходит звезда от Иакова и восстает жезл от Израиля, и разит князей Моава и сокрушает всех сынов Сифовых.
\vs Num 24:18 Едом будет под владением, Сеир будет под владением врагов своих, а Израиль явит силу \bibemph{свою}.
\vs Num 24:19 \bibemph{Происшедший} от Иакова овладеет и погубит оставшееся от города.
\vs Num 24:20 И увидел он Амалика, и произнес притчу свою, и сказал: первый из народов Амалик, но конец его~--- гибель.
\vs Num 24:21 И увидел он Кенеев, и произнес притчу свою, и сказал: крепко жилище твое, и на скале положено гнездо твое;
\vs Num 24:22 но разорен будет Каин, и недолго до того, что Ассур уведет тебя в плен.
\vs Num 24:23 И [увидев Ога,] произнес притчу свою, и сказал: горе, [горе,] кто уцелеет, когда наведет сие Бог!
\vs Num 24:24 \bibemph{придут} корабли от Киттима, и смирят Ассура, и смирят Евера; но и им гибель!
\vs Num 24:25 И встал Валаам и пошел обратно в свое место, а Валак также пошел своею дорогою.
\vs Num 25:1 И жил Израиль в Ситтиме, и начал народ блудодействовать с дочерями Моава,
\vs Num 25:2 и приглашали они народ к жертвам богов своих, и ел народ [жертвы их] и кланялся богам их.
\vs Num 25:3 И прилепился Израиль к Ваал-Фегору. И воспламенился гнев Господень на Израиля.
\vs Num 25:4 И сказал Господь Моисею: возьми всех начальников народа и повесь их Господу перед солнцем, и отвратится от Израиля ярость гнева Господня.
\vs Num 25:5 И сказал Моисей судьям Израилевым: убейте каждый людей своих, прилепившихся к Ваал-Фегору.
\rsbpar\vs Num 25:6 И вот, некто из сынов Израилевых пришел и привел к братьям своим Мадианитянку, в глазах Моисея и в глазах всего общества сынов Израилевых, когда они плакали у входа скинии собрания.
\vs Num 25:7 Финеес, сын Елеазара, сына Аарона священника, увидев это, встал из среды общества и взял в руку свою копье,
\vs Num 25:8 и вошел вслед за Израильтянином в спальню и пронзил обоих их, Израильтянина и женщину в чрево ее: и прекратилось поражение сынов Израилевых.
\vs Num 25:9 Умерших же от поражения было двадцать четыре тысячи.
\rsbpar\vs Num 25:10 И сказал Господь Моисею, говоря:
\vs Num 25:11 Финеес, сын Елеазара, сына Аарона священника, отвратил ярость Мою от сынов Израилевых, возревновав по Мне среди их, и Я не истребил сынов Израилевых в ревности Моей;
\vs Num 25:12 посему скажи: вот, Я даю ему Мой завет мира,
\vs Num 25:13 и будет он ему и потомству его по нем заветом священства вечного, за то, что он показал ревность по Боге своем и заступил сынов Израилевых.
\vs Num 25:14 Имя убитого Израильтянина, который убит с Мадианитянкою, было Зимри, сын Салу, начальник поколения Симеонова;
\vs Num 25:15 а имя убитой Мадианитянки Хазва; она была дочь Цура, начальника Оммофа, племени Мадиамского.
\rsbpar\vs Num 25:16 И сказал Господь Моисею, говоря:
\vs Num 25:17 враждуйте с Мадианитянами, и поражайте их,
\vs Num 25:18 ибо они враждебно поступили с вами в коварстве своем, прельстив вас Фегором и Хазвою, дочерью начальника Мадиамского, сестрою своею, убитою в день поражения за Фегора.
\vs Num 26:1 После сего поражения сказал Господь Моисею и Елеазару, сыну Аарона, священнику, говоря:
\vs Num 26:2 исчислите все общество сынов Израилевых от двадцати лет и выше, по семействам их, всех годных для войны у Израиля.
\vs Num 26:3 И сказал им Моисей и Елеазар священник на равнинах Моавитских у Иордана, против Иерихона, говоря:
\vs Num 26:4 \bibemph{исчислите всех} от двадцати лет и выше, как повелел Господь Моисею и сынам Израилевым, которые вышли из земли Египетской:
\vs Num 26:5 Рувим, первенец Израиля. Сыны Рувима: от Ханоха поколение Ханохово, от Фаллу поколение Фаллуево,
\vs Num 26:6 от Хецрона поколение Хецроново, от Харми поколение Хармиево;
\vs Num 26:7 вот поколения Рувимовы; и исчислено их сорок три тысячи семьсот тридцать.
\vs Num 26:8 И сыны Фаллуя: Елиав.
\vs Num 26:9 Сыны Елиава: Немуил, Дафан и Авирон. Это те Дафан и Авирон, призываемые в собрание, которые произвели возмущение против Моисея и Аарона вместе с сообщниками Корея, когда сии произвели возмущение против Господа;
\vs Num 26:10 и разверзла земля уста свои, и поглотила их и Корея; вместе с \bibemph{ними} умерли и сообщники их, когда огонь пожрал двести пятьдесят человек, и стали они в знамение;
\vs Num 26:11 но сыны Кореевы не умерли.
\vs Num 26:12 Сыны Симеона по поколениям их: от Немуила поколение Немуилово, от Ямина поколение Яминово, от Яхина поколение Яхиново,
\vs Num 26:13 от Зары поколение Зарино, от Саула поколение Саулово;
\vs Num 26:14 вот поколения Симеоновы [при исчислении их]: двадцать две тысячи двести.
\vs Num 26:15 Сыны Гада по поколениям их: от Цефона поколение Цефоново, от Хаггия поколение Хаггиево, от Шуния поколение Шуниево,
\vs Num 26:16 от Озния поколение Озниево, от Ерия поколение Ериево,
\vs Num 26:17 от Арода поколение Ародово, от Арелия поколение Арелиево;
\vs Num 26:18 вот поколения сынов Гадовых, по исчислению их: сорок тысяч пятьсот.
\vs Num 26:19 Сыны Иуды: Ир и Онан, [Шела, Фарес и Зара]; но Ир и Онан умерли в земле Ханаанской;
\vs Num 26:20 и были сыны Иуды по поколениям их: от Шелы поколение Шелино, от Фареса поколение Фаресово, от Зары поколение Зарино;
\vs Num 26:21 и были сыны Фаресовы: от Есрома поколение Есромово, от Хамула поколение Хамулово;
\vs Num 26:22 вот поколения Иудины, по исчислению их: семьдесят шесть тысяч пятьсот.
\vs Num 26:23 Сыны Иссахаровы по поколениям их: от Фолы поколение Фолино, от Фувы поколение Фувино,
\vs Num 26:24 от Иашува поколение Иашувово, от Шимрона поколение Шимроново;
\vs Num 26:25 вот поколения Иссахаровы, по исчислению их: шестьдесят четыре тысячи триста.
\vs Num 26:26 Сыны Завулона по поколениям их: от Середа поколение Середово, от Елона поколение Елоново, от Иахлеила поколение Иахлеилово;
\vs Num 26:27 вот поколения Завулоновы, по исчислению их: шестьдесят тысяч пятьсот.
\vs Num 26:28 Сыны Иосифа по поколениям их: Манассия и Ефрем.
\vs Num 26:29 Сыны Манассии: от Махира поколение Махирово; от Махира родился Галаад, от Галаада поколение Галаадово.
\vs Num 26:30 Вот сыны Галаадовы: от Иезера поколение Иезерово, от Хелека поколение Хелеково,
\vs Num 26:31 от Асриила поколение Асриилово, от Шехема поколение Шехемово,
\vs Num 26:32 от Шемиды поколение Шемидино, от Хефера поколение Хеферово.
\vs Num 26:33 У Салпаада, сына Хеферова, не было сыновей, а только дочери; имя дочерей Салпаадовых: Махла, Ноа, Хогла, Милка и Фирца.
\vs Num 26:34 Вот поколения Манассиины; а исчислено их пятьдесят две тысячи семьсот.
\vs Num 26:35 Вот сыны Ефремовы по поколениям их: от Шутелы поколение Шутелино, от Бехера поколение Бехерово, от Тахана поколение Таханово;
\vs Num 26:36 и вот сыны Шутелы: от Арана поколение Араново;
\vs Num 26:37 вот поколения сынов Ефремовых, по исчислению их: тридцать две тысячи пятьсот. Вот сыны Иосифовы по поколениям их.
\vs Num 26:38 Сыны Вениамина по поколениям их: от Белы поколение Белино, от Ашбела поколение Ашбелово, от Ахирама поколение Ахирамово,
\vs Num 26:39 от Шефуфама поколение Шефуфамово, от Хуфама поколение Хуфамово;
\vs Num 26:40 и были сыны Белы: Ард и Нааман; [от Арда] поколение Ардово, от Наамана поколение Нааманово;
\vs Num 26:41 вот сыны Вениамина по поколениям их; а исчислено их сорок пять тысяч шестьсот.
\vs Num 26:42 Вот сыны Дановы по поколениям их: от Шухама поколение Шухамово; вот семейства Дановы по поколениям их.
\vs Num 26:43 И всех поколений Шухама, по исчислению их: шестьдесят четыре тысячи четыреста.
\vs Num 26:44 Сыны Асировы по поколениям их: от Имны поколение Имнино, от Ишвы поколение Ишвино, от Верии поколение Вериино;
\vs Num 26:45 от сынов Верии, от Хевера поколение Хеверово, от Малхиила поколение Малхиилово;
\vs Num 26:46 имя дочери Асировой Сара;
\vs Num 26:47 вот поколения сынов Асировых, по исчислению их: пятьдесят три тысячи четыреста.
\vs Num 26:48 Сыны Неффалима по поколениям их: от Иахцеила поколение Иахцеилово, от Гуния поколение Гуниево,
\vs Num 26:49 от Иецера поколение Иецерово, от Шиллема поколение Шиллемово;
\vs Num 26:50 вот поколения Неффалимовы по поколениям их; исчислено же их сорок пять тысяч четыреста.
\vs Num 26:51 Вот \bibemph{число} вошедших в исчисление сынов Израилевых: шестьсот одна тысяча семьсот тридцать.
\rsbpar\vs Num 26:52 И сказал Господь Моисею, говоря:
\vs Num 26:53 сим в удел должно разделить землю по числу имен;
\vs Num 26:54 кто многочисленнее, тем дай удел более; а кто малочисленнее, тем дай удел менее: каждому должно дать удел соразмерно с числом вошедших в исчисление;
\vs Num 26:55 по жребию должно разделить землю, по именам колен отцов их должны они получить уделы;
\vs Num 26:56 по жребию должно разделить им уделы их, как многочисленным, так и малочисленным.
\rsbpar\vs Num 26:57 Сии суть вошедшие в исчисление левиты по поколениям их: от Гирсона поколение Гирсоново, от Каафа поколение Каафово, от Мерари поколение Мерарино.
\vs Num 26:58 Вот поколения Левиины: поколение Ливниево, поколение Хевроново, поколение Махлиево, поколение Мушиево, поколение Кореево. От Каафа родился Амрам.
\vs Num 26:59 Имя жены Амрамовой Иохаведа, дочь Левиина, которую родила \bibemph{жена} Левиина в Египте, а она Амраму родила Аарона, Моисея и Мариам, сестру их.
\vs Num 26:60 И родились у Аарона Надав и Авиуд, Елеазар и Ифамар;
\vs Num 26:61 но Надав и Авиуд умерли, когда принесли чуждый огонь пред Господа [в пустыне Синайской].
\vs Num 26:62 И было исчислено двадцать три тысячи всех мужеского пола, от одного месяца и выше; ибо они не были исчислены вместе с сынами Израилевыми, потому что не дано им удела среди сынов Израилевых.
\rsbpar\vs Num 26:63 Вот исчисленные Моисеем и Елеазаром священником, которые исчисляли сынов Израилевых на равнинах Моавитских у Иордана, против Иерихона;
\vs Num 26:64 в числе их не было ни одного человека из исчисленных Моисеем и Аароном священником, которые исчисляли сынов Израилевых в пустыне Синайской;
\vs Num 26:65 ибо Господь сказал им, что умрут они в пустыне,~--- и не осталось из них никого, кроме Халева, сына Иефонниина, и Иисуса, сына Навина.
\vs Num 27:1 И пришли дочери Салпаада, сына Хеферова, сына Галаадова, сына Махирова, сына Манассиина из поколения Манассии, сына Иосифова, и вот имена дочерей его: Махла, Ноа, Хогла, Милка и Фирца;
\vs Num 27:2 и предстали пред Моисея и пред Елеазара священника, и пред князей и пред все общество, у входа скинии собрания, и сказали:
\vs Num 27:3 отец наш умер в пустыне, и он не был в числе сообщников, собравшихся против Господа со скопищем Кореевым, но за свой грех умер, и сыновей у него не было;
\vs Num 27:4 за что исчезать имени отца нашего из племени его, потому что нет у него сына? дай нам удел среди братьев отца нашего.
\vs Num 27:5 И представил Моисей дело их Господу.
\rsbpar\vs Num 27:6 И сказал Господь Моисею:
\vs Num 27:7 правду говорят дочери Салпаадовы; дай им наследственный удел среди братьев отца их и передай им удел отца их;
\vs Num 27:8 и сынам Израилевым объяви и скажи: если кто умрет, не имея у себя сына, то передавайте удел его дочери его;
\vs Num 27:9 если же нет у него дочери, передавайте удел его братьям его;
\vs Num 27:10 если же нет у него братьев, отдайте удел его братьям отца его;
\vs Num 27:11 если же нет братьев отца его, отдайте удел его близкому его родственнику из поколения его, чтоб он наследовал его; и да будет это для сынов Израилевых постановлено в закон, как повелел Господь Моисею.
\rsbpar\vs Num 27:12 И сказал Господь Моисею: взойди на сию гору Аварим, [которая по эту сторону Иордана, на сию гору Нево,] и посмотри на землю [Ханаанскую], которую Я даю сынам Израилевым [во владение];
\vs Num 27:13 и когда посмотришь на нее, приложись к народу своему и ты, как приложился Аарон, брат твой, [на горе Ор];
\vs Num 27:14 потому что вы не послушались повеления Моего в пустыне Син, во время распри общества, чтоб явить пред глазами их святость Мою при водах. [Это воды Меривы при Кадесе в пустыне Син.]
\vs Num 27:15 И сказал Моисей Господу, говоря:
\vs Num 27:16 да поставит Господь, Бог духов всякой плоти, над обществом сим человека,
\vs Num 27:17 который выходил бы пред ними и который входил бы пред ними, который выводил бы их и который приводил бы их, чтобы не осталось общество Господне, как овцы, у которых нет пастыря.
\vs Num 27:18 И сказал Господь Моисею: возьми себе Иисуса, сына Навина, человека, в котором есть Дух, и возложи на него руку твою,
\vs Num 27:19 и поставь его пред Елеазаром священником и пред всем обществом, и дай ему наставление пред глазами их,
\vs Num 27:20 и дай ему от славы твоей, чтобы слушало его все общество сынов Израилевых;
\vs Num 27:21 и будет он обращаться к Елеазару священнику и спрашивать его о решении, посредством урима пред Господом; и по его слову должны выходить, и по его слову должны входить он и все сыны Израилевы с ним и все общество.
\vs Num 27:22 И сделал Моисей, как повелел ему Господь [Бог], и взял Иисуса, и поставил его пред Елеазаром священником и пред всем обществом;
\vs Num 27:23 и возложил на него руки свои и дал ему наставление, как говорил Господь чрез Моисея.
\vs Num 28:1 И сказал Господь Моисею, говоря:
\vs Num 28:2 повели сынам Израилевым и скажи им: наблюдайте, чтобы приношение Мое, хлеб Мой в жертву Мне, в приятное благоухание Мне, приносимо было Мне в свое время.
\vs Num 28:3 И скажи им: вот жертва, которую вы должны приносить Господу: два агнца однолетних без порока на день, во всесожжение постоянное;
\vs Num 28:4 одного агнца приноси утром, а другого агнца приноси вечером;
\vs Num 28:5 и в приношение хлебное [приноси] десятую часть \bibemph{ефы} пшеничной муки, смешанной с четвертью гина выбитого елея;
\vs Num 28:6 это~--- всесожжение постоянное, какое совершено было при горе Синае, в приятное благоухание, в жертву Господу;
\vs Num 28:7 и возлияния при ней четверть гина на одного агнца: на святом месте возливай возлияние, вино Господу.
\vs Num 28:8 Другого агнца приноси вечером, с таким хлебным приношением, как поутру, и с таким же возлиянием при нем приноси его в жертву, в приятное благоухание Господу.
\vs Num 28:9 А в субботу [приносите] двух агнцев однолетних без порока, и в приношение хлебное две десятых части \bibemph{ефы} пшеничной муки, смешанной с елеем, и возлияние при нем:
\vs Num 28:10 это~--- субботнее всесожжение в каждую субботу, сверх постоянного всесожжения и возлияния при нем.
\vs Num 28:11 И в новомесячия ваши приносите всесожжение Господу: из крупного скота двух тельцов, одного овна и семь однолетних агнцев без порока,
\vs Num 28:12 и три десятых части \bibemph{ефы} пшеничной муки, смешанной с елеем, в приношение хлебное на одного тельца, и две десятых части \bibemph{ефы} пшеничной муки, смешанной с елеем, в приношение хлебное на овна,
\vs Num 28:13 и по десятой части \bibemph{ефы} пшеничной муки, смешанной с елеем, в приношение хлебное на каждого агнца; \bibemph{это}~--- всесожжение, приятное благоухание, жертва Господу;
\vs Num 28:14 и возлияния при них должно быть полгина вина на тельца, треть гина на овна и четверть гина на агнца; это всесожжение в каждое новомесячие \bibemph{во все} месяцы года.
\vs Num 28:15 И одного козла приносите Господу в жертву за грех; сверх всесожжения постоянного должно приносить его с возлиянием его.
\rsbpar\vs Num 28:16 В первый месяц, в четырнадцатый день месяца~--- Пасха Господня.
\vs Num 28:17 И в пятнадцатый день сего месяца праздник; семь дней должно есть опресноки.
\vs Num 28:18 В первый день [да будет у вас] священное собрание; никакой работы не работайте;
\vs Num 28:19 и приносите жертву, всесожжение Господу: из крупного скота двух тельцов, одного овна и семь однолетних агнцев; без порока они должны быть у вас;
\vs Num 28:20 и при них в приношение хлебное приносите пшеничной муки, смешанной с елеем, три десятых части \bibemph{ефы} на каждого тельца, и две десятых части \bibemph{ефы} на овна,
\vs Num 28:21 и по десятой части \bibemph{ефы} приноси на каждого из семи агнцев,
\vs Num 28:22 и одного козла в жертву за грех, для очищения вас;
\vs Num 28:23 сверх утреннего всесожжения, которое есть всесожжение постоянное, приносите сие.
\vs Num 28:24 Так приносите и в каждый из семи дней; \bibemph{это} хлеб, жертва, приятное благоухание Господу; сверх всесожжения постоянного и возлияния его, должно приносить \bibemph{сие}.
\vs Num 28:25 И в седьмой день да будет у вас священное собрание; никакой работы не работайте.
\vs Num 28:26 И в день первых плодов, когда приносите Господу новое приношение хлебное в седмицы ваши, да будет у вас священное собрание; никакой работы не работайте;
\vs Num 28:27 и приносите всесожжение в приятное благоухание Господу: из крупного скота двух тельцов, одного овна и семь однолетних агнцев [без порока],
\vs Num 28:28 и при них в приношение хлебное пшеничной муки, смешанной с елеем, три десятых части \bibemph{ефы} на каждого тельца, две десятых части \bibemph{ефы} на овна,
\vs Num 28:29 и по десятой части \bibemph{ефы} на каждого из семи агнцев,
\vs Num 28:30 и одного козла [в жертву за грех], для очищения вас;
\vs Num 28:31 сверх постоянного всесожжения и хлебного приношения при нем, приносите [сие Мне] с возлиянием их; без порока должны быть они у вас.
\vs Num 29:1 И в седьмой месяц, в первый [день] месяца, да будет у вас священное собрание; никакой работы не работайте; пусть будет \bibemph{это} у вас день трубного звука;
\vs Num 29:2 и приносите всесожжение в приятное благоухание Господу: одного тельца, одного овна, семь однолетних агнцев, без порока,
\vs Num 29:3 и при них в приношение хлебное пшеничной муки, смешанной с елеем, три десятых части \bibemph{ефы} на тельца, две десятых части \bibemph{ефы} на овна,
\vs Num 29:4 и одну десятую часть \bibemph{ефы} на каждого из семи агнцев,
\vs Num 29:5 и одного козла в жертву за грех, для очищения вас,
\vs Num 29:6 сверх новомесячного всесожжения и хлебного приношения его, и \bibemph{сверх} постоянного всесожжения и хлебного приношения его, и возлияний их, по уставу, в приятное благоухание Господу.
\vs Num 29:7 И в десятый [день] сего седьмого месяца пусть будет у вас священное собрание: смиряйте \bibemph{тогда} души ваши и никакого дела не делайте;
\vs Num 29:8 и приносите всесожжение Господу в приятное благоухание: одного тельца, одного овна, семь однолетних агнцев; без порока пусть будут они у вас;
\vs Num 29:9 и при них в приношение хлебное пшеничной муки, смешанной с елеем, три десятых части \bibemph{ефы} на тельца, две десятых части \bibemph{ефы} на овна,
\vs Num 29:10 и по десятой части \bibemph{ефы} на каждого из семи агнцев,
\vs Num 29:11 и одного козла в жертву за грех, [для очищения вас,] сверх жертвы за грех, \bibemph{приносимой в день} очищения, и \bibemph{сверх} всесожжения постоянного и хлебного приношения его, и возлияния их, [по уставу \bibemph{приносимых} в приятное благоухание, в жертву Господу].
\vs Num 29:12 И в пятнадцатый день седьмого месяца пусть будет у вас священное собрание; никакой работы не работайте и празднуйте праздник Господень семь дней;
\vs Num 29:13 и приносите всесожжение, жертву, приятное благоухание Господу: тринадцать тельцов, двух овнов, четырнадцать однолетних агнцев; без порока пусть будут они;
\vs Num 29:14 и при них в приношение хлебное пшеничной муки, смешанной с елеем, три десятых части \bibemph{ефы} на каждого из тринадцати тельцов, две десятых части \bibemph{ефы} на каждого из двух овнов,
\vs Num 29:15 и по десятой части \bibemph{ефы} на каждого из четырнадцати агнцев,
\vs Num 29:16 и одного козла в жертву за грех, сверх всесожжения постоянного и хлебного приношения его и возлияния его.
\vs Num 29:17 И во второй день двенадцать тельцов, двух овнов, четырнадцать однолетних агнцев, без порока,
\vs Num 29:18 и при них приношение хлебное и возлияние для тельцов, овнов и агнцев, по числу их, по уставу,
\vs Num 29:19 и одного козла в жертву за грех, сверх всесожжения постоянного и хлебного приношения и возлияния их.
\vs Num 29:20 И в третий день одиннадцать тельцов, двух овнов, четырнадцать однолетних агнцев, без порока,
\vs Num 29:21 и при них приношение хлебное и возлияние для тельцов, овнов и агнцев, по числу их, по уставу,
\vs Num 29:22 и одного козла в жертву за грех, сверх всесожжения постоянного и хлебного приношения и возлияния его.
\vs Num 29:23 И в четвертый день десять тельцов, двух овнов, четырнадцать однолетних агнцев, без порока,
\vs Num 29:24 и при них приношение хлебное и возлияние для тельцов, овнов и агнцев, по числу их, по уставу,
\vs Num 29:25 и одного козла в жертву за грех, сверх всесожжения постоянного и хлебного приношения и возлияния его.
\vs Num 29:26 И в пятый день девять тельцов, двух овнов, четырнадцать однолетних агнцев, без порока,
\vs Num 29:27 и при них приношение хлебное и возлияние для тельцов, овнов и агнцев, по числу их, по уставу,
\vs Num 29:28 и одного козла в жертву за грех, сверх всесожжения постоянного и хлебного приношения и возлияния его.
\vs Num 29:29 И в шестой день восемь тельцов, двух овнов, четырнадцать однолетних агнцев, без порока,
\vs Num 29:30 и при них приношение хлебное и возлияние для тельцов, овнов и агнцев, по числу их, по уставу,
\vs Num 29:31 и одного козла в жертву за грех, сверх всесожжения постоянного и хлебного приношения и возлияния его.
\vs Num 29:32 И в седьмой день семь тельцов, двух овнов, четырнадцать однолетних агнцев, без порока,
\vs Num 29:33 и при них приношение хлебное и возлияние для тельцов, овнов и агнцев, по числу их, по уставу,
\vs Num 29:34 и одного козла в жертву за грех, сверх всесожжения постоянного и хлебного приношения и возлияния его.
\vs Num 29:35 В восьмой день пусть будет у вас отдание праздника; никакой работы не работайте;
\vs Num 29:36 и приносите всесожжение, жертву, приятное благоухание Господу: одного тельца, одного овна, семь однолетних агнцев, без порока,
\vs Num 29:37 и при них приношение хлебное и возлияние для тельца, овна и агнцев по числу их, по уставу,
\vs Num 29:38 и одного козла в жертву за грех, сверх всесожжения постоянного и приношения хлебного и возлияния его.
\vs Num 29:39 Приносите это Господу в праздники ваши, сверх \bibemph{приносимых} вами, по обету или по усердию, всесожжений ваших и хлебных приношений ваших, и возлияний ваших и мирных жертв ваших.
\vs Num 30:1 И пересказал Моисей сынам Израилевым все, что повелел Господь Моисею.
\vs Num 30:2 И сказал Моисей начальникам колен сынов Израилевых, говоря: вот что повелел Господь:
\vs Num 30:3 если кто даст обет Господу, или поклянется клятвою, положив зарок на душу свою, то он не должен нарушать слова своего, но должен исполнить все, что вышло из уст его.
\vs Num 30:4 Если женщина даст обет Господу и положит \bibemph{на себя} зарок в доме отца своего, в юности своей,
\vs Num 30:5 и услышит отец обет ее и зарок, который она положила на душу свою, и промолчит о том отец ее, то все обеты ее состоятся, и всякий зарок ее, который она положила на душу свою, состоится;
\vs Num 30:6 если же отец ее, услышав, запретит ей, то все обеты ее и зароки, которые она возложила на душу свою, не состоятся, и Господь простит ей, потому что запретил ей отец ее.
\vs Num 30:7 Если она выйдет в замужество, а на ней обет ее, или слово уст ее, которым она связала себя,
\vs Num 30:8 и услышит муж ее и, услышав, промолчит: то обеты ее состоятся, и зароки ее, которые она возложила на душу свою, состоятся;
\vs Num 30:9 если же муж ее, услышав, запретит ей и отвергнет обет ее, который на ней, и слово уст ее, которым она связала себя, [то они не состоятся, потому что запретил ей муж ее,] и Господь простит ей.
\vs Num 30:10 Обет же вдовы и разведенной, какой бы она ни возложила зарок на душу свою, состоится.
\vs Num 30:11 Если \bibemph{жена} в доме мужа своего дала обет, или возложила зарок на душу свою с клятвою,
\vs Num 30:12 и муж ее слышал, и промолчал о том, и не запретил ей, то все обеты ее состоятся, и всякий зарок, который она возложила на душу свою, состоится;
\vs Num 30:13 если же муж ее, услышав, отвергнул их, то все вышедшие из уст ее обеты ее и зароки души ее не состоятся: муж ее уничтожил их, и Господь простит ей.
\vs Num 30:14 Всякий обет и всякий клятвенный зарок, чтобы смирить душу, муж ее может утвердить, и муж ее может отвергнуть;
\vs Num 30:15 если же муж ее молчал о том день за день, то он \bibemph{тем} утвердил все обеты ее и все зароки ее, которые на ней, утвердил, потому что он, услышав, молчал о том;
\vs Num 30:16 а если [муж] отвергнул их, после того как услышал, то он взял на себя грех ее.
\rsbpar\vs Num 30:17 Вот уставы, которые Господь заповедал Моисею об отношении между мужем и женою его, между отцом и дочерью его в юности ее, в доме отца ее.
\vs Num 31:1 И сказал Господь Моисею, говоря:
\vs Num 31:2 отмсти Мадианитянам за сынов Израилевых, и после отойдешь к народу твоему.
\vs Num 31:3 И сказал Моисей народу, говоря: вооружите из себя людей на войну, чтобы они пошли против Мадианитян, совершить мщение Господне над Мадианитянами;
\vs Num 31:4 по тысяче из колена, от всех колен [сынов] Израилевых пошлите на войну.
\vs Num 31:5 И выделено из тысяч Израилевых, по тысяче из колена, двенадцать тысяч вооруженных на войну.
\vs Num 31:6 И послал их Моисей на войну, по тысяче из колена, их и Финееса, сына Елеазара, [сына Аарона,] священника, на войну, и в руке его священные сосуды и трубы для тревоги.
\vs Num 31:7 И пошли войною на Мадиама, как повелел Господь Моисею, и убили всех мужеского пола;
\vs Num 31:8 и вместе с убитыми их убили царей Мадиамских: Евия, Рекема, Цура, Хура и Реву, пять царей Мадиамских, и Валаама, сына Веорова, убили мечом [вместе с убитыми их];
\vs Num 31:9 а жен Мадиамских и детей их сыны Израилевы взяли в плен, и весь скот их, и все стада их и все имение их взяли в добычу,
\vs Num 31:10 и все города их во владениях их и все селения их сожгли огнем;
\vs Num 31:11 и взяли все захваченное и всю добычу, от человека до скота;
\vs Num 31:12 и доставили пленных и добычу и захваченное к Моисею и к Елеазару священнику и к обществу сынов Израилевых, к стану, на равнины Моавитские, что у Иордана, против Иерихона.
\vs Num 31:13 И вышли Моисей и Елеазар священник и все князья общества навстречу им из стана.
\vs Num 31:14 И прогневался Моисей на военачальников, тысяченачальников и стоначальников, пришедших с войны,
\vs Num 31:15 и сказал им Моисей: [для чего] вы оставили в живых всех женщин?
\vs Num 31:16 вот они, по совету Валаамову, были для сынов Израилевых поводом к отступлению от Господа в угождение Фегору, \bibemph{за что} и поражение было в обществе Господнем;
\vs Num 31:17 итак убейте всех детей мужеского пола, и всех женщин, познавших мужа на мужеском ложе, убейте;
\vs Num 31:18 а всех детей женского пола, которые не познали мужеского ложа, оставьте в живых для себя;
\vs Num 31:19 и пробудьте вне стана семь дней; всякий, убивший человека и прикоснувшийся к убитому, очиститесь в третий день и в седьмой день, вы и пленные ваши;
\vs Num 31:20 и все одежды, и все кожаные вещи, и все сделанное из козьей \bibemph{шерсти}, и все деревянные сосуды очистите.
\vs Num 31:21 И сказал Елеазар священник воинам, ходившим на войну: вот постановление закона, который заповедал Господь Моисею:
\vs Num 31:22 золото, серебро, медь, железо, олово и свинец,
\vs Num 31:23 и все, что проходит через огонь, проведите через огонь, чтоб оно очистилось, а кроме того и очистительною водою должно очистить; все же, что не проходит через огонь, проведите через воду;
\vs Num 31:24 и одежды ваши вымойте в седьмой день, и очиститесь, и после того входите в стан.
\rsbpar\vs Num 31:25 И сказал Господь Моисею, говоря:
\vs Num 31:26 сочти добычу плена, от человека до скота, ты и Елеазар священник и начальники племен общества;
\vs Num 31:27 и раздели добычу пополам между воевавшими, ходившими на войну, и между всем обществом;
\vs Num 31:28 и от воинов, ходивших на войну, возьми дань Господу, по одной душе из пятисот, из людей и из крупного скота, и из ослов и из мелкого скота;
\vs Num 31:29 возьми это из половины их и отдай Елеазару священнику в возношение Господу;
\vs Num 31:30 и из половины сынов Израилевых возьми по одной доле из пятидесяти, из людей, из крупного скота, из ослов и из мелкого скота, и отдай это левитам, служащим при скинии Господней.
\vs Num 31:31 И сделал Моисей и Елеазар священник, как повелел Господь Моисею.
\vs Num 31:32 И было добычи, оставшейся от захваченного, что захватили бывшие на войне: мелкого скота шестьсот семьдесят пять тысяч,
\vs Num 31:33 крупного скота семьдесят две тысячи,
\vs Num 31:34 ослов шестьдесят одна тысяча,
\vs Num 31:35 людей, женщин, которые не знали мужеского ложа, всех душ тридцать две тысячи.
\vs Num 31:36 Половина, доля ходивших на войну, по расчислению была: мелкого скота триста тридцать семь тысяч пятьсот,
\vs Num 31:37 и дань Господу из мелкого скота шестьсот семьдесят пять;
\vs Num 31:38 крупного скота тридцать шесть тысяч, и дань из них Господу семьдесят два;
\vs Num 31:39 ослов тридцать тысяч пятьсот, и дань из них Господу шестьдесят один;
\vs Num 31:40 людей шестнадцать тысяч, и дань из них Господу тридцать две души.
\vs Num 31:41 И отдал Моисей дань, возношение Господу, Елеазару священнику, как повелел Господь Моисею.
\vs Num 31:42 И из половины сынов Израилевых, которую отделил Моисей у бывших на войне;
\vs Num 31:43 половина же \bibemph{на долю} общества была: мелкого скота триста тридцать семь тысяч пятьсот,
\vs Num 31:44 крупного скота тридцать шесть тысяч,
\vs Num 31:45 ослов тридцать тысяч пятьсот,
\vs Num 31:46 людей шестнадцать тысяч.
\vs Num 31:47 Из половины сынов Израилевых взял Моисей одну пятидесятую часть из людей и из скота и отдал это левитам, исполняющим службу при скинии Господней, как повелел Господь Моисею.
\vs Num 31:48 И пришли к Моисею начальники над тысячами войска, тысяченачальники и стоначальники,
\vs Num 31:49 и сказали Моисею: рабы твои сосчитали воинов, которые нам поручены, и не убыло ни одного из них;
\vs Num 31:50 и \bibemph{вот}, мы принесли приношение Господу, кто что достал из золотых вещей: цепочки, запястья, перстни, серьги и привески, для очищения душ наших пред Господом.
\vs Num 31:51 И взял у них Моисей и Елеазар священник золото во всех этих изделиях;
\vs Num 31:52 и было всего золота, которое принесено в возношение Господу, шестнадцать тысяч семьсот пятьдесят сиклей, от тысяченачальников и стоначальников.
\vs Num 31:53 Воины грабили каждый для себя.
\vs Num 31:54 И взял Моисей и Елеазар священник золото от тысяченачальников и стоначальников, и принесли его в скинию собрания, в память сынов Израилевых пред Господом.
\vs Num 32:1 У сынов Рувимовых и у сынов Гадовых стад было весьма много; и увидели они, что земля Иазер и земля Галаад есть место \bibemph{годное} для стад;
\vs Num 32:2 и пришли сыны Гадовы и сыны Рувимовы и сказали Моисею и Елеазару священнику и князьям общества, говоря:
\vs Num 32:3 Атароф и Дивон, и Иазер, и Нимра, и Есевон, и Елеале, и Севам, и Нево, и Веон,
\vs Num 32:4 земля, которую Господь поразил пред обществом Израилевым, есть земля \bibemph{годная} для стад, а у рабов твоих есть стада.
\vs Num 32:5 И сказали: если мы нашли благоволение в глазах твоих, отдай землю сию рабам твоим во владение; не переводи нас чрез Иордан.
\vs Num 32:6 И сказал Моисей сынам Гадовым и сынам Рувимовым: братья ваши пойдут на войну, а вы останетесь здесь?
\vs Num 32:7 для чего вы отвращаете сердце сынов Израилевых от перехода в землю, которую дает им Господь?
\vs Num 32:8 так поступили отцы ваши, когда я посылал их из Кадес-Варни для обозрения земли:
\vs Num 32:9 они доходили до долины Есхол, и видели землю, и отвратили сердце сынов Израилевых, чтобы не шли они в землю, которую Господь дает им;
\vs Num 32:10 и воспылал в тот день гнев Господа, и поклялся Он, говоря:
\vs Num 32:11 люди сии, вышедшие из Египта, от двадцати лет и выше [знающие добро и зло,] не увидят земли, о которой Я клялся Аврааму, Исааку и Иакову, потому что они не повиновались Мне,
\vs Num 32:12 кроме Халева, сына Иефонниина, Кенезеянина, и Иисуса, сына Навина, потому что они повиновались Господу.
\vs Num 32:13 И воспылал гнев Господа на Израиля, и водил Он их по пустыне сорок лет, доколе не кончился весь род, сделавший зло в очах Господних.
\vs Num 32:14 И вот, вместо отцов ваших восстали вы, отродье грешников, чтоб усилить еще ярость гнева Господня на Израиля.
\vs Num 32:15 Если вы отвратитесь от Него, то Он опять оставит его в пустыне, и вы погубите весь народ сей.
\vs Num 32:16 И подошли они к нему и сказали: мы построим здесь овчие дворы для стад наших и города для детей наших;
\vs Num 32:17 сами же мы первые вооружимся и пойдем пред сынами Израилевыми, доколе не приведем их в места их; а дети наши пусть останутся в укрепленных городах, \bibemph{для безопасности} от жителей земли;
\vs Num 32:18 не возвратимся в домы наши, доколе не вступят сыны Израилевы каждый в удел свой;
\vs Num 32:19 ибо мы не возьмем с ними удела по ту сторону Иордана и далее, если удел нам достанется по эту сторону Иордана, к востоку.
\vs Num 32:20 И сказал им Моисей: если вы это сделаете, если вооруженные пойдете на войну пред Господом,
\vs Num 32:21 и пойдет каждый из вас вооруженный за Иордан пред Господом, доколе не истребит Он врагов Своих пред Собою,
\vs Num 32:22 и покорена будет земля пред Господом, то после возвратитесь и будете неповинны пред Господом и пред Израилем, и будет земля сия у вас во владении пред Господом;
\vs Num 32:23 если же не сделаете так, то согрешите пред Господом, и испытаете \bibemph{наказание} за грех ваш, которое постигнет вас;
\vs Num 32:24 стройте себе города для детей ваших и дворы для овец ваших и делайте, что произнесено устами вашими.
\vs Num 32:25 И сказали сыны Гадовы и сыны Рувимовы Моисею: рабы твои сделают, как повелевает господин наш;
\vs Num 32:26 дети наши, жены наши, стада наши и весь скот наш останутся тут в городах Галаада,
\vs Num 32:27 а рабы твои, все, вооружившись, как воины, пойдут пред Господом на войну, как говорит господин наш.
\vs Num 32:28 И дал Моисей о них повеление Елеазару священнику и Иисусу, сыну Навину, и начальникам племен сынов Израилевых,
\vs Num 32:29 и сказал им Моисей: если сыны Гадовы и сыны Рувимовы перейдут с вами за Иордан, все вооружившись на войну пред Господом, и покорена будет пред вами земля, то отдайте им землю Галаад во владение;
\vs Num 32:30 если же не пойдут они с вами вооруженные [на войну пред Господом, то пошлите пред собою имение их, жен их и скот их в землю Ханаанскую], и они получат владение вместе с вами в земле Ханаанской.
\vs Num 32:31 И отвечали сыны Гадовы и сыны Рувимовы и сказали: как сказал Господь рабам твоим, так и сделаем;
\vs Num 32:32 мы пойдем вооруженные пред Господом в землю Ханаанскую, а удел владения нашего пусть будет по эту сторону Иордана.
\vs Num 32:33 И отдал Моисей им, сынам Гадовым и сынам Рувимовым, и половине колена Манассии, сына Иосифова, царство Сигона, царя Аморрейского, и царство Ога, царя Васанского, землю с городами ее и окрестностями,~--- города земли во все стороны.
\vs Num 32:34 И построили сыны Гадовы Дивон и Атароф, и Ароер,
\vs Num 32:35 и Атароф-Шофан, и Иазер, и Иогбегу,
\vs Num 32:36 и Беф-Нимру и Беф-Гаран, города укрепленные и дворы для овец.
\vs Num 32:37 И сыны Рувимовы построили Есевон, Елеале, Кириафаим,
\vs Num 32:38 и Нево, и Ваал-Меон, которых имена переменены, и Сивму, и дали имена городам, которые они построили.
\vs Num 32:39 И пошли сыны Махира, сына Манассиина, в Галаад, и взяли его, и выгнали Аморреев, которые были в нем;
\vs Num 32:40 и отдал Моисей Галаад Махиру, сыну Манассии, и он поселился в нем.
\vs Num 32:41 И Иаир, сын Манассии, пошел и взял селения их, и назвал их: селения Иаировы.
\vs Num 32:42 И Новах пошел и взял Кенаф и зависящие от него города, и назвал его своим именем: Новах.
\vs Num 33:1 Вот станы сынов Израилевых, которые вышли из земли Египетской по ополчениям своим, под начальством Моисея и Аарона.
\vs Num 33:2 Моисей, по повелению Господню, описал путешествие их по станам их, и вот станы путешествия их:
\vs Num 33:3 из Раамсеса отправились они в первый месяц, в пятнадцатый день первого месяца; на другой день Пасхи вышли сыны Израилевы под рукою высокою в глазах всего Египта;
\vs Num 33:4 между тем Египтяне хоронили всех первенцев, которых поразил у них Господь, и над богами их Господь совершил суд.
\vs Num 33:5 Так отправились сыны Израилевы из Раамсеса и расположились станом в Сокхофе.
\vs Num 33:6 И отправились из Сокхофа и расположились станом в Ефаме, что на краю пустыни.
\vs Num 33:7 И отправились из Ефама и обратились к Пи-Гахирофу, что пред Ваал-Цефоном, и расположились станом пред Мигдолом.
\vs Num 33:8 Отправившись от Гахирофа, прошли среди моря в пустыню, и шли три дня пути пустынею Ефам, и расположились станом в Мерре.
\vs Num 33:9 И отправились из Мерры и пришли в Елим; в Елиме же [было] двенадцать источников воды и семьдесят финиковых дерев, и расположились там станом.
\vs Num 33:10 И отправились из Елима и расположились станом у Чермного моря.
\vs Num 33:11 И отправились от Чермного моря и расположились станом в пустыне Син.
\vs Num 33:12 И отправились из пустыни Син и расположились станом в Дофке.
\vs Num 33:13 И отправились из Дофки и расположились станом в Алуше.
\vs Num 33:14 И отправились из Алуша и расположились станом в Рефидиме, и не было там воды, чтобы пить народу.
\vs Num 33:15 И отправились из Рефидима и расположились станом в пустыне Синайской.
\vs Num 33:16 И отправились из пустыни Синайской и расположились станом в Киброт-Гаттааве.
\vs Num 33:17 И отправились из Киброт-Гаттаавы и расположились станом в Асирофе.
\vs Num 33:18 И отправились из Асирофа и расположились станом в Рифме.
\vs Num 33:19 И отправились из Рифмы и расположились станом в Римнон-Фареце.
\vs Num 33:20 И отправились из Римнон-Фареца и расположились станом в Ливне.
\vs Num 33:21 И отправились из Ливны и расположились станом в Риссе.
\vs Num 33:22 И отправились из Риссы и расположились станом в Кегелафе.
\vs Num 33:23 И отправились из Кегелафы и расположились станом на горе Шафер.
\vs Num 33:24 И отправились от горы Шафер и расположились станом в Хараде.
\vs Num 33:25 И отправились из Харады и расположились станом в Макелофе.
\vs Num 33:26 И отправились из Макелофа и расположились станом в Тахафе.
\vs Num 33:27 И отправились из Тахафа и расположились станом в Тарахе.
\vs Num 33:28 И отправились из Тараха и расположились станом в Мифке.
\vs Num 33:29 И отправились из Мифки и расположились станом в Хашмоне.
\vs Num 33:30 И отправились из Хашмоны и расположились станом в Мосерофе.
\vs Num 33:31 И отправились из Мосерофа и расположились станом в Бене-Яакане.
\vs Num 33:32 И отправились из Бене-Яакана и расположились станом в Хор-Агидгаде.
\vs Num 33:33 И отправились из Хор-Агидгада и расположились станом в Иотвафе.
\vs Num 33:34 И отправились от Иотвафы и расположились станом в Авроне.
\vs Num 33:35 И отправились из Аврона и расположились станом в Ецион-Гавере.
\vs Num 33:36 И отправились из Ецион-Гавера и расположились станом в пустыне Син. [Отправившись из пустыни Син, расположились станом в пустыне Фаран,] она же Кадес.
\vs Num 33:37 И отправились из Кадеса и расположились станом на горе Ор, у пределов земли Едомской.
\vs Num 33:38 И взошел Аарон священник на гору Ор по повелению Господню и умер там в сороковой год по исшествии сынов Израилевых из земли Египетской, в пятый месяц, в первый день месяца;
\vs Num 33:39 Аарон был ста двадцати трех лет, когда умер на горе Ор.
\rsbpar\vs Num 33:40 Ханаанский царь Арада, который жил к югу земли Ханаанской, услышал тогда, что идут сыны Израилевы.
\vs Num 33:41 И отправились они от горы Ор и расположились станом в Салмоне.
\vs Num 33:42 И отправились из Салмона и расположились станом в Пуноне.
\vs Num 33:43 И отправились из Пунона и расположились станом в Овофе.
\vs Num 33:44 И отправились из Овофа и расположились станом в Ийм-Авариме, на пределах Моава.
\vs Num 33:45 И отправились из Ийма и расположились станом в Дивон-Гаде.
\vs Num 33:46 И отправились из Дивон-Гада и расположились станом в Алмон-Дивлафаиме.
\vs Num 33:47 И отправились из Алмон-Дивлафаима и расположились станом на горах Аваримских пред Нево.
\vs Num 33:48 И отправились от гор Аваримских и расположились станом на равнинах Моавитских у Иордана, против Иерихона;
\vs Num 33:49 они расположились станом у Иордана от Беф-Иешимофа до Аве-Ситтима на равнинах Моавитских.
\rsbpar\vs Num 33:50 И сказал Господь Моисею на равнинах Моавитских у Иордана, против Иерихона, говоря:
\vs Num 33:51 объяви сынам Израилевым и скажи им: когда перейдете через Иордан в землю Ханаанскую,
\vs Num 33:52 то прогон\acc{и}те от себя всех жителей земли и истребите все изображения их, и всех литых идолов их истребите и все высоты их разорите;
\vs Num 33:53 и возьмите во владение землю и поселитесь на ней, ибо Я вам даю землю сию во владение;
\vs Num 33:54 и разделите землю по жребию на уделы племенам вашим: многочисленному дайте удел более, а малочисленному дай удел менее; кому где выйдет жребий, там ему и будет \bibemph{удел}; по коленам отцов ваших возьмите себе уделы;
\vs Num 33:55 если же вы не прогоните от себя жителей земли, то оставшиеся из них будут тернами для глаз ваших и иглами для боков ваших и будут теснить вас на земле, в которой вы будете жить,
\vs Num 33:56 и тогда, что Я вознамерился сделать им, сделаю вам.
\vs Num 34:1 И сказал Господь Моисею, говоря:
\vs Num 34:2 дай повеление сынам Израилевым и скажи им: когда войдете в землю Ханаанскую, то вот земля, которая достанется вам в удел, земля Ханаанская с ее границами:
\vs Num 34:3 южная сторона будет у вас от пустыни Син, подле Едома, и пойдет у вас южная граница от конца Соленого моря с востока,
\vs Num 34:4 и направится граница на юг к возвышенности Акравима и пойдет через Син, и будут выступы ее на юг к Кадес-Варни, оттуда пойдет к Гацар-Аддару и пройдет через Ацмон;
\vs Num 34:5 от Ацмона направится граница к потоку Египетскому, и будут выступы ее к морю;
\vs Num 34:6 а границею западною будет у вас великое море: это будет у вас граница к западу;
\vs Num 34:7 к северу же будет у вас граница: от великого моря проведите ее к горе Ор,
\vs Num 34:8 от горы Ор проведите к Емафу, и будут выступы границы к Цедаду;
\vs Num 34:9 оттуда пойдет граница к Цифрону, и выступы ее будут к Гацар-Енану: это будет у вас граница северная;
\vs Num 34:10 границу восточную проведите себе от Гацар-Енана к Шефаму,
\vs Num 34:11 от Шефама пойдет граница к Рибле, с восточной стороны Аина, потом пойдет граница и коснется берегов моря Киннереф с восточной стороны;
\vs Num 34:12 и пойдет граница к Иордану, и будут выступы ее к Соленому морю. Это будет земля ваша по границам ее со всех сторон.
\vs Num 34:13 И дал повеление Моисей сынам Израилевым и сказал: вот земля, которую вы разделите на уделы по жребию, которую повелел Господь дать девяти коленам и половине колена [Манассиина];
\vs Num 34:14 ибо колено сынов Рувимовых по семействам их, и колено сынов Гадовых по семействам их, и половина колена Манассиина получили удел свой:
\vs Num 34:15 два колена и половина колена получили удел свой за Иорданом против Иерихона к востоку.
\rsbpar\vs Num 34:16 И сказал Господь Моисею, говоря:
\vs Num 34:17 вот имена мужей, которые будут делить вам землю: Елеазар священник и Иисус, сын Навин;
\vs Num 34:18 и по одному князю от колена возьмите для раздела земли.
\vs Num 34:19 И вот имена сих мужей: для колена Иудина Халев, сын Иефонниин;
\vs Num 34:20 для колена сынов Симеоновых Самуил, сын Аммиуда;
\vs Num 34:21 для колена Вениаминова Елидад, сын Кислона;
\vs Num 34:22 для колена сынов Дановых князь Буккий, сын Иоглии;
\vs Num 34:23 для сынов Иосифовых, для колена сынов Манассииных князь Ханниил, сын Ефода;
\vs Num 34:24 для колена сынов Ефремовых князь Кемуил, сын Шифтана;
\vs Num 34:25 для колена сынов Завулоновых князь Елицафан, сын Фарнака;
\vs Num 34:26 для колена сынов Иссахаровых князь Фалтиил, сын Аззана;
\vs Num 34:27 для колена сынов Асировых князь Ахиуд, сын Шеломия;
\vs Num 34:28 для колена сынов Неффалимовых князь Педаил, сын Аммиуда;
\vs Num 34:29 вот те, которым повелел Господь разделить уделы сынам Израилевым в земле Ханаанской.
\vs Num 35:1 И сказал Господь Моисею на равнинах Моавитских у Иордана против Иерихона, говоря:
\vs Num 35:2 повели сынам Израилевым, чтоб они из уделов владения своего дали левитам города для жительства, и поля при городах со всех сторон дайте левитам:
\vs Num 35:3 города будут им для жительства, а поля будут для скота их и для имения их и для всех житейских потребностей их;
\vs Num 35:4 поля при городах, которые вы должны дать левитам, от стены города \bibemph{должны простираться} на [две] тысячи локтей, во все стороны;
\vs Num 35:5 и отмерьте за городом к восточной стороне две тысячи локтей, и к южной стороне две тысячи локтей, и к западу две тысячи локтей, и к северной стороне две тысячи локтей, а посредине город: таковы будут у них поля при городах.
\vs Num 35:6 Из городов, которые вы дадите левитам, [будут] шесть городов для убежища, в которые вы позволите убегать убийце; и сверх их дайте сорок два города:
\vs Num 35:7 всех городов, которые вы должны дать левитам, \bibemph{будет} сорок восемь городов, с полями при них.
\vs Num 35:8 И когда будете давать города из владения сынов Израилевых, тогда из большего дайте более, из меньшего менее; каждое колено, смотря по уделу, какой получит, должно дать из городов своих левитам.
\rsbpar\vs Num 35:9 И сказал Господь Моисею, говоря:
\vs Num 35:10 объяви сынам Израилевым и скажи им: когда вы перейдете чрез Иордан в землю Ханаанскую,
\vs Num 35:11 выберите себе города, которые были бы у вас городами для убежища, куда мог бы убежать убийца, убивший человека неумышленно;
\vs Num 35:12 и будут у вас города сии убежищем от мстителя [за кровь], чтобы не был умерщвлен убивший, прежде нежели он предстанет пред общество на суд.
\vs Num 35:13 Городов же, которые должны вы дать, городов для убежища, должно быть у вас шесть:
\vs Num 35:14 три города дайте по эту сторону Иордана и три города дайте в земле Ханаанской; городами убежища должны быть они;
\vs Num 35:15 для сынов Израилевых и для пришельца и для поселенца между вами будут сии шесть городов убежищем, чтобы убегать туда всякому, убившему человека неумышленно.
\vs Num 35:16 Если кто ударит кого железным орудием так, что тот умрет, то он убийца: убийцу должно предать смерти;
\vs Num 35:17 и если кто ударит кого из руки камнем, от которого можно умереть, так что тот умрет, то он убийца: убийцу должно предать смерти;
\vs Num 35:18 или если деревянным орудием, от которого можно умереть, ударит из руки так, что тот умрет, то он убийца: убийцу должно предать смерти;
\vs Num 35:19 мститель за кровь сам может умертвить убийцу: лишь только встретит его, сам может умертвить его;
\vs Num 35:20 если кто толкнет кого по ненависти, или с умыслом бросит на него \bibemph{что-нибудь} так, что тот умрет,
\vs Num 35:21 или по вражде ударит его рукою так, что тот умрет, то ударившего должно предать смерти: он убийца; мститель за кровь может умертвить убийцу, лишь только встретит его.
\vs Num 35:22 Если же он толкнет его нечаянно, без вражды, или бросит на него что-нибудь без умысла,
\vs Num 35:23 или какой-нибудь камень, от которого можно умереть, не видя уронит на него так, что тот умрет, но он не был врагом его и не желал ему зла,
\vs Num 35:24 то общество должно рассудить между убийцею и мстителем за кровь по сим постановлениям;
\vs Num 35:25 и должно общество спасти убийцу от руки мстителя за кровь, и должно возвратить его общество в город убежища его, куда он убежал, чтоб он жил там до смерти великого священника, который помазан священным елеем;
\vs Num 35:26 если же убийца выйдет за предел города убежища, в который он убежал,
\vs Num 35:27 и найдет его мститель за кровь вне пределов города убежища его, и убьет убийцу сего мститель за кровь, то не будет на нем \bibemph{вины} кровопролития,
\vs Num 35:28 ибо тот должен был жить в городе убежища своего до смерти великого священника, а по смерти великого священника должен был возвратиться убийца в землю владения своего.
\vs Num 35:29 Да будет это у вас постановлением законным в роды ваши, во всех жилищах ваших.
\vs Num 35:30 Если кто убьет человека, то убийцу должно убить по словам свидетелей; но одного свидетеля недостаточно, \bibemph{чтобы осудить} на смерть.
\vs Num 35:31 И не берите выкупа за душу убийцы, который повинен смерти, но его должно предать смерти;
\vs Num 35:32 и не берите выкупа за убежавшего в город убежища, чтоб ему \bibemph{позволить} жить в земле \bibemph{своей} прежде смерти [великого] священника.
\vs Num 35:33 Не оскверняйте земли, на которой вы [будете жить]; ибо кровь оскверняет землю, и земля не иначе очищается от пролитой на ней крови, как кровью пролившего ее.
\vs Num 35:34 Не должно осквернять землю, на которой вы живете, среди которой обитаю Я; ибо Я Господь обитаю среди сынов Израилевых.
\vs Num 36:1 Пришли главы семейств от племени сынов Галаада, сына Махирова, сына Манассиина из племен сынов Иосифовых, и говорили пред Моисеем [и пред Елеазаром священником] и пред князьями, главами поколений сынов Израилевых,
\vs Num 36:2 и сказали: Господь повелел господину нашему дать землю в удел сынам Израилевым по жребию, и господину нашему повелено от Господа дать удел Салпаада, брата нашего, дочерям его;
\vs Num 36:3 если же они будут женами сынов которого-нибудь \bibemph{другого} колена сынов Израилевых, то удел их отнимется от удела отцов наших и прибавится к уделу того колена, в котором они будут [женами], и отнимется от доставшегося по жребию удела нашего;
\vs Num 36:4 и даже когда будет у сынов Израилевых юбилей, тогда удел их прибавится к уделу того колена, в котором они будут [женами], и от удела колена отцов наших отнимется удел их.
\vs Num 36:5 И дал Моисей повеление сынам Израилевым, по слову Господню, и сказал: правду говорит колено сынов Иосифовых;
\vs Num 36:6 вот что заповедует Господь о дочерях Салпаадовых: они могут быть женами тех, кто понравится глазам их, только должны быть женами в племени колена отца своего,
\vs Num 36:7 чтобы удел сынов Израилевых не переходил из колена в колено; ибо каждый из сынов Израилевых должен быть привязан к уделу колена отцов своих;
\vs Num 36:8 и всякая дочь, наследующая удел в коленах сынов Израилевых, должна быть женою кого-нибудь из племени колена отца своего, чтобы сыны Израилевы наследовали каждый удел отцов своих,
\vs Num 36:9 и чтобы не переходил удел из колена в другое колено; ибо каждое из колен сынов Израилевых должно быть привязано к своему уделу.
\vs Num 36:10 Как повелел Господь Моисею, так и сделали дочери Салпаадовы.
\vs Num 36:11 И вышли дочери Салпаадовы Махла, Фирца, Хогла, Милка и Ноа в замужество за сыновей дядей своих;
\vs Num 36:12 в племени сынов Манассии, сына Иосифова, они были женами, и остался удел их в колене племени отца их.
\rsbpar\vs Num 36:13 Сии суть заповеди и постановления, которые дал Господь сынам Израилевым чрез Моисея на равнинах Моавитских, у Иордана, против Иерихона.

\bibbookdescr{Deu}{
  inline={\LARGE Пятая книга Моисеева\\\Huge Второзаконие},
  toc={Второзаконие},
  bookmark={Второзаконие},
  header={Второзаконие},
  %headerleft={},
  %headerright={},
  abbr={Втор}
}
\vs Deu 1:1 Сии суть слова, которые говорил Моисей всем Израильтянам за Иорданом в пустыне на равнине против Суфа, между Фараном и Тофелом, и Лаваном, и Асирофом, и Дизагавом,
\vs Deu 1:2 в расстоянии одиннадцати дней пути от Хорива, по дороге от горы Сеир к Кадес-Варни.
\vs Deu 1:3 Сорокового года, одиннадцатого месяца, в первый [день] месяца говорил Моисей [всем] сынам Израилевым все, что заповедал ему Господь о них.
\vs Deu 1:4 По убиении им Сигона, царя Аморрейского, который жил в Есевоне, и Ога, царя Васанского, который жил в Аштерофе в Едреи,
\vs Deu 1:5 за Иорданом, в земле Моавитской, начал Моисей изъяснять закон сей и сказал:
\vs Deu 1:6 Господь, Бог наш, говорил нам в Хориве и сказал: <<полно вам жить на горе сей!
\vs Deu 1:7 обратитесь, отправьтесь в путь и пойдите на гору Аморреев и ко всем соседям их, на равнину, на гору, на низкие места и на южный край и к берегам моря, в землю Ханаанскую и к Ливану, даже до реки великой, реки Евфрата;
\vs Deu 1:8 вот, Я даю вам землю сию, пойдите, возьмите в наследие землю, которую Господь с клятвою обещал дать отцам вашим, Аврааму, Исааку и Иакову, им и потомству их>>.
\vs Deu 1:9 И я сказал вам в то время: не могу один водить вас;
\vs Deu 1:10 Господь, Бог ваш, размножил вас, и вот, вы ныне многочисленны, как звезды небесные;
\vs Deu 1:11 Господь, Бог отцов ваших, да умножит вас в тысячу крат против того, сколько вас \bibemph{теперь}, и да благословит вас, как Он говорил вам:
\vs Deu 1:12 как же мне одному носить тягости ваши, бремена ваши и распри ваши?
\vs Deu 1:13 изберите себе по коленам вашим мужей мудрых, разумных и испытанных, и я поставлю их начальниками вашими.
\vs Deu 1:14 Вы отвечали мне и сказали: хорошее дело велишь ты сделать.
\vs Deu 1:15 И взял я главных из колен ваших, мужей мудрых, [разумных] и испытанных, и сделал их начальниками над вами, тысяченачальниками, стоначальниками, пятидесятиначальниками, десятиначальниками и надзирателями по коленам вашим.
\vs Deu 1:16 И дал я повеление судьям вашим в то время, говоря: выслушивайте братьев ваших и судите справедливо, как брата с братом, так и пришельца его;
\vs Deu 1:17 не различайте лиц на суде, как малого, так и великого выслушивайте: не бойтесь лица человеческого, ибо суд~--- дело Божие; а дело, которое для вас трудно, доводите до меня, и я выслушаю его.
\vs Deu 1:18 И дал я вам в то время повеления обо всем, что надлежит вам делать.
\vs Deu 1:19 И отправились мы от Хорива, и шли по всей этой великой и страшной пустыне, которую вы видели, по пути к горе Аморрейской, как повелел Господь, Бог наш, и пришли в Кадес-Варни.
\vs Deu 1:20 И сказал я вам: вы пришли к горе Аморрейской, которую Господь, Бог наш, дает нам;
\vs Deu 1:21 вот, Господь, Бог твой, отдает тебе землю сию, иди, возьми ее во владение, как говорил тебе Господь, Бог отцов твоих, не бойся и не ужасайся.
\vs Deu 1:22 Но вы все подошли ко мне и сказали: пошлем пред собою людей, чтоб они исследовали нам землю и принесли нам известие о дороге, по которой идти нам, и о городах, в которые идти нам.
\vs Deu 1:23 Слово это мне понравилось, и я взял из вас двенадцать человек, по одному человеку от [каждого] колена.
\vs Deu 1:24 Они пошли, взошли на гору и дошли до долины Есхол, и обозрели ее;
\vs Deu 1:25 и взяли в руки свои плодов земли и доставили нам, и принесли нам известие и сказали: хороша земля, которую Господь, Бог наш, дает нам.
\vs Deu 1:26 Но вы не захотели идти и воспротивились повелению Господа, Бога вашего,
\vs Deu 1:27 и роптали в шатрах ваших и говорили: Господь, по ненависти к нам, вывел нас из земли Египетской, чтоб отдать нас в руки Аморреев \bibemph{и} истребить нас;
\vs Deu 1:28 куда мы пойдем? братья наши расслабили сердце наше, говоря: народ тот более, [многочисленнее] и выше нас, города \bibemph{там} большие и с укреплениями до небес, да и сынов Енаковых видели мы там.
\vs Deu 1:29 И я сказал вам: не страшитесь и не бойтесь их;
\vs Deu 1:30 Господь, Бог ваш, идет перед вами; Он будет сражаться за вас, как Он сделал с вами в Египте, пред глазами вашими,
\vs Deu 1:31 и в пустыне сей, где, как ты видел, Господь, Бог твой, носил тебя, как человек носит сына своего, на всем пути, которым вы проходили, до пришествия вашего на сие место.
\vs Deu 1:32 Но и при этом вы не верили Господу, Богу вашему,
\vs Deu 1:33 Который шел перед вами путем~--- искать вам места, где остановиться вам, ночью в огне, чтобы указывать вам дорогу, по которой идти, а днем в облаке.
\vs Deu 1:34 И Господь [Бог] услышал слова ваши, и разгневался, и поклялся, говоря:
\vs Deu 1:35 никто из людей сих, из сего злого рода, не увидит доброй земли, которую Я клялся дать отцам вашим;
\vs Deu 1:36 только Халев, сын Иефонниин, увидит ее; ему дам Я землю, по которой он проходил, и сынам его, за то, что он повиновался Господу.
\vs Deu 1:37 И на меня прогневался Господь за вас, говоря: и ты не войдешь туда;
\vs Deu 1:38 Иисус, сын Навин, который при тебе, он войдет туда; его утверди, ибо он введет Израиля во владение ею;
\vs Deu 1:39 дети ваши, о которых вы говорили, что они достанутся в добычу \bibemph{врагам}, и сыновья ваши, которые не знают ныне ни добра ни зла, они войдут туда, им дам ее, и они овладеют ею;
\vs Deu 1:40 а вы обратитесь и отправьтесь в пустыню по дороге к Чермному морю.
\vs Deu 1:41 И вы отвечали тогда и сказали мне: согрешили мы пред Господом, [Богом нашим,] пойдем и сразимся, как повелел нам Господь, Бог наш. И препоясались вы, каждый ратным оружием своим, и безрассудно решились взойти на гору.
\vs Deu 1:42 Но Господь сказал мне: скажи им: не всходите и не сражайтесь, потому что нет Меня среди вас, чтобы не поразили вас враги ваши.
\vs Deu 1:43 И я говорил вам, но вы не послушали и воспротивились повелению Господню и по упорству своему взошли на гору.
\vs Deu 1:44 И выступил против вас Аморрей, живший на горе той, и преследовали вас так, как делают пчелы, и поражали вас на Сеире до самой Хормы.
\vs Deu 1:45 И возвратились вы и плакали пред Господом: но Господь не услышал вопля вашего и не внял вам.
\vs Deu 1:46 И пробыли вы в Кадесе много времени, сколько времени вы \bibemph{там} были.
\vs Deu 2:1 И обратились мы и отправились в пустыню к Чермному морю, как говорил мне Господь, и много времени ходили вокруг горы Сеира.
\vs Deu 2:2 И сказал мне Господь, говоря:
\vs Deu 2:3 полно вам ходить вокруг этой горы, обратитесь к северу;
\vs Deu 2:4 и народу дай повеление и скажи: вы будете проходить пределы братьев ваших, сынов Исавовых, живущих на Сеире, и они убоятся вас; но остерегайтесь
\vs Deu 2:5 начинать с ними войну, ибо Я не дам вам земли их ни на стопу ноги, потому что гору Сеир Я дал во владение Исаву;
\vs Deu 2:6 пищу покупайте у них за серебро и ешьте; и воду покупайте у них за серебро и пейте;
\vs Deu 2:7 ибо Господь, Бог твой, благословил тебя во всяком деле рук твоих, покровительствовал \bibemph{тебе} во время путешествия твоего по великой [и страшной] пустыне сей; вот, сорок лет Господь, Бог твой, с тобою; ты ни в чем не терпел недостатка.
\vs Deu 2:8 И шли мы мимо братьев наших, сынов Исавовых, живущих на Сеире, путем равнины, от Елафа и Ецион-Гавера, и поворотили, и шли к пустыне Моава.
\vs Deu 2:9 И сказал мне Господь: не вступай во вражду с Моавом и не начинай с ними войны; ибо Я не дам тебе ничего от земли его во владение, потому что Ар отдал Я во владение сынам Лотовым;
\vs Deu 2:10 прежде жили там Эмимы, народ великий, многочисленный и высокий, как \bibemph{сыны} Енаковы,
\vs Deu 2:11 и они считались между Рефаимами, как \bibemph{сыны} Енаковы; Моавитяне же называют их Эмимами;
\vs Deu 2:12 а на Сеире жили прежде Хорреи; но сыны Исавовы прогнали их и истребили их от лица своего и поселились вместо их~--- так, как поступил Израиль с землею наследия своего, которую дал им Господь;
\vs Deu 2:13 итак встаньте и пройдите долину Заред. И прошли мы долину Заред.
\vs Deu 2:14 С тех пор, как мы пошли в Кадес-Варни и как прошли долину Заред, \bibemph{минуло} тридцать восемь лет, и у нас перевелся из среды стана весь род ходящих на войну, как клялся им Господь [Бог];
\vs Deu 2:15 да и рука Господня была на них, чтоб истреблять их из среды стана, пока не вымерли.
\vs Deu 2:16 Когда же перевелись все ходящие на войну и вымерли из среды народа,
\vs Deu 2:17 тогда сказал мне Господь, говоря:
\vs Deu 2:18 ты проходишь ныне мимо пределов Моава, мимо Ара,
\vs Deu 2:19 и приблизился к Аммонитянам; не вступай с ними во вражду, и не начинай с ними войны, ибо Я не дам тебе ничего от земли сынов Аммоновых во владение, потому что Я отдал ее во владение сынам Лотовым;
\vs Deu 2:20 и она считалась землею Рефаимов; прежде жили на ней Рефаимы; Аммонитяне же называют их Замзумимами;
\vs Deu 2:21 народ великий, многочисленный и высокий, как \bibemph{сыны} Енаковы, и истребил их Господь пред лицем их, и изгнали они их и поселились на месте их,
\vs Deu 2:22 как Он сделал для сынов Исавовых, живущих на Сеире, истребив пред лицем их Хорреев, и они изгнали их, и поселились на месте их, и \bibemph{живут} до сего дня;
\vs Deu 2:23 и Аввеев, живших в селениях до самой Газы, Кафторимы, исшедшие из Кафтора, истребили и поселились на месте их.
\vs Deu 2:24 Встаньте, отправьтесь и перейдите поток Арнон; вот, Я предаю в руку твою Сигона, царя Есевонского, Аморреянина, и землю его; начинай овладевать ею, и веди с ним войну;
\vs Deu 2:25 с сего дня Я начну распространять страх и ужас пред тобою на народы под всем небом; те, которые услышат о тебе, вострепещут и ужаснутся тебя.
\vs Deu 2:26 И послал я послов из пустыни Кедемоф к Сигону, царю Есевонскому, с словами мирными, чтобы сказать:
\vs Deu 2:27 позволь пройти мне землею твоею; я пойду дорогою, не сойду ни направо, ни налево;
\vs Deu 2:28 пищу продавай мне за серебро, и я буду есть, и воду для питья давай мне за серебро, и я буду пить, только ногами моими пройду~---
\vs Deu 2:29 так, как сделали мне сыны Исава, живущие на Сеире, и Моавитяне, живущие в Аре, доколе не перейду чрез Иордан в землю, которую Господь, Бог наш, дает нам.
\vs Deu 2:30 Но Сигон, царь Есевонский, не согласился позволить пройти нам через свою \bibemph{землю}, потому что Господь, Бог твой, ожесточил дух его и сердце его сделал упорным, чтобы предать его в руку твою, как \bibemph{это видно} ныне.
\vs Deu 2:31 И сказал мне Господь: вот, Я начинаю предавать тебе Сигона [царя Есевонского, Аморреянина,] и землю его; начинай овладевать землею его.
\vs Deu 2:32 И Сигон [царь Есевонский] со всем народом своим выступил против нас на сражение к Иааце;
\vs Deu 2:33 и предал его Господь, Бог наш, [в руки наши,] и мы поразили его и сынов его и весь народ его,
\vs Deu 2:34 и взяли в то время все города его, и предали заклятию все города, мужчин и женщин и детей, не оставили никого в живых;
\vs Deu 2:35 только взяли мы себе в добычу скот их и захваченное во взятых нами городах.
\vs Deu 2:36 От Ароера, который на берегу потока Арнона, и от города, который на долине, до [горы] Галаада не было города, который был бы неприступен для нас: всё предал Господь, Бог наш, [в руки наши].
\vs Deu 2:37 Только к земле Аммонитян ты не подходил, ни к \bibemph{местам} [лежащим] близ потока Иавока, ни к городам [которые] на горе, ни ко всему, к чему не повелел [нам] Господь, Бог наш.
\vs Deu 3:1 И обратились мы оттуда, и шли к Васану, и выступил против нас на войну Ог, царь Васанский, со всем народом своим, при Едреи.
\vs Deu 3:2 И сказал мне Господь: не бойся его, ибо Я отдам в руку твою его, и весь народ его, и всю землю его, и ты поступишь с ним так, как поступил с Сигоном, царем Аморрейским, который жил в Есевоне.
\vs Deu 3:3 И предал Господь, Бог наш, в руки наши и Ога, царя Васанского, и весь народ его; и мы поразили его, так что никого не осталось у него в живых;
\vs Deu 3:4 и взяли мы в то время все города его; не было города, которого мы не взяли бы у них: шестьдесят городов, всю область Аргов, царство Ога Васанского;
\vs Deu 3:5 все эти города укреплены были высокими стенами, воротами и запорами, кроме городов неукрепленных, весьма многих;
\vs Deu 3:6 и предали мы их заклятию, как поступили с Сигоном, царем Есевонским, предав заклятию всякий город с мужчинами, женщинами и детьми;
\vs Deu 3:7 но весь скот и захваченное в городах взяли себе в добычу.
\vs Deu 3:8 И взяли мы в то время из руки двух царей Аморрейских землю сию, которая по эту сторону Иордана, от потока Арнона до горы Ермона,~---
\vs Deu 3:9 Сидоняне Ермон называют Сирионом, а Аморреи называют его Сениром,~---
\vs Deu 3:10 все города на равнине, весь Галаад и весь Васан до Салхи и Едреи, город\acc{а} царства Ога Васанского;
\vs Deu 3:11 ибо только Ог, царь Васанский, оставался из Рефаимов. Вот, одр его, одр железный, и теперь в Равве, у сынов Аммоновых: длина его девять локтей, а ширина его четыре локтя, локтей мужеских.
\vs Deu 3:12 Землю сию взяли мы в то время начиная от Ароера, который у потока Арнона; и половину горы Галаада с городами ее отдал я \bibemph{колену} Рувимову и Гадову;
\vs Deu 3:13 а остаток Галаада и весь Васан, царство Ога, отдал я половине колена Манассиина, всю область Аргов со всем Васаном. [Она называется землею Рефаимов.]
\vs Deu 3:14 Иаир, сын Манассиин, взял всю область Аргов, до пределов Гесурских и Маахских, и назвал Васан, по имени своему, селениями Иаировыми, что и доныне;
\vs Deu 3:15 Махиру дал я Галаад;
\vs Deu 3:16 а \bibemph{колену} Рувимову и Гадову дал от Галаада до потока Арнона, \bibemph{землю} между потоком и пределом, до потока Иавока, предела сынов Аммоновых,
\vs Deu 3:17 также равнину и Иордан, \bibemph{который есть} и предел, от Киннерефа до моря равнины, моря Соленого, при подошве \bibemph{горы} Фасги к востоку.
\vs Deu 3:18 И дал я вам в то время повеление, говоря: Господь, Бог ваш, дал вам землю сию во владение; все способные к войне, вооружившись, идите впереди братьев ваших, сынов Израилевых;
\vs Deu 3:19 только жены ваши и дети ваши и скот ваш [\bibemph{ибо} я знаю, что скота у вас много,] пусть останутся в городах ваших, которые я дал вам,
\vs Deu 3:20 доколе Господь [Бог] не даст покоя братьям вашим, как вам, и доколе и они не получат во владение землю, которую Господь, Бог ваш, дает им за Иорданом; тогда возвратитесь каждый в свое владение, которое я дал вам.
\vs Deu 3:21 И Иисусу заповедал я в то время, говоря: глаза твои видели все, что сделал Господь, Бог ваш, с двумя царями сими; то же сделает Господь со всеми царствами, которые ты будешь проходить;
\vs Deu 3:22 не бойтесь их, ибо Господь, Бог ваш, Сам сражается за вас.
\vs Deu 3:23 И молился я Господу в то время, говоря:
\vs Deu 3:24 Владыко Господи, Ты начал показывать рабу Твоему величие Твое [и силу Твою,] и крепкую руку Твою [и высокую мышцу]; ибо какой бог есть на небе, или на земле, который мог бы делать такие дела, как Твои, и с могуществом таким, как Твое?
\vs Deu 3:25 дай мне перейти и увидеть ту добрую землю, которая за Иорданом, и ту прекрасную гору и Ливан.
\vs Deu 3:26 Но Господь гневался на меня за вас и не послушал меня, и сказал мне Господь: полно тебе, впредь не говори Мне более об этом;
\vs Deu 3:27 взойди на вершину Фасги и взгляни глазами твоими к морю и к северу, и к югу и к востоку, и посмотри глазами твоими, потому что ты не перейдешь за Иордан сей;
\vs Deu 3:28 и дай наставление Иисусу, и укрепи его, и утверди его; ибо он будет предшествовать народу сему и он разделит им на уделы [всю] землю, на которую ты посмотришь.
\vs Deu 3:29 И остановились мы на долине, напротив Беф-Фегора.
\vs Deu 4:1 Итак, Израиль, слушай постановления и законы, которые я [сегодня] научаю вас исполнять, дабы вы были живы [и размножились], и пошли и наследовали ту землю, которую Господь, Бог отцов ваших, дает вам [в наследие];
\vs Deu 4:2 не прибавляйте к тому, что я заповедую вам, и не убавляйте от того; соблюдайте заповеди Господа, Бога вашего, которые я вам [сегодня] заповедую.
\vs Deu 4:3 Глаза ваши видели [все], что сделал Господь [Бог наш] с Ваал-Фегором: всякого человека, последовавшего Ваал-Фегору, истребил Господь, Бог твой, из среды тебя;
\vs Deu 4:4 а вы, прилепившиеся к Господу, Богу вашему, живы все доныне.
\vs Deu 4:5 Вот, я научил вас постановлениям и законам, как повелел мне Господь, Бог мой, дабы вы так поступали в той земле, в которую вы вступаете, чтоб овладеть ею;
\vs Deu 4:6 итак храните и исполняйте их, ибо в этом мудрость ваша и разум ваш пред глазами народов, которые, услышав о всех сих постановлениях, скажут: только этот великий народ есть народ мудрый и разумный.
\vs Deu 4:7 Ибо есть ли какой великий народ, к которому боги \bibemph{его} были бы столь близки, как близок к нам Господь, Бог наш, когда ни призовем Его?
\vs Deu 4:8 и есть ли какой великий народ, у которого были бы такие справедливые постановления и законы, как весь закон сей, который я предлагаю вам сегодня?
\vs Deu 4:9 Только берегись и тщательно храни душу твою, чтобы тебе не забыть тех дел, которые видели глаза твои, и чтобы они не выходили из сердца твоего во все дни жизни твоей; и поведай о них сынам твоим и сынам сынов твоих,~---
\vs Deu 4:10 о том дне, когда ты стоял пред Господом, Богом твоим, при Хориве, [в день собрания,] и когда сказал Господь мне: собери ко Мне народ, и Я возвещу им слова Мои, из которых они научатся бояться Меня во все дни жизни своей на земле и научат сыновей своих.
\vs Deu 4:11 Вы приблизились и стали под горою, а гора горела огнем до самых небес, \bibemph{и была} тьма, облако и мрак.
\vs Deu 4:12 И говорил Господь к вам [на горе] из среды огня; глас слов [Его] вы слышали, но образа не видели, а только глас;
\vs Deu 4:13 и объявил Он вам завет Свой, который повелел вам исполнять, десятословие, и написал его на двух каменных скрижалях;
\vs Deu 4:14 и повелел мне Господь в то время научить вас постановлениям и законам, дабы вы исполняли их в той земле, в которую вы входите, чтоб овладеть ею.
\vs Deu 4:15 Твердо держите в душах ваших, что вы не видели никакого образа в тот день, когда говорил к вам Господь на [горе] Хориве из среды огня,
\vs Deu 4:16 дабы вы не развратились и не сделали себе изваяний, изображений какого-либо кумира, представляющих мужчину или женщину,
\vs Deu 4:17 изображения какого-либо скота, который на земле, изображения какой-либо птицы крылатой, которая летает под небесами,
\vs Deu 4:18 изображения какого-либо [гада,] ползающего по земле, изображения какой-либо рыбы, которая в водах ниже земли;
\vs Deu 4:19 и дабы ты, взглянув на небо и увидев солнце, луну и звезды [и] все воинство небесное, не прельстился и не поклонился им и не служил им, так как Господь, Бог твой, уделил их всем народам под всем небом.
\vs Deu 4:20 А вас взял Господь [Бог] и вывел вас из печи железной, из Египта, дабы вы были народом Его удела, как это ныне \bibemph{видно}.
\vs Deu 4:21 И Господь [Бог] прогневался на меня за вас, и клялся, что я не перейду за Иордан и не войду в ту добрую землю, которую Господь, Бог твой, дает тебе в удел;
\vs Deu 4:22 я умру в сей земле, не перейдя за Иордан, а вы перейдете и овладеете тою доброю землею.
\vs Deu 4:23 Берегитесь, чтобы не забыть вам завета Господа, Бога вашего, который Он поставил с вами, и чтобы не делать себе кумиров, изображающих что-либо, как повелел тебе Господь, Бог твой;
\vs Deu 4:24 ибо Господь, Бог твой, есть огнь поядающий, Бог ревнитель.
\vs Deu 4:25 Если же родятся у тебя сыны и сыны у сынов [твоих], и, долго жив на земле, вы развратитесь и сделаете изваяние, изображающее что-либо, и сделаете зло сие пред очами Господа, Бога вашего, и раздражите Его,
\vs Deu 4:26 то свидетельствуюсь вам сегодня небом и землею, что скоро потеряете землю, для наследования которой вы переходите за Иордан; не пробудете много времени на ней, но погибнете;
\vs Deu 4:27 и рассеет вас Господь по [всем] народам, и останетесь в малом числе между народами, к которым отведет вас Господь;
\vs Deu 4:28 и будете там служить [другим] богам, сделанным руками человеческими из дерева и камня, которые не видят и не слышат, и не едят и не обоняют.
\vs Deu 4:29 Но когда ты взыщешь там Господа, Бога твоего, то найдешь [Его], если будешь искать Его всем сердцем твоим и всею душею твоею.
\vs Deu 4:30 Когда ты будешь в скорби, и когда все это постигнет тебя в последствие времени, то обратишься к Господу, Богу твоему, и послушаешь гласа Его.
\vs Deu 4:31 Господь, Бог твой, есть Бог [благий и] милосердый; Он не оставит тебя и не погубит тебя, и не забудет завета с отцами твоими, который Он клятвою утвердил им.
\vs Deu 4:32 Ибо спроси у времен прежних, бывших прежде тебя, с того дня, в который сотворил Бог человека на земле, и от края неба до края неба: бывало ли что-нибудь такое, как сие великое дело, или слыхано ли подобное сему?
\vs Deu 4:33 слышал ли [какой] народ глас Бога [живаго], говорящего из среды огня, и остался жив, как слышал ты?
\vs Deu 4:34 или покушался ли \bibemph{какой} бог пойти, взять себе народ из среды \bibemph{другого} народа казнями, знамениями и чудесами, и войною, и рукою крепкою, и мышцею высокою, и великими ужасами, как сделал для вас Господь, Бог ваш, в Египте пред глазами твоими?
\vs Deu 4:35 Тебе дано видеть \bibemph{это}, чтобы ты знал, что \bibemph{только} Господь [Бог твой] есть Бог, [и] нет еще кроме Его;
\vs Deu 4:36 с неба дал Он слышать тебе глас Свой, дабы научить тебя, и на земле показал тебе великий огнь Свой, и ты слышал слова Его из среды огня;
\vs Deu 4:37 и так как Он возлюбил отцов твоих и избрал [вас,] потомство их после них, то и вывел тебя Сам великою силою Своею из Египта,
\vs Deu 4:38 чтобы прогнать от лица твоего народы, которые больше и сильнее тебя, \bibemph{и} ввести тебя \bibemph{и} дать тебе землю их в удел, как это ныне \bibemph{видно}.
\vs Deu 4:39 Итак знай ныне и положи на сердце твое, что Господь [Бог твой] есть Бог на небе вверху и на земле внизу, [и] нет еще [кроме Его];
\vs Deu 4:40 и храни постановления Его и заповеди Его, которые я заповедую тебе ныне, чтобы хорошо было тебе и сынам твоим после тебя, и чтобы ты много времени пробыл на той земле, которую Господь, Бог твой, дает тебе навсегда.
\rsbpar\vs Deu 4:41 Тогда отделил Моисей три города по эту сторону Иордана на восток солнца,
\vs Deu 4:42 чтоб убегал туда убийца, который убьет ближнего своего без намерения, не быв врагом ему ни вчера, ни третьего дня, \bibemph{и} чтоб, убежав в один из этих городов, остался жив:
\vs Deu 4:43 Бецер в пустыне, на равнине в \bibemph{колене} Рувимовом, и Рамоф в Галааде в \bibemph{колене} Гадовом, и Голан в Васане в \bibemph{колене} Манассиином.
\rsbpar\vs Deu 4:44 Вот закон, который предложил Моисей сынам Израилевым;
\vs Deu 4:45 вот повеления, постановления и уставы, которые изрек Моисей сынам Израилевым [в пустыне], по исшествии их из Египта,
\vs Deu 4:46 за Иорданом, на долине против Беф-Фегора, в земле Сигона, царя Аморрейского, жившего в Есевоне, которого поразил Моисей с сынами Израилевыми, по исшествии их из Египта.
\vs Deu 4:47 И овладели они землею его и землею Ога, царя Васанского, двух царей Аморрейских, которая за Иорданом к востоку солнца,
\vs Deu 4:48 \bibemph{начиная} от Ароера, который \bibemph{лежит} на берегу потока Арнона, до горы Сиона, она же Ермон,
\vs Deu 4:49 и всею равниною по эту сторону Иордана к востоку, до самого моря равнины при подошве Фасги.
\vs Deu 5:1 И созвал Моисей весь Израиль и сказал им: слушай, Израиль, постановления и законы, которые я изреку сегодня в уши ваши, и выучите их и старайтесь исполнять их.
\vs Deu 5:2 Господь, Бог наш, поставил с нами завет на Хориве;
\vs Deu 5:3 не с отцами нашими поставил Господь завет сей, но с нами, \bibemph{которые} здесь сегодня все живы.
\vs Deu 5:4 Лицем к лицу говорил Господь с вами на горе из среды огня;
\vs Deu 5:5 я же стоял между Господом и между вами в то время, дабы пересказывать вам слово Господа, ибо вы боялись огня и не восходили на гору. Он \bibemph{тогда} сказал:
\rsbpar\vs Deu 5:6 Я Господь, Бог твой, Который вывел тебя из земли Египетской, из дома рабства;
\vs Deu 5:7 да не будет у тебя других богов перед лицем Моим.
\rsbpar\vs Deu 5:8 Не делай себе кумира и никакого изображения того, что на небе вверху и что на земле внизу, и что в водах ниже земли,
\vs Deu 5:9 не поклоняйся им и не служи им; ибо Я Господь, Бог твой, Бог ревнитель, за вину отцов наказывающий детей до третьего и четвертого рода, ненавидящих Меня,
\vs Deu 5:10 и творящий милость до тысячи \bibemph{родов} любящим Меня и соблюдающим заповеди Мои.
\rsbpar\vs Deu 5:11 Не произноси имени Господа, Бога твоего, напрасно; ибо не оставит Господь [Бог твой] без наказания того, кто употребляет имя Его напрасно.
\rsbpar\vs Deu 5:12 Наблюдай день субботний, чтобы свято хранить его, как заповедал тебе Господь, Бог твой;
\vs Deu 5:13 шесть дней работай и делай всякие дела твои,
\vs Deu 5:14 а день седьмой~--- суббота Господу, Богу твоему. Не делай [в оный] никакого дела, ни ты, ни сын твой, ни дочь твоя, ни раб твой, ни раба твоя, ни вол твой, ни осел твой, ни всякий скот твой, ни пришелец твой, который у тебя, чтобы отдохнул раб твой, и раба твоя [и осел твой,] как и ты;
\vs Deu 5:15 и помни, что [ты] был рабом в земле Египетской, но Господь, Бог твой, вывел тебя оттуда рукою крепкою и мышцею высокою, потому и повелел тебе Господь, Бог твой, соблюдать день субботний [и свято хранить его].
\rsbpar\vs Deu 5:16 Почитай отца твоего и матерь твою, как повелел тебе Господь, Бог твой, чтобы продлились дни твои, и чтобы хорошо тебе было на той земле, которую Господь, Бог твой, дает тебе.
\rsbpar\vs Deu 5:17 Не убивай.
\rsbpar\vs Deu 5:18 Не прелюбодействуй.
\rsbpar\vs Deu 5:19 Не кради.
\rsbpar\vs Deu 5:20 Не произноси ложного свидетельства на ближнего твоего.
\rsbpar\vs Deu 5:21 Не желай жены ближнего твоего и не желай дома ближнего твоего, ни поля его, ни раба его, ни рабы его, ни вола его, ни осла его, [ни всякого скота его,] ни всего, что есть у ближнего твоего.
\rsbpar\vs Deu 5:22 Слова сии изрек Господь ко всему собранию вашему на горе из среды огня, облака и мрака [и бури] громогласно, и более не говорил, и написал их на двух каменных скрижалях, и дал их мне.
\vs Deu 5:23 И когда вы услышали глас из среды мрака, и гора горела огнем, то вы подошли ко мне, все начальники колен ваших и старейшины ваши,
\vs Deu 5:24 и сказали: вот, показал нам Господь, Бог наш, славу Свою и величие Свое, и глас Его слышали мы из среды огня; сегодня видели мы, что Бог говорит с человеком, и сей остается жив;
\vs Deu 5:25 но теперь для чего нам умирать? ибо великий огонь сей пожрет нас; если мы еще услышим глас Господа, Бога нашего, то умрем,
\vs Deu 5:26 ибо есть ли какая плоть, которая слышала бы глас Бога живаго, говорящего из среды огня, как мы, и осталась жива?
\vs Deu 5:27 приступи ты и слушай все, что скажет [тебе] Господь, Бог наш, и ты пересказывай нам все, что будет говорить тебе Господь, Бог наш, и мы будем слушать и исполнять.
\vs Deu 5:28 И Господь услышал слова ваши, как вы разговаривали со мною, и сказал мне Господь: слышал Я слова народа сего, которые они говорили тебе; все, что ни говорили они, хорошо;
\vs Deu 5:29 о, если бы сердце их было у них таково, чтобы бояться Меня и соблюдать все заповеди Мои во все дни, дабы хорошо было им и сынам их вовек!
\vs Deu 5:30 пойди, скажи им: <<возвратитесь в шатры свои>>;
\vs Deu 5:31 а ты здесь останься со Мною, и Я изреку тебе все заповеди и постановления и законы, которым ты должен научить их, чтобы они [так] поступали на той земле, которую Я даю им во владение.
\vs Deu 5:32 Смотрите, поступайте так, как повелел вам Господь, Бог ваш; не уклоняйтесь ни направо, ни налево;
\vs Deu 5:33 ходите по тому пути, по которому повелел вам Господь, Бог ваш, дабы вы были живы, и хорошо было вам, и прожили много времени на той земле, которую получите во владение.
\vs Deu 6:1 Вот заповеди, постановления и законы, которым повелел Господь, Бог ваш, научить вас, чтобы вы поступали [так] в той земле, в которую вы идете, чтоб овладеть ею;
\vs Deu 6:2 дабы ты боялся Господа, Бога твоего, и все постановления Его и заповеди Его, которые [сегодня] заповедую тебе, соблюдал ты и сыны твои и сыны сынов твоих во все дни жизни твоей, дабы продлились дни твои.
\vs Deu 6:3 Итак слушай, Израиль, и старайся исполнить это, чтобы тебе хорошо было, и чтобы вы весьма размножились, как Господь, Бог отцов твоих, говорил тебе, [что Он даст тебе] землю, где течет молоко и мед. [Сии суть постановления и законы, которые заповедал Господь Бог сынам Израилевым в пустыне, по исшествии их из земли Египетской.]
\rsbpar\vs Deu 6:4 Слушай, Израиль: Господь, Бог наш, Господь един есть;
\vs Deu 6:5 и люби Господа, Бога твоего, всем сердцем твоим, и всею душею твоею и всеми силами твоими.
\vs Deu 6:6 И да будут слова сии, которые Я заповедую тебе сегодня, в сердце твоем [и в душе твоей];
\vs Deu 6:7 и внушай их детям твоим и говори о них, сидя в доме твоем и идя дорогою, и ложась и вставая;
\vs Deu 6:8 и навяжи их в знак на руку твою, и да будут они повязкою над глазами твоими,
\vs Deu 6:9 и напиши их на косяках дома твоего и на воротах твоих.
\vs Deu 6:10 Когда же введет тебя Господь, Бог твой, в ту землю, которую Он клялся отцам твоим, Аврааму, Исааку и Иакову, дать тебе с большими и хорошими городами, которых ты не строил,
\vs Deu 6:11 и с домами, наполненными всяким добром, которых ты не наполнял, и с колодезями, высеченными \bibemph{из камня}, которых ты не высекал, с виноградниками и маслинами, которых ты не садил, и будешь есть и насыщаться,
\vs Deu 6:12 тогда берегись, чтобы [не обольстилось сердце твое и] не забыл ты Господа, Который вывел тебя из земли Египетской, из дома рабства.
\rsbpar\vs Deu 6:13 Господа, Бога твоего, бойся, и Ему [одному] служи, [и к Нему прилепись,] и Его именем клянись.
\vs Deu 6:14 Не последуйте иным богам, богам тех народов, которые будут вокруг вас;
\vs Deu 6:15 ибо Господь, Бог твой, Который среди тебя, есть Бог ревнитель; чтобы не воспламенился гнев Господа, Бога твоего, на тебя, и не истребил Он тебя с лица земли.
\rsbpar\vs Deu 6:16 Не искушайте Господа, Бога вашего, как вы искушали Его в Массе.
\rsbpar\vs Deu 6:17 Твердо храните заповеди Господа, Бога вашего, и уставы Его и постановления, которые Он заповедал тебе;
\vs Deu 6:18 и делай справедливое и доброе пред очами Господа [Бога твоего], дабы хорошо тебе было, и дабы ты вошел и овладел доброю землею, которую Господь с клятвою обещал отцам твоим,
\vs Deu 6:19 и чтобы Он прогнал всех врагов твоих от лица твоего, как говорил Господь.
\vs Deu 6:20 Если спросит у тебя сын твой в последующее время, говоря: <<что \bibemph{значат} сии уставы, постановления и законы, которые заповедал вам Господь, Бог ваш?>>
\vs Deu 6:21 то скажи сыну твоему: <<рабами были мы у фараона в Египте, но Господь [Бог] вывел нас из Египта рукою крепкою [и мышцею высокою],
\vs Deu 6:22 и явил Господь [Бог] знамения и чудеса великие и казни над Египтом, над фараоном и над всем домом его [и над войском его] пред глазами нашими;
\vs Deu 6:23 а нас вывел оттуда [Господь, Бог наш,] чтобы ввести нас и дать нам землю, которую [Господь, Бог наш,] клялся отцам нашим [дать нам];
\vs Deu 6:24 и заповедал нам Господь исполнять все постановления сии, чтобы мы боялись Господа, Бога нашего, дабы хорошо было нам во все дни, дабы сохранить нашу жизнь, как и теперь;
\vs Deu 6:25 и в сем будет наша праведность, если мы будем стараться исполнять все сии заповеди [закона] пред лицем Господа, Бога нашего, как Он заповедал нам>>.
\vs Deu 7:1 Когда введет тебя Господь, Бог твой, в землю, в которую ты идешь, чтоб овладеть ею, и изгонит от лица твоего многочисленные народы, Хеттеев, Гергесеев, Аморреев, Хананеев, Ферезеев, Евеев и Иевусеев, семь народов, которые многочисленнее и сильнее тебя,
\vs Deu 7:2 и предаст их тебе Господь, Бог твой, и поразишь их, тогда предай их заклятию, не вступай с ними в союз и не щади их;
\vs Deu 7:3 и не вступай с ними в родство: дочери твоей не отдавай за сына его, и дочери его не бери за сына твоего;
\vs Deu 7:4 ибо они отвратят сынов твоих от Меня, чтобы служить иным богам, и \bibemph{тогда} воспламенится на вас гнев Господа, и Он скоро истребит тебя.
\vs Deu 7:5 Но поступите с ними так: жертвенники их разрушьте, столбы их сокрушите, и рощи их вырубите, и истуканов [богов] их сожгите огнем;
\vs Deu 7:6 ибо ты народ святый у Господа, Бога твоего: тебя избрал Господь, Бог твой, чтобы ты был собственным Его народом из всех народов, которые на земле.
\vs Deu 7:7 Не потому, чтобы вы были многочисленнее всех народов, принял вас Господь и избрал вас,~--- ибо вы малочисленнее всех народов,~---
\vs Deu 7:8 но потому, что любит вас Господь, и для того, чтобы сохранить клятву, которою Он клялся отцам вашим, вывел вас Господь рукою крепкою [и мышцею высокою] и освободил тебя из дома рабства, из руки фараона, царя Египетского.
\vs Deu 7:9 Итак знай, что Господь, Бог твой, есть Бог, Бог верный, Который хранит завет [Свой] и милость к любящим Его и сохраняющим заповеди Его до тысячи родов,
\vs Deu 7:10 и воздает ненавидящим Его в лице их, погубляя их; Он не замедлит, ненавидящему Его самому лично воздаст.
\rsbpar\vs Deu 7:11 Итак, соблюдай заповеди и постановления и законы, которые сегодня заповедую тебе исполнять.
\vs Deu 7:12 И если вы будете слушать законы сии и хранить и исполнять их, то Господь, Бог твой, будет хранить завет и милость к тебе, как Он клялся отцам твоим,
\vs Deu 7:13 и возлюбит тебя, и благословит тебя, и размножит тебя, и благословит плод чрева твоего и плод земли твоей, и хлеб твой, и вино твое, и елей твой, рождаемое от крупного скота твоего и от стада овец твоих, на той земле, которую Он клялся отцам твоим дать тебе;
\vs Deu 7:14 благословен ты будешь больше всех народов; не будет ни бесплодного, ни бесплодной, ни у тебя, ни в скоте твоем;
\vs Deu 7:15 и отдалит от тебя Господь [Бог твой] всякую немощь, и никаких лютых болезней Египетских, [которые ты видел и] которые ты знаешь, не наведет на тебя, но наведет их на всех, ненавидящих тебя;
\vs Deu 7:16 и истребишь все народы, которые Господь, Бог твой, дает тебе: да не пощадит их глаз твой; и не служи богам их, ибо это сеть для тебя.
\vs Deu 7:17 Если скажешь в сердце твоем: <<народы сии многочисленнее меня; как я могу изгнать их?>>
\vs Deu 7:18 Не бойся их, вспомни то, что сделал Господь, Бог твой, с фараоном и всем Египтом,
\vs Deu 7:19 те великие испытания, которые видели глаза твои, [великие] знамения, чудеса, и руку крепкую и мышцу высокую, с какими вывел тебя Господь, Бог твой; то же сделает Господь, Бог твой, со всеми народами, которых ты боишься;
\vs Deu 7:20 и шершней нашлет Господь, Бог твой, на них, доколе не погибнут оставшиеся и скрывшиеся от лица твоего;
\vs Deu 7:21 не страшись их, ибо Господь, Бог твой, среди тебя, Бог великий и страшный.
\vs Deu 7:22 И будет Господь, Бог твой, изгонять пред тобою народы сии мало-помалу; не можешь ты истребить их скоро, чтобы [земля не сделалась пуста и] не умножились против тебя полевые звери;
\vs Deu 7:23 но предаст их тебе Господь, Бог твой, и приведет их в великое смятение, так что они погибнут;
\vs Deu 7:24 и предаст царей их в руки твои, и ты истребишь имя их из поднебесной: не устоит никто против тебя, доколе не искоренишь их.
\vs Deu 7:25 Кумиры богов их сожгите огнем; не пожелай взять себе серебра или золота, которое на них, дабы это не было для тебя сетью, ибо это мерзость для Господа, Бога твоего;
\vs Deu 7:26 и не вноси мерзости в дом твой, дабы не подпасть заклятию, как она; отвращайся сего и гнушайся сего, ибо это заклятое.
\vs Deu 8:1 Все заповеди, которые я заповедую вам сегодня, старайтесь исполнять, дабы вы были живы и размножились, и пошли и завладели [доброю] землею, которую с клятвою обещал Господь [Бог] отцам вашим.
\vs Deu 8:2 И помни весь путь, которым вел тебя Господь, Бог твой, по пустыне, вот уже сорок лет, чтобы смирить тебя, чтобы испытать тебя и узнать, что в сердце твоем, будешь ли хранить заповеди Его, или нет;
\vs Deu 8:3 Он смирял тебя, томил тебя голодом и питал тебя манною, которой не знал ты и не знали отцы твои, дабы показать тебе, что не одним хлебом живет человек, но всяким [словом], исходящим из уст Господа, живет человек;
\vs Deu 8:4 одежда твоя не ветшала на тебе, и нога твоя не пухла, вот уже сорок лет.
\vs Deu 8:5 И знай в сердце твоем, что Господь, Бог твой, учит тебя, как человек учит сына своего.
\vs Deu 8:6 Итак храни заповеди Господа, Бога твоего, ходя путями Его и боясь Его.
\vs Deu 8:7 Ибо Господь, Бог твой, ведет тебя в землю добрую, в землю, \bibemph{где} потоки вод, источники и озера выходят из долин и гор,
\vs Deu 8:8 в землю, [где] пшеница, ячмень, виноградные лозы, смоковницы и гранатовые деревья, в землю, \bibemph{где} масличные деревья и мед,
\vs Deu 8:9 в землю, в которой без скудости будешь есть хлеб твой и ни в чем не будешь иметь недостатка, в землю, в которой камни~--- железо, и из гор которой будешь высекать медь.
\vs Deu 8:10 И когда будешь есть и насыщаться, тогда благословляй Господа, Бога твоего, за добрую землю, которую Он дал тебе.
\vs Deu 8:11 Берегись, чтобы ты не забыл Господа, Бога твоего, не соблюдая заповедей Его, и законов Его, и постановлений Его, которые сегодня заповедую тебе.
\vs Deu 8:12 Когда будешь есть и насыщаться, и построишь хорошие домы и будешь жить [в них],
\vs Deu 8:13 и когда будет у тебя много крупного и мелкого скота, и будет много серебра и золота, и всего у тебя будет много,~---
\vs Deu 8:14 то смотри, чтобы не надмилось сердце твое и не забыл ты Господа, Бога твоего, Который вывел тебя из земли Египетской, из дома рабства;
\vs Deu 8:15 Который провел тебя по пустыне великой и страшной, \bibemph{где} змеи, василиски, скорпионы и места сухие, на которых нет воды; Который источил для тебя [источник] воды из скалы гранитной,
\vs Deu 8:16 питал тебя в пустыне манною, которой [не знал ты и] не знали отцы твои, дабы смирить тебя и испытать тебя, чтобы впоследствии сделать тебе добро,
\vs Deu 8:17 и чтобы ты не сказал в сердце твоем: <<моя сила и крепость руки моей приобрели мне богатство сие>>,
\vs Deu 8:18 но чтобы помнил Господа, Бога твоего, ибо Он дает тебе силу приобретать богатство, дабы исполнить, как ныне, завет Свой, который Он клятвою утвердил отцам твоим.
\vs Deu 8:19 Если же ты забудешь Господа, Бога твоего, и пойдешь вслед богов других, и будешь служить им и поклоняться им, то свидетельствуюсь вам сегодня [небом и землею], что вы погибнете;
\vs Deu 8:20 как народы, которые Господь [Бог] истребляет от лица вашего, так погибнете \bibemph{и вы} за то, что не послушаете гласа Господа, Бога вашего.
\vs Deu 9:1 Слушай, Израиль: ты теперь идешь за Иордан, чтобы пойти овладеть народами, которые больше и сильнее тебя, городами большими, с укреплениями до небес,
\vs Deu 9:2 народом [великим,] многочисленным и великорослым, сынами Енаковыми, о которых ты знаешь и слышал: <<кто устоит против сынов Енаковых?>>
\vs Deu 9:3 Знай же ныне, что Господь, Бог твой, идет пред тобою, \bibemph{как} огнь поядающий; Он будет истреблять их и низлагать их пред тобою, и ты изгонишь их, и погубишь их скоро, как говорил тебе Господь.
\vs Deu 9:4 Когда будет изгонять их Господь, Бог твой, от лица твоего, не говори в сердце твоем, что за праведность мою привел меня Господь овладеть сею [доброю] землею, и что за нечестие народов сих Господь изгоняет их от лица твоего;
\vs Deu 9:5 не за праведность твою и не за правоту сердца твоего идешь ты наследовать землю их, но за нечестие [и беззакония] народов сих Господь, Бог твой, изгоняет их от лица твоего, и дабы исполнить слово, которым клялся Господь отцам твоим Аврааму, Исааку и Иакову;
\vs Deu 9:6 посему знай [ныне], что не за праведность твою Господь, Бог твой, дает тебе овладеть сею доброю землею, ибо ты народ жестоковыйный.
\vs Deu 9:7 Помни, не забудь, сколько ты раздражал Господа, Бога твоего, в пустыне: с самого того дня, как вышел ты из земли Египетской, и до самого прихода вашего на место сие вы противились Господу.
\vs Deu 9:8 И при Хориве вы раздражали Господа, и прогневался на вас Господь, так что \bibemph{хотел} истребить вас,
\vs Deu 9:9 когда я взошел на гору, чтобы принять скрижали каменные, скрижали завета, который поставил Господь с вами, и пробыл на горе сорок дней и сорок ночей, хлеба не ел и воды не пил,
\vs Deu 9:10 и дал мне Господь две скрижали каменные, написанные перстом Божиим, а на них [написаны были] все слова, которые изрек вам Господь на горе из среды огня в день собрания.
\vs Deu 9:11 По окончании же сорока дней и сорока ночей дал мне Господь две скрижали каменные, скрижали завета,
\vs Deu 9:12 и сказал мне Господь: встань, пойди скорее отсюда, ибо развратился народ твой, который ты вывел из Египта; скоро уклонились они от пути, который Я заповедал им; они сделали себе литой истукан.
\vs Deu 9:13 И сказал мне Господь: [Я говорил тебе один и другой раз:] вижу Я народ сей, вот он народ жестоковыйный;
\vs Deu 9:14 не удерживай Меня, и Я истреблю их, и изглажу имя их из поднебесной, а от тебя произведу народ, \bibemph{который будет} [больше,] сильнее и многочисленнее их.
\vs Deu 9:15 Я обратился и пошел с горы, гора же горела огнем; две скрижали завета \bibemph{были} в обеих руках моих;
\vs Deu 9:16 и видел я, что вы согрешили против Господа, Бога вашего, сделали себе литого тельца, скоро уклонились от пути, которого [держаться] заповедал вам Господь;
\vs Deu 9:17 и взял я обе скрижали, и бросил их из обеих рук своих, и разбил их пред глазами вашими.
\vs Deu 9:18 И [вторично] повергшись пред Господом, молился я, как прежде, сорок дней и сорок ночей, хлеба не ел и воды не пил, за все грехи ваши, которыми вы согрешили, сделав зло в очах Господа [Бога вашего] и раздражив Его;
\vs Deu 9:19 ибо я страшился гнева и ярости, которыми Господь прогневался на вас \bibemph{и хотел} погубить вас. И послушал меня Господь и на сей раз.
\vs Deu 9:20 И на Аарона весьма прогневался Господь \bibemph{и хотел} погубить его; но я молился и за Аарона в то время.
\vs Deu 9:21 Грех же ваш, который вы сделали,~--- тельца я взял, сожег его в огне, разбил его и всего истер до того, что он стал мелок, как прах, и я бросил прах сей в поток, текущий с горы.
\vs Deu 9:22 И в Тавере, в Массе и в Киброт-Гаттааве вы раздражили Господа [Бога вашего].
\vs Deu 9:23 И когда посылал вас Господь из Кадес-Варни, говоря: пойдите, овладейте землею, которую Я даю вам,~--- то вы воспротивились повелению Господа Бога вашего, и не поверили Ему, и не послушали гласа Его.
\vs Deu 9:24 Вы были непокорны Господу с того самого дня, как я стал знать вас.
\vs Deu 9:25 И повергшись пред Господом, умолял я сорок дней и сорок ночей, в которые я молился, ибо Господь хотел погубить вас;
\vs Deu 9:26 и молился я Господу и сказал: Владыка Господи, [Царь богов,] не погубляй народа Твоего и удела Твоего, который Ты избавил величием [крепости] Твоей, который вывел Ты из Египта рукою сильною [и мышцею Твоею высокою];
\vs Deu 9:27 вспомни рабов Твоих, Авраама, Исаака и Иакова, [которым Ты клялся Собою]; не смотри на ожесточение народа сего и на нечестие его и на грехи его,
\vs Deu 9:28 дабы [живущие] в той земле, откуда Ты вывел нас, не сказали: <<Господь не мог ввести их в землю, которую обещал им, и, ненавидя их, вывел Он их, чтоб умертвить их в пустыне>>.
\vs Deu 9:29 А они Твой народ и Твой удел, который Ты вывел [из земли Египетской] силою Твоею великою и мышцею Твоею высокою.
\vs Deu 10:1 В то время сказал мне Господь: вытеши себе две скрижали каменные, подобные первым, и взойди ко Мне на гору, и сделай себе деревянный ковчег;
\vs Deu 10:2 и Я напишу на скрижалях те слова, которые были на прежних скрижалях, которые ты разбил; и положи их в ковчег.
\vs Deu 10:3 И сделал я ковчег из дерева ситтим, и вытесал две каменные скрижали, как прежние, и пошел на гору; и две сии скрижали \bibemph{были} в руках моих.
\vs Deu 10:4 И написал Он на скрижалях, как написано было прежде, те десять слов, которые изрек вам Господь на горе из среды огня в день собрания, и отдал их Господь мне.
\vs Deu 10:5 И обратился я, и сошел с горы, и положил скрижали в ковчег, который я сделал, чтоб они там были, как повелел мне Господь.
\vs Deu 10:6 И сыны Израилевы отправились из Беероф-Бене-Яакана в Мозер; там умер Аарон и погребен там, и стал священником вместо него сын его Елеазар.
\vs Deu 10:7 Оттуда отправились в Гудгод, из Гудгода в Иотвафу, в землю, где потоки вод.
\vs Deu 10:8 В то время отделил Господь колено Левиино, чтобы носить ковчег завета Господня, предстоять пред Господом, служить Ему [и молиться] и благословлять именем Его, \bibemph{как это продолжается} до сего дня;
\vs Deu 10:9 потому нет левиту части и удела с братьями его: Сам Господь есть удел его, как говорил ему Господь, Бог твой.
\vs Deu 10:10 И пробыл я на горе, как и в прежнее время, сорок дней и сорок ночей; и послушал меня Господь и на сей раз, [и] не восхотел Господь погубить тебя;
\vs Deu 10:11 и сказал мне Господь: встань, пойди в путь пред народом [сим]; пусть они пойдут и овладеют землею, которую Я клялся отцам их дать им.
\vs Deu 10:12 Итак, Израиль, чего требует от тебя Господь, Бог твой? Того только, чтобы ты боялся Господа, Бога твоего, ходил всеми путями Его, и любил Его, и служил Господу, Богу твоему, от всего сердца твоего и от всей души твоей,
\vs Deu 10:13 чтобы соблюдал заповеди Господа [Бога твоего] и постановления Его, которые сегодня заповедую тебе, дабы тебе было хорошо.
\vs Deu 10:14 Вот у Господа, Бога твоего, небо и небеса небес, земля и все, что на ней;
\vs Deu 10:15 но только отцов твоих принял Господь и возлюбил их, и избрал вас, семя их после них, из всех народов, как ныне \bibemph{видишь}.
\vs Deu 10:16 Итак обрежьте крайнюю плоть сердца вашего и не будьте впредь жестоковыйны;
\vs Deu 10:17 ибо Господь, Бог ваш, есть Бог богов и Владыка владык, Бог великий, сильный и страшный, Который не смотрит на лица и не берет даров,
\vs Deu 10:18 Который дает суд сироте и вдове, и любит пришельца, и дает ему хлеб и одежду.
\vs Deu 10:19 Люб\acc{и}те и вы пришельца, ибо \bibemph{сами} были пришельцами в земле Египетской.
\vs Deu 10:20 Господа, Бога твоего, бойся [и] Ему [одному] служи, и к Нему прилепись и Его именем клянись:
\vs Deu 10:21 Он хвала твоя и Он Бог твой, Который сделал с тобою те великие и страшные \bibemph{дела}, какие видели глаза твои;
\vs Deu 10:22 в семидесяти [пяти] душах пришли отцы твои в Египет, а ныне Господь, Бог твой, сделал тебя многочисленным, как звезды небесные.
\vs Deu 11:1 Итак люби Господа, Бога твоего, и соблюдай, что повелено Им соблюдать, и постановления Его и законы Его и заповеди Его во все дни.
\vs Deu 11:2 И вспомните ныне,~--- ибо \bibemph{я говорю} не с сынами вашими, которые не знают и не видели наказания Господа Бога вашего,~--- Его величие [и] Его крепкую руку и высокую мышцу Его,
\vs Deu 11:3 знамения Его и дела Его, которые Он сделал среди Египта с фараоном, царем Египетским, и со всею землею его,
\vs Deu 11:4 и что Он сделал с войском Египетским, с конями его и колесницами его, которых Он потопил в водах Чермного моря, когда они гнались за вами,~--- и погубил их Господь [Бог] даже до сего дня;
\vs Deu 11:5 и что Он делал для вас в пустыне, доколе вы не дошли до места сего,
\vs Deu 11:6 и что Он сделал с Дафаном и Авироном, сынами Елиава, сына Рувимова, когда земля разверзла уста свои и среди всего Израиля поглотила их и семейства их, и шатры их, и все имущество их, которое было у них;
\vs Deu 11:7 ибо глаза ваши видели все великие дела Господа, которые Он сделал.
\vs Deu 11:8 Итак соблюдайте все заповеди [Его], которые я заповедую вам сегодня, дабы вы [были живы,] укрепились и пошли и овладели землею, в которую вы переходите [за Иордан], чтоб овладеть ею;
\vs Deu 11:9 и дабы вы жили много времени на той земле, которую клялся Господь отцам вашим дать им и семени их, на земле, в которой течет молоко и мед.
\vs Deu 11:10 Ибо земля, в которую ты идешь, чтоб овладеть ею, не такова, как земля Египетская, из которой вышли вы, где ты, посеяв семя твое, поливал [ее] при помощи ног твоих, как масличный сад;
\vs Deu 11:11 но земля, в которую вы переходите, чтоб овладеть ею, есть земля с горами и долинами, и от дождя небесного напояется водою,~---
\vs Deu 11:12 земля, о которой Господь, Бог твой, печется: очи Господа, Бога твоего, непрестанно на ней, от начала года и до конца года.
\vs Deu 11:13 Если вы будете слушать заповеди Мои, которые заповедую вам сегодня, любить Господа, Бога вашего, и служить Ему от всего сердца вашего и от всей души вашей,
\vs Deu 11:14 то дам земле вашей дождь в свое время, ранний и поздний; и ты соберешь хлеб твой и вино твое и елей твой;
\vs Deu 11:15 и дам траву на поле твоем для скота твоего, и будешь есть и насыщаться.
\vs Deu 11:16 Берегитесь, чтобы не обольстилось сердце ваше, и вы не уклонились и не стали служить иным богам и не поклонились им;
\vs Deu 11:17 и тогда воспламенится гнев Господа на вас, и заключит Он небо, и не будет дождя, и земля не принесет произведений своих, и вы скоро погибнете с доброй земли, которую Господь дает вам.
\vs Deu 11:18 Итак положите сии слова Мои в сердце ваше и в душу вашу, и навяжите их в знак на руку свою, и да будут они повязкою над глазами вашими;
\vs Deu 11:19 и учите им сыновей своих, говоря о них, когда ты сидишь в доме твоем, и когда идешь дорогою, и когда ложишься, и когда встаешь;
\vs Deu 11:20 и напиши их на косяках дома твоего и на воротах твоих,
\vs Deu 11:21 дабы столько же много было дней ваших и дней детей ваших на той земле, которую Господь клялся дать отцам вашим, сколько дней небо будет над землею.
\vs Deu 11:22 Ибо если вы будете соблюдать все заповеди сии, которые заповедую вам исполнять, будете любить Господа, Бога вашего, ходить всеми путями Его и прилепляться к Нему,
\vs Deu 11:23 то изгонит Господь все народы сии от лица вашего, и вы овладеете народами, которые больше и сильнее вас;
\vs Deu 11:24 всякое место, на которое ступит нога ваша, будет ваше; от пустыни и Ливана, от реки, реки Евфрата, даже до моря западного будут пределы ваши;
\vs Deu 11:25 никто не устоит против вас: Господь, Бог ваш, наведет страх и трепет пред вами на всякую землю, на которую вы ступите, как Он говорил вам.
\vs Deu 11:26 Вот, я предлагаю вам сегодня благословение и проклятие:
\vs Deu 11:27 благословение, если послушаете заповедей Господа, Бога вашего, которые я заповедую вам сегодня,
\vs Deu 11:28 а проклятие, если не послушаете заповедей Господа, Бога вашего, и уклонитесь от пути, который заповедую вам сегодня, и пойдете вслед богов иных, которых вы не знаете.
\vs Deu 11:29 Когда введет тебя Господь, Бог твой, в ту землю, в которую ты идешь, чтоб овладеть ею, тогда произнеси благословение на горе Гаризим, а проклятие на горе Гевал:
\vs Deu 11:30 вот они за Иорданом, по дороге к захождению солнца, в земле Хананеев, живущих на равнине, против Галгала, близ дубравы Мор\acc{е}.
\vs Deu 11:31 Ибо вы переходите Иордан, чтобы пойти овладеть землею, которую Господь, Бог ваш, дает вам [в удел навсегда], и овладеете ею и будете жить на ней.
\vs Deu 11:32 Итак старайтесь соблюдать все постановления и законы [Его], которые предлагаю я вам сегодня.
\vs Deu 12:1 Вот постановления и законы, которые вы должны стараться исполнять в земле, которую Господь, Бог отцов твоих, дает тебе во владение, во все дни, которые вы будете жить на той земле.
\vs Deu 12:2 Истребите все места, где народы, которыми вы овладеете, служили богам своим, на высоких горах и на холмах, и под всяким ветвистым деревом;
\vs Deu 12:3 и разрушьте жертвенники их, и сокрушите столбы их, и сожгите огнем рощи их, и разбейте истуканы богов их, и истребите имя их от места того.
\vs Deu 12:4 Не то должны вы делать для Господа, Бога вашего;
\vs Deu 12:5 но к месту, какое изберет Господь, Бог ваш, из всех колен ваших, чтобы пребывать имени Его там, обращайтесь и туда приходите,
\vs Deu 12:6 и туда приносите всесожжения ваши, и жертвы ваши, и десятины ваши, и возношение рук ваших, и обеты ваши, и добровольные приношения ваши, [и мирные жертвы ваши,] и первенцев крупного скота вашего и мелкого скота вашего;
\vs Deu 12:7 и ешьте там пред Господом, Богом вашим, и веселитесь вы и семейства ваши о всем, что делалось руками вашими, чем благословил тебя Господь, Бог твой.
\vs Deu 12:8 Там вы не должны делать всего, как мы теперь здесь делаем, каждый, что ему кажется правильным;
\vs Deu 12:9 ибо вы ныне еще не вступили в место покоя и в удел, который Господь, Бог твой, дает тебе.
\vs Deu 12:10 Но когда перейдете Иордан и поселитесь на земле, которую Господь, Бог ваш, дает вам в удел, и когда Он успокоит вас от всех врагов ваших, окружающих \bibemph{вас}, и будете жить безопасно,
\vs Deu 12:11 тогда, какое место изберет Господь, Бог ваш, чтобы пребывать имени Его там, туда приносите всё, что я заповедую вам [сегодня]: всесожжения ваши и жертвы ваши, десятины ваши и возношение рук ваших, и все, избранное по обетам вашим, что вы обещали Господу [Богу вашему];
\vs Deu 12:12 и веселитесь пред Господом, Богом вашим, вы и сыны ваши, и дочери ваши, и рабы ваши, и рабыни ваши, и левит, который посреди жилищ ваших, ибо нет ему части и удела с вами.
\rsbpar\vs Deu 12:13 Берегись приносить всесожжения твои на всяком месте, которое ты увидишь;
\vs Deu 12:14 но на том только месте, которое изберет Господь [Бог твой] в одном из колен твоих, приноси всесожжения твои и делай все, что заповедую тебе [сегодня].
\vs Deu 12:15 Впрочем, когда только пожелает душа твоя, можешь заколать и есть, по благословению Господа, Бога твоего, мясо, которое Он дал тебе, во всех жилищах твоих: нечистый и чистый могут есть сие, как серну и как оленя;
\vs Deu 12:16 только крови не ешьте: на землю выливайте ее, как воду.
\vs Deu 12:17 Нельзя тебе есть в жилищах твоих десятины хлеба твоего, и вина твоего, и елея твоего, и первенцев крупного скота твоего и мелкого скота твоего, и всех обетов твоих, которые ты обещал, и добровольных приношений твоих, и возношения рук твоих;
\vs Deu 12:18 но ешь сие [только] пред Господом, Богом твоим, на том месте, которое изберет Господь, Бог твой,~--- ты и сын твой, и дочь твоя, и раб твой, и раба твоя, и левит, [и пришелец,] который в жилищах твоих, и веселись пред Господом, Богом твоим, о всем, что делалось руками твоими.
\vs Deu 12:19 Смотри, не оставляй левита во все дни, [которые будешь жить] на земле твоей.
\rsbpar\vs Deu 12:20 Когда распространит Господь, Бог твой, пределы твои, как Он говорил тебе, и ты скажешь: <<поем я мяса>>, потому что душа твоя пожелает есть мяса,~--- тогда, по желанию души твоей, ешь мясо.
\vs Deu 12:21 Если далеко будет от тебя то место, которое изберет Господь, Бог твой, чтобы пребывать имени Его там, то заколай из крупного и мелкого скота твоего, который дал тебе Господь [Бог твой], как я повелел тебе, и ешь в жилищах твоих, по желанию души твоей;
\vs Deu 12:22 но ешь их так, как едят серну и оленя; нечистый как и чистый [у тебя] могут есть сие;
\vs Deu 12:23 только строго наблюдай, чтобы не есть крови, потому что кровь есть душа: не ешь души вместе с мясом;
\vs Deu 12:24 не ешь ее: выливай ее на землю, как воду;
\vs Deu 12:25 не ешь ее, дабы хорошо было тебе и детям твоим после тебя [во веки], если будешь делать [доброе и] справедливое пред очами Господа [Бога твоего].
\vs Deu 12:26 Только святыни твои, какие будут у тебя, и обеты твои приноси, и приходи на то место, которое изберет Господь [Бог твой, чтобы призываемо было там имя Его];
\vs Deu 12:27 и совершай всесожжения твои, мясо и кровь, на жертвеннике Господа, Бога твоего; но кровь \bibemph{других} жертв твоих должна быть проливаема у жертвенника Господа, Бога твоего, а мясо ешь.
\vs Deu 12:28 Слушай и исполняй все слова сии, которые заповедую тебе, дабы хорошо было тебе и детям твоим после тебя во век, если будешь делать доброе и угодное пред очами Господа, Бога твоего.
\rsbpar\vs Deu 12:29 Когда Господь, Бог твой, истребит от лица твоего народы, к которым ты идешь, чтобы взять их во владение, и ты, взяв их, поселишься в земле их;
\vs Deu 12:30 тогда берегись, чтобы ты не попал в сеть, последуя им, по истреблении их от лица твоего, и не искал богов их, говоря: <<как служили народы сии богам своим, так буду и я делать>>;
\vs Deu 12:31 не делай так Господу, Богу твоему, ибо все, чего гнушается Господь, что ненавидит Он, они делают богам своим: они и сыновей своих и дочерей своих сожигают на огне богам своим.
\vs Deu 12:32 Все, что я заповедую вам, старайтесь исполнить; не прибавляй к тому и не убавляй от того.
\vs Deu 13:1 Если восстанет среди тебя пророк, или сновидец, и представит тебе знамение или чудо,
\vs Deu 13:2 и сбудется то знамение или чудо, о котором он говорил тебе, и скажет притом: <<пойдем вслед богов иных, которых ты не знаешь, и будем служить им>>,~---
\vs Deu 13:3 то не слушай слов пророка сего, или сновидца сего; ибо \bibemph{чрез сие} искушает вас Господь, Бог ваш, чтобы узнать, любите ли вы Господа, Бога вашего, от всего сердца вашего и от всей души вашей;
\vs Deu 13:4 Господу, Богу вашему, последуйте и Его бойтесь, заповеди Его соблюдайте и гласа Его слушайте, и Ему служите, и к Нему прилепляйтесь;
\vs Deu 13:5 а пророка того или сновидца того должно предать смерти за то, что он уговаривал вас отступить от Господа, Бога вашего, выведшего вас из земли Египетской и избавившего тебя из дома рабства, желая совратить тебя с пути, по которому заповедал тебе идти Господь, Бог твой; и \bibemph{так} истреби зло из среды себя.
\vs Deu 13:6 Если будет уговаривать тебя тайно брат твой, [сын отца твоего или] сын матери твоей, или сын твой, или дочь твоя, или жена на лоне твоем, или друг твой, который для тебя, как душа твоя, говоря: <<пойдем и будем служить богам иным, которых не знал ты и отцы твои>>,
\vs Deu 13:7 богам тех народов, которые вокруг тебя, близких к тебе или отдаленных от тебя, от одного края земли до другого,~---
\vs Deu 13:8 то не соглашайся с ним и не слушай его; и да не пощадит его глаз твой, не жалей его и не прикрывай его,
\vs Deu 13:9 но убей его; твоя рука прежде \bibemph{всех} должна быть на нем, чтоб убить его, а потом руки всего народа;
\vs Deu 13:10 побей его камнями до смерти, ибо он покушался отвратить тебя от Господа, Бога твоего, Который вывел тебя из земли Египетской, из дома рабства;
\vs Deu 13:11 весь Израиль услышит сие и убоится, и не станут впредь делать среди тебя такого зла.
\vs Deu 13:12 Если услышишь о каком-либо из городов твоих, которые Господь, Бог твой, дает тебе для жительства,
\vs Deu 13:13 что появились в нем нечестивые люди из среды тебя и соблазнили жителей города их, говоря: <<пойдем и будем служить богам иным, которых вы не знали>>,~---
\vs Deu 13:14 то ты разыщи, исследуй и хорошо расспроси; и если это точная правда, что случилась мерзость сия среди тебя,
\vs Deu 13:15 порази жителей того города острием меча, предай заклятию его и все, что в нем, и скот его \bibemph{порази} острием меча;
\vs Deu 13:16 всю же добычу его собери на средину площади его и сожги огнем город и всю добычу его во всесожжение Господу, Богу твоему, и да будет он вечно в развалинах, не должно никогда вновь созидать его;
\vs Deu 13:17 ничто из заклятого да не прилипнет к руке твоей, дабы укротил Господь ярость гнева Своего, и дал тебе милость и помиловал тебя, и размножил тебя, [как Он говорил тебе,] как клялся отцам твоим,
\vs Deu 13:18 если будешь слушать гласа Господа, Бога твоего, соблюдая все заповеди Его, которые ныне заповедую тебе, делая [доброе и] угодное пред очами Господа, Бога твоего.
\vs Deu 14:1 Вы сыны Господа Бога вашего; не делайте нарезов \bibemph{на теле вашем} и не выстригайте волос над глазами вашими по умершем;
\vs Deu 14:2 ибо ты народ святой у Господа Бога твоего, и тебя избрал Господь, чтобы ты был собственным Его народом из всех народов, которые на земле.
\rsbpar\vs Deu 14:3 Не ешь никакой мерзости.
\vs Deu 14:4 Вот скот, который вам можно есть: волы, овцы, козы,
\vs Deu 14:5 олень и серна, и буйвол, и лань, и зубр, и орикс, и камелопард.
\vs Deu 14:6 Всякий скот, у которого раздвоены копыта и на обоих копытах глубокий разрез, и который скот жует жвачку, тот ешьте;
\vs Deu 14:7 только сих не ешьте из жующих жвачку и имеющих раздвоенные копыта с глубоким разрезом: верблюда, зайца и тушканчика, потому что, хотя они жуют жвачку, но копыта у них не раздвоены: нечисты они для вас;
\vs Deu 14:8 и свиньи, потому что копыта у нее раздвоены, но не жует жвачки: нечиста она для вас; не ешьте мяса их, и к трупам их не прикасайтесь.
\vs Deu 14:9 Из всех \bibemph{животных}, которые в воде, ешьте всех, у которых есть перья и чешуя;
\vs Deu 14:10 а всех тех, у которых нет перьев и чешуи, не ешьте: нечисто это для вас.
\vs Deu 14:11 Всякую птицу чистую ешьте;
\vs Deu 14:12 но сих не должно вам есть из них: орла, грифа и морского орла,
\vs Deu 14:13 и коршуна, и сокола, и кречета с породою их,
\vs Deu 14:14 и всякого ворона с породою его,
\vs Deu 14:15 и страуса, и совы, и чайки, и ястреба с породою его,
\vs Deu 14:16 и филина, и ибиса, и лебедя,
\vs Deu 14:17 и пеликана, и сипа, и рыболова,
\vs Deu 14:18 и цапли, и зуя с породою его, и удода, и нетопыря.
\vs Deu 14:19 Все крылатые пресмыкающиеся нечисты для вас, не ешьте [их].
\vs Deu 14:20 Всякую птицу чистую ешьте.
\vs Deu 14:21 Не ешьте никакой мертвечины; иноземцу, который \bibemph{случится} в жилищах твоих, отдай ее, он пусть ест ее, или продай ему, ибо ты народ святой у Господа Бога твоего. Не вари козленка в молоке матери его.
\rsbpar\vs Deu 14:22 Отделяй десятину от всего произведения семян твоих, которое приходит с поля [твоего] каждогодно,
\vs Deu 14:23 и ешь пред Господом, Богом твоим, на том месте, которое изберет Он, чтобы пребывать имени Его там; [приноси] десятину хлеба твоего, вина твоего и елея твоего, и первенцев крупного скота твоего и мелкого скота твоего, дабы ты научился бояться Господа, Бога твоего, во все дни.
\vs Deu 14:24 Если же длинна будет для тебя дорога, так что ты не можешь нести сего, потому что далеко от тебя то место, которое изберет Господь, Бог твой, чтоб положить там имя Свое, и Господь, Бог твой, благословил тебя,
\vs Deu 14:25 то променяй это на серебро и возьми серебро в руку твою и приходи на место, которое изберет Господь, Бог твой;
\vs Deu 14:26 и покупай на серебро сие всего, чего пожелает душа твоя, волов, овец, вина, сикера и всего, чего потребует от тебя душа твоя; и ешь там пред Господом, Богом твоим, и веселись ты и семейство твое.
\vs Deu 14:27 И левита, который в жилищах твоих, не оставь, ибо нет ему части и удела с тобою.
\vs Deu 14:28 По прошествии же трех лет отделяй все десятины произведений твоих в тот год и клади [сие] в жилищах твоих;
\vs Deu 14:29 и пусть придет левит, ибо ему нет части и удела с тобою, и пришелец, и сирота, и вдова, которые \bibemph{находятся} в жилищах твоих, и пусть едят и насыщаются, дабы благословил тебя Господь, Бог твой, во всяком деле рук твоих, которое ты будешь делать.
\vs Deu 15:1 В седьмой год делай прощение.
\vs Deu 15:2 Прощение же состоит в том, чтобы всякий заимодавец, который дал взаймы ближнему своему, простил \bibemph{долг} и не взыскивал с ближнего своего или с брата своего, ибо провозглашено прощение ради Господа [Бога твоего];
\vs Deu 15:3 с иноземца взыскивай, а что будет твое у брата твоего, прости.
\vs Deu 15:4 Разве только не будет у тебя нищего: ибо благословит тебя Господь на той земле, которую Господь, Бог твой, дает тебе в удел, чтобы ты взял ее в наследство,
\vs Deu 15:5 если только будешь слушать гласа Господа, Бога твоего, и стараться исполнять все заповеди сии, которые я сегодня заповедую тебе;
\vs Deu 15:6 ибо Господь, Бог твой, благословит тебя, как Он говорил тебе, и ты будешь давать взаймы многим народам, а сам не будешь брать взаймы; и господствовать будешь над многими народами, а они над тобою не будут господствовать.
\vs Deu 15:7 Если же будет у тебя нищий кто-либо из братьев твоих, в одном из жилищ твоих, на земле твоей, которую Господь, Бог твой, дает тебе, то не ожесточи сердца твоего и не сожми руки твоей пред нищим братом твоим,
\vs Deu 15:8 но открой ему руку твою и дай ему взаймы, смотря по его нужде, в чем он нуждается;
\vs Deu 15:9 берегись, чтобы не вошла в сердце твое беззаконная мысль: <<приближается седьмой год, год прощения>>, и чтоб \bibemph{оттого} глаз твой не сделался немилостив к нищему брату твоему, и ты не отказал ему; ибо он возопиет на тебя к Господу, и будет на тебе [великий] грех;
\vs Deu 15:10 дай ему [и взаймы дай ему, сколько он просит и сколько ему нужно], и когда будешь давать ему, не должно скорбеть сердце твое, ибо за то благословит тебя Господь, Бог твой, во всех делах твоих и во всем, что будет делаться твоими руками;
\vs Deu 15:11 ибо нищие всегда будут среди земли [твоей]; потому я и повелеваю тебе: отверзай руку твою брату твоему, бедному твоему и нищему твоему на земле твоей.
\vs Deu 15:12 Если продастся тебе брат твой, Еврей, или Евреянка, то шесть лет должен он быть рабом тебе, а в седьмой год отпусти его от себя на свободу;
\vs Deu 15:13 когда же будешь отпускать его от себя на свободу, не отпусти его с пустыми \bibemph{руками},
\vs Deu 15:14 но снабди его от стад твоих, от гумна твоего и от точила твоего: дай ему, чем благословил тебя Господь, Бог твой:
\vs Deu 15:15 помни, что [и] ты был рабом в земле Египетской и избавил тебя Господь, Бог твой, потому я сегодня и заповедую тебе сие.
\vs Deu 15:16 Если же он скажет тебе: <<не пойду я от тебя, потому что я люблю тебя и дом твой>>, потому что хорошо ему у тебя,
\vs Deu 15:17 то возьми шило и проколи ухо его к двери; и будет он рабом твоим на век. Так поступай и с рабою твоею.
\vs Deu 15:18 Не считай этого для себя тяжким, что ты должен отпустить его от себя на свободу, ибо он в шесть лет заработал тебе вдвое против платы наемника; и благословит тебя Господь, Бог твой, во всем, что ни будешь делать.
\rsbpar\vs Deu 15:19 Все первородное мужеского пола, что родится от крупного скота твоего и от мелкого скота твоего, посвящай Господу, Богу твоему: не работай на первородном воле твоем и не стриги первородного из мелкого скота твоего;
\vs Deu 15:20 пред Господом, Богом твоим, каждогодно съедай это ты и семейство твое, на месте, которое изберет Господь [Бог твой];
\vs Deu 15:21 если же будет на нем порок, хромота или слепота [или] другой какой-нибудь порок, то не приноси его в жертву Господу, Богу твоему,
\vs Deu 15:22 но в жилищах твоих ешь его; нечистый, как и чистый, [могут есть,] как серну и как оленя;
\vs Deu 15:23 только крови его не ешь: на землю выливай ее, как воду.
\vs Deu 16:1 Наблюдай месяц Авив, и совершай Пасху Господу, Богу твоему, потому что в месяце Авиве вывел тебя Господь, Бог твой, из Египта ночью.
\vs Deu 16:2 И заколай Пасху Господу, Богу твоему, из мелкого и крупного скота на месте, которое изберет Господь, чтобы пребывало там имя Его.
\vs Deu 16:3 Не ешь с нею квасного; семь дней ешь с нею опресноки, хлебы бедствия, ибо ты с поспешностью вышел из земли Египетской, дабы ты помнил день исшествия своего из земли Египетской во все дни жизни твоей;
\vs Deu 16:4 не должно находиться у тебя ничто квасное во всем уделе твоем в продолжение семи дней, и из мяса, которое ты принес в жертву вечером в первый день, ничто не должно оставаться до утра.
\vs Deu 16:5 Не можешь ты заколать Пасху в котором-нибудь из жилищ твоих, которые Господь, Бог твой, даст тебе;
\vs Deu 16:6 но только на том месте, которое изберет Господь, Бог твой, чтобы пребывало там имя Его, заколай Пасху вечером при захождении солнца, в то самое время, в которое ты вышел из Египта;
\vs Deu 16:7 и испеки и съешь на том месте, которое изберет Господь, Бог твой, а на другой день можешь возвратиться и войти в шатры твои.
\vs Deu 16:8 Шесть дней ешь пресные хлебы, а в седьмой день отдание праздника Господу, Богу твоему; не занимайся работою.
\rsbpar\vs Deu 16:9 Семь седмиц отсчитай себе; начинай считать семь седмиц с того времени, как появится серп на жатве;
\vs Deu 16:10 тогда совершай праздник седмиц Господу, Богу твоему, по усердию руки твоей, сколько ты дашь, смотря по тому, чем благословит тебя Господь, Бог твой;
\vs Deu 16:11 и веселись пред Господом, Богом твоим, ты, и сын твой, и дочь твоя, и раб твой, и раба твоя, и левит, который в жилищах твоих, и пришелец, и сирота, и вдова, которые среди тебя, на месте, которое изберет Господь, Бог твой, чтобы пребывало там имя Его;
\vs Deu 16:12 помни, что ты был рабом в Египте, и соблюдай и исполняй постановления сии.
\rsbpar\vs Deu 16:13 Праздник кущей совершай у себя семь дней, когда уберешь с гумна твоего и из точила твоего;
\vs Deu 16:14 и веселись в праздник твой ты и сын твой, и дочь твоя, и раб твой, и раба твоя, и левит, и пришелец, и сирота, и вдова, которые в жилищах твоих;
\vs Deu 16:15 семь дней празднуй Господу, Богу твоему, на месте, которое изберет Господь, Бог твой, [чтобы призываемо было там имя Его]; ибо благословит тебя Господь, Бог твой, во всех произведениях твоих и во всяком деле рук твоих, и ты будешь только веселиться.
\rsbpar\vs Deu 16:16 Три раза в году весь мужеский пол должен являться пред лице Господа, Бога твоего, на место, которое изберет Он: в праздник опресноков, в праздник седмиц и в праздник кущей; и \bibemph{никто} не должен являться пред лице Господа с пустыми \bibemph{руками},
\vs Deu 16:17 но каждый с даром в руке своей, смотря по благословению Господа, Бога твоего, какое Он дал тебе.
\rsbpar\vs Deu 16:18 Во всех жилищах твоих, которые Господь, Бог твой, даст тебе, поставь себе судей и надзирателей по коленам твоим, чтоб они судили народ судом праведным;
\vs Deu 16:19 не извращай закона, не смотри на лица и не бери даров, ибо дары ослепляют глаза мудрых и превращают дело правых;
\vs Deu 16:20 правды, правды ищи, дабы ты был жив и овладел землею, которую Господь, Бог твой, дает тебе.
\vs Deu 16:21 Не сади себе рощи из каких-либо дерев при жертвеннике Господа, Бога твоего, который ты сделаешь себе,
\vs Deu 16:22 и не ставь себе столба, что ненавидит Господь Бог твой.
\vs Deu 17:1 Не приноси в жертву Господу, Богу твоему, вола, или овцы, на которой будет порок, \bibemph{или} что-нибудь худое, ибо это мерзость для Господа, Бога твоего.
\vs Deu 17:2 Если найдется среди тебя в каком-либо из жилищ твоих, которые Господь, Бог твой, дает тебе, мужчина или женщина, кто сделает зло пред очами Господа, Бога твоего, преступив завет Его,
\vs Deu 17:3 и пойдет и станет служить иным богам, и поклонится им, или солнцу, или луне, или всему воинству небесному, чего я не повелел,
\vs Deu 17:4 и тебе возвещено будет, и ты услышишь, то ты хорошо разыщи; и если это точная правда, если сделана мерзость сия в Израиле,
\vs Deu 17:5 то выведи мужчину того, или женщину ту, которые сделали зло сие, к воротам твоим и побей их камнями до смерти.
\vs Deu 17:6 По словам двух свидетелей, или трех свидетелей, должен умереть осуждаемый на смерть: не должно предавать смерти по словам одного свидетеля;
\vs Deu 17:7 рука свидетелей должна быть на нем прежде \bibemph{всех}, чтоб убить его, потом рука всего народа; и \bibemph{так} истреби зло из среды себя.
\vs Deu 17:8 Если по какому делу затруднительным будет для тебя рассудить между кровью и кровью, между судом и судом, между побоями и побоями, \bibemph{и будут} несогласные мнения в воротах твоих, то встань и пойди на место, которое изберет Господь, Бог твой, [чтобы призываемо было там имя Его,]
\vs Deu 17:9 и приди к священникам левитам и к судье, который будет в те дни, и спроси их, и они скажут тебе, как рассудить;
\vs Deu 17:10 и поступи по слову, какое они скажут тебе, на том месте, которое изберет Господь [Бог твой, чтобы призываемо было там имя Его,] и постарайся исполнить все, чему они научат тебя;
\vs Deu 17:11 по закону, которому научат они тебя, и по определению, какое они скажут тебе, поступи, и не уклоняйся ни направо, ни налево от того, что они скажут тебе.
\vs Deu 17:12 А кто поступит так дерзко, что не послушает священника, стоящего там на служении пред Господом, Богом твоим, или судьи, [который будет в те дни], тот должен умереть,~--- и \bibemph{так} истреби зло от Израиля;
\vs Deu 17:13 и весь народ услышит и убоится, и не будут впредь поступать дерзко.
\rsbpar\vs Deu 17:14 Когда ты придешь в землю, которую Господь, Бог твой, дает тебе, и овладеешь ею, и поселишься на ней, и скажешь: <<поставлю я над собою царя, подобно прочим народам, которые вокруг меня>>,
\vs Deu 17:15 то поставь над собою царя, которого изберет Господь, Бог твой; из среды братьев твоих поставь над собою царя; не можешь поставить над собою [царем] иноземца, который не брат тебе.
\vs Deu 17:16 Только чтоб он не умножал себе коней и не возвращал народа в Египет для умножения себе коней, ибо Господь сказал вам: <<не возвращайтесь более путем сим>>;
\vs Deu 17:17 и чтобы не умножал себе жен, дабы не развратилось сердце его, и чтобы серебра и золота не умножал себе чрезмерно.
\vs Deu 17:18 Но когда он сядет на престоле царства своего, должен списать для себя список закона сего с книги, \bibemph{находящейся} у священников левитов,
\vs Deu 17:19 и пусть он будет у него, и пусть он читает его во все дни жизни своей, дабы научался бояться Господа, Бога своего, и старался исполнять все слова закона сего и постановления сии;
\vs Deu 17:20 чтобы не надмевалось сердце его пред братьями его, и чтобы не уклонялся он от закона ни направо, ни налево, дабы долгие дни пребыл на царстве своем он и сыновья его посреди Израиля.
\vs Deu 18:1 Священникам левитам, всему колену Левиину, не будет части и удела с Израилем: они должны питаться жертвами Господа и Его частью;
\vs Deu 18:2 удела же не будет ему между братьями его: Сам Господь удел его, как говорил Он ему.
\vs Deu 18:3 Вот что должно быть положено священникам от народа, от приносящих в жертву волов или овец: должно отдавать священнику плечо, челюсти и желудок;
\vs Deu 18:4 также начатки от хлеба твоего, вина твоего и елея твоего, и начатки от шерсти овец твоих отдавай ему,
\vs Deu 18:5 ибо его избрал Господь Бог твой из всех колен твоих, чтобы он предстоял [пред Господом, Богом твоим], служил [и благословлял] во имя Господа, сам и сыны его во все дни.
\vs Deu 18:6 И если левит придет из одного из жилищ твоих, из всей \bibemph{земли} [сынов] Израилевых, где он жил, и придет по желанию души своей на место, которое изберет Господь,
\vs Deu 18:7 и будет служить во имя Господа Бога своего, как и все братья его левиты, предстоящие там пред Господом,~---
\vs Deu 18:8 то пусть они пользуются одинаковою частью, сверх полученного от продажи отцовского \bibemph{имущества}.
\rsbpar\vs Deu 18:9 Когда ты войдешь в землю, которую дает тебе Господь Бог твой, тогда не научись делать мерзости, какие делали народы сии:
\vs Deu 18:10 не должен находиться у тебя проводящий сына своего или дочь свою чрез огонь, прорицатель, гадатель, ворожея, чародей,
\vs Deu 18:11 обаятель, вызывающий духов, волшебник и вопрошающий мертвых;
\vs Deu 18:12 ибо мерзок пред Господом всякий, делающий это, и за сии-то мерзости Господь Бог твой изгоняет их от лица твоего;
\vs Deu 18:13 будь непорочен пред Господом Богом твоим;
\vs Deu 18:14 ибо народы сии, которых ты изгоняешь, слушают гадателей и прорицателей, а тебе не то дал Господь Бог твой.
\vs Deu 18:15 Пророка из среды тебя, из братьев твоих, как меня, воздвигнет тебе Господь Бог твой,~--- Его слушайте,~---
\vs Deu 18:16 так как ты просил у Господа Бога твоего при Хориве в день собрания, говоря: да не услышу впредь гласа Господа Бога моего и огня сего великого да не увижу более, дабы мне не умереть.
\vs Deu 18:17 И сказал мне Господь: хорошо то, что они говорили [тебе];
\vs Deu 18:18 Я воздвигну им Пророка из среды братьев их, такого как ты, и вложу слова Мои в уста Его, и Он будет говорить им все, что Я повелю Ему;
\vs Deu 18:19 а кто не послушает слов Моих, которые [Пророк тот] будет говорить Моим именем, с того Я взыщу;
\vs Deu 18:20 но пророка, который дерзнет говорить Моим именем то, чего Я не повелел ему говорить, и который будет говорить именем богов иных, такого пророка предайте смерти.
\vs Deu 18:21 И если скажешь в сердце твоем: <<как мы узнаем слово, которое не Господь говорил?>>
\vs Deu 18:22 Если пророк скажет именем Господа, но слово то не сбудется и не исполнится, то не Господь говорил сие слово, но говорил сие пророк по дерзости своей,~--- не бойся его.
\vs Deu 19:1 Когда Господь Бог твой истребит народы, которых землю дает тебе Господь Бог твой и ты вступишь в наследие после них, и поселишься в городах их и домах их,
\vs Deu 19:2 тогда отдели себе три города среди земли твоей, которую Господь Бог твой дает тебе во владение;
\vs Deu 19:3 устрой себе дорогу и раздели на три части всю землю твою, которую Господь Бог твой дает тебе в удел; они будут служить убежищем всякому убийце.
\vs Deu 19:4 И вот какой убийца может убегать туда и остаться жив: кто убьет ближнего своего без намерения, не быв врагом ему вчера и третьего дня;
\vs Deu 19:5 кто пойдет с ближним своим в лес рубить дрова, и размахнется рука его с топором, чтобы срубить дерево, и соскочит железо с топорища и попадет в ближнего, и он умрет,~--- такой пусть убежит в один из городов тех, чтоб остаться живым,
\vs Deu 19:6 дабы мститель за кровь в горячности сердца своего не погнался за убийцею и не настиг его, если далек будет путь, и не убил его, между тем как он не \bibemph{подлежит} осуждению на смерть, ибо не был врагом ему вчера и третьего дня;
\vs Deu 19:7 посему я и дал тебе повеление, говоря: отдели себе три города.
\vs Deu 19:8 Когда же Господь Бог твой распространит пределы твои, как Он клялся отцам твоим, и даст тебе всю землю, которую Он обещал дать отцам твоим,
\vs Deu 19:9 если ты будешь стараться исполнять все сии заповеди, которые я заповедую тебе сегодня, любить Господа Бога твоего и ходить путями Его во все дни,~--- тогда к сим трем городам прибавь еще три города,
\vs Deu 19:10 дабы не проливалась кровь невинного среди земли твоей, которую Господь Бог твой дает тебе в удел, и чтобы не было на тебе [вины] крови.
\vs Deu 19:11 Но если кто [у тебя] будет врагом ближнему своему и будет подстерегать его, и восстанет на него и убьет его до смерти, и убежит в один из городов тех,
\vs Deu 19:12 то старейшины города его должны послать, чтобы взять его оттуда и предать его в руки мстителя за кровь, чтоб он умер;
\vs Deu 19:13 да не пощадит его глаз твой; смой с Израиля кровь невинного, и будет тебе хорошо.
\rsbpar\vs Deu 19:14 Не нарушай межи ближнего твоего, которую положили предки в уделе твоем, доставшемся тебе в земле, которую Господь Бог твой дает тебе во владение.
\rsbpar\vs Deu 19:15 Недостаточно одного свидетеля против кого-либо в какой-нибудь вине и в каком-нибудь преступлении и в каком-нибудь грехе, которым он согрешит: при словах двух свидетелей, или при словах трех свидетелей состоится [всякое] дело.
\vs Deu 19:16 Если выступит против кого свидетель несправедливый, обвиняя его в преступлении,
\vs Deu 19:17 то пусть предстанут оба сии человека, у которых тяжба, пред Господа, пред священников и пред судей, которые будут в те дни;
\vs Deu 19:18 судьи должны хорошо исследовать, и если свидетель тот свидетель ложный, ложно донес на брата своего,
\vs Deu 19:19 то сделайте ему то, что он умышлял сделать брату своему; и \bibemph{так} истреби зло из среды себя;
\vs Deu 19:20 и прочие услышат, и убоятся, и не станут впредь делать такое зло среди тебя;
\vs Deu 19:21 да не пощадит [его] глаз твой: душу за душу, глаз за глаз, зуб за зуб, руку за руку, ногу за ногу. [Какой кто сделает вред ближнему своему, тем должно отплатить ему.]
\vs Deu 20:1 Когда ты выйдешь на войну против врага твоего и увидишь коней и колесницы [и] народа более, нежели у тебя, то не бойся их, ибо с тобою Господь Бог твой, Который вывел тебя из земли Египетской.
\vs Deu 20:2 Когда же приступаете к сражению, тогда пусть подойдет священник, и говорит народу,
\vs Deu 20:3 и скажет ему: слушай, Израиль! вы сегодня вступаете в сражение с врагами вашими, да не ослабеет сердце ваше, не бойтесь, не смущайтесь и не ужасайтесь их,
\vs Deu 20:4 ибо Господь Бог ваш идет с вами, чтобы сразиться за вас с врагами вашими [и] спасти вас.
\vs Deu 20:5 Надзиратели же пусть объявят народу, говоря: кто построил новый дом и не обновил его, тот пусть идет и возвратится в дом свой, дабы не умер на сражении, и другой не обновил его;
\vs Deu 20:6 и кто насадил виноградник и не пользовался им, тот пусть идет и возвратится в дом свой, дабы не умер на сражении, и другой не воспользовался им;
\vs Deu 20:7 и кто обручился с женою и не взял ее, тот пусть идет и возвратится в дом свой, дабы не умер на сражении, и другой не взял ее.
\vs Deu 20:8 И еще объявят надзиратели народу, и скажут: кто боязлив и малодушен, тот пусть идет и возвратится в дом свой, дабы он не сделал робкими сердца братьев его, как его сердце.
\vs Deu 20:9 Когда надзиратели скажут все это народу, тогда должно поставить военных начальников в вожди народу.
\rsbpar\vs Deu 20:10 Когда подойдешь к городу, чтобы завоевать его, предложи ему мир;
\vs Deu 20:11 если он согласится на мир с тобою и отворит тебе \bibemph{ворота}, то весь народ, который найдется в нем, будет платить тебе дань и служить тебе;
\vs Deu 20:12 если же он не согласится на мир с тобою и будет вести с тобою войну, то осади его,
\vs Deu 20:13 и \bibemph{когда} Господь Бог твой предаст его в руки твои, порази в нем весь мужеский пол острием меча;
\vs Deu 20:14 только жен и детей и скот и все, что в городе, всю добычу его возьми себе и пользуйся добычею врагов твоих, которых предал тебе Господь Бог твой;
\vs Deu 20:15 так поступай со всеми городами, которые от тебя весьма далеко, которые не из \bibemph{числа} городов народов сих.
\vs Deu 20:16 А в городах сих народов, которых Господь Бог твой дает тебе во владение, не оставляй в живых ни одной души,
\vs Deu 20:17 но предай их заклятию: Хеттеев и Аморреев, и Хананеев, и Ферезеев, и Евеев, и Иевусеев, [и Гергесеев,] как повелел тебе Господь Бог твой,
\vs Deu 20:18 дабы они не научили вас делать такие же мерзости, какие они делали для богов своих, и дабы вы не грешили пред Господом Богом вашим.
\vs Deu 20:19 Если долгое время будешь держать в осаде [какой-нибудь] город, чтобы завоевать его и взять его, то не порти дерев его, от которых можно питаться, и не опустошай окрестностей, ибо дерево на поле не человек, чтобы могло уйти от тебя в укрепление;
\vs Deu 20:20 только те дерева, о которых ты знаешь, что они ничего не приносят в пищу, можешь портить и рубить, и строить укрепление против города, который ведет с тобою войну, доколе не покоришь его.
\vs Deu 21:1 Если в земле, которую Господь, Бог твой, дает тебе во владение, найден будет убитый, лежащий на поле, и неизвестно, кто убил его,
\vs Deu 21:2 то пусть выйдут старейшины твои и судьи твои и измерят \bibemph{расстояние} до городов, которые вокруг убитого;
\vs Deu 21:3 и старейшины города того, который будет ближайшим к убитому, пусть возьмут телицу, на которой не работали, [и] которая не носила ярма,
\vs Deu 21:4 и пусть старейшины того города отведут сию телицу в дикую долину, которая не разработана и не засеяна, и заколют там телицу в долине;
\vs Deu 21:5 и придут священники, сыны Левиины [ибо их избрал Господь Бог твой служить Ему и благословлять именем Господа, и по слову их должно \bibemph{решить} всякое спорное дело и всякий причиненный вред,]
\vs Deu 21:6 и все старейшины города того, ближайшие к убитому, пусть омоют руки свои над [головою] телицы, зарезанной в долине,
\vs Deu 21:7 и объявят и скажут: руки наши не пролили крови сей, и глаза наши не видели;
\vs Deu 21:8 очисти народ Твой, Израиля, который Ты, Господи, освободил [из земли Египетской], и не вмени народу Твоему, Израилю, невинной крови. И они очистятся от крови.
\vs Deu 21:9 \bibemph{Так} должен ты смывать у себя кровь невинного, если хочешь делать [доброе и] справедливое пред очами Господа [Бога твоего].
\rsbpar\vs Deu 21:10 Когда выйдешь на войну против врагов твоих, и Господь Бог твой предаст их в руки твои, и возьмешь их в плен,
\vs Deu 21:11 и увидишь между пленными женщину, красивую видом, и полюбишь ее, и захочешь взять ее себе в жену,
\vs Deu 21:12 то приведи ее в дом свой, и пусть она острижет голову свою и обрежет ногти свои,
\vs Deu 21:13 и снимет с себя пленническую одежду свою, и живет в доме твоем, и оплакивает отца своего и матерь свою в продолжение месяца; и после того ты можешь войти к ней и сделаться ее мужем, и она будет твоею женою;
\vs Deu 21:14 если же она \bibemph{после} не понравится тебе, то отпусти ее, \bibemph{куда} она захочет, но не продавай ее за серебро и не обращай ее в рабство, потому что ты смирил ее.
\rsbpar\vs Deu 21:15 Если у кого будут две жены~--- одна любимая, а другая нелюбимая, и как любимая, \bibemph{так} и нелюбимая родят ему сыновей, и первенцем будет сын нелюбимой,~---
\vs Deu 21:16 то, при разделе сыновьям своим имения своего, он не может сыну жены любимой дать первенство пред первородным сыном нелюбимой;
\vs Deu 21:17 но первенцем должен признать сына нелюбимой [и] дать ему двойную часть из всего, что у него найдется, ибо он есть начаток силы его, ему \bibemph{принадлежит} право первородства.
\rsbpar\vs Deu 21:18 Если у кого будет сын буйный и непокорный, не повинующийся голосу отца своего и голосу матери своей, и они наказывали его, но он не слушает их,~---
\vs Deu 21:19 то отец его и мать его пусть возьмут его и приведут его к старейшинам города своего и к воротам своего местопребывания
\vs Deu 21:20 и скажут старейшинам города своего: <<сей сын наш буен и непокорен, не слушает слов наших, мот и пьяница>>;
\vs Deu 21:21 тогда все жители города его пусть побьют его камнями до смерти; и \bibemph{так} истреби зло из среды себя, и все Израильтяне услышат и убоятся.
\rsbpar\vs Deu 21:22 Если в ком найдется преступление, достойное смерти, и он будет умерщвлен, и ты повесишь его на дереве,
\vs Deu 21:23 то тело его не должно ночевать на дереве, но погреби его в тот же день, ибо проклят пред Богом [всякий] повешенный [на дереве], и не оскверняй земли твоей, которую Господь Бог твой дает тебе в удел.
\vs Deu 22:1 Когда увидишь вола брата твоего или овцу его заблудившихся, не оставляй их, но возврати их брату твоему;
\vs Deu 22:2 если же не близко будет к тебе брат твой, или ты не знаешь его, то прибери их в дом свой, и пусть они будут у тебя, доколе брат твой не будет искать их, и тогда возврати ему их;
\vs Deu 22:3 так поступай и с ослом его, так поступай с одеждой его, так поступай со всякою потерянною \bibemph{вещью} брата твоего, которая будет им потеряна и которую ты найдешь; нельзя тебе уклоняться \bibemph{от сего}.
\vs Deu 22:4 Когда увидишь осла брата твоего или вола его упадших на пути, не оставляй их, но подними их с ним вместе.
\rsbpar\vs Deu 22:5 На женщине не должно быть мужской одежды, и мужчина не должен одеваться в женское платье, ибо мерзок пред Господом Богом твоим всякий делающий сие.
\rsbpar\vs Deu 22:6 Если попадется тебе на дороге птичье гнездо на каком-либо дереве или на земле, с птенцами или яйцами, и мать сидит на птенцах или на яйцах, то не бери матери вместе с детьми:
\vs Deu 22:7 мать пусти, а детей возьми себе, чтобы тебе было хорошо, и чтобы продлились дни твои.
\rsbpar\vs Deu 22:8 Если будешь строить новый дом, то сделай перила около кровли твоей, чтобы не навести тебе крови на дом твой, когда кто-нибудь упадет с него.
\rsbpar\vs Deu 22:9 Не засевай виноградника своего двумя родами семян, чтобы не сделать тебе заклятым сбора семян, которые ты посеешь вместе с плодами виноградника [своего].
\vs Deu 22:10 Не паши на воле и осле вместе.
\vs Deu 22:11 Не надевай одежды, сделанной из разных веществ, из шерсти и льна вместе.
\vs Deu 22:12 Сделай себе кисточки на четырех углах покрывала твоего, которым ты покрываешься.
\rsbpar\vs Deu 22:13 Если кто возьмет жену, и войдет к ней, и возненавидит ее,
\vs Deu 22:14 и будет возводить на нее порочные дела, и пустит о ней худую молву, и скажет: <<я взял сию жену, и вошел к ней, и не нашел у нее девства>>,
\vs Deu 22:15 то отец отроковицы и мать ее пусть возьмут и вынесут \bibemph{признаки} девства отроковицы к старейшинам города, к воротам;
\vs Deu 22:16 и отец отроковицы скажет старейшинам: дочь мою я отдал в жену сему человеку, и [ныне] он возненавидел ее,
\vs Deu 22:17 и вот, он взводит [на нее] порочные дела, говоря: <<я не нашел у дочери твоей девства>>; но вот признаки девства дочери моей. И расстелют одежду пред старейшинами города.
\vs Deu 22:18 Тогда старейшины того города пусть возьмут мужа и накажут его,
\vs Deu 22:19 и наложат на него сто \bibemph{сиклей} серебра пени и отдадут отцу отроковицы за то, что он пустил худую молву о девице Израильской; она же пусть останется его женою, и он не может развестись с нею во всю жизнь свою.
\vs Deu 22:20 Если же сказанное будет истинно, и не найдется девства у отроковицы,
\vs Deu 22:21 то отроковицу пусть приведут к дверям дома отца ее, и жители города ее побьют ее камнями до смерти, ибо она сделала срамное дело среди Израиля, блудодействовав в доме отца своего; и \bibemph{так} истреби зло из среды себя.
\vs Deu 22:22 Если найден будет кто лежащий с женою замужнею, то должно предать смерти обоих: и мужчину, лежавшего с женщиною, и женщину; и \bibemph{так} истреби зло от Израиля.
\vs Deu 22:23 Если будет молодая девица обручена мужу, и кто-нибудь встретится с нею в городе и ляжет с нею,
\vs Deu 22:24 то обоих их приведите к воротам того города, и побейте их камнями до смерти: отроковицу за то, что она не кричала в городе, а мужчину за то, что он опорочил жену ближнего своего; и \bibemph{так} истреби зло из среды себя.
\vs Deu 22:25 Если же кто в поле встретится с отроковицею обрученною и, схватив ее, ляжет с нею, то должно предать смерти только мужчину, лежавшего с нею,
\vs Deu 22:26 а отроковице ничего не делай; на отроковице нет преступления смертного: ибо это то же, как если бы кто восстал на ближнего своего и убил его;
\vs Deu 22:27 ибо он встретился с нею в поле, и \bibemph{хотя} отроковица обрученная кричала, но некому было спасти ее.
\vs Deu 22:28 Если кто-нибудь встретится с девицею необрученною, и схватит ее и ляжет с нею, и застанут их,
\vs Deu 22:29 то лежавший с нею должен дать отцу отроковицы пятьдесят [сиклей] серебра, а она пусть будет его женою, потому что он опорочил ее; во всю жизнь свою он не может развестись с нею.
\rsbpar\vs Deu 22:30 Никто не должен брать жены отца своего и открывать край \bibemph{одежды} отца своего.
\vs Deu 23:1 У кого раздавлены ятра или отрезан детородный член, тот не может войти в общество Господне.
\vs Deu 23:2 Сын блудницы не может войти в общество Господне, и десятое поколение его не может войти в общество Господне.
\vs Deu 23:3 Аммонитянин и Моавитянин не может войти в общество Господне, и десятое поколение их не может войти в общество Господне во веки,
\vs Deu 23:4 потому что они не встретили вас с хлебом и водою на пути, когда вы шли из Египта, и потому что они наняли против тебя Валаама, сына Веорова, из Пефора Месопотамского, чтобы проклясть тебя;
\vs Deu 23:5 но Господь, Бог твой, не восхотел слушать Валаама и обратил Господь Бог твой проклятие его в благословение тебе, ибо Господь Бог твой любит тебя.
\vs Deu 23:6 Не желай им мира и благополучия во все дни твои, во веки.
\vs Deu 23:7 Не гнушайся Идумеянином, ибо он брат твой; не гнушайся Египтянином, ибо ты был пришельцем в земле его;
\vs Deu 23:8 дети, которые у них родятся, в третьем поколении могут войти в общество Господне.
\rsbpar\vs Deu 23:9 Когда пойдешь в поход против врагов твоих, берегись всего худого.
\vs Deu 23:10 Если у тебя будет кто нечист от случившегося [ему] ночью, то он должен выйти вон из стана и не входить в стан,
\vs Deu 23:11 а при наступлении вечера должен омыть [тело свое] водою, и по захождении солнца может войти в стан.
\vs Deu 23:12 Место должно быть у тебя вне стана, куда бы тебе выходить;
\vs Deu 23:13 кроме оружия твоего должна быть у тебя лопатка; и когда будешь садиться вне \bibemph{стана}, выкопай ею [яму] и опять зарой [ею] испражнение твое;
\vs Deu 23:14 ибо Господь Бог твой ходит среди стана твоего, чтобы избавлять тебя и предавать врагов твоих [в руки твои], а \bibemph{посему} стан твой должен быть свят, чтобы Он не увидел у тебя чего срамного и не отступил от тебя.
\rsbpar\vs Deu 23:15 Не выдавай раба господину его, когда он прибежит к тебе от господина своего;
\vs Deu 23:16 пусть он у тебя живет, среди вас [пусть он живет] на месте, которое он изберет в каком-нибудь из жилищ твоих, где ему понравится; не притесняй его.
\rsbpar\vs Deu 23:17 Не должно быть блудницы из дочерей Израилевых и не должно быть блудника из сынов Израилевых.
\vs Deu 23:18 Не вноси платы блудницы и цены пса в дом Господа Бога твоего ни по какому обету, ибо то и другое есть мерзость пред Господом Богом твоим.
\rsbpar\vs Deu 23:19 Не отдавай в рост брату твоему ни серебра, ни хлеба, ни чего-либо другого, что \bibemph{можно} отдавать в рост;
\vs Deu 23:20 иноземцу отдавай в рост, а брату твоему не отдавай в рост, чтобы Господь Бог твой благословил тебя во всем, что делается руками твоими, на земле, в которую ты идешь, чтобы овладеть ею.
\rsbpar\vs Deu 23:21 Если дашь обет Господу Богу твоему, немедленно исполни его, ибо Господь Бог твой взыщет его с тебя, и на тебе будет грех;
\vs Deu 23:22 если же ты не дал обета, то не будет на тебе греха.
\vs Deu 23:23 Что вышло из уст твоих, соблюдай и исполняй так, как обещал ты Господу Богу твоему добровольное приношение, о котором сказал ты устами своими.
\rsbpar\vs Deu 23:24 Когда войдешь в виноградник ближнего твоего, можешь есть ягоды досыта, сколько \bibemph{хочет} душа твоя, а в сосуд твой не клади.
\vs Deu 23:25 Когда придешь на жатву ближнего твоего, срывай колосья руками твоими, но серпа не заноси на жатву ближнего твоего.
\vs Deu 24:1 Если кто возьмет жену и сделается ее мужем, и она не найдет благоволения в глазах его, потому что он находит в ней что-нибудь противное, и напишет ей разводное письмо, и даст ей в руки, и отпустит ее из дома своего,
\vs Deu 24:2 и она выйдет из дома его, пойдет, и выйдет за другого мужа,
\vs Deu 24:3 но и сей последний муж возненавидит ее и напишет ей разводное письмо, и даст ей в руки, и отпустит ее из дома своего, или умрет сей последний муж ее, взявший ее себе в жену,~---
\vs Deu 24:4 то не может первый ее муж, отпустивший ее, опять взять ее себе в жену, после того как она осквернена, ибо сие есть мерзость пред Господом [Богом твоим], и не порочь земли, которую Господь Бог твой дает тебе в удел.
\rsbpar\vs Deu 24:5 Если кто взял жену недавно, то пусть не идет на войну, и ничего не должно возлагать на него; пусть он остается свободен в доме своем в продолжение одного года и увеселяет жену свою, которую взял.
\rsbpar\vs Deu 24:6 Никто не должен брать в залог верхнего и нижнего жернова, ибо таковой берет в залог душу.
\rsbpar\vs Deu 24:7 Если найдут кого, что он украл кого-нибудь из братьев своих, из сынов Израилевых, и поработил его, и продал его, то такого вора должно предать смерти; и \bibemph{так} истреби зло из среды себя.
\rsbpar\vs Deu 24:8 Смотри, в язве проказы тщательно соблюдай и исполняй весь [закон], которому научат вас священники левиты; тщательно исполняйте, что я повелел им;
\vs Deu 24:9 помни, что Господь Бог твой сделал Мариами на пути, когда вы шли из Египта.
\rsbpar\vs Deu 24:10 Если ты ближнему твоему дашь что-нибудь взаймы, то не ходи к нему в дом, чтобы взять у него залог,
\vs Deu 24:11 постой на улице, а тот, которому ты дал взаймы, вынесет тебе залог свой на улицу;
\vs Deu 24:12 если же он будет человек бедный, то ты не ложись спать, имея [у себя] залог его:
\vs Deu 24:13 возврати ему залог при захождении солнца, чтоб он лег спать в одежде своей и благословил тебя,~--- и тебе поставится \bibemph{сие} в праведность пред Господом Богом твоим.
\vs Deu 24:14 Не обижай наемника, бедного и нищего, из братьев твоих или из пришельцев твоих, которые в земле твоей, в жилищах твоих;
\vs Deu 24:15 в тот же день отдай плату его, чтобы солнце не зашло прежде того, ибо он беден, и ждет ее душа его; чтоб он не возопил на тебя к Господу, и не было на тебе греха.
\rsbpar\vs Deu 24:16 Отцы не должны быть наказываемы смертью за детей, и дети не должны быть наказываемы смертью за отцов; каждый должен быть наказываем смертью за свое преступление.
\rsbpar\vs Deu 24:17 Не суди превратно пришельца, сироту [и вдову], и у вдовы не бери одежды в залог;
\vs Deu 24:18 помни, что и ты был рабом в Египте, и Господь [Бог твой] освободил тебя оттуда: посему я и повелеваю тебе делать сие.
\rsbpar\vs Deu 24:19 Когда будешь жать на поле твоем, и забудешь сноп на поле, то не возвращайся взять его; пусть он остается пришельцу, [нищему,] сироте и вдове, чтобы Господь Бог твой благословил тебя во всех делах рук твоих.
\vs Deu 24:20 Когда будешь обивать маслину твою, то не пересматривай за собою ветвей: пусть остается пришельцу, сироте и вдове. [И помни, что ты был рабом в земле Египетской: посему я и повелеваю тебе делать сие.]
\vs Deu 24:21 Когда будешь снимать плоды в винограднике твоем, не собирай остатков за собою: пусть остается пришельцу, сироте и вдове;
\vs Deu 24:22 и помни, что ты был рабом в земле Египетской: посему я и повелеваю тебе делать сие.
\vs Deu 25:1 Если будет тяжба между людьми, то пусть приведут их в суд и рассудят их, правого пусть оправдают, а виновного осудят;
\vs Deu 25:2 и если виновный достоин будет побоев, то судья пусть прикажет положить его и бить при себе, смотря по вине его, по счету;
\vs Deu 25:3 сорок ударов можно дать ему, а не более, чтобы от многих ударов брат твой не был обезображен пред глазами твоими.
\rsbpar\vs Deu 25:4 Не заграждай рта волу, когда он молотит.
\rsbpar\vs Deu 25:5 Если братья живут вместе и один из них умрет, не имея у себя сына, то жена умершего не должна выходить на сторону за человека чужого, но деверь ее должен войти к ней и взять ее себе в жену, и жить с нею,~---
\vs Deu 25:6 и первенец, которого она родит, останется с именем брата его умершего, чтоб имя его не изгладилось в Израиле.
\vs Deu 25:7 Если же он не захочет взять невестку свою, то невестка его пойдет к воротам, к старейшинам, и скажет: <<деверь мой отказывается восставить имя брата своего в Израиле, не хочет жениться на мне>>;
\vs Deu 25:8 тогда старейшины города его должны призвать его и уговаривать его, и если он станет и скажет: <<не хочу взять ее>>,
\vs Deu 25:9 \bibemph{тогда} невестка его пусть пойдет к нему в глазах старейшин, и снимет сапог его с ноги его, и плюнет в лице его, и скажет: <<так поступают с человеком, который не созидает дома брату своему [у Израиля]>>;
\vs Deu 25:10 и нарекут ему имя в Израиле: дом разутого.
\rsbpar\vs Deu 25:11 Когда дерутся между собою мужчины, и жена одного [из них] подойдет, чтобы отнять мужа своего из рук бьющего его, и протянув руку свою, схватит его за срамный уд,
\vs Deu 25:12 то отсеки руку ее: да не пощадит [ее] глаз твой.
\rsbpar\vs Deu 25:13 В кисе твоей не должны быть двоякие гири, б\acc{о}льшие и меньшие;
\vs Deu 25:14 в доме твоем не должна быть двоякая ефа, б\acc{о}льшая и меньшая;
\vs Deu 25:15 гиря у тебя должна быть точная и правильная, и ефа у тебя должна быть точная и правильная, чтобы продлились дни твои на земле, которую Господь Бог твой дает тебе [в удел];
\vs Deu 25:16 ибо мерзок пред Господом Богом твоим всякий делающий неправду.
\vs Deu 25:17 Помни, как поступил с тобою Амалик на пути, когда вы шли из Египта:
\vs Deu 25:18 как он встретил тебя на пути, и побил сзади тебя всех ослабевших, когда ты устал и утомился, и не побоялся он Бога;
\vs Deu 25:19 итак, когда Господь Бог твой успокоит тебя от всех врагов твоих со всех сторон, на земле, которую Господь Бог твой дает тебе в удел, чтоб овладеть ею, изгладь память Амалика из поднебесной; не забудь.
\vs Deu 26:1 Когда ты придешь в землю, которую Господь Бог твой дает тебе в удел, и овладеешь ею, и поселишься в ней;
\vs Deu 26:2 то возьми начатков всех плодов земли, которые ты получишь от земли твоей, которую Господь Бог твой дает тебе, и положи в корзину, и пойди на то место, которое Господь Бог твой изберет, чтобы пребывало там имя Его;
\vs Deu 26:3 и приди к священнику, который будет в те дни, и скажи ему: сегодня исповедую пред Господом Богом твоим, что я вошел в ту землю, которую Господь клялся отцам нашим дать нам.
\vs Deu 26:4 Священник возьмет корзину из руки твоей и поставит ее пред жертвенником Господа Бога твоего.
\vs Deu 26:5 Ты же отвечай и скажи пред Господом Богом твоим: отец мой был странствующий Арамеянин, и пошел в Египет и поселился там с немногими людьми, и произошел там от него народ великий, сильный и многочисленный;
\vs Deu 26:6 но Египтяне худо поступали с нами, и притесняли нас, и налагали на нас тяжкие работы;
\vs Deu 26:7 и возопили мы к Господу Богу отцов наших, и услышал Господь вопль наш и увидел бедствие наше, труды наши и угнетение наше;
\vs Deu 26:8 и вывел нас Господь из Египта [Сам крепостию Своею великою и] рукою сильною и мышцею простертою, великим ужасом, знамениями и чудесами,
\vs Deu 26:9 и привел нас на место сие, и дал нам землю сию, землю, в которой течет молоко и мед;
\vs Deu 26:10 итак вот, я принес начатки плодов от земли, которую Ты, Господи, дал мне, [от земли, где течет молоко и мед]. И поставь это пред Господом Богом твоим, и поклонись пред Господом Богом твоим,
\vs Deu 26:11 и веселись о всех благах, которые Господь Бог твой дал тебе и дому твоему, ты и левит и пришелец, который будет у тебя.
\vs Deu 26:12 Когда ты отделишь все десятины произведений [земли] твоей в третий год, год десятин, и отдашь левиту, пришельцу, сироте и вдове, чтоб они ели в жилищах твоих и насыщались,
\vs Deu 26:13 тогда скажи пред Господом Богом твоим: я отобрал от дома [моего] святыню и отдал ее левиту, пришельцу, сироте и вдове, по всем повелениям Твоим, которые Ты заповедал мне: я не преступил заповедей Твоих и не забыл;
\vs Deu 26:14 я не ел от нее в печали моей, и не отделял ее в нечистоте, и не давал из нее для мертвого; я повиновался гласу Господа Бога моего, исполнил все, что Ты заповедал мне;
\vs Deu 26:15 призри от святого жилища Твоего, с небес, и благослови народ Твой, Израиля, и землю, которую Ты дал нам~--- так как Ты клялся отцам нашим [дать нам] землю, в которой течет молоко и мед.
\vs Deu 26:16 В день сей Господь Бог твой завещевает тебе исполнять [все] постановления сии и законы: соблюдай и исполняй их от всего сердца твоего и от всей души твоей.
\vs Deu 26:17 Господу сказал ты ныне, что Он будет твоим Богом, и что ты будешь ходить путями Его и хранить постановления Его и заповеди Его и законы Его, и слушать гласа Его;
\vs Deu 26:18 и Господь обещал тебе ныне, что ты будешь собственным Его народом, как Он говорил тебе, если ты будешь хранить все заповеди Его,
\vs Deu 26:19 и что Он поставит тебя выше всех народов, которых Он сотворил, в чести, славе и великолепии, и что ты будешь святым народом у Господа Бога твоего, как Он говорил.
\vs Deu 27:1 И заповедал Моисей и старейшины [сынов] Израилевых народу, говоря: исполняйте все заповеди, которые заповедую вам ныне.
\vs Deu 27:2 И когда перейдете за Иордан, в землю, которую Господь Бог твой дает тебе, тогда поставь себе большие камни и обмажь их известью;
\vs Deu 27:3 и напиши на [камнях] сих все слова закона сего, когда перейдешь [Иордан], чтобы вступить в землю, которую Господь Бог твой дает тебе, в землю, где течет молоко и мед, как говорил тебе Господь Бог отцов твоих.
\vs Deu 27:4 Когда перейдете Иордан, поставьте камни те, как я повелеваю вам сегодня, на горе Гевал, и обмажьте их известью;
\vs Deu 27:5 и устрой там жертвенник Господу Богу твоему, жертвенник из камней, не поднимая на них железа;
\vs Deu 27:6 из камней цельных устрой жертвенник Господа Бога твоего, и возноси на нем всесожжения Господу Богу твоему,
\vs Deu 27:7 и приноси жертвы мирные, и ешь [и насыщайся] там, и веселись пред Господом Богом твоим;
\vs Deu 27:8 и напиши на камнях [сих] все слова закона сего очень явственно.
\rsbpar\vs Deu 27:9 И сказал Моисей и священники левиты всему Израилю, говоря: внимай и слушай, Израиль: в день сей ты сделался народом Господа Бога твоего;
\vs Deu 27:10 итак слушай гласа Господа Бога твоего и исполняй [все] заповеди Его и постановления Его, которые заповедую тебе сегодня.
\vs Deu 27:11 И заповедал Моисей народу в день тот, говоря:
\vs Deu 27:12 сии должны стать на горе Гаризим, чтобы благословлять народ, когда перейдете Иордан: Симеон, Левий, Иуда, Иссахар, Иосиф и Вениамин;
\vs Deu 27:13 а сии должны стать на горе Гевал, чтобы \bibemph{произносить} проклятие: Рувим, Гад, Асир, Завулон, Дан и Неффалим.
\vs Deu 27:14 Левиты возгласят и скажут всем Израильтянам громким голосом:
\vs Deu 27:15 проклят, кто сделает изваянный или литой кумир, мерзость пред Господом, произведение рук художника, и поставит его в тайном месте! Весь народ возгласит и скажет: аминь.
\vs Deu 27:16 Проклят злословящий отца своего или матерь свою! И весь народ скажет: аминь.
\vs Deu 27:17 Проклят нарушающий межи ближнего своего! И весь народ скажет: аминь.
\vs Deu 27:18 Проклят, кто слепого сбивает с пути! И весь народ скажет: аминь.
\vs Deu 27:19 Проклят, кто превратно судит пришельца, сироту и вдову! И весь народ скажет: аминь.
\vs Deu 27:20 Проклят, кто ляжет с женою отца своего, ибо он открыл край \bibemph{одежды} отца своего! И весь народ скажет: аминь.
\vs Deu 27:21 Проклят, кто ляжет с каким-либо скотом! И весь народ скажет: аминь.
\vs Deu 27:22 Проклят, кто ляжет с сестрою своею, с дочерью отца своего, или дочерью матери своей! И весь народ скажет: аминь.
\vs Deu 27:23 Проклят, кто ляжет с тещею своею! И весь народ скажет: аминь. [Проклят, кто ляжет с сестрою жены своей! И весь народ скажет: аминь.]
\rsbpar\vs Deu 27:24 Проклят, кто тайно убивает ближнего своего! И весь народ скажет: аминь.
\vs Deu 27:25 Проклят, кто берет подкуп, чтоб убить душу \bibemph{и пролить} кровь невинную! И весь народ скажет: аминь.
\vs Deu 27:26 Проклят [всякий человек], кто не исполнит [всех] слов закона сего и не будет поступать по ним! И весь народ скажет: аминь.
\vs Deu 28:1 Если ты, когда перейдете [за Иордан в землю, которую Господь Бог ваш дает вам], будешь слушать гласа Господа Бога твоего, тщательно исполнять все заповеди Его, которые заповедую тебе сегодня, то Господь Бог твой поставит тебя выше всех народов земли;
\vs Deu 28:2 и придут на тебя все благословения сии и исполнятся на тебе, если будешь слушать гласа Господа, Бога твоего.
\vs Deu 28:3 Благословен ты в городе и благословен на поле.
\vs Deu 28:4 Благословен плод чрева твоего, и плод земли твоей, и плод скота твоего, и плод твоих волов, и плод овец твоих.
\vs Deu 28:5 Благословенны житницы твои и кладовые твои.
\vs Deu 28:6 Благословен ты при входе твоем и благословен ты при выходе твоем.
\vs Deu 28:7 Поразит пред тобою Господь врагов твоих, восстающих на тебя; одним путем они выступят против тебя, а семью путями побегут от тебя.
\vs Deu 28:8 Пошлет Господь тебе благословение в житницах твоих и во всяком деле рук твоих; и благословит тебя на земле, которую Господь Бог твой дает тебе.
\vs Deu 28:9 Поставит тебя Господь [Бог твой] народом святым Своим, как Он клялся тебе [и отцам твоим], если ты будешь соблюдать заповеди Господа Бога твоего и будешь ходить путями Его;
\vs Deu 28:10 и увидят все народы земли, что имя Господа [Бога твоего] нарицается на тебе, и убоятся тебя.
\vs Deu 28:11 И даст тебе Господь [Бог твой] изобилие во всех благах, в плоде чрева твоего, и в плоде скота твоего, и в плоде полей твоих на земле, которую Господь клялся отцам твоим дать тебе.
\vs Deu 28:12 Откроет тебе Господь добрую сокровищницу Свою, небо, чтоб оно давало дождь земле твоей во время свое, и чтобы благословлять все дела рук твоих: и будешь давать взаймы многим народам, а сам не будешь брать взаймы [и будешь господствовать над многими народами, а они над тобою не будут господствовать].
\vs Deu 28:13 Сделает тебя Господь [Бог твой] главою, а не хвостом, и будешь только на высоте, а не будешь внизу, если будешь повиноваться заповедям Господа Бога твоего, которые заповедую тебе сегодня хранить и исполнять,
\vs Deu 28:14 и не отступишь от всех слов, которые заповедую вам сегодня, ни направо ни налево, чтобы пойти вслед иных богов \bibemph{и} служить им.
\vs Deu 28:15 Если же не будешь слушать гласа Господа Бога твоего и не будешь стараться исполнять все заповеди Его и постановления Его, которые я заповедую тебе сегодня, то придут на тебя все проклятия сии и постигнут тебя.
\vs Deu 28:16 Проклят ты [будешь] в городе и проклят ты [будешь] на поле.
\vs Deu 28:17 Прокляты [будут] житницы твои и кладовые твои.
\vs Deu 28:18 Проклят [будет] плод чрева твоего и плод земли твоей, плод твоих волов и плод овец твоих.
\vs Deu 28:19 Проклят ты [будешь] при входе твоем и проклят при выходе твоем.
\vs Deu 28:20 Пошлет Господь на тебя проклятие, смятение и несчастье во всяком деле рук твоих, какое ни станешь ты делать, доколе не будешь истреблен,~--- и ты скоро погибнешь за злые дела твои, за то, что ты оставил Меня.
\vs Deu 28:21 Пошлет Господь на тебя моровую язву, доколе не истребит Он тебя с земли, в которую ты идешь, чтобы владеть ею.
\vs Deu 28:22 Поразит тебя Господь чахлостью, горячкою, лихорадкою, воспалением, засухою, палящим ветром и ржавчиною, и они будут преследовать тебя, доколе не погибнешь.
\vs Deu 28:23 И небеса твои, которые над головою твоею, сделаются медью, и земля под тобою железом;
\vs Deu 28:24 вместо дождя Господь даст земле твоей пыль, и прах с неба будет падать, падать на тебя, [доколе не погубит тебя и] доколе не будешь истреблен.
\vs Deu 28:25 Предаст тебя Господь на поражение врагам твоим; одним путем выступишь против них, а семью путями побежишь от них; и будешь рассеян по всем царствам земли.
\vs Deu 28:26 И будут трупы твои пищею всем птицам небесным и зверям, и не будет отгоняющего их.
\vs Deu 28:27 Поразит тебя Господь проказою Египетскою, почечуем, коростою и чесоткою, от которых ты не возможешь исцелиться;
\vs Deu 28:28 поразит тебя Господь сумасшествием, слепотою и оцепенением сердца.
\vs Deu 28:29 И ты будешь ощупью ходить в полдень, как слепой ощупью ходит впотьмах, и не будешь иметь успеха в путях твоих, и будут теснить и обижать тебя всякий день, и никто не защитит тебя.
\vs Deu 28:30 С женою обручишься, и другой будет спать с нею; дом построишь, и не будешь жить в нем; виноградник насадишь, и не будешь пользоваться им.
\vs Deu 28:31 Вола твоего заколют в глазах твоих, и не будешь есть его; осла твоего уведут от тебя и не возвратят тебе; овцы твои отданы будут врагам твоим, и никто не защитит тебя.
\vs Deu 28:32 Сыновья твои и дочери твои будут отданы другому народу; глаза твои будут видеть и всякий день истаевать о них, и не будет силы в руках твоих.
\vs Deu 28:33 Плоды земли твоей и все труды твои будет есть народ, которого ты не знал; и ты будешь только притесняем и мучим во все дни.
\vs Deu 28:34 И сойдешь с ума от того, что будут видеть глаза твои.
\vs Deu 28:35 Поразит тебя Господь злою проказою на коленях и голенях, от которой ты не возможешь исцелиться, от подошвы ноги твоей до самого темени [головы] твоей.
\vs Deu 28:36 Отведет Господь тебя и царя твоего, которого ты поставишь над собою, к народу, которого не знал ни ты, ни отцы твои, и там будешь служить иным богам, деревянным и каменным;
\vs Deu 28:37 и будешь ужасом, притчею и посмешищем у всех народов, к которым отведет тебя Господь [Бог].
\vs Deu 28:38 Семян много вынесешь в поле, а соберешь мало, потому что поест их саранча.
\vs Deu 28:39 Виноградники будешь садить и возделывать, а вина не будешь пить, и не соберешь \bibemph{плодов} [их], потому что поест их червь.
\vs Deu 28:40 Маслины будут у тебя во всех пределах твоих, но елеем не помажешься, потому что осыплется маслина твоя.
\vs Deu 28:41 Сынов и дочерей родишь, но их не будет у тебя, потому что пойдут в плен.
\vs Deu 28:42 Все дерева твои и плоды земли твоей погубит ржавчина.
\vs Deu 28:43 Пришелец, который среди тебя, будет возвышаться над тобою выше и выше, а ты опускаться будешь ниже и ниже;
\vs Deu 28:44 он будет давать тебе взаймы, а ты не будешь давать ему взаймы; он будет главою, а ты будешь хвостом.
\vs Deu 28:45 И придут на тебя все проклятия сии, и будут преследовать тебя и постигнут тебя, доколе не будешь истреблен, за то, что ты не слушал гласа Господа Бога твоего и не соблюдал заповедей Его и постановлений Его, которые Он заповедал тебе:
\vs Deu 28:46 они будут знамением и указанием на тебе и на семени твоем вовек.
\vs Deu 28:47 За то, что ты не служил Господу Богу твоему с веселием и радостью сердца, при изобилии всего,
\vs Deu 28:48 будешь служить врагу твоему, которого пошлет на тебя Господь [Бог твой], в голоде, и жажде, и наготе и во всяком недостатке; он возложит на шею твою железное ярмо, так что измучит тебя.
\vs Deu 28:49 Пошлет на тебя Господь народ издалека, от края земли: как орел налетит народ, которого языка ты не разумеешь,
\vs Deu 28:50 народ наглый, который не уважит старца и не пощадит юноши;
\vs Deu 28:51 и будет он есть плод скота твоего и плод земли твоей, доколе не разорит тебя, так что не оставит тебе ни хлеба, ни вина, ни елея, ни плода волов твоих, ни плода овец твоих, доколе не погубит тебя;
\vs Deu 28:52 и будет теснить тебя во всех жилищах твоих, доколе во всей земле твоей не разрушит высоких и крепких стен твоих, на которые ты надеешься; и будет теснить тебя во всех жилищах твоих, во всей земле твоей, которую Господь Бог твой дал тебе.
\vs Deu 28:53 И ты будешь есть плод чрева твоего, плоть сынов твоих и дочерей твоих, которых Господь Бог твой дал тебе, в осаде и в стеснении, в котором стеснит тебя враг твой.
\vs Deu 28:54 Муж, изнеженный и живший между вами в великой роскоши, безжалостным оком будет смотреть на брата своего, на жену недра своего и на остальных детей своих, которые останутся у него,
\vs Deu 28:55 и не даст ни одному из них плоти детей своих, которых он будет есть, потому что у него не останется ничего в осаде и в стеснении, в котором стеснит тебя враг твой во всех жилищах твоих.
\vs Deu 28:56 [Женщина] жившая у тебя в неге и роскоши, которая никогда ноги своей не ставила на землю по причине роскоши и изнеженности, будет безжалостным оком смотреть на мужа недра своего и на сына своего и на дочь свою
\vs Deu 28:57 и \bibemph{не даст} им последа, выходящего из среды ног ее, и детей, которых она родит; потому что она, при недостатке во всем, тайно будет есть их, в осаде и стеснении, в котором стеснит тебя враг твой в жилищах твоих.
\vs Deu 28:58 Если не будешь стараться исполнять все слова закона сего, написанные в книге сей, и не будешь бояться сего славного и страшного имени Господа Бога твоего,
\vs Deu 28:59 то Господь поразит тебя и потомство твое необычайными язвами, язвами великими и постоянными, и болезнями злыми и постоянными;
\vs Deu 28:60 и наведет на тебя все [злые] язвы Египетские, которых ты боялся, и они прилипнут к тебе;
\vs Deu 28:61 и всякую болезнь и всякую язву, не написанную [и всякую написанную] в книге закона сего, Господь наведет на тебя, доколе не будешь истреблен;
\vs Deu 28:62 и останется вас немного, тогда как множеством вы подобны были звездам небесным, ибо ты не слушал гласа Господа Бога твоего.
\vs Deu 28:63 И как радовался Господь, делая вам добро и умножая вас, так будет радоваться Господь, погубляя вас и истребляя вас, и извержены будете из земли, в которую ты идешь, чтобы владеть ею.
\vs Deu 28:64 И рассеет тебя Господь [Бог твой] по всем народам, от края земли до края земли, и будешь там служить иным богам, которых не знал ни ты, ни отцы твои, дереву и камням.
\vs Deu 28:65 Но и между этими народами не успокоишься, и не будет места покоя для ноги твоей, и Господь даст тебе там трепещущее сердце, истаевание очей и изнывание души;
\vs Deu 28:66 жизнь твоя будет висеть пред тобою, и будешь трепетать ночью и днем, и не будешь уверен в жизни твоей;
\vs Deu 28:67 от трепета сердца твоего, которым ты будешь объят, и от того, что ты будешь видеть глазами твоими, утром ты скажешь: <<о, если бы пришел вечер!>>, а вечером скажешь: <<о, если бы наступило утро!>>
\vs Deu 28:68 и возвратит тебя Господь в Египет на кораблях тем путем, о котором я сказал тебе: <<ты более не увидишь его>>; и там будете продаваться врагам вашим в рабов и в рабынь, и не будет покупающего.
\vs Deu 29:1 Вот слова завета, который Господь повелел Моисею поставить с сынами Израилевыми в земле Моавитской, кроме завета, который Господь поставил с ними на Хориве.
\rsbpar\vs Deu 29:2 И созвал Моисей всех [сынов] Израилевых и сказал им: вы видели всё, что сделал Господь пред глазами вашими в земле Египетской с фараоном и всеми рабами его и всею землею его;
\vs Deu 29:3 те великие казни, которые видели глаза твои, и те великие знамения и чудеса, [руку крепкую и мышцу простертую];
\vs Deu 29:4 но до сего дня не дал вам Господь [Бог] сердца, чтобы разуметь, очей, чтобы видеть, и ушей, чтобы слышать.
\vs Deu 29:5 Сорок лет водил вас по пустыне, и одежды ваши на вас не обветшали, и обувь твоя не обветшала на ноге твоей;
\vs Deu 29:6 хлеба вы не ели и вина и сикера не пили, дабы вы знали, что Я Господь Бог ваш.
\vs Deu 29:7 И когда пришли вы на место сие, выступил против нас Сигон, царь Есевонский, и Ог, царь Васанский, чтобы сразиться \bibemph{с нами}, и мы поразили их;
\vs Deu 29:8 и взяли землю их и отдали ее в удел \bibemph{колену} Рувимову и Гадову и половине колена Манассиина.
\vs Deu 29:9 Соблюдайте же [все] слова завета сего и исполняйте их, чтобы вам иметь успех во всем, что ни будете делать.
\vs Deu 29:10 Все вы сегодня стоите пред лицем Господа Бога вашего, начальники колен ваших, старейшины ваши, [судьи ваши,] надзиратели ваши, все Израильтяне,
\vs Deu 29:11 дети ваши, жены ваши и пришельцы твои, находящиеся в стане твоем, от секущего дрова твои до черпающего воду твою,
\vs Deu 29:12 чтобы вступить тебе в завет Господа Бога твоего и в клятвенный договор с Ним, который Господь Бог твой сегодня поставляет с тобою,
\vs Deu 29:13 дабы соделать тебя сегодня Его народом, и Ему быть тебе Богом, как Он говорил тебе и как клялся отцам твоим Аврааму, Исааку и Иакову.
\vs Deu 29:14 Не с вами только одними я поставляю сей завет и сей клятвенный договор,
\vs Deu 29:15 но как с теми, которые сегодня здесь с нами стоят пред лицем Господа Бога нашего, так и с теми, которых нет здесь с нами сегодня.
\vs Deu 29:16 Ибо вы знаете, как мы жили в земле Египетской и как мы проходили посреди народов, чрез которые вы прошли,
\vs Deu 29:17 и видели мерзости их и кумиры их, деревянные и каменные, серебряные и золотые, которые у них.
\vs Deu 29:18 Да не будет между вами мужчины или женщины, или рода или колена, которых сердце уклонилось бы ныне от Господа Бога нашего, чтобы ходить служить богам тех народов; да не будет между вами корня, произращающего яд и полынь,
\vs Deu 29:19 такого человека, который, услышав слова проклятия сего, похвалялся бы в сердце своем, говоря: <<я буду счастлив, несмотря на то, что буду ходить по произволу сердца моего>>; и пропадет таким образом сытый с голодным;
\vs Deu 29:20 не простит Господь такому, но тотчас возгорится гнев Господа и ярость Его на такого человека, и падет на него все проклятие [завета сего], написанное в сей книге [закона], и изгладит Господь имя его из поднебесной;
\vs Deu 29:21 и отделит его Господь на погибель от всех колен Израилевых, сообразно со всеми проклятиями завета, написанными в сей книге закона.
\vs Deu 29:22 И скажет последующий род, дети ваши, которые будут после вас, и чужеземец, который придет из земли дальней, увидев поражение земли сей и болезни, которыми изнурит ее Господь:
\vs Deu 29:23 сера и соль, пожарище~--- вся земля; не засевается и не произращает она, и не выходит на ней никакой травы, как по истреблении Содома, Гоморры, Адмы и Севоима, которые ниспроверг Господь во гневе Своем и в ярости Своей.
\vs Deu 29:24 И скажут все народы: за что Господь так поступил с сею землею? какая великая ярость гнева Его!
\vs Deu 29:25 И скажут: за то, что они оставили завет Господа Бога отцов своих, который Он поставил с ними, когда вывел их из земли Египетской,
\vs Deu 29:26 и пошли и стали служить иным богам и поклоняться им, богам, которых они не знали и \bibemph{которых} Он не назначал им:
\vs Deu 29:27 \bibemph{за то} возгорелся гнев Господа на землю сию, и навел Он на нее все проклятия [завета], написанные в сей книге [закона],
\vs Deu 29:28 и извергнул их Господь из земли их в гневе, ярости и великом негодовании, и поверг их на другую землю, как ныне \bibemph{видим}.
\vs Deu 29:29 Сокрытое \bibemph{принадлежит} Господу Богу нашему, а открытое~--- нам и сынам нашим до века, чтобы мы исполняли все слова закона сего.
\vs Deu 30:1 Когда придут на тебя все слова сии~--- благословение и проклятие, которые изложил я тебе, и примешь \bibemph{их} к сердцу своему среди всех народов, в которых рассеет тебя Господь Бог твой,
\vs Deu 30:2 и обратишься к Господу Богу твоему и послушаешь гласа Его, как я заповедую тебе сегодня, ты и сыны твои от всего сердца твоего и от всей души твоей,~---
\vs Deu 30:3 тогда Господь Бог твой возвратит пленных твоих и умилосердится над тобою, и опять соберет тебя от всех народов, между которыми рассеет тебя Господь Бог твой.
\vs Deu 30:4 Хотя бы ты был рассеян [от края неба] до края неба, и оттуда соберет тебя Господь Бог твой, и оттуда возьмет тебя,
\vs Deu 30:5 и [оттуда] приведет тебя Господь Бог твой в землю, которою владели отцы твои, и получишь ее во владение, и облагодетельствует тебя и размножит тебя более отцов твоих;
\vs Deu 30:6 и обрежет Господь Бог твой сердце твое и сердце потомства твоего, чтобы ты любил Господа Бога твоего от всего сердца твоего и от всей души твоей, дабы жить тебе;
\vs Deu 30:7 тогда Господь Бог твой все проклятия сии обратит на врагов твоих и ненавидящих тебя, которые гнали тебя,
\vs Deu 30:8 а ты обратишься и будешь слушать гласа Господа [Бога твоего] и исполнять все заповеди Его, которые заповедую тебе сегодня;
\vs Deu 30:9 с избытком даст тебе Господь Бог твой успех во всяком деле рук твоих, в плоде чрева твоего, в плоде скота твоего, в плоде земли твоей; ибо снова радоваться будет Господь [Бог твой] о тебе, благодетельствуя \bibemph{тебе}, как Он радовался об отцах твоих,
\vs Deu 30:10 если будешь слушать гласа Господа Бога твоего, соблюдая [и исполняя все] заповеди Его и постановления Его [и законы Его], написанные в сей книге закона, и если обратишься к Господу Богу твоему всем сердцем твоим и всею душею твоею.
\vs Deu 30:11 Ибо заповедь сия, которую я заповедую тебе сегодня, не недоступна для тебя и не далека;
\vs Deu 30:12 она не на небе, чтобы можно \bibemph{было} говорить: <<кто взошел бы для нас на небо и принес бы ее нам, и дал бы нам услышать ее, и мы исполнили бы ее?>>
\vs Deu 30:13 и не за морем она, чтобы можно \bibemph{было} говорить: <<кто сходил бы для нас за море и принес бы ее нам, и дал бы нам услышать ее, и мы исполнили бы ее?>>
\vs Deu 30:14 но весьма близко к тебе слово сие: \bibemph{оно} в устах твоих и в сердце твоем, чтобы исполнять его.
\vs Deu 30:15 Вот, я сегодня предложил тебе жизнь и добро, смерть и зло.
\vs Deu 30:16 [Если будешь слушать заповеди Господа Бога твоего,] которые заповедую тебе сегодня, любить Господа Бога твоего, ходить по [всем] путям Его и исполнять заповеди Его и постановления Его и законы Его, то будешь жить и размножишься, и благословит тебя Господь Бог твой на земле, в которую ты идешь, чтоб овладеть ею;
\vs Deu 30:17 если же отвратится сердце твое, и не будешь слушать, и заблудишь, и станешь поклоняться иным богам и будешь служить им,
\vs Deu 30:18 то я возвещаю вам сегодня, что вы погибнете и не пробудете долго на земле, [которую Господь Бог дает тебе,] для овладения которою ты переходишь Иордан.
\vs Deu 30:19 Во свидетели пред вами призываю сегодня небо и землю: жизнь и смерть предложил я тебе, благословение и проклятие. Избери жизнь, дабы жил ты и потомство твое,
\vs Deu 30:20 любил Господа Бога твоего, слушал глас Его и прилеплялся к Нему; ибо в этом жизнь твоя и долгота дней твоих, чтобы пребывать тебе на земле, которую Господь [Бог] с клятвою обещал отцам твоим Аврааму, Исааку и Иакову дать им.
\vs Deu 31:1 И пошел Моисей, и говорил слова сии всем [сынам] Израиля,
\vs Deu 31:2 и сказал им: теперь мне сто двадцать лет, я не могу уже выходить и входить, и Господь сказал мне: <<ты не перейдешь Иордан сей>>;
\vs Deu 31:3 Господь Бог твой Сам пойдет пред тобою; Он истребит народы сии от лица твоего, и ты овладеешь ими; Иисус пойдет пред тобою, как говорил Господь;
\vs Deu 31:4 и поступит Господь с ними так же, как Он поступил с Сигоном и Огом, царями Аморрейскими, [которые были по эту сторону Иордана,] и с землею их, которых он истребил;
\vs Deu 31:5 и предаст их Господь вам, и вы поступите с ними по всем заповедям, какие заповедал я вам;
\vs Deu 31:6 будьте тверды и мужественны, не бойтесь, [не ужасайтесь] и не страшитесь их, ибо Господь Бог твой Сам пойдет с тобою [и] не отступит от тебя и не оставит тебя.
\rsbpar\vs Deu 31:7 И призвал Моисей Иисуса и пред очами всех Израильтян сказал ему: будь тверд и мужествен, ибо ты войдешь с народом сим в землю, которую Господь клялся отцам его дать ему, и ты разделишь ее на уделы ему;
\vs Deu 31:8 Господь Сам пойдет пред тобою, Сам будет с тобою, не отступит от тебя и не оставит тебя, не бойся и не ужасайся.
\vs Deu 31:9 И написал Моисей закон сей, и отдал его священникам, сынам Левииным, носящим ковчег завета Господня, и всем старейшинам [сынов] Израилевых.
\vs Deu 31:10 И завещал им Моисей и сказал: по прошествии семи лет, в год отпущения, в праздник кущей,
\vs Deu 31:11 когда весь Израиль придет явиться пред лице Господа Бога твоего на место, которое изберет [Господь], читай сей закон пред всем Израилем вслух его;
\vs Deu 31:12 собери народ, мужей и жен, и детей, и пришельцев твоих, которые будут в жилищах твоих, чтоб они слушали и учились, и чтобы боялись Господа Бога вашего, и старались исполнять все слова закона сего;
\vs Deu 31:13 и сыны их, которые не знают \bibemph{сего}, услышат и научатся бояться Господа Бога вашего во все дни, доколе вы будете жить на земле, в которую вы переходите за Иордан, чтоб овладеть ею.
\rsbpar\vs Deu 31:14 И сказал Господь Моисею: вот, дни твои приблизились к смерти; призови Иисуса и станьте у [входа] скинии собрания, и Я дам ему наставления. И пришел Моисей и Иисус, и стали у [входа] скинии собрания.
\vs Deu 31:15 И явился Господь в скинии в столпе облачном, и стал столп облачный у входа скинии [собрания].
\vs Deu 31:16 И сказал Господь Моисею: вот, ты почиешь с отцами твоими, и станет народ сей блудно ходить вслед чужих богов той земли, в которую он вступает, и оставит Меня, и нарушит завет Мой, который Я поставил с ним;
\vs Deu 31:17 и возгорится гнев Мой на него в тот день, и Я оставлю их и сокрою лице Мое от них, и он истреблен будет, и постигнут его многие бедствия и скорби, и скажет он в тот день: <<не потому ли постигли меня сии бедствия, что нет [Господа] Бога моего среди меня?>>
\vs Deu 31:18 и Я сокрою лице Мое [от него] в тот день за все беззакония его, которые он сделает, обратившись к иным богам.
\vs Deu 31:19 Итак напишите себе [слова] песни сей, и научи ей сынов Израилевых, и вложи ее в уста их, чтобы песнь сия была Мне свидетельством на сынов Израилевых;
\vs Deu 31:20 ибо Я введу их в землю [добрую], как Я клялся отцам их, где течет молоко и мед, и они будут есть и насыщаться, и утучнеют, и обратятся к иным богам, и будут служить им, а Меня отвергнут и нарушат завет Мой, [который Я завещал им];
\vs Deu 31:21 и когда постигнут их многие бедствия и скорби, тогда песнь сия будет против них свидетельством, ибо она не выйдет [из уст их и] из уст потомства их. Я знаю мысли их, которые они имеют ныне, прежде нежели Я ввел их в [добрую] землю, о которой Я клялся [отцам их].
\vs Deu 31:22 И написал Моисей песнь сию в тот день и научил ей сынов Израилевых.
\vs Deu 31:23 И заповедал Господь Иисусу, сыну Навину, и сказал [ему]: будь тверд и мужествен, ибо ты введешь сынов Израилевых в землю, о которой Я клялся им, и Я буду с тобою.
\rsbpar\vs Deu 31:24 Когда Моисей вписал в книгу все слова закона сего до конца,
\vs Deu 31:25 тогда Моисей повелел левитам, носящим ковчег завета Господня, сказав:
\vs Deu 31:26 возьмите сию книгу закона и положите ее одесную ковчега завета Господа Бога вашего, и она там будет свидетельством против тебя;
\vs Deu 31:27 ибо я знаю упорство твое и жестоковыйность твою: вот и теперь, когда я живу с вами ныне, вы упорны пред Господом; не тем ли более по смерти моей?
\vs Deu 31:28 соберите ко мне всех старейшин колен ваших [и судей ваших] и надзирателей ваших, и я скажу вслух их слова сии и призову во свидетельство на них небо и землю;
\vs Deu 31:29 ибо я знаю, что по смерти моей вы развратитесь и уклонитесь от пути, который я завещал вам, и в последствие времени постигнут вас бедствия за то, что вы будете делать зло пред очами Господа [Бога], раздражая Его делами рук своих.
\vs Deu 31:30 И изрек Моисей вслух всего собрания Израильтян слова песни сей до конца:
\vs Deu 32:1 Внимай, небо, я буду говорить; и слушай, земля, слова уст моих.
\vs Deu 32:2 Польется как дождь учение мое, как роса речь моя, как мелкий дождь на зелень, как ливень на траву.
\vs Deu 32:3 Имя Господа прославляю; воздайте славу Богу нашему.
\vs Deu 32:4 Он твердыня; совершенны дела Его, и все пути Его праведны; Бог верен, и нет неправды [в Нем]; Он праведен и истинен;
\vs Deu 32:5 но они развратились пред Ним, они не дети Его по своим порокам, род строптивый и развращенный.
\vs Deu 32:6 Сие ли воздаете вы Господу, народ глупый и несмысленный? не Он ли Отец твой, \bibemph{Который} усвоил тебя, создал тебя и устроил тебя?
\vs Deu 32:7 Вспомни дни древние, помысли о летах прежних родов; спроси отца твоего, и он возвестит тебе, старцев твоих, и они скажут тебе.
\vs Deu 32:8 Когда Всевышний давал уделы народам и расселял сынов человеческих, тогда поставил пределы народов по числу сынов Израилевых\fns{В греческом переводе: по числу Ангелов Божиих.};
\vs Deu 32:9 ибо часть Господа народ Его, Иаков наследственный удел Его.
\vs Deu 32:10 Он нашел его в пустыне, в степи печальной и дикой, ограждал его, смотрел за ним, хранил его, как зеницу ока Своего;
\vs Deu 32:11 как орел вызывает гнездо свое, носится над птенцами своими, распростирает крылья свои, берет их и носит их на перьях своих,
\vs Deu 32:12 так Господь один водил его, и не было с Ним чужого бога.
\vs Deu 32:13 Он вознес его на высоту земли и кормил произведениями полей, и питал его медом из камня и елеем из твердой скалы,
\vs Deu 32:14 маслом коровьим и молоком овечьим, и туком агнцев и овнов Васанских и козлов, и тучною пшеницею, и ты пил вино, кровь виноградных ягод.
\vs Deu 32:15 И [ел Иаков, и] утучнел Израиль, и стал упрям; утучнел, отолстел и разжирел; и оставил он Бога, создавшего его, и презрел твердыню спасения своего.
\vs Deu 32:16 \bibemph{Богами} чуждыми они раздражили Его и мерзостями [своими] разгневали Его:
\vs Deu 32:17 приносили жертвы бесам, а не Богу, богам, которых они не знали, новым, \bibemph{которые} пришли от соседей и о которых не помышляли отцы ваши.
\vs Deu 32:18 А Заступника, родившего тебя, ты забыл, и не помнил Бога, создавшего тебя.
\vs Deu 32:19 Господь увидел [и вознегодовал], и в негодовании пренебрег сынов Своих и дочерей Своих,
\vs Deu 32:20 и сказал: сокрою лице Мое от них [и] увижу, какой будет конец их; ибо они род развращенный; дети, в которых нет верности;
\vs Deu 32:21 они раздражили Меня не богом, суетными своими огорчили Меня: и Я раздражу их не народом, народом бессмысленным огорчу их;
\vs Deu 32:22 ибо огонь возгорелся во гневе Моем, жжет до ада преисподнего, и поядает землю и произведения ее, и попаляет основания гор;
\vs Deu 32:23 соберу на них бедствия и истощу на них стрелы Мои:
\vs Deu 32:24 \bibemph{будут} истощены голодом, истреблены горячкою и лютою заразою; и пошлю на них зубы зверей и яд ползающих по земле;
\vs Deu 32:25 извне будет губить их меч, а в домах ужас~--- и юношу, и девицу, и грудного младенца, и покрытого сединою старца.
\vs Deu 32:26 Я сказал бы: рассею их и изглажу из среды людей память о них;
\vs Deu 32:27 но отложил это ради озлобления врагов, чтобы враги его не возомнили и не сказали: наша рука высока, и не Господь сделал все сие.
\vs Deu 32:28 Ибо они народ, потерявший рассудок, и нет в них смысла.
\vs Deu 32:29 О, если бы они рассудили, подумали о сем, уразумели, что с ними будет!
\vs Deu 32:30 Как бы мог один преследовать тысячу и двое прогонять тьму, если бы Заступник их не предал их, и Господь не отдал их!
\vs Deu 32:31 Ибо заступник их не таков, как наш Заступник; сами враги наши судьи в том.
\vs Deu 32:32 Ибо виноград их от виноградной лозы Содомской и с полей Гоморрских; ягоды их ягоды ядовитые, грозды их горькие;
\vs Deu 32:33 вино их яд драконов и гибельная отрава аспидов.
\vs Deu 32:34 Не сокрыто ли это у Меня? не запечатано ли в хранилищах Моих?
\vs Deu 32:35 У Меня отмщение и воздаяние, когда поколеблется нога их; ибо близок день погибели их, скоро наступит уготованное для них.
\vs Deu 32:36 Но Господь будет судить народ Свой и над рабами Своими умилосердится, когда Он увидит, что рука их ослабела, и не стало ни заключенных, ни оставшихся \bibemph{вне}.
\vs Deu 32:37 Тогда скажет [Господь]: где боги их, твердыня, на которую они надеялись,
\vs Deu 32:38 которые ели тук жертв их [и] пили вино возлияний их? пусть они восстанут и помогут вам, пусть будут для вас покровом!
\vs Deu 32:39 Видите ныне, [видите,] что это Я, Я~--- и нет Бога, кроме Меня: Я умерщвляю и оживляю, Я поражаю и Я исцеляю, и никто не избавит от руки Моей.
\vs Deu 32:40 Я подъемлю к небесам руку Мою и [клянусь десницею Моею и] говорю: живу Я вовек!
\vs Deu 32:41 Когда изострю сверкающий меч Мой, и рука Моя приимет суд, то отмщу врагам Моим и ненавидящим Меня воздам;
\vs Deu 32:42 упою стрелы Мои кровью, и меч Мой насытится плотью, кровью убитых и пленных, головами начальников врага.
\vs Deu 32:43 [Веселитесь, небеса, вместе с Ним, и поклонитесь Ему, все Ангелы Божии.] Веселитесь, язычники, с народом Его [и да укрепятся все сыны Божии]! ибо Он отмстит за кровь рабов Своих, и воздаст мщение врагам Своим, [и ненавидящим Его воздаст,] и очистит [Господь] землю Свою \bibemph{и} народ Свой!
\rsbpar\vs Deu 32:44 [И написал Моисей песнь сию в тот день, и научил ей сынов Израилевых.] И пришел Моисей [к народу] и изрек все слова песни сей вслух народа, он и Иисус, сын Навин.
\vs Deu 32:45 Когда Моисей изрек все слова сии всему Израилю,
\vs Deu 32:46 тогда сказал им: положите на сердце ваше все слова, которые я объявил вам сегодня, и завещевайте их детям своим, чтобы они старались исполнять все слова закона сего;
\vs Deu 32:47 ибо это не пустое для вас, но это жизнь ваша, и чрез это вы долгое время пробудете на той земле, в которую вы идете чрез Иордан, чтоб овладеть ею.
\rsbpar\vs Deu 32:48 И говорил Господь Моисею в тот же самый день и сказал:
\vs Deu 32:49 взойди на сию гору Аварим, на гору Нево, которая в земле Моавитской, против Иерихона, и посмотри на землю Ханаанскую, которую я даю во владение сынам Израилевым;
\vs Deu 32:50 и умри на горе, на которую ты взойдешь, и приложись к народу твоему, как умер Аарон, брат твой, на горе Ор, и приложился к народу своему,
\vs Deu 32:51 за то, что вы согрешили против Меня среди сынов Израилевых при водах Меривы в Кадесе, в пустыне Син, за то, что не явили святости Моей среди сынов Израилевых;
\vs Deu 32:52 пред \bibemph{собою} ты увидишь землю, а не войдешь туда, в землю, которую Я даю сынам Израилевым.
\vs Deu 33:1 Вот благословение, которым Моисей, человек Божий, благословил сынов Израилевых пред смертью своею.
\vs Deu 33:2 Он сказал: Господь пришел от Синая, открылся им от Сеира, воссиял от горы Фарана и шел со тьмами святых; одесную Его огнь закона.
\vs Deu 33:3 Истинно Он любит народ [Свой]; все святые его в руке Твоей, и они припали к стопам Твоим, чтобы внимать словам Твоим.
\vs Deu 33:4 Закон дал нам Моисей, наследие обществу Иакова.
\vs Deu 33:5 И он был царь Израиля, когда собирались главы народа вместе с коленами Израилевыми.
\vs Deu 33:6 Да живет Рувим и да не умирает, и [Симеон] да \bibemph{не} будет малочислен!
\vs Deu 33:7 Но об Иуде сказал сие: услыши, Господи, глас Иуды и приведи его к народу его; руками своими да защитит он себя, и Ты будь помощником против врагов его.
\vs Deu 33:8 И о Левии сказал: туммим Твой и урим Твой на святом муже Твоем, которого Ты искусил в Массе, с которым Ты препирался при водах Меривы,
\vs Deu 33:9 который говорит об отце своем и матери своей: <<я на них не смотрю>>, и братьев своих не признает, и сыновей своих не знает; ибо они, \bibemph{левиты}, слова Твои хранят и завет Твой соблюдают,
\vs Deu 33:10 учат законам Твоим Иакова и заповедям Твоим Израиля, возлагают курение пред лице Твое и всесожжения на жертвенник Твой;
\vs Deu 33:11 благослови, Господи, силу его и о деле рук его благоволи, порази чресла восстающих на него и ненавидящих его, чтобы они не могли стоять.
\vs Deu 33:12 О Вениамине сказал: возлюбленный Господом обитает у Него безопасно, [Бог] покровительствует ему всякий день, и он покоится между раменами Его.
\vs Deu 33:13 Об Иосифе сказал: да благословит Господь землю его вожделенными дарами неба, росою и \bibemph{дарами} бездны, лежащей внизу,
\vs Deu 33:14 вожделенными плодами от солнца и вожделенными произведениями луны,
\vs Deu 33:15 превосходнейшими произведениями гор древних и вожделенными дарами холмов вечных,
\vs Deu 33:16 и вожделенными дарами земли и того, что наполняет ее; благословение Явившегося в терновом кусте да приидет на главу Иосифа и на темя наилучшего из братьев своих;
\vs Deu 33:17 крепость его как первородного тельца, и роги его, как роги буйвола; ими избодет он народы все до пределов земли: это тьмы Ефремовы, это тысячи Манассиины.
\vs Deu 33:18 О Завулоне сказал: веселись, Завулон, в путях твоих, и Иссахар, в шатрах твоих;
\vs Deu 33:19 созывают они народ на гору, там заколают законные жертвы, ибо они питаются богатством моря и сокровищами, сокрытыми в песке.
\vs Deu 33:20 О Гаде сказал: благословен распространивший Гада; он покоится как лев и сокрушает и мышцу и голову;
\vs Deu 33:21 он избрал себе начаток \bibemph{земли}, там почтен уделом от законодателя, и пришел с главами народа, и исполнил правду Господа и суды с Израилем.
\vs Deu 33:22 О Дане сказал: Дан молодой лев, который выбегает из Васана.
\vs Deu 33:23 О Неффалиме сказал: Неффалим насыщен благоволением и исполнен благословения Господа; море и юг во владении \bibemph{его}.
\vs Deu 33:24 Об Асире сказал: благословен между сынами Асир, он будет любим братьями своими, и окунет в елей ногу свою;
\vs Deu 33:25 железо и медь~--- запоры твои; как дни твои, \bibemph{будет умножаться} богатство твое.
\vs Deu 33:26 Нет подобного Богу Израилеву, Который по небесам принесся на помощь тебе и во славе Своей на облаках;
\vs Deu 33:27 прибежище [твое] Бог древний, и [ты] под мышцами вечными; Он прогонит врагов от лица твоего и скажет: истребляй!
\vs Deu 33:28 Израиль живет безопасно, один; око Иакова \bibemph{видит пред собою} землю обильную хлебом и вином, и небеса его каплют росу.
\vs Deu 33:29 Блажен ты, Израиль! кто подобен тебе, народ, хранимый Господом, Который есть щит, охраняющий тебя, и меч славы твоей? Враги твои раболепствуют тебе, и ты попираешь выи их.
\vs Deu 34:1 И взошел Моисей с равнин Моавитских на гору Нево, на вершину Фасги, что против Иерихона, и показал ему Господь всю землю Галаад до самого Дана,
\vs Deu 34:2 и всю [землю] Неффалимову, и [всю] землю Ефремову и Манассиину, и всю землю Иудину, даже до самого западного моря,
\vs Deu 34:3 и полуденную страну и равнину долины Иерихона, город Пальм, до Сигора.
\vs Deu 34:4 И сказал ему Господь: вот земля, о которой Я клялся Аврааму, Исааку и Иакову, говоря: <<семени твоему дам ее>>; Я дал тебе увидеть ее глазами твоими, но в нее ты не войдешь.
\rsbpar\vs Deu 34:5 И умер там Моисей, раб Господень, в земле Моавитской, по слову Господню;
\vs Deu 34:6 и погребен на долине в земле Моавитской против Беф-Фегора, и никто не знает \bibemph{места} погребения его даже до сего дня.
\vs Deu 34:7 Моисею было сто двадцать лет, когда он умер; но зрение его не притупилось, и крепость в нем не истощилась.
\vs Deu 34:8 И оплакивали Моисея сыны Израилевы на равнинах Моавитских [у Иордана близ Иерихона] тридцать дней. И прошли дни плача и сетования о Моисее.
\vs Deu 34:9 И Иисус, сын Навин, исполнился духа премудрости, потому что Моисей возложил на него руки свои, и повиновались ему сыны Израилевы и делали так, как повелел Господь Моисею.
\vs Deu 34:10 И не было более у Израиля пророка такого, как Моисей, которого Господь знал лицем к лицу,
\vs Deu 34:11 по всем знамениям и чудесам, которые послал его Господь сделать в земле Египетской над фараоном и над всеми рабами его и над всею землею его,
\vs Deu 34:12 и по руке сильной и по великим чудесам, которые Моисей совершил пред глазами всего Израиля.

\bibbookdescr{Jos}{
  inline={\LARGE Книга\\\Huge Иисуса Навина},
  toc={Иисус Навин},
  bookmark={Иисус Навин},
  header={Иисус Навин},
  %headerleft={},
  %headerright={},
  abbr={Нав}
}
\vs Jos 1:1 По смерти Моисея, раба Господня, Господь сказал Иисусу, сыну Навину, служителю Моисееву:
\vs Jos 1:2 Моисей, раб Мой, умер; итак встань, перейди через Иордан сей, ты и весь народ сей, в землю, которую Я даю им, сынам Израилевым.
\vs Jos 1:3 Всякое место, на которое ступят стопы ног ваших, Я даю вам, как Я сказал Моисею:
\vs Jos 1:4 от пустыни и Ливана сего до реки великой, реки Евфрата, всю землю Хеттеев; и до великого моря к западу солнца будут пределы ваши.
\vs Jos 1:5 Никто не устоит пред тобою во все дни жизни твоей; и как Я был с Моисеем, так буду и с тобою: не отступлю от тебя и не оставлю тебя.
\vs Jos 1:6 Будь тверд и мужествен; ибо ты народу сему передашь во владение землю, которую Я клялся отцам их дать им;
\vs Jos 1:7 только будь тверд и очень мужествен, и тщательно храни и исполняй весь закон, который завещал тебе Моисей, раб Мой; не уклоняйся от него ни направо ни налево, дабы поступать благоразумно во всех предприятиях твоих.
\vs Jos 1:8 Да не отходит сия книга закона от уст твоих; но поучайся в ней день и ночь, дабы в точности исполнять все, что в ней написано: тогда ты будешь успешен в путях твоих и будешь поступать благоразумно.
\vs Jos 1:9 Вот Я повелеваю тебе: будь тверд и мужествен, не страшись и не ужасайся; ибо с тобою Господь Бог твой везде, куда ни пойдешь.
\rsbpar\vs Jos 1:10 И дал Иисус повеление надзирателям народа и сказал:
\vs Jos 1:11 пройдите по стану и дайте повеление народу и скажите: заготовляйте себе пищу для пути, потому что, спустя три дня, вы пойдете за Иордан сей, дабы прийти взять землю, которую Господь Бог [отцов] ваших дает вам в наследие.
\vs Jos 1:12 А колену Рувимову, Гадову и половине колена Манассиина Иисус сказал:
\vs Jos 1:13 вспомните, что заповедал вам Моисей, раб Господень, говоря: Господь Бог ваш успокоил вас и дал вам землю сию;
\vs Jos 1:14 жены ваши, дети ваши и скот ваш пусть останутся в земле, которую дал вам Моисей за Иорданом; а вы все, могущие сражаться, вооружившись идите пред братьями вашими и помогайте им,
\vs Jos 1:15 доколе Господь [Бог ваш] не успокоит братьев ваших, как и вас; доколе и они не получат в наследие землю, которую Господь Бог ваш дает им; тогда возвратитесь в наследие ваше и владейте землею, которую Моисей, раб Господень, дал вам за Иорданом к востоку солнца.
\vs Jos 1:16 Они в ответ Иисусу сказали: все, что ни повелишь нам, сделаем, и куда ни пошлешь нас, пойдем;
\vs Jos 1:17 как слушали мы Моисея, так будем слушать и тебя: только Господь, Бог твой, да будет с тобою, как Он был с Моисеем;
\vs Jos 1:18 всякий, кто воспротивится повелению твоему и не послушает слов твоих во всем, что ты ни повелишь ему, будет предан смерти. Только будь тверд и мужествен!
\vs Jos 2:1 И послал Иисус, сын Навин, из Ситтима двух соглядатаев тайно и сказал: пойдите, осмотрите землю и Иерихон. [Два юноши] пошли и пришли [в Иерихон и вошли] в дом блудницы, которой имя Раав, и остались ночевать там.
\vs Jos 2:2 И сказано было царю Иерихонскому: вот, какие-то люди из сынов Израилевых пришли сюда в эту ночь, чтобы высмотреть землю.
\vs Jos 2:3 Царь Иерихонский послал сказать Рааве: выдай людей, пришедших к тебе, которые вошли в твой дом [ночью], ибо они пришли высмотреть всю землю.
\vs Jos 2:4 Но женщина взяла двух человек тех и скрыла их и сказала: точно приходили ко мне люди, но я не знала, откуда они;
\vs Jos 2:5 когда же в сумерки надлежало затворять ворота, тогда они ушли; не знаю, куда они пошли; гонитесь скорее за ними, вы догоните их.
\vs Jos 2:6 А сама отвела их на кровлю и скрыла их в снопах льна, разложенных у нее на кровле.
\vs Jos 2:7 \bibemph{Посланные} гнались за ними по дороге к Иордану до самой переправы; ворота же \bibemph{тотчас} затворили, после того как вышли погнавшиеся за ними.
\vs Jos 2:8 Прежде нежели они легли спать, она взошла к ним на кровлю
\vs Jos 2:9 и сказала им: я знаю, что Господь отдал землю сию вам, ибо вы навели на нас ужас, и все жители земли сей пришли от вас в робость;
\vs Jos 2:10 ибо мы слышали, как Господь [Бог] иссушил пред вами воду Чермного моря, когда вы шли из Египта, и как поступили вы с двумя царями Аморрейскими за Иорданом, с Сигоном и Огом, которых вы истребили;
\vs Jos 2:11 когда мы услышали об этом, ослабело сердце наше, и ни в ком [из нас] не стало духа против вас; ибо Господь Бог ваш есть Бог на небе вверху и на земле внизу;
\vs Jos 2:12 итак поклянитесь мне Господом [Богом вашим], что, как я сделала вам милость, так и вы сделаете милость дому отца моего, и дайте мне верный знак,
\vs Jos 2:13 что вы сохраните в живых отца моего и матерь мою, и братьев моих и сестер моих, и всех, кто есть у них, и избавите души наши от смерти.
\vs Jos 2:14 Эти люди сказали ей: душа наша вместо вас \bibemph{да будет предана} смерти, если вы [ныне] не откроете сего дела нашего; когда же Господь предаст нам землю, мы окажем тебе милость и истину.
\vs Jos 2:15 И спустила она их по веревке чрез окно, ибо дом ее был в городской стене, и она жила в стене;
\vs Jos 2:16 и сказала им: идите на гору, чтобы не встретили вас преследующие, и скрывайтесь там три дня, доколе не возвратятся погнавшиеся [за вами]; а после пойдете в путь ваш.
\vs Jos 2:17 И сказали ей те люди: мы свободны будем от твоей клятвы, которою ты нас закляла, \bibemph{если не сделаешь так}:
\vs Jos 2:18 вот, когда мы придем в эту землю, ты привяжи червленую веревку к окну, чрез которое ты нас спустила, а отца твоего и матерь твою и братьев твоих, все семейство отца твоего собери к себе в дом твой;
\vs Jos 2:19 и если кто-нибудь выйдет из дверей твоего дома вон, того кровь на голове его, а мы свободны [будем от сей клятвы твоей]; а кто будет с тобою в [твоем] доме, того кровь на голове нашей, если чья рука коснется его;
\vs Jos 2:20 если же [кто нас обидит, или] ты откроешь сие наше дело, то мы также свободны будем от клятвы твоей, которою ты нас закляла.
\vs Jos 2:21 Она сказала: да будет по словам вашим! И отпустила их, и они пошли, а она привязала к окну червленую веревку.
\vs Jos 2:22 Они пошли и пришли на гору, и пробыли там три дня, доколе не возвратились гнавшиеся \bibemph{за ними}. Гнавшиеся искали их по всей дороге и не нашли.
\vs Jos 2:23 Таким образом два сии человека пошли назад, сошли с горы, перешли [Иордан] и пришли к Иисусу, сыну Навину, и пересказали ему все, что с ними случилось.
\vs Jos 2:24 И сказали Иисусу: Господь [Бог наш] предал всю землю сию в руки наши, и все жители земли в страхе от нас.
\vs Jos 3:1 И встал Иисус рано поутру, и двинулись они от Ситтима и пришли к Иордану, он и все сыны Израилевы, и ночевали там, еще не переходя \bibemph{его}.
\vs Jos 3:2 Чрез три дня пошли надзиратели по стану
\vs Jos 3:3 и дали народу повеление, говоря: когда увидите ковчег завета Господа Бога вашего и священников [наших и] левитов, несущих его, то и вы двиньтесь с места своего и идите за ним;
\vs Jos 3:4 впрочем расстояние между вами и им должно быть до двух тысяч локтей мерою; не подходите к нему близко, чтобы знать вам путь, по которому идти; ибо вы не ходили сим путем ни вчера, ни третьего дня.
\vs Jos 3:5 И сказал Иисус народу: освятитесь [к утру], ибо завтра сотворит Господь среди вас чудеса.
\vs Jos 3:6 Священникам же сказал Иисус: возьмите ковчег завета [Господня] и идите пред народом. [Священники] взяли ковчег завета [Господня] и пошли пред народом.
\rsbpar\vs Jos 3:7 Тогда Господь сказал Иисусу: в сей день Я начну прославлять тебя пред очами всех [сынов] Израиля, дабы они узнали, что как Я был с Моисеем, так буду и с тобою;
\vs Jos 3:8 а ты дай повеление священникам, несущим ковчег завета, и скажи: как только войдете в воды Иордана, остановитесь в Иордане.
\vs Jos 3:9 Иисус сказал сынам Израилевым: подойдите сюда и выслушайте слова Господа, Бога вашего.
\vs Jos 3:10 И сказал Иисус: из сего узнаете, что среди вас есть Бог живый, Который прогонит от вас Хананеев и Хеттеев, и Евеев, и Ферезеев, и Гергесеев, и Аморреев, и Иевусеев:
\vs Jos 3:11 вот, ковчег завета Господа всей земли пойдет пред вами чрез Иордан;
\vs Jos 3:12 и возьмите себе двенадцать человек из колен Израилевых, по одному человеку из колена;
\vs Jos 3:13 и как только стопы ног священников, несущих ковчег Господа, Владыки всей земли, ступят в воду Иордана, вода Иорданская иссякнет, текущая же сверху вода остановится стеною.
\vs Jos 3:14 Итак, когда народ двинулся от своих шатров, чтобы переходить Иордан, и священники понесли ковчег завета [Господня] пред народом,
\vs Jos 3:15 то, лишь только несущие ковчег [завета Господня] вошли в Иордан, и ноги священников, несших ковчег, погрузились в воду Иордана~--- Иордан же выступает из всех берегов своих во все дни жатвы пшеницы,~---
\vs Jos 3:16 вода, текущая сверху, остановилась и стала стеною на весьма большое расстояние, до города Адама, который подле Цартана; а текущая в море равнины, в море Соленое, ушла и иссякла.
\vs Jos 3:17 И народ переходил против Иерихона; священники же, несшие ковчег завета Господня, стояли на суше среди Иордана твердою ногою. Все [сыны] Израилевы переходили по суше, доколе весь народ не перешел чрез Иордан.
\vs Jos 4:1 Когда весь народ перешел чрез Иордан, Господь сказал Иисусу:
\vs Jos 4:2 возьмите себе из народа двенадцать человек, по одному человеку из колена,
\vs Jos 4:3 и дайте им повеление и скажите: возьмите себе отсюда, из средины Иордана, где стояли ноги священников неподвижно, двенадцать камней, и перенесите их с собою, и положите их на ночлеге, где будете ночевать в эту ночь.
\vs Jos 4:4 Иисус призвал двенадцать человек, которых назначил из сынов Израилевых, по одному человеку из колена,
\vs Jos 4:5 и сказал им Иисус: пойдите пред ковчегом Господа Бога вашего в средину Иордана и [возьмите оттуда и] положите на плечо свое каждый по одному камню, по числу колен сынов Израилевых,
\vs Jos 4:6 чтобы они были у вас [лежащим всегда] знамением; когда спросят вас в последующее время сыны ваши и скажут: <<к чему у вас эти камни?>>,
\vs Jos 4:7 вы скажете им: <<\bibemph{в память того}, что вода Иордана разделилась пред ковчегом завета Господа [всей земли]; когда он переходил чрез Иордан, тогда вода Иордана разделилась>>; таким образом камни сии будут [у вас] для сынов Израилевых памятником на век.
\vs Jos 4:8 И сделали сыны Израилевы так, как приказал Иисус: взяли двенадцать камней из Иордана, как говорил Господь Иисусу, по числу колен сынов Израилевых, и перенесли их с собою на ночлег, и положили их там.
\vs Jos 4:9 И [другие] двенадцать камней поставил Иисус среди Иордана на месте, где стояли ноги священников, несших ковчег завета [Господня]. Они там и до сего дня.
\vs Jos 4:10 Священники, несшие ковчег [завета Господня], стояли среди Иордана, доколе не окончено было [Иисусом] все, что Господь повелел Иисусу сказать народу~--- так, как завещал Моисей Иисусу; а народ между тем поспешно переходил.
\vs Jos 4:11 Когда весь народ перешел [Иордан], тогда перешел и ковчег [завета] Господня, и священники пред народом;
\vs Jos 4:12 и сыны Рувима и сыны Гада и половина колена Манассиина перешли вооруженные впереди сынов Израилевых, как говорил им Моисей.
\vs Jos 4:13 Около сорока тысяч вооруженных на брань перешло пред Господом на равнины Иерихонские, чтобы сразиться.
\rsbpar\vs Jos 4:14 В тот день прославил Господь Иисуса пред очами всего Израиля и стали бояться его, как боялись Моисея, во все дни жизни его.
\vs Jos 4:15 И сказал Господь Иисусу, говоря:
\vs Jos 4:16 прикажи священникам, несущим ковчег откровения, выйти из Иордана.
\vs Jos 4:17 Иисус приказал священникам и сказал: выйдите из Иордана.
\vs Jos 4:18 И когда священники, несшие ковчег завета Господня, вышли из Иордана, то, лишь только стопы ног их ступили на сушу, вода Иордана устремилась по своему месту и пошла, как вчера и третьего дня, выше всех берегов своих.
\vs Jos 4:19 И вышел народ из Иордана в десятый день первого месяца и поставил стан в Галгале, на восточной стороне Иерихона.
\vs Jos 4:20 И двенадцать камней, которые взяли они из Иордана, Иисус поставил в Галгале
\vs Jos 4:21 и сказал сынам Израилевым: когда спросят в последующее время сыны ваши отцов своих: <<что значат эти камни?>>,
\vs Jos 4:22 скажите сынам вашим: <<Израиль перешел чрез Иордан сей по суше>>,
\vs Jos 4:23 ибо Господь Бог ваш иссушил воды Иордана для вас, доколе вы не перешли его, так же, как Господь Бог ваш сделал с Чермным морем, которое иссушил [Господь, Бог ваш,] пред нами, доколе мы не перешли его,
\vs Jos 4:24 дабы все народы земли познали, что рука Господня сильна, и дабы вы боялись Господа Бога вашего во все дни.
\vs Jos 5:1 Когда все цари Аморрейские, которые жили по эту сторону Иордана к морю, и все цари Ханаанские, которые при море, услышали, что Господь [Бог] иссушил воды Иордана пред сынами Израилевыми, доколе переходили они, тогда ослабело сердце их, [они ужаснулись] и не стало уже в них духа против сынов Израилевых.
\rsbpar\vs Jos 5:2 В то время сказал Господь Иисусу: сделай себе острые [каменные] ножи и обрежь сынов Израилевых во второй раз.
\vs Jos 5:3 И сделал себе Иисус острые [каменные] ножи и обрезал сынов Израилевых на [месте, названном]: Холм обрезания.
\vs Jos 5:4 Вот причина, почему обрезал Иисус [сынов Израилевых, которые тогда родились на пути, и которые из вышедших из Египта не были тогда обрезаны, всех их обрезал Иисус]: весь народ, вышедший из Египта, мужеского пола, все способные к войне умерли в пустыне на пути, по исшествии из Египта;
\vs Jos 5:5 весь же вышедший народ был обрезан, но весь народ, родившийся в пустыне на пути, после того как вышел из Египта, не был обрезан;
\vs Jos 5:6 ибо сыны Израилевы сорок [два] года ходили в пустыне [потому многие и не были обрезаны], доколе не перемер весь народ, способный к войне, вышедший из Египта, которые не слушали гласа Господня, и которым Господь клялся, что они не увидят земли, которую Господь с клятвою обещал отцам их, дать нам землю, где течет молоко и мед,
\vs Jos 5:7 а вместо их воздвиг сынов их. Сих обрезал Иисус, ибо они были необрезаны; потому что их, [как родившихся] на пути, не обрезывали.
\vs Jos 5:8 Когда весь народ был обрезан, оставался он на своем месте в стане, доколе не выздоровел.
\vs Jos 5:9 И сказал Господь Иисусу: ныне Я снял с вас посрамление Египетское. Почему и называется то место <<Галгал>>, даже до сего дня.
\vs Jos 5:10 И стояли сыны Израилевы станом в Галгале и совершили Пасху в четырнадцатый день месяца вечером на равнинах Иерихонских;
\vs Jos 5:11 и на другой день Пасхи стали есть из произведений земли сей, опресноки и сушеные зерна в самый тот день;
\vs Jos 5:12 а манна перестала падать на другой день после того, как они стали есть произведения земли, и не было более манны у сынов Израилевых, но они ели в тот год произведения земли Ханаанской.
\rsbpar\vs Jos 5:13 Иисус, находясь близ Иерихона, взглянул, и видит, и вот стоит пред ним человек, и в руке его обнаженный меч. Иисус подошел к нему и сказал ему: наш ли ты, или из неприятелей наших?
\vs Jos 5:14 Он сказал: нет; я вождь воинства Господня, теперь пришел [сюда]. Иисус пал лицем своим на землю, и поклонился и сказал ему: что господин мой скажет рабу своему?
\vs Jos 5:15 Вождь воинства Господня сказал Иисусу: сними обувь твою с ног твоих, ибо место, на котором ты стоишь, свято. Иисус так и сделал.
\rsbpar\vs Jos 5:16 Иерихон заперся и был заперт от \bibemph{страха} сынов Израилевых: никто не выходил [из него] и никто не входил.
\vs Jos 6:1 Тогда сказал Господь Иисусу: вот, Я предаю в руки твои Иерихон и царя его, [и находящихся в нем] людей сильных;
\vs Jos 6:2 пойдите вокруг города все способные к войне и обходите город однажды [в день]; и это делай шесть дней;
\vs Jos 6:3 и семь священников пусть несут семь труб юбилейных пред ковчегом; а в седьмой день обойдите вокруг города семь раз, и священники пусть трубят трубами;
\vs Jos 6:4 когда затрубит юбилейный рог, когда услышите звук трубы, тогда весь народ пусть воскликнет громким голосом, и стена города обрушится до своего основания, и [весь] народ пойдет [в город, устремившись] каждый с своей стороны.
\rsbpar\vs Jos 6:5 И призвал Иисус, сын Навин, священников [Израилевых] и сказал им: несите ковчег завета; а семь священников пусть несут семь труб юбилейных пред ковчегом Господним.
\vs Jos 6:6 И сказал [им, чтоб они сказали] народу: пойдите и обойдите вокруг города; вооруженные же пусть идут пред ковчегом Господним.
\vs Jos 6:7 Как скоро Иисус сказал народу, семь священников, несших семь труб юбилейных пред Господом, пошли и затрубили [громогласно] трубами, и ковчег завета Господня шел за ними;
\vs Jos 6:8 вооруженные же шли впереди священников, которые трубили трубами; а идущие позади следовали за ковчегом [завета Господня], во время шествия трубя трубами.
\vs Jos 6:9 Народу же Иисус дал повеление и сказал: не восклицайте и не давайте слышать голоса вашего, и чтобы слово не выходило из уст ваших до \bibemph{того} дня, доколе я не скажу вам: <<воскликните!>> и тогда воскликните.
\vs Jos 6:10 Таким образом ковчег [завета] Господня пошел вокруг города и обошел однажды; и пришли в стан и ночевали в стане.
\vs Jos 6:11 [На другой день] Иисус встал рано поутру, и священники понесли ковчег [завета] Господня;
\vs Jos 6:12 и семь священников, несших семь труб юбилейных пред ковчегом Господним, шли и трубили трубами; вооруженные же шли впереди их, а идущие позади следовали за ковчегом [завета] Господня и идучи трубили трубами.
\vs Jos 6:13 Таким образом и на другой день обошли вокруг города однажды и возвратились в стан. И делали это шесть дней.
\vs Jos 6:14 В седьмой день встали рано, при появлении зари, и обошли таким же образом вокруг города семь раз; только в этот день обошли вокруг города семь раз.
\vs Jos 6:15 Когда в седьмой раз священники трубили трубами, Иисус сказал народу: воскликните, ибо Господь предал вам город!
\vs Jos 6:16 город будет под заклятием, и все, что в нем~--- Господу [сил]; только Раав блудница пусть останется в живых, она и всякий, кто у нее в доме; потому что она укрыла посланных, которых мы посылали;
\vs Jos 6:17 но вы берегитесь заклятого, чтоб и самим не подвергнуться заклятию, если возьмете что-нибудь из заклятого, и чтобы на стан [сынов] Израилевых не навести заклятия и не сделать ему беды;
\vs Jos 6:18 и все серебро и золото, и сосуды медные и железные да будут святынею Господу и войдут в сокровищницу Господню.
\vs Jos 6:19 Народ воскликнул, и затрубили трубами. Как скоро услышал народ голос трубы, воскликнул народ [весь вместе] громким [и сильным] голосом, и обрушилась [вся] стена [города] до своего основания, и [весь] народ пошел в город, каждый с своей стороны, и взяли город.
\vs Jos 6:20 И предали заклятию всё, что в городе, и мужей и жен, и молодых и старых, и волов, и овец, и ослов, [всё] \bibemph{истребили} мечом.
\vs Jos 6:21 А двум юношам, высматривавшим землю, Иисус сказал: пойдите в дом оной блудницы и выведите оттуда ее и всех, которые у нее, так как вы поклялись ей.
\vs Jos 6:22 И пошли юноши, высматривавшие [город, в дом женщины] и вывели Раав [блудницу] и отца ее и мать ее, и братьев ее, и всех, которые у нее \bibemph{были}, и всех родственников ее вывели, и поставили их вне стана Израильского.
\vs Jos 6:23 А город и все, что в нем, сожгли огнем; только серебро и золото и сосуды медные и железные отдали, [чтобы внести Господу] в сокровищницу дома Господня.
\vs Jos 6:24 Раав же блудницу и дом отца ее и всех, которые у нее \bibemph{были}, Иисус оставил в живых, и она живет среди Израиля до сего дня, потому что она укрыла посланных, которых посылал Иисус для высмотрения Иерихона.
\rsbpar\vs Jos 6:25 В то время Иисус поклялся и сказал: проклят пред Господом тот, кто восставит и построит город сей Иерихон; на первенце своем он положит основание его и на младшем своем поставит врата его. [Так и сделал Азан, родом из Вефиля: он на Авироне, первенце своем, основал его и на меньшем, спасенном, поставил ворота его.]
\vs Jos 6:26 И Господь был с Иисусом, и слава его носилась по всей земле.
\vs Jos 7:1 Но сыны Израилевы сделали [великое] преступление [и взяли] из заклятого. Ахан, сын Хармия, сына Завдия, сына Зары, из колена Иудина, взял из заклятого, и гнев Господень возгорелся на сынов Израиля.
\vs Jos 7:2 Иисус из Иерихона послал людей в Гай, что близ Беф-Авена, с восточной стороны Вефиля, и сказал им: пойдите, осмотрите землю. Они пошли и осмотрели Гай.
\vs Jos 7:3 И возвратившись к Иисусу, сказали ему: не весь народ пусть идет, а пусть пойдет около двух тысяч или около трех тысяч человек, и поразят Гай; всего народа не утруждай туда, ибо их мало [там].
\vs Jos 7:4 Итак пошло туда из народа около трех тысяч человек, но они обратились в бегство от жителей Гайских;
\vs Jos 7:5 жители Гайские убили из них до тридцати шести человек, и преследовали их от ворот до Севарим и разбили их на спуске с горы; отчего сердце народа растаяло и стало, как вода.
\vs Jos 7:6 Иисус разодрал одежды свои и пал лицем своим на землю пред ковчегом Господним \bibemph{и лежал} до самого вечера, он и старейшины Израилевы, и посыпали прахом головы свои.
\vs Jos 7:7 И сказал Иисус: о, Господи Владыка! для чего Ты перевел народ сей чрез Иордан, дабы предать нас в руки Аморреев и погубить нас? о, если бы мы остались и жили за Иорданом!
\vs Jos 7:8 О, Господи! что сказать мне после того, как Израиль обратил тыл врагам своим?
\vs Jos 7:9 Хананеи и все жители земли услышат и окружат нас и истребят имя наше с земли. И что сделаешь \bibemph{тогда} имени Твоему великому?
\rsbpar\vs Jos 7:10 Господь сказал Иисусу: встань, для чего ты пал на лице твое?
\vs Jos 7:11 Израиль согрешил, и преступили они завет Мой, который Я завещал им; и взяли из заклятого, и украли, и утаили, и положили между своими вещами;
\vs Jos 7:12 за то сыны Израилевы не могли устоять пред врагами своими и обратили тыл врагам своим, ибо они подпали заклятию; не буду более с вами, если не истребите из среды вашей заклятого.
\vs Jos 7:13 Встань, освяти народ и скажи: освятитесь к утру, ибо так говорит Господь Бог Израилев: <<заклятое среди тебя, Израиль; посему ты не можешь устоять пред врагами твоими, доколе не отдалишь от себя заклятого>>;
\vs Jos 7:14 завтра подходите [все] по коленам вашим; колено же, которое укажет Господь, пусть подходит по племенам; племя, которое укажет Господь, пусть подходит по семействам; семейство, которое укажет Господь, пусть подходит по одному человеку;
\vs Jos 7:15 и обличенного в \bibemph{похищении} заклятого пусть сожгут огнем, его и все, что у него, за то, что он преступил завет Господень и сделал беззаконие среди Израиля.
\rsbpar\vs Jos 7:16 Иисус, встав рано поутру, велел подходить Израилю по коленам его, и указано колено Иудино;
\vs Jos 7:17 потом велел подходить племенам Иуды, и указано племя Зары; велел подходить племени Зарину по семействам, и указано [семейство] Завдиево;
\vs Jos 7:18 велел подходить семейству его по одному человеку, и указан Ахан, сын Хармия, сына Завдия, сына Зары, из колена Иудина.
\vs Jos 7:19 Тогда Иисус сказал Ахану: сын мой! воздай славу Господу, Богу Израилеву и сделай пред Ним исповедание и объяви мне, что ты сделал; не скрой от меня.
\vs Jos 7:20 В ответ Иисусу Ахан сказал: точно, я согрешил пред Господом Богом Израилевым и сделал то и то:
\vs Jos 7:21 между добычею увидел я одну прекрасную Сеннаарскую одежду и двести сиклей серебра и слиток золота весом в пятьдесят сиклей; это мне полюбилось и я взял это; и вот, оно спрятано в земле среди шатра моего, и серебро под ним [спрятано].
\vs Jos 7:22 Иисус послал людей, и они побежали в шатер [в стан]; и вот, \bibemph{все} это спрятано было в шатре его, и серебро под ним.
\vs Jos 7:23 Они взяли это из шатра и принесли к Иисусу и ко всем сынам Израилевым и положили пред Господом.
\vs Jos 7:24 Иисус и все Израильтяне с ним взяли Ахана, сына Зарина, и серебро, и одежду, и слиток золота, и сыновей его и дочерей его, и волов его и ослов его, и овец его и шатер его, и все, что у него \bibemph{было}, и вывели их [со всем] на долину Ахор.
\vs Jos 7:25 И сказал Иисус: за то, что ты навел на нас беду, Господь на тебя наводит беду в день сей. И побили его все Израильтяне камнями, и сожгли их огнем, и наметали на них камни.
\vs Jos 7:26 И набросали на него большую груду камней, \bibemph{которая уцелела} и до сего дня. После сего утихла ярость гнева Господня. Посему то место называется долиною Ахор даже до сего дня.
\vs Jos 8:1 Господь сказал Иисусу: не бойся и не ужасайся; возьми с собою весь народ, способный к войне, и встав пойди к Гаю; вот, Я предаю в руки твои царя Гайского и народ его, город его и землю его;
\vs Jos 8:2 сделай с Гаем и царем его то же, что сделал ты с Иерихоном и царем его, только добычу его и скот его разделите себе; сделай засаду позади города.
\rsbpar\vs Jos 8:3 Иисус и весь народ, способный к войне, встал, чтобы идти к Гаю, и выбрал Иисус тридцать тысяч человек храбрых и послал их ночью,
\vs Jos 8:4 и дал им приказание и сказал: смотрите, вы будете составлять засаду у города позади города; не отходите далеко от города и будьте все готовы;
\vs Jos 8:5 а я и весь народ, который со мною, подойдем к городу; и когда [жители Гая] выступят против нас, как и прежде, то мы побежим от них;
\vs Jos 8:6 они пойдут за нами, так что мы отвлечем их от города; ибо они скажут: <<бегут от нас, как и прежде>>; когда мы побежим от них,
\vs Jos 8:7 тогда вы встаньте из засады и завладейте городом, и Господь Бог ваш предаст его в руки ваши;
\vs Jos 8:8 когда возьмете город, зажгите город огнем, по слову Господню сделайте; смотрите, я повелеваю вам.
\vs Jos 8:9 Таким образом послал их Иисус, и они пошли в засаду и засели между Вефилем и между Гаем, с западной стороны Гая; а Иисус в ту ночь ночевал среди народа.
\rsbpar\vs Jos 8:10 Встав рано поутру, Иисус осмотрел народ, и пошел он и старейшины Израилевы впереди народа к Гаю;
\vs Jos 8:11 и весь народ, способный к войне, который был с ним, пошел, приблизился и подошел к городу [с восточной стороны, засада же была к западу от города],
\vs Jos 8:12 и поставил стан с северной стороны Гая, а между ним и Гаем была долина. Потом взял он около пяти тысяч человек и посадил их в засаде между Вефилем и Гаем, с западной стороны города.
\vs Jos 8:13 И народ расположил весь стан, который был с северной стороны города, так, что задняя часть была с западной стороны города. И пришел Иисус в ту ночь на средину долины.
\vs Jos 8:14 Когда увидел это царь Гайский, тотчас с жителями города, встав рано, выступил против Израиля на сражение, он и весь народ его, на назначенное место пред равниною; а он не знал, что для него есть засада позади города [его].
\vs Jos 8:15 Иисус и весь Израиль, будто пораженные ими, побежали к пустыне;
\vs Jos 8:16 а они кликнули весь народ, который был в городе, чтобы преследовать их, и, преследуя Иисуса, отдалились от города;
\vs Jos 8:17 в Гае и Вефиле не осталось ни одного человека, который не погнался бы за Израилем; и город свой они оставили отворенным, преследуя Израиля.
\vs Jos 8:18 Тогда Господь сказал Иисусу: простри копье, которое в руке твоей, к Гаю, ибо Я предам его в руки твои [и засада тотчас встанет с места своего]. Иисус простер [руку свою и] копье, которое было в его руке, к городу.
\vs Jos 8:19 Сидевшие в засаде тотчас встали с места своего и побежали, как скоро он простер руку свою, вошли в город и взяли его и тотчас зажгли город огнем.
\vs Jos 8:20 Жители Гая, оглянувшись назад, увидели, что дым от города восходил к небу. И не было для них места, куда бы бежать~--- ни туда, ни сюда; ибо народ, бежавший к пустыне, обратился на преследователей.
\vs Jos 8:21 Иисус и весь Израиль, увидев, что сидевшие в засаде взяли город, и дым от города восходил [к небу], возвратились и стали поражать жителей Гая;
\vs Jos 8:22 а те из города вышли навстречу им, так что они находились в средине между Израильтянами, \bibemph{из которых} одни были с той стороны, а другие с другой; так поражали их, что не оставили ни одного из них, уцелевшего или убежавшего;
\vs Jos 8:23 а царя Гайского взяли живого и привели его к Иисусу.
\rsbpar\vs Jos 8:24 Когда Израильтяне перебили всех жителей Гая на поле, в пустыне, куда они преследовали их, и когда все они до последнего пали от острия меча, тогда все Израильтяне обратились к Гаю и поразили его острием меча.
\vs Jos 8:25 Падших в тот день мужей и жен, всех жителей Гая, было двенадцать тысяч.
\vs Jos 8:26 Иисус не опускал руки своей, которую простер с копьем, доколе не предал заклятию всех жителей Гая;
\vs Jos 8:27 только скот и добычу города сего [сыны] Израиля разделили между собою, по слову Господа, которое [Господь] сказал Иисусу.
\vs Jos 8:28 И сожег Иисус Гай и обратил его в вечные развалины, в пустыню, до сего дня;
\vs Jos 8:29 а царя Гайского повесил на дереве, [и был он на дереве] до вечера; по захождении же солнца приказал Иисус, и сняли труп его с дерева, и бросили его у ворот городских, и набросали над ним большую груду камней, \bibemph{которая уцелела} даже до сего дня.
\vs Jos 8:30 Тогда Иисус устроил жертвенник Господу Богу Израилеву на горе Гевал,
\vs Jos 8:31 как заповедал Моисей, раб Господень, сынам Израилевым, о чем написано в книге закона Моисеева,~--- жертвенник из камней цельных, на которые не поднимали железа; и принесли на нем всесожжение Господу и совершили жертвы мирные.
\vs Jos 8:32 И написал [Иисус] там на камнях список с закона Моисеева, который он написал пред сынами Израилевыми.
\vs Jos 8:33 Весь Израиль, старейшины его и надзиратели [его] и судьи его, стали с той и другой стороны ковчега против священников [и] левитов, носящих ковчег завета Господня, как пришельцы, так и природные жители, одна половина их у горы Гаризим, а другая половина у горы Гевал, как прежде повелел Моисей, раб Господень, благословлять народ Израилев.
\vs Jos 8:34 И потом прочитал [Иисус] все слова закона, благословение и проклятие, как написано в книге закона;
\vs Jos 8:35 из всего, что Моисей заповедал [Иисусу], не было \bibemph{ни одного} слова, которого Иисус не прочитал бы пред всем собранием Израиля, [пред мужами,] и женами, и детьми, и пришельцами, находившимися среди них.
\vs Jos 9:1 Услышав сие, все цари [Аморрейские], которые за Иорданом, на горе и на равнине и по всему берегу великого моря, [и которые] близ Ливана, Хеттеи, Аморреи, [Гергесеи,] Хананеи, Ферезеи, Евеи и Иевусеи,
\vs Jos 9:2 собрались вместе, дабы единодушно сразиться с Иисусом и Израилем.
\vs Jos 9:3 Но жители Гаваона, услышав, что Иисус сделал с Иерихоном и Гаем,
\vs Jos 9:4 употребили хитрость: пошли, запаслись хлебом на дорогу и положили ветхие мешки на ослов своих и ветхие, изорванные и заплатанные мехи вина;
\vs Jos 9:5 и обувь на ногах их была ветхая с заплатами, и одежда на них ветхая; и весь дорожный хлеб их был сухой и заплесневелый [и раскрошенный].
\vs Jos 9:6 Они пришли к Иисусу в стан [Израильский] в Галгал и сказали ему и всем Израильтянам: из весьма дальней земли пришли мы; итак заключите с нами союз.
\vs Jos 9:7 Израильтяне же сказали Евеям: может быть, вы живете близ нас? как нам заключить с вами союз?
\vs Jos 9:8 Они сказали Иисусу: мы рабы твои. Иисус же сказал им: кто вы и откуда пришли?
\vs Jos 9:9 Они сказали ему: из весьма дальней земли пришли рабы твои во имя Господа Бога твоего; ибо мы слышали славу Его и все, что сделал Он в Египте,
\vs Jos 9:10 и все, что Он сделал двум царям Аморрейским, которые [были] по ту сторону Иордана, Сигону, царю Есевонскому, и Огу, царю Васанскому, который [жил] в Астарофе [и Едреи].
\vs Jos 9:11 [Слыша сие,] старейшины наши и все жители нашей земли сказали нам: возьмите в руки ваши хлеба на дорогу и пойдите навстречу им и скажите им: <<мы рабы ваши; итак заключите с нами союз>>.
\vs Jos 9:12 Этот хлеб наш из домов наших мы взяли теплый в тот день, когда пошли к вам, а теперь вот, он сделался сухой и заплесневелый;
\vs Jos 9:13 и эти мехи с вином, которые мы налили новые, вот, изорвались; и эта одежда наша и обувь наша обветшала от весьма дальней дороги.
\vs Jos 9:14 Израильтяне взяли их хлеба, а Господа не вопросили.
\vs Jos 9:15 И заключил Иисус с ними мир и постановил с ними условие в том, что он сохранит им жизнь; и поклялись им начальники общества.
\vs Jos 9:16 А чрез три дня, как заключили они с ними союз, услышали, что они соседи их и живут близ них;
\vs Jos 9:17 ибо сыны Израилевы, отправившись в путь, пришли в города их на третий день; города же их [были]: Гаваон, Кефира, Беероф и Кириаф-Иарим.
\vs Jos 9:18 [Иисус и] сыны Израилевы не побили их, потому что [все] начальники общества клялись им Господом Богом Израилевым. За это все общество [Израилево] возроптало на начальников.
\vs Jos 9:19 Все начальники сказали всему обществу: мы клялись им Господом Богом Израилевым и потому не можем коснуться их;
\vs Jos 9:20 а вот что сделаем с ними: оставим их в живых, чтобы не постиг нас гнев за клятву, которою мы клялись им.
\vs Jos 9:21 И сказали им начальники: пусть они живут, но будут рубить дрова и черпать воду для всего общества. [И сделало все общество] так, как сказали им начальники.
\vs Jos 9:22 Иисус призвал их и сказал: для чего вы обманули нас, сказав: <<мы весьма далеко от вас>>, тогда как вы живете близ нас?
\vs Jos 9:23 за это прокляты вы! без конца вы будете рабами, будете рубить дрова и черпать воду для [меня и для] дома Бога моего!
\vs Jos 9:24 Они в ответ Иисусу сказали: дошло до сведения рабов твоих, что Господь Бог твой повелел Моисею, рабу Своему, дать вам всю землю и погубить [нас и] всех жителей сей земли пред лицем вашим; посему мы весьма боялись, чтобы вы не лишили нас жизни, и сделали это дело;
\vs Jos 9:25 теперь вот мы в руке твоей: как лучше и справедливее тебе покажется поступить с нами, так и поступи.
\vs Jos 9:26 И поступил с ними так: избавил их от руки сынов Израилевых, и они не умертвили их;
\vs Jos 9:27 и определил в тот день Иисус, чтобы они рубили дрова и черпали воду для [всего] общества и для жертвенника Господня; [посему жители Гаваона сделались дровосеками и водоносами для жертвенника Божия] даже до сего дня, на месте, какое ни избрал бы [Господь].
\vs Jos 10:1 Когда Адониседек, царь Иерусалимский, услышал, что Иисус взял Гай и предал его заклятию, и что так же поступил с Гаем и царем его, как поступил с Иерихоном и царем его, и что жители Гаваона заключили мир [с Иисусом и] с Израилем и остались среди их,
\vs Jos 10:2 тогда он весьма испугался, потому что Гаваон [был] город большой, как один из царских городов, и больше Гая, и все жители его люди храбрые.
\vs Jos 10:3 Посему Адониседек, царь Иерусалимский, послал к Гогаму, царю Хевронскому, и к Фираму, царю Иармуфскому, и к Яфию, царю Лахисскому, и к Девиру, царю Еглонскому, чтобы сказать:
\vs Jos 10:4 придите ко мне и помогите мне поразить Гаваон за то, что он заключил мир с Иисусом и сынами Израилевыми.
\vs Jos 10:5 Они собрались, и пошли пять царей Аморрейских: царь Иерусалимский, царь Хевронский, царь Иармуфский, царь Лахисский, царь Еглонский, они и все ополчение их, и расположились станом подле Гаваона, чтобы воевать против него.
\vs Jos 10:6 Жители Гаваона послали к Иисусу в стан [Израильский], в Галгал, сказать: не отними руки твоей от рабов твоих; приди к нам скорее, спаси нас и подай нам помощь; ибо собрались против нас все цари Аморрейские, живущие на горах.
\vs Jos 10:7 Иисус пошел из Галгала сам, и с ним весь народ, способный к войне, и все мужи храбрые.
\rsbpar\vs Jos 10:8 И сказал Господь Иисусу: не бойся их, ибо Я предал их в руки твои: никто из них не устоит пред лицем твоим.
\vs Jos 10:9 И пришел на них Иисус внезапно, [потому что] всю ночь шел он из Галгала.
\vs Jos 10:10 Господь привел их в смятение при виде Израильтян, и они поразили их в Гаваоне сильным поражением, и преследовали их по дороге к возвышенности Вефорона, и поражали их до Азека и до Македа.
\vs Jos 10:11 Когда же они бежали от Израильтян по скату горы Вефоронской, Господь бросал на них с небес большие камни [града] до самого Азека, и они умирали; больше было тех, которые умерли от камней града, нежели тех, которых умертвили сыны Израилевы мечом [на сражении].
\vs Jos 10:12 Иисус воззвал к Господу в тот день, в который предал Господь [Бог] Аморрея в руки Израилю, когда побил их в Гаваоне, и они побиты были пред лицем сынов Израилевых, и сказал пред Израильтянами: стой, солнце, над Гаваоном, и луна, над долиною Аиалонскою!
\vs Jos 10:13 И остановилось солнце, и луна стояла, доколе народ мстил врагам своим. Не это ли написано в книге Праведного: <<стояло солнце среди неба и не спешило к западу почти целый день>>?
\vs Jos 10:14 И не было такого дня ни прежде ни после того, в который Господь [так] слушал бы гласа человеческого. Ибо Господь сражался за Израиля.
\vs Jos 10:15 Потом возвратился Иисус и весь Израиль с ним в стан, в Галгал.
\vs Jos 10:16 А те пять царей убежали и скрылись в пещере в Македе.
\vs Jos 10:17 Когда донесено было Иисусу и сказано: <<нашлись пять царей, они скрываются в пещере в Македе>>,
\vs Jos 10:18 Иисус сказал: <<привалите большие камни к отверстию пещеры и приставьте к ней людей стеречь их;
\vs Jos 10:19 а вы не останавливайтесь [здесь], но преследуйте врагов ваших и истребляйте заднюю часть войска их и не давайте им уйти в города их, ибо Господь Бог ваш предал их в руки ваши>>.
\rsbpar\vs Jos 10:20 После того, как Иисус и сыны Израилевы совершенно поразили их весьма великим поражением, и оставшиеся из них убежали в города укрепленные,
\vs Jos 10:21 весь народ возвратился в стан к Иисусу в Макед с миром, и никто на сынов Израилевых не пошевелил языком своим.
\vs Jos 10:22 Тогда Иисус сказал: откройте отверстие пещеры и выведите ко мне из пещеры пятерых царей тех.
\vs Jos 10:23 Так и сделали: вывели к нему из пещеры пятерых царей тех: царя Иерусалимского, царя Хевронского, царя Иармуфского, царя Лахисского и царя Еглонского.
\vs Jos 10:24 Когда вывели царей сих к Иисусу, Иисус призвал всех Израильтян и сказал вождям воинов, ходившим с ним: подойдите, наступите ногами вашими на выи царей сих. Они подошли и наступили ногами своими на выи их.
\vs Jos 10:25 Иисус сказал им: не бойтесь и не ужасайтесь, будьте тверды и мужественны; ибо так поступит Господь со всеми врагами вашими, с которыми будете воевать.
\vs Jos 10:26 Потом поразил их Иисус и убил их и повесил их на пяти деревах; и висели они на деревах до вечера.
\vs Jos 10:27 При захождении солнца приказал Иисус, и сняли их с дерев, и бросили их в пещеру, в которой они скрывались, и привалили большие камни к отверстию пещеры, \bibemph{которые там} даже до сего дня.
\vs Jos 10:28 В тот же день взял Иисус Макед, и поразил [его] мечом и царя его, и предал заклятию их и все дышащее, что находилось в нем: никого не оставил, кто бы уцелел [и избежал]; и поступил с царем Македским так же, как поступил с царем Иерихонским.
\vs Jos 10:29 И пошел Иисус и все Израильтяне с ним из Македа к Ливне и воевал против Ливны;
\vs Jos 10:30 и предал Господь и ее в руки Израиля, [и взяли ее] и царя ее, и истребил ее Иисус мечом и все дышащее, что \bibemph{находилось} в ней: никого не оставил в ней, кто бы уцелел [и избежал], и поступил с царем ее так же, как поступил с царем Иерихонским.
\vs Jos 10:31 Из Ливны пошел Иисус и все Израильтяне с ним к Лахису и расположился подле него станом и воевал против него;
\vs Jos 10:32 и предал Господь Лахис в руки Израиля, и взял он его на другой день, и поразил его мечом и все дышащее, что было в нем, [и истребил его] так, как поступил с Ливною.
\vs Jos 10:33 Тогда пришел на помощь Лахису Горам, царь Газерский; но Иисус поразил его и народ его [мечом] так, что никого у него не оставил, кто бы уцелел [и избежал].
\vs Jos 10:34 И пошел Иисус и все Израильтяне с ним из Лахиса к Еглону и расположились подле него станом и воевали против него;
\vs Jos 10:35 [и предал его Господь в руки Израиля,] и взяли его в тот же день и поразили его мечом, и все дышащее, что находилось в нем в тот день, предал он заклятию, как поступил с Лахисом.
\vs Jos 10:36 И пошел Иисус и все Израильтяне с ним из Еглона к Хеврону и воевали против него;
\vs Jos 10:37 и взяли его и поразили его мечом, и царя его, и все города его, и все дышащее, что находилось в нем; никого не оставил, кто уцелел бы, как поступил он и с Еглоном: предал заклятию его и все дышащее, что находилось в нем.
\vs Jos 10:38 Потом обратился Иисус и весь Израиль с ним к Давиру и воевал против него;
\vs Jos 10:39 и взял его и царя его и все города его, и поразили их мечом, и предали заклятию [их и] все дышащее, что находилось в нем: никого не осталось, кто уцелел бы; как поступил с Хевроном и царем его, так поступил с Давиром и царем его, и как поступил с Ливною и царем ее.
\vs Jos 10:40 И поразил Иисус всю землю нагорную и полуденную, и низменные места и землю, лежащую у гор, и всех царей их: никого не оставил, кто уцелел бы, и все дышащее предал заклятию, как повелел Господь Бог Израилев;
\vs Jos 10:41 поразил их Иисус от Кадес-Варни до Газы, и всю землю Гошен даже до Гаваона;
\vs Jos 10:42 и всех царей сих и земли их Иисус взял одним разом, ибо Господь Бог Израилев сражался за Израиля.
\vs Jos 10:43 Потом Иисус и все Израильтяне с ним возвратились в стан, в Галгал.
\vs Jos 11:1 Услышав \bibemph{сие}, Иавин, царь Асорский, послал к Иоваву, царю Мадонскому, и к царю Шимронскому, и к царю Ахсафскому,
\vs Jos 11:2 и к царям, которые \bibemph{жили} к северу на горе и на равнине с южной стороны Хиннарофа, и на низменных местах, и в Нафоф-Доре к западу,
\vs Jos 11:3 к Хананеям, \bibemph{которые жили} к востоку и к морю, к Аморреям и Хеттеям, к Ферезеям и к Иевусеям, \bibemph{жившим} на горе, и к Евеям, \bibemph{жившим} подле Ермона в земле Массифе.
\vs Jos 11:4 И выступили они и все ополчение их с ними, многочисленный народ, который множеством равнялся песку на берегу морском; и коней и колесниц \bibemph{было} весьма много.
\vs Jos 11:5 И собрались все цари сии, и пришли и расположились станом вместе при водах Меромских, чтобы сразиться с Израилем.
\vs Jos 11:6 Но Господь сказал Иисусу: не бойся их, ибо завтра, около сего времени, Я предам всех [их] на избиение [сынам] Израиля; коням же их перережь жилы и колесницы их сожги огнем.
\rsbpar\vs Jos 11:7 Иисус и с ним весь народ, способный к войне, внезапно вышли на них к водам Меромским и напали на них.
\vs Jos 11:8 И предал их Господь в руки Израильтян, и поразили они их, и преследовали их до Сидона великого и до Мисрефоф-Маима, и до долины Мицфы к востоку, и перебили их, так что никого из них не осталось, кто уцелел бы [и избежал].
\vs Jos 11:9 И поступил Иисус с ними так, как сказал ему Господь: коням их перерезал жилы и колесницы их сожег огнем.
\vs Jos 11:10 В то же время возвратившись Иисус взял Асор и царя его убил мечом (Асор же прежде был главою всех царств сих);
\vs Jos 11:11 и побили все дышащее, что было в нем, мечом, [все] предав заклятию: не осталось ни одной души; а Асор сожег он огнем.
\vs Jos 11:12 И все города царей сих и всех царей их взял Иисус и побил мечом, предав их заклятию, как повелел Моисей, раб Господень;
\vs Jos 11:13 впрочем всех городов, лежавших на возвышенности, не жгли Израильтяне, кроме одного Асора, \bibemph{который} сжег Иисус.
\vs Jos 11:14 А всю добычу городов сих и [весь] скот разграбили сыны Израилевы себе; людей же всех перебили мечом, так что истребили \bibemph{всех} их: не оставили [из них] ни одной души.
\vs Jos 11:15 Как повелел Господь Моисею, рабу Своему, так Моисей заповедал Иисусу, а Иисус так и сделал: не отступил ни от одного слова во всем, что повелел Господь Моисею.
\rsbpar\vs Jos 11:16 Таким образом Иисус взял всю эту нагорную землю, всю землю полуденную, всю землю Гошен и низменные места, и равнину и гору Израилеву, и низменные места [при горе],
\vs Jos 11:17 от горы Халак, простирающейся к Сеиру, до Ваал-Гада в долине Ливанской, подле горы Ермона, и всех царей их взял, поразил их и убил.
\vs Jos 11:18 Долгое время вел Иисус войну со всеми сими царями.
\vs Jos 11:19 Не было [ни одного] города, который заключил бы мир с сынами Израилевыми, кроме Евеев, жителей Гаваона: все взяли они войною;
\vs Jos 11:20 ибо от Господа было то, что они ожесточили сердце свое и войною встречали Израиля~--- для того, чтобы преданы были заклятию и чтобы не было им помилования, но чтобы истреблены были так, как повелел Господь Моисею.
\rsbpar\vs Jos 11:21 В то же время пришел Иисус и поразил [всех] Енакимов на горе, в Хевроне, в Давире, в Анаве, на всей горе Иудиной и на всей горе Израилевой; с городами их предал их Иисус заклятию;
\vs Jos 11:22 не осталось [ни одного] из Енакимов в земле сынов Израилевых, остались только в Газе, в Гефе и в Азоте.
\vs Jos 11:23 Таким образом взял Иисус всю землю, как говорил Господь Моисею, и отдал ее Иисусу в удел Израильтянам, по разделению между коленами их. И успокоилась земля от войны.
\vs Jos 12:1 Вот цари той земли, которых поразили сыны Израилевы и которых землю взяли в наследие по ту сторону Иордана к востоку солнца, от потока Арнона до горы Ермона, и всю равнину к востоку:
\vs Jos 12:2 Сигон, царь Аморрейский, живший в Есевоне, владевший от Ароера, что при береге потока Арнона, и от средины потока, половиною Галаада, до потока Иавока, предела Аммонитян,
\vs Jos 12:3 и равниною до самого моря Хиннерефского к востоку и до моря равнины, моря Соленого, к востоку по дороге к Беф-Иешимофу, а к югу местами, лежащими при подошве Фасги;
\vs Jos 12:4 сопредельный \bibemph{ему} Ог, царь Васанский, последний из Рефаимов, живший в Астарофе и в Едреи,
\vs Jos 12:5 владевший горою Ермоном и Салхою и всем Васаном, до предела Гессурского и Маахского, и половиною Галаада, до предела Сигона, царя Есевонского.
\vs Jos 12:6 Моисей, раб Господень, и сыны Израилевы убили их; и дал ее Моисей, раб Господень, в наследие \bibemph{колену} Рувимову и Гадову и половине колена Манассиина.
\rsbpar\vs Jos 12:7 И вот цари [Аморрейской] земли, которых поразил Иисус и сыны Израилевы по эту сторону Иордана к западу, от Ваал-Гада на долине Ливанской до Халака, горы, простирающейся к Сеиру, которую отдал Иисус коленам Израилевым в наследие, по разделению их,
\vs Jos 12:8 на горе, на низменных местах, на равнине, на местах, лежащих при горах, и в пустыне и на юге, Хеттеев, Аморреев, Хананеев, Ферезеев, Евеев и Иевусеев:
\vs Jos 12:9 один царь Иерихона, один царь Гая, что близ Вефиля,
\vs Jos 12:10 один царь Иерусалима, один царь Хеврона,
\vs Jos 12:11 один царь Иармуфа, один царь Лахиса,
\vs Jos 12:12 один царь Еглона, один царь Газера,
\vs Jos 12:13 один царь Давира, один царь Гадера,
\vs Jos 12:14 один царь Хормы, один царь Арада,
\vs Jos 12:15 один царь Ливны, один царь Одоллама,
\vs Jos 12:16 один царь Македа, один царь Вефиля,
\vs Jos 12:17 один царь Таппуаха, один царь Хефера.
\vs Jos 12:18 Один царь Афека, один царь Шарона,
\vs Jos 12:19 один царь Мадона, один царь Асора,
\vs Jos 12:20 один царь Шимрон-Мерона, один царь Ахсафа,
\vs Jos 12:21 один царь Фаанаха, один царь Мегиддона,
\vs Jos 12:22 один царь Кедеса, один царь Иокнеама при Кармиле,
\vs Jos 12:23 один царь Дора при Нафаф-Доре, один царь Гоима в Галгале,
\vs Jos 12:24 один царь Фирцы. Всех царей тридцать один.
\vs Jos 13:1 Когда Иисус состарился, вошел в лета \bibemph{преклонные}, тогда Господь сказал ему: ты состарился, вошел в лета \bibemph{преклонные}, а земли брать в наследие остается еще очень много.
\vs Jos 13:2 Остается сия земля: все округи Филистимские и вся \bibemph{земля} Гессурская [и Хананейская].
\vs Jos 13:3 От Сихора, что пред Египтом, до пределов Екрона к северу, считаются Ханаанскими пять владельцев Филистимских: Газский, Азотский, Аскалонский, Гефский, Екронский и Аввейский;
\vs Jos 13:4 к югу же вся земля Ханаанская от Меары Сидонской до Афека, до пределов Аморрейских,
\vs Jos 13:5 также [Филистимская] земля Гевла и весь Ливан к востоку солнца от Ваал-Гада, \bibemph{что} подле горы Ермона, до входа в Емаф.
\vs Jos 13:6 Всех горных жителей от Ливана до Мисрефоф-Маима, всех Сидонян Я изгоню от лица сынов Израилевых. Раздели же ее в удел Израилю, как Я повелел тебе;
\vs Jos 13:7 раздели землю сию в удел девяти коленам и половине колена Манассиина [от Иордана до моря великого к западу отдай ее \bibemph{им}; великое море будет пределом].
\vs Jos 13:8 А \bibemph{колено} Рувимово и Гадово с другою половиною колена Манассиина получили удел свой от Моисея за Иорданом к востоку [солнца], как дал им Моисей, раб Господень,
\vs Jos 13:9 от Ароера, который на берегу потока Арнона, и город, который среди потока, и всю равнину Медеву до Дивона;
\vs Jos 13:10 также все города Сигона, царя Аморрейского, который царствовал в Есевоне, до пределов Аммонитских,
\vs Jos 13:11 также Галаад и область Гессурскую и Маахскую, и всю гору Ермон и весь Васан до Салхи,
\vs Jos 13:12 все царство Ога Васанского, который царствовал в Астарофе и в Едреи. Он оставался один из Рефаимов, которых Моисей поразил и прогнал.
\vs Jos 13:13 Но сыны Израилевы не выгнали жителей Гессура и Маахи [и Хананеев], и живет Гессур и Мааха среди Израиля до сего дня.
\vs Jos 13:14 Только колену Левиину не дал он удела: жертвы Господа Бога Израилева суть удел его, как сказал ему Господь.\rsbpar [Вот разделение, какое сделал Моисей сынам Израилевым по племенам их на равнинах Моавитских за Иорданом, напротив Иерихона:]
\vs Jos 13:15 колену сынов Рувимовых по племенам их дал \bibemph{удел} Моисей:
\vs Jos 13:16 пределом их был Ароер, который на берегу потока Арнона, и город, который среди потока, и вся равнина при Медеве,
\vs Jos 13:17 Есевон и все города его, которые на равнине, и Дивон, Вамоф-Ваали Беф-Ваал-Меон,
\vs Jos 13:18 Иааца, Кедемоф и Мефааф,
\vs Jos 13:19 Кириафаим, Сивма и Цереф-Шахар на горе Емек,
\vs Jos 13:20 Беф-Фегор и места при подошве Фасги и Беф-Иешимоф,
\vs Jos 13:21 и все города на равнине, и все царство Сигона, царя Аморрейского, который царствовал в Есевоне, которого убил Моисей, равно как и вождей Мадиамских: Евия, и Рекема, и Цура, и Хура, и Реву, князей Сигоновых, живших в земле [той];
\vs Jos 13:22 также Валаама, сына Веорова, прорицателя, убили сыны Израилевы мечом в числе убитых ими.
\vs Jos 13:23 Пределом сынов Рувимовых был Иордан. Вот удел сынов Рувимовых по племенам их, города и села их.
\vs Jos 13:24 Моисей дал также \bibemph{удел} колену Гадову, сынам Гадовым, по племенам их:
\vs Jos 13:25 пределом их был Иазер и все города Галаадские, и половина земли сынов Аммоновых до Ароера, что пред Раввою,
\vs Jos 13:26 и \bibemph{земли} от Есевона до Рамаф-Мицфы и Ветонима и от Маханаима до пределов Давира,
\vs Jos 13:27 и на долине Беф-Гарам и Беф-Нимра и Сокхоф и Цафон, остаток царства Сигона, царя Есевонского; пределом его был Иордан до моря Хиннерефского за Иорданом к востоку.
\vs Jos 13:28 Вот удел сынов Гадовых по племенам их, города и села их.
\vs Jos 13:29 Моисей дал \bibemph{удел} и половине колена Манассиина, который \bibemph{принадлежал} половине колена сынов Манассииных, по племенам их;
\vs Jos 13:30 предел их был: от Маханаима весь Васан, все царство Ога, царя Васанского, и все селения Иаировы, что в Васане, шестьдесят городов;
\vs Jos 13:31 а половина Галаада и Астароф и Едрея, царственные города Ога Васанского, [даны] сынам Махира, сына Манассиина, половине сынов Махировых, по племенам их.
\vs Jos 13:32 Вот что Моисей дал в удел на равнинах Моавитских за Иорданом против Иерихона к востоку.
\vs Jos 13:33 Но колену Левиину Моисей не дал удела: Господь Бог Израилев Сам есть удел их, как Он говорил им.
\vs Jos 14:1 Вот что получили в удел сыны Израилевы в земле Ханаанской, что разделили им в удел Елеазар священник и Иисус, сын Навин, и начальники поколений в коленах сынов Израилевых;
\vs Jos 14:2 по жребию делили они, как повелел Господь чрез Моисея, девяти коленам и половине колена [Манассиина],
\vs Jos 14:3 ибо двум коленам и половине колена [Манассиина] Моисей дал удел за Иорданом, левитам же не дал удела между ними;
\vs Jos 14:4 ибо от сынов Иосифовых произошли два колена: Манассиино и Ефремово; посему они и не дали левитам части в земле, [а только] города для жительства с предместьями их для скота их и для \bibemph{других} выгод их.
\vs Jos 14:5 Как повелел Господь Моисею, так \bibemph{и} сделали сыны Израилевы, когда делили на уделы землю.
\vs Jos 14:6 Сыны Иудины пришли в Галгал к Иисусу. И сказал ему Халев, сын Иефоннии, Кенезеянин: ты знаешь, что говорил Господь Моисею, человеку Божию, о мне и о тебе в Кадес-Варне;
\vs Jos 14:7 я был сорока лет, когда Моисей, раб Господень, посылал меня из Кадес-Варни осмотреть землю, и я принес ему в ответ, что было у меня на сердце:
\vs Jos 14:8 братья мои, которые ходили со мною, привели в робость сердце народа, а я в точности следовал Господу Богу моему;
\vs Jos 14:9 и клялся Моисей в тот день и сказал: <<земля, по которой ходила нога твоя, будет уделом тебе и детям твоим на век, ибо ты в точности последовал Господу Богу моему>>;
\vs Jos 14:10 итак, вот, Господь сохранил меня в живых, как Он говорил; уже сорок пять лет \bibemph{прошло} от того времени, когда Господь сказал Моисею слово сие, и Израиль ходил по пустыне; теперь, вот, мне восемьдесят пять лет;
\vs Jos 14:11 но и ныне я столько же крепок, как и тогда, когда посылал меня Моисей: сколько тогда было у меня силы, столько и теперь есть для того, чтобы воевать и выходить и входить;
\vs Jos 14:12 итак дай мне сию гору, о которой говорил Господь в тот день; ибо ты слышал в тот день, что там [живут] сыны Енаковы, и города \bibemph{у них} большие и укрепленные; может быть, Господь [будет] со мною, и я изгоню их, как говорил Господь.
\vs Jos 14:13 Иисус благословил его, и дал в удел Халеву, сыну Иефонниину, [Кенезеянину,] Хеврон.
\vs Jos 14:14 Таким образом Хеврон остался уделом Халева, сына Иефонниина, Кенезеянина, до сего дня, за то, что он в точности последовал [повелению] Господа Бога Израилева.
\vs Jos 14:15 Имя Хеврону прежде было Кириаф-Арбы, как назывался между сынами Енака один человек великий. И земля успокоилась от войны.
\vs Jos 15:1 Жребий колену сынов Иудиных, по племенам их, выпал такой: в смежности с Идумеею была пустыня Син, к югу, при конце Фемана;
\vs Jos 15:2 южным пределом их был край моря Соленого от простирающегося к югу залива;
\vs Jos 15:3 на юге идет он к возвышенности Акраввимской, проходит Цин и, восходя с южной стороны к Кадес-Варне, проходит Хецрон и, восходя до Аддара, [идет на западной стороне Кадеса,] поворачивает к Каркае,
\vs Jos 15:4 потом проходит Ацмон, идет к потоку Египетскому, так что конец сего предела есть море. Сей будет южный ваш предел.
\vs Jos 15:5 Пределом же к востоку [все] море Соленое, до устья Иордана; а предел с северной стороны \bibemph{начинается} от залива моря, от устья Иордана;
\vs Jos 15:6 отсюда предел восходит к Беф-Хогле и проходит с северной стороны к Беф-Араве, и идет предел вверх до камня Богана, сына Рувимова;
\vs Jos 15:7 потом восходит предел к Давиру от долины Ахор и на севере поворачивает к Галгалу, который против возвышенности Адуммима, лежащего с южной стороны потока; отсюда предел проходит к водам Ен-Шемеша и оканчивается у Ен-Рогела;
\vs Jos 15:8 отсюда предел идет вверх к долине сына Енномова с южной стороны Иевуса, который \bibemph{есть} Иерусалим, и восходит предел на вершину горы, которая к западу против долины Енномовой, которая на краю долины Рефаимов к северу;
\vs Jos 15:9 от вершины горы предел поворачивает к источнику вод Нефтоах и идет к городам горы Ефрона, и поворачивает предел к Ваалу, который \bibemph{есть} Кириаф-Иарим;
\vs Jos 15:10 потом поворачивает предел от Ваала к морю [и идет] к горе Сеиру, и идет северною стороною горы Иеарим, которая \bibemph{есть} Кесалон, и, нисходя к Вефсамису, проходит чрез Фимну;
\vs Jos 15:11 отсюда предел идет северною стороною Екрона, и поворачивает предел к Шикарону, проходит чрез гору [земли] Ваал и доходит до Иавнеила, и оканчивается предел у моря. Западный предел составляет великое море.
\vs Jos 15:12 Вот предел сынов Иудиных с племенами их со всех сторон.
\rsbpar\vs Jos 15:13 И Халеву, сыну Иефонниину, [Иисус] дал часть среди сынов Иудиных, как повелел Господь Иисусу; [и дал ему Иисус] Кириаф-Арбы, отца Енакова, иначе Хеврон.
\vs Jos 15:14 И выгнал оттуда Халев [сын Иефонниин] трех сынов Енаковых: Шешая, Ахимана и Фалмая, детей Енаковых.
\vs Jos 15:15 Отсюда [Халев] пошел против жителей Давира (имя Давиру прежде \bibemph{было} Кириаф-Сефер).
\vs Jos 15:16 И сказал Халев: кто поразит Кириаф-Сефер и возьмет его, тому отдам Ахсу, дочь мою, в жену.
\vs Jos 15:17 И взял его Гофониил, [младший] сын Кеназа, брата Халевова, и отдал он в жену ему Ахсу, дочь свою.
\vs Jos 15:18 Когда надлежало ей идти, ее научили просить у отца ее поле, и она сошла с осла. Халев сказал ей: что тебе?
\vs Jos 15:19 Она сказала: дай мне благословение; ты дал мне землю полуденную, дай мне и источники вод. И дал он ей источники верхние и источники нижние.
\rsbpar\vs Jos 15:20 Вот удел колена сынов Иудиных, по племенам их:
\vs Jos 15:21 города с края колена сынов Иудиных в смежности с Идумеею на юге были: Кавцеил, Едер и Иагур,
\vs Jos 15:22 Кина, Димона, Адада,
\vs Jos 15:23 Кедес, Асор и Ифнан,
\vs Jos 15:24 Зиф, Телем и Валоф,
\vs Jos 15:25 Гацор-Хадафа, Кириаф, Хецрон, иначе Гацор,
\vs Jos 15:26 Амам, Шема и Молада,
\vs Jos 15:27 Хацар-Гадда, Хешмон и Веф-Палет,
\vs Jos 15:28 Хацар-Шуал, Вирсавия и Визиофея [и села их и предместья их,]
\vs Jos 15:29 Ваала, Иим и Ацем,
\vs Jos 15:30 Елфолад, Кесил и Хорма,
\vs Jos 15:31 Циклаг, Мадмана и Сансана,
\vs Jos 15:32 Леваоф, Шелихим, Аин и Риммон: всех двадцать девять городов с их селами.
\vs Jos 15:33 На низменных местах: Ештаол, Цора и Ашна,
\vs Jos 15:34 Заноах, Ен-Ганним, Таппуах и Гаенам,
\vs Jos 15:35 Иармуф, Одоллам, [Немра,] Сохо и Азека,
\vs Jos 15:36 Шаараим, Адифаим, Гедера или Гедерофаим: четырнадцать городов с их селами.
\vs Jos 15:37 Ценан, Хадаша, Мигдал-Гад,
\vs Jos 15:38 Дилеан, Мицфе и Иокфеил,
\vs Jos 15:39 Лахис, Воцкаф и Еглон,
\vs Jos 15:40 Хаббон, Лахмас и Хифлис,
\vs Jos 15:41 Гедероф, Беф-Дагон, Наема и Макед: шестнадцать городов с их селами.
\vs Jos 15:42 Ливна, Ефер и Ашан,
\vs Jos 15:43 Иффах, Ашна и Нецив,
\vs Jos 15:44 Кеила, Ахзив и Мареша [и Едом]: девять городов с их селами.
\vs Jos 15:45 Екрон с зависящими от него \bibemph{городами} и селами его,
\vs Jos 15:46 и от Екрона к морю все, что находится около Азота, с селами их,
\vs Jos 15:47 Азот, зависящие от него города и села его, Газа, зависящие от нее города и села ее, до самого потока Египетского и великого моря, которое \bibemph{есть} предел.
\vs Jos 15:48 На горах: Шамир, Иаттир и Сохо,
\vs Jos 15:49 Данна, Кириаф-Санна, иначе Давир,
\vs Jos 15:50 Анаф, Ештемо и Аним,
\vs Jos 15:51 Гошен, Холон и Гило: одиннадцать городов с их селами.
\vs Jos 15:52 Арав, Дума и Ешан,
\vs Jos 15:53 Ианум, Беф-Таппуах и Афека,
\vs Jos 15:54 Хумта, Кириаф-Арбы, иначе Хеврон, и Цигор: девять городов с их селами.
\vs Jos 15:55 Маон, Кармил, Зиф и Юта,
\vs Jos 15:56 Изреель, Иокдам и Заноах,
\vs Jos 15:57 Каин, Гива и Фимна: десять городов с их селами.
\vs Jos 15:58 Халхул, Беф-Цур и Гедор,
\vs Jos 15:59 Маараф, Беф-Аноф и Елтекон: шесть городов с их селами. [Феко, Ефрафа, иначе Вифлеем, Фагор, Етам, Кулон, Татами, Сорес, Карем, Галлим, Вефир и Манохо: одиннадцать городов с их селами.]
\vs Jos 15:60 Кириаф-Ваал, иначе Кириаф-Иарим, и Аравва: два города с их селами [и предместьями].
\vs Jos 15:61 В пустыне: Беф-Арава, Миддин и Секаха,
\vs Jos 15:62 Нившан, Ир-Мелах и Ен-Геди: шесть городов с их селами.
\vs Jos 15:63 Но Иевусеев, жителей Иерусалима, не могли изгнать сыны Иудины, и потому Иевусеи живут с сынами Иуды в Иерусалиме даже до сего дня.
\vs Jos 16:1 Потом выпал жребий сынам Иосифа: от Иордана подле Иерихона, у вод Иерихонских на восток, пустыня, простирающаяся от Иерихона к горе Вефильской;
\vs Jos 16:2 от Вефиля идет \bibemph{предел} к Лузу и переходит к пределу Архи до Атарофа,
\vs Jos 16:3 и спускается к морю, к пределу Иафлета, до предела нижнего Беф-Орона и до Газера, и оканчивается у моря.
\vs Jos 16:4 Это получили в удел сыны Иосифа: Манассия и Ефрем.
\rsbpar\vs Jos 16:5 Предел сынов Ефремовых по племенам их был сей: от востока пределом удела их был Атароф-Адар до Беф-Орона верхнего [и Газары];
\vs Jos 16:6 потом идет предел к морю северною стороною Михмефафа и поворачивает к восточной стороне Фаанаф-Силома и проходит его с восточной стороны Ианоха;
\vs Jos 16:7 от Ианоха, нисходя к Атарофу и Наарафу, примыкает к Иерихону и доходит до Иордана;
\vs Jos 16:8 от Таппуаха идет предел к морю, к потоку Кане, и оканчивается морем. Вот удел колена сынов Ефремовых, по племенам их.
\vs Jos 16:9 И города отделены сынам Ефремовым в уделе сынов Манассииных, все города с селами их.
\vs Jos 16:10 Но [Ефремляне] не изгнали Хананеев, живших в Газере; посему Хананеи жили среди Ефремлян до сего дня, платя им дань. [Наконец пришел фараон, царь Египетский, и взял город, и сжег его огнем, и Хананеев и Ферезеев и жителей Газера перебили, и отдал его фараон в приданое дочери своей.]
\vs Jos 17:1 И выпал жребий колену Манассии, так как он был первенец Иосифа. Махиру, первенцу Манассии, отцу Галаада, который \bibemph{был} храбр на войне, достался Галаад и Васан.
\vs Jos 17:2 Достались \bibemph{уделы} и прочим сынам Манассии, по племенам их, и сынам Авиезера, и сынам Хелека, и сынам Асриила, и сынам Шехема, и сынам Хефера, и сынам Шемиды. Вот дети Манассии, сына Иосифова, мужеского пола, по племенам их.
\vs Jos 17:3 У Салпаада же, сына Хеферова, сына Галаадова, сына Махирова, сына Манассиина, не было сыновей, а [только] дочери. Вот имена дочерей его: Махла, Ноа, Хогла, Милка и Фирца.
\vs Jos 17:4 Они пришли к священнику Елеазару и к Иисусу, сыну Навину, и к начальникам, и сказали: Господь повелел Моисею дать нам удел между братьями нашими. И дан им удел, по повелению Господню, между братьями отца их.
\vs Jos 17:5 И выпало Манассии десять участков, кроме земли Галаадской и Васанской, которая за Иорданом;
\vs Jos 17:6 ибо дочери [сынов] Манассии получили удел среди сыновей его, а земля Галаадская досталась прочим сыновьям Манассии.
\vs Jos 17:7 Предел [сынов] Манассии идет от Асира к Михмефафу, который против Сихема; отсюда предел идет направо к жителям Ен-Таппуаха.
\vs Jos 17:8 Земля Таппуах досталась Манассии, а \bibemph{город} Таппуах у предела Манассиина~--- сынам Ефремовым.
\vs Jos 17:9 Отсюда предел нисходит к потоку Кане, с южной стороны потока. Города сии \bibemph{принадлежат} Ефрему, \bibemph{хотя находятся} среди городов Манассии. Предел Манассии~--- на северной стороне потока и оканчивается морем.
\vs Jos 17:10 Что к югу, то Ефремово, а что к северу, то Манассиино; море же было пределом их; к Асиру примыкали они с северной стороны и к Иссахару с восточной.
\vs Jos 17:11 У Иссахара и Асира \bibemph{принадлежат} Манассии Беф-Сан и зависящие от него места, Ивлеам и зависящие от него места, жители Дора и зависящие от него места, жители Ен-Дора и зависящие от него места, жители Фаанаха и зависящие от него места, жители Мегиддона и зависящие от него места, и третья часть Нафефа [с селами его].
\vs Jos 17:12 Сыны Манассиины не могли выгнать \bibemph{жителей} городов сих, и Хананеи остались жить в земле сей.
\vs Jos 17:13 Когда же сыны Израилевы пришли в силу, тогда Хананеев сделали они данниками, но изгнать не изгнали их.
\rsbpar\vs Jos 17:14 Сыны Иосифа говорили Иисусу и сказали: почему ты дал мне в удел один жребий и один участок, тогда как я многолюден, потому что так благословил меня Господь?
\vs Jos 17:15 Иисус сказал им: если ты многолюден, то пойди в леса и там, в земле Ферезеев и Рефаимов, расчисти себе [место], если гора Ефремова для тебя тесна.
\vs Jos 17:16 Сыны Иосифа сказали: не останется за нами гора, потому что железные колесницы у всех Хананеев, живущих на долине, как у тех, которые в Беф-Сане и в зависящих от него местах, так и у тех, которые на долине Изреельской.
\vs Jos 17:17 Но Иисус сказал дому Иосифову, Ефрему и Манассии: ты многолюден и сила у тебя велика; не один жребий будет у тебя:
\vs Jos 17:18 и гора будет твоею, и лес сей; ты расчистишь его, и он будет твой до самого конца его; ибо ты изгонишь Хананеев, хотя у них колесницы железные, и хотя они сильны, [ты одолеешь их].
\vs Jos 18:1 Все общество сынов Израилевых собралось в Силом, и поставили там скинию собрания, ибо земля была покорена ими.
\vs Jos 18:2 Из сынов же Израилевых оставалось семь колен, которые еще не получили удела своего.
\vs Jos 18:3 И сказал Иисус сынам Израилевым: долго ли вы будете нерадеть о том, чтобы пойти \bibemph{и} взять в наследие землю, которую дал вам Господь Бог отцов ваших?
\vs Jos 18:4 дайте от себя по три человека из колена; я пошлю их, и они встав пройдут по земле и опишут ее, как надобно разделить им на уделы, и придут ко мне;
\vs Jos 18:5 пусть разделят ее на семь уделов; Иуда пусть остается в пределе своем на юге, а дом Иосифов пусть остается в пределе своем на севере;
\vs Jos 18:6 а вы распишите землю на семь уделов и представьте мне сюда: я брошу вам жребий здесь пред лицем Господа Бога нашего;
\vs Jos 18:7 а левитам нет части между вами, ибо священство Господне есть удел их; Гад же, Рувим и половина колена Манассиина получили удел свой за Иорданом к востоку, который дал им Моисей, раб Господень.
\vs Jos 18:8 Эти люди встали и пошли. Иисус же пошедшим описывать землю дал такое приказание: пойдите, обойдите землю, опишите ее и возвратитесь ко мне; а я здесь брошу вам жребий пред лицем Господним, в Силоме.
\vs Jos 18:9 Они пошли, прошли по земле, [осмотрели ее] и описали ее, по городам ее, на семь уделов, в книге, и пришли к Иисусу в стан, в Силом.
\vs Jos 18:10 Иисус бросил им жребий в Силоме пред Господом, и разделил там Иисус землю сынам Израилевым по участкам их.
\rsbpar\vs Jos 18:11 [Первый] жребий вышел колену сынов Вениаминовых, по племенам их. Предел их по жребию шел между сынами Иуды и между сынами Иосифа;
\vs Jos 18:12 предел их на северной стороне начинается у Иордана, и проходит предел сей подле Иерихона с севера, и восходит на гору к западу, и оканчивается в пустыне Бефавен;
\vs Jos 18:13 оттуда предел идет к Лузу, к южной стороне Луза, иначе Вефиля, и нисходит предел к Атароф-Адару, к горе, которая на южной стороне Беф-Орона нижнего;
\vs Jos 18:14 потом предел поворачивает и склоняется к морской стороне на юг от горы, которая на юге пред Беф-Ороном, и оканчивается у Кириаф-Ваала, иначе Кириаф-Иарима, города сынов Иудиных. Это западная сторона.
\vs Jos 18:15 Южною же стороною от Кириаф-Иарима идет предел к морю и доходит до источника вод Нефтоаха;
\vs Jos 18:16 потом предел нисходит к концу горы, которая пред долиною сына Енномова, на долине Рефаимов, к северу, и нисходит долиною Еннома к южной стороне Иевуса, и идет к Ен-Рогелу;
\vs Jos 18:17 потом поворачивает от севера и идет к Ен-Шемешу, и идет к Гелилофу, который против возвышенности Адуммима, и нисходит к камню Богана, сына Рувимова;
\vs Jos 18:18 потом проходит близ равнины к северу и нисходит на равнину;
\vs Jos 18:19 отсюда проходит предел подле Беф-Хоглы к северу, и оканчивается предел у северного залива моря Соленого, у южного конца Иордана. Вот предел южный. С восточной же стороны пределом служит Иордан.
\vs Jos 18:20 Вот удел сынов Вениаминовых, с пределами его со всех сторон, по племенам их.
\vs Jos 18:21 Города колену сынов Вениаминовых, по племенам их, принадлежали сии: Иерихон, Беф-Хогла и Емек-Кециц,
\vs Jos 18:22 Беф-Арава, Цемараим и Вефиль,
\vs Jos 18:23 Аввим, Фара и Офра,
\vs Jos 18:24 Кефар-Аммонай, Афни и Гева: двенадцать городов с их селами.
\vs Jos 18:25 Гаваон, Рама и Бероф,
\vs Jos 18:26 Мицфе, Кефира и Моца,
\vs Jos 18:27 Рекем, Ирфеил и Фарала,
\vs Jos 18:28 Цела, Елеф и Иевус, иначе Иерусалим, Гивеаф и Кириаф: четырнадцать городов с их селами. Вот удел сынов Вениаминовых, по племенам их.
\vs Jos 19:1 Второй жребий вышел Симеону, колену сынов Симеоновых, по племенам их; и был удел их среди удела сынов Иудиных.
\vs Jos 19:2 В уделе их были: Вирсавия или Шева, Молада,
\vs Jos 19:3 Хацар-Шуал, Вала и Ацем,
\vs Jos 19:4 Елтолад, Вефул и Хорма,
\vs Jos 19:5 Циклаг, Беф-Маркавоф и Хацар-Суса,
\vs Jos 19:6 Беф-Леваоф и Шарухен: тринадцать городов с их селами.
\vs Jos 19:7 Аин, Риммон, Ефер и Ашан: четыре города с селами их,
\vs Jos 19:8 и все села, которые находились вокруг городов сих даже до Ваалаф-Беера, или южной Рамы. Вот удел колена сынов Симеоновых, по племенам их.
\vs Jos 19:9 От участка сынов Иудиных \bibemph{выделен} удел [колену] сынов Симеоновых. Так как участок сынов Иудиных был слишком велик для них, то сыны Симеоновы и получили удел среди их удела.
\vs Jos 19:10 Третий жребий выпал сынам Завулоновым по племенам их, и простирался предел удела их до Сарида;
\vs Jos 19:11 предел их восходит к морю и Марале и примыкает к Дабешефу и примыкает к потоку, который пред Иокнеамом;
\vs Jos 19:12 от Сарида идет назад к восточной стороне, к востоку солнца, до предела Кислоф-Фавора; отсюда идет к Даврафу и восходит к Иафие;
\vs Jos 19:13 отсюда проходит к востоку в Геф-Хефер, в Итту-Кацин, и идет к Риммону, Мифоару и Нее;
\vs Jos 19:14 и поворачивает предел от севера к Ханнафону и оканчивается долиною Ифтах-Ел;
\vs Jos 19:15 далее: Каттаф, Нагалал, Шимрон, Идеала и Вифлеем: двенадцать городов с их селами.
\vs Jos 19:16 Вот удел сынов Завулоновых, по их племенам; вот города и села их.
\rsbpar\vs Jos 19:17 Четвертый жребий вышел Иссахару, сынам Иссахара, по племенам их;
\vs Jos 19:18 пределом их был: Изреель, Кесуллоф и Сунем,
\vs Jos 19:19 Хафараим, Шион и Анахараф,
\vs Jos 19:20 Раввиф, Кишион и Авец,
\vs Jos 19:21 Ремеф, Ен-Ганним, Ен-Хадда и Беф-Пацец;
\vs Jos 19:22 и примыкает предел к Фавору и Шагациме и Вефсамису, и оканчивается предел их у Иордана: шестнадцать городов с селами их.
\vs Jos 19:23 Вот удел колена сынов Иссахаровых по племенам их; вот города и села их.
\rsbpar\vs Jos 19:24 Пятый жребий вышел колену сынов Асировых, по племенам их;
\vs Jos 19:25 пределом их были: Хелкаф, Хали, Ветен и Ахсаф,
\vs Jos 19:26 Аламелех, Амад и Мишал; и примыкает \bibemph{предел} к Кармилу с западной стороны и к Шихор-Ливнафу;
\vs Jos 19:27 потом идет назад к востоку солнца в Беф-Дагон, и примыкает к Завулону и к долине Ифтах-Ел с севера, [и входит в пределы Асафы] в Беф-Емек и Неиел, и идет у Кавула, с левой стороны;
\vs Jos 19:28 далее: Еврон, Рехов, Хаммон и Кана, до Сидона великого;
\vs Jos 19:29 потом предел возвращается к Раме до укрепленного города Тира, и поворачивает предел к Хоссе, и оканчивается у моря, в местечке Ахзиве;
\vs Jos 19:30 далее: Умма, Афек и Рехов: двадцать два города с селами их.
\vs Jos 19:31 Вот удел колена сынов Асировых, по племенам их; вот города и села их.
\rsbpar\vs Jos 19:32 Шестой жребий вышел сынам Неффалима, сынам Неффалима по племенам их;
\vs Jos 19:33 предел их шел от Хелефа [и] от дубравы, \bibemph{что} в Цананниме, к Адами-Некеву и Иавнеилу, до Лаккума, и оканчивался у Иордана;
\vs Jos 19:34 отсюда возвращается предел на запад к Азноф-Фавору и идет оттуда к Хуккоку, и примыкает к Завулону с юга, и к Асиру примыкает с запада, и к Иуде у Иордана, от востока солнца.
\vs Jos 19:35 Города укрепленные: Циддим, Цер, Хамаф, Раккаф и Хиннереф,
\vs Jos 19:36 Адама, Рама и Асор,
\vs Jos 19:37 Кедес, Едрея и Ен-Гацор,
\vs Jos 19:38 Иреон, Мигдал-Ел, Хорем, Беф-Анаф и Вефсамис: девятнадцать городов с их селами.
\vs Jos 19:39 Вот удел колена сынов Неффалимовых по племенам их; вот города и села их.
\rsbpar\vs Jos 19:40 Колену сынов Дановых, по племенам их, вышел жребий седьмой;
\vs Jos 19:41 пределом удела их были: Цора, Ештаол и Ир-Шемеш,
\vs Jos 19:42 Шаалаввин, Аиалон и Ифла,
\vs Jos 19:43 Елон, Фимнафа и Екрон,
\vs Jos 19:44 Елтеке, Гиввефон и Ваалаф,
\vs Jos 19:45 Игуд, Бене-Верак и Гаф-Риммон,
\vs Jos 19:46 Ме-Иаркон и Ракон с пределом близ Иоппии. И вышел предел сынов Дановых мал для них.
\vs Jos 19:47 И сыны Дановы пошли войною на Ласем и взяли его, и поразили его мечом, и получили его в наследие, и поселились в нем, и назвали Ласем Даном по имени Дана, отца своего. [Аморреи оставались жить в Еломе и Саламине, но рука Ефремова одолела их, и сделались они данниками ему.]
\vs Jos 19:48 Вот удел колена сынов Дановых, по племенам их; вот города и села их. [Сыны Дановы не истребили Аморреев, которые стеснили их на горе, и не давали им Аморреи выходить на долину и отняли у них предел их участка.]
\rsbpar\vs Jos 19:49 Когда окончили разделение земли, по пределам ее, тогда сыны Израилевы дали среди себя удел Иисусу, сыну Навину:
\vs Jos 19:50 по повелению Господню дали ему город Фамнаф-Сараи, которого он просил, на горе Ефремовой; и построил он город и жил в нем.
\vs Jos 19:51 Вот уделы, которые Елеазар священник, Иисус, сын Навин, и начальники поколений разделили коленам сынов Израилевых, по жребию, в Силоме, пред лицем Господним, у входа скинии собрания. И кончили разделение земли.
\vs Jos 20:1 И сказал Господь Иисусу, говоря:
\vs Jos 20:2 скажи сынам Израилевым: сделайте у себя города убежища, о которых Я говорил вам чрез Моисея,
\vs Jos 20:3 чтобы мог убегать туда убийца, убивший человека по ошибке, без умысла; пусть [города сии] будут у вас убежищем [чтобы не умер убивший] от мстящего за кровь, [доколе не предстанет пред общество на суд].
\vs Jos 20:4 И кто убежит в один из городов сих, пусть станет у ворот города и расскажет вслух старейшин города сего дело свое; и они примут его к себе в город и дадут ему место, чтоб он жил у них;
\vs Jos 20:5 и когда погонится за ним мстящий за кровь, то они не должны выдавать в руки его убийцу, потому что он без умысла убил ближнего своего, не имел к нему ненависти ни вчера, ни третьего дня;
\vs Jos 20:6 пусть он живет в этом городе, доколе не предстанет пред общество на суд, доколе не умрет великий священник, который будет в те дни. А потом пусть возвратится убийца и пойдет в город свой и в дом свой, в город, из которого он убежал.
\vs Jos 20:7 И отделили Кедес в Галилее на горе Неффалимовой, Сихем на горе Ефремовой, и Кириаф-Арбы, иначе Хеврон, на горе Иудиной;
\vs Jos 20:8 за Иорданом, против Иерихона к востоку, отделили: Бецер в пустыне, на равнине, от колена Рувимова, и Рамоф в Галааде от колена Гадова, и Голан в Васане от колена Манассиина;
\vs Jos 20:9 сии города назначены для всех сынов Израилевых и для пришельцев, живущих у них, дабы убегал туда всякий, убивший человека по ошибке, дабы не умер он от руки мстящего за кровь, доколе не предстанет пред общество [на суд].
\vs Jos 21:1 Начальники поколений левитских пришли к Елеазару священнику и к Иисусу, сыну Навину, и к начальникам поколений сынов Израилевых,
\vs Jos 21:2 и говорили им в Силоме, в земле Ханаанской, и сказали: Господь повелел чрез Моисея дать нам города для жительства и предместья их для скота нашего.
\vs Jos 21:3 И дали сыны Израилевы левитам из уделов своих, по повелению Господню, сии города с предместьями их.
\vs Jos 21:4 Вышел жребий племенам Каафовым; и досталось по жребию сынам Аарона священника, левитам, от колена Иудина, и от колена Симеонова, и от колена Вениаминова, тринадцать городов;
\vs Jos 21:5 а прочим сынам Каафа от племен колен Ефремова, и от колена Данова, и от половины колена Манассиина, по жребию, \bibemph{досталось} десять городов;
\vs Jos 21:6 сынам Гирсоновым~--- от племен колена Иссахарова, и от колена Асирова, и от колена Неффалимова, и от половины колена Манассиина в Васане, по жребию, \bibemph{досталось} тринадцать городов;
\vs Jos 21:7 сынам Мерариным, по их племенам, от колена Рувимова, от колена Гадова и от колена Завулонова~--- двенадцать городов.
\vs Jos 21:8 И отдали сыны Израилевы левитам сии города с предместьями их, как повелел Господь чрез Моисея, по жребию.
\vs Jos 21:9 От колена сынов Иудиных, и от колена сынов Симеоновых, [и от колена сынов Вениаминовых] дали города, которые \bibemph{здесь} названы по имени:
\vs Jos 21:10 сынам Аарона, из племен Каафовых, из сынов Левия [так как жребий их был первый],
\vs Jos 21:11 дали Кириаф-Арбы, отца Енакова, иначе Хеврон, на горе Иудиной, и предместья его вокруг его;
\vs Jos 21:12 а поле сего города и сёла его отдали в собственность Халеву, сыну Иефонниину.
\vs Jos 21:13 Итак сынам Аарона священника дали город убежища для убийцы~--- Хеврон и предместья его, Ливну и предместья ее,
\vs Jos 21:14 Иаттир и предместья его, Ештемо и предместья его,
\vs Jos 21:15 Холон и предместья его, Давир и предместья его,
\vs Jos 21:16 Аин и предместья его, Ютту и предместья ее, Беф-Шемеш и предместья его: девять городов от двух колен сих;
\vs Jos 21:17 а от колена Вениаминова: Гаваон и предместья его, Геву и предместья ее,
\vs Jos 21:18 Анафоф и предместья его, Алмон и предместья его: четыре города.
\vs Jos 21:19 Всех городов сынам Аароновым, священникам, \bibemph{досталось} тринадцать городов с предместьями их.
\vs Jos 21:20 И племенам сынов Каафовых, левитов, прочим из сынов Каафовых, по жребию их, достались города от колена Ефремова;
\vs Jos 21:21 дали им город убежища для убийцы~--- Сихем и предместья его, на горе Ефремовой, Гезер и предместья его,
\vs Jos 21:22 Кивцаим и предместья его, Беф-Орон и предместья его: четыре города;
\vs Jos 21:23 от колена Данова: Елфеке и предместья его, Гиввефон и предместья его,
\vs Jos 21:24 Аиалон и предместья его, Гаф-Риммон и предместья его: четыре города;
\vs Jos 21:25 от половины колена Манассиина: Фаанах и предместья его, Гаф-Риммон и предместья его: два города.
\vs Jos 21:26 Всех городов с предместьями их прочим племенам сынов Каафовых \bibemph{досталось} десять.
\vs Jos 21:27 А сынам Гирсоновым, из племен левитских \bibemph{дали}: от половины колена Манассиина город убежища для убийцы~--- Голан в Васане и предместья его, и Беештеру и предместья ее: два города;
\vs Jos 21:28 от колена Иссахарова: Кишион и предместья его, Давраф и предместья его,
\vs Jos 21:29 Иармуф и предместья его, Ен-Ганним и предместья его: четыре города;
\vs Jos 21:30 от колена Асирова: Мишал и предместья его, Авдон и предместья его,
\vs Jos 21:31 Хелкаф и предместья его, Рехов и предместья его: четыре города;
\vs Jos 21:32 от колена Неффалимова город убежища для убийцы~--- Кедес в Галилее и предместья его, Хамоф-Дор и предместья его, Карфан и предместья его: три города.
\vs Jos 21:33 Всех городов сынам Гирсоновым, по племенам их, \bibemph{досталось} тринадцать городов с предместьями их.
\vs Jos 21:34 Племенам сынов Мерариных, остальным левитам, \bibemph{дали}: от колена Завулонова Иокнеам и предместья его, Карфу и предместья ее,
\vs Jos 21:35 Димну и предместья ее, Нагалал и предместья его: четыре города;
\vs Jos 21:36 [по ту сторону Иордана против Иерихона] от колена Рувимова [дан город убежища для убийцы] Бецер [в пустыне Мисор] и предместья его, Иааца и предместья ее,
\vs Jos 21:37 Кедемоф и предместья его, Мефааф и предместья его: четыре города;
\vs Jos 21:38 от колена Гадова: города убежища для убийцы~--- Рамоф в Галааде и предместья его, Маханаим и предместья его,
\vs Jos 21:39 Есевон и предместья его, Иазер и предместья его: всех городов четыре.
\vs Jos 21:40 Всех городов сынам Мерариным по племенам их, остальным племенам левитским, по жребию досталось двенадцать городов.
\rsbpar\vs Jos 21:41 Всех городов левитских среди владения сынов Израилевых \bibemph{было} сорок восемь городов с предместьями их.
\vs Jos 21:42 При городах сих были при каждом городе предместья вокруг него: так было при всех городах сих. [Когда Иисус кончил разделение земли по пределам ее, тогда сыны Израилевы дали часть Иисусу по повелению Господню, дали ему город, которого он просил, Фимнаф-Сару дали ему на горе Ефремовой, и построил Иисус город, которого просил, и жил в нем. И взял Иисус каменные ножи, которыми обрезал сынов Израилевых, родившихся на пути в пустыне, ибо они не были обрезаны в пустыне; и положил их в Фимнаф-Саре.]
\vs Jos 21:43 Таким образом отдал Господь Израилю всю землю, которую дать клялся отцам их, и они получили ее в наследие и поселились на ней.
\vs Jos 21:44 И дал им Господь покой со всех сторон, как клялся отцам их, и никто из всех врагов их не устоял против них; всех врагов их предал Господь в руки их.
\vs Jos 21:45 Не осталось неисполнившимся ни одно слово из всех добрых слов, которые Господь говорил дому Израилеву; все сбылось.
\vs Jos 22:1 Тогда Иисус призвал \bibemph{колено} Рувимово, Гадово и половину колена Манассиина и сказал им:
\vs Jos 22:2 вы исполнили всё, что повелел вам Моисей, раб Господень, и слушались слов моих во всем, что я приказывал вам;
\vs Jos 22:3 вы не оставляли братьев своих в продолжение многих дней до сего дня и исполнили, что надлежало исполнить по повелению Господа, Бога вашего:
\vs Jos 22:4 ныне Господь, Бог ваш, успокоил братьев ваших, как говорил им; итак возвратитесь и пойдите в шатры ваши, в землю вашего владения, которую дал вам Моисей, раб Господень, за Иорданом;
\vs Jos 22:5 только старайтесь тщательно исполнять заповеди и закон, который завещал вам Моисей, раб Господень: любить Господа Бога вашего, ходить всеми путями Его, хранить заповеди Его, прилепляться к Нему и служить Ему всем сердцем вашим и всею душею вашею.
\vs Jos 22:6 Потом Иисус благословил их и отпустил их, и они разошлись по шатрам своим.
\vs Jos 22:7 Одной половине колена Манассиина дал Моисей удел в Васане, а другой половине его дал Иисус \bibemph{удел} с братьями его по эту сторону Иордана к западу. И когда отпускал их Иисус в шатры их и благословил их,
\vs Jos 22:8 то сказал им: с великим богатством возвращаетесь вы в шатры ваши, с великим множеством скота, с серебром, с золотом, с медью и с железом, и с великим множеством одежд; разделите же добычу, \bibemph{взятую} у врагов ваших, с братьями своими.
\vs Jos 22:9 И возвратились, и пошли сыны Рувимовы и сыны Гадовы и половина колена Манассиина от сынов Израилевых из Силома, который в земле Ханаанской, чтоб идти в землю Галаад, в землю своего владения, которую получили во владение по повелению Господню, \bibemph{данному} чрез Моисея.
\rsbpar\vs Jos 22:10 Придя в окрестности Иордана, что в земле Ханаанской, сыны Рувимовы и сыны Гадовы и половина колена Манассиина соорудили там подле Иордана жертвенник, жертвенник большой по виду.
\vs Jos 22:11 И услышали сыны Израилевы, что говорят: вот, сыны Рувимовы и сыны Гадовы и половина колена Манассиина соорудили жертвенник на земле Ханаанской, в окрестностях Иордана, напротив сынов Израилевых.
\vs Jos 22:12 Когда услышали \bibemph{сие} сыны Израилевы, то собралось все общество сынов Израилевых в Силом, чтоб идти против них войною.
\vs Jos 22:13 Впрочем сыны Израилевы \bibemph{прежде} послали к сынам Рувимовым и к сынам Гадовым и к половине колена Манассиина в землю Галаадскую Финееса, сына Елеазара, священника,
\vs Jos 22:14 и с ним десять начальников, по начальнику поколения от всех колен Израилевых; каждый из них был начальником поколения в тысячах Израилевых.
\vs Jos 22:15 И пришли они к сынам Рувимовым и к сынам Гадовым и к половине колена Манассиина в землю Галаад и говорили им и сказали:
\vs Jos 22:16 так говорит все общество Господне: что это за преступление сделали вы пред [Господом] Богом Израилевым, отступив ныне от Господа [Бога Израилева], соорудив себе жертвенник и восстав ныне против Господа?
\vs Jos 22:17 Разве мало для нас беззакония Фегорова, от которого мы не очистились до сего дня и \bibemph{за которое} поражено было общество Господне?
\vs Jos 22:18 А вы отступаете сегодня от Господа! Сегодня вы восстаете против Господа, а завтра прогневается [Господь] на все общество Израилево;
\vs Jos 22:19 если же земля вашего владения кажется вам нечистою, то перейдите в землю владения Господня, в которой находится скиния Господня, возьмите удел среди нас, но не восставайте против Господа и против нас не восставайте, сооружая себе жертвенник, кроме жертвенника Господа, Бога нашего;
\vs Jos 22:20 не \bibemph{один} ли Ахан, сын Зары, сделал преступление, \bibemph{взяв} из заклятого, а гнев был на все общество Израилево? не один он умер за свое беззаконие.
\vs Jos 22:21 Сыны Рувимовы и сыны Гадовы и половина колена Манассиина в ответ \bibemph{на сие} говорили начальникам тысяч Израилевых:
\vs Jos 22:22 Бог богов Господь, Бог богов Господь, Он знает, и Израиль да знает! Если мы восстаем и отступаем от Господа, то да не пощадит нас \bibemph{Господь} в сей день!
\vs Jos 22:23 Если мы соорудили жертвенник для того, чтоб отступить от Господа [Бога нашего], и для того, чтобы приносить на нем всесожжение и приношение хлебное и чтобы совершать на нем жертвы мирные, то да взыщет Сам Господь!
\vs Jos 22:24 Но мы сделали сие по опасению того, чтобы в последующее время не сказали ваши сыны нашим сынам: <<что вам до Господа Бога Израилева!
\vs Jos 22:25 Господь поставил пределом между нами и вами, сыны Рувимовы и сыны Гадовы, Иордан: нет вам части в Господе>>. Таким образом ваши сыны не допустили бы наших сынов чтить Господа.
\vs Jos 22:26 Поэтому мы сказали: соорудим себе жертвенник не для всесожжения и не для жертв,
\vs Jos 22:27 но чтобы он между нами и вами, между последующими родами нашими, был свидетелем, что мы можем служить Господу всесожжениями нашими и жертвами нашими и благодарениями нашими, и чтобы в последующее время не сказали ваши сыны сынам нашим: <<нет вам части в Господе>>.
\vs Jos 22:28 Мы говорили: если скажут так нам и родам нашим в последующее время, то мы скажем: видите подобие жертвенника Господа, которое сделали отцы наши не для всесожжения и не для жертвы, но чтобы это было свидетелем между нами и вами [и между сынами нашими].
\vs Jos 22:29 Да не будет этого, чтобы восстать нам против Господа и отступить ныне от Господа, и соорудить жертвенник для всесожжения и для приношения хлебного и для жертв, кроме жертвенника Господа Бога нашего, который пред скиниею Его.
\vs Jos 22:30 Финеес священник, [все] начальники общества и головы тысяч Израилевых, которые были с ним, услышав слова, которые говорили сыны Рувимовы и сыны Гадовы и сыны Манассиины, одобрили их.
\vs Jos 22:31 И сказал Финеес, сын Елеазара, священник, сынам Рувимовым и сынам Гадовым и сынам Манассииным: сегодня мы узнали, что Господь среди нас, что вы не сделали пред Господом преступления сего; теперь вы избавили сынов Израиля от руки Господней.
\vs Jos 22:32 И возвратился Финеес, сын Елеазара, священник, и начальники от сынов Рувимовых и от сынов Гадовых [и от половины колена Манассиина] в землю Ханаанскую к сынам Израилевым и принесли им ответ.
\vs Jos 22:33 И сыны Израилевы одобрили это, и благословили сыны Израилевы Бога и отложили идти против них войною, чтобы разорить землю, на которой жили сыны Рувимовы и сыны Гадовы [и половина колена Манассиина].
\vs Jos 22:34 И назвали сыны Рувимовы и сыны Гадовы [и половина колена Манассиина] жертвенник: \bibemph{Ед}\fns{Свидетель.}, потому что, \bibemph{сказали они}, он свидетель между нами, что Господь есть Бог наш.
\vs Jos 23:1 Спустя много времени после того, как Господь [Бог] успокоил Израиля от всех врагов его со всех сторон, Иисус состарился, вошел в \bibemph{преклонные} лета.
\vs Jos 23:2 И призвал Иисус всех [сынов] Израилевых, старейшин их, начальников их, судей их и надзирателей их, и сказал им: я состарился, вошел в \bibemph{преклонные} лета.
\vs Jos 23:3 Вы видели всё, что сделал Господь Бог ваш пред лицем вашим со всеми сими народами, ибо Господь Бог ваш Сам сражался за вас.
\vs Jos 23:4 Вот, я разделил вам по жребию оставшиеся народы сии в удел коленам вашим, все народы, которые я истребил, от Иордана до великого моря, на запад солнца.
\vs Jos 23:5 Господь Бог ваш Сам прогонит их от вас [доколе не погибнут; и пошлет на них диких зверей, доколе не истребит их и царей их от лица вашего], и истребит их пред вами, дабы вы получили в наследие землю их, как говорил вам Господь Бог ваш.
\vs Jos 23:6 Посему во всей точности старайтесь хранить и исполнять все написанное в книге закона Моисеева, не уклоняясь от него ни направо, ни налево.
\vs Jos 23:7 Не сообщайтесь с сими народами, которые остались между вами, не воспоминайте имени богов их, не клянитесь [ими] и не служите им и не поклоняйтесь им,
\vs Jos 23:8 но прилепитесь к Господу Богу вашему, как вы делали до сего дня.
\vs Jos 23:9 Господь прогнал от вас народы великие и сильные, и пред вами никто не устоял до сего дня;
\vs Jos 23:10 один из вас прогоняет тысячу, ибо Господь Бог ваш Сам сражается за вас, как говорил вам.
\vs Jos 23:11 Посему всячески старайтесь любить Господа Бога вашего.
\vs Jos 23:12 Если же вы отвратитесь и пристанете к оставшимся из народов сих, которые остались между вами, и вступите в родство с ними и будете ходить к ним и они к вам,
\vs Jos 23:13 то знайте, что Господь Бог ваш не будет уже прогонять от вас народы сии, но они будут для вас петлею и сетью, бичом для ребр ваших и терном для глаз ваших, доколе не будете истреблены с сей доброй земли, которую дал вам Господь Бог ваш.
\vs Jos 23:14 Вот, я ныне отхожу в путь всей земли. А вы знаете всем сердцем вашим и всею душею вашею, что не осталось тщетным ни одно слово из всех добрых слов, которые говорил о вас Господь Бог ваш; все сбылось для вас, ни одно слово не осталось неисполнившимся.
\vs Jos 23:15 Но как сбылось над вами всякое доброе слово, которое говорил вам Господь Бог ваш, так Господь исполнит над вами всякое злое слово, доколе не истребит вас с этой доброй земли, которую дал вам Господь Бог ваш.
\vs Jos 23:16 Если вы преступите завет Господа Бога вашего, который Он поставил с вами, и пойдете и будете служить другим богам и поклоняться им, то возгорится на вас гнев Господень, и скоро сгибнете с этой доброй земли, которую дал вам [Господь].
\vs Jos 24:1 И собрал Иисус все колена Израилевы в Сихем и призвал старейшин Израиля и начальников его, и судей его и надзирателей его, и предстали пред [Господа] Бога.
\vs Jos 24:2 И сказал Иисус всему народу: так говорит Господь Бог Израилев: <<за рекою жили отцы ваши издревле, Фарра, отец Авраама и отец Нахора, и служили иным богам.
\vs Jos 24:3 Но Я взял отца вашего Авраама из-за реки и водил его по всей земле Ханаанской, и размножил семя его и дал ему Исаака.
\vs Jos 24:4 Исааку дал Иакова и Исава. Исаву дал Я гору Сеир в наследие; Иаков же и сыны его перешли в Египет [и сделались там народом великим, сильным и многочисленным, и стали притеснять их Египтяне].
\vs Jos 24:5 И послал Я Моисея и Аарона и поразил Египет язвами, которые делал Я среди его, и потом вывел вас.
\vs Jos 24:6 Я вывел отцов ваших из Египта, и вы пришли к [Чермному] морю. Тогда Египтяне гнались за отцами вашими с колесницами и всадниками до Чермного моря;
\vs Jos 24:7 но они возопили к Господу, и Он положил [облако и] тьму между вами и Египтянами и навел на них море, которое их и покрыло. Глаза ваши видели, что Я сделал в Египте. \bibemph{Потом} много времени пробыли вы в пустыне.
\vs Jos 24:8 И привел Я вас к земле Аморреев, живших за Иорданом; они сразились с вами, но Я предал их в руки ваши, и вы получили в наследие землю их, и Я истребил их пред вами.
\vs Jos 24:9 Восстал Валак, сын Сепфоров, царь Моавитский, и пошел войною на Израиля, и послал и призвал Валаама, сына Веорова, чтоб он проклял вас;
\vs Jos 24:10 но Я не хотел послушать Валаама,~--- и он благословил вас, и Я избавил вас из рук его.
\vs Jos 24:11 Вы перешли Иордан и пришли к Иерихону. И стали воевать с вами жители Иерихона, Аморреи, и Ферезеи, и Хананеи, и Хеттеи, и Гергесеи, и Евеи, и Иевусеи, но Я предал их в руки ваши.
\vs Jos 24:12 Я послал пред вами шершней, которые прогнали их от вас, двух царей Аморрейских; не мечом твоим и не луком твоим \bibemph{сделано это}.
\vs Jos 24:13 И дал Я вам землю, над которою ты не трудился, и города, которых вы не строили, и вы живете в них; из виноградных и масличных садов, которых вы не насаждали, вы едите \bibemph{плоды}>>.
\vs Jos 24:14 Итак бойтесь Господа и служите Ему в чистоте и искренности; отвергните богов, которым служили отцы ваши за рекою и в Египте, а служ\acc{и}те Господу.
\vs Jos 24:15 Если же не угодно вам служить Господу, то изберите себе ныне, кому служить, богам ли, которым служили отцы ваши, бывшие за рекою, или богам Аморреев, в земле которых живете; а я и дом мой будем служить Господу, [ибо Он свят].
\vs Jos 24:16 И отвечал народ и сказал: нет, не будет того, чтобы мы оставили Господа и стали служить другим богам!
\vs Jos 24:17 Ибо Господь~--- Бог наш, Он вывел нас и отцов наших из земли Египетской, из дома рабства, и делал пред глазами нашими великие знамения и хранил нас на всем пути, по которому мы шли, и среди всех народов, чрез которые мы проходили.
\vs Jos 24:18 Господь прогнал от нас все народы и Аморреев, живших в сей земле. Посему и мы будем служить Господу, ибо Он~--- Бог наш.
\vs Jos 24:19 Иисус сказал народу: не возможете служить Господу [Богу], ибо Он Бог святый, Бог ревнитель, не потерпит беззакония вашего и грехов ваших.
\vs Jos 24:20 Если вы оставите Господа и будете служить чужим богам, то Он наведет на вас зло и истребит вас, после того как благотворил вам.
\vs Jos 24:21 И сказал народ Иисусу: нет, мы Господу будем служить.
\vs Jos 24:22 Иисус сказал народу: вы свидетели о себе, что вы избрали себе Господа~--- служить Ему? Они отвечали: свидетели.
\vs Jos 24:23 Итак отвергните чужих богов, которые у вас, и обратите сердце свое к Господу Богу Израилеву.
\vs Jos 24:24 Народ сказал Иисусу: Господу Богу нашему будем служить и гласа Его будем слушать.
\vs Jos 24:25 И заключил Иисус с народом завет в тот день и дал ему постановления и закон в Сихеме [пред скиниею Господа Бога Израилева].
\vs Jos 24:26 И вписал Иисус слова сии в книгу закона Божия, и взял большой камень и положил его там под дубом, который подле святилища Господня.
\vs Jos 24:27 И сказал Иисус всему народу: вот, камень сей будет нам свидетелем, ибо он слышал все слова Господа, которые Он говорил с нами [сегодня]; он да будет свидетелем против вас [в последующие дни], чтобы вы не солгали пред [Господом] Богом вашим.
\vs Jos 24:28 И отпустил Иисус народ, каждого в свой удел.
\rsbpar\vs Jos 24:29 После сего умер Иисус, сын Навин, раб Господень, будучи ста десяти лет.
\vs Jos 24:30 И похоронили его в пределе его удела в Фамнаф-Сараи, что на горе Ефремовой, на север от горы Гааша. [И положили там с ним во гробе, в котором похоронили его, каменные ножи, которыми Иисус обрезал сынов Израилевых в Галгале, когда вывел их из Египта, как повелел Господь; и они там даже до сего дня.]
\vs Jos 24:31 И служил Израиль Господу во все дни Иисуса и во все дни старейшин, которых жизнь продлилась после Иисуса и которые видели все дела Господа, какие Он сделал Израилю.
\vs Jos 24:32 И кости Иосифа, которые вынесли сыны Израилевы из Египта, схоронили в Сихеме, в участке поля, которое купил Иаков у сынов Еммора, отца Сихемова, за сто монет и которое досталось в удел сынам Иосифовым.
\vs Jos 24:33 [После сего] умер и Елеазар, сын Аарона [первосвященник], и похоронили его на холме Финееса, сына его, который дан ему на горе Ефремовой.
\vs Jos 24:34 [В тот день сыны Израилевы, взяв ковчег Божий, носили с собою, и Финеес был священником вместо Елеазара, отца своего, доколе не умер и не был погребен в \bibemph{городе} своем Гаваафе.
\vs Jos 24:35 И сыны Израилевы пошли каждый в свое место и в свой город.
\vs Jos 24:36 И стали сыны Израилевы служить Астарте и Астарофу и богам окрестных народов; и предал их Господь в руки Еглона, царя Моавитского, и он владел ими восемнадцать лет.]

\bibbookdescr{Jdg}{
  inline={\LARGE Книга\\\Huge Судей Израилевых},
  toc={Книга Судей},
  bookmark={Книга Судей},
  header={Книга Судей},
  %headerleft={},
  %headerright={},
  abbr={Суд}
}
\vs Jdg 1:1 По смерти Иисуса вопрошали сыны Израилевы Господа, говоря: кто из нас прежде пойдет на Хананеев~--- воевать с ними?
\vs Jdg 1:2 И сказал Господь: Иуда пойдет; вот, Я предаю землю в руки его.
\vs Jdg 1:3 Иуда же сказал Симеону, брату своему: войди со мною в жребий мой, и будем воевать с Хананеями; и я войду с тобою в твой жребий. И пошел с ним Симеон.
\vs Jdg 1:4 И пошел Иуда, и предал Господь Хананеев и Ферезеев в руки их, и побили они из них в Везеке десять тысяч человек.
\vs Jdg 1:5 В Везеке встретились они с Адони-Везеком, сразились с ним и разбили Хананеев и Ферезеев.
\vs Jdg 1:6 Адони-Везек побежал, но они погнались за ним и поймали его и отсекли большие пальцы на руках его и на ногах его.
\vs Jdg 1:7 Тогда сказал Адони-Везек: семьдесят царей с отсеченными на руках и на ногах их большими пальцами собирали [крохи] под столом моим; как делал я, так и мне воздал Бог. И привели его в Иерусалим, и он умер там.
\vs Jdg 1:8 И воевали сыны Иудины против Иерусалима и взяли его, и поразили его мечом и город предали огню.
\vs Jdg 1:9 Потом пошли сыны Иудины воевать с Хананеями, которые жили на горах и на полуденной земле и на низменных местах.
\vs Jdg 1:10 И пошел Иуда на Хананеев, которые жили в Хевроне (имя же Хеврону \bibemph{было} прежде Кириаф-Арбы), и поразили Шешая, Ахимана и Фалмая [от рода Енакова].
\vs Jdg 1:11 Оттуда пошел он против жителей Давира; имя Давиру \bibemph{было} прежде Кириаф-Сефер.
\vs Jdg 1:12 И сказал Халев: кто поразит Кириаф-Сефер и возьмет его, тому отдам Ахсу, дочь мою, в жену.
\vs Jdg 1:13 И взял его Гофониил, сын Кеназа, младшего брата Халевова, и \bibemph{Халев} отдал в жену ему Ахсу, дочь свою.
\vs Jdg 1:14 Когда надлежало ей идти, \bibemph{Гофониил} научил ее просить у отца ее поле, и она сошла с осла. Халев сказал ей: что тебе?
\vs Jdg 1:15 [Ахса] сказала ему: дай мне благословение; ты дал мне землю полуденную, дай мне и источники воды. И дал ей [Халев по желанию ее] источники верхние и источники нижние.
\vs Jdg 1:16 И сыны [Иофора] Кенеянина, тестя Моисеева, пошли из города Пальм с сынами Иудиными в пустыню Иудину, которая на юг от Арада, и пришли и поселились среди народа.
\vs Jdg 1:17 И пошел Иуда с Симеоном, братом своим, и поразили Хананеев, живших в Цефафе, и предали его заклятию, и \bibemph{оттого} называется город сей Хорма.
\vs Jdg 1:18 Иуда взял также Газу с пределами ее, Аскалон с пределами его, и Екрон с пределами его [и Азот с окрестностями его].
\rsbpar\vs Jdg 1:19 Господь был с Иудою, и он овладел горою; но жителей долины не мог прогнать, потому что у них были железные колесницы.
\vs Jdg 1:20 И отдали Халеву Хеврон, как говорил Моисей, [и получил он там в наследие три города сынов Енаковых] и изгнал оттуда трех сынов Енаковых.
\vs Jdg 1:21 Но Иевусеев, которые жили в Иерусалиме, не изгнали сыны Вениаминовы, и живут Иевусеи с сынами Вениамина в Иерусалиме до сего дня.
\vs Jdg 1:22 И сыны Иосифа пошли также на Вефиль, и Господь был с ними.
\vs Jdg 1:23 [И остановились] и высматривали сыны Иосифовы Вефиль (имя же городу \bibemph{было} прежде Луз).
\vs Jdg 1:24 И увидели стражи человека, идущего из города, [и взяли его] и сказали ему: покажи нам вход в город, и сделаем с тобою милость.
\vs Jdg 1:25 Он показал им вход в город, и поразили они город мечом, а человека сего и все родство его отпустили.
\vs Jdg 1:26 Человек сей пошел в землю Хеттеев, и построил [там] город и нарек имя ему Луз. Это имя его до сего дня.
\vs Jdg 1:27 И Манассия не выгнал \bibemph{жителей} Бефсана [который есть Скифополь] и зависящих от него городов, Фаанаха и зависящих от него городов, жителей Дора и зависящих от него городов, жителей Ивлеама и зависящих от него городов, жителей Мегиддона и зависящих от него городов; и остались Хананеи жить в земле сей.
\vs Jdg 1:28 Когда Израиль пришел в силу, тогда сделал он Хананеев данниками, но изгнать не изгнал их.
\vs Jdg 1:29 И Ефрем не изгнал Хананеев, живущих в Газере; и жили Хананеи среди их в Газере [и платили им дань].
\vs Jdg 1:30 И Завулон не изгнал жителей Китрона и жителей Наглола, и жили Хананеи среди их и платили им дань.
\vs Jdg 1:31 И Асир не изгнал жителей Акко [которые платили ему дань, и жителей Дора] и жителей Сидона и Ахлава, Ахзива, Хелвы, Афека и Рехова.
\vs Jdg 1:32 И жил Асир среди Хананеев, жителей земли той, ибо не изгнал их.
\vs Jdg 1:33 И Неффалим не изгнал жителей Вефсамиса и жителей Бефанафа и жил среди Хананеев, жителей земли той; жители же Вефсамиса и Бефанафа были его данниками.
\vs Jdg 1:34 И стеснили Аморреи сынов Дановых в горах, ибо не давали им сходить на долину.
\vs Jdg 1:35 И остались Аморреи жить на горе Херес [где медведи и лисицы], в Аиалоне и Шаалвиме; но рука сынов Иосифовых одолела [Аморреев], и сделались они данниками им.
\vs Jdg 1:36 Пределы Аморреев от возвышенности Акравим и от Селы простирались и далее.
\vs Jdg 2:1 И пришел Ангел Господень из Галгала в Бохим [и в Вефиль и к дому Израилеву] и сказал [им: так говорит Господь]: Я вывел вас из Египта и ввел вас в землю, о которой клялся отцам вашим [дать вам], и сказал Я: <<не нарушу завета Моего с вами вовек;
\vs Jdg 2:2 и вы не вступайте в союз с жителями земли сей; [богам их не поклоняйтесь, изваяния их разбейте,] жертвенники их разрушьте>>. Но вы не послушали гласа Моего. Что вы это сделали?
\vs Jdg 2:3 И потому говорю Я: [не стану уже переселять людей сих, которых Я хотел изгнать,] не изгоню их от вас, и будут они вам петлею, и боги их будут для вас сетью.
\vs Jdg 2:4 Когда Ангел Господень сказал слова сии всем сынам Израилевым, то народ поднял громкий вопль и заплакал.
\vs Jdg 2:5 От сего и называют то место Бохим\fns{Плачущие.}. Там принесли они жертву Господу.
\rsbpar\vs Jdg 2:6 Когда Иисус распустил народ, и пошли сыны Израилевы, [каждый в свой дом и] каждый в свой удел, чтобы получить в наследие землю,
\vs Jdg 2:7 тогда народ служил Господу во все дни Иисуса и во все дни старейшин, которых жизнь продлилась после Иисуса и которые видели все великие дела Господни, какие Он сделал Израилю.
\vs Jdg 2:8 Но когда умер Иисус, сын Навин, раб Господень, будучи ста десяти лет,
\vs Jdg 2:9 и похоронили его в пределе удела его в Фамнаф-Сараи, на горе Ефремовой, на север от горы Гааша;
\vs Jdg 2:10 и когда весь народ оный отошел к отцам своим, и восстал после них другой род, который не знал Господа и дел Его, какие Он делал Израилю,~---
\vs Jdg 2:11 тогда сыны Израилевы стали делать злое пред очами Господа и стали служить Ваалам;
\vs Jdg 2:12 оставили Господа Бога отцов своих, Который вывел их из земли Египетской, и обратились к другим богам, богам народов, окружавших их, и стали поклоняться им, и раздражили Господа;
\vs Jdg 2:13 оставили Господа и стали служить Ваалу и Астартам.
\rsbpar\vs Jdg 2:14 И воспылал гнев Господень на Израиля, и предал их в руки грабителей, и грабили их; и предал их в руки врагов, окружавших их, и не могли уже устоять пред врагами своими.
\vs Jdg 2:15 Куда они ни пойдут, рука Господня везде была им во зло, как говорил им Господь и как клялся им Господь. И им было весьма тесно.
\vs Jdg 2:16 И воздвигал [им] Господь судей, которые спасали их от рук грабителей их;
\vs Jdg 2:17 но и судей они не слушали, а ходили блудно вслед других богов и поклонялись им [и раздражали Господа], скоро уклонялись от пути, коим ходили отцы их, повинуясь заповедям Господним. Они так не делали.
\vs Jdg 2:18 Когда Господь воздвигал им судей, то Сам Господь был с судьею и спасал их от врагов их во все дни судьи: ибо жалел \bibemph{их} Господь, слыша стон их от угнетавших и притеснявших их.
\vs Jdg 2:19 Но как скоро умирал судья, они опять делали хуже отцов своих, уклоняясь к другим богам, служа им и поклоняясь им. Не отставали от дел своих и [не отступали] от стропотного пути своего.
\vs Jdg 2:20 И воспылал гнев Господень на Израиля, и сказал Он: за то, что народ сей преступает завет Мой, который Я поставил с отцами их, и не слушает гласа Моего,
\vs Jdg 2:21 и Я не стану уже изгонять от них ни одного из тех народов, которых оставил Иисус, [сын Навин, на земле,] когда умирал,~---
\vs Jdg 2:22 чтобы искушать ими Израиля: станут ли они держаться пути Господня и ходить по нему, как держались отцы их, или нет?
\vs Jdg 2:23 И оставил Господь народы сии и не изгнал их вскоре и не предал их в руки Иисуса.
\vs Jdg 3:1 Вот те народы, которых оставил Господь, чтобы искушать ими Израильтян, всех, которые не знали о всех войнах Ханаанских,~---
\vs Jdg 3:2 для того только, чтобы знали и учились войне последующие роды сынов Израилевых, которые прежде не знали ее:
\vs Jdg 3:3 пять владельцев Филистимских, все Хананеи, Сидоняне и Евеи, живущие на горе Ливане, от горы Ваал-Ермона до входа в Емаф.
\vs Jdg 3:4 Они были \bibemph{оставлены}, чтобы искушать ими Израильтян и узнать, повинуются ли они заповедям Господним, которые Он заповедал отцам их чрез Моисея.
\vs Jdg 3:5 И жили сыны Израилевы среди Хананеев, Хеттеев, Аморреев, Ферезеев, Евеев, [Гергесеев] и Иевусеев,
\vs Jdg 3:6 и брали дочерей их себе в жены, и своих дочерей отдавали за сыновей их, и служили богам их.
\vs Jdg 3:7 И сделали сыны Израилевы злое пред очами Господа, и забыли Господа Бога своего, и служили Ваалам и Астартам.
\vs Jdg 3:8 И воспылал гнев Господень на Израиля, и предал их в руки Хусарсафема, царя Месопотамского, и служили сыны Израилевы Хусарсафему восемь лет.
\vs Jdg 3:9 Тогда возопили сыны Израилевы к Господу, и воздвигнул Господь спасителя сынам Израилевым, который спас их, Гофониила, сына Кеназа, младшего брата Халевова.
\vs Jdg 3:10 На нем был Дух Господень, и был он судьею Израиля. Он вышел на войну [против Хусарсафема], и предал Господь в руки его Хусарсафема, царя Месопотамского, и преодолела рука его Хусарсафема.
\vs Jdg 3:11 И покоилась земля сорок лет. И умер Гофониил, сын Кеназа.
\rsbpar\vs Jdg 3:12 Сыны Израилевы опять стали делать злое пред очами Господа, и укрепил Господь Еглона, царя Моавитского, против Израильтян, за то, что они делали злое пред очами Господа.
\vs Jdg 3:13 Он собрал к себе [всех] Аммонитян и Амаликитян, и пошел и поразил Израиля, и овладели они городом Пальм.
\vs Jdg 3:14 И служили сыны Израилевы Еглону, царю Моавитскому, восемнадцать лет.
\vs Jdg 3:15 Тогда возопили сыны Израилевы к Господу, и Господь воздвигнул им спасителя Аода, сына Геры, сына Иеминиева, который был левша. И послали сыны Израилевы с ним дары Еглону, царю Моавитскому.
\vs Jdg 3:16 Аод сделал себе меч с двумя остриями, длиною в локоть, и припоясал его под плащом своим к правому бедру,
\vs Jdg 3:17 [и пришел,] и поднес дары Еглону, царю Моавитскому; Еглон же был человек очень тучный.
\vs Jdg 3:18 Когда поднес \bibemph{Аод} все дары и проводил людей, принесших дары,
\vs Jdg 3:19 то сам возвратился от истуканов, которые в Галгале, и сказал: у меня есть тайное слово до тебя, царь. Он сказал: тише! И вышли от него все стоявшие при нем.
\vs Jdg 3:20 Аод вошел к нему: он сидел в прохладной горнице, которая была у него отдельно. И сказал Аод: у меня есть до тебя, [царь,] слово Божие. [Еглон] встал со стула [пред ним].
\vs Jdg 3:21 [Когда он встал,] Аод простер левую руку свою и взял меч с правого бедра своего и вонзил его в чрево его,
\vs Jdg 3:22 так что вошла за острием и рукоять, и тук закрыл острие, ибо \bibemph{Аод} не вынул меча из чрева его, и он прошел в задние части.
\vs Jdg 3:23 И вышел Аод в преддверие, и затворил за собою двери горницы, и замкнул.
\vs Jdg 3:24 Когда он вышел, рабы \bibemph{Еглона} пришли и видят, вот, двери горницы замкнуты, и говорят: верно он для нужды в прохладной комнате.
\vs Jdg 3:25 Ждали довольно долго, но видя, что никто не отпирает дверей горницы, взяли ключ и отперли, и вот, господин их лежит на земле мертвый.
\vs Jdg 3:26 Пока они недоумевали, Аод между тем ушел, [и никто о нем не думал,] прошел мимо истуканов и спасся в Сеираф.
\vs Jdg 3:27 Придя же [в землю Израилеву, Аод] вострубил трубою на горе Ефремовой, и сошли с ним сыны Израилевы с горы, и он \bibemph{шел} впереди их.
\vs Jdg 3:28 И сказал им: идите за мною, ибо предал Господь [Бог] врагов ваших Моавитян в руки ваши. И пошли за ним, и перехватили переправу через Иордан к Моаву, и не давали никому переходить.
\vs Jdg 3:29 И побили в то время Моавитян около десяти тысяч человек, всё здоровых и сильных, и никто не убежал.
\vs Jdg 3:30 Так смирились в тот день Моавитяне пред Израилем, и покоилась земля восемьдесят лет. [И был Аод судьею их до самой смерти.]
\vs Jdg 3:31 После него был Самегар, сын Анафов, который шестьсот человек Филистимлян побил воловьим рожном; и он также спас Израиля.
\vs Jdg 4:1 Когда умер Аод, сыны Израилевы стали опять делать злое пред очами Господа.
\vs Jdg 4:2 И предал их Господь в руки Иавина, царя Ханаанского, который царствовал в Асоре; военачальником у него был Сисара, который жил в Харошеф-Гоиме.
\vs Jdg 4:3 И возопили сыны Израилевы к Господу, ибо у него было девятьсот железных колесниц, и он жестоко угнетал сынов Израилевых двадцать лет.
\rsbpar\vs Jdg 4:4 В то время была судьею Израиля Девора пророчица, жена Лапидофова;
\vs Jdg 4:5 она жила под Пальмою Девориною, между Рамою и Вефилем, на горе Ефремовой; и приходили к ней [туда] сыны Израилевы на суд.
\vs Jdg 4:6 [Девора] послала и призвала Варака, сына Авиноамова, из Кедеса Неффалимова, и сказала ему: повелевает [тебе] Господь Бог Израилев: пойди, взойди на гору Фавор и возьми с собою десять тысяч человек из сынов Неффалимовых и сынов Завулоновых;
\vs Jdg 4:7 а Я приведу к тебе, к потоку Киссону, Сисару, военачальника Иавинова, и колесницы его и многолюдное [войско] его, и предам его в руки твои.
\vs Jdg 4:8 Варак сказал ей: если ты пойдешь со мною, пойду; а если не пойдешь со мною, не пойду; [ибо я не знаю дня, в который пошлет Господь Ангела со мною].
\vs Jdg 4:9 Она сказала [ему]: пойти пойду с тобою; только [знай, что] не тебе уже будет слава на сем пути, в который ты идешь; но в руки женщины предаст Господь Сисару. И встала Девора и пошла с Вараком в Кедес.
\vs Jdg 4:10 Варак созвал Завулонян и Неффалимлян в Кедес, и пошли вслед за ним десять тысяч человек, и Девора пошла с ним.
\vs Jdg 4:11 Хевер Кенеянин отделился \bibemph{тогда} от Кенеян, сынов Ховава, родственника Моисеева, и раскинул шатер свой у дубравы в Цаанниме близ Кедеса.
\rsbpar\vs Jdg 4:12 И донесли Сисаре, что Варак, сын Авиноамов, взошел на гору Фавор.
\vs Jdg 4:13 Сисара созвал все колесницы свои, девятьсот железных колесниц, и весь народ, который у него, из Харошеф-Гоима к потоку Киссону.
\vs Jdg 4:14 И сказала Девора Вараку: встань, ибо это тот день, в который Господь предаст Сисару в руки твои; Сам Господь пойдет пред тобою. И сошел Варак с горы Фавора, и за ним десять тысяч человек.
\vs Jdg 4:15 Тогда Господь привел в замешательство Сисару и все колесницы его и все ополчение его от меча Варакова, и сошел Сисара с колесницы [своей] и побежал пеший.
\vs Jdg 4:16 Варак преследовал колесницы [его] и ополчение до Харошеф-Гоима, и пало все ополчение Сисарино от меча, не осталось никого.
\vs Jdg 4:17 Сисара же убежал пеший в шатер Иаили, жены Хевера Кенеянина; ибо между Иавином, царем Асорским, и домом Хевера Кенеянина был мир.
\vs Jdg 4:18 И вышла Иаиль навстречу Сисаре и сказала ему: зайди, господин мой, зайди ко мне, не бойся. Он зашел к ней в шатер, и она покрыла его ковром [своим].
\vs Jdg 4:19 [Сисара] сказал ей: дай мне немного воды напиться, я пить хочу. Она развязала мех с молоком, и напоила его и \bibemph{опять} покрыла его.
\vs Jdg 4:20 [Сисара] сказал ей: стань у дверей шатра, и если кто придет и спросит у тебя и скажет: <<нет ли здесь кого?>>, ты скажи: <<нет>>.
\vs Jdg 4:21 Иаиль, жена Хеверова, взяла кол от шатра, и взяла молот в руку свою, и подошла к нему тихонько, и вонзила кол в висок его так, что приколола к земле; а он спал от усталости~--- и умер.
\vs Jdg 4:22 И вот, Варак гонится за Сисарою. Иаиль вышла навстречу ему и сказала ему: войди, я покажу тебе человека, которого ты ищешь. Он вошел к ней, и вот, Сисара лежит мертвый, и кол в виске его.
\rsbpar\vs Jdg 4:23 И смирил [Господь] Бог в тот день Иавина, царя Ханаанского, пред сынами Израилевыми.
\vs Jdg 4:24 Рука сынов Израилевых усиливалась более и более над Иавином, царем Ханаанским, доколе не истребили они Иавина, царя Ханаанского.
\vs Jdg 5:1 В тот день воспела Девора и Варак, сын Авиноамов, сими словами:
\vs Jdg 5:2 Израиль отмщен, народ показал рвение; прославьте Господа!
\vs Jdg 5:3 Слушайте, цари, внимайте, вельможи: я Господу, я пою, бряцаю Господу Богу Израилеву.
\vs Jdg 5:4 Когда выходил Ты, Господи, от Сеира, когда шел с поля Едомского, тогда земля тряслась, и небо капало, и облака проливали воду;
\vs Jdg 5:5 горы таяли от лица Господа, даже этот Синай от лица Господа Бога Израилева.
\vs Jdg 5:6 Во дни Самегара, сына Анафова, во дни Иаили, были пусты дороги, и ходившие прежде путями прямыми ходили тогда окольными дорогами.
\vs Jdg 5:7 Не стало обитателей в селениях у Израиля, не стало, доколе не восстала я, Девора, доколе не восстала я, мать в Израиле.
\vs Jdg 5:8 Избрали новых богов, оттого война у ворот. Виден ли был щит и копье у сорока тысяч Израиля?
\vs Jdg 5:9 Сердце мое к вам, начальники Израилевы, к ревнителям в народе; прославьте Господа!
\vs Jdg 5:10 Ездящие на ослицах белых, сидящие на коврах и ходящие по дороге, пойте песнь!
\vs Jdg 5:11 Среди голосов собирающих стада при колодезях, там да воспоют хвалу Господу, хвалу вождям Израиля! Тогда выступил ко вратам народ Господень.
\vs Jdg 5:12 Воспряни, воспряни, Девора! воспряни, воспряни! воспой песнь! Восстань, Варак! и веди пленников твоих, сын Авиноамов!
\vs Jdg 5:13 Тогда немногим из сильных подчинил Он народ; Господь подчинил мне храбрых.
\vs Jdg 5:14 От Ефрема пришли укоренившиеся в земле Амалика; за тобою Вениамин, среди народа твоего; от Махира шли начальники, и от Завулона владеющие тростью писца.
\vs Jdg 5:15 И князья Иссахаровы с Деворою, и Иссахар так же, как Варак, бросился в долину пеший. В племенах Рувимовых большое разногласие.
\vs Jdg 5:16 Что сидишь ты между овчарнями, слушая блеяние стад? В племенах Рувимовых большое разногласие.
\vs Jdg 5:17 Галаад живет \bibemph{спокойно} за Иорданом, и Дану чего бояться с кораблями? Асир сидит на берегу моря и у пристаней своих живет спокойно.
\vs Jdg 5:18 Завулон~--- народ, обрекший душу свою на смерть, и Неффалим~--- на высотах поля.
\vs Jdg 5:19 Пришли цари, сразились, тогда сразились цари Ханаанские в Фанаахе у вод Мегиддонских, но не получили нимало серебра.
\vs Jdg 5:20 С неба сражались, звезды с путей своих сражались с Сисарою.
\vs Jdg 5:21 Поток Киссон увлек их, поток Кедумим, поток Киссон. Попирай, душа моя, силу!
\vs Jdg 5:22 Тогда ломались копыта конские от побега, от побега сильных его.
\vs Jdg 5:23 Прокляните Мероз, говорит Ангел Господень, прокляните, прокляните жителей его за то, что не пришли на помощь Господу, на помощь Господу с храбрыми.
\vs Jdg 5:24 Да будет благословенна между женами Иаиль, жена Хевера Кенеянина, между женами в шатрах да будет благословенна!
\vs Jdg 5:25 Воды просил он: молока подала она, в чаше вельможеской принесла молока лучшего.
\vs Jdg 5:26 [Левую] руку свою протянула к колу, а правую свою к молоту работников; ударила Сисару, поразила голову его, разбила и пронзила висок его.
\vs Jdg 5:27 К ногам ее склонился, пал и лежал, к ногам ее склонился, пал; где склонился, там и пал сраженный.
\vs Jdg 5:28 В окно выглядывает и вопит мать Сисарина сквозь решетку: что долго не идет конница его, что медлят колеса колесниц его?
\vs Jdg 5:29 Умные из ее женщин отвечают ей, и сама она отвечает на слова свои:
\vs Jdg 5:30 верно, они нашли, делят добычу, по девице, по две девицы на каждого воина, в добычу полученная разноцветная \bibemph{одежда} Сисаре, полученная в добычу разноцветная одежда, вышитая с обеих сторон, снятая с плеч пленника.
\vs Jdg 5:31 Так да погибнут все враги Твои, Господи! Любящие же Его \bibemph{да будут} как солнце, восходящее во всей силе своей!~--- И покоилась земля сорок лет.
\vs Jdg 6:1 Сыны Израилевы стали \bibemph{опять} делать злое пред очами Господа, и предал их Господь в руки Мадианитян на семь лет.
\vs Jdg 6:2 Тяжела была рука Мадианитян над Израилем, и сыны Израилевы сделали себе от Мадианитян ущелья в горах и пещеры и укрепления.
\vs Jdg 6:3 Когда посеет Израиль, придут Мадианитяне и Амаликитяне и жители востока и ходят у них;
\vs Jdg 6:4 и стоят у них шатрами, и истребляют произведения земли до самой Газы, и не оставляют для пропитания Израилю ни овцы, ни вола, ни осла.
\vs Jdg 6:5 Ибо они приходили со скотом своим и с шатрами своими, приходили в таком множестве, как саранча; им и верблюдам их не было числа, и ходили по земле Израилевой, чтоб опустошать ее.
\vs Jdg 6:6 И весьма обнищал Израиль от Мадианитян, и возопили сыны Израилевы к Господу.
\rsbpar\vs Jdg 6:7 И когда возопили сыны Израилевы к Господу на Мадианитян,
\vs Jdg 6:8 послал Господь пророка к сынам Израилевым, и сказал им: так говорит Господь Бог Израилев: Я вывел вас из Египта, вывел вас из дома рабства;
\vs Jdg 6:9 избавил вас из руки Египтян и из руки всех, угнетавших вас, прогнал их от вас, и дал вам землю их,
\vs Jdg 6:10 и сказал вам: <<Я~--- Господь Бог ваш; не чтите богов Аморрейских, в земле которых вы живете>>; но вы не послушали гласа Моего.
\vs Jdg 6:11 И пришел Ангел Господень и сел в Офре под дубом, принадлежащим Иоасу, потомку Авиезерову; сын его Гедеон выколачивал тогда пшеницу в точиле, чтобы скрыться от Мадианитян.
\vs Jdg 6:12 И явился ему Ангел Господень и сказал ему: Господь с тобою, муж сильный!
\vs Jdg 6:13 Гедеон сказал ему: господин мой! если Господь с нами, то отчего постигло нас все это [бедствие]? и где все чудеса Его, о которых рассказывали нам отцы наши, говоря: <<из Египта вывел нас Господь>>? Ныне оставил нас Господь и предал нас в руки Мадианитян.
\vs Jdg 6:14 Господь, воззрев на него, сказал: иди с этою силою твоею и спаси Израиля от руки Мадианитян; Я посылаю тебя.
\vs Jdg 6:15 [Гедеон] сказал ему: Господи! как спасу я Израиля? вот, и племя мое в \bibemph{колене} Манассиином самое бедное, и я в доме отца моего младший.
\vs Jdg 6:16 И сказал ему Господь: Я буду с тобою, и ты поразишь Мадианитян, как одного человека.
\vs Jdg 6:17 [Гедеон] сказал Ему: если я обрел благодать пред очами Твоими, то сделай мне знамение, что Ты говоришь со мною:
\vs Jdg 6:18 не уходи отсюда, доколе я не приду к Тебе и не принесу дара моего и не предложу Тебе. Он сказал: Я останусь до возвращения твоего.
\vs Jdg 6:19 Гедеон пошел и приготовил козленка и опресноков из ефы муки; мясо положил в корзину, а похлебку влил в горшок и принес к Нему под дуб и предложил.
\vs Jdg 6:20 И сказал ему Ангел Божий: возьми мясо и опресноки, и положи на сей камень, и вылей похлебку. Он так и сделал.
\vs Jdg 6:21 Ангел Господень простер конец жезла, который был в руке его, прикоснулся к мясу и опреснокам; и вышел огонь из камня и поел мясо и опресноки; и Ангел Господень скрылся от глаз его.
\vs Jdg 6:22 И увидел Гедеон, что это Ангел Господень, и сказал Гедеон: \bibemph{увы мне}, Владыка Господи! потому что я видел Ангела Господня лицем к лицу.
\vs Jdg 6:23 Господь сказал ему: мир тебе, не бойся, не умрешь.
\vs Jdg 6:24 И устроил там Гедеон жертвенник Господу и назвал его: Иегова Шалом\fns{Господь~--- мир.}. Он еще до сего дня в Офре Авиезеровой.
\vs Jdg 6:25 В ту ночь сказал ему Господь: возьми тельца из стада отца твоего и другого тельца семилетнего, и разрушь жертвенник Ваала, который у отца твоего, и сруби священное дерево, которое при нем,
\vs Jdg 6:26 и поставь жертвенник Господу Богу твоему, [явившемуся тебе] на вершине скалы сей, в порядке, и возьми второго тельца и принеси во всесожжение на дровах дерева, которое срубишь.
\vs Jdg 6:27 Гедеон взял десять человек из рабов своих и сделал, как говорил ему Господь; но как сделать это днем он боялся домашних отца своего и жителей города, то сделал ночью.
\vs Jdg 6:28 Поутру встали жители города, и вот, жертвенник Ваалов разрушен, и дерево при нем срублено, и второй телец вознесен во всесожжение на новоустроенном жертвеннике.
\vs Jdg 6:29 И говорили друг другу: кто это сделал? Искали, расспрашивали и сказали: Гедеон, сын Иоасов, сделал это.
\vs Jdg 6:30 И сказали жители города Иоасу: выведи сына твоего; он должен умереть за то, что разрушил жертвенник Ваала и срубил дерево, которое было при нем.
\vs Jdg 6:31 Иоас сказал всем приступившим к нему: вам ли вступаться за Ваала, вам ли защищать его? кто вступится за него, тот будет предан смерти в это же утро; если он Бог, то пусть сам вступится за себя, потому что он разрушил его жертвенник.
\vs Jdg 6:32 И стал звать его с того дня Иероваалом, потому что сказал: пусть Ваал сам судится с ним за то, что он разрушил жертвенник его.
\vs Jdg 6:33 Между тем все Мадианитяне и Амаликитяне и жители востока собрались вместе, перешли [реку] и стали станом на долине Изреельской.
\vs Jdg 6:34 И Дух Господень объял Гедеона; он вострубил трубою, и созвано было племя Авиезерово идти за ним.
\vs Jdg 6:35 И послал послов по всему колену Манассиину, и оно вызвалось идти за ним; также послал послов к Асиру, Завулону и Неффалиму, и сии пришли навстречу им.
\vs Jdg 6:36 И сказал Гедеон Богу: если Ты спасешь Израиля рукою моею, как говорил Ты,
\vs Jdg 6:37 то вот, я расстелю \bibemph{здесь} на гумне стриженую шерсть: если роса будет только на шерсти, а на всей земле сухо, то буду знать, что спасешь рукою моею Израиля, как говорил Ты.
\vs Jdg 6:38 Так и сделалось: на другой день, встав рано, он стал выжимать шерсть и выжал из шерсти росы целую чашу воды.
\vs Jdg 6:39 И сказал Гедеон Богу: не прогневайся на меня, если еще раз скажу и еще только однажды сделаю испытание над шерстью: пусть будет сухо на одной только шерсти, а на всей земле пусть будет роса.
\vs Jdg 6:40 Бог так и сделал в ту ночь: только на шерсти было сухо, а на всей земле была роса.
\vs Jdg 7:1 Иероваал, он же и Гедеон, встал поутру и весь народ, бывший с ним, и расположились станом у источника Харода; Мадиамский же стан был от него к северу у холма Мор\acc{е} в долине.
\vs Jdg 7:2 И сказал Господь Гедеону: народа с тобою слишком много, не могу Я предать Мадианитян в руки их, чтобы не возгордился Израиль предо Мною и не сказал: <<моя рука спасла меня>>;
\vs Jdg 7:3 итак провозгласи вслух народа и скажи: <<кто боязлив и робок, тот пусть возвратится и пойдет назад с горы Галаада>>. И возвратилось народа двадцать две тысячи, а десять тысяч осталось.
\vs Jdg 7:4 И сказал Господь Гедеону: все еще много народа; веди их к воде, там Я выберу их тебе; о ком Я скажу: <<пусть идет с тобою>>, тот и пусть идет с тобою; а о ком скажу тебе: <<не должен идти с тобою>>, тот пусть и не идет.
\vs Jdg 7:5 Он привел народ к воде. И сказал Господь Гедеону: кто будет лакать воду языком своим, как лакает пес, того ставь особо, также и тех всех, которые будут наклоняться на колени свои и пить.
\vs Jdg 7:6 И было число лакавших ртом своим с руки триста человек; весь же остальной народ наклонялся на колени свои пить воду.
\vs Jdg 7:7 И сказал Господь Гедеону: тремя стами лакавших Я спасу вас и предам Мадианитян в руки ваши, а весь народ пусть идет, каждый в свое место.
\vs Jdg 7:8 И взяли они съестной запас у народа себе и трубы их, и отпустил Гедеон всех Израильтян по шатрам и удержал у себя триста человек; стан же Мадиамский был у него внизу в долине.
\rsbpar\vs Jdg 7:9 В ту ночь сказал ему Господь: встань, сойди в стан, Я предаю его в руки твои;
\vs Jdg 7:10 если же ты боишься идти \bibemph{один}, то пойди в стан ты и Фура, слуга твой;
\vs Jdg 7:11 и услышишь, что говорят, и тогда укрепятся руки твои, и пойдешь в стан. И сошел он и Фура, слуга его, к самому \bibemph{полку} вооруженных, которые были в стане.
\vs Jdg 7:12 Мадианитяне же и Амаликитяне и все жители востока расположились на долине в таком множестве, как саранча; верблюдам их не было числа, много было их, как песку на берегу моря.
\vs Jdg 7:13 Гедеон пришел. И вот, один рассказывает другому сон и говорит: снилось мне, будто круглый ячменный хлеб катился по стану Мадиамскому и, прикатившись к шатру, ударил в него так, что он упал, опрокинул его, и шатер распался.
\vs Jdg 7:14 Другой сказал в ответ ему: это не иное что, как меч Гедеона, сына Иоасова, Израильтянина; предал Бог в руки его Мадианитян и весь стан.
\vs Jdg 7:15 Гедеон, услышав рассказ сна и толкование его, поклонился [Господу] и возвратился в стан Израильский и сказал: вставайте! предал Господь в руки ваши стан Мадиамский.
\vs Jdg 7:16 И разделил триста человек на три отряда и дал в руки всем им трубы и пустые кувшины и в кувшины светильники.
\vs Jdg 7:17 И сказал им: смотрите на меня и делайте то же; вот, я подойду к стану, и что буду делать, то и вы делайте;
\vs Jdg 7:18 когда я и находящиеся со мною затрубим трубою, трубите и вы трубами вашими вокруг всего стана и кричите: [меч] Господа и Гедеона!
\vs Jdg 7:19 И подошел Гедеон и сто человек с ним к стану, в начале средней стражи, и разбудили стражей, и затрубили трубами и разбили кувшины, которые были в руках их.
\vs Jdg 7:20 И затрубили \bibemph{все} три отряда трубами, и разбили кувшины, и держали в левой руке своей светильники, а в правой руке трубы, и трубили, и кричали: меч Господа и Гедеона!
\vs Jdg 7:21 И стоял всякий на своем месте вокруг стана; и стали бегать во всем стане, и кричали, и обратились в бегство.
\vs Jdg 7:22 Между тем как триста человек трубили трубами, обратил Господь меч одного на другого во всем стане, и бежало ополчение до Бефшитты к Царере, до предела Авелмехолы, близ Табафы.
\vs Jdg 7:23 И созваны Израильтяне из колена Неффалимова, Асирова и всего колена Манассиина, и погнались за Мадианитянами.
\vs Jdg 7:24 Гедеон же послал послов на всю гору Ефремову сказать: выйдите навстречу Мадианитянам и перехватите у них \bibemph{переправу через} воду до Бефвары и Иордан. И созваны все Ефремляне и перехватили \bibemph{переправы через} воду до Бефвары и Иордан;
\vs Jdg 7:25 и поймали двух князей Мадиамских: Орива и Зива, и убили Орива в Цур-Ориве, а Зива в Иекев-Зиве и преследовали Мадианитян; головы же Орива и Зива принесли к Гедеону за Иордан.
\vs Jdg 8:1 И сказали ему Ефремляне: зачем ты это сделал, что не позвал нас, когда шел воевать с Мадианитянами? И сильно ссорились с ним.
\vs Jdg 8:2 [Гедеон] отвечал им: сделал ли я что такое, как вы ныне? Не счастливее ли Ефрем добирал виноград, нежели Авиезер обирал?
\vs Jdg 8:3 В ваши руки предал Бог князей Мадиамских Орива и Зива, и что мог сделать я такое, как вы? Тогда успокоился дух их против него, когда сказал он им такие слова.
\rsbpar\vs Jdg 8:4 И пришел Гедеон к Иордану, и перешел сам и триста человек, бывшие с ним. Они были утомлены [и голодны], преследуя \bibemph{врагов}.
\vs Jdg 8:5 И сказал он жителям Сокхофа: дайте хлеба народу, который идет за мною; они утомились, а я преследую Зевея и Салмана, царей Мадиамских.
\vs Jdg 8:6 Князья Сокхофа сказали: разве рука Зевея и Салмана уже в твоей руке, чтобы нам войску твоему давать хлеб?
\vs Jdg 8:7 И сказал Гедеон: за это, когда предаст Господь Зевея и Салмана в руки мои, я растерзаю тело ваше терновником пустынным и молотильными зубчатыми досками.
\vs Jdg 8:8 Оттуда пошел он в Пенуэл и то же сказал жителям его, и жители Пенуэла отвечали ему то же, что отвечали жители Сокхофа.
\vs Jdg 8:9 Он сказал и жителям Пенуэла: когда я возвращусь в мире, разрушу башню сию.
\vs Jdg 8:10 Зевей же и Салман были в Каркоре и с ними их ополчение до пятнадцати тысяч, все, что осталось из всего ополчения жителей востока; пало же сто двадцать тысяч человек, обнажающих меч.
\vs Jdg 8:11 Гедеон пошел к живущим в шатрах на восток от Новы и Иогбеги и поразил стан, когда стан стоял беспечно.
\vs Jdg 8:12 Зевей и Салман побежали; он погнался за ними и схватил обоих царей Мадиамских, Зевея и Салмана, и весь стан привел в замешательство.
\rsbpar\vs Jdg 8:13 И возвратился Гедеон, сын Иоаса, с войны от возвышенности Хереса.
\vs Jdg 8:14 И захватил юношу из жителей Сокхофа и выспросил у него; и он написал ему князей и старейшин Сокхофских семьдесят семь человек.
\vs Jdg 8:15 И пришел он к жителям Сокхофским, и сказал: вот Зевей и Салман, за которых вы посмеялись надо мною, говоря: разве рука Зевея и Салмана уже в твоей руке, чтобы нам давать хлеб утомившимся людям твоим?
\vs Jdg 8:16 И взял старейшин города и терновник пустынный и зубчатые молотильные доски и наказал ими жителей Сокхофа;
\vs Jdg 8:17 и башню Пенуэльскую разрушил, и перебил жителей города.
\vs Jdg 8:18 И сказал Зевею и Салману: каковы были те, которых вы убили на Фаворе? Они сказали: они были такие, как ты, каждый имел вид сынов царских.
\vs Jdg 8:19 [Гедеон] сказал: это были братья мои, сыны матери моей. Жив Господь! если бы вы оставили их в живых, я не убил бы вас.
\vs Jdg 8:20 И сказал Иеферу, первенцу своему: встань, убей их. Но юноша не извлек меча своего, потому что боялся, так как был еще молод.
\vs Jdg 8:21 И сказали Зевей и Салман: встань сам и порази нас, потому что по человеку и сила его. И встал Гедеон, и убил Зевея и Салмана, и взял пряжки, бывшие на шеях верблюдов их.
\vs Jdg 8:22 И сказали Израильтяне Гедеону: владей нами ты и сын твой и сын сына твоего, ибо ты спас нас из руки Мадианитян.
\vs Jdg 8:23 Гедеон сказал им: ни я не буду владеть вами, ни мой сын не будет владеть вами; Господь да владеет вами.
\vs Jdg 8:24 И сказал им Гедеон: прошу у вас одного, дайте мне каждый по серьге из добычи своей. (Ибо у \bibemph{неприятелей} много было золотых серег, потому что они были Измаильтяне.)
\vs Jdg 8:25 Они сказали: дадим. И разостлали одежду и бросали туда каждый по серьге из добычи своей.
\vs Jdg 8:26 Весу в золотых серьгах, которые он выпросил, было тысяча семьсот золотых [сиклей], кроме пряжек, пуговиц и пурпуровых одежд, которые были на царях Мадиамских, и кроме [золотых] цепочек, которые были на шее у верблюдов их.
\vs Jdg 8:27 Из этого сделал Гедеон ефод и положил его в своем городе, в Офре, и стали все Израильтяне блудно ходить туда за ним, и был он сетью Гедеону и всему дому его.
\rsbpar\vs Jdg 8:28 Так смирились Мадианитяне пред сынами Израиля и не стали уже поднимать головы своей, и покоилась земля сорок лет во дни Гедеона.
\vs Jdg 8:29 И пошел Иероваал, сын Иоасов, и жил в доме своем.
\rsbpar\vs Jdg 8:30 У Гедеона было семьдесят сыновей, происшедших от чресл его, потому что у него много было жен.
\vs Jdg 8:31 Также и наложница, жившая в Сихеме, родила ему сына, и он дал ему имя Авимелех.
\vs Jdg 8:32 И умер Гедеон, сын Иоасов, в глубокой старости, и погребен во гробе отца своего Иоаса, в Офре Авиезеровой.
\vs Jdg 8:33 Когда умер Гедеон, сыны Израилевы опять стали блудно ходить вслед Ваалов и поставили себе богом Ваалверифа;
\vs Jdg 8:34 и не вспомнили сыны Израилевы Господа Бога своего, Который избавлял их из руки всех врагов, окружавших их;
\vs Jdg 8:35 и дому Иероваалову, \bibemph{или} Гедеонову, не сделали милости за все благодеяния, какие он сделал Израилю.
\vs Jdg 9:1 Авимелех, сын Иероваалов, пошел в Сихем к братьям матери своей и говорил им и всему племени отца матери своей, и сказал:
\vs Jdg 9:2 внушите всем жителям Сихемским: что лучше для вас, чтобы владели вами все семьдесят сынов Иеровааловых, или чтобы владел один? и вспомните, что я кость ваша и плоть ваша.
\vs Jdg 9:3 Братья матери его внушили о нем все сии слова жителям Сихемским; и склонилось сердце их к Авимелеху, ибо говорили они: он брат наш.
\vs Jdg 9:4 И дали ему семьдесят \bibemph{сиклей} серебра из дома Ваалверифа; Авимелех нанял на оные праздных и своевольных людей, которые и пошли за ним.
\vs Jdg 9:5 И пришел он в дом отца своего в Офру и убил братьев своих, семьдесят сынов Иеровааловых, на одном камне. Остался только Иофам, младший сын Иероваалов, потому что скрылся.
\vs Jdg 9:6 И собрались все жители Сихемские и весь дом Милло, и пошли и поставили царем Авимелеха у дуба, что близ Сихема.
\rsbpar\vs Jdg 9:7 Когда рассказали об этом Иофаму, он пошел и стал на вершине горы Гаризима и, возвысив голос свой, кричал и говорил им: послушайте меня, жители Сихема, и послушает вас Бог!
\vs Jdg 9:8 Пошли некогда дерева помазать над собою царя и сказали маслине: царствуй над нами.
\vs Jdg 9:9 Маслина сказала им: оставлю ли я тук мой, которым чествуют богов и людей и пойду ли скитаться по деревам?
\vs Jdg 9:10 И сказали дерева смоковнице: иди ты, царствуй над нами.
\vs Jdg 9:11 Смоковница сказала им: оставлю ли я сладость мою и хороший плод мой и пойду ли скитаться по деревам?
\vs Jdg 9:12 И сказали дерева виноградной лозе: иди ты, царствуй над нами.
\vs Jdg 9:13 Виноградная лоза сказала им: оставлю ли я сок мой, который веселит богов и человеков, и пойду ли скитаться по деревам?
\vs Jdg 9:14 Наконец сказали все дерева терновнику: иди ты, царствуй над нами.
\vs Jdg 9:15 Терновник сказал деревам: если вы по истине поставляете меня царем над собою, то идите, покойтесь под тенью моею; если же нет, то выйдет огонь из терновника и пожжет кедры Ливанские.
\vs Jdg 9:16 Итак смотрите, по истине ли и по правде ли вы поступили, поставив Авимелеха царем? И хорошо ли вы поступили с Иероваалом и домом его, и сообразно ли с его благодеяниями поступили вы?
\vs Jdg 9:17 За вас отец мой сражался, не дорожил жизнью своею и избавил вас от руки Мадианитян;
\vs Jdg 9:18 а вы теперь восстали против дома отца моего, и убили семьдесят сынов отца моего на одном камне, и поставили царем над жителями Сихемскими Авимелеха, сына рабыни его, потому что он брат ваш.
\vs Jdg 9:19 Если вы ныне по истине и по правде поступили с Иероваалом и домом его, то [да будет на вас благословение и] радуйтесь об Авимелехе, и он пусть радуется о вас;
\vs Jdg 9:20 если же нет, то да изыдет огонь от Авимелеха и да пожжет жителей Сихемских и весь дом Милло и да изыдет огонь от жителей Сихемских и от дома Милло, и да пожжет Авимелеха.
\vs Jdg 9:21 И побежал Иофам, и убежал и пошел в Беэр, и жил там, \bibemph{укрываясь} от брата своего Авимелеха.
\rsbpar\vs Jdg 9:22 Авимелех же царствовал над Израилем три года.
\vs Jdg 9:23 И послал Бог злого духа между Авимелехом и между жителями Сихема, и не стали покоряться жители Сихемские Авимелеху,
\vs Jdg 9:24 дабы таким образом совершилось мщение за семьдесят сынов Иеровааловых, и кровь их обратилась на Авимелеха, брата их, который убил их, и на жителей Сихемских, которые подкрепили руки его, чтоб убить братьев своих.
\vs Jdg 9:25 Жители Сихемские посадили против него в засаду людей на вершинах гор, которые грабили всякого проходящего мимо их по дороге. О сем донесено было Авимелеху.
\vs Jdg 9:26 Пришел же и Гаал, сын Еведов, с братьями своими в Сихем, и ходили они по Сихему, и жители Сихемские положились на него.
\vs Jdg 9:27 И вышли в поле, и собирали виноград свой, и давили в точилах, и делали праздники, ходили в дом бога своего, и ели и пили, и проклинали Авимелеха.
\vs Jdg 9:28 Гаал, сын Еведов, говорил: кто Авимелех и что Сихем, чтобы нам служить ему? Не сын ли он Иероваалов, и не Зевул ли главный начальник его? Служите лучше потомкам Еммора, отца Сихемова, а ему для чего нам служить?
\vs Jdg 9:29 Если бы кто дал народ сей в руки мои, я прогнал бы Авимелеха. И сказано было Авимелеху: умножь войско твое и выходи.
\vs Jdg 9:30 Зевул, начальник города, услышал слова Гаала, сына Еведова, и воспылал гнев его.
\vs Jdg 9:31 Он хитрым образом отправляет послов к Авимелеху, чтобы сказать: вот, Гаал, сын Еведов, и братья его пришли в Сихем, и вот, они возмущают против тебя город;
\vs Jdg 9:32 итак, встань ночью, ты и народ, находящийся с тобою, и поставь засаду в поле;
\vs Jdg 9:33 поутру же, при восхождении солнца, встань рано и приступи к городу; и когда он и народ, который у него, выйдут к тебе, тогда делай с ними, что может рука твоя.
\vs Jdg 9:34 И встал ночью Авимелех и весь народ, находившийся с ним, и поставили в засаду у Сихема четыре отряда.
\vs Jdg 9:35 [Поутру] Гаал, сын Еведов, вышел и стал у ворот городских; и встал Авимелех и народ, бывший с ним, из засады.
\vs Jdg 9:36 Гаал, увидев народ, говорит Зевулу: вот, народ спускается с вершины гор. А Зевул сказал ему: тень гор тебе кажется людьми.
\vs Jdg 9:37 Гаал опять говорил и сказал: вот, народ спускается с возвышенности, и один отряд идет от дуба Меонним.
\vs Jdg 9:38 И сказал ему Зевул: где уста твои, которые говорили: <<кто Авимелех, чтобы мы стали служить ему?>> Это тот народ, который ты пренебрегал; выходи теперь и сразись с ним.
\vs Jdg 9:39 И пошел Гаал впереди жителей Сихемских и сразился с Авимелехом.
\vs Jdg 9:40 И погнался за ним Авимелех, и побежал он от него, и много пало убитых до самых ворот города.
\vs Jdg 9:41 И остался Авимелех в Аруме, а Гаала и братьев его Зевул выгнал, чтоб они не жили в Сихеме.
\vs Jdg 9:42 На другой день вышел народ в поле, и донесли о сем Авимелеху.
\vs Jdg 9:43 Он взял свой народ и разделил его на три отряда и поставил в засаду в поле. И увидев, что народ вышел из города, восстал на них и побил их.
\vs Jdg 9:44 Между тем как Авимелех и отряды, бывшие с ним, приступили и стали у ворот городских, другие два отряда напали на всех, бывших в поле, и убивали их.
\vs Jdg 9:45 И сражался Авимелех с городом весь тот день, и взял город, и побил народ, бывший в нем, и разрушил город и засеял его солью.
\vs Jdg 9:46 Услышав об этом, все бывшие в башне Сихемской ушли в башню капища \bibemph{Ваал-}Верифа.
\vs Jdg 9:47 Авимелеху донесено, что собрались \bibemph{туда} все бывшие в башне Сихемской.
\vs Jdg 9:48 И пошел Авимелех на гору Селмон, сам и весь народ, бывший с ним, и взял Авимелех топоры с собою и нарубил сучьев древесных, и положил на плечи свои, и сказал народу, бывшему с ним: вы видели, что я делал; скорее делайте и вы то же, что я.
\vs Jdg 9:49 И нарубил каждый из всего народа сучьев, и пошли за Авимелехом, и положили к башне, и сожгли посредством их башню огнем, и умерли все бывшие в башне Сихемской, около тысячи мужчин и женщин.
\vs Jdg 9:50 Потом пошел Авимелех в Тевец и осадил Тевец и взял его.
\vs Jdg 9:51 Среди города была крепкая башня, и убежали туда все мужчины и женщины и все жители города, и заперлись и взошли на кровлю башни.
\vs Jdg 9:52 Авимелех пришел к башне и окружил ее и подошел к дверям башни, чтобы сжечь ее огнем.
\vs Jdg 9:53 Тогда одна женщина бросила обломок жернова на голову Авимелеху и проломила ему череп.
\vs Jdg 9:54 [Авимелех] тотчас призвал отрока, оруженосца своего, и сказал ему: обнажи меч твой и умертви меня, чтобы не сказали обо мне: женщина убила его. И пронзил его отрок его, и он умер.
\vs Jdg 9:55 Израильтяне, видя, что умер Авимелех, пошли каждый в свое место.
\vs Jdg 9:56 Так воздал Бог Авимелеху за злодеяние, которое он сделал отцу своему, убив семьдесят братьев своих.
\vs Jdg 9:57 И все злодеяния жителей Сихемских обратил Бог на голову их; и постигло их проклятие Иофама, сына Иероваалова.
\vs Jdg 10:1 После Авимелеха восстал для спасения Израиля Фола, сын Фуи, сына Додова, из колена Иссахарова. Он жил в Шамире на горе Ефремовой.
\vs Jdg 10:2 Он был судьею Израиля двадцать три года, и умер, и погребен в Шамире.
\vs Jdg 10:3 После него восстал Иаир из Галаада и был судьею Израиля двадцать два года.
\vs Jdg 10:4 У него было тридцать [два] сына, ездивших на тридцати [двух] молодых ослах, и тридцать [два] города было у них; их до сего дня называют селениями Иаира, что в земле Галаадской.
\vs Jdg 10:5 И умер Иаир и погребен в Камоне.
\rsbpar\vs Jdg 10:6 Сыны Израилевы продолжали делать злое пред очами Господа и служили Ваалам и Астартам, и богам Арамейским, и богам Сидонским, и богам Моавитским, и богам Аммонитским, и богам Филистимским; а Господа оставили и не служили Ему.
\vs Jdg 10:7 И воспылал гнев Господа на Израиля, и Он предал их в руки Филистимлян и в руки Аммонитян;
\vs Jdg 10:8 они теснили и мучили сынов Израилевых с того года восемнадцать лет, всех сынов Израилевых по ту сторону Иордана в земле Аморрейской, которая в Галааде.
\vs Jdg 10:9 Наконец Аммонитяне перешли Иордан, чтобы вести войну с Иудою и Вениамином и с домом Ефремовым. И весьма тесно было сынам Израиля.
\vs Jdg 10:10 И возопили сыны Израилевы к Господу, и говорили: согрешили мы пред Тобою, потому что оставили Бога нашего и служили Ваалам.
\vs Jdg 10:11 И сказал Господь сынам Израилевым: не угнетали ли вас Египтяне, и Аморреи, и Аммонитяне, и Филистимляне,
\vs Jdg 10:12 и Сидоняне, и Амаликитяне, и Моавитяне, и когда вы взывали ко Мне, не спасал ли Я вас от рук их?
\vs Jdg 10:13 А вы оставили Меня и стали служить другим богам; за то Я не буду уже спасать вас:
\vs Jdg 10:14 пойдите, взывайте к богам, которых вы избрали, пусть они спасают вас в тесное для вас время.
\vs Jdg 10:15 И сказали сыны Израилевы Господу: согрешили мы; делай с нами все, что Тебе угодно, только избавь нас ныне.
\vs Jdg 10:16 И отвергли от себя чужих богов и стали служить [только] Господу. И не потерпела душа Его страдания Израилева.
\vs Jdg 10:17 Аммонитяне собрались и расположились станом в Галааде; собрались также сыны Израилевы и стали станом в Массифе.
\vs Jdg 10:18 Народ \bibemph{и} князья Галаадские сказали друг другу: кто начнет войну против Аммонитян, тот будет начальником всех жителей Галаадских.
\vs Jdg 11:1 Иеффай Галаадитянин был человек храбрый. Он был сын блудницы; от Галаада родился Иеффай.
\vs Jdg 11:2 И жена Галаадова родила ему сыновей. Когда возмужали сыновья жены, изгнали они Иеффая, сказав ему: ты не наследник в доме отца нашего, потому что ты сын другой женщины.
\vs Jdg 11:3 И убежал Иеффай от братьев своих и жил в земле Тов; и собрались к Иеффаю праздные люди и выходили с ним.
\rsbpar\vs Jdg 11:4 Чрез несколько времени Аммонитяне пошли войною на Израиля.
\vs Jdg 11:5 Во время войны Аммонитян с Израильтянами пришли старейшины Галаадские взять Иеффая из земли Тов
\vs Jdg 11:6 и сказали Иеффаю: приди, будь у нас вождем, и сразимся с Аммонитянами.
\vs Jdg 11:7 Иеффай сказал старейшинам Галаадским: не вы ли возненавидели меня и выгнали из дома отца моего? зачем же пришли ко мне ныне, когда вы в беде?
\vs Jdg 11:8 Старейшины Галаадские сказали Иеффаю: для того мы теперь пришли к тебе, чтобы ты пошел с нами и сразился с Аммонитянами и был у нас начальником всех жителей Галаадских.
\vs Jdg 11:9 И сказал Иеффай старейшинам Галаадским: если вы возвратите меня, чтобы сразиться с Аммонитянами, и Господь предаст мне их, то останусь ли я у вас начальником?
\vs Jdg 11:10 Старейшины Галаадские сказали Иеффаю: Господь да будет свидетелем между нами, что мы сделаем по слову твоему!
\vs Jdg 11:11 И пошел Иеффай со старейшинами Галаадскими, и народ поставил его над собою начальником и вождем, и Иеффай произнес все слова свои пред лицем Господа в Массифе.
\vs Jdg 11:12 И послал Иеффай послов к царю Аммонитскому сказать: что тебе до меня, что ты пришел ко мне воевать на земле моей?
\vs Jdg 11:13 Царь Аммонитский сказал послам Иеффая: Израиль, когда шел из Египта, взял землю мою от Арнона до Иавока и Иордана; итак возврати мне ее с миром [и я отступлю].
\vs Jdg 11:14 [И возвратились послы к Иеффаю.] Иеффай в другой раз послал послов к царю Аммонитскому,
\vs Jdg 11:15 сказать ему: так говорит Иеффай: Израиль не взял земли Моавитской и земли Аммонитской;
\vs Jdg 11:16 ибо когда шли из Египта, Израиль пошел в пустыню к Чермному морю и пришел в Кадес;
\vs Jdg 11:17 оттуда послал Израиль послов к царю Едомскому сказать: <<позволь мне пройти землею твоею>>; но царь Едомский не послушал; и к царю Моавитскому он посылал, но и тот не согласился; посему Израиль оставался в Кадесе.
\vs Jdg 11:18 И пошел пустынею, и миновал землю Едомскую и землю Моавитскую, и, придя к восточному пределу земли Моавитской, расположился станом за Арноном; но не входил в пределы Моавитские, ибо Арнон есть предел Моава.
\vs Jdg 11:19 И послал Израиль послов к Сигону, царю Аморрейскому, царю Есевонскому, и сказал ему Израиль: позволь нам пройти землею твоею в свое место.
\vs Jdg 11:20 Но Сигон не согласился пропустить Израиля чрез пределы свои, и собрал Сигон весь народ свой, и расположился станом в Иааце, и сразился с Израилем.
\vs Jdg 11:21 И предал Господь Бог Израилев Сигона и весь народ его в руки Израилю, и он побил их; и получил Израиль в наследие всю землю Аморрея, жившего в земле той;
\vs Jdg 11:22 и получили они в наследие все пределы Аморрея от Арнона до Иавока и от пустыни до Иордана.
\vs Jdg 11:23 Итак Господь Бог Израилев изгнал Аморрея от лица народа Своего Израиля, а ты хочешь взять его наследие?
\vs Jdg 11:24 Не владеешь ли ты тем, что дал тебе Хамос, бог твой? И мы владеем всем тем, что дал нам в наследие Господь Бог наш.
\vs Jdg 11:25 Разве ты лучше Валака, сына Сепфорова, царя Моавитского? Ссорился ли он с Израилем, или воевал ли с ними?
\vs Jdg 11:26 Израиль уже живет триста лет в Есевоне и в зависящих от него \bibemph{городах}, в Ароере и зависящих от него \bibemph{городах}, и во всех городах, которые близ Арнона; для чего вы в то время не отнимали [их]?
\vs Jdg 11:27 А я не виновен пред тобою, и ты делаешь мне зло, выступив против меня войною. Господь Судия да будет ныне судьею между сынами Израиля и между Аммонитянами!
\vs Jdg 11:28 Но царь Аммонитский не послушал слов Иеффая, с которыми он посылал к нему.
\rsbpar\vs Jdg 11:29 И был на Иеффае Дух Господень, и прошел он Галаад и Манассию, и прошел Массифу Галаадскую, и из Массифы Галаадской пошел к Аммонитянам.
\vs Jdg 11:30 И дал Иеффай обет Господу и сказал: если Ты предашь Аммонитян в руки мои,
\vs Jdg 11:31 то по возвращении моем с миром от Аммонитян, что выйдет из ворот дома моего навстречу мне, будет Господу, и вознесу сие на всесожжение.
\vs Jdg 11:32 И пришел Иеффай к Аммонитянам~--- сразиться с ними, и предал их Господь в руки его;
\vs Jdg 11:33 и поразил их поражением весьма великим, от Ароера до Минифа двадцать городов, и до Авель-Керамима, и смирились Аммонитяне пред сынами Израилевыми.
\vs Jdg 11:34 И пришел Иеффай в Массифу в дом свой, и вот, дочь его выходит навстречу ему с тимпанами и ликами: она была у него только одна, и не было у него еще ни сына, ни дочери.
\vs Jdg 11:35 Когда он увидел ее, разодрал одежду свою и сказал: ах, дочь моя! ты сразила меня; и ты в числе нарушителей покоя моего! я отверз [о тебе] уста мои пред Господом и не могу отречься.
\vs Jdg 11:36 Она сказала ему: отец мой! ты отверз уста твои пред Господом~--- и делай со мною то, что произнесли уста твои, когда Господь совершил чрез тебя отмщение врагам твоим Аммонитянам.
\vs Jdg 11:37 И сказала отцу своему: сделай мне только вот что: отпусти меня на два месяца; я пойду, взойду на горы и опл\acc{а}чу девство мое с подругами моими.
\vs Jdg 11:38 Он сказал: пойди. И отпустил ее на два месяца. Она пошла с подругами своими и оплакивала девство свое в горах.
\vs Jdg 11:39 По прошествии двух месяцев она возвратилась к отцу своему, и он совершил над нею обет свой, который дал, и она не познала мужа. И вошло в обычай у Израиля,
\vs Jdg 11:40 что ежегодно дочери Израилевы ходили оплакивать дочь Иеффая Галаадитянина, четыре дня в году.
\vs Jdg 12:1 Ефремляне собрались и перешли в Севину и сказали Иеффаю: для чего ты ходил воевать с Аммонитянами, а нас не позвал с собою? мы сожжем дом твой огнем и с тобою вместе.
\vs Jdg 12:2 Иеффай сказал им: я и народ мой имели с Аммонитянами сильную ссору; я звал вас, но вы не спасли меня от руки их;
\vs Jdg 12:3 видя, что ты не спасаешь меня, я подверг опасности жизнь мою и пошел на Аммонитян, и предал их Господь в руки мои; зачем же вы пришли ныне воевать со мною?
\vs Jdg 12:4 И собрал Иеффай всех жителей Галаадских и сразился с Ефремлянами, и побили жители Галаадские Ефремлян, говоря: вы беглецы Ефремовы, Галаад же среди Ефрема и среди Манассии.
\vs Jdg 12:5 И перехватили Галаадитяне переправу чрез Иордан от Ефремлян, и когда кто из уцелевших Ефремлян говорил: <<позвольте мне переправиться>>, то жители Галаадские говорили ему: не Ефремлянин ли ты? Он говорил: нет.
\vs Jdg 12:6 Они говорили ему <<скажи: шибболет>>, а он говорил: <<сибболет>>, и не мог иначе выговорить. Тогда они, взяв его, заколали у переправы чрез Иордан. И пало в то время из Ефремлян сорок две тысячи.
\rsbpar\vs Jdg 12:7 Иеффай был судьею Израиля шесть лет, и умер Иеффай Галаадитянин и погребен в одном из городов Галаадских.
\vs Jdg 12:8 После него был судьею Израиля Есевон из Вифлеема.
\vs Jdg 12:9 У него было тридцать сыновей, и тридцать дочерей отпустил он из дома [в замужество], а тридцать дочерей взял со стороны за сыновей своих, и был судьею Израиля семь лет.
\vs Jdg 12:10 И умер Есевон и погребен в Вифлееме.
\vs Jdg 12:11 После него был судьею Израиля Елон Завулонянин и судил Израиля десять лет.
\vs Jdg 12:12 И умер Елон Завулонянин и погребен в Аиалоне, в земле Завулоновой.
\vs Jdg 12:13 После него был судьею Израиля Авдон, сын Гиллела, Пирафонянин.
\vs Jdg 12:14 У него было сорок сыновей и тридцать внуков, ездивших на семидесяти молодых ослах; он судил Израиля восемь лет.
\vs Jdg 12:15 И умер Авдон, сын Гиллела, Пирафонянин, и погребен в Пирафоне в земле Ефремовой, на горе Амаликовой.
\vs Jdg 13:1 Сыны Израилевы продолжали делать злое пред очами Господа, и предал их Господь в руки Филистимлян на сорок лет.
\rsbpar\vs Jdg 13:2 В то время был человек из Цоры, от племени Данова, именем Маной; жена его была неплодна и не рождала.
\vs Jdg 13:3 И явился Ангел Господень жене и сказал ей: вот, ты неплодна и не рождаешь; но зачнешь, и родишь сына;
\vs Jdg 13:4 итак берегись, не пей вина и сикера, и не ешь ничего нечистого;
\vs Jdg 13:5 ибо вот, ты зачнешь и родишь сына, и бритва не коснется головы его, потому что от самого чрева младенец сей будет назорей Божий, и он начнет спасать Израиля от руки Филистимлян.
\vs Jdg 13:6 Жена пришла и сказала мужу своему: человек Божий приходил ко мне, которого вид, как вид Ангела Божия, весьма почтенный; я не спросила его, откуда он, и он не сказал мне имени своего;
\vs Jdg 13:7 он сказал мне: <<вот, ты зачнешь и родишь сына; итак не пей вина и сикера и не ешь ничего нечистого, ибо младенец от самого чрева до смерти своей будет назорей Божий>>.
\vs Jdg 13:8 Маной помолился Господу и сказал: Господи! пусть придет опять к нам человек Божий, которого посылал Ты, и научит нас, что нам делать с имеющим родиться младенцем.
\vs Jdg 13:9 И услышал Бог голос Маноя, и Ангел Божий опять пришел к жене, когда она была в поле, и Маноя, мужа ее, не было с нею.
\vs Jdg 13:10 Жена тотчас побежала и известила мужа своего и сказала ему: вот, явился мне человек, приходивший ко мне тогда.
\vs Jdg 13:11 Маной встал и пошел с женою своею, и пришел к тому человеку и сказал ему: ты ли тот человек, который говорил с сею женщиною? [Ангел] сказал: я.
\vs Jdg 13:12 И сказал Маной: итак, если исполнится слово твое, как нам поступать с младенцем сим и что делать с ним?
\vs Jdg 13:13 Ангел Господень сказал Маною: пусть он остерегается всего, о чем я сказал жене;
\vs Jdg 13:14 пусть не ест ничего, что производит виноградная лоза; пусть не пьет вина и сикера и не ест ничего нечистого и соблюдает все, что я приказал ей.
\vs Jdg 13:15 И сказал Маной Ангелу Господню: позволь удержать тебя, пока мы изготовим для тебя козленка.
\vs Jdg 13:16 Ангел Господень сказал Маною: хотя бы ты и удержал меня, но я не буду есть хлеба твоего; если же хочешь совершить всесожжение Господу, то вознеси его. Маной же не знал, что это Ангел Господень.
\vs Jdg 13:17 И сказал Маной Ангелу Господню: как тебе имя? чтобы нам прославить тебя, когда исполнится слово твое.
\vs Jdg 13:18 Ангел Господень сказал ему: что ты спрашиваешь об имени моем? оно чудно.
\vs Jdg 13:19 И взял Маной козленка и хлебное приношение и вознес Господу на камне. И сделал Он чудо, которое видели Маной и жена его.
\vs Jdg 13:20 Когда пламень стал подниматься от жертвенника к небу, Ангел Господень поднялся в пламени жертвенника. Видя это, Маной и жена его пали лицем на землю.
\vs Jdg 13:21 И невидим стал Ангел Господень Маною и жене его. Тогда Маной узнал, что это Ангел Господень.
\vs Jdg 13:22 И сказал Маной жене своей: верно мы умрем, ибо видели мы Бога.
\vs Jdg 13:23 Жена его сказала ему: если бы Господь хотел умертвить нас, то не принял бы от рук наших всесожжения и хлебного приношения, и не показал бы нам всего того, и теперь не открыл бы нам сего.
\vs Jdg 13:24 И родила жена сына, и нарекла имя ему: Самсон. И рос младенец, и благословлял его Господь.
\vs Jdg 13:25 И начал Дух Господень действовать в нем в стане Дановом, между Цорою и Естаолом.
\vs Jdg 14:1 И пошел Самсон в Фимнафу и увидел в Фимнафе женщину из дочерей Филистимских [и она понравилась ему].
\vs Jdg 14:2 Он пошел и объявил отцу своему и матери своей и сказал: я видел в Фимнафе женщину из дочерей Филистимских; возьмите ее мне в жену.
\vs Jdg 14:3 Отец и мать его сказали ему: разве нет женщин между дочерями братьев твоих и во всем народе моем, что ты идешь взять жену у Филистимлян необрезанных? И сказал Самсон отцу своему: ее возьми мне, потому что она мне понравилась.
\vs Jdg 14:4 Отец его и мать его не знали, что это от Господа, и что он ищет случая \bibemph{отмстить} Филистимлянам. А в то время Филистимляне господствовали над Израилем.
\vs Jdg 14:5 И пошел Самсон с отцом своим и с матерью своею в Фимнафу, и когда подходили к виноградникам Фимнафским, вот, молодой лев рыкая \bibemph{идет} навстречу ему.
\vs Jdg 14:6 И сошел на него Дух Господень, и он растерзал \bibemph{льва} как козленка; а в руке у него ничего не было. И не сказал отцу своему и матери своей, что он сделал.
\vs Jdg 14:7 И пришел и поговорил с женщиною, и она понравилась Самсону.
\vs Jdg 14:8 Спустя несколько дней, опять пошел он, чтобы взять ее, и зашел посмотреть труп льва, и вот, рой пчел в трупе львином и мед.
\vs Jdg 14:9 Он взял его в руки свои и пошел, и ел дорогою; и когда пришел к отцу своему и матери своей, дал и им, и они ели; но не сказал им, что из львиного трупа взял мед сей.
\vs Jdg 14:10 И пришел отец его к женщине, и сделал там Самсон [семидневный] пир, как обыкновенно делают женихи.
\vs Jdg 14:11 И как там увидели его, выбрали тридцать брачных друзей, которые были бы при нем.
\vs Jdg 14:12 И сказал им Самсон: загадаю я вам загадку; если вы отгадаете мне ее в семь дней пира и отгадаете верно, то я дам вам тридцать синдонов\fns{Рубашка из тонкого полотна.} и тридцать перемен одежд;
\vs Jdg 14:13 если же не сможете отгадать мне, то вы дайте мне тридцать синдонов и тридцать перемен одежд. Они сказали ему: загадай загадку твою, послушаем.
\vs Jdg 14:14 И сказал им: из ядущего вышло ядомое, и из сильного вышло сладкое. И не могли отгадать загадку в три дня.
\vs Jdg 14:15 В седьмой день сказали они жене Самсоновой: уговори мужа твоего, чтоб он разгадал нам загадку; иначе сожжем огнем тебя и дом отца твоего; разве вы призвали нас, чтоб обобрать нас?
\vs Jdg 14:16 И плакала жена Самсонова пред ним и говорила: ты ненавидишь меня и не любишь; ты загадал загадку сынам народа моего, а мне не разгадаешь ее. Он сказал ей: отцу моему и матери моей не разгадал ее; и тебе ли разгадаю?
\vs Jdg 14:17 И плакала она пред ним семь дней, в которые продолжался у них пир. Наконец в седьмой день разгадал ей, ибо она усиленно просила его. А она разгадала загадку сынам народа своего.
\vs Jdg 14:18 И в седьмой день до захождения солнечного сказали ему граждане: что слаще меда, и что сильнее льва! Он сказал им: если бы вы не орали на моей телице, то не отгадали бы моей загадки.
\vs Jdg 14:19 И сошел на него Дух Господень, и пошел он в Аскалон, и, убив там тридцать человек, снял с них одежды, и отдал перемены \bibemph{платья} их разгадавшим загадку. И воспылал гнев его, и ушел он в дом отца своего.
\vs Jdg 14:20 А жена Самсонова вышла за брачного друга его, который был при нем другом.
\vs Jdg 15:1 Чрез несколько дней, во время жатвы пшеницы, пришел Самсон повидаться с женою своею, принеся с собою козленка; и когда сказал: <<войду к жене моей в спальню>>, отец ее не дал ему войти.
\vs Jdg 15:2 И сказал отец ее: я подумал, что ты возненавидел ее, и я отдал ее другу твоему; вот, меньшая сестра красивее ее; пусть она будет тебе вместо ее.
\vs Jdg 15:3 Но Самсон сказал им: теперь я буду прав пред Филистимлянами, если сделаю им зло.
\vs Jdg 15:4 И пошел Самсон, и поймал триста лисиц, и взял факелы, и связал хвост с хвостом, и привязал по факелу между двумя хвостами;
\vs Jdg 15:5 и зажег факелы, и пустил их на жатву Филистимскую, и выжег и копны и нежатый хлеб, и виноградные сады \bibemph{и} масличные.
\vs Jdg 15:6 И говорили Филистимляне: кто это сделал? И сказали: Самсон, зять Фимнафянина, ибо этот взял жену его и отдал другу его. И пошли Филистимляне и сожгли огнем ее и [дом] отца ее.
\vs Jdg 15:7 Самсон сказал им: хотя вы сделали это, но я отмщу вам самим и тогда только успокоюсь.
\vs Jdg 15:8 И перебил он им голени и бедра, и пошел и засел в ущелье скалы Етама.
\vs Jdg 15:9 И пошли Филистимляне, и расположились станом в Иудее, и протянулись до Лехи.
\vs Jdg 15:10 И сказали жители Иудеи: за что вы вышли против нас? Они сказали: мы пришли связать Самсона, чтобы поступить с ним, как он поступил с нами.
\vs Jdg 15:11 И пошли три тысячи человек из Иудеи к ущелью скалы Етама и сказали Самсону: разве ты не знаешь, что Филистимляне господствуют над нами? что ты это сделал нам? Он сказал им: как они со мною поступили, так и я поступил с ними.
\vs Jdg 15:12 И сказали ему: мы пришли связать тебя, чтобы отдать тебя в руки Филистимлянам. И сказал им Самсон: поклянитесь мне, что вы не убьете меня.
\vs Jdg 15:13 И сказали ему: нет, мы только свяжем тебя и отдадим тебя в руки их, а умертвить не умертвим. И связали его двумя новыми веревками и повели его из ущелья.
\vs Jdg 15:14 Когда он подошел к Лехе, Филистимляне с криком встретили его. И сошел на него Дух Господень, и веревки, бывшие на руках его, сделались, как перегоревший лен, и упали узы его с рук его.
\vs Jdg 15:15 Нашел он свежую ослиную челюсть и, протянув руку свою, взял ее, и убил ею тысячу человек.
\vs Jdg 15:16 И сказал Самсон: челюстью ослиною толпу, две толпы, челюстью ослиною убил я тысячу человек.
\vs Jdg 15:17 Сказав это, бросил челюсть из руки своей и назвал то место: Рамаф-Лехи\fns{Брошенная челюсть.}.
\vs Jdg 15:18 И почувствовал сильную жажду и воззвал к Господу и сказал: Ты соделал рукою раба Твоего великое спасение сие; а теперь умру я от жажды, и попаду в руки необрезанных.
\vs Jdg 15:19 И разверз Бог ямину в Лехе, и потекла из нее вода. Он напился, и возвратился дух его, и он ожил; оттого и наречено имя месту сему: <<Источник взывающего>>, который в Лехе до сего дня.
\vs Jdg 15:20 И был он судьею Израиля во дни Филистимлян двадцать лет.
\vs Jdg 16:1 Пришел однажды Самсон в Газу и, увидев там блудницу, вошел к ней.
\vs Jdg 16:2 Жителям Газы сказали: Самсон пришел сюда. И ходили они кругом, и подстерегали его всю ночь в воротах города, и таились всю ночь, говоря: до света утреннего \bibemph{подождем, и} убьем его.
\vs Jdg 16:3 А Самсон спал до полуночи; в полночь же встав, схватил двери городских ворот с обоими косяками, поднял их вместе с запором, положил на плечи свои и отнес их на вершину горы, которая на пути к Хеврону, [и положил их там].
\vs Jdg 16:4 После того полюбил он одну женщину, жившую на долине Сорек; имя ей Далида.
\vs Jdg 16:5 К ней пришли владельцы Филистимские и говорят ей: уговори его, и выведай, в чем великая сила его и как нам одолеть его, чтобы связать его и усмирить его; а мы дадим тебе за то каждый тысячу сто \bibemph{сиклей} серебра.
\vs Jdg 16:6 И сказала Далида Самсону: скажи мне, в чем великая сила твоя и чем связать тебя, чтобы усмирить тебя?
\vs Jdg 16:7 Самсон сказал ей: если свяжут меня семью сырыми тетивами, которые не засушены, то я сделаюсь бессилен и буду как и прочие люди.
\vs Jdg 16:8 И принесли ей владельцы Филистимские семь сырых тетив, которые не засохли, и она связала его ими.
\vs Jdg 16:9 (Между тем один скрытно сидел у нее в спальне.) И сказала ему: Самсон! Филистимляне \bibemph{идут} на тебя. Он разорвал тетивы, как разрывают нитку из пакли, когда пережжет ее огонь. И не узнана сила его.
\vs Jdg 16:10 И сказала Далида Самсону: вот, ты обманул меня и говорил мне ложь; скажи же теперь мне, чем связать тебя?
\vs Jdg 16:11 Он сказал ей: если свяжут меня новыми веревками, которые не были в деле, то я сделаюсь бессилен и буду, как прочие люди.
\vs Jdg 16:12 Далида взяла новые веревки и связала его и сказала ему: Самсон! Филистимляне \bibemph{идут} на тебя. (Между тем один скрытно сидел в спальне.) И сорвал он их с рук своих, как нитки.
\vs Jdg 16:13 И сказала Далида Самсону: все ты обманываешь меня и говоришь мне ложь; скажи мне, чем бы связать тебя? Он сказал ей: если ты воткешь семь кос головы моей в ткань и прибьешь ее гвоздем к ткальной колоде, [то я буду бессилен, как и прочие люди].
\vs Jdg 16:14 [И усыпила его Далида на коленях своих. И когда он уснул, взяла Далида семь кос головы его,] и прикрепила их к колоде, и сказала ему: Филистимляне \bibemph{идут} на тебя, Самсон! Он пробудился от сна своего и выдернул ткальную колоду вместе с тканью; [и не узнана сила его].
\vs Jdg 16:15 И сказала ему [Далида]: как же ты говоришь: <<люблю тебя>>, а сердце твое не со мною? вот, ты трижды обманул меня, и не сказал мне, в чем великая сила твоя.
\vs Jdg 16:16 И как она словами своими тяготила его всякий день и мучила его, то душе его тяжело стало до смерти.
\vs Jdg 16:17 И он открыл ей все сердце свое, и сказал ей: бритва не касалась головы моей, ибо я назорей Божий от чрева матери моей; если же остричь меня, то отступит от меня сила моя; я сделаюсь слаб и буду, как прочие люди.
\vs Jdg 16:18 Далида, видя, что он открыл ей все сердце свое, послала и звала владельцев Филистимских, сказав им: идите теперь; он открыл мне все сердце свое. И пришли к ней владельцы Филистимские и принесли серебро в руках своих.
\vs Jdg 16:19 И усыпила его [Далида] на коленях своих, и призвала человека, и велела ему остричь семь кос головы его. И начал он ослабевать, и отступила от него сила его.
\vs Jdg 16:20 Она сказала: Филистимляне \bibemph{идут} на тебя, Самсон! Он пробудился от сна своего, и сказал: пойду, как и прежде, и освобожусь. А не знал, что Господь отступил от него.
\vs Jdg 16:21 Филистимляне взяли его и выкололи ему глаза, привели его в Газу и оковали его двумя медными цепями, и он молол в доме узников.
\vs Jdg 16:22 Между тем волосы на голове его начали расти, где они были острижены.
\vs Jdg 16:23 Владельцы Филистимские собрались, чтобы принести великую жертву Дагону, богу своему, и повеселиться, и сказали: бог наш предал Самсона, врага нашего, в руки наши.
\vs Jdg 16:24 Также и народ, видя его, прославлял бога своего, говоря: бог наш предал в руки наши врага нашего и опустошителя земли нашей, который побил многих из нас.
\vs Jdg 16:25 И когда развеселилось сердце их, сказали: позовите Самсона [из дома темничного], пусть он позабавит нас. И призвали Самсона из дома узников, и он забавлял их, [и заушали его] и поставили его между столбами.
\vs Jdg 16:26 И сказал Самсон отроку, который водил его за руку: подведи меня, чтобы ощупать мне столбы, на которых утвержден дом, и прислониться к ним. [Отрок так и сделал.]
\vs Jdg 16:27 Дом же был полон мужчин и женщин; там были все владельцы Филистимские, и на кровле было до трех тысяч мужчин и женщин, смотревших на забавляющего \bibemph{их} Самсона.
\vs Jdg 16:28 И воззвал Самсон к Господу и сказал: Господи Боже! вспомни меня и укрепи меня только теперь, о Боже! чтобы мне в один раз отмстить Филистимлянам за два глаза мои.
\vs Jdg 16:29 И сдвинул Самсон с места два средних столба, на которых утвержден был дом, упершись в них, в один правою рукою своею, а в другой левою.
\vs Jdg 16:30 И сказал Самсон: умри, душа моя, с Филистимлянами! И уперся \bibemph{всею} силою, и обрушился дом на владельцев и на весь народ, бывший в нем. И было умерших, которых умертвил [Самсон] при смерти своей, более, нежели сколько умертвил он в жизни своей.
\vs Jdg 16:31 И пришли братья его и весь дом отца его, и взяли его, и пошли и похоронили его между Цорою и Естаолом, во гробе Маноя, отца его. Он был судьею Израиля двадцать лет. [После Самсона восстал Емегар, сын Енана, и убил из иноплеменников шестьсот человек, кроме скота. И он спас Израиля.]
\vs Jdg 17:1 Был некто на горе Ефремовой, именем Миха.
\vs Jdg 17:2 Он сказал матери своей: тысяча сто \bibemph{сиклей} серебра, которые у тебя взяты и за которые ты при мне изрекла проклятие, это серебро у меня, я взял его. Мать его сказала: благословен сын мой у Господа!
\vs Jdg 17:3 И возвратил он матери своей тысячу сто \bibemph{сиклей} серебра. И сказала мать его: это серебро я от себя посвятила Господу для [тебя,] сына моего, чтобы сделать из него истукан и литой кумир; итак отдаю оное тебе.
\vs Jdg 17:4 Но он возвратил серебро матери своей. Мать его взяла двести \bibemph{сиклей} серебра и отдала их плавильщику. Он сделал из них истукан и литой кумир, который и находился в доме Михи.
\vs Jdg 17:5 И был у Михи дом Божий. И сделал он ефод и терафим и посвятил одного из сыновей своих, чтоб он был у него священником.
\rsbpar\vs Jdg 17:6 В те дни не было царя у Израиля; каждый делал то, что ему казалось справедливым.
\vs Jdg 17:7 Один юноша из Вифлеема Иудейского, из колена Иудина, левит, тогда жил там;
\vs Jdg 17:8 этот человек пошел из города Вифлеема Иудейского, чтобы пожить, где случится, и идя дорогою, пришел на гору Ефремову к дому Михи.
\vs Jdg 17:9 И сказал ему Миха: откуда ты идешь? Он сказал ему: я левит из Вифлеема Иудейского и иду пожить, где случится.
\vs Jdg 17:10 И сказал ему Миха: останься у меня и будь у меня отцом и священником; я буду давать тебе по десяти \bibemph{сиклей} серебра на год, потребное одеяние и пропитание.
\vs Jdg 17:11 Левит пошел к нему и согласился левит остаться у этого человека, и был юноша у него, как один из сыновей его.
\vs Jdg 17:12 Миха посвятил левита, и этот юноша был у него священником и жил в доме у Михи.
\vs Jdg 17:13 И сказал Миха: теперь я знаю, что Господь будет мне благотворить, потому что левит у меня священником.
\vs Jdg 18:1 В те дни не было царя у Израиля; и в те дни колено Даново искало себе удела, где бы поселиться, потому что дотоле не выпало ему \bibemph{полного} удела между коленами Израилевыми.
\vs Jdg 18:2 И послали сыны Дановы от племени своего пять человек, мужей сильных, из Цоры и Естаола, чтоб осмотреть землю и узнать ее, и сказали им: пойдите, узнайте землю. Они пришли на гору Ефремову к дому Михи и ночевали там.
\vs Jdg 18:3 Находясь у дома Михи, узнали они голос молодого левита и зашли туда и спрашивали его: кто тебя привел сюда? что ты здесь делаешь и зачем ты здесь?
\vs Jdg 18:4 Он сказал им: то и то сделал для меня Миха, нанял меня, и я у него священником.
\vs Jdg 18:5 Они сказали ему: вопроси Бога, чтобы знать нам, успешен ли будет путь наш, в который мы идем.
\vs Jdg 18:6 Священник сказал им: идите с миром; пред Господом путь ваш, в который вы идете.
\vs Jdg 18:7 И пошли те пять мужей, и пришли в Лаис, и увидели народ, который в нем, что он живет покойно, по обычаю Сидонян, тих и беспечен, и что не было в земле той, кто обижал бы в чем, или имел бы власть: от Сидонян они жили далеко, и ни с кем не было у них никакого дела.
\vs Jdg 18:8 И возвратились [оные пять человек] к братьям своим в Цору и Естаол, и сказали им братья их: с чем вы?
\vs Jdg 18:9 Они сказали: встанем и пойдем на них; мы видели землю, она весьма хороша; а вы задумались: не медлите пойти и взять в наследие ту землю;
\vs Jdg 18:10 когда пойдете вы, придете к народу беспечному, и земля та обширна; Бог предает ее в руки ваши; это такое место, где нет ни в чем недостатка, что \bibemph{получается} от земли.
\vs Jdg 18:11 И отправились оттуда из колена Данова, из Цоры и Естаола, шестьсот мужей, препоясавшись воинским оружием.
\vs Jdg 18:12 Они пошли и стали станом в Кириаф-Иариме, в Иудее. Посему и называют то место станом Дановым до сего дня. Он позади Кириаф-Иарима.
\vs Jdg 18:13 Оттуда отправились они на гору Ефремову и пришли к дому Михи.
\vs Jdg 18:14 И сказали те пять мужей, которые ходили осматривать землю Лаис, братьям своим: знаете ли, что в одном из домов сих есть ефод, терафим, истукан и литой кумир? итак подумайте, что сделать.
\vs Jdg 18:15 И зашли туда, и вошли в дом молодого левита, в дом Михи, и приветствовали его.
\vs Jdg 18:16 А шестьсот человек из сынов Дановых, перепоясанные воинским оружием, стояли у ворот.
\vs Jdg 18:17 Пять же человек, ходивших осматривать землю, пошли, вошли туда, взяли истукан и ефод и терафим и литой кумир. Священник стоял у ворот с теми шестьюстами человек, препоясанных воинским оружием.
\vs Jdg 18:18 Когда они вошли в дом Михи и взяли истукан, ефод, терафим и литой кумир, священник сказал им: что вы делаете?
\vs Jdg 18:19 Они сказали ему: молчи, положи руку твою на уста твои и иди с нами и будь у нас отцом и священником; лучше ли тебе быть священником в доме одного человека, нежели быть священником в колене или в племени Израилевом?
\vs Jdg 18:20 Священник обрадовался, и взял ефод, терафим и истукан [и литой кумир], и пошел с народом.
\vs Jdg 18:21 Они обратились и пошли, и отпустили детей, скот и тяжести вперед.
\vs Jdg 18:22 Когда они удалились от дома Михи, [Миха и] жители домов соседних с домом Михи собрались и погнались за сынами Дана,
\vs Jdg 18:23 и кричали сынам Дана. [Сыны Дановы] оборотились и сказали Михе: что тебе, что ты так кричишь?
\vs Jdg 18:24 [Миха] сказал: вы взяли богов моих, которых я сделал, и священника, и ушли; чего еще более? как же вы говорите: что тебе?
\vs Jdg 18:25 Сыны Дановы сказали ему: \bibemph{молчи}, чтобы мы не слышали голоса твоего; иначе некоторые из нас, рассердившись, нападут на вас, и ты погубишь себя и семейство твое.
\vs Jdg 18:26 И пошли сыны Дановы путем своим; Миха же, видя, что они сильнее его, пошел назад и возвратился в дом свой.
\vs Jdg 18:27 А [сыны Дановы] взяли то, что сделал Миха, и священника, который был у него, и пошли в Лаис, против народа спокойного и беспечного, и побили его мечом, а город сожгли огнем.
\vs Jdg 18:28 Некому было помочь, потому что он был отдален от Сидона и ни с кем не имел дела. [Город сей] находился в долине, что близ Беф-Рехова. И построили \bibemph{снова} город и поселились в нем,
\vs Jdg 18:29 и нарекли имя городу: Дан, по имени отца своего Дана, сына Израилева; а прежде имя городу тому было: Лаис.
\vs Jdg 18:30 И поставили у себя сыны Дановы истукан; Ионафан же, сын Гирсона, сына Манассии, сам и сыновья его были священниками в колене Дановом до дня переселения \bibemph{жителей той} земли;
\vs Jdg 18:31 и имели у себя истукан, сделанный Михою, во все то время, когда дом Божий находился в Силоме.
\vs Jdg 19:1 В те дни, когда не было царя у Израиля, жил один левит на склоне горы Ефремовой. Он взял себе наложницу из Вифлеема Иудейского.
\vs Jdg 19:2 Наложница его поссорилась с ним и ушла от него в дом отца своего в Вифлеем Иудейский и была там четыре месяца.
\vs Jdg 19:3 Муж ее встал и пошел за нею, чтобы поговорить к сердцу ее и возвратить ее к себе. С ним был слуга его и пара ослов. Она ввела его в дом отца своего.
\vs Jdg 19:4 Отец этой молодой женщины, увидев его, с радостью встретил его, и удержал его тесть его, отец молодой женщины. И пробыл он у него три дня; они ели и пили и ночевали там.
\vs Jdg 19:5 В четвертый день встали они рано, и он встал, чтоб идти. И сказал отец молодой женщины зятю своему: подкрепи сердце твое куском хлеба, и потом пойдете.
\vs Jdg 19:6 Они остались, и оба вместе ели и пили. И сказал отец молодой женщины человеку тому: останься еще на ночь, и пусть повеселится сердце твое.
\vs Jdg 19:7 Человек тот встал, было, чтоб идти, но тесть его упросил его, и он опять ночевал там.
\vs Jdg 19:8 На пятый день встал он поутру, чтоб идти. И сказал отец молодой женщины той: подкрепи сердце твое [хлебом], и помедлите, доколе преклонится день. И ели оба они [и пили].
\vs Jdg 19:9 И встал тот человек, чтоб идти, сам он, наложница его и слуга его. И сказал ему тесть его, отец молодой женщины: вот, день преклонился к вечеру, ночуйте, пожалуйте; вот, дню скоро конец, ночуй здесь, пусть повеселится сердце твое; завтра пораньше встанете в путь ваш, и пойдешь в дом твой.
\vs Jdg 19:10 Но муж не согласился ночевать, встал и пошел; и пришел к Иевусу, что \bibemph{ныне} Иерусалим; с ним пара навьюченных ослов и наложница его с ним.
\vs Jdg 19:11 Когда они были близ Иевуса, день уже очень преклонился. И сказал слуга господину своему: зайдем в этот город Иевусеев и ночуем в нем.
\vs Jdg 19:12 Господин его сказал ему: нет, не пойдем в город иноплеменников, которые не из сынов Израилевых, но дойдем до Гивы.
\vs Jdg 19:13 И сказал слуге своему: дойдем до одного из сих мест и ночуем в Гиве, или в Раме.
\vs Jdg 19:14 И пошли, и шли, и закатилось солнце подле Гивы Вениаминовой.
\vs Jdg 19:15 И повернули они туда, чтобы пойти ночевать в Гиве. И пришел он и сел на улице в городе; но никто не приглашал их в дом для ночлега.
\vs Jdg 19:16 И вот, идет один старик с работы своей с поля вечером; он родом был с горы Ефремовой и жил в Гиве. Жители же места сего были сыны Вениаминовы.
\vs Jdg 19:17 Он, подняв глаза свои, увидел прохожего на улице городской. И сказал старик: куда идешь? и откуда ты пришел?
\vs Jdg 19:18 Он сказал ему: мы идем из Вифлеема Иудейского к горе Ефремовой, откуда я; я ходил в Вифлеем Иудейский, а теперь иду к дому Господа; и никто не приглашает меня в дом;
\vs Jdg 19:19 у нас есть и солома и корм для ослов наших; также хлеб и вино для меня и для рабы твоей и для сего слуги есть у рабов твоих; ни в чем нет недостатка.
\vs Jdg 19:20 Старик сказал ему: будь спокоен: весь недостаток твой на мне, только не ночуй на улице.
\vs Jdg 19:21 И ввел его в дом свой и дал корму ослам [его], а сами они омыли ноги свои и ели и пили.
\vs Jdg 19:22 Тогда как они развеселили сердца свои, вот, жители города, люди развратные, окружили дом, стучались в двери и говорили старику, хозяину дома: выведи человека, вошедшего в дом твой, мы познаем его.
\vs Jdg 19:23 Хозяин дома вышел к ним и сказал им: нет, братья мои, не делайте зла, когда человек сей вошел в дом мой, не делайте этого безумия;
\vs Jdg 19:24 вот у меня дочь девица, и у него наложница, выведу я их, смирите их и делайте с ними, что вам угодно; а с человеком сим не делайте этого безумия.
\vs Jdg 19:25 Но они не хотели слушать его. Тогда муж взял свою наложницу и вывел к ним на улицу. Они познали ее, и ругались над нею всю ночь до утра. И отпустили ее при появлении зари.
\vs Jdg 19:26 И пришла женщина пред появлением зари, и упала у дверей дома того человека, у которого был господин ее, \bibemph{и лежала} до света.
\vs Jdg 19:27 Господин ее встал поутру, отворил двери дома и вышел, чтоб идти в путь свой: и вот, наложница его лежит у дверей дома, и руки ее на пороге.
\vs Jdg 19:28 Он сказал ей: вставай, пойдем. Но ответа не было, [потому что она умерла]. Он положил ее на осла, встал и пошел в свое место.
\vs Jdg 19:29 Придя в дом свой, взял нож и, взяв наложницу свою, разрезал ее по членам ее на двенадцать частей и послал во все пределы Израилевы.
\vs Jdg 19:30 Всякий, видевший это, говорил: не бывало и не видано было подобного сему от дня исшествия сынов Израилевых из земли Египетской до сего дня. [Посланным же от себя людям он дал приказание и сказал: так говорите всему Израилю: бывало ли когда подобное сему?] Обратите внимание на это, посоветуйтесь и скажите.
\vs Jdg 20:1 И вышли все сыны Израилевы, и собралось \bibemph{все} общество, как один человек, от Дана до Вирсавии, и земля Галаадская пред Господа в Массифу.
\vs Jdg 20:2 И собрались [пред Господа] начальники всего народа, все колена Израилевы, в собрание народа Божия, четыреста тысяч пеших, обнажающих меч.
\vs Jdg 20:3 И сыны Вениаминовы услышали, что сыны Израилевы пришли в Массифу. И сказали сыны Израилевы: скажите, как происходило это зло?
\vs Jdg 20:4 Левит, муж оной убитой женщины, отвечал и сказал: я с наложницею моею пришел ночевать в Гиву Вениаминову;
\vs Jdg 20:5 и восстали на меня жители Гивы и окружили из-за меня дом ночью; меня намеревались убить, и наложницу мою замучили, [надругавшись над нею,] так, что она умерла;
\vs Jdg 20:6 я взял наложницу мою, разрезал ее и послал ее во все области владения Израилева, ибо они сделали беззаконное и срамное дело в Израиле;
\vs Jdg 20:7 вот все вы, сыны Израилевы, рассмотрите это дело и решите здесь.
\vs Jdg 20:8 И восстал весь народ, как один человек, и сказал: не пойдем никто в шатер свой и не возвратимся никто в дом свой;
\vs Jdg 20:9 и вот что мы сделаем ныне с Гивою: [пойдем] на нее по жребию;
\vs Jdg 20:10 и возьмем по десяти человек из ста от всех колен Израилевых, по сто от тысячи и по тысяче от тьмы, чтоб они принесли съестных припасов для народа, который пойдет против Гивы Вениаминовой, наказать ее за срамное дело, которое она сделала в Израиле.
\vs Jdg 20:11 И собрались все Израильтяне против города единодушно, как один человек.
\vs Jdg 20:12 И послали колена Израилевы во все колено Вениаминово сказать: какое это гнусное дело сделано у вас!
\vs Jdg 20:13 Выдайте развращенных оных людей, которые в Гиве; мы умертвим их и искореним зло из Израиля. Но сыны Вениаминовы не хотели послушать голоса братьев своих, сынов Израилевых;
\vs Jdg 20:14 а собрались сыны Вениаминовы из городов в Гиву, чтобы пойти войною против сынов Израилевых.
\vs Jdg 20:15 И насчиталось в тот день сынов Вениаминовых, \bibemph{собравшихся} из городов, двадцать шесть тысяч человек, обнажающих меч; кроме того, из жителей Гивы насчитано семьсот отборных;
\vs Jdg 20:16 из всего народа сего было семьсот человек отборных, которые были левши, и все сии, бросая из пращей камни в волос, не бросали мимо.
\vs Jdg 20:17 Израильтян же, кроме сынов Вениаминовых, насчиталось четыреста тысяч человек, обнажающих меч; все они были способны к войне.
\vs Jdg 20:18 И встали и пошли в дом Божий, и вопрошали Бога и сказали сыны Израилевы: кто из нас прежде пойдет на войну с сынами Вениамина? И сказал Господь: Иуда [пойдет] впереди.
\vs Jdg 20:19 И встали сыны Израилевы поутру и расположились станом подле Гивы;
\vs Jdg 20:20 и выступили Израильтяне на войну против Вениамина, и стали сыны Израилевы в боевой порядок близ Гивы.
\vs Jdg 20:21 И вышли сыны Вениаминовы из Гивы и положили в тот день двадцать две тысячи Израильтян на землю.
\vs Jdg 20:22 Но народ Израильский ободрился, и опять стали в боевой порядок на том месте, где стояли в прежний день.
\vs Jdg 20:23 И пошли сыны Израилевы, и плакали пред Господом до вечера, и вопрошали Господа: вступать ли мне еще в сражение с сынами Вениамина, брата моего? Господь сказал: идите против него.
\vs Jdg 20:24 И подступили сыны Израилевы к сынам Вениамина во второй день.
\vs Jdg 20:25 Вениамин вышел против них из Гивы во второй день, и еще положили на землю из сынов Израилевых восемнадцать тысяч человек, обнажающих меч.
\rsbpar\vs Jdg 20:26 Тогда все сыны Израилевы и весь народ пошли и пришли в дом Божий и, сидя там, плакали пред Господом, и постились в тот день до вечера, и вознесли всесожжения и мирные жертвы пред Господом.
\vs Jdg 20:27 И вопрошали сыны Израилевы Господа (в то время ковчег завета Божия находился там,
\vs Jdg 20:28 и Финеес, сын Елеазара, сына Ааронова, предстоял пред ним): выходить ли мне еще на сражение с сынами Вениамина, брата моего, или нет? Господь сказал: идите; Я завтра предам его в руки ваши.
\rsbpar\vs Jdg 20:29 И поставил Израиль засаду вокруг Гивы.
\vs Jdg 20:30 И пошли сыны Израилевы на сынов Вениамина в третий день и стали в боевой порядок пред Гивою, как прежде.
\vs Jdg 20:31 Сыны Вениаминовы выступили против народа и отдалились от города, и начали, как прежде, убивать из народа на дорогах, из которых одна идет к Вефилю, а другая к Гиве полем, и \bibemph{убили} до тридцати человек из Израильтян.
\vs Jdg 20:32 И сказали сыны Вениаминовы: они падают пред нами, как и прежде. А сыны Израилевы сказали: побежим от них и отвлечем их от города на дороги. [И сделали так.]
\vs Jdg 20:33 И все Израильтяне встали с своего места и выстроились в Ваал-Фамаре. И засада Израилева устремилась из своего места, с западной стороны Гивы.
\vs Jdg 20:34 И пришли пред Гиву десять тысяч человек отборных из всего Израиля, и началось жестокое сражение; но \bibemph{сыны Вениамина} не знали, что предстоит им беда.
\vs Jdg 20:35 И поразил Господь Вениамина пред Израильтянами, и положили в тот день Израильтяне из сынов Вениамина двадцать пять тысяч сто человек, обнажавших меч.
\vs Jdg 20:36 Когда сыны Вениамина увидели, что они поражены, тогда Израильтяне уступили место сынам Вениамина, ибо надеялись на засаду, которую они поставили близ Гивы.
\vs Jdg 20:37 Засада же поспешила и устремилась к Гиве, и вступила и поразила весь город мечом.
\vs Jdg 20:38 Израильтяне поставили с засадою \bibemph{условленным} знаком к нападению поднимающийся дым из города.
\vs Jdg 20:39 Итак, когда Израильтяне отступили с места сражения, и Вениамин начал поражать и поверг Израильтян до тридцати человек и говорил: <<опять падают они пред нами, как и в прежние сражения>>,
\vs Jdg 20:40 тогда начал подниматься из города дым столбом. Вениамин оглянулся назад, и вот, \bibemph{дым} от всего города восходит к небу.
\vs Jdg 20:41 Израильтяне воротились, а Вениамин оробел, ибо увидел, что постигла его беда.
\vs Jdg 20:42 И побежали они от Израильтян по дороге к пустыне; но сеча преследовала их, и выходившие из городов побивали их там;
\vs Jdg 20:43 окружили Вениамина, и преследовали его до Менухи и поражали до самой восточной стороны Гивы.
\vs Jdg 20:44 И пало из сынов Вениамина восемнадцать тысяч человек, людей сильных.
\vs Jdg 20:45 [Оставшиеся] оборотились и побежали к пустыне, к скале Риммону, и побили еще [Израильтяне] на дорогах пять тысяч человек; и гнались за ними до Гидома и еще убили из них две тысячи человек.
\vs Jdg 20:46 Всех же сынов Вениаминовых, павших в тот день, было двадцать пять тысяч человек, обнажавших меч, и все они были мужи сильные.
\vs Jdg 20:47 И обратились [оставшиеся] и убежали в пустыню, к скале Риммону, шестьсот человек, и оставались там в каменной горе Риммоне четыре месяца.
\vs Jdg 20:48 Израильтяне же опять пошли к сынам Вениаминовым и поразили их мечом, и людей в городе, и скот, и все, что ни встречалось [во всех городах], и все находившиеся \bibemph{на пути} города сожгли огнем.
\vs Jdg 21:1 И поклялись Израильтяне в Массифе, говоря: никто из нас не отдаст дочери своей сынам Вениамина в замужество.
\vs Jdg 21:2 И пришел народ в дом Божий, и сидели там до вечера пред Богом, и подняли громкий вопль, и сильно плакали,
\vs Jdg 21:3 и сказали: Господи, Боже Израилев! для чего случилось это в Израиле, что не стало теперь у Израиля одного колена?
\vs Jdg 21:4 На другой день встал народ поутру, и устроили там жертвенник, и вознесли всесожжения и мирные жертвы.
\vs Jdg 21:5 И сказали сыны Израилевы: кто не приходил в собрание пред Господа из всех колен Израилевых? Ибо великое проклятие \bibemph{произнесено} было на тех, которые не пришли пред Господа в Массифу, и сказано было, что те преданы будут смерти.
\vs Jdg 21:6 И сжалились сыны Израилевы над Вениамином, братом своим, и сказали: ныне отсечено одно колено от Израиля;
\vs Jdg 21:7 как поступить нам с оставшимися из них \bibemph{касательно} жен, когда мы поклялись Господом не давать им жен из дочерей наших?
\vs Jdg 21:8 И сказали: нет ли кого из колен Израилевых, кто не приходил пред Господа в Массифу? И оказалось, что из Иависа Галаадского никто не приходил пред Господа в стан на собрание.
\vs Jdg 21:9 И осмотрен народ, и вот, не было там ни одного из жителей Иависа Галаадского.
\vs Jdg 21:10 И послало туда общество двенадцать тысяч человек, мужей сильных, и дали им приказание, говоря: идите и поразите жителей Иависа Галаадского мечом, и женщин и детей;
\vs Jdg 21:11 и вот что сделайте: всякого мужчину и всякую женщину, познавшую ложе мужеское, предайте заклятию, [а девиц оставляйте в живых. И сделали так].
\vs Jdg 21:12 И нашли они между жителями Иависа Галаадского четыреста девиц, не познавших ложа мужеского, и привели их в стан в Силом, что в земле Ханаанской.
\vs Jdg 21:13 И послало все общество переговорить с сынами Вениамина, бывшими в скале Риммоне, и объявило им мир.
\vs Jdg 21:14 Тогда возвратились сыны Вениамина [к Израильтянам], и дали им [Израильтяне] жен, которых оставили в живых из женщин Иависа Галаадского; но оказалось, что этого было недостаточно.
\vs Jdg 21:15 Народ же сожалел о Вениамине, что Господь не сохранил целости колен Израилевых.
\vs Jdg 21:16 И сказали старейшины общества: что нам делать с оставшимися \bibemph{касательно} жен, ибо истреблены женщины у Вениамина?
\vs Jdg 21:17 И сказали: наследственная земля пусть остается уцелевшим сынам Вениамина, чтобы не исчезло колено от Израиля;
\vs Jdg 21:18 но мы не можем дать им жен из дочерей наших; ибо сыны Израилевы поклялись, говоря: проклят, кто даст жену Вениамину.
\vs Jdg 21:19 И сказали: вот, каждый год бывает праздник Господень в Силоме, который на север от Вефиля и на восток от дороги, ведущей от Вефиля в Сихем, и на юг от Левоны.
\vs Jdg 21:20 И приказали сынам Вениамина и сказали: пойдите и засядьте в виноградниках,
\vs Jdg 21:21 и смотрите, когда выйдут девицы Силомские плясать в хороводах, тогда выйдите из виноградников и схватите себе каждый жену из девиц Силомских и идите в землю Вениаминову;
\vs Jdg 21:22 и когда придут отцы их, или братья их с жалобою к нам, мы скажем им: простите нас за них, ибо мы не взяли для каждого из них жены на войне, и вы не дали им; теперь вы виновны.
\vs Jdg 21:23 Сыны Вениамина так и сделали, и взяли жен по числу своему из бывших в хороводе, которых они похитили, и пошли и возвратились в удел свой, и построили города и стали жить в них.
\vs Jdg 21:24 В то же время Израильтяне разошлись оттуда каждый в колено свое и в племя свое, и пошли оттуда каждый в удел свой.
\rsbpar\vs Jdg 21:25 В те дни не было царя у Израиля; каждый делал то, что ему казалось справедливым.

\bibbookdescr{Rut}{
  inline={\LARGE Книга\\\Huge Руфь},
  toc={Руфь},
  bookmark={Руфь},
  header={Руфь},
  %headerleft={},
  %headerright={},
  abbr={Руфь}
}
\vs Rut 1:1 В те дни, когда управляли судьи, случился голод на земле. И пошел один человек из Вифлеема Иудейского со своею женою и двумя сыновьями своими жить на полях Моавитских.
\vs Rut 1:2 Имя человека того Елимелех, имя жены его Ноеминь, а имена двух сынов его Махлон и Хилеон; \bibemph{они были} Ефрафяне из Вифлеема Иудейского. И пришли они на поля Моавитские и остались там.
\vs Rut 1:3 И умер Елимелех, муж Ноемини, и осталась она с двумя сыновьями своими.
\vs Rut 1:4 Они взяли себе жен из Моавитянок, имя одной Орфа, а имя другой Руфь, и жили там около десяти лет.
\vs Rut 1:5 Но потом и оба [сына ее], Махлон и Хилеон, умерли, и осталась та женщина после обоих своих сыновей и после мужа своего.
\vs Rut 1:6 И встала она со снохами своими и пошла обратно с полей Моавитских, ибо услышала на полях Моавитских, что Бог посетил народ Свой и дал им хлеб.
\vs Rut 1:7 И вышла она из того места, в котором жила, и обе снохи ее с нею. Когда они шли по дороге, возвращаясь в землю Иудейскую,
\vs Rut 1:8 Ноеминь сказала двум снохам своим: пойдите, возвратитесь каждая в дом матери своей; да сотворит Господь с вами милость, как вы поступали с умершими и со мною!
\vs Rut 1:9 да даст вам Господь, чтобы вы нашли пристанище каждая в доме своего мужа! И поцеловала их. Но они подняли вопль и плакали
\vs Rut 1:10 и сказали: нет, мы с тобою возвратимся к народу твоему.
\vs Rut 1:11 Ноеминь же сказала: возвратитесь, дочери мои; зачем вам идти со мною? Разве еще есть у меня сыновья в моем чреве, которые были бы вам мужьями?
\vs Rut 1:12 Возвратитесь, дочери мои, пойдите, ибо я уже стара, чтоб быть замужем. Да если б я и сказала: <<есть мне еще надежда>>, и даже если бы я сию же ночь была с мужем и потом родила сыновей,~---
\vs Rut 1:13 то можно ли вам ждать, пока они выросли бы? можно ли вам медлить и не выходить замуж? Нет, дочери мои, я весьма сокрушаюсь о вас, ибо рука Господня постигла меня.
\vs Rut 1:14 Они подняли вопль и опять стали плакать. И Орфа простилась со свекровью своею [и возвратилась к народу своему], а Руфь осталась с нею.
\vs Rut 1:15 [Ноеминь] сказала [Руфи]: вот, невестка твоя возвратилась к народу своему и к своим богам; возвратись и ты вслед за невесткою твоею.
\vs Rut 1:16 Но Руфь сказала: не принуждай меня оставить тебя и возвратиться от тебя; но куда ты пойдешь, туда и я пойду, и где ты жить будешь, там и я буду жить; народ твой будет моим народом, и твой Бог~--- моим Богом;
\vs Rut 1:17 и где ты умрешь, там и я умру и погребена буду; пусть то и то сделает мне Господь, и еще больше сделает; смерть одна разлучит меня с тобою.
\vs Rut 1:18 [Ноеминь,] видя, что она твердо решилась идти с нею, перестала уговаривать ее.
\vs Rut 1:19 И шли обе они, доколе не пришли в Вифлеем. Когда пришли они в Вифлеем, весь город пришел в движение от них, и говорили: это Ноеминь?
\vs Rut 1:20 Она сказала им: не называйте меня Ноеминью\fns{Приятная.}, а называйте меня Марою\fns{Горькая.}, потому что Вседержитель послал мне великую горесть;
\vs Rut 1:21 я вышла отсюда с достатком, а возвратил меня Господь с пустыми руками; зачем называть меня Ноеминью, когда Господь заставил меня страдать, и Вседержитель послал мне несчастье?
\vs Rut 1:22 И возвратилась Ноеминь, и с нею сноха ее Руфь Моавитянка, пришедшая с полей Моавитских, и пришли они в Вифлеем в начале жатвы ячменя.
\vs Rut 2:1 У Ноемини был родственник по мужу ее, человек весьма знатный, из племени Елимелехова, имя ему Вооз.
\vs Rut 2:2 И сказала Руфь Моавитянка Ноемини: пойду я на поле и буду подбирать колосья по следам того, у кого найду благоволение. Она сказала ей: пойди, дочь моя.
\vs Rut 2:3 Она пошла, и пришла, и подбирала в поле \bibemph{колосья} позади жнецов. И случилось, что та часть поля принадлежала Воозу, который из племени Елимелехова.
\vs Rut 2:4 И вот, Вооз пришел из Вифлеема и сказал жнецам: Господь с вами! Они сказали ему: да благословит тебя Господь!
\vs Rut 2:5 И сказал Вооз слуге своему, приставленному к жнецам: чья это молодая женщина?
\vs Rut 2:6 Слуга, приставленный к жнецам, отвечал и сказал: эта молодая женщина~--- Моавитянка, пришедшая с Ноеминью с полей Моавитских;
\vs Rut 2:7 она сказала: <<буду я подбирать и собирать между снопами позади жнецов>>; и пришла, и находится \bibemph{здесь} с самого утра доселе; мало бывает она дома.
\vs Rut 2:8 И сказал Вооз Руфи: послушай, дочь моя, не ходи подбирать на другом поле и не переходи отсюда, но будь здесь с моими служанками;
\vs Rut 2:9 пусть в глазах твоих будет то поле, где они жнут, и ходи за ними; вот, я приказал слугам моим не трогать тебя; когда захочешь пить, иди к сосудам и пей, откуда черпают слуги мои.
\vs Rut 2:10 Она пала на лице свое и поклонилась до земли и сказала ему: чем снискала я в глазах твоих милость, что ты принимаешь меня, хотя я и чужеземка?
\vs Rut 2:11 Вооз отвечал и сказал ей: мне сказано все, что сделала ты для свекрови своей по смерти мужа твоего, что ты оставила твоего отца и твою мать и твою родину и пришла к народу, которого ты не знала вчера и третьего дня;
\vs Rut 2:12 да воздаст Господь за это дело твое, и да будет тебе полная награда от Господа Бога Израилева, к Которому ты пришла, чтоб успокоиться под Его крылами!
\vs Rut 2:13 Она сказала: да буду я в милости пред очами твоими, господин мой! Ты утешил меня и говорил по сердцу рабы твоей, между тем как я не ст\acc{о}ю ни одной из рабынь твоих.
\vs Rut 2:14 И сказал ей Вооз: время обеда; приди сюда и ешь хлеб и обмакивай кусок твой в уксус. И села она возле жнецов. Он подал ей хлеба; она ела, наелась, и еще осталось.
\vs Rut 2:15 И встала, чтобы подбирать. Вооз дал приказ слугам своим, сказав: пусть подбирает она и между снопами, и не обижайте ее;
\vs Rut 2:16 да и от снопов откидывайте ей и оставляйте, пусть она подбирает [и ест], и не браните ее.
\vs Rut 2:17 Так подбирала она на поле до вечера и вымолотила собранное, и вышло около ефы ячменя.
\vs Rut 2:18 Взяв это, она пошла в город, и свекровь ее увидела, что она набрала. И вынула [Руфь из пазухи своей] и дала ей то, что оставила, наевшись сама.
\vs Rut 2:19 И сказала ей свекровь ее: где ты собирала сегодня и где работала? да будет благословен принявший тебя! [Руфь] объявила свекрови своей, у кого она работала, и сказала: человеку тому, у которого я сегодня работала, имя Вооз.
\vs Rut 2:20 И сказала Ноеминь снохе своей: благословен он от Господа за то, что не лишил милости своей ни живых, ни мертвых! И сказала ей Ноеминь: человек этот близок к нам; он из наших родственников.
\vs Rut 2:21 Руфь Моавитянка сказала [свекрови своей]: он даже сказал мне: будь с моими служанками, доколе не докончат они жатвы моей.
\vs Rut 2:22 И сказала Ноеминь снохе своей Руфи: хорошо, дочь моя, что ты будешь ходить со служанками его, и не будут оскорблять тебя на другом поле.
\vs Rut 2:23 Так была она со служанками Воозовыми и подбирала [колосья], доколе не кончилась жатва ячменя и жатва пшеницы, и жила у свекрови своей.
\vs Rut 3:1 И сказала ей Ноеминь, свекровь ее: дочь моя, не поискать ли тебе пристанища, чтобы тебе хорошо было?
\vs Rut 3:2 Вот, Вооз, со служанками которого ты была, родственник наш; вот, он в эту ночь веет на гумне ячмень;
\vs Rut 3:3 умойся, помажься, надень на себя [нарядные] одежды твои и пойди на гумно, но не показывайся ему, доколе не кончит есть и пить;
\vs Rut 3:4 когда же он ляжет спать, узнай место, где он ляжет; тогда придешь и откроешь у ног его и ляжешь; он скажет тебе, что тебе делать.
\vs Rut 3:5 [Руфь] сказала ей: сделаю все, что ты сказала мне.
\vs Rut 3:6 И пошла на гумно и сделала все так, как приказывала ей свекровь ее.
\vs Rut 3:7 Вооз наелся и напился, и развеселил сердце свое, и пошел \bibemph{и лег} спать подле скирда. И она пришла тихонько, открыла у ног его и легла.
\vs Rut 3:8 В полночь он содрогнулся, приподнялся, и вот, у ног его лежит женщина.
\vs Rut 3:9 И сказал [ей Вооз]: кто ты? Она сказала: я Руфь, раба твоя, простри крыло твое на рабу твою, ибо ты родственник.
\vs Rut 3:10 [Вооз] сказал: благословенна ты от Господа [Бога], дочь моя! это последнее твое доброе дело сделала ты еще лучше прежнего, что ты не пошла искать молодых людей, ни бедных, ни богатых;
\vs Rut 3:11 итак, дочь моя, не бойся, я сделаю тебе все, что ты сказала; ибо у всех ворот народа моего знают, что ты женщина добродетельная;
\vs Rut 3:12 хотя и правда, что я родственник, но есть еще родственник ближе меня;
\vs Rut 3:13 переночуй эту ночь; завтра же, если он примет тебя, то хорошо, пусть примет; а если он не захочет принять тебя, то я приму; жив Господь! Спи до утра.
\vs Rut 3:14 И спала она у ног его до утра и встала прежде, нежели могли они распознать друг друга. И сказал Вооз: пусть не знают, что женщина приходила на гумно.
\vs Rut 3:15 И сказал ей: подай верхнюю одежду, которая на тебе, подержи ее. Она держала, и он отмерил [ей] шесть мер ячменя, и положил на нее, и пошел в город.
\vs Rut 3:16 А [Руфь] пришла к свекрови своей. Та сказала [ей]: что, дочь моя? Она пересказала ей все, что сделал ей человек тот.
\vs Rut 3:17 И сказала [ей]: эти шесть мер ячменя он дал мне и сказал мне: не ходи к свекрови своей с пустыми руками.
\vs Rut 3:18 Та сказала: подожди, дочь моя, доколе не узнаешь, чем кончится дело; ибо человек тот не останется в покое, не кончив сегодня дела.
\vs Rut 4:1 Вооз вышел к воротам и сидел там. И вот, идет мимо родственник, о котором говорил Вооз. И сказал ему [Вооз]: зайди сюда и сядь здесь. Тот зашел и сел.
\vs Rut 4:2 [Вооз] взял десять человек из старейшин города и сказал: сядьте здесь. И они сели.
\vs Rut 4:3 И сказал [Вооз] родственнику: Ноеминь, возвратившаяся с полей Моавитских, продает часть поля, принадлежащую брату нашему Елимелеху;
\vs Rut 4:4 я решился довести до ушей твоих и сказать: купи при сидящих здесь и при старейшинах народа моего; если хочешь выкупить, выкуп\acc{а}й; а если не хочешь выкупить, скажи мне, и я буду знать; ибо кроме тебя некому выкупить; а по тебе я. Тот сказал: я выкуп\acc{а}ю.
\vs Rut 4:5 Вооз сказал: когда ты купишь поле у Ноемини, то должен купить и у Руфи Моавитянки, жены умершего, и должен взять ее в замужество, чтобы восстановить имя умершего в уделе его.
\vs Rut 4:6 И сказал тот родственник: не могу я взять ее себе, чтобы не расстроить своего удела; прими ее ты, ибо я не могу принять.
\vs Rut 4:7 Прежде такой был \bibemph{обычай} у Израиля при выкупе и при мене для подтверждения какого-либо дела: один снимал сапог свой и давал другому, [который принимал право родственника,] и это было свидетельством у Израиля.
\vs Rut 4:8 И сказал тот родственник Воозу: купи себе. И снял сапог свой [и дал ему].
\vs Rut 4:9 И сказал Вооз старейшинам и всему народу: вы теперь свидетели тому, что я покупаю у Ноемини все Елимелехово и все Хилеоново и Махлоново;
\vs Rut 4:10 также и Руфь Моавитянку, жену Махлонову, беру себе в жену, чтоб оставить имя умершего в уделе его, и чтобы не исчезло имя умершего между братьями его и у ворот местопребывания его: вы сегодня свидетели тому.
\vs Rut 4:11 И сказал весь народ, который при воротах, и старейшины: мы свидетели; да соделает Господь жену, входящую в дом твой, как Рахиль и как Лию, которые обе устроили дом Израилев; приобретай богатство в Ефрафе, и да славится имя твое в Вифлееме;
\vs Rut 4:12 и да будет дом твой, как дом Фареса, которого родила Фамарь Иуде, от того семени, которое даст тебе Господь от этой молодой женщины.
\vs Rut 4:13 И взял Вооз Руфь, и она сделалась его женою. И вошел он к ней, и Господь дал ей беременность, и она родила сына.
\vs Rut 4:14 И говорили женщины Ноемини: благословен Господь, что Он не оставил тебя ныне без наследника! И да будет славно имя его в Израиле!
\vs Rut 4:15 Он будет тебе отрадою и питателем в старости твоей, ибо его родила сноха твоя, которая любит тебя, которая для тебя лучше семи сыновей.
\vs Rut 4:16 И взяла Ноеминь дитя сие, и носила его в объятиях своих, и была ему нянькою.
\vs Rut 4:17 Соседки нарекли ему имя и говорили: <<у Ноемини родился сын>>, и нарекли ему имя: Овид. Он отец Иессея, отца Давидова.
\rsbpar\vs Rut 4:18 И вот род Фаресов: Фарес родил Есрома;
\vs Rut 4:19 Есром родил Арама; Арам родил Аминадава;
\vs Rut 4:20 Аминадав родил Наассона; Наассон родил Салмона;
\vs Rut 4:21 Салмон родил Вооза; Вооз родил Овида;
\vs Rut 4:22 Овид родил Иессея; Иессей родил Давида.
\newbookpage
\bibbookdescr{1Sa}{
  inline={\LARGE Первая книга\\\Huge Царств\fns{У Евреев: <<Первая Самуила>>.}},
  toc={1-я Царств},
  bookmark={1-я Царств},
  header={1-я Царств},
  %headerleft={},
  %headerright={},
  abbr={1~Цар}
}
\vs 1Sa 1:1 Был один человек из Рамафаим-Цофима, с горы Ефремовой, имя ему Елкана, сын Иерохама, сына Илия, сына Тоху, сына Цуфа,~--- Ефрафянин;
\vs 1Sa 1:2 у него были две жены: имя одной Анна, а имя другой Феннана; у Феннаны были дети, у Анны же не было детей.
\vs 1Sa 1:3 И ходил этот человек из города своего в положенные дни поклоняться и приносить жертву Господу Саваофу в Силом; там \bibemph{были} [Илий и] два сына его, Офни и Финеес, священниками Господа.
\rsbpar\vs 1Sa 1:4 В тот день, когда Елкана приносил жертву, давал Феннане, жене своей, и всем сыновьям ее и дочерям ее части;
\vs 1Sa 1:5 Анне же давал часть особую, [так как у нее не было детей], ибо любил Анну [более, нежели Феннану], хотя Господь заключил чрево ее.
\vs 1Sa 1:6 Соперница ее сильно огорчала ее, побуждая ее к ропоту на то, что Господь заключил чрево ее.
\vs 1Sa 1:7 Так бывало каждый год, когда ходила она в дом Господень; та огорчала ее, а эта плакала [и сетовала] и не ела.
\vs 1Sa 1:8 И сказал ей Елкана, муж ее: Анна! [Она отвечала ему: вот я. И сказал ей:] что ты плачешь и почему не ешь, и отчего скорбит сердце твое? не лучше ли я для тебя десяти сыновей?
\vs 1Sa 1:9 И встала Анна после того, как они ели и пили в Силоме, [и стала пред Господом]. Илий же священник сидел тогда на седалище у входа в храм Господень.
\vs 1Sa 1:10 И была она в скорби души, и молилась Господу, и горько плакала,
\vs 1Sa 1:11 и дала обет, говоря: Господи [Всемогущий Боже] Саваоф! если Ты призришь на скорбь рабы Твоей и вспомнишь обо мне, и не забудешь рабы Твоей и дашь рабе Твоей дитя мужеского пола, то я отдам его Господу [в дар] на все дни жизни его, [и вина и сикера не будет он пить,] и бритва не коснется головы его.
\vs 1Sa 1:12 Между тем как она долго молилась пред Господом, Илий смотрел на уста ее;
\vs 1Sa 1:13 и как Анна говорила в сердце своем, а уста ее только двигались, и не было слышно голоса ее, то Илий счел ее пьяною.
\vs 1Sa 1:14 И сказал ей Илий: доколе ты будешь пьяною? вытрезвись от вина твоего [и иди от лица Господня].
\vs 1Sa 1:15 И отвечала Анна, и сказала: нет, господин мой; я~--- жена, скорбящая духом, вина и сикера я не пила, но изливаю душу мою пред Господом;
\vs 1Sa 1:16 не считай рабы твоей негодною женщиною, ибо от великой печали моей и от скорби моей я говорила доселе.
\vs 1Sa 1:17 И отвечал Илий и сказал: иди с миром, и Бог Израилев исполнит прошение твое, чего ты просила у Него.
\vs 1Sa 1:18 Она же сказала: да найдет раба твоя милость в очах твоих! И пошла она в путь свой, и ела, и лице ее не было уже \bibemph{печально}, как прежде.
\vs 1Sa 1:19 И встали они поутру, и поклонились пред Господом, и возвратились, и пришли в дом свой в Раму. И познал Елкана Анну, жену свою, и вспомнил о ней Господь.
\vs 1Sa 1:20 Чрез несколько времени зачала Анна и родила сына и дала ему имя: Самуил, ибо, [говорила она], от Господа [Бога Саваофа] я испросила его.
\vs 1Sa 1:21 И пошел муж ее Елкана и все семейство его [в Силом] совершить годичную жертву Господу и обеты свои [и все десятины от земли своей].
\vs 1Sa 1:22 Анна же не пошла [с ним], сказав мужу своему: когда младенец отнят будет от груди и подрастет, тогда я отведу его, и он явится пред Господом и останется там навсегда.
\vs 1Sa 1:23 И сказал ей Елкана, муж ее: делай, что тебе угодно; оставайся, доколе не вскормишь его грудью; только да утвердит Господь слово, [вышедшее из уст твоих]. И осталась жена \bibemph{его}, и кормила грудью сына своего, доколе не вскормила.
\rsbpar\vs 1Sa 1:24 Когда же вскормила его, пошла с ним в Силом, \bibemph{взяв} три тельца [и хлебы] и одну ефу муки и мех вина, и пришла в дом Господа в Силом, [и отрок с ними]; отрок же был еще дитя.
\vs 1Sa 1:25 [И привели его пред лице Господа; и принес отец его жертву, какую в установленные дни приносил Господу. И привели отрока] и закололи тельца; и привела отрока [Анна мать] к Илию
\vs 1Sa 1:26 и сказала: о, господин мой! да живет душа твоя, господин мой! я~--- та самая женщина, которая здесь при тебе стояла и молилась Господу;
\vs 1Sa 1:27 о сем дитяти молилась я, и исполнил мне Господь прошение мое, чего я просила у Него;
\vs 1Sa 1:28 и я отдаю его Господу на все дни жизни его, служить Господу. И поклонилась там Господу.
\vs 1Sa 2:1 И молилась Анна и говорила: возрадовалось сердце мое в Господе; вознесся рог мой в Боге моем; широко разверзлись уста мои на врагов моих, ибо я радуюсь о спасении Твоем.
\vs 1Sa 2:2 Нет \bibemph{столь} святаго, как Господь; ибо нет другого, кроме Тебя; и нет твердыни, как Бог наш.
\vs 1Sa 2:3 Не умножайте речей надменных; дерзкие слова да не исходят из уст ваших; ибо Господь есть Бог ведения, и дела у Него взвешены.
\vs 1Sa 2:4 Лук сильных преломляется, а немощные препоясываются силою;
\vs 1Sa 2:5 сытые работают из хлеба, а голодные отдыхают; даже бесплодная рождает семь раз, а многочадная изнемогает.
\vs 1Sa 2:6 Господь умерщвляет и оживляет, низводит в преисподнюю и возводит;
\vs 1Sa 2:7 Господь делает нищим и обогащает, унижает и возвышает.
\vs 1Sa 2:8 Из праха подъемлет Он бедного, из брения возвышает нищего, посаждая с вельможами, и престол славы дает им в наследие; ибо у Господа основания земли, и Он утвердил на них вселенную.
\vs 1Sa 2:9 Стопы святых Своих Он блюдет, а беззаконные во тьме исчезают; ибо не силою крепок человек.
\vs 1Sa 2:10 Господь сотрет препирающихся с Ним; с небес возгремит на них. [Господь свят. Да не хвалится мудрый мудростью своею, и да не хвалится сильный силою своею, и да не хвалится богатый богатством своим, но желающий хвалиться да хвалится тем, что разумеет и знает Господа.] \bibemph{Господь} будет судить концы земли, и даст крепость царю Своему и вознесет рог помазанника Своего.
\rsbpar\vs 1Sa 2:11 [И оставили Самуила там пред Господом,] и пошел Елкана в Раму в дом свой, а отрок остался служить Господу при Илии священнике.
\vs 1Sa 2:12 Сыновья же Илия были люди негодные; они не знали Господа
\vs 1Sa 2:13 и долга священников в отношении к народу. Когда кто приносил жертву, отрок священнический, во время варения мяса, приходил с вилкой в руке своей
\vs 1Sa 2:14 и опускал ее в котел, или в кастрюлю, или на сковороду, или в горшок, и что вынет вилка, то брал себе священник. Так поступали они со всеми Израильтянами, приходившими туда в Силом.
\vs 1Sa 2:15 Даже прежде, нежели сожигали тук, приходил отрок священнический и говорил приносившему жертву: дай мяса на жаркое священнику; он не возьмет у тебя вареного мяса, а дай сырое.
\vs 1Sa 2:16 И \bibemph{если} кто говорил ему: пусть сожгут прежде тук, как должно, и \bibemph{потом} возьми себе, сколько пожелает душа твоя, то он говорил: нет, теперь же дай, а если нет, то силою возьму.
\vs 1Sa 2:17 И грех этих молодых людей был весьма велик пред Господом, ибо они отвращали от жертвоприношений Господу.
\vs 1Sa 2:18 Отрок же Самуил служил пред Господом, надевая льняной ефод.
\vs 1Sa 2:19 Верхнюю одежду малую делала ему мать его и приносила ему ежегодно, когда приходила с мужем своим для принесения положенной жертвы.
\vs 1Sa 2:20 И благословил Илий Елкану и жену его и сказал: да даст тебе Господь детей от жены сей вместо данного, которого ты отдал Господу! И пошли они в место свое.
\vs 1Sa 2:21 И посетил Господь Анну, и зачала она и родила еще трех сыновей и двух дочерей; а отрок Самуил возрастал у Господа.
\rsbpar\vs 1Sa 2:22 Илий же был весьма стар и слышал все, как поступают сыновья его со всеми Израильтянами, и что они спят с женщинами, собиравшимися у входа в скинию собрания.
\vs 1Sa 2:23 И сказал им: для чего вы делаете такие дела? ибо я слышу худые речи о вас от всего народа [Господня].
\vs 1Sa 2:24 Нет, дети мои, нехороша молва, которую я слышу [о вас, не делайте так, ибо нехороша молва, которую я слышу]; вы развращаете народ Господень;
\vs 1Sa 2:25 если согрешит человек против человека, то помолятся о нем Богу; если же человек согрешит против Господа, то кто будет ходатаем о нем? Но они не слушали голоса отца своего, ибо Господь решил уже предать их смерти.
\rsbpar\vs 1Sa 2:26 Отрок же Самуил более и более приходил в возраст и в благоволение у Господа и у людей.
\vs 1Sa 2:27 И пришел человек Божий к Илию и сказал ему: так говорит Господь: не открылся ли Я дому отца твоего, когда еще были они в Египте, в доме фараона?
\vs 1Sa 2:28 И не избрал ли его из всех колен Израилевых Себе во священника, чтоб он восходил к жертвеннику Моему, чтобы воскурял фимиам, чтобы носил ефод предо Мною? И не дал ли Я дому отца твоего от всех огнем сожигаемых жертв сынов Израилевых?
\vs 1Sa 2:29 Для чего же вы попираете ногами жертвы Мои и хлебные приношения Мои, которые заповедал Я для жилища \bibemph{Моего}, и для чего ты предпочитаешь Мне сыновей своих, утучняя себя начатками всех приношений народа Моего~--- Израиля?
\vs 1Sa 2:30 Посему так говорит Господь Бог Израилев: Я сказал \bibemph{тогда}: <<дом твой и дом отца твоего будут ходить пред лицем Моим вовек>>. Но теперь говорит Господь: да не будет так, ибо Я прославлю прославляющих Меня, а бесславящие Меня будут посрамлены.
\vs 1Sa 2:31 Вот, наступают дни, в \bibemph{которые} Я подсеку мышцу твою и мышцу дома отца твоего, так что не будет старца в доме твоем [никогда];
\vs 1Sa 2:32 и ты будешь видеть бедствие жилища Моего, при всем том, что \bibemph{Господь} благотворит Израилю и не будет в доме твоем старца во все дни,
\vs 1Sa 2:33 Я не отрешу у тебя \bibemph{всех} от жертвенника Моего, чтобы томить глаза твои и мучить душу твою; но все потомство дома твоего будет умирать в средних летах.
\vs 1Sa 2:34 И вот тебе знамение, которое последует с двумя сыновьями твоими, Офни и Финеесом: оба они умрут в один день.
\vs 1Sa 2:35 И поставлю Себе священника верного; он будет поступать по сердцу Моему и по душе Моей; и дом его сделаю твердым, и он будет ходить пред помазанником Моим во все дни;
\vs 1Sa 2:36 и всякий, оставшийся из дома твоего, придет кланяться ему из-за геры серебра и куска хлеба и скажет: <<причисли меня к какой-либо левитской должности, чтоб иметь пропитание>>.
\vs 1Sa 3:1 Отрок Самуил служил Господу при Илии; слово Господне было редко в те дни, видения \bibemph{были} не часты.
\vs 1Sa 3:2 И было в то время, когда Илий лежал на своем месте,~--- глаза же его начали смежаться, и он не мог видеть,~---
\vs 1Sa 3:3 и светильник Божий еще не погас, и Самуил лежал в храме Господнем, где ковчег Божий;
\vs 1Sa 3:4 воззвал Господь к Самуилу: [Самуил, Самуил!] И отвечал он: вот я!
\vs 1Sa 3:5 И побежал к Илию и сказал: вот я! ты звал меня. Но тот сказал: я не звал тебя; пойди назад, ложись. И он пошел и лег.
\vs 1Sa 3:6 Но Господь в другой раз воззвал к Самуилу: [Самуил, Самуил!] Он встал, и пришел к Илию вторично, и сказал: вот я! ты звал меня. Но тот сказал: я не звал тебя, сын мой; пойди назад, ложись.
\vs 1Sa 3:7 Самуил еще не знал тогда \bibemph{голоса} Господа, и еще не открывалось ему слово Господне.
\vs 1Sa 3:8 И воззвал Господь к Самуилу еще в третий раз. Он встал и пришел к Илию и сказал: вот я! ты звал меня. Тогда понял Илий, что Господь зовет отрока.
\vs 1Sa 3:9 И сказал Илий Самуилу: пойди назад и ложись, и когда [Зовущий] позовет тебя, ты скажи: говори, Господи, ибо слышит раб Твой. И пошел Самуил и лег на месте своем.
\vs 1Sa 3:10 И пришел Господь, и стал, и воззвал, как в тот и другой раз: Самуил, Самуил! И сказал Самуил: говори, [Господи,] ибо слышит раб Твой.
\vs 1Sa 3:11 И сказал Господь Самуилу: вот, Я сделаю дело в Израиле, о котором кто услышит, у того зазвенит в обоих ушах;
\vs 1Sa 3:12 в тот день Я исполню над Илием все то, что Я говорил о доме его; Я начну и окончу;
\vs 1Sa 3:13 Я объявил ему, что Я накажу дом его на веки за ту вину, что он знал, как сыновья его нечествуют, и не обуздывал их;
\vs 1Sa 3:14 и посему клянусь дому Илия, что вина дома Илиева не загладится ни жертвами, ни приношениями хлебными вовек.
\vs 1Sa 3:15 И спал Самуил до утра, [и встал рано] и отворил двери дома Господня; и боялся Самуил объявить видение сие Илию.
\vs 1Sa 3:16 Но Илий позвал Самуила и сказал: сын мой Самуил! Тот сказал: вот я!
\vs 1Sa 3:17 И сказал \bibemph{Илий}: что сказано тебе? не скрой от меня; то и то сделает с тобою Бог, и еще больше сделает, если ты утаишь от меня что-либо из всего того, что сказано тебе.
\vs 1Sa 3:18 И объявил ему Самуил все и не скрыл от него \bibemph{ничего}. Тогда сказал [Илий]: Он~--- Господь; что Ему угодно, то да сотворит.
\rsbpar\vs 1Sa 3:19 И возрос Самуил, и Господь был с ним; и не осталось ни одного из слов его неисполнившимся.
\vs 1Sa 3:20 И узнал весь Израиль от Дана до Вирсавии, что Самуил удостоен быть пророком Господним.
\vs 1Sa 3:21 И продолжал Господь являться в Силоме после того, как открыл Себя Самуилу в Силоме чрез слово Господне. [И уверились во всем Израиле, от конца до конца земли, что Самуил есть пророк Господень. Илий же сделался очень стар, а сыновья его продолжали ходить беззаконным путем своим пред Господом.]
\vs 1Sa 4:1 [И собрались Филистимляне воевать с Израильтянами.] И было слово Самуила ко всему Израилю. И выступили Израильтяне против Филистимлян на войну и расположились станом при Авен-Езере, а Филистимляне расположились при Афеке.
\vs 1Sa 4:2 И выстроились Филистимляне против Израильтян, и произошла битва, и были поражены Израильтяне Филистимлянами, которые побили на поле сражения около четырех тысяч человек.
\vs 1Sa 4:3 И пришел народ в стан; и сказали старейшины Израилевы: за что поразил нас Господь сегодня пред Филистимлянами? возьмем себе из Силома ковчег завета Господня, и он пойдет среди нас и спасет нас от руки врагов наших.
\vs 1Sa 4:4 И послал народ в Силом, и принесли оттуда ковчег завета Господа Саваофа, сидящего на херувимах; а при ковчеге завета Божия были и два сына Илиевы, Офни и Финеес.
\vs 1Sa 4:5 И когда прибыл ковчег завета Господня в стан, весь Израиль поднял такой сильный крик, что земля стонала.
\vs 1Sa 4:6 И услышали Филистимляне шум восклицаний и сказали: отчего такие громкие восклицания в стане Евреев? И узнали, что ковчег Господень прибыл в стан.
\vs 1Sa 4:7 И устрашились Филистимляне, ибо сказали: Бог тот пришел к ним в стан. И сказали: горе нам! ибо не бывало подобного ни вчера, ни третьего дня;
\vs 1Sa 4:8 горе нам! кто избавит нас от руки этого сильного Бога? Это~--- тот Бог, Который поразил Египтян всякими казнями в пустыне;
\vs 1Sa 4:9 укрепитесь и будьте мужественны, Филистимляне, чтобы вам не быть в порабощении у Евреев, как они у вас в порабощении; будьте мужественны и сразитесь с ними.
\vs 1Sa 4:10 И сразились Филистимляне, и поражены были Израильтяне, и каждый побежал в шатер свой, и было поражение весьма великое, и пало из Израильтян тридцать тысяч пеших.
\vs 1Sa 4:11 И ковчег Божий был взят, и два сына Илиевы, Офни и Финеес, умерли.
\vs 1Sa 4:12 И побежал один Вениамитянин с места сражения и пришел в Силом в тот же день; одежда на нем была разодрана и прах на голове его.
\vs 1Sa 4:13 Когда пришел он, Илий сидел на седалище при дороге у ворот и смотрел, ибо сердце его трепетало за ковчег Божий. И когда человек тот пришел и объявил в городе, то громко восстенал весь город.
\vs 1Sa 4:14 И услышал Илий звуки вопля и сказал: отчего такой шум? И тотчас подошел человек тот и объявил Илию.
\vs 1Sa 4:15 Илий был тогда девяноста восьми лет; и глаза его померкли, и он не мог видеть.
\vs 1Sa 4:16 И сказал тот человек Илию: я пришел из стана, сегодня же бежал я с места сражения. И сказал \bibemph{Илий}: что произошло, сын мой?
\vs 1Sa 4:17 И отвечал вестник и сказал: побежал Израиль пред Филистимлянами, и поражение великое произошло в народе, и оба сына твои, Офни и Финеес, умерли, и ковчег Божий взят.
\vs 1Sa 4:18 Когда упомянул он о ковчеге Божием, \bibemph{Илий} упал с седалища навзничь у ворот, сломал себе хребет и умер; ибо он \bibemph{был} стар и тяжел. Был же он судьею Израиля сорок лет.
\vs 1Sa 4:19 Невестка его, жена Финеесова, была беременна уже пред родами. И когда услышала она известие о взятии ковчега Божия и о смерти свекра своего и мужа своего, то упала на колени и родила, ибо приступили к ней боли ее.
\vs 1Sa 4:20 И когда умирала она, стоявшие при ней женщины говорили ей: не бойся, ты родила сына. Но она не отвечала и не обращала внимания.
\vs 1Sa 4:21 И назвала младенца: Ихавод\fns{Бесславие.}, сказав: <<отошла слава от Израиля>>~--- со взятием ковчега Божия и [со смертью] свекра ее и мужа ее.
\vs 1Sa 4:22 Она сказала: отошла слава от Израиля, ибо взят ковчег Божий.
\vs 1Sa 5:1 Филистимляне же взяли ковчег Божий и принесли его из Авен-Езера в Азот.
\vs 1Sa 5:2 И взяли Филистимляне ковчег Божий, и внесли его в храм Дагона, и поставили его подле Дагона.
\vs 1Sa 5:3 И встали Азотяне рано на другой день, и вот, Дагон лежит лицем своим к земле пред ковчегом Господним. И взяли они Дагона и опять поставили его на свое место.
\vs 1Sa 5:4 И встали они поутру на следующий день, и вот, Дагон лежит ниц на земле пред ковчегом Господним; голова Дагонова и [обе ноги его и] обе руки его [лежали] отсеченные, каждая особо, на пороге, осталось только туловище Дагона.
\vs 1Sa 5:5 Посему жрецы Дагоновы и все приходящие в капище Дагона в Азот не ступают на порог Дагонов до сего дня, [а переступают чрез него].
\vs 1Sa 5:6 И отяготела рука Господня над Азотянами, и Он поражал их и наказал их мучительными наростами, в Азоте и в окрестностях его, [а внутри страны размножились мыши, и было в городе великое отчаяние].
\vs 1Sa 5:7 И увидели это Азотяне и сказали: да не останется ковчег Бога Израилева у нас, ибо тяжка рука Его и для нас и для Дагона, бога нашего.
\vs 1Sa 5:8 И послали, и собрали к себе всех владетелей Филистимских, и сказали: что нам делать с ковчегом Бога Израилева? И сказали [Гефяне]: пусть ковчег Бога Израилева перейдет [к нам] в Геф. И отправили ковчег Бога Израилева в Геф.
\vs 1Sa 5:9 После того, как отправили его, была рука Господа на городе~--- ужас весьма великий, и поразил Господь жителей города от малого до большого, и показались на них наросты.
\vs 1Sa 5:10 И отослали они ковчег Божий в Аскалон; и когда пришел ковчег Божий в Аскалон, возопили Аскалонитяне, говоря: принесли к нам ковчег Бога Израилева, чтоб умертвить нас и народ наш.
\vs 1Sa 5:11 И послали, и собрали всех владетелей Филистимских, и сказали: отошлите ковчег Бога Израилева; пусть он возвратится в свое место, чтобы не умертвил он нас и народа нашего. Ибо смертельный ужас был во всем городе; весьма отяготела рука Божия на них, [когда пришел туда ковчег Бога Израилева].
\vs 1Sa 5:12 И те, которые не умерли, поражены были наростами, так что вопль города восходил до небес.
\vs 1Sa 6:1 И пробыл ковчег Господень в области Филистимской семь месяцев, [и наполнилась земля та мышами].
\vs 1Sa 6:2 И призвали Филистимляне жрецов и прорицателей [и заклинателей] и сказали: что нам делать с ковчегом Господним? научите нас, как нам отпустить его в свое место.
\vs 1Sa 6:3 Те сказали: если вы хотите отпустить ковчег [завета Господа] Бога Израилева, то не отпускайте его ни с чем, но принесите Ему жертву повинности; тогда исцелитесь и узнаете, за что не отступает от вас рука Его.
\vs 1Sa 6:4 И сказали они: какую жертву повинности должны мы принести Ему? Те сказали: по числу владетелей Филистимских пять наростов золотых и пять мышей золотых; ибо казнь одна на всех вас и на владетелях ваших;
\vs 1Sa 6:5 итак сделайте изваяния наростов ваших и изваяния мышей ваших, опустошающих землю, и воздайте славу Богу Израилеву; может быть, Он облегчит руку Свою над вами и над богами вашими и над землею вашею;
\vs 1Sa 6:6 и для чего вам ожесточать сердце ваше, как ожесточили сердце свое Египтяне и фараон? вот, когда Господь показал силу Свою над ними, то они отпустили их, и те пошли;
\vs 1Sa 6:7 итак возьмите, сделайте одну колесницу новую и возьмите двух первородивших коров, на которых не было ярма, и впрягите коров в колесницу, а телят их отведите от них домой;
\vs 1Sa 6:8 и возьмите ковчег Господень, и поставьте его на колесницу, а золотые вещи, которые принесете Ему в жертву повинности, положите в ящик сбоку его; и отпустите его, и пусть пойдет;
\vs 1Sa 6:9 и смотрите, если он пойдет к пределам своим, к Вефсамису, то Он великое сие зло сделал нам; если же нет, то мы будем знать, что не Его рука поразила нас, а сделалось это с нами случайно.
\vs 1Sa 6:10 И сделали они так: и взяли двух первородивших коров и впрягли их в колесницу, а телят их удержали дома;
\vs 1Sa 6:11 и поставили ковчег Господа на колесницу и ящик с золотыми мышами и изваяниями наростов.
\vs 1Sa 6:12 И пошли коровы прямо на дорогу к Вефсамису; одною дорогою шли, шли и мычали, но не уклонялись ни направо, ни налево; владетели же Филистимские следовали за ними до пределов Вефсамиса.
\vs 1Sa 6:13 \bibemph{Жители} Вефсамиса жали тогда пшеницу в долине, и взглянув увидели ковчег Господень, и обрадовались, что увидели его.
\vs 1Sa 6:14 Колесница же пришла на поле Иисуса Вефсамитянина и остановилась там; и был тут большой камень, и раскололи колесницу на дрова, а коров принесли во всесожжение Господу.
\vs 1Sa 6:15 Левиты сняли ковчег Господа и ящик, бывший при нем, в котором \bibemph{были} золотые вещи, и поставили на большом том камне; жители же Вефсамиса принесли в тот день всесожжения и закололи жертвы Господу.
\vs 1Sa 6:16 И пять владетелей Филистимских видели \bibemph{это} и возвратились в тот день в Аккарон.
\vs 1Sa 6:17 Золотые эти наросты, которые принесли Филистимляне в жертву повинности Господу, были: один за Азот, один за Газу, один за Аскалон, один за Геф, один за Аккарон;
\vs 1Sa 6:18 а золотые мыши \bibemph{были} по числу всех городов Филистимских~--- пяти владетелей, от городов укрепленных и до открытых сел, до большого камня, на котором поставили ковчег Господа и \bibemph{который находится} до сего дня на поле Иисуса Вефсамитянина.
\vs 1Sa 6:19 [Не порадовались сыны Иехониины среди мужей Вефсамисских, что видели ковчег Господа]. И поразил Он жителей Вефсамиса за то, что они заглядывали в ковчег Господа, и убил из народа пятьдесят тысяч семьдесят человек; и заплакал народ, ибо поразил Господь народ поражением великим.
\vs 1Sa 6:20 И сказали жители Вефсамиса: кто может стоять пред Господом, сим святым Богом? и к кому Он пойдет от нас?
\vs 1Sa 6:21 И послали послов к жителям Кириаф-Иарима сказать: Филистимляне возвратили ковчег Господа; придите, возьмите его к себе.
\vs 1Sa 7:1 И пришли жители Кириаф-Иарима, и взяли ковчег Господа, и принесли его в дом Аминадава, на холм, а Елеазара, сына его, посвятили, чтобы он хранил ковчег Господа.
\vs 1Sa 7:2 С того дня, как остался ковчег в Кириаф-Иариме, прошло много времени, лет двадцать. И обратился весь дом Израилев к Господу.
\vs 1Sa 7:3 И сказал Самуил всему дому Израилеву, говоря: если вы всем сердцем своим обращаетесь к Господу, то удалите из среды себя богов иноземных и Астарт и расположите сердце ваше к Господу, и служите Ему одному, и Он избавит вас от руки Филистимлян.
\vs 1Sa 7:4 И удалили сыны Израилевы Ваалов и Астарт и стали служить одному Господу.
\vs 1Sa 7:5 И сказал Самуил: соберите всех Израильтян в Массифу и я помолюсь о вас Господу.
\vs 1Sa 7:6 И собрались в Массифу, и черпали воду, и проливали пред Господом, и постились в тот день, говоря: согрешили мы пред Господом. И судил Самуил сынов Израилевых в Массифе.
\vs 1Sa 7:7 Когда же услышали Филистимляне, что собрались сыны Израилевы в Массифу, тогда пошли владетели Филистимские на Израиля. Израильтяне, услышав \bibemph{о том}, убоялись Филистимлян.
\vs 1Sa 7:8 И сказали сыны Израилевы Самуилу: не переставай взывать о нас к Господу Богу нашему, чтоб Он спас нас от руки Филистимлян. [И сказал Самуил: да не будет этого со мною, чтоб отступить от Господа Бога моего, и не взывать о вас в молитве!]
\vs 1Sa 7:9 И взял Самуил одного ягненка от сосцов, и принес его [со всем народом] во всесожжение Господу, и воззвал Самуил к Господу о Израиле, и услышал его Господь.
\vs 1Sa 7:10 И когда Самуил возносил всесожжение, Филистимляне пришли воевать с Израилем. Но Господь возгремел в тот день сильным громом над Филистимлянами и навел на них ужас, и они были поражены пред Израилем.
\vs 1Sa 7:11 И выступили Израильтяне из Массифы, и преследовали Филистимлян, и поражали их до места под Вефхором.
\vs 1Sa 7:12 И взял Самуил один камень, и поставил между Массифою и между Сеном, и назвал его Авен-Езер\fns{Камень помощи.}, сказав: до сего места помог нам Господь.
\vs 1Sa 7:13 Так усмирены были Филистимляне, и не стали более ходить в пределы Израилевы; и была рука Господня на Филистимлянах во все дни Самуила.
\vs 1Sa 7:14 И возвращены были Израилю города, которые взяли Филистимляне у Израиля, от Аккарона и до Гефа, и пределы их освободил Израиль из рук Филистимлян, и был мир между Израилем и Аморреями.
\rsbpar\vs 1Sa 7:15 И был Самуил судьею Израиля во все дни жизни своей:
\vs 1Sa 7:16 из года в год он ходил и обходил Вефиль, и Галгал и Массифу и судил Израиля во всех сих местах;
\vs 1Sa 7:17 потом возвращался в Раму; ибо \bibemph{там} был дом его, и там судил он Израиля, и построил там жертвенник Господу.
\vs 1Sa 8:1 Когда же состарился Самуил, то поставил сыновей своих судьями над Израилем.
\vs 1Sa 8:2 Имя старшему сыну его Иоиль, а имя второму \bibemph{сыну} его Авия; они \bibemph{были} судьями в Вирсавии.
\vs 1Sa 8:3 Но сыновья его не ходили путями его, а уклонились в корысть и брали подарки, и судили превратно.
\vs 1Sa 8:4 И собрались все старейшины Израиля, и пришли к Самуилу в Раму,
\vs 1Sa 8:5 и сказали ему: вот, ты состарился, а сыновья твои не ходят путями твоими; итак поставь над нами царя, чтобы он судил нас, как у прочих народов.
\vs 1Sa 8:6 И не понравилось слово сие Самуилу, когда они сказали: дай нам царя, чтобы он судил нас. И молился Самуил Господу.
\rsbpar\vs 1Sa 8:7 И сказал Господь Самуилу: послушай голоса народа во всем, что они говорят тебе; ибо не тебя они отвергли, но отвергли Меня, чтоб Я не царствовал над ними;
\vs 1Sa 8:8 как они поступали с того дня, в который Я вывел их из Египта, и до сего дня, оставляли Меня и служили иным богам, так поступают они с тобою;
\vs 1Sa 8:9 итак послушай голоса их; только представь им и объяви им права царя, который будет царствовать над ними.
\vs 1Sa 8:10 И пересказал Самуил все слова Господа народу, просящему у него царя,
\vs 1Sa 8:11 и сказал: вот какие будут права царя, который будет царствовать над вами: сыновей ваших он возьмет и приставит их к колесницам своим и \bibemph{сделает} всадниками своими, и будут они бегать пред колесницами его;
\vs 1Sa 8:12 и поставит \bibemph{их} у себя тысяченачальниками и пятидесятниками, и чтобы они возделывали поля его, и жали хлеб его, и делали ему воинское оружие и колесничный прибор его;
\vs 1Sa 8:13 и дочерей ваших возьмет, чтоб они составляли масти, варили кушанье и пекли хлебы;
\vs 1Sa 8:14 и поля ваши и виноградные и масличные сады ваши лучшие возьмет, и отдаст слугам своим;
\vs 1Sa 8:15 и от посевов ваших и из виноградных садов ваших возьмет десятую часть и отдаст евнухам своим и слугам своим;
\vs 1Sa 8:16 и рабов ваших и рабынь ваших, и юношей ваших лучших, и ослов ваших возьмет и употребит на свои дела;
\vs 1Sa 8:17 от мелкого скота вашего возьмет десятую часть, и сами вы будете ему рабами;
\vs 1Sa 8:18 и восстенаете тогда от царя вашего, которого вы избрали себе; и не будет Господь отвечать вам тогда.
\vs 1Sa 8:19 Но народ не согласился послушаться голоса Самуила, и сказал: нет, пусть царь будет над нами,
\vs 1Sa 8:20 и мы будем как прочие народы: будет судить нас царь наш, и ходить пред нами, и вести войны наши.
\vs 1Sa 8:21 И выслушал Самуил все слова народа, и пересказал их вслух Господа.
\vs 1Sa 8:22 И сказал Господь Самуилу: послушай голоса их и поставь им царя. И сказал Самуил Израильтянам: пойдите каждый в свой город.
\vs 1Sa 9:1 Был некто из сынов Вениамина, имя его Кис, сын Авиила, сына Церона, сына Бехорафа, сына Афия, сына некоего Вениамитянина, человек знатный.
\vs 1Sa 9:2 У него был сын, имя его Саул, молодой и красивый; и не было никого из Израильтян красивее его; он от плеч своих был выше всего народа.
\vs 1Sa 9:3 И пропали ослицы у Киса, отца Саулова, и сказал Кис Саулу, сыну своему: возьми с собою одного из слуг и встань, пойди, поищи ослиц.
\vs 1Sa 9:4 И прошел он гору Ефремову и прошел землю Шалишу, но не нашли; и прошли землю Шаалим, и \bibemph{там их} нет; и прошел он землю Вениаминову, и не нашли.
\vs 1Sa 9:5 Когда они пришли в землю Цуф, Саул сказал слуге своему, который был с ним: пойдем назад, чтобы отец мой, оставив ослиц, не стал беспокоиться о нас.
\vs 1Sa 9:6 Но слуга сказал ему: вот в этом городе есть человек Божий, человек уважаемый; все, что он ни скажет, сбывается; сходим теперь туда; может быть, он укажет нам путь наш, по которому нам идти.
\vs 1Sa 9:7 И сказал Саул слуге своему: вот мы пойдем, а что мы принесем тому человеку? ибо хлеба не стало в сумах наших, и подарка нет, чтобы поднести человеку Божию; что у нас?
\vs 1Sa 9:8 И опять отвечал слуга Саулу и сказал: вот в руке моей четверть сикля серебра; я отдам человеку Божию, и он укажет нам путь наш.
\vs 1Sa 9:9 Прежде у Израиля, когда кто-нибудь шел вопрошать Бога, говорили так: <<пойдем к прозорливцу>>; ибо тот, кого \bibemph{называют} ныне пророком, прежде назывался прозорливцем.
\vs 1Sa 9:10 И сказал Саул слуге своему: хорошо ты говоришь; пойдем. И пошли в город, где человек Божий.
\vs 1Sa 9:11 Когда они поднимались вверх в город, то встретили девиц, вышедших черпать воду, и сказали им: есть ли здесь прозорливец?
\vs 1Sa 9:12 Те отвечали им и сказали: есть; вот, он впереди тебя; только поспешай, ибо он сегодня пришел в город, потому что сегодня у народа жертвоприношение на высоте;
\vs 1Sa 9:13 когда придете в город, застанете его, пока он еще не пошел на ту высоту, на обед; ибо народ не начнет есть, доколе он не придет; потому что он благословит жертву, и после того станут есть званые; итак ступайте, теперь еще застанете его.
\vs 1Sa 9:14 И пошли они в город. Когда же вошли в средину города, то вот и Самуил выходит навстречу им, чтоб идти на высоту.
\vs 1Sa 9:15 А Господь открыл Самуилу за день до прихода Саулова и сказал:
\vs 1Sa 9:16 завтра в это время Я пришлю к тебе человека из земли Вениаминовой, и ты помажь его в правителя народу Моему~--- Израилю, и он спасет народ Мой от руки Филистимлян; ибо Я призрел на народ Мой, так как вопль его достиг до Меня.
\vs 1Sa 9:17 Когда Самуил увидел Саула, то Господь сказал ему: вот человек, о котором Я говорил тебе; он будет управлять народом Моим.
\vs 1Sa 9:18 И подошел Саул к Самуилу в воротах и спросил его: скажи мне, где дом прозорливца?
\vs 1Sa 9:19 И отвечал Самуил Саулу, и сказал: я прозорливец, иди впереди меня на высоту; и вы будете обедать со мною сегодня, и отпущу тебя утром, и все, что у тебя на сердце, скажу тебе;
\vs 1Sa 9:20 а об ослицах, которые у тебя пропали уже три дня, не заботься; они нашлись. И кому все вожделенное в Израиле? Не тебе ли и всему дому отца твоего?
\vs 1Sa 9:21 И отвечал Саул и сказал: не сын ли я Вениамина, одного из меньших колен Израилевых? И племя мое не малейшее ли между всеми племенами колена Вениаминова? К чему же ты говоришь мне это?
\vs 1Sa 9:22 И взял Самуил Саула и слугу его, и ввел их в комнату, и дал им первое место между зваными, которых было около тридцати человек.
\vs 1Sa 9:23 И сказал Самуил повару: подай ту часть, которую я дал тебе и о которой я сказал тебе: <<отложи ее у себя>>.
\vs 1Sa 9:24 И взял повар плечо и что было при нем и положил пред Саулом. И сказал [Самуил]: вот это оставлено, положи пред собою \bibemph{и} ешь, ибо к сему времени сбережено \bibemph{это} для тебя, когда я созывал народ. И обедал Саул с Самуилом в тот день.
\vs 1Sa 9:25 И сошли они с высоты в город, и Самуил разговаривал с Саулом на кровле, [и постлали Саулу на кровле, и он спал].
\vs 1Sa 9:26 Утром встали они так: когда взошла заря, Самуил воззвал к Саулу на кровлю и сказал: встань, я провожу тебя. И встал Саул, и вышли оба они из дома, он и Самуил.
\vs 1Sa 9:27 Когда подходили они к концу города, Самуил сказал Саулу: скажи слуге, чтобы он пошел впереди нас,~--- и он пошел вперед;~--- а ты остановись теперь, и я открою тебе, что сказал Бог.
\vs 1Sa 10:1 И взял Самуил сосуд с елеем и вылил на голову его, и поцеловал его и сказал: вот, Господь помазывает тебя в правителя наследия Своего [в Израиле, и ты будешь царствовать над народом Господним и спасешь их от руки врагов их, окружающих их, и вот тебе знамение, что помазал тебя Господь в царя над наследием Своим]:
\vs 1Sa 10:2 когда ты теперь пойдешь от меня, то встретишь двух человек близ гроба Рахили, на пределах Вениаминовых, в Целцахе, и они скажут тебе: <<нашлись ослицы, которых ты ходил искать, и вот отец твой, забыв об ослицах, беспокоится о вас, говоря: что с сыном моим?>>
\vs 1Sa 10:3 И пойдешь оттуда далее и придешь к дубраве Фаворской, и встретят тебя там три человека, идущих к Богу в Вефиль: один несет трех козлят, другой несет три хлеба, а третий несет мех с вином;
\vs 1Sa 10:4 и будут приветствовать они тебя и дадут тебе два хлеба, и ты возьмешь из рук их.
\vs 1Sa 10:5 После того ты придешь на холм Божий, где охранный отряд Филистимский; [там начальники Филистимские;] и когда войдешь там в город, встретишь сонм пророков, сходящих с высоты, и пред ними псалтирь и тимпан, и свирель и гусли, и они пророчествуют;
\vs 1Sa 10:6 и найдет на тебя Дух Господень, и ты будешь пророчествовать с ними и сделаешься иным человеком.
\vs 1Sa 10:7 Когда эти знамения сбудутся с тобою, тогда делай, что может рука твоя, ибо с тобою Бог.
\vs 1Sa 10:8 И ты пойди прежде меня в Галгал, куда и я приду к тебе для принесения всесожжений и мирных жертв; семь дней жди, доколе я не приду к тебе, и тогда укажу тебе, что тебе делать.
\vs 1Sa 10:9 Как скоро Саул обратился, чтоб идти от Самуила, Бог дал ему иное сердце, и сбылись все те знамения в тот же день.
\vs 1Sa 10:10 Когда пришли они к холму, вот встречается им сонм пророков, и сошел на него Дух Божий, и он пророчествовал среди них.
\vs 1Sa 10:11 Все знавшие его вчера и третьего дня, увидев, что он с пророками пророчествует, говорили в народе друг другу: что это сталось с сыном Кисовым? неужели и Саул во пророках?
\vs 1Sa 10:12 И отвечал один из бывших там и сказал: а у тех кто отец? Посему вошло в пословицу: <<неужели и Саул во пророках?>>
\vs 1Sa 10:13 И перестал он пророчествовать, и пошел на высоту.
\vs 1Sa 10:14 И сказал дядя Саулов ему и слуге его: куда вы ходили? Он сказал: искать ослиц, но, видя, что \bibemph{их} нет, зашли к Самуилу.
\vs 1Sa 10:15 И сказал дядя Саулов: расскажи мне, что сказал вам Самуил.
\vs 1Sa 10:16 И сказал Саул дяде своему: он объявил нам, что ослицы нашлись. А того, что сказал ему Самуил о царстве, не открыл ему.
\rsbpar\vs 1Sa 10:17 И созвал Самуил народ к Господу в Массифу
\vs 1Sa 10:18 и сказал сынам Израилевым: так говорит Господь Бог Израилев: Я вывел Израиля из Египта и избавил вас от руки Египтян и от руки всех царств, угнетавших вас.
\vs 1Sa 10:19 А вы теперь отвергли Бога вашего, Который спасает вас от всех бедствий ваших и скорбей ваших, и сказали Ему: <<царя поставь над нами>>. Итак предстаньте теперь пред Господом по коленам вашим и по племенам вашим.
\vs 1Sa 10:20 И велел Самуил подходить всем коленам Израилевым, и указано колено Вениаминово.
\vs 1Sa 10:21 И велел подходить колену Вениаминову по племенам его, и указано племя Матриево; и приводят племя Матриево по мужам, и назван Саул, сын Кисов; и искали его, и не находили.
\vs 1Sa 10:22 И вопросили еще Господа: придет ли еще он сюда? И сказал Господь: вот он скрывается в обозе.
\vs 1Sa 10:23 И побежали и взяли его оттуда, и он стал среди народа и был от плеч своих выше всего народа.
\vs 1Sa 10:24 И сказал Самуил всему народу: видите ли, кого избрал Господь? подобного ему нет во всем народе. Тогда весь народ воскликнул и сказал: да живет царь!
\vs 1Sa 10:25 И изложил Самуил народу права царства, и написал в книгу, и положил пред Господом. И отпустил весь народ, каждого в дом свой.
\vs 1Sa 10:26 Также и Саул пошел в дом свой, в Гиву; и пошли с ним храбрые, которых с\acc{е}рдца коснулся Бог.
\vs 1Sa 10:27 А негодные люди говорили: ему ли спасать нас? И презрели его и не поднесли ему даров; но он как бы не замечал того.
\vs 1Sa 11:1 И [было спустя около месяца,] пришел Наас Аммонитянин и осадил Иавис Галаадский. И сказали все жители Иависа Наасу: заключи с нами союз, и мы будем служить тебе.
\vs 1Sa 11:2 И сказал им Наас Аммонитянин: я заключу с вами союз, но с тем, чтобы выколоть у каждого из вас правый глаз и тем положить бесчестие на всего Израиля.
\vs 1Sa 11:3 И сказали ему старейшины Иависа: дай нам сроку семь дней, чтобы послать нам послов во все пределы Израильские, и если никто не поможет нам, то мы выйдем к тебе.
\vs 1Sa 11:4 И пришли послы в Гиву Саулову и пересказали слова сии вслух народа; и весь народ поднял вопль и заплакал.
\vs 1Sa 11:5 И вот, пришел Саул позади волов с поля и сказал: что \bibemph{сделалось} с народом, что он плачет? И пересказали ему слова жителей Иависа.
\vs 1Sa 11:6 И сошел Дух Божий на Саула, когда он услышал слова сии, и сильно воспламенился гнев его;
\vs 1Sa 11:7 и взял он пару волов, и рассек их на части, и послал во все пределы Израильские чрез тех послов, объявляя, что так будет поступлено с волами того, кто не пойдет вслед Саула и Самуила. И напал страх Господень на народ, и выступили \bibemph{все}, как один человек.
\vs 1Sa 11:8 \bibemph{Саул} осмотрел их в Везеке, и нашлось сынов Израилевых триста тысяч и мужей Иудиных тридцать тысяч.
\vs 1Sa 11:9 И сказали пришедшим послам: так скажите жителям Иависа Галаадского: завтра будет к вам помощь, когда обогреет солнце. И пришли послы и объявили жителям Иависа, и они обрадовались.
\vs 1Sa 11:10 И сказали жители Иависа [Наасу]: завтра выйдем к вам, и поступайте с нами, как вам угодно.
\vs 1Sa 11:11 В следующий день Саул разделил народ на три отряда, и они проникли в средину стана во время утренней стражи и поразили Аммонитян до дневного зноя; оставшиеся рассеялись, так что не осталось из них двоих вместе.
\vs 1Sa 11:12 Тогда сказал народ Самуилу: кто говорил: <<Саулу ли царствовать над нами>>? дайте этих людей, и мы умертвим их.
\vs 1Sa 11:13 Но Саул сказал: в сей день никого не должно умерщвлять, ибо сегодня Господь совершил спасение в Израиле.
\vs 1Sa 11:14 И сказал Самуил народу: пойдем в Галгал, и обновим там царство.
\vs 1Sa 11:15 И пошел весь народ в Галгал, и поставили там Саула царем пред Господом в Галгале, и принесли там мирные жертвы пред Господом. И весьма веселились там Саул и все Израильтяне.
\vs 1Sa 12:1 И сказал Самуил всему Израилю: вот, я послушался голоса вашего во всем, что вы говорили мне, и поставил над вами царя,
\vs 1Sa 12:2 и вот, царь ходит пред вами; а я состарился и поседел; и сыновья мои с вами; я же ходил пред вами от юности моей и до сего дня;
\vs 1Sa 12:3 вот я; свидетельствуйте на меня пред Господом и пред помазанником Его, у кого взял я вола, у кого взял осла, кого обидел и кого притеснил, у кого взял дар и закрыл в \bibemph{деле} его глаза мои,~--- и я возвращу вам.
\vs 1Sa 12:4 И отвечали: ты не обижал нас и не притеснял нас и ничего ни у кого не взял.
\vs 1Sa 12:5 И сказал он им: свидетель на вас Господь, и свидетель помазанник Его в сей день, что вы не нашли ничего за мною. И сказали: свидетель.
\vs 1Sa 12:6 Тогда Самуил сказал народу: [свидетель] Господь, Который поставил Моисея и Аарона и Который вывел отцов ваших из земли Египетской.
\vs 1Sa 12:7 Теперь же предстаньте, и я буду судиться с вами пред Господом о всех благодеяниях, которые оказал Он вам и отцам вашим.
\vs 1Sa 12:8 Когда пришел Иаков в Египет, и отцы ваши возопили к Господу, то Господь послал Моисея и Аарона, и они вывели отцов ваших из Египта и поселили их на месте сем.
\vs 1Sa 12:9 Но они забыли Господа Бога своего, и Он предал их в руки Сисары, военачальника Асорского, и в руки Филистимлян и в руки царя Моавитского, \bibemph{которые} воевали против них.
\vs 1Sa 12:10 Но когда они возопили к Господу и сказали: <<согрешили мы, ибо оставили Господа и стали служить Ваалам и Астартам, теперь избавь нас от руки врагов наших, и мы будем служить Тебе>>,
\vs 1Sa 12:11 тогда Господь послал Иероваала, и Варака, и Иеффая, и Самуила, и избавил вас от руки врагов ваших, окружавших вас, и вы жили безопасно.
\vs 1Sa 12:12 Но увидев, что Наас, царь Аммонитский, идет против вас, вы сказали мне: <<нет, царь пусть царствует над нами>>, тогда как Господь Бог ваш~--- Царь ваш.
\vs 1Sa 12:13 Итак, вот царь, которого вы избрали, которого вы требовали: вот, Господь поставил над вами царя.
\vs 1Sa 12:14 Если будете бояться Господа и служить Ему и слушать гласа Его, и не станете противиться повелениям Господа, и будете и вы и царь ваш, который царствует над вами, \bibemph{ходить} вслед Господа, Бога вашего, [то рука Господа не будет против вас];
\vs 1Sa 12:15 а если не будете слушать гласа Господа и станете противиться повелениям Господа, то рука Господа будет против вас, \bibemph{как была} против отцов ваших.
\vs 1Sa 12:16 Теперь станьте и посмотрите на дело великое, которое Господь совершит пред глазами вашими:
\vs 1Sa 12:17 не жатва ли пшеницы ныне? Но я воззову к Господу, и пошлет Он гром и дождь, и вы узнаете и увидите, как велик грех, который вы сделали пред очами Господа, прося себе царя.
\vs 1Sa 12:18 И воззвал Самуил к Господу, и Господь послал гром и дождь в тот день; и пришел весь народ в большой страх от Господа и Самуила.
\vs 1Sa 12:19 И сказал весь народ Самуилу: помолись о рабах твоих пред Господом Богом твоим, чтобы не умереть нам; ибо ко всем грехам нашим мы прибавили еще грех, когда просили себе царя.
\vs 1Sa 12:20 И отвечал Самуил народу: не бойтесь, грех этот вами сделан, но вы не отступайте только от Господа и служите Господу всем сердцем вашим
\vs 1Sa 12:21 и не обращайтесь вслед ничтожных \bibemph{богов}, которые не принесут пользы и не избавят; ибо они~--- ничто;
\vs 1Sa 12:22 Господь же не оставит народа Своего ради великого имени Своего, ибо Господу угодно было избрать вас народом Своим;
\vs 1Sa 12:23 и я также не допущу себе греха пред Господом, чтобы перестать молиться за вас, и буду наставлять вас на путь добрый и прямой;
\vs 1Sa 12:24 только бойтесь Господа и служите Ему истинно, от всего сердца вашего, ибо вы видели, какие великие дела Он сделал с вами;
\vs 1Sa 12:25 если же вы будете делать зло, то и вы и царь ваш погибнете.
\vs 1Sa 13:1 Год был по воцарении Саула, и другой год царствовал он над Израилем, как выбрал Саул себе три тысячи из Израильтян:
\vs 1Sa 13:2 две тысячи были с Саулом в Михмасе и на горе Вефильской, тысяча же была с Ионафаном в Гиве Вениаминовой; а прочий народ отпустил он по домам своим.
\vs 1Sa 13:3 И разбил Ионафан охранный отряд Филистимский, который был в Гиве; и услышали об этом Филистимляне, а Саул протрубил трубою по всей стране, возглашая: да услышат Евреи!
\vs 1Sa 13:4 Когда весь Израиль услышал, что разбил Саул охранный отряд Филистимский и что Израиль сделался ненавистным для Филистимлян, то народ собрался к Саулу в Галгал.
\vs 1Sa 13:5 И собрались Филистимляне на войну против Израиля: тридцать тысяч колесниц и шесть тысяч конницы, и народа множество, как песок на берегу моря; и пришли и расположились станом в Михмасе, с восточной стороны Беф-Авена.
\vs 1Sa 13:6 Израильтяне, видя, что они в опасности, потому что народ был стеснен, укрывались в пещерах и в ущельях, и между скалами, и в башнях, и во рвах;
\vs 1Sa 13:7 а \bibemph{некоторые} из Евреев переправились за Иордан в страну Гадову и Галаадскую; Саул же находился еще в Галгале, и весь народ, бывший с ним, находился в страхе.
\vs 1Sa 13:8 И ждал он семь дней, до срока, \bibemph{назначенного} Самуилом, а Самуил не приходил в Галгал; и стал народ разбегаться от него.
\vs 1Sa 13:9 И сказал Саул: приведите ко мне, что \bibemph{назначено} для жертвы всесожжения и для жертв мирных. И вознес всесожжение.
\vs 1Sa 13:10 Но едва кончил он возношение всесожжения, вот, приходит Самуил; и вышел Саул к нему навстречу, чтобы приветствовать его.
\vs 1Sa 13:11 Но Самуил сказал: что ты сделал? Саул отвечал: я видел, что народ разбегается от меня, а ты не приходил к назначенному времени; Филистимляне же собрались в Михмасе;
\vs 1Sa 13:12 тогда подумал я: <<теперь придут на меня Филистимляне в Галгал, а я еще не вопросил Господа>>, и потому решился принести всесожжение.
\vs 1Sa 13:13 И сказал Самуил Саулу: худо поступил ты, что не исполнил повеления Господа Бога твоего, которое дано было тебе, ибо ныне упрочил бы Господь царствование твое над Израилем навсегда;
\vs 1Sa 13:14 но теперь не устоять царствованию твоему; Господь найдет Себе мужа по сердцу Своему, и повелит ему Господь быть вождем народа Своего, так как ты не исполнил того, что было повелено тебе Господом.
\vs 1Sa 13:15 И встал Самуил и пошел из Галгала в Гиву Вениаминову; [оставшиеся люди пошли за Саулом навстречу неприятельскому ополчению, которое нападало на них, когда они шли из Галгал в Гиву Вениаминову;] а Саул пересчитал людей, бывших с ним, до шестисот человек.
\vs 1Sa 13:16 Саул с сыном своим Ионафаном и людьми, находившимися при них, засели в Гиве Вениаминовой [и плакали]; Филистимляне же стояли станом в Михмасе.
\vs 1Sa 13:17 И вышли из стана Филистимского три отряда для опустошения земли: один направился по дороге к Офре, в округ Суаль,
\vs 1Sa 13:18 другой отряд направился по дороге Вефоронской, а третий направился по дороге к границе долины Цевоим, к пустыне.
\vs 1Sa 13:19 Кузнецов не было во всей земле Израильской; ибо Филистимляне опасались, чтобы Евреи не сделали меча или копья.
\vs 1Sa 13:20 И должны были ходить все Израильтяне к Филистимлянам оттачивать свои сошники, и свои заступы, и свои топоры, и свои кирки,
\vs 1Sa 13:21 когда сделается щербина на острие у сошников, и у заступов, и у вил, и у топоров, или нужно рожон поправить.
\vs 1Sa 13:22 Поэтому во время войны [Михмасской] не было ни меча, ни копья у всего народа, бывшего с Саулом и Ионафаном, а \bibemph{только} нашлись они у Саула и Ионафана, сына его.
\vs 1Sa 13:23 И вышел передовой отряд Филистимский к переправе Михмасской.
\vs 1Sa 14:1 В один день сказал Ионафан, сын Саулов, слуге оруженосцу своему: ступай, перейдем к отряду Филистимскому, что на той стороне. А отцу своему не сказал \bibemph{об этом}.
\vs 1Sa 14:2 Саул же находился в окраине Гивы, под гранатовым деревом, что в Мигроне. С ним было около шестисот человек народа
\vs 1Sa 14:3 и Ахия, сын Ахитува, брата Иохаведа, сына Финееса, сына Илия, священник Господа в Силоме, носивший ефод. Народ же не знал, что Ионафан пошел.
\vs 1Sa 14:4 Между переходами, по которым Ионафан искал пробраться к отряду Филистимскому, была острая скала с одной стороны и острая скала с другой: имя одной Боцец, а имя другой Сене;
\vs 1Sa 14:5 одна скала выдавалась с севера к Михмасу, другая с юга к Гиве.
\vs 1Sa 14:6 И сказал Ионафан слуге оруженосцу своему: ступай, перейдем к отряду этих необрезанных; может быть, Господь поможет нам, ибо для Господа нетрудно спасти чрез многих, или немногих.
\vs 1Sa 14:7 И отвечал оруженосец: делай все, что на сердце у тебя; иди, вот я с тобою, куда тебе угодно.
\vs 1Sa 14:8 И сказал Ионафан: вот, мы перейдем к этим людям и станем на виду у них;
\vs 1Sa 14:9 если они так скажут нам: <<остановитесь, пока мы подойдем к вам>>, то мы остановимся на своих местах и не взойдем к ним;
\vs 1Sa 14:10 а если так скажут: <<поднимитесь к нам>>, то мы взойдем, ибо Господь предал их в руки наши; и это будет знаком для нас.
\vs 1Sa 14:11 Когда оба они стали на виду у отряда Филистимского, то Филистимляне сказали: вот, Евреи выходят из ущелий, в которых попрятались они.
\vs 1Sa 14:12 И закричали люди, составлявшие отряд, к Ионафану и оруженосцу его, говоря: взойдите к нам, и мы вам скажем нечто. Тогда Ионафан сказал оруженосцу своему: следуй за мною, ибо Господь предал их в руки Израиля.
\vs 1Sa 14:13 И начал всходить Ионафан, \bibemph{цепляясь} руками и ногами, и оруженосец его за ним. И падали \bibemph{Филистимляне} пред Ионафаном, а оруженосец добивал их за ним.
\vs 1Sa 14:14 И пало от этого первого поражения, нанесенного Ионафаном и оруженосцем его, около двадцати человек, на половине поля, обрабатываемого парою волов в день.
\vs 1Sa 14:15 И произошел ужас в стане на поле и во всем народе; передовые отряды и опустошавшие землю пришли в трепет [и не хотели сражаться]; дрогнула вся земля, и был ужас великий от Господа.
\vs 1Sa 14:16 И увидели стражи Саула в Гиве Вениаминовой, что толпа рассеивается и бежит туда и сюда.
\vs 1Sa 14:17 И сказал Саул к народу, бывшему с ним; пересмотрите и узнайте, кто из наших вышел. И пересмотрели, и вот нет Ионафана и оруженосца его.
\vs 1Sa 14:18 И сказал Саул Ахии: <<принеси кивот\fns{В греческом переводе: ефод.} Божий>>, ибо кивот Божий в то время был с сынами Израильскими.
\vs 1Sa 14:19 Саул еще говорил к священнику, как смятение в стане Филистимском более и более [распространялось и] увеличивалось. Тогда сказал Саул священнику: сложи руки твои.
\vs 1Sa 14:20 И воскликнул Саул и весь народ, бывший с ним, и пришли к месту сражения, и вот, там меч каждого \bibemph{обращен} был против ближнего своего; смятение \bibemph{было} очень великое.
\vs 1Sa 14:21 Тогда и Евреи, которые вчера и третьего дня были у Филистимлян и которые повсюду ходили с ними в стане, пристали к Израильтянам, находившимся с Саулом и Ионафаном;
\vs 1Sa 14:22 и все Израильтяне, скрывавшиеся в горе Ефремовой, услышав, что Филистимляне побежали, также пристали к своим в сражении.
\vs 1Sa 14:23 И спас Господь в тот день Израиля; битва же простерлась даже до Беф-Авена. [Всех людей было с Саулом до десяти тысяч, и битва происходила по всему городу на горе Ефремовой.]
\rsbpar\vs 1Sa 14:24 Люди Израильские были истомлены в тот день; а Саул [весьма безрассудно] заклял народ, сказав: проклят, кто вкусит хлеба до вечера, доколе я не отомщу врагам моим. И никто из народа не вкусил пищи.
\vs 1Sa 14:25 И пошел весь народ в лес, и был там на поляне мед.
\vs 1Sa 14:26 И вошел народ в лес, говоря: вот, течет мед. Но никто не протянул руки своей ко рту своему, ибо народ боялся заклятия.
\vs 1Sa 14:27 Ионафан же не слышал, когда отец его заклинал народ, и, протянув конец палки, которая была в руке его, обмакнул ее в сот медовый и обратил рукою к устам своим, и просветлели глаза его.
\vs 1Sa 14:28 И сказал ему один из народа, говоря: отец твой заклял народ, сказав: <<проклят, кто сегодня вкусит пищи>>; от этого народ истомился.
\vs 1Sa 14:29 И сказал Ионафан: смутил отец мой землю; смотрите, у меня просветлели глаза, когда я вкусил немного этого меду;
\vs 1Sa 14:30 если бы поел сегодня народ из добычи, какую нашел у врагов своих, то не большее ли было бы поражение Филистимлян?
\vs 1Sa 14:31 И поражали Филистимлян в тот день от Михмаса до Аиалона, и народ очень истомился.
\vs 1Sa 14:32 И кинулся народ на добычу, и брали овец, волов и телят, и заколали на земле, и ел народ с кровью.
\vs 1Sa 14:33 И возвестили Саулу, говоря: вот, народ грешит пред Господом, ест с кровью. И сказал Саул: вы согрешили; привалите ко мне теперь большой камень.
\vs 1Sa 14:34 Потом сказал Саул: пройдите между народом и скажите ему: пусть каждый приводит ко мне своего вола и каждый свою овцу, и заколайте здесь и ешьте, и не грешите пред Господом, не ешьте с кровью. И приводили все из народа, каждый своею рукою, вола своего [и свою овцу] ночью, и заколали там.
\vs 1Sa 14:35 И устроил Саул жертвенник Господу: то был первый жертвенник, поставленный им Господу.
\vs 1Sa 14:36 И сказал Саул: пойдем \bibemph{в погоню} за Филистимлянами ночью и оберем их до рассвета и не оставим у них ни одного человека. И сказали: делай все, что хорошо в глазах твоих. Священник же сказал: приступим здесь к Богу.
\vs 1Sa 14:37 И вопросил Саул Бога: идти ли мне \bibemph{в погоню} за Филистимлянами? предашь ли их в руки Израиля? Но Он не отвечал ему в тот день.
\vs 1Sa 14:38 Тогда сказал Саул: пусть подойдут сюда все начальники народа и разведают и узнают, на ком грех ныне?
\vs 1Sa 14:39 ибо,~--- жив Господь, спасший Израиля,~--- если окажется и на Ионафане, сыне моем, то и он умрет непременно. Но никто не отвечал ему из всего народа.
\vs 1Sa 14:40 И сказал \bibemph{Саул} всем Израильтянам: станьте вы по одну сторону, а я и сын мой Ионафан станем по другую сторону. И отвечал народ Саулу: делай, что хорошо в глазах твоих.
\vs 1Sa 14:41 И сказал Саул: Господи, Боже Израилев! [отчего Ты ныне не отвечал рабу Твоему? моя ли в том вина, или сына моего Ионафана? Господи, Боже Израилев!] дай знамение. [Если же она в народе Твоем Израиле, дай ему освящение.] И уличены были Ионафан и Саул, а народ вышел \bibemph{правым}.
\vs 1Sa 14:42 Тогда сказал Саул: бросьте жребий между мною и между Ионафаном, сыном моим, [и кого объявит Господь, тот да умрет. И сказал народ Саулу: да не будет так! Но Саул настоял. И бросили жребий между ним и Ионафаном, сыном его,] и пал жребий на Ионафана.
\vs 1Sa 14:43 И сказал Саул Ионафану: расскажи мне, что сделал ты? И рассказал ему Ионафан и сказал: я отведал концом палки, которая в руке моей, немного меду; и вот, я должен умереть.
\vs 1Sa 14:44 И сказал Саул: пусть то и то сделает мне Бог, и еще больше сделает; ты, Ионафан, должен сегодня умереть!
\vs 1Sa 14:45 Но народ сказал Саулу: Ионафану ли умереть, который доставил столь великое спасение Израилю? Да не будет этого! Жив Господь, и волос не упадет с головы его на землю, ибо с Богом он действовал ныне. И освободил народ Ионафана, и не умер он.
\vs 1Sa 14:46 И возвратился Саул от преследования Филистимлян; Филистимляне же пошли в свое место.
\rsbpar\vs 1Sa 14:47 И утвердил Саул свое царствование над Израилем, и воевал со всеми окрестными врагами своими, с Моавом и с Аммонитянами, и с Едомом [и с Вефором] и с царями Совы и с Филистимлянами, и везде, против кого ни обращался, имел успех.
\vs 1Sa 14:48 И устроил войско, и поразил Амалика, и освободил Израиля от руки грабителей его.
\vs 1Sa 14:49 Сыновья у Саула были: Ионафан, Иессуи и Мелхисуа; а имена двух дочерей его: имя старшей~--- Мерова, а имя младшей~--- Мелхола.
\vs 1Sa 14:50 Имя же жены Сауловой~--- Ахиноамь, дочь Ахимааца; а имя начальника войска его~--- Авенир, сын Нира, дяди Саулова.
\vs 1Sa 14:51 Кис, отец Саулов, и Нир, отец Авенира, были сыновьями Авиила.
\vs 1Sa 14:52 И была упорная война против Филистимлян во все время Саулово. И когда Саул видел какого-либо человека сильного и воинственного, брал его к себе.
\vs 1Sa 15:1 И сказал Самуил Саулу: Господь послал меня помазать тебя царем над народом Его, над Израилем; теперь послушай гласа Господа.
\vs 1Sa 15:2 Так говорит Господь Саваоф: вспомнил Я о том, что сделал Амалик Израилю, как он противостал ему на пути, когда он шел из Египта;
\vs 1Sa 15:3 теперь иди и порази Амалика [и Иерима] и истреби все, что у него; [не бери себе ничего у них, но уничтожь и предай заклятию все, что у него;] и не давай пощады ему, но предай смерти от мужа до жены, от отрока до грудного младенца, от вола до овцы, от верблюда до осла.
\vs 1Sa 15:4 И собрал Саул народ и насчитал их в Телаиме двести тысяч Израильтян пеших и десять тысяч из колена Иудина.
\vs 1Sa 15:5 И дошел Саул до города Амаликова, и сделал засаду в долине.
\vs 1Sa 15:6 И сказал Саул Кинеянам: пойдите, отделитесь, выйдите из среды Амалика, чтобы мне не погубить вас с ним, ибо вы оказали благосклонность всем Израильтянам, когда они шли из Египта. И отделились Кинеяне из среды Амалика.
\vs 1Sa 15:7 И поразил Саул Амалика от Хавилы до окрестностей Сура, что пред Египтом;
\vs 1Sa 15:8 и Агага, царя Амаликова, захватил живого, а народ весь истребил мечом [и Иерима умертвил].
\vs 1Sa 15:9 Но Саул и народ пощадили Агага и лучших из овец и волов и откормленных ягнят, и все хорошее, и не хотели истребить, а все вещи маловажные и худые истребили.
\rsbpar\vs 1Sa 15:10 И было слово Господа к Самуилу такое:
\vs 1Sa 15:11 жалею, что поставил Я Саула царем, ибо он отвратился от Меня и слова Моего не исполнил. И опечалился Самуил и взывал к Господу целую ночь.
\vs 1Sa 15:12 И встал Самуил рано утром \bibemph{и пошел} навстречу Саулу. И известили Самуила, что Саул ходил на Кармил и там поставил себе памятник, [но оттуда возвратил колесницу] и сошел в Галгал.
\vs 1Sa 15:13 Когда пришел Самуил к Саулу, то Саул сказал ему: благословен ты у Господа; я исполнил слово Господа.
\vs 1Sa 15:14 И сказал Самуил: а что это за блеяние овец в ушах моих и мычание волов, которое я слышу?
\vs 1Sa 15:15 И сказал Саул: привели их от Амалика, так как народ пощадил лучших из овец и волов для жертвоприношения Господу Богу твоему; прочее же мы истребили.
\vs 1Sa 15:16 И сказал Самуил Саулу: подожди, я скажу тебе, что сказал мне Господь ночью. И сказал ему Саул: говори.
\vs 1Sa 15:17 И сказал Самуил: не малым ли ты был в глазах твоих, когда сделался главою колен Израилевых, и Господь помазал тебя царем над Израилем?
\vs 1Sa 15:18 И послал тебя Господь в путь, сказав: <<иди и предай заклятию нечестивых Амаликитян и воюй против них, доколе не уничтожишь их>>.
\vs 1Sa 15:19 Зачем же ты не послушал гласа Господа и бросился на добычу, и сделал зло пред очами Господа?
\vs 1Sa 15:20 И сказал Саул Самуилу: я послушал гласа Господа и пошел в путь, куда послал меня Господь, и привел Агага, царя Амаликитского, а Амалика истребил;
\vs 1Sa 15:21 народ же из добычи, из овец и волов, взял лучшее из заклятого, для жертвоприношения Господу Богу твоему, в Галгале.
\vs 1Sa 15:22 И отвечал Самуил: неужели всесожжения и жертвы столько же приятны Господу, как послушание гласу Господа? Послушание лучше жертвы и повиновение лучше тука овнов;
\vs 1Sa 15:23 ибо непокорность есть \bibemph{такой же} грех, что волшебство, и противление \bibemph{то же, что} идолопоклонство; за то, что ты отверг слово Господа, и Он отверг тебя, чтобы ты не был царем [над Израилем].
\vs 1Sa 15:24 И сказал Саул Самуилу: согрешил я, ибо преступил повеление Господа и слово твое; но я боялся народа и послушал голоса их;
\vs 1Sa 15:25 теперь же сними с меня грех мой и воротись со мною, чтобы я поклонился Господу [Богу твоему].
\vs 1Sa 15:26 И отвечал Самуил Саулу: не ворочусь я с тобою, ибо ты отверг слово Господа, и Господь отверг тебя, чтобы ты не был царем над Израилем.
\vs 1Sa 15:27 И обратился Самуил, чтобы уйти. Но [Саул] ухватился за край одежды его и разодрал ее.
\vs 1Sa 15:28 Тогда сказал Самуил: ныне отторг Господь царство Израильское от тебя и отдал его ближнему твоему, лучшему тебя;
\vs 1Sa 15:29 и не скажет неправды и не раскается Верный Израилев; ибо не человек Он, чтобы раскаяться Ему.
\vs 1Sa 15:30 И сказал [Саул]: я согрешил, но почти меня ныне пред старейшинами народа моего и пред Израилем и воротись со мною, и я поклонюсь Господу Богу твоему.
\vs 1Sa 15:31 И возвратился Самуил за Саулом, и поклонился Саул Господу.
\vs 1Sa 15:32 Потом сказал Самуил: приведите ко мне Агага, царя Амаликитского. И подошел к нему Агаг дрожащий, и сказал Агаг: конечно горечь смерти миновалась?
\vs 1Sa 15:33 Но Самуил сказал: как меч твой жен лишал детей, так мать твоя между женами пусть лишена будет \bibemph{сына}. И разрубил Самуил Агага пред Господом в Галгале.
\vs 1Sa 15:34 И отошел Самуил в Раму, а Саул пошел в дом свой, в Гиву Саулову.
\vs 1Sa 15:35 И более не видался Самуил с Саулом до дня смерти своей; но печалился Самуил о Сауле, потому что Господь раскаялся, что воцарил Саула над Израилем.
\vs 1Sa 16:1 И сказал Господь Самуилу: доколе будешь ты печалиться о Сауле, которого Я отверг, чтоб он не был царем над Израилем? Наполни рог твой елеем и пойди; Я пошлю тебя к Иессею Вифлеемлянину, ибо между сыновьями его Я усмотрел Себе царя.
\vs 1Sa 16:2 И сказал Самуил: как я пойду? Саул услышит и убьет меня. Господь сказал: возьми в руку твою телицу из стада и скажи: <<я пришел для жертвоприношения Господу>>;
\vs 1Sa 16:3 и пригласи Иессея [и сыновей его] к жертве; Я укажу тебе, что делать тебе, и ты помажешь Мне того, о котором Я скажу тебе.
\vs 1Sa 16:4 И сделал Самуил так, как сказал ему Господь. Когда пришел он в Вифлеем, то старейшины города с трепетом вышли навстречу ему и сказали: мирен ли приход твой?
\vs 1Sa 16:5 И отвечал он: мирен, для жертвоприношения Господу пришел я; освятитесь и идите со мною к жертвоприношению. И освятил Иессея и сыновей его и пригласил их к жертве.
\vs 1Sa 16:6 И когда они пришли, он, увидев Елиава, сказал: верно, сей пред Господом помазанник Его!
\vs 1Sa 16:7 Но Господь сказал Самуилу: не смотри на вид его и на высоту роста его; Я отринул его; Я \bibemph{смотрю не так}, как смотрит человек; ибо человек смотрит на лице, а Господь смотрит на сердце.
\vs 1Sa 16:8 И позвал Иессей Аминадава и подвел его к Самуилу, и сказал Самуил: и этого не избрал Господь.
\vs 1Sa 16:9 И подвел Иессей Самму, и сказал \bibemph{Самуил}: и этого не избрал Господь.
\vs 1Sa 16:10 Так подводил Иессей к Самуилу семерых сыновей своих, но Самуил сказал Иессею: \bibemph{никого} из этих не избрал Господь.
\vs 1Sa 16:11 И сказал Самуил Иессею: все ли дети здесь? И отвечал Иессей: есть еще меньший; он пасет овец. И сказал Самуил Иессею: пошли и возьми его, ибо мы не сядем обедать, доколе не придет он сюда.
\vs 1Sa 16:12 И послал \bibemph{Иессей} и привели его. Он был белокур, с красивыми глазами и приятным лицем. И сказал Господь: встань, помажь его, ибо это он.
\rsbpar\vs 1Sa 16:13 И взял Самуил рог с елеем и помазал его среди братьев его, и почивал Дух Господень на Давиде с того дня и после; Самуил же встал и отошел в Раму.
\vs 1Sa 16:14 А от Саула отступил Дух Господень, и возмущал его злой дух от Господа.
\vs 1Sa 16:15 И сказали слуги Сауловы ему: вот, злой дух от Бога возмущает тебя;
\vs 1Sa 16:16 пусть господин наш прикажет слугам своим, \bibemph{которые} пред тобою, поискать человека, искусного в игре на гуслях, и когда придет на тебя злой дух от Бога, то он, играя рукою своею, будет успокоивать тебя.
\vs 1Sa 16:17 И отвечал Саул слугам своим: найдите мне человека, хорошо играющего, и представьте его ко мне.
\vs 1Sa 16:18 Тогда один из слуг его сказал: вот, я видел у Иессея Вифлеемлянина сына, умеющего играть, человека храброго и воинственного, и разумного в речах и видного собою, и Господь с ним.
\vs 1Sa 16:19 И послал Саул вестников к Иессею и сказал: пошли ко мне Давида, сына твоего, который при стаде.
\vs 1Sa 16:20 И взял Иессей осла с хлебом и мех с вином и одного козленка, и послал с Давидом, сыном своим, к Саулу.
\vs 1Sa 16:21 И пришел Давид к Саулу и служил пред ним, и очень понравился ему и сделался его оруженосцем.
\vs 1Sa 16:22 И послал Саул сказать Иессею: пусть Давид служит при мне, ибо он снискал благоволение в глазах моих.
\vs 1Sa 16:23 И когда дух от Бога бывал на Сауле, то Давид, взяв гусли, играл,~--- и отраднее и лучше становилось Саулу, и дух злой отступал от него.
\vs 1Sa 17:1 Филистимляне собрали войска свои для войны и собрались в Сокхофе, что в Иудее, и расположились станом между Сокхофом и Азеком в Ефес-Даммиме.
\vs 1Sa 17:2 А Саул и Израильтяне собрались и расположились станом в долине дуба и приготовились к войне против Филистимлян.
\vs 1Sa 17:3 И стали Филистимляне на горе с одной стороны, и Израильтяне на горе с другой стороны, а между ними была долина.
\vs 1Sa 17:4 И выступил из стана Филистимского единоборец, по имени Голиаф, из Гефа; ростом он~--- шести локтей и пяди.
\vs 1Sa 17:5 Медный шлем на голове его; и одет он был в чешуйчатую броню, и вес брони его~--- пять тысяч сиклей меди;
\vs 1Sa 17:6 медные наколенники на ногах его, и медный щит за плечами его;
\vs 1Sa 17:7 и древко копья его, как навой у ткачей; а самое копье его в шестьсот сиклей железа, и пред ним шел оруженосец.
\vs 1Sa 17:8 И стал \bibemph{он} и кричал к полкам Израильским, говоря им: зачем вышли вы воевать? Не Филистимлянин ли я, а вы рабы Сауловы? Выберите у себя человека, и пусть сойдет ко мне;
\vs 1Sa 17:9 если он может сразиться со мною и убьет меня, то мы будем вашими рабами; если же я одолею его и убью его, то вы будете нашими рабами и будете служить нам.
\vs 1Sa 17:10 И сказал Филистимлянин: сегодня я посрамлю полки Израильские; дайте мне человека, и мы сразимся вдвоем.
\vs 1Sa 17:11 И услышали Саул и все Израильтяне эти слова Филистимлянина, и очень испугались и ужаснулись.
\rsbpar\vs 1Sa 17:12 Давид же был сын Ефрафянина из Вифлеема Иудина, по имени Иессея, у которого было восемь сыновей. Этот человек во дни Саула достиг старости и был старший между мужами.
\vs 1Sa 17:13 Три старших сына Иессеевы пошли с Саулом на войну; имена трех сыновей его, пошедших на войну: старший~--- Елиав, второй за ним~--- Аминадав, и третий~--- Самма;
\vs 1Sa 17:14 Давид же был меньший. Трое старших пошли с Саулом,
\vs 1Sa 17:15 а Давид возвратился от Саула, чтобы пасти овец отца своего в Вифлееме.
\vs 1Sa 17:16 И выступал Филистимлянин тот утром и вечером и выставлял себя сорок дней.
\vs 1Sa 17:17 И сказал Иессей Давиду, сыну своему: возьми для братьев своих ефу сушеных зерен и десять этих хлебов и отнеси поскорее в стан к твоим братьям;
\vs 1Sa 17:18 а эти десять сыров отнеси тысяченачальнику и наведайся о здоровье братьев и узнай о нуждах их.
\vs 1Sa 17:19 Саул и они и все Израильтяне \bibemph{находились} в долине дуба и готовились к сражению с Филистимлянами.
\vs 1Sa 17:20 И встал Давид рано утром, и поручил овец сторожу, и, взяв ношу, пошел, как приказал ему Иессей, и пришел к обозу, когда войско выведено было в строй и с криком готовилось к сражению.
\vs 1Sa 17:21 И расположили Израильтяне и Филистимляне строй против строя.
\vs 1Sa 17:22 Давид оставил свою ношу обозному сторожу и побежал в ряды и, придя, спросил братьев своих о здоровье.
\vs 1Sa 17:23 И вот, когда он разговаривал с ними, единоборец, по имени Голиаф, Филистимлянин из Гефа, выступает из рядов Филистимских и говорит те слова, и Давид услышал \bibemph{их}.
\vs 1Sa 17:24 И все Израильтяне, увидев этого человека, убегали от него и весьма боялись.
\vs 1Sa 17:25 И говорили Израильтяне: видите этого выступающего человека? Он выступает, чтобы поносить Израиля. Если бы кто убил его, одарил бы того царь великим богатством, и дочь свою выдал бы за него, и дом отца его сделал бы свободным в Израиле.
\vs 1Sa 17:26 И сказал Давид людям, стоящим с ним: что сделают тому, кто убьет этого Филистимлянина и снимет поношение с Израиля? ибо кто этот необрезанный Филистимлянин, что так поносит воинство Бога живаго?
\vs 1Sa 17:27 И сказал ему народ те же слова, говоря: вот что сделано будет тому человеку, который убьет его.
\vs 1Sa 17:28 И услышал Елиав, старший брат Давида, что говорил он с людьми, и рассердился Елиав на Давида и сказал: зачем ты сюда пришел и на кого оставил немногих овец тех в пустыне? Я знаю высокомерие твое и дурное сердце твое, ты пришел посмотреть на сражение.
\vs 1Sa 17:29 И сказал Давид: что же я сделал? не слова ли это?
\vs 1Sa 17:30 И отворотился от него к другому и говорил те же слова, и отвечал ему народ по-прежнему.
\vs 1Sa 17:31 И услышали слова, которые говорил Давид, и пересказали Саулу, и тот призвал его.
\vs 1Sa 17:32 И сказал Давид Саулу: пусть никто не падает духом из-за него; раб твой пойдет и сразится с этим Филистимлянином.
\vs 1Sa 17:33 И сказал Саул Давиду: не можешь ты идти против этого Филистимлянина, чтобы сразиться с ним, ибо ты еще юноша, а он воин от юности своей.
\vs 1Sa 17:34 И сказал Давид Саулу: раб твой пас овец у отца своего, и когда, бывало, приходил лев или медведь и уносил овцу из стада,
\vs 1Sa 17:35 то я гнался за ним и нападал на него и отнимал из пасти его; а если он бросался на меня, то я брал его за космы и поражал его и умерщвлял его;
\vs 1Sa 17:36 и льва и медведя убивал раб твой, и с этим Филистимлянином необрезанным будет то же, что с ними, потому что так поносит воинство Бога живаго. [Не пойти ли мне и поразить его, чтобы снять поношение с Израиля? Ибо кто этот необрезанный?]
\vs 1Sa 17:37 И сказал Давид: Господь, Который избавлял меня от льва и медведя, избавит меня и от руки этого Филистимлянина. И сказал Саул Давиду: иди, и да будет Господь с тобою.
\vs 1Sa 17:38 И одел Саул Давида в свои одежды, и возложил на голову его медный шлем, и надел на него броню.
\vs 1Sa 17:39 И опоясался Давид мечом его сверх одежды и начал ходить, ибо не привык \bibemph{к такому вооружению}; потом сказал Давид Саулу: я не могу ходить в этом, я не привык. И снял Давид все это с себя.
\vs 1Sa 17:40 И взял посох свой в руку свою, и выбрал себе пять гладких камней из ручья, и положил их в пастушескую сумку, которая была с ним; и с сумкою и с пращею в руке своей выступил против Филистимлянина.
\vs 1Sa 17:41 Выступил и Филистимлянин, идя и приближаясь к Давиду, и оруженосец шел впереди его.
\vs 1Sa 17:42 И взглянул Филистимлянин и, увидев Давида, с презрением посмотрел на него, ибо он был молод, белокур и красив лицем.
\vs 1Sa 17:43 И сказал Филистимлянин Давиду: что ты идешь на меня с палкою [и с камнями]? разве я собака? [И сказал Давид: нет, но хуже собаки.] И проклял Филистимлянин Давида своими богами.
\vs 1Sa 17:44 И сказал Филистимлянин Давиду: подойди ко мне, и я отдам тело твое птицам небесным и зверям полевым.
\vs 1Sa 17:45 А Давид отвечал Филистимлянину: ты идешь против меня с мечом и копьем и щитом, а я иду против тебя во имя Господа Саваофа, Бога воинств Израильских, которые ты поносил;
\vs 1Sa 17:46 ныне предаст тебя Господь в руку мою, и я убью тебя, и сниму с тебя голову твою, и отдам [труп твой и] трупы войска Филистимского птицам небесным и зверям земным, и узнает вся земля, что есть Бог в Израиле;
\vs 1Sa 17:47 и узнает весь этот сонм, что не мечом и копьем спасает Господь, ибо это война Господа, и Он предаст вас в руки наши.
\vs 1Sa 17:48 Когда Филистимлянин поднялся и стал подходить и приближаться навстречу Давиду, Давид поспешно побежал к строю навстречу Филистимлянину.
\vs 1Sa 17:49 И опустил Давид руку свою в сумку и взял оттуда камень, и бросил из пращи и поразил Филистимлянина в лоб, так что камень вонзился в лоб его, и он упал лицем на землю.
\vs 1Sa 17:50 Так одолел Давид Филистимлянина пращею и камнем, и поразил Филистимлянина и убил его; меча же не было в руках Давида.
\vs 1Sa 17:51 Тогда Давид подбежал и, наступив на Филистимлянина, взял меч его и вынул его из ножен, ударил его и отсек им голову его; Филистимляне, увидев, что силач их умер, побежали.
\vs 1Sa 17:52 И поднялись мужи Израильские и Иудейские, и воскликнули и гнали Филистимлян до входа в долину и до ворот Аккарона. И падали поражаемые Филистимляне по дороге Шааримской до Гефа и до Аккарона.
\vs 1Sa 17:53 И возвратились сыны Израилевы из погони за Филистимлянами и разграбили стан их.
\vs 1Sa 17:54 И взял Давид голову Филистимлянина и отнес ее в Иерусалим, а оружие его положил в шатре своем.
\vs 1Sa 17:55 Когда Саул увидел Давида, выходившего против Филистимлянина, то сказал Авениру, начальнику войска: Авенир, чей сын этот юноша? Авенир сказал: да живет душа твоя, царь; я не знаю.
\vs 1Sa 17:56 И сказал царь: так спроси, чей сын этот юноша?
\vs 1Sa 17:57 Когда же Давид возвращался после поражения Филистимлянина, то Авенир взял его и привел к Саулу, и голова Филистимлянина была в руке его.
\vs 1Sa 17:58 И спросил его Саул: чей ты сын, юноша? И отвечал Давид: сын раба твоего Иессея из Вифлеема.
\vs 1Sa 18:1 Когда кончил \bibemph{Давид} разговор с Саулом, душа Ионафана прилепилась к душе его, и полюбил его Ионафан, как свою душу.
\vs 1Sa 18:2 И взял его Саул в тот день и не позволил ему возвратиться в дом отца его.
\vs 1Sa 18:3 Ионафан же заключил с Давидом союз, ибо полюбил его, как свою душу.
\vs 1Sa 18:4 И снял Ионафан верхнюю одежду свою, которая была на нем, и отдал ее Давиду, также и прочие одежды свои, и меч свой, и лук свой, и пояс свой.
\rsbpar\vs 1Sa 18:5 И Давид действовал благоразумно везде, куда ни посылал его Саул, и сделал его Саул начальником над военными людьми; и это понравилось всему народу и слугам Сауловым.
\vs 1Sa 18:6 Когда они шли, при возвращении Давида с победы над Филистимлянином, то женщины из всех городов Израильских выходили навстречу Саулу царю с пением и плясками, с торжественными тимпанами и с кимвалами.
\vs 1Sa 18:7 И восклицали игравшие женщины, говоря: Саул победил тысячи, а Давид~--- десятки тысяч!
\vs 1Sa 18:8 И Саул сильно огорчился, и неприятно было ему это слово, и он сказал: Давиду дали десятки тысяч, а мне тысячи; ему недостает только царства.
\vs 1Sa 18:9 И с того дня и потом подозрительно смотрел Саул на Давида.
\vs 1Sa 18:10 И было на другой день: напал злой дух от Бога на Саула, и он бесновался в доме своем, а Давид играл рукою своею на струнах, как и в другие дни; в руке у Саула было копье.
\vs 1Sa 18:11 И бросил Саул копье, подумав: пригвожду Давида к стене; но Давид два раза уклонился от него.
\vs 1Sa 18:12 И стал бояться Саул Давида, потому что Господь был с ним, а от Саула отступил.
\vs 1Sa 18:13 И удалил его Саул от себя и поставил его у себя тысяченачальником, и он выходил и входил пред народом.
\vs 1Sa 18:14 И Давид во всех делах своих поступал благоразумно, и Господь \bibemph{был} с ним.
\vs 1Sa 18:15 И Саул видел, что он очень благоразумен, и боялся его.
\vs 1Sa 18:16 А весь Израиль и Иуда любили Давида, ибо он выходил и входил пред ними.
\vs 1Sa 18:17 И сказал Саул Давиду: вот старшая дочь моя, Мерова; я дам ее тебе в жену, только будь у меня храбрым и веди войны Господни. Ибо Саул думал: пусть не моя рука будет на нем, но рука Филистимлян будет на нем.
\vs 1Sa 18:18 Но Давид сказал Саулу: кто я, и что жизнь моя и род отца моего в Израиле, чтобы мне быть зятем царя?
\vs 1Sa 18:19 А когда наступило время отдать Мерову, дочь Саула, Давиду, то она выдана была в замужество за Адриэла из Мехолы.
\vs 1Sa 18:20 Но Давида полюбила \bibemph{другая} дочь Саула, Мелхола; и когда возвестили \bibemph{об этом} Саулу, то это было приятно ему.
\vs 1Sa 18:21 Саул думал: отдам ее за него, и она будет ему сетью, и рука Филистимлян будет на нем. И сказал Саул Давиду: чрез другую ты породнишься ныне со мною.
\vs 1Sa 18:22 И приказал Саул слугам своим: скажите Давиду тайно: вот, царь благоволит к тебе, и все слуги его любят тебя; итак будь зятем царя.
\vs 1Sa 18:23 И передали слуги Сауловы в уши Давиду все слова эти. И сказал Давид: разве легко кажется вам быть зятем царя? я~--- человек бедный и незначительный.
\vs 1Sa 18:24 И донесли Саулу слуги его и сказали: вот что говорит Давид.
\vs 1Sa 18:25 И сказал Саул: так скажите Давиду: царь не хочет вена, кроме ста краеобрезаний Филистимских, в отмщение врагам царя. Ибо Саул имел в мыслях погубить Давида руками Филистимлян.
\vs 1Sa 18:26 И пересказали слуги его Давиду эти слова, и понравилось Давиду сделаться зятем царя.
\vs 1Sa 18:27 Еще не прошли назначенные дни, как Давид встал и пошел сам и люди его с ним, и убил двести человек Филистимлян, и принес Давид краеобрезания их, и представил их в полном количестве царю, чтобы сделаться зятем царя. И выдал Саул за него Мелхолу, дочь свою, в замужество.
\vs 1Sa 18:28 И увидел Саул и узнал, что Господь с Давидом [и весь Израиль любит его,] и что Мелхола, дочь Саула, любила \bibemph{Давида}.
\vs 1Sa 18:29 И стал Саул еще больше бояться Давида и сделался врагом его на всю жизнь.
\vs 1Sa 18:30 И когда вожди Филистимские вышли \bibemph{на войну}, Давид, с самого выхода их, действовал благоразумнее всех слуг Сауловых, и весьма прославилось имя его.
\vs 1Sa 19:1 И говорил Саул Ионафану, сыну своему, и всем слугам своим, чтобы умертвить Давида; но Ионафан, сын Саула, очень любил Давида.
\vs 1Sa 19:2 И известил Ионафан Давида, говоря: отец мой Саул ищет умертвить тебя; итак берегись завтра; скройся и будь в потаенном месте;
\vs 1Sa 19:3 а я выйду и стану подле отца моего на поле, где ты будешь, и поговорю о тебе отцу моему, и что увижу, расскажу тебе.
\vs 1Sa 19:4 И говорил Ионафан доброе о Давиде Саулу, отцу своему, и сказал ему: да не грешит царь против раба своего Давида, ибо он ничем не согрешил против тебя, и дела его весьма полезны для тебя;
\vs 1Sa 19:5 он подвергал опасности душу свою, чтобы поразить Филистимлянина, и Господь соделал великое спасение всему Израилю; ты видел \bibemph{это} и радовался; для чего же ты хочешь согрешить \bibemph{против} невинной крови и умертвить Давида без причины?
\vs 1Sa 19:6 И послушал Саул голоса Ионафана и поклялся Саул: жив Господь, \bibemph{Давид} не умрет.
\vs 1Sa 19:7 И позвал Ионафан Давида, и пересказал ему Ионафан все слова сии, и привел Ионафан Давида к Саулу, и он был при нем, как вчера и третьего дня.
\vs 1Sa 19:8 Опять началась война, и вышел Давид, и воевал с Филистимлянами, и нанес им великое поражение, и они побежали от него.
\vs 1Sa 19:9 И злой дух от Бога напал на Саула, и он сидел в доме своем, и копье его было в руке его, а Давид играл рукою своею на струнах.
\vs 1Sa 19:10 И хотел Саул пригвоздить Давида копьем к стене, но Давид отскочил от Саула, и копье вонзилось в стену; Давид же убежал и спасся в ту ночь.
\vs 1Sa 19:11 И послал Саул слуг в дом к Давиду, чтобы стеречь его и убить его до утра. И сказала Давиду Мелхола, жена его: если ты не спасешь души твоей в эту ночь, то завтра будешь убит.
\vs 1Sa 19:12 И спустила Мелхола Давида из окна, и он пошел, и убежал и спасся.
\vs 1Sa 19:13 Мелхола же взяла статую и положила на постель, а в изголовье ее положила козью кожу, и покрыла одеждою.
\vs 1Sa 19:14 И послал Саул слуг, чтобы взять Давида; но \bibemph{Мелхола} сказала: он болен.
\vs 1Sa 19:15 И послал Саул слуг, чтобы осмотреть Давида, говоря: принесите его ко мне на постели, чтоб убить его.
\vs 1Sa 19:16 И пришли слуги, и вот, на постели статуя, а в изголовье ее козья кожа.
\vs 1Sa 19:17 Тогда Саул сказал Мелхоле: для чего ты так обманула меня и отпустила врага моего, чтоб он убежал? И сказала Мелхола Саулу: он сказал мне: отпусти меня, иначе я убью тебя.
\vs 1Sa 19:18 И убежал Давид и спасся, и пришел к Самуилу в Раму и рассказал ему все, что делал с ним Саул. И пошел он с Самуилом, и остановились они в Навафе [в Раме].
\vs 1Sa 19:19 И донесли Саулу, говоря: вот, Давид в Навафе, в Раме.
\vs 1Sa 19:20 И послал Саул слуг взять Давида, и \bibemph{когда} увидели они сонм пророков пророчествующих и Самуила, начальствующего над ними, то Дух Божий сошел на слуг Саула, и они стали пророчествовать.
\vs 1Sa 19:21 Донесли \bibemph{об этом} Саулу, и он послал других слуг, но и эти стали пророчествовать. Потом послал Саул третьих слуг, и эти стали пророчествовать.
\vs 1Sa 19:22 [Разгневавшись,] Саул сам пошел в Раму, и дошел до большого источника, что в Сефе, и спросил, говоря: где Самуил и Давид? И сказали: вот, в Навафе, в Раме.
\vs 1Sa 19:23 И пошел он туда в Наваф в Раме, и на него сошел Дух Божий, и он шел и пророчествовал, доколе не пришел в Наваф в Раме.
\vs 1Sa 19:24 И снял и он одежды свои, и пророчествовал пред Самуилом, и весь день тот и всю ту ночь лежал неодетый; поэтому говорят: <<неужели и Саул во пророках?>>
\vs 1Sa 20:1 Давид убежал из Навафа в Раме и пришел и сказал Ионафану: что сделал я, в чем неправда моя, чем согрешил я пред отцом твоим, что он ищет души моей?
\vs 1Sa 20:2 И сказал ему [Ионафан]: нет, ты не умрешь; вот, отец мой не делает ни большого, ни малого дела, не открыв ушам моим; для чего же бы отцу моему скрывать от меня это дело? этого не будет.
\vs 1Sa 20:3 Давид клялся и говорил: отец твой хорошо знает, что я нашел благоволение в очах твоих, и потому говорит сам в себе: <<пусть не знает о том Ионафан, чтобы не огорчился>>; но жив Господь и жива душа твоя! один только шаг между мною и смертью.
\vs 1Sa 20:4 И сказал Ионафан Давиду: чего желает душа твоя, я сделаю для тебя.
\vs 1Sa 20:5 И сказал Давид Ионафану: вот, завтра новомесячие, и я должен сидеть с царем за столом; но отпусти меня, и я скроюсь в поле до вечера третьего дня.
\vs 1Sa 20:6 Если отец твой спросит обо мне, ты скажи: <<Давид выпросился у меня сходить в свой город Вифлеем; потому что там годичное жертвоприношение всего родства его>>.
\vs 1Sa 20:7 Если на это он скажет: <<хорошо>>, то мир рабу твоему; а если он разгневается, то знай, что злое дело решено у него.
\vs 1Sa 20:8 Ты же сделай милость рабу твоему,~--- ибо ты принял раба твоего в завет Господень с тобою,~--- и если есть какая вина на мне, то умертви ты меня; зачем тебе вести меня к отцу твоему?
\vs 1Sa 20:9 И сказал Ионафан: никак не будет этого с тобою; ибо, если я узнаю наверное, что у отца моего решено злое дело совершить над тобою, то неужели не извещу тебя об этом?
\vs 1Sa 20:10 И сказал Давид Ионафану: кто известит меня, если отец твой ответит тебе сурово?
\vs 1Sa 20:11 И сказал Ионафан Давиду: иди, выйдем в поле. И вышли оба в поле.
\vs 1Sa 20:12 И сказал Ионафан Давиду: жив Господь Бог Израилев! я завтра около этого времени, или послезавтра, выпытаю у отца моего; и если он благосклонен к Давиду, и я тогда же не пошлю к тебе и не открою пред ушами твоими,
\vs 1Sa 20:13 пусть то и то сделает Господь с Ионафаном и еще больше сделает. Если же отец мой замышляет сделать тебе зло, и это открою в уши твои, и отпущу тебя, и тогда иди с миром: и да будет Господь с тобою, как был с отцом моим!
\vs 1Sa 20:14 Но и ты, если я буду еще жив, окажи мне милость Господню.
\vs 1Sa 20:15 А если я умру, то не отними милости твоей от дома моего во веки, даже и тогда, когда Господь истребит с лица земли всех врагов Давида.
\vs 1Sa 20:16 Так заключил Ионафан завет с домом Давида \bibemph{и сказал}: да взыщет Господь с врагов Давида!
\vs 1Sa 20:17 И снова Ионафан клялся Давиду своею любовью к нему, ибо любил его, как свою душу.
\vs 1Sa 20:18 И сказал ему Ионафан: завтра новомесячие, и о тебе спросят, ибо место твое будет не занято;
\vs 1Sa 20:19 поэтому на третий день ты спустись и поспеши на то место, где скрывался ты прежде, и сядь у камня Азель;
\vs 1Sa 20:20 а я в ту сторону пущу три стрелы, как будто стреляя в цель;
\vs 1Sa 20:21 потом пошлю отрока, \bibemph{говоря}: <<пойди, найди стрелы>>; и если я скажу отроку: <<вот, стрелы сзади тебя, возьми их>>, то приди ко мне, ибо мир тебе, и, жив Господь, ничего \bibemph{тебе не будет};
\vs 1Sa 20:22 если же так скажу отроку: <<вот, стрелы впереди тебя>>, то ты уходи, ибо отпускает тебя Господь;
\vs 1Sa 20:23 а тому, что мы говорили, я и ты, \bibemph{свидетель} Господь между мною и тобою во веки.
\rsbpar\vs 1Sa 20:24 И скрылся Давид на поле. И наступило новомесячие, и сел царь обедать.
\vs 1Sa 20:25 Царь сел на своем месте, по обычаю, на седалище у стены, и Ионафан встал, и Авенир сел подле Саула; место же Давида осталось праздным.
\vs 1Sa 20:26 И не сказал Саул в тот день ничего, ибо подумал, что это случайность, что \bibemph{Давид} нечист, не очистился.
\vs 1Sa 20:27 Наступил и второй день новомесячия, а место Давида оставалось праздным. Тогда сказал Саул сыну своему Ионафану: почему сын Иессеев не пришел к обеду ни вчера, ни сегодня?
\vs 1Sa 20:28 И отвечал Ионафан Саулу: Давид выпросился у меня в Вифлеем;
\vs 1Sa 20:29 он говорил: <<отпусти меня, ибо у нас в городе родственное жертвоприношение, и мой брат пригласил меня; итак, если я нашел благоволение в очах твоих, схожу я и повидаюсь со своими братьями>>; поэтому он и не пришел к обеду царя.
\vs 1Sa 20:30 Тогда сильно разгневался Саул на Ионафана и сказал ему: сын негодный и непокорный! разве я не знаю, что ты подружился с сыном Иессеевым на срам себе и на срам матери твоей?
\vs 1Sa 20:31 ибо во все дни, доколе сын Иессеев будет жить на земле, не устоишь ни ты, ни царство твое; теперь же пошли и приведи его ко мне, ибо он обречен на смерть.
\vs 1Sa 20:32 И отвечал Ионафан Саулу, отцу своему, и сказал ему: за что умерщвлять его? что он сделал?
\vs 1Sa 20:33 Тогда Саул бросил копье в него, чтобы поразить его. И Ионафан понял, что отец его решился убить Давида.
\vs 1Sa 20:34 И встал Ионафан из-за стола в великом гневе и не обедал во второй день новомесячия, потому что скорбел о Давиде и потому что обидел его отец его.
\vs 1Sa 20:35 На другой день утром вышел Ионафан в поле, во время, которое назначил Давиду, и малый отрок с ним.
\vs 1Sa 20:36 И сказал он отроку: беги, ищи стрелы, которые я пускаю. Отрок побежал, а он пускал стрелы так, что они летели дальше \bibemph{отрока}.
\vs 1Sa 20:37 И побежал отрок туда, куда Ионафан пускал стрелы, и закричал Ионафан вслед отроку и сказал: смотри, стрела впереди тебя.
\vs 1Sa 20:38 И опять кричал Ионафан вслед отроку: скорей беги, не останавливайся. И собрал отрок Ионафанов стрелы и пришел к своему господину.
\vs 1Sa 20:39 Отрок же не знал ничего; только Ионафан и Давид знали, в чем дело.
\vs 1Sa 20:40 И отдал Ионафан оружие свое отроку, бывшему при нем, и сказал ему: ступай, отнеси в город.
\vs 1Sa 20:41 Отрок пошел, а Давид поднялся с южной стороны и пал лицем своим на землю и трижды поклонился; и целовали они друг друга, и плакали оба вместе, но Давид плакал более.
\vs 1Sa 20:42 И сказал Ионафан Давиду: иди с миром; а в чем клялись мы оба именем Господа, говоря: <<Господь да будет между мною и между тобою и между семенем моим и семенем твоим>>, то да будет на веки.
\vs 1Sa 20:43 И встал [Давид] и пошел, а Ионафан возвратился в город.
\vs 1Sa 21:1 И пришел Давид в Номву к Ахимелеху священнику, и смутился Ахимелех при встрече с Давидом и сказал ему: почему ты один, и никого нет с тобою?
\vs 1Sa 21:2 И сказал Давид Ахимелеху священнику: царь поручил мне дело и сказал мне: <<пусть никто не знает, за чем я послал тебя и что поручил тебе>>; поэтому людей я оставил на известном месте;
\vs 1Sa 21:3 итак, что есть у тебя под рукою, дай мне, хлебов пять, или что найдется.
\vs 1Sa 21:4 И отвечал священник Давиду, говоря: нет у меня под рукою простого хлеба, а есть хлеб священный; если только люди \bibemph{твои} воздержались от женщин, [пусть съедят].
\vs 1Sa 21:5 И отвечал Давид священнику и сказал ему: женщин при нас не было ни вчера, ни третьего дня, со времени, как я вышел, и сосуды отроков чисты, а если дорога нечиста, то \bibemph{хлеб} останется чистым в сосудах.
\vs 1Sa 21:6 И дал ему священник священного хлеба; ибо не было у него хлеба, кроме хлебов предложения, которые взяты были от лица Господа, чтобы по снятии их положить теплые хлебы.
\vs 1Sa 21:7 Там находился в тот день пред Господом один из слуг Сауловых, по имени Доик, Идумеянин, начальник пастухов Сауловых.
\vs 1Sa 21:8 И сказал Давид Ахимелеху: нет ли здесь у тебя под рукою копья или меча? ибо я не взял с собою ни меча, ни другого оружия, так как поручение царя было спешное.
\vs 1Sa 21:9 И сказал священник: вот меч Голиафа Филистимлянина, которого ты поразил в долине дуба, завернутый в одежду, позади ефода; если хочешь, возьми его; другого кроме этого нет здесь. И сказал Давид: нет ему подобного, дай мне его. [И дал ему.]
\vs 1Sa 21:10 И встал Давид, и убежал в тот же день от Саула, и пришел к Анхусу, царю Гефскому.
\vs 1Sa 21:11 И сказали Анхусу слуги его: не это ли Давид, царь той страны? не ему ли пели в хороводах и говорили: <<Саул поразил тысячи, а Давид~--- десятки тысяч>>?
\vs 1Sa 21:12 Давид положил слова эти в сердце своем и сильно боялся Анхуса, царя Гефского.
\vs 1Sa 21:13 И изменил лице свое пред ними, и притворился безумным в их глазах, и чертил на дверях, [кидался на руки свои] и пускал слюну по бороде своей.
\vs 1Sa 21:14 И сказал Анхус рабам своим: видите, он человек сумасшедший; для чего вы привели его ко мне?
\vs 1Sa 21:15 разве мало у меня сумасшедших, что вы привели его, чтобы он юродствовал предо мною? неужели он войдет в дом мой?
\vs 1Sa 22:1 И вышел Давид оттуда и убежал в пещеру Одолламскую, и услышали братья его и весь дом отца его и пришли к нему туда.
\vs 1Sa 22:2 И собрались к нему все притесненные и все должники и все огорченные душею, и сделался он начальником над ними; и было с ним около четырехсот человек.
\vs 1Sa 22:3 Оттуда пошел Давид в Массифу Моавитскую и сказал царю Моавитскому: пусть отец мой и мать моя побудут у вас, доколе я не узнаю, что сделает со мною Бог.
\vs 1Sa 22:4 И привел их к царю Моавитскому, и жили они у него все время, доколе Давид был в оном убежище.
\vs 1Sa 22:5 Но пророк Гад сказал Давиду: не оставайся в этом убежище, но ступай, иди в землю Иудину. И пошел Давид и пришел в лес Херет.
\vs 1Sa 22:6 И услышал Саул, что Давид появился и люди, бывшие с ним. Саул сидел тогда в Гиве под дубом на горе, с копьем в руке, и все слуги его окружали его.
\vs 1Sa 22:7 И сказал Саул слугам своим, окружавшим его: послушайте, сыны Вениаминовы, неужели всем вам даст сын Иессея поля и виноградники и всех вас поставит тысяченачальниками и сотниками,
\vs 1Sa 22:8 что вы все сговорились против меня, и никто не открыл мне, когда сын мой вступил в дружбу с сыном Иессея, и никто из вас не пожалел о мне и не открыл мне, что сын мой возбудил против меня раба моего строить мне ковы, как это ныне видно?
\vs 1Sa 22:9 И отвечал Доик Идумеянин, стоявший со слугами Сауловыми, и сказал: я видел, как сын Иессея приходил в Номву к Ахимелеху, сыну Ахитува,
\vs 1Sa 22:10 и тот вопросил о нем Господа, и дал ему продовольствие, и меч Голиафа Филистимлянина отдал ему.
\vs 1Sa 22:11 И послал царь призвать Ахимелеха, сына Ахитувова, священника, и весь дом отца его, священников, что в Номве; и пришли они все к царю.
\vs 1Sa 22:12 И сказал Саул: послушай, сын Ахитува. И тот отвечал: вот я, господин мой.
\vs 1Sa 22:13 И сказал ему Саул: для чего вы сговорились против меня, ты и сын Иессея, что ты дал ему хлебы и меч и вопросил о нем Бога, чтоб он восстал против меня и строил мне ковы, как это ныне видно?
\vs 1Sa 22:14 И отвечал Ахимелех царю и сказал: кто из всех рабов твоих верен как Давид? он и зять царя, и исполнитель повелений твоих, и почтен в доме твоем.
\vs 1Sa 22:15 Теперь ли я стал вопрошать для него Бога? Нет, не обвиняй в этом, царь, раба твоего и весь дом отца моего, ибо во всем этом деле не знает раб твой ни малого, ни великого.
\vs 1Sa 22:16 И сказал царь: ты должен умереть, Ахимелех, ты и весь дом отца твоего.
\vs 1Sa 22:17 И сказал царь телохранителям, стоявшим при нем: ступайте, умертвите священников Господних, ибо и их рука с Давидом, и они знали, что он убежал, и не открыли мне. Но слуги царя не хотели поднять рук своих на убиение священников Господних.
\vs 1Sa 22:18 И сказал царь Доику: ступай ты и умертви священников. И пошел Доик Идумеянин, и напал на священников, и умертвил в тот день восемьдесят пять\fns{В греческом переводе: триста пять.} мужей, носивших льняной ефод;
\vs 1Sa 22:19 и Номву, город священников, поразил мечом; и мужчин и женщин, и юношей и младенцев, и волов и ослов и овец поразил мечом.
\vs 1Sa 22:20 Спасся один только сын Ахимелеха, сына Ахитува, по имени Авиафар, и убежал к Давиду.
\vs 1Sa 22:21 И рассказал Авиафар Давиду, что Саул умертвил священников Господних.
\vs 1Sa 22:22 И сказал Давид Авиафару: я знал в тот день, когда там был Доик Идумеянин, что он непременно донесет Саулу; я виновен во всех душах дома отца твоего;
\vs 1Sa 22:23 останься у меня, не бойся, ибо кто будет искать моей души, будет искать и твоей души; ты будешь у меня под охранением.
\vs 1Sa 23:1 И известили Давида, говоря: вот, Филистимляне напали на Кеиль и расхищают гумна.
\vs 1Sa 23:2 И вопросил Давид Господа, говоря: идти ли мне, и поражу ли я этих Филистимлян? И отвечал Господь Давиду: иди, ты поразишь Филистимлян и спасешь Кеиль.
\vs 1Sa 23:3 Но бывшие с Давидом сказали ему: вот, мы боимся здесь в Иудее, как же нам идти в Кеиль против ополчений Филистимских? [мы попадем в плен к Филистимлянам.]
\vs 1Sa 23:4 Тогда снова вопросил Давид Господа, и отвечал ему Господь и сказал: встань и иди в Кеиль, ибо Я предам Филистимлян в руки твои.
\vs 1Sa 23:5 И пошел Давид с людьми своими в Кеиль, и воевал с Филистимлянами, и угнал скот их, и нанес им великое поражение, и спас Давид жителей Кеиля.
\vs 1Sa 23:6 Когда Авиафар, сын Ахимелеха, прибежал к Давиду [и пошел с ним] в Кеиль, то принес с собою и ефод.
\vs 1Sa 23:7 И донесли Саулу, что Давид пришел в Кеиль, и Саул сказал: Бог предал его в руки мои, ибо он запер себя, войдя в город с воротами и запорами.
\vs 1Sa 23:8 И созвал Саул весь народ на войну, чтоб идти к Кеилю, осадить Давида и людей его.
\rsbpar\vs 1Sa 23:9 Когда узнал Давид, что Саул задумал против него злое, сказал священнику Авиафару: принеси ефод [Господень].
\vs 1Sa 23:10 И сказал Давид: Господи Боже Израилев! раб Твой услышал, что Саул хочет прийти в Кеиль, разорить город ради меня.
\vs 1Sa 23:11 Предадут ли меня жители Кеиля в руки его? И придет ли сюда Саул, как слышал раб Твой? Господи Боже Израилев! открой рабу Твоему. И сказал Господь: придет.
\vs 1Sa 23:12 И сказал Давид: предадут ли жители Кеиля меня и людей моих в руки Саула? И сказал Господь: предадут.
\vs 1Sa 23:13 Тогда поднялся Давид и люди его, около шестисот человек, и вышли из Кеиля и ходили, где могли. Саулу же было донесено, что Давид убежал из Кеиля, и тогда он отменил поход.
\vs 1Sa 23:14 Давид же пребывал в пустыне в неприступных местах и потом на горе в пустыне Зиф. Саул искал его всякий день; но Бог не предал \bibemph{Давида} в руки его.
\vs 1Sa 23:15 И видел Давид, что Саул вышел искать души его; Давид же был в пустыне Зиф в лесу.
\vs 1Sa 23:16 И встал Ионафан, сын Саула, и пришел к Давиду в лес, и укрепил его упованием на Бога,
\vs 1Sa 23:17 и сказал ему: не бойся, ибо не найдет тебя рука отца моего Саула, и ты будешь царствовать над Израилем, а я буду вторым по тебе; и Саул, отец мой, знает это.
\vs 1Sa 23:18 И заключили они между собою завет пред лицем Господа; и Давид остался в лесу, а Ионафан пошел в дом свой.
\vs 1Sa 23:19 И пришли Зифеи к Саулу в Гиву, говоря: вот, Давид скрывается у нас в неприступных местах, в лесу, на холме Гахила, что направо от Иесимона;
\vs 1Sa 23:20 итак по желанию души твоей, царь, иди; а наше дело будет предать его в руки царя.
\vs 1Sa 23:21 И сказал им Саул: благословенны вы у Господа за то, что пожалели о мне;
\vs 1Sa 23:22 идите, удостоверьтесь еще, разведайте \bibemph{и} высмотрите место его, где будет нога его, \bibemph{и} кто видел его там, ибо мне говорят, что он очень хитер;
\vs 1Sa 23:23 и высмотрите, и разведайте о всех убежищах, в которых он скрывается, и возвратитесь ко мне с верным известием, и я пойду с вами; и если он в этой земле, я буду искать его во всех тысячах Иудиных.
\vs 1Sa 23:24 И встали они и пошли в Зиф прежде Саула. Давид же и люди его были в пустыне Маон, на равнине, направо от Иесимона.
\vs 1Sa 23:25 И пошел Саул с людьми своими искать \bibemph{его}. Но Давида известили об этом, и он перешел к скале и оставался в пустыне Маон. И услышал Саул, и погнался за Давидом в пустыню Маон.
\vs 1Sa 23:26 И шел Саул по одной стороне горы, а Давид с людьми своими был на другой стороне горы. И когда Давид спешил уйти от Саула, а Саул с людьми своими шел в обход Давиду и людям его, чтобы захватить их;
\vs 1Sa 23:27 тогда пришел к Саулу вестник, говоря: поспешай и приходи, ибо Филистимляне напали на землю.
\vs 1Sa 23:28 И возвратился Саул от преследования Давида и пошел навстречу Филистимлянам; посему и назвали это место: Села-Гаммахлекоф\fns{Скала разделений.}.
\vs 1Sa 24:1 И вышел Давид оттуда и жил в безопасных местах Ен-Гадди.
\vs 1Sa 24:2 Когда Саул возвратился от Филистимлян, его известили, говоря: вот, Давид в пустыне Ен-Гадди.
\vs 1Sa 24:3 И взял Саул три тысячи отборных мужей из всего Израиля и пошел искать Давида и людей его по горам, где живут серны.
\vs 1Sa 24:4 И пришел к загону овечьему, при дороге; там была пещера, и зашел туда Саул для нужды; Давид же и люди его сидели в глубине пещеры.
\vs 1Sa 24:5 И говорили Давиду люди его: вот день, о котором говорил тебе Господь: <<вот, Я предам врага твоего в руки твои, и сделаешь с ним, что тебе угодно>>. Давид встал и тихонько отрезал край от верхней одежды Саула.
\vs 1Sa 24:6 Но после сего больно стало сердцу Давида, что он отрезал край от одежды Саула.
\vs 1Sa 24:7 И сказал он людям своим: да не попустит мне Господь сделать это господину моему, помазаннику Господню, чтобы наложить руку мою на него, ибо он помазанник Господень.
\vs 1Sa 24:8 И удержал Давид людей своих сими словами и не дал им восстать на Саула. А Саул встал и вышел из пещеры на дорогу.
\vs 1Sa 24:9 Потом встал и Давид, и вышел из пещеры, и закричал вслед Саула, говоря: господин мой, царь! Саул оглянулся назад, и Давид пал лицем на землю и поклонился [ему].
\vs 1Sa 24:10 И сказал Давид Саулу: зачем ты слушаешь речи людей, которые говорят: <<вот, Давид умышляет зло на тебя>>?
\vs 1Sa 24:11 Вот, сегодня видят глаза твои, что Господь предавал тебя ныне в руки мои в пещере; и мне говорили, чтоб убить тебя; но я пощадил тебя и сказал: <<не подниму руки моей на господина моего, ибо он помазанник Господа>>.
\vs 1Sa 24:12 Отец мой! посмотри на край одежды твоей в руке моей; я отрезал край одежды твоей, а тебя не убил: узнай и убедись, что нет в руке моей зла, ни коварства, и я не согрешил против тебя; а ты ищешь души моей, чтоб отнять ее.
\vs 1Sa 24:13 Да рассудит Господь между мною и тобою, и да отмстит тебе Господь за меня; но рука моя не будет на тебе,
\vs 1Sa 24:14 как говорит древняя притча: <<от беззаконных исходит беззаконие>>. А моя рука не будет на тебе.
\vs 1Sa 24:15 Против кого вышел царь Израильский? За кем ты гоняешься? За мертвым псом, за одною блохою.
\vs 1Sa 24:16 Господь да будет судьею и рассудит между мною и тобою. Он рассмотрит, разберет дело мое, и спасет меня от руки твоей.
\vs 1Sa 24:17 Когда кончил Давид говорить слова сии к Саулу, Саул сказал: твой ли это голос, сын мой Давид? И возвысил Саул голос свой, и плакал,
\vs 1Sa 24:18 и сказал Давиду: ты правее меня, ибо ты воздал мне добром, а я воздавал тебе злом;
\vs 1Sa 24:19 ты показал это сегодня, поступив со мною милостиво, когда Господь предавал меня в руки твои, ты не убил меня.
\vs 1Sa 24:20 Кто, найдя врага своего, отпустил бы его в добрый путь? Господь воздаст тебе добром за то, что сделал ты мне сегодня.
\vs 1Sa 24:21 И теперь я знаю, что ты непременно будешь царствовать, и царство Израилево будет твердо в руке твоей.
\vs 1Sa 24:22 Итак поклянись мне Господом, что ты не искоренишь потомства моего после меня и не уничтожишь имени моего в доме отца моего.
\vs 1Sa 24:23 И поклялся Давид Саулу. И пошел Саул в дом свой, Давид же и люди его взошли в место укрепленное.
\vs 1Sa 25:1 И умер Самуил; и собрались все Израильтяне, и плакали по нем, и погребли его в доме его, в Раме. Давид встал и сошел к пустыне Фаран.
\rsbpar\vs 1Sa 25:2 Был некто в Маоне, а имение его на Кармиле, человек очень богатый; у него было три тысячи овец и тысяча коз; и был он при стрижке овец своих на Кармиле.
\vs 1Sa 25:3 Имя человека того~--- Навал, а имя жены его~--- Авигея; эта женщина \bibemph{была} весьма умная и красивая лицем, а он~--- человек жестокий и злой нравом; он был из рода Халева.
\vs 1Sa 25:4 И услышал Давид в пустыне, что Навал стрижет [на Кармиле] овец своих.
\vs 1Sa 25:5 И послал Давид десять отроков, и сказал Давид отрокам: взойдите на Кармил и пойдите к Навалу, и приветствуйте его от моего имени,
\vs 1Sa 25:6 и скажите так: <<[здравствуй,] мир тебе, мир дому твоему, мир всему твоему;
\vs 1Sa 25:7 ныне я услышал, что у тебя стригут \bibemph{овец}. Вот, пастухи твои были с нами, и мы не обижали их, и ничего у них не пропало во все время их пребывания на Кармиле;
\vs 1Sa 25:8 спроси слуг твоих, и они скажут тебе; итак да найдут отроки благоволение в глазах твоих, ибо в добрый день пришли мы; дай же рабам твоим и сыну твоему Давиду, что найдет рука твоя>>.
\vs 1Sa 25:9 И пошли люди Давидовы, и сказали Навалу от имени Давида все эти слова, и умолкли.
\vs 1Sa 25:10 И [вскочил] Навал, [и] отвечал слугам Давидовым, и сказал: кто такой Давид, и кто такой сын Иессеев? ныне стало много рабов, бегающих от господ своих;
\vs 1Sa 25:11 неужели мне взять хлебы мои и воду мою, [и вино мое] и мясо, приготовленное мною для стригущих овец у меня, и отдать людям, о которых не знаю, откуда они?
\vs 1Sa 25:12 И пошли назад люди Давида своим путем и возвратились, и пришли и пересказали ему все слова сии.
\vs 1Sa 25:13 Тогда Давид сказал людям своим: опояшьтесь каждый мечом своим. И все опоясались мечами своими, опоясался и сам Давид своим мечом, и пошли за Давидом около четырехсот человек, а двести остались при обозе.
\vs 1Sa 25:14 Авигею же, жену Навала, известил один из слуг, сказав: вот, Давид присылал из пустыни послов приветствовать нашего господина, но он обошелся с ними грубо;
\vs 1Sa 25:15 а эти люди очень добры к нам, не обижали нас, и ничего не пропало у нас во все время, когда мы ходили с ними, быв в поле;
\vs 1Sa 25:16 они были для нас оградою и днем и ночью во все время, когда мы пасли стада вблизи их;
\vs 1Sa 25:17 итак подумай и посмотри, что делать; ибо неминуемо угрожает беда господину нашему и всему дому его, а он~--- человек злой, нельзя говорить с ним.
\vs 1Sa 25:18 Тогда Авигея поспешно взяла двести хлебов, и два меха с вином, и пять овец приготовленных, и пять мер сушеных зерен, и сто связок изюму, и двести связок смокв, и навьючила на ослов,
\vs 1Sa 25:19 и сказала слугам своим: ступайте впереди меня, вот, я пойду за вами. А мужу своему Навалу ничего не сказала.
\vs 1Sa 25:20 Когда же она, сидя на осле, спускалась по извилинам горы, вот, навстречу ей идет Давид и люди его, и она встретилась с ними.
\vs 1Sa 25:21 И Давид сказал: да, напрасно я охранял в пустыне все имущество этого человека, и ничего не пропало из принадлежащего ему; он платит мне злом за добро;
\vs 1Sa 25:22 пусть то и то сделает Бог с врагами Давида, и еще больше сделает, если до рассвета утреннего из всего, что принадлежит Навалу, я оставлю мочащегося к стене.
\vs 1Sa 25:23 Когда Авигея увидела Давида, то поспешила сойти с осла и пала пред Давидом на лице свое и поклонилась до земли;
\vs 1Sa 25:24 и пала к ногам его и сказала: на мне грех, господин мой; позволь рабе твоей говорить в уши твои и послушай слов рабы твоей.
\vs 1Sa 25:25 Пусть господин мой не обращает внимания на этого злого человека, на Навала; ибо каково имя его, таков и он. Навал~--- имя его, и безумие его с ним\fns{<<Навал>>~--- безумный.}. А я, раба твоя, не видела слуг господина моего, которых ты присылал.
\vs 1Sa 25:26 И ныне, господин мой, жив Господь и жива душа твоя, Господь не попустит тебе идти на пролитие крови и удержит руку твою от мщения, и ныне да будут, как Навал, враги твои и злоумышляющие против господина моего.
\vs 1Sa 25:27 Вот эти дары, которые принесла раба твоя господину моему, чтобы дать их отрокам, служащим господину моему.
\vs 1Sa 25:28 Прости вину рабы твоей; Господь непременно устроит господину моему дом твердый, ибо войны Господа ведет господин мой, и зло не найдется в тебе во всю жизнь твою.
\vs 1Sa 25:29 Если восстанет человек преследовать тебя и искать души твоей, то душа господина моего будет завязана в узле жизни у Господа Бога твоего, а душу врагов твоих бросит Он как бы пращею.
\vs 1Sa 25:30 И когда сделает Господь господину моему все, что говорил о тебе доброго, и поставит тебя вождем над Израилем,
\vs 1Sa 25:31 то не будет это сердцу господина моего огорчением и беспокойством, что не пролил напрасно крови и сберег себя от мщения. И Господь облагодетельствует господина моего, и вспомнишь рабу твою [и окажешь милость ей].
\vs 1Sa 25:32 И сказал Давид Авигее: благословен Господь Бог Израилев, Который послал тебя ныне навстречу мне,
\vs 1Sa 25:33 и благословен разум твой, и благословенна ты за то, что ты теперь не допустила меня идти на пролитие крови и отмстить за себя.
\vs 1Sa 25:34 Но,~--- жив Господь Бог Израилев, удержавший меня от нанесения зла тебе,~--- если бы ты не поспешила и не пришла навстречу мне, то до рассвета утреннего я не оставил бы Навалу мочащегося к стене.
\vs 1Sa 25:35 И принял Давид из рук ее то, что она принесла ему, и сказал ей: иди с миром в дом твой; вот, я послушался голоса твоего и почтил лице твое.
\vs 1Sa 25:36 И пришла Авигея к Навалу, и вот, у него пир в доме его, как пир царский, и сердце Навала было весело; он же был очень пьян; и не сказала ему ни слова, ни большого, ни малого, до утра.
\vs 1Sa 25:37 Утром же, когда Навал отрезвился, жена его рассказала ему об этом, и замерло в нем сердце его, и стал он, как камень.
\vs 1Sa 25:38 Дней через десять поразил Господь Навала, и он умер.
\vs 1Sa 25:39 И услышал Давид, что Навал умер, и сказал: благословен Господь, воздавший за посрамление, нанесенное мне Навалом, и сохранивший раба Своего от зла; Господь обратил злобу Навала на его же голову. И послал Давид сказать Авигее, что он берет ее себе в жену.
\vs 1Sa 25:40 И пришли слуги Давидовы к Авигее на Кармил и сказали ей так: Давид послал нас к тебе, чтобы взять тебя ему в жену.
\vs 1Sa 25:41 Она встала и поклонилась лицем до земли и сказала: вот, раба твоя \bibemph{готова} быть служанкою, чтобы омывать ноги слуг господина моего.
\vs 1Sa 25:42 И собралась Авигея поспешно и села на осла, и пять служанок сопровождали ее; и пошла она за послами Давида и сделалась его женою.
\vs 1Sa 25:43 И Ахиноаму из Изрееля взял Давид, и обе они были его женами.
\vs 1Sa 25:44 Саул же отдал дочь свою Мелхолу, жену Давидову, Фалтию, сыну Лаиша, что из Галлима.
\vs 1Sa 26:1 Пришли Зифеи [с юга] к Саулу в Гиву и сказали: вот, Давид скрывается у нас на холме Гахила, что направо от Иесимона.
\vs 1Sa 26:2 И встал Саул и спустился в пустыню Зиф, и с ним три тысячи отборных мужей Израильских, чтоб искать Давида в пустыне Зиф.
\vs 1Sa 26:3 И расположился Саул на холме Гахила, что направо от Иесимона, при дороге; Давид же находился в пустыне и видел, что Саул шел за ним в пустыню;
\vs 1Sa 26:4 и послал Давид соглядатаев и узнал, что Саул действительно пришел [из Кеиля].
\vs 1Sa 26:5 И встал Давид [тайно] и пошел к месту, на котором Саул расположился станом, и увидел Давид место, где спал Саул и Авенир, сын Ниров, военачальник его. Саул же спал в шатре, а народ расположился вокруг него.
\vs 1Sa 26:6 И обратился Давид и сказал Ахимелеху Хеттеянину и Авессе, сыну Саруину, брату Иоава, говоря: кто пойдет со мною к Саулу в стан? И отвечал Авесса: я пойду с тобою.
\vs 1Sa 26:7 И пришел Давид с Авессою к людям [Сауловым] ночью; и вот, Саул лежит, спит в шатре, и копье его воткнуто в землю у изголовья его; Авенир же и народ лежат вокруг него.
\vs 1Sa 26:8 Авесса сказал Давиду: предал Бог ныне врага твоего в руки твои; итак позволь, я пригвожду его копьем к земле одним ударом и не повторю \bibemph{удара}.
\vs 1Sa 26:9 Но Давид сказал Авессе: не убивай его; ибо кто, подняв руку на помазанника Господня, останется ненаказанным?
\vs 1Sa 26:10 И сказал Давид: жив Господь! пусть поразит его Господь, или придет день его, и он умрет, или пойдет на войну и погибнет; меня же да не попустит Господь поднять руку мою на помазанника Господня;
\vs 1Sa 26:11 а возьми его копье, которое у изголовья его, и сосуд с водою, и пойдем к себе.
\vs 1Sa 26:12 И взял Давид копье и сосуд с водою у изголовья Саула, и пошли они к себе; и никто не видел, и никто не знал, и никто не проснулся, но все спали, ибо сон от Господа напал на них.
\vs 1Sa 26:13 И перешел Давид на другую сторону и стал на вершине горы вдали; большое расстояние \bibemph{было} между ними.
\vs 1Sa 26:14 И воззвал Давид к народу и Авениру, сыну Нирову, говоря: отвечай, Авенир. И отвечал Авенир и сказал: кто ты, что кричишь и \bibemph{беспокоишь} царя?
\vs 1Sa 26:15 И сказал Давид Авениру: не муж ли ты, и кто равен тебе в Израиле? Для чего же ты не бережешь господина твоего, царя? ибо приходил некто из народа, чтобы погубить царя, господина твоего.
\vs 1Sa 26:16 Нехорошо ты это делаешь; жив Господь! вы достойны смерти за то, что не бережете господина вашего, помазанника Господня. Посмотри, где копье царя и сосуд с водою, что \bibemph{были} у изголовья его?
\vs 1Sa 26:17 И узнал Саул голос Давида и сказал: твой ли это голос, сын мой Давид? И сказал Давид: мой голос, господин мой, царь.
\vs 1Sa 26:18 И сказал \bibemph{еще}: за что господин мой преследует раба своего? что я сделал? какое зло в руке моей?
\vs 1Sa 26:19 И ныне пусть выслушает господин мой, царь, слова раба своего: если Господь возбудил тебя против меня, то да будет это от тебя благовонною жертвою; если же~--- сыны человеческие, то прокляты они пред Господом, ибо они изгнали меня ныне, чтобы не принадлежать мне к наследию Господа, говоря: <<ступай, служи богам чужим>>.
\vs 1Sa 26:20 Да не прольется же кровь моя на землю пред лицем Господа; ибо царь Израилев вышел искать одну блоху, как гоняются за куропаткою по горам.
\vs 1Sa 26:21 И сказал Саул: согрешил я; возвратись, сын мой Давид, ибо я не буду больше делать тебе зла, потому что душа моя была дорога ныне в глазах твоих; безумно поступал я и очень много погрешал.
\vs 1Sa 26:22 И отвечал Давид и сказал: вот копье царя; пусть один из отроков придет и возьмет его;
\vs 1Sa 26:23 и да воздаст Господь каждому по правде его и по истине его, так как Господь предавал тебя в руки \bibemph{мои}, но я не захотел поднять руки моей на помазанника Господня;
\vs 1Sa 26:24 и пусть, как драгоценна была жизнь твоя ныне в глазах моих, так ценится моя жизнь в очах Господа, [и да покроет Он меня] и да избавит меня от всякой беды!
\vs 1Sa 26:25 И сказал Саул Давиду: благословен ты, сын мой Давид; и дело сделаешь, и превозмочь превозможешь. И пошел Давид своим путем, а Саул возвратился в свое место.
\vs 1Sa 27:1 И сказал Давид в сердце своем: когда-нибудь попаду я в руки Саула, и нет для меня ничего лучшего, как убежать в землю Филистимскую; и отстанет от меня Саул \bibemph{и не будет} искать меня более по всем пределам Израильским, и я спасусь от руки его.
\vs 1Sa 27:2 И встал Давид, и отправился сам и шестьсот мужей, бывших с ним, к Анхусу, сыну Маоха, царю Гефскому.
\vs 1Sa 27:3 И жил Давид у Анхуса в Гефе, сам и люди его, каждый с семейством своим, Давид и обе жены его~--- Ахиноама Изреелитянка и Авигея, \bibemph{бывшая} жена Навала, Кармилитянка.
\vs 1Sa 27:4 И донесли Саулу, что Давид убежал в Геф, и не стал он более искать его.
\vs 1Sa 27:5 И сказал Давид Анхусу: если я приобрел благоволение в глазах твоих, то пусть дано будет мне место в одном из малых городов, и я буду жить там; для чего рабу твоему жить в царском городе вместе с тобою?
\vs 1Sa 27:6 Тогда дал ему Анхус Секелаг, посему Секелаг и остался за царями Иудейскими доныне.
\vs 1Sa 27:7 Всего времени, какое прожил Давид в стране Филистимской, было год и четыре месяца.
\vs 1Sa 27:8 И выходил Давид с людьми своими и нападал на Гессурян и Гирзеян и Амаликитян, которые издавна населяли эту страну до Сура и даже до земли Египетской.
\vs 1Sa 27:9 И опустошал Давид ту страну, и не оставлял в живых ни мужчины, ни женщины, и забирал овец, и волов, и ослов, и верблюдов, и одежду; и возвращался, и приходил к Анхусу.
\vs 1Sa 27:10 И сказал Анхус Давиду: на кого нападали ныне? Давид сказал: на полуденную страну Иудеи и на полуденную страну Иерахмеела и на полуденную страну Кенеи.
\vs 1Sa 27:11 И не оставлял Давид в живых ни мужчины, ни женщины, и не приводил в Геф, говоря: они могут донести на нас и сказать: <<так поступил Давид, и таков образ действий его во все время пребывания в стране Филистимской>>.
\vs 1Sa 27:12 И доверился Анхус Давиду, говоря: он опротивел народу своему Израилю и будет слугою моим вовек.
\vs 1Sa 28:1 В то время Филистимляне собрали войска свои для войны, чтобы воевать с Израилем. И сказал Анхус Давиду: да будет тебе известно, что ты пойдешь со мною в ополчение, ты и люди твои.
\vs 1Sa 28:2 И сказал Давид Анхусу: ныне ты узнаешь, что сделает раб твой. И сказал Анхус Давиду: за то я сделаю тебя хранителем головы моей на все время.
\rsbpar\vs 1Sa 28:3 И умер Самуил, и оплакивали его все Израильтяне и погребли его в Раме, в городе его. Саул же изгнал волшебников и гадателей из страны.
\rsbpar\vs 1Sa 28:4 И собрались Филистимляне и пошли и стали станом в Сонаме; собрал и Саул весь народ Израильский, и стали станом на Гелвуе.
\vs 1Sa 28:5 И увидел Саул стан Филистимский и испугался, и крепко дрогнуло сердце его.
\vs 1Sa 28:6 И вопросил Саул Господа; но Господь не отвечал ему ни во сне, ни чрез урим, ни чрез пророков.
\vs 1Sa 28:7 Тогда Саул сказал слугам своим: сыщите мне женщину волшебницу, и я пойду к ней и спрошу ее. И отвечали ему слуги его: здесь в Аэндоре есть женщина волшебница.
\vs 1Sa 28:8 И снял с себя Саул одежды свои и надел другие, и пошел сам и два человека с ним, и пришли они к женщине ночью. И сказал ей \bibemph{Саул}: прошу тебя, поворожи мне и выведи мне, о ком я скажу тебе.
\vs 1Sa 28:9 Но женщина отвечала ему: ты знаешь, что сделал Саул, как выгнал он из страны волшебников и гадателей; для чего же ты расставляешь сеть душе моей на погибель мне?
\vs 1Sa 28:10 И поклялся ей Саул Господом, говоря: жив Господь! не будет тебе беды за это дело.
\vs 1Sa 28:11 Тогда женщина спросила: кого же вывесть тебе? И отвечал он: Самуила выведи мне.
\vs 1Sa 28:12 И увидела женщина Самуила и громко вскрикнула; и обратилась женщина к Саулу, говоря: зачем ты обманул меня? ты~--- Саул.
\vs 1Sa 28:13 И сказал ей царь: не бойся; [скажи,] что ты видишь? И отвечала женщина: вижу как бы бога, выходящего из земли.
\vs 1Sa 28:14 Какой он видом?~--- спросил у нее \bibemph{Саул}. Она сказала: выходит из земли муж престарелый, одетый в длинную одежду. Тогда узнал Саул, что это Самуил, и пал лицем на землю и поклонился.
\vs 1Sa 28:15 И сказал Самуил Саулу: для чего ты тревожишь меня, чтобы я вышел? И отвечал Саул: тяжело мне очень; Филистимляне воюют против меня, а Бог отступил от меня и более не отвечает мне ни чрез пророков, ни во сне, [ни в видении]; потому я вызвал тебя, чтобы ты научил меня, что мне делать.
\vs 1Sa 28:16 И сказал Самуил: для чего же ты спрашиваешь меня, когда Господь отступил от тебя и сделался врагом твоим?
\vs 1Sa 28:17 Господь сделает то, что говорил чрез меня; отнимет Господь царство из рук твоих и отдаст его ближнему твоему, Давиду.
\vs 1Sa 28:18 Так как ты не послушал гласа Господня и не выполнил ярости гнева Его на Амалика, то Господь и делает это над тобою ныне.
\vs 1Sa 28:19 И предаст Господь Израиля вместе с тобою в руки Филистимлян: завтра ты и сыны твои \bibemph{будете} со мною, и стан Израильский предаст Господь в руки Филистимлян.
\vs 1Sa 28:20 Тогда Саул вдруг пал всем телом своим на землю, ибо сильно испугался слов Самуила; притом и силы не стало в нем, ибо он не ел хлеба весь тот день и всю ночь.
\vs 1Sa 28:21 И подошла женщина та к Саулу, и увидела, что он очень испугался, и сказала: вот, раба твоя послушалась голоса твоего и подвергала жизнь свою опасности и исполнила приказание, которое ты дал мне;
\vs 1Sa 28:22 теперь прошу, послушайся и ты голоса рабы твоей: я предложу тебе кусок хлеба, поешь, и будет в тебе крепость, когда пойдешь в путь.
\vs 1Sa 28:23 Но он отказался и сказал: не буду есть. И стали уговаривать его слуги его, а также и женщина; и он послушался голоса их, и встал с земли и сел на ложе.
\vs 1Sa 28:24 У женщины же был в доме откормленный теленок, и она поспешила заколоть его и, взяв муки, замесила и испекла опресноки,
\vs 1Sa 28:25 и предложила Саулу и слугам его, и они поели, и встали, и ушли в ту же ночь.
\vs 1Sa 29:1 И собрали Филистимляне все ополчения свои в Афеке, а Израильтяне расположились станом у источника, что в Изрееле.
\vs 1Sa 29:2 Князья Филистимские шли с сотнями и тысячами, Давид же и люди его шли позади с Анхусом.
\vs 1Sa 29:3 И говорили князья Филистимские: это что за Евреи? Анхус отвечал князьям Филистимским: разве не знаете, что это Давид, раб Саула, царя Израильского? он при мне уже более года, и я не нашел в нем ничего худого со времени его прихода до сего дня.
\vs 1Sa 29:4 И вознегодовали на него князья Филистимские, и сказали ему князья Филистимские: отпусти ты этого человека, пусть он сидит в своем месте, которое ты ему назначил, чтоб он не шел с нами на войну и не сделался противником нашим на войне. Чем он может умилостивить господина своего, как не головами сих мужей?
\vs 1Sa 29:5 Не тот ли это Давид, которому пели в хороводах, говоря: <<Саул поразил тысячи, а Давид~--- десятки тысяч>>?
\vs 1Sa 29:6 И призвал Анхус Давида и сказал ему: жив Господь! ты честен, и глазам моим приятно было бы, чтобы ты выходил и входил со мною в ополчении; ибо я не заметил в тебе худого со времени прихода твоего ко мне до сего дня; но в глазах князей ты не хорош.
\vs 1Sa 29:7 Итак, возвратись теперь, и иди с миром и не раздражай князей Филистимских.
\vs 1Sa 29:8 Но Давид сказал Анхусу: что я сделал, и что ты нашел в рабе твоем с того времени, как я пред лицем твоим, и до сего дня, почему бы мне не идти и не воевать с врагами господина моего, царя?
\vs 1Sa 29:9 И отвечал Анхус Давиду: будь уверен, что в моих глазах ты хорош, как Ангел Божий; но князья Филистимские сказали: <<пусть он не идет с нами на войну>>.
\vs 1Sa 29:10 Итак встань утром, ты и рабы господина твоего, которые пришли с тобою, [и идите на место, которое я назначил вам, и не имей худой мысли на сердце твоем, ибо ты предо мною хорош]; и встаньте поутру, и когда светло будет, идите.
\vs 1Sa 29:11 И встал Давид, сам и люди его, чтобы идти утром и возвратиться в землю Филистимскую. А Филистимляне пошли [на войну] в Изреель.
\vs 1Sa 30:1 В третий день после того, как Давид и люди его пошли в Секелаг, Амаликитяне напали с юга на Секелаг и взяли Секелаг и сожгли его огнем,
\vs 1Sa 30:2 а женщин [и всех], бывших в нем, от малого до большого, не умертвили, но увели в плен, и ушли своим путем.
\vs 1Sa 30:3 И пришел Давид и люди его к городу, и вот, он сожжен огнем, а жены их и сыновья их и дочери их взяты в плен.
\vs 1Sa 30:4 И поднял Давид и народ, бывший с ним, вопль, и плакали, доколе не стало в них силы плакать.
\vs 1Sa 30:5 Взяты были в плен и обе жены Давида: Ахиноама Изреелитянка и Авигея, \bibemph{бывшая} жена Навала, Кармилитянка.
\vs 1Sa 30:6 Давид сильно был смущен, так как народ хотел побить его камнями; ибо скорбел душею весь народ, каждый о сыновьях своих и дочерях своих.
\vs 1Sa 30:7 Но Давид укрепился \bibemph{надеждою} на Господа Бога своего, и сказал Давид Авиафару священнику, сыну Ахимелехову: принеси мне ефод. И принес Авиафар ефод к Давиду.
\vs 1Sa 30:8 И вопросил Давид Господа, говоря: преследовать ли мне это полчище, и догоню ли их? И сказано ему: преследуй, догонишь и отнимешь.
\vs 1Sa 30:9 И пошел Давид сам и шестьсот мужей, бывших с ним; и пришли к потоку Восор и усталые остановились там.
\vs 1Sa 30:10 И преследовал Давид сам и четыреста человек; двести же человек остановились, потому что были не в силах перейти поток Восорский.
\vs 1Sa 30:11 И нашли Египтянина в поле, и привели его к Давиду, и дали ему хлеба, и он ел, и напоили его водою;
\vs 1Sa 30:12 и дали ему часть связки смокв и две связки изюму, и он ел и укрепился, ибо он не ел хлеба и не пил воды три дня и три ночи.
\vs 1Sa 30:13 И сказал ему Давид: чей ты и откуда ты? И сказал он: я~--- отрок Египтянина, раб одного Амаликитянина, и бросил меня господин мой, ибо уже три дня, как я заболел;
\vs 1Sa 30:14 мы вторгались в полуденную часть Керети и в область Иудину и в полуденную часть Халева, а Секелаг сожгли огнем.
\vs 1Sa 30:15 И сказал ему Давид: доведешь ли меня до этого полчища? И сказал он: поклянись мне Богом, что ты не умертвишь меня и не предашь меня в руки господина моего, и я доведу тебя до этого полчища.
\vs 1Sa 30:16 [Давид поклялся ему,] и он повел его; и вот, \bibemph{Амаликитяне}, рассыпавшись по всей той стране, едят и пьют и празднуют по причине великой добычи, которую они взяли из земли Филистимской и из земли Иудейской.
\vs 1Sa 30:17 [И напал на них] и поражал их Давид от сумерек до вечера другого дня, и никто из них не спасся, кроме четырехсот юношей, которые сели на верблюдов и убежали.
\vs 1Sa 30:18 И отнял Давид все, что взяли Амаликитяне, и обеих жен своих отнял Давид.
\vs 1Sa 30:19 И не пропало у них ничего, ни малого, ни большого, ни из сыновей, ни из дочерей, ни из добычи, ни из всего, что \bibemph{Амаликитяне} взяли у них; все возвратил Давид,
\vs 1Sa 30:20 и взял Давид весь мелкий и крупный скот, и гнали его пред своим скотом и говорили: это~--- добыча Давида.
\vs 1Sa 30:21 И пришел Давид к тем двум стам человек, которые не были в силах идти за ним, и \bibemph{которых} он оставил у потока Восор, и вышли они навстречу Давиду и навстречу людям, бывшим с ним. И подошел Давид к этим людям и приветствовал их.
\vs 1Sa 30:22 Тогда злые и негодные из людей, ходивших с Давидом, стали говорить: за то, что они не ходили с нами, не дадим им из добычи, которую мы отняли; пусть каждый возьмет только свою жену и детей и идет.
\vs 1Sa 30:23 Но Давид сказал: не делайте так, братья мои, после того, как Господь дал нам это и сохранил нас и предал в руки наши полчище, приходившее против нас.
\vs 1Sa 30:24 И кто послушает вас в этом деле? [Они не хуже нас.] Какова часть ходившим на войну, такова часть должна быть и оставшимся при обозе: на всех должно разделить.
\vs 1Sa 30:25 Так было с этого времени и после; и поставил он это в закон и в правило для Израиля до сего дня.
\rsbpar\vs 1Sa 30:26 И пришел Давид в Секелаг и послал из добычи к старейшинам Иудиным, друзьям своим, говоря: <<вот вам подарок из добычи, \bibemph{взятой} у врагов Господних>>,~---
\vs 1Sa 30:27 тем, которые в Вефиле, и в Рамофе южном, и в Иаттире, [и в Гефоре,]
\vs 1Sa 30:28 и в Ароере, [и в Аммаде,] и в Шифмофе, и в Естемоа, [и в Гефе,]
\vs 1Sa 30:29 [в Кинане, в Сафене, в Фимафе,] и в Рахале, и в городах Иерахмеельских, и в городах Кенейских,
\vs 1Sa 30:30 и в Хорме, и в Хорашане, и в Атахе,
\vs 1Sa 30:31 и в Хевроне, и во всех местах, где ходил Давид сам и люди его.
\vs 1Sa 31:1 Филистимляне же воевали с Израильтянами, и побежали мужи Израильские от Филистимлян и пали пораженные на горе Гелвуе.
\vs 1Sa 31:2 И догнали Филистимляне Саула и сыновей его, и убили Филистимляне Ионафана, и Аминадава, и Малхисуа, сыновей Саула.
\vs 1Sa 31:3 И битва против Саула сделалась жестокая, и стрелки из луков поражали его, и он очень изранен был стрелками.
\vs 1Sa 31:4 И сказал Саул оруженосцу своему: обнажи твой меч и заколи меня им, чтобы не пришли эти необрезанные и не убили меня и не издевались надо мною. Но оруженосец не хотел, ибо очень боялся. Тогда Саул взял меч свой и пал на него.
\vs 1Sa 31:5 Оруженосец его, увидев, что Саул умер, и сам пал на свой меч и умер с ним.
\vs 1Sa 31:6 Так умер в тот день Саул и три сына его, и оруженосец его, а также и все люди его вместе.
\vs 1Sa 31:7 Израильтяне, жившие на стороне долины и за Иорданом, видя, что люди Израильские побежали и что умер Саул и сыновья его, оставили города свои и бежали, а Филистимляне пришли и засели в них.
\vs 1Sa 31:8 На другой день Филистимляне пришли грабить убитых, и нашли Саула и трех сыновей его, павших на горе Гелвуйской.
\vs 1Sa 31:9 И [поворотили его и] отсекли ему голову, и сняли с него оружие и послали по всей земле Филистимской, чтобы возвестить о сем в капищах идолов своих и народу;
\vs 1Sa 31:10 и положили оружие его в капище Астарты, а тело его повесили на стене Беф-Сана.
\vs 1Sa 31:11 И услышали жители Иависа Галаадского о том, как поступили Филистимляне с Саулом,
\vs 1Sa 31:12 и поднялись все люди сильные, и шли всю ночь, и взяли тело Саула и тела сыновей его со стены Беф-Сана, и пришли в Иавис, и сожгли их там;
\vs 1Sa 31:13 и взяли кости их, и погребли под дубом в Иависе, и постились семь дней.

\bibbookdescr{2Sa}{
  inline={\LARGE Вторая книга\\\Huge Царств\fns{У Евреев: <<Вторая Самуила>>.}},
  toc={2-я Царств},
  bookmark={2-я Царств},
  header={2-я Царств},
  %headerleft={},
  %headerright={},
  abbr={2~Цар}
}
\vs 2Sa 1:1 По смерти Саула, когда Давид возвратился от поражения Амаликитян и пробыл в Секелаге два дня,
\vs 2Sa 1:2 вот, на третий день приходит человек из стана Саулова; одежда на нем разодрана и прах на голове его. Придя к Давиду, он пал на землю и поклонился [ему].
\vs 2Sa 1:3 И сказал ему Давид: откуда ты пришел? И сказал тот: я убежал из стана Израильского.
\vs 2Sa 1:4 И сказал ему Давид: что произошло? расскажи мне. И тот сказал: народ побежал со сражения, и множество из народа пало и умерло, и умерли и Саул и сын его Ионафан.
\vs 2Sa 1:5 И сказал Давид отроку, рассказывавшему ему: как ты знаешь, что Саул и сын его Ионафан умерли?
\vs 2Sa 1:6 И сказал отрок, рассказывавший ему: я случайно пришел на гору Гелвуйскую, и вот, Саул пал на свое копье, колесницы же и всадники настигали его.
\vs 2Sa 1:7 Тогда он оглянулся назад и, увидев меня, позвал меня.
\vs 2Sa 1:8 И я сказал: вот я. Он сказал мне: кто ты? И я сказал ему: я~--- Амаликитянин.
\vs 2Sa 1:9 Тогда он сказал мне: подойди ко мне и убей меня, ибо тоска смертная объяла меня, душа моя все еще во мне.
\vs 2Sa 1:10 И я подошел к нему и убил его, ибо знал, что он не будет жив после своего падения; и взял я [царский] венец, бывший на голове его, и запястье, бывшее на руке его, и принес их к господину моему сюда.
\vs 2Sa 1:11 Тогда схватил Давид одежды свои и разодрал их, также и все люди, бывшие с ним, [разодрали одежды свои,]
\vs 2Sa 1:12 и рыдали и плакали, и постились до вечера о Сауле и о сыне его Ионафане, и о народе Господнем и о доме Израилевом, что пали они от меча.
\vs 2Sa 1:13 И сказал Давид отроку, рассказывавшему ему: откуда ты? И сказал он: я~--- сын пришельца Амаликитянина.
\vs 2Sa 1:14 Тогда Давид сказал ему: как не побоялся ты поднять руку, чтобы убить помазанника Господня?
\vs 2Sa 1:15 И призвал Давид одного из отроков и сказал ему: подойди, убей его.
\vs 2Sa 1:16 И \bibemph{тот} убил его, и он умер. И сказал к нему Давид: кровь твоя на голове твоей, ибо твои уста свидетельствовали на тебя, когда ты говорил: я убил помазанника Господня.
\rsbpar\vs 2Sa 1:17 И оплакал Давид Саула и сына его Ионафана сею плачевною песнью,
\vs 2Sa 1:18 и повелел научить сынов Иудиных луку, как написано в книге Праведного, и сказал:
\vs 2Sa 1:19 краса твоя, о Израиль, поражена на высотах твоих! как пали сильные!
\vs 2Sa 1:20 Не рассказывайте в Гефе, не возвещайте на улицах Аскалона, чтобы не радовались дочери Филистимлян, чтобы не торжествовали дочери необрезанных.
\vs 2Sa 1:21 Горы Гелвуйские! да [не сойдет] ни роса, ни дождь на вас, и да не будет \bibemph{на вас} полей с плодами, ибо там повержен щит сильных, щит Саула, как бы не был он помазан елеем.
\vs 2Sa 1:22 Без крови раненых, без тука сильных лук Ионафана не возвращался назад, и меч Саула не возвращался даром.
\vs 2Sa 1:23 Саул и Ионафан, любезные и согласные в жизни своей, не разлучились и в смерти своей; быстрее орлов, сильнее львов \bibemph{они были}.
\vs 2Sa 1:24 Дочери Израильские! плачьте о Сауле, который одевал вас в багряницу с украшениями и доставлял на одежды ваши золотые уборы.
\vs 2Sa 1:25 Как пали сильные на брани! Сражен Ионафан на высотах твоих.
\vs 2Sa 1:26 Скорблю о тебе, брат мой Ионафан; ты был очень дорог для меня; любовь твоя была для меня превыше любви женской.
\vs 2Sa 1:27 Как пали сильные, погибло оружие бранное!
\vs 2Sa 2:1 После сего Давид вопросил Господа, говоря: идти ли мне в какой-либо из городов Иудиных? И сказал ему Господь: иди. И сказал Давид: куда идти? И сказал Он: в Хеврон.
\vs 2Sa 2:2 И пошел туда Давид и обе жены его, Ахиноама Изреелитянка и Авигея, \bibemph{бывшая} жена Навала, Кармилитянка.
\vs 2Sa 2:3 И людей, бывших с ним, привел Давид, каждого с семейством его, и поселились в городе Хевроне.
\vs 2Sa 2:4 И пришли мужи Иудины и помазали там Давида на царство над домом Иудиным. И донесли Давиду, что жители Иависа Галаадского погребли Саула.
\rsbpar\vs 2Sa 2:5 И отправил Давид послов к жителям Иависа Галаадского, сказать им: благословенны вы у Господа за то, что оказали эту милость господину своему Саулу, [помазаннику Господню,] и погребли его [и Ионафана, сына его],
\vs 2Sa 2:6 и ныне да воздаст вам Господь милостью и истиною; и я сделаю вам благодеяние за то, что вы это сделали;
\vs 2Sa 2:7 ныне да укрепятся руки ваши, и будьте мужественны; ибо господин ваш Саул умер, а меня помазал дом Иудин царем над собою.
\vs 2Sa 2:8 Но Авенир, сын Ниров, начальник войска Саулова, взял Иевосфея, сына Саулова, и привел его в Маханаим,
\vs 2Sa 2:9 и воцарил его над Галаадом, и Ашуром, и Изреелем, и Ефремом, и Вениамином, и над всем Израилем.
\vs 2Sa 2:10 Сорок лет было Иевосфею, сыну Саулову, когда он воцарился над Израилем, и царствовал два года. Только дом Иудин остался с Давидом.
\vs 2Sa 2:11 Всего времени, в которое Давид царствовал в Хевроне над домом Иудиным, было семь лет и шесть месяцев.
\vs 2Sa 2:12 И вышел Авенир, сын Ниров, и слуги Иевосфея, сына Саулова, из Маханаима в Гаваон.
\vs 2Sa 2:13 Вышел и Иоав, сын Саруи, со слугами Давида, и встретились у Гаваонского пруда, и засели те на одной стороне пруда, а эти на другой стороне пруда.
\vs 2Sa 2:14 И сказал Авенир Иоаву: пусть встанут юноши и поиграют пред нами. И сказал Иоав: пусть встанут.
\vs 2Sa 2:15 И встали и пошли числом двенадцать Вениамитян со стороны Иевосфея, сына Саулова, и двенадцать из слуг Давидовых.
\vs 2Sa 2:16 Они схватили друг друга за голову, \bibemph{вонзили} меч один другому в бок и пали вместе. И было названо это место Хелкаф-Хаццурим, что в Гаваоне.
\vs 2Sa 2:17 И произошло в тот день жесточайшее сражение, и Авенир с людьми Израильскими был поражен слугами Давида.
\vs 2Sa 2:18 И были там три сына Саруи: Иоав, и Авесса, и Асаил. Асаил же был легок на ноги, как серна в поле.
\vs 2Sa 2:19 И погнался Асаил за Авениром и преследовал его, не уклоняясь ни направо, ни налево от следов Авенира.
\vs 2Sa 2:20 И оглянулся Авенир назад и сказал: ты ли это, Асаил? Тот сказал: я.
\vs 2Sa 2:21 И сказал ему Авенир: уклонись направо или налево, и выбери себе одного из отроков и возьми себе его вооружение. Но Асаил не захотел отстать от него.
\vs 2Sa 2:22 И повторил Авенир еще, говоря Асаилу: отстань от меня, чтоб я не поверг тебя на землю; тогда с каким лицем явлюсь я к Иоаву, брату твоему?
\vs 2Sa 2:23 [и где это бывает? возвратись к брату твоему Иоаву.] Но тот не захотел отстать. Тогда Авенир, поворотив копье, поразил его в живот; копье прошло насквозь его, и он упал там же и умер на месте. Все проходившие чрез то место, где пал и умер Асаил, останавливались.
\vs 2Sa 2:24 И преследовали Иоав и Авесса Авенира. Солнце уже зашло, когда они пришли к холму Амма, что против Гиаха, на дороге к пустыне Гаваонской.
\vs 2Sa 2:25 И собрались Вениамитяне вокруг Авенира и составили одно ополчение, и стали на вершине одного холма.
\vs 2Sa 2:26 И воззвал Авенир к Иоаву, и сказал: вечно ли будет пожирать меч? Или ты не знаешь, что последствия будут горестные? И доколе ты не скажешь людям, чтобы они перестали преследовать братьев своих?
\vs 2Sa 2:27 И сказал Иоав: жив Бог! если бы ты не говорил иначе, то еще утром перестали бы люди преследовать братьев своих.
\vs 2Sa 2:28 И затрубил Иоав трубою, и остановился весь народ, и не преследовали более Израильтян; сражение прекратилось.
\vs 2Sa 2:29 Авенир же и люди его шли равниною всю ту ночь и перешли Иордан, и прошли весь Битрон, и пришли в Маханаим.
\vs 2Sa 2:30 И возвратился Иоав от преследования Авенира и собрал весь народ, и недоставало из слуг Давидовых девятнадцати человек кроме Асаила.
\vs 2Sa 2:31 Слуги же Давидовы поразили Вениамитян и людей Авенировых; пало их триста шестьдесят человек.
\vs 2Sa 2:32 И взяли Асаила и похоронили его во гробе отца его, что в Вифлееме. Иоав же с людьми своими шел всю ночь и на рассвете прибыл в Хеврон.
\vs 2Sa 3:1 И была продолжительная распря между домом Сауловым и домом Давидовым. Давид все более и более усиливался, а дом Саулов более и более ослабевал.
\vs 2Sa 3:2 И родились у Давида [шесть] сыновей в Хевроне. Первенец его был Амнон от Ахиноамы Изреелитянки,
\vs 2Sa 3:3 а второй [сын] его~--- Далуиа от Авигеи, \bibemph{бывшей} жены Навала, Кармилитянки; третий~--- Авессалом, сын Маахи, дочери Фалмая, царя Гессурского;
\vs 2Sa 3:4 четвертый~--- Адония, сын Аггифы; пятый~--- Сафатия, сын Авиталы;
\vs 2Sa 3:5 шестой~--- Иефераам от Эглы, жены Давидовой. Они родились у Давида в Хевроне.
\rsbpar\vs 2Sa 3:6 Когда была распря между домом Саула и домом Давида, то Авенир поддерживал дом Саула.
\vs 2Sa 3:7 У Саула была наложница, по имени Рицпа, дочь Айя [и вошел к ней Авенир]. И сказал [Иевосфей] Авениру: зачем ты вошел к наложнице отца моего?
\vs 2Sa 3:8 Авенир же сильно разгневался на слова Иевосфея и сказал: разве я~--- собачья голова? Я против Иуды оказал ныне милость дому Саула, отца твоего, братьям его и друзьям его, и не предал тебя в руки Давида, а ты взыскиваешь ныне на мне грех из-за женщины.
\vs 2Sa 3:9 То и то пусть сделает Бог Авениру и еще больше сделает ему! Как клялся Господь Давиду, так и сделаю ему [в сей день]:
\vs 2Sa 3:10 отниму царство от дома Саулова и поставлю престол Давида над Израилем и над Иудою, от Дана до Вирсавии.
\vs 2Sa 3:11 И не мог Иевосфей возразить Авениру, ибо боялся его.
\vs 2Sa 3:12 И послал Авенир от себя послов к Давиду [в Хеврон, где он находился], сказать: чья эта земля? И еще сказать: заключи союз со мною, и рука моя будет с тобою, чтобы обратить к тебе весь народ Израильский.
\vs 2Sa 3:13 И сказал [Давид]: хорошо, я заключу союз с тобою, только прошу тебя об одном, именно~--- ты не увидишь лица моего, если не приведешь с собою Мелхолы, дочери Саула, когда придешь увидеться со мною.
\vs 2Sa 3:14 И отправил Давид послов к Иевосфею, сыну Саулову, сказать: отдай жену мою Мелхолу, которую я получил за сто краеобрезаний Филистимских.
\vs 2Sa 3:15 И послал Иевосфей и взял ее от мужа, от Фалтия, сына Лаишева.
\vs 2Sa 3:16 Пошел с нею и муж ее и с плачем провожал ее до Бахурима; но Авенир сказал ему: ступай назад. И он возвратился.
\vs 2Sa 3:17 И обратился Авенир к старейшинам Израильским, говоря: и вчера и третьего дня вы желали, чтобы Давид был царем над вами,
\vs 2Sa 3:18 теперь сделайте \bibemph{это}, ибо Господь сказал Давиду: <<рукою раба Моего Давида Я спасу народ Мой Израиля от руки Филистимлян и от руки всех врагов его>>.
\vs 2Sa 3:19 То же говорил Авенир и Вениамитянам. И пошел Авенир в Хеврон, чтобы пересказать Давиду все, чего желали Израиль и весь дом Вениаминов.
\vs 2Sa 3:20 И пришел Авенир к Давиду в Хеврон и с ним двадцать человек, и сделал Давид пир для Авенира и людей, бывших с ним.
\vs 2Sa 3:21 И сказал Авенир Давиду: я встану и пойду и соберу к господину моему царю весь народ Израильский, и они вступят в завет с тобою, и будешь царствовать над всеми, как желает душа твоя. И отпустил Давид Авенира, и он ушел с миром.
\vs 2Sa 3:22 И вот, слуги Давидовы с Иоавом пришли из похода и принесли с собою много добычи; но Авенира уже не было с Давидом в Хевроне, ибо \bibemph{Давид} отпустил его, и он ушел с миром.
\vs 2Sa 3:23 Когда Иоав и все войско, ходившее с ним, пришли, то Иоаву рассказали: приходил Авенир, сын Ниров, к царю, и тот отпустил его, и он ушел с миром.
\vs 2Sa 3:24 И пришел Иоав к царю и сказал: что ты сделал? Вот, приходил к тебе Авенир; зачем ты отпустил его, и он ушел?
\vs 2Sa 3:25 Ты знаешь Авенира, сына Нирова: он приходил обмануть тебя, узнать выход твой и вход твой и разведать все, что ты делаешь.
\vs 2Sa 3:26 И вышел Иоав от Давида и послал гонцов вслед за Авениром; и возвратили они его от колодезя Сира, без ведома Давида.
\vs 2Sa 3:27 Когда Авенир возвратился в Хеврон, то Иоав отвел его внутрь ворот, как будто для того, чтобы поговорить с ним тайно, и там поразил его в живот. И умер \bibemph{Авенир} за кровь Асаила, брата Иоавова.
\vs 2Sa 3:28 И услышал после Давид \bibemph{об этом} и сказал: невинен я и царство мое вовек пред Господом в крови Авенира, сына Нирова;
\vs 2Sa 3:29 пусть падет она на голову Иоава и на весь дом отца его; пусть никогда не остается дом Иоава без семеноточивого, или прокаженного, или опирающегося на посох, или падающего от меча, или нуждающегося в хлебе.
\vs 2Sa 3:30 Иоав же и брат его Авесса убили Авенира за то, что он умертвил брата их Асаила в сражении у Гаваона.
\vs 2Sa 3:31 И сказал Давид Иоаву и всем людям, бывшим с ним: раздерите одежды ваши и оденьтесь во вретища и плачьте над Авениром. И царь Давид шел за гробом \bibemph{его}.
\vs 2Sa 3:32 Когда погребали Авенира в Хевроне, то царь громко плакал над гробом Авенира; плакал и весь народ.
\vs 2Sa 3:33 И оплакал царь Авенира, говоря: смертью ли подлого умирать Авениру?
\vs 2Sa 3:34 Руки твои не были связаны, и ноги твои не в оковах, и ты пал, как падают от разбойников. И весь народ стал еще более плакать над ним.
\vs 2Sa 3:35 И пришел весь народ предложить Давиду хлеба, когда еще продолжался день; но Давид поклялся, говоря: то и то пусть сделает со мною Бог и еще больше сделает, если я до захождения солнца вкушу хлеба или чего-нибудь.
\vs 2Sa 3:36 И весь народ узнал это, и понравилось ему это, как и все, что делал царь, нравилось всему народу.
\vs 2Sa 3:37 И узнал весь народ и весь Израиль в тот день, что не от царя произошло умерщвление Авенира, сына Нирова.
\vs 2Sa 3:38 И сказал царь слугам своим: знаете ли, что вождь и великий муж пал в этот день в Израиле?
\vs 2Sa 3:39 Я теперь еще слаб, хотя и помазан на царство, а эти люди, сыновья Саруи, сильнее меня; пусть же воздаст Господь делающему злое по злобе его!
\vs 2Sa 4:1 И услышал [Иевосфей,] сын Саулов, что умер Авенир в Хевроне, и опустились руки его, и весь Израиль смутился.
\rsbpar\vs 2Sa 4:2 У [Иевосфея,] сына Саулова, два было предводителя войска; имя одного~--- Баана и имя другого~--- Рихав, сыновья Реммона Беерофянина, из потомков Вениаминовых, ибо и Беероф причислялся к Вениамину.
\vs 2Sa 4:3 И убежали Беерофяне в Гиффаим и остались там пришельцами до сего дня.
\vs 2Sa 4:4 У Ионафана, сына Саулова, был сын хромой. Пять лет было ему, когда пришло известие о Сауле и Ионафане из Изрееля, и нянька, взяв его, побежала. И когда она бежала поспешно, то он упал, и сделался хромым. Имя его Мемфивосфей.
\vs 2Sa 4:5 И пошли сыны Реммона Беерофянина, Рихав и Баана, и пришли в самый жар дня к дому Иевосфея; а он спал на постели в полдень.
\vs 2Sa 4:6 [А привратник дома, очищавший пшеницу, задремал и уснул] и Рихав и Баана, брат его, вошли внутрь дома, \bibemph{как бы} для того, чтобы взять пшеницы; и поразили его в живот и убежали.
\vs 2Sa 4:7 Когда они вошли в дом, [Иевосфей] лежал на постели своей, в спальной комнате своей; и они поразили его, и умертвили его, и отрубили голову его, и взяли голову его с собою, и шли пустынною дорогою всю ночь;
\vs 2Sa 4:8 и принесли голову Иевосфея к Давиду в Хеврон и сказали царю: вот голова Иевосфея, сына Саула, врага твоего, который искал души твоей; ныне Господь отмстил за господина моего царя Саулу [врагу твоему] и потомству его.
\vs 2Sa 4:9 И отвечал Давид Рихаву и Баане, брату его, сыновьям Реммона Беерофянина, и сказал им: жив Господь, избавивший душу мою от всякой скорби!
\vs 2Sa 4:10 если того, кто принес мне известие, сказав: <<вот, умер Саул, [и Ионафан]>>, и кто считал себя радостным вестником, я схватил и убил его в Секелаге, вместо того, чтобы дать ему награду,
\vs 2Sa 4:11 то теперь, когда негодные люди убили человека невинного в его доме на постели его, неужели я не взыщу крови его от руки вашей и не истреблю вас с земли?
\vs 2Sa 4:12 И приказал Давид слугам, и убили их, и отрубили им руки и ноги, и повесили их над прудом в Хевроне. А голову Иевосфея взяли и погребли во гробе Авенира, в Хевроне.
\vs 2Sa 5:1 И пришли все колена Израилевы к Давиду в Хеврон и сказали: вот, мы~--- кости твои и плоть твоя;
\vs 2Sa 5:2 еще вчера и третьего дня, когда Саул царствовал над нами, ты выводил и вводил Израиля; и сказал Господь тебе: <<ты будешь пасти народ Мой Израиля и ты будешь вождем Израиля>>.
\vs 2Sa 5:3 И пришли все старейшины Израиля к царю в Хеврон, и заключил с ними царь Давид завет в Хевроне пред Господом; и помазали Давида в царя над [всем] Израилем.
\rsbpar\vs 2Sa 5:4 Тридцать лет было Давиду, когда он воцарился; царствовал сорок лет.
\vs 2Sa 5:5 В Хевроне царствовал над Иудою семь лет и шесть месяцев, и в Иерусалиме царствовал тридцать три года над всем Израилем и Иудою.
\vs 2Sa 5:6 И пошел царь и люди его на Иерусалим против Иевусеев, жителей той страны; но они говорили Давиду: <<ты не войдешь сюда; тебя отгонят слепые и хромые>>,~--- это значило: <<не войдет сюда Давид>>.
\vs 2Sa 5:7 Но Давид взял крепость Сион: это~--- город Давидов.
\vs 2Sa 5:8 И сказал Давид в тот день: всякий, убивая Иевусеев, пусть поражает копьем и хромых и слепых, ненавидящих душу Давида. Посему и говорится: слепой и хромой не войдет в дом [Господень].
\vs 2Sa 5:9 И поселился Давид в крепости, и назвал ее городом Давидовым, и обстроил кругом от Милло и внутри.
\vs 2Sa 5:10 И преуспевал Давид и возвышался, и Господь Бог Саваоф \bibemph{был} с ним.
\rsbpar\vs 2Sa 5:11 И прислал Хирам, царь Тирский, послов к Давиду и кедровые деревья и плотников и каменщиков, и они построили дом Давиду.
\vs 2Sa 5:12 И уразумел Давид, что Господь утвердил его царем над Израилем и что возвысил царство его ради народа Своего Израиля.
\vs 2Sa 5:13 И взял Давид еще наложниц и жен из Иерусалима, после того, как пришел из Хеврона.
\vs 2Sa 5:14 И родились еще у Давида сыновья и дочери. И вот имена родившихся у него в Иерусалиме: Самус, и Совав, и Нафан, и Соломон,
\vs 2Sa 5:15 и Евеар, и Елисуа, и Нафек, и Иафиа,
\vs 2Sa 5:16 и Елисама, и Елидае, и Елифалеф, [Самае, Иосиваф, Нафан, Галамаан, Иеваар, Феисус, Елифалаф, Нагев, Нафек, Ионафан, Леасамис, Ваалимаф и Елифааф].
\rsbpar\vs 2Sa 5:17 Когда Филистимляне услышали, что Давида помазали на царство над Израилем, то поднялись все Филистимляне искать Давида. И услышал Давид и пошел в крепость.
\vs 2Sa 5:18 А Филистимляне пришли и расположились в долине Рефаим.
\vs 2Sa 5:19 И вопросил Давид Господа, говоря: идти ли мне против Филистимлян? предашь ли их в руки мои? И сказал Господь Давиду: иди, ибо Я предам Филистимлян в руки твои.
\vs 2Sa 5:20 И пошел Давид в Ваал-Перацим и поразил их там, и сказал Давид: Господь разнес врагов моих предо мною, как разносит вода. Посему и месту тому дано имя Ваал-Перацим.
\vs 2Sa 5:21 И оставили там [Филистимляне] истуканов своих, а Давид с людьми своими взял их [и велел сжечь их в огне].
\vs 2Sa 5:22 И пришли опять Филистимляне и расположились в долине Рефаим.
\vs 2Sa 5:23 И вопросил Давид Господа, [идти ли мне против Филистимлян, и предашь ли их в руки мои?] И Он отвечал ему: не выходи навстречу им, а зайди им с тылу и иди к ним со стороны тутовой рощи;
\vs 2Sa 5:24 и когда услышишь шум как бы идущего по вершинам тутовых дерев, то двинься, ибо тогда пошел Господь пред тобою, чтобы поразить войско Филистимское.
\vs 2Sa 5:25 И сделал Давид, как повелел ему Господь, и поразил Филистимлян от Гаваи до Газера.
\vs 2Sa 6:1 И собрал снова Давид всех отборных \bibemph{людей} из Израиля, тридцать тысяч\fns{В греческом переводе: около семидесяти тысяч.}.
\vs 2Sa 6:2 И встал и пошел Давид и весь народ, бывший с ним из Ваала Иудина, чтобы перенести оттуда ковчег Божий, на котором нарицается имя Господа Саваофа, сидящего на херувимах.
\vs 2Sa 6:3 И поставили ковчег Божий на новую колесницу и вывезли его из дома Аминадава, что на холме. Сыновья же Аминадава, Оза и Ахио, вели новую колесницу.
\vs 2Sa 6:4 И повезли ее с ковчегом Божиим из дома Аминадава, что на холме; и Ахио шел пред ковчегом [Господним].
\vs 2Sa 6:5 А Давид и все сыны Израилевы играли пред Господом на всяких музыкальных орудиях из кипарисового дерева, и на цитрах, и на псалтирях, и на тимпанах, и на систрах, и на кимвалах.
\vs 2Sa 6:6 И когда дошли до гумна Нахонова, Оза простер руку свою к ковчегу Божию [чтобы придержать его] и взялся за него, ибо волы наклонили его.
\vs 2Sa 6:7 Но Господь прогневался на Озу, и поразил его Бог там же за дерзновение, и умер он там у ковчега Божия.
\vs 2Sa 6:8 И опечалился Давид, что Господь поразил Озу. Место сие и доныне называется: <<поражение Озы>>.
\vs 2Sa 6:9 И устрашился Давид в тот день Господа и сказал: как войти ко мне ковчегу Господню?
\vs 2Sa 6:10 И не захотел Давид везти ковчег Господень к себе, в город Давидов, а обратил его в дом Аведдара Гефянина.
\vs 2Sa 6:11 И оставался ковчег Господень в доме Аведдара Гефянина три месяца, и благословил Господь Аведдара и весь дом его.
\vs 2Sa 6:12 Когда донесли царю Давиду, говоря: <<Господь благословил дом Аведдара и все, что было у него, ради ковчега Божия>>, то пошел Давид и с торжеством перенес ковчег Божий из дома Аведдара в город Давидов.
\vs 2Sa 6:13 И когда несшие ковчег Господень проходили по шести шагов, он приносил в жертву тельца и овна.
\vs 2Sa 6:14 Давид скакал из всей силы пред Господом; одет же был Давид в льняной ефод.
\vs 2Sa 6:15 Так Давид и весь дом Израилев несли ковчег Господень с восклицаниями и трубными звуками.
\rsbpar\vs 2Sa 6:16 Когда входил ковчег Господень в город Давидов, Мелхола, дочь Саула, смотрела в окно и, увидев царя Давида, скачущего и пляшущего пред Господом, уничижила его в сердце своем.
\vs 2Sa 6:17 И принесли ковчег Господень и поставили его на своем месте посреди скинии, которую устроил для него Давид; и принес Давид всесожжения пред Господом и жертвы мирные.
\vs 2Sa 6:18 Когда Давид окончил приношение всесожжений и жертв мирных, то благословил он народ именем Господа Саваофа;
\vs 2Sa 6:19 и раздал всему народу, всему множеству Израильтян [от Дана даже до Вирсавии], как мужчинам, так и женщинам, по одному хлебу и по куску жареного мяса и по одной лепешке каждому. И пошел весь народ, каждый в дом свой.
\vs 2Sa 6:20 Когда Давид возвратился, чтобы благословить дом свой, то Мелхола, дочь Саула, вышла к нему навстречу, [и приветствовала его] и сказала: как отличился сегодня царь Израилев, обнажившись сегодня пред глазами рабынь рабов своих, как обнажается какой-нибудь пустой человек!
\vs 2Sa 6:21 И сказал Давид Мелхоле: пред Господом [плясать буду. И благословен Господь], Который предпочел меня отцу твоему и всему дому его, утвердив меня вождем народа Господня, Израиля; пред Господом играть и плясать буду;
\vs 2Sa 6:22 и я еще больше уничижусь, и сделаюсь еще ничтожнее в глазах моих, и пред служанками, о которых ты говоришь, я буду славен.
\vs 2Sa 6:23 И у Мелхолы, дочери Сауловой, не было детей до дня смерти ее.
\vs 2Sa 7:1 Когда царь жил в доме своем, и Господь успокоил его от всех окрестных врагов его,
\vs 2Sa 7:2 тогда сказал царь пророку Нафану: вот, я живу в доме кедровом, а ковчег Божий находится под шатром.
\vs 2Sa 7:3 И сказал Нафан царю: все, что у тебя на сердце, иди, делай; ибо Господь с тобою.
\vs 2Sa 7:4 Но в ту же ночь было слово Господа к Нафану:
\vs 2Sa 7:5 пойди, скажи рабу Моему Давиду: так говорит Господь: ты ли построишь Мне дом для Моего обитания,
\vs 2Sa 7:6 когда Я не жил в доме с того времени, как вывел сынов Израилевых из Египта, и до сего дня, но переходил в шатре и в скинии?
\vs 2Sa 7:7 Где Я ни ходил со всеми сынами Израиля, говорил ли Я хотя слово какому-либо из колен, которому Я назначил пасти народ Мой Израиля: <<почему не построите Мне кедрового дома>>?
\vs 2Sa 7:8 И теперь так скажи рабу Моему Давиду: так говорит Господь Саваоф: Я взял тебя от стада овец, чтобы ты был вождем народа Моего, Израиля;
\vs 2Sa 7:9 и был с тобою везде, куда ни ходил ты, и истребил всех врагов твоих пред лицем твоим, и сделал имя твое великим, как имя великих на земле.
\vs 2Sa 7:10 И Я устрою место для народа Моего, для Израиля, и укореню его, и будет он спокойно жить на месте своем, и не будет тревожиться больше, и люди нечестивые не станут более теснить его, как прежде,
\vs 2Sa 7:11 с того времени, как Я поставил судей над народом Моим, Израилем; и Я успокою тебя от всех врагов твоих. И Господь возвещает тебе, что Он устроит тебе дом.
\vs 2Sa 7:12 Когда же исполнятся дни твои, и ты почиешь с отцами твоими, то Я восставлю после тебя семя твое, которое произойдет из чресл твоих, и упрочу царство его.
\vs 2Sa 7:13 Он построит дом имени Моему, и Я утвержу престол царства его на веки.
\vs 2Sa 7:14 Я буду ему отцом, и он будет Мне сыном; и если он согрешит, Я накажу его жезлом мужей и ударами сынов человеческих;
\vs 2Sa 7:15 но милости Моей не отниму от него, как Я отнял от Саула, которого Я отверг пред лицем твоим.
\vs 2Sa 7:16 И будет непоколебим дом твой и царство твое на веки пред лицем Моим, и престол твой устоит во веки.
\vs 2Sa 7:17 Все эти слова и все это видение Нафан пересказал Давиду.
\rsbpar\vs 2Sa 7:18 И пошел царь Давид, и предстал пред лицем Господа, и сказал: кто я, Господи [мой], Господи, и что такое дом мой, что Ты меня так возвеличил!
\vs 2Sa 7:19 И этого еще мало показалось в очах Твоих, Господи мой, Господи; но Ты возвестил еще о доме раба Твоего вдаль. Это уже по-человечески. Господи мой, Господи!
\vs 2Sa 7:20 Что еще может сказать Тебе Давид? Ты знаешь раба Твоего, Господи мой, Господи!
\vs 2Sa 7:21 Ради слова Твоего и по сердцу Твоему Ты делаешь это, открывая все это великое рабу Твоему.
\vs 2Sa 7:22 По всему велик Ты, Господи мой, Господи! ибо нет подобного Тебе и нет Бога, кроме Тебя, по всему, что слышали мы своими ушами.
\vs 2Sa 7:23 И кто подобен народу Твоему, Израилю, единственному народу на земле, для которого приходил Бог, чтобы приобрести \bibemph{его} Себе в народ и прославить Свое имя \bibemph{и} совершить великое и страшное пред народом Твоим, который Ты приобрел Себе от Египтян, изгнав народы и богов их?
\vs 2Sa 7:24 И Ты укрепил за Собою народ Твой, Израиля, как собственный народ, на веки, и Ты, Господи, сделался его Богом.
\vs 2Sa 7:25 И ныне, Господи Боже, утверди на веки слово, которое изрек Ты о рабе Твоем и о доме его, и исполни то, что Ты изрек.
\vs 2Sa 7:26 И да возвеличится имя Твое во веки, чтобы говорили: <<Господь Саваоф~--- Бог над Израилем>>. И дом раба Твоего Давида да будет тверд пред лицем Твоим.
\vs 2Sa 7:27 Так как ты, Господи Саваоф, Боже Израилев, открыл рабу Твоему, говоря: <<устрою тебе дом>>, то раб Твой уготовал сердце свое, чтобы молиться Тебе такою молитвою.
\vs 2Sa 7:28 Итак, Господи мой, Господи! Ты Бог, и слова Твои непреложны, и Ты возвестил рабу Твоему такое благо!
\vs 2Sa 7:29 И ныне начни и благослови дом раба Твоего, чтоб он был вечно пред лицем Твоим, ибо Ты, Господи мой, Господи, возвестил это, и благословением Твоим соделается дом раба Твоего благословенным, [чтоб быть ему пред Тобою] во веки.
\vs 2Sa 8:1 После сего Давид поразил Филистимлян и смирил их, и взял Давид Мефег-Гаамма из рук Филистимлян.
\vs 2Sa 8:2 И поразил Моавитян и смерил их веревкою, положив их на землю; и отмерил две веревки на умерщвление, а одну веревку на оставление в живых. И сделались Моавитяне у Давида рабами, платящими дань.
\vs 2Sa 8:3 И поразил Давид Адраазара, сына Реховова, царя Сувского, когда тот шел, чтоб восстановить свое владычество при реке [Евфрате];
\vs 2Sa 8:4 и взял Давид у него тысячу семьсот всадников\fns{В греческом переводе: тысячу колесниц и семь тысяч всадников.} и двадцать тысяч человек пеших, и подрезал Давид жилы у всех коней колесничных, оставив [себе] из них для ста колесниц.
\vs 2Sa 8:5 И пришли Сирийцы Дамасские на помощь к Адраазару, царю Сувскому; но Давид поразил двадцать две тысячи человек Сирийцев.
\vs 2Sa 8:6 И поставил Давид охранные войска в Сирии Дамасской, и стали Сирийцы у Давида рабами, платящими дань. И хранил Господь Давида везде, куда он ни ходил.
\vs 2Sa 8:7 И взял Давид золотые щиты, которые были у рабов Адраазара, и принес их в Иерусалим. [Их взял \bibemph{потом} Сусаким, царь Египетский, во время нашествия своего на Иерусалим, во дни Ровоама, сына Соломонова.]
\vs 2Sa 8:8 А в Бефе и Берофе, городах Адраазаровых, взял царь Давид весьма много меди, [из которой Соломон устроил медное море и столбы, и умывальницы и все сосуды].
\vs 2Sa 8:9 И услышал Фой, царь Имафа, что Давид поразил все войско Адраазарово,
\vs 2Sa 8:10 и послал Фой Иорама, сына своего, к царю Давиду, приветствовать его и благодарить его за то, что он воевал с Адраазаром и поразил его; ибо Адраазар вел войны с Фоем. В руках же \bibemph{Иорама} были сосуды серебряные, золотые и медные.
\vs 2Sa 8:11 Их также посвятил царь Давид Господу, вместе с серебром и золотом, которое посвятил из \bibemph{отнятого} у всех покоренных им народов:
\vs 2Sa 8:12 у Сирийцев, и Моавитян, и Аммонитян, и Филистимлян, и Амаликитян, и из отнятого у Адраазара, сына Реховова, царя Сувского.
\vs 2Sa 8:13 И сделал Давид себе имя, возвращаясь с поражения восемнадцати тысяч Сирийцев в долине Соленой.
\vs 2Sa 8:14 И поставил он охранные войска в Идумее; во всей Идумее поставил охранные войска, и все Идумеяне были рабами Давиду. И хранил Господь Давида везде, куда он ни ходил.
\vs 2Sa 8:15 И царствовал Давид над всем Израилем, и творил Давид суд и правду над всем народом своим.
\vs 2Sa 8:16 Иоав же, сын Саруи, \bibemph{был начальником} войска; и Иосафат, сын Ахилуда,~--- дееписателем;
\vs 2Sa 8:17 Садок, сын Ахитува, и Ахимелех, сын Авиафара,~--- священниками, Сераия~--- писцом;
\vs 2Sa 8:18 и Ванея, сын Иодая~--- \bibemph{начальником} над Хелефеями и Фелефеями, и сыновья Давида~--- первыми при дворе.
\vs 2Sa 9:1 И сказал Давид: не остался ли еще кто-нибудь из дома Саулова? я оказал бы ему милость ради Ионафана.
\vs 2Sa 9:2 В доме Саула был раб, по имени Сива; и позвали его к Давиду, и сказал ему царь: ты ли Сива? И тот сказал: я, раб твой.
\vs 2Sa 9:3 И сказал царь: нет ли еще кого-нибудь из дома Саулова? я оказал бы ему милость Божию. И сказал Сива царю: есть сын Ионафана, хромой ногами.
\vs 2Sa 9:4 И сказал ему царь: где он? И сказал Сива царю: вот, он в доме Махира, сына Аммиэлова, в Лодеваре.
\vs 2Sa 9:5 И послал царь Давид, и взяли его из дома Махира, сына Аммиэлова, из Лодевара.
\vs 2Sa 9:6 И пришел Мемфивосфей, сын Ионафана, сына Саулова, к Давиду, и пал на лице свое, и поклонился [царю]. И сказал Давид: Мемфивосфей! И сказал тот: вот раб твой.
\vs 2Sa 9:7 И сказал ему Давид: не бойся; я окажу тебе милость ради отца твоего Ионафана и возвращу тебе все поля Саула, отца твоего, и ты всегда будешь есть хлеб за моим столом.
\vs 2Sa 9:8 И поклонился [Мемфивосфей] и сказал: что такое раб твой, что ты призрел на такого мертвого пса, как я?
\vs 2Sa 9:9 И призвал царь Сиву, слугу Саула, и сказал ему: все, что принадлежало Саулу и всему дому его, я отдаю сыну господина твоего;
\vs 2Sa 9:10 итак обрабатывай для него землю ты и сыновья твои и рабы твои, и доставляй \bibemph{плоды ее}, чтобы у сына господина твоего был хлеб для пропитания; Мемфивосфей же, сын господина твоего, всегда будет есть за моим столом. У Сивы было пятнадцать сыновей и двадцать рабов.
\vs 2Sa 9:11 И сказал Сива царю: все, что приказывает господин мой царь рабу своему, исполнит раб твой. Мемфивосфей ел за столом [Давида], как один из сыновей царя.
\vs 2Sa 9:12 У Мемфивосфея был малолетний сын, по имени Миха. Все живущие в доме Сивы были рабами Мемфивосфея.
\vs 2Sa 9:13 И жил Мемфивосфей в Иерусалиме, ибо он ел всегда за царским столом. Он был хром на обе ноги.
\vs 2Sa 10:1 Спустя несколько времени умер царь Аммонитский, и воцарился вместо него сын его Аннон.
\vs 2Sa 10:2 И сказал Давид: окажу я милость Аннону, сыну Наасову, за благодеяние, которое оказал мне отец его. И послал Давид слуг своих утешить Аннона об отце его. И пришли слуги Давидовы в землю Аммонитскую.
\vs 2Sa 10:3 Но князья Аммонитские сказали Аннону, господину своему: неужели ты думаешь, что Давид из уважения к отцу твоему прислал к тебе утешителей? не для того ли, чтобы осмотреть город и высмотреть в нем и \bibemph{после} разрушить его, прислал Давид слуг своих к тебе?
\vs 2Sa 10:4 И взял Аннон слуг Давидовых, и обрил каждому из них половину бороды, и обрезал одежды их наполовину, до чресл, и отпустил их.
\vs 2Sa 10:5 Когда донесли об этом Давиду, то он послал к ним навстречу, так как они были очень обесчещены. И велел царь сказать им: оставайтесь в Иерихоне, пока отрастут бороды ваши, и \bibemph{тогда} возвратитесь.
\vs 2Sa 10:6 И увидели Аммонитяне, что они сделались ненавистными для Давида; и послали Аммонитяне нанять Сирийцев из Беф-Рехова и Сирийцев Сувы двадцать тысяч пеших, у царя [Амаликитского] Маахи тысячу человек и из Истова двенадцать тысяч человек.
\vs 2Sa 10:7 Когда услышал об этом Давид, то послал Иоава со всем войском храбрых.
\vs 2Sa 10:8 И вышли Аммонитяне и расположились к сражению у ворот, а Сирийцы Сувы и Рехова, и Истова, и Маахи, \bibemph{стали} отдельно в поле.
\vs 2Sa 10:9 И увидел Иоав, что неприятельское войско было поставлено против него и спереди и сзади, и избрал \bibemph{воинов} из всех отборных в Израиле, и выстроил их против Сирийцев;
\vs 2Sa 10:10 остальную же часть людей поручил Авессе, брату своему, чтоб он выстроил их против Аммонитян.
\vs 2Sa 10:11 И сказал \bibemph{Иоав}: если Сирийцы будут одолевать меня, ты поможешь мне; а если Аммонитяне тебя будут одолевать, я приду к тебе на помощь;
\vs 2Sa 10:12 будь мужествен, и будем стоять твердо за народ наш и за города Бога нашего, а Господь сделает, что Ему угодно.
\vs 2Sa 10:13 И вступил Иоав и народ, который \bibemph{был} у него, в сражение с Сирийцами, и они побежали от него.
\vs 2Sa 10:14 Аммонитяне же, увидев, что Сирийцы бегут, побежали от Авессы и ушли в город. И возвратился Иоав от Аммонитян и пришел в Иерусалим.
\vs 2Sa 10:15 Сирийцы, видя, что они поражены Израильтянами, собрались вместе.
\vs 2Sa 10:16 И послал Адраазар и призвал Сирийцев, которые за рекою [Халамаком], и пришли они к Еламу; а Совак, военачальник Адраазаров, предводительствовал ими.
\vs 2Sa 10:17 Когда донесли \bibemph{об этом} Давиду, то он собрал всех Израильтян, и перешел Иордан и пришел к Еламу. Сирийцы выстроились против Давида и сразились с ним.
\vs 2Sa 10:18 И побежали Сирийцы от Израильтян. Давид истребил у Сирийцев семьсот колесниц и сорок тысяч всадников; поразил и военачальника Совака, который там и умер.
\vs 2Sa 10:19 Когда все цари покорные Адраазару увидели, что они поражены Израильтянами, то заключили мир с Израильтянами и покорились им. А Сирийцы боялись более помогать Аммонитянам.
\vs 2Sa 11:1 Через год, в то время, когда выходят цари \bibemph{в походы}, Давид послал Иоава и слуг своих с ним и всех Израильтян; и они поразили Аммонитян и осадили Равву; Давид же оставался в Иерусалиме.
\rsbpar\vs 2Sa 11:2 Однажды под вечер Давид, встав с постели, прогуливался на кровле царского дома и увидел с кровли купающуюся женщину; а та женщина была очень красива.
\vs 2Sa 11:3 И послал Давид разведать, кто эта женщина? И сказали ему: это Вирсавия, дочь Елиама, жена Урии Хеттеянина.
\vs 2Sa 11:4 Давид послал слуг взять ее; и она пришла к нему, и он спал с нею. Когда же она очистилась от нечистоты своей, возвратилась в дом свой.
\vs 2Sa 11:5 Женщина эта сделалась беременною и послала известить Давида, говоря: я беременна.
\vs 2Sa 11:6 И послал Давид \bibemph{сказать} Иоаву: пришли ко мне Урию Хеттеянина. И послал Иоав Урию к Давиду.
\vs 2Sa 11:7 И пришел к нему Урия, и расспросил \bibemph{его} Давид о положении Иоава и о положении народа, и о ходе войны.
\vs 2Sa 11:8 И сказал Давид Урии: иди домой и омой ноги свои. И вышел Урия из дома царского, а вслед за ним понесли и царское кушанье.
\vs 2Sa 11:9 Но Урия спал у ворот царского дома со всеми слугами своего господина, и не пошел в свой дом.
\vs 2Sa 11:10 И донесли Давиду, говоря: не пошел Урия в дом свой. И сказал Давид Урии: вот, ты пришел с дороги; отчего же не пошел ты в дом свой?
\vs 2Sa 11:11 И сказал Урия Давиду: ковчег [Божий] и Израиль и Иуда находятся в шатрах, и господин мой Иоав и рабы господина моего пребывают в поле, а я вошел бы в дом свой и есть и пить и спать со своею женою! Клянусь твоею жизнью и жизнью души твоей, этого я не сделаю.
\vs 2Sa 11:12 И сказал Давид Урии: останься здесь и на этот день, а завтра я отпущу тебя. И остался Урия в Иерусалиме на этот день до завтра.
\vs 2Sa 11:13 И пригласил его Давид, и ел \bibemph{Урия} пред ним и пил, и напоил его \bibemph{Давид}. Но вечером \bibemph{Урия} пошел спать на постель свою с рабами господина своего, а в свой дом не пошел.
\vs 2Sa 11:14 Поутру Давид написал письмо к Иоаву и послал \bibemph{его} с Уриею.
\vs 2Sa 11:15 В письме он написал так: поставьте Урию там, где \bibemph{будет} самое сильное сражение, и отступите от него, чтоб он был поражен и умер.
\vs 2Sa 11:16 Посему, когда Иоав осаждал город, то поставил он Урию на таком месте, о котором знал, что там храбрые люди.
\vs 2Sa 11:17 И вышли люди из города и сразились с Иоавом, и пало несколько из народа, из слуг Давидовых; был убит также и Урия Хеттеянин.
\vs 2Sa 11:18 И послал Иоав донести Давиду о всем ходе сражения.
\vs 2Sa 11:19 И приказал посланному, говоря: когда ты расскажешь царю о всем ходе сражения
\vs 2Sa 11:20 и увидишь, что царь разгневается, и скажет тебе: <<зачем вы так близко подходили к городу сражаться? разве вы не знали, что со стены будут бросать на вас?
\vs 2Sa 11:21 кто убил Авимелеха, сына Иероваалова? не женщина ли бросила на него со стены обломок жернова [и поразила его], и он умер в Тевеце? Зачем же вы близко подходили к стене?>> тогда ты скажи: и раб твой Урия Хеттеянин также [поражен и] умер.
\vs 2Sa 11:22 И пошел [посланный от Иоава к царю в Иерусалим], и пришел, и рассказал Давиду обо всем, для чего послал его Иоав, обо всем ходе сражения. [И разгневался Давид на Иоава и сказал посланному: зачем вы близко подходили к городу сражаться? разве вы не знали, что вас поражать будут со стены? кто убил Авимелеха, сына Иероваалова? не женщина ли бросила на него со стены обломок жернова, и он умер в Тевеце? Зачем вы близко подходили к стене?]
\vs 2Sa 11:23 Тогда посланный сказал Давиду: одолевали нас те люди и вышли к нам в поле, и мы преследовали их до входа в ворота;
\vs 2Sa 11:24 тогда стреляли стрелки со стены на рабов твоих, и умерли \bibemph{некоторые} из рабов царя; умер также и раб твой Урия Хеттеянин.
\vs 2Sa 11:25 Тогда сказал Давид посланному: так скажи Иоаву: <<пусть не смущает тебя это дело, ибо меч поядает иногда того, иногда сего; усиль войну твою против города и разрушь его>>. Так ободри его.
\vs 2Sa 11:26 И услышала жена Урии, что умер Урия, муж ее, и плакала по муже своем.
\vs 2Sa 11:27 Когда кончилось время плача, Давид послал, и взял ее в дом свой, и она сделалась его женою и родила ему сына. И было это дело, которое сделал Давид, зло в очах Господа.
\vs 2Sa 12:1 И послал Господь Нафана [пророка] к Давиду, и тот пришел к нему и сказал ему: в одном городе были два человека, один богатый, а другой бедный;
\vs 2Sa 12:2 у богатого было очень много мелкого и крупного скота,
\vs 2Sa 12:3 а у бедного ничего, кроме одной овечки, которую он купил маленькую и выкормил, и она выросла у него вместе с детьми его; от хлеба его она ела, и из его чаши пила, и на груди у него спала, и была для него, как дочь;
\vs 2Sa 12:4 и пришел к богатому человеку странник, и тот пожалел взять из своих овец или волов, чтобы приготовить [обед] для странника, который пришел к нему, а взял овечку бедняка и приготовил ее для человека, который пришел к нему.
\vs 2Sa 12:5 Сильно разгневался Давид на этого человека и сказал Нафану: жив Господь! достоин смерти человек, сделавший это;
\vs 2Sa 12:6 и за овечку он должен заплатить вчетверо, за то, что он сделал это, и за то, что не имел сострадания.
\vs 2Sa 12:7 И сказал Нафан Давиду: ты~--- тот человек, [который сделал это]. Так говорит Господь Бог Израилев: Я помазал тебя в царя над Израилем и Я избавил тебя от руки Саула,
\vs 2Sa 12:8 и дал тебе дом господина твоего и жен господина твоего на лоно твое, и дал тебе дом Израилев и Иудин, и, если этого [для тебя] мало, прибавил бы тебе еще больше;
\vs 2Sa 12:9 зачем же ты пренебрег слово Господа, сделав злое пред очами Его? Урию Хеттеянина ты поразил мечом; жену его взял себе в жену, а его ты убил мечом Аммонитян;
\vs 2Sa 12:10 итак не отступит меч от дома твоего во веки, за то, что ты пренебрег Меня и взял жену Урии Хеттеянина, чтоб она была тебе женою.
\vs 2Sa 12:11 Так говорит Господь: вот, Я воздвигну на тебя зло из дома твоего, и возьму жен твоих пред глазами твоими, и отдам ближнему твоему, и будет он спать с женами твоими пред этим солнцем;
\vs 2Sa 12:12 ты сделал тайно, а Я сделаю это пред всем Израилем и пред солнцем.
\vs 2Sa 12:13 И сказал Давид Нафану: согрешил я пред Господом. И сказал Нафан Давиду: и Господь снял \bibemph{с тебя} грех твой; ты не умрешь;
\vs 2Sa 12:14 но как ты этим делом подал повод врагам Господа хулить Его, то умрет родившийся у тебя сын.
\vs 2Sa 12:15 И пошел Нафан в дом свой.\rsbpar И поразил Господь дитя, которое родила жена Урии Давиду, и оно заболело.
\vs 2Sa 12:16 И молился Давид Богу о младенце, и постился Давид, и, уединившись провел ночь, лежа на земле.
\vs 2Sa 12:17 И вошли к нему старейшины дома его, чтобы поднять его с земли; но он не хотел, и не ел с ними хлеба.
\vs 2Sa 12:18 На седьмой день дитя умерло, и слуги Давидовы боялись донести ему, что умер младенец; ибо, говорили они, когда дитя было еще живо, и мы уговаривали его, и он не слушал голоса нашего, как же мы скажем ему: <<умерло дитя>>? Он сделает что-нибудь худое.
\vs 2Sa 12:19 И увидел Давид, что слуги его перешептываются между собою, и понял Давид, что дитя умерло, и спросил Давид слуг своих: умерло дитя? И сказали: умерло.
\vs 2Sa 12:20 Тогда Давид встал с земли и умылся, и помазался, и переменил одежды свои, и пошел в дом Господень, и молился. Возвратившись домой, потребовал, чтобы подали ему хлеба, и он ел.
\vs 2Sa 12:21 И сказали ему слуги его: что значит, что ты так поступаешь: когда дитя было еще живо, ты постился и плакал [и не спал]; а когда дитя умерло, ты встал и ел хлеб [и пил]?
\vs 2Sa 12:22 И сказал Давид: доколе дитя было живо, я постился и плакал, ибо думал: кто знает, не помилует ли меня Господь, и дитя останется живо?
\vs 2Sa 12:23 А теперь оно умерло; зачем же мне поститься? Разве я могу возвратить его? Я пойду к нему, а оно не возвратится ко мне.
\vs 2Sa 12:24 И утешил Давид Вирсавию, жену свою, и вошел к ней и спал с нею; и она [зачала и] родила сына, и нарекла ему имя: Соломон. И Господь возлюбил его
\vs 2Sa 12:25 и послал пророка Нафана, и он нарек ему имя: Иедидиа\fns{Возлюбленный Богом.} по слову Господа.
\rsbpar\vs 2Sa 12:26 Иоав воевал против Раввы Аммонитской и взял \bibemph{почти} царственный город.
\vs 2Sa 12:27 И послал Иоав к Давиду сказать ему: я нападал на Равву и овладел водою города;
\vs 2Sa 12:28 теперь собери остальной народ и подступи к городу и возьми его; ибо, если я возьму его, то мое имя будет наречено ему.
\vs 2Sa 12:29 И собрал Давид весь народ и пошел к Равве, и воевал против нее и взял ее.
\vs 2Sa 12:30 И взял Давид венец царя их с головы его,~--- а в нем было золота талант и драгоценный камень,~--- и возложил его Давид на свою голову, и добычи из города вынес очень много.
\vs 2Sa 12:31 А народ, бывший в нем, он вывел и положил их под пилы, под железные молотилки, под железные топоры, и бросил их в обжигательные печи. Так он поступил со всеми городами Аммонитскими. И возвратился после того Давид и весь народ в Иерусалим.
\vs 2Sa 13:1 И было после того: у Авессалома, сына Давидова, \bibemph{была} сестра красивая, по имени Фамарь, и полюбил ее Амнон, сын Давида.
\vs 2Sa 13:2 И скорбел Амнон до того, что заболел из-за Фамари, сестры своей; ибо она была девица, и Амнону казалось трудным что-нибудь сделать с нею.
\vs 2Sa 13:3 Но у Амнона был друг, по имени Ионадав, сын Самая, брата Давидова; и Ионадав был человек очень хитрый.
\vs 2Sa 13:4 И он сказал ему: отчего ты так худеешь с каждым днем, сын царев,~--- не откроешь ли мне? И сказал ему Амнон: Фамарь, сестру Авессалома, брата моего, люблю я.
\vs 2Sa 13:5 И сказал ему Ионадав: ложись в постель твою, и притворись больным; и когда отец твой придет навестить тебя: скажи ему: пусть придет Фамарь, сестра моя, и подкрепит меня пищею, приготовив кушанье при моих глазах, чтоб я видел, и ел из рук ее.
\vs 2Sa 13:6 И лег Амнон и притворился больным, и пришел царь навестить его; и сказал Амнон царю: пусть придет Фамарь, сестра моя, и испечет при моих глазах лепешку, или две, и я поем из рук ее.
\vs 2Sa 13:7 И послал Давид к Фамари в дом сказать: пойди в дом Амнона, брата твоего, и приготовь ему кушанье.
\vs 2Sa 13:8 И пошла она в дом брата своего Амнона; а он лежит. И взяла она муки и замесила, и изготовила пред глазами его и испекла лепешки,
\vs 2Sa 13:9 и взяла сковороду и выложила пред ним; но он не хотел есть. И сказал Амнон: пусть все выйдут от меня. И вышли от него все люди,
\vs 2Sa 13:10 и сказал Амнон Фамари: отнеси кушанье во внутреннюю комнату, и я поем из рук твоих. И взяла Фамарь лепешки, которые приготовила, и отнесла Амнону, брату своему, во внутреннюю комнату.
\vs 2Sa 13:11 И когда она поставила пред ним, чтоб он ел, то он схватил ее, и сказал ей: иди, ложись со мною, сестра моя.
\vs 2Sa 13:12 Но она сказала: нет, брат мой, не бесчести меня, ибо не делается так в Израиле; не делай этого безумия.
\vs 2Sa 13:13 И я, куда пойду я с моим бесчестием? И ты, ты будешь одним из безумных в Израиле. Ты поговори с царем; он не откажет отдать меня тебе.
\vs 2Sa 13:14 Но он не хотел слушать слов ее, и преодолел ее, и изнасиловал ее, и лежал с нею.
\vs 2Sa 13:15 Потом возненавидел ее Амнон величайшею ненавистью, так что ненависть, какою он возненавидел ее, была сильнее любви, какую имел к ней; и сказал ей Амнон: встань, уйди.
\vs 2Sa 13:16 И [Фамарь] сказала ему: нет, [брат]; прогнать меня~--- это зло больше первого, которое ты сделал со мною. Но он не хотел слушать ее.
\vs 2Sa 13:17 И позвал отрока своего, который служил ему, и сказал: прогони эту от меня вон и запри дверь за нею.
\vs 2Sa 13:18 На ней была разноцветная одежда, ибо такие верхние одежды носили царские дочери-девицы. И вывел ее слуга вон и запер за нею дверь.
\vs 2Sa 13:19 И посыпала Фамарь пеплом голову свою, и разодрала разноцветную одежду, которую имела на себе, и положила руки свои на голову свою, и так шла и вопила.
\vs 2Sa 13:20 И сказал ей Авессалом, брат ее: не Амнон ли, брат твой, был с тобою?~--- но теперь молчи, сестра моя; он~--- брат твой; не сокрушайся сердцем твоим об этом деле. И жила Фамарь в одиночестве в доме Авессалома, брата своего.
\vs 2Sa 13:21 И услышал царь Давид обо всем этом, и сильно разгневался, [но не опечалил духа Амнона, сына своего, ибо любил его, потому что он был первенец его].
\vs 2Sa 13:22 Авессалом же не говорил с Амноном ни худого, ни хорошего; ибо возненавидел Авессалом Амнона за то, что он обесчестил Фамарь, сестру его.
\rsbpar\vs 2Sa 13:23 Чрез два года было стрижение \bibemph{овец} у Авессалома в Ваал-Гацоре, что у Ефрема, и позвал Авессалом всех сыновей царских.
\vs 2Sa 13:24 И пришел Авессалом к царю и сказал: вот, ныне стрижение \bibemph{овец} у раба твоего; пусть пойдет царь и слуги его с рабом твоим.
\vs 2Sa 13:25 Но царь сказал Авессалому: нет, сын мой, мы не пойдем все, чтобы не быть тебе в тягость. И сильно упрашивал его \bibemph{Авессалом}; но он не захотел идти, и благословил его.
\vs 2Sa 13:26 И сказал ему Авессалом: по крайней мере пусть пойдет с нами Амнон, брат мой. И сказал ему царь: зачем ему идти с тобою?
\vs 2Sa 13:27 Но Авессалом упросил его, и он отпустил с ним Амнона и всех царских сыновей; [и сделал Авессалом пир, как царь делает пир].
\vs 2Sa 13:28 Авессалом же приказал отрокам своим, сказав: смотрите, как только развеселится сердце Амнона от вина, и я скажу вам: <<поразите Амнона>>, тогда убейте его, не бойтесь; это я приказываю вам, будьте смелы и мужественны.
\vs 2Sa 13:29 И поступили отроки Авессалома с Амноном, как приказал Авессалом. Тогда встали все царские сыновья, сели каждый на мула своего и убежали.
\vs 2Sa 13:30 Когда они были еще на пути, дошел слух до Давида, что Авессалом умертвил всех царских сыновей, и не осталось ни одного из них.
\vs 2Sa 13:31 И встал царь, и разодрал одежды свои, и повергся на землю, и все слуги его, предстоящие ему, разодрали одежды свои.
\vs 2Sa 13:32 Но Ионадав, сын Самая, брата Давидова, сказал: пусть не думает господин мой [царь], что всех отроков, царских сыновей, умертвили; один только Амнон умер, ибо у Авессалома был этот замысел с того дня, как \bibemph{Амнон} обесчестил сестру его;
\vs 2Sa 13:33 итак пусть господин мой, царь, не тревожится мыслью о том, будто умерли все царские сыновья: умер один только Амнон.
\vs 2Sa 13:34 И убежал Авессалом. И поднял отрок, стоявший на страже, глаза свои, и увидел: вот, много народа идет по дороге по скату горы. [И пришел страж, и возвестил царю, и сказал: я видел людей на дороге Оронской на скате горы.]
\vs 2Sa 13:35 Тогда Ионадав сказал царю: это идут царские сыновья; как говорил раб твой, так и есть.
\vs 2Sa 13:36 И едва только сказал он это, вот пришли царские сыновья, и подняли вопль и плакали. И сам царь и все слуги его плакали очень великим плачем.
\vs 2Sa 13:37 Авессалом же убежал и пошел к Фалмаю, сыну Емиуда, царю Гессурскому [в землю Хамаахадскую]. И плакал [царь] Давид о сыне своем во все дни.
\vs 2Sa 13:38 Авессалом убежал и пришел в Гессур и пробыл там три года.
\vs 2Sa 13:39 И не стал царь Давид преследовать Авессалома; ибо утешился о смерти Амнона.
\vs 2Sa 14:1 И заметил Иоав, сын Саруи, что сердце царя обратилось к Авессалому.
\vs 2Sa 14:2 И послал Иоав в Фекою, и взял оттуда умную женщину и сказал ей: притворись плачущею и надень печальную одежду, и не мажься елеем, и представься женщиною, много дней плакавшею по умершем;
\vs 2Sa 14:3 и пойди к царю и скажи ему так и так. И вложил Иоав в уста ее, что сказать.
\vs 2Sa 14:4 И вошла женщина Фекоитянка к царю и пала лицем своим на землю, и поклонилась и сказала: помоги, царь, [помоги]!
\vs 2Sa 14:5 И сказал ей царь: что тебе? И сказала она: я [давно] вдова, муж мой умер;
\vs 2Sa 14:6 и у рабы твоей \bibemph{было} два сына; они поссорились в поле, и некому было разнять их, и поразил один другого и умертвил его.
\vs 2Sa 14:7 И вот, восстало все родство на рабу твою, и говорят: <<отдай убийцу брата своего; мы убьем его за душу брата его, которую он погубил, и истребим даже наследника>>. И так они погасят остальную искру мою, чтобы не оставить мужу моему имени и потомства на лице земли.
\vs 2Sa 14:8 И сказал царь женщине: иди спокойно домой, я дам приказание о тебе.
\vs 2Sa 14:9 Но женщина Фекоитянка сказала царю: на мне, господин мой царь, да будет вина и на доме отца моего, царь же и престол его неповинен.
\vs 2Sa 14:10 И сказал царь: того, кто будет против тебя, приведи ко мне, и он более не тронет тебя.
\vs 2Sa 14:11 Она сказала: помяни, царь, Господа Бога твоего, чтобы не умножились мстители за кровь и не погубили сына моего. И сказал \bibemph{царь}: жив Господь! не падет и волос сына твоего на землю.
\vs 2Sa 14:12 И сказала женщина: позволь рабе твоей сказать \bibemph{еще} слово господину моему царю.
\vs 2Sa 14:13 Он сказал: говори. И сказала женщина: почему ты так мыслишь против народа Божия? Царь, произнеся это слово, обвинил себя самого, потому что не возвращает изгнанника своего.
\vs 2Sa 14:14 Мы умрем и \bibemph{будем} как вода, вылитая на землю, которую нельзя собрать; но Бог не желает погубить душу и помышляет, как бы не отвергнуть от Себя и отверженного.
\vs 2Sa 14:15 И теперь я пришла сказать царю, господину моему, эти слова, потому что народ пугает меня; и раба твоя сказала: поговорю я с царем, не сделает ли он по слову рабы своей;
\vs 2Sa 14:16 верно царь выслушает и избавит рабу свою от руки людей, \bibemph{хотящих} истребить меня вместе с сыном моим из наследия Божия.
\vs 2Sa 14:17 И сказала раба твоя: да будет слово господина моего царя в утешение мне, ибо господин мой царь, как Ангел Божий, и может выслушать и доброе и худое. И Господь Бог твой будет с тобою.
\vs 2Sa 14:18 И отвечал царь и сказал женщине: не скрой от меня, о чем я спрошу тебя. И сказала женщина: говори, господин мой царь.
\vs 2Sa 14:19 И сказал царь: не рука ли Иоава во всем этом с тобою? И отвечала женщина и сказала: да живет душа твоя, господин мой царь; ни направо, ни налево нельзя уклониться от того, что сказал господин мой, царь; точно, раб твой Иоав приказал мне, и он вложил в уста рабы твоей все эти слова;
\vs 2Sa 14:20 чтобы притчею дать делу такой вид, раб твой Иоав научил меня; но господин мой [царь] мудр, как мудр Ангел Божий, чтобы знать все, что на земле.
\vs 2Sa 14:21 И сказал царь Иоаву: вот, я сделал [по слову твоему]; пойди же, возврати отрока Авессалома.
\vs 2Sa 14:22 Тогда Иоав пал лицем на землю и поклонился, и благословил царя и сказал: теперь знает раб твой, что обрел благоволение пред очами твоими, господин мой царь, так как царь сделал по слову раба своего.
\vs 2Sa 14:23 И встал Иоав, и пошел в Гессур, и привел Авессалома в Иерусалим.
\vs 2Sa 14:24 И сказал царь: пусть он возвратится в дом свой, а лица моего не видит. И пошел Авессалом в свой дом, а лица царского не видал.
\rsbpar\vs 2Sa 14:25 Не было во всем Израиле мужчины столь красивого, как Авессалом, и столько хвалимого, как он; от подошвы ног до верха головы его не было у него недостатка.
\vs 2Sa 14:26 Когда он стриг голову свою,~--- а он стриг ее каждый год, потому что она отягощала его,~--- то волоса с головы его весили двести сиклей по весу царскому.
\vs 2Sa 14:27 И родились у Авессалома три сына и одна дочь, по имени Фамарь; она была женщина красивая [и сделалась женою Ровоама, сына Соломонова, и родила ему Авию].
\vs 2Sa 14:28 И оставался Авессалом в Иерусалиме два года, а лица царского не видал.
\vs 2Sa 14:29 И послал Авессалом за Иоавом, чтобы послать его к царю, но тот не захотел прийти к нему. Послал и в другой раз; но тот не захотел прийти.
\vs 2Sa 14:30 И сказал [Авессалом] слугам своим: видите участок поля Иоава подле моего, и у него там ячмень; пойдите, выжгите его огнем. И выжгли слуги Авессалома тот участок поля огнем. [И пришли слуги Иоава к нему, разодрав одежды свои, и сказали: слуги Авессалома выжгли участок твой огнем.]
\vs 2Sa 14:31 И встал Иоав, и пришел к Авессалому в дом, и сказал ему: зачем слуги твои выжгли мой участок огнем?
\vs 2Sa 14:32 И сказал Авессалом Иоаву: вот, я посылал за тобою, говоря: приди сюда, и я пошлю тебя к царю сказать: зачем я пришел из Гессура? Лучше было бы мне оставаться там. Я хочу увидеть лице царя. Если же я виноват, то убей меня.
\vs 2Sa 14:33 И пошел Иоав к царю и пересказал ему \bibemph{это}. И позвал \bibemph{царь} Авессалома; он пришел к царю, [поклонился ему] и пал лицем своим на землю пред царем; и поцеловал царь Авессалома.
\vs 2Sa 15:1 После сего Авессалом завел у себя колесницы и лошадей и пятьдесят скороходов.
\vs 2Sa 15:2 И вставал Авессалом рано утром, и становился при дороге у ворот, и когда кто-нибудь, имея тяжбу, шел к царю на суд, то Авессалом подзывал его к себе и спрашивал: из какого города ты? И когда тот отвечал: из такого-то колена Израилева раб твой,
\vs 2Sa 15:3 тогда говорил ему Авессалом: вот, дело твое доброе и справедливое, но у царя некому выслушать тебя.
\vs 2Sa 15:4 И говорил Авессалом: о, если бы меня поставили судьею в этой земле! ко мне приходил бы всякий, кто имеет спор и тяжбу, и я судил бы его по правде.
\vs 2Sa 15:5 И когда подходил кто-нибудь поклониться ему, то он простирал руку свою и обнимал его и целовал его.
\vs 2Sa 15:6 Так поступал Авессалом со всяким Израильтянином, приходившим на суд к царю, и вкрадывался Авессалом в сердце Израильтян.
\rsbpar\vs 2Sa 15:7 По прошествии сорока лет \bibemph{царствования Давида}, Авессалом сказал царю: пойду я и исполню обет мой, который я дал Господу, в Хевроне;
\vs 2Sa 15:8 ибо я, раб твой, живя в Гессуре в Сирии, дал обет: если Господь возвратит меня в Иерусалим, то я принесу жертву Господу.
\vs 2Sa 15:9 И сказал ему царь: иди с миром. И встал он и пошел в Хеврон.
\vs 2Sa 15:10 И разослал Авессалом лазутчиков во все колена Израилевы, сказав: когда вы услышите звук трубы, то говорите: Авессалом воцарился в Хевроне.
\vs 2Sa 15:11 С Авессаломом пошли из Иерусалима двести человек, которые были приглашены им, и пошли по простоте своей, не зная, в чем дело.
\vs 2Sa 15:12 Во время жертвоприношения Авессалом послал и призвал Ахитофела Гилонянина, советника Давидова, из его города Гило. И составился сильный заговор, и народ стекался и умножался около Авессалома.
\vs 2Sa 15:13 И пришел вестник к Давиду и сказал: сердце Израильтян уклонилось на сторону Авессалома.
\vs 2Sa 15:14 И сказал Давид всем слугам своим, которые были при нем в Иерусалиме: встаньте, убежим, ибо не будет нам спасения от Авессалома; спешите, чтобы нам уйти, чтоб он не застиг и не захватил нас, и не навел на нас беды и не истребил города мечом.
\vs 2Sa 15:15 И сказали слуги царские царю: во всем, что угодно господину нашему царю, мы~--- рабы твои.
\vs 2Sa 15:16 И вышел царь и весь дом его за ним пешком. Оставил же царь десять жен, наложниц [своих], для хранения дома.
\vs 2Sa 15:17 И вышел царь и весь народ пешие, и остановились у Беф-Мерхата.
\vs 2Sa 15:18 И все слуги его шли по сторонам его, и все Хелефеи, и все Фелефеи, и все Гефяне до шестисот человек, пришедшие вместе с ним из Гефа, шли впереди царя.
\vs 2Sa 15:19 И сказал царь Еффею Гефянину: зачем и ты идешь с нами? Возвратись и оставайся с тем царем; ибо ты~--- чужеземец и пришел сюда из своего места;
\vs 2Sa 15:20 вчера ты пришел, а сегодня я заставлю тебя идти с нами? Я иду, куда случится; возвратись и возврати братьев своих с собою, [да сотворит Господь] милость и истину [с тобою]!
\vs 2Sa 15:21 И отвечал Еффей царю и сказал: жив Господь, и да живет господин мой царь: где бы ни был господин мой царь, в жизни ли, в смерти ли, там будет и раб твой.
\vs 2Sa 15:22 И сказал Давид Еффею: итак иди и ходи со мною. И пошел Еффей Гефянин и все люди его и все дети, бывшие с ним.
\vs 2Sa 15:23 И плакала вся земля громким голосом. И весь народ переходил, и царь перешел поток Кедрон; и пошел весь народ [и царь] по дороге к пустыне.
\vs 2Sa 15:24 Вот и Садок [священник], и все левиты с ним несли ковчег завета Божия из Вефары и поставили ковчег Божий; Авиафар же стоял на возвышении, доколе весь народ не вышел из города.
\vs 2Sa 15:25 И сказал царь Садоку: возврати ковчег Божий в город [и пусть он стоит на своем месте]. Если я обрету милость пред очами Господа, то Он возвратит меня и даст мне видеть его и жилище его.
\vs 2Sa 15:26 А если Он скажет так: <<нет Моего благоволения к тебе>>, то вот я; пусть творит со мною, что Ему благоугодно.
\vs 2Sa 15:27 И сказал царь Садоку священнику: видишь ли,~--- возвратись в город с миром, и Ахимаас, сын твой, и Ионафан, сын Авиафара, оба сына ваши с вами;
\vs 2Sa 15:28 видите ли, я помедлю на равнине в пустыне, доколе не придет известие от вас ко мне.
\vs 2Sa 15:29 И возвратили Садок и Авиафар ковчег Божий в Иерусалим, и остались там.
\vs 2Sa 15:30 А Давид пошел на гору Елеонскую, шел и плакал; голова у него была покрыта; он шел босой, и все люди, бывшие с ним, покрыли каждый голову свою, шли и плакали.
\vs 2Sa 15:31 Донесли Давиду и сказали: и Ахитофел в числе заговорщиков с Авессаломом. И сказал Давид: Господи [Боже мой!] разрушь совет Ахитофела.
\vs 2Sa 15:32 Когда Давид взошел на вершину горы, где он поклонялся Богу, вот навстречу ему идет Хусий Архитянин, друг Давидов; одежда на нем была разодрана, и прах на голове его.
\vs 2Sa 15:33 И сказал ему Давид: если ты пойдешь со мною, то будешь мне в тягость;
\vs 2Sa 15:34 но если возвратишься в город и скажешь Авессалому: <<царь, [прошли мимо братья твои, и царь отец твой прошел, и ныне] я раб твой; [оставь меня в живых;] доселе я был рабом отца твоего, а теперь я~--- твой раб>>: то ты расстроишь для меня совет Ахитофела.
\vs 2Sa 15:35 Вот, там с тобою Садок и Авиафар священники, и всякое слово, какое услышишь из дома царя, пересказывай Садоку и Авиафару священникам.
\vs 2Sa 15:36 Там с ними и два сына их, Ахимаас, сын Садока, и Ионафан, сын Авиафара; чрез них посылайте ко мне всякое известие, какое услышите.
\vs 2Sa 15:37 И пришел Хусий, друг Давидов, в город; Авессалом же вступал тогда в Иерусалим.
\vs 2Sa 16:1 Когда Давид немного сошел с вершины горы, вот встречается ему Сива, слуга Мемфивосфея, с парою навьюченных ослов, и на них двести хлебов, сто связок изюму, сто связок смокв и мех с вином.
\vs 2Sa 16:2 И сказал царь Сиве: для чего это у тебя? И отвечал Сива: ослы для дома царского, для езды, а хлеб и плоды для пищи отрокам, а вино для питья ослабевшим в пустыне.
\vs 2Sa 16:3 И сказал царь: где сын господина твоего? И отвечал Сива царю: вот, он остался в Иерусалиме и говорит: теперь-то дом Израилев возвратит мне царство отца моего.
\vs 2Sa 16:4 И сказал царь Сиве: вот тебе все, что у Мемфивосфея. И отвечал Сива, поклонившись: да обрету милость в глазах господина моего царя!
\vs 2Sa 16:5 Когда дошел царь Давид до Бахурима, вот вышел оттуда человек из рода дома Саулова, по имени Семей, сын Геры; он шел и злословил,
\vs 2Sa 16:6 и бросал камнями на Давида и на всех рабов царя Давида; все же люди и все храбрые были по правую и по левую сторону [царя].
\vs 2Sa 16:7 Так говорил Семей, злословя его: уходи, уходи, убийца и беззаконник!
\vs 2Sa 16:8 Господь обратил на тебя всю кровь дома Саулова, вместо которого ты воцарился, и предал Господь царство в руки Авессалома, сына твоего; и вот, ты в беде, ибо ты~--- кровопийца.
\vs 2Sa 16:9 И сказал Авесса, сын Саруин, царю: зачем злословит этот мертвый пес господина моего царя? пойду я и сниму с него голову.
\vs 2Sa 16:10 И сказал царь: что мне и вам, сыны Саруины? [оставьте его,] пусть он злословит, ибо Господь повелел ему злословить Давида. Кто же может сказать: зачем ты так делаешь?
\vs 2Sa 16:11 И сказал Давид Авессе и всем слугам своим: вот, если мой сын, который вышел из чресл моих, ищет души моей, тем больше сын Вениамитянина; оставьте его, пусть злословит, ибо Господь повелел ему;
\vs 2Sa 16:12 может быть, Господь призрит на уничижение мое, и воздаст мне Господь благостью за теперешнее его злословие.
\vs 2Sa 16:13 И шел Давид и люди его \bibemph{своим} путем, а Семей шел по окраине горы, со стороны его, шел и злословил, и бросал камнями на сторону его и пылью.
\vs 2Sa 16:14 И пришел царь и весь народ, бывший с ним, утомленный, и отдыхал там.
\rsbpar\vs 2Sa 16:15 Авессалом же и весь народ Израильский пришли в Иерусалим, и Ахитофел с ним.
\vs 2Sa 16:16 Когда Хусий Архитянин, друг Давидов, пришел к Авессалому, то сказал Хусий Авессалому: да живет царь, да живет царь!
\vs 2Sa 16:17 И сказал Авессалом Хусию: таково-то усердие твое к твоему другу! отчего ты не пошел с другом твоим?
\vs 2Sa 16:18 И сказал Хусий Авессалому: нет, [я пойду вслед того,] кого избрал Господь и этот народ и весь Израиль, с тем и я, и с ним останусь.
\vs 2Sa 16:19 И притом кому я буду служить? Не сыну ли его? Как служил я отцу твоему, так буду служить и тебе.
\vs 2Sa 16:20 И сказал Авессалом Ахитофелу: дайте совет, что нам делать.
\vs 2Sa 16:21 И сказал Ахитофел Авессалому: войди к наложницам отца твоего, которых он оставил охранять дом свой; и услышат все Израильтяне, что ты сделался ненавистным для отца твоего, и укрепятся руки всех, которые с тобою.
\vs 2Sa 16:22 И поставили для Авессалома палатку на кровле, и вошел Авессалом к наложницам отца своего пред глазами всего Израиля.
\vs 2Sa 16:23 Советы же Ахитофела, которые он давал, в то время \bibemph{считались}, как если бы кто спрашивал наставления у Бога. Таков был всякий совет Ахитофела как для Давида, так и для Авессалома.
\vs 2Sa 17:1 И сказал Ахитофел Авессалому: выберу я двенадцать тысяч человек и встану и пойду в погоню за Давидом в эту ночь;
\vs 2Sa 17:2 и нападу на него, когда он будет утомлен и с опущенными руками, и приведу его в страх; и все люди, которые с ним, разбегутся; и я убью одного царя
\vs 2Sa 17:3 и всех людей обращу к тебе; и когда не будет одного, душу которого ты ищешь, тогда весь народ будет в мире.
\vs 2Sa 17:4 И понравилось это слово Авессалому и всем старейшинам Израилевым.
\vs 2Sa 17:5 И сказал Авессалом: позовите Хусия Архитянина; послушаем, что он скажет.
\vs 2Sa 17:6 И пришел Хусий к Авессалому, и сказал ему Авессалом, говоря: вот что говорит Ахитофел; сделать ли по его словам? а если нет, то говори ты.
\vs 2Sa 17:7 И сказал Хусий Авессалому: нехорош на этот раз совет, который дал Ахитофел.
\vs 2Sa 17:8 И продолжал Хусий: ты знаешь твоего отца и людей его; они храбры и сильно раздражены, как медведица в поле, у которой отняли детей, [и как вепрь свирепый на поле,] и отец твой~--- человек воинственный; он не остановится ночевать с народом.
\vs 2Sa 17:9 Вот, теперь он скрывается в какой-нибудь пещере, или в другом месте, и если кто падет при первом нападении на них, и услышат и скажут: <<было поражение людей, последовавших за Авессаломом>>,
\vs 2Sa 17:10 тогда и самый храбрый, у которого сердце, как сердце львиное, упадет духом; ибо всему Израилю известно, как храбр отец твой и мужественны те, которые с ним.
\vs 2Sa 17:11 Посему я советую: пусть соберется к тебе весь Израиль, от Дана до Вирсавии, во множестве, как песок при море, и ты сам пойдешь посреди его;
\vs 2Sa 17:12 и тогда мы пойдем против него, в каком бы месте он ни находился, и нападем на него, как падает роса на землю; и не останется у него ни одного человека из всех, которые с ним;
\vs 2Sa 17:13 а если он войдет в какой-либо город, то весь Израиль принесет к тому городу веревки, и мы стащим его в реку, так что не останется ни одного камешка.
\vs 2Sa 17:14 И сказал Авессалом и весь Израиль: совет Хусия Архитянина лучше совета Ахитофелова. Так Господь судил разрушить лучший совет Ахитофела, чтобы навести Господу бедствие на Авессалома.
\vs 2Sa 17:15 И сказал Хусий Садоку и Авиафару священникам: так и так советовал Ахитофел Авессалому и старейшинам Израилевым, а так и так посоветовал я.
\vs 2Sa 17:16 И теперь пошлите поскорее и скажите Давиду так: не оставайся в эту ночь на равнине в пустыне, но поскорее перейди, чтобы не погибнуть царю и всем людям, которые с ним.
\vs 2Sa 17:17 Ионафан и Ахимаас стояли у источника Рогель. И пошла служанка и рассказала им, а они пошли и известили царя Давида; ибо они не могли показаться в городе.
\vs 2Sa 17:18 И увидел их отрок и донес Авессалому; но они оба скоро ушли и пришли в Бахурим, в дом одного человека, у которого на дворе был колодезь, и спустились туда.
\vs 2Sa 17:19 А женщина взяла и растянула над устьем колодезя покрывало и насыпала на него крупы, так что не было ничего заметно.
\vs 2Sa 17:20 И пришли рабы Авессалома к женщине в дом, и сказали: где Ахимаас и Ионафан? И сказала им женщина: они перешли вброд реку. И искали они, и не нашли, и возвратились в Иерусалим.
\vs 2Sa 17:21 Когда они ушли, те вышли из колодезя, пошли и известили царя Давида и сказали Давиду: встаньте и поскорее перейдите воду; ибо так и так советовал о вас Ахитофел.
\vs 2Sa 17:22 И встал Давид и все люди, бывшие с ним, и перешли Иордан; к рассвету не осталось ни одного, который не перешел бы Иордана.
\vs 2Sa 17:23 И увидел Ахитофел, что не исполнен совет его, и оседлал осла, и собрался, и пошел в дом свой, в город свой, и сделал завещание дому своему, и удавился, и умер, и был погребен в гробе отца своего.
\vs 2Sa 17:24 И пришел Давид в Маханаим, а Авессалом перешел Иордан, сам и весь Израиль с ним.
\vs 2Sa 17:25 Авессалом поставил Амессая, вместо Иоава, над войском. Амессай был сын одного человека, по имени Иефера из Изрееля, который вошел к Авигее, дочери Нааса, сестре Саруи, матери Иоава.
\vs 2Sa 17:26 И Израиль с Авессаломом расположился станом в земле Галаадской.
\vs 2Sa 17:27 Когда Давид пришел в Маханаим, то Сови, сын Нааса, из Раввы Аммонитской, и Махир, сын Аммиила, из Лодавара, и Верзеллий Галаадитянин из Роглима,
\vs 2Sa 17:28 принесли [десять приготовленных] постелей, [десять] блюд и глиняных сосудов, и пшеницы, и ячменя, и муки, и пшена, и бобов, и чечевицы, и жареных зерен,
\vs 2Sa 17:29 и меду, и масла, и овец, и сыра коровьего, принесли Давиду и людям, бывшим с ним, в пищу; ибо говорили они: народ голоден и утомлен и терпел жажду в пустыне.
\vs 2Sa 18:1 И осмотрел Давид людей, бывших с ним, и поставил над ними тысяченачальников и сотников.
\vs 2Sa 18:2 И отправил Давид людей~--- третью часть под предводительством Иоава, третью часть под предводительством Авессы, сына Саруина, брата Иоава, третью часть под предводительством Еффея Гефянина. И сказал царь людям: я сам пойду с вами.
\vs 2Sa 18:3 Но люди отвечали ему: не ходи; ибо, если мы и побежим, то не обратят внимания на это; если и умрет половина из нас, также не обратят внимания; а ты один то же, что нас десять тысяч; итак для нас лучше, чтобы ты помогал нам из города.
\vs 2Sa 18:4 И сказал им царь: что угодно в глазах ваших, то и сделаю. И стал царь у ворот, и весь народ выходил по сотням и по тысячам.
\vs 2Sa 18:5 И приказал царь Иоаву и Авессе и Еффею, говоря: сберегите мне отрока Авессалома. И все люди слышали, как приказывал царь всем начальникам об Авессаломе.
\vs 2Sa 18:6 И вышли люди в поле навстречу Израильтянам, и было сражение в лесу Ефремовом.
\vs 2Sa 18:7 И был поражен народ Израильский рабами Давида; было там поражение великое в тот день,~--- поражены двадцать тысяч [человек].
\vs 2Sa 18:8 Сражение распространилось по всей той стране, и лес погубил народа больше, чем сколько истребил меч, в тот день.
\vs 2Sa 18:9 И встретился Авессалом с рабами Давидовыми; он был на муле. Когда мул вбежал с ним под ветви большого дуба, то \bibemph{Авессалом} запутался волосами своими в ветвях дуба и повис между небом и землею, а мул, бывший под ним, убежал.
\vs 2Sa 18:10 И увидел это некто и донес Иоаву, говоря: вот, я видел Авессалома висящим на дубе.
\vs 2Sa 18:11 И сказал Иоав человеку, донесшему об этом: вот, ты видел; зачем же ты не поверг его там на землю? я дал бы тебе десять сиклей серебра и один пояс.
\vs 2Sa 18:12 И отвечал тот Иоаву: если бы положили на руки мои и тысячу сиклей серебра, и тогда я не поднял бы руки на царского сына; ибо вслух нас царь приказывал тебе и Авессе и Еффею, говоря: <<сберегите мне отрока Авессалома>>;
\vs 2Sa 18:13 и если бы я поступил иначе с опасностью жизни моей, то это не скрылось бы от царя, и ты же восстал бы против меня.
\vs 2Sa 18:14 Иоав сказал: нечего мне медлить с тобою. И взял в руки три стрелы и вонзил их в сердце Авессалома, который был еще жив на дубе.
\vs 2Sa 18:15 И окружили Авессалома десять отроков, оруженосцев Иоава, и поразили и умертвили его.
\vs 2Sa 18:16 И затрубил Иоав трубою, и возвратились люди из погони за Израилем, ибо Иоав щадил народ.
\vs 2Sa 18:17 И взяли Авессалома, и бросили его в лесу в глубокую яму, и наметали над ним огромную кучу камней. И все Израильтяне разбежались, каждый в шатер свой.
\vs 2Sa 18:18 Авессалом еще при жизни своей взял и поставил себе памятник в царской долине; ибо сказал он: нет у меня сына, чтобы сохранилась память имени моего. И назвал памятник своим именем. И называется он <<памятник Авессалома>> до сего дня.
\rsbpar\vs 2Sa 18:19 Ахимаас, сын Садоков, сказал Иоаву: побегу я, извещу царя, что Господь судом Своим избавил его от рук врагов его.
\vs 2Sa 18:20 Но Иоав сказал ему: не будешь ты сегодня добрым вестником; известишь в другой день, а не сегодня, ибо умер сын царя.
\vs 2Sa 18:21 И сказал Иоав Хусию: пойди, донеси царю, что видел ты. И поклонился Хусий Иоаву и побежал.
\vs 2Sa 18:22 Но Ахимаас, сын Садоков, настаивал и говорил Иоаву: что бы ни было, но и я побегу за Хусием. Иоав же отвечал: зачем бежать тебе, сын мой? не принесешь ты доброй вести.
\vs 2Sa 18:23 [И сказал Ахимаас:] пусть так, но я побегу. И сказал ему [Иоав]: беги. И побежал Ахимаас по прямой дороге и опередил Хусия.
\vs 2Sa 18:24 Давид тогда сидел между двумя воротами. И сторож взошел на кровлю ворот к стене и, подняв глаза, увидел: вот, бежит один человек.
\vs 2Sa 18:25 И закричал сторож и известил царя. И сказал царь: если один, то весть в устах его. А тот подходил все ближе и ближе.
\vs 2Sa 18:26 Сторож увидел и другого бегущего человека; и закричал сторож привратнику: вот, еще бежит один человек. Царь сказал: и это~--- вестник.
\vs 2Sa 18:27 Сторож сказал: я вижу походку первого, похожую на походку Ахимааса, сына Садокова. И сказал царь: это человек хороший и идет с хорошею вестью.
\vs 2Sa 18:28 И воскликнул Ахимаас и сказал царю: мир. И поклонился царю лицем своим до земли и сказал: благословен Господь Бог твой, предавший людей, которые подняли руки свои на господина моего царя!
\vs 2Sa 18:29 И сказал царь: благополучен ли отрок Авессалом? И сказал Ахимаас: я видел большое волнение, когда раб царев Иоав посылал раба твоего; но я не знаю, что [там] было.
\vs 2Sa 18:30 И сказал царь: отойди, стань здесь. Он отошел и стал.
\vs 2Sa 18:31 Вот, пришел и Хусий [вслед за ним]. И сказал Хусий [царю]: добрая весть господину моему царю! Господь явил тебе ныне правду в избавлении от руки всех восставших против тебя.
\vs 2Sa 18:32 И сказал царь Хусию: благополучен ли отрок Авессалом? И сказал Хусий: да будет с врагами господина моего царя и со всеми, злоумышляющими против тебя то же, что постигло отрока!
\vs 2Sa 18:33 И смутился царь, и пошел в горницу над воротами, и плакал, и когда шел, говорил так: сын мой Авессалом! сын мой, сын мой Авессалом! о, кто дал бы мне умереть вместо тебя, Авессалом, сын мой, сын мой!
\vs 2Sa 19:1 И сказали Иоаву: вот, царь плачет и рыдает об Авессаломе.
\vs 2Sa 19:2 И обратилась победа того дня в плач для всего народа; ибо народ услышал в тот день и говорил, что царь скорбит о своем сыне.
\vs 2Sa 19:3 И входил тогда народ в город украдкою, как крадутся люди стыдящиеся, которые во время сражения обратились в бегство.
\vs 2Sa 19:4 А царь закрыл лице свое и громко взывал: сын мой Авессалом! Авессалом, сын мой, сын мой!
\vs 2Sa 19:5 И пришел Иоав к царю в дом и сказал: ты в стыд привел сегодня всех слуг твоих, спасших ныне жизнь твою и жизнь сыновей и дочерей твоих, и жизнь жен и жизнь наложниц твоих;
\vs 2Sa 19:6 ты любишь ненавидящих тебя и ненавидишь любящих тебя, ибо ты показал сегодня, что ничто для тебя и вожди и слуги; сегодня я узнал, что если бы Авессалом остался жив, а мы все умерли, то тебе было бы приятнее;
\vs 2Sa 19:7 итак встань, выйди и поговори к сердцу рабов твоих, ибо клянусь Господом, что, если ты не выйдешь, в эту ночь не останется у тебя ни одного человека; и это будет для тебя хуже всех бедствий, какие находили на тебя от юности твоей доныне.
\vs 2Sa 19:8 И встал царь и сел у ворот, а всему народу возвестили, что царь сидит у ворот. И пришел весь народ пред лице царя [к воротам]; Израильтяне же разбежались по своим шатрам.
\vs 2Sa 19:9 И весь народ во всех коленах Израилевых спорил и говорил: царь [Давид] избавил нас от рук врагов наших и освободил нас от рук Филистимлян, а теперь сам бежал из земли сей [из царства своего] от Авессалома.
\vs 2Sa 19:10 Но Авессалом, которого мы помазали \bibemph{в царя} над нами, умер на войне; почему же теперь вы медлите возвратить царя? [И эти слова всего Израиля дошли до царя.]
\vs 2Sa 19:11 И царь Давид послал сказать священникам Садоку и Авиафару: скажите старейшинам Иудиным: зачем хотите вы быть последними, чтобы возвратить царя в дом его, тогда как слова всего Израиля дошли до царя в дом его?
\vs 2Sa 19:12 Вы братья мои, кости мои и плоть моя~--- вы; зачем хотите вы быть последними в возвращении царя в дом его?
\vs 2Sa 19:13 И Амессаю скажите: не кость ли моя и плоть моя~--- ты? Пусть то и то сделает со мною Бог и еще больше сделает, если ты не будешь военачальником при мне, вместо Иоава, навсегда!
\vs 2Sa 19:14 И склонил он сердце всех Иудеев, как одного человека; и послали они к царю \bibemph{сказать}: возвратись ты и все слуги твои.
\vs 2Sa 19:15 И возвратился царь, и пришел к Иордану, а Иудеи пришли в Галгал, чтобы встретить царя и перевезти царя чрез Иордан.
\vs 2Sa 19:16 И поспешил Семей, сын Геры, Вениамитянин из Бахурима, и пошел с Иудеями навстречу царю Давиду,
\vs 2Sa 19:17 и тысяча человек из Вениамитян с ним, и Сива, слуга дома Саулова, с пятнадцатью сыновьями своими и двадцатью рабами своими; и перешли они Иордан пред лицем царя [и приготовили для царя переправу чрез Иордан].
\vs 2Sa 19:18 Когда переправили судно, чтобы перевезти дом царя и послужить ему, тогда Семей, сын Геры, пал [на лице свое] пред царем, как только он перешел Иордан,
\vs 2Sa 19:19 и сказал царю: не поставь мне, господин мой, в преступление, и не помяни того, чем согрешил раб твой в тот день, когда господин мой царь выходил из Иерусалима, и не держи \bibemph{того}, царь, на сердце своем;
\vs 2Sa 19:20 ибо знает раб твой, что согрешил, и вот, ныне я пришел первый из всего дома Иосифова, чтобы выйти навстречу господину моему царю.
\vs 2Sa 19:21 И отвечал Авесса, сын Саруин, и сказал: неужели Семей не умрет за то, что злословил помазанника Господня?
\vs 2Sa 19:22 И сказал Давид: что мне и вам, сыны Саруины, что вы делаетесь ныне мне наветниками? Ныне ли умерщвлять кого-либо в Израиле? Не вижу ли я, что ныне я~--- царь над Израилем?
\vs 2Sa 19:23 И сказал царь Семею: ты не умрешь. И поклялся ему царь.
\vs 2Sa 19:24 И Мемфивосфей, сын [Ионафана, сына] Саулова, вышел навстречу царю. Он не омывал ног своих, [не обрезывал ногтей,] не заботился о бороде своей и не мыл одежд своих с того дня, как вышел царь, до дня, когда он возвратился с миром.
\vs 2Sa 19:25 Когда он вышел из Иерусалима навстречу царю, царь сказал ему: почему ты, Мемфивосфей, не пошел со мною?
\vs 2Sa 19:26 Тот отвечал: господин мой царь! слуга мой обманул меня; ибо я, раб твой, говорил: <<оседлаю себе осла и сяду на нем и поеду с царем>>, так как раб твой хром.
\vs 2Sa 19:27 А он оклеветал раба твоего пред господином моим царем. Но господин мой царь, как Ангел Божий; делай, что тебе угодно;
\vs 2Sa 19:28 хотя весь дом отца моего был повинен смерти пред господином моим царем, но ты посадил раба твоего между ядущими за столом твоим; какое же имею я право жаловаться еще пред царем?
\vs 2Sa 19:29 И сказал ему царь: к чему ты говоришь все это? я сказал, чтобы ты и Сива разделили \bibemph{между собою} поля.
\vs 2Sa 19:30 Но Мемфивосфей отвечал царю: пусть он возьмет даже все, после того как господин мой царь с миром возвратился в дом свой.
\vs 2Sa 19:31 И Верзеллий Галаадитянин пришел из Роглима и перешел с царем Иордан, чтобы проводить его за Иордан.
\vs 2Sa 19:32 Верзеллий же был очень стар, лет восьмидесяти. Он продовольствовал царя в пребывание его в Маханаиме, потому что был человек богатый.
\vs 2Sa 19:33 И сказал царь Верзеллию: иди со мною, и я буду продовольствовать тебя в Иерусалиме.
\vs 2Sa 19:34 Но Верзеллий отвечал царю: долго ли мне осталось жить, чтоб идти с царем в Иерусалим?
\vs 2Sa 19:35 Мне теперь восемьдесят лет; различу ли хорошее от худого? Узнает ли раб твой вкус в том, что буду есть, и в том, что буду пить? И буду ли в состоянии слышать голос певцов и певиц? Зачем же рабу твоему быть в тягость господину моему царю?
\vs 2Sa 19:36 Еще немного пройдет раб твой с царем за Иордан; за что же царю награждать меня такою милостью?
\vs 2Sa 19:37 Позволь рабу твоему возвратиться, чтобы умереть в своем городе, около гроба отца моего и матери моей. Но вот, раб твой [сын мой] Кимгам пусть пойдет с господином моим, царем, и поступи с ним, как тебе угодно.
\vs 2Sa 19:38 И сказал царь: пусть идет со мною Кимгам, и я сделаю для него, что тебе угодно; и все, чего бы ни пожелал ты от меня, я сделаю для тебя.
\vs 2Sa 19:39 И перешел весь народ Иордан, и царь \bibemph{также}. И поцеловал царь Верзеллия и благословил его, и он возвратился в место свое.
\vs 2Sa 19:40 И отправился царь в Галгал, отправился с ним и Кимгам; и весь народ Иудейский провожал царя, и половина народа Израильского.
\vs 2Sa 19:41 И вот, все Израильтяне пришли к царю и сказали царю: зачем братья наши, мужи Иудины, похитили тебя и проводили царя и дом его и всех людей Давида с ним через Иордан?
\vs 2Sa 19:42 И отвечали все мужи Иудины Израильтянам: затем, что царь ближний нам; и из-за чего сердиться вам на это? Разве мы что-нибудь съели у царя, или получили от него подарки? [Или от податей освободил он нас?]
\vs 2Sa 19:43 И отвечали Израильтяне мужам Иудиным и сказали: мы десять частей у царя, также и у Давида мы более, нежели вы; [мы первенец, а не вы;] зачем же вы унизили нас? Не нам ли принадлежало первое слово о том, чтобы возвратить нашего царя? Но слово мужей Иудиных было сильнее, нежели слово Израильтян.
\vs 2Sa 20:1 Там случайно находился один негодный человек, по имени Савей, сын Бихри, Вениамитянин; он затрубил трубою и сказал: нет нам части в Давиде, и нет нам доли в сыне Иессеевом; все по шатрам своим, Израильтяне!
\vs 2Sa 20:2 И отделились все Израильтяне от Давида \bibemph{и пошли} за Савеем, сыном Бихри; Иудеи же остались на стороне царя своего, от Иордана до Иерусалима.
\vs 2Sa 20:3 И пришел Давид в свой дом в Иерусалиме, и взял царь десять жен наложниц, которых он оставлял стеречь дом, и поместил их в особый дом под надзор, и содержал их, но не ходил к ним. И содержались они там до дня смерти своей, живя как вдовы.
\vs 2Sa 20:4 И сказал Давид Амессаю: созови ко мне Иудеев в течение трех дней и сам явись сюда.
\vs 2Sa 20:5 И пошел Амессай созвать Иудеев, но промедлил более назначенного ему времени.
\vs 2Sa 20:6 Тогда Давид сказал Авессе: теперь наделает нам зла Савей, сын Бихри, больше нежели Авессалом; возьми ты слуг господина твоего и преследуй его, чтобы он не нашел себе укрепленных городов и не скрылся от глаз наших.
\vs 2Sa 20:7 И вышли за ним люди Иоавовы, и Хелефеи и Фелефеи, и все храбрые пошли из Иерусалима преследовать Савея, сына Бихри.
\vs 2Sa 20:8 И когда они были близ большого камня, что у Гаваона, то встретился с ними Амессай. Иоав был одет в воинское одеяние свое и препоясан мечом, который висел при бедре в ножнах и который легко выходил из них и входил.
\vs 2Sa 20:9 И сказал Иоав Амессаю: здоров ли ты, брат мой? И взял Иоав правою рукою Амессая за бороду, чтобы поцеловать его.
\vs 2Sa 20:10 Амессай же не остерегся меча, бывшего в руке Иоава, и тот поразил его им в живот, так что выпали внутренности его на землю, и не повторил ему \bibemph{удара}, и он умер. Иоав и Авесса, брат его, погнались за Савеем, сыном Бихри.
\vs 2Sa 20:11 Один из отроков Иоавовых стоял над \bibemph{Амессаем} и говорил: тот, кто предан Иоаву и кто за Давида, \bibemph{пусть идет} за Иоавом!
\vs 2Sa 20:12 Амессай же [мертвый] лежал в крови среди дороги; и тот человек, увидев, что весь народ останавливается над ним, стащил Амессая с дороги в поле и набросил на него одежду, так как он видел, что всякий проходящий останавливался над ним.
\vs 2Sa 20:13 Но когда он был стащен с дороги, то весь народ Израильский пошел вслед за Иоавом преследовать Савея, сына Бихри.
\vs 2Sa 20:14 А он прошел чрез все колена Израильские до Авела-Беф-Мааха и чрез весь Берим; и [все жители городов] собирались и шли за ним.
\vs 2Sa 20:15 И пришли и осадили его в Авеле-Беф-Маахе; и насыпали вал пред городом и подступили к стене, и все люди, бывшие с Иоавом, старались разрушить стену.
\vs 2Sa 20:16 \bibemph{Тогда} одна умная женщина закричала со стены города: послушайте, послушайте, скажите Иоаву, чтоб он подошел сюда, и я поговорю с ним.
\vs 2Sa 20:17 И подошел к ней Иоав, и сказала женщина: ты ли Иоав? И сказал: я. Она сказала: послушай слов рабы твоей. И сказал он: слушаю.
\vs 2Sa 20:18 Она сказала: прежде говаривали: <<кто хочет спросить, спроси в Авеле>>; и так решали дело. [Остались ли такие, которые положили пребыть верными Израильтянами? Пусть спросят в Авеле: остались ли?]
\vs 2Sa 20:19 Я из мирных, верных \bibemph{городов} Израиля; а ты хочешь уничтожить город, и \bibemph{притом} мать [городов] в Израиле; для чего тебе разрушать наследие Господне?
\vs 2Sa 20:20 И отвечал Иоав и сказал: да не будет этого от меня, чтобы я уничтожил или разрушил!
\vs 2Sa 20:21 Это не так; но человек с горы Ефремовой, по имени Савей, сын Бихри, поднял руку свою на царя Давида; выдайте мне его одного, и я отступлю от города. И сказала женщина Иоаву: вот, голова его \bibemph{будет} тебе брошена со стены.
\vs 2Sa 20:22 И пошла женщина ко всему народу со своим умным словом [и говорила ко всему городу, чтобы отсекли голову Савею, сыну Бихри]; и отсекли голову Савею, сыну Бихри, и бросили Иоаву. Тогда [Иоав] затрубил трубою, и разошлись от города все [люди] по своим шатрам; Иоав же возвратился в Иерусалим к царю.
\vs 2Sa 20:23 И был Иоав \bibemph{поставлен} над всем войском Израильским, а Ванея, сын Иодаев,~--- над Хелефеями и над Фелефеями;
\vs 2Sa 20:24 Адорам~--- над сбором податей; Иосафат, сын Ахилуда~--- дееписателем;
\vs 2Sa 20:25 Суса~--- писцом; Садок и Авиафар~--- священниками;
\vs 2Sa 20:26 также и Ира Иаритянин был священником у Давида.
\vs 2Sa 21:1 Был голод на земле во дни Давида три года, год за годом. И вопросил Давид Господа. И сказал Господь: это ради Саула и кровожадного дома его, за то, что он умертвил Гаваонитян.
\vs 2Sa 21:2 Тогда царь призвал Гаваонитян и говорил с ними. Гаваонитяне были не из сынов Израилевых, но из остатков Аморреев; Израильтяне же дали им клятву, но Саул хотел истребить их по ревности своей о потомках Израиля и Иуды.
\vs 2Sa 21:3 И сказал Давид Гаваонитянам: что мне сделать для вас, и чем примирить вас, чтобы вы благословили наследие Господне?
\vs 2Sa 21:4 И сказали ему Гаваонитяне: не нужно нам ни серебра, ни золота от Саула, или от дома его, и не нужно нам, чтоб умертвили кого в Израиле. Он сказал: чего же вы хотите? я сделаю для вас.
\vs 2Sa 21:5 И сказали они царю: того человека, который губил нас и хотел истребить нас, чтобы не было нас ни в одном из пределов Израилевых,~---
\vs 2Sa 21:6 из его потомков выдай нам семь человек, и мы повесим их [на солнце] пред Господом в Гиве Саула, избранного Господом. И сказал царь: я выдам.
\vs 2Sa 21:7 Но пощадил царь Мемфивосфея, сына Ионафана, сына Саулова, ради клятвы именем Господним, которая была между ними, между Давидом и Ионафаном, сыном Сауловым.
\vs 2Sa 21:8 И взял царь двух сыновей Рицпы, дочери Айя, которая родила Саулу Армона и Мемфивосфея, и пять сыновей Мелхолы, дочери Сауловой, которых она родила Адриэлу, сыну Верзеллия из Мехолы,
\vs 2Sa 21:9 и отдал их в руки Гаваонитян, и они повесили их [на солнце] на горе пред Господом. И погибли все семь вместе; они умерщвлены в первые дни жатвы, в начале жатвы ячменя.
\vs 2Sa 21:10 Тогда Рицпа, дочь Айя, взяла вретище и разостлала его себе на той горе \bibemph{и сидела} от начала жатвы до того времени, пока не полились на них воды Божии с неба, и не допускала касаться их птицам небесным днем и зверям полевым ночью.
\vs 2Sa 21:11 И донесли Давиду, что сделала Рицпа, дочь Айя, наложница Саула. [И истлели они; и взял их Дан, сын Иои, из потомков исполинов.]
\vs 2Sa 21:12 И пошел Давид и взял кости Саула и кости Ионафана, сына его, у жителей Иависа Галаадского, которые тайно взяли их с площади Беф-Сана, где они были повешены Филистимлянами, когда убили Филистимляне Саула на Гелвуе.
\vs 2Sa 21:13 И перенес он оттуда кости Саула и кости Ионафана, сына его; и собрали кости повешенных [на солнце].
\vs 2Sa 21:14 И похоронили кости Саула и Ионафана, сына его, [и кости повешенных на солнце] в земле Вениаминовой, в Цела, во гробе Киса, отца его. И сделали всё, что повелел царь, и умилостивился Бог над страною после того.
\rsbpar\vs 2Sa 21:15 И открылась снова война между Филистимлянами и Израильтянами. И вышел Давид и слуги его с ним, и воевали с Филистимлянами; и Давид утомился.
\vs 2Sa 21:16 Тогда Иесвий, один из потомков Рефаимов, у которого копье было весом в триста сиклей меди и который опоясан был новым мечом, хотел поразить Давида.
\vs 2Sa 21:17 Но ему помог Авесса, сын Саруин, [и спас Давида Авесса] и поразил Филистимлянина и умертвил его. Тогда люди Давидовы поклялись, говоря: не выйдешь ты больше с нами на войну, чтобы не угас светильник Израиля.
\vs 2Sa 21:18 Потом была снова война с Филистимлянами в Гобе; тогда Совохай Хушатянин убил Сафута, одного из потомков Рефаимов.
\vs 2Sa 21:19 Было и другое сражение в Гобе; тогда убил Елханан, сын Ягаре-Оргима Вифлеемского, Голиафа Гефянина, у которого древко копья было, как навой у ткачей.
\vs 2Sa 21:20 Было еще сражение в Гефе; и был \bibemph{там} один человек рослый, имевший по шести пальцев на руках и на ногах, всего двадцать четыре, также из потомков Рефаимов,
\vs 2Sa 21:21 и он поносил Израильтян; но его убил Ионафан, сын Сафая, брата Давидова.
\vs 2Sa 21:22 Эти четыре были из рода Рефаимов в Гефе, и они пали от руки Давида и слуг его.
\vs 2Sa 22:1 И воспел Давид песнь Господу в день, когда Господь избавил его от руки всех врагов его и от руки Саула, и сказал:
\vs 2Sa 22:2 Господь~--- твердыня моя и крепость моя и избавитель мой.
\vs 2Sa 22:3 Бог мой~--- скала моя; на Него я уповаю; щит мой, рог спасения моего, ограждение мое и убежище мое; Спаситель мой, от бед Ты избавил меня!
\vs 2Sa 22:4 Призову Господа достопоклоняемого и от врагов моих спасусь.
\vs 2Sa 22:5 Объяли меня волны смерти, и потоки беззакония устрашили меня;
\vs 2Sa 22:6 цепи ада облегли меня, и сети смерти опутали меня.
\vs 2Sa 22:7 Но в тесноте моей я призвал Господа и к Богу моему воззвал, и Он услышал из [святого] чертога Своего голос мой, и вопль мой \bibemph{дошел} до слуха Его.
\vs 2Sa 22:8 Потряслась, всколебалась земля, дрогнули и подвиглись основания небес, ибо разгневался [на них Господь].
\vs 2Sa 22:9 Поднялся дым от гнева Его и из уст Его огонь поядающий; горящие угли сыпались от Него.
\vs 2Sa 22:10 Наклонил Он небеса и сошел; и мрак под ногами Его;
\vs 2Sa 22:11 и воссел на Херувимов, и полетел, и понесся на крыльях ветра;
\vs 2Sa 22:12 и мраком покрыл Себя, как сению, сгустив воды облаков небесных;
\vs 2Sa 22:13 от блистания пред Ним разгорались угли огненные.
\vs 2Sa 22:14 Возгремел с небес Господь, и Всевышний дал глас Свой;
\vs 2Sa 22:15 пустил стрелы и рассеял их; [блеснул] молниею и истребил их.
\vs 2Sa 22:16 И открылись источники моря, обнажились основания вселенной от грозного гласа Господа, от дуновения духа гнева Его.
\vs 2Sa 22:17 Простер Он \bibemph{руку} с высоты и взял меня, и извлек меня из вод многих;
\vs 2Sa 22:18 избавил меня от врага моего сильного, от ненавидящих меня, которые были сильнее меня.
\vs 2Sa 22:19 Они восстали на меня в день бедствия моего; но Господь был опорою для меня
\vs 2Sa 22:20 и вывел меня на пространное место, избавил меня, ибо Он благоволит ко мне.
\vs 2Sa 22:21 Воздал мне Господь по правде моей, по чистоте рук моих вознаградил меня.
\vs 2Sa 22:22 Ибо я хранил пути Господа и не был нечестивым пред Богом моим,
\vs 2Sa 22:23 ибо все заповеди Его предо мною, и от уставов Его я не отступал,
\vs 2Sa 22:24 и был непорочен пред Ним, и остерегался, чтобы не согрешить мне.
\vs 2Sa 22:25 И воздал мне Господь по правде моей, по чистоте моей пред очами Его.
\vs 2Sa 22:26 С милостивым Ты поступаешь милостиво, с мужем искренним~--- искренно,
\vs 2Sa 22:27 с чистым~--- чисто, а с лукавым~--- по лукавству его.
\vs 2Sa 22:28 Людей угнетенных Ты спасаешь и взором Своим унижаешь надменных.
\vs 2Sa 22:29 Ты, Господи, светильник мой; Господь просвещает тьму мою.
\vs 2Sa 22:30 С Тобою я поражаю войско; с Богом моим восхожу на стену.
\vs 2Sa 22:31 Бог!~--- непорочен путь Его, чисто слово Господа, щит Он для всех, надеющихся на Него.
\vs 2Sa 22:32 Ибо кто Бог, кроме Господа, и кто защита, кроме Бога нашего?
\vs 2Sa 22:33 Бог препоясует меня силою, устрояет мне верный путь;
\vs 2Sa 22:34 делает ноги мои, как оленьи, и на высотах поставляет меня;
\vs 2Sa 22:35 научает руки мои брани и мышцы мои напрягает, как медный лук.
\vs 2Sa 22:36 Ты даешь мне щит спасения Твоего, и милость Твоя возвеличивает меня.
\vs 2Sa 22:37 Ты расширяешь шаг мой подо мною, и не колеблются ноги мои.
\vs 2Sa 22:38 Я гоняюсь за врагами моими и истребляю их, и не возвращаюсь, доколе не уничтожу их;
\vs 2Sa 22:39 и истребляю их и поражаю их, и не встают и падают под ноги мои.
\vs 2Sa 22:40 Ты препоясываешь меня силою для войны и низлагаешь предо мною восстающих на меня;
\vs 2Sa 22:41 Ты обращаешь ко мне тыл врагов моих, и я истребляю ненавидящих меня.
\vs 2Sa 22:42 Они взывают, но нет спасающего,~--- ко Господу, но Он не внемлет им.
\vs 2Sa 22:43 Я рассеваю их, как прах земной, как грязь уличную мну их и топчу их.
\vs 2Sa 22:44 Ты избавил меня от мятежа народа моего; Ты сохранил меня, чтоб быть мне главою над иноплеменниками; народ, которого я не знал, служит мне.
\vs 2Sa 22:45 Иноплеменники ласкательствуют предо мною; по слуху \bibemph{обо мне} повинуются мне.
\vs 2Sa 22:46 Иноплеменники бледнеют и трепещут в укреплениях своих.
\vs 2Sa 22:47 Жив Господь и благословен защитник мой! Да будет превознесен Бог, убежище спасения моего,
\vs 2Sa 22:48 Бог, мстящий за меня и покоряющий мне народы
\vs 2Sa 22:49 и избавляющий меня от врагов моих! Над восстающими против меня Ты возвысил меня; от человека жестокого Ты избавил меня.
\vs 2Sa 22:50 За то я буду славить Тебя, Господи, между иноплеменниками и буду петь имени Твоему,
\vs 2Sa 22:51 величественно спасающий царя Своего и творящий милость помазаннику Своему Давиду и потомству его во веки!
\vs 2Sa 23:1 Вот последние слова Давида, изречение Давида, сына Иессеева, изречение мужа, поставленного высоко, помазанника Бога Иаковлева и сладкого певца Израилева:
\vs 2Sa 23:2 Дух Господень говорит во мне, и слово Его на языке у меня.
\vs 2Sa 23:3 Сказал Бог Израилев, говорил о мне скала Израилева: владычествующий над людьми будет праведен, владычествуя в страхе Божием.
\vs 2Sa 23:4 И как на рассвете утра, при восходе солнца на безоблачном небе, от сияния после дождя вырастает трава из земли,
\vs 2Sa 23:5 не так ли дом мой у Бога? Ибо завет вечный положил Он со мною, твердый и непреложный. Не так ли исходит от Него все спасение мое и все хотение мое?
\vs 2Sa 23:6 А нечестивые будут, как выброшенное терние, которого не берут рукою;
\vs 2Sa 23:7 но кто касается его, вооружается железом или деревом копья, и огнем сожигают его на месте.
\rsbpar\vs 2Sa 23:8 Вот имена храбрых у Давида: Исбосеф Ахаманитянин, главный из трех; он поднял копье свое на восемьсот человек и поразил их в один раз.
\vs 2Sa 23:9 По нем Елеазар, сын Додо, сына Ахохи, из трех храбрых, бывших с Давидом, когда они порицанием вызывали Филистимлян, собравшихся на войну;
\vs 2Sa 23:10 израильтяне вышли против них, и он стал и поражал Филистимлян до того, что рука его утомилась и прилипла к мечу. И даровал Господь в тот день великую победу, и народ последовал за ним для того только, чтоб обирать \bibemph{убитых}.
\vs 2Sa 23:11 За ним Шамма, сын Аге, Гараритянин. Когда Филистимляне собрались в Фирию, где было поле, засеянное чечевицею, и народ побежал от Филистимлян,
\vs 2Sa 23:12 то он стал среди поля и сберег его и поразил Филистимлян. И даровал тогда Господь великую победу.
\vs 2Sa 23:13 Трое сих главных из тридцати вождей пошли и вошли во время жатвы к Давиду в пещеру Одоллам, когда толпы Филистимлян стояли в долине Рефаимов.
\vs 2Sa 23:14 Давид был тогда в укрепленном месте, а отряд Филистимлян~--- в Вифлееме.
\vs 2Sa 23:15 И захотел Давид пить, и сказал: кто напоит меня водою из колодезя Вифлеемского, что у ворот?
\vs 2Sa 23:16 Тогда трое этих храбрых пробились сквозь стан Филистимский и почерпнули воды из колодезя Вифлеемского, что у ворот, и взяли и принесли Давиду. Но он не захотел пить ее и вылил ее во славу Господа,
\vs 2Sa 23:17 и сказал: сохрани меня Господь, чтоб я сделал это! не кровь ли это людей, ходивших с опасностью собственной жизни? И не захотел пить ее. Вот что сделали эти трое храбрых!
\vs 2Sa 23:18 И Авесса, брат Иоава, сын Саруин, был главным из трех; он убил копьем своим триста человек и был в славе у тех троих.
\vs 2Sa 23:19 Из трех он был знатнейшим и был начальником, но с теми тремя не равнялся.
\vs 2Sa 23:20 Ванея, сын Иодая, мужа храброго, великий по делам, из Кавцеила; он поразил двух сыновей Ариила Моавитского; он же сошел и убил льва во рве в снежное время;
\vs 2Sa 23:21 он же убил одного Египтянина человека видного; в руке Египтянина было копье, а он пошел к нему с палкою и отнял копье из руки Египтянина, и убил его собственным его копьем:
\vs 2Sa 23:22 вот что сделал Ванея, сын Иодаев, и он был в славе у трех храбрых;
\vs 2Sa 23:23 он был знатнее тридцати, но с теми тремя не равнялся. И поставил его Давид ближайшим исполнителем своих приказаний.
\rsbpar\vs 2Sa 23:24 [Вот имена сильных царя Давида:] Асаил, брат Иоава~--- в числе тридцати; Елханан, сын Додо, из Вифлеема,
\vs 2Sa 23:25 Шамма Хародитянин, Елика Хародитянин,
\vs 2Sa 23:26 Херец Палтитянин, Ира, сын Икеша, Фекоитянин,
\vs 2Sa 23:27 Евиезер Анафофянин, Мебуннай Хушатянин,
\vs 2Sa 23:28 Цалмон Ахохитянин, Магарай Нетофафянин,
\vs 2Sa 23:29 Хелев, сын Бааны, Нетофафянин, Иттай, сын Рибая, из Гивы сынов Вениаминовых,
\vs 2Sa 23:30 Ванея Пирафонянин, Иддай из Нахле-Гааша,
\vs 2Sa 23:31 Ави-Албон Арбатитянин, Азмавет Бархюмитянин,
\vs 2Sa 23:32 Елияхба Шаалбонянин; из сыновей Яшена~--- Ионафан,
\vs 2Sa 23:33 Шама Гараритянин, Ахиам, сын Шарара, Араритянин,
\vs 2Sa 23:34 Елифелет, сын Ахасбая, сына Магахати, Елиам, сын Ахитофела, Гилонянин,
\vs 2Sa 23:35 Хецрай Кармилитянин, Паарай Арбитянин,
\vs 2Sa 23:36 Игал, сын Нафана, из Цобы, Бани Гадитянин,
\vs 2Sa 23:37 Целек Аммонитянин, Нахарай Беротянин, оруженосец Иоава, сына Саруи,
\vs 2Sa 23:38 Ира Итритянин, Гареб Итритянин,
\vs 2Sa 23:39 Урия Хеттеянин. Всех тридцать семь.
\vs 2Sa 24:1 Гнев Господень опять возгорелся на Израильтян, и возбудил он в них Давида сказать: пойди, исчисли Израиля и Иуду.
\vs 2Sa 24:2 И сказал царь Иоаву военачальнику, который был при нем: пройди по всем коленам Израилевым [и Иудиным] от Дана до Вирсавии, и исчислите народ, чтобы мне знать число народа.
\vs 2Sa 24:3 И сказал Иоав царю: Господь Бог твой да умножит столько народа, сколько есть, и еще во сто раз столько, а очи господина моего царя да увидят \bibemph{это}; но для чего господин мой царь желает этого дела?
\vs 2Sa 24:4 Но слово царя Иоаву и военачальникам превозмогло; и пошел Иоав с военачальниками от царя считать народ Израильский.
\vs 2Sa 24:5 И перешли они Иордан и остановились в Ароере, на правой стороне города, который среди долины Гадовой, к Иазеру;
\vs 2Sa 24:6 и пришли в Галаад и в землю Тахтим-Ходши; и пришли в Дан-Яан и обошли Сидон;
\vs 2Sa 24:7 и пришли к укреплению Тира и во все города Хивеян и Хананеян и вышли на юг Иудеи в Вирсавию;
\vs 2Sa 24:8 и обошли всю землю и пришли чрез девять месяцев и двадцать дней в Иерусалим.
\vs 2Sa 24:9 И подал Иоав список народной переписи царю; и оказалось, что Израильтян было восемьсот тысяч мужей сильных, способных к войне, а Иудеян пятьсот тысяч.
\rsbpar\vs 2Sa 24:10 И вздрогнуло сердце Давидово после того, как он сосчитал народ. И сказал Давид Господу: тяжко согрешил я, поступив так; и ныне молю Тебя, Господи, прости грех раба Твоего, ибо крайне неразумно поступил я.
\vs 2Sa 24:11 Когда Давид встал на другой день утром, то было слово Господа к Гаду пророку, прозорливцу Давида:
\vs 2Sa 24:12 пойди и скажи Давиду: так говорит Господь: три \bibemph{наказания} предлагаю Я тебе; выбери себе одно из них, которое совершилось бы над тобою.
\vs 2Sa 24:13 И пришел Гад к Давиду, и возвестил ему, и сказал ему: избирай себе, быть ли голоду в стране твоей семь лет, или чтобы ты три месяца бегал от неприятелей твоих, и они преследовали тебя, или чтобы в продолжение трех дней была моровая язва в стране твоей? теперь рассуди и реши, что мне отвечать Пославшему меня.
\vs 2Sa 24:14 И сказал Давид Гаду: тяжело мне очень; но пусть впаду я в руки Господа, ибо велико милосердие Его; только бы в руки человеческие не впасть мне. [И избрал себе Давид моровую язву во время жатвы пшеницы.]
\vs 2Sa 24:15 И послал Господь язву на Израильтян от утра до назначенного времени; [и началась язва в народе] и умерло из народа, от Дана до Вирсавии, семьдесят тысяч человек.
\vs 2Sa 24:16 И простер Ангел [Божий] руку свою на Иерусалим, чтобы опустошить его; но Господь пожалел о бедствии и сказал Ангелу, поражавшему народ: довольно, теперь опусти руку твою. Ангел же Господень был тогда у гумна Орны Иевусеянина.
\vs 2Sa 24:17 И сказал Давид Господу, когда увидел Ангела, поражавшего народ, говоря: вот, я согрешил, я [пастырь] поступил беззаконно; а эти овцы, что сделали они? пусть же рука Твоя обратится на меня и на дом отца моего.
\vs 2Sa 24:18 И пришел в тот день Гад к Давиду и сказал: иди, поставь жертвенник Господу на гумне Орны Иевусеянина.
\vs 2Sa 24:19 И пошел Давид по слову Гада [пророка], как повелел Господь.
\vs 2Sa 24:20 И взглянул Орна и увидел царя и слуг его, шедших к нему, и вышел Орна и поклонился царю лицем своим до земли.
\vs 2Sa 24:21 И сказал Орна: зачем пришел господин мой царь к рабу своему? И сказал Давид: купить у тебя гумно для устроения жертвенника Господу, чтобы прекратилось поражение народа.
\vs 2Sa 24:22 И сказал Орна Давиду: пусть возьмет и вознесет \bibemph{в жертву} господин мой, царь, что ему угодно. Вот волы для всесожжения и повозки и упряжь воловья на дрова.
\vs 2Sa 24:23 Все это, царь, Орна отдает царю. Еще сказал Орна царю: Господь Бог твой да будет милостив к тебе!
\vs 2Sa 24:24 Но царь сказал Орне: нет, я заплач\acc{у} тебе, что ст\acc{о}ит, и не вознесу Господу Богу моему жертвы, \bibemph{взятой} даром. И купил Давид гумно и волов за пятьдесят сиклей серебра.
\vs 2Sa 24:25 И соорудил там Давид жертвенник Господу и принес всесожжения и мирные жертвы. [После Соломон распространил жертвенник, потому что он мал был.] И умилостивился Господь над страною, и прекратилось поражение Израильтян.

\bibbookdescr{1Ki}{
  inline={\LARGE Третья книга\\\Huge Царств\fns{У Евреев: <<Первая царей>>.}},
  toc={3-я Царств},
  bookmark={3-я Царств},
  header={3-я Царств},
  %headerleft={},
  %headerright={},
  abbr={3~Цар}
}
\vs 1Ki 1:1 Когда царь Давид состарился, вошел в \bibemph{преклонные} лета, то покрывали его одеждами, но не мог он согреться.
\vs 1Ki 1:2 И сказали ему слуги его: пусть поищут для господина нашего царя молодую девицу, чтоб она предстояла царю и ходила за ним и лежала с ним,~--- и будет тепло господину нашему, царю.
\vs 1Ki 1:3 И искали красивой девицы во всех пределах Израильских, и нашли Ависагу Сунамитянку, и привели ее к царю.
\vs 1Ki 1:4 Девица была очень красива, и ходила она за царем и прислуживала ему; но царь не познал ее.
\rsbpar\vs 1Ki 1:5 Адония, сын Аггифы, возгордившись говорил: я буду царем. И завел себе колесницы и всадников и пятьдесят человек скороходов.
\vs 1Ki 1:6 Отец же никогда не стеснял его вопросом: для чего ты это делаешь? Он же был очень красив и родился ему после Авессалома.
\vs 1Ki 1:7 И советовался он с Иоавом, сыном Саруиным, и с Авиафаром священником, и они помогали Адонии.
\vs 1Ki 1:8 Но священник Садок и Ванея, сын Иодаев, и пророк Нафан, и Семей, и Рисий, и сильные Давидовы не были на стороне Адонии.
\vs 1Ki 1:9 И заколол Адония овец и волов и тельцов у камня Зохелет, что у источника Рогель, и пригласил всех братьев своих, сыновей царя, со всеми Иудеянами, служившими у царя.
\vs 1Ki 1:10 Пророка же Нафана и Ванею, и тех сильных, и Соломона, брата своего, не пригласил.
\vs 1Ki 1:11 Тогда Нафан сказал Вирсавии, матери Соломона, говоря: слышала ли ты, что Адония, сын Аггифин, сделался царем, а господин наш Давид не знает \bibemph{о том}?
\vs 1Ki 1:12 Теперь, вот, я советую тебе: спасай жизнь твою и жизнь сына твоего Соломона.
\vs 1Ki 1:13 Иди и войди к царю Давиду и скажи ему: не клялся ли ты, господин мой царь, рабе твоей, говоря: <<сын твой Соломон будет царем после меня и он сядет на престоле моем>>? Почему же воцарился Адония?
\vs 1Ki 1:14 И вот, когда ты еще будешь говорить там с царем, войду и я вслед за тобою и дополню слова твои.
\vs 1Ki 1:15 Вирсавия пошла к царю в спальню; царь был очень стар, и Ависага Сунамитянка прислуживала царю;
\vs 1Ki 1:16 и наклонилась Вирсавия и поклонилась царю; и сказал царь: что тебе?
\vs 1Ki 1:17 Она сказала ему: господин мой царь! ты клялся рабе твоей Господом Богом твоим: <<сын твой Соломон будет царствовать после меня и он сядет на престоле моем>>.
\vs 1Ki 1:18 А теперь, вот, Адония \bibemph{воцарился}, и ты, господин мой царь, не знаешь \bibemph{о том}.
\vs 1Ki 1:19 И заколол он множество волов, тельцов и овец, и пригласил всех сыновей царских и священника Авиафара, и военачальника Иоава; Соломона же, раба твоего, не пригласил.
\vs 1Ki 1:20 Но ты, господин мой,~--- царь, и глаза всех Израильтян \bibemph{устремлены} на тебя, чтобы ты объявил им, кто сядет на престоле господина моего царя после него;
\vs 1Ki 1:21 иначе, когда господин мой царь почиет с отцами своими, падет обвинение на меня и на сына моего Соломона.
\vs 1Ki 1:22 Когда она еще говорила с царем, пришел и пророк Нафан.
\vs 1Ki 1:23 И сказали царю, говоря: вот Нафан пророк. И вошел он к царю и поклонился царю лицем до земли.
\vs 1Ki 1:24 И сказал Нафан: господин мой царь! сказал ли ты: <<Адония будет царствовать после меня и он сядет на престоле моем>>?
\vs 1Ki 1:25 Потому что он ныне сошел и заколол множество волов, тельцов и овец, и пригласил всех сыновей царских и военачальников и священника Авиафара, и вот, они едят и пьют у него и говорят: да живет царь Адония!
\vs 1Ki 1:26 А меня, раба твоего, и священника Садока, и Ванею, сына Иодаева, и Соломона, раба твоего, не пригласил.
\vs 1Ki 1:27 Не сталось ли это по \bibemph{воле} господина моего царя, и для чего ты не открыл рабу твоему, кто сядет на престоле господина моего царя после него?
\vs 1Ki 1:28 И отвечал царь Давид и сказал: позовите ко мне Вирсавию. И вошла она и стала пред царем.
\vs 1Ki 1:29 И клялся царь и сказал: жив Господь, избавлявший душу мою от всякой беды!
\vs 1Ki 1:30 Как я клялся тебе Господом Богом Израилевым, говоря, что Соломон, сын твой, будет царствовать после меня и он сядет на престоле моем вместо меня, так я и сделаю это сегодня.
\vs 1Ki 1:31 И наклонилась Вирсавия лицем до земли, и поклонилась царю, и сказала: да живет господин мой царь Давид во веки!
\rsbpar\vs 1Ki 1:32 И сказал царь Давид: позовите ко мне священника Садока и пророка Нафана и Ванею, сына Иодаева. И вошли они к царю.
\vs 1Ki 1:33 И сказал им царь: возьмите с собою слуг господина вашего и посадите Соломона, сына моего, на мула моего, и сведите его к Гиону,
\vs 1Ki 1:34 и да помажет его там Садок священник и Нафан пророк в царя над Израилем, и затрубите трубою и возгласите: да живет царь Соломон!
\vs 1Ki 1:35 Потом проводите его назад, и он придет и сядет на престоле моем; он будет царствовать вместо меня; ему завещал я быть вождем Израиля и Иуды.
\vs 1Ki 1:36 И отвечал Ванея, сын Иодаев, царю и сказал: аминь,~--- да скажет так Господь Бог господина моего царя!
\vs 1Ki 1:37 Как был Господь Бог с господином моим царем, так да будет Он с Соломоном и да возвеличит престол его более престола господина моего царя Давида!
\vs 1Ki 1:38 И пошли Садок священник и Нафан пророк и Ванея, сын Иодая, и Хелефеи и Фелефеи, и посадили Соломона на мула царя Давида, и повели его к Гиону.
\vs 1Ki 1:39 И взял Садок священник рог с елеем из скинии и помазал Соломона. И затрубили трубою, и весь народ восклицал: да живет царь Соломон!
\vs 1Ki 1:40 И весь народ провожал Соломона, и играл народ на свирелях, и весьма радовался, так что земля расседалась от криков его.
\vs 1Ki 1:41 И услышал Адония и все приглашенные им, как только перестали есть; а Иоав, услышав звук трубы, сказал: отчего этот шум волнующегося города?
\vs 1Ki 1:42 Еще он говорил, как пришел Ионафан, сын священника Авиафара. И сказал Адония: войди; ты~--- честный человек и несешь добрую весть.
\vs 1Ki 1:43 И отвечал Ионафан и сказал Адонии: да, господин наш царь Давид поставил Соломона царем;
\vs 1Ki 1:44 и послал царь с ним Садока священника и Нафана пророка, и Ванею, сына Иодая, и Хелефеев и Фелефеев, и они посадили его на мула царского;
\vs 1Ki 1:45 и помазали его Садок священник и Нафан пророк в царя в Гионе, и оттуда отправились с радостью, и пришел в движение город. Вот отчего шум, который вы слышите.
\vs 1Ki 1:46 И Соломон уже сел на царском престоле.
\vs 1Ki 1:47 И слуги царя приходили поздравить господина нашего царя Давида, говоря: Бог твой да прославит имя Соломона более твоего имени и да возвеличит престол его более твоего престола. И поклонился царь на ложе своем,
\vs 1Ki 1:48 и сказал царь так: <<благословен Господь Бог Израилев, Который сегодня дал [от семени моего] сидящего на престоле моем, и очи мои видят это!>>
\vs 1Ki 1:49 \bibemph{Тогда} испугались и встали все приглашенные, которые были у Адонии, и пошли каждый своею дорогою.
\vs 1Ki 1:50 Адония же, боясь Соломона, встал и пошел и ухватился за роги жертвенника.
\vs 1Ki 1:51 И донесли Соломону, говоря: вот, Адония боится царя Соломона, и вот, он держится за роги жертвенника, говоря: пусть поклянется мне теперь царь Соломон, что он не умертвит раба своего мечом.
\vs 1Ki 1:52 И сказал Соломон: если он будет человеком честным, то ни один волос его не упадет на землю; если же найдется в нем лукавство, то умрет.
\vs 1Ki 1:53 И послал царь Соломон, и привели его от жертвенника. И он пришел и поклонился царю Соломону; и сказал ему Соломон: иди в дом свой.
\vs 1Ki 2:1 Приблизилось время умереть Давиду, и завещал он сыну своему Соломону, говоря:
\vs 1Ki 2:2 вот, я отхожу в путь всей земли, ты же будь тверд и будь мужествен
\vs 1Ki 2:3 и храни завет Господа Бога твоего, ходя путями Его и соблюдая уставы Его и заповеди Его, и определения Его и постановления Его, как написано в законе Моисеевом, чтобы быть тебе благоразумным во всем, что ни будешь делать, и везде, куда ни обратишься;
\vs 1Ki 2:4 чтобы Господь исполнил слово Свое, которое Он сказал обо мне, говоря: <<если сыны твои будут наблюдать за путями своими, чтобы ходить предо Мною в истине от всего сердца своего и от всей души своей, то не прекратится муж от тебя на престоле Израилевом>>.
\vs 1Ki 2:5 Еще: ты знаешь, что сделал мне Иоав, сын Саруин, как поступил он с двумя вождями войска Израильского, с Авениром, сыном Нировым, и Амессаем, сыном Иеферовым, как он умертвил их и пролил кровь бранную во время мира, обагрив кровью бранною пояс на чреслах своих и обувь на ногах своих:
\vs 1Ki 2:6 поступи по мудрости твоей, чтобы не отпустить седины его мирно в преисподнюю.
\vs 1Ki 2:7 А сынам Верзеллия Галаадитянина окажи милость, чтоб они были между питающимися твоим столом, ибо они пришли ко мне, когда я бежал от Авессалома, брата твоего.
\vs 1Ki 2:8 Вот еще у тебя Семей, сын Геры Вениамитянина из Бахурима; он злословил меня тяжким злословием, когда я шел в Маханаим; но он вышел навстречу мне у Иордана, и я поклялся ему Господом, говоря: <<я не умерщвлю тебя мечом>>.
\vs 1Ki 2:9 Ты же не оставь его безнаказанным; ибо ты человек мудрый и знаешь, что тебе сделать с ним, чтобы низвести седину его в крови в преисподнюю.
\vs 1Ki 2:10 И почил Давид с отцами своими и погребен был в городе Давидовом.
\vs 1Ki 2:11 Времени царствования Давида над Израилем было сорок лет: в Хевроне царствовал он семь лет и тридцать три года царствовал в Иерусалиме.
\rsbpar\vs 1Ki 2:12 И сел Соломон на престоле Давида, отца своего, и царствование его было очень твердо.
\vs 1Ki 2:13 И пришел Адония, сын Аггифы, к Вирсавии, матери Соломона, [и поклонился ей]. Она сказала: с миром ли приход твой? И сказал он: с миром.
\vs 1Ki 2:14 И сказал он: у меня есть слово к тебе. Она сказала: говори.
\vs 1Ki 2:15 И сказал он: ты знаешь, что царство принадлежало мне, и весь Израиль обращал на меня взоры свои, как на будущего царя; но царство отошло от меня и досталось брату моему, ибо от Господа это было ему;
\vs 1Ki 2:16 теперь я прошу тебя об одном, не откажи мне. Она сказала ему: говори.
\vs 1Ki 2:17 И сказал он: прошу тебя, поговори царю Соломону, ибо он не откажет тебе, чтоб он дал мне Ависагу Сунамитянку в жену.
\vs 1Ki 2:18 И сказала Вирсавия: хорошо, я поговорю о тебе царю.
\vs 1Ki 2:19 И вошла Вирсавия к царю Соломону говорить ему об Адонии. Царь встал перед нею, и поклонился ей, и сел на престоле своем. Поставили престол и для матери царя, и она села по правую руку его
\vs 1Ki 2:20 и сказала: я имею к тебе одну небольшую просьбу, не откажи мне. И сказал ей царь: проси, мать моя; я не откажу тебе.
\vs 1Ki 2:21 И сказала она: дай Ависагу Сунамитянку Адонии, брату твоему, в жену.
\vs 1Ki 2:22 И отвечал царь Соломон и сказал матери своей: а зачем ты просишь Ависагу Сунамитянку для Адонии? проси ему \bibemph{также} и царства; ибо он мой старший брат, и ему священник Авиафар и Иоав, сын Саруин, [военачальник, друг].
\vs 1Ki 2:23 И поклялся царь Соломон Господом, говоря: то и то пусть сделает со мною Бог и еще больше сделает, если не на свою душу сказал Адония такое слово;
\vs 1Ki 2:24 ныне же,~--- жив Господь, укрепивший меня и посадивший меня на престоле Давида, отца моего, и устроивший мне дом, как говорил Он,~--- ныне же Адония должен умереть.
\vs 1Ki 2:25 И послал царь Соломон Ванею, сына Иодаева, который поразил его, и он умер.
\vs 1Ki 2:26 А священнику Авиафару царь сказал: ступай в Анафоф на твое поле; ты достоин смерти, но в настоящее время я не умерщвлю тебя, ибо ты носил ковчег Владыки Господа пред Давидом, отцом моим, и терпел все, что терпел отец мой.
\vs 1Ki 2:27 И удалил Соломон Авиафара от священства Господня, и исполнилось слово Господа, которое сказал Он о доме Илия в Силоме.
\vs 1Ki 2:28 Слух \bibemph{об этом} дошел до Иоава,~--- так как Иоав склонялся на сторону Адонии, а на сторону Соломона не склонялся,~--- и убежал Иоав в скинию Господню и ухватился за роги жертвенника.
\vs 1Ki 2:29 И донесли царю Соломону, что Иоав убежал в скинию Господню и что он у жертвенника. И послал Соломон Ванею, сына Иодаева, говоря: пойди, умертви его [и похорони его].
\vs 1Ki 2:30 И пришел Ванея в скинию Господню и сказал ему: так сказал царь: выходи. И сказал тот: нет, я хочу умереть здесь. Ванея передал это царю, говоря: так сказал Иоав, и так отвечал мне.
\vs 1Ki 2:31 Царь сказал ему: сделай, как он сказал, и умертви его и похорони его, и сними невинную кровь, пролитую Иоавом, с меня и с дома отца моего;
\vs 1Ki 2:32 да обратит Господь кровь его на голову его за то, что он убил двух мужей невинных и лучших его: поразил мечом, без ведома отца моего Давида, Авенира, сына Нирова, военачальника Израильского, и Амессая, сына Иеферова, военачальника Иудейского;
\vs 1Ki 2:33 да обратится кровь их на голову Иоава и на голову потомства его на веки, а Давиду и потомству его, и дому его и престолу его да будет мир на веки от Господа!
\vs 1Ki 2:34 И пошел Ванея, сын Иодаев, и поразил Иоава, и умертвил его, и он был похоронен в доме своем в пустыне.
\vs 1Ki 2:35 И поставил царь Соломон Ванею, сына Иодаева, вместо его над войском; [управление же царством было в Иерусалиме,] а Садока священника поставил царь [первосвященником] вместо Авиафара.\rsbpar [И дал Господь Соломону разум и мудрость весьма великую и обширный ум, как песок при море. И Соломон имел разум выше разума всех сынов востока и всех мудрых Египтян. И взял за себя дочь фараона и ввел ее в город Давидов, доколе не построил дома своего и, во-первых, дома Господня и стены вокруг Иерусалима; в семь лет окончил он строение. И было у Соломона семьдесят тысяч человек, носящих тяжести, и восемьдесят тысяч каменосеков в горах. И сделал Соломон море и подпоры, и большие бани и столбы, и источник на дворе и медное море, и построил замок и укрепления его, и разделил город Давидов. Тогда дочь фараона перешла из города Давидова в дом свой, который он построил ей; тогда построил Соломон стену вокруг города. И приносил Соломон три раза в год всесожжения и мирные жертвы на жертвеннике, который он устроил Господу, и курение совершал на нем пред Господом, и окончил строение дома. Главных приставников над работами Соломоновыми было три тысячи шестьсот, которые управляли народом, производившим работы. И построил он Ассур, и Магдон, и Газер, и Вефорон вышний и Валалаф; но эти города он построил после построения дома Господня и стены вокруг Иерусалима. И еще при жизни Давид завещал Соломону, говоря: вот у тебя Семей, сын Геры, сына Иеминиина из Бахурима; он злословил меня тяжким злословием, как я шел в Маханаим; но он вышел навстречу мне у Иордана, и я поклялся ему Господом, говоря: <<я не умерщвлю тебя мечом>>. Ты же не оставь его безнаказанным, ибо ты человек мудрый и знаешь, что тебе сделать с ним, чтобы низвести седину его в крови в преисподнюю.]
\vs 1Ki 2:36 И послав царь призвал Семея и сказал ему: построй себе дом в Иерусалиме и живи здесь, и никуда не выходи отсюда;
\vs 1Ki 2:37 и знай, что в тот день, в который ты выйдешь и перейдешь поток Кедрон, непременно умрешь; кровь твоя будет на голове твоей.
\vs 1Ki 2:38 И сказал Семей царю: хорошо; как приказал господин мой царь, так сделает раб твой. И жил Семей в Иерусалиме долгое время.
\vs 1Ki 2:39 Но через три года случилось, что у Семея двое рабов убежали к Анхусу, сыну Маахи, царю Гефскому. И сказали Семею, говоря: вот, рабы твои в Гефе.
\vs 1Ki 2:40 И встал Семей, и оседлал осла своего, и отправился в Геф к Анхусу искать рабов своих. И возвратился Семей и привел рабов своих из Гефа.
\vs 1Ki 2:41 И донесли Соломону, что Семей ходил из Иерусалима в Геф и возвратился.
\vs 1Ki 2:42 И послав призвал царь Семея и сказал ему: не клялся ли я тебе Господом и не объявлял ли тебе, говоря: <<знай, что в тот день, в который ты выйдешь и пойдешь куда-нибудь, непременно умрешь>>? и ты сказал мне: <<хорошо>>;
\vs 1Ki 2:43 зачем же ты не соблюл приказания, которое я дал тебе пред Господом с клятвою?
\vs 1Ki 2:44 И сказал царь Семею: ты знаешь и знает сердце твое все зло, какое ты сделал отцу моему Давиду; да обратит же Господь злобу твою на голову твою!
\vs 1Ki 2:45 а царь Соломон да будет благословен, и престол Давида да будет непоколебим пред Господом во веки!
\vs 1Ki 2:46 и повелел царь Ванее, сыну Иодаеву, и он пошел и поразил Семея, и тот умер.
\vs 1Ki 3:1 [Когда утвердилось царство в руках Соломона,] Соломон породнился с фараоном, царем Египетским, и взял за себя дочь фараона и ввел ее в город Давидов, доколе не построил дома своего и дома Господня и стены вокруг Иерусалима.
\vs 1Ki 3:2 Народ еще приносил жертвы на высотах, ибо не был построен дом имени Господа до того времени.
\vs 1Ki 3:3 И возлюбил Соломон Господа, ходя по уставу Давида, отца своего; но и он приносил жертвы и курения на высотах.
\vs 1Ki 3:4 И пошел царь в Гаваон, чтобы принести там жертву, ибо там был главный жертвенник. Тысячу всесожжений вознес Соломон на том жертвеннике.
\rsbpar\vs 1Ki 3:5 В Гаваоне явился Господь Соломону во сне ночью, и сказал Бог: проси, что дать тебе.
\vs 1Ki 3:6 И сказал Соломон: Ты сделал рабу Твоему Давиду, отцу моему, великую милость; и за то, что он ходил пред Тобою в истине и правде и с искренним сердцем пред Тобою, Ты сохранил ему эту великую милость и даровал ему сына, который сидел бы на престоле его, как это и есть ныне;
\vs 1Ki 3:7 и ныне, Господи Боже мой, Ты поставил раба Твоего царем вместо Давида, отца моего; но я отрок малый, не знаю ни моего выхода, ни входа;
\vs 1Ki 3:8 и раб Твой~--- среди народа Твоего, который избрал Ты, народа столь многочисленного, что по множеству его нельзя ни исчислить его, ни обозреть;
\vs 1Ki 3:9 даруй же рабу Твоему сердце разумное, чтобы судить народ Твой и различать, что добро и что зло; ибо кто может управлять этим многочисленным народом Твоим?
\vs 1Ki 3:10 И благоугодно было Господу, что Соломон просил этого.
\vs 1Ki 3:11 И сказал ему Бог: за то, что ты просил этого и не просил себе долгой жизни, не просил себе богатства, не просил себе душ врагов твоих, но просил себе разума, чтоб уметь судить,~---
\vs 1Ki 3:12 вот, Я сделаю по слову твоему: вот, Я даю тебе сердце мудрое и разумное, так что подобного тебе не было прежде тебя, и после тебя не восстанет подобный тебе;
\vs 1Ki 3:13 и то, чего ты не просил, Я даю тебе, и богатство и славу, так что не будет подобного тебе между царями во все дни твои;
\vs 1Ki 3:14 и если будешь ходить путем Моим, сохраняя уставы Мои и заповеди Мои, как ходил отец твой Давид, Я продолжу и дни твои.
\vs 1Ki 3:15 И пробудился Соломон, и вот, \bibemph{это было} сновидение. И пошел он в Иерусалим и стал [пред жертвенником] пред ковчегом завета Господня, и принес всесожжения и совершил \bibemph{жертвы} мирные, и сделал большой пир для всех слуг своих.
\rsbpar\vs 1Ki 3:16 Тогда пришли две женщины блудницы к царю и стали пред ним.
\vs 1Ki 3:17 И сказала одна женщина: о, господин мой! я и эта женщина живем в одном доме; и я родила при ней в этом доме;
\vs 1Ki 3:18 на третий день после того, как я родила, родила и эта женщина; и были мы вместе, и в доме никого постороннего с нами не было; только мы две были в доме;
\vs 1Ki 3:19 и умер сын этой женщины ночью, ибо она заспала его;
\vs 1Ki 3:20 и встала она ночью, и взяла сына моего от меня, когда я, раба твоя, спала, и положила его к своей груди, а своего мертвого сына положила к моей груди;
\vs 1Ki 3:21 утром я встала, чтобы покормить сына моего, и вот, он был мертвый; а когда я всмотрелась в него утром, то это был не мой сын, которого я родила.
\vs 1Ki 3:22 И сказала другая женщина: нет, мой сын живой, а твой сын мертвый. А та говорила ей: нет, твой сын мертвый, а мой живой. И говорили они так пред царем.
\vs 1Ki 3:23 И сказал царь: эта говорит: мой сын живой, а твой сын мертвый; а та говорит: нет, твой сын мертвый, а мой сын живой.
\vs 1Ki 3:24 И сказал царь: подайте мне меч. И принесли меч к царю.
\vs 1Ki 3:25 И сказал царь: рассеките живое дитя надвое и отдайте половину одной и половину другой.
\vs 1Ki 3:26 И отвечала та женщина, которой сын был живой, царю, ибо взволновалась вся внутренность ее от жалости к сыну своему: о, господин мой! отдайте ей этого ребенка живого и не умерщвляйте его. А другая говорила: пусть же не будет ни мне, ни тебе, рубите.
\vs 1Ki 3:27 И отвечал царь и сказал: отдайте этой живое дитя, и не умерщвляйте его: она~--- его мать.
\vs 1Ki 3:28 И услышал весь Израиль о суде, как рассудил царь; и стали бояться царя, ибо увидели, что мудрость Божия в нем, чтобы производить суд.
\vs 1Ki 4:1 И был царь Соломон царем над всем Израилем.
\vs 1Ki 4:2 И вот начальники, которые \bibemph{были} у него: Азария, сын Садока священника;
\vs 1Ki 4:3 Елихореф и Ахия, сыновья Сивы, писцы; Иосафат, сын Ахилуда, дееписатель;
\vs 1Ki 4:4 Ванея, сын Иодая, военачальник; Садок и Авиафар~--- священники;
\vs 1Ki 4:5 Азария, сын Нафана, начальник над приставниками, и Завуф, сын Нафана священника~--- друг царя;
\vs 1Ki 4:6 Ахисар~--- начальник над домом \bibemph{царским}, и Адонирам, сын Авды,~--- над податями.
\rsbpar\vs 1Ki 4:7 И было у Соломона двенадцать приставников над всем Израилем, и они доставляли продовольствие царю и дому его; каждый должен был доставлять продовольствие на один месяц в году.
\vs 1Ki 4:8 Вот имена их: Бен-Хур~--- на горе Ефремовой;
\vs 1Ki 4:9 Бен-Декер~--- в Макаце и в Шаалбиме, в Вефсамисе и в Елоне и в Беф-Ханане;
\vs 1Ki 4:10 Бен-Хесед~--- в Арюбофе; ему же принадлежал Соко и вся земля Хефер;
\vs 1Ki 4:11 Бен-Авинадав~--- \bibemph{над} всем Нафаф-Дором; Тафафь, дочь Соломона, была его женою;
\vs 1Ki 4:12 Ваана, сын Ахилуда, в Фаанахе и Мегиддо и во всем Беф-Сане, что близ Цартана ниже Иезрееля, от Беф-Сана до Абел-Мехола, и даже за Иокмеам;
\vs 1Ki 4:13 Бен-Гевер~--- в Рамофе Галаадском; у него были селения Иаира, сына Манассиина, что в Галааде; у него также область Аргов, что в Васане, шестьдесят больших городов со стенами и медными затворами;
\vs 1Ki 4:14 Ахинадав, сын Гиддо, в Маханаиме;
\vs 1Ki 4:15 Ахимаас~--- в \bibemph{земле} Неффалимовой; он взял себе в жену Васемафу, дочь Соломона;
\vs 1Ki 4:16 Ваана, сын Хушая, в \bibemph{земле} Асировой и в Баалофе;
\vs 1Ki 4:17 Иосафат, сын Паруаха, в \bibemph{земле} Иссахаровой;
\vs 1Ki 4:18 Шимей, сын Елы, в \bibemph{земле} Вениаминовой;
\vs 1Ki 4:19 Гевер, сын Урия, в земле Галаадской, в земле Сигона, царя Аморрейского, и Ога, царя Васанского. Он был приставник в этой земле.
\vs 1Ki 4:20 Иуда и Израиль, многочисленные как песок у моря, ели, пили и веселились.
\vs 1Ki 4:21 Соломон владел всеми царствами от реки \bibemph{Евфрата} до земли Филистимской и до пределов Египта. Они приносили дары и служили Соломону во все дни жизни его.
\vs 1Ki 4:22 Продовольствие Соломона на каждый день составляли: тридцать к\acc{о}ров муки пшеничной и шестьдесят к\acc{о}ров прочей муки,
\vs 1Ki 4:23 десять волов откормленных и двадцать волов с пастбища, и сто овец, кроме оленей, и серн, и сайгаков, и откормленных птиц;
\vs 1Ki 4:24 ибо он владычествовал над всею землею по эту сторону реки, от Типсаха до Газы, над всеми царями по эту сторону реки, и был у него мир со всеми окрестными странами.
\vs 1Ki 4:25 И жили Иуда и Израиль спокойно, каждый под виноградником своим и под смоковницею своею, от Дана до Вирсавии, во все дни Соломона.
\vs 1Ki 4:26 И было у Соломона сорок тысяч стойл для коней колесничных и двенадцать тысяч для конницы.
\vs 1Ki 4:27 И те приставники доставляли царю Соломону все принадлежащее к столу царя, каждый в свой месяц, и не допускали недостатка ни в чем.
\vs 1Ki 4:28 И ячмень и солому для коней и для мулов доставляли каждый в свою очередь на место, где находился царь.
\rsbpar\vs 1Ki 4:29 И дал Бог Соломону мудрость и весьма великий разум, и обширный ум, как песок на берегу моря.
\vs 1Ki 4:30 И была мудрость Соломона выше мудрости всех сынов востока и всей мудрости Египтян.
\vs 1Ki 4:31 Он был мудрее всех людей, мудрее и Ефана Езрахитянина, и Емана, и Халкола, и Дарды, сыновей Махола, и имя его было в славе у всех окрестных народов.
\vs 1Ki 4:32 И изрек он три тысячи притчей, и песней его было тысяча и пять;
\vs 1Ki 4:33 и говорил он о деревах, от кедра, что в Ливане, до иссопа, вырастающего из стены; говорил и о животных, и о птицах, и о пресмыкающихся, и о рыбах.
\vs 1Ki 4:34 И приходили от всех народов послушать мудрости Соломона, от всех царей земных, которые слышали о мудрости его.
\vs 1Ki 5:1 И послал Хирам, царь Тирский, слуг своих к Соломону, когда услышал, что его помазали в царя на место отца его; ибо Хирам был другом Давида во всю жизнь.
\vs 1Ki 5:2 И послал также и Соломон к Хираму сказать:
\vs 1Ki 5:3 ты знаешь, что Давид, отец мой, не мог построить дом имени Господа Бога своего по причине войн с окрестными народами, доколе Господь не покорил их под стопы ног его;
\vs 1Ki 5:4 ныне же Господь Бог мой даровал мне покой отовсюду: нет противника и нет более препон;
\vs 1Ki 5:5 и вот, я намерен построить дом имени Господа Бога моего, как сказал Господь отцу моему Давиду, говоря: <<сын твой, которого Я посажу вместо тебя на престоле твоем, он построит дом имени Моему>>;
\vs 1Ki 5:6 итак прикажи нарубить для меня кедров с Ливана; и вот, рабы мои будут вместе с твоими рабами, и я буду давать тебе плату за рабов твоих, какую ты назначишь; ибо ты знаешь, что у нас нет людей, которые умели бы рубить дерева так, как Сидоняне.
\vs 1Ki 5:7 Когда услышал Хирам слова Соломона, очень обрадовался и сказал: благословен ныне Господь, Который дал Давиду сына мудрого \bibemph{для управления} этим многочисленным народом!
\vs 1Ki 5:8 И послал Хирам к Соломону сказать: я выслушал то, за чем ты посылал ко мне, и исполню все желание твое о деревах кедровых и деревах кипарисовых;
\vs 1Ki 5:9 рабы мои свезут их с Ливана к морю, и я плотами доставлю их морем к месту, которое ты назначишь мне, и там сложу их, и ты возьмешь; но и ты исполни мое желание, чтобы доставлять хлеб для моего дома.
\vs 1Ki 5:10 И давал Хирам Соломону дерева кедровые и дерева кипарисовые, вполне по его желанию.
\vs 1Ki 5:11 А Соломон давал Хираму двадцать тысяч к\acc{о}ров пшеницы для продовольствия дома его и двадцать к\acc{о}ров оливкового выбитого масла: столько давал Соломон Хираму каждый год.
\rsbpar\vs 1Ki 5:12 Господь дал мудрость Соломону, как обещал ему. И был мир между Хирамом и Соломоном, и они заключили между собою союз.
\vs 1Ki 5:13 И обложил царь Соломон повинностью весь Израиль; повинность же состояла в тридцати тысячах человек.
\vs 1Ki 5:14 И посылал их на Ливан, по десяти тысяч на месяц, попеременно; месяц они были на Ливане, а два месяца в доме своем. Адонирам же начальствовал над ними.
\vs 1Ki 5:15 Еще у Соломона было семьдесят тысяч носящих тяжести и восемьдесят тысяч каменосеков в горах,
\vs 1Ki 5:16 кроме трех тысяч трехсот начальников, поставленных Соломоном над работою для надзора за народом, который производил работу.
\vs 1Ki 5:17 И повелел царь привозить камни большие, камни дорогие, для основания дома, камни обделанные.
\vs 1Ki 5:18 Обтесывали же их работники Соломоновы и работники Хирамовы и Гивлитяне, и приготовляли дерева и камни для строения дома [три года].
\vs 1Ki 6:1 В четыреста восьмидесятом году по исшествии сынов Израилевых из земли Египетской, в четвертый год царствования Соломонова над Израилем, в месяц Зиф, который есть второй месяц, начал он строить храм Господу.
\vs 1Ki 6:2 Храм, который построил царь Соломон Господу, длиною был в шестьдесят локтей, шириною в двадцать и вышиною в тридцать локтей,
\vs 1Ki 6:3 и притвор пред храмом в двадцать локтей длины, соответственно ширине храма, и в десять локтей ширины пред храмом.
\vs 1Ki 6:4 И сделал он в доме окна решетчатые, глухие с откосами.
\vs 1Ki 6:5 И сделал пристройку вокруг стен храма, вокруг храма и давира\fns{Отделение для Святаго Святых.}; и сделал боковые комнаты кругом.
\vs 1Ki 6:6 Нижний \bibemph{ярус} пристройки шириною был в пять локтей, средний шириною в шесть локтей, а третий шириною в семь локтей; ибо вокруг храма извне сделаны были уступы, дабы пристройка не прикасалась к стенам храма.
\vs 1Ki 6:7 Когда строился храм, на строение употребляемы были обтесанные камни; ни молота, ни тесла, ни всякого другого железного орудия не было слышно в храме при строении его.
\vs 1Ki 6:8 Вход в средний ярус был с правой стороны храма. По круглым лестницам всходили в средний \bibemph{ярус}, а от среднего в третий.
\vs 1Ki 6:9 И построил он храм, и кончил его, и обшил храм кедровыми досками.
\vs 1Ki 6:10 И пристроил ко всему храму боковые комнаты вышиною в пять локтей; они прикреплены были к храму посредством кедровых бревен.
\rsbpar\vs 1Ki 6:11 И было слово Господа к Соломону, и сказано ему:
\vs 1Ki 6:12 вот, ты строишь храм; если ты будешь ходить по уставам Моим, и поступать по определениям Моим и соблюдать все заповеди Мои, поступая по ним, то Я исполню на тебе слово Мое, которое Я сказал Давиду, отцу твоему,
\vs 1Ki 6:13 и буду жить среди сынов Израилевых, и не оставлю народа Моего Израиля.
\rsbpar\vs 1Ki 6:14 И построил Соломон храм и кончил его.
\vs 1Ki 6:15 И обложил стены храма внутри кедровыми досками; от пола храма до потолка внутри обложил деревом и покрыл пол храма кипарисовыми досками.
\vs 1Ki 6:16 И устроил в задней стороне храма, в двадцати локтях от края, стену, и обложил стены и потолок кедровыми досками, и устроил давир для Святаго Святых.
\vs 1Ki 6:17 Сорока локтей \bibemph{был} храм, то есть передняя часть храма.
\vs 1Ki 6:18 На кедрах внутри храма были вырезаны \bibemph{подобия} огурцов и распускающихся цветов; все было покрыто кедром, камня не видно было.
\vs 1Ki 6:19 Давир же внутри храма он приготовил для того, чтобы поставить там ковчег завета Господня.
\vs 1Ki 6:20 И давир был длиною в двадцать локтей, шириною в двадцать локтей и вышиною в двадцать локтей; он обложил его чистым золотом; обложил также и кедровый жертвенник.
\vs 1Ki 6:21 И обложил Соломон храм внутри чистым золотом, и протянул золотые цепи пред давиром, и обложил его золотом.
\vs 1Ki 6:22 Весь храм он обложил золотом, весь храм до конца, и весь жертвенник, который пред давиром, обложил золотом.
\vs 1Ki 6:23 И сделал в давире двух херувимов из масличного дерева, вышиною в десять локтей.
\vs 1Ki 6:24 Одно крыло херувима было в пять локтей и другое крыло херувима в пять локтей; десять локтей было от одного конца крыльев его до другого конца крыльев его.
\vs 1Ki 6:25 В десять локтей \bibemph{был} и другой херувим; одинаковой меры и одинакового вида \bibemph{были} оба херувима.
\vs 1Ki 6:26 Высота одного херувима \bibemph{была} десять локтей, также и другого херувима.
\vs 1Ki 6:27 И поставил он херувимов среди внутренней части храма. Крылья же херувимов были распростерты, и касалось крыло одного \bibemph{одной} стены, а крыло другого херувима касалось другой стены; другие же крылья их среди храма сходились крыло с крылом.
\vs 1Ki 6:28 И обложил он херувимов золотом.
\vs 1Ki 6:29 И на всех стенах храма кругом сделал резные изображения херувимов и пальмовых дерев и распускающихся цветов, внутри и вне.
\vs 1Ki 6:30 И пол в храме обложил золотом во внутренней и передней части.
\vs 1Ki 6:31 Для входа в давир сделал двери из масличного дерева, с пятиугольными косяками.
\vs 1Ki 6:32 На двух половинах дверей из масличного дерева он сделал резных херувимов и пальмы и распускающиеся цветы и обложил золотом; покрыл золотом и херувимов и пальмы.
\vs 1Ki 6:33 И у входа в храм сделал косяки из масличного дерева четырехугольные,
\vs 1Ki 6:34 и две двери из кипарисового дерева; обе половинки одной двери были подвижные, и обе половинки другой двери были подвижные.
\vs 1Ki 6:35 И вырезал \bibemph{на них} херувимов и пальмы и распускающиеся цветы и обложил золотом по резьбе.
\vs 1Ki 6:36 И построил внутренний двор из трех рядов обтесанного камня и из ряда кедровых брусьев.
\vs 1Ki 6:37 В четвертый год, в месяц Зиф, [в месяц второй,] положил он основание храму Господа,
\vs 1Ki 6:38 а на одиннадцатом году, в месяце Буле,~--- это месяц восьмой,~--- он окончил храм со всеми принадлежностями его и по всем предначертаниям его; строил его семь лет.
\vs 1Ki 7:1 А свой дом Соломон строил тринадцать лет и окончил весь дом свой.
\vs 1Ki 7:2 И построил он дом из дерева Ливанского, длиною во сто локтей, шириною в пятьдесят локтей, а вышиною в тридцать локтей, на четырех рядах кедровых столбов; и кедровые бревна \bibemph{положены были} на столбах.
\vs 1Ki 7:3 И настлан был помост из кедра над бревнами на сорока пяти столбах, по пятнадцати в ряд.
\vs 1Ki 7:4 Оконных косяков \bibemph{было} три ряда; и три ряда \bibemph{окон}, окно против окна.
\vs 1Ki 7:5 И все двери и дверные косяки были четырехугольные, и окно против окна, в три ряда.
\vs 1Ki 7:6 И притвор из столбов сделал он длиною в пятьдесят локтей, шириною в тридцать локтей, и пред ними крыльцо, и столбы, и порог пред ними.
\vs 1Ki 7:7 Еще притвор с престолом, с которого он судил, притвор для судилища сделал он и покрыл все полы кедром.
\vs 1Ki 7:8 В доме, где он жил, другой двор позади притвора был такого же устройства. И в доме дочери фараоновой, которую взял за себя Соломон, он сделал такой же притвор.
\vs 1Ki 7:9 Все это сделано было из дорогих камней, обтесанных по размеру, обрезанных пилою, с внутренней и наружной стороны, от основания до выступов, и с наружной стороны до большого двора.
\vs 1Ki 7:10 И в основание положены были камни дорогие, камни большие, камни в десять локтей и камни в восемь локтей,
\vs 1Ki 7:11 и сверху дорогие камни, обтесанные по размеру, и кедр.
\vs 1Ki 7:12 Большой двор огорожен был кругом тремя рядами тесаных камней и одним рядом кедровых бревен; также и внутренний двор храма Господа и притвор храма.
\rsbpar\vs 1Ki 7:13 И послал царь Соломон и взял из Тира Хирама,
\vs 1Ki 7:14 сына одной вдовы, из колена Неффалимова. Отец его Тирянин был медник; он владел способностью, искусством и уменьем выделывать всякие вещи из меди. И пришел он к царю Соломону и производил у него всякие работы:
\vs 1Ki 7:15 и сделал он два медных столба, каждый в восемнадцать локтей вышиною, и снурок в двенадцать локтей обнимал \bibemph{окружность} того и другого столба;
\vs 1Ki 7:16 и два венца, вылитых из меди, он сделал, чтобы положить наверху столбов: пять локтей вышины в одном венце и пять локтей вышины в другом венце;
\vs 1Ki 7:17 сетки плетеной работы и снурки в виде цепочек для венцов, которые были на верху столбов: семь на одном венце и семь на другом венце.
\vs 1Ki 7:18 Так сделал он столбы и два ряда гранатовых яблок вокруг сетки, чтобы покрыть венцы, которые на верху столбов; то же сделал и для другого венца.
\vs 1Ki 7:19 А в притворе венцы на верху столбов сделаны \bibemph{наподобие} лилии, в четыре локтя,
\vs 1Ki 7:20 и венцы на обоих столбах вверху, прямо над выпуклостью, которая подле сетки; и на другом венце, рядами кругом, двести гранатовых яблок.
\vs 1Ki 7:21 И поставил столбы к притвору храма; поставил столб на правой стороне и дал ему имя Иахин, и поставил столб на левой стороне и дал ему имя Воаз.
\vs 1Ki 7:22 И над столбами поставил \bibemph{венцы}, сделанные \bibemph{наподобие} лилии; так окончена работа над столбами.
\vs 1Ki 7:23 И сделал литое \bibemph{из меди} море,~--- от края его до края его десять локтей,~--- совсем круглое, вышиною в пять локтей, и снурок в тридцать локтей обнимал его кругом.
\vs 1Ki 7:24 \bibemph{Подобия} огурцов под краями его окружали его по десяти на локоть, окружали море со всех сторон в два ряда; \bibemph{подобия} огурцов были вылиты с ним одним литьем.
\vs 1Ki 7:25 Оно стояло на двенадцати волах: три глядели к северу, три глядели к западу, три глядели к югу и три глядели к востоку; море лежало на них, и зады их \bibemph{обращены были} внутрь под него.
\vs 1Ki 7:26 Толщиною оно было в ладонь, и края его, сделанные подобно краям чаши, \bibemph{походили} на распустившуюся лилию. Оно вмещало две тысячи батов.
\vs 1Ki 7:27 И сделал он десять медных подстав; длина каждой подставы~--- четыре локтя, ширина~--- четыре локтя и три локтя~--- вышина.
\vs 1Ki 7:28 И вот устройство подстав: у них стенки, стенки между наугольными пластинками;
\vs 1Ki 7:29 на стенках, которые между наугольниками, \bibemph{изображены} были львы, волы и херувимы; также и на наугольниках, а выше и ниже львов и волов~--- развесистые венки;
\vs 1Ki 7:30 у каждой подставы по четыре медных колеса и оси медные. На четырех углах выступы наподобие плеч, выступы литые внизу, под чашею, подле каждого венка.
\vs 1Ki 7:31 Отверстие от внутреннего венка до верха в один локоть; отверстие его круглое, подобно подножию столбов, в полтора локтя, и при отверстии его изваяния; но боковые стенки четырехугольные, не круглые.
\vs 1Ki 7:32 Под стенками было четыре колеса, и оси колес в подставах; вышина каждого колеса~--- полтора локтя.
\vs 1Ki 7:33 Устройство колес такое же, как устройство колес в колеснице; оси их, и ободья их, и спицы их, и ступицы их, все было литое.
\vs 1Ki 7:34 Четыре выступа на четырех углах каждой подставы; из подставы \bibemph{выходили} выступы ее.
\vs 1Ki 7:35 И на верху подставы круглое возвышение на пол-локтя вышины; и на верху подставы рукоятки ее и стенки ее из одной с нею массы.
\vs 1Ki 7:36 И изваял он на дощечках ее рукоятки и на стенках ее херувимов, львов и пальмы, сколько где позволяло место, и вокруг развесистые венки.
\vs 1Ki 7:37 Так сделал он десять подстав: у всех их одно литье, одна мера, один вид.
\vs 1Ki 7:38 И сделал десять медных умывальниц: каждая умывальница вмещала сорок батов, каждая умывальница была в четыре локтя, каждая умывальница стояла на одной из десяти подстав.
\vs 1Ki 7:39 И расставил подставы~--- пять на правой стороне храма и пять на левой стороне храма, а море поставил на правой стороне храма, на восточно-южной стороне.
\vs 1Ki 7:40 И сделал Хирам умывальницы и лопатки и чаши. И кончил Хирам всю работу, которую производил у царя Соломона для храма Господня:
\vs 1Ki 7:41 два столба и две опояски венцов, которые на верху столбов, и две сетки для покрытия двух опоясок венцов, которые на верху столбов;
\vs 1Ki 7:42 и четыреста гранатовых яблок на двух сетках; два ряда гранатовых яблок для каждой сетки, для покрытия двух опоясок венцов, которые на столбах;
\vs 1Ki 7:43 и десять подстав и десять умывальниц на подставах;
\vs 1Ki 7:44 одно море и двенадцать волов под морем;
\vs 1Ki 7:45 и тазы, и лопатки, и чаши. Все вещи, которые сделал Хирам царю Соломону для храма Господа, \bibemph{были} из полированной меди.
\vs 1Ki 7:46 Царь выливал их в глинистой земле, в окрестности Иордана, между Сокхофом и Цартаном.
\vs 1Ki 7:47 И поставил Соломон все сии вещи \bibemph{на место}. По причине чрезвычайного их множества, вес меди не определен.
\rsbpar\vs 1Ki 7:48 И сделал Соломон все вещи, которые в храме Господа: золотой жертвенник и золотой стол, на котором хлебы предложения;
\vs 1Ki 7:49 и светильники~--- пять по правую сторону и пять по левую сторону, пред задним отделением храма, из чистого золота, и цветы, и лампадки, и щипцы из золота;
\vs 1Ki 7:50 и блюда, и ножи, и чаши, и лотки, и кадильницы из чистого золота, и петли у дверей внутреннего храма во Святом Святых и у дверей в храме из золота же.
\vs 1Ki 7:51 Так совершена вся работа, которую производил царь Соломон для храма Господа. И принес Соломон посвященное Давидом, отцом его; серебро и золото и вещи отдал в сокровищницы храма Господня.
\vs 1Ki 8:1 Тогда созвал Соломон старейшин Израилевых и всех начальников колен, глав поколений сынов Израилевых, к царю Соломону в Иерусалим, чтобы перенести ковчег завета Господня из города Давидова, то есть Сиона.
\vs 1Ki 8:2 И собрались к царю Соломону на праздник все Израильтяне в месяце Афаниме, который есть седьмой месяц.
\vs 1Ki 8:3 И пришли все старейшины Израилевы; и подняли священники ковчег,
\vs 1Ki 8:4 и понесли ковчег Господень и скинию собрания и все священные вещи, которые были в скинии; и несли их священники и левиты.
\vs 1Ki 8:5 А царь Соломон и с ним все общество Израилево, собравшееся к нему, шли пред ковчегом, принося жертвы из мелкого и крупного скота, которых невозможно исчислить и определить, по множеству их.
\vs 1Ki 8:6 И внесли священники ковчег завета Господня на место его, в давир храма, во Святое Святых, под крылья херувимов.
\vs 1Ki 8:7 Ибо херувимы простирали крылья над местом ковчега, и покрывали херувимы сверху ковчег и шесты его.
\vs 1Ki 8:8 И выдвинулись шесты так, что головки шестов видны были из святилища пред давиром, но не выказывались наружу; они там и до сего дня.
\vs 1Ki 8:9 В ковчеге ничего не было, кроме двух каменных скрижалей, которые положил туда Моисей на Хориве, когда Господь заключил завет с сынами Израилевыми, по исшествии их из земли Египетской.
\vs 1Ki 8:10 Когда священники вышли из святилища, облако наполнило дом Господень;
\vs 1Ki 8:11 и не могли священники стоять на служении, по причине облака, ибо слава Господня наполнила храм Господень.
\vs 1Ki 8:12 Тогда сказал Соломон: Господь сказал, что Он благоволит обитать во мгле;
\vs 1Ki 8:13 я построил храм в жилище Тебе, место, чтобы пребывать Тебе во веки.
\vs 1Ki 8:14 И обратился царь лицем своим, и благословил все собрание Израильтян; все собрание Израильтян стояло,~---
\vs 1Ki 8:15 и сказал: благословен Господь Бог Израилев, Который сказал Своими устами Давиду, отцу моему, и ныне исполнил рукою Своею! Он говорил:
\vs 1Ki 8:16 <<с того дня, как Я вывел народ Мой Израиля из Египта, Я не избрал города ни в одном из колен Израилевых, чтобы построен был дом, в котором пребывало бы имя Мое; [но избрал Иерусалим для пребывания в нем имени Моего] и избрал Давида, чтобы быть ему над народом Моим Израилем>>.
\vs 1Ki 8:17 У Давида, отца моего, было на сердце построить храм имени Господа Бога Израилева;
\vs 1Ki 8:18 но Господь сказал Давиду, отцу моему: <<у тебя есть на сердце построить храм имени Моему; хорошо, что это у тебя лежит на сердце;
\vs 1Ki 8:19 однако не ты построишь храм, а сын твой, исшедший из чресл твоих, он построит храм имени Моему>>.
\vs 1Ki 8:20 И исполнил Господь слово Свое, которое изрек. Я вступил на место отца моего Давида и сел на престоле Израилевом, как сказал Господь, и построил храм имени Господа Бога Израилева;
\vs 1Ki 8:21 и приготовил там место для ковчега, в котором завет Господа, заключенный Им с отцами нашими, когда Он вывел их из земли Египетской.
\rsbpar\vs 1Ki 8:22 И стал Соломон пред жертвенником Господним впереди всего собрания Израильтян, и воздвиг руки свои к небу,
\vs 1Ki 8:23 и сказал: Господи Боже Израилев! нет подобного Тебе Бога на небесах вверху и на земле внизу; Ты хранишь завет и милость к рабам Твоим, ходящим пред Тобою всем сердцем своим.
\vs 1Ki 8:24 Ты исполнил рабу Твоему Давиду, отцу моему, что говорил ему; что изрек Ты устами Твоими, то в сей день совершил рукою Твоею.
\vs 1Ki 8:25 И ныне, Господи Боже Израилев, исполни рабу Твоему Давиду, отцу моему, то, что говорил Ты ему, сказав: <<не прекратится у тебя пред лицем Моим сидящий на престоле Израилевом, если только сыновья твои будут держаться пути своего, ходя предо Мною так, как ты ходил предо Мною>>.
\vs 1Ki 8:26 И ныне, Боже Израилев, да будет верно слово Твое, которое Ты изрек рабу Твоему Давиду, отцу моему!
\vs 1Ki 8:27 Поистине, Богу ли жить на земле? Небо и небо небес не вмещают Тебя, тем менее сей храм, который я построил [имени Твоему];
\vs 1Ki 8:28 но призри на молитву раба Твоего и на прошение его, Господи Боже мой; услышь воззвание и молитву, которою раб Твой умоляет Тебя ныне.
\vs 1Ki 8:29 Да будут очи Твои отверсты на храм сей день и ночь, на сие место, о котором Ты сказал: <<Мое имя будет там>>; услышь молитву, которою будет молиться раб Твой на месте сем.
\vs 1Ki 8:30 Услышь моление раба Твоего и народа Твоего Израиля, когда они будут молиться на месте сем; услышь на месте обитания Твоего, на небесах, услышь и помилуй.
\vs 1Ki 8:31 Когда кто согрешит против ближнего своего, и потребует от него клятвы, чтобы он поклялся, и для клятвы придут пред жертвенник Твой в храм сей,
\vs 1Ki 8:32 тогда Ты услышь с неба и произведи суд над рабами Твоими, обвини виновного, возложив поступок его на голову его, и оправдай правого, воздав ему по правде его.
\vs 1Ki 8:33 Когда народ Твой Израиль будет поражен неприятелем за то, что согрешил пред Тобою, и когда они обратятся к Тебе, и исповедают имя Твое, и будут просить и умолять Тебя в сем храме,
\vs 1Ki 8:34 тогда Ты услышь с неба и прости грех народа Твоего Израиля, и возврати их в землю, которую Ты дал отцам их.
\vs 1Ki 8:35 Когда заключится небо и не будет дождя за то, что они согрешат пред Тобою, и когда помолятся на месте сем и исповедают имя Твое и обратятся от греха своего, ибо Ты смирил их,
\vs 1Ki 8:36 тогда услышь с неба и прости грех рабов Твоих и народа Твоего Израиля, указав им добрый путь, по которому идти, и пошли дождь на землю Твою, которую Ты дал народу Твоему в наследие.
\vs 1Ki 8:37 Будет ли на земле голод, будет ли моровая язва, будет ли палящий ветер, ржавчина, саранча, червь, неприятель ли будет теснить его в земле его, \bibemph{будет ли} какое бедствие, какая болезнь,~---
\vs 1Ki 8:38 при всякой молитве, при всяком прошении, какое будет от какого-либо человека во всем народе Твоем Израиле, когда они почувствуют бедствие в сердце своем и прострут руки свои к храму сему,
\vs 1Ki 8:39 Ты услышь с неба, с места обитания Твоего, и помилуй; соделай и воздай каждому по путям его, как Ты усмотришь сердце его, ибо Ты один знаешь сердце всех сынов человеческих:
\vs 1Ki 8:40 чтобы они боялись Тебя во все дни, доколе живут на земле, которую Ты дал отцам нашим.
\vs 1Ki 8:41 Если и иноплеменник, который не от Твоего народа Израиля, придет из земли далекой ради имени Твоего,~---
\vs 1Ki 8:42 ибо и они услышат о Твоем имени великом и о Твоей руке сильной и о Твоей мышце простертой,~--- и придет он и помолится у храма сего,
\vs 1Ki 8:43 услышь с неба, с места обитания Твоего, и сделай все, о чем будет взывать к Тебе иноплеменник, чтобы все народы земли знали имя Твое, чтобы боялись Тебя, как народ Твой Израиль, чтобы знали, что именем Твоим называется храм сей, который я построил.
\vs 1Ki 8:44 Когда выйдет народ Твой на войну против врага своего путем, которым Ты пошлешь его, и будет молиться Господу, обратившись к городу, который Ты избрал, и к храму, который я построил имени Твоему,
\vs 1Ki 8:45 тогда услышь с неба молитву их и прошение их и сделай, что потребно для них.
\vs 1Ki 8:46 Когда они согрешат пред Тобою,~--- ибо нет человека, который не грешил бы,~--- и Ты прогневаешься на них и предашь их врагам, и пленившие их отведут их в неприятельскую землю, далекую или близкую;
\vs 1Ki 8:47 и когда они в земле, в которой будут находиться в плену, войдут в себя и обратятся и будут молиться Тебе в земле пленивших их, говоря: <<мы согрешили, сделали беззаконие, мы виновны>>;
\vs 1Ki 8:48 и когда обратятся к Тебе всем сердцем своим и всею душею своею в земле врагов, которые пленили их, и будут молиться Тебе, обратившись к земле своей, которую Ты дал отцам их, к городу, который Ты избрал, и к храму, который я построил имени Твоему,
\vs 1Ki 8:49 тогда услышь с неба, с места обитания Твоего, молитву и прошение их и сделай, что потребно для них;
\vs 1Ki 8:50 и прости народу Твоему, в чем он согрешил пред Тобою, и все проступки его, которые он сделал пред Тобою, и возбуди сострадание к ним в пленивших их, чтобы они были милостивы к ним:
\vs 1Ki 8:51 ибо они Твой народ и Твой удел, который Ты вывел из Египта, из железной печи.
\vs 1Ki 8:52 Да будут [уши Твои и] очи Твои отверсты на молитву раба Твоего и на молитву народа Твоего Израиля, чтобы слышать их всегда, когда они будут призывать Тебя,
\vs 1Ki 8:53 ибо Ты отделил их Себе в удел из всех народов земли, как Ты изрек чрез Моисея, раба Твоего, когда вывел отцов наших из Египта, Владыка Господи!
\rsbpar\vs 1Ki 8:54 Когда Соломон произнес все сие моление и прошение к Господу, тогда встал с колен от жертвенника Господня, \bibemph{руки же} его были распростерты к небу.
\vs 1Ki 8:55 И стоя благословил все собрание Израильтян, громким голосом говоря:
\vs 1Ki 8:56 благословен Господь [Бог], Который дал покой народу Своему Израилю, как говорил! не осталось неисполненным ни одного слова из всех благих слов Его, которые Он изрек чрез раба Своего Моисея;
\vs 1Ki 8:57 да будет с нами Господь Бог наш, как был Он с отцами нашими, да не оставит нас, да не покинет нас,
\vs 1Ki 8:58 наклоняя к Себе сердце наше, чтобы мы ходили по всем путям Его и соблюдали заповеди Его и уставы Его и законы Его, которые Он заповедал отцам нашим;
\vs 1Ki 8:59 и да будут слова сии, которыми я молился [ныне] пред Господом, близки к Господу Богу нашему день и ночь, дабы Он делал, что потребно для раба Своего, и что потребно для народа Своего Израиля, изо дня в день,
\vs 1Ki 8:60 чтобы все народы познали, что Господь есть Бог и нет кроме Его;
\vs 1Ki 8:61 да будет сердце ваше вполне предано Господу Богу нашему, чтобы ходить по уставам Его и соблюдать заповеди Его, как ныне.
\rsbpar\vs 1Ki 8:62 И царь и все Израильтяне с ним принесли жертву Господу.
\vs 1Ki 8:63 И принес Соломон в мирную жертву, которую принес он Господу, двадцать две тысячи крупного скота и сто двадцать тысяч мелкого скота. Так освятили храм Господу царь и все сыны Израилевы.
\vs 1Ki 8:64 В тот же день освятил царь среднюю часть двора, который пред храмом Господним, совершив там всесожжение и хлебное приношение и \bibemph{вознеся} тук мирных жертв, потому что медный жертвенник, который пред Господом, был мал для помещения всесожжения и хлебного приношения и тука мирных жертв.
\vs 1Ki 8:65 И сделал Соломон в это время праздник, и весь Израиль с ним,~--- большое собрание, \bibemph{сошедшееся} от входа в Емаф до реки Египетской пред Господом Богом нашим; [и ели, и пили, и молились пред Господом Богом нашим у построенного храма]~--- семь дней и еще семь дней, четырнадцать дней.
\vs 1Ki 8:66 В восьмой день Соломон отпустил народ. И благословили царя и пошли в шатры свои, радуясь и веселясь в сердце о всем добром, что сделал Господь рабу Своему Давиду и народу Своему Израилю.
\vs 1Ki 9:1 После того, как Соломон кончил строение храма Господня и дома царского и все, что Соломон желал сделать,
\vs 1Ki 9:2 явился Соломону Господь во второй раз, как явился ему в Гаваоне.
\vs 1Ki 9:3 И сказал ему Господь: Я услышал молитву твою и прошение твое, о чем ты просил Меня; [сделал все по молитве твоей]. Я освятил сей храм, который ты построил, чтобы пребывать имени Моему там вовек; и будут очи Мои и сердце Мое там во все дни.
\vs 1Ki 9:4 И если ты будешь ходить пред лицем Моим, как ходил отец твой Давид, в чистоте сердца и в правоте, исполняя все, что Я заповедал тебе, и если будешь хранить уставы Мои и законы Мои,
\vs 1Ki 9:5 то Я поставлю царский престол твой над Израилем вовек, как Я сказал отцу твоему Давиду, говоря: <<не прекратится у тебя сидящий на престоле Израилевом>>.
\vs 1Ki 9:6 Если же вы и сыновья ваши отступите от Меня и не будете соблюдать заповедей Моих и уставов Моих, которые Я дал вам, и пойдете и станете служить иным богам и поклоняться им,
\vs 1Ki 9:7 то Я истреблю Израиля с лица земли, которую Я дал ему, и храм, который Я освятил имени Моему, отвергну от лица Моего, и будет Израиль притчею и посмешищем у всех народов.
\vs 1Ki 9:8 И о храме сем высоком всякий, проходящий мимо его, ужаснется и свистнет, и скажет: <<за что Господь поступил так с сею землею и с сим храмом?>>
\vs 1Ki 9:9 И скажут: <<за то, что они оставили Господа Бога своего, Который вывел отцов их из земли Египетской, и приняли других богов, и поклонялись им и служили им,~--- за это навел на них Господь все сие бедствие>>.
\vs 1Ki 9:10 По окончании двадцати лет, в которые Соломон построил два дома,~--- дом Господень и дом царский,~---
\vs 1Ki 9:11 на что Хирам, царь Тирский, доставлял Соломону дерева кедровые и дерева кипарисовые и золото, по его желанию,~--- царь Соломон дал Хираму двадцать городов в земле Галилейской.
\vs 1Ki 9:12 И вышел Хирам из Тира посмотреть города, которые дал ему Соломон, и они не понравились ему.
\vs 1Ki 9:13 И сказал он: что это за города, которые ты, брат мой, дал мне? И назвал их землею Кавул, \bibemph{как называются они} до сего дня.
\vs 1Ki 9:14 И послал Хирам царю сто двадцать талантов золота.
\vs 1Ki 9:15 Вот распоряжение о подати, которую наложил царь Соломон, чтобы построить храм Господень и дом свой, и Милло, и стену Иерусалимскую, Гацор, и Мегиддо, и Газер.
\rsbpar\vs 1Ki 9:16 Фараон, царь Египетский, пришел и взял Газер, и сжег его огнем, и Хананеев, живших в городе, побил, и отдал его в приданое дочери своей, жене Соломоновой.
\vs 1Ki 9:17 И построил Соломон Газер и нижний Бефорон,
\vs 1Ki 9:18 и Ваалаф и Фадмор в пустыне,
\vs 1Ki 9:19 и все города для запасов, которые были у Соломона, и города для колесниц, и города для конницы и все то, что Соломон хотел построить в Иерусалиме и на Ливане и во всей земле своего владения.
\vs 1Ki 9:20 Весь народ, оставшийся от Аморреев, Хеттеев, Ферезеев, [Хананеев,] Евеев, Иевусеев и [Гергесеев], которые были не из сынов Израилевых,
\vs 1Ki 9:21 детей их, оставшихся после них на земле, которых сыны Израилевы не могли истребить, Соломон сделал оброчными работниками до сего дня.
\vs 1Ki 9:22 Сынов же Израилевых Соломон не делал работниками, но они были его воинами, его слугами, его вельможами, его военачальниками и вождями его колесниц и его всадников.
\vs 1Ki 9:23 Вот главные приставники над работами Соломоновыми: управлявших народом, который производил работы, было пятьсот пятьдесят.
\vs 1Ki 9:24 Дочь фараонова перешла из города Давидова в свой дом, который построил для нее Соломон; потом построил он Милло.
\vs 1Ki 9:25 И приносил Соломон три раза в год всесожжения и мирные жертвы на жертвеннике, который он построил Господу, и курение на нем совершал пред Господом. И окончил он \bibemph{строение} дома.
\vs 1Ki 9:26 Царь Соломон также сделал корабль в Ецион-Гавере, что при Елафе, на берегу Чермного моря, в земле Идумейской.
\vs 1Ki 9:27 И послал Хирам на корабле своих подданных корабельщиков, знающих море, с подданными Соломоновыми;
\vs 1Ki 9:28 и отправились они в Офир, и взяли оттуда золота четыреста двадцать талантов, и привезли царю Соломону.
\vs 1Ki 10:1 Царица Савская, услышав о славе Соломона во имя Господа, пришла испытать его загадками.
\vs 1Ki 10:2 И пришла она в Иерусалим с весьма большим богатством: верблюды навьючены \bibemph{были} благовониями и великим множеством золота и драгоценными камнями; и пришла к Соломону и беседовала с ним обо всем, что было у нее на сердце.
\vs 1Ki 10:3 И объяснил ей Соломон все слова ее, и не было ничего незнакомого царю, чего бы он не изъяснил ей.
\vs 1Ki 10:4 И увидела царица Савская всю мудрость Соломона и дом, который он построил,
\vs 1Ki 10:5 и пищу за столом его, и жилище рабов его, и стройность слуг его, и одежду их, и виночерпиев его, и всесожжения его, которые он приносил в храме Господнем. И не могла она более удержаться
\vs 1Ki 10:6 и сказала царю: верно то, что я слышала в земле своей о делах твоих и о мудрости твоей;
\vs 1Ki 10:7 но я не верила словам, доколе не пришла, и не увидели глаза мои: и вот, мне и в половину не сказано; мудрости и богатства у тебя больше, нежели как я слышала.
\vs 1Ki 10:8 Блаженны люди твои и блаженны сии слуги твои, которые всегда предстоят пред тобою и слышат мудрость твою!
\vs 1Ki 10:9 Да будет благословен Господь Бог твой, Который благоволил посадить тебя на престол Израилев! Господь, по вечной любви Своей к Израилю, поставил тебя царем, творить суд и правду.
\vs 1Ki 10:10 И подарила она царю сто двадцать талантов золота и великое множество благовоний и драгоценные камни; никогда еще не приходило такого множества благовоний, какое подарила царица Савская царю Соломону.
\vs 1Ki 10:11 И корабль Хирамов, который привозил золото из Офира, привез из Офира великое множество красного дерева и драгоценных камней.
\vs 1Ki 10:12 И сделал царь из сего красного дерева перила для храма Господня и для дома царского, и гусли и псалтири для певцов; никогда не приходило столько красного дерева и не видано было до сего дня.
\vs 1Ki 10:13 И царь Соломон дал царице Савской все, чего она желала и чего просила, сверх того, что подарил ей царь Соломон своими руками. И отправилась она обратно в свою землю, она и все слуги ее.
\rsbpar\vs 1Ki 10:14 В золоте, которое приходило Соломону в каждый год, весу было шестьсот шестьдесят шесть талантов золотых,
\vs 1Ki 10:15 сверх того, что \bibemph{получаемо было} от разносчиков товара и от торговли купцов, и от всех царей Аравийских и от областных начальников.
\vs 1Ki 10:16 И сделал царь Соломон двести больших щитов из кованого золота, по шестисот \bibemph{сиклей} пошло на каждый щит;
\vs 1Ki 10:17 и триста меньших щитов из кованого золота, по три мины золота пошло на каждый щит; и поставил их царь в доме из Ливанского дерева.
\vs 1Ki 10:18 И сделал царь большой престол из слоновой кости и обложил его чистым золотом;
\vs 1Ki 10:19 к престолу было шесть ступеней; верх сзади у престола был круглый, и были с обеих сторон у места сиденья локотники, и два льва стояли у локотников;
\vs 1Ki 10:20 и еще двенадцать львов стояли там на шести ступенях по обе стороны. Подобного сему не бывало ни в одном царстве.
\vs 1Ki 10:21 И все сосуды для питья у царя Соломона \bibemph{были} золотые, и все сосуды в доме из Ливанского дерева были из чистого золота; из серебра ничего не было, потому что серебро во дни Соломоновы считалось ни за что;
\vs 1Ki 10:22 ибо у царя был на море Фарсисский корабль с кораблем Хирамовым; в три года раз приходил Фарсисский корабль, привозивший золото и серебро, и слоновую кость, и обезьян, и павлинов.
\rsbpar\vs 1Ki 10:23 Царь Соломон превосходил всех царей земли богатством и мудростью.
\vs 1Ki 10:24 И все [цари] на земле искали видеть Соломона, чтобы послушать мудрости его, которую вложил Бог в сердце его.
\vs 1Ki 10:25 И они подносили ему, каждый от себя, в дар: сосуды серебряные и сосуды золотые, и одежды, и оружие, и благовония, коней и мулов, каждый год.
\vs 1Ki 10:26 И набрал Соломон колесниц и всадников; у него было тысяча четыреста колесниц\fns{В греческом переводе: сорок тысяч коней колесничных.} и двенадцать тысяч всадников; и разместил он их по колесничным городам и при царе в Иерусалиме. [И господствовал он над всеми морями от реки до земли Филистимской и до пределов Египта.]
\vs 1Ki 10:27 И сделал царь серебро в Иерусалиме равноценным с простыми камнями, а кедры, по их множеству, сделал равноценными с сикоморами, \bibemph{растущими} на низких местах.
\vs 1Ki 10:28 Коней же царю Соломону приводили из Египта и из Кувы; царские купцы покупали их из Кувы за деньги.
\vs 1Ki 10:29 Колесница из Египта получаема и доставляема была за шестьсот \bibemph{сиклей} серебра, а конь за сто пятьдесят. Таким же образом они руками своими доставляли \bibemph{все это} царям Хеттейским и царям Арамейским.
\vs 1Ki 11:1 И полюбил царь Соломон многих чужестранных женщин, кроме дочери фараоновой, Моавитянок, Аммонитянок, Идумеянок, Сидонянок, Хеттеянок,
\vs 1Ki 11:2 из тех народов, о которых Господь сказал сынам Израилевым: <<не входите к ним, и они пусть не входят к вам, чтобы они не склонили сердца вашего к своим богам>>; к ним прилепился Соломон любовью.
\vs 1Ki 11:3 И было у него семьсот жен и триста наложниц; и развратили жены его сердце его.
\vs 1Ki 11:4 Во время старости Соломона жены его склонили сердце его к иным богам, и сердце его не было вполне предано Господу Богу своему, как сердце Давида, отца его.
\vs 1Ki 11:5 И стал Соломон служить Астарте, божеству Сидонскому, и Милхому, мерзости Аммонитской.
\vs 1Ki 11:6 И делал Соломон неугодное пред очами Господа и не вполне последовал Господу, как Давид, отец его.
\vs 1Ki 11:7 Тогда построил Соломон капище Хамосу, мерзости Моавитской, на горе, которая пред Иерусалимом, и Молоху, мерзости Аммонитской.
\vs 1Ki 11:8 Так сделал он для всех своих чужестранных жен, которые кадили и приносили жертвы своим богам.
\vs 1Ki 11:9 И разгневался Господь на Соломона за то, что он уклонил сердце свое от Господа Бога Израилева, Который два раза являлся ему
\vs 1Ki 11:10 и заповедал ему, чтобы он не следовал иным богам; но он не исполнил того, что заповедал ему Господь [Бог].
\vs 1Ki 11:11 И сказал Господь Соломону: за то, что так у тебя делается, и ты не сохранил завета Моего и уставов Моих, которые Я заповедал тебе, Я отторгну от тебя царство и отдам его рабу твоему;
\vs 1Ki 11:12 но во дни твои Я не сделаю сего ради Давида, отца твоего; из руки сына твоего исторгну его;
\vs 1Ki 11:13 и не все царство исторгну; одно колено дам сыну твоему ради Давида, раба Моего, и ради Иерусалима, который Я избрал.
\rsbpar\vs 1Ki 11:14 И воздвиг Господь противника на Соломона, Адера Идумеянина, из царского Идумейского рода.
\vs 1Ki 11:15 Когда Давид был в Идумее, и военачальник Иоав пришел для погребения убитых и избил весь мужеский пол в Идумее,~---
\vs 1Ki 11:16 ибо шесть месяцев прожил там Иоав и все Израильтяне, доколе не истребили всего мужеского пола в Идумее,~---
\vs 1Ki 11:17 тогда сей Адер убежал в Египет и с ним несколько Идумеян, служивших при отце его; Адер \bibemph{был тогда} малым ребенком.
\vs 1Ki 11:18 Отправившись из Мадиама, они пришли в Фаран и взяли с собою людей из Фарана и пришли в Египет к фараону, царю Египетскому. [Адер вошел к фараону, и] он дал ему дом, и назначил ему содержание, и дал ему землю.
\vs 1Ki 11:19 Адер снискал у фараона большую милость, так что он дал ему в жену сестру своей жены, сестру царицы Тахпенесы.
\vs 1Ki 11:20 И родила ему сестра Тахпенесы сына Генувата. Тахпенеса воспитывала его в доме фараоновом; и жил Генуват в доме фараоновом вместе с сыновьями фараоновыми.
\vs 1Ki 11:21 Когда Адер услышал, что Давид почил с отцами своими и что военачальник Иоав умер, то сказал фараону: отпусти меня, я пойду в свою землю.
\vs 1Ki 11:22 И сказал ему фараон: разве ты нуждаешься в чем у меня, что хочешь идти в свою землю? Он отвечал: нет, но отпусти меня. [И возвратился Адер в землю свою.]
\vs 1Ki 11:23 И воздвиг Бог против Соломона еще противника, Разона, сына Елиады, который убежал от государя своего Адраазара, царя Сувского,
\vs 1Ki 11:24 и, собрав около себя людей, сделался начальником шайки, после того, как Давид разбил \bibemph{Адраазара}; и пошли они в Дамаск, и водворились там, и владычествовали в Дамаске.
\vs 1Ki 11:25 И был он противником Израиля во все дни Соломона. Кроме зла, \bibemph{причиненного} Адером, он всегда вредил Израилю и сделался царем Сирии.
\vs 1Ki 11:26 И Иеровоам, сын Наватов, Ефремлянин из Цареды,~--- имя матери его вдовы: Церуа,~--- раб Соломонов, поднял руку на царя.
\vs 1Ki 11:27 И вот обстоятельство, по которому он поднял руку на царя: Соломон строил Милло, починивал повреждения в городе Давида, отца своего.
\vs 1Ki 11:28 Иеровоам был человек мужественный. Соломон, заметив, что этот молодой человек умеет делать дело, поставил его смотрителем над оброчными из дома Иосифова.
\vs 1Ki 11:29 В то время случилось Иеровоаму выйти из Иерусалима; и встретил его на дороге пророк Ахия Силомлянин, и на нем была новая одежда. На поле их было только двое.
\vs 1Ki 11:30 И взял Ахия новую одежду, которая была на нем, и разодрал ее на двенадцать частей,
\vs 1Ki 11:31 и сказал Иеровоаму: возьми себе десять частей, ибо так говорит Господь Бог Израилев: вот, Я исторгаю царство из руки Соломоновой и даю тебе десять колен,
\vs 1Ki 11:32 а одно колено\fns{В греческом переводе: два колена.} останется за ним ради раба Моего Давида и ради города Иерусалима, который Я избрал из всех колен Израилевых.
\vs 1Ki 11:33 Это за то, что они оставили Меня и стали поклоняться Астарте, божеству Сидонскому, и Хамосу, богу Моавитскому, и Милхому, богу Аммонитскому, и не пошли путями Моими, чтобы делать угодное пред очами Моими и \bibemph{соблюдать} уставы Мои и заповеди Мои, подобно Давиду, отцу его.
\vs 1Ki 11:34 Я не беру всего царства из руки его, но Я оставляю его владыкою на все дни жизни его ради Давида, раба Моего, которого Я избрал, который соблюдал заповеди Мои и уставы Мои;
\vs 1Ki 11:35 но возьму царство из руки сына его и дам тебе из него десять колен;
\vs 1Ki 11:36 а сыну его дам одно колено, дабы оставался светильник Давида, раба Моего, во все дни пред лицем Моим, в городе Иерусалиме, который Я избрал Себе для пребывания там имени Моего.
\vs 1Ki 11:37 Тебя Я избираю, и ты будешь владычествовать над всем, чего пожелает душа твоя, и будешь царем над Израилем;
\vs 1Ki 11:38 и если будешь соблюдать все, что Я заповедую тебе, и будешь ходить путями Моими и делать угодное пред очами Моими, соблюдая уставы Мои и заповеди Мои, как делал раб Мой Давид, то Я буду с тобою и устрою тебе дом твердый, как Я устроил Давиду, и отдам тебе Израиля;
\vs 1Ki 11:39 и смирю Я род Давидов за сие, но не на все дни.
\vs 1Ki 11:40 Соломон же хотел умертвить Иеровоама; но Иеровоам встал и убежал в Египет к Сусакиму, царю Египетскому, и жил в Египте до смерти Соломоновой.
\rsbpar\vs 1Ki 11:41 Прочие события Соломоновы и все, что он делал, и мудрость его описаны в книге дел Соломоновых.
\vs 1Ki 11:42 Времени царствования Соломонова в Иерусалиме над всем Израилем \bibemph{было} сорок лет.
\vs 1Ki 11:43 И почил Соломон с отцами своими и погребен был в городе Давида, отца своего, и воцарился вместо него сын его Ровоам.
\vs 1Ki 12:1 И пошел Ровоам в Сихем, ибо в Сихем пришли все Израильтяне, чтобы воцарить его.
\vs 1Ki 12:2 И услышал о том Иеровоам, сын Наватов, когда находился еще в Египте, куда убежал от царя Соломона, и возвратился Иеровоам из Египта;
\vs 1Ki 12:3 и послали за ним и призвали его. Тогда Иеровоам и все собрание Израильтян пришли и говорили [царю] Ровоаму и сказали:
\vs 1Ki 12:4 отец твой наложил на нас тяжкое иго, ты же облегчи нам жестокую работу отца твоего и тяжкое иго, которое он наложил на нас, и тогда мы будем служить тебе.
\vs 1Ki 12:5 И сказал он им: пойдите и чрез три дня опять придите ко мне. И пошел народ.
\vs 1Ki 12:6 Царь Ровоам советовался со старцами, которые предстояли пред Соломоном, отцом его, при жизни его, и говорил: как посоветуете вы мне отвечать сему народу?
\vs 1Ki 12:7 Они говорили ему и сказали: если ты на сей день будешь слугою народу сему и услужишь ему, и удовлетворишь им и будешь говорить им ласково, то они будут твоими рабами на все дни.
\vs 1Ki 12:8 Но он пренебрег совет старцев, что они советовали ему, и советовался с молодыми людьми, которые выросли вместе с ним и которые предстояли пред ним,
\vs 1Ki 12:9 и сказал им: что вы посоветуете мне отвечать народу сему, который говорил мне и сказал: <<облегчи иго, которое наложил на нас отец твой>>?
\vs 1Ki 12:10 И говорили ему молодые люди, которые выросли вместе с ним, и сказали: так скажи народу сему, который говорил тебе и сказал: <<отец твой наложил на нас тяжкое иго, ты же облегчи нас>>; так скажи им: <<мой мизинец толще чресл отца моего;
\vs 1Ki 12:11 итак, если отец мой обременял вас тяжким игом, то я увеличу иго ваше; отец мой наказывал вас бичами, а я буду наказывать вас скорпионами>>.
\vs 1Ki 12:12 Иеровоам и весь народ пришли к Ровоаму на третий день, как приказал царь, сказав: придите ко мне на третий день.
\vs 1Ki 12:13 И отвечал царь народу сурово и пренебрег совет старцев, что они советовали ему;
\vs 1Ki 12:14 и говорил он по совету молодых людей и сказал: отец мой наложил на вас тяжкое иго, а я увеличу иго ваше; отец мой наказывал вас бичами, а я буду наказывать вас скорпионами.
\vs 1Ki 12:15 И не послушал царь народа, ибо так суждено было Господом, чтобы исполнилось слово Его, которое изрек Господь чрез Ахию Силомлянина Иеровоаму, сыну Наватову.
\vs 1Ki 12:16 И увидели все Израильтяне, что царь не послушал их. И отвечал народ царю и сказал: какая нам часть в Давиде? нет нам доли в сыне Иессеевом; по шатрам своим, Израиль! теперь знай свой дом, Давид! И разошелся Израиль по шатрам своим.
\vs 1Ki 12:17 Только над сынами Израилевыми, жившими в городах Иудиных, царствовал Ровоам.
\vs 1Ki 12:18 И послал царь Ровоам Адонирама, начальника над податью; но все Израильтяне забросали его каменьями, и он умер; царь же Ровоам поспешно взошел на колесницу, чтоб убежать в Иерусалим.
\vs 1Ki 12:19 И отложился Израиль от дома Давидова до сего дня.
\vs 1Ki 12:20 Когда услышали все Израильтяне, что Иеровоам возвратился [из Египта], то послали и призвали его в собрание, и воцарили его над всеми Израильтянами. За домом Давидовым не осталось никого, кроме колена Иудина [и Вениаминова].
\vs 1Ki 12:21 Ровоам, прибыв в Иерусалим, собрал из всего дома Иудина и из колена Вениаминова сто восемьдесят тысяч отборных воинов, дабы воевать с домом Израилевым для того, чтобы возвратить царство Ровоаму, сыну Соломонову.
\rsbpar\vs 1Ki 12:22 И было слово Божие к Самею, человеку Божию, и сказано:
\vs 1Ki 12:23 скажи Ровоаму, сыну Соломонову, царю Иудейскому, и всему дому Иудину и Вениаминову и прочему народу:
\vs 1Ki 12:24 так говорит Господь: не ходите и не начинайте войны с братьями вашими, сынами Израилевыми; возвратитесь каждый в дом свой, ибо от Меня это было. И послушались они слова Господня и пошли назад по слову Господню.
\rsbpar\vs 1Ki 12:25 И обстроил Иеровоам Сихем на горе Ефремовой и поселился в нем; оттуда пошел и построил Пенуил.
\vs 1Ki 12:26 И говорил Иеровоам в сердце своем: царство может опять перейти к дому Давидову;
\vs 1Ki 12:27 если народ сей будет ходить в Иерусалим для жертвоприношения в доме Господнем, то сердце народа сего обратится к государю своему, к Ровоаму, царю Иудейскому, и убьют они меня и возвратятся к Ровоаму, царю Иудейскому.
\vs 1Ki 12:28 И посоветовавшись царь сделал двух золотых тельцов и сказал [народу]: не нужно вам ходить в Иерусалим; вот боги твои, Израиль, которые вывели тебя из земли Египетской.
\vs 1Ki 12:29 И поставил одного в Вефиле, а другого в Дане.
\vs 1Ki 12:30 И повело это ко греху, ибо народ стал ходить к одному \bibemph{из них}, даже в Дан, [и оставили храм Господень].
\vs 1Ki 12:31 И построил он капище на высоте и поставил из народа священников, которые не были из сынов Левииных.
\vs 1Ki 12:32 И установил Иеровоам праздник в восьмой месяц, в пятнадцатый день месяца, подобный тому празднику, какой был в Иудее, и приносил жертвы на жертвеннике; то же сделал он в Вефиле, чтобы приносить жертву тельцам, которых сделал. И поставил в Вефиле священников высот, которые устроил,
\vs 1Ki 12:33 и принес жертвы на жертвеннике, который он сделал в Вефиле, в пятнадцатый день восьмого месяца, месяца, который он произвольно назначил; и установил праздник для сынов Израилевых, и подошел к жертвеннику, чтобы совершить курение.
\vs 1Ki 13:1 И вот, человек Божий пришел из Иудеи по слову Господню в Вефиль, в то время, как Иеровоам стоял у жертвенника, чтобы совершить курение.
\vs 1Ki 13:2 И произнес к жертвеннику слово Господне и сказал: жертвенник, жертвенник! так говорит Господь: вот, родится сын дому Давидову, имя ему Иосия, и принесет на тебе в жертву священников высот, совершающих на тебе курение, и человеческие кости сожжет на тебе.
\vs 1Ki 13:3 И дал в тот день знамение, сказав: вот знамение того, что это изрек Господь: вот, этот жертвенник распадется, и пепел, который на нем, рассыплется.
\vs 1Ki 13:4 Когда царь услышал слово человека Божия, произнесенное к жертвеннику в Вефиле, то простер Иеровоам руку свою от жертвенника, говоря: возьмите его. И одеревенела рука его, которую он простер на него, и не мог он поворотить ее к себе.
\vs 1Ki 13:5 И жертвенник распался, и пепел с жертвенника рассыпался, по знамению, которое дал человек Божий словом Господним.
\vs 1Ki 13:6 И сказал царь [Иеровоам] человеку Божию: умилостиви лице Господа Бога твоего и помолись обо мне, чтобы рука моя могла поворотиться ко мне. И умилостивил человек Божий лице Господа, и рука царя поворотилась к нему и стала, как прежде.
\vs 1Ki 13:7 И сказал царь человеку Божию: зайди со мною в дом и подкрепи себя пищею, и я дам тебе подарок.
\vs 1Ki 13:8 Но человек Божий сказал царю: хотя бы ты давал мне полдома твоего, я не пойду с тобою и не буду есть хлеба и не буду пить воды в этом месте,
\vs 1Ki 13:9 ибо так заповедано мне словом Господним: <<не ешь там хлеба и не пей воды и не возвращайся тою дорогою, которою ты шел>>.
\vs 1Ki 13:10 И пошел он другою дорогою и не пошел обратно тою дорогою, которою пришел в Вефиль.
\rsbpar\vs 1Ki 13:11 В Вефиле жил один пророк-старец. Сын его пришел и рассказал ему все, что сделал сегодня человек Божий в Вефиле; и слова, какие он говорил царю, пересказали \bibemph{сыновья} отцу своему.
\vs 1Ki 13:12 И спросил их отец их: какою дорогою он пошел? И показали сыновья его, какою дорогою пошел человек Божий, приходивший из Иудеи.
\vs 1Ki 13:13 И сказал он сыновьям своим: оседлайте мне осла. И оседлали ему осла, и он сел на него.
\vs 1Ki 13:14 И поехал за человеком Божиим, и нашел его сидящего под дубом, и сказал ему: ты ли человек Божий, пришедший из Иудеи? И сказал тот: я.
\vs 1Ki 13:15 И сказал ему: зайди ко мне в дом и поешь хлеба.
\vs 1Ki 13:16 Тот сказал: я не могу возвратиться с тобою и пойти к тебе; не буду есть хлеба и не буду пить у тебя воды в сем месте,
\vs 1Ki 13:17 ибо словом Господним сказано мне: <<не ешь хлеба и не пей там воды и не возвращайся тою дорогою, которою ты шел>>.
\vs 1Ki 13:18 И сказал он ему: и я пророк такой же, как ты, и Ангел говорил мне словом Господним, и сказал: <<вороти его к себе в дом; пусть поест он хлеба и напьется воды>>.~--- Он солгал ему.
\vs 1Ki 13:19 И тот воротился с ним, и поел хлеба в его доме, и напился воды.
\vs 1Ki 13:20 Когда они еще сидели за столом, слово Господне было к пророку, воротившему его.
\vs 1Ki 13:21 И произнес он к человеку Божию, пришедшему из Иудеи, и сказал: так говорит Господь: за то, что ты не повиновался устам Господа и не соблюл повеления, которое заповедал тебе Господь Бог твой,
\vs 1Ki 13:22 но воротился, ел хлеб и пил воду в том месте, о котором Он сказал тебе: <<не ешь хлеба и не пей воды>>, тело твое не войдет в гробницу отцов твоих.
\vs 1Ki 13:23 После того, как тот поел хлеба и напился, он оседлал осла для пророка, которого он воротил.
\vs 1Ki 13:24 И отправился тот. И встретил его на дороге лев и умертвил его. И лежало тело его, брошенное на дороге; осел же стоял подле него, и лев стоял подле тела.
\vs 1Ki 13:25 И вот, проходившие мимо люди увидели тело, брошенное на дороге, и льва, стоящего подле тела, и пошли и рассказали в городе, в котором жил пророк-старец.
\vs 1Ki 13:26 Пророк, воротивший его с дороги, услышав \bibemph{это}, сказал: это тот человек Божий, который не повиновался устам Господа; Господь предал его льву, который изломал его и умертвил его, по слову Господа, которое Он изрек ему.
\vs 1Ki 13:27 И сказал сыновьям своим: оседлайте мне осла. И оседлали они.
\vs 1Ki 13:28 Он отправился и нашел тело его, брошенное на дороге; осел же и лев стояли подле тела; лев не съел тела и не изломал осла.
\vs 1Ki 13:29 И поднял пророк тело человека Божия, и положил его на осла, и повез его обратно. И пошел пророк-старец в город \bibemph{свой}, чтобы оплакать и похоронить его.
\vs 1Ki 13:30 И положил тело его в своей гробнице и плакал по нем: увы, брат мой!
\vs 1Ki 13:31 После погребения его он сказал сыновьям своим: когда я умру, похороните меня в гробнице, в которой погребен человек Божий; подле костей его положите кости мои;
\vs 1Ki 13:32 ибо сбудется слово, которое он по повелению Господню произнес о жертвеннике в Вефиле и о всех капищах на высотах, в городах Самарийских.
\vs 1Ki 13:33 И после сего события Иеровоам не сошел со своей худой дороги, но продолжал ставить из народа священников высот; кто хотел, того он посвящал, и тот становился священником высот.
\vs 1Ki 13:34 Это вело дом Иеровоамов ко греху и к погибели и к истреблению его с лица земли.
\vs 1Ki 14:1 В то время заболел Авия, сын Иеровоамов.
\vs 1Ki 14:2 И сказал Иеровоам жене своей: встань и переоденься, чтобы не узнали, что ты жена Иеровоамова, и пойди в Силом. Там есть пророк Ахия; он предсказал мне, что я буду царем сего народа.
\vs 1Ki 14:3 И возьми с собою [для человека Божия] десять хлебов, и лепешек, и кувшин меду, и пойди к нему: он скажет тебе, что будет с отроком.
\vs 1Ki 14:4 Жена Иеровоама так и сделала: встала, пошла в Силом и пришла в дом Ахии. Ахия уже не мог видеть, ибо глаза его сделались неподвижны от старости.
\vs 1Ki 14:5 И сказал Господь Ахии: вот, идет жена Иеровоамова спросить тебя о сыне своем, ибо он болен; так и так говори ей; она придет переодетая.
\vs 1Ki 14:6 Ахия, услышав шорох от ног ее, когда она вошла в дверь, сказал: войди, жена Иеровоамова; для чего было тебе переодеваться? Я грозный посланник к тебе.
\vs 1Ki 14:7 Пойди, скажи Иеровоаму: так говорит Господь Бог Израилев: Я возвысил тебя из среды простого народа и поставил вождем народа Моего Израиля,
\vs 1Ki 14:8 и отторг царство от дома Давидова и дал его тебе; а ты не таков, как раб Мой Давид, который соблюдал заповеди Мои и который последовал Мне всем сердцем своим, делая только угодное пред очами Моими;
\vs 1Ki 14:9 ты поступал хуже всех, которые были прежде тебя, и пошел, и сделал себе иных богов и истуканов, чтобы раздражить Меня, Меня же отбросил назад;
\vs 1Ki 14:10 за это Я наведу беды на дом Иеровоамов и истреблю у Иеровоама \bibemph{до} мочащегося к стене, заключенного и оставшегося в Израиле, и вымету дом Иеровоамов, как выметают сор, дочиста;
\vs 1Ki 14:11 кто умрет у Иеровоама в городе, того съедят псы, а кто умрет на поле, того склюют птицы небесные; так Господь сказал.
\vs 1Ki 14:12 Встань и иди в дом твой; и как скоро нога твоя ступит в город, умрет дитя;
\vs 1Ki 14:13 и оплачут его все Израильтяне и похоронят его, ибо он один у Иеровоама войдет в гробницу, так как в нем, из дома Иеровоамова, нашлось нечто доброе пред Господом Богом Израилевым.
\vs 1Ki 14:14 И восставит Себе Господь над Израилем царя, который истребит дом Иеровоамов в тот день; и что? даже теперь.
\vs 1Ki 14:15 И поразит Господь Израиля, и \bibemph{будет он}, как тростник, колеблемый в воде, и извергнет Израильтян из этой доброй земли, которую дал отцам их, и развеет их за реку, за то, что они сделали у себя идолов, раздражая Господа;
\vs 1Ki 14:16 и предаст [Господь] Израиля за грехи Иеровоама, которые он сам сделал и которыми ввел в грех Израиля.
\vs 1Ki 14:17 И встала жена Иеровоамова, и пошла, и пришла в Фирцу; и лишь только переступила чрез порог дома, дитя умерло.
\vs 1Ki 14:18 И похоронили его, и оплакали его все Израильтяне, по слову Господа, которое Он изрек чрез раба Своего Ахию пророка.
\rsbpar\vs 1Ki 14:19 Прочие дела Иеровоама, как он воевал и как царствовал, описаны в летописи царей Израильских.
\vs 1Ki 14:20 Времени царствования Иеровоамова было двадцать два года; и почил он с отцами своими, и воцарился Нават, сын его, вместо него.
\rsbpar\vs 1Ki 14:21 Ровоам, сын Соломонов, царствовал в Иудее. Сорок один год было Ровоаму, когда он воцарился, и семнадцать лет царствовал в Иерусалиме, в городе, который избрал Господь из всех колен Израилевых, чтобы пребывало там имя Его. Имя матери его Наама Аммонитянка.
\vs 1Ki 14:22 И делал Иуда неугодное пред очами Господа, и раздражали Его более всего того, что сделали отцы их своими грехами, какими они грешили.
\vs 1Ki 14:23 И устроили они у себя высоты и статуи и капища на всяком высоком холме и под всяким тенистым деревом.
\vs 1Ki 14:24 И блудники были также в этой земле и делали все мерзости тех народов, которых Господь прогнал от лица сынов Израилевых.
\vs 1Ki 14:25 На пятом году царствования Ровоамова, Сусаким, царь Египетский, вышел против Иерусалима
\vs 1Ki 14:26 и взял сокровища дома Господня и сокровища дома царского [и золотые щиты, которые взял Давид от рабов Адраазара, царя Сувского, и внес в Иерусалим]. Всё взял; взял и все золотые щиты, которые сделал Соломон.
\vs 1Ki 14:27 И сделал царь Ровоам вместо них медные щиты и отдал их на руки начальникам телохранителей, которые охраняли вход в дом царя.
\vs 1Ki 14:28 Когда царь выходил в дом Господень, телохранители несли их, и потом опять относили их в палату телохранителей.
\vs 1Ki 14:29 Прочее о Ровоаме и обо всем, что он делал, описано в летописи царей Иудейских.
\vs 1Ki 14:30 Между Ровоамом и Иеровоамом была война во все дни \bibemph{жизни их}.
\vs 1Ki 14:31 И почил Ровоам с отцами своими и погребен с отцами своими в городе Давидовом. Имя матери его Наама Аммонитянка. И воцарился Авия, сын его, вместо него.
\vs 1Ki 15:1 В восемнадцатый год царствования Иеровоама, сына Наватова, Авия воцарился над Иудеями.
\vs 1Ki 15:2 Три года он царствовал в Иерусалиме; имя матери его Мааха, дочь Авессалома.
\vs 1Ki 15:3 Он ходил во всех грехах отца своего, которые тот делал прежде него, и сердце его не было предано Господу Богу его, как сердце Давида, отца его.
\vs 1Ki 15:4 Но ради Давида Господь Бог его дал ему светильник в Иерусалиме, восставив по нем сына его и утвердив Иерусалим,
\vs 1Ki 15:5 потому что Давид делал угодное пред очами Господа и не отступал от всего того, что Он заповедал ему, во все дни жизни своей, кроме поступка с Уриею Хеттеянином.
\vs 1Ki 15:6 И война была между Ровоамом и Иеровоамом во все дни жизни их.
\rsbpar\vs 1Ki 15:7 Прочие дела Авии, всё, что он сделал, описано в летописи царей Иудейских. И была война между Авиею и Иеровоамом.
\vs 1Ki 15:8 И почил Авия с отцами своими, и похоронили его в городе Давидовом. И воцарился Аса, сын его, вместо него.
\rsbpar\vs 1Ki 15:9 В двадцатый год \bibemph{царствования} Иеровоама, царя Израильского, воцарился Аса над Иудеями
\vs 1Ki 15:10 и сорок один год царствовал в Иерусалиме; имя матери его Ан\acc{а}, дочь Авессалома.
\vs 1Ki 15:11 Аса делал угодное пред очами Господа, как Давид, отец его.
\vs 1Ki 15:12 Он изгнал блудников из земли и отверг всех идолов, которых сделали отцы его,
\vs 1Ki 15:13 и даже мать свою Ан\acc{у} лишил звания царицы за то, что она сделала истукан Астарты; и изрубил Аса истукан ее и сжег у потока Кедрона.
\vs 1Ki 15:14 Высоты же не были уничтожены. Но сердце Асы было предано Господу во все дни его.
\vs 1Ki 15:15 И внес он в дом Господень вещи, посвященные отцом его, и вещи, посвященные им: серебро и золото и сосуды.
\vs 1Ki 15:16 И война была между Асою и Ваасою, царем Израильским, во все дни их.
\vs 1Ki 15:17 И вышел Вааса, царь Израильский, против Иудеи и начал строить Раму, чтобы никто не выходил и не уходил к Асе, царю Иудейскому.
\vs 1Ki 15:18 И взял Аса все серебро и золото, остававшееся в сокровищницах дома Господня и в сокровищницах дома царского, и дал его в руки слуг своих, и послал их царь Аса к Венададу, сыну Тавримона, сына Хезионова, царю Сирийскому, жившему в Дамаске, и сказал:
\vs 1Ki 15:19 союз да будет между мною и между тобою, \bibemph{как был} между отцом моим и между отцом твоим; вот, я посылаю тебе в дар серебро и золото; расторгни союз твой с Ваасою, царем Израильским, чтобы он отошел от меня.
\vs 1Ki 15:20 И послушался Венадад царя Асы, и послал военачальников своих против городов Израильских, и поразил Аин и Дан и Авел-Беф-Мааху и весь Киннероф, по всей земле Неффалима.
\vs 1Ki 15:21 Услышав \bibemph{о сем}, Вааса перестал строить Раму и возвратился в Фирцу.
\vs 1Ki 15:22 Царь же Аса созвал всех Иудеев, никого не исключая, и вынесли они из Рамы камни и дерева, которые Вааса употреблял для строения. И выстроил из них царь Аса Гиву Вениаминову и Мицпу.
\rsbpar\vs 1Ki 15:23 Все прочие дела Асы и все подвиги его, и всё, что он сделал, и города, которые он построил, описаны в летописи царей Иудейских, кроме того, что в старости своей он был болен ногами.
\vs 1Ki 15:24 И почил Аса с отцами своими и погребен с отцами своими в городе Давида, отца своего. И воцарился Иосафат, сын его, вместо него.
\rsbpar\vs 1Ki 15:25 Нават же, сын Иеровоамов, воцарился над Израилем во второй год Асы, царя Иудейского, и царствовал над Израилем два года.
\vs 1Ki 15:26 И делал он неугодное пред очами Господа, ходил путем отца своего и во грехах его, которыми тот ввел Израиля в грех.
\vs 1Ki 15:27 И сделал против него заговор Вааса, сын Ахии, из дома Иссахарова, и убил его Вааса при Гавафоне Филистимском, когда Нават и все Израильтяне осаждали Гавафон;
\vs 1Ki 15:28 и умертвил его Вааса в третий год Асы, царя Иудейского, и воцарился вместо него.
\vs 1Ki 15:29 Когда он воцарился, то избил весь дом Иеровоамов, не оставил ни души у Иеровоама, доколе не истребил его, по слову Господа, которое Он изрек чрез раба Своего Ахию Силомлянина,
\vs 1Ki 15:30 за грехи Иеровоама, которые он сам делал и которыми ввел в грех Израиля, за оскорбление, которым он прогневал Господа Бога Израилева.
\rsbpar\vs 1Ki 15:31 Прочие дела Навата, всё, что он сделал, описано в летописи царей Израильских.
\vs 1Ki 15:32 И война была между Асою и Ваасою, царем Израильским, во все дни их.
\rsbpar\vs 1Ki 15:33 В третий год Асы, царя Иудейского, воцарился Вааса, сын Ахии, над всеми Израильтянами в Фирце \bibemph{и царствовал} двадцать четыре года.
\vs 1Ki 15:34 И делал неугодное пред очами Господними и ходил путем Иеровоама и во грехах его, которыми тот ввел в грех Израиля.
\vs 1Ki 16:1 И было слово Господне к Иую, сыну Ананиеву, о Ваасе:
\vs 1Ki 16:2 за то, что Я поднял тебя из праха и сделал тебя вождем народа Моего Израиля, ты же пошел путем Иеровоама и ввел в грех народ Мой Израильтян, чтобы он прогневлял Меня грехами своими,
\vs 1Ki 16:3 вот, Я отвергну дом Ваасы и дом потомства его и сделаю с домом твоим то же, что с домом Иеровоама, сына Наватова;
\vs 1Ki 16:4 кто умрет у Ваасы в городе, того съедят псы; а кто умрет у него на поле, того склюют птицы небесные.
\rsbpar\vs 1Ki 16:5 Прочие дела Ваасы, всё, чт\acc{о} он сделал, и подвиги его описаны в летописи царей Израильских.
\vs 1Ki 16:6 И почил Вааса с отцами своими, и погребен в Фирце. И воцарился Ила, сын его, вместо него.
\vs 1Ki 16:7 Но чрез Иуя, сына Ананиева, уже было \bibemph{сказано} слово Господне о Ваасе и о доме его и о всем зле, какое он делал пред очами Господа, раздражая Его делами рук своих, подражая дому Иеровоамову, за чт\acc{о} он истреблен был.
\rsbpar\vs 1Ki 16:8 В двадцать шестой год Асы, царя Иудейского, воцарился Ила, сын Ваасы, над Израилем в Фирце, \bibemph{и царствовал} два года.
\vs 1Ki 16:9 И составил против него заговор раб его Замврий, начальствовавший над половиною колесниц. Когда он в Фирце напился допьяна в доме Арсы, начальствующего над дворцом в Фирце,
\vs 1Ki 16:10 тогда вошел Замврий, поразил его и умертвил его, в двадцать седьмой год Асы, царя Иудейского, и воцарился вместо него.
\vs 1Ki 16:11 Когда он воцарился и сел на престоле его, то истребил весь дом Ваасы, не оставив ему мочащегося к стене, ни родственников его, ни друзей его.
\vs 1Ki 16:12 И истребил Замврий весь дом Ваасы, по слову Господа, которое Он изрек о Ваасе чрез Иуя пророка,
\vs 1Ki 16:13 за все грехи Ваасы и за грехи Илы, сына его, которые они сами делали и которыми вводили Израиля в грех, раздражая Господа Бога Израилева своими идолами.
\rsbpar\vs 1Ki 16:14 Прочие дела Илы, все, что он сделал, описано в летописи царей Израильских.
\rsbpar\vs 1Ki 16:15 В двадцать седьмой год Асы, царя Иудейского, воцарился Замврий и царствовал семь дней в Фирце, когда народ осаждал Гавафон Филистимский.
\vs 1Ki 16:16 Когда народ осаждавший услышал, что Замврий сделал заговор и умертвил царя, то все Израильтяне воцарили Амврия, военачальника, над Израилем в тот же день, в стане.
\vs 1Ki 16:17 И отступили Амврий и все Израильтяне с ним от Гавафона и осадили Фирцу.
\vs 1Ki 16:18 Когда увидел Замврий, что город взят, вошел во внутреннюю комнату царского дома и зажег за собою царский дом огнем и погиб
\vs 1Ki 16:19 за свои грехи, в чем он согрешил, делая неугодное пред очами Господними, ходя путем Иеровоама и во грехах его, которые тот сделал, чтобы ввести Израиля в грех.
\rsbpar\vs 1Ki 16:20 Прочие дела Замврия и заговор его, который он составил, описаны в летописи царей Израильских.
\rsbpar\vs 1Ki 16:21 Тогда разделился народ Израильский надвое: половина народа стояла за Фамния, сына Гонафова, чтобы воцарить его, а половина за Амврия.
\vs 1Ki 16:22 И одержал верх народ, который за Амврия, над народом, который за Фамния, сына Гонафова, и умер Фамний, и воцарился Амврий.
\rsbpar\vs 1Ki 16:23 В тридцать первый год Асы, царя Иудейского, воцарился Амврий над Израилем \bibemph{и царствовал} двенадцать лет. В Фирце он царствовал шесть лет.
\vs 1Ki 16:24 И купил Амврий гору Семерон у Семира за два таланта серебра, и застроил гору, и назвал построенный им город Самариею, по имени Семира, владельца горы.
\vs 1Ki 16:25 И делал Амврий неугодное пред очами Господа и поступал хуже всех бывших перед ним.
\vs 1Ki 16:26 Он во всем ходил путем Иеровоама, сына Наватова, и во грехах его, которыми тот ввел в грех Израильтян, чтобы прогневлять Господа Бога Израилева идолами своими.
\rsbpar\vs 1Ki 16:27 Прочие дела Амврия, которые он сделал, и мужество, которое он показал, описаны в летописи царей Израильских.
\vs 1Ki 16:28 И почил Амврий с отцами своими и погребен в Самарии. И воцарился Ахав, сын его, вместо него.
\rsbpar\vs 1Ki 16:29 Ахав, сын Амвриев, воцарился над Израилем в тридцать восьмой год Асы, царя Иудейского, и царствовал Ахав, сын Амврия, над Израилем в Самарии двадцать два года.
\vs 1Ki 16:30 И делал Ахав, сын Амврия, неугодное пред очами Господа более всех бывших прежде него.
\vs 1Ki 16:31 Мало было для него впадать в грехи Иеровоама, сына Наватова; он взял себе в жену Иезавель, дочь Ефваала царя Сидонского, и стал служить Ваалу и поклоняться ему.
\vs 1Ki 16:32 И поставил он Ваалу жертвенник в капище Ваала, который построил в Самарии.
\vs 1Ki 16:33 И сделал Ахав дубраву, и более всех царей Израильских, которые были прежде него, Ахав делал то, что раздражает Господа Бога Израилева, [и погубил душу свою].
\vs 1Ki 16:34 В его дни Ахиил Вефилянин построил Иерихон: на первенце своем Авираме он положил основание его и на младшем своем \bibemph{сыне} Сегубе поставил ворота его, по слову Господа, которое Он изрек чрез Иисуса, сына Навина.
\vs 1Ki 17:1 И сказал Илия [пророк], Фесвитянин, из жителей Галаадских, Ахаву: жив Господь Бог Израилев, пред Которым я стою! в сии годы не будет ни росы, ни дождя, разве только по моему слову.
\vs 1Ki 17:2 И было к нему слово Господне:
\vs 1Ki 17:3 пойди отсюда и обратись на восток и скройся у потока Хорафа, что против Иордана;
\vs 1Ki 17:4 из этого потока ты будешь пить, а в\acc{о}ронам Я повелел кормить тебя там.
\vs 1Ki 17:5 И пошел он и сделал по слову Господню; пошел и остался у потока Хорафа, что против Иордана.
\vs 1Ki 17:6 И в\acc{о}роны приносили ему хлеб и мясо поутру, и хлеб и мясо по вечеру, а из потока он пил.
\vs 1Ki 17:7 По прошествии некоторого времени этот поток высох, ибо не было дождя на землю.
\vs 1Ki 17:8 И было к нему слово Господне:
\vs 1Ki 17:9 встань и пойди в Сарепту Сидонскую, и оставайся там; Я повелел там женщине вдове кормить тебя.
\vs 1Ki 17:10 И встал он и пошел в Сарепту; и когда пришел к воротам города, вот, там женщина вдова собирает дрова. И подозвал он ее и сказал: дай мне немного воды в сосуде напиться.
\vs 1Ki 17:11 И пошла она, чтобы взять; а он закричал вслед ей и сказал: возьми для меня и кусок хлеба в руки свои.
\vs 1Ki 17:12 Она сказала: жив Господь Бог твой! у меня ничего нет печеного, а только есть горсть муки в кадке и немного масла в кувшине; и вот, я наберу полена два дров, и пойду, и приготовлю это для себя и для сына моего; съедим это и умрем.
\vs 1Ki 17:13 И сказал ей Илия: не бойся, пойди, сделай, чт\acc{о} ты сказала; но прежде из этого сделай небольшой опреснок для меня и принеси мне; а для себя и для своего сына сделаешь после;
\vs 1Ki 17:14 ибо так говорит Господь Бог Израилев: мука в кадке не истощится, и масло в кувшине не убудет до того дня, когда Господь даст дождь на землю.
\vs 1Ki 17:15 И пошла она и сделала так, как сказал Илия; и кормилась она, и он, и дом ее несколько времени.
\vs 1Ki 17:16 Мука в кадке не истощалась, и масло в кувшине не убывало, по слову Господа, которое Он изрек чрез Илию.
\vs 1Ki 17:17 После этого заболел сын этой женщины, хозяйки дома, и болезнь его была так сильна, что не осталось в нем дыхания.
\vs 1Ki 17:18 И сказала она Илии: что мне и тебе, человек Божий? ты пришел ко мне напомнить грехи мои и умертвить сына моего.
\vs 1Ki 17:19 И сказал он ей: дай мне сына твоего. И взял его с рук ее, и понес его в горницу, где он жил, и положил его на свою постель,
\vs 1Ki 17:20 и воззвал к Господу и сказал: Господи Боже мой! неужели Ты и вдове, у которой я пребываю, сделаешь зло, умертвив сына ее?
\vs 1Ki 17:21 И простершись над отроком трижды, он воззвал к Господу и сказал: Господи Боже мой! да возвратится душа отрока сего в него!
\vs 1Ki 17:22 И услышал Господь голос Илии, и возвратилась душа отрока сего в него, и он ожил.
\vs 1Ki 17:23 И взял Илия отрока, и свел его из горницы в дом, и отдал его матери его, и сказал Илия: смотри, сын твой жив.
\vs 1Ki 17:24 И сказала та женщина Илии: теперь-то я узнала, что ты человек Божий, и что слово Господне в устах твоих истинно.
\vs 1Ki 18:1 По прошествии многих дней было слово Господне к Илии в третий год: пойди и покажись Ахаву, и Я дам дождь на землю.
\vs 1Ki 18:2 И пошел Илия, чтобы показаться Ахаву. Голод же сильный был в Самарии.
\vs 1Ki 18:3 И призвал Ахав Авдия, начальствовавшего над дворцом. Авдий же был человек весьма богобоязненный,
\vs 1Ki 18:4 и когда Иезавель истребляла пророков Господних, Авдий взял сто пророков, и скрывал их, по пятидесяти человек, в пещерах, и питал их хлебом и водою.
\vs 1Ki 18:5 И сказал Ахав Авдию: пойди по земле ко всем источникам водным и ко всем потокам на земле, не найдем ли где травы, чтобы нам прокормить коней и лошаков и не лишиться скота.
\vs 1Ki 18:6 И разделили они между собою землю, чтобы обойти ее: Ахав особо пошел одною дорогою, и Авдий особо пошел другою дорогою.
\rsbpar\vs 1Ki 18:7 Когда Авдий шел дорогою, вот, навстречу ему идет Илия. Он узнал его и пал на лице свое и сказал: ты ли это, господин мой Илия?
\vs 1Ki 18:8 Тот сказал ему: я; пойди, скажи господину твоему: <<Илия здесь>>.
\vs 1Ki 18:9 Он сказал: чем я провинился, что ты предаешь раба твоего в руки Ахава, чтоб умертвить меня?
\vs 1Ki 18:10 Жив Господь Бог твой! нет ни одного народа и царства, куда бы не посылал государь мой искать тебя; и когда ему говорили, \bibemph{что тебя} нет, он брал клятву с того царства и народа, что не могли отыскать тебя;
\vs 1Ki 18:11 а ты теперь говоришь: <<пойди, скажи господину твоему: Илия здесь>>.
\vs 1Ki 18:12 Когда я пойду от тебя, тогда Дух Господень унесет тебя, не знаю, куда; и если я пойду уведомить Ахава, и он не найдет тебя, то он убьет меня; а раб твой богобоязнен от юности своей.
\vs 1Ki 18:13 Разве не сказано господину моему, чт\acc{о} я сделал, когда Иезавель убивала пророков Господних, как я скрывал сто человек пророков Господних, по пятидесяти человек, в пещерах и питал их хлебом и водою?
\vs 1Ki 18:14 А ты теперь говоришь: <<пойди, скажи господину твоему: Илия здесь>>; он убьет меня.
\vs 1Ki 18:15 И сказал Илия: жив Господь Саваоф, пред Которым я стою! сегодня я покажусь ему.
\vs 1Ki 18:16 И пошел Авдий навстречу Ахаву и донес ему. И пошел Ахав навстречу Илии.
\vs 1Ki 18:17 Когда Ахав увидел Илию, то сказал Ахав ему: ты ли это, смущающий Израиля?
\vs 1Ki 18:18 И сказал Илия: не я смущаю Израиля, а ты и дом отца твоего, тем, что вы презрели повеления Господни и идете вслед Ваалам;
\vs 1Ki 18:19 теперь пошли и собери ко мне всего Израиля на гору Кармил, и четыреста пятьдесят пророков Вааловых, и четыреста пророков дубравных, питающихся от стола Иезавели.
\vs 1Ki 18:20 И послал Ахав ко всем сынам Израилевым и собрал всех пророков на гору Кармил.
\vs 1Ki 18:21 И подошел Илия ко всему народу и сказал: долго ли вам хромать на оба колена? если Господь есть Бог, то последуйте Ему; а если Ваал, то ему последуйте. И не отвечал народ ему ни слова.
\vs 1Ki 18:22 И сказал Илия народу: я один остался пророк Господень, а пророков Вааловых четыреста пятьдесят человек [и четыреста пророков дубравных];
\vs 1Ki 18:23 пусть дадут нам двух тельцов, и пусть они выберут себе одного тельца, и рассекут его, и положат на дрова, но огня пусть не подкладывают; а я приготовлю другого тельца и положу на дрова, а огня не подложу;
\vs 1Ki 18:24 и призовите вы имя бога вашего, а я призову имя Господа Бога моего. Тот Бог, Который даст ответ посредством огня, есть Бог. И отвечал весь народ и сказал: хорошо, [пусть будет так].
\vs 1Ki 18:25 И сказал Илия пророкам Вааловым: выберите себе одного тельца и приготовьте вы прежде, ибо вас много; и призовите имя бога вашего, но огня не подкладывайте.
\vs 1Ki 18:26 И взяли они тельца, который дан был им, и приготовили, и призывали имя Ваала от утра до полудня, говоря: Ваале, услышь нас! Но не было ни голоса, ни ответа. И скакали они у жертвенника, который сделали.
\vs 1Ki 18:27 В полдень Илия стал смеяться над ними и говорил: кричите громким голосом, ибо он бог; может быть, он задумался, или занят чем-либо, или в дороге, а может быть, и спит, так он проснется!
\vs 1Ki 18:28 И стали они кричать громким голосом, и кололи себя по своему обыкновению ножами и копьями, так что кровь лилась по ним.
\vs 1Ki 18:29 Прошел полдень, а они всё еще бесновались до самого времени вечернего жертвоприношения; но не было ни голоса, ни ответа, ни слуха. [И сказал Илия Фесвитянин пророкам Вааловым: теперь отойдите, чтоб и я совершил мое жертвоприношение. Они отошли и умолкли.]
\vs 1Ki 18:30 Тогда Илия сказал всему народу: подойдите ко мне. И подошел весь народ к нему. Он восстановил разрушенный жертвенник Господень.
\vs 1Ki 18:31 И взял Илия двенадцать камней, по числу колен сынов Иакова, которому Господь сказал так: Израиль будет имя твое.
\vs 1Ki 18:32 И построил из сих камней жертвенник во имя Господа, и сделал вокруг жертвенника ров, вместимостью в две саты зерен,
\vs 1Ki 18:33 и положил дрова [на жертвенник], и рассек тельца, и возложил его на дрова,
\vs 1Ki 18:34 и сказал: наполните четыре ведра воды и выливайте на всесожигаемую жертву и на дрова. [И сделали так.] Потом сказал: повторите. И они повторили. И сказал: сделайте \bibemph{то же} в третий раз. И сделали в третий раз,
\vs 1Ki 18:35 и вода полилась вокруг жертвенника, и ров наполнился водою.
\vs 1Ki 18:36 Во время приношения вечерней жертвы подошел Илия пророк [и воззвал на небо] и сказал: Господи, Боже Авраамов, Исааков и Израилев! [Услышь меня, Господи, услышь меня ныне в огне!] Да познают в сей день [люди сии], что Ты один Бог в Израиле, и что я раб Твой и сделал всё по слову Твоему.
\vs 1Ki 18:37 Услышь меня, Господи, услышь меня! Да познает народ сей, что Ты, Господи, Бог, и Ты обратишь сердце их [к Тебе].
\vs 1Ki 18:38 И ниспал огонь Господень и пожрал всесожжение, и дрова, и камни, и прах, и поглотил воду, которая во рве.
\vs 1Ki 18:39 Увидев \bibemph{это}, весь народ пал на лице свое и сказал: Господь есть Бог, Господь есть Бог!
\vs 1Ki 18:40 И сказал им Илия: схватите пророков Вааловых, чтобы ни один из них не укрылся. И схватили их, и отвел их Илия к потоку Киссону и заколол их там.
\vs 1Ki 18:41 И сказал Илия Ахаву: пойди, ешь и пей, ибо слышен шум дождя.
\vs 1Ki 18:42 И пошел Ахав есть и пить, а Илия взошел на верх Кармила и наклонился к земле, и положил лице свое между коленами своими,
\vs 1Ki 18:43 и сказал отроку своему: пойди, посмотри к морю. Тот пошел и посмотрел, и сказал: ничего нет. Он сказал: продолжай \bibemph{это} до семи раз.
\vs 1Ki 18:44 В седьмой раз тот сказал: вот, небольшое облако поднимается от моря, величиною в ладонь человеческую. Он сказал: пойди, скажи Ахаву: <<запрягай [колесницу твою] и поезжай, чтобы не застал тебя дождь>>.
\vs 1Ki 18:45 Между тем небо сделалось мрачно от туч и от ветра, и пошел большой дождь. Ахав же сел в колесницу, [заплакал] и поехал в Изреель.
\vs 1Ki 18:46 И была на Илии рука Господня. Он опоясал чресла свои и бежал пред Ахавом до самого Изрееля.
\vs 1Ki 19:1 И пересказал Ахав Иезавели всё, что сделал Илия, и то, что он убил всех пророков мечом.
\vs 1Ki 19:2 И послала Иезавель посланца к Илии сказать: [если ты Илия, а я Иезавель, то] пусть то и то сделают мне боги, и еще больше сделают, если я завтра к этому времени не сделаю с твоею душею того, что \bibemph{сделано} с душею каждого из них.
\vs 1Ki 19:3 Увидев это, он встал и пошел, чтобы спасти жизнь свою, и пришел в Вирсавию, которая в Иудее, и оставил отрока своего там.
\vs 1Ki 19:4 А сам отошел в пустыню на день пути и, придя, сел под можжевеловым кустом, и просил смерти себе и сказал: довольно уже, Господи; возьми душу мою, ибо я не лучше отцов моих.
\vs 1Ki 19:5 И лег и заснул под можжевеловым кустом. И вот, Ангел коснулся его и сказал ему: встань, ешь [и пей].
\vs 1Ki 19:6 И взглянул Илия, и вот, у изголовья его печеная лепешка и кувшин воды. Он поел и напился и опять заснул.
\vs 1Ki 19:7 И возвратился Ангел Господень во второй раз, коснулся его и сказал: встань, ешь [и пей], ибо дальняя дорога пред тобою.
\vs 1Ki 19:8 И встал он, поел и напился, и, подкрепившись тою пищею, шел сорок дней и сорок ночей до горы Божией Хорива.
\vs 1Ki 19:9 И вошел он там в пещеру и ночевал в ней. И вот, было к нему слово Господне, и сказал ему \bibemph{Господь}: что ты здесь, Илия?
\vs 1Ki 19:10 Он сказал: возревновал я о Господе Боге Саваофе, ибо сыны Израилевы оставили завет Твой, разрушили Твои жертвенники и пророков Твоих убили мечом; остался я один, но и моей души ищут, чтобы отнять ее.
\vs 1Ki 19:11 И сказал: выйди и стань на горе пред лицем Господним, и вот, Господь пройдет, и большой и сильный ветер, раздирающий горы и сокрушающий скалы пред Господом, но не в ветре Господь; после ветра землетрясение, но не в землетрясении Господь;
\vs 1Ki 19:12 после землетрясения огонь, но не в огне Господь; после огня веяние тихого ветра, [и там Господь].
\vs 1Ki 19:13 Услышав \bibemph{сие}, Илия закрыл лице свое милотью своею, и вышел, и стал у входа в пещеру. И был к нему голос и сказал ему: что ты здесь, Илия?
\vs 1Ki 19:14 Он сказал: возревновал я о Господе Боге Саваофе, ибо сыны Израилевы оставили завет Твой, разрушили жертвенники Твои и пророков Твоих убили мечом; остался я один, но и моей души ищут, чтоб отнять ее.
\vs 1Ki 19:15 И сказал ему Господь: пойди обратно своею дорогою чрез пустыню в Дамаск, и когда придешь, то помажь Азаила в царя над Сириею,
\vs 1Ki 19:16 а Ииуя, сына Намессиина, помажь в царя над Израилем; Елисея же, сына Сафатова, из Авел-Мехолы, помажь в пророка вместо себя;
\vs 1Ki 19:17 кто убежит от меча Азаилова, того умертвит Ииуй; а кто спасется от меча Ииуева, того умертвит Елисей.
\vs 1Ki 19:18 Впрочем, Я оставил между Израильтянами семь тысяч [мужей]; всех сих колени не преклонялись пред Ваалом, и всех сих уста не лобызали его.
\vs 1Ki 19:19 И пошел он оттуда, и нашел Елисея, сына Сафатова, когда он орал; двенадцать пар [волов] было у него, и сам он был при двенадцатой. Илия, проходя мимо него, бросил на него милоть свою.
\vs 1Ki 19:20 И оставил [Елисей] волов, и побежал за Илиею, и сказал: позволь мне поцеловать отца моего и мать мою, и я пойду за тобою. Он сказал ему: пойди и приходи назад, ибо что сделал я тебе?
\vs 1Ki 19:21 Он, отойдя от него, взял пару волов и заколол их и, зажегши плуг волов, изжарил мясо их, и раздал людям, и они ели. А сам встал и пошел за Илиею, и стал служить ему.
\vs 1Ki 20:1 Венадад, царь Сирийский, собрал все свое войско, и с ним были тридцать два царя, и кони и колесницы, и пошел, осадил Самарию и воевал против нее.
\vs 1Ki 20:2 И послал послов к Ахаву, царю Израильскому, в город,
\vs 1Ki 20:3 и сказал ему: так говорит Венадад: серебро твое и золото твое~--- мои, и жены твои и лучшие сыновья твои~--- мои.
\vs 1Ki 20:4 И отвечал царь Израильский и сказал: да будет по слову твоему, господин мой царь: я и все мое~--- твое.
\vs 1Ki 20:5 И опять пришли послы и сказали: так говорит Венадад: я послал к тебе сказать: <<серебро твое, и золото твое, и жён твоих, и сыновей твоих отдай мне>>;
\vs 1Ki 20:6 поэтому я завтра, к этому времени, пришлю к тебе рабов моих, чтобы они осмотрели твой дом и домы служащих при тебе, и все дорогое для глаз твоих взяли в свои руки и унесли.
\vs 1Ki 20:7 И созвал царь Израильский всех старейшин земли и сказал: замечайте и смотрите, он замышляет зло; когда он присылал ко мне за жёнами моими, и сыновьями моими, и серебром моим, и золотом моим, я ему не отказал.
\vs 1Ki 20:8 И сказали ему все старейшины и весь народ: не слушай и не соглашайся.
\vs 1Ki 20:9 И сказал он послам Венадада: скажите господину моему царю: все, о чем ты присылал в первый раз к рабу твоему, я готов сделать, а этого не могу сделать. И пошли послы и отнесли ему ответ.
\vs 1Ki 20:10 И прислал к нему Венадад сказать: пусть то и то сделают мне боги, и еще больше сделают, если праха Самарийского достанет по горсти для всех людей, идущих за мною.
\vs 1Ki 20:11 И отвечал царь Израильский и сказал: скажите: пусть не хвалится подпоясывающийся, как распоясывающийся.
\vs 1Ki 20:12 Услышав это слово, Венадад, который пил вместе с царями в палатках, сказал рабам своим: осаждайте город. И они осадили город.
\rsbpar\vs 1Ki 20:13 И вот, один пророк подошел к Ахаву, царю Израильскому, и сказал: так говорит Господь: видишь ли все это большое полчище? вот, Я сегодня предам его в руку твою, чтобы ты знал, что Я Господь.
\vs 1Ki 20:14 И сказал Ахав: чрез кого? Он сказал: так говорит Господь: чрез слуг областных начальников. И сказал [Ахав]: кто начнет сражение? Он сказал: ты.
\vs 1Ki 20:15 [Ахав] счел слуг областных начальников, и нашлось их двести тридцать два; после них счел весь народ, всех сынов Израилевых, семь тысяч.
\vs 1Ki 20:16 И они выступили в полдень. Венадад же напился допьяна в палатках вместе с царями, с тридцатью двумя царями, помогавшими ему.
\vs 1Ki 20:17 И выступили прежде слуги областных начальников. И послал Венадад, и донесли ему, что люди вышли из Самарии.
\vs 1Ki 20:18 Он сказал: если за миром вышли они, то схватите их живыми, и если на войну вышли, также схватите их живыми.
\vs 1Ki 20:19 Вышли из города слуги областных начальников, и войско за ними.
\vs 1Ki 20:20 И поражал каждый противника своего; и побежали Сирияне, а Израильтяне погнались за ними. Венадад же, царь Сирийский, спасся на коне с всадниками.
\vs 1Ki 20:21 И вышел царь Израильский, и взял коней и колесниц, и произвел большое поражение у Сириян.
\rsbpar\vs 1Ki 20:22 И подошел пророк к царю Израильскому и сказал ему: пойди, укрепись, и знай и смотри, что тебе делать, ибо по прошествии года царь Сирийский опять пойдет против тебя.
\vs 1Ki 20:23 Слуги царя Сирийского сказали ему: Бог их есть Бог гор, [а не Бог долин,] поэтому они одолели нас; если же мы сразимся с ними на равнине, то верно одолеем их.
\vs 1Ki 20:24 Итак вот что сделай: удали царей, каждого с места его, и вместо них поставь областеначальников;
\vs 1Ki 20:25 и набери себе войска столько, сколько пало у тебя, и коней, сколько было коней, и колесниц, сколько было колесниц; и сразимся с ними на равнине, и тогда верно одолеем их. И послушался он голоса их и сделал так.
\rsbpar\vs 1Ki 20:26 По прошествии года Венадад собрал Сириян и выступил к Афеку, чтобы сразиться с Израилем.
\vs 1Ki 20:27 Собраны были и сыны Израилевы и, взяв продовольствие, пошли навстречу им. И расположились сыны Израилевы станом пред ними, как бы два небольшие стада коз, а Сирияне наполнили землю.
\vs 1Ki 20:28 И подошел человек Божий, и сказал царю Израильскому: так говорит Господь: за то, что Сирияне говорят: <<Господь есть Бог гор, а не Бог долин>>, Я все это большое полчище предам в руку твою, чтобы вы знали, что Я~--- Господь.
\vs 1Ki 20:29 И стояли станом одни против других семь дней. В седьмой день началась битва, и сыны Израилевы поразили сто тысяч пеших Сириян в один день.
\vs 1Ki 20:30 Остальные убежали в город Афек; \bibemph{там} упала стена на остальных двадцать семь тысяч человек. А Венадад ушел в город и бегал из одной внутренней комнаты в другую.
\vs 1Ki 20:31 И сказали ему слуги его: мы слышали, что цари дома Израилева цари милостивые; позволь нам возложить вретища на чресла свои и веревки на головы свои и пойти к царю Израильскому; может быть, он пощадит жизнь твою.
\vs 1Ki 20:32 И опоясали они вретищами чресла свои и возложили веревки на головы свои, и пришли к царю Израильскому и сказали: раб твой Венадад говорит: <<пощади жизнь мою>>. Тот сказал: разве он жив? он брат мой.
\vs 1Ki 20:33 Люди сии приняли это за \bibemph{хороший} знак и поспешно подхватили слово из уст его и сказали: брат твой Венадад. И сказал он: пойдите, приведите его. И вышел к нему Венадад, и он посадил его \bibemph{с собою} на колесницу.
\vs 1Ki 20:34 И сказал ему \bibemph{Венадад}: города, которые взял мой отец у твоего отца, я возвращу, и площади ты можешь иметь для себя в Дамаске, как отец мой имел в Самарии. \bibemph{Ахав сказал}: после договора я отпущу тебя. И, заключив с ним договор, отпустил его.
\rsbpar\vs 1Ki 20:35 Тогда один человек из сынов пророческих сказал другому, по слову Господа: бей меня. Но этот человек не согласился бить его.
\vs 1Ki 20:36 И сказал ему: за то, что ты не слушаешь гласа Господня, убьет тебя лев, когда пойдешь от меня. Он пошел от него, и лев, встретив его, убил его.
\vs 1Ki 20:37 И нашел он другого человека, и сказал: бей меня. Этот человек бил его до того, что изранил побоями.
\vs 1Ki 20:38 И пошел пророк и предстал пред царя на дороге, прикрыв покрывалом глаза свои.
\vs 1Ki 20:39 Когда царь проезжал мимо, он закричал царю и сказал: раб твой ходил на сражение, и вот, один человек, отошедший в сторону, подвел ко мне человека и сказал: <<стереги этого человека; если его не станет, то твоя душа будет за его душу, или ты должен будешь отвесить талант серебра>>.
\vs 1Ki 20:40 Когда раб твой занялся теми и другими делами, его не стало.~--- И сказал ему царь Израильский: таков тебе и приговор, ты сам решил.
\vs 1Ki 20:41 Он тотчас снял покрывало с глаз своих, и узнал его царь, что он из пророков.
\vs 1Ki 20:42 И сказал ему: так говорит Господь: за то, что ты выпустил из рук твоих человека, заклятого Мною, душа твоя будет вместо его души, народ твой вместо его народа.
\vs 1Ki 20:43 И отправился царь Израильский домой встревоженный и огорченный, и прибыл в Самарию.
\vs 1Ki 21:1 И было после сих происшествий: у Навуфея Изреелитянина в Изреели был виноградник подле дворца Ахава, царя Самарийского.
\vs 1Ki 21:2 И сказал Ахав Навуфею, говоря: отдай мне свой виноградник; из него будет у меня овощной сад, ибо он близко к моему дому; а вместо него я дам тебе виноградник лучше этого, или, если угодно тебе, дам тебе серебра, сколько он стоит.
\vs 1Ki 21:3 Но Навуфей сказал Ахаву: сохрани меня Господь, чтоб я отдал тебе наследство отцов моих!
\vs 1Ki 21:4 И пришел Ахав домой встревоженный и огорченный тем словом, которое сказал ему Навуфей Изреелитянин, говоря: не отдам тебе наследства отцов моих. И [в смущенном духе] лег на постель свою, и отворотил лице свое, и хлеба не ел.
\vs 1Ki 21:5 И вошла к нему жена его Иезавель и сказала ему: отчего встревожен дух твой, что ты и хлеба не ешь?
\vs 1Ki 21:6 Он сказал ей: когда я стал говорить Навуфею Изреелитянину и сказал ему: <<отдай мне виноградник твой за серебро, или, если хочешь, я дам тебе \bibemph{другой} виноградник вместо него>>, тогда он сказал: <<не отдам тебе виноградника моего, [наследства отцов моих]>>.
\vs 1Ki 21:7 И сказала ему Иезавель, жена его: что за царство было бы в Израиле, если бы ты так поступал? встань, ешь хлеб и будь спокоен; я доставлю тебе виноградник Навуфея Изреелитянина.
\vs 1Ki 21:8 И написала она от имени Ахава письма, и запечатала их его печатью, и послала эти письма к старейшинам и знатным в его городе, живущим с Навуфеем.
\vs 1Ki 21:9 В письмах она писала так: объявите пост и посадите Навуфея на первое место в народе;
\vs 1Ki 21:10 и против него посадите двух негодных людей, которые свидетельствовали бы на него и сказали: <<ты хулил Бога и царя>>; и потом выведите его, и побейте его камнями, чтоб он умер.
\vs 1Ki 21:11 И сделали мужи города его, старейшины и знатные, жившие в городе его, как приказала им Иезавель, так, как писано в письмах, которые она послала к ним.
\vs 1Ki 21:12 Объявили пост и посадили Навуфея во главе народа;
\vs 1Ki 21:13 и выступили два негодных человека и сели против него, и свидетельствовали на него эти недобрые люди пред народом, и говорили: Навуфей хулил Бога и царя. И вывели его за город, и побили его камнями, и он умер.
\vs 1Ki 21:14 И послали к Иезавели сказать: Навуфей побит камнями и умер.
\vs 1Ki 21:15 Услышав, что Навуфей побит камнями и умер, Иезавель сказала Ахаву: встань, возьми во владение виноградник Навуфея Изреелитянина, который не хотел отдать тебе за серебро; ибо Навуфея нет в живых, он умер.
\vs 1Ki 21:16 Когда услышал Ахав, что Навуфей [Изреелитянин] был убит, [разодрал одежды свои и надел на себя вретище, а потом] встал Ахав, чтобы пойти в виноградник Навуфея Изреелитянина и взять его во владение.
\rsbpar\vs 1Ki 21:17 И было слово Господне к Илии Фесвитянину:
\vs 1Ki 21:18 встань, пойди навстречу Ахаву, царю Израильскому, который в Самарии, вот, он теперь в винограднике Навуфея, куда пришел, чтобы взять \bibemph{его} во владение;
\vs 1Ki 21:19 и скажи ему: <<так говорит Господь: ты убил, и еще вступаешь в наследство?>> и скажи ему: <<так говорит Господь: на том месте, где псы лизали кровь Навуфея, псы будут лизать и твою кровь>>.
\vs 1Ki 21:20 И сказал Ахав Илии: нашел ты меня, враг мой! Он сказал: нашел, ибо ты предался тому, чтобы делать неугодное пред очами Господа [и раздражать Его].
\vs 1Ki 21:21 [Так говорит Господь:] вот, Я наведу на тебя беды и вымету за тобою и истреблю у Ахава мочащегося к стене и заключенного и оставшегося в Израиле.
\vs 1Ki 21:22 И поступлю с домом твоим так, как поступил Я с домом Иеровоама, сына Наватова, и с домом Ваасы, сына Ахиина, за оскорбление, которым ты раздражил \bibemph{Меня} и ввел Израиля в грех.
\vs 1Ki 21:23 Также и о Иезавели сказал Господь: псы съедят Иезавель за стеною Изрееля.
\vs 1Ki 21:24 Кто умрет у Ахава в городе, того съедят псы, а кто умрет на поле, того расклюют птицы небесные;
\vs 1Ki 21:25 не было еще такого, как Ахав, который предался бы тому, чтобы делать неугодное пред очами Господа, к чему подущала его жена его Иезавель;
\vs 1Ki 21:26 он поступал весьма гнусно, последуя идолам, как делали Аморреи, которых Господь прогнал от лица сынов Израилевых.
\vs 1Ki 21:27 Выслушав все слова сии, Ахав [умилился пред Господом, ходил и плакал,] разодрал одежды свои, и возложил на тело свое вретище, и постился, и спал во вретище, и ходил печально.
\vs 1Ki 21:28 И было слово Господне к Илии Фесвитянину [об Ахаве], и сказал Господь:
\vs 1Ki 21:29 видишь, как смирился предо Мною Ахав? За то, что он смирился предо Мною, Я не наведу бед в его дни; во дни сына его наведу беды на дом его.
\vs 1Ki 22:1 Прожили три года, и не было войны между Сириею и Израилем.
\vs 1Ki 22:2 На третий год Иосафат, царь Иудейский, пошел к царю Израильскому.
\vs 1Ki 22:3 И сказал царь Израильский слугам своим: знаете ли, что Рамоф Галаадский наш? А мы так долго молчим, и не берем его из руки царя Сирийского.
\vs 1Ki 22:4 И сказал он Иосафату: пойдешь ли ты со мною на войну против Рамофа Галаадского? И сказал Иосафат царю Израильскому: как ты, так и я; как твой народ, так и мой народ; как твои кони, так и мои кони.
\vs 1Ki 22:5 И сказал Иосафат царю Израильскому: спроси сегодня, что скажет Господь.
\vs 1Ki 22:6 И собрал царь Израильский пророков, около четырехсот человек и сказал им: идти ли мне войною на Рамоф Галаадский, или нет? Они сказали: иди, Господь предаст \bibemph{его} в руки царя.
\vs 1Ki 22:7 И сказал Иосафат: нет ли здесь еще пророка Господня, чтобы нам вопросить чрез него Господа?
\vs 1Ki 22:8 И сказал царь Израильский Иосафату: есть еще один человек, чрез которого можно вопросить Господа, но я не люблю его, ибо он не пророчествует о мне доброго, а только худое,~--- это Михей, сын Иемвлая. И сказал Иосафат: не говори, царь, так.
\vs 1Ki 22:9 И позвал царь Израильский одного евнуха и сказал: сходи поскорее за Михеем, сыном Иемвлая.
\vs 1Ki 22:10 Царь Израильский и Иосафат, царь Иудейский, сидели каждый на седалище своем, одетые в \bibemph{царские} одежды, на площади у ворот Самарии, и все пророки пророчествовали пред ними.
\vs 1Ki 22:11 И сделал себе Седекия, сын Хенааны, железные рога, и сказал: так говорит Господь: сими избодешь Сириян до истребления их.
\vs 1Ki 22:12 И все пророки пророчествовали то же, говоря: иди на Рамоф Галаадский, будет успех, Господь предаст \bibemph{его} в руку царя.
\vs 1Ki 22:13 Посланный, который пошел позвать Михея, говорил ему: вот, речи пророков единогласно \bibemph{предвещают} царю доброе; пусть бы и твое слово было согласно с словом каждого из них; изреки и ты доброе.
\vs 1Ki 22:14 И сказал Михей: жив Господь! я изреку то, что скажет мне Господь.
\vs 1Ki 22:15 И пришел он к царю. Царь сказал ему: Михей! идти ли нам войною на Рамоф Галаадский, или нет? И сказал тот ему: иди, будет успех, Господь предаст \bibemph{его} в руку царя.
\vs 1Ki 22:16 И сказал ему царь: еще и еще заклинаю тебя, чтобы ты не говорил мне ничего, кроме истины во имя Господа.
\vs 1Ki 22:17 И сказал он: я вижу всех Израильтян, рассеянных по горам, как овец, у которых нет пастыря. И сказал Господь: нет у них начальника, пусть возвращаются с миром каждый в свой дом.
\vs 1Ki 22:18 И сказал царь Израильский Иосафату: не говорил ли я тебе, что он не пророчествует о мне доброго, а только худое?
\vs 1Ki 22:19 И сказал [Михей]: [не так; не я, а] выслушай слово Господне: я видел Господа, сидящего на престоле Своем, и все воинство небесное стояло при Нем, по правую и по левую руку Его;
\vs 1Ki 22:20 и сказал Господь: кто склонил бы Ахава, чтобы он пошел и пал в Рамофе Галаадском? И один говорил так, другой говорил иначе;
\vs 1Ki 22:21 и выступил один дух, стал пред лицем Господа и сказал: я склоню его. И сказал ему Господь: чем?
\vs 1Ki 22:22 Он сказал: я выйду и сделаюсь духом лживым в устах всех пророков его. \bibemph{Господь} сказал: ты склонишь его и выполнишь это; пойди и сделай так.
\vs 1Ki 22:23 И вот, теперь попустил Господь духа лживого в уста всех сих пророков твоих; но Господь изрек о тебе недоброе.
\vs 1Ki 22:24 И подошел Седекия, сын Хенааны, и, ударив Михея по щеке, сказал: как, неужели от меня отошел Дух Господень, чтобы говорить в тебе?
\vs 1Ki 22:25 И сказал Михей: вот, ты увидишь \bibemph{это} в тот день, когда будешь бегать из одной комнаты в другую, чтоб укрыться,
\vs 1Ki 22:26 и сказал царь Израильский: возьмите Михея и отведите его к Амону градоначальнику и к Иоасу, сыну царя,
\vs 1Ki 22:27 и скажите: так говорит царь: посадите этого в темницу и кормите его скудно хлебом и скудно водою, доколе я не возвращусь в мире.
\vs 1Ki 22:28 И сказал Михей: если возвратишься в мире, то не Господь говорил чрез меня. И сказал: слушай, весь народ!
\rsbpar\vs 1Ki 22:29 И пошел царь Израильский и Иосафат, царь Иудейский, к Рамофу Галаадскому.
\vs 1Ki 22:30 И сказал царь Израильский Иосафату: я переоденусь и вступлю в сражение, а ты надень твои \bibemph{царские} одежды. И переоделся царь Израильский и вступил в сражение.
\vs 1Ki 22:31 Сирийский царь повелел начальникам колесниц, которых у него было тридцать два, сказав: не сражайтесь ни с малым, ни с великим, а только с одним царем Израильским.
\vs 1Ki 22:32 Начальники колесниц, увидев Иосафата, подумали: <<верно это царь Израильский>>, и поворотили на него, чтобы сразиться \bibemph{с ним}. И закричал Иосафат.
\vs 1Ki 22:33 Начальники колесниц, видя, что это не Израильский царь, поворотили от него.
\vs 1Ki 22:34 А один человек случайно натянул лук и ранил царя Израильского сквозь швы лат. И сказал он своему вознице: повороти назад и вывези меня из войска, ибо я ранен.
\vs 1Ki 22:35 Но сражение в тот день усилилось, и царь стоял на колеснице против Сириян, и вечером умер, и кровь из раны лилась в колесницу.
\vs 1Ki 22:36 И провозглашено было по всему стану при захождении солнца: каждый иди в свой город, каждый в свою землю!
\vs 1Ki 22:37 И умер царь, и привезен был в Самарию, и похоронили царя в Самарии.
\vs 1Ki 22:38 И обмыли колесницу на пруде Самарийском, и псы лизали кровь его, и омывали блудницы, по слову Господа, которое Он изрек.
\rsbpar\vs 1Ki 22:39 Прочие дела Ахава, все, что он делал, и дом из слоновой кости, который он построил, и все города, которые он строил, описаны в летописи царей Израильских.
\vs 1Ki 22:40 И почил Ахав с отцами своими, и воцарился Охозия, сын его, вместо него.
\rsbpar\vs 1Ki 22:41 Иосафат, сын Асы, воцарился над Иудеею в четвертый год Ахава, царя Израильского.
\vs 1Ki 22:42 Тридцати пяти лет был Иосафат, когда воцарился, и двадцать пять лет царствовал в Иерусалиме. Имя матери его Азува, дочь Салаиля.
\vs 1Ki 22:43 Он ходил во всем путем отца своего Асы, не сходил с него, делая угодное пред очами Господними. Только высоты не были отменены; народ еще совершал жертвы и курения на высотах.
\vs 1Ki 22:44 Иосафат заключил мир с царем Израильским.
\rsbpar\vs 1Ki 22:45 Прочие дела Иосафата и подвиги его, какие он совершил, и как он воевал, описаны в летописи царей Иудейских.
\vs 1Ki 22:46 И остатки блудников, которые остались во дни отца его Асы, он истребил с земли.
\vs 1Ki 22:47 В Идумее тогда не было царя; \bibemph{был} наместник царский.
\vs 1Ki 22:48 [Царь] Иосафат сделал корабли на море, чтобы ходить в Офир за золотом; но они не дошли, ибо разбились в Ецион-Гавере.
\vs 1Ki 22:49 Тогда сказал Охозия, сын Ахава, Иосафату: пусть мои слуги пойдут с твоими слугами на кораблях. Но Иосафат не согласился.
\vs 1Ki 22:50 И почил Иосафат с отцами своими и был погребен с отцами своими в городе Давида, отца своего. И воцарился Иорам, сын его, вместо него.
\rsbpar\vs 1Ki 22:51 Охозия, сын Ахава, воцарился над Израилем в Самарии, в семнадцатый год Иосафата, царя Иудейского, и царствовал над Израилем [в Самарии] два года,
\vs 1Ki 22:52 и делал неугодное пред очами Господа, и ходил путем отца своего и путем матери своей и путем Иеровоама, сына Наватова, который ввел Израиля в грех:
\vs 1Ki 22:53 он служил Ваалу и поклонялся ему и прогневал Господа Бога Израилева всем тем, что делал отец его.

\bibbookdescr{2Ki}{
  inline={\LARGE Четвертая книга\\\Huge Царств\fns{У Евреев: <<Вторая царей>>.}},
  toc={4-я Царств},
  bookmark={4-я Царств},
  header={4-я Царств},
  %headerleft={},
  %headerright={},
  abbr={4~Цар}
}
\vs 2Ki 1:1 И отложился Моав от Израиля по смерти Ахава.
\vs 2Ki 1:2 Охозия же упал чрез решетку с горницы своей, что в Самарии, и занемог. И послал послов, и сказал им: пойдите, спросите у Веельзевула, божества Аккаронского: выздоровею ли я от сей болезни? [И пошли они спрашивать.]
\rsbpar\vs 2Ki 1:3 Тогда Ангел Господень сказал Илии Фесвитянину: встань, пойди навстречу посланным от царя Самарийского и скажи им: разве нет Бога в Израиле, что вы идете вопрошать Веельзевула, божество Аккаронское?
\vs 2Ki 1:4 За это так говорит Господь: с постели, на которую ты лег, не сойдешь с нее, но умрешь. И пошел Илия, [и сказал им].
\vs 2Ki 1:5 И возвратились к \bibemph{Охозии} посланные. И он сказал им: что это вы возвратились?
\vs 2Ki 1:6 И сказали ему: навстречу нам вышел человек и сказал нам: пойдите, возвратитесь к царю, который послал вас, и скажите ему: так говорит Господь: разве нет Бога в Израиле, что ты посылаешь вопрошать Веельзевула, божество Аккаронское? За то с постели, на которую ты лег, не сойдешь с нее, но умрешь.
\vs 2Ki 1:7 И сказал им: каков видом тот человек, который вышел навстречу вам и говорил вам слова сии?
\vs 2Ki 1:8 Они сказали ему: человек тот весь в волосах и кожаным поясом подпоясан по чреслам своим. И сказал он: это Илия Фесвитянин.
\vs 2Ki 1:9 И послал к нему пятидесятника с его пятидесятком. И он взошел к нему, когда Илия сидел на верху горы, и сказал ему: человек Божий! царь говорит: сойди.
\vs 2Ki 1:10 И отвечал Илия, и сказал пятидесятнику: если я человек Божий, то пусть сойдет огонь с неба и попалит тебя и твой пятидесяток. И сошел огонь с неба и попалил его и пятидесяток его.
\vs 2Ki 1:11 И послал к нему царь другого пятидесятника с его пятидесятком. И он стал говорить ему: человек Божий! так сказал царь: сойди скорее.
\vs 2Ki 1:12 И отвечал Илия и сказал ему: если я человек Божий, то пусть сойдет огонь с неба и попалит тебя и твой пятидесяток. И сошел огонь Божий с неба, и попалил его и пятидесяток его.
\vs 2Ki 1:13 И еще послал в третий раз пятидесятника с его пятидесятком. И поднялся, и пришел пятидесятник третий, и пал на колена свои пред Илиею, и умолял его, и говорил ему: человек Божий! да не будет презрена душа моя и душа рабов твоих~--- сих пятидесяти~--- пред очами твоими;
\vs 2Ki 1:14 вот, сошел огонь с неба, и попалил двух пятидесятников прежних с их пятидесятками; но теперь да не будет презрена душа моя пред очами твоими!
\vs 2Ki 1:15 И сказал Ангел Господень Илии: пойди с ним, не бойся его. И он встал, и пошел с ним к царю.
\vs 2Ki 1:16 И сказал ему: так говорит Господь: за то, что ты посылал послов вопрошать Веельзевула, божество Аккаронское, как будто в Израиле нет Бога, чтобы вопрошать о слове Его,~--- с постели, на которую ты лег, не сойдешь с нее, но умрешь.
\vs 2Ki 1:17 И умер он по слову Господню, которое изрек Илия. И воцарился Иорам [брат Охозии], вместо него, во второй год Иорама, сына Иосафатова, царя Иудейского, так как сына у того не было.
\rsbpar\vs 2Ki 1:18 Прочее об Охозии, что он сделал, написано в летописи царей Израильских.
\vs 2Ki 2:1 В то время, как Господь восхотел вознести Илию в вихре на небо, шел Илия с Елисеем из Галгала.
\vs 2Ki 2:2 И сказал Илия Елисею: останься здесь, ибо Господь посылает меня в Вефиль. Но Елисей сказал: жив Господь и жива душа твоя! не оставлю тебя. И пошли они в Вефиль.
\vs 2Ki 2:3 И вышли сыны пророков, которые в Вефиле, к Елисею и сказали ему: знаешь ли, что сегодня Господь вознесет господина твоего над главою твоею? Он сказал: я также знаю, молчите.
\vs 2Ki 2:4 И сказал ему Илия: Елисей, останься здесь, ибо Господь посылает меня в Иерихон. И сказал он: жив Господь и жива душа твоя! не оставлю тебя. И пришли в Иерихон.
\vs 2Ki 2:5 И подошли сыны пророков, которые в Иерихоне, к Елисею и сказали ему: знаешь ли, что сегодня Господь берет господина твоего и вознесет над главою твоею? Он сказал: я также знаю, молчите.
\vs 2Ki 2:6 И сказал ему Илия: останься здесь, ибо Господь посылает меня к Иордану. И сказал он: жив Господь и жива душа твоя! не оставлю тебя. И пошли оба.
\vs 2Ki 2:7 Пятьдесят человек из сынов пророческих пошли и стали вдали напротив их, а они оба стояли у Иордана.
\vs 2Ki 2:8 И взял Илия милоть свою, и свернул, и ударил ею по воде, и расступилась она туда и сюда, и перешли оба посуху.
\vs 2Ki 2:9 Когда они перешли, Илия сказал Елисею: проси, чт\acc{о} сделать тебе, прежде нежели я буду взят от тебя. И сказал Елисей: дух, который в тебе, пусть будет на мне вдвойне.
\vs 2Ki 2:10 И сказал он: трудного ты просишь. Если увидишь, как я буду взят от тебя, то будет тебе так, а если не увидишь, не будет.
\rsbpar\vs 2Ki 2:11 Когда они шли и дорогою разговаривали, вдруг явилась колесница огненная и кони огненные, и разлучили их обоих, и понесся Илия в вихре на небо.
\vs 2Ki 2:12 Елисей же смотрел и воскликнул: отец мой, отец мой, колесница Израиля и конница его! И не видел его более. И схватил он одежды свои и разодрал их на две части.
\vs 2Ki 2:13 И поднял милоть Илии, упавшую с него, и пошел назад, и стал на берегу Иордана;
\vs 2Ki 2:14 и взял милоть Илии, упавшую с него, и ударил ею по воде, и сказал: где Господь, Бог Илии,~--- Он Самый? И ударил по воде, и она расступилась туда и сюда, и перешел Елисей.
\vs 2Ki 2:15 И увидели его сыны пророков, которые в Иерихоне, издали, и сказали: опочил дух Илии на Елисее. И пошли навстречу ему, и поклонились ему до земли,
\vs 2Ki 2:16 и сказали ему: вот, есть \bibemph{у нас}, рабов твоих, человек пятьдесят, люди сильные; пусть бы они пошли и поискали господина твоего; может быть, унес его Дух Господень и поверг его на одной из гор, или на одной из долин. Он же сказал: не посылайте.
\vs 2Ki 2:17 Но они приступали к нему долго, так что наскучили ему, и он сказал: пошлите. И послали пятьдесят человек, и искали три дня, и не нашли его,
\vs 2Ki 2:18 и возвратились к нему, между тем как он оставался в Иерихоне, и сказал им: не говорил ли я вам: не ходите?
\vs 2Ki 2:19 И сказали жители того города Елисею: вот, положение этого города хорошо, как видит господин мой; но вода нехороша и земля бесплодна.
\vs 2Ki 2:20 И сказал он: дайте мне новую чашу и положите туда соли. И дали ему.
\vs 2Ki 2:21 И вышел он к истоку воды, и бросил туда соли, и сказал: так говорит Господь: Я сделал воду сию здоровою, не будет от нее впредь ни смерти, ни бесплодия.
\vs 2Ki 2:22 И вода стала здоровою до сего дня, по слову Елисея, которое он сказал.
\vs 2Ki 2:23 И пошел он оттуда в Вефиль. Когда он шел дорогою, малые дети вышли из города и насмехались над ним и говорили ему: иди, плешивый! иди, плешивый!
\vs 2Ki 2:24 Он оглянулся и увидел их и проклял их именем Господним. И вышли две медведицы из леса и растерзали из них сорок два ребенка.
\vs 2Ki 2:25 Отсюда пошел он на гору Кармил, а оттуда возвратился в Самарию.
\vs 2Ki 3:1 Иорам, сын Ахава, воцарился над Израилем в Самарии в восемнадцатый год Иосафата, царя Иудейского, и царствовал двенадцать лет,
\vs 2Ki 3:2 и делал неугодное в очах Господних, хотя не так, как отец его и мать его: он снял статую Ваала, которую сделал отец его;
\vs 2Ki 3:3 однако же грехов Иеровоама, сына Наватова, который ввел в грех Израиля, он держался, не отставал от них.
\rsbpar\vs 2Ki 3:4 Меса, царь Моавитский, был богат скотом и присылал царю Израильскому по сто тысяч овец и по сто тысяч неостриженных баранов.
\vs 2Ki 3:5 Но когда умер Ахав, царь Моавитский отложился от царя Израильского.
\vs 2Ki 3:6 И выступил царь Иорам в то время из Самарии и сделал смотр всем Израильтянам;
\vs 2Ki 3:7 и пошел и послал к Иосафату, царю Иудейскому, сказать: царь Моавитский отложился от меня, пойдешь ли со мной на войну против Моава? Он сказал: пойду; как ты, так и я, как твой народ, так и мой народ; как твои кони, так и мои кони.
\vs 2Ki 3:8 И сказал: какою дорогою идти нам? Он сказал: дорогою пустыни Едомской.
\vs 2Ki 3:9 И пошел царь Израильский, и царь Иудейский, и царь Едомский, и шли они обходом семь дней, и не было воды для войска и для скота, который \bibemph{шел} за ними.
\vs 2Ki 3:10 И сказал царь Израильский: ах! созвал Господь трех царей сих, чтобы предать их в руку Моава.
\vs 2Ki 3:11 И сказал Иосафат: нет ли здесь пророка Господня, чтобы нам вопросить Господа чрез него? И отвечал один из слуг царя Израильского и сказал: здесь Елисей, сын Сафатов, который подавал воду на руки Илии.
\vs 2Ki 3:12 И сказал Иосафат: есть у него слово Господне. И пошли к нему царь Израильский, и Иосафат, и царь Едомский.
\vs 2Ki 3:13 И сказал Елисей царю Израильскому: что мне и тебе? пойди к пророкам отца твоего и к пророкам матери твоей. И сказал ему царь Израильский: нет, потому что Господь созвал сюда трех царей сих, чтобы предать их в руку Моава.
\vs 2Ki 3:14 И сказал Елисей: жив Господь Саваоф, пред Которым я стою! Если бы я не почитал Иосафата, царя Иудейского, то не взглянул бы на тебя и не видел бы тебя;
\vs 2Ki 3:15 теперь позовите мне гуслиста. И когда гуслист играл на гуслях, тогда рука Господня коснулась Елисея,
\vs 2Ki 3:16 и он сказал: так говорит Господь: делайте на сей долине рвы за рвами,
\vs 2Ki 3:17 ибо так говорит Господь: не увидите ветра и не увидите дождя, а долина сия наполнится водою, которую будете пить вы и мелкий и крупный скот ваш;
\vs 2Ki 3:18 но этого мало пред очами Господа; Он и Моава предаст в руки ваши,
\vs 2Ki 3:19 и вы поразите все города укрепленные и все города главные, и все лучшие деревья срубите, и все источники водные запрудите, и все лучшие участки полевые испортите каменьями.
\vs 2Ki 3:20 Поутру, когда возносят хлебное приношение, вдруг полилась вода по пути от Едома, и наполнилась земля водою.
\rsbpar\vs 2Ki 3:21 Когда Моавитяне услышали, что идут цари воевать с ними, тогда собраны были все, начиная от носящего пояс и старше, и стали на границе.
\vs 2Ki 3:22 Поутру встали они рано, и когда солнце воссияло над водою, Моавитянам издали показалась эта вода красною, как кровь.
\vs 2Ki 3:23 И сказали они: это кровь; сразились цари между собою и истребили друг друга; теперь на добычу, Моав!
\vs 2Ki 3:24 И пришли они к стану Израильскому. И встали Израильтяне и стали бить Моавитян, и те побежали от них, а они продолжали идти на них и бить Моавитян.
\vs 2Ki 3:25 И города разрушили, и на всякий лучший участок в поле бросили каждый по камню и закидали его; и все протоки вод запрудили и все дерева лучшие срубили, так что оставались только каменья в Кир-Харешете. И обступили его пращники и разрушили его.
\vs 2Ki 3:26 И увидел царь Моавитский, что битва одолевает его, и взял с собою семьсот человек, владеющих мечом, чтобы пробиться к царю Едомскому; но не могли.
\vs 2Ki 3:27 И взял он сына своего первенца, которому следовало царствовать вместо него, и вознес его во всесожжение на стене. Это произвело большое негодование в Израильтянах, и они отступили от него и возвратились в свою землю.
\vs 2Ki 4:1 Одна из жен сынов пророческих с воплем говорила Елисею: раб твой, мой муж, умер; а ты знаешь, что раб твой боялся Господа; теперь пришел заимодавец взять обоих детей моих в рабы себе.
\vs 2Ki 4:2 И сказал ей Елисей: что мне сделать тебе? скажи мне, что есть у тебя в доме? Она сказала: нет у рабы твоей ничего в доме, кроме сосуда с елеем.
\vs 2Ki 4:3 И сказал он: пойди, попроси себе сосудов на стороне, у всех соседей твоих, сосудов порожних; набери немало,
\vs 2Ki 4:4 и пойди, запри дверь за собою и за сыновьями твоими, и наливай во все эти сосуды; полные отставляй.
\vs 2Ki 4:5 И пошла от него и заперла дверь за собой и за сыновьями своими. Они подавали ей, а она наливала.
\vs 2Ki 4:6 Когда наполнены были сосуды, она сказала сыну своему: подай мне еще сосуд. Он сказал ей: нет более сосудов. И остановилось масло.
\vs 2Ki 4:7 И пришла она, и пересказала человеку Божию. Он сказал: пойди, продай масло и заплати долги твои; а что останется, тем будешь жить с сыновьями твоими.
\rsbpar\vs 2Ki 4:8 В один день пришел Елисей в Сонам. Там одна богатая женщина упросила его \bibemph{к себе} есть хлеба; и когда он ни проходил, всегда заходил туда есть хлеба.
\vs 2Ki 4:9 И сказала она мужу своему: вот, я знаю, что человек Божий, который проходит мимо нас постоянно, святой;
\vs 2Ki 4:10 сделаем небольшую горницу над стеною и поставим ему там постель, и стол, и седалище, и светильник; и когда он будет приходить к нам, пусть заходит туда.
\vs 2Ki 4:11 В один день он пришел туда, и зашел в горницу, и лег там,
\vs 2Ki 4:12 и сказал Гиезию, слуге своему: позови эту Сонамитянку. И позвал ее, и она стала пред ним.
\vs 2Ki 4:13 И сказал ему: скажи ей: <<вот, ты так заботишься о нас; что сделать бы тебе? не нужно ли поговорить о тебе с царем, или с военачальником?>> Она сказала: нет, среди своего народа я живу.
\vs 2Ki 4:14 И сказал он: что же сделать ей? И сказал Гиезий: да вот, сына нет у нее, а муж ее стар.
\vs 2Ki 4:15 И сказал он: позови ее. Он позвал ее, и стала она в дверях.
\vs 2Ki 4:16 И сказал он: через год, в это самое время ты будешь держать на руках сына. И сказала она: нет, господин мой, человек Божий, не обманывай рабы твоей.
\vs 2Ki 4:17 И женщина стала беременною и родила сына на другой год, в то самое время, как сказал ей Елисей.
\vs 2Ki 4:18 И подрос ребенок и в один день пошел к отцу своему, к жнецам.
\vs 2Ki 4:19 И сказал отцу своему: голова моя! голова моя болит! И сказал тот слуге своему: отнеси его к матери его.
\vs 2Ki 4:20 И понес его и принес его к матери его. И он сидел на коленях у нее до полудня, и умер.
\vs 2Ki 4:21 И пошла она, и положила его на постели человека Божия, и заперла его, и вышла,
\vs 2Ki 4:22 и позвала мужа своего и сказала: пришли мне одного из слуг и одну из ослиц, я поеду к человеку Божию и возвращусь.
\vs 2Ki 4:23 Он сказал: зачем тебе ехать к нему? сегодня не новомесячие и не суббота. Но она сказала: хорошо.
\vs 2Ki 4:24 И оседлала ослицу и сказала слуге своему: веди и иди; не останавливайся, доколе не скажу тебе.
\vs 2Ki 4:25 И отправилась и прибыла к человеку Божию, к горе Кармил. И когда увидел человек Божий ее издали, то сказал слуге своему Гиезию: это та Сонамитянка.
\vs 2Ki 4:26 Побеги к ней навстречу и скажи ей: <<здорова ли ты? здоров ли муж твой? здоров ли ребенок?>>~--- Она сказала: здоровы.
\vs 2Ki 4:27 Когда же пришла к человеку Божию на гору, ухватилась за ноги его. И подошел Гиезий, чтобы отвести ее; но человек Божий сказал: оставь ее, душа у нее огорчена, а Господь скрыл от меня и не объявил мне.
\vs 2Ki 4:28 И сказала она: просила ли я сына у господина моего? не говорила ли я: <<не обманывай меня>>?
\vs 2Ki 4:29 И сказал он Гиезию: опояшь чресла твои и возьми жезл мой в руку твою, и пойди; если встретишь кого, не приветствуй его, и если кто будет тебя приветствовать, не отвечай ему; и положи посох мой на лице ребенка.
\vs 2Ki 4:30 И сказала мать ребенка: жив Господь и жива душа твоя! не отстану от тебя. И он встал и пошел за нею.
\vs 2Ki 4:31 Гиезий пошел впереди их и положил жезл на лице ребенка. Но не было ни голоса, ни ответа. И вышел навстречу ему, и донес ему, и сказал: не пробуждается ребенок.
\vs 2Ki 4:32 И вошел Елисей в дом, и вот, ребенок умерший лежит на постели его.
\vs 2Ki 4:33 И вошел, и запер дверь за собою, и помолился Господу.
\vs 2Ki 4:34 И поднялся и лег над ребенком, и приложил свои уста к его устам, и свои глаза к его глазам, и свои ладони к его ладоням, и простерся на нем, и согрелось тело ребенка.
\vs 2Ki 4:35 И встал и прошел по горнице взад и вперед; потом опять поднялся и простерся на нем. И чихнул ребенок раз семь, и открыл ребенок глаза свои.
\vs 2Ki 4:36 И позвал он Гиезия и сказал: позови эту Сонамитянку. И тот позвал ее. Она пришла к нему, и он сказал: возьми сына твоего.
\vs 2Ki 4:37 И подошла, и упала ему в ноги, и поклонилась до земли; и взяла сына своего и пошла.
\vs 2Ki 4:38 Елисей же возвратился в Галгал.\rsbpar И был голод в земле той, и сыны пророков сидели пред ним. И сказал он слуге своему: поставь большой котел и свари похлебку для сынов пророческих.
\vs 2Ki 4:39 И вышел один из них в поле собирать овощи, и нашел дикое вьющееся растение, и набрал с него диких плодов полную одежду свою; и пришел и накрошил их в котел с похлебкою, так как они не знали \bibemph{их}.
\vs 2Ki 4:40 И налили им есть. Но как скоро они стали есть похлебку, то подняли крик и говорили: смерть в котле, человек Божий! И не могли есть.
\vs 2Ki 4:41 И сказал он: подайте муки. И всыпал ее в котел и сказал [Гиезию]: наливай людям, пусть едят. И не стало ничего вредного в котле.
\vs 2Ki 4:42 Пришел некто из Ваал-Шалиши, и принес человеку Божию хлебный начаток~--- двадцать ячменных хлебцев и сырые зерна в шелухе. И сказал Елисей: отдай людям, пусть едят.
\vs 2Ki 4:43 И сказал слуга его: что тут я дам ста человекам? И сказал он: отдай людям, пусть едят, ибо так говорит Господь: <<насытятся, и останется>>.
\vs 2Ki 4:44 Он подал им, и они насытились, и еще осталось, по слову Господню.
\vs 2Ki 5:1 Нееман, военачальник царя Сирийского, был великий человек у господина своего и уважаемый, потому что чрез него дал Господь победу Сириянам; и человек сей был отличный воин, но прокаженный.
\vs 2Ki 5:2 Сирияне \bibemph{однажды} пошли отрядами и взяли в плен из земли Израильской маленькую девочку, и она служила жене Неемановой.
\vs 2Ki 5:3 И сказала она госпоже своей: о, если бы господин мой побывал у пророка, который в Самарии, то он снял бы с него проказу его!
\vs 2Ki 5:4 И пошел \bibemph{Нееман} и передал это господину своему, говоря: так и так говорит девочка, которая из земли Израильской.
\vs 2Ki 5:5 И сказал царь Сирийский [Нееману]: пойди, сходи, а я пошлю письмо к царю Израильскому. Он пошел и взял с собою десять талантов серебра и шесть тысяч \bibemph{сиклей} золота, и десять перемен одежд;
\vs 2Ki 5:6 и принес письмо царю Израильскому, в котором было сказано: вместе с письмом сим, вот, я посылаю к тебе Неемана, слугу моего, чтобы ты снял с него проказу его.
\vs 2Ki 5:7 Царь Израильский, прочитав письмо, разодрал одежды свои и сказал: разве я Бог, чтобы умерщвлять и оживлять, что он посылает ко мне, чтобы я снял с человека проказу его? вот, теперь знайте и смотрите, что он ищет предлога враждовать против меня.
\rsbpar\vs 2Ki 5:8 Когда услышал Елисей, человек Божий, что царь Израильский разодрал одежды свои, то послал сказать царю: для чего ты разодрал одежды свои? пусть он придет ко мне, и узнает, что есть пророк в Израиле.
\vs 2Ki 5:9 И прибыл Нееман на конях своих и на колеснице своей, и остановился у входа в дом Елисеев.
\vs 2Ki 5:10 И выслал к нему Елисей слугу сказать: пойди, омойся семь раз в Иордане, и обновится тело твое у тебя, и будешь чист.
\vs 2Ki 5:11 И разгневался Нееман, и пошел, и сказал: вот, я думал, что он выйдет, станет и призовет имя Господа Бога своего, и возложит руку свою на то место и снимет проказу;
\vs 2Ki 5:12 разве Авана и Фарфар, реки Дамасские, не лучше всех вод Израильских? разве я не мог бы омыться в них и очиститься? И оборотился и удалился в гневе.
\vs 2Ki 5:13 И подошли рабы его и говорили ему, и сказали: отец мой, \bibemph{если бы} что-нибудь важное сказал тебе пророк, то не сделал ли бы ты? а тем более, когда он сказал тебе только: <<омойся, и будешь чист>>.
\vs 2Ki 5:14 И пошел он и окунулся в Иордане семь раз, по слову человека Божия, и обновилось тело его, как тело малого ребенка, и очистился.
\vs 2Ki 5:15 И возвратился к человеку Божию он и все сопровождавшие его, и пришел, и стал пред ним, и сказал: вот, я узнал, что на всей земле нет Бога, как только у Израиля; итак прими дар от раба твоего.
\vs 2Ki 5:16 И сказал он: жив Господь, пред лицем Которого стою! не приму. И тот принуждал его взять, но он не согласился.
\vs 2Ki 5:17 И сказал Нееман: если уже не так, то пусть рабу твоему дадут земли, сколько снесут два лошака, потому что не будет впредь раб твой приносить всесожжения и жертвы другим богам, кроме Господа;
\vs 2Ki 5:18 только вот в чем да простит Господь раба твоего: когда пойдет господин мой в дом Риммона для поклонения там и опрется на руку мою, и поклонюсь я в доме Риммона, то, за мое поклонение в доме Риммона, да простит Господь раба твоего в случае сем.
\vs 2Ki 5:19 И сказал ему: иди с миром. И он отъехал от него на небольшое пространство земли.
\rsbpar\vs 2Ki 5:20 И сказал Гиезий, слуга Елисея, человека Божия: вот, господин мой отказался взять из руки Неемана, этого Сириянина, то, что он приносил. Жив Господь! Побегу я за ним, и возьму у него что-нибудь.
\vs 2Ki 5:21 И погнался Гиезий за Нееманом. И увидел Нееман бегущего за собою, и сошел с колесницы навстречу ему, и сказал: с миром ли?
\vs 2Ki 5:22 Он отвечал: с миром; господин мой послал меня сказать: <<вот, теперь пришли ко мне с горы Ефремовой два молодых человека из сынов пророческих; дай им талант серебра и две перемены одежд>>.
\vs 2Ki 5:23 И сказал Нееман: возьми, пожалуй, два таланта. И упрашивал его. И завязал он два таланта серебра в два мешка и две перемены одежд и отдал двум слугам своим, и понесли перед ним.
\vs 2Ki 5:24 Когда он пришел к холму, то взял из рук их и спрятал дома. И отпустил людей, и они ушли.
\vs 2Ki 5:25 Когда он пришел и явился к господину своему, Елисей сказал ему: откуда, Гиезий? И сказал он: никуда не ходил раб твой.
\vs 2Ki 5:26 И сказал он ему: разве сердце мое не сопутствовало тебе, когда обратился навстречу тебе человек тот с колесницы своей? время ли брать серебро и брать одежды, или масличные деревья и виноградники, и мелкий или крупный скот, и рабов или рабынь?
\vs 2Ki 5:27 Пусть же проказа Нееманова пристанет к тебе и к потомству твоему навек. И вышел он от него \bibemph{белый} от проказы, как снег.
\vs 2Ki 6:1 И сказали сыны пророков Елисею: вот, место, где мы живем при тебе, тесно для нас;
\vs 2Ki 6:2 пойдем к Иордану и возьмем оттуда каждый по одному бревну и сделаем себе там место для жительства. Он сказал: пойдите.
\vs 2Ki 6:3 И сказал один: сделай милость, пойди и ты с рабами твоими. И сказал он: пойду.
\vs 2Ki 6:4 И пошел с ними, и пришли к Иордану и стали рубить деревья.
\vs 2Ki 6:5 И когда один валил бревно, топор его упал в воду. И закричал он и сказал: ах, господин мой! а он взят был на подержание!
\vs 2Ki 6:6 И сказал человек Божий: где он упал? Он указал ему место. И отрубил он \bibemph{кусок} дерева и бросил туда, и всплыл топор.
\vs 2Ki 6:7 И сказал он: возьми себе. Он протянул руку свою и взял его.
\rsbpar\vs 2Ki 6:8 Царь Сирийский пошел войною на Израильтян, и советовался со слугами своими, говоря: в таком-то и в таком-то месте я расположу свой стан.
\vs 2Ki 6:9 И посылал человек Божий к царю Израильскому сказать: берегись проходить сим местом, ибо там Сирияне залегли.
\vs 2Ki 6:10 И посылал царь Израильский на то место, о котором говорил ему человек Божий и предостерегал его; и сберег себя там не раз и не два.
\vs 2Ki 6:11 И встревожилось сердце царя Сирийского по сему случаю, и призвал он рабов своих и сказал им: скажите мне, кто из наших \bibemph{в сношении} с царем Израильским?
\vs 2Ki 6:12 И сказал один из слуг его: никто, господин мой царь; а Елисей пророк, который у Израиля, пересказывает царю Израильскому и те слова, которые ты говоришь в спальной комнате твоей.
\vs 2Ki 6:13 И сказал он: пойдите, узнайте, где он; я пошлю и возьму его. И донесли ему и сказали: вот, он в Дофаиме.
\vs 2Ki 6:14 И послал туда коней и колесницы и много войска. И пришли ночью и окружили город.
\vs 2Ki 6:15 Поутру служитель человека Божия встал и вышел; и вот, войско вокруг города, и кони и колесницы. И сказал ему слуга его: увы! господин мой, что нам делать?
\vs 2Ki 6:16 И сказал он: не бойся, потому что тех, которые с нами, больше, нежели тех, которые с ними.
\vs 2Ki 6:17 И молился Елисей, и говорил: Господи! открой ему глаза, чтоб он увидел. И открыл Господь глаза слуге, и он увидел, и вот, вся гора наполнена конями и колесницами огненными кругом Елисея.
\vs 2Ki 6:18 Когда пошли к нему Сирияне, Елисей помолился Господу и сказал: порази их слепотою. И Он поразил их слепотою по слову Елисея.
\vs 2Ki 6:19 И сказал им Елисей: это не та дорога и не тот город; идите за мною, и я провожу вас к тому человеку, которого вы ищете. И привел их в Самарию.
\vs 2Ki 6:20 Когда они пришли в Самарию, Елисей сказал: Господи! открой глаза им, чтобы они видели. И открыл Господь глаза их, и увидели, что они в средине Самарии.
\vs 2Ki 6:21 И сказал царь Израильский Елисею, увидев их: не избить ли их, отец мой?
\vs 2Ki 6:22 И сказал он: не убивай. Разве мечом твоим и луком твоим ты пленил их, чтобы убивать их? Предложи им хлеба и воды; пусть едят и пьют, и пойдут к государю своему.
\vs 2Ki 6:23 И приготовил им большой обед, и они ели и пили. И отпустил их, и пошли к государю своему. И не ходили более те полчища Сирийские в землю Израилеву.
\rsbpar\vs 2Ki 6:24 После того собрал Венадад, царь Сирийский, все войско свое и выступил, и осадил Самарию.
\vs 2Ki 6:25 И был большой голод в Самарии, когда они осадили ее, так что ослиная голова продавалась по восьмидесяти сиклей серебра, и четвертая часть каба голубиного помета~--- по пяти сиклей серебра.
\vs 2Ki 6:26 Однажды царь Израильский проходил по стене, и женщина с воплем говорила ему: помоги, господин мой царь.
\vs 2Ki 6:27 И сказал он: если не поможет тебе Господь, из чего я помогу тебе? с гумна ли, с точила ли?
\vs 2Ki 6:28 И сказал ей царь: что тебе? И сказала она: эта женщина говорила мне: <<отдай своего сына, съедим его сегодня, а сына моего съедим завтра>>.
\vs 2Ki 6:29 И сварили мы моего сына, и съели его. И я сказала ей на другой день: <<отдай же твоего сына, и съедим его>>. Но она спрятала своего сына.
\vs 2Ki 6:30 Царь, выслушав слова женщины, разодрал одежды свои; и проходил он по стене, и народ видел, что вретище на самом теле его.
\vs 2Ki 6:31 И сказал: пусть то и то сделает мне Бог, и еще более сделает, если останется голова Елисея, сына Сафатова, на нем сегодня.
\vs 2Ki 6:32 Елисей же сидел в своем доме, и старцы сидели у него. И послал [царь] человека от себя. Прежде нежели пришел посланный к нему, он сказал старцам: видите ли, что этот сын убийцы послал снять с меня голову? Смотрите, когда придет посланный, затворите дверь и прижмите его дверью. А вот и топот ног господина его за ним!
\vs 2Ki 6:33 Еще говорил он с ними, и вот посланный приходит к нему, и сказал: вот какое бедствие от Господа! чего мне впредь ждать от Господа?
\vs 2Ki 7:1 И сказал Елисей: выслушайте слово Господне: так говорит Господь: завтра в это время мера муки лучшей \bibemph{будет} по сиклю и две меры ячменя по сиклю у ворот Самарии.
\vs 2Ki 7:2 И отвечал сановник, на руку которого царь опирался, человеку Божию, и сказал: если бы Господь и открыл окна на небе, и тогда может ли это быть? И сказал тот: вот увидишь глазами твоими, но есть этого не будешь.
\vs 2Ki 7:3 Четыре человека прокаженных находились при входе в ворота и говорили они друг другу: что нам сидеть здесь, ожидая смерти?
\vs 2Ki 7:4 Если решиться нам пойти в город, то в городе голод, и мы там умрем; если же сидеть здесь, то также умрем. Пойдем лучше в стан Сирийский. Если оставят нас в живых, будем жить, а если умертвят, умрем.
\vs 2Ki 7:5 И встали в сумерки, чтобы пойти в стан Сирийский. И пришли к краю стана Сирийского, и вот, нет там ни одного человека.
\vs 2Ki 7:6 Господь сделал то, что стану Сирийскому послышался стук колесниц и ржание коней, шум войска большого. И сказали они друг другу: верно нанял против нас царь Израильский царей Хеттейских и Египетских, чтобы пойти на нас.
\vs 2Ki 7:7 И встали и побежали в сумерки, и оставили шатры свои, и коней своих, и ослов своих, весь стан, как он был, и побежали, спасая себя.
\vs 2Ki 7:8 И пришли те прокаженные к краю стана, и вошли в один шатер, и ели и пили, и взяли оттуда серебро, и золото, и одежды, и пошли и спрятали. Пошли еще в другой шатер, и там взяли, и пошли и спрятали.
\vs 2Ki 7:9 И сказали друг другу: не так мы делаем. День сей~--- день радостной вести, если мы замедлим и будем дожидаться утреннего света, то падет на нас вина. Пойдем же и уведомим дом царский.
\vs 2Ki 7:10 И пришли, и позвали привратников городских, и рассказали им, говоря: мы ходили в стан Сирийский, и вот, нет там ни человека, ни голоса человеческого, а только кони привязанные, и ослы привязанные, и шатры, как быть им.
\vs 2Ki 7:11 И позвали привратников, и они передали весть в самый дворец царский.
\vs 2Ki 7:12 И встал царь ночью, и сказал слугам своим: скажу вам, что делают с нами Сирияне. Они знают, что мы терпим голод, и вышли из стана, чтобы спрятаться в поле, думая так: <<когда они выйдут из города, мы захватим их живыми и вторгнемся в город>>.
\vs 2Ki 7:13 И отвечал один из служащих при нем, и сказал: пусть возьмут пять из остальных коней, которые остались в городе, (из всего ополчения Израильтян только и осталось в нем, из всего ополчения Израильтян, которое погибло), и пошлем, и посмотрим.
\vs 2Ki 7:14 И взяли две пары коней, запряженных в колесницы. И послал царь вслед Сирийского войска, сказав: пойдите, посмотрите.
\vs 2Ki 7:15 И ехали за ним до Иордана, и вот вся дорога устлана одеждами и вещами, которые побросали Сирияне при торопливом побеге своем. И возвратились посланные, и донесли царю.
\vs 2Ki 7:16 И вышел народ, и разграбил стан Сирийский, и была мера муки лучшей по сиклю, и две меры ячменя по сиклю, по слову Господню.
\vs 2Ki 7:17 И царь поставил того сановника, на руку которого опирался, у ворот; и растоптал его народ в воротах, и он умер, как сказал человек Божий, который говорил, когда приходил к нему царь.
\vs 2Ki 7:18 Когда говорил человек Божий царю так: <<две меры ячменя по сиклю, и мера муки лучшей по сиклю будут завтра в это время у ворот Самарии>>,
\vs 2Ki 7:19 тогда отвечал этот сановник человеку Божию и сказал: <<если бы Господь и открыл окна на небе, и тогда может ли это быть?>> А он сказал: <<увидишь твоими глазами, но есть этого не будешь>>.
\vs 2Ki 7:20 Так и сбылось с ним; и затоптал его народ в воротах, и он умер.
\vs 2Ki 8:1 И говорил Елисей женщине, сына которой воскресил он, и сказал: встань, и пойди, ты и дом твой, и поживи там, где можешь пожить, ибо призвал Господь голод, и он придет на сию землю на семь лет.
\vs 2Ki 8:2 И встала та женщина, и сделала по слову человека Божия; и пошла она и дом ее, и жила в земле Филистимской семь лет.
\vs 2Ki 8:3 По прошествии семи лет возвратилась эта женщина из земли Филистимской и пришла просить царя о доме своем и о поле своем.
\vs 2Ki 8:4 Царь тогда разговаривал с Гиезием, слугою человека Божия, и сказал: расскажи мне все замечательное, чт\acc{о} сделал Елисей.
\vs 2Ki 8:5 И между тем как он рассказывал царю, что тот воскресил умершего, женщина, которой сына воскресил он, просила царя о доме своем и о поле своем. И сказал Гиезий: господин мой царь, это та самая женщина и тот самый сын ее, которого воскресил Елисей.
\vs 2Ki 8:6 И спросил царь у женщины, и она рассказала ему. И дал ей царь одного из придворных, сказав: возвратить ей все принадлежащее ей и все доходы с поля, с того дня, как она оставила землю, поныне.
\rsbpar\vs 2Ki 8:7 И пришел Елисей в Дамаск, когда Венадад, царь Сирийский, был болен. И донесли ему, говоря: пришел человек Божий сюда.
\vs 2Ki 8:8 И сказал царь Азаилу: возьми в руку твою дар и пойди навстречу человеку Божию, и вопроси Господа чрез него, говоря: выздоровею ли я от сей болезни?
\vs 2Ki 8:9 И пошел Азаил навстречу ему, и взял дар в руку свою и всего лучшего в Дамаске, сколько могут нести сорок верблюдов, и пришел и стал пред лице его, и сказал: сын твой Венадад, царь Сирийский, послал меня к тебе спросить: <<выздоровею ли я от сей болезни?>>
\vs 2Ki 8:10 И сказал ему Елисей: пойди, скажи ему: <<выздоровеешь>>; однако ж открыл мне Господь, что он умрет.
\vs 2Ki 8:11 И устремил на него \bibemph{Елисей} взор свой, и так оставался до того, что привел его в смущение; и заплакал человек Божий.
\vs 2Ki 8:12 И сказал Азаил: отчего господин мой плачет? И сказал он: оттого, что я знаю, какое наделаешь ты сынам Израилевым зло; крепости их предашь огню, и юношей их мечом умертвишь, и грудных детей их побьешь, и беременных \bibemph{женщин} у них разрубишь.
\vs 2Ki 8:13 И сказал Азаил: что такое раб твой, пес [мертвый], чтобы мог сделать такое большое дело? И сказал Елисей: указал мне Господь в тебе царя Сирии.
\vs 2Ki 8:14 И пошел он от Елисея, и пришел к государю своему. И сказал ему \bibemph{этот}: что говорил тебе Елисей? И сказал: он говорил мне, что ты выздоровеешь.
\vs 2Ki 8:15 А на другой день он взял одеяло, намочил его водою, и положил на лице его, и он умер. И воцарился Азаил вместо него.
\rsbpar\vs 2Ki 8:16 В пятый год Иорама, сына Ахавова, царя Израильского, за Иосафатом, царем Иудейским, воцарился Иорам, сын Иосафатов, царь Иудейский.
\vs 2Ki 8:17 Тридцати двух лет был он, когда воцарился, и восемь лет царствовал в Иерусалиме,
\vs 2Ki 8:18 и ходил путем царей Израильских, как поступал дом Ахавов, потому что дочь Ахава была женою его, и делал неугодное в очах Господних.
\vs 2Ki 8:19 Однако ж не хотел Господь погубить Иуду, ради Давида, раба Своего, так как Он обещал дать ему светильник в детях его на все времена.
\vs 2Ki 8:20 Во дни его выступил Едом из-под руки Иуды, и поставили они над собою царя.
\vs 2Ki 8:21 И пошел Иорам в Цаир, и все колесницы с ним; и встал он ночью, и поразил Идумеян, окружавших его, и начальников над колесницами, но народ убежал в шатры свои.
\vs 2Ki 8:22 И выступил Едом из-под руки Иуды до сего дня. В то же время выступила и Ливна.
\rsbpar\vs 2Ki 8:23 Прочее об Иораме и обо всем, что он сделал, написано в летописи царей Иудейских.
\vs 2Ki 8:24 И почил Иорам с отцами своими, и погребен с отцами своими в городе Давидовом. И воцарился Охозия, сын его, вместо него.
\rsbpar\vs 2Ki 8:25 В двенадцатый год Иорама, сына Ахавова, царя Израильского, воцарился Охозия, сын Иорама, царя Иудейского.
\vs 2Ki 8:26 Двадцати двух лет был Охозия, когда воцарился, и один год царствовал в Иерусалиме. Имя же матери его Гофолия, дочь Амврия, царя Израильского.
\vs 2Ki 8:27 И ходил путем дома Ахавова, и делал неугодное в очах Господних, подобно дому Ахавову, потому что он был в родстве с домом Ахавовым.
\vs 2Ki 8:28 И пошел он с Иорамом, сыном Ахавовым, на войну с Азаилом, царем Сирийским, в Рамоф Галаадский, и ранили Сирияне Иорама.
\vs 2Ki 8:29 И возвратился Иорам царь, чтобы лечиться в Изрееле от ран, которые причинили ему Сирияне в Рамофе, когда он воевал с Азаилом, царем Сирийским. И Охозия, сын Иорама, царь Иудейский, пришел посетить Иорама, сына Ахавова, в Изреель, так как он был болен.
\vs 2Ki 9:1 Елисей пророк призвал одного из сынов пророческих и сказал ему: опояшь чресла твои, и возьми сей сосуд с елеем в руку твою, и пойди в Рамоф Галаадский.
\vs 2Ki 9:2 Придя туда, отыщи там Ииуя, сына Иосафата, сына Намессиева, и подойди, и вели выступить ему из среды братьев своих, и введи его во внутреннюю комнату;
\vs 2Ki 9:3 и возьми сосуд с елеем, и вылей на голову его, и скажи: <<так говорит Господь: помазую тебя в царя над Израилем>>. Потом отвори дверь, и беги, и не жди.
\vs 2Ki 9:4 И пошел отрок, слуга пророка, в Рамоф Галаадский,
\vs 2Ki 9:5 и пришел, и вот сидят военачальники. И сказал: у меня слово до тебя, военачальник. И сказал Ииуй: до кого из всех нас? И сказал он: до тебя, военачальник.
\vs 2Ki 9:6 И встал он, и вошел в дом. И \bibemph{отрок} вылил елей на голову его, и сказал ему: так говорит Господь Бог Израилев: <<помазую тебя в царя над народом Господним, над Израилем,
\vs 2Ki 9:7 и ты истребишь дом Ахава, господина твоего, чтобы Мне отмстить за кровь рабов Моих пророков и за кровь всех рабов Господних, \bibemph{павших} от руки Иезавели;
\vs 2Ki 9:8 и погибнет весь дом Ахава, и истреблю у Ахава мочащегося к стене, и заключенного и оставшегося в Израиле,
\vs 2Ki 9:9 и сделаю дом Ахава, как дом Иеровоама, сына Наватова, и как дом Ваасы, сына Ахиина;
\vs 2Ki 9:10 Иезавель же съедят псы на поле Изреельском, и никто не похоронит ее>>. И отворил дверь, и убежал.
\vs 2Ki 9:11 И вышел Ииуй к слугам господина своего, и сказали ему: с миром ли? Зачем приходил этот неистовый к тебе? И сказал им: вы знаете этого человека и что он говорит.
\vs 2Ki 9:12 И сказали: неправда, скажи нам. И сказал он: то и то он сказал мне, говоря: <<так говорит Господь: помазую тебя в царя над Израилем>>.
\vs 2Ki 9:13 И поспешили они, и взяли каждый одежду свою, и подостлали ему на самых ступенях, и затрубили трубою, и сказали: воцарился Ииуй!
\vs 2Ki 9:14 И восстал Ииуй, сын Иосафата, сына Намессиева, против Иорама; Иорам же находился со всеми Израильтянами в Рамофе Галаадском на страже против Азаила, царя Сирийского.
\vs 2Ki 9:15 Впрочем сам царь Иорам возвратился, чтобы лечиться в Изрееле от ран, которые причинили ему Сирияне, когда он воевал с Азаилом, царем Сирийским. И сказал Ииуй: если вы согласны [со мною], то пусть никто не уходит из города, чтобы идти подать весть в Изрееле.
\vs 2Ki 9:16 И сел Ииуй на коня, и поехал в Изреель, где лежал Иорам [царь Израильский, для лечения ран, которые причинили ему Сирияне в Рамофе, когда он воевал с Азаилом, царем Сирийским, сильным и могущественным], и куда Охозия, царь Иудейский, пришел посетить Иорама.
\vs 2Ki 9:17 На башне в Изрееле стоял сторож, и увидел он полчище Ииуево, когда оно шло, и сказал: полчище вижу я. И сказал Иорам: возьми всадника, и пошли навстречу им, и пусть скажет: с миром ли?
\vs 2Ki 9:18 И выехал всадник на коне навстречу ему, и сказал: так говорит царь: с миром ли? И сказал Ииуй: что тебе до мира? Поезжай за мною. И донес сторож, и сказал: доехал до них, но не возвращается.
\vs 2Ki 9:19 И послали другого всадника, и он приехал к ним, и сказал: так говорит царь: с миром ли? И сказал Ииуй: что тебе до мира? Поезжай за мною.
\vs 2Ki 9:20 И донес сторож, сказав: доехал до них, и не возвращается, а походка, как будто Ииуя, сына Намессиева, потому что он идет стремительно.
\vs 2Ki 9:21 И сказал Иорам: запрягай. И запрягли колесницу его. И выступил Иорам, царь Израильский, и Охозия, царь Иудейский, каждый на колеснице своей. И выступили навстречу Ииую, и встретились с ним на поле Навуфея Изреелитянина.
\vs 2Ki 9:22 И когда увидел Иорам Ииуя, то сказал: с миром ли Ииуй? И сказал он: какой мир при любодействе Иезавели, матери твоей, и при многих волхвованиях ее?
\vs 2Ki 9:23 И поворотил Иорам руки свои, и побежал, и сказал Охозии: измена, Охозия!
\vs 2Ki 9:24 А Ииуй натянул лук рукою своею, и поразил Иорама между плечами его, и прошла стрела чрез сердце его, и пал он на колеснице своей.
\vs 2Ki 9:25 И сказал Ииуй Бидекару, сановнику своему: возьми, брось его на участок поля Навуфея Изреелитянина, ибо вспомни, как мы с тобою ехали вдвоем сзади Ахава, отца его, и как Господь изрек на него такое пророчество:
\vs 2Ki 9:26 истинно, кровь Навуфея и кровь сыновей его видел Я вчера, говорит Господь, и отмщу тебе на сем поле. Итак возьми, брось его на поле, по слову Господню.
\vs 2Ki 9:27 Охозия, царь Иудейский, увидев сие, побежал по дороге к дому, что в саду. И погнался за ним Ииуй, и сказал: и его бейте на колеснице. \bibemph{Это было} на возвышенности Гур, что при Ивлеаме. И побежал он в Мегиддон, и умер там.
\vs 2Ki 9:28 И отвезли его рабы его в Иерусалим, и похоронили его в гробнице его, с отцами его, в городе Давидовом.
\rsbpar\vs 2Ki 9:29 В одиннадцатый год Иорама, сына Ахавова, воцарился Охозия в Иудее.
\vs 2Ki 9:30 И прибыл Ииуй в Изреель. Иезавель же, получив весть, нарумянила лице свое и украсила голову свою, и глядела в окно.
\vs 2Ki 9:31 Когда Ииуй вошел в ворота, она сказала: мир ли Замврию, убийце государя своего?
\vs 2Ki 9:32 И поднял он лице свое к окну и сказал: кто со мною, кто? И выглянули к нему два, три евнуха.
\vs 2Ki 9:33 И сказал он: выбросьте ее. И выбросили ее. И брызнула кровь ее на стену и на коней, и растоптали ее.
\vs 2Ki 9:34 И пришел Ииуй, и ел, и пил, и сказал: отыщите эту проклятую и похороните ее, так как царская дочь она.
\vs 2Ki 9:35 И пошли хоронить ее, и не нашли от нее ничего, кроме черепа, и ног, и кистей рук.
\vs 2Ki 9:36 И возвратились, и донесли ему. И сказал он: таково было слово Господа, которое Он изрек чрез раба Своего Илию Фесвитянина, сказав: на поле Изреельском съедят псы тело Иезавели,
\vs 2Ki 9:37 и будет труп Иезавели на участке Изреельском, как навоз на поле, так что никто не скажет: это Иезавель.
\vs 2Ki 10:1 У Ахава было семьдесят сыновей в Самарии. И написал Ииуй письма, и послал в Самарию к начальникам Изреельским, старейшинам и воспитателям детей Ахавовых, такого содержания:
\vs 2Ki 10:2 когда придет это письмо к вам, то, так как у вас и сыновья господина вашего, у вас же и колесницы, и кони, и укрепленный город, и оружие,~---
\vs 2Ki 10:3 выберите лучшего и достойнейшего из сыновей государя своего, и посадите на престол отца его, и воюйте за дом государя своего.
\vs 2Ki 10:4 Они испугались чрезвычайно и сказали: вот, два царя не устояли перед ним, как же нам устоять?
\vs 2Ki 10:5 И послал начальствующий над домом \bibemph{царским}, и градоначальник, и старейшины, и воспитатели к Ииую, сказать: мы рабы твои, и что скажешь нам, то и сделаем; мы никого не поставим царем, что угодно тебе, то и делай.
\vs 2Ki 10:6 И написал он к ним письмо во второй раз такое: если вы мои и слову моему повинуетесь, то возьмите головы сыновей государя своего, и придите ко мне завтра в это время в Изреель. (Царских же сыновей было семьдесят человек; воспитывали их знатнейшие в городе.)
\vs 2Ki 10:7 Когда пришло к ним письмо, они взяли царских сыновей, и закололи их~--- семьдесят человек, и положили головы их в корзины, и послали к нему в Изреель.
\vs 2Ki 10:8 И пришел посланный, и донес ему, и сказал: принесли головы сыновей царских. И сказал он: разложите их на две груды у входа в ворота, до утра.
\vs 2Ki 10:9 Поутру он вышел, и стал, и сказал всему народу: вы невиновны. Вот я восстал против государя моего и умертвил его, а их всех кто убил?
\vs 2Ki 10:10 Знайте же теперь, что не падет на землю ни одно слово Господа, которое Он изрек о доме Ахава; Господь сделал то, что изрек чрез раба Своего Илию.
\vs 2Ki 10:11 И умертвил Ииуй всех оставшихся из дома Ахава в Изрееле, и всех вельмож его, и близких его, и священников его, так что не осталось от него ни одного уцелевшего.
\vs 2Ki 10:12 И встал, и пошел, и пришел в Самарию. Находясь на пути при Беф-Екеде пастушеском,
\vs 2Ki 10:13 встретил Ииуй братьев Охозии, царя Иудейского, и сказал: кто вы? Они сказали: мы братья Охозии, идем узнать о здоровье сыновей царя и сыновей государыни.
\vs 2Ki 10:14 И сказал он: возьмите их живых. И взяли их живых, и закололи их~--- сорок два человека, при колодезе Беф-Екеда, и не осталось из них ни одного.
\vs 2Ki 10:15 И поехал оттуда, и встретился с Ионадавом, сыном Рихавовым, \bibemph{шедшим} навстречу ему, и приветствовал его, и сказал ему: расположено ли твое сердце так, как мое сердце к твоему сердцу? И сказал Ионадав: да. Если так, то дай руку твою. И подал он руку свою, и приподнял он его к себе в колесницу,
\vs 2Ki 10:16 и сказал: поезжай со мною, и смотри на мою ревность о Господе. И посадили его в колесницу.
\vs 2Ki 10:17 Прибыв в Самарию, он убил всех, остававшихся у Ахава в Самарии, так что совсем истребил его, по слову Господа, которое Он изрек Илии.
\rsbpar\vs 2Ki 10:18 И собрал Ииуй весь народ и сказал им: Ахав мало служил Ваалу; Ииуй будет служить ему более.
\vs 2Ki 10:19 Итак созовите ко мне всех пророков Ваала, всех служителей его и всех священников его, чтобы никто не был в отсутствии, потому что у меня будет великая жертва Ваалу. А всякий, кто не явится, не останется жив. Ииуй делал \bibemph{это} с хитрым намерением, чтобы истребить служителей Ваала.
\vs 2Ki 10:20 И сказал Ииуй: назначьте праздничное собрание ради Ваала. И провозгласили \bibemph{собрание}.
\vs 2Ki 10:21 И послал Ииуй по всему Израилю, и пришли все служители Ваала; не оставалось ни одного человека, кто бы не пришел; и вошли в дом Ваалов, и наполнился дом Ваалов от края до края.
\vs 2Ki 10:22 И сказал он хранителю одежд: принеси одежду для всех служителей Ваала. И он принес им одежду.
\vs 2Ki 10:23 И вошел Ииуй с Ионадавом, сыном Рихавовым, в дом Ваалов, и сказал служителям Ваала: разведайте и разглядите, не находится ли у вас кто-нибудь из служителей Господних, так как здесь должны находиться только одни служители Ваала.
\vs 2Ki 10:24 И приступили они к совершению жертв и всесожжений. А Ииуй поставил вне \bibemph{дома} восемьдесят человек и сказал: душа того, у которого спасется кто-либо из людей, которых я отдаю вам в руки, будет вместо души \bibemph{спасшегося}.
\rsbpar\vs 2Ki 10:25 Когда кончено было всесожжение, сказал Ииуй скороходам и начальникам: пойдите, бейте их, чтобы ни один не ушел. И поразили их острием меча и бросили \bibemph{их} скороходы и начальники, и пошли в город, где было капище Ваалово.
\vs 2Ki 10:26 И вынесли статуи из капища Ваалова и сожгли их.
\vs 2Ki 10:27 И разбили статую Ваала, и разрушили капище Ваалово; и сделали из него место нечистот, до сего дня.
\vs 2Ki 10:28 И истребил Ииуй Ваала с земли Израильской.
\vs 2Ki 10:29 Впрочем от грехов Иеровоама, сына Наватова, который ввел Израиля в грех, от них не отступал Ииуй,~--- от золотых тельцов, которые в Вефиле и которые в Дане.
\rsbpar\vs 2Ki 10:30 И сказал Господь Ииую: за то, что ты охотно сделал, что было праведно в очах Моих, выполнил над домом Ахавовым все, что было на сердце у Меня, сыновья твои до четвертого рода будут сидеть на престоле Израилевом.
\vs 2Ki 10:31 Но Ииуй не старался ходить в законе Господа Бога Израилева, от всего сердца. Он не отступал от грехов Иеровоама, который ввел Израиля в грех.
\rsbpar\vs 2Ki 10:32 В те дни начал Господь отрезать части от Израильтян, и поражал их Азаил во всем пределе Израилевом,
\vs 2Ki 10:33 на восток от Иордана, всю землю Галаад, \bibemph{колено} Гадово, Рувимово, Манассиино, \bibemph{начиная} от Ароера, который при потоке Арноне, и Галаад и Васан.
\rsbpar\vs 2Ki 10:34 Прочее об Ииуе и обо всем, что он сделал, и о мужественных подвигах его написано в летописи царей Израильских.
\vs 2Ki 10:35 И почил Ииуй с отцами своими, и похоронили его в Самарии. И воцарился Иоахаз, сын его, вместо него.
\vs 2Ki 10:36 Времени же царствования Ииуева над Израилем, в Самарии, было двадцать восемь лет.
\vs 2Ki 11:1 Гофолия, мать Охозии, видя, что сын ее умер, встала и истребила все царское племя.
\vs 2Ki 11:2 Но Иосавеф, дочь царя Иорама, сестра Охозии, взяла Иоаса, сына Охозии, и тайно увела его из среды умерщвляемых сыновей царских, его и кормилицу его, в постельную комнату; и скрыли его от Гофолии, и он не умерщвлен.
\vs 2Ki 11:3 И был он с нею скрываем в доме Господнем шесть лет, между тем как Гофолия царствовала над землею.
\rsbpar\vs 2Ki 11:4 В седьмой год послал Иодай, и взял сотников из телохранителей и скороходов, и привел их к себе в дом Господень, и сделал с ними договор, и взял с них клятву в доме Господнем, и показал им царского сына.
\vs 2Ki 11:5 И дал им приказание, сказав: вот что вы сделайте: третья часть из вас, из приходящих в субботу, будет содержать стражу при царском доме;
\vs 2Ki 11:6 третья часть у ворот Сур, и третья часть у ворот сзади телохранителей, и содержите стражу дома, чтобы не было повреждения;
\vs 2Ki 11:7 и две части из вас, из всех отходящих в субботу, будут содержать стражу при доме Господнем для царя;
\vs 2Ki 11:8 и окружите царя со всех сторон, каждый с оружием своим в руке своей; и кто вошел бы в ряды, тот да будет умерщвлен. И будьте при царе, когда он выходит и когда входит.
\vs 2Ki 11:9 И сделали сотники всё, что приказал Иодай священник, и взяли каждый людей своих, приходящих в субботу и отходящих в субботу, и пришли к Иодаю священнику.
\vs 2Ki 11:10 И раздал священник сотникам копья и щиты царя Давида, которые были в доме Господнем.
\vs 2Ki 11:11 И стали скороходы, каждый с оружием в руке своей, от правой стороны дома до левой стороны дома, у жертвенника и у дома, вокруг царя.
\vs 2Ki 11:12 И вывел он царского сына, и возложил на него \bibemph{царский} венец и украшения, и воцарили его, и помазали его, и рукоплескали и восклицали: да живет царь!
\vs 2Ki 11:13 И услышала Гофолия голос бегущего народа, и пошла к народу в дом Господень.
\vs 2Ki 11:14 И видит, и вот царь стоит на возвышении, по обычаю, и князья и трубы подле царя; и весь народ земли веселится, и трубят трубами. И разодрала Гофолия одежды свои, и закричала: заговор! заговор!
\vs 2Ki 11:15 И дал приказание Иодай священник сотникам, начальствующим над войском, и сказал им: <<выведите ее за ряды, а кто пойдет за нею, умерщвляйте мечом>>, так как думал священник, чтобы не умертвили ее в доме Господнем.
\vs 2Ki 11:16 И дали ей место, и она прошла чрез вход конский к дому царскому, и умерщвлена там.
\rsbpar\vs 2Ki 11:17 И заключил Иодай завет между Господом и между царем и народом, чтоб он был народом Господним, и между царем и народом.
\vs 2Ki 11:18 И пошел весь народ земли в дом Ваала, и разрушили жертвенники его, и изображения его совершенно разбили, и Матфана, жреца Ваалова, убили пред жертвенниками. И учредил священник наблюдение над домом Господним.
\vs 2Ki 11:19 И взял сотников и телохранителей и скороходов и весь народ земли, и проводили царя из дома Господня, и пришли по дороге чрез ворота телохранителей в дом царский; и он воссел на престоле царей.
\vs 2Ki 11:20 И веселился весь народ земли, и город успокоился. А Гофолию умертвили мечом в царском доме.
\vs 2Ki 11:21 Семи лет был Иоас, когда воцарился.
\vs 2Ki 12:1 В седьмой год Ииуя воцарился Иоас и сорок лет царствовал в Иерусалиме. Имя матери его Цивья, из Вирсавии.
\vs 2Ki 12:2 И делал Иоас угодное в очах Господних во все дни свои, доколе наставлял его священник Иодай;
\vs 2Ki 12:3 только высоты не были отменены; народ еще приносил жертвы и курения на высотах.
\vs 2Ki 12:4 И сказал Иоас священникам: все серебро посвящаемое, которое приносят в дом Господень, серебро от приходящих, серебро, \bibemph{вносимое} за каждую душу по оценке, все серебро, сколько кому приходит на сердце принести в дом Господень,
\vs 2Ki 12:5 пусть берут священники себе, каждый от своего знакомого, и пусть исправляют они поврежденное в храме, везде, где найдется повреждение.
\vs 2Ki 12:6 Но как до двадцать третьего года царя Иоаса священники не исправляли повреждений в храме,
\vs 2Ki 12:7 то царь Иоас позвал священника Иодая и священников и сказал им: почему вы не исправляете повреждений в храме? Не берите же отныне серебра у знакомых своих, а на \bibemph{починку} повреждений в храме отдайте его.
\vs 2Ki 12:8 И согласились священники не брать серебра у народа на исправление повреждений в храме.
\vs 2Ki 12:9 И взял священник Иодай один ящик, и сделал отверстие сверху его, и поставил его подле жертвенника на правой стороне, где входили в дом Господень. И полагали туда священники, стоящие на страже у порога, все серебро, приносимое в дом Господень.
\vs 2Ki 12:10 И когда видели, что много серебра в ящике, приходили писец царский и первосвященник, и завязывали \bibemph{в мешки}, и пересчитывали серебро, найденное в доме Господнем;
\vs 2Ki 12:11 и отдавали сосчитанное серебро в руки производителям работ, приставленным к дому Господню, а сии издерживали его на плотников и строителей, работавших в доме Господнем,
\vs 2Ki 12:12 и на делателей стен и на каменотесов, также на покупку дерев и тесаных камней, для починки повреждений в доме Господнем, и на все, что расходовалось для поддержания храма.
\vs 2Ki 12:13 Но не сделано было для дома Господня серебряных блюд, ножей, чаш \bibemph{для окропления}, труб, всяких сосудов золотых и сосудов серебряных из серебра, приносимого в дом Господень,
\vs 2Ki 12:14 а производителям работ отдавали его, и починивали им дом Господень.
\vs 2Ki 12:15 И не требовали отчета от тех людей, которым поручали серебро для раздачи производителям работ, ибо они действовали честно.
\vs 2Ki 12:16 Серебро за жертву о преступлении и серебро за жертву о грехе не вносилось в дом Господень: священникам оно принадлежало.
\rsbpar\vs 2Ki 12:17 Тогда выступил в поход Азаил, царь Сирийский, и пошел войною на Геф, и взял его; и вознамерился Азаил идти на Иерусалим.
\vs 2Ki 12:18 Но Иоас, царь Иудейский, взял все пожертвованное, что пожертвовали \bibemph{храму} Иосафат, и Иорам и Охозия, отцы его, цари Иудейские, и что он сам пожертвовал, и все золото, найденное в сокровищницах дома Господня и дома царского, и послал Азаилу, царю Сирийскому; и он отступил от Иерусалима.
\rsbpar\vs 2Ki 12:19 Прочее об Иоасе и обо всем, что он сделал, написано в летописи царей Иудейских.
\rsbpar\vs 2Ki 12:20 И восстали слуги его, и составили заговор, и убили Иоаса в доме Милло, на дороге к Силле.
\vs 2Ki 12:21 Его убили слуги его: Иозакар, сын Шимеаты, и Иегозавад, сын Шомеры; и он умер, и похоронили его с отцами его в городе Давидовом. И воцарился Амасия, сын его, вместо него.
\vs 2Ki 13:1 В двадцать третий год Иоаса, сына Охозиина, царя Иудейского, воцарился Иоахаз, сын Ииуя, над Израилем в Самарии, \bibemph{и царствовал} семнадцать лет,
\vs 2Ki 13:2 и делал неугодное в очах Господних, и ходил в грехах Иеровоама, сына Наватова, который ввел Израиля в грех, и не отставал от них.
\vs 2Ki 13:3 И возгорелся гнев Господа на Израиля, и Он предавал их в руку Азаила, царя Сирийского, и в руку Венадада, сына Азаилова, во все дни.
\vs 2Ki 13:4 И помолился Иоахаз лицу Господню, и услышал его Господь, потому что видел стеснение Израильтян, как теснил их царь Сирийский.
\vs 2Ki 13:5 И дал Господь Израильтянам избавителя, и вышли они из-под руки Сириян, и жили сыны Израилевы в шатрах своих, как вчера и третьего дня.
\vs 2Ki 13:6 Однако ж не отступали от грехов дома Иеровоама, который ввел Израиля в грех; ходили в них, и дубрава стояла в Самарии.
\vs 2Ki 13:7 У Иоахаза оставалось войска только пятьдесят всадников, десять колесниц и десять тысяч пеших, оттого, что истребил их царь Сирийский и обратил их в прах на попрание.
\rsbpar\vs 2Ki 13:8 Прочее об Иоахазе и обо всем, что он сделал, и о мужественных подвигах его, написано в летописи царей Израильских.
\vs 2Ki 13:9 И почил Иоахаз с отцами своими, и похоронили его в Самарии. И воцарился Иоас, сын его, вместо него.
\rsbpar\vs 2Ki 13:10 В тридцать седьмой год Иоаса, царя Иудейского, воцарился Иоас, сын Иоахазов, над Израилем в Самарии, \bibemph{и царствовал} шестнадцать лет,
\vs 2Ki 13:11 и делал неугодное в очах Господних; не отставал от всех грехов Иеровоама, сына Наватова, который ввел Израиля в грех, но ходил в них.
\rsbpar\vs 2Ki 13:12 Прочее об Иоасе и обо всем, что он сделал, и о мужественных подвигах его, как он воевал с Амасиею, царем Иудейским, написано в летописи царей Израильских.
\vs 2Ki 13:13 И почил Иоас с отцами своими, а Иеровоам сел на престоле его. И погребен Иоас в Самарии с царями Израильскими.
\rsbpar\vs 2Ki 13:14 Елисей заболел болезнью, от которой \bibemph{потом} и умер. И пришел к нему Иоас, царь Израильский, и плакал над ним, и говорил: отец мой! отец мой! колесница Израиля и конница его!
\vs 2Ki 13:15 И сказал ему Елисей: возьми лук и стрелы. И взял он лук и стрелы.
\vs 2Ki 13:16 И сказал царю Израильскому: положи руку твою на лук. И положил он руку свою. И наложил Елисей руки свои на руки царя,
\vs 2Ki 13:17 и сказал: отвори окно на восток. И он отворил. И сказал Елисей: выстрели. И он выстрелил. И сказал: эта стрела избавления от Господа и стрела избавления против Сирии, и ты поразишь Сириян в Афеке вконец.
\vs 2Ki 13:18 И сказал [Елисей]: возьми стрелы. И он взял. И сказал царю Израильскому: бей по земле. И ударил он три раза, и остановился.
\vs 2Ki 13:19 И разгневался на него человек Божий, и сказал: надобно было бы бить пять или шесть раз, тогда ты побил бы Сириян совершенно, а теперь \bibemph{только} три раза поразишь Сириян.
\vs 2Ki 13:20 И умер Елисей, и похоронили его. И полчища Моавитян пришли в землю в следующем году.
\vs 2Ki 13:21 И было, что, когда погребали одного человека, то, увидев это полчище, \bibemph{погребавшие} бросили того человека в гроб Елисеев; и он при падении своем коснулся костей Елисея, и ожил, и встал на ноги свои.
\rsbpar\vs 2Ki 13:22 Азаил, царь Сирийский, теснил Израильтян во все дни Иоахаза.
\vs 2Ki 13:23 Но Господь умилосердился над ними, и помиловал их, и обратился к ним ради завета Своего с Авраамом, Исааком и Иаковом, и не хотел истребить их, и не отверг их от лица Своего доныне.
\vs 2Ki 13:24 И умер Азаил, царь Сирийский, и воцарился Венадад, сын его, вместо него.
\vs 2Ki 13:25 И взял назад Иоас, сын Иоахаза, из руки Венадада, сына Азаила, города, которые он взял войною из руки отца его Иоахаза. Три раза разбил его Иоас и возвратил города Израилевы.
\vs 2Ki 14:1 Во второй год Иоаса, сына Иоахазова, царя Израильского, воцарился Амасия, сын Иоаса, царь Иудейский:
\vs 2Ki 14:2 двадцати пяти лет был он, когда воцарился, и двадцать девять лет царствовал в Иерусалиме. Имя матери его Иегоаддань, из Иерусалима.
\vs 2Ki 14:3 И делал он угодное в очах Господних, впрочем не так, как отец его Давид: он во всем поступал так, как отец его Иоас.
\vs 2Ki 14:4 Только высоты не были отменены: народ совершал еще жертвы и курения на высотах.
\rsbpar\vs 2Ki 14:5 Когда утвердилось царство в руках его, тогда он умертвил слуг своих, убивших царя, отца его.
\vs 2Ki 14:6 Но детей убийц не умертвил, так как написано в книге закона Моисеева, в которой заповедал Господь, говоря: <<не должны быть наказываемы смертью отцы за детей, и дети не должны быть наказываемы смертью за отцов, но каждый за свое преступление должен быть наказываем смертью>>.
\vs 2Ki 14:7 Он поразил десять тысяч Идумеян на долине Соляной, и взял Селу войною, и дал ей имя Иокфеил, которое \bibemph{остается и} до сего дня.
\rsbpar\vs 2Ki 14:8 Тогда послал Амасия послов к Иоасу, царю Израильскому, сыну Иоахаза, сына Ииуева, сказать: выйди, повидаемся лично.
\vs 2Ki 14:9 И послал Иоас, царь Израильский, к Амасии, царю Иудейскому, сказать: терн, который на Ливане, послал к кедру, который на Ливане же, сказать: <<отдай дочь свою в жену сыну моему>>. Но прошли дикие звери, что на Ливане, и истоптали этот терн.
\vs 2Ki 14:10 Ты поразил Идумеян, и возгордилось сердце твое. Величайся и сиди у себя дома. К чему тебе затевать ссору на свою беду? Падешь ты и Иуда с тобою.
\vs 2Ki 14:11 Но не послушался Амасия. И выступил Иоас, царь Израильский, и увиделись лично он и Амасия, царь Иудейский, в Вефсамисе, что в Иудее.
\vs 2Ki 14:12 И разбиты были Иудеи Израильтянами, и разбежались по шатрам своим.
\vs 2Ki 14:13 И Амасию, царя Иудейского, сына Иоаса, сына Охозиина, захватил Иоас, царь Израильский, в Вефсамисе, и пошел в Иерусалим и разрушил стену Иерусалимскую от ворот Ефремовых до ворот угольных на четыреста локтей.
\vs 2Ki 14:14 И взял все золото и серебро, и все сосуды, какие нашлись в доме Господнем и в сокровищницах царского дома, и заложников, и возвратился в Самарию.
\rsbpar\vs 2Ki 14:15 Прочее об Иоасе, что он сделал, и о мужественных подвигах его, и как он воевал с Амасиею, царем Иудейским, написано в летописи царей Израильских.
\vs 2Ki 14:16 И почил Иоас с отцами своими, и погребен в Самарии с царями Израильскими. И воцарился Иеровоам, сын его, вместо него.
\rsbpar\vs 2Ki 14:17 И жил Амасия, сын Иоасов, царь Иудейский, по смерти Иоаса, сына Иоахазова, царя Израильского, пятнадцать лет.
\rsbpar\vs 2Ki 14:18 Прочие дела Амасии записаны в летописи царей Иудейских.
\vs 2Ki 14:19 И составили против него заговор в Иерусалиме, и убежал он в Лахис. И послали за ним в Лахис, и умертвили его там.
\vs 2Ki 14:20 И привезли его на конях, и погребен он был в Иерусалиме с отцами своими в городе Давидовом.
\vs 2Ki 14:21 И взял весь народ Иудейский Азарию, которому было шестнадцать лет, и воцарили его вместо отца его Амасии.
\vs 2Ki 14:22 Он обстроил Елаф, и возвратил его Иуде, после того как царь почил с отцами своими.
\rsbpar\vs 2Ki 14:23 В пятнадцатый год Амасии, сына Иоасова, царя Иудейского, воцарился Иеровоам, сын Иоасов, царь Израильский, в Самарии, и \bibemph{царствовал} сорок один год,
\vs 2Ki 14:24 и делал он неугодное в очах Господних: не отступал от всех грехов Иеровоама, сына Наватова, который ввел Израиля в грех.
\vs 2Ki 14:25 Он восстановил пределы Израиля, от входа в Емаф до моря пустыни, по слову Господа Бога Израилева, которое Он изрек чрез раба Своего Иону, сына Амафиина, пророка из Гафхефера,
\vs 2Ki 14:26 ибо Господь видел бедствие Израиля, весьма горькое, так что не оставалось ни заключенного, ни оставшегося, и не было помощника у Израиля.
\vs 2Ki 14:27 И не восхотел Господь искоренить имя Израильтян из поднебесной, и спас их рукою Иеровоама, сына Иоасова.
\rsbpar\vs 2Ki 14:28 Прочее об Иеровоаме и обо всем, что он сделал, и о мужественных подвигах его, как он воевал и как возвратил Израилю Дамаск и Емаф, принадлежавший Иуде, написано в летописи царей Израильских.
\vs 2Ki 14:29 И почил Иеровоам с отцами своими, с царями Израильскими. И воцарился Захария, сын его, вместо него.
\vs 2Ki 15:1 В двадцать седьмой год Иеровоама, царя Израильского, воцарился Азария, сын Амасии, царь Иудейский:
\vs 2Ki 15:2 шестнадцати лет был он, когда воцарился, и пятьдесят два года царствовал в Иерусалиме. Имя матери его Иехолия, из Иерусалима.
\vs 2Ki 15:3 Он делал угодное в очах Господних во всем так, как поступал Амасия, отец его.
\vs 2Ki 15:4 Только высоты не были отменены: народ совершал еще жертвы и курения на высотах.
\vs 2Ki 15:5 И поразил Господь царя, и был он прокаженным до дня смерти своей и жил в отдельном доме. А Иофам, сын царя, \bibemph{начальствовал} над дворцом и управлял народом земли.
\rsbpar\vs 2Ki 15:6 Прочее об Азарии и обо всем, что он сделал, написано в летописи царей Иудейских.
\vs 2Ki 15:7 И почил Азария с отцами своими, и похоронили его с отцами его в городе Давидовом. И воцарился Иофам, сын его, вместо него.
\rsbpar\vs 2Ki 15:8 В тридцать восьмой год Азарии, царя Иудейского, воцарился Захария, сын Иеровоама, над Израилем в Самарии \bibemph{и царствовал} шесть месяцев.
\vs 2Ki 15:9 Он делал неугодное в очах Господних, как делали отцы его: не отставал от грехов Иеровоама, сына Наватова, который ввел Израиля в грех.
\vs 2Ki 15:10 И составил против него заговор Селлум, сын Иависа, и поразил его пред народом и убил его, и воцарился вместо него.
\rsbpar\vs 2Ki 15:11 Прочее о Захарии написано в летописи царей Израильских.
\vs 2Ki 15:12 Таково было слово Господа, которое он изрек Ииую, сказав: сыновья твои до четвертого рода будут сидеть на престоле Израилевом. И сбылось так.
\rsbpar\vs 2Ki 15:13 Селлум, сын Иависа, воцарился в тридцать девятый год Азарии, царя Иудейского, и царствовал один месяц в Самарии.
\vs 2Ki 15:14 И пошел Менаим, сын Гадия из Фирцы, и пришел в Самарию, и поразил Селлума, сына Иависова, в Самарии и умертвил его, и воцарился вместо него.
\rsbpar\vs 2Ki 15:15 Прочее о Селлуме и о заговоре его, который он составил, написано в летописи царей Израильских.
\vs 2Ki 15:16 И поразил Менаим Типсах и всех, которые были в нем и в пределах его, \bibemph{начиная} от Фирцы, за то, что \bibemph{город} не отворил \bibemph{ворот}, и разбил \bibemph{его}, и всех беременных женщин в нем разрубил.
\rsbpar\vs 2Ki 15:17 В тридцать девятом году Азарии, царя Иудейского, воцарился Менаим, сын Гадия, над Израилем \bibemph{и царствовал} десять лет в Самарии;
\vs 2Ki 15:18 и делал он неугодное в очах Господних; не отставал от грехов Иеровоама, сына Наватова, который ввел Израиля в грех, во все дни свои.
\vs 2Ki 15:19 Тогда пришел Фул, царь Ассирийский, на землю [Израилеву]. И дал Менаим Фулу тысячу талантов серебра, чтобы руки его были за него и чтобы утвердить царство в руке своей.
\vs 2Ki 15:20 И разложил Менаим это серебро на Израильтян, на всех людей богатых, по пятидесяти сиклей серебра на каждого человека, чтобы отдать царю Ассирийскому. И пошел назад царь Ассирийский и не остался там в земле.
\rsbpar\vs 2Ki 15:21 Прочее о Менаиме и обо всем, что он сделал, написано в летописи царей Израильских.
\vs 2Ki 15:22 И почил Менаим с отцами своими. И воцарился Факия, сын его, вместо него.
\rsbpar\vs 2Ki 15:23 В пятидесятый год Азарии, царя Иудейского, воцарился Факия, сын Менаима, над Израилем в Самарии \bibemph{и царствовал} два года;
\vs 2Ki 15:24 и делал он неугодное в очах Господних; не отставал от грехов Иеровоама, сына Наватова, который ввел Израиля в грех.
\vs 2Ki 15:25 И составил против него заговор Факей, сын Ремалии, сановник его, и поразил его в Самарии в палате царского дома, с Арговом и Арием, имея с собою пятьдесят человек Галаадитян, и умертвил его, и воцарился вместо него.
\rsbpar\vs 2Ki 15:26 Прочее о Факии и обо всем, что он сделал, написано в летописи царей Израильских.
\rsbpar\vs 2Ki 15:27 В пятьдесят второй год Азарии, царя Иудейского, воцарился Факей, сын Ремалии, над Израилем в Самарии \bibemph{и царствовал} двадцать лет;
\vs 2Ki 15:28 и делал он неугодное в очах Господних: не отставал от грехов Иеровоама, сына Наватова, который ввел Израиля в грех.
\vs 2Ki 15:29 Во дни Факея, царя Израильского, пришел Феглаффелласар, царь Ассирийский, и взял Ион, Авел-Беф-Мааху, и Ианох, и Кедес, и Асор, и Галаад, и Галилею, всю землю Неффалимову, и переселил их в Ассирию.
\vs 2Ki 15:30 И составил заговор Осия, сын Илы, против Факея, сына Ремалиина, и поразил его, и умертвил его, и воцарился вместо него в двадцатый год Иоафама, сына Озиина.
\rsbpar\vs 2Ki 15:31 Прочее о Факее и обо всем, что он сделал, написано в летописи царей Израильских.
\rsbpar\vs 2Ki 15:32 Во второй год Факея, сына Ремалиина, царя Израильского, воцарился Иоафам, сын Озии, царя Иудейского.
\vs 2Ki 15:33 Двадцати пяти лет был он, когда воцарился, и шестнадцать лет царствовал в Иерусалиме. Имя матери его Иеруша, дочь Садока.
\vs 2Ki 15:34 Он делал угодное в очах Господних: во всем, как поступал Озия, отец его, так поступал и он.
\vs 2Ki 15:35 Только высоты не были отменены: народ совершал еще жертвы и курения на высотах. Он построил верхние ворота при доме Господнем.
\rsbpar\vs 2Ki 15:36 Прочее об Иоафаме и обо всем, что он сделал, написано в летописи царей Иудейских.
\vs 2Ki 15:37 В те дни начал Господь посылать на Иудею Рецина, царя Сирийского, и Факея, сына Ремалиина.
\vs 2Ki 15:38 И почил Иоафам с отцами своими, и погребен с отцами своими в городе Давида, отца его. И воцарился Ахаз, сын его, вместо него.
\vs 2Ki 16:1 В семнадцатый год Факея, сына Ремалиина, воцарился Ахаз, сын Иоафама, царя Иудейского.
\vs 2Ki 16:2 Двадцати лет был Ахаз, когда воцарился, и шестнадцать лет царствовал в Иерусалиме, и не делал угодного в очах Господа Бога своего, как Давид, отец его,
\vs 2Ki 16:3 но ходил путем царей Израильских, и даже сына своего провел чрез огонь, \bibemph{подражая} мерзостям народов, которых прогнал Господь от лица сынов Израилевых,
\vs 2Ki 16:4 и совершал жертвы и курения на высотах и на холмах и под всяким тенистым деревом.
\vs 2Ki 16:5 Тогда пошел Рецин, царь Сирийский, и Факей, сын Ремалиин, царь Израильский, против Иерусалима, чтобы завоевать его, и держали Ахаза в осаде, но одолеть не могли.
\vs 2Ki 16:6 В то время Рецин, царь Сирийский, возвратил Сирии Елаф и изгнал Иудеев из Елафа; и Идумеяне вступили в Елаф, и живут там до сего дня.
\vs 2Ki 16:7 И послал Ахаз послов к Феглаффелласару, царю Ассирийскому, сказать: раб твой и сын твой я; приди и защити меня от руки царя Сирийского и от руки царя Израильского, восставших на меня.
\vs 2Ki 16:8 И взял Ахаз серебро и золото, какое нашлось в доме Господнем и в сокровищницах дома царского, и послал царю Ассирийскому в дар.
\vs 2Ki 16:9 И послушал его царь Ассирийский; и пошел царь Ассирийский в Дамаск, и взял его, и переселил жителей его в Кир, а Рецина умертвил.
\vs 2Ki 16:10 И пошел царь Ахаз навстречу Феглаффелласару, царю Ассирийскому, в Дамаск, и увидел жертвенник, который в Дамаске, и послал царь Ахаз к Урии священнику изображение жертвенника и чертеж всего устройства его.
\vs 2Ki 16:11 И построил священник Урия жертвенник по образцу, который прислал царь Ахаз из Дамаска; и сделал так священник Урия до прибытия царя Ахаза из Дамаска.
\vs 2Ki 16:12 И пришел царь из Дамаска, и увидел царь жертвенник, и подошел царь к жертвеннику, и принес на нем жертву;
\vs 2Ki 16:13 и сожег всесожжение свое и хлебное приношение, и совершил возлияние свое, и окропил кровью мирной жертвы свой жертвенник.
\vs 2Ki 16:14 А медный жертвенник, который пред лицем Господним, он передвинул от лицевой стороны храма, с \bibemph{места} между жертвенником \bibemph{новым} и домом Господним, и поставил его сбоку \bibemph{сего} жертвенника на север.
\vs 2Ki 16:15 И дал приказание царь Ахаз священнику Урии, сказав: на большом жертвеннике сожигай утреннее всесожжение и вечернее хлебное приношение, и всесожжение от царя и хлебное приношение от него, и всесожжение от всех людей земли и хлебное приношение от них, и возлияние от них, и всякою кровью всесожжений и всякою кровью жертв окропляй его, а жертвенник медный останется до моего усмотрения.
\vs 2Ki 16:16 И сделал священник Урия все так, как приказал царь Ахаз.
\vs 2Ki 16:17 И обломал царь Ахаз ободки у подстав, и снял с них умывальницы, и море снял с медных волов, которые \bibemph{были} под ним, и поставил его на каменный пол.
\vs 2Ki 16:18 И отменил крытый субботний ход, который построили при храме, и внешний царский вход к дому Господню, ради царя Ассирийского.
\rsbpar\vs 2Ki 16:19 Прочее об Ахазе, что он сделал, написано в летописи царей Иудейских.
\vs 2Ki 16:20 И почил Ахаз с отцами своими, и погребен с отцами своими в городе Давидовом. И воцарился Езекия, сын его, вместо него.
\vs 2Ki 17:1 В двенадцатый год Ахаза, царя Иудейского, воцарился Осия, сын Илы, в Самарии над Израилем \bibemph{и царствовал} девять лет.
\vs 2Ki 17:2 И делал он неугодное в очах Господних, но не так, как цари Израильские, которые были прежде него.
\vs 2Ki 17:3 Против него выступил Салманассар, царь Ассирийский, и сделался Осия подвластным ему и давал ему дань.
\vs 2Ki 17:4 И заметил царь Ассирийский в Осии измену, так как он посылал послов к Сигору, царю Египетскому, и не доставлял дани царю Ассирийскому каждый год; и взял его царь Ассирийский под стражу, и заключил его в дом темничный.
\vs 2Ki 17:5 И пошел царь Ассирийский на всю землю, и приступил к Самарии, и держал ее в осаде три года.
\vs 2Ki 17:6 В девятый год Осии взял царь Ассирийский Самарию, и переселил Израильтян в Ассирию, и поселил их в Халахе и в Хаворе, при реке Гозан, и в городах Мидийских.
\rsbpar\vs 2Ki 17:7 Когда стали грешить сыны Израилевы пред Господом Богом своим, Который вывел их из земли Египетской, из-под руки фараона, царя Египетского, и стали чтить богов иных,
\vs 2Ki 17:8 и стали поступать по обычаям народов, которых прогнал Господь от лица сынов Израилевых, и \bibemph{по обычаям} царей Израильских, как поступали они;
\vs 2Ki 17:9 и стали делать сыны Израилевы дела неугодные Господу Богу своему, и построили себе высоты во всех городах своих, \bibemph{начиная} от сторожевой башни до укрепленного города,
\vs 2Ki 17:10 и поставили у себя статуи и изображения Астарт на всяком высоком холме и под всяким тенистым деревом,
\vs 2Ki 17:11 и стали там совершать курения на всех высотах, подобно народам, которых изгнал от них Господь, и делали худые дела, прогневляющие Господа,
\vs 2Ki 17:12 и служили идолам, о которых говорил им Господь: <<не делайте сего>>;
\vs 2Ki 17:13 тогда Господь чрез всех пророков Своих, чрез всякого прозорливца предостерегал Израиля и Иуду, говоря: возвратитесь со злых путей ваших и соблюдайте заповеди Мои, уставы Мои, по всему учению, которое Я заповедал отцам вашим и которое Я преподал вам чрез рабов Моих, пророков.
\vs 2Ki 17:14 Но они не слушали и ожесточили выю свою, как была выя отцов их, которые не веровали в Господа, Бога своего;
\vs 2Ki 17:15 и презирали уставы Его, и завет Его, который Он заключил с отцами их, и откровения Его, какими Он предостерегал их, и пошли вслед суеты и осуетились, и вслед народов окрестных, о которых Господь заповедал им, чтобы не поступали так, как они,
\vs 2Ki 17:16 и оставили все заповеди Господа Бога своего, и сделали себе литые изображения двух тельцов, и устроили дубраву, и поклонялись всему воинству небесному, и служили Ваалу,
\vs 2Ki 17:17 и проводили сыновей своих и дочерей своих чрез огонь, и гадали, и волшебствовали, и предались тому, чтобы делать неугодное в очах Господа и прогневлять Его.
\rsbpar\vs 2Ki 17:18 И прогневался Господь сильно на Израильтян, и отверг их от лица Своего. Не осталось никого, кроме одного колена Иудина.
\vs 2Ki 17:19 И Иуда также не соблюдал заповедей Господа Бога своего, и поступал по обычаям Израильтян, как поступали они.
\vs 2Ki 17:20 И отвратился Господь от всех потомков Израиля, и смирил их, и отдавал их в руки грабителям, и наконец отверг их от лица Своего.
\vs 2Ki 17:21 Израильтяне отторглись от дома Давидова и воцарили Иеровоама, сына Наватова; и отклонил Иеровоам Израильтян от Господа, и вовлек их в великий грех.
\vs 2Ki 17:22 И поступали сыны Израилевы по всем грехам Иеровоама, какие он делал, не отставали от них,
\vs 2Ki 17:23 доколе Господь не отверг Израиля от лица Своего, как говорил чрез всех рабов Своих, пророков. И переселен Израиль из земли своей в Ассирию, где он и до сего дня.
\rsbpar\vs 2Ki 17:24 И перевел царь Ассирийский людей из Вавилона, и из Куты, и из Аввы, и из Емафа, и из Сепарваима, и поселил \bibemph{их} в городах Самарийских вместо сынов Израилевых. И они овладели Самариею, и стали жить в городах ее.
\vs 2Ki 17:25 И как в начале жительства своего там они не чтили Господа, то Господь посылал на них львов, которые умерщвляли их.
\vs 2Ki 17:26 И донесли царю Ассирийскому, и сказали: народы, которых ты переселил и поселил в городах Самарийских, не знают закона Бога той земли, и за то Он посылает на них львов, и вот они умерщвляют их, потому что они не знают закона Бога той земли.
\vs 2Ki 17:27 И повелел царь Ассирийский, и сказал: отправьте туда одного из священников, которых вы выселили оттуда; пусть пойдет и живет там, и он научит их закону Бога той земли.
\vs 2Ki 17:28 И пришел один из священников, которых выселили из Самарии, и жил в Вефиле, и учил их, как чтить Господа.
\vs 2Ki 17:29 Притом сделал каждый народ и своих богов и поставил в капищах высот, какие устроили Самаряне,~--- каждый народ в своих городах, где живут они.
\vs 2Ki 17:30 Вавилоняне сделали Суккот-Беноф, Кутийцы сделали Нергала, Емафяне сделали Ашиму,
\vs 2Ki 17:31 Аввийцы сделали Нивхаза и Тартака, а Сепарваимцы сожигали сыновей своих в огне Адрамелеху и Анамелеху, богам Сепарваимским.
\vs 2Ki 17:32 Между тем чтили и Господа, и сделали у себя священников высот из среды своей, и они служили у них в капищах высот.
\vs 2Ki 17:33 Господа они чтили, и богам своим они служили по обычаю народов, из которых выселили их.
\vs 2Ki 17:34 До сего дня поступают они по прежним своим обычаям: не боятся Господа и не поступают по уставам и по обрядам, и по закону и по заповедям, которые заповедал Господь сынам Иакова, которому дал Он имя Израиля.
\vs 2Ki 17:35 Заключил Господь с ними завет и заповедал им, говоря: не чтите богов иных, и не поклоняйтесь им, и не служите им, и не приносите жертв им,
\vs 2Ki 17:36 но Господа, Который вывел вас из земли Египетской силою великою и мышцею простертою,~--- Его чтите и Ему поклоняйтесь, и Ему приносите жертвы,
\vs 2Ki 17:37 и уставы, и учреждения, и закон, и заповеди, которые Он написал вам, старайтесь исполнять во все дни, и не чтите богов иных;
\vs 2Ki 17:38 и завета, который Я заключил с вами, не забывайте, и не чтите богов иных,
\vs 2Ki 17:39 только Господа Бога вашего чтите, и Он избавит вас от руки всех врагов ваших.
\vs 2Ki 17:40 Но они не послушали, а поступали по прежним своим обычаям.
\vs 2Ki 17:41 Народы сии чтили Господа, но и истуканам своим служили. Да и дети их и дети детей их до сего дня поступают так же, как поступали отцы их.
\vs 2Ki 18:1 В третий год Осии, сына Илы, царя Израильского, воцарился Езекия, сын Ахаза, царя Иудейского.
\vs 2Ki 18:2 Двадцати пяти лет был он, когда воцарился, и двадцать девять лет царствовал в Иерусалиме; имя матери его Ави, дочь Захарии.
\vs 2Ki 18:3 И делал он угодное в очах Господних во всем так, как делал Давид, отец его;
\vs 2Ki 18:4 он отменил высоты, разбил статуи, срубил дубраву и истребил медного змея, которого сделал Моисей, потому что до самых тех дней сыны Израилевы кадили ему и называли его Нехуштан.
\vs 2Ki 18:5 На Господа Бога Израилева уповал он; и такого, как он, не было между всеми царями Иудейскими и после него и прежде него.
\vs 2Ki 18:6 И прилепился он к Господу и не отступал от Него, и соблюдал заповеди Его, какие заповедал Господь Моисею.
\vs 2Ki 18:7 И был Господь с ним: везде, куда он ни ходил, поступал он благоразумно. И отложился он от царя Ассирийского, и не стал служить ему.
\vs 2Ki 18:8 Он поразил Филистимлян до Газы и в пределах ее, от сторожевой башни до укрепленного города.
\rsbpar\vs 2Ki 18:9 В четвертый год царя Езекии, то есть в седьмой год Осии, сына Илы, царя Израильского, пошел Салманассар, царь Ассирийский, на Самарию, и осадил ее,
\vs 2Ki 18:10 и взял ее через три года; в шестой год Езекии, то есть в девятый год Осии, царя Израильского, взята Самария.
\vs 2Ki 18:11 И переселил царь Ассирийский Израильтян в Ассирию, и поселил их в Халахе и в Хаворе, при реке Гозан, и в городах Мидийских,
\vs 2Ki 18:12 за то, что они не слушали гласа Господа Бога своего и преступили завет Его, всё, что заповедал Моисей раб Господень, они и не слушали и не исполняли.
\rsbpar\vs 2Ki 18:13 В четырнадцатый год царя Езекии, пошел Сеннахирим, царь Ассирийский, против всех укрепленных городов Иуды и взял их.
\vs 2Ki 18:14 И послал Езекия, царь Иудейский, к царю Ассирийскому в Лахис сказать: виновен я; отойди от меня; что наложишь на меня, я внесу. И наложил царь Ассирийский на Езекию, царя Иудейского, триста талантов серебра и тридцать талантов золота.
\vs 2Ki 18:15 И отдал Езекия все серебро, какое нашлось в доме Господнем и в сокровищницах дома царского.
\vs 2Ki 18:16 В то время снял Езекия \bibemph{золото} с дверей дома Господня и с дверных столбов, которые позолотил Езекия, царь Иудейский, и отдал его царю Ассирийскому.
\rsbpar\vs 2Ki 18:17 И послал царь Ассирийский Тартана и Рабсариса и Рабсака из Лахиса к царю Езекии с большим войском в Иерусалим. И пошли, и пришли к Иерусалиму; и пошли, и пришли, и стали у водопровода верхнего пруда, который на дороге поля белильничьего.
\vs 2Ki 18:18 И звали они царя. И вышел к ним Елиаким, сын Хелкиин, начальник дворца, и Севна писец, и Иоах, сын Асафов, дееписатель.
\vs 2Ki 18:19 И сказал им Рабсак: скажите Езекии: так говорит царь великий, царь Ассирийский: что это за упование, на которое ты уповаешь?
\vs 2Ki 18:20 Ты говорил только пустые слова: для войны нужны совет и сила. Ныне же на кого ты уповаешь, что отложился от меня?
\vs 2Ki 18:21 Вот, ты думаешь опереться на Египет, на эту трость надломленную, которая, если кто опрется на нее, войдет ему в руку и проколет ее. Таков фараон, царь Египетский, для всех уповающих на него.
\vs 2Ki 18:22 А если вы скажете мне: <<на Господа Бога нашего мы уповаем>>, то на Того ли, Которого высоты и жертвенники отменил Езекия, и сказал Иуде и Иерусалиму: <<пред сим только жертвенником поклоняйтесь в Иерусалиме>>?
\vs 2Ki 18:23 Итак вступи в союз с господином моим царем Ассирийским: я дам тебе две тысячи коней, можешь ли достать себе всадников на них?
\vs 2Ki 18:24 Как тебе одолеть и одного вождя из малейших слуг господина моего? И уповаешь на Египет ради колесниц и коней?
\vs 2Ki 18:25 Притом же разве я без воли Господней пошел на место сие, чтобы разорить его? Господь сказал мне: <<пойди на землю сию и разори ее>>.
\vs 2Ki 18:26 И сказал Елиаким, сын Хелкиин, и Севна и Иоах Рабсаку: говори рабам твоим по-арамейски, потому что понимаем мы, а не говори с нами по-иудейски вслух народа, который на стене.
\vs 2Ki 18:27 И сказал им Рабсак: разве \bibemph{только} к господину твоему и к тебе послал меня господин мой сказать сии слова? Нет, также и к людям, которые сидят на стене, чтобы есть помет свой и пить мочу свою с вами.
\vs 2Ki 18:28 И встал Рабсак и возгласил громким голосом по-иудейски, и говорил, и сказал: слушайте слово царя великого, царя Ассирийского!
\vs 2Ki 18:29 Так говорит царь: пусть не обольщает вас Езекия, ибо он не может вас спасти от руки моей;
\vs 2Ki 18:30 и пусть не обнадеживает вас Езекия Господом, говоря: <<спасет нас Господь и не будет город сей отдан в руки царя Ассирийского>>.
\vs 2Ki 18:31 Не слушайте Езекии. Ибо так говорит царь Ассирийский: примиритесь со мною и выйдите ко мне, и пусть каждый ест \bibemph{плоды} виноградной лозы своей и смоковницы своей, и пусть каждый пьет воду из своего колодезя,
\vs 2Ki 18:32 пока я не приду и не возьму вас в землю такую же, как и ваша земля, в землю хлеба и вина, в землю плодов и виноградников, в землю масличных дерев и меда, и будете жить, и не умрете. Не слушайте же Езекии, который обольщает вас, говоря: <<Господь спасет нас>>.
\vs 2Ki 18:33 Спасли ли боги народов, каждый свою землю, от руки царя Ассирийского?
\vs 2Ki 18:34 Где боги Емафа и Арпада? Где боги Сепарваима, Ены и Иввы? Спасли ли они Самарию от руки моей?
\vs 2Ki 18:35 Кто из всех богов земель сих спас землю свою от руки моей? Так неужели Господь спасет Иерусалим от руки моей?
\vs 2Ki 18:36 И молчал народ и не отвечали ему ни слова, потому что было приказание царя: <<не отвечайте ему>>.
\vs 2Ki 18:37 И пришел Елиаким, сын Хелкиин, начальник дворца, и Севна писец и Иоах, сын Асафов, дееписатель, к Езекии в разодранных одеждах, и пересказали ему слова Рабсаковы.
\vs 2Ki 19:1 Когда услышал \bibemph{это} царь Езекия, то разодрал одежды свои и покрылся вретищем, и пошел в дом Господень.
\vs 2Ki 19:2 И послал Елиакима, начальника дворца, и Севну писца, и старших священников, покрытых вретищами, к Исаии пророку, сыну Амосову.
\vs 2Ki 19:3 И они сказали ему: так говорит Езекия: день скорби и наказания и посрамления~--- день сей; ибо дошли младенцы до отверстия утробы матерней, а силы нет родить.
\vs 2Ki 19:4 Может быть, услышит Господь Бог твой все слова Рабсака, которого послал царь Ассирийский, господин его, хулить Бога живаго и поносить словами, какие слышал Господь Бог твой. Принеси же молитву об оставшихся, которые находятся еще в живых.
\vs 2Ki 19:5 И пришли слуги царя Езекии к Исаии,
\vs 2Ki 19:6 и сказал им Исаия: так скажите господину вашему: так говорит Господь: не бойся слов, которые ты слышал, которыми поносили Меня слуги царя Ассирийского.
\vs 2Ki 19:7 Вот Я пошлю в него дух, и он услышит весть, и возвратится в землю свою, и Я поражу его мечом в земле его.
\rsbpar\vs 2Ki 19:8 И возвратился Рабсак, и нашел царя Ассирийского воюющим против Ливны, ибо он слышал, что тот отошел от Лахиса.
\vs 2Ki 19:9 И услышал он о Тиргаке, царе Ефиопском; ему сказали: вот, он вышел сразиться с тобою. И снова послал он послов к Езекии сказать:
\vs 2Ki 19:10 так скажите Езекии, царю Иудейскому: пусть не обманывает тебя Бог твой, на Которого ты уповаешь, думая: <<не будет отдан Иерусалим в руки царя Ассирийского>>.
\vs 2Ki 19:11 Ведь ты слышал, что сделали цари Ассирийские со всеми землями, положив на них заклятие,~--- и ты ли уцелеешь?
\vs 2Ki 19:12 Боги народов, которых разорили отцы мои, спасли ли их? \bibemph{Спасли ли} Гозан, и Харан, и Рецеф, и сынов Едена, что в Фалассаре?
\vs 2Ki 19:13 Где царь Емафа, и царь Арпада, и царь города Сепарваима, Ены и Иввы?
\vs 2Ki 19:14 И взял Езекия письмо из руки послов, и прочитал его, и пошел в дом Господень, и развернул его Езекия пред лицем Господним,
\vs 2Ki 19:15 и молился Езекия пред лицем Господним и говорил: Господи Боже Израилев, сидящий на Херувимах! Ты один Бог всех царств земли, Ты сотворил небо и землю.
\vs 2Ki 19:16 Приклони, Господи, ухо Твое и услышь [меня]; открой, Господи, очи Твои и воззри, и услышь слова Сеннахирима, который послал поносить [Тебя,] Бога живаго!
\vs 2Ki 19:17 Правда, о, Господи, цари Ассирийские разорили народы и земли их,
\vs 2Ki 19:18 и побросали богов их в огонь; но это не боги, а изделие рук человеческих, дерево и камень; потому и истребили их.
\vs 2Ki 19:19 И ныне, Господи Боже наш, спаси нас от руки его, и узнают все царства земли, что Ты, Господи, Бог один.
\rsbpar\vs 2Ki 19:20 И послал Исаия, сын Амосов, к Езекии сказать: так говорит Господь Бог Израилев: то, о чем ты молился Мне против Сеннахирима, царя Ассирийского, Я услышал.
\vs 2Ki 19:21 Вот слово, которое изрек Господь о нем: презрит тебя, посмеется над тобою девствующая дочь Сиона; вслед тебя покачает головою дочь Иерусалима.
\vs 2Ki 19:22 Кого ты порицал и поносил? И на кого ты возвысил голос и поднял так высоко глаза свои? На Святаго Израилева!
\vs 2Ki 19:23 Чрез послов твоих ты порицал Господа и сказал: <<со множеством колесниц моих я взошел на высоту гор, на ребра Ливана, и срубил рослые кедры его, отличные кипарисы его, и пришел на самое крайнее пристанище его, в рощу сада его;
\vs 2Ki 19:24 и откапывал я и пил воду чужую, и осушу ступнями ног моих все реки Египетские>>.
\vs 2Ki 19:25 Разве ты не слышал, что Я издавна сделал это, в древние дни предначертал это, а ныне выполнил тем, что ты опустошаешь укрепленные города, \bibemph{превращая} в груды развалин?
\vs 2Ki 19:26 И жители их сделались маломощны, трепещут и остаются в стыде. Они стали \bibemph{как} трава на поле и нежная зелень, \bibemph{как} порост на кровлях и опаленный хлеб, прежде нежели выколосился.
\vs 2Ki 19:27 Сядешь ли ты, выйдешь ли, войдешь ли, Я все знаю; \bibemph{знаю} и дерзость твою против Меня.
\vs 2Ki 19:28 За твою дерзость против Меня и \bibemph{за то, что} надмение твое дошло до ушей Моих, Я вложу кольцо Мое в ноздри твои и удила Мои в рот твой, и возвращу тебя назад тою же дорогою, которою пришел ты.
\vs 2Ki 19:29 И вот тебе, [Езекия,] знамение: ешьте в этот год выросшее от упавшего зерна, и в другой год~--- самородное, а на третий год сейте и жните, и садите виноградные сады и ешьте плоды их.
\vs 2Ki 19:30 И уцелевшее в доме Иудином, оставшееся пустит опять корень внизу и принесет плод вверху,
\vs 2Ki 19:31 ибо из Иерусалима произойдет остаток, и спасенное от горы Сиона. Ревность Господа Саваофа сделает сие.
\vs 2Ki 19:32 Посему так говорит Господь о царе Ассирийском: <<не войдет он в сей город, и не бросит туда стрелы, и не приступит к нему со щитом, и не насыплет против него вала.
\vs 2Ki 19:33 Тою же дорогою, которою пришел, возвратится, и в город сей не войдет, говорит Господь.
\vs 2Ki 19:34 Я буду охранять город сей, чтобы спасти его ради Себя и ради Давида, раба Моего>>.
\rsbpar\vs 2Ki 19:35 И случилось в ту ночь: пошел Ангел Господень и поразил в стане Ассирийском сто восемьдесят пять тысяч. И встали поутру, и вот все тела мертвые.
\vs 2Ki 19:36 И отправился, и пошел, и возвратился Сеннахирим, царь Ассирийский, и жил в Ниневии.
\vs 2Ki 19:37 И когда он поклонялся в доме Нисроха, бога своего, то Адрамелех и Шарецер, сыновья его, убили его мечом, а сами убежали в землю Араратскую. И воцарился Асардан, сын его, вместо него.
\vs 2Ki 20:1 В те дни заболел Езекия смертельно, и пришел к нему Исаия, сын Амосов, пророк, и сказал ему: так говорит Господь: сделай завещание для дома твоего, ибо умрешь ты и не выздоровеешь.
\vs 2Ki 20:2 И отворотился [Езекия] лицем своим к стене и молился Господу, говоря:
\vs 2Ki 20:3 <<О, Господи! вспомни, что я ходил пред лицем Твоим верно и с преданным \bibemph{Тебе} сердцем, и делал угодное в очах Твоих>>. И заплакал Езекия сильно.
\vs 2Ki 20:4 Исаия еще не вышел из города, как было к нему слово Господне:
\vs 2Ki 20:5 возвратись и скажи Езекии, владыке народа Моего: так говорит Господь Бог Давида, отца твоего: Я услышал молитву твою, увидел слезы твои. Вот, Я исцелю тебя; в третий день пойдешь в дом Господень;
\vs 2Ki 20:6 и прибавлю ко дням твоим пятнадцать лет, и от руки царя Ассирийского спасу тебя и город сей, и защищу город сей ради Себя и ради Давида, раба Моего.
\vs 2Ki 20:7 И сказал Исаия: возьмите пласт смокв. И взяли, и приложили к нарыву; и он выздоровел.
\vs 2Ki 20:8 И сказал Езекия Исаии: какое знамение, что Господь исцелит меня, и что пойду я на третий день в дом Господень?
\vs 2Ki 20:9 И сказал Исаия: вот тебе знамение от Господа, что исполнит Господь слово, которое Он изрек: вперед ли пройти тени на десять ступеней, или воротиться на десять ступеней?
\vs 2Ki 20:10 И сказал Езекия: легко тени подвинуться вперед на десять ступеней; нет, пусть воротится тень назад на десять ступеней.
\vs 2Ki 20:11 И воззвал Исаия пророк к Господу, и возвратил тень назад на ступенях, где она спускалась по ступеням Ахазовым, на десять ступеней.
\rsbpar\vs 2Ki 20:12 В то время послал Беродах Баладан, сын Баладана, царь Вавилонский, письма и подарок Езекии, ибо он слышал, что Езекия был болен.
\vs 2Ki 20:13 Езекия, выслушав посланных, показал им кладовые свои, серебро и золото, и ароматы, и масти дорогие, и весь оружейный дом свой и все, что находилось в сокровищницах его; не оставалось ни одной вещи, которой не показал бы им Езекия в доме своем и во всем владении своем.
\rsbpar\vs 2Ki 20:14 И пришел Исаия пророк к царю Езекии и сказал ему: что говорили эти люди, и откуда они приходили к тебе? И сказал Езекия: из земли далекой они приходили, из Вавилона.
\vs 2Ki 20:15 И сказал \bibemph{Исаия}: что они видели в доме твоем? И сказал Езекия: все, что в доме моем, они видели, не осталось ни одной вещи, которой я не показал бы им в сокровищницах моих.
\vs 2Ki 20:16 И сказал Исаия Езекии: выслушай слово Господне:
\vs 2Ki 20:17 вот придут дни, и взято будет все, что в доме твоем, и что собрали отцы твои до сего дня, в Вавилон; ничего не останется, говорит Господь.
\vs 2Ki 20:18 Из сынов твоих, которые произойдут от тебя, которых ты родишь, возьмут, и будут они евнухами во дворце царя Вавилонского.
\vs 2Ki 20:19 И сказал Езекия Исаии: благо слово Господне, которое ты изрек. И продолжал: да будет мир и благосостояние во дни мои!
\rsbpar\vs 2Ki 20:20 Прочее об Езекии и о всех подвигах его, и о том, что он сделал пруд и водопровод и провел воду в город, написано в летописи царей Иудейских.
\vs 2Ki 20:21 И почил Езекия с отцами своими, и воцарился Манассия, сын его, вместо него.
\vs 2Ki 21:1 Двенадцати лет был Манассия, когда воцарился, и пятьдесят лет царствовал в Иерусалиме; имя матери его Хефциба.
\vs 2Ki 21:2 И делал он неугодное в очах Господних, \bibemph{подражая} мерзостям народов, которых прогнал Господь от лица сынов Израилевых.
\vs 2Ki 21:3 И снова устроил высоты, которые уничтожил отец его Езекия, и поставил жертвенники Ваалу, и сделал дубраву, как сделал Ахав, царь Израильский; и поклонялся всему воинству небесному, и служил ему.
\vs 2Ki 21:4 И соорудил жертвенники в доме Господнем, о котором сказал Господь: <<в Иерусалиме положу имя Мое>>.
\vs 2Ki 21:5 И соорудил жертвенники всему воинству небесному на обоих дворах дома Господня,
\vs 2Ki 21:6 и провел сына своего чрез огонь, и гадал, и ворожил, и завел вызывателей мертвецов и волшебников; много сделал неугодного в очах Господа, чтобы прогневать Его.
\vs 2Ki 21:7 И поставил истукан Астарты, который сделал в доме, о котором говорил Господь Давиду и Соломону, сыну его: <<в доме сем и в Иерусалиме, который Я избрал из всех колен Израилевых, Я полагаю имя Мое на век;
\vs 2Ki 21:8 и не дам впредь выступить ноге Израильтянина из земли, которую Я дал отцам их, если только они будут стараться поступать согласно со всем тем, что Я повелел им, и со всем законом, который заповедал им раб Мой Моисей>>.
\vs 2Ki 21:9 Но они не послушались; и совратил их Манассия до того, что они поступали хуже тех народов, которых истребил Господь от лица сынов Израилевых.
\rsbpar\vs 2Ki 21:10 И говорил Господь чрез рабов Своих пророков и сказал:
\vs 2Ki 21:11 за то, что сделал Манассия, царь Иудейский, такие мерзости, хуже всего того, что делали Аморреи, которые были прежде его, и ввел Иуду в грех идолами своими,
\vs 2Ki 21:12 за то, так говорит Господь, Бог Израилев, вот, Я наведу такое зло на Иерусалим и на Иуду, о котором кто услышит, зазвенит в обоих ушах у того;
\vs 2Ki 21:13 и протяну на Иерусалим мерную вервь Самарии и отвес дома Ахавова, и вытру Иерусалим так, как вытирают чашу,~--- вытрут и опрокинут ее;
\vs 2Ki 21:14 и отвергну остаток удела Моего, и отдам их в руку врагов их, и будут на расхищение и разграбление всем неприятелям своим,
\vs 2Ki 21:15 за то, что они делали неугодное в очах Моих и прогневляли Меня с того дня, как вышли отцы их из Египта, и до сего дня.
\vs 2Ki 21:16 Еще же пролил Манассия и весьма много невинной крови, так что наполнил \bibemph{ею} Иерусалим от края до края, сверх своего греха, что он завлек Иуду в грех~--- делать неугодное в очах Господних.
\rsbpar\vs 2Ki 21:17 Прочее о Манассии и обо всем, что он сделал, и о грехах его, в чем он согрешил, написано в летописи царей Иудейских.
\vs 2Ki 21:18 И почил Манассия с отцами своими, и погребен в саду при доме его, в саду Уззы. И воцарился Аммон, сын его, вместо него.
\rsbpar\vs 2Ki 21:19 Двадцати двух лет был Аммон, когда воцарился, и два года царствовал в Иерусалиме; имя матери его Мешуллемеф, дочь Харуца, из Ятбы.
\vs 2Ki 21:20 И делал он неугодное в очах Господних так, как делал Манассия, отец его;
\vs 2Ki 21:21 и ходил тою же точно дорогою, которою ходил отец его, и служил идолам, которым служил отец его, и поклонялся им,
\vs 2Ki 21:22 и оставил Господа Бога отцов своих, не ходил путем Господним.
\vs 2Ki 21:23 И составили заговор слуги Аммоновы против него, и умертвили царя в доме его.
\vs 2Ki 21:24 Но народ земли перебил всех, бывших в заговоре против царя Аммона; и воцарил народ земли Иосию, сына его, вместо него.
\rsbpar\vs 2Ki 21:25 Прочее об Аммоне, что он сделал, написано в летописи царей Иудейских.
\vs 2Ki 21:26 И похоронили его в гробнице его, в саду Уззы. И воцарился Иосия, сын его, вместо него.
\vs 2Ki 22:1 Восьми лет был Иосия, когда воцарился, и тридцать один год царствовал в Иерусалиме; имя матери его Иедида, дочь Адаии, из Боцкафы.
\vs 2Ki 22:2 И делал он угодное в очах Господних, и ходил во всем путем Давида, отца своего, и не уклонялся ни направо, ни налево.
\rsbpar\vs 2Ki 22:3 В восемнадцатый год царя Иосии, послал царь Шафана, сына Ацалии, сына Мешулламова, писца, в дом Господень, сказав:
\vs 2Ki 22:4 пойди к Хелкии первосвященнику, пусть он пересчитает серебро, принесенное в дом Господень, которое собрали от народа стоящие на страже у порога,
\vs 2Ki 22:5 и пусть отдадут его в руки производителям работ, приставленным к дому Господню, а сии пусть раздают его работающим в доме Господнем, на исправление повреждений дома,
\vs 2Ki 22:6 плотникам и каменщикам, и делателям стен, и на покупку дерев и тесаных камней для исправления дома;
\vs 2Ki 22:7 впрочем не требовать у них отчета в серебре, переданном в руки их, потому что они поступают честно.
\rsbpar\vs 2Ki 22:8 И сказал Хелкия первосвященник Шафану писцу: книгу закона я нашел в доме Господнем. И подал Хелкия книгу Шафану, и он читал ее.
\vs 2Ki 22:9 И пришел Шафан писец к царю, и принес царю ответ, и сказал: взяли рабы твои серебро, найденное в доме, и передали его в руки производителям работ, приставленным к дому Господню.
\vs 2Ki 22:10 И донес Шафан писец царю, говоря: книгу дал мне Хелкия священник. И читал ее Шафан пред царем.
\vs 2Ki 22:11 Когда услышал царь слова книги закона, то разодрал одежды свои.
\vs 2Ki 22:12 И повелел царь Хелкии священнику, и Ахикаму, сыну Шафанову, и Ахбору, сыну Михеину, и Шафану писцу, и Асаии, слуге царскому, говоря:
\vs 2Ki 22:13 пойдите, вопросите Господа за меня и за народ и за всю Иудею о словах сей найденной книги, потому что велик гнев Господень, который воспылал на нас за то, что не слушали отцы наши слов книги сей, чтобы поступать согласно с предписанным нам.
\vs 2Ki 22:14 И пошел Хелкия священник, и Ахикам, и Ахбор, и Шафан, и Асаия к Олдаме пророчице, жене Шаллума, сына Тиквы, сына Хархаса, хранителя одежд,~--- жила же она в Иерусалиме, во второй части,~--- и говорили с нею.
\vs 2Ki 22:15 И она сказала им: так говорит Господь, Бог Израилев: скажите человеку, который послал вас ко мне:
\vs 2Ki 22:16 так говорит Господь: наведу зло на место сие и на жителей его,~--- все слова книги, которую читал царь Иудейский.
\vs 2Ki 22:17 За то, что оставили Меня, и кадят другим богам, чтобы раздражать Меня всеми делами рук своих, воспылал гнев Мой на место сие, и не погаснет.
\vs 2Ki 22:18 А царю Иудейскому, пославшему вас вопросить Господа, скажите: так говорит Господь Бог Израилев, о словах, которые ты слышал:
\vs 2Ki 22:19 так как смягчилось сердце твое, и ты смирился пред Господом, услышав то, что Я изрек на место сие и на жителей его, что они будут предметом ужаса и проклятия, и ты разодрал одежды свои, и плакал предо Мною, то и Я услышал тебя, говорит Господь.
\vs 2Ki 22:20 За это, вот, Я приложу тебя к отцам твоим, и ты положен будешь в гробницу твою в мире, и не увидят глаза твои всего того бедствия, которое Я наведу на место сие. И принесли царю ответ.
\vs 2Ki 23:1 И послал царь, и собрали к нему всех старейшин Иуды и Иерусалима.
\vs 2Ki 23:2 И пошел царь в дом Господень, и все Иудеи, и все жители Иерусалима с ним, и священники, и пророки, и весь народ, от малого до большого, и прочел вслух их все слова книги завета, найденной в доме Господнем.
\vs 2Ki 23:3 Потом стал царь на возвышенное место и заключил пред лицем Господним завет~--- последовать Господу и соблюдать заповеди Его и откровения Его и уставы Его от всего сердца и от всей души, чтобы выполнить слова завета сего, написанные в книге сей. И весь народ вступил в завет.
\rsbpar\vs 2Ki 23:4 И повелел царь Хелкии первосвященнику и вторым священникам и стоящим на страже у порога вынести из храма Господня все вещи, сделанные для Ваала и для Астарты и для всего воинства небесного, и сжег их за Иерусалимом в долине Кедрон, и \bibemph{велел} прах их отнести в Вефиль.
\vs 2Ki 23:5 И отставил жрецов, которых поставили цари Иудейские, чтобы совершать курения на высотах в городах Иудейских и окрестностях Иерусалима,~--- и которые кадили Ваалу, солнцу, и луне, и созвездиям, и всему воинству небесному;
\vs 2Ki 23:6 и вынес Астарту из дома Господня за Иерусалим к потоку Кедрону, и сжег ее у потока Кедрона, и истер ее в прах, и бросил прах ее на кладбище общенародное;
\vs 2Ki 23:7 и разрушил домы блудилищные, которые \bibemph{были} при храме Господнем, где женщины ткали одежды для Астарты;
\vs 2Ki 23:8 и вывел всех жрецов из городов Иудейских, и осквернил высоты, на которых совершали курения жрецы, от Гевы до Вирсавии, и разрушил высоты \bibemph{пред} воротами,~--- ту, которая у входа в ворота Иисуса градоначальника, и ту, которая на левой стороне у городских ворот.
\vs 2Ki 23:9 Впрочем жрецы высот не приносили жертв на жертвеннике Господнем в Иерусалиме, опресноки же ели вместе с братьями своими.
\vs 2Ki 23:10 И осквернил он Тофет, что на долине сыновей Еннома, чтобы никто не проводил сына своего и дочери своей чрез огонь Молоху;
\vs 2Ki 23:11 и отменил коней, которых ставили цари Иудейские солнцу пред входом в дом Господень близ комнат Нефан-Мелеха евнуха, что в Фаруриме, колесницы же солнца сжег огнем.
\vs 2Ki 23:12 И жертвенники на кровле горницы Ахазовой, которые сделали цари Иудейские, и жертвенники, которые сделал Манассия на обоих дворах дома Господня, разрушил царь, и низверг оттуда, и бросил прах их в поток Кедрон.
\vs 2Ki 23:13 И высоты, которые пред Иерусалимом, направо от Масличной горы, которые устроил Соломон, царь Израилев, Астарте, мерзости Сидонской, и Хамосу, мерзости Моавитской, и Милхому, мерзости Аммонитской, осквернил царь;
\vs 2Ki 23:14 и изломал статуи, и срубил дубравы, и наполнил место их костями человеческими.
\vs 2Ki 23:15 Также и жертвенник, который в Вефиле, высоту, устроенную Иеровоамом, сыном Наватовым, который ввел Израиля в грех,~--- также и жертвенник тот и высоту он разрушил, и сжег сию высоту, стер в прах, и сжег дубраву.
\vs 2Ki 23:16 И взглянул Иосия и увидел могилы, которые \bibemph{были} там на горе, и послал и взял кости из могил, и сжег на жертвеннике, и осквернил его по слову Господню, которое провозгласил человек Божий, предрекший события сии, [когда Иеровоам, во время праздника, стоял пред жертвенником. Потом, обратившись, увидел могилу человека Божия, предрекшего сии события,]
\vs 2Ki 23:17 и сказал \bibemph{Иосия}: что это за памятник, который я вижу? И сказали ему жители города: \bibemph{это} могила человека Божия, который приходил из Иудеи и провозгласил о том, что ты делаешь над жертвенником Вефильским.
\vs 2Ki 23:18 И сказал он: оставьте его в покое, никто не трогай костей его. И сохранили кости его и кости пророка, который приходил из Самарии.
\vs 2Ki 23:19 Также и все капища высот в городах Самарийских, которые построили цари Израильские, прогневляя \bibemph{Господа}, разрушил Иосия, и сделал с ними то же, что сделал в Вефиле;
\vs 2Ki 23:20 и заколол всех жрецов высот, которые там были, на жертвенниках, и сожег кости человеческие на них,~--- и возвратился в Иерусалим.
\vs 2Ki 23:21 И повелел царь всему народу, сказав: <<совершите пасху Господу Богу вашему, как написано в сей книге завета>>,~---
\vs 2Ki 23:22 потому что не была совершена такая пасха от дней судей, которые судили Израиля, и во все дни царей Израильских и царей Иудейских;
\vs 2Ki 23:23 а в восемнадцатый год царя Иосии была совершена сия пасха Господу в Иерусалиме.
\vs 2Ki 23:24 И вызывателей мертвых, и волшебников, и терафимов, и идолов, и все мерзости, которые появлялись в земле Иудейской и в Иерусалиме, истребил Иосия, чтоб исполнить слова закона, написанные в книге, которую нашел Хелкия священник в доме Господнем.
\vs 2Ki 23:25 Подобного ему не было царя прежде его, который обратился бы к Господу всем сердцем своим, и всею душею своею, и всеми силами своими, по всему закону Моисееву; и после него не восстал подобный ему.
\vs 2Ki 23:26 Однако ж Господь не отложил великой ярости гнева Своего, какою воспылал гнев Его на Иуду за все оскорбления, какими прогневал Его Манассия.
\vs 2Ki 23:27 И сказал Господь: и Иуду отрину от лица Моего, как отринул Я Израиля, и отвергну город сей Иерусалим, который Я избрал, и дом, о котором Я сказал: <<будет имя Мое там>>.
\rsbpar\vs 2Ki 23:28 Прочее об Иосии и обо всем, что он сделал, написано в летописи царей Иудейских.
\rsbpar\vs 2Ki 23:29 Во дни его пошел фараон Нехао, царь Египетский, против царя Ассирийского на реку Евфрат. И вышел царь Иосия навстречу ему, и тот умертвил его в Мегиддоне, когда увидел его.
\vs 2Ki 23:30 И рабы его повезли его мертвого из Мегиддона, и привезли его в Иерусалим, и похоронили его в гробнице его. И взял народ земли Иоахаза, сына Иосиина, и помазали его и воцарили его вместо отца его.
\rsbpar\vs 2Ki 23:31 Двадцати трех лет был Иоахаз, когда воцарился, и три месяца царствовал в Иерусалиме; имя матери его Хамуталь, дочь Иеремии, из Ливны.
\vs 2Ki 23:32 И делал он неугодное в очах Господних во всем так, как делали отцы его.
\vs 2Ki 23:33 И задержал его фараон Нехао в Ривле, в земле Емафской, чтобы он не царствовал в Иерусалиме,~--- и наложил пени на землю сто талантов серебра и [сто] талантов золота.
\vs 2Ki 23:34 И воцарил фараон Нехао Елиакима, сына Иосиина, вместо Иосии, отца его, и переменил имя его на Иоакима; Иоахаза же взял и отвел в Египет, где он и умер.
\vs 2Ki 23:35 И серебро и золото давал Иоаким фараону; он сделал оценку земле, чтобы давать серебро по приказанию фараона; от каждого из народа земли, по оценке своей, он взыскивал серебро и золото для того, чтобы отдавать фараону Нехао.
\rsbpar\vs 2Ki 23:36 Двадцати пяти лет был Иоаким, когда воцарился, и одиннадцать лет царствовал в Иерусалиме; имя матери его Зебудда, дочь Федаии, из Румы.
\vs 2Ki 23:37 И делал он неугодное в очах Господних во всем так, как делали отцы его.
\vs 2Ki 24:1 Во дни его выступил Навуходоносор, царь Вавилонский, и сделался Иоаким подвластным ему на три года, но потом отложился от него.
\vs 2Ki 24:2 И посылал на него Господь полчища Халдеев, и полчища Сириян, и полчища Моавитян, и полчища Аммонитян,~--- посылал их на Иуду, чтобы погубить его по слову Господа, которое Он изрек чрез рабов Своих пророков.
\vs 2Ki 24:3 По повелению Господа было \bibemph{это} с Иудою, чтобы отвергнуть \bibemph{его} от лица Его за грехи Манассии, за всё, что он сделал;
\vs 2Ki 24:4 и за кровь невинную, которую он пролил, наполнив Иерусалим кровью невинною, Господь не захотел простить.
\rsbpar\vs 2Ki 24:5 Прочее об Иоакиме и обо всем, что он сделал, написано в летописи царей Иудейских.
\vs 2Ki 24:6 И почил Иоаким с отцами своими, и воцарился Иехония, сын его, вместо него.
\vs 2Ki 24:7 Царь Египетский не выходил более из земли своей, потому что взял царь Вавилонский все, от потока Египетского до реки Евфрата, что принадлежало царю Египетскому.
\rsbpar\vs 2Ki 24:8 Восемнадцати лет был Иехония, когда воцарился, и три месяца царствовал в Иерусалиме; имя матери его Нехушта, дочь Елнафана, из Иерусалима.
\vs 2Ki 24:9 И делал он неугодное в очах Господних во всем так, как делал отец его.
\rsbpar\vs 2Ki 24:10 В то время подступили рабы Навуходоносора, царя Вавилонского, к Иерусалиму, и подвергся город осаде.
\vs 2Ki 24:11 И пришел Навуходоносор, царь Вавилонский, к городу, когда рабы его осаждали его.
\vs 2Ki 24:12 И вышел Иехония, царь Иудейский, к царю Вавилонскому, он и мать его, и слуги его, и князья его, и евнухи его,~--- и взял его царь Вавилонский в восьмой год своего царствования.
\vs 2Ki 24:13 И вывез он оттуда все сокровища дома Господня и сокровища царского дома; и изломал, как изрек Господь, все золотые сосуды, которые Соломон, царь Израилев, сделал в храме Господнем;
\vs 2Ki 24:14 и выселил весь Иерусалим, и всех князей, и все храброе войско,~--- десять тысяч было переселенных,~--- и всех плотников и кузнецов; никого не осталось, кроме бедного народа земли.
\vs 2Ki 24:15 И переселил он Иехонию в Вавилон; и мать царя, и жен царя, и евнухов его, и сильных земли отвел на поселение из Иерусалима в Вавилон.
\vs 2Ki 24:16 И все войско \bibemph{числом} семь тысяч, и художников и строителей тысячу, всех храбрых, ходящих на войну, отвел царь Вавилонский на поселение в Вавилон.
\vs 2Ki 24:17 И воцарил царь Вавилонский Матфанию, дядю \bibemph{Иехонии}, вместо него, и переменил имя его на Седекию.
\rsbpar\vs 2Ki 24:18 Двадцати одного года был Седекия, когда воцарился, и одиннадцать лет царствовал в Иерусалиме; имя матери его Хамуталь, дочь Иеремии, из Ливны.
\vs 2Ki 24:19 И делал он неугодное в очах Господних во всем так, как делал Иоаким.
\vs 2Ki 24:20 Гнев Господень был над Иерусалимом и над Иудою до того, что Он отверг их от лица Своего. И отложился Седекия от царя Вавилонского.
\vs 2Ki 25:1 В девятый год царствования своего, в десятый месяц, в десятый день месяца, пришел Навуходоносор, царь Вавилонский, со всем войском своим к Иерусалиму, и осадил его, и устроил вокруг него вал.
\vs 2Ki 25:2 И находился город в осаде до одиннадцатого года царя Седекии.
\vs 2Ki 25:3 В девятый день месяца усилился голод в городе, и не было хлеба у народа земли.
\vs 2Ki 25:4 И взят был город, и \bibemph{побежали} все военные ночью по дороге к воротам, между двумя стенами, что подле царского сада; Халдеи же стояли вокруг города, и \bibemph{царь} ушел дорогою к равнине.
\vs 2Ki 25:5 И погналось войско Халдейское за царем, и настигли его на равнинах Иерихонских, и все войско его разбежалось от него.
\vs 2Ki 25:6 И взяли царя, и отвели его к царю Вавилонскому в Ривлу, и произвели над ним суд:
\vs 2Ki 25:7 и сыновей Седекии закололи пред глазами его, а \bibemph{самому} Седекии ослепили глаза и сковали его оковами, и отвели его в Вавилон.
\rsbpar\vs 2Ki 25:8 В пятый месяц, в седьмой день месяца, то есть в девятнадцатый год Навуходоносора, царя Вавилонского, пришел Навузардан, начальник телохранителей, слуга царя Вавилонского, в Иерусалим
\vs 2Ki 25:9 и сжег дом Господень и дом царя, и все домы в Иерусалиме, и все домы большие сожег огнем;
\vs 2Ki 25:10 и стены вокруг Иерусалима разрушило войско Халдейское, бывшее у начальника телохранителей.
\vs 2Ki 25:11 И прочий народ, остававшийся в городе, и переметчиков, которые передались царю Вавилонскому, и прочий простой народ выселил Навузардан, начальник телохранителей.
\vs 2Ki 25:12 Только несколько из бедного народа земли оставил начальник телохранителей работниками в виноградниках и землепашцами.
\vs 2Ki 25:13 И столбы медные, которые были у дома Господня, и подставы, и море медное, которое в доме Господнем, изломали Халдеи, и отнесли медь их в Вавилон;
\vs 2Ki 25:14 и тазы, и лопатки, и ножи, и ложки, и все сосуды медные, которые употреблялись при служении, взяли;
\vs 2Ki 25:15 и кадильницы, и чаши, что было золотое и что было серебряное, взял начальник телохранителей:
\vs 2Ki 25:16 столбы \bibemph{числом} два, море одно, и подставы, которые сделал Соломон в дом Господень,~--- меди во всех сих вещах не было весу.
\vs 2Ki 25:17 Восемнадцать локтей вышины в одном столбе; венец на нем медный, а вышина венца три локтя, и сетка и гранатовые яблоки вокруг венца~--- все из меди. То же и на другом столбе с сеткою.
\vs 2Ki 25:18 И взял начальник телохранителей Сераию первосвященника и Цефанию, священника второго, и трех, стоявших на страже у порога.
\vs 2Ki 25:19 И из города взял одного евнуха, который был начальствующим над людьми военными, и пять человек, предстоявших лицу царя, которые находились в городе, и писца главного в войске, записывавшего в войско народ земли, и шестьдесят человек из народа земли, находившихся в городе.
\vs 2Ki 25:20 И взял их Навузардан, начальник телохранителей, и отвел их к царю Вавилонскому в Ривлу.
\vs 2Ki 25:21 И поразил их царь Вавилонский, и умертвил их в Ривле, в земле Емаф. И выселены Иудеи из земли своей.
\vs 2Ki 25:22 Над народом же, остававшимся в земле Иудейской, который оставил Навуходоносор, царь Вавилонский,~--- над ними поставил начальником Годолию, сына Ахикама, сына Шафанова.
\vs 2Ki 25:23 Когда услышали все военачальники, они и люди их, что царь Вавилонский поставил начальником Годолию, то пришли к Годолии в Массифу, и \bibemph{именно}: Исмаил, сын Нефании, и Иоханан, сын Карея, и Сераия, сын Танхумефа из Нетофафа, и Иезания, сын Маахитянина, они и люди их.
\vs 2Ki 25:24 И поклялся Годолия им и людям их, и сказал им: не бойтесь быть подвластными Халдеям, селитесь на земле и служите царю Вавилонскому, и будет хорошо вам.
\vs 2Ki 25:25 Но в седьмой месяц пришел Исмаил, сын Нефании, сына Елишамы, из племени царского, с десятью человеками, и поразил Годолию, и он умер, и Иудеев и Халдеев, которые были с ним в Массифе.
\vs 2Ki 25:26 И встал весь народ, от малого до большого, и военачальники, и пошли в Египет, потому что боялись Халдеев.
\rsbpar\vs 2Ki 25:27 В тридцать седьмой год переселения Иехонии, царя Иудейского, в двенадцатый месяц, в двадцать седьмой день месяца, Евилмеродах, царь Вавилонский, в год своего воцарения, вывел Иехонию, царя Иудейского, из дома темничного
\vs 2Ki 25:28 и говорил с ним дружелюбно, и поставил престол его выше престола царей, которые были у него в Вавилоне;
\vs 2Ki 25:29 и переменил темничные одежды его, и он всегда имел пищу у него, во все дни жизни его.
\vs 2Ki 25:30 И содержание его, содержание постоянное, выдаваемо было ему от царя, изо дня в день, во все дни жизни его.
\newbookpage
\bibbookdescr{1Ch}{
  inline={\LARGE Первая книга\\\Huge Паралипоменон\fns{У Евреев: <<Летопись>>.}},
  toc={1-я Паралипоменон},
  bookmark={1-я Паралипоменон},
  header={1-я Паралипоменон},
  %headerleft={},
  %headerright={},
  abbr={1~Пар}
}
\vs 1Ch 1:1 Адам, Сиф, Енос,
\vs 1Ch 1:2 Каинан, Малелеил, Иаред,
\vs 1Ch 1:3 Енох, Мафусал, Ламех,
\vs 1Ch 1:4 Ной, Сим, Хам и Иафет.
\rsbpar\vs 1Ch 1:5 Сыновья Иафета: Гомер, Магог, Мадай, Иаван, [Елиса,] Фувал, Мешех и Фирас.
\vs 1Ch 1:6 Сыновья Гомера: Аскеназ, Рифат и Фогарма.
\vs 1Ch 1:7 Сыновья Иавана: Елиса, Фарсис, Киттим и Доданим.
\rsbpar\vs 1Ch 1:8 Сыновья Хама: Хуш, Мицраим, Фут и Ханаан.
\vs 1Ch 1:9 Сыновья Хуша: Сева, Хавила, Савта, Раама и Савтеха. Сыновья Раамы: Шева и Дедан.
\vs 1Ch 1:10 Хуш родил \bibemph{также} Нимрода: сей начал быть сильным на земле.
\vs 1Ch 1:11 Мицраим родил: Лудима, Анамима, Легавима, Нафтухима,
\vs 1Ch 1:12 Патрусима, Каслухима, от которого произошли Филистимляне, и Кафторима.
\vs 1Ch 1:13 Ханаан родил Сидона, первенца своего, Хета,
\vs 1Ch 1:14 Иевусея, Аморрея, Гергесея,
\vs 1Ch 1:15 Евея, Аркея, Синея,
\vs 1Ch 1:16 Арвадея, Цемарея и Хамафея.
\rsbpar\vs 1Ch 1:17 Сыновья Сима: Елам, Ассур, Арфаксад, Луд и Арам. [Сыновья Арама:] Уц, Хул, Гефер и Мешех.
\vs 1Ch 1:18 Арфаксад [родил Каинана, Каинан же] родил Салу, Сала же родил Евера.
\vs 1Ch 1:19 У Евера родились два сына: имя одному Фалек, потому что во дни его разделилась земля; имя брату его Иоктан.
\vs 1Ch 1:20 Иоктан родил Алмодада, Шалефа, Хацармавета, Иераха,
\vs 1Ch 1:21 Гадорама, Узала, Диклу,
\vs 1Ch 1:22 Евала, Авимаила, Шеву,
\vs 1Ch 1:23 Офира, Хавилу и Иовава. Все эти сыновья Иоктана.
\vs 1Ch 1:24 [Сыновья же] Симовы: Арфаксад, [Каинан,] Сала,
\vs 1Ch 1:25 Евер, Фалек, Рагав,
\vs 1Ch 1:26 Серух, Нахор, Фарра,
\vs 1Ch 1:27 Аврам, он же Авраам.
\rsbpar\vs 1Ch 1:28 Сыновья Авраама: Исаак и Измаил.
\vs 1Ch 1:29 Вот родословие их: первенец Измаилов Наваиоф, \bibemph{за ним} Кедар, Адбеел, Мивсам,
\vs 1Ch 1:30 Мишма, Дума, Масса, Хадад, Фема,
\vs 1Ch 1:31 Иетур, Нафиш и Кедма. Это сыновья Измаиловы.
\vs 1Ch 1:32 Сыновья Хеттуры, наложницы Авраамовой: она родила Зимрана, Иокшана, Медана, Мадиана, Ишбака и Шуаха. Сыновья Иокшана: Шева и Дедан. [Сыновья Дедановы: Рагуил, Навдеил, Ассуриим, Астусиим и Асомин.]
\vs 1Ch 1:33 Сыновья Мадиана: Ефа, Ефер, Ханох, Авида и Елдага. Все эти сыновья Хеттуры.
\rsbpar\vs 1Ch 1:34 И родил Авраам Исаака. Сыновья Исаака: Исав и Израиль.
\rsbpar\vs 1Ch 1:35 Сыновья Исава: Елифаз, Рагуил, Иеус, Иеглом и Корей.
\vs 1Ch 1:36 Сыновья Елифаза: Феман, Омар, Цефо, Гафам, Кеназ; [Фимна же, наложница Елифазова, родила ему] Амалика.
\vs 1Ch 1:37 Сыновья Рагуила: Нахаф, Зерах, Шамма и Миза.
\vs 1Ch 1:38 Сыновья Сеира: Лотан, Шовал, Цивеон, Ана, Дишон, Ецер и Дишан.
\vs 1Ch 1:39 Сыновья Лотана: Хори и Гемам; а сестра у Лотана: Фимна.
\vs 1Ch 1:40 Сыновья Шовала: Алеан, Манахаф, Евал, Шефо и Онам. Сыновья Цивеона: Аиа и Ана.
\vs 1Ch 1:41 Дети Аны: Дишон [и Оливема дочь Аны]. Сыновья Дишона: Хемдан, Ешбан, Ифран и Херан.
\vs 1Ch 1:42 Сыновья Ецера: Билган, Зааван и Акан. Сыновья Дишана: Уц и Аран.
\rsbpar\vs 1Ch 1:43 Сии суть цари, царствовавшие в земле Едома, прежде нежели воцарился царь над сынами Израилевыми: Бела, сын Веора, и имя городу его~--- Дингава;
\vs 1Ch 1:44 и умер Бела, и воцарился по нем Иовав, сын Зераха, из Восоры.
\vs 1Ch 1:45 И умер Иовав, и воцарился по нем Хушам, из земли Феманитян.
\vs 1Ch 1:46 И умер Хушам, и воцарился по нем Гадад, сын Бедадов, который поразил Мадианитян на поле Моава; имя городу его: Авив.
\vs 1Ch 1:47 И умер Гадад, и воцарился по нем Самла, из Масреки.
\vs 1Ch 1:48 И умер Самла, и воцарился по нем Саул из Реховофа, \bibemph{что} при реке.
\vs 1Ch 1:49 И умер Саул, и воцарился по нем Баал-Ханан, сын Ахбора.
\vs 1Ch 1:50 И умер Баал-Ханан, и воцарился по нем Гадар; имя городу его Пау; имя жене его Мегетавеель, дочь Матреда, дочь Мезагава.
\rsbpar\vs 1Ch 1:51 И умер Гадар. И были старейшины у Едома: старейшина Фимна, старейшина Алва, старейшина Иетеф,
\vs 1Ch 1:52 старейшина Оливема, старейшина Эла, старейшина Пинон,
\vs 1Ch 1:53 старейшина Кеназ, старейшина Феман, старейшина Мивцар,
\vs 1Ch 1:54 старейшина Магдиил, старейшина Ирам. Вот старейшины Идумейские.
\vs 1Ch 2:1 Вот сыновья Израиля: Рувим, Симеон, Левий, Иуда, Иссахар, Завулон,
\vs 1Ch 2:2 Дан, Иосиф, Вениамин, Неффалим, Гад и Асир.
\rsbpar\vs 1Ch 2:3 Сыновья Иуды: Ир, Онан и Силом,~--- трое родились у него от дочери Шуевой, Хананеянки. И был Ир, первенец Иудин, не благоугоден в очах Господа, и Он умертвил его.
\rsbpar\vs 1Ch 2:4 И Фамарь, невестка его, родила ему Фареса и Зару. Всех сыновей у Иуды было пятеро.
\vs 1Ch 2:5 Сыновья Фареса: Есром и Хамул.
\vs 1Ch 2:6 Сыновья Зары: Зимри, Ефан, Еман, Халкол и Дара; всех их пятеро.
\rsbpar\vs 1Ch 2:7 Сыновья Харми: Ахар, наведший беду на Израиля, нарушив заклятие.
\vs 1Ch 2:8 Сын Ефана: Азария.
\vs 1Ch 2:9 Сыновья Есрома, которые родились у него: Иерахмеил, Арам и Хелувай.
\rsbpar\vs 1Ch 2:10 Арам же родил Аминадава; Аминадав родил Наассона, князя сынов Иудиных;
\vs 1Ch 2:11 Наассон родил Салмона, Салмон родил Вооза;
\vs 1Ch 2:12 Вооз родил Овида, Овид родил Иессея;
\vs 1Ch 2:13 Иессей родил первенца своего Елиава, второго~--- Аминадава, третьего~--- Самму,
\vs 1Ch 2:14 четвертого~--- Нафанаила, пятого~--- Раддая,
\vs 1Ch 2:15 шестого~--- Оцема, седьмого~--- Давида.
\vs 1Ch 2:16 Сестры их: Саруия и Авигея. Сыновья Саруии: Авесса, Иоав и Азаил, трое.
\vs 1Ch 2:17 Авигея родила Амессу; отец же Амессы~--- Иефер, Измаильтянин.
\rsbpar\vs 1Ch 2:18 Халев, сын Есрома, родил от Азувы, жены \bibemph{своей}, и от Иериофы, и вот сыновья его: Иешер, Шовав и Ардон.
\vs 1Ch 2:19 И умерла Азува; и взял себе Халев [жену] Ефрафу, и она родила ему Хура.
\vs 1Ch 2:20 Хур родил Урия, Урий родил Веселиила.
\rsbpar\vs 1Ch 2:21 После Есром вошел к дочери Махира, отца Галаадова, и взял ее, будучи шестидесяти лет, и она родила ему Сегува.
\vs 1Ch 2:22 Сегув родил Иаира, и было у него двадцать три города в земле Галаадской.
\vs 1Ch 2:23 Но Гессуряне и Сирияне взяли у них селения Иаира, Кенаф и зависящие от него города,~--- шестьдесят городов. Все эти города сыновей Махира, отца Галаадова.
\rsbpar\vs 1Ch 2:24 По смерти Есрома в Халев-Ефрафе, жена Есромова, Авия, родила ему Ашхура, отца Фекои.
\rsbpar\vs 1Ch 2:25 Сыновья Иерахмеила, первенца Есромова, были: первенец Рам, \bibemph{за ним} Вуна, Орен, Оцем и Ахия.
\vs 1Ch 2:26 Была у Иерахмеила и другая жена, имя ее Афара; она мать Онама.
\vs 1Ch 2:27 Сыновья Рама, первенца Иерахмеилова, были: Маац, Иамин и Екер.
\vs 1Ch 2:28 Сыновья Онама были: Шаммай и Иада. Сыновья Шаммая: Надав и Авишур.
\vs 1Ch 2:29 Имя жене Авишуровой Авихаиль, и она родила ему Ахбана и Молида.
\vs 1Ch 2:30 Сыновья Надава: Селед и Афаим. И умер Селед бездетным.
\vs 1Ch 2:31 Сын Афаима: Иший. Сын Ишия: Шешан. Сын Шешана: Ахлай.
\vs 1Ch 2:32 Сыновья Иады, брата Шаммаева: Иефер и Ионафан. Иефер умер бездетным.
\vs 1Ch 2:33 Сыновья Ионафана: Пелеф и Заза. Это сыновья Иерахмеила.
\vs 1Ch 2:34 У Шешана не было сыновей, а только дочери. У Шешана \bibemph{был} раб, Египтянин, имя его Иарха;
\vs 1Ch 2:35 Шешан отдал дочь свою Иархе [рабу своему] в жену: и она родила ему Аттая.
\vs 1Ch 2:36 Аттай родил Нафана, Нафан родил Завада;
\vs 1Ch 2:37 Завад родил Ефлала, Ефлал родил Овида;
\vs 1Ch 2:38 Овид родил Иеуя, Иеуй родил Азарию;
\vs 1Ch 2:39 Азария родил Хелеца, Хелец родил Елеасу;
\vs 1Ch 2:40 Елеаса родил Сисмая, Сисмай родил Саллума;
\vs 1Ch 2:41 Саллум родил Иекамию, Иекамия родил Елишаму.
\rsbpar\vs 1Ch 2:42 Сыновья Халева, брата Иерахмеилова: Меша, первенец его,~--- он отец Зифа; и сыновья Мареши, отца Хеврона.
\vs 1Ch 2:43 Сыновья Хеврона: Корей и Таппуах, и Рекем и Шема.
\vs 1Ch 2:44 Шема родил Рахама, отца Иоркеамова, а Рекем родил Шаммая.
\vs 1Ch 2:45 Сын Шаммая Маон, а Маон~--- отец Беф-Цура.
\vs 1Ch 2:46 И Ефа, наложница Халевова, родила Харана, Моцу и Газеза. И Харан родил Газеза.
\vs 1Ch 2:47 Сыновья Иегдая: Регем, Иофам, Гешан, Пелет, Ефа и Шааф.
\vs 1Ch 2:48 Наложница Халевова, Мааха, родила Шевера и Фирхану;
\vs 1Ch 2:49 она же родила Шаафа, отца Мадманны, Шеву, отца Махбены и отца Гивеи. Дочь же Халева~--- Ахса.
\rsbpar\vs 1Ch 2:50 Вот сыновья Халева: сын Хур, первенец Ефрафы; Шовал, отец Кириаф-Иарима;
\vs 1Ch 2:51 Салма, отец Вифлеема; Хареф, отец Бефгадера.
\vs 1Ch 2:52 У Шовала, отца Кириаф-Иарима, были сыновья: Гарое, Хаци, Галменюхот.
\vs 1Ch 2:53 Племена Кириаф-Иарима: Ифрияне, Футияне, Шумафане и Мидраитяне. От сих произошли Цоряне и Ештаоляне.
\vs 1Ch 2:54 Сыновья Салмы: Вифлеемляне и Нетофафяне, венец дома Иоавова и половина Менухотян~--- Цоряне,
\vs 1Ch 2:55 и племена Соферийцев, живших в Иабеце, Тирейцы, Шимейцы, Сухайцы: это Кинеяне, происшедшие от Хамафа, отца Бетрехава.
\vs 1Ch 3:1 Сыновья Давида, родившиеся у него в Хевроне, были: первенец Амнон, от Ахиноамы Изреелитянки; второй~--- Далуия, от Авигеи Кармилитянки;
\vs 1Ch 3:2 третий~--- Авессалом, сын Маахи, дочери Фалмая, царя Гессурского; четвертый~--- Адония, сын Аггифы;
\vs 1Ch 3:3 пятый~--- Сафатия, от Авиталы; шестой~--- Ифреам, от Аглаи, жены его,~---
\vs 1Ch 3:4 шесть родившихся у него в Хевроне; царствовал же он там семь лет и шесть месяцев; а тридцать три года царствовал в Иерусалиме.
\rsbpar\vs 1Ch 3:5 А сии родились у него в Иерусалиме: Шима, Шовав, Нафан и Соломон, четверо от Вирсавии, дочери Аммииловой;
\vs 1Ch 3:6 Ивхар, Елишама, Елифелет,
\vs 1Ch 3:7 Ногаг, Нефег, Иафиа,
\vs 1Ch 3:8 Елишама, Елиада и Елифелет~--- девятеро.
\vs 1Ch 3:9 \bibemph{Вот} все сыновья Давида, кроме сыновей от наложниц. Сестра их Фамарь.
\rsbpar\vs 1Ch 3:10 Сын Соломона Ровоам; его сын Авия, его сын Аса, его сын Иосафат,
\vs 1Ch 3:11 его сын Иорам, его сын Охозия, его сын Иоас,
\vs 1Ch 3:12 его сын Амасия, его сын Азария, его сын Иофам,
\vs 1Ch 3:13 его сын Ахаз, его сын Езекия, его сын Манассия,
\vs 1Ch 3:14 его сын Амон, его сын Иосия.
\vs 1Ch 3:15 Сыновья Иосии: первенец Иоахаз, второй Иоаким, третий Седекия, четвертый Селлум.
\vs 1Ch 3:16 Сыновья Иоакима: Иехония, сын его; Седекия, сын его.
\rsbpar\vs 1Ch 3:17 Сыновья Иехонии: Асир, Салафиил, сын его;
\vs 1Ch 3:18 Малкирам, Федаия, Шенацар, Иезекия, Гошама и Савадия.
\vs 1Ch 3:19 И сыновья Федаии: Зоровавель и Шимей. Сыновья же Зоровавеля: Мешуллам и Ханания, и Шеломиф, сестра их,
\vs 1Ch 3:20 и еще пять: Хашува, Огел, Берехия, Хасадия и Иушав-Хесед.
\vs 1Ch 3:21 И сыновья Ханании: Фелатия и Исаия; его сын Рефаия, его сын Арнан, его сын Овадия, его сын Шехания.
\vs 1Ch 3:22 Сын Шехании: Шемаия; сыновья Шемаии: Хаттуш, Игеал, Бариах, Неария и Шафат, шестеро.
\vs 1Ch 3:23 Сыновья Неарии: Елиоенай, Езекия и Азрикам, трое.
\vs 1Ch 3:24 Сыновья Елиоеная: Годавьягу, Елеашив, Фелаия, Аккув, Иоханан, Делаия и Анани, семеро.
\vs 1Ch 4:1 Сыновья Иуды: Фарес, Есром, Харми, Хур и Шовал.
\vs 1Ch 4:2 Реаия, сын Шовала, родил Иахафа; Иахаф родил Ахума и Лагада: от них племена Цорян.
\vs 1Ch 4:3 И сии сыновья Етама: Изреель, Ишма и Идбаш, и сестра их, по имени Гацлелпони,
\vs 1Ch 4:4 Пенуел, отец Гедора, и Езер, отец Хуша. Вот сыновья Хура, первенца Ефрафы, отца Вифлеема.
\rsbpar\vs 1Ch 4:5 У Ахшура, отца Фекои, были две жены: Хела и Наара.
\vs 1Ch 4:6 И родила ему Наара Ахузама, Хефера, Фимни и Ахашфари; это сыновья Наары.
\vs 1Ch 4:7 Сыновья Хелы: Цереф, Цохар и Ефнан.
\rsbpar\vs 1Ch 4:8 Коц родил: Анува и Цовева и племена Ахархела, сына Гарумова.
\rsbpar\vs 1Ch 4:9 Иавис был знаменитее своих братьев. Мать дала ему имя Иавис, сказав: я родила его с болезнью.
\vs 1Ch 4:10 И воззвал Иавис к Богу Израилеву \bibemph{и} сказал: о, если бы Ты благословил меня Твоим благословением, распространил пределы мои, и рука Твоя была со мною, охраняя \bibemph{меня} от зла, чтобы я не горевал!.. И Бог ниспослал \bibemph{ему}, чего он просил.
\rsbpar\vs 1Ch 4:11 Хелув же, брат Шухи, родил Махира; он есть отец Ештона.
\vs 1Ch 4:12 Ештон родил Беф-Рафу, Пасеаха и Техинну, отца города Нааса [брата Селома Кенезиина и Ахазова]; это жители Рехи.
\rsbpar\vs 1Ch 4:13 Сыновья Кеназа: Гофониил и Сераия. Сын Гофониила: Хафаф.
\vs 1Ch 4:14 Меонофай родил Офру, а Сераия родил Иоава, родоначальника долины плотников, потому что они были плотники.
\rsbpar\vs 1Ch 4:15 Сыновья Халева, сына Иефонниина: Ир, Ила и Наам. Сын Илы: Кеназ.
\vs 1Ch 4:16 Сыновья Иегаллелела: Зиф, Зифа, Фирия и Асареел.
\vs 1Ch 4:17 Сыновья Езры: Иефер, Меред, Ефер и Иалон; Иефер же родил Мерома, Шаммая и Ишбаха, отца Ешфемои.
\vs 1Ch 4:18 И жена его Иудия родила Иереда, отца Гедора, и Хевера, отца Сохо, и Иекуфиила, отца Занаоха. Это сыновья Бифьи, дочери фараоновой, которую взял Меред.
\vs 1Ch 4:19 Сыновья жены его Годии, сестры Нахама, отца Кеилы: Гарми и Ешфемоа~--- Маахатянин.
\vs 1Ch 4:20 Сыновья Симеона: Амнон, Ринна, Бенханан и Филон. Сыновья Ишия: Зохеф и Бензохеф.
\rsbpar\vs 1Ch 4:21 Сыновья Силома, сына Иудина: Ир, отец Лехи, и Лаеда, отец Мареши, и семейства выделывавших виссон, из дома Ашбеи,
\vs 1Ch 4:22 и Иоким, и жители Хозевы, и Иоаш и Сараф, которые имели владение в Моаве, и Иашувилехем; но это события древние.
\vs 1Ch 4:23 Они \bibemph{были} горшечники, и жили при садах и в огородах; у царя для работ его жили они там.
\rsbpar\vs 1Ch 4:24 Сыновья Симеона: Немуил, Иамин, Иарив, Зерах и Саул.
\vs 1Ch 4:25 Шаллум сын его; его сын Мивсам; его сын Мишма.
\vs 1Ch 4:26 Сыновья Мишмы: Хаммуил, сын его; его сын Закур; его сын Шимей.
\vs 1Ch 4:27 У Шимея \bibemph{было} шестнадцать сыновей и шесть дочерей; у братьев же его сыновей \bibemph{было} немного, и все племя их не так было многочисленно, как племя сынов Иуды.
\rsbpar\vs 1Ch 4:28 Они жили в Вирсавии, Моладе, Хацаршуале,
\vs 1Ch 4:29 в Билге, в Ецеме, в Фоладе,
\vs 1Ch 4:30 в Вефуиле, в Хорме, в Циклаге,
\vs 1Ch 4:31 в Беф-Маркавофе, в Хацарсусиме, в Беф-Биреи и в Шаариме. Вот города их до царствования Давидова,
\vs 1Ch 4:32 с селами их: Етам, Аин, Риммон, Фокен и Ашан,~--- пять городов.
\vs 1Ch 4:33 И все селения их, которые находились вокруг сих городов до Ваала; вот места жительства их и родословия их.
\vs 1Ch 4:34 Мешовав, Иамлех и Иосия, сын Амассии,
\vs 1Ch 4:35 Иоил и Иегу, сын Иошиви, сына Сераии, сына Асиилова,
\vs 1Ch 4:36 Елиоенай, Иакова, Ишохаия, Асаия, Адиил, Ишимиил и Ванея,
\vs 1Ch 4:37 и Зиза, сын Шифия, сын Аллона, сын Иедаии, сын Шимрия, сын Шемаии.
\rsbpar\vs 1Ch 4:38 Сии поименованные \bibemph{были} князьями племен своих, и дом отцов их разделился на многие отрасли.
\vs 1Ch 4:39 Они доходили до Герары и до восточной стороны долины, чтобы найти пастбища для стад своих;
\vs 1Ch 4:40 и нашли пастбища тучные и хорошие и землю обширную, спокойную и безопасную, потому что до них жило там \bibemph{только} немного Хамитян.
\vs 1Ch 4:41 И пришли сии, по именам записанные, во дни Езекии, царя Иудейского, и перебили кочующих и оседлых, которые там находились, и истребили их навсегда и поселились на месте их, ибо там были пастбища для стад их.
\vs 1Ch 4:42 Из них же, из сынов Симеоновых, пошли к горе Сеир пятьсот человек: Фелатия, Неария, Рефаия и Узиил, сыновья Ишия, \bibemph{были} во главе их;
\vs 1Ch 4:43 и побили уцелевший там остаток Амаликитян, и живут там до сего дня.
\vs 1Ch 5:1 Сыновья Рувима, первенца Израилева,~--- он первенец; но, когда осквернил он постель отца своего, первенство его отдано сыновьям Иосифа, сына Израилева, с тем однако ж, чтобы не писаться им первородными;
\vs 1Ch 5:2 потому что Иуда был сильнейшим из братьев своих, и вождь от него, но первенство \bibemph{перенесено} на Иосифа.
\vs 1Ch 5:3 Сыновья Рувима, первенца Израилева: Ханох, Фаллу, Хецрон и Харми.
\rsbpar\vs 1Ch 5:4 Сыновья Иоиля: Шемая, сын его; его сын Гог, его сын Шимей,
\vs 1Ch 5:5 его сын Миха, его сын Реаия, его сын Ваал,
\vs 1Ch 5:6 его сын Беера, которого отвел в плен Феглафелласар, царь Ассирийский. Он \bibemph{был} князем Рувимлян.
\vs 1Ch 5:7 И братья его, по племенам их, по родословному списку их, были: главный Иеиель, потом Захария,
\vs 1Ch 5:8 и Бела, сын Азаза, сына Шемы, сына Иоиля. Он обитал в Ароере до Нево и Ваал-Меона;
\vs 1Ch 5:9 а к востоку он обитал до входа в пустыню, идущую от реки Евфрата, потому что стада их были многочисленны в земле Галаадской.
\vs 1Ch 5:10 Во дни Саула они вели войну с Агарянами, которые пали от рук их, а они стали жить в шатрах и по всей восточной стороне Галаада.
\rsbpar\vs 1Ch 5:11 Сыновья Гада жили напротив их в земле Васанской до Салхи:
\vs 1Ch 5:12 в Васане Иоиль был главный, Шафан второй, потом Иаанай и Шафат.
\vs 1Ch 5:13 Братьев их с семействами их было семь: Михаил, Мешуллам, Шева, Иорай, Иаакан, Зия и Евер.
\vs 1Ch 5:14 Вот сыновья Авихаила, сына Хурия, сына Иароаха, сына Галаада, сына Михаила, сына Иешишая, сына Иахдо, сына Буза.
\vs 1Ch 5:15 Ахи, сын Авдиила, сына Гуниева, \bibemph{был} главою своего рода.
\vs 1Ch 5:16 Они жили в Галааде, в Васане и в зависящих от него городах и во всех окрестностях Сарона, до исхода их.
\vs 1Ch 5:17 Все они перечислены во дни Иоафама, царя Иудейского, и во дни Иеровоама, царя Израильского.
\rsbpar\vs 1Ch 5:18 У потомков Рувима и Гада и полуплемени Манассиина было людей воинственных, мужей носящих щит и меч, стреляющих из лука и приученных к битве, сорок четыре тысячи семьсот шестьдесят, выходящих на войну.
\vs 1Ch 5:19 И воевали они с Агарянами, Иетуром, Нафишем и Надавом.
\vs 1Ch 5:20 И подана была им помощь против них, и преданы были в руки их Агаряне и все, что у них было, потому что они во время сражения воззвали к Богу, и Он услышал их, за то, что они уповали на Него.
\vs 1Ch 5:21 И взяли они стада их: верблюдов пятьдесят тысяч, из мелкого скота двести пятьдесят тысяч, ослов две тысячи, и сто тысяч душ людей,
\vs 1Ch 5:22 потому что много пало убитых, так как от Бога было сражение сие. И жили они на месте их до переселения.
\rsbpar\vs 1Ch 5:23 Потомки полуколена Манассиина жили в той земле, от Васана до Ваал-Ермона и Сенира и до горы Ермона; и их было много.
\vs 1Ch 5:24 И вот главы поколений их: Ефер, Ишьи, Елиил, Азриил, Иеремия, Годавия и Иагдиил, мужи мощные, мужи именитые, главы родов своих.
\rsbpar\vs 1Ch 5:25 Но когда они согрешили против Бога отцов своих и стали блудно ходить вслед богов народов той земли, которых изгнал Бог от лица их,
\vs 1Ch 5:26 тогда Бог Израилев возбудил дух Фула, царя Ассирийского, и дух Феглафелласара, царя Ассирийского, и он выселил Рувимлян и Гадитян и половину колена Манассиина, и отвел их в Халах, и Хавор, и Ару, и на реку Гозан,~--- \bibemph{где они} до сего дня.
\vs 1Ch 6:1 Сыновья Левия: Гирсон, Кааф и Мерари.
\vs 1Ch 6:2 Сыновья Каафа: Амрам, Ицгар, Хеврон и Узиил.
\vs 1Ch 6:3 Дети Амрама: Аарон, Моисей и Мариам. Сыновья Аарона: Надав, Авиуд, Елеазар и Ифамар.
\vs 1Ch 6:4 Елеазар родил Финееса, Финеес родил Авишуя;
\vs 1Ch 6:5 Авишуй родил Буккия, Буккий родил Озию;
\vs 1Ch 6:6 Озия родил Зерахию, Зерахия родил Мераиофа;
\vs 1Ch 6:7 Мераиоф родил Амарию, Амария родил Ахитува;
\vs 1Ch 6:8 Ахитув родил Садока, Садок родил Ахимааса;
\vs 1Ch 6:9 Ахимаас родил Азарию, Азария родил Иоанана;
\vs 1Ch 6:10 Иоанан родил Азарию,~--- это тот, который был священником в храме, построенном Соломоном в Иерусалиме.
\vs 1Ch 6:11 И родил Азария Амарию, Амария родил Ахитува;
\vs 1Ch 6:12 Ахитув родил Садока, Садок родил Селлума;
\vs 1Ch 6:13 Селлум родил Хелкию, Хелкия родил Азарию;
\vs 1Ch 6:14 Азария родил Сераию, Сераия родил Иоседека.
\vs 1Ch 6:15 Иоседек пошел \bibemph{в плен}, когда Господь переселил Иудеев и Иерусалимлян рукою Навуходоносора.
\rsbpar\vs 1Ch 6:16 Итак сыновья Левия: Гирсон, Кааф и Мерари.
\vs 1Ch 6:17 Вот имена сыновей Гирсоновых: Ливни и Шимей.
\vs 1Ch 6:18 Сыновья Каафа: Амрам, Ицгар, Хеврон и Узиил.
\vs 1Ch 6:19 Сыновья Мерари: Махли и Муши. Вот потомки Левия по родам их.
\vs 1Ch 6:20 У Гирсона: Ливни, сын его; Иахав, сын его; Зимма, сын его;
\vs 1Ch 6:21 Иоах, сын его; Иддо, сын его; Зерах, сын его; Иеафрай, сын его.
\vs 1Ch 6:22 Сыновья Каафа: Аминадав, сын его; Корей, сын его; Асир, сын его;
\vs 1Ch 6:23 Елкана, сын его; Евиасаф, сын его; Асир, сын его;
\vs 1Ch 6:24 Тахаф, сын его; Уриил, сын его; Узия, сын его; Саул, сын его.
\vs 1Ch 6:25 Сыновья Елканы: Амасай и Ахимоф.
\vs 1Ch 6:26 Елкана, сын его; Цофай, сын его; Нахаф, сын его;
\vs 1Ch 6:27 Елиаф, сын его; Иерохам, сын его, Елкана, сын его; [Самуил, сын его].
\vs 1Ch 6:28 Сыновья Самуила: первенец Иоиль, второй Авия.
\vs 1Ch 6:29 Сыновья Мерари: Махли; Ливни, сын его; Шимей, сын его; Уза, сын его;
\vs 1Ch 6:30 Шима, сын его; Хаггия, сын его; Асаия, сын его.
\vs 1Ch 6:31 Вот те, которых Давид поставил начальниками над певцами в доме Господнем, со времени поставления в нем ковчега.
\vs 1Ch 6:32 Они служили певцами пред скиниею собрания, доколе Соломон не построил дома Господня в Иерусалиме. И они становились на службу свою по уставу своему.
\rsbpar\vs 1Ch 6:33 Вот те, которые становились с сыновьями своими: из сыновей Каафовых~--- Еман певец, сын Иоиля, сын Самуила,
\vs 1Ch 6:34 сын Елканы, сын Иерохама, сын Елиила, сын Тоаха,
\vs 1Ch 6:35 сын Цуфа, сын Елканы, сын Махафа, сын Амасая,
\vs 1Ch 6:36 сын Елканы, сын Иоиля, сын Азарии, сын Цефании,
\vs 1Ch 6:37 сын Тахафа, сын Асира, сын Авиасафа, сын Корея,
\vs 1Ch 6:38 сын Ицгара, сын Каафа, сын Левия, сын Израиля;
\vs 1Ch 6:39 и брат его Асаф, стоявший на правой стороне его,~--- Асаф, сын Берехии, сын Шимы,
\vs 1Ch 6:40 сын Михаила, сын Ваасеи, сын Малхии,
\vs 1Ch 6:41 сын Ефния, сын Зераха, сын Адаии,
\vs 1Ch 6:42 сын Ефана, сын Зиммы, сын Шимия,
\vs 1Ch 6:43 сын Иахафа, сын Гирсона, сын Левия.
\rsbpar\vs 1Ch 6:44 А из сыновей Мерари, братьев их,~--- на левой стороне: Ефан, сын Кишия, сын Авдия, сын Маллуха,
\vs 1Ch 6:45 сын Хашавии, сын Амасии, сын Хелкии,
\vs 1Ch 6:46 сын Амция, сын Вания, сын Шемера,
\vs 1Ch 6:47 сын Махлия, сын Мушия, сын Мерари, сын Левия.
\vs 1Ch 6:48 Братья их левиты определены на всякие службы при доме Божием;
\rsbpar\vs 1Ch 6:49 Аарон же и сыновья его сожигали на жертвеннике всесожжения и на жертвеннике кадильном, и совершали всякое священнодействие во Святом Святых и для очищения Израиля во всем, как заповедал раб Божий Моисей.
\vs 1Ch 6:50 Вот сыновья Аарона: Елеазар, сын его; Финеес, сын его; Авиуд, сын его;
\vs 1Ch 6:51 Буккий, сын его; Уззий, сын его; Зерахия, сын его;
\vs 1Ch 6:52 Мераиоф, сын его; Амария, сын его; Ахитув, сын его;
\vs 1Ch 6:53 Садок, сын его; Ахимаас, сын его.
\rsbpar\vs 1Ch 6:54 И вот жилища их по селениям их в пределах их: сыновьям Аарона из племени Каафова, так как жребий выпал им,
\vs 1Ch 6:55 дали Хеврон, в земле Иудиной, и предместья его вокруг его;
\vs 1Ch 6:56 поля же сего города и села его отдали Халеву, сыну Иефонниину.
\vs 1Ch 6:57 Сыновьям Аарона дали также города убежищ: Хеврон и Ливну с их предместьями, Иаттир и Ештемоа и предместья его,
\vs 1Ch 6:58 и Хилен и предместья его, Давир и предместья его,
\vs 1Ch 6:59 и Ашан и предместья его, Вефсамис и предместья его,
\vs 1Ch 6:60 а от колена Вениаминова~--- Геву и предместья ее, и Аллемеф и предместья его, и Анафоф и предместья его: всех городов их в племенах их тринадцать городов.
\vs 1Ch 6:61 Остальным сыновьям Каафа, из семейств этого колена, \bibemph{дано} по жребию десять городов из удела половины колена Манассиина.
\vs 1Ch 6:62 Сыновьям Гирсона по племенам их, от колена Иссахарова, и от колена Асирова, и от колена Неффалимова, и от колена Манассиина в Васане, \bibemph{дано} тринадцать городов.
\vs 1Ch 6:63 Сыновьям Мерари по племенам их, от колена Рувимова, и от колена Гадова, и от колена Завулонова, \bibemph{дано} по жребию двенадцать городов.
\rsbpar\vs 1Ch 6:64 Так дали сыны Израилевы левитам города и предместья их.
\vs 1Ch 6:65 Дали они по жребию от колена сыновей Иудиных, и от колена сыновей Симеоновых, и от колена сыновей Вениаминовых те города, которые они назвали по именам.
\vs 1Ch 6:66 Некоторым же племенам сыновей Каафовых даны были города от колена Ефремова.
\vs 1Ch 6:67 И дали им города убежищ: Сихем и предместья его на горе Ефремовой, и Гезер и предместья его,
\vs 1Ch 6:68 и Иокмеам и предместья его, и Беф-Орон и предместья его,
\vs 1Ch 6:69 и Аиалон и предместья его, и Гаф-Риммон и предместья его;
\vs 1Ch 6:70 от половины колена Манассиина~--- Анер и предместья его, Билеам и предместья его. Это поколению остальных сыновей Каафовых.
\vs 1Ch 6:71 Сыновьям Гирсона от племени полуколена Манассиина \bibemph{дали} Голан в Васане и предместья его, и Аштароф и предместья его.
\vs 1Ch 6:72 От колена Иссахарова~--- Кедес и предместья его, Давраф и предместья его,
\vs 1Ch 6:73 и Рамоф и предместья его, и Анем и предместья его;
\vs 1Ch 6:74 от колена Асирова~--- Машал и предместья его, и Авдон и предместья его,
\vs 1Ch 6:75 и Хукок и предместья его, и Рехов и предместья его;
\vs 1Ch 6:76 от колена Неффалимова~--- Кедес в Галилее и предместья его, и Хаммон и предместья его, и Кириафаим и предместья его.
\vs 1Ch 6:77 А прочим сыновьям Мерариным~--- от колена Завулонова Риммон и предместья его, Фавор и предместья его.
\vs 1Ch 6:78 По ту сторону Иордана, против Иерихона, на восток от Иордана, от колена Рувимова \bibemph{дали} Восор в пустыне и предместья его, и Иаацу и предместья ее,
\vs 1Ch 6:79 и Кедемоф и предместья его, и Мефааф и предместья его;
\vs 1Ch 6:80 от колена Гадова~--- Рамоф в Галааде и предместья его, и Маханаим и предместья его,
\vs 1Ch 6:81 и Есевон и предместья его, и Иазер и предместья его.
\vs 1Ch 7:1 Сыновья Иссахара: Фола, Фуа, Иашув и Шимрон, четверо.
\vs 1Ch 7:2 Сыновья Фолы: Уззий, Рефаия, Иериил, Иахмай, Ивсам и Самуил, главные в поколениях Фолы, люди воинственные в своих поколениях; число их во дни Давида было двадцать две тысячи и шестьсот.
\vs 1Ch 7:3 Сын Уззия: Израхия; а сыновья Израхии: Михаил, Овадиа, Иоиль и Ишшия, пятеро. Все они главные.
\vs 1Ch 7:4 У них, по родам их, по поколениям их, было готово к сражению войска тридцать шесть тысяч; потому что у них было много жен и сыновей.
\vs 1Ch 7:5 Братьев же их, во всех поколениях Иссахаровых, людей воинственных, было восемьдесят семь тысяч, внесенных в родословные записи.
\rsbpar\vs 1Ch 7:6 У Вениамина: Бела, Бехер и Иедиаил, трое.
\vs 1Ch 7:7 Сыновья Белы: Ецбон, Уззий, Уззиил, Иеримоф и Ири, пятеро, главы поколений, люди воинственные. В родословных списках записано их двадцать две тысячи тридцать четыре.
\vs 1Ch 7:8 Сыновья Бехера: Земира, Иоаш, Елиезер, Елиоенай, Омри, Иремоф, Авия, Анафоф и Алемеф: все эти сыновья Бехера.
\vs 1Ch 7:9 В родословных списках записано их по родам их, по главам поколений, людей воинственных~--- двадцать тысяч и двести.
\vs 1Ch 7:10 Сын Иедиаила: Билган. Сыновья Билгана: Иеус, Вениамин, Егуд, Хенаана, Зефан, Фарсис и Ахишахар.
\vs 1Ch 7:11 Все эти сыновья Иедиаила были главами поколений, люди воинственные; семнадцать тысяч и двести было выходящих на войну.
\vs 1Ch 7:12 И Шупим и Хупим, сыновья Ира; Хушим, сын Ахера.
\rsbpar\vs 1Ch 7:13 Сыновья Неффалима: Иахцеил, Гуни, Иецер и Шиллем, дети Валлы.
\rsbpar\vs 1Ch 7:14 Сыновья Манассии: Асриил, которого родила наложница его Арамеянка; она же родила Махира, отца Галаадова.
\vs 1Ch 7:15 Махир взял в жену сестру Хупима и Шупима,~--- имя сестры их Мааха; имя второму Салпаад. У Салпаада были \bibemph{только} дочери.
\vs 1Ch 7:16 Мааха, жена Махирова, родила сына и нарекла ему имя Кереш, а имя брату его Шереш. Сыновья его: Улам и Рекем.
\vs 1Ch 7:17 Сын Улама: Бедан. Вот сыновья Галаада, сына Махира, сына Манассиина.
\vs 1Ch 7:18 Сестра его Молехеф родила Ишгода, Авиезера и Махлу.
\vs 1Ch 7:19 Сыновья Шемиды были: Ахиан, Шехем, Ликхи и Аниам.
\rsbpar\vs 1Ch 7:20 Сыновья Ефрема: Шутелах, и Беред, сын его, и Фахаф, сын его, и Елеада, сын его, и Фахаф, сын его,
\vs 1Ch 7:21 и Завад, сын его, и Шутелах, сын его, и Езер и Елеад. И убили их жители Гефа, уроженцы той земли, за то, что они пошли захватить стада их.
\vs 1Ch 7:22 И плакал о них Ефрем, отец их, много дней, и приходили братья его утешать его.
\vs 1Ch 7:23 Потом он вошел к жене своей, и она зачала и родила сына, и он нарек ему имя: Берия, потому что несчастье постигло дом его.
\vs 1Ch 7:24 И дочь у него \bibemph{была} Шеера. Она построила Беф-Орон нижний и верхний и Уззен-Шееру.
\vs 1Ch 7:25 И Рефай, сын его, и Решеф, и Фелах, сын его, и Фахан, сын его,
\vs 1Ch 7:26 Лаедан, сын его, Аммиуд, сын его, Елишама, сын его,
\vs 1Ch 7:27 Нон, сын его, Иисус, сын его.
\vs 1Ch 7:28 Владения их и места жительства их \bibemph{были}: Вефиль и зависящие от него города; к востоку Нааран, к западу Гезер и зависящие от него города; Сихем и зависящие от него города до Газы и зависящих от нее городов.
\vs 1Ch 7:29 А со стороны сыновей Манассииных: Беф-Сан и зависящие от него города, Фаанах и зависящие от него города, Мегиддо и зависящие от него города, Дор и зависящие от него города. В них жили сыновья Иосифа, сына Израилева.
\rsbpar\vs 1Ch 7:30 Сыновья Асира: Имна, Ишва, Ишви и Берия, и сестра их Серах.
\vs 1Ch 7:31 Сыновья Берии: Хевер и Малхиил. Он отец Бирзаифа.
\vs 1Ch 7:32 Хевер родил Иафлета, Шомера и Хофама, и Шую, сестру их.
\vs 1Ch 7:33 Сыновья Иафлета: Пасах, Бимгал и Ашваф. Вот сыновья Иафлета.
\vs 1Ch 7:34 Сыновья Шемера: Ахи, Рохга, Ихубба и Арам.
\vs 1Ch 7:35 Сыновья Гелема, брата его: Цофах, Имна, Шелеш и Амал.
\vs 1Ch 7:36 Сыновья Цофаха: Суах, Харнефер, Шуал, Бери, Имра,
\vs 1Ch 7:37 Бецер, Год, Шамма, Шилша, Ифран и Беера.
\vs 1Ch 7:38 Сыновья Иефера: Иефунни, Фиспа и Ара.
\vs 1Ch 7:39 Сыновья Уллы: Арах, Ханниил и Риция.
\vs 1Ch 7:40 Все эти сыновья Асира, главы поколений, люди отборные, воинственные, главные начальники. Записано у них в родословных списках в войске, для войны, по счету двадцать шесть тысяч человек.
\vs 1Ch 8:1 Вениамин родил Белу, первенца своего, второго Ашбела, третьего Ахрая,
\vs 1Ch 8:2 четвертого Ноху и пятого Рафу.
\vs 1Ch 8:3 Сыновья Белы были: Аддар, Гера, Авиуд,
\vs 1Ch 8:4 Авишуа, Нааман, Ахоах,
\vs 1Ch 8:5 Гера, Шефуфан и Хурам.
\rsbpar\vs 1Ch 8:6 И вот сыновья Егуда, которые были главами родов, живших в Геве и переселенных в Манахаф:
\vs 1Ch 8:7 Нааман, Ахия и Гера, который переселил их; он родил Уззу и Ахихуда.
\rsbpar\vs 1Ch 8:8 Шегараим родил детей в земле Моавитской после того, как отпустил от \bibemph{себя} Хушиму и Баару, жен своих.
\vs 1Ch 8:9 И родил он от Ходеши, жены своей, Иовава, Цивию, Мешу, Малхама,
\vs 1Ch 8:10 Иеуца, Шахию и Мирму: вот сыновья его, главы поколений.
\vs 1Ch 8:11 От Хушимы родил он Авитува и Елпаала.
\vs 1Ch 8:12 Сыновья Елпаала: Евер, Мишам и Шемер, который построил Оно и Лод и зависящие от него города,~---
\vs 1Ch 8:13 и Берия и Шема. Они были главами поколений жителей Аиалона. Они выгнали жителей Гефа.
\vs 1Ch 8:14 Ахио, Шашак, Иремоф,
\vs 1Ch 8:15 Зевадия, Арад, Едер,
\vs 1Ch 8:16 Михаил, Ишфа и Иоха~--- сыновья Берии.
\vs 1Ch 8:17 Зевадия, Мешуллам, Хизкий, Хевер,
\vs 1Ch 8:18 Ишмерай, Излия и Иовав~--- сыновья Елпаала.
\vs 1Ch 8:19 Иаким, Зихрий, Завдий,
\vs 1Ch 8:20 Елиенай, Цилфай, Елиил,
\vs 1Ch 8:21 Адаия, Бераия и Шимраф~--- сыновья Шимея.
\vs 1Ch 8:22 Ишпан, Евер, Елиил,
\vs 1Ch 8:23 Авдон, Зихрий, Ханан,
\vs 1Ch 8:24 Ханания, Елам, Антофия,
\vs 1Ch 8:25 Ифдия и Фенуил~--- сыновья Шашака.
\vs 1Ch 8:26 Шамшерай, Шехария, Афалия,
\vs 1Ch 8:27 Иаарешия, Елия и Зихрий, сыновья Иерохама.
\vs 1Ch 8:28 Это главы поколений, в родах своих главные. Они жили в Иерусалиме.
\rsbpar\vs 1Ch 8:29 В Гаваоне жили: [Иеил,] отец Гаваонитян,~--- имя жены его Мааха,~---
\vs 1Ch 8:30 и сын его, первенец Авдон, \bibemph{за ним} Цур, Кис, Ваал, Надав, [Нер,]
\vs 1Ch 8:31 Гедор, Ахио, Зехер и Миклоф.
\vs 1Ch 8:32 Миклоф родил Шимея. И они подле братьев своих жили в Иерусалиме, вместе с братьями своими.
\vs 1Ch 8:33 Нер родил Киса; Кис родил Саула; Саул родил Ионафана, Мелхисуя, Авинадава и Ешбаала.
\vs 1Ch 8:34 Сын Ионафана Мериббаал; Мериббаал родил Миху.
\vs 1Ch 8:35 Сыновья Михи: Пифон, Мелег, Фаарея и Ахаз.
\vs 1Ch 8:36 Ахаз родил Иоиадду; Иоиадда родил Алемефа, Азмавефа и Замврия; Замврий родил Моцу;
\vs 1Ch 8:37 Моца родил Бинею. Рефаия, сын его; Елеаса, сын его; Ацел, сын его.
\vs 1Ch 8:38 У Ацела шесть сыновей, и вот имена их: Азрикам, Бохру, Исмаил, Шеария, Овадия и Ханан; все они сыновья Ацела.
\vs 1Ch 8:39 Сыновья Ешека, брата его: Улам, первенец его, второй Иеуш, третий Елифелет.
\vs 1Ch 8:40 Сыновья Улама были люди воинственные, стрелявшие из лука, имевшие много сыновей и внуков: сто пятьдесят. Все они от сынов Вениамина.
\vs 1Ch 9:1 Так были перечислены по родам своим все Израильтяне, и вот они записаны в книге царей Израильских. Иудеи же за беззакония свои переселены в Вавилон.
\rsbpar\vs 1Ch 9:2 Первые жители, которые \bibemph{жили} во владениях своих, по городам Израильским, были Израильтяне, священники, левиты и нефинеи.
\vs 1Ch 9:3 В Иерусалиме жили некоторые из сынов Иудиных и из сынов Вениаминовых, и из сынов Ефремовых и Манассииных:
\vs 1Ch 9:4 Уфай, сын Аммиуда, сын Омри, сын Имрия, сын Вания,~--- из сыновей Фареса, сына Иудина;
\vs 1Ch 9:5 из сыновей Шилона~--- Асаия первенец и сыновья его;
\vs 1Ch 9:6 из сыновей Зары~--- Иеуил и братья их,~--- шестьсот девяносто;
\vs 1Ch 9:7 из сыновей Вениаминовых Саллу, сын Мешуллама, сын Годавии, сын Гассенуи;
\vs 1Ch 9:8 и Ивния, сын Иерохама, и Эла, сын Уззия, сына Михриева, и Мешуллам, сын Шефатии, сына Регуила, сына Ивнии,
\vs 1Ch 9:9 и братья их, по родам их: девятьсот пятьдесят шесть,~--- все сии мужи были главы родов в поколениях своих.
\rsbpar\vs 1Ch 9:10 А из священников: Иедаия, Иоиарив, Иахин,
\vs 1Ch 9:11 и Азария, сын Хелкии, сын Мешуллама, сын Садока, сын Мераиофа, сын Ахитува, начальствующий в доме Божием;
\vs 1Ch 9:12 и Адаия, сын Иерохама, сын Пашхура, сын Малхии; и Маасай, сын Адиела, сын Иахзера, сын Мешуллама, сын Мешиллемифа, сын Иммера;
\vs 1Ch 9:13 и братья их, главы родов своих: тысяча семьсот шестьдесят,~--- люди отличные в деле служения в доме Божием.
\rsbpar\vs 1Ch 9:14 А из левитов: Шемаия, сын Хашува, сын Азрикама, сын Хашавии,~--- из сыновей Мерариных;
\vs 1Ch 9:15 и Вакбакар, Хереш, Галал, и Матфания, сын Михи, сын Зихрия, сын Асафа;
\vs 1Ch 9:16 и Овадия, сын Шемаии, сын Галала, сын Идифуна, и Берехия, сын Асы, сын Елканы, живший в селениях Нетофафских.
\rsbpar\vs 1Ch 9:17 А привратники: Шаллум, Аккуб, Талмон и Ахиман, и братья их; Шаллум \bibemph{был} главным.
\vs 1Ch 9:18 И доныне сии привратники у ворот царских, к востоку, содержат стражу сынов Левииных.
\vs 1Ch 9:19 Шаллум, сын Коре, сын Евиасафа, сын Корея, и братья его из рода его, Кореяне, по делу служения своего, были стражами у порогов скинии, а отцы их охраняли вход в стан Господень.
\vs 1Ch 9:20 Финеес, сын Елеазаров, был прежде начальником над ними, и Господь был с ним.
\vs 1Ch 9:21 Захария, сын Мешелемии, \bibemph{был} привратником у дверей скинии собрания.
\vs 1Ch 9:22 Всех их, выбранных в привратники к порогам, было двести двенадцать. Они внесены в список по селениям своим. Их поставил Давид и Самуил-прозорливец за верность их.
\vs 1Ch 9:23 И они и сыновья их были на страже у ворот дома Господня, при доме скинии.
\vs 1Ch 9:24 На четырех сторонах находились привратники: на восточной, западной, северной и южной.
\vs 1Ch 9:25 Братья же их жили в селениях своих, приходя к ним от времени до времени на семь дней.
\vs 1Ch 9:26 Сии четыре начальника привратников, левиты, были в доверенности; они же были приставлены к жилищам и к сокровищам дома Божия.
\vs 1Ch 9:27 Вокруг дома Божия они и ночь проводили, потому что на них \bibemph{лежало} охранение, и они должны были каждое утро отпирать двери.
\vs 1Ch 9:28 \bibemph{Одни} из них были приставлены к служебным сосудам, так что счетом принимали их и счетом выдавали.
\vs 1Ch 9:29 \bibemph{Другим} из них поручена была прочая утварь и все священные потребности: мука лучшая, и вино, и елей, и ладан, и благовония.
\rsbpar\vs 1Ch 9:30 А из сыновей священнических \bibemph{некоторые} составляли миро из веществ благовонных.
\vs 1Ch 9:31 Маттафии из левитов,~--- он первенец Селлума Кореянина,~--- вверено было приготовляемое на сковородах.
\vs 1Ch 9:32 \bibemph{Некоторым} из братьев их, из сынов Каафовых, поручено было \bibemph{заготовление} хлебов предложения, чтобы представлять \bibemph{их} каждую субботу.
\vs 1Ch 9:33 Певцы же, главные в поколениях левитских, в комнатах храма свободны были от занятий, потому что день и ночь они обязаны были \bibemph{заниматься} искусством \bibemph{своим}.
\vs 1Ch 9:34 Это главы поколений левитских, в родах своих главные. Они жили в Иерусалиме.
\rsbpar\vs 1Ch 9:35 В Гаваоне жили: отец Гаваонитян Иеил,~--- имя жены его Мааха,
\vs 1Ch 9:36 и сын его первенец Авдон, \bibemph{за ним} Цур, Кис, Ваал, Нер, Надав,
\vs 1Ch 9:37 Гедор, Ахио, Захария и Миклоф.
\vs 1Ch 9:38 Миклоф родил Шимеама. И они подле братьев своих жили в Иерусалиме вместе с братьями своими.
\vs 1Ch 9:39 Нер родил Киса, Кис родил Саула, Саул родил Ионафана, Мелхисуя, Авинадава и Ешбаала.
\vs 1Ch 9:40 Сын Ионафана Мериббаал; Мериббаал родил Миху.
\vs 1Ch 9:41 Сыновья Михи: Пифон, Мелех, Фарей [и Ахаз].
\vs 1Ch 9:42 Ахаз родил Иаеру; Иаера родил Алемефа, Азмавефа и Замврия; Замврий родил Моцу;
\vs 1Ch 9:43 Моца родил Бинею: Рефаия, сын его; Елеаса, сын его; Ацел, сын его.
\vs 1Ch 9:44 У Ацела шесть сыновей, и вот имена их: Азрикам, Бохру, Исмаил, Шеария, Овадия и Ханан. Это сыновья Ацела.
\vs 1Ch 10:1 Филистимляне воевали с Израилем, и побежали Израильтяне от Филистимлян, и падали пораженные на горе Гелвуе.
\vs 1Ch 10:2 И погнались Филистимляне за Саулом и сыновьями его, и убили Филистимляне Ионафана и Авинадава и Мелхисуя, сыновей Сауловых.
\vs 1Ch 10:3 Сражение против Саула усилилось, и стрелки устремились на него, так что он изранен был стрелками.
\vs 1Ch 10:4 И сказал Саул оруженосцу своему: обнажи меч твой и заколи меня им, чтобы не пришли эти необрезанные и не надругались надо мною. Но оруженосец не решился, потому что очень испугался. Тогда Саул взял меч и пал на него.
\vs 1Ch 10:5 Оруженосец его, увидев, что Саул умер, и сам пал на меч и умер.
\vs 1Ch 10:6 И умер Саул, и три сына его, и весь дом его вместе с ним умер.
\vs 1Ch 10:7 Когда увидели Израильтяне, которые были в долине, что все бегут и что Саул и сыновья его умерли, то оставили города свои и разбежались; а Филистимляне пришли и поселились в них.
\vs 1Ch 10:8 На другой день пришли Филистимляне обирать убитых, и нашли Саула и сыновей его, павших на горе Гелвуйской,
\vs 1Ch 10:9 и раздели его, и сняли с него голову его и оружие его, и послали по земле Филистимской, чтобы возвестить \bibemph{о сем} пред идолами их и пред народом.
\vs 1Ch 10:10 И положили оружие его в капище богов своих, и голову его воткнули в доме Дагона.
\vs 1Ch 10:11 И услышал весь Иавис Галаадский все, что сделали Филистимляне с Саулом.
\vs 1Ch 10:12 И поднялись все люди сильные, взяли тело Саулово и тела сыновей его, и принесли их в Иавис, и похоронили кости их под дубом в Иависе, и постились семь дней.
\rsbpar\vs 1Ch 10:13 Так умер Саул за свое беззаконие, которое он сделал пред Господом, за то, что не соблюл слова Господня и обратился к волшебнице с вопросом,
\vs 1Ch 10:14 а не взыскал Господа. \bibemph{За то} Он и умертвил его, и передал царство Давиду, сыну Иессееву.
\vs 1Ch 11:1 И собрались все Израильтяне к Давиду в Хеврон и сказали: вот, мы кость твоя и плоть твоя;
\vs 1Ch 11:2 и вчера, и третьего дня, когда еще Саул был царем, ты выводил и вводил Израиля, и Господь Бог твой сказал тебе: <<ты будешь пасти народ Мой, Израиля и ты будешь вождем народа Моего Израиля>>.
\vs 1Ch 11:3 И пришли все старейшины Израилевы к царю в Хеврон, и заключил с ними Давид завет в Хевроне пред лицем Господним; и они помазали Давида в царя над Израилем, по слову Господню, чрез Самуила.
\rsbpar\vs 1Ch 11:4 И пошел Давид и весь Израиль к Иерусалиму, то есть к Иевусу. А там были Иевусеи, жители той земли.
\vs 1Ch 11:5 И сказали жители Иевуса Давиду: не войдешь сюда. Но Давид взял крепость Сион; это город Давидов.
\vs 1Ch 11:6 И сказал Давид: кто прежде всех поразит Иевусеев, тот будет главою и военачальником. И взошел прежде всех Иоав, сын Саруи, и сделался главою.
\vs 1Ch 11:7 Давид жил в той крепости, потому и называли ее городом Давидовым.
\vs 1Ch 11:8 И он обстроил город кругом, \bibemph{начиная} от Милло, всю окружность, а Иоав возобновил остальные \bibemph{части} города.
\vs 1Ch 11:9 И преуспевал Давид, и возвышался более и более, и Господь Саваоф \bibemph{был} с ним.
\rsbpar\vs 1Ch 11:10 Вот главные из сильных у Давида, которые крепко подвизались с ним в царстве его, вместе со всем Израилем, чтобы воцарить его, по слову Господню, над Израилем,
\vs 1Ch 11:11 и вот число храбрых, которые были у Давида: Иесваал, сын Ахамани, главный из тридцати. Он поднял копье свое на триста человек и поразил их в один раз.
\vs 1Ch 11:12 По нем Елеазар, сын Додо Ахохиянина, из трех храбрых:
\vs 1Ch 11:13 он был с Давидом в Фасдамиме, куда Филистимляне собрались на войну. Там часть поля была засеяна ячменем, и народ побежал от Филистимлян;
\vs 1Ch 11:14 но они стали среди поля, сберегли его и поразили Филистимлян. И даровал Господь спасение великое!
\vs 1Ch 11:15 Трое сих главных из тридцати вождей взошли на скалу к Давиду, в пещеру Одоллам, когда стан Филистимлян был расположен в долине Рефаимов.
\vs 1Ch 11:16 Давид тогда был в укрепленном месте, а охранное войско Филистимлян было тогда в Вифлееме.
\vs 1Ch 11:17 И сильно захотелось \bibemph{пить} Давиду, и он сказал: кто напоит меня водою из колодезя Вифлеемского, что у ворот?
\vs 1Ch 11:18 Тогда эти трое пробились сквозь стан Филистимский и почерпнули воды из колодезя Вифлеемского, что у ворот, и взяли, и принесли Давиду. Но Давид не захотел пить ее и вылил ее во славу Господа,
\vs 1Ch 11:19 и сказал: сохрани меня Господь, чтоб я сделал это! Стану ли я пить кровь мужей сих, полагавших души свои! Ибо с опасностью собственной жизни они принесли \bibemph{воду}. И не захотел пить ее. Вот что сделали трое этих храбрых.
\rsbpar\vs 1Ch 11:20 И Авесса, брат Иоава, был главным из трех: он убил копьем своим триста человек, и был в славе у тех троих.
\vs 1Ch 11:21 Из трех он был знатнейшим и был начальником; но с теми тремя не равнялся.
\rsbpar\vs 1Ch 11:22 Ванея, сын Иодая, мужа храброго, великий по делам, из Кавцеила: он поразил двух Ариилов Моавитских; он же сошел и убил льва во рве, в снежное время;
\vs 1Ch 11:23 он же убил Египтянина, человека ростом в пять локтей: в руке Египтянина было копье, как навой у ткачей, а он подошел к нему с палкою и, вырвав копье из руки Египтянина, убил его его же копьем:
\vs 1Ch 11:24 вот что сделал Ванея, сын Иодая. И он был в славе у тех троих храбрых;
\vs 1Ch 11:25 он был знатнее тридцати, но с тремя не равнялся, и Давид поставил его ближайшим исполнителем своих приказаний.
\rsbpar\vs 1Ch 11:26 А главные из воинов: Асаил, брат Иоава; Елханан, сын Додо, из Вифлеема;
\vs 1Ch 11:27 Шамма Гародитянин; Херец Пелонитянин;
\vs 1Ch 11:28 Ира, сын Икеша, Фекоитянин; Евиезер Анафофянин;
\vs 1Ch 11:29 Сивхай Хушатянин; Илай Ахохиянин;
\vs 1Ch 11:30 Магарай Нетофафянин; Хелед, сын Вааны, Нетофафянин;
\vs 1Ch 11:31 Иттай, сын Рибая, из Гивы Вениаминовой; Ванея Пирафонянин;
\vs 1Ch 11:32 Хурай из Нагале-Гааша; Авиел из Аравы;
\vs 1Ch 11:33 Азмавеф Бахарумиянин; Елияхба Шаалбонянин.
\vs 1Ch 11:34 Сыновья Гашема Гизонитянина: Ионафан, сын Шаге, Гараритянин;
\vs 1Ch 11:35 Ахиам, сын Сахара, Гараритянин; Елифал, сын Уры;
\vs 1Ch 11:36 Хефер из Махеры; Ахиа Пелонитянин;
\vs 1Ch 11:37 Хецрой Кармилитянин; Наарай, сын Езбая;
\vs 1Ch 11:38 Иоиль, брат Нафана; Мивхар, сын Гагрия;
\vs 1Ch 11:39 Целек Аммонитянин; Нахарай Берофянин, оруженосец Иоава, сына Саруи;
\vs 1Ch 11:40 Ира Ифриянин; Гареб Ифриянин;
\vs 1Ch 11:41 Урия Хеттеянин; Завад, сын Ахлая;
\vs 1Ch 11:42 Адина, сын Шизы, Рувимлянин, глава Рувимлян, и у него \bibemph{было} тридцать;
\vs 1Ch 11:43 Ханан, сын Маахи; Иосафат Мифниянин;
\vs 1Ch 11:44 Уззия Аштерофянин; Шама и Иеиел, сыновья Хофама Ароерянина;
\vs 1Ch 11:45 Иедиаел, сын Шимрия, и Иоха, брат его, Фициянин;
\vs 1Ch 11:46 Елиел из Махавима, и Иеривай и Иошавия, сыновья Елнаама, и Ифма Моавитянин;
\vs 1Ch 11:47 Елиел, Овед и Иасиел из Мецоваи.
\vs 1Ch 12:1 И сии также пришли к Давиду в Секелаг, когда он еще укрывался от Саула, сына Кисова, и были из храбрых, помогавших в сражении.
\vs 1Ch 12:2 Вооруженные луком, правою и левою рукою \bibemph{бросавшие} каменья и \bibemph{стрелявшие} стрелами из лука,~--- из братьев Саула, от Вениамина:
\vs 1Ch 12:3 главный Ахиезер, за ним Иоас, сыновья Шемаи, из Гивы; Иезиел и Фелет, сыновья Азмавефа; Бераха и Иегу из Анафофа;
\vs 1Ch 12:4 Ишмаия Гаваонитянин, храбрый из тридцати и \bibemph{начальствовавший} над тридцатью; Иеремия, Иахазиил, Иоханан и Иозавад из Гедеры.
\vs 1Ch 12:5 Елузай, Иеримоф, Веалия, Шемария, Сафатия Харифиянин;
\vs 1Ch 12:6 Елкана, Ишшияху, Азариил, Иоезер и Иошавам, Кореяне;
\vs 1Ch 12:7 и Иоела и Зевадия, сыновья Иерохама, из Гедора.
\rsbpar\vs 1Ch 12:8 И из Гадитян перешли к Давиду в укрепление, в пустыню, люди мужественные, воинственные, вооруженные щитом и копьем; лица львиные~--- лица их, и они быстры как серны на горах.
\vs 1Ch 12:9 Главный Езер, второй Овадия, третий Елиав,
\vs 1Ch 12:10 четвертый Мишманна, пятый Иеремия,
\vs 1Ch 12:11 шестой Афай, седьмой Елиел,
\vs 1Ch 12:12 восьмой Иоханан, девятый Елзавад,
\vs 1Ch 12:13 десятый Иеремия, одиннадцатый Махбанай.
\vs 1Ch 12:14 Они из сыновей Гадовых \bibemph{были} главами в войске: меньший над сотнею, и больший над тысячею.
\vs 1Ch 12:15 Они-то перешли Иордан в первый месяц, когда он выступает из берегов своих, и разогнали всех живших в долинах к востоку и западу.
\rsbpar\vs 1Ch 12:16 Пришли также и из сыновей Вениаминовых и Иудиных в укрепление к Давиду.
\vs 1Ch 12:17 Давид вышел навстречу им и сказал им: если с миром пришли вы ко мне, чтобы помогать мне, то да будет у меня с вами одно сердце; а если для того, чтобы коварно предать меня врагам моим, тогда как нет порока на руках моих, то да видит Бог отцов наших и рассудит.
\vs 1Ch 12:18 И объял дух Амасая, главу тридцати, \bibemph{и сказал он}: мир тебе Давид, и с тобою, сын Иессеев; мир тебе, и мир помощникам твоим; ибо помогает тебе Бог твой. Тогда принял их Давид и поставил их во главе войска.
\rsbpar\vs 1Ch 12:19 И из колена Манассиина перешли \bibemph{некоторые} к Давиду, когда он шел с Филистимлянами на войну против Саула, но не помогал им, потому что предводители Филистимские, посоветовавшись, отослали его, говоря: на нашу голову он перейдет к господину своему Саулу.
\vs 1Ch 12:20 Когда он возвращался в Секелаг, тогда перешли к нему из Манассиян: Аднах, Иозавад, Иедиаел, Михаил, Иозавад, Елигу и Цилльфай, тысяченачальники у Манассиян.
\vs 1Ch 12:21 И они помогали Давиду против полчищ, ибо все это были люди храбрые и были начальниками в войске.
\vs 1Ch 12:22 Так с каждым днем приходили к Давиду на помощь до того, что его ополчение стало велико, как ополчение Божие.
\rsbpar\vs 1Ch 12:23 Вот число главных в войске, которые пришли к Давиду в Хеврон, чтобы передать ему царство Саулово, по слову Господню:
\vs 1Ch 12:24 сыновей Иудиных, носящих щит и копье, было шесть тысяч восемьсот готовых к войне;
\vs 1Ch 12:25 из сыновей Симеоновых, людей храбрых, в войске было семь тысяч и сто;
\vs 1Ch 12:26 из сыновей Левииных четыре тысячи шестьсот;
\vs 1Ch 12:27 и Иоддай, князь от \bibemph{племени} Аарона, и с ним три тысячи семьсот;
\vs 1Ch 12:28 и Садок, мужественный юноша, и род его, двадцать два начальника;
\vs 1Ch 12:29 из сыновей Вениаминовых, братьев Сауловых, три тысячи,~--- но еще многие из них держались дома Саулова;
\vs 1Ch 12:30 из сыновей Ефремовых двадцать тысяч восемьсот людей мужественных, людей именитых в родах своих;
\vs 1Ch 12:31 из полуколена Манассиина восемнадцать тысяч, которые вызваны были поименно, чтобы пойти воцарить Давида;
\vs 1Ch 12:32 из сынов Иссахаровых \bibemph{пришли} люди разумные, которые знали, чт\acc{о} когда надлежало делать Израилю,~--- их было двести главных, и все братья их следовали слову их;
\vs 1Ch 12:33 из \bibemph{колена} Завулонова готовых к сражению, вооруженных всякими военными оружиями, пятьдесят тысяч, в строю, единодушных;
\vs 1Ch 12:34 из \bibemph{колена} Неффалимова тысяча вождей и с ними тридцать семь тысяч с щитами и копьями;
\vs 1Ch 12:35 из \bibemph{колена} Данова готовых к войне двадцать восемь тысяч шестьсот;
\vs 1Ch 12:36 от Асира воинов, готовых к сражению, сорок тысяч;
\vs 1Ch 12:37 из-за Иордана, от колена Рувимова, Гадова и полуколена Манассиина, сто двадцать тысяч, со всяким воинским оружием.
\rsbpar\vs 1Ch 12:38 Все эти воины, в строю, от полного сердца пришли в Хеврон воцарить Давида над всем Израилем. Да и все прочие Израильтяне были единодушны, чтобы воцарить Давида.
\vs 1Ch 12:39 И пробыли там у Давида три дня, ели и пили, потому что братья их \bibemph{всё} приготовили для них;
\vs 1Ch 12:40 да и близкие к ним, даже до \bibemph{колена} Иссахарова, Завулонова и Неффалимова, привозили все съестное на ослах, и верблюдах, и мулах, и волах: муку, смоквы, и изюм, и вино, и елей, и крупного и мелкого скота множество, так как радость была для Израиля.
\vs 1Ch 13:1 И советовался Давид с тысяченачальниками, сотниками и со всеми вождями,
\vs 1Ch 13:2 и сказал [Давид] всему собранию Израильтян: если угодно вам, и если на то будет воля Господа Бога нашего, пошлем повсюду к прочим братьям нашим, по всей земле Израильской, и вместе с ними к священникам и левитам, в города и селения их, чтобы они собрались к нам;
\vs 1Ch 13:3 и перенесем к себе ковчег Бога нашего, потому что во дни Саула мы не обращались к нему.
\vs 1Ch 13:4 И сказало все собрание: <<да будет так>>, потому что это дело всему народу казалось справедливым.
\vs 1Ch 13:5 Так собрал Давид всех Израильтян, от Шихора Египетского до входа в Емаф, чтобы перенести ковчег Божий из Кириаф-Иарима.
\vs 1Ch 13:6 И пошел Давид и весь Израиль в Кириаф-Иарим, что в Иудее, чтобы перенести оттуда ковчег Бога, Господа, сидящего на Херувимах, на котором нарицается имя \bibemph{Его}.
\vs 1Ch 13:7 И повезли ковчег Божий на новой колеснице из дома Авинадава; и Оза и Ахия вели колесницу.
\vs 1Ch 13:8 Давид же и все Израильтяне играли пред Богом из всей силы, с пением, на цитрах и псалтирях, и тимпанах, и кимвалах и трубах.
\vs 1Ch 13:9 Когда дошли до гумна Хидона, Оза простер руку свою, чтобы придержать ковчег, ибо волы наклонили его.
\vs 1Ch 13:10 Но Господь разгневался на Озу, и поразил его за то, что он простер руку свою к ковчегу; и он умер тут же пред лицем Божиим.
\vs 1Ch 13:11 И опечалился Давид, что Господь поразил Озу. И назвал то место поражением Озы; так называется оно и до сего дня.
\vs 1Ch 13:12 И устрашился Давид Бога в день тот, и сказал: как я внесу к себе ковчег Божий?
\vs 1Ch 13:13 И не повез Давид ковчега к себе, в город Давидов, а обратил его к дому Аведдара Гефянина.
\vs 1Ch 13:14 И оставался ковчег Божий у Аведдара, в доме его, три месяца, и благословил Господь дом Аведдара и все, что у него.
\vs 1Ch 14:1 И послал Хирам, царь Тирский, к Давиду послов, и кедровые деревья, и каменщиков, и плотников, чтобы построить ему дом.
\vs 1Ch 14:2 Когда узнал Давид, что утвердил его Господь царем над Израилем, что вознесено высоко царство его, ради народа его Израиля,
\vs 1Ch 14:3 тогда взял Давид еще жен в Иерусалиме, и родил Давид еще сыновей и дочерей.
\vs 1Ch 14:4 И вот имена родившихся у него в Иерусалиме: Самус, Совав, Нафан, Соломон,
\vs 1Ch 14:5 Евеар, Елисуа, Елфалет,
\vs 1Ch 14:6 Ногах, Нафек, Иафиа,
\vs 1Ch 14:7 и Елисама, Веелиада и Елифалеф.
\rsbpar\vs 1Ch 14:8 И услышали Филистимляне, что помазан Давид в царя над всем Израилем, и поднялись все Филистимляне искать Давида. И услышал Давид \bibemph{об этом} и пошел против них.
\vs 1Ch 14:9 И Филистимляне пришли и расположились в долине Рефаимов.
\vs 1Ch 14:10 И вопросил Давид Бога, говоря: идти ли мне против Филистимлян, и предашь ли их в руки мои? И сказал ему Господь: иди, и Я предам их в руки твои.
\vs 1Ch 14:11 И пошли они в Ваал-Перацим, и поразил их там Давид; и сказал Давид: сломил Бог врагов моих рукою моею, как прорыв воды. Посему и дали имя месту тому: Ваал-Перацим.
\vs 1Ch 14:12 И оставили там \bibemph{Филистимляне} богов своих, и повелел Давид, и сожжены они огнем.
\vs 1Ch 14:13 И \bibemph{пришли} опять Филистимляне и расположились по долине.
\vs 1Ch 14:14 И еще вопросил Давид Бога, и сказал ему Бог: не ходи \bibemph{прямо} на них, уклонись от них и иди к ним со стороны тутовых дерев;
\vs 1Ch 14:15 и когда услышишь шум как бы шагов на вершинах тутовых дерев, тогда вступи в битву, ибо вышел Бог пред тобою, чтобы поразить стан Филистимлян.
\vs 1Ch 14:16 И сделал Давид, как повелел ему Бог; и поразили стан Филистимский, от Гаваона до Газера.
\vs 1Ch 14:17 И пронеслось имя Давидово по всем землям, и Господь сделал его страшным для всех народов.
\vs 1Ch 15:1 И построил он себе домы в городе Давидовом, и приготовил место для ковчега Божия, и устроил для него скинию.
\vs 1Ch 15:2 Тогда сказал Давид: \bibemph{никто} не должен носить ковчега Божия, кроме левитов, потому что их избрал Господь на то, чтобы носить ковчег Божий и служить Ему во веки.
\vs 1Ch 15:3 И собрал Давид всех Израильтян в Иерусалим, чтобы внести ковчег Господень на место его, которое он для него приготовил.
\vs 1Ch 15:4 И созвал Давид сыновей Аароновых и левитов:
\vs 1Ch 15:5 из сыновей Каафовых, Уриила начальника и братьев его~--- сто двадцать \bibemph{человек};
\vs 1Ch 15:6 из сыновей Мерариных, Асаию начальника и братьев его~--- двести двадцать \bibemph{человек};
\vs 1Ch 15:7 из сыновей Гирсоновых, Иоиля начальника и братьев его~--- сто тридцать \bibemph{человек};
\vs 1Ch 15:8 из сыновей Елисафановых, Шемаию начальника и братьев его~--- двести;
\vs 1Ch 15:9 из сыновей Хевроновых, Елиела начальника и братьев его~--- восемьдесят;
\vs 1Ch 15:10 из сыновей Уззииловых, Аминадава начальника и братьев его~--- сто двенадцать.
\vs 1Ch 15:11 И призвал Давид священников: Садока и Авиафара, и левитов: Уриила, Асаию, Иоиля, Шемаию, Елиела и Аминадава,
\vs 1Ch 15:12 и сказал им: вы, начальники родов левитских, освятитесь сами и братья ваши, и принесите ковчег Господа Бога Израилева на \bibemph{место, которое} я приготовил для него;
\vs 1Ch 15:13 ибо как прежде не вы это \bibemph{делали}, то Господь Бог наш поразил нас за то, что мы не взыскали Его, как должно.
\rsbpar\vs 1Ch 15:14 И освятились священники и левиты для того, чтобы нести ковчег Господа, Бога Израилева.
\vs 1Ch 15:15 И понесли сыновья левитов ковчег Божий, как заповедал Моисей по слову Господа, на плечах, на шестах.
\vs 1Ch 15:16 И приказал Давид начальникам левитов поставить братьев своих певцов с музыкальными орудиями, с псалтирями и цитрами и кимвалами, чтобы они громко возвещали глас радования.
\vs 1Ch 15:17 И поставили левиты Емана, сына Иоилева, и из братьев его, Асафа, сына Верехиина, а из сыновей Мерариных, братьев их, Ефана, сына Кушаии;
\vs 1Ch 15:18 и с ними братьев их второстепенных: Захарию, Бена, Иаазиила, Шемирамофа, Иехиила, Унния, Елиава, Ванею, Маасея, Маттафию, Елифлеуя, Микнея и Овед-Едома и Иеиела, привратников.
\vs 1Ch 15:19 Еман, Асаф и Ефан играли громко на медных кимвалах,
\vs 1Ch 15:20 а Захария, Азиил, Шемирамоф, Иехиил, Унний, Елиав, Маасей и Ванея~--- на псалтирях, тонким голосом.
\vs 1Ch 15:21 Маттафия же, Елифлеуй, Микней, Овед-Едом, Иеиел и Азазия~--- на цитрах, чтобы делать начало.
\vs 1Ch 15:22 А Хенания, начальник левитов, был учитель пения, потому что был искусен в нем.
\vs 1Ch 15:23 Верехия и Елкана были придверниками у ковчега.
\vs 1Ch 15:24 Шевания, Иосафат, Нафанаил, Амасай, Захария, Ванея и Елиезер, священники, трубили трубами пред ковчегом Божиим. Овед-Едом и Иехия \bibemph{были} придверниками у ковчега.
\rsbpar\vs 1Ch 15:25 Так Давид и старейшины Израилевы и тысяченачальники пошли перенести ковчег завета Господня из дома Овед-Едомова с веселием.
\vs 1Ch 15:26 И когда Бог помог левитам, несшим ковчег завета Господня, тогда закололи в жертву семь тельцов и семь овнов.
\vs 1Ch 15:27 Давид был одет в виссонную одежду, \bibemph{а также} и все левиты, несшие ковчег, и певцы, и Хенания начальник музыкантов и певцов. На Давиде же был \bibemph{еще} льняной ефод.
\vs 1Ch 15:28 Так весь Израиль вносил ковчег завета Господня с восклицанием, при звуке рога и труб и кимвалов, играя на псалтирях и цитрах.
\rsbpar\vs 1Ch 15:29 Когда ковчег завета Господня входил в город Давидов, Мелхола, дочь Саулова, смотрела в окно и, увидев царя Давида, скачущего и веселящегося, уничижила его в сердце своем.
\vs 1Ch 16:1 И принесли ковчег Божий, и поставили его среди скинии, которую устроил для него Давид, и вознесли Богу всесожжения и мирные жертвы.
\rsbpar\vs 1Ch 16:2 Когда Давид окончил всесожжения и приношение мирных жертв, то благословил народ именем Господа
\vs 1Ch 16:3 и раздал всем Израильтянам, и мужчинам и женщинам, по одному хлебу и по куску мяса и по кружке вина,
\rsbpar\vs 1Ch 16:4 и поставил на службу пред ковчегом Господним \bibemph{некоторых} из левитов, чтобы они славословили, благодарили и превозносили Господа Бога Израилева:
\vs 1Ch 16:5 Асафа главным, вторым по нем Захарию, Иеиела, Шемирамофа, Иехиила, Маттафию, Елиава, и Ванею, Овед-Едома и Иеиела с псалтирями и цитрами, и Асафа для игры на кимвалах,
\vs 1Ch 16:6 а Ванею и Озиила, священников, \bibemph{чтобы} постоянно \bibemph{трубили} пред ковчегом завета Божия.
\rsbpar\vs 1Ch 16:7 В этот день Давид в первый раз дал псалом для славословия Господу чрез Асафа и братьев его:
\vs 1Ch 16:8 славьте Господа, провозглашайте имя Его; возвещайте в народах дела Его;
\vs 1Ch 16:9 пойте Ему, бряцайте Ему; поведайте о всех чудесах Его;
\vs 1Ch 16:10 хвалитесь именем Его святым; да веселится сердце ищущих Господа;
\vs 1Ch 16:11 взыщите Господа и силы Его, ищите непрестанно лица Его;
\vs 1Ch 16:12 поминайте чудеса, которые Он сотворил, знамения Его и суды уст Его,
\vs 1Ch 16:13 \bibemph{вы}, семя Израилево, рабы Его, сыны Иакова, избранные Его!
\vs 1Ch 16:14 Он Господь Бог наш; суды Его по всей земле.
\vs 1Ch 16:15 Помните вечно завет Его, слово, которое Он заповедал в тысячу родов,
\vs 1Ch 16:16 то, что завещал Аврааму, и в чем клялся Исааку,
\vs 1Ch 16:17 и что поставил Иакову в закон и Израилю в завет вечный,
\vs 1Ch 16:18 говоря: <<тебе дам Я землю Ханаанскую, в наследственный удел вам>>.
\vs 1Ch 16:19 Они были тогда малочисленны и ничтожны, и пришельцы в ней,
\vs 1Ch 16:20 и переходили от народа к народу и из одного царства к другому народу;
\vs 1Ch 16:21 но Он никому не позволил обижать их, и обличал за них царей:
\vs 1Ch 16:22 <<Не прикасайтеся к помазанным Моим, и пророкам Моим не делайте зла>>.
\vs 1Ch 16:23 Пойте Господу, вся земля, благовествуйте изо дня в день спасение Его.
\vs 1Ch 16:24 Возвещайте язычникам славу Его, всем народам чудеса Его,
\vs 1Ch 16:25 ибо велик Господь и достохвален, страшен паче всех богов.
\vs 1Ch 16:26 Ибо все боги народов ничто, а Господь небеса сотворил.
\vs 1Ch 16:27 Слава и величие пред лицем Его, могущество и радость на месте [святом] Его.
\vs 1Ch 16:28 Воздайте Господу, племена народов, воздайте Господу славу и честь,
\vs 1Ch 16:29 воздайте Господу славу имени Его. Возьмите дар, идите пред лице Его, поклонитесь Господу в благолепии святыни Его.
\vs 1Ch 16:30 Трепещи пред Ним, вся земля, ибо Он основал вселенную, она не поколеблется.
\vs 1Ch 16:31 Да веселятся небеса, да торжествует земля, и да скажут в народах: Господь царствует!
\vs 1Ch 16:32 Да плещет море и что наполняет его, да радуется поле и все, что на нем.
\vs 1Ch 16:33 Да ликуют вместе все дерева дубравные пред лицем Господа, ибо Он идет судить землю.
\vs 1Ch 16:34 Славьте Господа, ибо вовек милость Его,
\vs 1Ch 16:35 и скажите: спаси нас, Боже, Спаситель наш! Собери нас и избавь нас от народов, да славим святое имя Твое и да хвалимся славою Твоею!
\vs 1Ch 16:36 Благословен Господь Бог Израилев, от века и до века! И сказал весь народ: аминь! аллилуия!
\rsbpar\vs 1Ch 16:37 Давид оставил там, пред ковчегом завета Господня, Асафа и братьев его, чтоб они служили пред ковчегом постоянно, каждый день,
\vs 1Ch 16:38 и Овед-Едома и братьев его, шестьдесят восемь \bibemph{человек}; Овед-Едома, сына Идифунова, и Хосу~--- привратниками,
\vs 1Ch 16:39 а Садока священника и братьев его священников пред жилищем Господним, что на высоте в Гаваоне,
\vs 1Ch 16:40 для возношения всесожжений Господу на жертвеннике всесожжения постоянно, утром и вечером, и для всего, что написано в законе Господа, который Он заповедал Израилю;
\vs 1Ch 16:41 и с ними Емана и Идифуна и прочих избранных, которые назначены поименно, чтобы славить Господа, ибо навек милость Его.
\vs 1Ch 16:42 При них Еман и Идифун прославляли Бога, играя на трубах, кимвалах и разных музыкальных орудиях; сыновей же Идифуна \bibemph{поставил} при вратах.
\rsbpar\vs 1Ch 16:43 И пошел весь народ, каждый в свой дом; возвратился и Давид, чтобы благословить дом свой.
\vs 1Ch 17:1 Когда Давид жил в доме своем, то сказал Давид Нафану пророку: вот, я живу в доме кедровом, а ковчег завета Господня под шатром.
\vs 1Ch 17:2 И сказал Нафан Давиду: все, что у тебя на сердце, делай, ибо с тобою Бог.
\rsbpar\vs 1Ch 17:3 Но в ту же ночь было слово Божие к Нафану:
\vs 1Ch 17:4 пойди и скажи рабу Моему Давиду: так говорит Господь: не ты построишь Мне дом для обитания,
\vs 1Ch 17:5 ибо Я не жил в доме с того дня, как вывел сынов Израиля, и до сего дня, а \bibemph{ходил} из скинии в скинию и из жилища \bibemph{в жилище}.
\vs 1Ch 17:6 Где ни ходил Я со всем Израилем, сказал ли Я хотя слово которому-либо из судей Израильских, которым Я повелел пасти народ Мой: зачем вы не построите Мне дома кедрового?
\vs 1Ch 17:7 И теперь так скажи рабу Моему Давиду: так говорит Господь Саваоф: Я взял тебя от стада овец, чтобы ты был вождем народа Моего Израиля;
\vs 1Ch 17:8 и был с тобою везде, куда ты ни ходил, и истребил всех врагов твоих пред лицем твоим, и сделал имя твое, как имя великих на земле;
\vs 1Ch 17:9 и Я устроил место для народа Моего Израиля, и укоренил его, и будет он спокойно жить на месте своем, и не будет более тревожим, и нечестивые не станут больше теснить его, как прежде,
\vs 1Ch 17:10 в те дни, когда Я поставил судей над народом Моим Израилем, и Я смирил всех врагов твоих, и возвещаю тебе, что Господь устроит тебе дом.
\vs 1Ch 17:11 Когда исполнятся дни твои, и ты отойдешь к отцам твоим, тогда Я восставлю семя твое после тебя, которое будет из сынов твоих, и утвержу царство его.
\vs 1Ch 17:12 Он построит Мне дом, и утвержу престол его на веки.
\vs 1Ch 17:13 Я буду ему отцом, и он будет Мне сыном,~--- и милости Моей не отниму от него, как Я отнял от того, который был прежде тебя.
\vs 1Ch 17:14 Я поставлю его в доме Моем и в царстве Моем на веки, и престол его будет тверд вечно.
\vs 1Ch 17:15 Все эти слова и все видение точно пересказал Нафан Давиду.
\rsbpar\vs 1Ch 17:16 И пришел царь Давид, и стал пред лицем Господним, и сказал: кто я, Господи Боже, и что такое дом мой, что Ты так возвысил меня?
\vs 1Ch 17:17 Но и этого еще мало показалось в очах Твоих, Боже; Ты возвещаешь о доме раба Твоего вдаль, и взираешь на меня, как на человека великого, Господи Боже!
\vs 1Ch 17:18 Что еще может прибавить пред Тобою Давид для возвеличения раба Твоего? Ты знаешь раба Твоего!
\vs 1Ch 17:19 Господи! для раба Твоего, по сердцу Твоему, Ты делаешь все это великое, чтобы явить всякое величие.
\vs 1Ch 17:20 Господи! Нет подобного Тебе, и нет Бога, кроме Тебя, по всему, что слышали мы своими ушами.
\vs 1Ch 17:21 И кто подобен народу Твоему Израилю, единственному народу на земле, к которому приходил Бог, \bibemph{чтоб} искупить его Себе в народ, сделать Себе имя великим и страшным делом~--- прогнанием народов от лица народа Твоего, который Ты избавил из Египта.
\vs 1Ch 17:22 Ты соделал народ Твой Израиля Своим собственным народом навек, и Ты, Господи, стал Богом его.
\vs 1Ch 17:23 Итак теперь, о, Господи, слово, которое Ты сказал о рабе Твоем и о доме его, утверди навек, и сделай, как Ты сказал.
\vs 1Ch 17:24 И да пребудет и возвеличится имя Твое во веки, чтобы говорили: Господь Саваоф, Бог Израилев, есть Бог над Израилем, и дом раба Твоего Давида да будет тверд пред лицем Твоим.
\vs 1Ch 17:25 Ибо Ты, Боже мой, открыл рабу Твоему, что Ты устроишь ему дом, поэтому раб Твой и дерзнул молиться пред Тобою.
\vs 1Ch 17:26 И ныне, Господи, Ты Бог, и Ты сказал о рабе Твоем такое благо.
\vs 1Ch 17:27 Начни же благословлять дом раба Твоего, чтоб он был вечно пред лицем Твоим. Ибо если Ты, Господи, благословишь, то будет он благословен вовек.
\vs 1Ch 18:1 После сего Давид поразил Филистимлян и смирил их, и взял Геф и зависящие от него города из руки Филистимлян.
\vs 1Ch 18:2 Он поразил также Моавитян,~--- и сделались Моавитяне рабами Давида, принося ему дань.
\rsbpar\vs 1Ch 18:3 И поразил Давид Адраазара, царя Сувского, в Емафе, когда тот шел утвердить власть свою при реке Евфрате.
\vs 1Ch 18:4 И взял Давид у него тысячу колесниц, семь тысяч всадников и двадцать тысяч пеших, и разрушил Давид все колесницы, оставив из них \bibemph{только} сто.
\vs 1Ch 18:5 Сирияне Дамасские пришли было на помощь к Адраазару, царю Сувскому, но Давид поразил двадцать две тысячи Сириян.
\vs 1Ch 18:6 И поставил Давид \bibemph{охранное войско} в Сирии Дамасской, и сделались Сирияне рабами Давида, принося ему дань. И помогал Господь Давиду везде, куда он ни ходил.
\vs 1Ch 18:7 И взял Давид золотые щиты, которые были у рабов Адраазара, и принес их в Иерусалим.
\vs 1Ch 18:8 А из Тивхавы и Куна, городов Адраазаровых, взял Давид весьма много меди. Из нее Соломон сделал медное море и столбы и медные сосуды.
\vs 1Ch 18:9 И услышал Фой, царь Имафа, что Давид поразил все войско Адраазара, царя Сувского.
\vs 1Ch 18:10 И послал Иорама, сына своего, к царю Давиду, приветствовать его и благодарить за то, что он воевал с Адраазаром и поразил его, ибо Фой был в войне с Адраазаром,~--- и \bibemph{с ним} всякие сосуды золотые, серебряные и медные.
\vs 1Ch 18:11 И посвятил их царь Давид Господу вместе с серебром и золотом, которое он взял от всех народов: от Идумеян, Моавитян, Аммонитян, Филистимлян и от Амаликитян.
\vs 1Ch 18:12 И Авесса, сын Саруи, поразил Идумеян на долине Соляной восемнадцать тысяч;
\vs 1Ch 18:13 и поставил в Идумее охранное войско, и сделались все Идумеяне рабами Давиду. Господь помогал Давиду везде, куда он ни ходил.
\rsbpar\vs 1Ch 18:14 И царствовал Давид над всем Израилем, и творил суд и правду всему народу своему.
\vs 1Ch 18:15 Иоав, сын Саруи, \bibemph{был} начальником войска, Иосафат, сын Ахилуда, дееписателем,
\vs 1Ch 18:16 Садок, сын Ахитува, и Авимелех, сын Авиафара, священниками, а Суса писцом,
\vs 1Ch 18:17 Ванея, сын Иодая, над Хелефеями и Фелефеями, а сыновья Давидовы~--- первыми при царе.
\vs 1Ch 19:1 После сего умер Наас, царь Аммонитский, и воцарился сын его вместо него.
\vs 1Ch 19:2 И сказал Давид: окажу я милость Аннону, сыну Наасову, за благодеяние, которое отец его оказал мне. И послал Давид послов утешить его об отце его; и пришли слуги Давидовы в землю Аммонитскую, к Аннону, чтобы утешить его.
\vs 1Ch 19:3 Но князья Аммонитские сказали Аннону: неужели ты думаешь, что Давид из уважения к отцу твоему прислал к тебе утешителей? Не для того ли пришли слуги его к тебе, чтобы разведать и высмотреть землю и разорить ее?
\vs 1Ch 19:4 И взял Аннон слуг Давидовых и обрил их, и обрезал одежды их наполовину до чресл и отпустил их.
\vs 1Ch 19:5 И пошли они. И донесено было Давиду о людях сих, и он послал им навстречу, так как они были очень обесчещены; и сказал царь: останьтесь в Иерихоне, пока отрастут бороды ваши, и тогда возвратитесь.
\rsbpar\vs 1Ch 19:6 Когда Аммонитяне увидели, что они сделались ненавистными Давиду, тогда послал Аннон и Аммонитяне тысячу талантов серебра, чтобы нанять себе колесниц и всадников из Сирии Месопотамской и из Сирии Мааха и из Сувы.
\vs 1Ch 19:7 И наняли себе тридцать две тысячи колесниц и царя Мааха с народом его, которые пришли и расположились станом пред Медевою. И Аммонитяне собрались из городов своих и выступили на войну.
\rsbpar\vs 1Ch 19:8 Когда услышал об этом Давид, то послал Иоава со всем войском храбрых.
\vs 1Ch 19:9 И выступили Аммонитяне и выстроились к сражению у ворот города, а цари, которые пришли, отдельно в поле.
\vs 1Ch 19:10 Иоав, видя, что предстоит ему сражение спереди и сзади, избрал воинов из всех отборных в Израиле и выстроил \bibemph{их} против Сириян.
\vs 1Ch 19:11 А остальную часть народа поручил Авессе, брату своему, чтоб они выстроились против Аммонитян.
\vs 1Ch 19:12 И сказал он: если Сирияне будут одолевать меня, то ты поможешь мне, а если Аммонитяне будут одолевать тебя, то я помогу тебе.
\vs 1Ch 19:13 Будь мужествен, и будем твердо стоять за народ наш и за города Бога нашего,~--- и Господь пусть сделает, что ему угодно.
\vs 1Ch 19:14 И вступил Иоав и люди, которые были у него, в сражение с Сириянами, и они побежали от него.
\vs 1Ch 19:15 Аммонитяне же, увидев, что Сирияне бегут, и сами побежали от Авессы, брата его, и ушли в город. И пришел Иоав в Иерусалим.
\vs 1Ch 19:16 Сирияне, видя, что они поражены Израильтянами, отправили послов и вывели Сириян, которые были по ту сторону реки, и Совак, военачальник Адраазаров, предводительствовал ими.
\vs 1Ch 19:17 Когда донесли об этом Давиду, он собрал всех Израильтян, перешел Иордан и, придя к ним, выстроился против них; и вступил Давид в сражение с Сириянами, и они сразились с ним.
\vs 1Ch 19:18 И Сирияне побежали от Израильтян, и истребил Давид у Сириян семь тысяч колесниц и сорок тысяч пеших, и Совака военачальника умертвил.
\vs 1Ch 19:19 Когда увидели слуги Адраазара, что они поражены Израильтянами, заключили с Давидом мир и подчинились ему. И не хотели Сирияне помогать более Аммонитянам.
\vs 1Ch 20:1 Через год, в то время когда цари выходят \bibemph{на войну}, вывел Иоав войско и стал разорять землю Аммонитян, и пришел и осадил Равву. Давид же оставался в Иерусалиме. Иоав, завоевав Равву, разрушил ее.
\vs 1Ch 20:2 И взял Давид венец царя их с головы его, и в нем оказалось весу талант золота, и драгоценные камни были на нем; и был он возложен на голову Давида. И добычи очень много вынес из города.
\vs 1Ch 20:3 А народ, который был в нем, вывел и умерщвлял их пилами, железными молотилами и секирами. Так поступил Давид со всеми городами Аммонитян, и возвратился Давид и весь народ в Иерусалим.
\rsbpar\vs 1Ch 20:4 После того началась война с Филистимлянами в Газере. Тогда Совохай Хушатянин поразил Сафа, одного из потомков Рефаимов. И они усмирились.
\vs 1Ch 20:5 И опять была война с Филистимлянами. Тогда Елханам, сын Иаира, поразил Лахмия, брата Голиафова, Гефянина, у которого древко копья было, как навой у ткачей.
\vs 1Ch 20:6 Было еще сражение в Гефе. Там был один рослый человек, у которого было по шести пальцев, \bibemph{всего} двадцать четыре. И он также был из потомков Рефаимов.
\vs 1Ch 20:7 Он поносил Израиля, но Ионафан, сын Шимы, брата Давидова, поразил его.
\vs 1Ch 20:8 Это были родившиеся от Рефаимов в Гефе, и пали от руки Давида и от руки слуг его.
\vs 1Ch 21:1 И восстал сатана на Израиля, и возбудил Давида сделать счисление Израильтян.
\vs 1Ch 21:2 И сказал Давид Иоаву и начальствующим в народе: пойдите исчислите Израильтян, от Вирсавии до Дана, и представьте мне, чтоб я знал число их.
\vs 1Ch 21:3 И сказал Иоав: да умножит Господь народ Свой во сто раз против того, сколько есть его. Не все ли они, господин мой царь, рабы господина моего? Для чего же требует сего господин мой? Чтобы вменилось это в вину Израилю?
\vs 1Ch 21:4 Но царское слово превозмогло Иоава; и пошел Иоав, и обошел всего Израиля, и пришел в Иерусалим.
\vs 1Ch 21:5 И подал Иоав Давиду список народной переписи, и было всех Израильтян тысяча тысяч, и сто тысяч мужей, обнажающих меч, и Иудеев~--- четыреста семьдесят тысяч, обнажающих меч.
\vs 1Ch 21:6 А левитов и Вениаминян он не исчислял между ними, потому что царское слово противно было Иоаву.
\rsbpar\vs 1Ch 21:7 И не угодно было в очах Божиих дело сие, и Он поразил Израиля.
\vs 1Ch 21:8 И сказал Давид Богу: весьма согрешил я, что сделал это. И ныне прости вину раба Твоего, ибо я поступил очень безрассудно.
\vs 1Ch 21:9 И говорил Господь Гаду, прозорливцу Давидову, и сказал:
\vs 1Ch 21:10 пойди и скажи Давиду: так говорит Господь: три \bibemph{наказания} Я предлагаю тебе, избери себе одно из них,~--- и Я пошлю его на тебя.
\vs 1Ch 21:11 И пришел Гад к Давиду и сказал ему: так говорит Господь: избирай себе:
\vs 1Ch 21:12 или три года~--- голод, или три месяца будешь ты преследуем неприятелями твоими и меч врагов твоих будет досягать \bibemph{до тебя}; или три дня~--- меч Господень и язва на земле и Ангел Господень, истребляющий во всех пределах Израиля. Итак, рассмотри, что мне отвечать Пославшему меня с словом.
\vs 1Ch 21:13 И сказал Давид Гаду: тяжело мне очень, но пусть лучше впаду в руки Господа, ибо весьма велико милосердие Его, только бы не впасть мне в руки человеческие.
\vs 1Ch 21:14 И послал Господь язву на Израиля, и умерло Израильтян семьдесят тысяч человек.
\vs 1Ch 21:15 И послал Бог Ангела в Иерусалим, чтобы истреблять его. И когда он начал истреблять, увидел Господь и пожалел о сем бедствии, и сказал Ангелу-истребителю: довольно! теперь опусти руку твою. Ангел же Господень стоял \bibemph{тогда} над гумном Орны Иевусеянина.
\vs 1Ch 21:16 И поднял Давид глаза свои, и увидел Ангела Господня, стоящего между землею и небом, с обнаженным в руке его мечом, простертым на Иерусалим; и пал Давид и старейшины, покрытые вретищем, на лица свои.
\vs 1Ch 21:17 И сказал Давид Богу: не я ли велел исчислить народ? я согрешил, я сделал зло, а эти овцы что сделали? Господи, Боже мой! да будет рука Твоя на мне и на доме отца моего, а не на народе Твоем, чтобы погубить \bibemph{его}.
\vs 1Ch 21:18 И Ангел Господень сказал Гаду, чтобы тот сказал Давиду: пусть Давид придет и поставит жертвенник Господу на гумне Орны Иевусеянина.
\rsbpar\vs 1Ch 21:19 И пошел Давид, по слову Гада, которое он говорил именем Господним.
\vs 1Ch 21:20 Орна обратился, увидел Ангела, и четыре сына его с ним скрылись. Орна молотил тогда пшеницу.
\vs 1Ch 21:21 И пришел Давид к Орне. Орна, взглянув и увидев Давида, вышел из гумна и поклонился Давиду лицем до земли.
\vs 1Ch 21:22 И сказал Давид Орне: отдай мне место под гумном, я построю на нем жертвенник Господу; за настоящую цену отдай мне его, чтобы прекратилось истребление народа.
\vs 1Ch 21:23 И сказал Орна Давиду: возьми себе; пусть делает господин мой царь что ему угодно; вот я отдаю и волов на всесожжение, и молотильные орудия на дрова, и пшеницу на приношение; все это отдаю даром.
\vs 1Ch 21:24 И сказал царь Давид Орне: нет, я хочу купить у тебя за настоящую цену, ибо не стану я приносить твоей собственности Господу, и не буду приносить во всесожжение \bibemph{взятого} даром.
\vs 1Ch 21:25 И дал Давид Орне за это место шестьсот сиклей золота.
\vs 1Ch 21:26 И соорудил там Давид жертвенник Господу и вознес всесожжения и мирные жертвы; и призвал Господа, и Он услышал его, \bibemph{послав} огонь с неба на жертвенник всесожжения.
\vs 1Ch 21:27 И сказал Господь Ангелу: возврати меч твой в ножны его.
\vs 1Ch 21:28 В это время Давид, видя, что Господь услышал его на гумне Орны Иевусеянина, принес там жертву.
\rsbpar\vs 1Ch 21:29 Скиния же Господня, которую сделал Моисей в пустыне, и жертвенник всесожжения \bibemph{находились} в то время на высоте в Гаваоне.
\vs 1Ch 21:30 И не мог Давид пойти туда, чтобы взыскать Бога, потому что устрашен был мечом Ангела Господня.
\vs 1Ch 22:1 И сказал Давид: вот дом Господа Бога и вот жертвенник для всесожжений Израиля.
\vs 1Ch 22:2 И приказал Давид собрать пришельцев, находившихся в земле Израильской, и поставил каменотесов, чтобы обтесывать камни для построения дома Божия.
\vs 1Ch 22:3 И множество железа для гвоздей к дверям ворот и для связей заготовил Давид, и множество меди без весу,
\vs 1Ch 22:4 и кедровых дерев без счету, потому что Сидоняне и Тиряне доставили Давиду множество кедровых дерев.
\vs 1Ch 22:5 И сказал Давид: Соломон, сын мой, молод и малосилен, а дом, который следует выстроить для Господа, должен быть весьма величествен, на славу и украшение пред всеми землями: итак буду я заготовлять для него. И заготовил Давид до смерти своей много.
\rsbpar\vs 1Ch 22:6 И призвал Соломона, сына своего, и завещал ему построить дом Господу Богу Израилеву.
\vs 1Ch 22:7 И сказал Давид Соломону: сын мой! у меня было на сердце построить дом во имя Господа, Бога моего,
\vs 1Ch 22:8 но было ко мне слово Господне, и сказано: <<ты пролил много крови и вел большие войны; ты не должен строить дома имени Моему, потому что пролил много крови на землю пред лицем Моим.
\vs 1Ch 22:9 Вот, у тебя родится сын: он будет человек мирный; Я дам ему покой от всех врагов его кругом: посему имя ему будет Соломон. И мир и покой дам Израилю во дни его.
\vs 1Ch 22:10 Он построит дом имени Моему, и он будет Мне сыном, а Я ему отцом, и утвержу престол царства его над Израилем навек>>.
\vs 1Ch 22:11 И ныне, сын мой! да будет Господь с тобою, чтобы ты был благоуспешен и построил дом Господу Богу твоему, как Он говорил о тебе.
\vs 1Ch 22:12 Да даст тебе Господь смысл и разум, и поставит тебя над Израилем; и соблюди закон Господа Бога твоего.
\vs 1Ch 22:13 Тогда ты будешь благоуспешен, если будешь стараться исполнять уставы и законы, которые заповедал Господь Моисею для Израиля. Будь тверд и мужествен, не бойся и не унывай.
\vs 1Ch 22:14 И вот, я при скудости моей приготовил для дома Господня сто тысяч талантов золота и тысячу тысяч талантов серебра, а меди и железу нет веса, потому что их множество; и дерева и камни я также заготовил, а ты еще прибавь к этому.
\vs 1Ch 22:15 У тебя множество рабочих, и каменотесов, резчиков и плотников, и всяких способных на всякое дело;
\vs 1Ch 22:16 золоту, серебру и меди и железу нет счета: начни и делай; Господь будет с тобою.
\rsbpar\vs 1Ch 22:17 И завещал Давид всем князьям Израилевым помогать Соломону, сыну его:
\vs 1Ch 22:18 не с вами ли Господь Бог наш, давший вам покой со всех сторон? потому что Он предал в руки мои жителей земли, и покорилась земля пред Господом и пред народом Его.
\vs 1Ch 22:19 Итак расположите сердце ваше и душу вашу к тому, чтобы взыскать Господа Бога вашего. Встаньте и постройте святилище Господу Богу, чтобы перенести ковчег завета Господня и священные сосуды Божии в дом, созидаемый имени Господню.
\vs 1Ch 23:1 Давид, состарившись и насытившись \bibemph{жизнью}, воцарил над Израилем сына своего Соломона.
\vs 1Ch 23:2 И собрал всех князей Израилевых и священников и левитов,
\vs 1Ch 23:3 и исчислены были левиты, от тридцати лет и выше, и было число их, считая поголовно, тридцать восемь тысяч человек.
\vs 1Ch 23:4 Из них \bibemph{назначены} для дела в доме Господнем двадцать четыре тысячи, писцов же и судей шесть тысяч,
\vs 1Ch 23:5 и четыре тысячи привратников, и четыре тысячи прославляющих Господа на \bibemph{музыкальных} орудиях, которые он сделал для прославления.
\rsbpar\vs 1Ch 23:6 И разделил их Давид на череды по сынам Левия~--- Гирсону, Каафу и Мерари.
\vs 1Ch 23:7 Из Гирсонян~--- Лаедан и Шимей.
\vs 1Ch 23:8 Сыновья Лаедана: первый Иехиил, Зефам и Иоиль, трое.
\vs 1Ch 23:9 Сыновья Шимея: Шеломиф, Хазиил и Гаран, трое. Они главы поколений Лаедановых.
\vs 1Ch 23:10 Еще сыновья Шимея: Иахаф, Зиза, Иеуш и Берия. Это сыновья Шимея, четверо.
\vs 1Ch 23:11 Иахаф был главным, Зиза вторым; Иеуш и Берия имели детей немного, и потому они были в одном счете при доме отца.
\vs 1Ch 23:12 Сыновья Каафа: Амрам, Ицгар, Хеврон и Озиил, четверо.
\rsbpar\vs 1Ch 23:13 Сыновья Амрама: Аарон и Моисей. Аарон отделен был на посвящение ко Святому Святых, он и сыновья его, на веки, чтобы совершать курение пред лицем Господа, чтобы служить Ему и благословлять именем Его на веки.
\vs 1Ch 23:14 А Моисей, человек Божий, \bibemph{и} сыновья его причтены к колену Левиину.
\vs 1Ch 23:15 Сыновья Моисея: Гирсон и Елиезер.
\vs 1Ch 23:16 Сыновья Гирсона: первый был Шевуил.
\vs 1Ch 23:17 Сыновья Елиезера были: первый Рехавия. И не было у Елиезера других сыновей; у Рехавии же было очень много сыновей.
\vs 1Ch 23:18 Сыновья Ицгара: первый Шеломиф.
\vs 1Ch 23:19 Сыновья Хеврона: первый Иерия и второй Амария, третий Иахазиил и четвертый Иекамам.
\vs 1Ch 23:20 Сыновья Озиила: первый Миха и второй Ишшия.
\vs 1Ch 23:21 Сыновья Мерарины: Махли и Муши. Сыновья Махлия: Елеазар и Кис.
\vs 1Ch 23:22 И умер Елеазар, и не было у него сыновей, а только дочери; и взяли их за себя сыновья Киса, братья их.
\vs 1Ch 23:23 Сыновья Мушия: Махли, Едер и Иремоф~--- трое.
\vs 1Ch 23:24 Вот сыновья Левиины, по домам отцов их, главы семейств, по именному счислению их поголовно, которые отправляли дела служения в доме Господнем, от двадцати лет и выше.
\vs 1Ch 23:25 Ибо Давид сказал: Господь, Бог Израилев, дал покой народу Своему и водворил его в Иерусалиме на веки,
\vs 1Ch 23:26 и левитам не нужно носить скинию и всякие вещи ее для служения в ней.
\rsbpar\vs 1Ch 23:27 Посему, по последним повелениям Давида, исчислены левиты от двадцати лет и выше,
\vs 1Ch 23:28 чтоб они были при сынах Аароновых, для служения дому Господню, во дворе и в пристройках, для соблюдения чистоты всего святилища и для исполнения всякой службы при доме Божием,
\vs 1Ch 23:29 для наблюдения за хлебами предложения и пшеничною мукою для хлебного приношения и пресными лепешками, за печеным, жареным и за всякою мерою и весом,
\vs 1Ch 23:30 и чтобы становились каждое утро благодарить и славословить Господа, также и вечером,
\vs 1Ch 23:31 и при всех всесожжениях, возносимых Господу в субботы, в новомесячия и в праздники по числу, как предписано о них,~--- постоянно пред лицем Господа,
\vs 1Ch 23:32 и чтобы охраняли скинию откровения и святилище и сынов Аароновых, братьев своих, при службах дому Господню.
\vs 1Ch 24:1 И вот распределения сыновей Аароновых: сыновья Аарона: Надав, Авиуд, Елеазар и Ифамар.
\vs 1Ch 24:2 Надав и Авиуд умерли прежде отца своего, сыновей же не было у них, и потому священствовали Елеазар и Ифамар.
\vs 1Ch 24:3 И распределил их Давид~--- Садока из сыновей Елеазара, и Ахимелеха из сыновей Ифамара, поочередно на службу их.
\vs 1Ch 24:4 И нашлось, что между сынами Елеазара глав поколений более, нежели между сынами Ифамара. И он распределил их \bibemph{так}: из сынов Елеазара шестнадцать глав семейств, а из сынов Ифамара восемь.
\vs 1Ch 24:5 Распределял же их по жребиям, потому что главными во святилище и главными пред Богом были из сынов Елеазара и из сынов Ифамара,
\vs 1Ch 24:6 и записывал их Шемаия, сын Нафанаила, писец из левитов, пред лицем царя и князей и пред священником Садоком и Ахимелехом, сыном Авиафара, и пред главами семейств священнических и левитских: брали \bibemph{при бросании жребия} одно семейство из \bibemph{рода} Елеазарова, потом брали из \bibemph{рода} Ифамарова.
\vs 1Ch 24:7 И вышел первый жребий Иегоиариву, второй Иедаии,
\vs 1Ch 24:8 третий Хариму, четвертый Сеориму,
\vs 1Ch 24:9 пятый Малхию, шестой Миямину,
\vs 1Ch 24:10 седьмой Гаккоцу, восьмой Авии,
\vs 1Ch 24:11 девятый Иешую, десятый Шехании,
\vs 1Ch 24:12 одиннадцатый Елиашиву, двенадцатый Иакиму,
\vs 1Ch 24:13 тринадцатый Хушаю, четырнадцатый Иешеваву,
\vs 1Ch 24:14 пятнадцатый Вилге, шестнадцатый Имеру,
\vs 1Ch 24:15 семнадцатый Хезиру, восемнадцатый Гапицецу,
\vs 1Ch 24:16 девятнадцатый Петахии, двадцатый Иезекиилю,
\vs 1Ch 24:17 двадцать первый Иахину, двадцать второй Гамулу,
\vs 1Ch 24:18 двадцать третий Делаии, двадцать четвертый Маазии.
\vs 1Ch 24:19 Вот порядок их при служении их, как \bibemph{им} приходить в дом Господень, по уставу их чрез Аарона, отца их, как заповедал ему Господь Бог Израилев.
\rsbpar\vs 1Ch 24:20 У прочих сыновей Левия~--- \bibemph{распределение}: из сынов Амрама: Шуваил; из сынов Шуваила: Иедия;
\vs 1Ch 24:21 от Рехавии: из сынов Рехавии Ишшия был первый;
\vs 1Ch 24:22 от Ицгара: Шеломоф; из сыновей Шеломофа: Иахав;
\vs 1Ch 24:23 из сыновей \bibemph{Хеврона}: первый Иерия, второй Амария, третий Иахазиил, четвертый Иекамам.
\vs 1Ch 24:24 \bibemph{Из} сыновей Озиила: Миха; из сыновей Михи: Шамир.
\vs 1Ch 24:25 Брат Михи Ишшия; из сыновей Ишшии: Захария.
\vs 1Ch 24:26 Сыновья Мерари: Махли и Муши; \bibemph{из} сыновей Иаазии: Бено.
\vs 1Ch 24:27 \bibemph{Из} сыновей Мерари у Иаазии: Бено и Шогам, и Заккур и Иври.
\vs 1Ch 24:28 У Махлия~--- Елеазар; у него сыновей не было.
\vs 1Ch 24:29 У Киса: \bibemph{из} сыновей Киса: Иерахмиил;
\vs 1Ch 24:30 сыновья Мушия: Махли, Едер и Иеримоф. Вот сыновья левитов по поколениям их.
\vs 1Ch 24:31 Бросали и они жребий, наравне с братьями своими, сыновьями Аароновыми, пред лицем царя Давида и Садока и Ахимелеха, и глав семейств священнических и левитских: глава семейства наравне с меньшим братом своим.
\vs 1Ch 25:1 И отделил Давид и начальники войска на службу сыновей Асафа, Емана и Идифуна, чтобы они провещавали на цитрах, псалтирях и кимвалах; и были отчислены они на дело служения своего:
\vs 1Ch 25:2 из сыновей Асафа: Заккур, Иосиф, Нефания и Ашарела сыновья Асафа, под руководством Асафа, игравшего по наставлению царя.
\vs 1Ch 25:3 От Идифуна сыновья Идифуна: Гедалия, Цери, Исаия, Семей, Хашавия и Маттафия, шестеро, под руководством отца своего Идифуна, игравшего на цитре во славу и хвалу Господа.
\vs 1Ch 25:4 От Емана сыновья Емана: Буккия, Матфания, Озиил, Шевуил и Иеримоф, Ханания, Ханани, Елиафа, Гиддалти, Ромамти-Езер, Иошбекаша, Маллофи, Гофир и Махазиоф.
\vs 1Ch 25:5 Все эти сыновья Емана, прозорливца царского, по словам Божиим, чтобы возвышать славу его. И дал Бог Еману четырнадцать сыновей и трех дочерей.
\vs 1Ch 25:6 Все они под руководством отца своего пели в доме Господнем с кимвалами, псалтирями и цитрами в служении в доме Божием, по указанию царя, или Асафа, Идифуна и Емана.
\vs 1Ch 25:7 И было число их с братьями их, обученными петь пред Господом, всех знающих \bibemph{сие дело}, двести восемьдесят восемь.
\rsbpar\vs 1Ch 25:8 И бросили они жребий о череде служения, малый наравне с большим, учители \bibemph{наравне} с учениками.
\vs 1Ch 25:9 И вышел первый жребий Асафу, для Иосифа; второй Гедалии с братьями его и сыновьями его; их было двенадцать;
\vs 1Ch 25:10 третий Заккуру с сыновьями его и братьями его; их~--- двенадцать;
\vs 1Ch 25:11 четвертый Ицрию с сыновьями его и братьями его; их~--- двенадцать;
\vs 1Ch 25:12 пятый Нефании с сыновьями его и братьями его; их~--- двенадцать;
\vs 1Ch 25:13 шестой Буккии с сыновьями его и братьями его; их~--- двенадцать;
\vs 1Ch 25:14 седьмой Иесареле с сыновьями его и братьями его; их~--- двенадцать;
\vs 1Ch 25:15 восьмой Исаии с сыновьями его и братьями его; их~--- двенадцать;
\vs 1Ch 25:16 девятый Матфании с сыновьями его и братьями его; их~--- двенадцать;
\vs 1Ch 25:17 десятый Шимею с сыновьями его и братьями его; их~--- двенадцать;
\vs 1Ch 25:18 одиннадцатый Азариилу с сыновьями его и братьями его; их~--- двенадцать;
\vs 1Ch 25:19 двенадцатый Хашавии с сыновьями его и братьями его; их~--- двенадцать;
\vs 1Ch 25:20 тринадцатый Шуваилу с сыновьями его и братьями его; их~--- двенадцать;
\vs 1Ch 25:21 четырнадцатый Маттафии с сыновьями его и братьями его; их~--- двенадцать;
\vs 1Ch 25:22 пятнадцатый Иеримофу с сыновьями его и братьями его; их~--- двенадцать;
\vs 1Ch 25:23 шестнадцатый Ханании с сыновьями его и братьями его; их~--- двенадцать;
\vs 1Ch 25:24 семнадцатый Иошбекаше с сыновьями его и братьями его; их~--- двенадцать;
\vs 1Ch 25:25 восемнадцатый Ханани с сыновьями его и братьями его; их~--- двенадцать;
\vs 1Ch 25:26 девятнадцатый Маллофию с сыновьями его и братьями его; их~--- двенадцать;
\vs 1Ch 25:27 двадцатый Елиафе с сыновьями его и братьями его; их~--- двенадцать;
\vs 1Ch 25:28 двадцать первый Гофиру с сыновьями его и братьями его; их~--- двенадцать;
\vs 1Ch 25:29 двадцать второй Гиддалтию с сыновьями его и братьями его; их~--- двенадцать;
\vs 1Ch 25:30 двадцать третий Махазиофу с сыновьями его и братьями его; их~--- двенадцать;
\vs 1Ch 25:31 двадцать четвертый Ромамти-Езеру с сыновьями его и братьями его; их~--- двенадцать.
\vs 1Ch 26:1 Вот распределение привратников: из Кореян: Мешелемия, сын Корея, из сыновей Асафовых.
\vs 1Ch 26:2 Сыновья Мешелемии: первенец Захария, второй Иедиаил, третий Зевадия, четвертый Иафниил,
\vs 1Ch 26:3 пятый Елам, шестой Иегоханан, седьмой Елиегоэнай.
\vs 1Ch 26:4 Сыновья Овед-Едома: первенец Шемаия, второй Иегозавад, третий Иоах, четвертый Сахар, пятый Нафанаил,
\vs 1Ch 26:5 шестой Аммиил, седьмой Иссахар, восьмой Пеульфай, потому что Бог благословил его.
\vs 1Ch 26:6 У сына его Шемаии родились также сыновья, начальствовавшие в своем роде, потому что они были люди сильные.
\vs 1Ch 26:7 Сыновья Шемаии: Офни, Рефаил, Овед и Елзавад, братья его, люди сильные, Елия, Семахия [и Иеваком].
\vs 1Ch 26:8 Все они из сыновей Овед-Едома; они и сыновья их, и братья их были люди прилежные и к службе способные: их было у Овед-Едома шестьдесят два.
\vs 1Ch 26:9 У Мешелемии сыновей и братьев, людей способных, \bibemph{было} восемнадцать.
\vs 1Ch 26:10 У Хосы, из сыновей Мерариных, сыновья: Шимри главный,~--- хотя он не был первенцем, но отец его поставил его главным;
\vs 1Ch 26:11 второй Хелкия, третий Тевалия, четвертый Захария; всех сыновей и братьев у Хосы было тринадцать.
\vs 1Ch 26:12 Вот распределение привратников по главам семейств, способных на службу вместе с братьями их, для служения в доме Господнем.
\vs 1Ch 26:13 И бросили они жребии, как малый, так и большой, по своим семействам, на каждые ворота.
\vs 1Ch 26:14 И выпал жребий на восток Шелемии; и Захарии, сыну его, умному советнику, бросили жребий, и вышел ему жребий на север;
\vs 1Ch 26:15 Овед-Едому на юг, а сыновьям его при кладовых.
\vs 1Ch 26:16 Шупиму и Хосе на запад, у ворот Шаллехет, где дорога поднимается и где стража против стражи.
\vs 1Ch 26:17 К востоку по шести левитов, к северу по четыре, к югу по четыре, а у кладовых по два.
\vs 1Ch 26:18 К западу у притвора на дороге по четыре, а у самого притвора по два.
\vs 1Ch 26:19 Вот распределение привратников из сыновей Кореевых и сыновей Мерариных.
\rsbpar\vs 1Ch 26:20 Левиты же, братья их, \bibemph{смотрели} за сокровищами дома Божия и за сокровищницами посвященных вещей.
\vs 1Ch 26:21 Сыновья Лаедана, сына Герсонова~--- от Лаедана, гл\acc{а}вы семейств от Лаедана Герсонского: Иехиел.
\vs 1Ch 26:22 Сыновья Иехиела: Зефам и Иоиль, брат его, \bibemph{смотрели} за сокровищами дома Господня,
\vs 1Ch 26:23 вместе с потомками Амрама, Ицгара, Хеврона, Озиила.
\vs 1Ch 26:24 Шевуил, сын Гирсона, сына Моисеева, \bibemph{был} главным смотрителем за сокровищницами.
\vs 1Ch 26:25 У брата его Елиезера сын Рехавия, у него сын Исаия, у него сын Иорам, у него сын Зихрий, у него сын Шеломиф.
\vs 1Ch 26:26 Шеломиф и братья его \bibemph{смотрели} за всеми сокровищницами посвященных вещей, которые посвятил царь Давид и главы семейств и тысяченачальники, стоначальники и предводители войска.
\vs 1Ch 26:27 Из завоеваний и из добыч они посвящали на поддержание дома Господня.
\vs 1Ch 26:28 И все, что посвятил Самуил пророк, и Саул, сын Киса, и Авенир, сын Нира, и Иоав, сын Саруи, все посвященное \bibemph{было} на руках у Шеломифа и братьев его.
\rsbpar\vs 1Ch 26:29 Из племени Ицгарова: Хенания и сыновья его \bibemph{определены} на внешнее служение у Израильтян, писцами и судьями.
\vs 1Ch 26:30 Из племени Хевронова: Хашавия и братья его, люди мужественные, тысяча семьсот, имели надзор над Израилем по эту сторону Иордана к западу, по всяким делам \bibemph{служения} Господня и по службе царской.
\vs 1Ch 26:31 У племени Хевронова Иерия \bibemph{был} главою Хевронян, в их родах, в поколениях. В сороковой год царствования Давида они исчислены, и найдены между ними люди мужественные в Иазере Галаадском.
\vs 1Ch 26:32 И братья его, люди способные, две тысячи семьсот, были главы семейств. Их поставил царь Давид над коленом Рувимовым и Гадовым и полуколеном Манассииным, по всем делам Божиим и делам царя.
\vs 1Ch 27:1 Вот сыны Израилевы по числу их, главы семейств, тысяченачальники и стоначальники и управители, которые по отделениям служили царю во всех делах, приходя и отходя каждый месяц, во все месяцы года. В каждом отделении было их по двадцать четыре тысячи.
\vs 1Ch 27:2 Над первым отделением, для первого месяца, \bibemph{начальствовал} Иашовам, сын Завдиила; в его отделении было двадцать четыре тысячи;
\vs 1Ch 27:3 он \bibemph{был} из сынов Фареса, главный над всеми военачальниками в первый месяц.
\vs 1Ch 27:4 Над отделением второго месяца был Додай Ахохиянин; в отделении его был и князь Миклоф, и в его отделении было двадцать четыре тысячи.
\vs 1Ch 27:5 Третий главный военачальник, для третьего месяца, Ванея, сын Иодая, священника, и в его отделении было двадцать четыре тысячи:
\vs 1Ch 27:6 этот Ванея~--- \bibemph{один} из тридцати храбрых и \bibemph{начальник} над ними, и в его отделе \bibemph{находился} Аммизавад, сын его.
\vs 1Ch 27:7 Четвертый, для четвертого месяца, был Асаил, брат Иоава, и по нем Завадия, сын его, и в его отделении двадцать четыре тысячи.
\vs 1Ch 27:8 Пятый, для пятого месяца, князь Шамгуф Израхитянин, и в его отделении двадцать четыре тысячи.
\vs 1Ch 27:9 Шестой, для шестого месяца, Ира, сын Иккеша, Фекоянин, и в его отделении двадцать четыре тысячи.
\vs 1Ch 27:10 Седьмой, для седьмого месяца, Хелец Пелонитянин, из сынов Ефремовых, и в его отделении двадцать четыре тысячи.
\vs 1Ch 27:11 Восьмой, для восьмого месяца, Совохай Хушатянин, из племени Зары, и в его отделении двадцать четыре тысячи.
\vs 1Ch 27:12 Девятый, для девятого месяца, Авиезер Анафофянин, из сыновей Вениаминовых, и в его отделении двадцать четыре тысячи.
\vs 1Ch 27:13 Десятый, для десятого месяца, Магарай Нетофафянин, из племени Зары, и в его отделении двадцать четыре тысячи.
\vs 1Ch 27:14 Одиннадцатый, для одиннадцатого месяца, Ванея Пирафонянин, из сынов Ефремовых, и в его отделении двадцать четыре тысячи.
\vs 1Ch 27:15 Двенадцатый, для двенадцатого месяца, Хелдай Нетофафянин, из потомков Гофониила, и в его отделении двадцать четыре тысячи.
\rsbpar\vs 1Ch 27:16 А над коленами Израилевыми,~--- у Рувимлян главным начальником \bibemph{был} Елиезер, сын Зихри; у Симеона~--- Сафатия, сын Маахи;
\vs 1Ch 27:17 у Левия~--- Хашавия, сын Кемуила; у Аарона~--- Садок;
\vs 1Ch 27:18 у Иуды~--- Елиав, из братьев Давида; у Иссахара~--- Омри, сын Михаила;
\vs 1Ch 27:19 у Завулона~--- Ишмаия, сын Овадии; у Неффалима~--- Иеримоф, сын Азриила;
\vs 1Ch 27:20 у сыновей Ефремовых~--- Осия, сын Азазии; у полуколена Манассиина~--- Иоиль, сын Федаии;
\vs 1Ch 27:21 у полуколена Манассии в Галааде~--- Иддо, сын Захарии; у Вениамина~--- Иаасиил, сын Авенира;
\vs 1Ch 27:22 у Дана~--- Азариил, сын Иерохама. Вот вожди колен Израилевых.
\vs 1Ch 27:23 Давид не делал счисления тех, которые были от двадцати лет и ниже, потому что Господь сказал, что Он умножит Израиля, как звезды небесные.
\vs 1Ch 27:24 Иоав, сын Саруи, начал делать счисление, но не кончил. И был за это гнев Божий на Израиля, и не вошло то счисление в летопись царя Давида.
\rsbpar\vs 1Ch 27:25 Над сокровищами царскими был Азмавеф, сын Адиилов, а над запасами в поле, в городах, и в селах и в башнях~--- Ионафан, сын Уззии;
\vs 1Ch 27:26 над занимающимися полевыми работами, земледелием~--- Езрий, сын Хелува;
\vs 1Ch 27:27 над виноградниками~--- Шимей из Рамы, а над запасами вина в виноградниках~--- Завдий из Шефама;
\vs 1Ch 27:28 над маслинами и смоковницами в долине~--- Баал-Ханан Гедеритянин, а над запасами деревянного масла~--- Иоас;
\vs 1Ch 27:29 над крупным скотом, пасущимся в Шароне~--- Шитрай Шаронянин, а над скотом в долинах~--- Шафат, сын Адлая;
\vs 1Ch 27:30 над верблюдами~--- Овил Исмаильтянин; над ослицами~--- Иехдия Меронифянин;
\vs 1Ch 27:31 над мелким скотом~--- Иазиз Агаритянин. Все эти были начальниками над имением, которое \bibemph{было} у царя Давида.
\rsbpar\vs 1Ch 27:32 Ионафан, дядя Давидов, \bibemph{был} советником, человек умный и писец; Иехиил, сын Хахмониев, \bibemph{был} при сыновьях царя;
\vs 1Ch 27:33 Ахитофел \bibemph{был} советником царя; Хусий Архитянин~--- другом царя;
\vs 1Ch 27:34 после же Ахитофела Иодай, сын Ванеи, и Авиафар, а Иоав был военачальником у царя.
\vs 1Ch 28:1 И собрал Давид в Иерусалим всех вождей Израильских, начальников колен и начальников отделов, служивших царю, и тысяченачальников, и стоначальников, и заведовавших всем имением и стадами царя и сыновей его с евнухами, военачальников и всех храбрых мужей.
\vs 1Ch 28:2 И стал Давид царь на ноги свои и сказал: послушайте меня, братья мои и народ мой! \bibemph{было} у меня на сердце построить дом покоя для ковчега завета Господня и в подножие ногам Бога нашего, и \bibemph{потребное} для строения я приготовил.
\vs 1Ch 28:3 Но Бог сказал мне: не строй д\acc{о}ма имени Моему, потому что ты человек воинственный и проливал кровь.
\vs 1Ch 28:4 Однако же избрал Господь Бог Израилев меня из всего дома отца моего, чтоб быть \bibemph{мне} царем над Израилем вечно, потому что Иуду избрал Он князем, а в доме Иуды дом отца моего, а из сыновей отца моего меня благоволил поставить царем над всем Израилем,
\vs 1Ch 28:5 из всех же сыновей моих,~--- ибо много сыновей дал мне Господь,~--- Он избрал Соломона, сына моего, сидеть на престоле царства Господня над Израилем,
\vs 1Ch 28:6 и сказал мне: Соломон, сын твой, построит дом Мой и дворы Мои, потому что Я избрал его Себе в сына, и Я буду ему Отцом;
\vs 1Ch 28:7 и утвержу царство его на веки, если он будет тверд в исполнении заповедей Моих и уставов Моих, как до сего дня.
\vs 1Ch 28:8 И теперь пред очами всего Израиля, собрания Господня, и во уши Бога нашего \bibemph{говорю}: соблюдайте и держитесь всех заповедей Господа Бога вашего, чтобы владеть вам сею доброю землею и оставить ее после себя в наследство детям своим на век;
\vs 1Ch 28:9 и ты, Соломон, сын мой, знай Бога отца твоего и служи Ему от всего сердца и от всей души, ибо Господь испытует все сердца и знает все движения мыслей. Если будешь искать Его, то найдешь Его, а если оставишь Его, Он оставит тебя навсегда.
\vs 1Ch 28:10 Смотри же, когда Господь избрал тебя построить дом для святилища, будь тверд и делай.
\rsbpar\vs 1Ch 28:11 И отдал Давид Соломону, сыну своему, чертеж притвора и домов его, и кладовых его, и горниц его, и внутренних покоев его, и дома для ковчега,
\vs 1Ch 28:12 и чертеж всего, что было у него на душе, дворов дома Господня и всех комнат кругом, сокровищниц дома Божия и сокровищниц вещей посвященных,
\vs 1Ch 28:13 и священнических и левитских отделений, и всякого служебного дела в доме Господнем, и всех служебных сосудов дома Господня,
\vs 1Ch 28:14 золотых вещей, с \bibemph{означением} веса, для всякого из служебных сосудов, всех вещей серебряных, с \bibemph{означением} веса, для всякого из сосудов служебных.
\vs 1Ch 28:15 И дал золота для светильников и золотых лампад их, с означением веса каждого из светильников и лампад его, также светильников серебряных, с означением веса каждого из светильников и лампад его, смотря по служебному назначению каждого светильника;
\vs 1Ch 28:16 и золота для столов предложения хлебов, для каждого \bibemph{золотого} стола, и серебра для столов серебряных,
\vs 1Ch 28:17 и вилок, и чаш и кропильниц из чистого золота, и золотых блюд, с означением веса каждого блюда, и серебряных блюд, с означением веса каждого блюда,
\vs 1Ch 28:18 и для жертвенника курения из литого золота с означением веса, и устройства колесницы с золотыми херувимами, распростирающими \bibemph{крылья} и покрывающими ковчег завета Господня.
\vs 1Ch 28:19 Все сие в письмени от Господа, \bibemph{говорил Давид, как} Он вразумил меня на все дела постройки.
\rsbpar\vs 1Ch 28:20 И сказал Давид сыну своему Соломону: будь тверд и мужествен, и приступай к делу, не бойся и не ужасайся, ибо Господь Бог, Бог мой, с тобою; Он не отступит от тебя и не оставит тебя, доколе не совершишь всего дела, требуемого для дома Господня.
\vs 1Ch 28:21 И вот отделы священников и левитов, для всякой службы при доме Божием. И у тебя есть для всякого дела усердные люди, искусные для всякой работы, и начальники и весь народ \bibemph{готовы} на все твои приказания.
\vs 1Ch 29:1 И сказал царь Давид всему собранию: Соломон, сын мой, которого одного избрал Бог, молод и малосилен, а дело сие велико, потому что не для человека здание сие, а для Господа Бога.
\vs 1Ch 29:2 Всеми силами я заготовил для дома Бога моего золото для золотых вещей и серебро для серебряных, и медь для медных, железо для железных, и дерев\acc{а} для деревянных, камни оникса и \bibemph{камни} вставные, камни красивые и разноцветные, и всякие дорогие камни, и множество мрамора;
\vs 1Ch 29:3 и еще по любви моей к дому Бога моего, есть у меня сокровище собственное из золота и серебра, \bibemph{и его} я отдаю для дома Бога моего, сверх всего, что заготовил я для святаго дома:
\vs 1Ch 29:4 три тысячи талантов золота, золота Офирского, и семь тысяч талантов серебра чистого, для обложения стен в домах,
\vs 1Ch 29:5 для каждой из золотых вещей, и для каждой из серебряных, и для всякого изделия рук художнических. Не поусердствует ли \bibemph{еще} кто жертвовать сегодня для Господа?
\rsbpar\vs 1Ch 29:6 И стали жертвовать начальники семейств и начальники колен Израилевых, и начальники тысяч и сотен, и начальники над имениями царя.
\vs 1Ch 29:7 И дали на устроение дома Божия пять тысяч талантов и десять тысяч драхм золота, и серебра десять тысяч талантов, и меди восемнадцать тысяч талантов, и железа сто тысяч талантов.
\vs 1Ch 29:8 И у кого нашлись \bibemph{дорогие} камни, те отдавали и их в сокровищницу дома Господня, на руки Иехиилу Герсонитянину.
\vs 1Ch 29:9 И радовался народ усердию их, потому что они от всего сердца жертвовали Господу, также и царь Давид весьма радовался.
\rsbpar\vs 1Ch 29:10 И благословил Давид Господа пред всем собранием, и сказал Давид: благословен Ты, Господи Боже Израиля, отца нашего, от века и до века!
\vs 1Ch 29:11 Твое, Господи, величие, и могущество, и слава, и победа и великолепие, и все, \bibemph{что} на небе и на земле, \bibemph{Твое}: Твое, Господи, царство, и Ты превыше всего, как Владычествующий.
\vs 1Ch 29:12 И богатство и слава от лица Твоего, и Ты владычествуешь над всем, и в руке Твоей сила и могущество, и во власти Твоей возвеличить и укрепить все.
\vs 1Ch 29:13 И ныне, Боже наш, мы славословим Тебя и хвалим величественное имя Твое.
\vs 1Ch 29:14 Ибо кто я и кто народ мой, что мы имели возможность так жертвовать? Но от Тебя все, и от руки Твоей \bibemph{полученное} мы отдали Тебе,
\vs 1Ch 29:15 потому что странники мы пред Тобою и пришельцы, как и все отцы наши, как тень дни наши на земле, и нет ничего прочного.
\vs 1Ch 29:16 Господи Боже наш! все это множество, которое приготовили мы для построения дома Тебе, святому имени Твоему, от руки Твоей оно, и все Твое.
\vs 1Ch 29:17 Знаю, Боже мой, что Ты испытуешь сердце и любишь чистосердечие; я от чистого сердца моего пожертвовал все сие, и ныне вижу, что и народ Твой, здесь находящийся, с радостью жертвует Тебе.
\vs 1Ch 29:18 Господи, Боже Авраама, Исаака и Израиля, отцов наших! сохрани сие навек, \bibemph{сие} расположение мыслей сердца народа Твоего, и направь сердце их к Тебе.
\vs 1Ch 29:19 Соломону же, сыну моему, дай сердце правое, чтобы соблюдать заповеди Твои, откровения Твои и уставы Твои, и исполнить все это и построить здание, для которого я сделал приготовление.
\rsbpar\vs 1Ch 29:20 И сказал Давид всему собранию: благословите Господа Бога нашего.~--- И благословило все собрание Господа Бога отцов своих, и пало, и поклонилось Господу и царю.
\vs 1Ch 29:21 И принесли Господу жертвы, и вознесли всесожжения Господу, на другой после сего день: тысячу тельцов, тысячу овнов, тысячу агнцев с их возлияниями, и множество жертв от всего Израиля.
\vs 1Ch 29:22 И ели и пили пред Господом в тот день, с великою радостью; и в другой раз воцарили Соломона, сына Давидова, и помазали пред Господом в правителя верховного, а Садока во священника.
\vs 1Ch 29:23 И сел Соломон на престоле Господнем, как царь, вместо Давида, отца своего, и был благоуспешен, и весь Израиль повиновался ему.
\vs 1Ch 29:24 И все начальники и сильные, также и все сыновья царя Давида подчинились Соломону царю.
\vs 1Ch 29:25 И возвеличил Господь Соломона пред очами всего Израиля, и даровал ему славу царства, какой не имел прежде его ни один царь у Израиля.
\rsbpar\vs 1Ch 29:26 И Давид, сын Иессеев, царствовал над всем Израилем.
\vs 1Ch 29:27 Времени царствования его над Израилем \bibemph{было} сорок лет: в Хевроне царствовал он семь лет, и в Иерусалиме царствовал тридцать три \bibemph{года}.
\vs 1Ch 29:28 И умер в доброй старости, насыщенный жизнью, богатством и славою; и воцарился Соломон, сын его, вместо него.
\rsbpar\vs 1Ch 29:29 Дела царя Давида, первые и последние, описаны в записях Самуила провидца и в записях Нафана пророка и в записях Гада прозорливца,
\vs 1Ch 29:30 равно и все царствование его, и мужество его, и происшествия, случившиеся с ним и с Израилем и со всеми земными царствами.

\bibbookdescr{2Ch}{
  inline={\LARGE Вторая книга\\\Huge Паралипоменон\fns{У Евреев: <<Летопись>>.}},
  toc={2-я Паралипоменон},
  bookmark={2-я Паралипоменон},
  header={2-я Паралипоменон},
  %headerleft={},
  %headerright={},
  abbr={2~Пар}
}
\vs 2Ch 1:1 И утвердился Соломон, сын Давидов, в царстве своем; и Господь Бог его \bibemph{был} с ним, и вознес его высоко.
\vs 2Ch 1:2 И приказал Соломон \bibemph{собраться} всему Израилю: тысяченачальникам и стоначальникам, и судьям, и всем начальствующим во всем Израиле~--- главам поколений.
\vs 2Ch 1:3 И пошли Соломон и все собрание с ним на высоту, что в Гаваоне, ибо там была Божия скиния собрания, которую устроил Моисей, раб Господень, в пустыне.
\vs 2Ch 1:4 Ковчег Божий принес Давид из Кириаф-Иарима на место, которое приготовил для него Давид, устроив для него скинию в Иерусалиме.
\vs 2Ch 1:5 А медный жертвенник, который сделал Веселеил, сын Урия, сына Орова, \bibemph{оставался} там, пред скиниею Господнею, и взыскал его Соломон с собранием.
\vs 2Ch 1:6 И там пред лицем Господа, на медном жертвеннике, который пред скиниею собрания, вознес Соломон тысячу всесожжений.
\rsbpar\vs 2Ch 1:7 В ту ночь явился Бог Соломону и сказал ему: проси, что Мне дать тебе.
\vs 2Ch 1:8 И сказал Соломон Богу: Ты сотворил Давиду, отцу моему, великую милость и поставил меня царем вместо него.
\vs 2Ch 1:9 Да исполнится же, Господи Боже, слово Твое к Давиду, отцу моему. Так как Ты воцарил меня над народом многочисленным, как прах земной,
\vs 2Ch 1:10 то ныне дай мне премудрость и знание, чтобы я \bibemph{умел} выходить пред народом сим и входить, ибо кто может управлять сим народом Твоим великим?
\vs 2Ch 1:11 И сказал Бог Соломону: за то, что это было на сердце твоем, и ты не просил богатства, имения и славы и души неприятелей твоих, и также не просил ты многих дней, а просил себе премудрости и знания, чтобы управлять народом Моим, над которым Я воцарил тебя,
\vs 2Ch 1:12 премудрость и знание дается тебе, а богатство и имение и славу Я дам тебе такие, подобных которым не бывало у царей прежде тебя и не будет после тебя.
\rsbpar\vs 2Ch 1:13 И пришел Соломон с высоты, что в Гаваоне, от скинии собрания, в Иерусалим и царствовал над Израилем.
\vs 2Ch 1:14 И набрал Соломон колесниц и всадников; и было у него тысяча четыреста колесниц и двенадцать тысяч всадников; и он разместил их в колесничных городах и при царе в Иерусалиме.
\vs 2Ch 1:15 И сделал царь серебро и золото в Иерусалиме равноценным \bibemph{простому} камню, а кедры, по множеству их, сделал равноценными сикоморам, которые на низких местах.
\vs 2Ch 1:16 Коней Соломону приводили из Египта и из Кувы; купцы царские из Кувы получали их за деньги.
\vs 2Ch 1:17 Колесница получаема и доставляема была из Египта за шестьсот \bibemph{сиклей} серебра, а конь за сто пятьдесят. Таким же образом они руками своими доставляли \bibemph{это} всем царям Хеттейским и царям Арамейским.
\vs 2Ch 2:1 И положил Соломон построить дом имени Господню и дом царский для себя.
\vs 2Ch 2:2 И отчислил Соломон семьдесят тысяч носильщиков и восемьдесят тысяч каменосеков в горах, и надзирателей над ними три тысячи шестьсот.
\vs 2Ch 2:3 И послал Соломон к Хираму, царю Тирскому, сказать: как поступал ты с Давидом, отцом моим, и присылал ему кедры на построение дома для его жительства, \bibemph{так поступи и со мною}.
\vs 2Ch 2:4 Вот я строю дом имени Господа Бога моего, для посвящения Ему, чтобы возжигать пред Ним благовонное курение, представлять постоянно хлебы предложения и \bibemph{возносить там} всесожжения утром и вечером в субботы, и в новомесячия, и в праздники Господа Бога нашего, что навсегда заповедано Израилю.
\vs 2Ch 2:5 И дом, который я строю, велик, потому что велик Бог наш, выше всех богов.
\vs 2Ch 2:6 И достанет ли у кого силы построить Ему дом, когда небо и небеса небес не вмещают Его? И кто я, чтобы мог построить Ему дом? Разве \bibemph{только} для курения пред лицем Его.
\vs 2Ch 2:7 Итак пришли мне человека, умеющего делать \bibemph{изделия} из золота, и из серебра, и из меди, и из железа, и из \bibemph{пряжи} пурпурового, багряного и яхонтового \bibemph{цвета}, и знающего вырезывать резную работу, вместе с художниками, какие есть у меня в Иудее и в Иерусалиме, которых приготовил Давид, отец мой.
\vs 2Ch 2:8 И пришли мне кедровых дерев, и кипарису и певгового дерева с Ливана, ибо я знаю, что рабы твои умеют рубить дерева Ливанские. И вот рабы мои пойдут с рабами твоими,
\vs 2Ch 2:9 чтобы мне приготовить множество дерев, потому что дом, который я строю, великий и чудный.
\vs 2Ch 2:10 И вот древосекам, рубящим дерева, рабам твоим, я даю в пищу: пшеницы двадцать тысяч к\acc{о}ров, и ячменю двадцать тысяч к\acc{о}ров, и вина двадцать тысяч батов, и оливкового масла двадцать тысяч батов.
\rsbpar\vs 2Ch 2:11 И отвечал Хирам, царь Тирский, письмом, которое прислал к Соломону: по любви к народу Своему, Господь поставил тебя царем над ним.
\vs 2Ch 2:12 И \bibemph{еще} сказал Хирам: благословен Господь Бог Израилев, создавший небо и землю, давший царю Давиду сына мудрого, имеющего смысл и разум, который намерен строить дом Господу и дом царский для себя.
\vs 2Ch 2:13 Итак я посылаю [тебе] человека умного, имеющего знания, Хирам-Авия,
\vs 2Ch 2:14 сына \bibemph{одной} женщины из дочерей Дановых,~--- а отец его Тирянин,~--- умеющего делать \bibemph{изделия} из золота и из серебра, из меди, из железа, из камней и из дерев, из \bibemph{пряжи} пурпурового, яхонтового \bibemph{цвета}, и из виссона, и из багряницы, и вырезывать всякую резьбу, и исполнять все, что будет поручено ему вместе с художниками твоими и с художниками господина моего Давида, отца твоего.
\vs 2Ch 2:15 А пшеницу и ячмень, оливковое масло и вино, о которых говорил ты, господин мой, пошли рабам твоим.
\vs 2Ch 2:16 Мы же нарубим дерев с Ливана, сколько нужно тебе, и пригоним их в плотах по морю в Яфу, а ты отвезешь их в Иерусалим.
\rsbpar\vs 2Ch 2:17 И исчислил Соломон всех пришельцев, бывших \bibemph{тогда} в земле Израилевой, после исчисления их, сделанного Давидом, отцом его,~--- и нашлось их сто пятьдесят три тысячи шестьсот.
\vs 2Ch 2:18 И сделал он из них семьдесят тысяч носильщиков и восемьдесят тысяч каменосеков на горах и три тысячи шестьсот надзирателей, чтобы они побуждали народ к работе.
\vs 2Ch 3:1 И начал Соломон строить дом Господень в Иерусалиме на горе Мориа, которая указана была Давиду, отцу его, на месте, которое приготовил Давид, на гумне Орны Иевусеянина.
\vs 2Ch 3:2 Начал же он строить во второй \bibemph{день} второго месяца, в четвертый год царствования своего.
\vs 2Ch 3:3 И вот основание, \bibemph{положенное} Соломоном при строении дома Божия: длина \bibemph{его} шестьдесят локтей, по прежней мере, а ширина двадцать локтей;
\vs 2Ch 3:4 и притвор, который пред домом, длиною по ширине дома в двадцать локтей, а вышиною во сто двадцать. И обложил его внутри чистым золотом.
\vs 2Ch 3:5 Дом же главный обшил деревом кипарисовым и обложил его лучшим золотом, и выделал на нем пальмы и цепочки.
\vs 2Ch 3:6 И обложил дом дорогими камнями для красоты; золото же \bibemph{было} золото Парваимское.
\vs 2Ch 3:7 И покрыл дом, бревна, пороги и стены его, [и окна] и двери его золотом, и вырезал на стенах херувимов.
\vs 2Ch 3:8 И сделал Святое Святых: длина его по широте дома в двадцать локтей, и ширина его в двадцать локтей; и покрыл его лучшим золотом на шестьсот талантов.
\vs 2Ch 3:9 В гвоздях весу до пятидесяти сиклей золота [в каждом гвозде]. Горницы также покрыл золотом.
\vs 2Ch 3:10 И сделал он во Святом Святых двух херувимов резной работы и покрыл их золотом.
\vs 2Ch 3:11 Крылья херувимов длиною \bibemph{были} в двадцать локтей. Одно крыло в пять локтей касалось стены дома, а другое крыло в пять же локтей сходилось с крылом другого херувима;
\vs 2Ch 3:12 \bibemph{равно} и крыло другого херувима в пять локтей касалось стены дома, а другое крыло в пять локтей сходилось с крылом другого херувима.
\vs 2Ch 3:13 Крылья сих херувимов \bibemph{были} распростерты на двадцать локтей; и они стояли на ногах своих, лицами своими к храму.
\vs 2Ch 3:14 И сделал завесу из яхонтовой, пурпуровой и багряной \bibemph{ткани} и из виссона и изобразил на ней херувимов.
\vs 2Ch 3:15 И сделал пред храмом два столба, длиною по тридцати пяти локтей, и капитель на верху каждого в пять локтей.
\vs 2Ch 3:16 И сделал цепочки, \bibemph{как} во святилище, и положил на верху столбов, и сделал сто гранатовых яблок и положил на цепочки.
\vs 2Ch 3:17 И поставил столбы пред храмом, один по правую сторону, другой по левую, и дал имя правому Иахин, а левому имя Воаз.
\vs 2Ch 4:1 И сделал медный жертвенник: двадцать локтей длина его и двадцать локтей ширина его и десять локтей вышина его.
\vs 2Ch 4:2 И сделал море литое,~--- от края его до края его десять локтей,~--- все круглое, вышиною в пять локтей; и снурок в тридцать локтей обнимал его кругом;
\vs 2Ch 4:3 и \bibemph{литые} подобия волов стояли под ним кругом со всех сторон; на десять локтей окружали море кругом два ряда волов, вылитых одним литьем с ним.
\vs 2Ch 4:4 Стояло оно на двенадцати волах: три глядели к северу и три глядели к западу, и три глядели к югу, и три глядели к востоку,~--- и море на них сверху; зады же их были обращены внутрь под него.
\vs 2Ch 4:5 Толщиною оно \bibemph{было} в ладонь; и края его, сделанные, как края чаши, \bibemph{походили} на распустившуюся лилию. Оно вмещало до трех тысяч батов.
\vs 2Ch 4:6 И сделал десять омывальниц, и поставил пять по правую сторону и пять по левую, чтоб омывать в них,~--- приготовляемое ко всесожжению омывали в них; море же~--- для священников, чтоб они омывались в нем.
\vs 2Ch 4:7 И сделал десять золотых светильников, как им быть надлежало, и поставил в храме, пять по правую сторону и пять по левую.
\vs 2Ch 4:8 И сделал десять столов и поставил в храме, пять по правую сторону и пять по левую, и сделал сто золотых чаш.
\vs 2Ch 4:9 И сделал священнический двор и большой двор и двери к двору, и вереи их обложил медью.
\vs 2Ch 4:10 Море поставил на правой стороне, к юго-востоку.
\vs 2Ch 4:11 И сделал Хирам тазы, и лопатки, и чаши [и кадильницы, и все жертвенные сосуды]. И кончил Хирам работу, которую производил для царя Соломона в доме Божием:
\vs 2Ch 4:12 два столба и две опояски венцов на верху столбов, и две сетки для покрытия двух опоясок венцов, которые на главе столбов,
\vs 2Ch 4:13 и четыреста гранатовых яблок на двух сетках, два ряда гранатовых яблок для каждой сетки, для покрытия двух опоясок венцов, которые на столбах.
\vs 2Ch 4:14 И подставы сделал он, и омывальницы сделал на подставах;
\vs 2Ch 4:15 одно море, и двенадцать волов под ним,
\vs 2Ch 4:16 и тазы, и лопатки, и вилки, и весь прибор их сделал Хирам-Авий царю Соломону для дома Господня из полированной меди.
\vs 2Ch 4:17 В окрестности Иордана выливал их царь, в глинистой земле, между Сокхофом и Цередою.
\vs 2Ch 4:18 И сделал Соломон все вещи сии в великом множестве, так что не знали веса меди.
\vs 2Ch 4:19 Также сделал Соломон все вещи для дома Божия и золотой жертвенник, и столы, на которых хлебы предложения,
\vs 2Ch 4:20 и светильники и лампады их, чтобы возжигать их по уставу пред давиром, из чистого золота;
\vs 2Ch 4:21 и цветы, и лампады, и щипцы из золота, из самого чистого золота,
\vs 2Ch 4:22 и ножи, и кропильницы, и чаши, и лотки из золота самого чистого, и двери храма,~--- двери его внутренние во Святое Святых, и двери храма во святилище,~--- из золота.
\vs 2Ch 5:1 И окончилась вся работа, которую производил Соломон для дома Господня. И принес Соломон посвященное Давидом, отцом его, и серебро и золото и все вещи отдал в сокровищницы дома Божия.
\rsbpar\vs 2Ch 5:2 Тогда собрал Соломон старейшин Израилевых и всех глав колен, начальников поколений сынов Израилевых, в Иерусалим, для перенесения ковчега завета Господня из города Давидова, то есть \bibemph{с} Сиона.
\vs 2Ch 5:3 И собрались к царю все Израильтяне на праздник, в седьмой месяц.
\vs 2Ch 5:4 И пришли все старейшины Израилевы. Левиты взяли ковчег
\vs 2Ch 5:5 и понесли ковчег и скинию собрания и все вещи священные, которые в скинии,~--- понесли их священники и левиты.
\vs 2Ch 5:6 Царь же Соломон и все общество Израилево, собравшееся к нему пред ковчегом, приносили жертвы из овец и волов, которых невозможно исчислить и определить, по причине множества.
\vs 2Ch 5:7 И принесли священники ковчег завета Господня на место его, в давир храма~--- во Святое Святых, под крылья херувимов.
\vs 2Ch 5:8 И херувимы распростирали крылья над местом ковчега, и покрывали херувимы ковчег и шесты его сверху.
\vs 2Ch 5:9 И выдвинулись шесты, так что головки шестов ковчега видны были пред давиром, но не выказывались наружу, и они там до сего дня.
\vs 2Ch 5:10 Не было в ковчеге ничего кроме двух скрижалей, которые положил Моисей на Хориве, когда Господь заключил \bibemph{завет} с сынами Израилевыми, по исходе их из Египта.
\vs 2Ch 5:11 Когда священники вышли из святилища, ибо все священники, находившиеся там, освятились без различия отделов;
\vs 2Ch 5:12 и левиты певцы,~--- все они, \bibemph{то есть} Асаф, Еман, Идифун и сыновья их, и братья их,~--- одетые в виссон, с кимвалами и с псалтирями и цитрами стояли на восточной стороне жертвенника, и с ними сто двадцать священников, трубивших трубами,
\vs 2Ch 5:13 и были, как один, трубящие и поющие, издавая один голос к восхвалению и славословию Господа; и когда загремел звук труб и кимвалов и музыкальных орудий, и восхваляли Господа, ибо Он благ, ибо вовек милость Его; тогда дом, дом Господень, наполнило облако,
\vs 2Ch 5:14 и не могли священники стоять на служении по причине облака, потому что слава Господня наполнила дом Божий.
\vs 2Ch 6:1 Тогда сказал Соломон: Господь сказал, что Он благоволит обитать во мгле,
\vs 2Ch 6:2 а я построил дом в жилище Тебе, [Святый,] место для вечного Твоего пребывания.
\vs 2Ch 6:3 И обратился царь лицем своим и благословил все собрание Израильтян,~--- все собрание Израильтян стояло,~---
\vs 2Ch 6:4 и сказал: благословен Господь Бог Израилев, Который, чт\acc{о} сказал устами Своими Давиду, отцу моему, исполнил \bibemph{ныне} рукою Своею! Он говорил:
\vs 2Ch 6:5 <<с того дня, как Я вывел народ Мой из земли Египетской, Я не избрал города ни в одном из колен Израилевых для построения дома, в котором пребывало бы имя Мое, и не избрал человека, который был бы правителем народа Моего Израиля,
\vs 2Ch 6:6 но избрал Иерусалим, чтобы там пребывало имя Мое, и избрал Давида, чтоб он был над народом Моим Израилем>>.
\vs 2Ch 6:7 И было на сердце у Давида, отца моего, построить дом имени Господа, Бога Израилева.
\vs 2Ch 6:8 Но Господь сказал Давиду, отцу моему: <<у тебя есть на сердце построить храм имени Моему; хорошо, что это на сердце у тебя.
\vs 2Ch 6:9 Однако не ты построишь храм, а сын твой, который произойдет из чресл твоих,~--- он построит храм имени Моему>>.
\vs 2Ch 6:10 И исполнил Господь слово Свое, которое изрек: я вступил на место Давида, отца моего, и воссел на престоле Израилевом, как сказал Господь, и построил дом имени Господа Бога Израилева.
\vs 2Ch 6:11 И я поставил там ковчег, в котором завет Господа, заключенный Им с сынами Израилевыми.
\rsbpar\vs 2Ch 6:12 И стал \bibemph{Соломон} у жертвенника Господня впереди всего собрания Израильтян, и воздвиг руки свои,~---
\vs 2Ch 6:13 ибо Соломон сделал медный амвон длиною в пять локтей и шириною в пять локтей, а вышиною в три локтя, и поставил его среди двора; и стал на нем, и преклонил колени впереди всего собрания Израильтян, и воздвиг руки свои к небу,~---
\vs 2Ch 6:14 и сказал: Господи Боже Израилев! Нет Бога, подобного Тебе, ни на небе, ни на земле. Ты хранишь завет и милость к рабам Твоим, ходящим пред Тобою всем сердцем своим:
\vs 2Ch 6:15 Ты исполнил рабу Твоему Давиду, отцу моему, что Ты говорил ему; что изрек Ты устами Твоими, то в день сей исполнил рукою Твоею.
\vs 2Ch 6:16 И ныне, Господи Боже Израилев! исполни рабу Твоему Давиду, отцу моему, то, что Ты сказал ему, говоря: не прекратится у тебя [муж,] сидящий пред лицем Моим на престоле Израилевом, если только сыновья твои будут наблюдать за путями своими, ходя по закону Моему так, как ты ходил предо Мною.
\vs 2Ch 6:17 И ныне, Господи Боже Израилев! да будет верно слово Твое, которое Ты изрек рабу Твоему Давиду.
\vs 2Ch 6:18 Поистине, Богу ли жить с человеками на земле? Если небо и небеса небес не вмещают Тебя, тем менее храм сей, который построил я.
\vs 2Ch 6:19 Но призри на молитву раба Твоего и на прошение его, Господи Боже мой! услышь воззвание и молитву, которою раб Твой молится пред Тобою.
\vs 2Ch 6:20 Да будут очи Твои отверсты на храм сей днем и ночью, на место, где Ты обещал положить имя Твое, чтобы слышать молитву, которою раб Твой будет молиться на месте сем.
\vs 2Ch 6:21 Услышь моления раба Твоего и народа Твоего Израиля, какими они будут молиться на месте сем; услышь с места обитания Твоего, с небес, услышь и помилуй!
\vs 2Ch 6:22 Когда кто согрешит против ближнего своего, и потребуют от него клятвы, чтоб он поклялся, и будет совершаться клятва пред жертвенником Твоим в храме сем,
\vs 2Ch 6:23 тогда Ты услышь с неба и соверши суд над рабами Твоими, воздай виновному, возложив поступок его на голову его, и оправдай правого, воздав ему по правде его.
\vs 2Ch 6:24 Когда поражен будет народ Твой Израиль неприятелем за то, что согрешил пред Тобою, и они обратятся \bibemph{к Тебе}, и исповедают имя Твое, и будут просить и молиться пред Тобою в храме сем,
\vs 2Ch 6:25 тогда Ты услышь с неба, и прости грех народа Твоего Израиля, и возврати их в землю, которую Ты дал им и отцам их.
\vs 2Ch 6:26 Когда заключится небо и не будет дождя за то, что они согрешили пред Тобою, и будут молиться на месте сем, и исповедают имя Твое, и обратятся от греха своего, потому что Ты смирил их,
\vs 2Ch 6:27 тогда Ты услышь с неба и прости грех рабов Твоих и народа Твоего Израиля, указав им добрый путь, по которому идти им, и пошли дождь на землю Твою, которую Ты дал народу Твоему в наследие.
\vs 2Ch 6:28 Голод ли будет на земле, будет ли язва моровая, будет ли ветер палящий или ржа, саранча или червь, будут ли теснить его неприятели его на земле владений его, \bibemph{будет ли} какое бедствие, какая болезнь,
\vs 2Ch 6:29 всякую молитву, всякое прошение, какое будет от какого-либо человека или от всего народа Твоего Израиля, когда они почувствуют каждый бедствие свое и горе свое и прострут руки свои к храму сему,
\vs 2Ch 6:30 Ты услышь с неба~--- места обитания Твоего, и прости, и воздай каждому по всем путям его, как Ты знаешь сердце его,~--- ибо Ты один знаешь сердце сынов человеческих,~---
\vs 2Ch 6:31 чтобы они боялись Тебя и ходили путями Твоими во все дни, доколе живут на земле, которую Ты дал отцам нашим.
\vs 2Ch 6:32 Даже и иноплеменник, который не от народа Твоего Израиля, когда он придет из земли далекой ради имени Твоего великого и руки Твоей могущественной и мышцы Твоей простертой, и придет и будет молиться у храма сего,
\vs 2Ch 6:33 Ты услышь с неба, с места обитания Твоего, и сделай все, о чем будет взывать к Тебе иноплеменник, чтобы все народы земли узнали имя Твое, и чтобы боялись Тебя, как народ Твой Израиль, и знали, что Твоим именем называется дом сей, который построил я.
\vs 2Ch 6:34 Когда выйдет народ Твой на войну против неприятелей своих путем, которым Ты пошлешь его, и будет молиться Тебе, обратившись к городу сему, который избрал Ты, и к храму, который я построил имени Твоему,
\vs 2Ch 6:35 тогда услышь с неба молитву их и прошение их и сделай, что потребно для них.
\vs 2Ch 6:36 Когда они согрешат пред Тобою,~--- ибо нет человека, который не согрешил бы,~--- и Ты прогневаешься на них, и предашь их врагу, и отведут их пленившие их в землю далекую или близкую,
\vs 2Ch 6:37 и когда они в земле, в которую будут пленены, войдут в себя и обратятся и будут молиться Тебе в земле пленения своего, говоря: мы согрешили, сделали беззаконие, мы виновны,
\vs 2Ch 6:38 и обратятся к Тебе всем сердцем своим и всею душею своею в земле пленения своего, куда отведут их в плен, и будут молиться, обратившись к земле своей, которую Ты дал отцам их, и к городу, который избрал Ты, и к храму, который я построил имени Твоему,~---
\vs 2Ch 6:39 тогда услышь с неба, с места обитания Твоего, молитву их и прошение их, и сделай, что потребно для них, и прости народу Твоему, в чем он согрешил пред Тобою.
\vs 2Ch 6:40 Боже мой! да будут очи Твои отверсты и уши Твои внимательны к молитве на месте сем.
\vs 2Ch 6:41 И ныне, Господи Боже, стань на \bibemph{место} покоя Твоего, Ты и ковчег могущества Твоего. Священники Твои, Господи Боже, да облекутся во спасение, и преподобные Твои да насладятся благами.
\vs 2Ch 6:42 Господи Боже! не отврати лица помазанника Твоего, помяни милости к Давиду, рабу Твоему.
\vs 2Ch 7:1 Когда окончил Соломон молитву, сошел огонь с неба и поглотил всесожжение и жертвы, и слава Господня наполнила дом.
\vs 2Ch 7:2 И не могли священники войти в дом Господень, потому что слава Господня наполнила дом Господень.
\vs 2Ch 7:3 И все сыны Израилевы, видя, как сошел огонь и слава Господня на дом, пали лицем на землю, на помост, и поклонились, и славословили Господа, ибо Он благ, ибо вовек милость Его.
\rsbpar\vs 2Ch 7:4 Царь же и весь народ стали приносить жертвы пред лицем Господа.
\vs 2Ch 7:5 И принес царь Соломон в жертву двадцать две тысячи волов и сто двадцать тысяч овец: так освятили дом Божий царь и весь народ.
\vs 2Ch 7:6 Священники стояли в служении своем, и левиты с музыкальными орудиями Господа, которые сделал царь Давид для прославления Господа, ибо вечна милость Его, так как Давид славословил чрез них; священники же трубили перед ним, и весь Израиль стоял.
\vs 2Ch 7:7 Освятил Соломон и внутреннюю часть двора, которая пред домом Господним: ибо принес там всесожжения и тук мирных жертв, так как жертвенник медный, сделанный Соломоном, не мог вмещать всесожжения и хлебного приношения, и туков.
\vs 2Ch 7:8 И сделал Соломон в то время семидневный праздник, и весь Израиль с ним~--- собрание весьма большое, \bibemph{сошедшееся} от входа в Емаф до реки Египетской;
\vs 2Ch 7:9 а в день восьмой сделали попразднство, ибо освящение жертвенника совершали семь дней и праздник семь дней.
\vs 2Ch 7:10 И в двадцать третий день седьмого месяца \bibemph{царь} отпустил народ в шатры их, радующийся и веселящийся в сердце о благе, какое сделал Господь Давиду и Соломону и Израилю, народу Своему.
\rsbpar\vs 2Ch 7:11 И окончил Соломон дом Господень и дом царский; и все, что предположил Соломон в сердце своем сделать в доме Господнем и в доме своем, совершил он успешно.
\vs 2Ch 7:12 И явился Господь Соломону ночью и сказал ему: Я услышал молитву твою и избрал Себе место сие в дом жертвоприношения.
\vs 2Ch 7:13 Если Я заключу небо и не будет дождя, и если повелю саранче поядать землю, или пошлю моровую язву на народ Мой,
\vs 2Ch 7:14 и смирится народ Мой, который именуется именем Моим, и будут молиться, и взыщут лица Моего, и обратятся от худых путей своих, то Я услышу с неба и прощу грехи их и исцелю землю их.
\vs 2Ch 7:15 Ныне очи Мои будут отверсты и уши Мои внимательны к молитве на месте сем.
\vs 2Ch 7:16 И ныне Я избрал и освятил дом сей, чтобы имя Мое было там во веки; и очи Мои и сердце Мое будут там во все дни.
\vs 2Ch 7:17 И если ты будешь ходить пред лицем Моим, как ходил Давид, отец твой, и будешь делать все, что Я повелел тебе, и будешь хранить уставы Мои и законы Мои,
\vs 2Ch 7:18 то утвержу престол царства твоего, как Я обещал Давиду, отцу твоему, говоря: не прекратится у тебя [муж,] владеющий Израилем.
\vs 2Ch 7:19 Если же вы отступите и оставите уставы Мои и заповеди Мои, которые Я дал вам, и пойдете и станете служить богам иным и поклоняться им,
\vs 2Ch 7:20 то Я истреблю \bibemph{Израиля} с лица земли Моей, которую Я дал им, и храм сей, который Я освятил имени Моему, отвергну от лица Моего и сделаю его притчею и посмешищем у всех народов.
\vs 2Ch 7:21 И о храме сем высоком всякий, проходящий мимо него, ужаснется и скажет: за что поступил так Господь с землею сею и с храмом сим?
\vs 2Ch 7:22 И скажут: за то, что они оставили Господа, Бога отцов своих, Который вывел их из земли Египетской, и прилепились к богам иным, и поклонялись им, и служили им,~--- за то Он навел на них все это бедствие.
\vs 2Ch 8:1 По окончании двадцати лет, в которые Соломон строил дом Господень и свой дом,
\vs 2Ch 8:2 Соломон обстроил и города, которые дал Соломону Хирам, и поселил в них сынов Израилевых.
\vs 2Ch 8:3 И пошел Соломон на Емаф-Сува и взял его.
\vs 2Ch 8:4 И построил он Фадмор в пустыне, и все города для запасов, какие основал в Емафе.
\vs 2Ch 8:5 Он обстроил Вефорон верхний и Вефорон нижний, города укрепленные, со стенами, воротами и запорами,
\vs 2Ch 8:6 и Ваалаф и все города для запасов, которые были у Соломона, и все города для колесниц, и города для конных, и все, что хотел Соломон построить в Иерусалиме и на Ливане и во всей земле владения своего.
\rsbpar\vs 2Ch 8:7 Весь народ, оставшийся от Хеттеев, и Аморреев, и Ферезеев, и Евеев и Иевусеев, которые были не из сынов Израилевых,~---
\vs 2Ch 8:8 детей их, оставшихся после них на земле, которых не истребили сыны Израилевы,~--- сделал Соломон оброчными до сего дня.
\vs 2Ch 8:9 Сынов же Израилевых не делал Соломон работниками по делам своим, но они были воинами, и начальниками телохранителей его, и вождями колесниц его и всадников его.
\vs 2Ch 8:10 И было главных приставников у царя Соломона, управлявших народом, двести пятьдесят.
\vs 2Ch 8:11 А дочь Фараонову перевел Соломон из города Давидова в дом, который построил для нее, потому что, говорил он, не должна жить женщина у меня в доме Давида, царя Израилева, ибо свят он, так как вошел в него ковчег Господень.
\rsbpar\vs 2Ch 8:12 Тогда стал возносить Соломон всесожжения Господу на жертвеннике Господнем, который он устроил пред притвором,
\vs 2Ch 8:13 чтобы по уставу каждого дня приносить всесожжения, по заповеди Моисеевой, в субботы, и в новомесячия, и в праздники три раза в год: в праздник опресноков, и в праздник седмиц, и в праздник кущей.
\vs 2Ch 8:14 И установил он, по распоряжению Давида, отца своего, череды священников по службе их и левитов по стражам их, чтобы они славословили и служили при священниках по уставу каждого дня, и привратников по чередам их, к каждым воротам, потому что таково было завещание Давида, человека Божия.
\vs 2Ch 8:15 И не отступали от повелений царя о священниках и левитах ни в чем, ни в отношении сокровищ.
\vs 2Ch 8:16 Так устроено было все дело Соломоново от дня основания дома Господня до совершенного окончания его~--- дома Господня.
\rsbpar\vs 2Ch 8:17 Тогда пошел Соломон в Ецион-Гавер и в Елаф, который на берегу моря, в земле Идумейской.
\vs 2Ch 8:18 И прислал ему Хирам чрез слуг своих корабли и рабов, знающих море, и отправились они с слугами Соломоновыми в Офир, и добыли оттуда четыреста пятьдесят талантов золота, и привезли царю Соломону.
\vs 2Ch 9:1 Царица Савская, услышав о славе Соломона, пришла испытать Соломона загадками в Иерусалим, с весьма большим богатством, и с верблюдами, навьюченными благовониями и множеством золота и драгоценных камней. И пришла к Соломону и беседовала с ним обо всем, что было на сердце у нее.
\vs 2Ch 9:2 И объяснил ей Соломон все слова ее, и не нашлось ничего незнакомого Соломону, чего он не объяснил бы ей.
\vs 2Ch 9:3 И увидела царица Савская мудрость Соломона и дом, который он построил,
\vs 2Ch 9:4 и пищу за столом его, и жилище рабов его, и чинность служащих ему и одежду их, и виночерпиев его и одежду их, и ход, которым он ходил в дом Господень,~--- и была она вне себя.
\vs 2Ch 9:5 И сказала царю: верно то, что я слышала в земле моей о делах твоих и о мудрости твоей,
\vs 2Ch 9:6 но я не верила словам о них, доколе не пришла и не увидела глазами своими. И вот, мне и вполовину не сказано о множестве мудрости твоей: ты превосходишь молву, какую я слышала.
\vs 2Ch 9:7 Блаженны люди твои, и блаженны сии слуги твои, всегда предстоящие пред тобою и слышащие мудрость твою!
\vs 2Ch 9:8 Да будет благословен Господь Бог твой, Который благоволил посадить тебя на престол Свой в царя у Господа Бога твоего. По любви Бога твоего к Израилю, чтоб утвердить его на веки, Он поставил тебя царем над ним~--- творить суд и правду.
\vs 2Ch 9:9 И подарила она царю сто двадцать талантов золота и великое множество благовоний и драгоценных камней; и не бывало таких благовоний, какие подарила царица Савская царю Соломону.
\vs 2Ch 9:10 И слуги Хирамовы и слуги Соломоновы, которые привезли золото из Офира, привезли и красного дерева и драгоценных камней.
\vs 2Ch 9:11 И сделал царь из этого красного дерева лестницы к дому Господню и к дому царскому, и цитры и псалтири для певцов. И не видано было подобного сему прежде в земле Иудейской.
\vs 2Ch 9:12 Царь же Соломон дал царице Савской все, чего она желала и чего она просила, кроме таких вещей, какие она привезла царю. И она отправилась обратно в землю свою, она и слуги ее.
\rsbpar\vs 2Ch 9:13 Весу в золоте, которое приходило к Соломону в один год, \bibemph{было} шестьсот шестьдесят шесть талантов золота.
\vs 2Ch 9:14 Сверх того, послы и купцы приносили, и все цари Аравийские и начальники областные приносили золото и серебро Соломону.
\vs 2Ch 9:15 И сделал царь Соломон двести больших щитов из кованого золота,~--- по шестисот \bibemph{сиклей} кованого золота пошло на каждый щит,~---
\vs 2Ch 9:16 и триста щитов меньших из кованого золота,~--- по триста \bibemph{сиклей} золота пошло на каждый щит; и поставил их царь в доме из Ливанского дерева.
\vs 2Ch 9:17 И сделал царь большой престол из слоновой кости и обложил его чистым золотом,
\vs 2Ch 9:18 и шесть ступеней к престолу и золотое подножие, к престолу приделанное, и локотники по обе стороны у места сидения, и двух львов, стоящих возле локотников,
\vs 2Ch 9:19 и \bibemph{еще} двенадцать львов, стоящих там на шести ступенях, по обе стороны. Не бывало такого [престола] ни в одном царстве.
\vs 2Ch 9:20 И все сосуды для питья у царя Соломона \bibemph{были} из золота, и все сосуды в доме из Ливанского дерева \bibemph{были} из золота отборного; серебро во дни Соломона вменялось ни во что,
\vs 2Ch 9:21 ибо корабли царя ходили в Фарсис с слугами Хирама, и в три года раз возвращались корабли из Фарсиса и привозили золото и серебро, слоновую кость и обезьян и павлинов.
\rsbpar\vs 2Ch 9:22 И превзошел царь Соломон всех царей земли богатством и мудростью.
\vs 2Ch 9:23 И все цари земли искали видеть Соломона, чтобы послушать мудрости его, которую вложил Бог в сердце его.
\vs 2Ch 9:24 И каждый из них подносил от себя в дар сосуды серебряные и сосуды золотые и одежды, оружие и благовония, коней и лошаков, из года в год.
\vs 2Ch 9:25 И было у Соломона четыре тысячи стойл для коней и колесниц и двенадцать тысяч всадников; и он разместил их в городах колесничных и при царе~--- в Иерусалиме;
\vs 2Ch 9:26 и господствовал он над всеми царями, от реки \bibemph{Евфрата} до земли Филистимской и до пределов Египта.
\vs 2Ch 9:27 И сделал царь [золото и] серебро в Иерусалиме равноценным \bibemph{простому} камню, а кедры, по их множеству, сделал равноценными сикоморам, которые на низких местах.
\vs 2Ch 9:28 Коней приводили Соломону из Египта и из всех земель.
\rsbpar\vs 2Ch 9:29 Прочие деяния Соломоновы, первые и последние, описаны в записях Нафана пророка и в пророчестве Ахии Силомлянина и в видениях прозорливца Иоиля о Иеровоаме, сыне Наватовом.
\vs 2Ch 9:30 Царствовал же Соломон в Иерусалиме над всем Израилем сорок лет.
\vs 2Ch 9:31 И почил Соломон с отцами своими, и похоронили его в городе Давида, отца его. И воцарился Ровоам, сын его, вместо него.
\vs 2Ch 10:1 И пошел Ровоам в Сихем, потому что в Сихем сошлись все Израильтяне, чтобы поставить его царем.
\vs 2Ch 10:2 Когда услышал \bibemph{о сем} Иеровоам, сын Наватов,~--- он находился в Египте, куда убежал от царя Соломона,~--- то возвратился Иеровоам из Египта.
\vs 2Ch 10:3 И послали и звали его; и пришел Иеровоам и весь Израиль, и говорили Ровоаму так:
\vs 2Ch 10:4 отец твой наложил на нас тяжкое иго; но ты облегчи жестокую работу отца твоего и тяжкое иго, которое он наложил на нас, и мы будем служить тебе.
\vs 2Ch 10:5 И сказал им \bibemph{Ровоам}: через три дня придите опять ко мне. И разошелся народ.
\vs 2Ch 10:6 И советовался царь Ровоам со старейшинами, которые предстояли пред лицем Соломона, отца его, при жизни его, и говорил: как вы посоветуете отвечать народу сему?
\vs 2Ch 10:7 Они сказали ему: если ты [ныне] будешь добр к народу сему и угодишь им и будешь говорить с ними ласково, то они будут тебе рабами на все дни.
\vs 2Ch 10:8 Но он оставил совет старейшин, который они давали ему, и стал советоваться с людьми молодыми, которые выросли вместе с ним, предстоящими пред лицем его;
\vs 2Ch 10:9 и сказал им: что вы посоветуете мне отвечать народу сему, говорившему мне так: облегчи иго, которое наложил на нас отец твой?
\vs 2Ch 10:10 И говорили ему молодые люди, выросшие вместе с ним, и сказали: так скажи народу, говорившему тебе: отец твой наложил на нас тяжкое иго, а ты облегчи нас,~--- так скажи им: мизинец мой толще чресл отца моего.
\vs 2Ch 10:11 Отец мой наложил на вас тяжкое иго, а я увеличу иго ваше; отец мой наказывал вас бичами, а я [буду бить вас] скорпионами.
\rsbpar\vs 2Ch 10:12 И пришел Иеровоам и весь народ к Ровоаму на третий день, как приказал царь, сказав: придите ко мне опять чрез три дня.
\vs 2Ch 10:13 Тогда царь отвечал им сурово, ибо оставил царь Ровоам совет старейшин, и говорил им по совету молодых людей так:
\vs 2Ch 10:14 отец мой наложил на вас тяжкое иго, а я увеличу его; отец мой наказывал вас бичами, а я [буду бить вас] скорпионами.
\vs 2Ch 10:15 И не послушал царь народа, потому что так устроено было от Бога, чтоб исполнить Господу слово Свое, которое изрек Он чрез Ахию Силомлянина Иеровоаму, сыну Наватову.
\rsbpar\vs 2Ch 10:16 Когда весь Израиль увидел, что не слушает его царь, то отвечал народ царю, говоря: какая нам часть в Давиде? Нет нам доли в сыне Иессеевом; по шатрам своим, Израиль! Теперь знай свой дом, Давид. И разошлись все Израильтяне по шатрам своим.
\vs 2Ch 10:17 Только над сынами Израилевыми, жившими в городах Иудиных, остался царем Ровоам.
\vs 2Ch 10:18 И послал царь Ровоам Адонирама, начальника над собиранием даней, и забросали его сыны Израилевы каменьями, и он умер. Царь же Ровоам поспешил сесть на колесницу, чтобы убежать в Иерусалим.
\vs 2Ch 10:19 Так отложились Израильтяне от дома Давидова до сего дня.
\vs 2Ch 11:1 И прибыл Ровоам в Иерусалим и созвал из дома Иудина и Вениаминова сто восемьдесят тысяч отборных воинов, чтобы воевать с Израилем и возвратить царство Ровоаму.
\vs 2Ch 11:2 И было слово Господне к Самею, человеку Божию, и сказано:
\vs 2Ch 11:3 скажи Ровоаму, сыну Соломонову, царю Иудейскому, и всему Израилю в \bibemph{колене} Иудином и Вениаминовом:
\vs 2Ch 11:4 так говорит Господь: не ходите и не начинайте войн\acc{ы} с братьями вашими; возвратитесь каждый в дом свой, ибо Мною сделано это. Они послушались слов Господних и возвратились из похода против Иеровоама.
\rsbpar\vs 2Ch 11:5 Ровоам жил в Иерусалиме; он обнес города в Иудее стенами.
\vs 2Ch 11:6 Он укрепил Вифлеем и Ефам, и Фекою,
\vs 2Ch 11:7 и Вефцур, и Сохо, и Одоллам,
\vs 2Ch 11:8 и Геф, и Марешу, и Зиф,
\vs 2Ch 11:9 и Адораим, и Лахис, и Азеку,
\vs 2Ch 11:10 и Цору, и Аиалон, и Хеврон, находившиеся в колене Иудином и Вениаминовом.
\vs 2Ch 11:11 И утвердил он крепости сии, и устроил в них начальников и хранилища для хлеба и деревянного масла и вина.
\vs 2Ch 11:12 И \bibemph{дал} в каждый город щиты и копья и утвердил их весьма сильно. И оставались за ним Иуда и Вениамин.
\rsbpar\vs 2Ch 11:13 И священники и левиты, какие \bibemph{были} по всей земле Израильской, собрались к нему из всех пределов,
\vs 2Ch 11:14 ибо оставили левиты свои городские предместья и свои владения и пришли в Иудею и в Иерусалим, так как отставил их Иеровоам и сыновья его от священства Господня
\vs 2Ch 11:15 и поставил у себя жрецов к высотам, и к козлам, и к тельцам, которых он сделал.
\vs 2Ch 11:16 А за ними и из всех колен Израилевых расположившие сердце свое, чтобы взыскать Господа Бога Израилева, приходили в Иерусалим, дабы приносить жертвы Господу Богу отцов своих.
\vs 2Ch 11:17 И укрепили они царство Иудино и поддерживали Ровоама, сына Соломонова, три года, потому что ходили путем Давида и Соломона в сии три года.
\rsbpar\vs 2Ch 11:18 И взял себе Ровоам в жену Махалафу, дочь Иеромофа, сына Давидова, и Авихаиль, дочь Елиава, сына Иессеева,
\vs 2Ch 11:19 и она родила ему сыновей: Иеуса и Шемарию и Загама.
\vs 2Ch 11:20 После нее он взял Мааху, дочь Авессалома, и она родила ему Авию и Аттая, и Зизу и Шеломифа.
\vs 2Ch 11:21 И любил Ровоам Мааху, дочь Авессалома, более всех жен и наложниц своих, ибо он имел восемнадцать жен и шестьдесят наложниц и родил двадцать восемь сыновей и шестьдесят дочерей.
\vs 2Ch 11:22 И поставил Ровоам Авию, сына Маахи, главою [и] князем над братьями его, потому что \bibemph{хотел} воцарить его.
\vs 2Ch 11:23 И действовал благоразумно, и разослал всех сыновей своих по всем землям Иуды и Вениамина во все укрепленные города, и дал им содержание большое и приискал много жен.
\vs 2Ch 12:1 Когда царство Ровоама утвердилось, и он сделался силен, тогда он оставил закон Господень, и весь Израиль с ним.
\rsbpar\vs 2Ch 12:2 На пятом году царствования Ровоама, Сусаким, царь Египетский, пошел на Иерусалим,~--- потому что они отступили от Господа,~---
\vs 2Ch 12:3 с тысячью и двумя стами колесниц и шестьюдесятью тысячами всадников; и не было числа народу, который пришел с ним из Египта, Ливиянам, Сукхитам и Ефиоплянам;
\vs 2Ch 12:4 и взял укрепленные города в Иудее и пришел к Иерусалиму.
\vs 2Ch 12:5 Тогда Самей пророк пришел к Ровоаму и князьям Иудеи, которые собрались в Иерусалим, \bibemph{спасаясь} от Сусакима, и сказал им: так говорит Господь: вы оставили Меня, за то и Я оставляю вас в руки Сусакиму.
\vs 2Ch 12:6 И смирились князья Израилевы и царь и сказали: праведен Господь!
\vs 2Ch 12:7 Когда увидел Господь, что они смирились, тогда было слово Господне к Самею, и сказано: они смирились; не истреблю их и вскоре дам им избавление, и не прольется гнев Мой на Иерусалим рукою Сусакима;
\vs 2Ch 12:8 однако же они будут слугами его, чтобы знали, каково служить Мне и служить царствам земным.
\vs 2Ch 12:9 И пришел Сусаким, царь Египетский, в Иерусалим и взял сокровища дома Господня и сокровища дома царского; всё взял он, взял и щиты золотые, которые сделал Соломон.
\vs 2Ch 12:10 И сделал царь Ровоам, вместо их, щиты медные, и отдал их на руки начальникам телохранителей, охранявших вход дома царского.
\vs 2Ch 12:11 Когда выходил царь в дом Господень, приходили телохранители и несли их, и потом опять относили их в палату телохранителей.
\vs 2Ch 12:12 И когда он смирился, тогда отвратился от него гнев Господа и не погубил его до конца; притом и в Иудее было нечто доброе.
\rsbpar\vs 2Ch 12:13 И утвердился царь Ровоам в Иерусалиме и царствовал. Сорок один год было Ровоаму, когда он воцарился, и семнадцать лет царствовал в Иерусалиме, в городе, который из всех колен Израилевых избрал Господь, чтобы там пребывало имя Его. Имя матери его Наама, Аммонитянка.
\vs 2Ch 12:14 И делал он зло, потому что не расположил сердца своего к тому, чтобы взыскать Господа.
\rsbpar\vs 2Ch 12:15 Деяния Ровоамовы, первые и последние, описаны в записях Самея пророка и Адды прозорливца при родословиях. И были войны у Ровоама с Иеровоамом во все дни.
\vs 2Ch 12:16 И почил Ровоам с отцами своими и погребен в городе Давидовом. И воцарился Авия, сын его, вместо него.
\vs 2Ch 13:1 В восемнадцатый год царствования Иеровоама воцарился Авия над Иудою.
\vs 2Ch 13:2 Три года он царствовал в Иерусалиме; имя матери его Михаия, дочь Уриилова, из Гивы. И была война у Авии с Иеровоамом.
\vs 2Ch 13:3 И вывел Авия на войну войско, состоявшее из людей храбрых, из четырехсот тысяч человек отборных; а Иеровоам выступил против него на войну с восемью стами тысяч человек, \bibemph{также} отборных, храбрых.
\vs 2Ch 13:4 И стал Авия на вершине горы Цемараимской, одной из гор Ефремовых, и говорил: послушайте меня, Иеровоам и все Израильтяне!
\vs 2Ch 13:5 Не знаете ли вы, что Господь Бог Израилев дал царство Давиду над Израилем навек, ему и сыновьям его, по завету соли [вечному]?
\vs 2Ch 13:6 Но восстал Иеровоам, сын Наватов, раб Соломона, сына Давидова, и возмутился против господина своего.
\vs 2Ch 13:7 И собрались вокруг него люди пустые, люди развращенные, и укрепились против Ровоама, сына Соломонова; Ровоам же был молод и слаб сердцем и не устоял против них.
\vs 2Ch 13:8 И ныне вы думаете устоять против царства Господня в руке сынов Давидовых, \bibemph{потому что} вас великое множество, и у вас золотые тельцы, которых Иеровоам сделал вам богами.
\vs 2Ch 13:9 Не вы ли изгнали священников Господних, сынов Аарона, и левитов, и поставили у себя священников, какие у народов \bibemph{других} земель? Всякий, кто приходит для посвящения своего с тельцом и с семью овнами, делается \bibemph{у вас} священником лжебогов.
\vs 2Ch 13:10 А у нас~--- Господь Бог наш; мы не оставляли Его, и Господу служат священники, сыны Аароновы, и левиты при \bibemph{своем} деле.
\vs 2Ch 13:11 И сожигают они Господу всесожжения каждое утро и каждый вечер, и благовонное курение, и полагают рядами хлебы на столе чистом, и \bibemph{зажигают} золотой светильник и лампады его, чтобы горели каждый вечер, потому что мы соблюдаем установление Господа Бога нашего, а вы оставили Его.
\vs 2Ch 13:12 И вот, у нас во главе Бог, и священники Его, и трубы громогласные, чтобы греметь против вас. Сыны Израилевы! не воюйте с Господом Богом отцов ваших, ибо не получите успеха.
\rsbpar\vs 2Ch 13:13 Между тем Иеровоам послал отряд в засаду с тыла им, так что \bibemph{сам он} был впереди Иудеев, а засада позади их.
\vs 2Ch 13:14 И оглянулись Иудеи, и вот, им битва спереди и сзади; и возопили они к Господу, а священники затрубили трубами.
\vs 2Ch 13:15 И воскликнули Иудеи. И когда воскликнули Иудеи, Бог поразил Иеровоама и всех Израильтян пред лицем Авии и Иуды.
\vs 2Ch 13:16 И побежали сыны Израилевы от Иудеев, и предал их Бог в руки им.
\vs 2Ch 13:17 И произвели у них Авия и народ его поражение сильное; и пало убитых у Израиля пятьсот тысяч человек отборных.
\vs 2Ch 13:18 И смирились тогда сыны Израилевы, и были сильны сыны Иудины, потому что уповали на Господа Бога отцов своих.
\vs 2Ch 13:19 И преследовал Авия Иеровоама и взял у него города: Вефиль и зависящие от него города, и Иешану и зависящие от нее города, и Ефрон и зависящие от него города.
\vs 2Ch 13:20 И не входил уже в силу Иеровоам во дни Авии. И поразил его Господь, и он умер.
\vs 2Ch 13:21 Авия же усилился; и взял себе четырнадцать жен и родил двадцать два сына и шестнадцать дочерей.
\rsbpar\vs 2Ch 13:22 Прочие деяния Авии и его поступки и слова описаны в сказании пророка Адды.
\vs 2Ch 14:1 И почил Авия с отцами своими, и похоронили его в городе Давидовом. И воцарился Аса, сын его, вместо него. Во дни его покоилась земля десять лет.
\vs 2Ch 14:2 И делал Аса доброе и угодное в очах Господа Бога своего:
\vs 2Ch 14:3 и отверг он жертвенники \bibemph{богов} чужих и высоты, и разбил статуи, и вырубил \bibemph{посвященные} дерева;
\vs 2Ch 14:4 и повелел Иудеям взыскать Господа Бога отцов своих, и исполнять закон [Его] и заповеди;
\vs 2Ch 14:5 и отменил он во всех городах Иудиных высоты и статуи солнца. И спокойно было при нем царство.
\vs 2Ch 14:6 И построил он укрепленные города в Иудее, ибо спокойна была земля, и не было у него войны в те годы, так как Господь дал покой ему.
\vs 2Ch 14:7 И сказал он Иудеям: построим города сии и обнесем их стенами с башнями, с воротами и запорами. Земля еще наша, потому что мы взыскали Господа Бога нашего: мы взыскали Его,~--- и Он дал нам покой со всех сторон. И стали строить, и имели успех.
\vs 2Ch 14:8 И было у Асы военной силы: вооруженных щитом и копьем из \bibemph{колена} Иудина триста тысяч, и из \bibemph{колена} Вениаминова вооруженных щитом и стрелявших из лука двести восемьдесят тысяч, людей храбрых.
\rsbpar\vs 2Ch 14:9 И вышел на них Зарай Ефиоплянин с войском в тысячу тысяч и с тремя стами колесниц и дошел до Мареши.
\vs 2Ch 14:10 И выступил Аса против него, и построились к сражению на долине Цефата у Мареши.
\vs 2Ch 14:11 И воззвал Аса к Господу Богу своему, и сказал: Господи! не в Твоей ли силе помочь сильному или бессильному? помоги же нам, Господи Боже наш: ибо мы на Тебя уповаем и во имя Твое вышли мы против множества сего. Господи! Ты Бог наш: да не превозможет Тебя человек.
\vs 2Ch 14:12 И поразил Господь Ефиоплян пред лицем Асы и пред лицем Иуды, и побежали Ефиопляне.
\vs 2Ch 14:13 И преследовал их Аса и народ, бывший с ним, до Герара, и пали Ефиопляне, так что у них никого \bibemph{не осталось} в живых, потому что они поражены были пред Господом и пред воинством Его. И набрали добычи великое множество.
\vs 2Ch 14:14 И разрушили все города вокруг Герара, потому что напал на них ужас от Господа; и разграбили все города и вынесли из них весьма много добычи.
\vs 2Ch 14:15 Также и пастушеские шалаши разорили и угнали множество стад мелкого скота и верблюдов и возвратились в Иерусалим.
\vs 2Ch 15:1 Тогда на Азарию, сына Одедова, сошел Дух Божий,
\vs 2Ch 15:2 и вышел он навстречу Асе и сказал ему: послушайте меня, Аса и весь Иуда и Вениамин: Господь с вами, когда вы с Ним; и если будете искать Его, Он будет найден вами; если же оставите Его, Он оставит вас.
\vs 2Ch 15:3 Многие дни Израиль \bibemph{будет} без Бога истинного, и без священника учащего, и без закона;
\rsbpar\vs 2Ch 15:4 но когда он обратится в тесноте своей к Господу Богу Израилеву и взыщет Его, Он даст им найти Себя.
\vs 2Ch 15:5 В те времена не будет мира ни выходящему, ни входящему, ибо великие волнения будут у всех жителей земель;
\vs 2Ch 15:6 народ будет сражаться с народом, и город с городом, потому что Бог приведет их в смятение всякими бедствиями.
\vs 2Ch 15:7 Но вы укрепитесь, и пусть не ослабевают руки ваши, потому что есть возмездие за дела ваши.
\rsbpar\vs 2Ch 15:8 Когда услышал Аса слова сии и пророчество [Азарии], \bibemph{сына} Одеда пророка, то ободрился и изверг мерзости \bibemph{языческие} из всей земли Иудиной и Вениаминовой и из городов, которые он взял на горе Ефремовой, и обновил жертвенник Господень, который пред притвором Господним.
\vs 2Ch 15:9 И собрал всего Иуду и Вениамина и живущих с ними переселенцев от Ефрема и Манассии и Симеона; ибо многие от Израиля перешли к нему, когда увидели, что Господь, Бог его, с ним.
\vs 2Ch 15:10 И собрались в Иерусалим в третий месяц, в пятнадцатый год царствования Асы;
\vs 2Ch 15:11 и принесли в день тот жертву Господу из добычи, которую привели, из крупного скота семьсот и из мелкого семь тысяч;
\vs 2Ch 15:12 и вступили в завет, чтобы взыскать Господа Бога отцов своих от всего сердца своего и от всей души своей;
\vs 2Ch 15:13 а всякий, кто не станет искать Господа Бога Израилева, должен умереть, малый ли он или большой, мужчина ли или женщина.
\vs 2Ch 15:14 И клялись Господу громогласно и с восклицанием и при \bibemph{звуке} труб и рогов.
\vs 2Ch 15:15 И радовались все Иудеи сей клятве, потому что от всего сердца своего клялись и со всем усердием взыскали Его, и Он дал им найти Себя. И дал им Господь покой со всех сторон.
\vs 2Ch 15:16 И Мааху, мать свою, царь Аса лишил царского достоинства за то, что она сделала истукан для дубравы. И ниспроверг Аса истукан ее, и изрубил в куски, и сжег на долине Кедрона.
\vs 2Ch 15:17 Хотя высоты не были отменены у Израиля, но сердце Асы было вполне предано \bibemph{Господу} во все дни его.
\vs 2Ch 15:18 И внес он посвященное отцом его и свое посвящение в дом Божий, серебро и золото и сосуды.
\vs 2Ch 15:19 И не было войны до тридцать пятого года царствования Асы.
\vs 2Ch 16:1 В тридцать шестой год царствования Асы, пошел Вааса, царь Израильский, на Иудею и начал строить Раму, чтобы не позволить \bibemph{никому} ни уходить от Асы, царя Иудейского, ни приходить \bibemph{к нему}.
\vs 2Ch 16:2 И вынес Аса серебро и золото из сокровищниц дома Господня и дома царского и послал к Венададу, царю Сирийскому, жившему в Дамаске, говоря:
\vs 2Ch 16:3 союз да будет между мною и тобою, как был между отцом моим и отцом твоим; вот, я посылаю тебе серебра и золота: пойди, расторгни союз твой с Ваасою, царем Израильским, чтоб он отступил от меня.
\vs 2Ch 16:4 И послушался Венадад царя Асы и послал военачальников, которые \bibemph{были} у него, против городов Израильских, и они опустошили Ийон и Дан и Авелмаим и все запасы в городах Неффалимовых.
\vs 2Ch 16:5 И когда услышал \bibemph{о сем} Вааса, то перестал строить Раму и прекратил работу свою.
\vs 2Ch 16:6 Аса же царь собрал всех Иудеев, и они вывезли \bibemph{из} Рамы камни и дерева, которые употреблял Вааса для строения,~--- и выстроил из них Геву и Мицфу.
\rsbpar\vs 2Ch 16:7 В то время пришел Ананий прозорливец к Асе, царю Иудейскому, и сказал ему: так как ты понадеялся на царя Сирийского и не уповал на Господа Бога твоего, потому и спаслось войско царя Сирийского от руки твоей.
\vs 2Ch 16:8 Не были ли Ефиопляне и Ливияне с силою большею и с колесницами и всадниками весьма многочисленными? Но как ты уповал на Господа, то Он предал их в руку твою,
\vs 2Ch 16:9 ибо очи Господа обозревают всю землю, чтобы поддерживать тех, \bibemph{чье} сердце вполне предано Ему. Безрассудно ты поступил теперь. За то отныне будут у тебя войны.
\vs 2Ch 16:10 И разгневался Аса на прозорливца, и заключил его в темницу, так как за это был в раздражении на него; притеснял Аса и \bibemph{некоторых} из народа в то время.
\rsbpar\vs 2Ch 16:11 И вот, деяния Асы, первые и последние, описаны в книге царей Иудейских и Израильских.
\vs 2Ch 16:12 И сделался Аса болен ногами на тридцать девятом году царствования своего, и болезнь его поднялась до верхних частей тела; но он в болезни своей взыскал не Господа, а врачей.
\vs 2Ch 16:13 И почил Аса с отцами своими, и умер на сорок первом году царствования своего.
\vs 2Ch 16:14 И похоронили его в гробнице, которую он устроил для себя в городе Давидовом; и положили его на одре, который наполнили благовониями и разными искусственными мастями, и сожгли их для него великое множество.
\vs 2Ch 17:1 И воцарился Иосафат, сын его, вместо него; и укрепился он против Израильтян.
\vs 2Ch 17:2 И поставил он войско во все укрепленные города Иудеи и расставил охранное войско по земле Иудейской и по городам Ефремовым, которыми овладел Аса, отец его.
\vs 2Ch 17:3 И был Господь с Иосафатом, потому что он ходил первыми путями Давида, отца своего, и не взыскал Ваалов,
\vs 2Ch 17:4 но взыскал он Бога отца своего и поступал по заповедям Его, а не по деяниям Израильтян.
\vs 2Ch 17:5 И утвердил Господь царство в руке его, и давали все Иудеи дары Иосафату, и было у него много богатства и славы.
\vs 2Ch 17:6 И возвысилось сердце его на путях Господних; притом и высоты отменил он и дубравы в Иудее.
\rsbpar\vs 2Ch 17:7 И в третий год царствования своего он послал князей своих Бенхаила и Овадию, и Захарию и Нафанаила и Михея, чтоб учили по городам Иудиным народ,
\vs 2Ch 17:8 и с ними левитов: Шемаию и Нефанию, и Зевадию и Азаила, и Шемирамофа и Ионафана, и Адонию и Товию и Тов-Адонию, и с ними Елишаму и Иорама, священников.
\vs 2Ch 17:9 И они учили в Иудее, имея с собою книгу закона Господня; и обходили все города Иудеи и учили народ.
\vs 2Ch 17:10 И был страх Господень на всех царствах земель, которые вокруг Иудеи, и не воевали с Иосафатом.
\vs 2Ch 17:11 А от Филистимлян приносили Иосафату дары и в дань серебро; также Аравитяне пригоняли к нему мелкий скот: овнов семь тысяч семьсот и козлов семь тысяч семьсот.
\rsbpar\vs 2Ch 17:12 И возвышался Иосафат все более и более и построил в Иудее крепости и города для запасов.
\vs 2Ch 17:13 Много было у него запасов в городах Иудейских, а в Иерусалиме людей военных, храбрых.
\vs 2Ch 17:14 И вот список их по поколениям их: у Иуды начальники тысяч: Адна начальник, и у него отличных воинов триста тысяч;
\vs 2Ch 17:15 за ним Иоханан начальник, и у него двести восемьдесят тысяч;
\vs 2Ch 17:16 за ним Амасия, сын Зихри, посвятивший себя Господу, и у него двести тысяч воинов отличных.
\vs 2Ch 17:17 У Вениамина: отличный воин Елиада, и у него вооруженных луком и щитом двести тысяч;
\vs 2Ch 17:18 за ним Иегозавад, и у него сто восемьдесят тысяч вооруженных воинов.
\vs 2Ch 17:19 Вот служившие царю, сверх тех, которых расставил царь в укрепленных городах по всей Иудее.
\vs 2Ch 18:1 И было у Иосафата много богатства и славы; и породнился он с Ахавом.
\vs 2Ch 18:2 И пошел чрез несколько лет к Ахаву в Самарию; и заколол для него Ахав множество скота мелкого и крупного, и для людей, бывших с ним, и склонял его идти на Рамоф Галаадский.
\vs 2Ch 18:3 И говорил Ахав, царь Израильский, Иосафату, царю Иудейскому: пойдешь ли со мною в Рамоф Галаадский? Тот сказал ему: как ты, так и я, как твой народ, так и мой народ: \bibemph{иду} с тобою на войну!
\vs 2Ch 18:4 И сказал Иосафат царю Израильскому: вопроси сегодня, что скажет Господь.
\vs 2Ch 18:5 И собрал царь Израильский пророков четыреста человек и сказал им: идти ли нам на Рамоф Галаадский войною, или удержаться? Они сказали: иди, и Бог предаст \bibemph{его} в руку царя.
\vs 2Ch 18:6 И сказал Иосафат: нет ли здесь еще пророка Господня? спросим и у него.
\vs 2Ch 18:7 И сказал царь Израильский Иосафату: есть еще один человек, чрез которого можно вопросить Господа; но я не люблю его, потому что он не пророчествует обо мне доброго, а постоянно пророчествует худое; это Михей, сын Иемвлая. И сказал Иосафат: не говори так, царь.
\vs 2Ch 18:8 И позвал царь Израильский одного евнуха, и сказал: сходи поскорее за Михеем, сыном Иемвлая.
\vs 2Ch 18:9 Царь же Израильский и Иосафат, царь Иудейский, сидели каждый на своем престоле, одетые в \bibemph{царские} одежды; сидели на площади у ворот Самарии, и все пророки пророчествовали пред ними.
\vs 2Ch 18:10 И сделал себе Седекия, сын Хенааны, железные рога и сказал: так говорит Господь: сими избодешь Сириян до истребления их.
\vs 2Ch 18:11 И все пророки пророчествовали то же, говоря: иди на Рамоф Галаадский; будет успех тебе, и предаст \bibemph{его} Господь в руку царя.
\rsbpar\vs 2Ch 18:12 Посланный, который пошел позвать Михея, говорил ему: вот, пророки единогласно предрекают доброе царю; пусть бы и твое слово было такое же, как каждого из них: изреки и ты доброе.
\vs 2Ch 18:13 И сказал Михей: жив Господь,~--- что скажет мне Бог мой, то изреку я.
\vs 2Ch 18:14 И пришел он к царю, и сказал ему царь: Михей, идти ли нам войной на Рамоф Галаадский, или удержаться? И сказал тот: идите, будет вам успех, и они преданы будут в руки ваши.
\vs 2Ch 18:15 И сказал ему царь: сколько раз мне заклинать тебя, чтобы ты не говорил мне ничего, кроме истины, во имя Господне?
\vs 2Ch 18:16 Тогда \bibemph{Михей} сказал: я видел всех сынов Израиля, рассеянных по горам, как овец, у которых нет пастыря,~--- и сказал Господь: нет у них начальника, пусть возвратятся каждый в дом свой с миром.
\vs 2Ch 18:17 И сказал царь Израильский Иосафату: не говорил ли я тебе, что он не пророчествует о мне доброго, а только худое?
\vs 2Ch 18:18 И сказал \bibemph{Михей}: так выслушайте слово Господне: я видел Господа, сидящего на престоле Своем, и все воинство небесное стояло по правую и по левую руку Его.
\vs 2Ch 18:19 И сказал Господь: кто увлек бы Ахава, царя Израильского, чтобы он пошел и пал в Рамофе Галаадском? И один говорил так, другой говорил иначе.
\vs 2Ch 18:20 И выступил один дух, и стал пред лицем Господа, и сказал: я увлеку его. И сказал ему Господь: чем?
\vs 2Ch 18:21 Тот сказал: я выйду, и буду духом лжи в устах всех пророков его. И сказал Он: ты увлечешь его, и успеешь; пойди и сделай так.
\vs 2Ch 18:22 И теперь, вот попустил Господь духу лжи \bibemph{войти} в уста сих пророков твоих, но Господь изрек о тебе недоброе.
\vs 2Ch 18:23 И подошел Седекия, сын Хенааны, и ударил Михея по щеке, и сказал: по какой это дороге отошел от меня Дух Господень, чтобы говорить в тебе?
\vs 2Ch 18:24 И сказал Михей: вот, ты увидишь \bibemph{это} в тот день, когда будешь бегать из комнаты в комнату, чтобы укрыться.
\vs 2Ch 18:25 И сказал царь Израильский: возьмите Михея и отведите его к Амону градоначальнику и к Иоасу, сыну царя,
\vs 2Ch 18:26 и скажите: так говорит царь: посад\acc{и}те этого в темницу и кормите его хлебом и водою скудно, доколе я не возвращусь в мире.
\vs 2Ch 18:27 И сказал Михей: если ты возвратишься в мире, то не Господь говорил чрез меня. И сказал: слушайте \bibemph{это}, все люди!
\rsbpar\vs 2Ch 18:28 И пошел царь Израильский и Иосафат, царь Иудейский, к Рамофу Галаадскому.
\vs 2Ch 18:29 И сказал царь Израильский Иосафату: я переоденусь и вступлю в сражение, а ты надень свои \bibemph{царские} одежды. И переоделся царь Израильский, и вступили в сражение.
\vs 2Ch 18:30 И царь Сирийский повелел начальникам колесниц, бывших у него, сказав: не сражайтесь ни с малым, ни с великим, а только с одним царем Израильским.
\vs 2Ch 18:31 И когда увидели Иосафата начальники колесниц, то подумали: это царь Израильский,~--- и окружили его, чтобы сразиться с ним. Но Иосафат закричал, и Господь помог ему, и отвел их Бог от него.
\vs 2Ch 18:32 И когда увидели начальники колесниц, что \bibemph{это} не был царь Израильский, то поворотили от него.
\vs 2Ch 18:33 Между тем один человек случайно натянул лук свой, и ранил царя Израильского сквозь швы лат. И сказал он вознице: повороти назад, и вези меня от войска, ибо я ранен.
\vs 2Ch 18:34 Но сражение в тот день усилилось; и царь Израильский стоял на колеснице напротив Сириян до вечера и умер на закате солнца.
\vs 2Ch 19:1 И возвращался Иосафат, царь Иудейский, в мире в дом свой в Иерусалим.
\vs 2Ch 19:2 И выступил навстречу ему Ииуй, сын Анании, прозорливец, и сказал царю Иосафату: \bibemph{следовало} ли тебе помогать нечестивцу и любить ненавидящих Господа? За это на тебя гнев от лица Господня.
\vs 2Ch 19:3 Впрочем и доброе найдено в тебе, потому что ты истребил кумиры в земле [Иудейской] и расположил сердце свое к тому, чтобы взыскать Бога.
\rsbpar\vs 2Ch 19:4 И жил Иосафат в Иерусалиме. И опять стал он обходить народ \bibemph{свой} от Вирсавии до горы Ефремовой, и обращал их к Господу, Богу отцов их.
\vs 2Ch 19:5 И поставил судей на земле по всем укрепленным городам Иудеи в каждом городе,
\vs 2Ch 19:6 и сказал судьям: смотрите, что вы делаете, вы творите не суд человеческий, но суд Господа; и \bibemph{Он} с вами в деле суда.
\vs 2Ch 19:7 Итак да будет страх Господень на вас: действуйте осмотрительно, ибо нет у Господа Бога нашего неправды, ни лицеприятия, ни мздоимства.
\vs 2Ch 19:8 И в Иерусалиме приставил Иосафат \bibemph{некоторых} из левитов и священников и глав поколений у Израиля~--- к суду Господню и к тяжбам. И возвратились в Иерусалим.
\vs 2Ch 19:9 И дал им повеление, говоря: так действуйте в страхе Господнем, с верностью и с чистым сердцем:
\vs 2Ch 19:10 во всяком деле спорном, какое поступит к вам от братьев ваших, живущих в городах своих, о кровопролитии ли, или о законе, заповеди, уставах и обрядах, наставляйте их, чтобы они не провинились пред Господом, и не было бы гнева \bibemph{Его} на вас и на братьев ваших; так действуйте,~--- и вы не погрешите.
\vs 2Ch 19:11 И вот Амария первосвященник, над вами во всяком деле Господнем, а Зевадия, сын Исмаилов, князь дома Иудина, во всяком деле царя, и надзиратели левиты пред вами. Будьте тверды и действуйте, и будет Господь с добрым.
\vs 2Ch 20:1 После сего Моавитяне и Аммонитяне, а с ними некоторые из страны Маонитской, пошли войною на Иосафата.
\vs 2Ch 20:2 И пришли, и донесли Иосафату, говоря: идет на тебя множество великое из-за моря, от Сирии, и вот они в Хацацон-Фамаре, то есть в Енгедди.
\vs 2Ch 20:3 И убоялся Иосафат, и обратил лице свое взыскать Господа, и объявил пост по всей Иудее.
\vs 2Ch 20:4 И собрались Иудеи просить \bibemph{помощи} у Господа; из всех городов Иудиных пришли они умолять Господа.
\rsbpar\vs 2Ch 20:5 И стал Иосафат в собрании Иудеев и Иерусалимлян в доме Господнем, пред новым двором,
\vs 2Ch 20:6 и сказал: Господи Боже отцов наших! Не Ты ли Бог на небе? И Ты владычествуешь над всеми царствами народов, и в Твоей руке сила и крепость, и никто не устоит против Тебя!
\vs 2Ch 20:7 Не Ты ли, Боже наш, изгнал жителей земли сей пред лицем народа Твоего Израиля и отдал ее семени Авраама, друга Твоего, навек?
\vs 2Ch 20:8 И они поселились на ней и построили Тебе на ней святилище во имя Твое, говоря:
\vs 2Ch 20:9 если придет на нас бедствие: меч наказующий, или язва, или голод, то мы станем пред домом сим и пред лицем Твоим, ибо имя Твое в доме сем; и воззовем к Тебе в тесноте нашей, и Ты услышишь и спасешь.
\vs 2Ch 20:10 И ныне вот Аммонитяне и Моавитяне и \bibemph{обитатели} горы Сеира, чрез земли которых Ты не позволил пройти Израильтянам, когда они шли из земли Египетской, а потому они миновали их и не истребили их,~---
\vs 2Ch 20:11 вот они платят нам \bibemph{тем}, что пришли выгнать нас из наследственного владения Твоего, которое Ты отдал нам.
\vs 2Ch 20:12 Боже наш! Ты суди их. Ибо нет в нас силы против множества сего великого, пришедшего на нас, и мы не знаем, чт\acc{о} делать, но к Тебе очи наши!
\vs 2Ch 20:13 И все Иудеи стояли пред лицем Господним, и малые дети их, жены их и сыновья их.
\rsbpar\vs 2Ch 20:14 Тогда на Иозиила, сына Захарии, сына Ванеи, сына Иеиела, сына Матфании, левита из сынов Асафовых, сошел Дух Господень среди собрания
\vs 2Ch 20:15 и сказал он: слушайте, все Иудеи и жители Иерусалима и царь Иосафат! Так говорит Господь к вам: не бойтесь и не ужасайтесь множества сего великого, ибо не ваша война, а Божия.
\vs 2Ch 20:16 Завтра выступите против них: вот они всходят на возвышенность Циц, и вы найдете их на конце долины, пред пустынею Иеруилом.
\vs 2Ch 20:17 Не вам сражаться на сей раз; вы станьте, стойте и смотрите на спасение Господне, \bibemph{посылаемое} вам. Иуда и Иерусалим! не бойтесь и не ужасайтесь. Завтра выступите навстречу им, и Господь будет с вами.
\vs 2Ch 20:18 И преклонился Иосафат лицем до земли, и все Иудеи и жители Иерусалима пали пред Господом, чтобы поклониться Господу.
\vs 2Ch 20:19 И встали левиты из сынов Каафовых и из сынов Кореевых~--- хвалить Господа Бога Израилева, голосом весьма громким.
\rsbpar\vs 2Ch 20:20 И встали они рано утром, и выступили к пустыне Фекойской; и когда они выступили, стал Иосафат и сказал: послушайте меня, Иудеи и жители Иерусалима! Верьте Господу Богу вашему, и будете тверды; верьте пророкам Его, и будет успех вам.
\vs 2Ch 20:21 И совещался он с народом, и поставил певцов Господу, чтобы они в благолепии святыни, выступая впереди вооруженных, славословили и говорили: славьте Господа, ибо вовек милость Его!
\vs 2Ch 20:22 И в то время, \bibemph{как} они стали восклицать и славословить, Господь возбудил несогласие между Аммонитянами, Моавитянами и \bibemph{обитателями} горы Сеира, пришедшими на Иудею, и были они поражены:
\vs 2Ch 20:23 ибо восстали Аммонитяне и Моавитяне на обитателей горы Сеира, побивая и истребляя \bibemph{их}, а когда покончили с жителями Сеира, тогда стали истреблять друг друга.
\vs 2Ch 20:24 И когда Иудеи пришли на возвышенность к пустыне и взглянули на то многолюдство, и вот~--- трупы, лежащие на земле, и нет уцелевшего.
\vs 2Ch 20:25 И пришел Иосафат и народ его забирать добычу, и нашли у них во множестве и имущество, и одежды, и драгоценные вещи, и набрали себе столько, что не \bibemph{могли} нести. И три дня они забирали добычу; так велика \bibemph{была} она!
\rsbpar\vs 2Ch 20:26 А в четвертый день собрались на долину благословения, так как там они благословили Господа. Посему и называют то место долиною благословения до сего дня.
\vs 2Ch 20:27 И пошли назад все Иудеи и Иерусалимляне и Иосафат во главе их, чтобы возвратиться в Иерусалим с веселием, потому что дал им Господь торжество над врагами их.
\vs 2Ch 20:28 И пришли в Иерусалим с псалтирями, и цитрами, и трубами, к дому Господню.
\vs 2Ch 20:29 И был страх Божий на всех царствах земных, когда они услышали, что \bibemph{Сам} Господь воевал против врагов Израиля.
\vs 2Ch 20:30 И спокойно стало царство Иосафатово, и дал ему Бог его покой со всех сторон.
\rsbpar\vs 2Ch 20:31 Так царствовал Иосафат над Иудеею: тридцати пяти лет он \bibemph{был}, когда воцарился, и двадцать пять лет царствовал в Иерусалиме. Имя матери его Азува, дочь Салаила.
\vs 2Ch 20:32 И ходил он путем отца своего Асы и не уклонился от него, делая угодное в очах Господних.
\vs 2Ch 20:33 Только высоты не были отменены, и народ еще не обратил твердо сердца своего к Богу отцов своих.
\rsbpar\vs 2Ch 20:34 Прочие деяния Иосафата, первые и последние, описаны в записях Ииуя, сына Ананиева, которые внесены в книгу царей Израилевых.
\rsbpar\vs 2Ch 20:35 Но после того вступил Иосафат, царь Иудейский в общение с Охозиею, царем Израильским, который поступал беззаконно,
\vs 2Ch 20:36 и соединился с ним, чтобы построить корабли для отправления в Фарсис; и построили они корабли в Ецион-Гавере.
\vs 2Ch 20:37 И изрек \bibemph{тогда} Елиезер, сын Додавы из Мареши, пророчество на Иосафата, говоря: так как ты вступил в общение с Охозиею, то разрушил Господь дело твое.~--- И разбились корабли, и не могли идти в Фарсис.
\vs 2Ch 21:1 И почил Иосафат с отцами своими, и похоронен с отцами своими в городе Давидовом. И воцарился Иорам, сын его, вместо него.
\vs 2Ch 21:2 И у него \bibemph{были} братья, сыновья Иосафата: Азария и Иехиил, и Захария и Азария, и Михаил и Сафатия: все сии сыновья Иосафата, царя Израилева.
\vs 2Ch 21:3 И дал им отец их большие подарки серебром и золотом и драгоценностями, вместе с укрепленными городами в Иудее; царство же отдал Иораму, потому что он первенец.
\vs 2Ch 21:4 И вступил Иорам на царство отца своего и утвердился, и умертвил всех братьев своих мечом и также \bibemph{некоторых} из князей Израилевых.
\rsbpar\vs 2Ch 21:5 Тридцати двух лет \bibemph{был} Иорам, когда воцарился, и восемь лет царствовал в Иерусалиме;
\vs 2Ch 21:6 и ходил он путем царей Израильских, как поступал дом Ахавов, потому что дочь Ахава была женою его,~--- и делал он неугодное в очах Господних.
\vs 2Ch 21:7 Однако же не хотел Господь погубить дома Давидова ради завета, который заключил с Давидом, и потому что обещал дать ему светильник и сыновьям его на все времена.
\rsbpar\vs 2Ch 21:8 Во дни его вышел Едом из-под власти Иуды, и поставили над собою царя.
\vs 2Ch 21:9 И пошел Иорам с военачальниками своими, и все колесницы с ним; и встав ночью, поразил Идумеян, которые окружили его, и начальствующих над колесницами [и побежал народ в жилища свои].
\vs 2Ch 21:10 Однако вышел Едом из-под власти Иуды до сего дня. В то же время вышла и Ливна из-под власти его, потому что он оставил Господа Бога отцов своих.
\vs 2Ch 21:11 Также высоты устроил он на горах Иудейских, и ввел в блужение жителей Иерусалима и соблазнил Иудею.
\rsbpar\vs 2Ch 21:12 И пришло к нему письмо от Илии пророка, в котором было сказано: так говорит Господь Бог Давида, отца твоего: за то, что ты не пошел путями Иосафата, отца твоего, и путями Асы, царя Иудейского,
\vs 2Ch 21:13 а пошел путем царей Израильских и ввел в блужение Иудею и жителей Иерусалима, как вводил в блужение дом Ахавов, еще же и братьев твоих, дом отца твоего, которые лучше тебя, ты умертвил,
\vs 2Ch 21:14 \bibemph{за то}, вот Господь поразит поражением великим народ твой и сыновей твоих, и жен твоих, и все имущество твое,
\vs 2Ch 21:15 тебя же \bibemph{самого}~--- болезнью сильною, болезнью внутренностей твоих до того, что будут выпадать внутренности твои от болезни со дня на день.
\rsbpar\vs 2Ch 21:16 И возбудил Господь против Иорама дух Филистимлян и Аравитян, сопредельных Ефиоплянам;
\vs 2Ch 21:17 и они пошли на Иудею и ворвались в нее, и захватили все имущество, находившееся в доме царя, также и сыновей его и жен его; и не осталось у него сына, кроме Охозии, меньшего из сыновей его.
\vs 2Ch 21:18 А после всего этого поразил Господь внутренности его болезнью неизлечимою.
\vs 2Ch 21:19 Так было со дня на день, а к концу второго года выпали внутренности его от болезни его, и он умер в жестоких страданиях; и не сожег для него народ его \bibemph{благовоний}, как делал то для отцов его.
\rsbpar\vs 2Ch 21:20 Тридцати двух \bibemph{лет} был он, когда воцарился, и восемь лет царствовал в Иерусалиме, и отошел неоплаканный, и похоронили его в городе Давидовом, но не в царских гробницах.
\vs 2Ch 22:1 И поставили царем жители Иерусалима Охозию, меньшего сына его, вместо него, так как всех старших избило полчище, приходившее с Аравитянами к стану,~--- и воцарился Охозия, сын Иорама, царя Иудейского.
\rsbpar\vs 2Ch 22:2 Двадцати двух лет \bibemph{был} Охозия, когда воцарился, и один год царствовал в Иерусалиме; имя матери его Гофолия, дочь Амврия.
\vs 2Ch 22:3 Он также ходил путями дома Ахавова, потому что мать его была советницею ему на беззаконные дела.
\vs 2Ch 22:4 И делал он неугодное в очах Господних, подобно дому Ахавову, потому что он был ему советником, по смерти отца его, на погибель ему.
\vs 2Ch 22:5 Также следуя их совету, он пошел с Иорамом, сыном Ахавовым, царем Израильским, на войну против Азаила, царя Сирийского, в Рамоф Галаадский. И ранили Сирияне Иорама,
\vs 2Ch 22:6 и возвратился он в Изреель лечиться от ран, которые причинили ему в Раме, когда он воевал с Азаилом, царем Сирийским. И Охозия, сын Иорама, царь Иудейский, пришел посетить Иорама, сына Ахавова, в Изреель, потому что тот был болен.
\vs 2Ch 22:7 И от Бога было это на погибель Охозии, что он пришел к Иораму: ибо, по приходе своем, он вышел с Иорамом против Ииуя, сына Намессиева, которого помазал Господь на истребление дома Ахавова.
\vs 2Ch 22:8 Когда совершал Ииуй суд над домом Ахава, тогда он нашел князей Иудейских и сыновей братьев Охозии, служивших Охозии, и умертвил их.
\vs 2Ch 22:9 И [велел] он искать Охозию, и взяли его, когда он скрывался в Самарии, и привели его к Ииую, и умертвили его, и похоронили его, ибо говорили: он сын Иосафата, который взыскал Господа от всего сердца своего. И не \bibemph{осталось} в доме Охозии, \bibemph{кто} мог бы царствовать.
\vs 2Ch 22:10 Ибо Гофолия, мать Охозии, увидев, что умер сын ее, встала и истребила все царское племя дома Иудина.
\vs 2Ch 22:11 Но Иосавеф, дочь царя, взяла Иоаса, сына Охозии, и похитила его из среды царских сыновей умерщвляемых, и поместила его и кормилицу его в спальной комнате; и таким образом Иосавеф, дочь царя Иорама, жена Иодая священника, сестра Охозии, скрыла Иоаса от Гофолии, и она не умертвила его.
\vs 2Ch 22:12 И был он у них в доме Божием скрываем шесть лет; Гофолия же царствовала над землею.
\vs 2Ch 23:1 Но в седьмой год ободрился Иодай и принял в союз с собою начальников сотен: Азарию, сына Иерохамова, и Исмаила, сына Иегохананова, и Азарию, сына Оведова, и Маасею, сына Адаии, и Елишафата, сына Зихри.
\vs 2Ch 23:2 И они прошли по Иудее и собрали левитов из всех городов Иудеи и глав поколений Израилевых, и пришли в Иерусалим.
\vs 2Ch 23:3 И заключило все собрание союз в доме Божием с царем. И сказал им \bibemph{Иодай}: вот сын царя должен быть царем, как изрек Господь о сыновьях Давидовых.
\vs 2Ch 23:4 Вот что вы сделайте: треть вас, приходящих в субботу, из священников и левитов, \bibemph{будет} привратниками у порогов,
\vs 2Ch 23:5 и треть при доме царском, и треть у ворот Иесод, а весь народ на дворах дома Господня.
\vs 2Ch 23:6 И \bibemph{никто} пусть не входит в дом Господень, кроме священников и служащих из левитов. Они могут войти, потому что освящены; весь же народ пусть стоит на страже Господней.
\vs 2Ch 23:7 И пусть левиты окружат царя со всех сторон, всякий с оружием своим в руке своей, и кто будет входить в храм, да будет умерщвлен. И будьте вы при царе, когда он будет входить и выходить.
\vs 2Ch 23:8 И сделали левиты и все Иудеи, что приказал Иодай священник; и взяли каждый людей своих, приходящих в субботу с отходящими в субботу, потому что не отпустил священник Иодай \bibemph{сменившихся} черед.
\vs 2Ch 23:9 И раздал Иодай священник начальникам сотен копья и малые и большие щиты царя Давида, которые \bibemph{были} в доме Божием;
\vs 2Ch 23:10 и поставил весь народ, каждого с оружием его в руке его, от правой стороны храма до левой стороны храма, у жертвенника и у дома, вокруг царя.
\vs 2Ch 23:11 И вывели сына царя, и возложили на него венец и украшения, и поставили его царем; и помазали его Иодай и сыновья его и сказали: да живет царь!
\vs 2Ch 23:12 И услышала Гофолия голос народа, бегущего и провозглашающего о царе, и вышла к народу в дом Господень,
\vs 2Ch 23:13 и увидела: и вот царь стоит на возвышении своем при входе, и князья и трубы подле царя, и весь народ земли веселится, и трубят трубами, и певцы с орудиями музыкальными и искусные в славословии. И разодрала Гофолия одежды свои и закричала: заговор! заговор!
\vs 2Ch 23:14 И вызвал Иодай священник начальников сотен, начальствующих над войском, и сказал им: выведите ее вон [из храма], и кто последует за нею, да будет умерщвлен мечом. Потому что священник сказал: не умертвите ее в доме Господнем.
\vs 2Ch 23:15 И дали ей место, и когда она пришла ко входу конских ворот царского дома, там умертвили ее.
\rsbpar\vs 2Ch 23:16 И заключил Иодай завет между собою и между всем народом и царем, чтобы быть \bibemph{им} народом Господним.
\vs 2Ch 23:17 И пошел весь народ в капище Ваала, и разрушили его, и жертвенники его и истуканов его сокрушили; и Матфана, жреца Ваалова, умертвили пред жертвенниками.
\vs 2Ch 23:18 И поручил Иодай дела дома Господня священникам и левитам, [и восстановил дневные череды священников и левитов,] как распределил Давид в доме Господнем, для возношения всесожжений Господу, как написано в законе Моисеевом, с радостью и пением, по уставу Давидову.
\vs 2Ch 23:19 И поставил он привратников у ворот дома Господня, чтобы не \bibemph{мог} входить нечистый почему-нибудь.
\vs 2Ch 23:20 И взял начальников сотен, и вельмож, и начальствующих в народе, и весь народ земли, и проводил царя из дома Господня, и прошли чрез верхние ворота в дом царский, и посадили царя на царский престол.
\vs 2Ch 23:21 И веселился весь народ земли, и город успокоился. А Гофолию умертвили мечом.
\vs 2Ch 24:1 Семи лет \bibemph{был} Иоас, когда воцарился, и сорок лет царствовал в Иерусалиме; имя матери его Цивья из Вирсавии.
\vs 2Ch 24:2 И делал Иоас угодное в очах Господних во все дни Иодая священника.
\vs 2Ch 24:3 И взял ему Иодай двух жен, и он имел \bibemph{от них} сыновей и дочерей.
\rsbpar\vs 2Ch 24:4 И после сего пришло на сердце Иоасу обновить дом Господень,
\vs 2Ch 24:5 и собрал он священников и левитов и сказал им: пойдите по городам Иудеи и собирайте со всех Израильтян серебро для поддержания дома Бога вашего из года в год, и поспешите в этом деле. Но не поспешили левиты.
\vs 2Ch 24:6 И призвал царь Иодая, главу \bibemph{их}, и сказал ему: почему ты не требуешь от левитов, чтобы они доставляли с Иудеи и Иерусалима дань, \bibemph{установленную} Моисеем, рабом Господним, и собранием Израильтян для скинии собрания?
\vs 2Ch 24:7 Ибо нечестивая Гофолия и сыновья ее разорили дом Божий и все посвященное для дома Господня употребили для Ваалов.
\vs 2Ch 24:8 И приказал царь, и сделали один ящик, и поставили его у входа в дом Господень извне.
\vs 2Ch 24:9 И провозгласили по Иудее и Иерусалиму, чтобы приносили Господу дань, \bibemph{наложенную} Моисеем, рабом Божиим, на Израильтян в пустыне.
\vs 2Ch 24:10 И обрадовались все начальствующие и весь народ, и приносили и клали в ящик дотоле, доколе он не наполнился.
\vs 2Ch 24:11 В то время, когда приносили ящик к царским чиновникам чрез левитов, и когда они видели, что серебра много, приходил писец царя и поверенный первосвященника, и высыпали из ящика, и относили его и ставили его на свое место. Так делали они изо дня в день, и собрали множество серебра.
\vs 2Ch 24:12 И отдавали его царь и Иодай производителям работ по дому Господню, и они нанимали каменотесов и плотников для подновления дома Господня, также кузнецов и медников для укрепления дома Господня.
\vs 2Ch 24:13 И работали производители работ, и совершилось исправление руками их, и привели дом Божий в надлежащее состояние его, и укрепили его.
\vs 2Ch 24:14 И кончив \bibemph{все}, они представили царю и Иодаю остаток серебра. И сделали из него сосуды для дома Господня, сосуды служебные и \bibemph{для} всесожжений, чаши и \bibemph{другие} сосуды золотые и серебряные. И приносили всесожжения в доме Господнем постоянно во все дни Иодая.
\rsbpar\vs 2Ch 24:15 И состарился Иодай и, насытившись днями \bibemph{жизни}, умер: сто тридцать лет \bibemph{было} ему, когда он умер.
\vs 2Ch 24:16 И похоронили его в городе Давидовом с царями, потому что он делал доброе в Израиле и для Бога, и для дома Его.
\rsbpar\vs 2Ch 24:17 Но по смерти Иодая пришли князья Иудейские и поклонились царю; тогда царь стал слушаться их.
\vs 2Ch 24:18 И оставили дом Господа Бога отцов своих и стали служить деревам \bibemph{посвященным} и идолам,~--- и был гнев \bibemph{Господень} на Иуду и Иерусалим за сию вину их.
\vs 2Ch 24:19 И он посылал к ним пророков для обращения их к Господу, и они увещевали их, но те не слушали.
\vs 2Ch 24:20 И Дух Божий облек Захарию, сына Иодая священника, и он стал на возвышении пред народом и сказал им: так говорит Господь: для чего вы преступаете повеления Господни? не будет успеха вам; и как вы оставили Господа, то и Он оставит вас.
\vs 2Ch 24:21 И сговорились против него, и побили его камнями, по приказанию царя [Иоаса], на дворе дома Господня.
\vs 2Ch 24:22 И не вспомнил царь Иоас благодеяния, какое сделал ему Иодай, отец его, и убил сына его. И он умирая говорил: да видит Господь и да взыщет!
\rsbpar\vs 2Ch 24:23 И по истечении года выступило против него войско Сирийское, и вошли в Иудею и в Иерусалим, и истребили из народа всех князей народа, и всю добычу, \bibemph{взятую} у них, отослали к царю в Дамаск.
\vs 2Ch 24:24 Хотя в небольшом числе людей приходило войско Сирийское, но Господь предал в руку их весьма многочисленную силу за то, что оставили Господа Бога отцов своих. И над Иоасом совершили они суд,
\vs 2Ch 24:25 и когда они ушли от него, оставив его в тяжкой болезни, то составили против него заговор рабы его, за кровь сына Иодая священника, и убили его на постели его, и он умер. И похоронили его в городе Давидовом, но не похоронили его в царских гробницах.
\vs 2Ch 24:26 Заговорщиками же против него были: Завад, сын Шимеафы Аммонитянки, и Иегозавад, сын Шимрифы Моавитянки.
\vs 2Ch 24:27 О сыновьях его и о множестве пророчеств против него и об устроении дома Божия написано в книге царей. И воцарился Амасия, сын его, вместо него.
\vs 2Ch 25:1 Двадцати пяти лет воцарился Амасия и двадцать девять лет царствовал в Иерусалиме; имя матери его Иегоаддань из Иерусалима.
\vs 2Ch 25:2 И делал он угодное в очах Господних, но не от полного сердца.
\vs 2Ch 25:3 Когда утвердилось за ним царство, тогда он умертвил рабов своих, убивших царя, отца его.
\vs 2Ch 25:4 Но детей их не умертвил, так как написано в законе, в книге Моисеевой, где заповедал Господь, говоря: не должны быть умерщвляемы отцы за детей, и дети не должны быть умерщвляемы за отцов, но каждый за свое преступление должен умереть.
\rsbpar\vs 2Ch 25:5 И собрал Амасия Иудеев и поставил их по поколениям под власть тысяченачальников и стоначальников, всех Иудеев и Вениаминян, и пересчитал их от двадцати лет и выше, и нашел их триста тысяч человек отборных, ходящих на войну, держащих копье и щит.
\vs 2Ch 25:6 И \bibemph{еще} нанял из Израильтян сто тысяч храбрых воинов за сто талантов серебра.
\vs 2Ch 25:7 Но человек Божий пришел к нему и сказал: царь! пусть не идет с тобою войско Израильское, потому что нет Господа с Израильтянами, со всеми сынами Ефрема.
\vs 2Ch 25:8 Но иди ты \bibemph{один}, делай дело, мужественно подвизайся на войне. \bibemph{Иначе} повергнет тебя Бог пред лицем врага, ибо есть сила у Бога поддержать и повергнуть.
\vs 2Ch 25:9 И сказал Амасия человеку Божию: что же делать со ста талантами, которые я отдал войску Израильскому? И сказал человек Божий: может Господь дать тебе более сего.
\vs 2Ch 25:10 И отделил их Амасия,~--- войско, пришедшее к нему из \bibemph{земли} Ефремовой,~--- чтоб они шли в свое место. И возгорелся сильно гнев их на Иудею, и они пошли назад в свое место, в пылу гнева.
\vs 2Ch 25:11 А Амасия отважился и повел народ свой, и пошел на долину Соляную и побил сынов Сеира десять тысяч;
\vs 2Ch 25:12 и десять тысяч живых взяли сыны Иудины в плен, и привели их на вершину скалы, и низринули их с вершины скалы, и все они разбились совершенно.
\vs 2Ch 25:13 Войско же, которое Амасия отослал обратно, чтоб оно не ходило с ним на войну, рассыпалось по городам Иудеи от Самарии до Вефорона и перебило в них три тысячи, и награбило множество добычи.
\rsbpar\vs 2Ch 25:14 Амасия, придя после поражения Идумеян, принес богов сынов Сеира и поставил их у себя богами, и пред ними кланялся и им кадил.
\vs 2Ch 25:15 И воспылал гнев Господа на Амасию, и послал Он к нему пророка, и тот сказал ему: зачем ты прибегаешь к богам народа сего, которые не избавили народа своего от руки твоей?
\vs 2Ch 25:16 Когда он говорил ему, \bibemph{царь} отвечал: разве советником царским поставили тебя? перестань, чтоб не убили тебя. И перестал пророк, сказав: знаю, что решил Бог погубить тебя, потому что ты сделал сие и не слушаешь совета моего.
\vs 2Ch 25:17 И посоветовался Амасия, царь Иудейский, и послал к Иоасу, сыну Иоахаза, сына Ииуева, царю Израильскому, сказать: выходи, повидаемся лично.
\vs 2Ch 25:18 И послал Иоас, царь Израильский, к Амасии, царю Иудейскому, сказать: терн, который на Ливане, послал к кедру, который на Ливане же, сказать: отдай дочь свою в жену сыну моему. Но прошли звери дикие, которые на Ливане, и истоптали этот терн.
\vs 2Ch 25:19 Ты говоришь: вот я побил Идумеян,~--- и вознеслось сердце твое до тщеславия. Сиди лучше у себя дома. К чему тебе затевать опасное дело? Падешь ты и Иудея с тобою.
\vs 2Ch 25:20 Но не послушался Амасия, так как от Бога \bibemph{было} это, дабы предать их в руку \bibemph{Иоаса} за то, что стали прибегать к богам Идумейским.
\vs 2Ch 25:21 И выступил Иоас, царь Израильский, и увиделись лично, он и Амасия, царь Иудейский, в Вефсамисе Иудейском.
\vs 2Ch 25:22 И были разбиты Иудеи Израильтянами, и разбежались каждый в шатер свой.
\vs 2Ch 25:23 И Амасию, царя Иудейского, сына Иоаса, сына Иоахазова, захватил Иоас, царь Израильский, в Вефсамисе и привел его в Иерусалим, и разрушил стену Иерусалимскую от ворот Ефремовых до ворот уг\acc{о}льных, на четыреста локтей;
\vs 2Ch 25:24 и \bibemph{взял} все золото и серебро, и все сосуды, находившиеся в доме Божием у Овед-Едома, и сокровища дома царского, и заложников, и возвратился в Самарию.
\rsbpar\vs 2Ch 25:25 И жил Амасия, сын Иоасов, царь Иудейский, по смерти Иоаса, сына Иоахазова, царя Израильского, пятнадцать лет.
\vs 2Ch 25:26 Прочие дела Амасии, первые и последние, описаны в книге царей Иудейских и Израильских.
\vs 2Ch 25:27 И после того времени, как Амасия отступил от Господа, составили против него заговор в Иерусалиме, и он убежал в Лахис. И послали за ним в Лахис, и умертвили его там.
\vs 2Ch 25:28 И привезли его на конях, и похоронили его с отцами его в городе Иудином.
\vs 2Ch 26:1 И взял весь народ Иудейский Озию, которому \bibemph{было} шестнадцать лет, и поставили его царем на место отца его Амасии.
\vs 2Ch 26:2 Он обстроил Елаф и возвратил его Иудее, после того как почил царь с отцами своими.
\rsbpar\vs 2Ch 26:3 Шестнадцати лет \bibemph{был} Озия, когда воцарился, и пятьдесят два года царствовал в Иерусалиме; имя матери его Иехолия из Иерусалима.
\vs 2Ch 26:4 И делал он угодное в очах Господних точно так, как делал Амасия, отец его;
\vs 2Ch 26:5 и прибегал он к Богу во дни Захарии, поучавшего страху Божию; и в те дни, когда он прибегал к Господу, споспешествовал ему Бог.
\vs 2Ch 26:6 И он вышел и сразился с Филистимлянами, и разрушил стены Гефа и стены Иавнеи и стены Азота; и построил города в \bibemph{области} Азотской и у Филистимлян.
\vs 2Ch 26:7 И помогал ему Бог против Филистимлян и против Аравитян, живущих в Гур-Ваале, и \bibemph{против} Меунитян;
\vs 2Ch 26:8 и давали Аммонитяне дань Озии, и дошло имя его до пределов Египта, потому что он был весьма силен.
\vs 2Ch 26:9 И построил Озия башни в Иерусалиме над воротами уг\acc{о}льными и над воротами долины и на углу, и укрепил их.
\vs 2Ch 26:10 И построил башни в пустыне, и иссек много водоемов, потому что имел много скота, и на низменности и на равнине, и земледельцев и садовников на горах и на Кармиле, ибо он любил земледелие.
\vs 2Ch 26:11 Было у Озии и войско, выходившее на войну отрядами, по счету в списке их, составленном рукою Иеиела писца и Маасеи надзирателя, под предводительством Ханании, \bibemph{одного} из главных сановников царских.
\vs 2Ch 26:12 Все число глав поколений, из храбрых воинов, \bibemph{было} две тысячи шестьсот,
\vs 2Ch 26:13 и под рукою их военной силы триста семь тысяч пятьсот, вступавших в сражение с воинским мужеством, на помощь царю против неприятеля.
\vs 2Ch 26:14 И заготовил для них Озия, для всего войска, щиты и копья, и шлемы и латы, и луки и пращные камни.
\vs 2Ch 26:15 И сделал он в Иерусалиме искусно придуманные машины, чтоб они находились на башнях и на углах для метания стрел и больших камней. И пронеслось имя его далеко, потому что он дивно оградил себя и сделался силен.
\rsbpar\vs 2Ch 26:16 Но когда он сделался силен, возгордилось сердце его на погибель \bibemph{его}, и он сделался преступником пред Господом Богом своим, ибо вошел в храм Господень, чтобы воскурить \bibemph{фимиам} на алтаре кадильном.
\vs 2Ch 26:17 И пошел за ним Азария священник, и с ним восемьдесят священников Господних, людей отличных,
\vs 2Ch 26:18 и воспротивились Озии царю и сказали ему: не тебе, Озия, кадить Господу; это \bibemph{дело} священников, сынов Аароновых, посвященных для каждения; выйди из святилища, ибо ты поступил беззаконно, и не [будет] тебе это в честь у Господа Бога.
\vs 2Ch 26:19 И разгневался Озия,~--- а в руке у него кадильница для каждения; и когда разгневался он на священников, проказа явилась на челе его, пред лицем священников, в доме Господнем, у алтаря кадильного.
\vs 2Ch 26:20 И взглянул на него Азария первосвященник и все священники; и вот у него проказа на челе его. И понуждали его выйти оттуда, да и сам он спешил удалиться, так как поразил его Господь.
\vs 2Ch 26:21 И был царь Озия прокаженным до дня смерти своей, и жил в отдельном доме и отлучен был от дома Господня. А Иоафам, сын его, начальствовал над домом царским и управлял народом земли.
\rsbpar\vs 2Ch 26:22 Прочие деяния Озии, первые и последние, описал Исаия, сын Амоса, пророк.
\vs 2Ch 26:23 И почил Озия с отцами своими, и похоронили его с отцами его на поле царских гробниц, ибо говорили: он прокаженный. И воцарился Иоафам, сын его, вместо него.
\vs 2Ch 27:1 Двадцати пяти лет \bibemph{был} Иоафам, когда воцарился, и шестнадцать лет царствовал в Иерусалиме; имя матери его Иеруша, дочь Садока.
\vs 2Ch 27:2 И делал он угодное в очах Господних точно так, как делал Озия, отец его, только он не входил в храм Господень, и народ продолжал еще грешить.
\vs 2Ch 27:3 Он построил верхние ворота дома Господня, и многое построил на стене Офела;
\vs 2Ch 27:4 и города построил на горе Иудейской, и в лесах построил дворцы и башни.
\vs 2Ch 27:5 Он воевал с царем Аммонитян и одолел их, и дали ему Аммонитяне в тот год сто талантов серебра и десять тысяч к\acc{о}ров пшеницы и ячменя десять тысяч. Это давали ему Аммонитяне и на другой год, и на третий.
\vs 2Ch 27:6 Так силен был Иоафам потому, что устроял пути свои пред лицем Господа Бога своего.
\rsbpar\vs 2Ch 27:7 Прочие деяния Иоафама и все войны его и поведение его описаны в книге царей Израильских и Иудейских:
\vs 2Ch 27:8 двадцати пяти лет был он, когда воцарился, и шестнадцать лет царствовал в Иерусалиме.
\vs 2Ch 27:9 И почил Иоафам с отцами своими, и похоронили его в городе Давидовом. И воцарился Ахаз, сын его, вместо него.
\vs 2Ch 28:1 Двадцати лет был Ахаз, когда воцарился, и шестнадцать лет царствовал в Иерусалиме; и он не делал угодного в очах Господних, как \bibemph{делал} Давид, отец его:
\vs 2Ch 28:2 он шел путями царей Израильских, и даже сделал литые статуи Ваалов;
\vs 2Ch 28:3 и он совершал курения на долине сынов Еннома, и проводил сыновей своих через огонь, подражая мерзостям народов, которых изгнал Господь пред лицем сынов Израилевых;
\vs 2Ch 28:4 и приносил жертвы и курения на высотах и на холмах и под всяким ветвистым деревом.
\vs 2Ch 28:5 И предал его Господь Бог его в руку царя Сириян, и они поразили его и взяли у него множество пленных и отвели в Дамаск. Также и в руку царя Израильского был предан он, и тот произвел у него великое поражение.
\vs 2Ch 28:6 И избил Факей, сын Ремалиин, [царь Израильский,] Иудеев сто двадцать тысяч в один день, людей воинственных, потому что они оставили Господа Бога отцов своих.
\vs 2Ch 28:7 Зихрий же, силач из Ефремлян, убил Маасею, сына царя, и Азрикама, начальствующего над дворцом, и Елкану, второго по царе.
\vs 2Ch 28:8 И взяли сыны Израилевы в плен у братьев своих, \bibemph{Иудеев}, двести тысяч жен, сыновей и дочерей; также и множество добычи награбили у них, и отправили добычу в Самарию.
\rsbpar\vs 2Ch 28:9 Там был пророк Господень, имя его Одед. Он вышел пред лице войска, шедшего в Самарию, и сказал им: вот Господь Бог отцов ваших, во гневе на Иудеев, предал их в руку вашу, и вы избили их с такою яростью, которая достигла до небес.
\vs 2Ch 28:10 И теперь вы думаете поработить сынов Иуды и Иерусалима в рабы и рабыни себе. А разве на самих вас нет вины пред Господом Богом вашим?
\vs 2Ch 28:11 Итак послушайте меня, и возвратите пленных, которых вы захватили из братьев ваших, ибо пламень гнева Господня на вас.
\vs 2Ch 28:12 И встали некоторые из начальников сынов Ефремовых: Азария, сын Иегоханана, Берехия, сын Мешиллемофа, и Езекия, сын Шаллума, и Амаса, сын Хадлая, против шедших с войны,
\vs 2Ch 28:13 и сказали им: не вводите сюда пленных, потому что грех был бы нам пред Господом. Неужели вы думаете прибавить к грехам нашим и к преступлениям нашим? велика вина наша, и пламень гнева [Господня] над Израилем.
\vs 2Ch 28:14 И оставили вооруженные пленных и добычу у военачальников и всего собрания.
\vs 2Ch 28:15 И встали мужи, упомянутые по именам, и взяли пленных, и всех нагих из них одели из добычи,~--- и одели их, и обули их, и накормили их, и напоили их, и помазали их елеем, и посадили на ослов всех слабых из них, и отправили их в Иерихон, город пальм, к братьям их, и возвратились в Самарию.
\rsbpar\vs 2Ch 28:16 В то время послал царь Ахаз к царям Ассирийским, чтоб они помогли ему,
\vs 2Ch 28:17 ибо Идумеяне и еще приходили, и \bibemph{многих} побили в Иудее, и взяли в плен;
\vs 2Ch 28:18 и Филистимляне рассыпались по городам низменного края и юга Иудеи и взяли Вефсамис и Аиалон, и Гедероф и Сохо и зависящие от него города, и Фимну и зависящие от нее города, и Гимзо и зависящие от него города, и поселились там.
\vs 2Ch 28:19 Так унизил Господь Иудею за Ахаза, царя Иудейского, потому что он развратил Иудею и тяжко грешил пред Господом.
\vs 2Ch 28:20 И пришел к нему Феглафелласар, царь Ассирийский, но был в тягость ему, вместо того, чтобы помочь ему,
\vs 2Ch 28:21 потому что Ахаз взял \bibemph{сокровища} из дома Господня и дома царского и у князей и отдал царю Ассирийскому, но не в помощь себе.
\rsbpar\vs 2Ch 28:22 И в тесное для себя время он продолжал беззаконно поступать пред Господом, он~--- царь Ахаз.
\vs 2Ch 28:23 И приносил он жертвы богам Дамасским, \bibemph{думая, что} они поражали его, и говорил: боги царей Сирийских помогают им; принесу я жертву им, и они помогут мне. Но они были на падение ему и всему Израилю.
\vs 2Ch 28:24 И собрал Ахаз сосуды дома Божия, и сокрушил сосуды дома Божия, и запер двери дома Господня, и устроил себе жертвенники по всем углам в Иерусалиме,
\vs 2Ch 28:25 и по всем городам Иудиным устроил высоты для каждения богам иным, и раздражал Господа Бога отцов своих.
\rsbpar\vs 2Ch 28:26 Прочие дела его и все поступки его, первые и последние, описаны в книге царей Иудейских и Израильских.
\vs 2Ch 28:27 И почил Ахаз с отцами своими, и похоронили его в городе, в Иерусалиме, но не внесли его в гробницы царей Израилевых. И воцарился Езекия, сын его, вместо него.
\vs 2Ch 29:1 Езекия воцарился двадцати пяти лет, и двадцать девять лет царствовал в Иерусалиме; имя матери его Авия, дочь Захарии.
\vs 2Ch 29:2 И делал он угодное в очах Господних точно так, как делал Давид, отец его.
\rsbpar\vs 2Ch 29:3 В первый же год царствования своего, в первый месяц, он отворил двери дома Господня и возобновил их,
\vs 2Ch 29:4 и велел прийти священникам и левитам, и собрал их на площади восточной,
\vs 2Ch 29:5 и сказал им: послушайте меня, левиты! Ныне освятитесь \bibemph{сами} и освятите дом Господа Бога отцов ваших, и выбросьте нечистоту из святилища.
\vs 2Ch 29:6 Ибо отцы наши поступали беззаконно, и делали неугодное в очах Господа Бога нашего, и оставили Его, и отвратили они лица свои от жилища Господня, и оборотились спиною,
\vs 2Ch 29:7 и заперли двери притвора, и погасили светильники, и не сожигали курения, и не возносили всесожжений во святилище Бога Израилева.
\vs 2Ch 29:8 И был гнев Господа на Иудею и на Иерусалим, и Он отдал их на позор, на опустошение и на посмеяние, как вы видите глазами вашими.
\vs 2Ch 29:9 И вот, пали отцы наши от меча, а сыновья наши и дочери наши и жены наши за это в плену [в земле не своей] доныне.
\vs 2Ch 29:10 Теперь у меня на сердце~--- заключить завет с Господом Богом Израилевым, да отвратит от нас пламень гнева Своего.
\vs 2Ch 29:11 Дети мои! не будьте небрежны, ибо вас избрал Господь предстоять лицу Его, служить Ему и быть у Него служителями и возжигателями курений.
\vs 2Ch 29:12 И встали левиты: Махаф, сын Амасая, и Иоель, сын Азарии, из сыновей Каафовых; и из сыновей Мерариных: Кис, сын Авдия, и Азария, сын Иегаллелела; и из племени Гирсонова: Иоах, сын Зиммы, и Еден, сын Иоаха;
\vs 2Ch 29:13 и из сыновей Елицафановых: Шимри и Иеиел; и из сыновей Асафовых: Захария и Матфания;
\vs 2Ch 29:14 и из сыновей Емановых: Иехиел и Шимей; и из сыновей Идифуновых: Шемаия и Уззиел.
\vs 2Ch 29:15 Они собрали братьев своих и освятились, и пошли по приказанию царя очищать дом Господень по словам Господа.
\vs 2Ch 29:16 И вошли священники внутрь дома Господня для очищения, и вынесли все нечистое, что нашли в храме Господнем, на двор дома Господня, а левиты взяли это, чтобы вынести вон к потоку Кедрону.
\vs 2Ch 29:17 И начали освящать в первый \bibemph{день} первого месяца, и в восьмой день \bibemph{того же} месяца вошли в притвор Господень; и освящали дом Господень восемь дней, и в шестнадцатый день первого месяца кончили.
\vs 2Ch 29:18 И пришли в дом к царю Езекии и сказали: мы очистили дом Господень, и жертвенник для всесожжения, и все сосуды его, и стол \bibemph{для хлебов} предложения, и все сосуды его;
\vs 2Ch 29:19 и все сосуды, которые забросил царь Ахаз во время царствования своего, в беззаконии своем, мы приготовили и освятили, и вот они пред жертвенником Господним.
\rsbpar\vs 2Ch 29:20 И встал царь Езекия рано утром и собрал начальников города, и пошел в дом Господень.
\vs 2Ch 29:21 И привели семь тельцов и семь овнов, и семь агнцев и семь козлов на жертву о грехе за царство и за святилище и за Иудею; и приказал он сынам Аароновым, священникам, вознести всесожжение на жертвенник Господень.
\vs 2Ch 29:22 И закололи тельцов, и взяли священники кровь, и окропили жертвенник, и закололи овнов, и окропили кровью жертвенник; и закололи агнцев, и окропили кровью жертвенник.
\vs 2Ch 29:23 И привели козлов за грех пред лице царя и собрания, и они возложили руки свои на них.
\vs 2Ch 29:24 И закололи их священники, и очистили кровью их жертвенник для заглаждения грехов всего Израиля, ибо за всего Израиля приказал царь \bibemph{принести} всесожжение и жертву о грехе.
\vs 2Ch 29:25 И поставил он левитов в доме Господнем с кимвалами, псалтирями и цитрами, по уставу Давида и Гада, прозорливца царева, и Нафана пророка, так как от Господа \bibemph{был} устав этот чрез пророков Его.
\vs 2Ch 29:26 И стали левиты с \bibemph{музыкальными} орудиями Давидовыми и священники с трубами.
\vs 2Ch 29:27 И приказал Езекия вознести всесожжение на жертвенник. И в то время, как началось всесожжение, началось пение Господу, при \bibemph{звуке} труб и орудий Давида, царя Израилева.
\vs 2Ch 29:28 И все собрание молилось, и певцы пели, и трубили трубы, доколе не окончилось всесожжение.
\vs 2Ch 29:29 По окончании же всесожжения царь и все находившиеся при нем преклонились и поклонились.
\vs 2Ch 29:30 И сказал царь Езекия и князья левитам, чтоб они славили Господа словами Давида и Асафа прозорливца, и они славили с радостью, и преклонялись и поклонялись.
\vs 2Ch 29:31 И продолжал Езекия и сказал: теперь вы посвятили себя Господу; приступайте и приносите жертвы и благодарственные приношения в дом Господень. И понесло \bibemph{все} собрание жертвы и благодарственные приношения, и всякий, кто расположен был сердцем,~--- всесожжения.
\vs 2Ch 29:32 И было число всесожжений, которые привели собравшиеся: семьдесят волов, сто овнов, двести агнцев~--- все это для всесожжения Господу.
\vs 2Ch 29:33 \bibemph{Других} священных жертв \bibemph{было}: шестьсот из крупного скота и три тысячи из мелкого скота.
\vs 2Ch 29:34 Но священников было мало, и они не могли сдирать кож со всех всесожжений, и помогали им братья их левиты, до окончания дела и доколе освятились \bibemph{прочие} священники, ибо левиты были более тщательны в освящении себя, нежели священники.
\vs 2Ch 29:35 Притом же всесожжений \bibemph{было} множество с туками мирных жертв и с возлияниями при всесожжении. Так восстановлено служение в доме Господнем.
\vs 2Ch 29:36 И радовался Езекия и весь народ о том, что Бог \bibemph{так} расположил народ, ибо это сделалось неожиданно.
\vs 2Ch 30:1 И послал Езекия по всей \bibemph{земле} Израильской и Иудее, и письма писал к Ефрему и Манассии, чтобы пришли в дом Господень, в Иерусалим, для совершения пасхи Господу Богу Израилеву.
\vs 2Ch 30:2 И положили на совете царь и князья его и все собрание в Иерусалиме~--- совершить пасху во второй месяц,
\vs 2Ch 30:3 ибо не могли совершить ее в свое время, потому что священники \bibemph{еще} не освятились в достаточном числе и народ не собрался в Иерусалим.
\vs 2Ch 30:4 И понравилось это царю и всему собранию.
\vs 2Ch 30:5 И определили объявить по всему Израилю, от Вирсавии до Дана, чтобы шли в Иерусалим для совершения пасхи Господу Богу Израилеву, потому что давно не совершали \bibemph{ее}, как предписано.
\vs 2Ch 30:6 И пошли гонцы с письмами от царя и от князей его по всей \bibemph{земле} Израильской и Иудее, и по повелению царя говорили: дети Израиля! обратитесь к Господу Богу Авраама, Исаака и Израиля, и Он обратится к остатку, уцелевшему у вас от руки царей Ассирийских.
\vs 2Ch 30:7 И не будьте таковы, как отцы ваши и братья ваши, которые беззаконно поступали пред Господом Богом отцов своих; и Он предал их на опустошение, как вы видите.
\vs 2Ch 30:8 Ныне не будьте жестоковыйны, как отцы ваши, покоритесь Господу и приходите во святилище Его, которое Он освятил навек; и служите Господу Богу вашему, и Он отвратит от вас пламень гнева Своего.
\vs 2Ch 30:9 Когда вы обратитесь к Господу, тогда братья ваши и дети ваши [будут] в милости у пленивших их и возвратятся в землю сию, ибо благ и милосерд Господь Бог ваш и не отвратит лица от вас, если вы обратитесь к Нему.
\vs 2Ch 30:10 И ходили гонцы из города в город по земле Ефремовой и Манассииной и до Завулоновой, но над ними смеялись и издевались.
\vs 2Ch 30:11 Однако некоторые из \bibemph{колена} Асирова, Манассиина и Завулонова смирились и пришли в Иерусалим.
\vs 2Ch 30:12 И над Иудеею была рука Божия, даровавшая им единое сердце, чтоб исполнить повеление царя и князей, по слову Господню.
\rsbpar\vs 2Ch 30:13 И собралось в Иерусалим множество народа для совершения праздника опресноков, во второй месяц,~--- собрание весьма многочисленное.
\vs 2Ch 30:14 И встали и ниспровергли жертвенники, которые были в Иерусалиме; и всё, на чем совершаемо было курение [идолам], разрушили и бросили в поток Кедрон;
\vs 2Ch 30:15 и закололи пасхального агнца в четырнадцатый \bibemph{день} второго месяца. Священники и левиты устыдившись освятились и принесли всесожжения в дом Господень,
\vs 2Ch 30:16 и стали на своем месте по уставу своему, по закону Моисея, человека Божия. Священники кропили кровью [принимая ее] из рук левитов.
\vs 2Ch 30:17 Так как много \bibemph{было} в собрании таких, которые не освятились, то вместо нечистых левиты закололи пасхального агнца, для посвящения Господу.
\vs 2Ch 30:18 Многие из народа, большею частью из колена Ефремова и Манассиина, Иссахарова и Завулонова, не очистились; однако же они ели пасху, не по уставу.
\vs 2Ch 30:19 Но Езекия помолился за них, говоря: Господь благий да простит каждого, кто расположил сердце свое к тому, чтобы взыскать Господа Бога, Бога отцов своих, хотя и без очищения священного!
\vs 2Ch 30:20 И услышал Господь Езекию и простил народ.
\rsbpar\vs 2Ch 30:21 И совершили сыны Израилевы, находившиеся в Иерусалиме, праздник опресноков в семь дней, с великим веселием; каждый день левиты и священники славили Господа на орудиях, \bibemph{устроенных} для славословия Господа.
\vs 2Ch 30:22 И говорил Езекия по сердцу всем левитам, имевшим доброе разумение \bibemph{в служении} Господу. И ели праздничное семь дней, принося жертвы мирные и славя Господа Бога отцов своих.
\vs 2Ch 30:23 И решило все собрание праздновать другие семь дней, и провели эти семь дней в веселии,
\vs 2Ch 30:24 потому что Езекия, царь Иудейский, выставил для собравшихся тысячу тельцов и десять тысяч мелкого скота, и вельможи выставили для собравшихся тысячу тельцов и десять тысяч мелкого скота; и священников освятилось \bibemph{уже} много.
\vs 2Ch 30:25 И веселились все собравшиеся из Иудеи, и священники и левиты, и все собрание, пришедшее от Израиля, и пришельцы, пришедшие из земли Израильской и обитавшие в Иудее.
\vs 2Ch 30:26 И было веселие великое в Иерусалиме, потому что со дней Соломона, сына Давидова, царя Израилева, \bibemph{не бывало} подобного сему в Иерусалиме.
\vs 2Ch 30:27 И встали священники и левиты, и благословили народ; и услышан был голос их, и взошла молитва их в святое жилище Его на небеса.
\vs 2Ch 31:1 И по окончании всего этого, пошли все Израильтяне, \bibemph{там} находившиеся, в города Иудейские и разбили статуи, срубили \bibemph{посвященные} дерева, и разрушили высоты и жертвенники во всей Иудее и в \bibemph{земле} Вениаминовой, Ефремовой и Манассииной, до конца. И \bibemph{потом} возвратились все сыны Израилевы, каждый во владение свое, в города свои.
\vs 2Ch 31:2 И поставил Езекия череды священников и левитов, по их распределению, каждого при деле своем, священническом или левитском, при всесожжении и при жертвах мирных, для службы, для хваления и славословия, у ворот дома Господня.
\vs 2Ch 31:3 И \bibemph{определил} царь часть из имущества своего на всесожжения: на всесожжения утренние и вечерние, и на всесожжения в субботы и в новомесячия, и в праздники, как написано в законе Господнем.
\vs 2Ch 31:4 И повелел он народу, живущему в Иерусалиме, давать определенное содержание священникам и левитам, чтоб они были ревностны в законе Господнем.
\rsbpar\vs 2Ch 31:5 Когда обнародовано было это повеление, тогда нанесли сыны Израилевы множество начатков хлеба, вина, и масла, и меду, и всяких произведений полевых; и десятин из всего нанесли множество.
\vs 2Ch 31:6 И Израильтяне и Иудеи, живущие по городам Иудейским, также представили десятины из крупного и мелкого скота и десятины из пожертвований, посвященных Господу Богу их; и наложили груды, груды.
\vs 2Ch 31:7 В третий месяц начали класть груды, и в седьмой месяц кончили.
\vs 2Ch 31:8 И пришли Езекия и вельможи, и увидели груды, и благодарили Господа и народ Его Израиля.
\vs 2Ch 31:9 И спросил Езекия священников и левитов об этих грудах.
\vs 2Ch 31:10 И отвечал ему Азария первосвященник из дома Садокова и сказал: с того времени, как начали носить приношения в дом Господень, мы ели досыта, и многое осталось, потому что Господь благословил народ Свой. Из оставшегося \bibemph{составилось} такое множество.
\vs 2Ch 31:11 И приказал Езекия приготовить комнаты при доме Господнем. И приготовили.
\vs 2Ch 31:12 И перенесли \bibemph{туда} приношения, и десятины, и пожертвования, со \bibemph{всею} точностью. И \bibemph{был} начальником при них Хонания левит, и Симей, брат его, вторым.
\vs 2Ch 31:13 А Иехиил и Азазия, и Нахаф и Асаил, и Иеримоф и Иозавад, и Елиел и Исмахия, и Махаф и Бенания \bibemph{были} смотрителями под рукою Хонании и Симея, брата его, по распоряжению царя Езекии и Азарии, начальника при доме Божием.
\vs 2Ch 31:14 Коре, сын Имны, левит, привратник на восточной стороне, \bibemph{был} при добровольных приношениях Богу, для выдачи принесенного Господу и важнейших из вещей посвященных.
\vs 2Ch 31:15 И под его \bibemph{ведением находились} Еден, и Миниамин, и Иешуа, и Шемаия, и Амария и Шехания в городах священнических, чтобы верно раздавать братьям своим части, как большому, так и малому,
\vs 2Ch 31:16 сверх списка их, \bibemph{всем} мужеского пола от трех лет и выше, всем ходящим в дом Господа для дел ежедневных, для служения их, по должностям их и по отделам их,
\vs 2Ch 31:17 и внесенным в список священникам, по поколениям их, и левитам от двадцати лет и выше, по должностям их, по отделам их,
\vs 2Ch 31:18 и внесенным в список, со всеми малолетними их, с женами их и с сыновьями их и с дочерями их,~--- всему обществу, ибо они со \bibemph{всею} верностью посвятили себя на священную службу.
\vs 2Ch 31:19 И для сынов Аароновых, священников в селениях вокруг городов их, при каждом городе \bibemph{поставлены были} мужи поименованные, чтобы раздавать участки всем мужеского пола у священников и всем внесенным в список у левитов.
\rsbpar\vs 2Ch 31:20 Вот что сделал Езекия во всей Иудее,~--- и он делал доброе, и справедливое, и истинное пред лицем Господа Бога своего.
\vs 2Ch 31:21 И во всем, что он предпринимал на служение дому Божию и для соблюдения закона и заповедей, помышляя о Боге своем, он действовал от всего сердца своего и имел успех.
\vs 2Ch 32:1 После таких дел и верности, пришел Сеннахирим, царь Ассирийский, и вступил в Иудею, и осадил укрепленные города, и думал отторгнуть их себе.
\vs 2Ch 32:2 Когда Езекия увидел, что пришел Сеннахирим с намерением воевать против Иерусалима,
\vs 2Ch 32:3 тогда решил с князьями своими и с военными людьми своими зас\acc{ы}пать источники воды, которые вне города, и те помогли ему.
\vs 2Ch 32:4 И собралось множество народа, и зас\acc{ы}пали все источники и поток, протекавший по стране, говоря: да не найдут цари Ассирийские, придя \bibemph{сюда}, много воды [и да не укрепятся].
\vs 2Ch 32:5 И ободрился он, и восстановил всю обрушившуюся стену, и поднял ее до башни, и извне \bibemph{построил} другую стену, и укрепил Милло в городе Давидовом, и наготовил множество оружия и щитов.
\vs 2Ch 32:6 И поставил военачальников над народом, и собрал их к себе на площадь у городских ворот, и говорил к сердцу их, и сказал:
\vs 2Ch 32:7 будьте тверды и мужественны, не бойтесь и не страшитесь царя Ассирийского и всего множества, которое с ним, потому что с нами более, нежели с ним;
\vs 2Ch 32:8 с ним мышца плотская, а с нами Господь Бог наш, чтобы помогать нам и сражаться на бранях наших. И подкрепился народ словами Езекии, царя Иудейского.
\rsbpar\vs 2Ch 32:9 После сего послал Сеннахирим, царь Ассирийский, рабов своих в Иерусалим,~--- сам он \bibemph{стоял} против Лахиса, и вся сила его с ним,~--- к Езекии, царю Иудейскому, и ко всем Иудеям, которые в Иерусалиме, сказать:
\vs 2Ch 32:10 так говорит Сеннахирим, царь Ассирийский: на что вы надеетесь и сидите в крепости в Иерусалиме?
\vs 2Ch 32:11 Не обольщает ли вас Езекия, чтобы предать вас смерти от голода и жажды, говоря: Господь Бог наш спасет нас от руки царя Ассирийского?
\vs 2Ch 32:12 Не этот ли Езекия разрушил высоты Его и жертвенники Его, и сказал Иудее и Иерусалиму: пред жертвенником единым поклоняйтесь и на нем совершайте курения?
\vs 2Ch 32:13 Разве вы не знаете, что сделал я и отцы мои со всеми народами земель? Могли ли боги народов земных спасти землю свою от руки моей?
\vs 2Ch 32:14 Кто из всех богов народов, истребленных отцами моими, мог спасти народ свой от руки моей? \bibemph{Как же} возможет ваш Бог спасти вас от руки моей?
\vs 2Ch 32:15 И ныне пусть не обольщает вас Езекия и не отклоняет вас таким образом; не верьте ему: если не в силах был ни один бог ни одного народа и царства спасти народ свой от руки моей и от руки отцов моих, то и ваш Бог не спасет вас от руки моей.
\vs 2Ch 32:16 И еще \bibemph{многое} говорили рабы его против Господа Бога и против Езекии, раба Его.
\vs 2Ch 32:17 И письма писал он, \bibemph{в которых} поносил Господа Бога Израилева и говорил против Него такие слова: как боги народов земных не спасли народов своих от руки моей, так Бог Езекии не спасет народа Своего от руки моей.
\vs 2Ch 32:18 И кричали громким голосом на Иудейском языке к народу Иерусалимскому, который \bibemph{был} на стене, чтоб устрашить его и напугать его, и взять город.
\vs 2Ch 32:19 И говорили о Боге Иерусалима, как о богах народов земли,~--- изделии рук человеческих.
\vs 2Ch 32:20 И помолился царь Езекия и Исаия, сын Амосов, пророк, и возопили к небу.
\vs 2Ch 32:21 И послал Господь Ангела, и он истребил всех храбрых и главноначальствующего и начальствующих в войске царя Ассирийского. И возвратился он со стыдом в землю свою; и когда пришел в дом бога своего,~--- исшедшие из чресл его поразили его там мечом.
\vs 2Ch 32:22 Так спас Господь Езекию и жителей Иерусалима от руки Сеннахирима, царя Ассирийского, и от руки всех и оберегал их отовсюду.
\vs 2Ch 32:23 Тогда многие приносили дары Господу в Иерусалим и дорогие вещи Езекии, царю Иудейскому. И он возвеличился после сего в глазах всех народов.
\rsbpar\vs 2Ch 32:24 В те дни заболел Езекия смертельно. И помолился Господу, и Он услышал его и дал ему знамение.
\vs 2Ch 32:25 Но не воздал Езекия за оказанные ему благодеяния, ибо возгордилось сердце его. И был на него гнев \bibemph{Божий} и на Иудею, и на Иерусалим.
\vs 2Ch 32:26 Но как смирился Езекия в гордости сердца своего,~--- сам и жители Иерусалима, то не пришел на них гнев Господень во дни Езекии.
\vs 2Ch 32:27 И было у Езекии богатства и славы весьма много, и хранилище он сделал у себя для серебра и золота, и камней драгоценных, и для ароматов и щитов, и для всяких драгоценных сосудов;
\vs 2Ch 32:28 и кладовые для произведений \bibemph{земли}, для хлеба, вина и масла, и стойла для всякого рода скота, и дворы для стад.
\vs 2Ch 32:29 И города построил себе. И стад мелкого и крупного скота \bibemph{было у него} множество, потому что дал ему Бог весьма большое имущество.
\vs 2Ch 32:30 Он же, Езекия, запер верхний проток вод Геона и провел их вниз к западной стороне города Давидова. И действовал успешно Езекия во всяком деле своем.
\vs 2Ch 32:31 Только при послах царей Вавилонских, которые присылали к нему спросить о знамении, бывшем на земле, оставил его Бог, чтоб испытать его и открыть все, что у него на сердце.
\rsbpar\vs 2Ch 32:32 Прочие деяния Езекии и добродетели его описаны в видении Исаии, сына Амосова, пророка, и в книге царей Иудейских и Израильских.
\vs 2Ch 32:33 И почил Езекия с отцами своими, и похоронили его над гробницами сыновей Давидовых, и почесть воздали ему по смерти его все Иудеи и жители Иерусалима. И воцарился Манассия, сын его, вместо него.
\vs 2Ch 33:1 Двенадцати лет \bibemph{был} Манассия, когда воцарился, и пятьдесят пять лет царствовал в Иерусалиме,
\vs 2Ch 33:2 и делал он неугодное в очах Господних, подражая мерзостям народов, которых прогнал Господь от лица сынов Израилевых,
\vs 2Ch 33:3 и снова построил высоты, которые разрушил Езекия, отец его, и поставил жертвенники Ваалам, и устроил дубравы, и поклонялся всему воинству небесному, и служил ему,
\vs 2Ch 33:4 и соорудил жертвенники в доме Господнем, о котором сказал Господь: в Иерусалиме будет имя Мое вечно;
\vs 2Ch 33:5 и соорудил жертвенники всему воинству небесному на обоих дворах дома Господня.
\vs 2Ch 33:6 Он же проводил сыновей своих чрез огонь в долине сына Енномова, и гадал, и ворожил, и чародействовал, и учредил вызывателей мертвецов и волшебников; много делал он неугодного в очах Господа, к прогневлению Его.
\vs 2Ch 33:7 И поставил резного идола, которого сделал, в доме Божием, о котором говорил Бог Давиду и Соломону, сыну его: в доме сем и в Иерусалиме, который Я избрал из всех колен Израилевых, Я положу имя Мое навек;
\vs 2Ch 33:8 и не дам впредь выступить ноге Израиля из земли сей, которую Я укрепил за отцами их, если только они будут стараться исполнять все, что Я заповедал им, по всему закону и уставам и повелениям, \bibemph{данным} рукою Моисея.
\vs 2Ch 33:9 Но Манассия довел Иудею и жителей Иерусалима до того, что они поступали хуже тех народов, которых истребил Господь от лица сынов Израилевых.
\rsbpar\vs 2Ch 33:10 И говорил Господь к Манассии и к народу его, но они не слушали.
\vs 2Ch 33:11 И привел Господь на них военачальников царя Ассирийского, и заковали они Манассию в кандалы и оковали его цепями, и отвели его в Вавилон.
\vs 2Ch 33:12 И в тесноте своей он стал умолять лице Господа Бога своего и глубоко смирился пред Богом отцов своих.
\vs 2Ch 33:13 И помолился Ему, и \bibemph{Бог} преклонился к нему и услышал моление его, и возвратил его в Иерусалим на царство его. И узнал Манассия, что Господь есть Бог.
\vs 2Ch 33:14 И после того построил внешнюю стену города Давидова, на западной стороне Геона, по лощине и до входа в Рыбные ворота, и провел ее вокруг Офела и высоко поднял ее. И поставил военачальников по всем укрепленным городам в Иудее,
\vs 2Ch 33:15 и низверг чужеземных богов и идола из дома Господня, и все капища, которые соорудил на горе дома Господня и в Иерусалиме, и выбросил их за город.
\vs 2Ch 33:16 И восстановил жертвенник Господень и принес на нем жертвы мирные и хвалебные, и сказал Иудеям, чтобы они служили Господу Богу Израилеву.
\vs 2Ch 33:17 Но народ еще приносил жертвы на высотах, хотя и Господу Богу своему.
\rsbpar\vs 2Ch 33:18 Прочие дела Манассии, и молитва его к Богу своему, и слова прозорливцев, говоривших к нему именем Господа Бога Израилева, находятся в записях царей Израилевых.
\vs 2Ch 33:19 И молитва его, и то, что \bibemph{Бог} преклонился к нему, и все грехи его и беззакония его, и места, на которых он построил высоты и поставил изображения Астарты и истуканов, прежде нежели смирился, описаны в записях Хозая.
\vs 2Ch 33:20 И почил Манассия с отцами своими, и похоронили его в доме его. И воцарился Амон, сын его, вместо него.
\rsbpar\vs 2Ch 33:21 Двадцати двух лет был Амон, когда воцарился, и два года царствовал в Иерусалиме.
\vs 2Ch 33:22 И делал неугодное в очах Господних так, как делал Манассия, отец его; и всем истуканам, которых сделал Манассия, отец его, приносил Амон жертвы и служил им.
\vs 2Ch 33:23 И не смирился пред лицем Господним, как смирился Манассия, отец его; напротив, Амон умножил \bibemph{свои} грехи.
\vs 2Ch 33:24 И составили против него заговор слуги его, и умертвили его в доме его.
\vs 2Ch 33:25 Но народ земли перебил всех, бывших в заговоре против царя Амона, и воцарил народ земли Иосию, сына его, вместо него.
\vs 2Ch 34:1 Восемь лет было Иосии, когда он воцарился, и тридцать один год царствовал в Иерусалиме,
\vs 2Ch 34:2 и делал он угодное в очах Господних, и ходил путями Давида, отца своего, и не уклонялся ни направо, ни налево.
\rsbpar\vs 2Ch 34:3 В восьмой год царствования своего, будучи еще отроком, он начал прибегать к Богу Давида, отца своего, а в двенадцатый год начал очищать Иудею и Иерусалим от высот и \bibemph{посвященных} дерев и от резных и литых кумиров.
\vs 2Ch 34:4 И разрушили пред лицем его жертвенники Ваалов и статуи, возвышавшиеся над ними; и \bibemph{посвященные} дерева он срубил, и резные и литые кумиры изломал и разбил в прах, и рассыпал на гробах тех, которые приносили им жертвы,
\vs 2Ch 34:5 и кости жрецов сжег на жертвенниках их, и очистил Иудею и Иерусалим,
\vs 2Ch 34:6 и в городах Манассии, и Ефрема, и Симеона, \bibemph{даже} до колена Неффалимова, и в опустошенных окрестностях их
\vs 2Ch 34:7 он разрушил жертвенники и \bibemph{посвященные} дерева, и кумиры разбил в прах, и все статуи сокрушил по всей земле Израильской, и возвратился в Иерусалим.
\rsbpar\vs 2Ch 34:8 В восемнадцатый год царствования своего, по очищении земли и дома \bibemph{Божия}, он послал Шафана, сына Ацалии, и Маасею градоначальника, и Иоаха, сына Иоахазова, дееписателя, возобновить дом Господа Бога своего.
\vs 2Ch 34:9 И пришли они к Хелкии первосвященнику, и отдали серебро, принесенное в дом Божий, которое левиты, стоящие на страже у порога, собрали из рук Манассии и Ефрема и всех прочих Израильтян, и от всего Иуды и Вениамина, и от жителей Иерусалима,
\vs 2Ch 34:10 и отдали в руки производителям работ, приставленным к дому Господню, чтоб они раздавали его работникам, которые работали в доме Господнем, при исправлении и возобновлении дома.
\vs 2Ch 34:11 И они раздавали плотникам и строителям на покупку тесаных камней и дерев для связей и для покрытия зданий, которые разорили цари Иудейские.
\vs 2Ch 34:12 Люди сии действовали честно при работе, и для надзора над ними поставлены были Иахаф и Овадия, левиты из сыновей Мерариных, и Захария и Мешуллам из сыновей Каафовых, и все левиты, умеющие играть на музыкальных орудиях.
\vs 2Ch 34:13 Они же \bibemph{были} приставниками над носильщиками и наблюдали над всеми работниками при каждой работе; из левитов же \bibemph{были и} писцы, и надзиратели, и привратники.
\rsbpar\vs 2Ch 34:14 Когда вынимали они серебро, принесенное в дом Господень, тогда Хелкия священник нашел книгу закона Господня, \bibemph{данную} рукою Моисея.
\vs 2Ch 34:15 И начал Хелкия, и сказал Шафану писцу: книгу закона нашел я в доме Господнем. И подал Хелкия ту книгу Шафану.
\vs 2Ch 34:16 И понес Шафан книгу к царю, и принес при этом царю известие: все, что поручено рабам твоим, они делают;
\vs 2Ch 34:17 и высыпали серебро, найденное в доме Господнем, и передали его в руки приставникам и в руки производителям работ.
\vs 2Ch 34:18 И \bibemph{также} донес Шафан писец царю, говоря: книгу дал мне Хелкия священник. И читал ее Шафан перед царем.
\vs 2Ch 34:19 Когда услышал царь слова закона, то разодрал одежды свои.
\vs 2Ch 34:20 И дал царь повеление Хелкии и Ахикаму, сыну Шафанову, и Авдону, сыну Михея, и Шафану писцу, и Асаии, слуге царскому, говоря:
\vs 2Ch 34:21 пойдите, вопросите Господа за меня и за оставшихся у Израиля и за Иуду о словах сей найденной книги, потому что велик гнев Господа, который воспылал на нас за то, что не соблюдали отцы наши слова Господня, чтобы поступать по всему написанному в книге сей.
\vs 2Ch 34:22 И пошел Хелкия и те, которые от царя, к Олдане пророчице, жене Шаллума, сына Тавкегафа, сына Хасры, хранителя одежд,~--- а жила она во второй части Иерусалима,~--- и говорили с нею об этом.
\vs 2Ch 34:23 И она сказала им: так говорит Господь Бог Израилев: скажите тому человеку, который послал вас ко мне:
\vs 2Ch 34:24 так говорит Господь: вот Я наведу бедствие на место сие и на жителей его все проклятия, написанные в книге, которую читали пред лицем царя Иудейского,
\vs 2Ch 34:25 за то, что они оставили Меня и кадили богам другим, чтобы прогневлять Меня всеми делами рук своих. И гнев Мой возгорится над местом сим и не угаснет.
\vs 2Ch 34:26 А царю Иудейскому, пославшему вас вопросить Господа, так скажите: так говорит Господь Бог Израилев о словах, которые ты слышал:
\vs 2Ch 34:27 так как смягчилось сердце твое, и ты смирился пред Богом, услышав слова Его о месте сем и о жителях его,~--- и ты смирился предо Мною, и разодрал одежды свои, и плакал предо Мною, то и Я услышал \bibemph{тебя}, говорит Господь.
\vs 2Ch 34:28 Вот Я приложу тебя к отцам твоим, и положен будешь в гробницу твою в мире, и не увидят глаза твои всего того бедствия, которое Я наведу на место сие и на жителей его. И принесли царю ответ.
\rsbpar\vs 2Ch 34:29 И послал царь, и собрал всех старейшин Иудеи и Иерусалима,
\vs 2Ch 34:30 и пошел царь в дом Господень, и \bibemph{с ним} все Иудеи и жители Иерусалима, и священники и левиты, и весь народ, от большого до малого; и он прочитал вслух их все слова книги завета, найденной в доме Господнем.
\vs 2Ch 34:31 И стал царь на месте своем, и заключил завет пред лицем Господа последовать Господу и соблюдать заповеди Его и откровения Его, и уставы Его, от всего сердца своего и от всей души своей, чтобы выполнить слова завета, написанные в книге сей.
\vs 2Ch 34:32 И велел царь подтвердить \bibemph{это} всем находившимся в Иерусалиме и в земле Вениаминовой; и стали поступать жители Иерусалима по завету Бога, Бога отцов своих.
\vs 2Ch 34:33 И изверг Иосия все мерзости из всех земель, которые у сынов Израилевых, и повелел всем, находившимся в \bibemph{земле} Израилевой служить Господу Богу своему. И во все дни \bibemph{жизни} его они не отступали от Господа Бога отцов своих.
\vs 2Ch 35:1 И совершил Иосия в Иерусалиме пасху Господу, и закололи пасхального агнца в четырнадцатый \bibemph{день} первого месяца.
\vs 2Ch 35:2 И поставил он священников на местах их, и ободрял их на служение в доме Господнем,
\vs 2Ch 35:3 и сказал левитам, наставникам всех Израильтян, посвященным Господу: поставьте ковчег святый в храме, который построил Соломон, сын Давидов, царь Израилев; нет вам нужды носить \bibemph{его} на раменах; служите теперь Господу Богу нашему и народу Его Израилю;
\vs 2Ch 35:4 станьте по поколениям вашим, по чередам вашим, как предписано Давидом, царем Израилевым, и как предписано Соломоном, сыном его,
\vs 2Ch 35:5 и стойте во святилище, по распределениям поколений у братьев ваших, сынов народа, и по разделению поколений у левитов,
\vs 2Ch 35:6 и заколите пасхального агнца, и освятитесь, и приготовьте его для братьев ваших, поступая согласно со словом Господним чрез Моисея.
\vs 2Ch 35:7 И дал Иосия в дар сынам народа, всем, находившимся там, из мелкого скота агнцев и козлов молодых, все для жертвы пасхальной, числом тридцать тысяч и три тысячи волов. Это из имущества царя.
\vs 2Ch 35:8 И князья его по усердию давали в дар народу, священникам и левитам: Хелкия и Захария и Иехиил, начальствующие в доме Божием, дали священникам для жертвы пасхальной две тысячи шестьсот [овец, агнцев и козлов] и триста волов;
\vs 2Ch 35:9 и Хонания, и Шемаия, и Нафанаил, братья его, и Хашавия, и Иеиел, и Иозавад, начальники левитов, подарили левитам для жертвы пасхальной [овец] пять тысяч и пятьсот волов.
\rsbpar\vs 2Ch 35:10 Так устроено было служение. И стали священники на место свое и левиты по чередам своим, по повелению царскому;
\vs 2Ch 35:11 и закололи пасхального агнца. И кропили священники \bibemph{кровью}, принимая ее из рук левитов, а левиты снимали кожу;
\vs 2Ch 35:12 и распределили \bibemph{назначенное} для всесожжения, чтобы раздать то по отделениям поколений у сынов народа, для принесения Господу, как написано в книге Моисеевой. То же \bibemph{сделали} и с волами.
\vs 2Ch 35:13 И испекли пасхального агнца на огне, по уставу; и священные жертвы сварили в котлах, горшках и кастрюлях, и поспешно раздали всему народу,
\vs 2Ch 35:14 а после приготовили для себя и для священников, ибо священники, сыны Аароновы, \bibemph{заняты были} приношением всесожжения и туков до ночи; потому-то и готовили левиты для себя и для священников, сынов Аароновых.
\vs 2Ch 35:15 И певцы, сыновья Асафовы, \bibemph{оставались} на местах своих, по установлению Давида и Асафа, и Емана и Идифуна, прозорливца царского, и привратники у каждых ворот: не для чего \bibemph{было} им отходить от служения своего, так как братья их левиты готовили для них.
\vs 2Ch 35:16 Так устроено было все служение Господу в тот день, чтобы совершить пасху и принести всесожжения на жертвеннике Господнем, по повелению царя Иосии.
\vs 2Ch 35:17 И совершали сыны Израилевы, находившиеся \bibemph{там}, пасху в то время и праздник опресноков в течение семи дней.
\vs 2Ch 35:18 И не была совершаема такая пасха у Израиля от дней Самуила пророка; и из всех царей Израилевых ни один не совершал такой пасхи, какую совершил Иосия, и священники, и левиты, и все Иудеи, и Израильтяне, \bibemph{там} находившиеся, и жители Иерусалима.
\vs 2Ch 35:19 В восемнадцатый год царствования Иосии совершена сия пасха.
\rsbpar\vs 2Ch 35:20 После всего того, что сделал Иосия в доме \bibemph{Божием} [и как сжег огнем царь Иосия и чревовещателей, и волхвов, и капища, и идолов, и дубравы, бывшие в Иерусалиме и Иудее, чтобы утвердить слова закона, написанные в книге, которую нашел Хелкия священник в доме Господнем, не было подобного ему прежде него, кто обратился бы к Господу всем сердцем своим, и всею душею своею, и всею крепостию своею, по всему закону Моисееву; не восстал и после него подобный ему. Однако же не отвратился Господь от великой ярости гнева Своего,~--- ярости, которою разгневался Господь на Иудею за все оскорбления, которыми прогневал Манассия. И сказал Господь: и Иуду отвергну от лица Моего, как отверг дом Израилев, и отвергну город Иерусалим, который избрал, и храм, о котором сказал: будет там имя Мое,] пошел Нехао, царь Египетский, на войну к Кархемису на Евфрате; и Иосия вышел навстречу ему.
\vs 2Ch 35:21 И послал к нему \bibemph{Нехао} послов сказать: что мне и тебе, царь Иудейский? Не против тебя теперь \bibemph{иду я}, но туда, где у меня война. И Бог повелел мне поспешать; не противься Богу, Который со мною, чтоб Он не погубил тебя.
\vs 2Ch 35:22 Но Иосия не отстранился от него, а приготовился, чтобы сразиться с ним, и не послушал слов Нехао от лица Божия и выступил на сражение на равнину Мегиддо.
\vs 2Ch 35:23 И выстрелили стрельцы в царя Иосию, и сказал царь слугам своим: уведите меня, потому что я тяжело ранен.
\vs 2Ch 35:24 И свели его слуги его с колесницы, и посадили его в другую повозку, которая \bibemph{была} у него, и отвезли его в Иерусалим. И умер он, и похоронен в гробницах отцов своих. И вся Иудея и Иерусалим оплакали Иосию.
\vs 2Ch 35:25 Оплакал Иосию и Иеремия в песне плачевной; и говорили все певцы и певицы об Иосии в плачевных песнях своих, \bibemph{известных} до сего дня, и передали их в употребление у Израиля; и вот они вписаны в \bibemph{книгу} плачевных песней.
\rsbpar\vs 2Ch 35:26 Прочие деяния Иосии и добродетели его, согласные с предписанным в законе Господнем,
\vs 2Ch 35:27 и деяния его, первые и последние, описаны в книге царей Израильских и Иудейских.
\vs 2Ch 36:1 И взял народ земли Иоахаза, сына Иосиина, [и помазали его] и воцарили его, вместо отца его, в Иерусалиме.
\vs 2Ch 36:2 Двадцати трех лет был Иоахаз, когда воцарился, и три месяца царствовал в Иерусалиме. [Имя матери его~--- Амитал, дочь Иеремии из Ловны. И сделал он лукавое пред Господом по всему, что сделали отцы его. И оковал его фараон Нехао в Девлафе, в земле Емафской, чтобы не царствовать ему в Иерусалиме.]
\vs 2Ch 36:3 И низложил его царь Египетский в Иерусалиме [и привел его царь в Египет], и наложил на землю пени сто талантов серебра и талант золота.
\vs 2Ch 36:4 И воцарил царь Египетский над Иудеею и Иерусалимом Елиакима, брата его, и переменил имя его на Иоакима, а Иоахаза, брата его, взял Нехао и отвел его в Египет [и он умер там. И серебро и золото давал фараону: тогда земля начала давать серебро по слову фараона, и каждый, по власти, требовал серебра и золота от народа земли для дани фараону Нехао].
\rsbpar\vs 2Ch 36:5 Двадцати пяти лет \bibemph{был} Иоаким, когда воцарился, и одиннадцать лет царствовал в Иерусалиме [имя матери его Зехора, дочь Нириева из Рамы]. И делал он неугодное в очах Господа Бога своего [по всему, что делали отцы его. Во дни его пришел Навуходоносор, царь Вавилонский, на землю, и он служил ему три года и отступил от него. И послал Господь на них Халдеев и разбойников Сирских, и разбойников Моавитских, и сынов Аммоновых и Самарийских, и отступили по слову сему,~--- по слову Господа устами рабов Его, пророков. Впрочем гнев Господа был на Иуде, чтоб отвергнуть его от лица Его, за все грехи Манассии, которые он сделал, и за кровь неповинную, которую пролил Иоаким и наполнил Иерусалим неповинною кровью. Но не восхотел Господь искоренить их].
\vs 2Ch 36:6 Против него вышел Навуходоносор, царь Вавилонский, и оковал его оковами, чтоб отвести его в Вавилон.
\vs 2Ch 36:7 И часть сосудов дома Господня перенес Навуходоносор в Вавилон и положил их в капище своем в Вавилоне.
\rsbpar\vs 2Ch 36:8 Прочие дела Иоакима и мерзости его, какие он делал и какие найдены в нем, описаны в книге царей Израильских и Иудейских. [И почил Иоаким с отцами своими, и погребен был в Ганозане с отцами своими.] И воцарился Иехония, сын его, вместо него.
\rsbpar\vs 2Ch 36:9 Восемнадцати лет \bibemph{был} Иехония, когда воцарился, и три месяца и десять дней царствовал в Иерусалиме, и делал он неугодное в очах Господних.
\vs 2Ch 36:10 По прошествии года послал царь Навуходоносор и велел взять его в Вавилон вместе с драгоценными сосудами дома Господня, и воцарил над Иудеею и Иерусалимом Седекию, брата его.
\rsbpar\vs 2Ch 36:11 Двадцати одного года \bibemph{был} Седекия, когда воцарился, и одиннадцать лет царствовал в Иерусалиме,
\vs 2Ch 36:12 и делал он неугодное в очах Господа Бога своего. Он не смирился пред Иеремиею пророком, \bibemph{пророчествовавшим} от уст Господних,
\vs 2Ch 36:13 и отложился от царя Навуходоносора, взявшего клятву с него \bibemph{именем} Бога,~--- и сделал упругою шею свою и ожесточил сердце свое до того, что не обратился к Господу Богу Израилеву.
\vs 2Ch 36:14 Да и все начальствующие над священниками и над народом много грешили, подражая всем мерзостям язычников, и сквернили дом Господа, который Он освятил в Иерусалиме.
\vs 2Ch 36:15 И посылал к ним Господь Бог отцов их, посланников Своих от раннего утра, потому что Он жалел Свой народ и Свое жилище.
\vs 2Ch 36:16 Но они издевались над посланными от Бога и пренебрегали словами Его, и ругались над пророками Его, доколе не сошел гнев Господа на народ Его, так что не было \bibemph{ему} спасения.
\vs 2Ch 36:17 И Он навел на них царя Халдейского,~--- и тот умертвил юношей их мечом в доме святыни их и не пощадил [ни Седекии,] ни юноши, ни девицы, ни старца, ни седовласого: все предал \bibemph{Бог} в руку его.
\vs 2Ch 36:18 И все сосуды дома Божия, большие и малые, и сокровища дома Господня, и сокровища царя и князей его, все принес он в Вавилон.
\vs 2Ch 36:19 И сожгли дом Божий, и разрушили стену Иерусалима, и все чертоги его сожгли огнем, и все драгоценности его истребили.
\vs 2Ch 36:20 И переселил он оставшихся от меча в Вавилон, и были они рабами его и сыновей его, до воцарения царя Персидского,
\vs 2Ch 36:21 доколе, во исполнение слова Господня, \bibemph{сказанного} устами Иеремии, земля не отпраздновала суббот своих. Во все дни запустения она субботствовала до исполнения семидесяти лет.
\rsbpar\vs 2Ch 36:22 А в первый год Кира, царя Персидского, во исполнение слова Господня, \bibemph{сказанного} устами Иеремии, возбудил Господь дух Кира, царя Персидского, и он велел объявить по всему царству своему, словесно и письменно, и сказать:
\vs 2Ch 36:23 так говорит Кир, царь Персидский: все царства земли дал мне Господь Бог небесный, и Он повелел мне построить Ему дом в Иерусалиме, что в Иудее. Кто есть из вас~--- из всего народа Его, [да будет] Господь Бог его с ним, и пусть он туда идет.
\chhdr{Молитва Манассии, царя Иудейского, когда он содержался в плену в Вавилоне.\fns{Переведена с греческого; в еврейском тексте ее нет.}}
\vs 2Ch 37:1 Господи Вседержителю, Боже отцев наших, Авраама и Исаака и Иакова, и семени их праведного,
\vs 2Ch 37:2 сотворивший небо и землю со всем благолепием их, связавший море словом повеления Твоего, заключивший бездну и запечатавший ее страшным и славным именем Твоим, которого все боятся, и трепещут от лица силы Твоея, потому что никто не может устоять пред великолепием славы Твоея, и нестерпим гнев
\vs 2Ch 37:3 прещения Твоего на грешников!
\vs 2Ch 37:4 Но безмерна и неисследима милость обетования Твоего,
\vs 2Ch 37:5 ибо Ты Господь вышний, благий, долготерпеливый и многомилостивый и кающийся о злобах человеческих. Ты, Господи, по множеству Твоей благости, обещал покаяние
\vs 2Ch 37:6 и отпущение согрешившим Тебе, и множеством щедрот Твоих определил покаяние грешникам во спасение. Итак Ты, Господи, Боже праведных, не положил покаяния праведным
\vs 2Ch 37:7 Аврааму и Исааку и Иакову, не согрешившим Тебе, но положил покаяние мне грешнику, потому что я согрешил паче числа песка морского.
\vs 2Ch 37:8 Многочисленны беззакония мои, Господи, многочисленны беззакония мои, и я недостоин взирать и смотреть на высоту небесную от множества неправд моих. Я согбен многими железными узами,
\vs 2Ch 37:9 так что не могу поднять головы моей, и нет мне отдохновения, потому что прогневал Тебя и сделал пред Тобою злое:
\vs 2Ch 37:10 не исполнил воли Твоей, не сохранил повелений Твоих, поставил мерзости и умножил соблазны. И ныне преклоняю колени сердца моего, умоляя Тебя о благости.
\vs 2Ch 37:11 Согрешил я, Господи, согрешил, и беззакония мои я знаю, но прошу, молясь Тебе: отпусти мне, Господи, отпусти мне, и не погуби меня с беззакониями моими и не осуди меня в преисподнюю. Ибо Ты Бог, Бог кающихся, и на мне яви всю благость Твою, спасши меня недостойного по великой милости Твоей, и буду прославлять Тебя во все дни жизни моей,
\vs 2Ch 37:12 потому что Тебя славят все силы небесные, и Твоя слава во веки веков. Аминь.
\newbookpage
\bibbookdescr{Ezr}{
  inline={\LARGE Первая книга\\\Huge Ездры},
  toc={1-я Ездры},
  bookmark={1-я Ездры},
  header={1-я Ездры},
  %headerleft={},
  %headerright={},
  abbr={1~Езд}
}
\vs Ezr 1:1 В первый год Кира, царя Персидского, во исполнение слова Господня из уст Иеремии, возбудил Господь дух Кира, царя Персидского, и он повелел объявить по всему царству своему, словесно и письменно:
\vs Ezr 1:2 так говорит Кир, царь Персидский: все царства земли дал мне Господь Бог небесный, и Он повелел мне построить Ему дом в Иерусалиме, что в Иудее.
\vs Ezr 1:3 Кто есть из вас, из всего народа Его,~--- да будет Бог его с ним,~--- и пусть он идет в Иерусалим, что в Иудее, и строит дом Господа Бога Израилева, Того Бога, Который в Иерусалиме.
\vs Ezr 1:4 А все оставшиеся во всех местах, где бы тот ни жил, пусть помогут ему жители места того серебром и золотом и \bibemph{иным} имуществом, и скотом, с доброхотным даянием для дома Божия, что в Иерусалиме.
\rsbpar\vs Ezr 1:5 И поднялись главы поколений Иудиных и Вениаминовых, и священники и левиты, всякий, \bibemph{в ком} возбудил Бог дух его, чтобы пойти строить дом Господень, который в Иерусалиме.
\vs Ezr 1:6 И все соседи их вспомоществовали им серебряными сосудами, золотом, \bibemph{иным} имуществом, и скотом, и дорогими вещами, сверх всякого доброхотного даяния \bibemph{для храма}.
\rsbpar\vs Ezr 1:7 И царь Кир вынес сосуды дома Господня, которые Навуходоносор взял из Иерусалима и положил в доме бога своего,~---
\vs Ezr 1:8 и вынес их Кир, царь Персидский, рукою Мифредата сокровищехранителя, а он счетом сдал их Шешбацару князю Иудину.
\vs Ezr 1:9 И вот число их: блюд золотых тридцать, блюд серебряных тысяча, ножей двадцать девять,
\vs Ezr 1:10 чаш золотых тридцать, чаш серебряных двойных четыреста десять, других сосудов тысяча:
\vs Ezr 1:11 всех сосудов, золотых и серебряных, пять тысяч четыреста. Все \bibemph{это} взял \bibemph{с собою} Шешбацар, при отправлении переселенцев из Вавилона в Иерусалим.
\vs Ezr 2:1 Вот сыны страны из пленников переселения, которых Навуходоносор, царь Вавилонский, отвел в Вавилон, возвратившиеся в Иерусалим и Иудею, каждый в свой город,~---
\vs Ezr 2:2 пришедшие с Зоровавелем, Иисусом, Неемиею, Сараием, Реелаем, Мардохеем, Билшаном, Мисфаром, Бигваем, Рехумом, Вааном. Число людей народа Израилева:
\vs Ezr 2:3 сыновей Пароша две тысячи сто семьдесят два;
\vs Ezr 2:4 сыновей Сафатии триста семьдесят два;
\vs Ezr 2:5 сыновей Араха семьсот семьдесят пять;
\vs Ezr 2:6 сыновей Пахаф-Моава, из сыновей Иисуса [и] Иоава, две тысячи восемьсот двенадцать;
\vs Ezr 2:7 сыновей Елама тысяча двести пятьдесят четыре;
\vs Ezr 2:8 сыновей Заттуя девятьсот сорок пять;
\vs Ezr 2:9 сыновей Закхая семьсот шестьдесят;
\vs Ezr 2:10 сыновей Вания шестьсот сорок два;
\vs Ezr 2:11 сыновей Бебая шестьсот двадцать три;
\vs Ezr 2:12 сыновей Азгада тысяча двести двадцать два;
\vs Ezr 2:13 сыновей Адоникама шестьсот шестьдесят шесть;
\vs Ezr 2:14 сыновей Бигвая две тысячи пятьдесят шесть;
\vs Ezr 2:15 сыновей Адина четыреста пятьдесят четыре;
\vs Ezr 2:16 сыновей Атера, из \bibemph{дома} Езекии, девяносто восемь;
\vs Ezr 2:17 сыновей Бецая триста двадцать три;
\vs Ezr 2:18 сыновей Иоры сто двенадцать;
\vs Ezr 2:19 сыновей Хашума двести двадцать три;
\vs Ezr 2:20 сыновей Гиббара девяносто пять;
\vs Ezr 2:21 уроженцев Вифлеема сто двадцать три;
\vs Ezr 2:22 жителей Нетофы пятьдесят шесть;
\vs Ezr 2:23 жителей Анафофа сто двадцать восемь;
\vs Ezr 2:24 уроженцев Азмавефа сорок два;
\vs Ezr 2:25 уроженцев Кириаф-Иарима, Кефиры и Беерофа семьсот сорок три;
\vs Ezr 2:26 уроженцев Рамы и Гевы шестьсот двадцать один;
\vs Ezr 2:27 жителей Михмаса сто двадцать два;
\vs Ezr 2:28 жителей Вефиля и Гая двести двадцать три;
\vs Ezr 2:29 уроженцев Нево пятьдесят два;
\vs Ezr 2:30 уроженцев Магбиша сто пятьдесят шесть;
\vs Ezr 2:31 сыновей другого Елама тысяча двести пятьдесят четыре;
\vs Ezr 2:32 сыновей Харима триста двадцать;
\vs Ezr 2:33 уроженцев Лидды, Хадида и Оно семьсот двадцать пять;
\vs Ezr 2:34 уроженцев Иерихона триста сорок пять;
\vs Ezr 2:35 уроженцев Сенаи три тысячи шестьсот тридцать.
\rsbpar\vs Ezr 2:36 Священников: сыновей Иедаии, из дома Иисусова, девятьсот семьдесят три;
\vs Ezr 2:37 сыновей Иммера тысяча пятьдесят два;
\vs Ezr 2:38 сыновей Пашхура тысяча двести сорок семь;
\vs Ezr 2:39 сыновей Харима тысяча семнадцать.
\rsbpar\vs Ezr 2:40 Левитов: сыновей Иисуса и Кадмиила, из сыновей Годавии, семьдесят четыре;
\vs Ezr 2:41 певцов: сыновей Асафа сто двадцать восемь;
\vs Ezr 2:42 сыновей привратников: сыновья Шаллума, сыновья Атера, сыновья Талмона, сыновья Аккува, сыновья Хатиты, сыновья Шовая,~--- всего сто тридцать девять.
\rsbpar\vs Ezr 2:43 Нефинеев: сыновья Цихи, сыновья Хасуфы, сыновья Таббаофа,
\vs Ezr 2:44 сыновья Кероса, сыновья Сиаги, сыновья Фадона,
\vs Ezr 2:45 сыновья Лебаны, сыновья Хагабы, сыновья Аккува,
\vs Ezr 2:46 сыновья Хагава, сыновья Шамлая, сыновья Ханана,
\vs Ezr 2:47 сыновья Гиддела, сыновья Гахара, сыновья Реаии,
\vs Ezr 2:48 сыновья Рецина, сыновья Некоды, сыновья Газзама,
\vs Ezr 2:49 сыновья Уззы, сыновья Пасеаха, сыновья Бесая,
\vs Ezr 2:50 сыновья Асны, сыновья Меунима, сыновья Нефисима,
\vs Ezr 2:51 сыновья Бакбука, сыновья Хакуфы, сыновья Хархура,
\vs Ezr 2:52 сыновья Бацлуфа, сыновья Мехиды, сыновья Харши,
\vs Ezr 2:53 сыновья Баркоса, сыновья Сисры, сыновья Фамаха,
\vs Ezr 2:54 сыновья Нециаха, сыновья Хатифы;
\vs Ezr 2:55 сыновья рабов Соломоновых: сыновья Сотая, сыновья Гассоферефа, сыновья Феруды,
\vs Ezr 2:56 сыновья Иаалы, сыновья Даркона, сыновья Гиддела,
\vs Ezr 2:57 сыновья Сефатии, сыновья Хаттила, сыновья Похереф-Гаццебайима, сыновья Амия,~---
\vs Ezr 2:58 всего~--- нефинеев и сыновей рабов Соломоновых триста девяносто два.
\rsbpar\vs Ezr 2:59 И вот вышедшие из Тел-Мелаха, Тел-Харши, Херуб-Аддан-Иммера, которые не могли показать о поколении своем и о племени своем~--- от Израиля ли они:
\vs Ezr 2:60 сыновья Делайи, сыновья Товии, сыновья Некоды, шестьсот пятьдесят два.
\vs Ezr 2:61 И из сыновей священнических: сыновья Хабайи, сыновья Гаккоца, сыновья Верзеллия, который взял жену из дочерей Верзеллия Галаадитянина и стал называться именем их.
\vs Ezr 2:62 Они искали своей записи родословной, и не нашлось ее, а \bibemph{потому} исключены из священства.
\vs Ezr 2:63 И Тиршафа сказал им, чтоб они не ели великой святыни, доколе не восстанет священник с уримом и туммимом.
\rsbpar\vs Ezr 2:64 Все общество вместе \bibemph{состояло} из сорока двух тысяч трехсот шестидесяти \bibemph{человек},
\vs Ezr 2:65 кроме рабов их и рабынь их, которых \bibemph{было} семь тысяч триста тридцать семь; и при них певцов и певиц двести.
\vs Ezr 2:66 Коней у них семьсот тридцать шесть, лошаков у них двести сорок пять;
\vs Ezr 2:67 верблюдов у них четыреста тридцать пять, ослов шесть тысяч семьсот двадцать.
\rsbpar\vs Ezr 2:68 Из глав поколений \bibemph{некоторые}, придя к дому Господню, что в Иерусалиме, доброхотно жертвовали на дом Божий, чтобы восстановить его на основании его.
\vs Ezr 2:69 По достатку своему, они дали в сокровищницу на \bibemph{производство} работ шестьдесят одну тысячу драхм золота и пять тысяч мин серебра и сто священнических одежд.
\rsbpar\vs Ezr 2:70 И стали жить священники и левиты, и народ и певцы, и привратники и нефинеи в городах своих, и весь Израиль в городах своих.
\vs Ezr 3:1 Когда наступил седьмой месяц, и сыны Израилевы \bibemph{уже были} в городах, тогда собрался народ, как один человек, в Иерусалиме.
\vs Ezr 3:2 И встал Иисус, сын Иоседеков, и братья его священники, и Зоровавель, сын Салафиилов, и братья его, и соорудили они жертвенник Богу Израилеву, чтобы возносить на нем всесожжения, как написано в законе Моисея, человека Божия.
\vs Ezr 3:3 И поставили жертвенник на основании его, так как они \bibemph{были} в страхе от иноземных народов; и стали возносить на нем всесожжения Господу, всесожжения утренние и вечерние.
\vs Ezr 3:4 И совершили праздник кущей, как предписано, с ежедневным всесожжением в определенном числе, по уставу \bibemph{каждого} дня.
\vs Ezr 3:5 И после того \bibemph{совершали} всесожжение постоянное, и в новомесячия, и во все праздники, посвященные Господу, и добровольное приношение Господу от всякого усердствующего.
\rsbpar\vs Ezr 3:6 С первого же дня седьмого месяца начали возносить всесожжения Господу. А храму Господню \bibemph{еще} не было положено основание.
\vs Ezr 3:7 И стали выдавать серебро каменотесам и плотникам, и пищу и питье и масло Сидонянам и Тирянам, чтоб они доставляли кедровый лес с Ливана по морю в Яфу, с дозволения им Кира, царя Персидского.
\rsbpar\vs Ezr 3:8 Во второй год по приходе своем к дому Божию в Иерусалим, во второй месяц Зоровавель, сын Салафиилов, и Иисус, сын Иоседеков, и прочие братья их, священники и левиты, и все пришедшие из плена в Иерусалим положили начало и поставили левитов от двадцати лет и выше для надзора за работами дома Господня.
\vs Ezr 3:9 И стали Иисус, сыновья его и братья его, Кадмиил и сыновья его, сыновья Иуды, как один \bibemph{человек}, для надзора за производителями работ в доме Божием, \bibemph{а также и} сыновья Хенадада, сыновья их и братья их левиты.
\rsbpar\vs Ezr 3:10 Когда строители положили основание храму Господню, тогда поставили священников в облачении их с трубами и левитов, сыновей Асафовых, с кимвалами, чтобы славить Господа по уставу Давида, царя Израилева.
\vs Ezr 3:11 И начали они попеременно петь: <<хвалите>> и: <<славьте Господа>>, <<ибо~--- благ, ибо вовек милость Его к Израилю>>. И весь народ восклицал громогласно, славя Господа за то, что положено основание дома Господня.
\vs Ezr 3:12 Впрочем многие из священников и левитов и глав поколений, старики, которые видели прежний храм, при основании этого храма пред глазами их, плакали громко, но многие и восклицали от радости громогласно.
\vs Ezr 3:13 И не мог народ распознать восклицаний радости от воплей плача народного, потому что народ восклицал громко, и голос слышен был далеко.
\vs Ezr 4:1 И услышали враги Иуды и Вениамина, что возвратившиеся из плена строят храм Господу Богу Израилеву;
\vs Ezr 4:2 и пришли они к Зоровавелю и к главам поколений, и сказали им: будем и мы строить с вами, потому что мы, как и вы, прибегаем к Богу вашему, и Ему приносим жертвы от дней Асардана, царя Сирийского, который перевел нас сюда.
\vs Ezr 4:3 И сказал им Зоровавель и Иисус и прочие главы поколений Израильских: не строить вам вместе с нами дом нашему Богу; мы одни будем строить \bibemph{дом} Господу, Богу Израилеву, как повелел нам царь Кир, царь Персидский.
\vs Ezr 4:4 И стал народ земли той ослаблять руки народа Иудейского и препятствовать ему в строении;
\vs Ezr 4:5 и подкупали против них советников, чтобы разрушить предприятие их, во все дни Кира, царя Персидского, и до царствования Дария, царя Персидского.
\rsbpar\vs Ezr 4:6 А в царствование Ахашвероша, в начале царствования его, написали обвинение на жителей Иудеи и Иерусалима.
\vs Ezr 4:7 И во дни Артаксеркса писали Бишлам, Мифредат, Табеел и прочие товарищи их к Артаксерксу, царю Персидскому. Письмо же написано \bibemph{было} буквами Сирийскими и на Сирийском языке.
\vs Ezr 4:8 Рехум советник и Шимшай писец писали одно письмо против Иерусалима к царю Артаксерксу такое:
\vs Ezr 4:9 Тогда-то. Рехум советник и Шимшай писец и прочие товарищи их,~--- Динеи и Афарсафхеи, Тарпелеи, Апарсы, Арехьяне, Вавилоняне, Сусанцы, Даги, Еламитяне,
\vs Ezr 4:10 и прочие народы, которых переселил Аснафар [Сеннахирим] великий и славный и поселил в городах Самарийских и в прочих \bibemph{городах} за рекою, и прочее.
\vs Ezr 4:11 И вот список с письма, которое послали к нему: Царю Артаксерксу~--- рабы твои, люди, \bibemph{живущие} за рекою, и прочее.
\vs Ezr 4:12 Да будет известно царю, что Иудеи, которые вышли от тебя, пришли к нам в Иерусалим, строят \bibemph{этот} мятежный и негодный город, и стены делают, и основания \bibemph{их уже} исправили.
\vs Ezr 4:13 Да будет же известно царю, что если этот город будет построен и стены восстановлены, то \bibemph{ни} подати, \bibemph{ни} налога, ни пошлины не будут давать, и царской казне сделан будет ущерб.
\vs Ezr 4:14 Так как мы едим соль от дворца царского, и ущерб для царя не можем видеть, поэтому мы посылаем донесение к царю:
\vs Ezr 4:15 пусть поищут в памятной книге отцов твоих,~--- и найдешь в книге памятной, и узнаешь, что город сей~--- город мятежный и вредный для царей и областей, и \bibemph{что} отпадения бывали в нем издавна, за что город сей и опустошен.
\vs Ezr 4:16 Посему мы уведомляем царя, что если город сей будет достроен и стены его доделаны, то после этого не будет у тебя владения за рекою.
\rsbpar\vs Ezr 4:17 Царь послал ответ Рехуму советнику и Шимшаю писцу и прочим товарищам их, которые живут в Самарии и \bibemph{в} прочих \bibemph{городах} заречных: Мир\dots\ и прочее.
\vs Ezr 4:18 Письмо, которое вы прислали нам, внятно прочитано предо мною;
\vs Ezr 4:19 и от меня дано повеление,~--- и разыскивали, и нашли, что город этот издавна восставал против царей, и производились в нем мятежи и волнения,
\vs Ezr 4:20 и \bibemph{что были} в Иерусалиме цари могущественные и владевшие всем заречьем, и им давали подать, налоги и пошлины.
\vs Ezr 4:21 Итак дайте приказание, чтобы люди сии перестали работать, и \bibemph{чтобы} город сей не строился, доколе от меня не будет дано повеление.
\vs Ezr 4:22 И будьте осторожны, чтобы не сделать в этом недосмотра. К чему допускать размножение вредного в ущерб царям?
\rsbpar\vs Ezr 4:23 Как скоро это письмо царя Артаксеркса было прочитано пред Рехумом и Шимшаем писцом и товарищами их, они немедленно пошли в Иерусалим к Иудеям, и сильною вооруженною рукою остановили работу их.
\vs Ezr 4:24 Тогда остановилась работа при доме Божием, который в Иерусалиме, и остановка сия продолжалась до второго года царствования Дария, царя Персидского.
\vs Ezr 5:1 Но пророк Аггей и пророк Захария, сын Адды, говорили Иудеям, которые в Иудее и Иерусалиме, пророческие речи во имя Бога Израилева.
\vs Ezr 5:2 Тогда встали Зоровавель, сын Салафиилов, и Иисус, сын Иоседеков, и начали строить дом Божий в Иерусалиме, и с ними пророки Божии, подкреплявшие их.
\rsbpar\vs Ezr 5:3 В то время пришел к ним Фафнай, заречный областеначальник, и Шефар-Бознай и товарищи их, и так сказали им: кто дал вам разрешение строить дом сей и доделывать стены сии?
\vs Ezr 5:4 Тогда мы сказали им имена тех людей, которые строят это здание.
\vs Ezr 5:5 Но око Бога их было над старейшинами Иудейскими, и те не возбраняли им, доколе дело не отправили к Дарию, и доколе не пришло решение по этому делу.
\rsbpar\vs Ezr 5:6 Вот содержание письма, которое послал Фафнай, заречный областеначальник, и Шефар-Бознай с товарищами своими~--- Афарсахеями, которые за рекою, к царю Дарию.
\vs Ezr 5:7 В донесении, которое они послали к нему, вот что написано: Дарию царю~--- всякий мир!
\vs Ezr 5:8 Да будет известно царю, что мы ходили в Иудейскую область, к дому Бога великого; и строится он из больших камней, и дерево вкладывается в стены; и работа сия производится быстро и с успехом идет в руках их.
\vs Ezr 5:9 Тогда мы спросили у старейшин тех и так сказали им: кто дал вам разрешение строить дом сей и стены сии доделывать?
\vs Ezr 5:10 И сверх того об именах их мы спросили их, чтобы дать знать тебе и написать имена тех людей, которые главными у них.
\vs Ezr 5:11 И они ответили нам такими словами: мы рабы Бога неба и земли и строим дом, который был построен за много лет прежде сего,~--- и великий царь у Израиля строил его и довершил его.
\vs Ezr 5:12 Когда же отцы наши прогневали Бога небесного, Он предал их в руку Навуходоносора, царя Вавилонского, Халдеянина; и дом сей он разрушил, и народ переселил в Вавилон.
\vs Ezr 5:13 Но в первый год Кира, царя Вавилонского, царь Кир дал разрешение построить сей дом Божий;
\vs Ezr 5:14 да и сосуды дома Божия, золотые и серебряные, которые Навуходоносор вынес из храма Иерусалимского и отнес в храм Вавилонский,~--- вынес Кир царь из храма Вавилонского; и отдали \bibemph{их} по имени Шешбацару, которого он назначил областеначальником,
\vs Ezr 5:15 и сказал ему: возьми сии сосуды, пойди и отнеси их в храм Иерусалимский, и пусть дом Божий строится на своем месте.
\vs Ezr 5:16 Тогда Шешбацар тот пришел, положил основания дома Божия в Иерусалиме; и с тех пор доселе он строится, и еще не кончен.
\vs Ezr 5:17 Итак, если царю благоугодно, пусть поищут в доме царских сокровищ, там в Вавилоне, точно ли царем Киром дано разрешение строить сей дом Божий в Иерусалиме, и царскую волю о сем пусть пришлют к нам.
\vs Ezr 6:1 Тогда царь Дарий дал повеление, и разыскивали в Вавилоне в книгохранилище, куда полагали сокровища.
\vs Ezr 6:2 И найден в Екбатане во дворце, который в области Мидии, один свиток, и в нем написано так: <<Для памяти:
\vs Ezr 6:3 в первый год царя Кира, царь Кир дал повеление о доме Божием в Иерусалиме: пусть строится дом на том месте, где приносят жертвы, и пусть будут положены прочные основания для него; вышина его в шестьдесят локтей, ширина его в шестьдесят локтей;
\vs Ezr 6:4 рядов из камней больших три, и ряд из дерева один; издержки же пусть выдаются из царского дома.
\vs Ezr 6:5 Да и сосуды дома Божия, золотые и серебряные, которые Навуходоносор вынес из храма Иерусалимского и отнес в Вавилон, пусть возвратятся и пойдут в храм Иерусалимский, \bibemph{каждый} на место свое, и помещены будут в доме Божием.
\vs Ezr 6:6 Итак, Фафнай, заречный областеначальник, и Шефар-Бознай, с товарищами вашими Афарсахеями, которые за рекою,~--- удалитесь оттуда.
\vs Ezr 6:7 Не останавливайте работы при сем доме Божием; пусть Иудейский областеначальник и Иудейские старейшины строят сей дом Божий на месте его.
\vs Ezr 6:8 И от меня дается повеление о том, чем вы должны содействовать старейшинам тем Иудейским в построении того дома Божия, и \bibemph{именно}: из имущества царского~--- \bibemph{из} заречной подати~--- немедленно берите и давайте тем людям, чтобы работа не останавливалась;
\vs Ezr 6:9 и сколько нужно~--- тельцов ли, или овнов и агнцев, на всесожжения Богу небесному, также пшеницы, соли, вина и масла, как скажут священники Иерусалимские, пусть будет выдаваемо им изо дня в день без задержки,
\vs Ezr 6:10 чтоб они приносили жертву приятную Богу небесному и молились о жизни царя и сыновей его.
\vs Ezr 6:11 Мною же дается повеление, что \bibemph{если} какой человек изменит это определение, то будет вынуто бревно из дома его, и будет поднят он и пригвожден к нему, а дом его за то будет обращен в развалины.
\vs Ezr 6:12 И Бог, Которого имя там обитает, да низложит всякого царя и народ, который простер бы руку свою, чтобы изменить \bibemph{сие} ко вреду этого дома Божия в Иерусалиме. Я, Дарий, дал это повеление; да будет оно в точности исполняемо>>.
\rsbpar\vs Ezr 6:13 Тогда Фафнай, заречный областеначальник, Шефар-Бознай и товарищи их,~--- как повелел царь Дарий, так в точности и делали.
\vs Ezr 6:14 И старейшины Иудейские строили и преуспевали, по пророчеству Аггея пророка и Захарии, сына Адды. И построили и окончили, по воле Бога Израилева и по воле Кира и Дария и Артаксеркса, царей Персидских.
\vs Ezr 6:15 И окончен дом сей к третьему дню месяца Адара, в шестой год царствования царя Дария.
\vs Ezr 6:16 И совершили сыны Израилевы, священники и левиты и прочие, возвратившиеся из плена, освящение сего дома Божия с радостью.
\vs Ezr 6:17 И принесли при освящении сего дома Божия: сто волов, двести овнов, четыреста агнцев и двенадцать козлов в жертву за грех за всего Израиля, по числу колен Израилевых.
\vs Ezr 6:18 И поставили священников по отделениям их, и левитов по чередам их на службу Божию в Иерусалиме, как предписано в книге Моисея.
\vs Ezr 6:19 И совершили возвратившиеся из плена пасху в четырнадцатый день первого месяца,
\vs Ezr 6:20 потому что очистились священники и левиты,~--- все они, как один, \bibemph{были} чисты; и закололи агнцев пасхальных для всех, возвратившихся из плена, для братьев своих священников и для себя.
\vs Ezr 6:21 И ели сыны Израилевы, возвратившиеся из переселения, и все отделившиеся к ним от нечистоты народов земли, чтобы прибегать к Господу Богу Израилеву.
\vs Ezr 6:22 И праздновали праздник опресноков семь дней в радости, потому что обрадовал их Господь и обратил к ним сердце царя Ассирийского, чтобы подкреплять руки их при строении дома Господа Бога Израилева.
\vs Ezr 7:1 После сих происшествий, в царствование Артаксеркса, царя Персидского, Ездра, сын Сераии, сын Азарии, сын Хелкии,
\vs Ezr 7:2 сын Шаллума, сын Садока, сын Ахитува,
\vs Ezr 7:3 сын Амарии, сын Азарии, сын Марайофа,
\vs Ezr 7:4 сын Захарии, сын Уззия, сын Буккия,
\vs Ezr 7:5 сын Авишуя, сын Финееса, сын Елеазара, сын Аарона первосвященника,~---
\vs Ezr 7:6 сей Ездра вышел из Вавилона. Он был книжник, сведущий в законе Моисеевом, который дал Господь Бог Израилев. И дал ему царь все по желанию его, так как рука Господа Бога его \bibemph{была} над ним.
\vs Ezr 7:7 \bibemph{С ним} пошли в Иерусалим и \bibemph{некоторые} из сынов Израилевых, и из священников и левитов, и певцов и привратников и нефинеев в седьмой год царя Артаксеркса.
\vs Ezr 7:8 И пришел он в Иерусалим в пятый месяц,~--- в седьмой же год царя.
\vs Ezr 7:9 Ибо в первый день первого месяца \bibemph{было} начало выхода из Вавилона, и в первый день пятого месяца он пришел в Иерусалим, так как благодеющая рука Бога его была над ним,
\vs Ezr 7:10 потому что Ездра расположил сердце свое к тому, чтобы изучать закон Господень и исполнять \bibemph{его}, и учить в Израиле закону и правде.
\rsbpar\vs Ezr 7:11 И вот содержание письма, которое дал царь Артаксеркс Ездре священнику, книжнику, учившему словам заповедей Господа и законов Его в Израиле:
\vs Ezr 7:12 Артаксеркс, царь царей, Ездре священнику, учителю закона Бога небесного совершенному, и прочее.
\vs Ezr 7:13 От меня дано повеление, чтобы в царстве моем всякий из народа Израилева и из священников его и левитов, желающий идти в Иерусалим, шел с тобою.
\vs Ezr 7:14 Так как ты посылаешься от царя и семи советников его, чтобы обозреть Иудею и Иерусалим по закону Бога твоего, находящемуся в руке твоей,
\vs Ezr 7:15 и чтобы доставить серебро и золото, которое царь и советники его пожертвовали Богу Израилеву, Которого жилище в Иерусалиме,
\vs Ezr 7:16 и все серебро и золото, которое ты соберешь во всей области Вавилонской, вместе с доброхотными даяниями от народа и священников, которые пожертвуют они для дома Бога своего, что в Иерусалиме;
\vs Ezr 7:17 поэтому немедленно купи на эти деньги волов, овнов, агнцев и хлебных приношений к ним и возлияний для них, и принеси их на жертвенник дома Бога вашего в Иерусалиме.
\vs Ezr 7:18 И что тебе и братьям твоим заблагорассудится сделать из остального серебра и золота, то по воле Бога вашего делайте.
\vs Ezr 7:19 И сосуды, которые даны тебе для служб \bibemph{в} доме Бога твоего, поставь пред Богом Иерусалимским.
\vs Ezr 7:20 И прочее потребное для дома Бога твоего, что ты признаешь нужным, давай из дома царских сокровищ.
\vs Ezr 7:21 И от меня царя Артаксеркса дается повеление всем сокровищехранителям, которые за рекою: все, чего потребует у вас Ездра священник, учитель закона Бога небесного, немедленно давайте:
\vs Ezr 7:22 серебра до ста талантов, и пшеницы до ста к\acc{о}ров, и вина до ста б\acc{а}тов, и до ста же батов масла, а соли без обозначения \bibemph{количества}.
\vs Ezr 7:23 Все, что повелено Богом небесным, должно делаться со тщанием для дома Бога небесного; [смотрите, чтобы кто не простер руки на дом Бога небесного,] дабы не \bibemph{было} гнева \bibemph{Его} на царство, царя и сыновей его.
\vs Ezr 7:24 И даем вам знать, чтобы \bibemph{ни} на кого \bibemph{из} священников или левитов, певцов, привратников, нефинеев и служащих при этом доме Божием, не налагать \bibemph{ни} подати, \bibemph{ни} налога, ни пошлины.
\vs Ezr 7:25 Ты же, Ездра, по премудрости Бога твоего, которая в руке твоей, поставь правителей и судей, чтоб они судили весь народ за рекою,~--- всех знающих законы Бога твоего, а кто не знает, тех учите.
\vs Ezr 7:26 Кто же не будет исполнять закон Бога твоего и закон царя, над тем немедленно пусть производят суд, на смерть ли, или на изгнание, или на денежную пеню, или на заключение в темницу.
\rsbpar\vs Ezr 7:27 Благословен Господь, Бог отцов наших, вложивший в сердце царя~--- украсить дом Господень, который в Иерусалиме,
\vs Ezr 7:28 и склонивший на меня милость царя и советников его, и всех могущественных князей царя! И я ободрился, ибо рука Господа Бога моего \bibemph{была} надо мною, и собрал я глав Израиля, чтоб они пошли со мною.
\vs Ezr 8:1 И вот главы поколений и родословие тех, которые вышли со мною из Вавилона, в царствование царя Артаксеркса:
\vs Ezr 8:2 из сыновей Финееса Гирсон; из сыновей Ифамара Даниил; из сыновей Давида Хаттуш;
\vs Ezr 8:3 из сыновей Шехании, из сыновей Пароша Захария, и с ним по списку родословному сто пятьдесят \bibemph{человек} мужеского пола;
\vs Ezr 8:4 из сыновей Пахаф-Моава Эльегоенай, сын Зерахии, и с ним двести \bibemph{человек} мужеского пола;
\vs Ezr 8:5 из сыновей [Зафоя] Шехания, сын Яхазиила, и с ним триста \bibemph{человек} мужеского пола;
\vs Ezr 8:6 из сыновей Адина Евед, сын Ионафана, и с ним пятьдесят \bibemph{человек} мужеского пола;
\vs Ezr 8:7 из сыновей Елама Иешаия, сын Афалии, и с ним семьдесят \bibemph{человек} мужеского пола;
\vs Ezr 8:8 из сыновей Сафатии Зевадия, сын Михаилов, и с ним восемьдесят \bibemph{человек} мужеского пола;
\vs Ezr 8:9 из сыновей Иоава Овадия, сын Иехиелов, и с ним двести восемнадцать \bibemph{человек} мужеского пола;
\vs Ezr 8:10 из сыновей [Ваания] Шеломиф, сын Иосифии, и с ним сто шестьдесят \bibemph{человек} мужеского пола;
\vs Ezr 8:11 из сыновей Бевая Захария, сын Бевая, и с ним двадцать восемь \bibemph{человек} мужеского пола;
\vs Ezr 8:12 из сыновей Азгада Иоханан, сын Гаккатана, и с ним сто десять \bibemph{человек} мужеского пола;
\vs Ezr 8:13 из сыновей Адоникама последние, и вот имена их: Елифелет, Иеиел и Шемаия, и с ними шестьдесят \bibemph{человек} мужеского пола;
\vs Ezr 8:14 из сыновей Бигвая, Уфай и Заббуд, и с ними семьдесят \bibemph{человек} мужеского пола.
\rsbpar\vs Ezr 8:15 Я собрал их у реки, втекающей в Агаву, и мы простояли там три дня, и когда я осмотрел народ и священников, то из сынов Левия \bibemph{никого} там не нашел.
\vs Ezr 8:16 И послал я позвать Елиезера, Ариэла, Шемаию, и Элнафана, и Иарива, и Элнафана, и Нафана, и Захарию, и Мешуллама~--- главных, и Иоярива и Элнафана~--- ученых;
\vs Ezr 8:17 и дал им поручение к Иддо, главному в местности Касифье, и вложил им в уста, что говорить к Иддо и братьям его, нефинеям в местности Касифье, чтобы они привели к нам служителей для дома Бога нашего.
\vs Ezr 8:18 И привели они к нам, так как благодеющая рука Бога нашего была над нами, человека умного из сыновей Махлия, сына Левиина, сына Израилева, именно Шеревию, и сыновей его и братьев его, восемнадцать \bibemph{человек};
\vs Ezr 8:19 и Хашавию и с ним Иешаию из сыновей Мерариных, братьев его и сыновей их двадцать;
\vs Ezr 8:20 и из нефинеев, которых дал Давид и князья \bibemph{его} на прислугу левитам, двести двадцать нефинеев; все они названы поименно.
\rsbpar\vs Ezr 8:21 И провозгласил я там пост у реки Агавы, чтобы смириться нам пред лицем Бога нашего, просить у Него благополучного пути для себя и для детей наших и для всего имущества нашего,
\vs Ezr 8:22 так как мне стыдно было просить у царя войска и всадников для охранения нашего от врага на пути, ибо мы, говоря с царем, сказали: рука Бога нашего для всех прибегающих к Нему \bibemph{есть} благодеющая, а на всех оставляющих Его~--- могущество Его и гнев Его!
\vs Ezr 8:23 Итак мы постились и просили Бога нашего о сем, и Он услышал нас.
\vs Ezr 8:24 И я отделил из начальствующих над священниками двенадцать \bibemph{человек}: Шеревию, Хашавию и с ними десять из братьев их;
\vs Ezr 8:25 и отдал им весом серебро, и золото, и сосуды,~--- все, пожертвованное \bibemph{для} дома Бога нашего, что пожертвовали царь, и советники его, и князья его, и все Израильтяне, \bibemph{там} находившиеся.
\vs Ezr 8:26 И отдал на руки им весом: серебра~--- шестьсот пятьдесят талантов, и серебряных сосудов на сто талантов, золота~--- сто талантов;
\vs Ezr 8:27 и чаш золотых~--- двадцать, в тысячу драхм, и два сосуда из лучшей блестящей меди, ценимой как золото.
\vs Ezr 8:28 И сказал я им: вы~--- святыня Господу, и сосуды~--- святыня, и серебро и золото~--- доброхотное даяние Господу Богу отцов ваших.
\vs Ezr 8:29 Будьте же бдительны и сберегите \bibemph{это}, доколе весом не сдадите начальствующим над священниками и левитами и главам поколений Израилевых в Иерусалиме, в хранилище при доме Господнем.
\vs Ezr 8:30 И приняли священники и левиты взвешенное серебро, и золото, и сосуды, чтоб отнести в Иерусалим в дом Бога нашего.
\rsbpar\vs Ezr 8:31 И отправились мы от реки Агавы в двенадцатый день первого месяца, чтобы идти в Иерусалим; и рука Бога нашего была над нами, и спасала нас от руки врага и от подстерегающих нас на пути.
\vs Ezr 8:32 И пришли мы в Иерусалим, и пробыли там три дня.
\vs Ezr 8:33 В четвертый день мы сдали весом серебро, и золото, и сосуды в дом Бога нашего, на руки Меремофу, сыну Урии, священнику, и с ним Елеазару, сыну Финеесову, и с ними Иозаваду, сыну Иисусову, и Ноадии, сыну Виннуя, левитам,
\vs Ezr 8:34 все счетом и весом. И все взвешенное записано в то же время.
\vs Ezr 8:35 Пришедшие из плена переселенцы принесли во всесожжение Богу Израилеву двенадцать тельцов из всего Израиля, девяносто шесть овнов, семьдесят семь агнцев и двенадцать козлов в жертву за грех: все это во всесожжение Господу.
\vs Ezr 8:36 И отдали царские повеления царским сатрапам и заречным областеначальникам, и они почтили народ и дом Божий.
\vs Ezr 9:1 По окончании сего, подошли ко мне начальствующие и сказали: народ Израилев и священники и левиты не отделились от народов иноплеменных с мерзостями их, от Хананеев, Хеттеев, Ферезеев, Иевусеев, Аммонитян, Моавитян, Египтян и Аморреев,
\vs Ezr 9:2 потому что взяли дочерей их за себя и за сыновей своих, и смешалось семя святое с народами иноплеменными, и притом рука знатнейших и главнейших была в сем беззаконии первою.
\rsbpar\vs Ezr 9:3 Услышав это слово, я разодрал нижнюю и верхнюю одежду мою и рвал волосы на голове моей и на бороде моей, и сидел печальный.
\vs Ezr 9:4 Тогда собрались ко мне все, убоявшиеся слов Бога Израилева по причине преступления переселенцев, и я сидел в печали до вечерней жертвы.
\vs Ezr 9:5 А во время вечерней жертвы я встал с места сетования моего, и в разодранной нижней и верхней одежде пал на колени мои и простер руки мои к Господу Богу моему
\vs Ezr 9:6 и сказал: Боже мой! стыжусь и боюсь поднять лице мое к Тебе, Боже мой, потому что беззакония наши стали выше головы, и вина наша возросла до небес.
\vs Ezr 9:7 Со дней отцов наших мы в великой вине до сего дня, и за беззакония наши преданы были мы, цари наши, священники наши, в руки царей иноземных, под меч, в плен и на разграбление и на посрамление, как это и ныне.
\vs Ezr 9:8 И вот, по малом времени, даровано нам помилование от Господа Бога нашего, и Он оставил у нас \bibemph{несколько} уцелевших и дал нам утвердиться на месте святыни Его, и просветил глаза наши Бог наш, и дал нам ожить немного в рабстве нашем.
\vs Ezr 9:9 Мы~--- рабы, но и в рабстве нашем не оставил нас Бог наш. И склонил Он к нам милость царей Персидских, чтоб они дали нам ожить, воздвигнуть дом Бога нашего и восстановить \bibemph{его} из развалин его, и дали нам ограждение в Иудее и в Иерусалиме.
\vs Ezr 9:10 И ныне, что скажем мы, Боже наш, после этого? Ибо мы отступили от заповедей Твоих,
\vs Ezr 9:11 которые заповедал Ты чрез рабов Твоих пророков, говоря: земля, в которую идете вы, чтоб овладеть ею, земля нечистая, она осквернена нечистотою иноплеменных народов, их мерзостями, которыми они наполнили ее от края до края в осквернениях своих.
\vs Ezr 9:12 Итак дочерей ваших не выдавайте за сыновей их, и дочерей их не берите за сыновей ваших, и не ищите мира их и блага их во веки, чтобы укрепиться вам и питаться благами земли той и передать ее в наследие сыновьям вашим на веки.
\vs Ezr 9:13 И после всего, постигшего нас за худые дела наши и за великую вину нашу,~--- ибо Ты, Боже наш, пощадил нас не по мере беззакония нашего и дал нам такое избавление,~---
\vs Ezr 9:14 неужели мы опять будем нарушать заповеди Твои и вступать в родство с этими отвратительными народами? Не прогневаешься ли Ты на нас даже до истребления \bibemph{нас}, так что не будет уцелевших и не будет спасения?
\vs Ezr 9:15 Господи Боже Израилев! праведен Ты. Ибо мы остались уцелевшими до сего дня; и вот мы в беззакониях наших пред лицем Твоим, хотя после этого не надлежало бы нам стоять пред лицем Твоим.
\vs Ezr 10:1 Когда \bibemph{так} молился Ездра и исповедовался, плача и повергаясь пред домом Божиим, стеклось к нему весьма большое собрание Израильтян, мужчин и женщин и детей, потому что и народ много плакал.
\vs Ezr 10:2 И отвечал Шехания, сын Иехиила из сыновей Еламовых, и сказал Ездре: мы сделали преступление пред Богом нашим, что взяли \bibemph{себе} жен иноплеменных из народов земли, но есть еще надежда для Израиля в этом деле;
\vs Ezr 10:3 заключим теперь завет с Богом нашим, что, по совету господина моего и благоговеющих пред заповедями Бога нашего, мы отпустим \bibemph{от себя} всех жен и \bibemph{детей}, рожденных ими,~--- и да будет по закону!
\vs Ezr 10:4 Встань, потому что это твое дело, и мы с тобою: ободрись и действуй!
\vs Ezr 10:5 И встал Ездра, и велел начальствующим над священниками, левитами и всем Израилем дать клятву, что они сделают так. И они дали клятву.
\vs Ezr 10:6 И встал Ездра и пошел от дома Божия в жилище Иоханана, сына Елияшивова, и пришел туда. Хлеба он не ел и воды не пил, потому что плакал о преступлении переселенцев.
\rsbpar\vs Ezr 10:7 И объявили в Иудее и в Иерусалиме всем \bibemph{бывшим} в плену, чтоб они собрались в Иерусалим;
\vs Ezr 10:8 а кто не придет чрез три дня, на все имение того, по определению начальствующих и старейшин, будет положено заклятие, и сам он будет отлучен от общества переселенцев.
\vs Ezr 10:9 И собрались все жители Иудеи и земли Вениаминовой в Иерусалим в три дня. Это \bibemph{было} в девятом месяце, в двадцатый день месяца. И сидел весь народ на площади у дома Божия, дрожа как по этому делу, так и от дождей.
\vs Ezr 10:10 И встал Ездра священник и сказал им: вы сделали преступление, взяв себе жен иноплеменных, и тем увеличили вину Израиля.
\vs Ezr 10:11 Итак покайтесь \bibemph{в сем} пред Господом Богом отцов ваших, и исполните волю Его, и отлучите себя от народов земли и от жен иноплеменных.
\vs Ezr 10:12 И отвечало все собрание, и сказало громким голосом: как ты сказал, так и сделаем.
\vs Ezr 10:13 Однако же народ многочислен и время \bibemph{теперь} дождливое, и нет возможности стоять на улице. Да и это дело не одного дня и не двух, потому что мы много в этом деле погрешили.
\vs Ezr 10:14 Пусть наши начальствующие заступят место всего общества, и все в городах наших, которые взяли жен иноплеменных, пусть приходят сюда в назначенные времена и с ними старейшины каждого города и с\acc{у}дьи его, доколе не отвратится от нас пылающий гнев Бога нашего за это дело.
\rsbpar\vs Ezr 10:15 Тогда Ионафан, сын Асаила, и Яхзеия, сын Фиквы, стали над этим делом, и Мешуллам и Шавфай левит были помощниками им.
\vs Ezr 10:16 И сделали так вышедшие из плена. И отделены \bibemph{на это} Ездра священник, главы поколений, от каждого поколения их, и все они \bibemph{названы} поименно. И сделали они заседание в первый день десятого месяца, для исследования сего дела;
\vs Ezr 10:17 и окончили \bibemph{исследование} о всех, которые взяли жен иноплеменных, к первому дню первого месяца.
\rsbpar\vs Ezr 10:18 И нашлись из сыновей священнических, которые взяли жен иноплеменных,~--- из сыновей Иисуса, сына Иоседекова, и братьев его: Маасея, Елиезер, Иарив и Гедалия;
\vs Ezr 10:19 и они дали руки свои \bibemph{во уверение}, что отпустят жен своих, и \bibemph{что они} повинны \bibemph{принести} в жертву овна за свою вину;
\vs Ezr 10:20 и из сыновей Иммера: Хананий и Зевадия;
\vs Ezr 10:21 и из сыновей Харима: Маасея, Елия, Шемаия, Иехиил и Уззия;
\vs Ezr 10:22 и из сыновей Пашхура: Елиоенай, Маасея, Исмаил, Нафанаил, Иозавад и Эласа;
\vs Ezr 10:23 и из левитов: Иозавад, Шимей и Келаия, он же Клита, Пафахия, Иуда и Елиезер;
\vs Ezr 10:24 и из певцов: Елияшив; и из привратников: Шаллум, Телем и Урий;
\vs Ezr 10:25 а из Израильтян,~--- из сыновей Пароша: Рамаия, Иззия, Малхия, Миямин, Елеазар, Малхия и Венаия;
\vs Ezr 10:26 и из сыновей Елама: Матфания, Захария, Иехиел, Авдий, Иремоф и Елия;
\vs Ezr 10:27 и из сыновей Заффу: Елиоенай, Елияшив, Матфания, Иремоф, Завад и Азиса;
\vs Ezr 10:28 и из сыновей Бевая: Иоханан, Ханания, Забвай и Афлай;
\vs Ezr 10:29 и из сыновей Вания: Мешуллам, Маллух, Адая, Иашув, Шеал и Иерамоф;
\vs Ezr 10:30 и из сыновей Пахаф-Моава: Адна, Хелал, Венаия, Маасея, Матфания, Веселеил, Биннуй и Манассия;
\vs Ezr 10:31 и из сыновей Харима: Елиезер, Ишшия, Малхия, Шемаия, Симеон,
\vs Ezr 10:32 Вениамин, Маллух, Шемария;
\vs Ezr 10:33 и из сыновей Хашума: Мафнай, Мафафа, Завад, Елифелет, Иеремай, Манассия и Шимей;
\vs Ezr 10:34 и из сыновей Вания: Маадай, Амрам и Уел,
\vs Ezr 10:35 Бенаия, Бидья, Келуги,
\vs Ezr 10:36 Ванея, Меремоф, Елиашив,
\vs Ezr 10:37 Матфания, Мафнай, Иаасай,
\vs Ezr 10:38 Ваний, Биннуй, Шимей,
\vs Ezr 10:39 Шелемия, Нафан, Адаия,
\vs Ezr 10:40 Махнадбай, Шашай, Шарай,
\vs Ezr 10:41 Азариел, Шелемиягу, Шемария,
\vs Ezr 10:42 Шаллум, Амария и Иосиф;
\vs Ezr 10:43 и из сыновей Нево: Иеиел, Матфифия, Завад, Зевина, Иаддай, Иоель и Бенаия.
\vs Ezr 10:44 Все сии взяли \bibemph{за себя} жен иноплеменных, и некоторые из сих жен родили им детей.

\bibbookdescr{Neh}{
  inline={\LARGE Книга\\\Huge Неемии},
  toc={Неемия},
  bookmark={Неемия},
  header={Неемия},
  %headerleft={},
  %headerright={},
  abbr={Неем}
}
\vs Neh 1:1 Слов\acc{а} Неемии, сына Ахалиина. В месяце Кислеве, в двадцатом году, я находился в Сузах, престольном городе.
\vs Neh 1:2 И пришел Ханани, один из братьев моих, он и \bibemph{несколько} человек из Иудеи. И спросил я их об уцелевших Иудеях, которые остались от плена, и об Иерусалиме.
\vs Neh 1:3 И сказали они мне: оставшиеся, которые остались от плена, \bibemph{находятся} там, в стране \bibemph{своей}, в великом бедствии и в уничижении; и стена Иерусалима разрушена, и ворота его сожжены огнем.
\rsbpar\vs Neh 1:4 Услышав эти слова, я сел и заплакал, и печален был несколько дней, и постился и молился пред Богом небесным
\vs Neh 1:5 и говорил: Господи Боже небес, Боже великий и страшный, хранящий завет и милость к любящим Тебя и соблюдающим заповеди Твои!
\vs Neh 1:6 Да будут уши Твои внимательны и очи Твои отверсты, чтобы услышать молитву раба Твоего, которою я теперь день и ночь молюсь пред Тобою о сынах Израилевых, рабах Твоих, и исповедуюсь во грехах сынов Израилевых, которыми согрешили мы пред Тобою, согрешили~--- и я и дом отца моего.
\vs Neh 1:7 Мы стали преступны пред Тобою и не сохранили заповедей и уставов и определений, которые Ты заповедал Моисею, рабу Твоему.
\vs Neh 1:8 Но помяни слово, которое Ты заповедал Моисею, рабу Твоему, говоря: \bibemph{если} вы сделаетесь преступниками, то Я рассею вас по народам;
\vs Neh 1:9 \bibemph{когда} же обратитесь ко Мне и будете хранить заповеди Мои и исполнять их, то хотя бы вы изгнаны были на край неба, и оттуда соберу вас и приведу вас на место, которое избрал Я, чтобы водворить там имя Мое.
\vs Neh 1:10 Они же рабы Твои и народ Твой, который Ты искупил силою Твоею великою и рукою Твоею могущественною.
\vs Neh 1:11 Молю Тебя, Господи! Да будет ухо Твое внимательно к молитве раба Твоего и к молитве рабов Твоих, любящих благоговеть пред именем Твоим. И благопоспеши рабу Твоему теперь, и введи его в милость у человека сего. Я был виночерпием у царя.
\vs Neh 2:1 В месяце Нисане, в двадцатый год царя Артаксеркса, \bibemph{было} перед ним вино. И я взял вино и подал царю, и, казалось, не был печален перед ним.
\vs Neh 2:2 Но царь сказал мне: отчего лице у тебя печально; ты не болен, этого нет, а верно печаль на сердце? Я сильно испугался
\vs Neh 2:3 и сказал царю: да живет царь во веки! Как не быть печальным лицу моему, когда город, дом гробов отцов моих, в запустении, и ворота его сожжены огнем!
\vs Neh 2:4 И сказал мне царь: чего же ты желаешь? Я помолился Богу небесному
\vs Neh 2:5 и сказал царю: если царю благоугодно, и если в благоволении раб твой пред лицем твоим, то пошли меня в Иудею, в город, \bibemph{где} гробы отцов моих, чтоб я обстроил его.
\vs Neh 2:6 И сказал мне царь и царица, которая сидела подле него: сколько времени продлится путь твой, и когда возвратишься? И благоугодно было царю послать меня, после того как я назначил время.
\vs Neh 2:7 И сказал я царю: если царю благоугодно, то дал бы мне письма к заречным областеначальникам, чтоб они давали мне пропуск, доколе я не дойду до Иудеи,
\vs Neh 2:8 и письмо к Асафу, хранителю царских лесов, чтоб он дал мне дерев для ворот крепости, которая при доме \bibemph{Божием}, и для городской стены, и для дома, в котором бы мне жить. И дал мне царь, так как благодеющая рука Бога моего была надо мною.
\vs Neh 2:9 И пришел я к заречным областеначальникам и отдал им царские письма. Послал же со мною царь воинских начальников со всадниками.
\rsbpar\vs Neh 2:10 Когда услышал \bibemph{сие} Санаваллат, Хоронит и Товия, Аммонитский раб, то им было весьма досадно, что пришел человек заботиться о благе сынов Израилевых.
\rsbpar\vs Neh 2:11 И пришел я в Иерусалим. И пробыв там три дня,
\vs Neh 2:12 встал я ночью с немногими людьми, \bibemph{бывшими} при мне, и никому не сказал, чт\acc{о} Бог мой положил мне на сердце сделать для Иерусалима; животного же не было со мною никакого, кроме того, на котором я ехал.
\vs Neh 2:13 И проехал я ночью через ворота Долины перед источником Драконовым к воротам Навозным, и осмотрел я стены Иерусалима разрушенные и его ворота, сожженные огнем.
\vs Neh 2:14 И подъехал я к воротам Источника и к царскому водоему, но \bibemph{там} не было места пройти животному, которое было подо мною,~---
\vs Neh 2:15 и я поднялся назад по лощине ночью и осматривал стену, и проехав \bibemph{опять} воротами Долины, возвратился.
\vs Neh 2:16 И начальствующие не знали, куда я ходил и что я делаю: ни Иудеям, ни священникам, ни знатнейшим, ни начальствующим, ни прочим производителям работ я дотоле ничего не открывал.
\vs Neh 2:17 И сказал я им: вы видите бедствие, в каком мы находимся; Иерусалим пуст и ворота его сожжены огнем; пойдем, построим стену Иерусалима, и не будем впредь \bibemph{в таком} уничижении.
\vs Neh 2:18 И я рассказал им о благодеявшей мне руке Бога моего, а также и слова царя, которые он говорил мне. И сказали они: будем строить,~--- и укрепили руки свои на благое \bibemph{дело}.
\rsbpar\vs Neh 2:19 Услышав это, Санаваллат, Хоронит и Товия, Аммонитский раб, и Гешем Аравитянин смеялись над нами и с презрением говорили: что это за дело, которое вы делаете? уже не думаете ли возмутиться против царя?
\vs Neh 2:20 Я дал им ответ и сказал им: Бог Небесный, Он благопоспешит нам, и мы, рабы Его, станем строить, а вам нет части и права и памяти в Иерусалиме.
\vs Neh 3:1 И встал Елияшив, великий священник, и братья его священники и построили Овечьи ворота: они освятили их и вставили двери их, и от башни Меа освятили их до башни Хананела.
\vs Neh 3:2 И подле него строили Иерихонцы, а подле них строил Закхур, сын Имрия.
\vs Neh 3:3 Ворота Рыбные строили уроженцы Сенаи: они покрыли их, и вставили двери их, замки их и засовы их.
\vs Neh 3:4 Подле них чинил \bibemph{стену} Меремоф, сын Урии, сын Гаккоца; подле них чинил Мешуллам, сын Берехии, сын Мешизабела; подле них чинил Садок, сын Бааны;
\vs Neh 3:5 подле них чинили Фекойцы; впрочем знатнейшие из них не наклонили шеи своей поработать для Господа своего.
\vs Neh 3:6 Старые ворота чинили Иоиада, сын Пасеаха, и Мешуллам, сын Бесодии: они покрыли их и вставили двери их, и замки их и засовы их.
\vs Neh 3:7 Подле них чинил Мелатия Гаваонитянин, и Иадон из Меронофа, с жителями Гаваона и Мицфы, подвластными заречному областеначальнику.
\vs Neh 3:8 Подле него чинил Уззиил, сын Харгаии, серебряник, а подле него чинил Ханания, сын Гараккахима. И восстановили Иерусалим до стены широкой.
\vs Neh 3:9 Подле них чинил Рефаия, сын Хура, начальник полуокруга Иерусалимского.
\vs Neh 3:10 Подле них и против дома своего чинил Иедаия, сын Харумафа, а подле него чинил Хаттуш, сын Хашавнии.
\vs Neh 3:11 На втором участке чинил Малхия, сын Харима, и Хашшув, сын Пахаф-Моава; \bibemph{они же чинили} и башню Печную.
\vs Neh 3:12 Подле них чинил Шаллум, сын Галлохеша, начальник полуокруга Иерусалимского, он и дочери его.
\vs Neh 3:13 Ворота Долины чинил Ханун и жители Заноаха: они построили их, и вставили двери их, замки их и засовы их, и \bibemph{еще чинили} они тысячу локтей стены до ворот Навозных.
\vs Neh 3:14 А ворота Навозные чинил Малхия, сын Рехава, начальник Бефкаремского округа: он построил их и вставил двери их, замки их и засовы их.
\vs Neh 3:15 Ворота Источника чинил Шаллум, сын Колхозея, начальник округа Мицфы: он построил их, и покрыл их, и вставил двери их, замки их и засовы их,~--- \bibemph{он же чинил} стену у водоема Селах против царского сада и до ступеней, спускающихся из города Давидова.
\vs Neh 3:16 За ним чинил Неемия, сын Азбука, начальник полуокруга Бефцурского, до гробниц Давидовых и до выкопанного пруда и до дома храбрых.
\vs Neh 3:17 За ним чинили левиты: Рехум, сын Вания; подле него чинил Хашавия, начальник полуокруга Кеильского, за свой округ.
\vs Neh 3:18 За ним чинили братья их: Баввай, сын Хенадада, начальник Кеильского полуокруга.
\vs Neh 3:19 А подле него чинил Езер, сын Иисуса, начальник Мицфы, на втором участке, напротив всхода к оружейне на углу.
\vs Neh 3:20 За ним ревностно чинил Варух, сын Забвая, на втором участке, от угла до дверей дома Елияшива, великого священника.
\vs Neh 3:21 За ним чинил Меремоф, сын Урии, сын Гаккоца, на втором участке, от дверей дома Елияшивова до конца дома Елияшивова.
\vs Neh 3:22 За ним чинили священники из окрестностей.
\vs Neh 3:23 За ними чинил Вениамин и Хашшув, против дома своего; за ними чинил Азария, сын Маасеи, сын Анании, возле дома своего.
\vs Neh 3:24 За ним чинил Биннуй, сын Хенадада, на втором участке, от дома Азарии до угла и поворота.
\vs Neh 3:25 \bibemph{За ним} Фалал, сын Узая, напротив угла и башни, выступающей от верхнего дома царского, которая у двора темничного. За ним Федаия, сын Пароша.
\vs Neh 3:26 Нефинеи же, \bibemph{которые} жили в Офеле, \bibemph{починили} напротив Водяных ворот к востоку и до выступающей башни.
\vs Neh 3:27 За ними чинили Фекойцы, на втором участке, от \bibemph{места} напротив большой выступающей башни до стены Офела.
\vs Neh 3:28 Далее ворот Конских чинили священники, каждый против своего дома.
\vs Neh 3:29 За ними чинил Садок, сын Иммера, против своего дома, а за ним чинил Шемаия, сын Шехании, сторож восточных ворот.
\vs Neh 3:30 За ним чинил Ханания, сын Шелемии, и Ханун, шестой сын Цалафа, на втором участке. За ним чинил Мешуллам, сын Берехии, против комнаты своей.
\vs Neh 3:31 За ним чинил Малхия, сын Гацорфия, до дома нефинеев и торговцев, против ворот Гаммифкад и до уг\acc{о}льного жилья.
\vs Neh 3:32 А между уг\acc{о}льным жильем до ворот Овечьих чинили серебряники и торговцы.
\vs Neh 4:1 Когда услышал Санаваллат, что мы строим стену, он рассердился и много досадовал и издевался над Иудеями;
\vs Neh 4:2 и говорил при братьях своих и при Самарийских военных людях, и сказал: что делают эти жалкие Иудеи? неужели им это дозволят? неужели будут они приносить жертвы? неужели они когда-либо кончат? неужели они оживят камни из груд праха, и притом пожженные?
\vs Neh 4:3 А Товия Аммонитянин, \bibemph{бывший} подле него, сказал: пусть их строят; пойдет лисица, и разрушит их каменную стену.
\rsbpar\vs Neh 4:4 Услыши, Боже наш, в каком мы презрении, и обрати ругательство их на их голову, и предай их презрению в земле пленения;
\vs Neh 4:5 и не покрой беззаконий их, и грех их да не изгладится пред лицем Твоим, потому что они огорчили строящих!
\vs Neh 4:6 Мы однако же строили стену, и сложена была вся стена до половины ее. И у народа доставало усердия работать.
\rsbpar\vs Neh 4:7 Когда услышал Санаваллат и Товия, и Аравитяне, и Аммонитяне, и Азотяне, что стены Иерусалимские восстановляются, что повреждения начали заделываться, то им было весьма досадно.
\vs Neh 4:8 И сговорились все вместе пойти войною на Иерусалим и разрушить его.
\vs Neh 4:9 И мы молились Богу нашему, и ставили против них стражу днем и ночью, для спасения от них.
\vs Neh 4:10 Но Иудеи сказали: ослабела сила у носильщиков, а мусору много; мы не в состоянии строить стену.
\vs Neh 4:11 А неприятели наши говорили: не узнают и не увидят, как \bibemph{вдруг} мы войдем в средину их и перебьем их, и остановим дело.
\vs Neh 4:12 Когда приходили Иудеи, жившие подле них, и говорили нам раз десять, со всех мест, что они нападут на нас:
\vs Neh 4:13 тогда в низменных местах у города, за стеною, на местах сухих поставил я народ поплеменно с мечами их, с копьями их и луками их.
\vs Neh 4:14 И осмотрел я, и стал, и сказал знатнейшим и начальствующим и прочему народу: не бойтесь их; помните Господа великого и страшного и сражайтесь за братьев своих, за сыновей своих и за дочерей своих, за жен своих и за домы свои.
\rsbpar\vs Neh 4:15 Когда услышали неприятели наши, что нам известно \bibemph{намерение их}, тогда разорил Бог замысел их, и все мы возвратились к стене, каждый на свою работу.
\vs Neh 4:16 С того дня половина молодых людей у меня занималась работою, а \bibemph{другая} половина их держала копья, щиты и луки и латы; и начальствующие \bibemph{находились} позади всего дома Иудина.
\vs Neh 4:17 Строившие стену и носившие тяжести, которые налагали \bibemph{на них}, одною рукою производили работу, а другою держали копье.
\vs Neh 4:18 Каждый из строивших препоясан был мечом по чреслам своим, и \bibemph{так} они строили. Возле меня находился трубач.
\vs Neh 4:19 И сказал я знатнейшим и начальствующим и прочему народу: работа велика и обширна, и мы рассеяны по стене и отдалены друг от друга;
\vs Neh 4:20 поэтому, откуда услышите вы звук трубы, в то место собирайтесь к нам: Бог наш будет сражаться за нас.
\vs Neh 4:21 Так производили мы работу; и половина держала копья от восхода зари до появления звезд.
\vs Neh 4:22 Сверх сего, в то же время я сказал народу, чтобы в Иерусалиме ночевали все с рабами своими,~--- и будут они у нас ночью на страже, а днем на работе.
\vs Neh 4:23 И ни я, ни братья мои, ни слуги мои, ни стражи, сопровождавшие меня, не снимали с себя одеяния своего, у каждого были под рукою меч и вода.
\vs Neh 5:1 И сделался большой ропот в народе и у жен его на братьев своих Иудеев.
\vs Neh 5:2 Были такие, которые говорили: нас, сыновей наших и дочерей наших много; и мы желали бы доставать хлеб и кормиться и жить.
\vs Neh 5:3 Были и такие, которые говорили: поля свои, и виноградники свои, и домы свои мы закладываем, чтобы достать хлеба от голода.
\vs Neh 5:4 Были и такие, которые говорили: мы занимаем серебро на подать царю \bibemph{под залог} полей наших и виноградников наших;
\vs Neh 5:5 у нас такие же тела, какие тела у братьев наших, и сыновья наши такие же, как их сыновья; а вот, мы должны отдавать сыновей наших и дочерей наших в рабы, и некоторые из дочерей наших уже находятся в порабощении. Нет никаких средств для выкупа в руках наших; и поля наши и виноградники наши у других.
\rsbpar\vs Neh 5:6 Когда я услышал ропот их и такие слова, я очень рассердился.
\vs Neh 5:7 Сердце мое возмутилось, и я строго выговорил знатнейшим и начальствующим и сказал им: вы берете лихву с братьев своих. И созвал я против них большое собрание
\vs Neh 5:8 и сказал им: мы выкупали братьев своих, Иудеев, проданных народам, сколько было сил у нас, а вы продаете братьев своих, и они продаются нам? Они молчали и не находили ответа.
\vs Neh 5:9 И сказал я: нехорошо вы делаете. Не в страхе ли Бога нашего должны ходить вы, дабы избегнуть поношения от народов, врагов наших?
\vs Neh 5:10 И я также, братья мои и \bibemph{служащие} при мне давали им в заем и серебро и хлеб: оставим им долг сей.
\vs Neh 5:11 Возвратите им ныне же поля их, виноградные и масличные сады их, и домы их, и рост с серебра и хлеба, и вина и масла, за который вы ссудили их.
\vs Neh 5:12 И сказали они: возвратим и не будем с них требовать; сделаем так, как ты говоришь. И позвал я священников и велел им дать клятву, что они так сделают.
\vs Neh 5:13 И вытряхнул я \bibemph{одежду} мою и сказал: так пусть вытряхнет Бог всякого человека, который не сдержит слова сего, из дома его и из имения его, и так да будет у него вытрясено и пусто! И сказало все собрание: аминь. И прославили Бога; и народ выполнил слово сие.
\rsbpar\vs Neh 5:14 Еще: с того дня, как определен я был областеначальником их в земле Иудейской, от двадцатого года до тридцать второго года царя Артаксеркса, в продолжение двенадцати лет я и братья мои не ели хлеба областеначальнического.
\vs Neh 5:15 А прежние областеначальники, которые \bibemph{были} до меня, отягощали народ и брали с них хлеб и вино, кроме сорока сиклей серебра; даже и слуги их господствовали над народом. Я же не делал так по страху Божию.
\vs Neh 5:16 При этом работы на стене сей я поддерживал; и полей мы не закупали, и все слуги мои собирались туда на работу.
\vs Neh 5:17 Иудеев и начальствующих по сто пятидесяти человек \bibemph{бывало} за столом у меня, кроме приходивших к нам из окрестных народов.
\vs Neh 5:18 И \bibemph{вот} что было приготовляемо на один день: один бык, шесть отборных овец и птицы приготовлялись у меня; и в десять дней \bibemph{издерживалось} множество всякого вина. И при \bibemph{всем} том, хлеба областеначальнического я не требовал, так как тяжелая служба \bibemph{лежала} на народе сем.
\vs Neh 5:19 Помяни, Боже мой, во благо мне все, что я сделал для народа сего!
\vs Neh 6:1 Когда дошло до слуха Санаваллата и Товии и Гешема Аравитянина и прочих неприятелей наших, что я отстроил стену, и не оставалось в ней повреждений~--- впрочем до того времени я еще не ставил дверей в ворота,~---
\vs Neh 6:2 тогда прислал Санаваллат и Гешем ко мне сказать: приди, и сойдемся в одном из сел на равнине Оно. Они замышляли сделать мне зло.
\vs Neh 6:3 Но я послал к ним послов сказать: я занят большим делом, не могу сойти; дело остановилось бы, если бы я оставил его и сошел к вам.
\vs Neh 6:4 Четыре раза присылали они ко мне с таким же приглашением, и я отвечал им то же.
\vs Neh 6:5 Тогда прислал ко мне Санаваллат в пятый раз своего слугу, у которого в руке было открытое письмо.
\vs Neh 6:6 В нем было написано: слух носится у народов, и Гешем говорит, будто ты и Иудеи задумали отпасть, для чего и строишь стену и хочешь быть у них царем, по тем же слухам;
\vs Neh 6:7 и пророков поставил ты, чтоб они разглашали о тебе в Иерусалиме и говорили: царь Иудейский! И такие речи дойдут до царя. Итак приходи, и посоветуемся вместе.
\vs Neh 6:8 Но я послал к нему сказать: ничего такого не было, о чем ты говоришь; ты выдумал это своим умом.
\vs Neh 6:9 Ибо все они стращали нас, думая: опустятся руки их от дела сего, и оно не состоится; но я тем более укрепил руки мои.
\rsbpar\vs Neh 6:10 Пришел я в дом Шемаии, сына Делаии, сына Мегетавелова, и он заперся и сказал: пойдем в дом Божий, внутрь храма, и запрем за собою двери храма, потому что придут убить тебя, и придут убить тебя ночью.
\vs Neh 6:11 Но я сказал: может ли бежать такой человек, как я? Может ли такой, как я, войти в храм, чтобы остаться живым? Не пойду.
\vs Neh 6:12 Я знал, что не Бог послал его, хотя он пророчески говорил мне, но что Товия и Санаваллат подкупили его.
\vs Neh 6:13 Для того он был подкуплен, чтоб я устрашился и сделал так и согрешил, и чтобы имели о мне худое мнение и преследовали меня за это укоризнами.
\vs Neh 6:14 Помяни, Боже мой, Товию и Санаваллата по сим делам их, а также пророчицу Ноадию и прочих пророков, которые хотели устрашить меня!
\rsbpar\vs Neh 6:15 Стена была совершена в двадцать пятый день месяца Елула, в пятьдесят два дня.
\vs Neh 6:16 Когда услышали об этом все неприятели наши, и увидели это все народы, которые вокруг нас, тогда они очень упали в глазах своих и познали, что это дело сделано Богом нашим.
\rsbpar\vs Neh 6:17 Сверх того в те дни знатнейшие Иудеи много писали писем, которые посылались к Товии, а Товиины письма приходили к ним.
\vs Neh 6:18 Ибо многие в Иудее были в клятвенном союзе с ним, потому что он был зять Шехании, сына Арахова, а сын его Иоханан взял \bibemph{за себя} дочь Мешуллама, сына Верехии.
\vs Neh 6:19 Даже о доброте его они говорили при мне, и мои слова переносились к нему. Товия присылал письма, чтоб устрашить меня.
\vs Neh 7:1 Когда стена была построена, и я вставил двери, и поставлены были на свое служение привратники и певцы и левиты,
\vs Neh 7:2 тогда приказал я брату моему Ханани и начальнику Иерусалимской крепости Хананию, ибо он более многих других был человек верный и богобоязненный,
\vs Neh 7:3 и сказал я им: пусть не отворяют ворот Иерусалимских, доколе не обогреет солнце, и доколе они стоят, пусть замыкают и запирают двери. И поставил я стражами жителей Иерусалима, каждого на свою стражу и каждого напротив дома его.
\vs Neh 7:4 Но город был пространен и велик, а народа в нем было немного, и домы не были построены.
\rsbpar\vs Neh 7:5 И положил мне Бог мой на сердце собрать знатнейших и начальствующих и народ, чтобы сделать перепись. И нашел я родословную перепись тех, которые сначала пришли, и в ней написано:
\vs Neh 7:6 вот жители страны, которые отправились из пленников, переселенных Навуходоносором, царем Вавилонским, и возвратились в Иерусалим и Иудею, каждый в свой город,~---
\vs Neh 7:7 те, которые пошли с Зоровавелем, Иисусом, Неемиею, Азариею, Раамиею, Нахманием, Мардохеем, Билшаном, Мисферефом, Бигваем, Нехумом, Вааною. Число людей народа Израилева:
\vs Neh 7:8 сыновей Пароша две тысячи сто семьдесят два.
\vs Neh 7:9 Сыновей Сафатии триста семьдесят два.
\vs Neh 7:10 Сыновей Араха шестьсот пятьдесят два.
\vs Neh 7:11 Сыновей Пахаф-Моава, из сыновей Иисуса и Иоава, две тысячи восемьсот восемнадцать.
\vs Neh 7:12 Сыновей Елама тысяча двести пятьдесят четыре.
\vs Neh 7:13 Сыновей Заффу восемьсот сорок пять.
\vs Neh 7:14 Сыновей Закхая семьсот шестьдесят.
\vs Neh 7:15 Сыновей Биннуя шестьсот сорок восемь.
\vs Neh 7:16 Сыновей Бевая шестьсот двадцать восемь.
\vs Neh 7:17 Сыновей Азгада две тысячи триста двадцать два.
\vs Neh 7:18 Сыновей Адоникама шестьсот шестьдесят семь.
\vs Neh 7:19 Сыновей Бигвая две тысячи шестьсот семь.
\vs Neh 7:20 Сыновей Адина шестьсот пятьдесят пять.
\vs Neh 7:21 Сыновей Атера из \bibemph{дома} Езекии девяносто восемь.
\vs Neh 7:22 Сыновей Хашума триста двадцать восемь.
\vs Neh 7:23 Сыновей Вецая триста двадцать четыре.
\vs Neh 7:24 Сыновей Харифа сто двенадцать.
\vs Neh 7:25 Уроженцев Гаваона девяносто пять.
\vs Neh 7:26 Жителей Вифлеема и Нетофы сто восемьдесят восемь.
\vs Neh 7:27 Жителей Анафофа сто двадцать восемь.
\vs Neh 7:28 Жителей Беф-Азмавефа сорок два.
\vs Neh 7:29 Жителей Кириаф-Иарима, Кефиры и Беерофа семьсот сорок три.
\vs Neh 7:30 Жителей Рамы и Гевы шестьсот двадцать один.
\vs Neh 7:31 Жителей Михмаса сто двадцать два.
\vs Neh 7:32 Жителей Вефиля и Гая сто двадцать три.
\vs Neh 7:33 Жителей Нево другого пятьдесят два.
\vs Neh 7:34 Сыновей Елама другого тысяча двести пятьдесят четыре.
\vs Neh 7:35 Сыновей Харима триста двадцать.
\vs Neh 7:36 Уроженцев Иерихона триста сорок пять.
\vs Neh 7:37 Уроженцев Лода, Хадида и Оно семьсот двадцать один.
\vs Neh 7:38 Уроженцев Сенаи три тысячи девятьсот тридцать.
\vs Neh 7:39 Священников, сыновей Иедаии, из дома Иисусова, девятьсот семьдесят три.
\vs Neh 7:40 Сыновей Иммера тысяча пятьдесят два.
\vs Neh 7:41 Сыновей Пашхура тысяча двести сорок семь.
\vs Neh 7:42 Сыновей Харима тысяча семнадцать.
\vs Neh 7:43 Левитов: сыновей Иисуса, из \bibemph{дома} Кадмиилова, из дома сыновей Годевы, семьдесят четыре.
\vs Neh 7:44 Певцов: сыновей Асафа сто сорок восемь.
\vs Neh 7:45 Привратники: сыновья Шаллума, сыновья Атера, сыновья Талмона, сыновья Аккува, сыновья Хатиты, сыновья Шовая~--- сто тридцать восемь.
\vs Neh 7:46 Нефинеи: сыновья Цихи, сыновья Хасуфы, сыновья Таббаофа,
\vs Neh 7:47 сыновья Кироса, сыновья Сии, сыновья Фадона,
\vs Neh 7:48 сыновья Леваны, сыновья Хагавы, сыновья Салмая,
\vs Neh 7:49 сыновья Ханана, сыновья Гиддела, сыновья Гахара,
\vs Neh 7:50 сыновья Реаии, сыновья Рецина, сыновья Некоды,
\vs Neh 7:51 сыновья Газзама, сыновья Уззы, сыновья Пасеаха,
\vs Neh 7:52 сыновья Весая, сыновья Меунима, сыновья Нефишсима,
\vs Neh 7:53 сыновья Бакбука, сыновья Хакуфы, сыновья Хархура,
\vs Neh 7:54 сыновья Бацлифа, сыновья Мехиды, сыновья Харши,
\vs Neh 7:55 сыновья Баркоса, сыновья Сисары, сыновья Фамаха,
\vs Neh 7:56 сыновья Нециаха, сыновья Хатифы.
\vs Neh 7:57 Сыновья рабов Соломоновых: сыновья Сотая, сыновья Соферефа, сыновья Фериды,
\vs Neh 7:58 сыновья Иаалы, сыновья Даркона, сыновья Гиддела,
\vs Neh 7:59 сыновья Сафатии, сыновья Хаттила, сыновья Похереф-Гаццевайима, сыновья Амона.
\vs Neh 7:60 Всех нефинеев и сыновей рабов Соломоновых триста девяносто два.
\vs Neh 7:61 И вот вышедшие из Тел-Мелаха, Тел-Харши, Херув-Аддона и Иммера; но они не могли показать о поколении своем и о племени своем, от Израиля ли они.
\vs Neh 7:62 Сыновья Делаии, сыновья Товии, сыновья Некоды~--- шестьсот сорок два.
\vs Neh 7:63 И из священников: сыновья Ховаии, сыновья Гаккоца, сыновья Верзеллия, который взял жену из дочерей Верзеллия Галаадитянина и стал называться их именем.
\vs Neh 7:64 Они искали родословной своей записи, и не нашлось, и потому исключены из священства.
\vs Neh 7:65 И Тиршафа сказал им, чтоб они не ели великой святыни, доколе не восстанет священник с уримом и туммимом.
\rsbpar\vs Neh 7:66 Все общество вместе \bibemph{состояло} из сорока двух тысяч трехсот шестидесяти \bibemph{человек},
\vs Neh 7:67 кроме рабов их и рабынь их, которых было семь тысяч триста тридцать семь; и при них певцов и певиц двести сорок пять.
\vs Neh 7:68 Коней у них было семьсот тридцать шесть, лошаков у них двести сорок пять,
\vs Neh 7:69 верблюдов четыреста тридцать пять, ослов шесть тысяч семьсот двадцать.
\vs Neh 7:70 Некоторые главы поколений дали вклады на производство работ. Тиршафа дал в сокровищницу золотом тысячу драхм, пятьдесят чаш, пятьсот тридцать священнических одежд.
\vs Neh 7:71 И некоторые из глав поколений дали в сокровищницу на производство работ двадцать тысяч драхм золота и две тысячи двести мин серебра.
\vs Neh 7:72 Прочие из народа дали двадцать тысяч драхм золота и две тысячи мин серебра и шестьдесят семь священнических одежд.
\vs Neh 7:73 И стали жить священники и левиты, и привратники и певцы, и народ и нефинеи, и весь Израиль в городах своих.
\vs Neh 8:1 Когда наступил седьмой месяц, и сыны Израилевы \bibemph{жили} по городам своим, тогда собрался весь народ, как один человек, на площадь, которая пред Водяными воротами, и сказали книжнику Ездре, чтоб он принес книгу закона Моисеева, который заповедал Господь Израилю.
\vs Neh 8:2 И принес священник Ездра закон пред собрание мужчин и женщин, и всех, которые могли понимать, в первый день седьмого месяца;
\vs Neh 8:3 и читал из него на площади, которая пред Водяными воротами, от рассвета до полудня, пред мужчинами и женщинами и всеми, которые могли понимать; и уши всего народа \bibemph{были приклонены} к книге закона.
\vs Neh 8:4 Книжник Ездра стоял на деревянном возвышении, которое для сего сделали, а подле него, по правую руку его, стояли Маттифия и Шема, и Анаия и Урия, и Хелкия и Маасея, а по левую руку его Федаия и Мисаил, и Малхия и Хашум, и Хашбаддана, и Захария и Мешуллам.
\vs Neh 8:5 И открыл Ездра книгу пред глазами всего народа, потому что он стоял выше всего народа. И когда он открыл ее, весь народ встал.
\vs Neh 8:6 И благословил Ездра Господа Бога великого. И весь народ отвечал: аминь, аминь, поднимая вверх руки свои,~--- и поклонялись и повергались пред Господом лицем до земли.
\vs Neh 8:7 Иисус, Ванаия, Шеревия, Иамин, Аккув, Шавтай, Годия, Маасея, Клита, Азария, Иозавад, Ханан, Фелаия и левиты поясняли народу закон, между тем как народ стоял на своем месте.
\vs Neh 8:8 И читали из книги, из закона Божия, внятно, и присоединяли толкование, и \bibemph{народ} понимал прочитанное.
\rsbpar\vs Neh 8:9 Тогда Неемия, он же Тиршафа, и книжник Ездра, священник, и левиты, учившие народ, сказали всему народу: день сей свят Господу Богу вашему; не печальтесь и не плачьте, потому что весь народ плакал, слушая слова закона.
\vs Neh 8:10 И сказал им: пойдите, ешьте тучное и пейте сладкое, и посылайте части тем, у кого ничего не приготовлено, потому что день сей свят Господу нашему. И не печальтесь, потому что радость пред Господом~--- подкрепление для вас.
\vs Neh 8:11 И левиты успокаивали весь народ, говоря: перестаньте, ибо день сей свят, не печальтесь.
\vs Neh 8:12 И пошел весь народ есть, и пить, и посылать части, и праздновать с великим веселием, ибо поняли слова, которые сказали им.
\rsbpar\vs Neh 8:13 На другой день собрались главы поколений от всего народа, священники и левиты к книжнику Ездре, чтобы он изъяснял им слова закона.
\vs Neh 8:14 И нашли написанное в законе, который Господь дал чрез Моисея, чтобы сыны Израилевы в седьмом месяце, в праздник, жили в кущах.
\vs Neh 8:15 И потому объявили и провозгласили по всем городам своим и в Иерусалиме, говоря: пойдите на гору и несите ветви маслины садовой и ветви маслины дикой, и ветви миртовые и ветви пальмовые, и ветви \bibemph{других} широколиственных дерев, чтобы сделать кущи по написанному.
\vs Neh 8:16 И пошел народ, и принесли, и сделали себе кущи, каждый на своей кровле и на дворах своих, и на дворах дома Божия, и на площади у Водяных ворот, и на площади у Ефремовых ворот.
\vs Neh 8:17 Все общество возвратившихся из плена сделало кущи и жило в кущах. От дней Иисуса, сына Навина, до этого дня не делали так сыны Израилевы. Радость была весьма великая.
\vs Neh 8:18 И читали из книги закона Божия каждый день, от первого дня до последнего дня. И праздновали праздник семь дней, а в восьмой день попразднество по уставу.
\vs Neh 9:1 В двадцать четвертый день этого месяца собрались все сыны Израилевы, постящиеся и во вретищах и с пеплом на головах своих.
\vs Neh 9:2 И отделилось семя Израилево от всех инородных, и встали и исповедовались во грехах своих и в преступлениях отцов своих.
\vs Neh 9:3 И стояли на своем месте, и четверть дня читали из книги закона Господа Бога своего, и четверть исповедовались и поклонялись Господу Богу своему.
\vs Neh 9:4 И стали на возвышенное место левитов: Иисус, Вания, Кадмиил, Шевания, Вунний, Шеревия, Вания, Хенани, и громко взывали к Господу Богу своему.
\vs Neh 9:5 И сказали левиты~--- Иисус, Кадмиил, Вания, Хашавния, Шеревия, Годия, Шевания, Петахия: встаньте, славьте Господа Бога вашего, от века и до века. Да славословят достославное и превысшее всякого славословия и хвалы имя Твое!
\vs Neh 9:6 [И сказал Ездра:] Ты, Господи, един, Ты создал небо, небеса небес и все воинство их, землю и все, что на ней, моря и все, что в них, и Ты живишь все сие, и небесные воинства Тебе поклоняются.
\vs Neh 9:7 Ты Сам, Господи Боже, избрал Аврама, и вывел его из Ура Халдейского, и дал ему имя Авраама,
\vs Neh 9:8 и нашел сердце его верным пред Тобою, и заключил с ним завет, чтобы дать [ему и] семени его землю Хананеев, Хеттеев, Аморреев, Ферезеев, Иевусеев и Гергесеев. И Ты исполнил слово Свое, потому что Ты праведен.
\vs Neh 9:9 Ты увидел бедствие отцов наших в Египте и услышал вопль их у Чермного моря,
\vs Neh 9:10 и явил знамения и чудеса над фараоном и над всеми рабами его, и над всем народом земли его, так как Ты знал, что они надменно поступали с ними, и сделал Ты Себе имя до сего дня.
\vs Neh 9:11 Ты рассек пред ними море, и они среди моря прошли посуху, и гнавшихся за ними Ты поверг в глубины, как камень в сильные воды.
\vs Neh 9:12 В столпе облачном Ты вел их днем и в столпе огненном~--- ночью, чтоб освещать им путь, по которому идти им.
\vs Neh 9:13 И снисшел Ты на гору Синай и говорил с ними с неба, и дал им суды справедливые, законы верные, уставы и заповеди добрые.
\vs Neh 9:14 И указал им святую Твою субботу и заповеди, и уставы и закон преподал им чрез раба Твоего Моисея.
\vs Neh 9:15 И хлеб с неба Ты давал им в голоде их, и воду из камня источал им в жажде их, и сказал им, чтоб они пошли и овладели землею, которую Ты, подняв руку Твою, \bibemph{клялся} дать им.
\vs Neh 9:16 Но они и отцы наши упрямствовали, и шею свою держали упруго, и не слушали заповедей Твоих;
\vs Neh 9:17 не захотели повиноваться и не вспомнили чудных дел Твоих, которые Ты делал с ними, и держали шею свою упруго, и, по упорству своему, поставили над собою вождя, чтобы возвратиться в рабство свое. Но Ты Бог, любящий прощать, благий и милосердый, долготерпеливый и многомилостивый, и Ты не оставил их.
\vs Neh 9:18 И хотя они сделали себе литого тельца, и сказали: вот бог твой, который вывел тебя из Египта, и хотя делали великие оскорбления,
\vs Neh 9:19 но Ты, по великому милосердию Твоему, не оставлял их в пустыне; столп облачный не отходил от них днем, чтобы вести их по пути, и столп огненный~--- ночью, чтобы светить им на пути, по которому им идти.
\vs Neh 9:20 И Ты дал им Духа Твоего благого, чтобы наставлять их, и манну Твою не отнимал от уст их, и воду давал им для утоления жажды их.
\vs Neh 9:21 Сорок лет Ты питал их в пустыне; они ни в чем не терпели недостатка; одежды их не ветшали, и ноги их не пухли.
\vs Neh 9:22 И Ты дал им царства и народы и разделил им, и они овладели землею Сигона, и землею царя Есевонского, и землею Ога, царя Васанского.
\vs Neh 9:23 И сыновей их Ты размножил, как звезды небесные, и ввел их в землю, о которой Ты говорил отцам их, что они придут владеть \bibemph{ею}.
\vs Neh 9:24 И вошли сыновья их, и овладели землею. И Ты покорил им жителей земли, Хананеев, и отдал их в руки их, и царей их, и народы земли, чтобы они поступали с ними по своей воле.
\vs Neh 9:25 И заняли они укрепленные города и тучную землю, и взяли во владение домы, наполненные всяким добром, водоемы, высеченные \bibemph{из камня}, виноградные и масличные сады и множество дерев \bibemph{с плодами} для пищи. Они ели, насыщались, тучнели и наслаждались по великой благости Твоей;
\vs Neh 9:26 и сделались упорны и возмутились против Тебя, и презрели закон Твой, убивали пророков Твоих, которые увещевали их обратиться к Тебе, и делали великие оскорбления.
\vs Neh 9:27 И Ты отдал их в руки врагов их, которые теснили их. Но когда, в тесное для них время, они взывали к Тебе, Ты выслушивал их с небес и, по великому милосердию Твоему, давал им спасителей, и они спасали их от рук врагов их.
\vs Neh 9:28 Когда же успокаивались, то снова начинали делать зло пред лицем Твоим, и Ты отдавал их в руки неприятелей их, и они господствовали над ними. Но когда они опять взывали к Тебе, Ты выслушивал их с небес и, по великому милосердию Твоему, избавлял их многократно.
\vs Neh 9:29 Ты напоминал им обратиться к закону Твоему, но они упорствовали и не слушали заповедей Твоих, и отклонялись от уставов Твоих, которыми жил бы человек, если бы исполнял их, и хребет \bibemph{свой} сделали упорным, и шею свою держали упруго, и не слушали.
\vs Neh 9:30 Ожидая их \bibemph{обращения}, Ты медлил многие годы и напоминал им Духом Твоим чрез пророков Твоих, но они не слушали. И Ты предал их в руки иноземных народов.
\vs Neh 9:31 Но, по великому милосердию Твоему, Ты не истребил их до конца, и не оставлял их, потому что Ты Бог благий и милостивый.
\vs Neh 9:32 И ныне, Боже наш, Боже великий, сильный и страшный, хранящий завет и милость! да не будет малым пред лицем Твоим все страдание, которое постигало нас, царей наших, князей наших, и священников наших, и пророков наших, и отцов наших и весь народ Твой от дней царей Ассирийских до сего дня.
\vs Neh 9:33 Во всем постигшем нас Ты праведен, потому что Ты делал по правде, а мы виновны.
\vs Neh 9:34 Цари наши, князья наши, священники наши и отцы наши не исполняли закона Твоего, и не внимали заповедям Твоим и напоминаниям Твоим, которыми Ты напоминал им.
\vs Neh 9:35 И в царстве своем, при великом добре Твоем, которое Ты давал им, и на обширной и тучной земле, которую Ты отделил им, они не служили Тебе и не обращались от злых дел своих.
\vs Neh 9:36 И вот, мы ныне рабы; на той земле, которую Ты дал отцам нашим, чтобы питаться ее плодами и ее добром, вот, мы рабствуем.
\vs Neh 9:37 И произведения свои она во множестве приносит для царей, которым Ты покорил нас за грехи наши. И телами нашими и скотом нашим они владеют по своему произволу, и мы в великом стеснении.
\vs Neh 9:38 По всему этому мы даем твердое обязательство и подписываем, и на подписи печать князей наших, левитов наших и священников наших.
\vs Neh 10:1 Приложившие печати были: Неемия-Тиршафа, сын Гахалии, и Седекия,
\vs Neh 10:2 Сераия, Азария, Иеремия,
\vs Neh 10:3 Пашхур, Амария, Малхия,
\vs Neh 10:4 Хаттуш, Шевания, Маллух,
\vs Neh 10:5 Харим, Меремоф, Овадия,
\vs Neh 10:6 Даниил, Гиннефон, Варух,
\vs Neh 10:7 Мешуллам, Авия, Миямин,
\vs Neh 10:8 Маазия, Вилгай, Шемаия: это священники.
\vs Neh 10:9 Левиты: Иисус, сын Азании, Биннуй, из сыновей Хенадада, Кадмиил,
\vs Neh 10:10 и братья их: Шевания, Годия, Клита, Фелаия, Ханан,
\vs Neh 10:11 Миха, Рехов, Хашавия,
\vs Neh 10:12 Закхур, Шеревия, Шевания,
\vs Neh 10:13 Годия, Ваний, Венинуй.
\vs Neh 10:14 Главы народа: Парош, Пахаф-Моав, Елам, Заффу, Вания,
\vs Neh 10:15 Вунний, Азгар, Бевай,
\vs Neh 10:16 Адония, Бигвай, Адин,
\vs Neh 10:17 Атер, Езекия, Азур,
\vs Neh 10:18 Годия, Хашум, Бецай,
\vs Neh 10:19 Хариф, Анафоф, Невай,
\vs Neh 10:20 Магпиаш, Мешуллам, Хезир,
\vs Neh 10:21 Мешезавел, Садок, Иаддуй,
\vs Neh 10:22 Фелатия, Ханан, Анаия,
\vs Neh 10:23 Осия, Ханания, Хашшув,
\vs Neh 10:24 Лохеш, Пилха, Шовек,
\vs Neh 10:25 Рехум, Хашавна, Маасея,
\vs Neh 10:26 Ахия, Ханан, Анан,
\vs Neh 10:27 Маллух, Харим, Ваана.
\vs Neh 10:28 И прочий народ, священники, левиты, привратники, певцы, нефинеи и все, отделившиеся от народов иноземных к закону Божию, жены их, сыновья их и дочери их, все, которые могли понимать,
\vs Neh 10:29 пристали к братьям своим, к почетнейшим из них, и вступили в обязательство с клятвою и проклятием~--- поступать по закону Божию, который дан рукою Моисея, раба Божия, и соблюдать и исполнять все заповеди Господа Бога нашего, и уставы Его и предписания Его,
\vs Neh 10:30 и не отдавать дочерей своих иноземным народам, и их дочерей не брать за сыновей своих;
\vs Neh 10:31 и когда иноземные народы будут привозить товары и все продажное в субботу, не брать у них в субботу и в священный день, и в седьмой год оставлять долги всякого рода.
\vs Neh 10:32 И поставили мы себе в закон давать от себя по трети сикля в год на потребности для дома Бога нашего:
\vs Neh 10:33 на хлебы предложения, на всегдашнее хлебное приношение и на всегдашнее всесожжение, на субботы, на новомесячия, на праздники, на священные вещи и на жертвы за грех для очищения Израиля, и на все, совершаемое в доме Бога нашего.
\vs Neh 10:34 И бросили мы жребии о доставке дров, священники, левиты и народ, когда которому поколению нашему в назначенные времена, из года в год, привозить \bibemph{их} к дому Бога нашего, чтоб они горели на жертвеннике Господа Бога нашего, по написанному в законе.
\vs Neh 10:35 \bibemph{И обязались мы} каждый год приносить в дом Господень начатки с земли нашей и начатки всяких плодов со всякого дерева;
\vs Neh 10:36 также приводить в дом Бога нашего к священникам, служащим в доме Бога нашего, первенцев из сыновей наших и из скота нашего, как написано в законе, и первородное от крупного и мелкого скота нашего.
\vs Neh 10:37 И начатки из молотого хлеба нашего и приношений наших, и плодов со всякого дерева, вина и масла мы будем доставлять священникам в кладовые при доме Бога нашего и десятину с земли нашей левитам. Они, левиты, будут брать десятину во всех городах, где у нас земледелие.
\vs Neh 10:38 При левитах, когда они будут брать левитскую десятину, будет находиться священник, сын Аарона, чтобы левиты десятину из своих десятин отвозили в дом Бога нашего в комнаты, \bibemph{отделенные} для кладовой,
\vs Neh 10:39 потому что в эти комнаты как сыны Израилевы, так и левиты должны доставлять приносимое в дар: хлеб, вино и масло. Там священные сосуды, и служащие священники, и привратники, и певцы. И мы не оставим д\acc{о}ма Бога нашего.
\vs Neh 11:1 И жили начальники народа в Иерусалиме, а прочие из народа бросили жребии, чтоб одна из десяти частей их шла на жительство в святой город Иерусалим, а девять \bibemph{оставались} в \bibemph{прочих} городах.
\rsbpar\vs Neh 11:2 И благословил народ всех, которые добровольно согласились жить в Иерусалиме.
\vs Neh 11:3 Вот главы страны, которые жили в Иерусалиме,~--- а в городах Иудеи жили, всякий в своем владении, по городам своим: Израильтяне, священники, левиты и нефинеи и сыновья рабов Соломоновых;~---
\vs Neh 11:4 в Иерусалиме жили из сыновей Иуды и из сыновей Вениамина. Из сыновей Иуды: Афаия, сын Уззии, сын Захарии, сын Амарии, сын Сафатии, сын Малелеила, из сыновей Фареса,
\vs Neh 11:5 и Маасея, сын Варуха, сын Колхозея, сын Хазаии, сын Адаии, сын Иоиарива, сын Захарии, сын Шилония.
\vs Neh 11:6 Всех сыновей Фареса, живших в Иерусалиме, четыреста шестьдесят восемь, люди отличные.
\vs Neh 11:7 И вот сыновья Вениамина: Саллу, сын Мешуллама, сын Иоеда, сын Федаии, сын Колаии, сын Маасеи, сын Ифиила, сын Исаии,
\vs Neh 11:8 и за ним Габбай, Саллай~--- девятьсот двадцать восемь.
\vs Neh 11:9 Иоиль, сын Зихри, был начальником над ними, а Иуда, сын Сенуи, был вторым над городом.
\vs Neh 11:10 Из священников: Иедаия, сын Иоиарива, Иахин,
\vs Neh 11:11 Сераия, сын Хелкии, сын Мешуллама, сын Садока, сын Мераиофа, сын Ахитува, начальствующий в доме Божием,
\vs Neh 11:12 и братья их, отправлявшие службу в доме \bibemph{Божием}~--- восемьсот двадцать два; и Адаия, сын Иерохама, сын Фелалии, сын Амция, сын Захарии, сын Пашхура, сын Малхии,
\vs Neh 11:13 и братья его, гл\acc{а}вы поколений~--- двести сорок два; и Амашсай, сын Азариила, сын Ахзая, сын Мешиллемофа, сын Иммера,
\vs Neh 11:14 и братья его, люди отличные~--- сто двадцать восемь. Начальником над ними был Завдиил, сын Гагедолима.
\vs Neh 11:15 А из левитов: Шемаия, сын Хашшува, сын Азрикама, сын Хашавии, сын Вунния,
\vs Neh 11:16 и Шавфай, и Иозавад из глав левитов по внешним делам дома Божия,
\vs Neh 11:17 и Матфания, сын Михи, сын Завдия, сын Асафа, главный начинатель славословия при молитве, и Бакбукия, второй \bibemph{по нем} из братьев его, и Авда, сын Шаммуя, сын Галала, сын Идифуна.
\vs Neh 11:18 Всех левитов во святом городе двести восемьдесят четыре.
\vs Neh 11:19 А привратники: Аккув, Талмон и братья их, содержавшие стражу у ворот~--- сто семьдесят два.
\rsbpar\vs Neh 11:20 Прочие Израильтяне, священники, левиты \bibemph{жили} по всем городам Иудеи, каждый в своем уделе.
\vs Neh 11:21 А нефинеи жили в Офеле; над нефинеями Циха и Гишфа.
\vs Neh 11:22 Начальником над левитами в Иерусалиме был Уззий, сын Вания, сын Хашавии, сын Матфании, сын Михи, из сыновей Асафовых, которые были певцами при служении в доме Божием,
\vs Neh 11:23 потому что от царя \bibemph{было} о них \bibemph{особое} повеление, и назначено было на каждый день для певцов определенное содержание.
\vs Neh 11:24 И Петахия, сын Мешезавела, из сыновей Зары, сына Иуды, был доверенным от царя по всяким делам, \bibemph{касающимся} до народа.
\vs Neh 11:25 Из \bibemph{живших} же в селах, на полях своих, сыновья Иуды жили в Кириаф-Арбе и зависящих от нее городах, в Дивоне и зависящих от него городах, в Иекавцеиле и селах его,
\vs Neh 11:26 в Иешуе, в Моладе и в Беф-Палете,
\vs Neh 11:27 в Хацар-Шуале, в Вирсавии и зависящих от нее городах,
\vs Neh 11:28 в Секелаге, в Мехоне и зависящих от нее городах,
\vs Neh 11:29 в Ен-Риммоне, в Цоре и в Иармуфе,
\vs Neh 11:30 в Заноахе, Одолламе и селах их, в Лахисе и на полях его, в Азеке и зависящих от нее городах. Они расположились от Вирсавии и до долины Енномовой.
\vs Neh 11:31 Сыновья Вениаминовы, \bibemph{начиная} от Гевы, в Михмасе, Гае, в Вефиле и зависящих от него городах,
\vs Neh 11:32 в Анафофе, Нове, Анании,
\vs Neh 11:33 Гацоре, Раме, Гиффаиме,
\vs Neh 11:34 Хадиде, Цевоиме, Неваллате,
\vs Neh 11:35 Лоде, Оно, в долине Харашиме.
\vs Neh 11:36 И левиты имели жилища свои в участках Иуды и Вениамина.
\vs Neh 12:1 Вот священники и левиты, которые пришли с Зоровавелем, сыном Салафииловым, и с Иисусом: Сераия, Иеремия, Ездра,
\vs Neh 12:2 Амария, Маллух, Хаттуш,
\vs Neh 12:3 Шехания, Рехум, Меремоф,
\vs Neh 12:4 Иддо, Гиннефой, Авия,
\vs Neh 12:5 Миямин, Маадия, Вилга,
\vs Neh 12:6 Шемаия, Иоиарив, Иедаия,
\vs Neh 12:7 Саллу, Амок, Хелкия, Иедаия. Это главы священников и братья их во дни Иисуса.
\vs Neh 12:8 А левиты: Иисус, Биннуй, Кадмиил, Шеревия, Иуда, Матфания, \bibemph{главный} при славословии, он и братья его,
\vs Neh 12:9 и Бакбукия и Унний, братья их, наряду с ними \bibemph{державшие} стражу.
\vs Neh 12:10 Иисус родил Иоакима, Иоаким родил Елиашива, Елиашив родил Иоиаду,
\vs Neh 12:11 Иоиада родил Ионафана, Ионафан родил Иаддуя.
\rsbpar\vs Neh 12:12 Во дни Иоакима были священники, гл\acc{а}вы поколений: из \bibemph{дома} Сераии Мераия, из \bibemph{дома} Иеремии Ханания,
\vs Neh 12:13 из \bibemph{дома} Ездры Мешуллам, из \bibemph{дома} Амарии Иоханан,
\vs Neh 12:14 из \bibemph{дома} Мелиху Ионафан, из \bibemph{дома} Шевании Иосиф,
\vs Neh 12:15 из \bibemph{дома} Харима Адна, из \bibemph{дома} Мераиофа Хелкия,
\vs Neh 12:16 из \bibemph{дома} Иддо Захария, из \bibemph{дома} Гиннефона Мешуллам,
\vs Neh 12:17 из \bibemph{дома} Авии Зихрий, из \bibemph{дома} Миниамина, из \bibemph{дома} Моадии Пилтай,
\vs Neh 12:18 из \bibemph{дома} Вилги Шаммуй, из \bibemph{дома} Шемаии Ионафан,
\vs Neh 12:19 из \bibemph{дома} Иоиарива Мафнай, из \bibemph{дома} Иедаии Уззий,
\vs Neh 12:20 из \bibemph{дома} Саллая Каллай, из \bibemph{дома} Амока Евер,
\vs Neh 12:21 из \bibemph{дома} Хелкии Хашавия, из \bibemph{дома} Иедаии Нафанаил.
\rsbpar\vs Neh 12:22 Левиты, гл\acc{а}вы поколений, внесены в запись во дни Елиашива, Иоиады, Иоханана и Иаддуя, и также священники в царствование Дария Персидского.
\vs Neh 12:23 Сыновья Левия, гл\acc{а}вы поколений, вписаны в летописи до дней Иоханана, сына Елиашивова.
\vs Neh 12:24 Гл\acc{а}вы левитов: Хашавия, Шеревия, и Иисус, сын Кадмиила, и братья их, при них \bibemph{поставленные} для славословия при благодарениях, по установлению Давида, человека Божия~--- смена за сменою.
\vs Neh 12:25 Матфания, Бакбукия, Овадия, Мешуллам, Талмон, Аккув~--- стражи, привратники на страже у порогов ворот.
\vs Neh 12:26 Они были во дни Иоакима, сына Иисусова, сына Иоседекова, и во дни областеначальника Неемии и книжника Ездры, священника.
\rsbpar\vs Neh 12:27 При освящении стены Иерусалимской потребовали левитов из всех мест их, приказывая им прийти в Иерусалим для совершения освящения и радостного празднества со славословиями и песнями при \bibemph{звуке} кимвалов, псалтирей и гуслей.
\vs Neh 12:28 И собрались сыновья певцов из округа Иерусалимского и из сел Нетофафских,
\vs Neh 12:29 и из Беф-Гаггилгала, и с полей Гевы и Азмавета, потому что певцы выстроили себе села в окрестностях Иерусалима.
\vs Neh 12:30 И очистились священники и левиты, и очистили народ и ворота, и стену.
\vs Neh 12:31 Тогда я повел начальствующих в Иудее на стену и поставил два больших хора для шествия, и один из них шел по правой стороне стены к Навозным воротам.
\vs Neh 12:32 За ними шел Гошаия и половина начальствующих в Иудее,
\vs Neh 12:33 Азария, Ездра и Мешуллам,
\vs Neh 12:34 Иуда и Вениамин, и Шемаия и Иеремия,
\vs Neh 12:35 а из сыновей священнических с трубами: Захария, сын Ионафана, сын Шемаии, сын Матфании, сын Михея, сын Закхура, сын Асафа,
\vs Neh 12:36 и братья его: Шемаия, Азариил, Милалай, Гилалай, Маай, Нафанаил, Иуда и Хананий с музыкальными орудиями Давида, человека Божия, и книжник Ездра впереди них.
\vs Neh 12:37 Подле ворот Источника, против них, они взошли по ступеням города Давидова, по лестнице, ведущей на стену сверх дома Давидова до Водяных ворот к востоку.
\vs Neh 12:38 Другой хор шел напротив них, и за ним я и половина народа, по стене от Печной башни и до широкой стены,
\vs Neh 12:39 и от ворот Ефремовых, мимо старых ворот и ворот Рыбных, и башни Хананела, и башни Меа, к Овечьим воротам, и остановились у ворот Темничных.
\vs Neh 12:40 Потом оба хора стали у дома Божия, и я и половина начальствующих со мною,
\vs Neh 12:41 и священники: Елиаким, Маасея, Миниамин, Михей, Елиоенай, Захария, Ханания с трубами,
\vs Neh 12:42 и Маасея и Шемаия, и Елеазар и Уззий, и Иоханан и Малхия, и Елам и Езер. И пели певцы громко; главным \bibemph{у них был} Израхия.
\rsbpar\vs Neh 12:43 И приносили в тот день большие жертвы и веселились, потому что Бог дал им великую радость. Веселились и жены и дети, и веселие Иерусалима далеко было слышно.
\vs Neh 12:44 В тот же день приставлены были люди к кладовым комнатам для приношений начатков и десятин, чтобы собирать с полей при городах части, положенные законом для священников и левитов, потому что Иудеям радостно было \bibemph{смотреть} на стоящих священников и левитов,
\vs Neh 12:45 которые совершали службу Богу своему и дела очищения и были певцами и привратниками по установлению Давида и сына его Соломона.
\vs Neh 12:46 Ибо издавна во дни Давида и Асафа были установлены главы певцов и песни Богу, хвалебные и благодарственные.
\vs Neh 12:47 Все Израильтяне во дни Зоровавеля и во дни Неемии давали части певцам и привратникам на каждый день и отдавали святыни левитам, а левиты отдавали святыни сынам Аарона.
\vs Neh 13:1 В тот день читано было из книги Моисеевой вслух народа и найдено написанное в ней: Аммонитянин и Моавитянин не может войти в общество Божие во веки,
\vs Neh 13:2 потому что они не встретили сынов Израиля с хлебом и водою и наняли против него Валаама, чтобы проклясть его, но Бог наш обратил проклятие в благословение.
\vs Neh 13:3 Услышав этот закон, они отделили все иноплеменное от Израиля.
\vs Neh 13:4 А прежде того священник Елиашив, приставленный к комнатам при доме Бога нашего, близкий родственник Товии,
\vs Neh 13:5 отделал для него большую комнату, в которую прежде клали хлебное приношение, ладан и сосуды, и десятины хлеба, вина и масла, положенные законом для левитов, певцов и привратников, и приношения для священников.
\vs Neh 13:6 Когда все это \bibemph{происходило}, я не был в Иерусалиме, потому что в тридцать втором году Вавилонского царя Артаксеркса я ходил к царю, и по прошествии нескольких дней \bibemph{опять} выпросился у царя.
\vs Neh 13:7 Когда я пришел в Иерусалим и узнал о худом деле, которое сделал Елиашив, отделав для Товии комнату на дворах дома Божия,
\vs Neh 13:8 тогда мне было весьма неприятно, и я выбросил все домашние вещи Товиины вон из комнаты
\vs Neh 13:9 и сказал, чтобы очистили комнаты, и велел опять внести туда сосуды дома Божия, хлебное приношение и ладан.
\vs Neh 13:10 Еще узнал я, что части левитам не отдаются, и что левиты и певцы, делавшие \bibemph{свое} дело, разбежались, каждый на свое поле.
\vs Neh 13:11 Я сделал \bibemph{за это} выговор начальствующим и сказал: зачем оставлен нами дом Божий? И я собрал их и поставил их на место их.
\vs Neh 13:12 И все Иудеи стали приносить десятины хлеба, вина и масла в кладовые.
\vs Neh 13:13 И приставил я к кладовым Шелемию священника и Садока книжника и Федаию из левитов, и при них Ханана, сына Закхура, сына Матфании, потому что они считались верными. И на них \bibemph{возложено} раздавать части братьям своим.
\vs Neh 13:14 Помяни меня за это, Боже мой, и не изгладь усердных дел моих, которые я сделал для дома Бога моего и для служения при нем!
\rsbpar\vs Neh 13:15 В те дни я увидел в Иудее, что в субботу топчут точила, возят снопы и навьючивают ослов вином, виноградом, смоквами и всяким грузом, и отвозят в субботний день в Иерусалим. И я строго выговорил \bibemph{им} в тот же день, когда они продавали съестное.
\vs Neh 13:16 И Тиряне жили в \bibemph{Иудее} и привозили рыбу и всякий товар и продавали в субботу жителям Иудеи и в Иерусалиме.
\vs Neh 13:17 И я сделал выговор знатнейшим из Иудеев и сказал им: зачем вы делаете такое зло и оскверняете день субботний?
\vs Neh 13:18 Не так ли поступали отцы ваши, и за то Бог наш навел на нас и на город сей все это бедствие? А вы увеличиваете гнев \bibemph{Его} на Израиля, оскверняя субботу.
\vs Neh 13:19 После сего, когда смеркалось у ворот Иерусалимских, перед субботою, я велел запирать двери и сказал, чтобы не отпирали их до \bibemph{утра} после субботы. И слуг моих я ставил у ворот, чтобы никакая ноша не проходила в день субботний.
\vs Neh 13:20 И ночевали торговцы и продавцы всякого товара вне Иерусалима раз и два.
\vs Neh 13:21 Но я строго выговорил им и сказал им: зачем вы ночуете возле стены? Если сделаете это в другой раз, я наложу руку на вас. С того времени они не приходили в субботу.
\vs Neh 13:22 И сказал я левитам, чтобы они очистились и пришли содержать стражу у ворот, дабы святить день субботний. И за сие помяни меня, Боже мой, и пощади меня по великой милости Твоей!
\rsbpar\vs Neh 13:23 Еще в те дни я видел Иудеев, которые взяли себе жен из Азотянок, Аммонитянок и Моавитянок;
\vs Neh 13:24 и оттого сыновья их в половину говорят по-азотски, или языком других народов, и не умеют говорить по-иудейски.
\vs Neh 13:25 Я сделал за это выговор и проклинал их, и некоторых из мужей бил, рвал у них волоса и заклинал их Богом, чтобы они не отдавали дочерей своих за сыновей их и не брали дочерей их за сыновей своих и за себя.
\vs Neh 13:26 Не из-за них ли, \bibemph{говорил я}, грешил Соломон, царь Израилев? У многих народов не было такого царя, как он. Он был любим Богом своим, и Бог поставил его царем над всеми Израильтянами; и однако же чужеземные жены ввели в грех и его.
\vs Neh 13:27 И можно ли нам слышать о вас, что вы делаете все сие великое зло, грешите пред Богом нашим, принимая в сожительство чужеземных жен?
\vs Neh 13:28 И из сыновей Иоиады, сына великого священника Елиашива, один был зятем Санаваллата, Хоронита. Я прогнал его от себя.
\vs Neh 13:29 Воспомяни им, Боже мой, что они опорочили священство и завет священнический и левитский!
\rsbpar\vs Neh 13:30 Так очистил я их от всего чужеземного и восстановил службы священников и левитов, каждого в деле его,
\vs Neh 13:31 и доставку дров в назначенные времена и начатки. Помяни меня, Боже мой, во благо \bibemph{мне}!

\bibbookdescr{2Ez}{
  inline={\LARGE Вторая книга\\\Huge Ездры\fns{Переведена с греческого.}},
  toc={2-я Ездры*},
  bookmark={2-я Ездры},
  header={2-я Ездры},
  %headerleft={},
  %headerright={},
  abbr={2~Езд}
}
\vs 2Ez 1:1 И совершил Иосия в Иерусалиме пасху Господу своему, и закололи пасхального агнца в четырнадцатый день первого месяца,
\vs 2Ez 1:2 поставив священников по чередам в облачении в храме Господнем.
\vs 2Ez 1:3 И сказал левитам, священнослужителям Израилевым: освятите себя Господу, для поставления святого ковчега Господня в храме, который построил царь Соломон, сын Давидов.
\vs 2Ez 1:4 Не нужно будет вам брать его на рамена; служите теперь Господу Богу вашему, и заботьтесь о народе Его Израиле, и устройтесь по родам и поколениям вашим, по расписанию Давида, царя Израилева, и по великолепию Соломона, сына его,
\vs 2Ez 1:5 и став во святилище, по родовым левитским разрядам вашим пред братьями вашими, сынами Израиля,
\vs 2Ez 1:6 заколите по уставу пасхального агнца и приготовьте жертвы для братьев ваших и совершите пасху по заповеди Господней, данной Моисею.
\rsbpar\vs 2Ez 1:7 И дал Иосия в дар находившемуся там народу тридцать тысяч агнцев и козлов и три тысячи тельцов; это по обету дано от царских стад народу и священникам и левитам.
\vs 2Ez 1:8 И дали Хелкия и Захария и Иеиил, начальствующие в храме, священникам на пасху две тысячи шестьсот овец и триста волов.
\vs 2Ez 1:9 И Иехония и Самей и Нафаниил, брат его, и Асавия и Охиил, и Иорам, тысяченачальники, дали левитам на пасху пять тысяч овец и семьсот волов.
\rsbpar\vs 2Ez 1:10 И когда это происходило, священники и левиты благолепно стояли по поколениям и родовым преимуществам, держа опресноки пред народом,
\vs 2Ez 1:11 чтобы приносить жертвы Господу по предписанному в книге Моисеевой. И это было в раннее время.
\vs 2Ez 1:12 И испекли пасхального агнца на огне, как надлежало, а жертвы сварили в медных сосудах и котлах с благовониями, и отнесли всему народу.
\vs 2Ez 1:13 А после того приготовили для себя и для священников, братьев своих, сынов Аарона.
\vs 2Ez 1:14 Ибо священники приносили тук до позднего времени, а потому левиты приготовляли для себя и для священников, братьев своих, сынов Аарона.
\vs 2Ez 1:15 Священнопевцы же, сыны Асафовы, находились на местах своих, по установлению Давида, и Асаф и Захария и Еддинус, который был от царя.
\vs 2Ez 1:16 И привратникам при каждых воротах не позволялось оставлять своей череды, потому что для них приготовляли братья их, левиты.
\rsbpar\vs 2Ez 1:17 И совершилось в тот день все, что принадлежало к жертвоприношению Господу при совершении пасхи,
\vs 2Ez 1:18 и к приношению всесожжений на жертвеннике Господнем, по повелению царя Иосии.
\vs 2Ez 1:19 И совершали сыны Израилевы, в то время находившиеся там, пасху и праздник опресноков семь дней.
\vs 2Ez 1:20 И не совершалось такой пасхи в Израиле от времен Самуила пророка.
\vs 2Ez 1:21 И ни один из всех царей Израильских не совершал такой пасхи, какую совершил Иосия, и священники и левиты, и Иудеи и все Израильтяне, находившиеся \bibemph{в то время} на жительстве в Иерусалиме.
\vs 2Ez 1:22 В восемнадцатый год царствования Иосии совершена сия пасха.
\vs 2Ez 1:23 И направлены были по прямому пути дела Иосии пред Господом от сердца, полного благочестия.
\rsbpar\vs 2Ez 1:24 Бывшее же при нем описано в прежних летописях о согрешавших и нечествовавших против Господа больше всякого народа и царства, и чем они сознательно оскорбляли Его, и за что слова Господа восстали против Израиля.
\rsbpar\vs 2Ez 1:25 И после всех сих деяний Иосии случилось, что фараон, царь Египетский, шел воевать в Каркамис при Евфрате, и Иосия вышел навстречу ему.
\vs 2Ez 1:26 Царь Египетский послал к нему сказать: что мне и тебе, царь Иудейский?
\vs 2Ez 1:27 Не против тебя послан я от Господа Бога; война моя на Евфрате, и ныне Господь со мною и Господь побуждает меня; отступи и не противься Господу.
\vs 2Ez 1:28 Но не возвратился Иосия на свою колесницу, а решился воевать с ним, не вняв словам Иеремии пророка из уст Господа.
\vs 2Ez 1:29 И вступил с ним в сражение на поле Мегиддо. И сошлись начальствующие к царю Иосии.
\vs 2Ez 1:30 И сказал царь слугам своим: унесите меня с поля сражения, потому что я очень изнемог. И слуги его тотчас вынесли его из строя.
\vs 2Ez 1:31 И взошел он на вторую колесницу свою и, возвратившись в Иерусалим, умер и погребен в гробнице отцов своих.
\vs 2Ez 1:32 И плакали об Иосии во всей Иудее, плакал об Иосии и пророк Иеремия, и начальствующие с женами оплакивали его до сего дня. И это передано навсегда всему роду Израилеву.
\rsbpar\vs 2Ez 1:33 Это написано в летописи царей Иудейских, и то, что сделано Иосиею, и слава его и его разумение закона Господня; прежние же дела его и ныне \bibemph{упоминаемые} описаны в книге царей Израильских и Иудейских.
\rsbpar\vs 2Ez 1:34 И взял народ Иехонию [Иоахаза], сына Иосии, и поставили его царем вместо Иосии, отца его, когда ему было двадцать три года.
\vs 2Ez 1:35 И царствовал он в Иудее и Иерусалиме три месяца, и отставил его царь Египетский, чтобы не царствовать ему в Иерусалиме.
\vs 2Ez 1:36 И наложил на народ сто талантов серебра и один талант золота.
\vs 2Ez 1:37 И поставил царь Египетский Иоакима, брата его, царем Иудеи и Иерусалима.
\vs 2Ez 1:38 И связал вельмож, а Заракина, брата его, отвел в Египет.
\rsbpar\vs 2Ez 1:39 Был же Иоаким двадцати пяти лет, когда воцарился над Иудеею и Иерусалимом, и делал он зло пред Господом.
\rsbpar\vs 2Ez 1:40 Против него вышел Навуходоносор, царь Вавилонский, и связал его медными узами и отвел в Вавилон.
\vs 2Ez 1:41 И, взяв некоторые из священных сосудов Господа, Навуходоносор перенес их и поставил в своем капище в Вавилоне.
\rsbpar\vs 2Ez 1:42 Сказания о нем, о его развращении и нечестии написаны в книге летописей царских.
\rsbpar\vs 2Ez 1:43 И воцарился вместо него Иоаким, сын его; был он восемнадцати лет, когда назначен царем.
\vs 2Ez 1:44 Царствовал же в Иерусалиме три месяца и десять дней, и сделал он зло пред Господом.
\vs 2Ez 1:45 И через год Навуходоносор послал и отвел его в Вавилон вместе со священными сосудами Господа,
\vs 2Ez 1:46 и назначил царем Иудеи и Иерусалима Седекию, который был двадцати одного года. Царствовал он одиннадцать лет.
\vs 2Ez 1:47 И делал он зло пред Господом, не вняв словам, сказанным пророком Иеремиею от уст Господа.
\vs 2Ez 1:48 И, быв связан от царя Навуходоносора клятвою во имя Господа, нарушил клятву, отложился и, ожесточив свою выю и сердце свое, преступил законы Господа Бога Израилева.
\vs 2Ez 1:49 Также и начальники народа и священников поступали весьма нечестиво, превосходя во всех нечистотах всех язычников, и осквернили освященный в Иерусалиме храм Господень.
\vs 2Ez 1:50 Бог отцов их посылал вестников Своих призывать их \bibemph{к обращению}, так как щадил Он их и жилище Свое;
\vs 2Ez 1:51 но они смеялись над вестниками Его: в тот самый день, в который Господь говорил, они насмехались над пророками Его,
\vs 2Ez 1:52 доколе Он, прогневавшись на народ Свой за нечестия, повелел восстать на них царям Халдейским.
\vs 2Ez 1:53 Они избили юношей их мечом вокруг святаго храма их и не пощадили ни юноши, ни девицы, ни старого, ни молодого, но все были преданы в руки их.
\vs 2Ez 1:54 И все священные сосуды Господни, большие и малые, и сосуды ковчега Господня и царские сокровища взяли они и отнесли в Вавилон.
\vs 2Ez 1:55 И сожгли дом Господень и разорили стены Иерусалима и башни его сожгли огнем,
\vs 2Ez 1:56 и все великолепие его обратили в ничто; оставшихся же от меча отвели в Вавилон.
\vs 2Ez 1:57 И они были рабами ему и сыновьям его до владычества Персов, в исполнение слова Господня из уст Иеремии:
\vs 2Ez 1:58 доколе земля не отпразднует суббот своих, во все время запустения своего, в продолжение семидесяти лет, она будет субботствовать.
\vs 2Ez 2:1 В первый год царствования Кира Персидского, в исполнение слова Господа из уст Иеремии,
\vs 2Ez 2:2 Господь подвиг дух Кира, царя Персидского, и он объявил по всему царству своему словесно и письменно:
\vs 2Ez 2:3 так говорит Кир, царь Персидский: Господь Израиля, Господь Всевышний поставил меня царем вселенной
\vs 2Ez 2:4 и повелел мне построить Ему дом в Иерусалиме, который в Иудее.
\vs 2Ez 2:5 Итак, кто есть из вас, из народа Его, да будет Господь его с ним, и пусть он, отправившись в Иерусалим, что в Иудее, строит дом Господа Израилева: Он есть Господь, живущий в Иерусалиме.
\vs 2Ez 2:6 Посему, сколько их живет по местам, жители места того пусть помогут им золотом и серебром,
\vs 2Ez 2:7 дарами коней и скота и другими обетными приношениями на храм Господа в Иерусалиме.
\rsbpar\vs 2Ez 2:8 И поднялись старейшины племен колена Иудина и Вениаминова и священники и левиты и все, которых дух подвиг Господь идти и строить дом Господу в Иерусалиме;
\vs 2Ez 2:9 а жившие в соседстве с ними всем помогали им: серебром и золотом, и конями и скотом и весьма многими обетными приношениями многих, которых дух подвигнут был.
\rsbpar\vs 2Ez 2:10 И царь Кир вынес священные сосуды Господа, которые Навуходоносор перенес из Иерусалима и поставил в своем капище.
\vs 2Ez 2:11 Вынеся же их, Кир, царь Персидский, передал их Мифридату, своему сокровищехранителю,
\vs 2Ez 2:12 а через него они переданы были Саманассару, князю Иудеи.
\vs 2Ez 2:13 Число же их было: возливальниц золотых тысяча, возливальниц серебряных тысяча, серебряных курильниц двадцать девять, чаш золотых тридцать, серебряных две тысячи четыреста десять, и других сосудов тысяча.
\vs 2Ez 2:14 Всех сосудов золотых и серебряных принесено пять тысяч четыреста шестьдесят девять.
\vs 2Ez 2:15 И принесены они Саманассаром и возвратившимися с ним из плена Вавилонского в Иерусалим.
\rsbpar\vs 2Ez 2:16 Во время же царствования Артаксеркса, царя Персидского, Вилем и Мифридат, и Тавеллий и Рафим, и Веелтефм и Самеллий писец и другие, согласившиеся с ними, обитавшие в Самарии и других местах, писали ему следующее письмо:
\vs 2Ez 2:17 Царю Артаксерксу, господину, рабы твои Рафим, описатель происшествий, и Самеллий писец, и прочие из совета их, и судьи, находящиеся в Келе-Сирии и Финикии.
\vs 2Ez 2:18 Да будет ныне известно господину царю, что вышедшие от вас к нам Иудеи, придя в Иерусалим, в этот отступнический и коварный город, устрояют площади его, возобновляют стены и полагают основание храма.
\vs 2Ez 2:19 Итак, если этот город будет отстроен и стены его окончены, то они не только не согласятся платить подати, но и восстанут против царей.
\vs 2Ez 2:20 И как уже начато построение храма, то мы за благо признали не пренебрегать этим,
\vs 2Ez 2:21 но известить господина царя, не благоугодно ли тебе посмотреть в книгах отцов твоих.
\vs 2Ez 2:22 Ты найдешь запись о том в памятных книгах, и узнаешь, что этот город был изменник и смущал царей и города,
\vs 2Ez 2:23 а Иудеи~--- отступники, вечно производившие в нем заговоры, по какой причине и был опустошен этот город.
\vs 2Ez 2:24 Итак, теперь извещаем тебя, господин царь, что если построится этот город и восстановятся стены его, то не будет для тебя прохода в Келе-Сирию и Финикию.
\rsbpar\vs 2Ez 2:25 Тогда царь написал в ответ Рафиму, описателю происшествий, и Веелтефму и Самеллию писцу и прочим, согласившимся с ними, и обитающим в Самарии и Сирии и Финикии, следующее:
\vs 2Ez 2:26 прочитал я письмо, которое вы прислали ко мне, и приказал рассмотреть; и найдено, что этот город издавна восстает против царей,
\vs 2Ez 2:27 и люди сии поднимают в нем мятежи и войны, и были цари в Иерусалиме сильные и могущественные, владевшие и собиравшие дань с Келе-Сирии и Финикии.
\vs 2Ez 2:28 Итак, теперь я приказал воспретить этим людям строить сей город, и наблюдать, чтобы ничего более не делалось
\vs 2Ez 2:29 и чтобы не имели дальнейшего успеха злонамеренные предприятия к беспокойству царей.
\vs 2Ez 2:30 По прочтении написанного от царя Артаксеркса, Рафим и Самеллий писец и согласившиеся с ними поспешно отправились в Иерусалим с конницею и ополчением народа
\vs 2Ez 2:31 и начали удерживать строящих. И остановилось строение Иерусалимского храма до второго года царствования Дария, царя Персидского.
\vs 2Ez 3:1 И сделал царь Дарий большой пир своим подданным и домашним своим и всем вельможам Мидии и Персии,
\vs 2Ez 3:2 и всем сатрапам и военачальникам, и начальникам подвластных ему стран от Индии и до Ефиопии в ста двадцати семи сатрапиях.
\vs 2Ez 3:3 И ели и пили и, насытившись, разошлись; царь же Дарий отправился в спальню свою и спал, и потом пробудился.
\rsbpar\vs 2Ez 3:4 Между тем трое юношей телохранителей, охранявших тело царя, сказали друг другу:
\vs 2Ez 3:5 пусть каждый из нас скажет одно слово о том, что всего сильнее? И чье слово окажется разумнее другого, даст тому царь Дарий великие дары и великую награду.
\vs 2Ez 3:6 И будет тот одеваться багряницею и пить из золотых сосудов, и спать на золоте, и ездить на колеснице с конями в золотых уздах, носить на голове повязку из виссона и ожерелье на шее,
\vs 2Ez 3:7 и сядет он вторым по Дарии за мудрость свою, и будет называться родственником Дария.
\vs 2Ez 3:8 И тотчас, написав каждый свое слово, запечатали и положили под изголовье царя Дария и сказали:
\vs 2Ez 3:9 когда царь встанет, подадут ему это писание, и за кем признает царь и трое вельмож Персидских, что слово его мудрее, тому дастся преимущество, как написано.
\vs 2Ez 3:10 Один написал: сильнее всего вино.
\vs 2Ez 3:11 Другой написал: сильнее царь.
\vs 2Ez 3:12 Третий написал: сильнее женщины, а над всем одерживает победу истина.
\vs 2Ez 3:13 И вот, когда царь встал, подали ему это писание, и он прочитал.
\vs 2Ez 3:14 И, послав, призвал всех вельмож Персии и Мидии, и сатрапов и военачальников, и начальников областей и советников,
\vs 2Ez 3:15 и сел в совещательной палате, и прочитано было пред ними писание.
\vs 2Ez 3:16 И сказал: призовите этих юношей, пусть они объяснят слова свои. И были призваны и вошли.
\vs 2Ez 3:17 И сказал им: объясните нам написанное.\rsbpar И начал первый, сказавший о силе вина, и говорил так:
\vs 2Ez 3:18 О, мужи! Как сильно вино! Оно приводит в омрачение ум всех людей, пьющих его;
\vs 2Ez 3:19 оно делает ум царя и сироты, раба и свободного, бедного и богатого, одним умом;
\vs 2Ez 3:20 и всякий ум превращает в веселие и радость, так что \bibemph{человек} не помнит никакой печали и никакого долга,
\vs 2Ez 3:21 и все сердца делает оно богатыми, так что \bibemph{никто} не думает ни о царе, ни о сатрапе, и всякого заставляет оно говорить о \bibemph{своих} талантах.
\vs 2Ez 3:22 И когда опьянеют, не помнят о приязни к друзьям и братьям и скоро обнажают мечи,
\vs 2Ez 3:23 а когда истрезвятся от вина, не помнят, что делали.
\vs 2Ez 3:24 О, мужи! Не сильнее ли всего вино, когда заставляет так поступать? И, сказав это, замолчал.
\vs 2Ez 4:1 И начал говорить второй, сказавший о силе царя.
\vs 2Ez 4:2 О, мужи! Не сильны ли люди, владеющие землею и морем и всем содержащимся в них?
\vs 2Ez 4:3 Но царь превозмогает и господствует над ними и повелевает ими, и во всем, что бы ни сказал им, они повинуются.
\vs 2Ez 4:4 Если скажет, чтоб они ополчались друг против друга, они исполняют; если пошлет их против неприятелей, они идут и разрушают горы и стены и башни,
\vs 2Ez 4:5 и убивают и бывают убиваемы, но не преступают слова царского; если же победят, всё приносят царю, что получат в добычу, и все прочее.
\vs 2Ez 4:6 И те, которые не ходят на войну и не сражаются, но возделывают землю, после посева, собрав жатву, также приносят царю
\vs 2Ez 4:7 и, понуждая один другого, приносят царю дани.
\vs 2Ez 4:8 И он один, если скажет убить~--- убивают; если скажет отпустить~--- отпускают; сказал бить~--- бьют;
\vs 2Ez 4:9 сказал опустошить~--- опустошают; сказал строить~--- строят; сказал срубить~--- срубают; сказал насадить~--- насаждают;
\vs 2Ez 4:10 и весь народ его и войско его повинуются ему. Кроме того, он возлежит, ест и пьет и спит,
\vs 2Ez 4:11 а они стерегут вокруг него и не могут никто отойти и делать дела свои, и не могут ослушаться его.
\vs 2Ez 4:12 О, мужи! Не сильнее ли всех царь, когда так повинуются ему?~--- И замолчал.
\rsbpar\vs 2Ez 4:13 Третий же, сказавший о женщинах и об истине,~--- это был Зоровавель,~--- начал говорить:
\vs 2Ez 4:14 О, мужи! Не велик ли царь, и многие из людей, и не сильно ли вино? Но кто господствует над ними и владеет ими? не женщины ли?
\vs 2Ez 4:15 Жены родили царя и весь народ, который владеет морем и землею;
\vs 2Ez 4:16 и от них родились и ими вскормлены насаждающие виноград, из которого делается вино;
\vs 2Ez 4:17 они делают одежды для людей и доставляют украшение людям, и люди не могут быть без жен.
\vs 2Ez 4:18 Если соберут золото и серебро и всякие драгоценности, а потом увидят одну женщину, хорошую лицом и красивую,
\vs 2Ez 4:19 оставив все, устремляются к ней и, раскрыв рот, смотрят на нее, и все прилепляются к ней более, чем к золоту и серебру и ко всякой дорогой вещи.
\vs 2Ez 4:20 Человек оставляет воспитавшего его отца и страну свою и прилепляется к жене своей,
\vs 2Ez 4:21 и с женою оставляет душу, и не помнит ни отца, ни матери, ни страны своей.
\vs 2Ez 4:22 И из этого должно вам познать, что женщины господствуют над вами. Не подъемлете ли вы трудов и не напрягаете ли усилий, и не отдаете ли и не приносите ли всего женам?
\vs 2Ez 4:23 Берет человек меч свой и отправляется, чтобы выходить на дороги и грабить и красть, и готов плавать по морю и рекам,
\vs 2Ez 4:24 льва встречает, и во тьме скитается; но лишь только украдет, похитит и ограбит, относит то к возлюбленной.
\vs 2Ez 4:25 И более любит человек жену свою, нежели отца и мать.
\vs 2Ez 4:26 Многие сошли с ума из-за женщин и сделались рабами через них.
\vs 2Ez 4:27 Многие погибли и сбились с пути и согрешили через женщин.
\vs 2Ez 4:28 Неужели теперь не поверите мне? Не велик ли царь властью своею? Не боятся ли все страны прикоснуться к нему?
\vs 2Ez 4:29 Я видел его и Апамину, дочь славного Вартака, царскую наложницу, сидящую по правую сторону царя;
\vs 2Ez 4:30 она снимала венец с головы царя и возлагала на себя, а левою рукою ударяла царя по щеке.
\vs 2Ez 4:31 И при всем том царь смотрел на нее, раскрыв рот: если она улыбнется ему, улыбается и он; если же она рассердится на него, он ласкает ее, чтобы помирилась с ним.
\vs 2Ez 4:32 О, мужи! Как же не сильны женщины, когда так поступают они?
\vs 2Ez 4:33 Тогда царь и вельможи взглянули друг на друга, а он начал говорить об истине.
\vs 2Ez 4:34 О, мужи! Не сильны ли женщины? Велика земля, и высоко небо, и быстро в своем течении солнце, ибо оно в один день обходит круг неба и опять возвращается на свое место.
\vs 2Ez 4:35 Не велик ли Тот, Кто совершает это? И истина велика и сильнее всего.
\vs 2Ez 4:36 Вся земля взывает к истине, и небо благословляет ее, и все дела трясутся и трепещут пред нею. И нет в ней неправды.
\vs 2Ez 4:37 Неправедно вино, неправеден царь, неправедны женщины, несправедливы все сыны человеческие и все дела их таковы, и нет в них истины, и они погибнут в неправде своей;
\vs 2Ez 4:38 а истина пребывает и остается сильною в век, и живет и владычествует в век века.
\vs 2Ez 4:39 И нет у ней лицеприятия и различения, но делает она справедливое, удаляясь от всего несправедливого и злого, и все одобряют дела ее.
\vs 2Ez 4:40 И нет в суде ее ничего неправого; она есть сила и царство и власть и величие всех веков: благословен Бог истины!
\vs 2Ez 4:41 И перестал он говорить. И все возгласили тогда и сказали: велика истина и сильнее всего.
\rsbpar\vs 2Ez 4:42 Тогда царь сказал ему: проси, чего хочешь, более написанного, и дадим тебе, так как ты оказался мудрейшим, и будешь сидеть подле меня и будешь называться родственником моим.
\vs 2Ez 4:43 Тогда сказал он царю: вспомни обещание, данное тобою в тот день, в который ты принял царство твое, что ты построишь Иерусалим
\vs 2Ez 4:44 и отошлешь все сосуды, взятые из Иерусалима, которые отобрал Кир, когда дал обеты разорить Вавилон, и обещался выслать \bibemph{их} туда.
\vs 2Ez 4:45 А ты обещался построить храм, который сожгли Идумеи, когда Иудея опустошена была Халдеями.
\vs 2Ez 4:46 И об этом самом теперь я прошу тебя, господин царь, и умоляю тебя, и в этом величие твое: прошу тебя исполнить обещание, которое ты устами твоими обещал Царю Небесному исполнить.
\rsbpar\vs 2Ez 4:47 Тогда царь Дарий, встав, поцеловал его, и написал ему письма ко всем правителям и начальникам областей и военачальникам и сатрапам, чтобы они пропустили его и с ним всех, идущих строить Иерусалим.
\vs 2Ez 4:48 Также писал письма ко всем местным начальникам в Келе-Сирии и Финикии и находящимся на Ливане, чтобы привозили с Ливана в Иерусалим кедровые дерева и помогали ему строить город.
\vs 2Ez 4:49 Писал о свободе и для всех Иудеев, отправляющихся из царства в Иудею, чтобы никто из имеющих власть, областной начальник и сатрап и правитель, не приходил к дверям их,
\vs 2Ez 4:50 но чтобы вся страна, которою они владеют, изъята была от даней, и чтоб Идумеи оставили селения Иудеев, которыми они владеют;
\vs 2Ez 4:51 также, чтобы даваемо было на построение храма каждогодно по двадцати талантов, доколе не будет построен;
\vs 2Ez 4:52 и для приношения на жертвенник каждодневных всесожжений, сверх семнадцати предписанных, даваемо было еще по десяти талантов в год;
\vs 2Ez 4:53 и чтобы всем отправляющимся из Вавилона была свобода строить город, как самим, так и потомкам их и всем священникам, которые пойдут.
\vs 2Ez 4:54 Писал также и о содержании и о священническом облачении, в котором служат.
\vs 2Ez 4:55 Написал давать содержание и левитам до того дня, когда совершится храм и построен будет Иерусалим;
\vs 2Ez 4:56 и всем, стерегущим город, предписал давать жалованье и продовольствие.
\vs 2Ez 4:57 Отпустил и все сосуды, которые отделил Кир из Вавилона; и всё, что велел сделать Кир, и он повелел исполнить и послать в Иерусалим.
\rsbpar\vs 2Ez 4:58 И когда вышел юноша, то устремил лице свое на небо против Иерусалима, возблагодарил Царя Небесного и сказал:
\vs 2Ez 4:59 от Тебя победа и от Тебя мудрость, и Твоя слава, а я Твой раб.
\vs 2Ez 4:60 Благословен Ты, даровавший мне мудрость, и благодарю Тебя, Господи, Боже отцов наших.
\vs 2Ez 4:61 И, взяв письма, отправился и пришел в Вавилон и объявил всем братьям своим.
\vs 2Ez 4:62 И они возблагодарили Бога отцов своих за то, что даровал им свободу и разрешение
\vs 2Ez 4:63 идти и строить Иерусалим и храм, на котором наречено имя Его. И ликовали с музыкою и веселием семь дней.
\vs 2Ez 5:1 После сего избраны были к отправлению родоначальники по коленам их, и жены их, и сыновья их, и дочери их, и рабы их, и рабыни их со скотом их.
\vs 2Ez 5:2 Дарий послал с ними тысячу конников, доколе они не введут их в Иерусалим с миром, с музыкою, с тимпанами и трубами.
\vs 2Ez 5:3 И все братья их веселились, и \bibemph{царь} дозволил им идти вместе.
\rsbpar\vs 2Ez 5:4 И вот имена мужей, шедших по племенам их в коленах по старшинству их:
\vs 2Ez 5:5 священники, сыны Финееса, сыны Аарона, Иисус, сын Иоседека, сына Сареева, и Иоаким, сын Зоровавеля, сына Салафииля из дома Давидова, из рода Фареса, колена же Иудова,
\vs 2Ez 5:6 который говорил пред Дарием, царем Персидским, мудрые слова на втором году царствования его, в месяце Нисане, месяце первом.
\vs 2Ez 5:7 Вот Иудеи, вышедшие из плена переселения, которых переселил в Вавилон Навуходоносор, царь Вавилонский,
\vs 2Ez 5:8 и которые возвратились в Иерусалим и в прочие места Иудеи, каждый в свой город,~--- вышедшие с Зоровавелем и Иисусом, Неемиею, Зареем, Рисеем, Енинеем, Мардохеем, Веельсаром, Асфарасом, Реелием, Роимом, Вааною, начальниками их.
\rsbpar\vs 2Ez 5:9 Число народа с начальниками их: сынов Фороса две тысячи сто семьдесят два; сынов Сафата четыреста семьдесят два;
\vs 2Ez 5:10 сынов Ареса семьсот пятьдесят шесть;
\vs 2Ez 5:11 сынов Фааф-Моава с сынами Иисуса и Иоава две тысячи восемьсот двенадцать;
\vs 2Ez 5:12 сынов Илама тысяча двести пятьдесят четыре; сынов Зафуи девятьсот семьдесят пять; сынов Хорве семьсот пять; сынов Ванния шестьсот сорок восемь;
\vs 2Ez 5:13 сынов Вивая шестьсот тридцать три; сынов Арге тысяча триста двадцать два;
\vs 2Ez 5:14 сынов Адоникама шестьсот тридцать семь; сынов Вагоя две тысячи шестьсот шесть; сынов Адина четыреста пятьдесят четыре;
\vs 2Ez 5:15 сынов Атира от Езекии девяносто два; сынов Килана и Азинана шестьдесят семь; сынов Азара четыреста тридцать два;
\vs 2Ez 5:16 сынов Анниса сто один; сынов Арома тридцать два; сынов Вассая триста двадцать три; сынов Арсифурифа сто два;
\vs 2Ez 5:17 сынов Ветируса три тысячи пять; сынов Вефломонских сто двадцать три;
\vs 2Ez 5:18 из Нетофаса пятьдесят пять; из Анафофа сто пятьдесят восемь; из Вефасмона сорок два;
\vs 2Ez 5:19 из Кариафири двадцать пять; из Кафира и Вирога семьсот сорок три;
\vs 2Ez 5:20 Хадиасеев и Аммидеев четыреста двадцать два; из Кирама и Гаввиса шестьсот двадцать один;
\vs 2Ez 5:21 из Макалона сто двадцать два; из Ветолия пятьдесят два; сынов Нифиса сто пятьдесят шесть;
\vs 2Ez 5:22 сынов Каламолала и Онуса семьсот двадцать пять; сынов Иереха двести сорок пять;
\vs 2Ez 5:23 сынов Санааса три тысячи триста один.
\vs 2Ez 5:24 Священников, сынов Иедду, сына Иисусова, с сынами Санасива, девятьсот семьдесят два; сынов Еммируфа тысяча пятьдесят два;
\vs 2Ez 5:25 сынов Фассура тысяча сорок семь; сынов Харми тысяча семнадцать.
\vs 2Ez 5:26 Левитов, сынов Иисуса и Кадмиила и Ванны и Судия, семьдесят четыре.
\vs 2Ez 5:27 Священнопевцов, сынов Асафа, сто сорок.
\vs 2Ez 5:28 Привратников, сынов Салума, сынов Атара, сынов Толмана, сынов Дакува, сынов Атита, сынов Товиса, всех сто тридцать девять.
\vs 2Ez 5:29 Служителей при храме, сынов Исава, сынов Асифа, сынов Таваофа, сынов Кираса, сынов Суда, сынов Фалея, сынов Лавана, сынов Аграва,
\vs 2Ez 5:30 сынов Акуда, сынов Ута, сынов Китава, сынов Аккава, сынов Сивая, сынов Анана, сынов Кафуа, сынов Геддура,
\vs 2Ez 5:31 сынов Иаира, сынов Десана, сынов Ноева, сынов Хасева, сынов Казира, сынов Озии, сынов Финое, сынов Асара, сынов Васфая, сынов Ассана, сынов Мани, сынов Нафиси, сынов Акуфа, сынов Ахива, сынов Асува, сынов Фаракема, сынов Васалема,
\vs 2Ez 5:32 сынов Меедда, сынов Куфа, сынов Хареа, сынов Вархуе, сынов Серара, сынов Фомоя, сынов Наси, сынов Атефа,
\vs 2Ez 5:33 сынов рабов Соломоновых, сынов Ассапфиофа, сынов Фарира, сынов Иеили, сынов Лозона, сынов Исдаила, сынов Сафии,
\vs 2Ez 5:34 сынов Агия, сынов Фахарефа, сынов Савии, сынов Сарофи, сынов Мисея, сынов Гаса, сынов Аддуса, сынов Сува, сынов Аферра, сынов Вародиса, сынов Сафага, сынов Аллома,
\vs 2Ez 5:35 всех служителей при храме и сынов рабов Соломоновых триста семьдесят два.
\rsbpar\vs 2Ez 5:36 Вот вышедшие из Фермелефа и Фелерса: начальник их Хараафалан и Аалар.
\vs 2Ez 5:37 Но они не могли показать отечеств своих и родов, что они от Израиля: сынов Далана, сына Ваенанова, сынов Некодана, шестьсот пятьдесят два.
\vs 2Ez 5:38 И из священников были исправлявшие священнослужение, но не найденные \bibemph{в списке}: сыны Овдия, сыны Аквоса, сыны Иадду, который взял в жену Авгию, из дочерей Верзеллия, и назывался его именем.
\vs 2Ez 5:39 И как родовая запись их по изыскании не найдена в списке, то они отлучены от священства.
\vs 2Ez 5:40 И сказал им Неемия и Атфария, чтобы они не участвовали в святынях, доколе не восстанет первосвященник, облеченный в урим и туммим.
\rsbpar\vs 2Ez 5:41 Всех же Израильтян от двенадцати лет и выше, кроме рабов и рабынь, было сорок две тысячи триста шестьдесят; рабов их и рабынь семь тысяч триста сорок семь; певцов и псалмопевцев двести сорок пять.
\vs 2Ez 5:42 Верблюдов четыреста тридцать пять, коней семь тысяч тридцать шесть, лошаков двести сорок пять, подъяремного скота пять тысяч пятьсот двадцать пять.
\vs 2Ez 5:43 Некоторые из родоначальников, когда пришли они ко храму Бога в Иерусалиме, дали обещание воздвигнуть сей дом на месте его по силе своей
\vs 2Ez 5:44 и дать в сокровищницу храма на построение тысячу мин золота и пять тысяч мин серебра и сто священнических одежд.
\vs 2Ez 5:45 И поселились священники и левиты и некоторые из народа в Иерусалиме и области его, а священнопевцы и привратники и весь Израиль в селениях своих.
\rsbpar\vs 2Ez 5:46 Когда же настал седьмой месяц и сыны Израиля были уже каждый во владении своем, собрались все единодушно на открытое место при первых воротах на восток.
\vs 2Ez 5:47 И встал Иисус, сын Иоседека, и братья его священники, и Зоровавель, сын Салафииля, и братья его, и устроили жертвенник Богу Израиля,
\vs 2Ez 5:48 чтобы возносить на нем всесожжения, как предписано в книге Моисея, человека Божия.
\vs 2Ez 5:49 И собрались к ним от иных народов, бывших в той земле, и устроили жертвенник на своем месте, ибо были во вражде с ними, и одолевали их все народы, бывшие в той земле; и они возносили жертвы в свое время и всесожжения Господу, утреннее и вечернее.
\vs 2Ez 5:50 И совершили праздник кущей, как предписано законом, \bibemph{вознося} каждодневные жертвы, как надлежало,
\vs 2Ez 5:51 и потом непрестанные приношения и жертвы суббот и новомесячий и всех святых праздников.
\vs 2Ez 5:52 И все те, которые обещали обеты Богу, с новомесячия седьмого месяца начали приносить жертвы Богу, хотя храм не был еще построен.
\vs 2Ez 5:53 И давали серебро каменотесам и плотникам и питье и пищу, и повозки Сидонянам и Тирянам, чтобы они привозили с Ливана кедровые дерева, доставляя их плотами в Иоппийскую пристань, по приказанию, данному им от Кира, царя Персидского.
\rsbpar\vs 2Ez 5:54 И на втором году во втором месяце, по прибытии ко храму Божию в Иерусалиме, Зоровавель, сын Салафииля, и Иисус, сын Иоседека, и братья их и священники, левиты и все, пришедшие в Иерусалим из плена,
\vs 2Ez 5:55 положили основание храму Божию в новолуние второго месяца второго года по прибытии их в Иудею и Иерусалим
\vs 2Ez 5:56 и приставили левитов от двадцати лет к делам Господним: и стал Иисус и сыновья его и братья, и Кадмиил брат и сыновья Имадавуна и сыновья Иода, сына Илиадудова, с сыновьями и братьями, все левиты, единодушно побуждая к делам в доме Господнем. И построили строители храм Господа.
\vs 2Ez 5:57 И стали священники в облачении с музыкальными инструментами и трубами и левиты, сыны Асафа, с кимвалами, воспевая Господу и прославляя Его по \bibemph{уставу} Давида, царя Израильского,
\vs 2Ez 5:58 и возглашали в песнях, прославляя Господа, что благость Его и слава вовек над всем Израилем.
\vs 2Ez 5:59 И весь народ трубил и взывал громким голосом, прославляя Господа за восстановление дома Господня.
\vs 2Ez 5:60 А старейшие из священников и левитов и родоначальников, видевшие прежний храм, пришли теперь на строение с плачем и громким воплем,
\vs 2Ez 5:61 а многие с трубами и радостными громкими восклицаниями,
\vs 2Ez 5:62 так что народ не мог слышать труб по причине воплей народных; хотя собрание громко трубило, так что далеко слышно было.
\rsbpar\vs 2Ez 5:63 И услышали враги колена Иудина и Вениаминова и пришли узнать, чт\acc{о} значит этот трубный звук.
\vs 2Ez 5:64 И узнали, что возвратившиеся из плена строят храм Господу Богу Израилеву.
\vs 2Ez 5:65 И, приступив к Зоровавелю и Иисусу и к родоначальникам, говорят им: будем и мы строить вместе с вами;
\vs 2Ez 5:66 ибо и мы, подобно вам, слушаем Господа вашего и приносим Ему жертвы от дней Асвакафаса, царя Ассирийского, который переселил нас сюда.
\vs 2Ez 5:67 Тогда сказал им Зоровавель и Иисус и начальники племен Израильских: не с вами нам строить дом Господу Богу нашему;
\vs 2Ez 5:68 мы одни будем строить его Господу Богу Израиля, соответственно тому, как повелел нам Кир, царь Персидский.
\vs 2Ez 5:69 Тогда народы той земли, нападая на обитающих в Иудее и осаждая их, препятствовали строению
\vs 2Ez 5:70 и, коварством увлекая народ и производя смуты, препятствовали довершить строение во все время жизни царя Кира и остановили строение на два года до воцарения Дария.
\vs 2Ez 6:1 Во второй год царствования Дария Аггей и Захария, сын Аддо, пророки, пророчествовали Иудеям, которые были в Иудее и Иерусалиме, от имени Господа Бога Израилева.
\vs 2Ez 6:2 Тогда встал Зоровавель, сын Салафииля, и Иисус, сын Иоседека, и начали строить дом Господа в Иерусалиме, в присутствии пророков Господних, помогавших им.
\rsbpar\vs 2Ez 6:3 В это время явился к ним Сисинни, правитель Сирии и Финикии, и Сафравузан и товарищи \bibemph{их} и сказали им:
\vs 2Ez 6:4 с чьего разрешения строите вы сей дом и сей кров, и все прочее совершаете? И кто строители, совершающие это?
\vs 2Ez 6:5 Но старейшины Иудейские обрели милость от Господа, призревшего на пленение,
\vs 2Ez 6:6 и им не запретили строить, пока возвещено будет о них Дарию. И получен был ответ.
\rsbpar\vs 2Ez 6:7 Вот список с письма, которое Сисинни писал и которое послали Дарию: Сисинни, правитель Сирии и Финикии, и Сафравузан и товарищи, начальники в Сирии и Финикии, царю Дарию радоваться.
\vs 2Ez 6:8 Да будет все известно господину нашему царю, что мы, придя в область Иудейскую и войдя в город Иерусалим, нашли в городе Иерусалиме возвратившихся из плена старейшин Иудейских,
\vs 2Ez 6:9 которые строят новый большой дом Господу из дорогих тесаных камней, полагая в стенах дерева;
\vs 2Ez 6:10 и работы сии производятся с ревностью, и дело успешно идет в руках их и совершается со всем великолепием и тщательностью.
\vs 2Ez 6:11 Тогда мы спросили этих старейшин, говоря: с чьего повеления строите вы этот дом и производите эти работы?
\vs 2Ez 6:12 И так мы спросили их, чтоб известить тебя и написать тебе о начальниках их, и требовали мы от них именной список предводителей их.
\vs 2Ez 6:13 Они же сказали нам в ответ: мы рабы Господа, создавшего небо и землю.
\vs 2Ez 6:14 И дом сей за много лет пред сим был строен царем Израильским, великим и сильным, и был окончен.
\vs 2Ez 6:15 Но как отцы наши грехами своими прогневали небесного Господа Израилева, то Он предал их в руки Навуходоносора, царя Вавилонского, царя Халдеев.
\vs 2Ez 6:16 Они, разрушив дом сей, сожгли, а народ отвели в плен в Вавилон.
\vs 2Ez 6:17 Но в первом году, по воцарении Кира над страною Вавилонскою, царь Кир предписал построить дом сей.
\vs 2Ez 6:18 И священные сосуды, золотые и серебряные, которые Навуходоносор вынес из храма Иерусалимского и поставил в своем капище, царь Кир опять вынес из капища Вавилонского и передал их князю Саманассару Зоровавелю.
\vs 2Ez 6:19 И повелено ему отнести все сии сосуды и положить в Иерусалимском храме и построить храм Господа на его месте.
\vs 2Ez 6:20 Тогда Саманассар, придя, положил основание дома Господа в Иерусалиме, и с того времени доныне он строился и не получил совершения.
\vs 2Ez 6:21 Итак, царь, если угодно, пусть поищут в царских книгохранилищах Кира,
\vs 2Ez 6:22 и если окажется, что строение дома Господня в Иерусалиме производилось по воле царя Кира, и угодно будет это господину царю нашему, пусть дано будет нам знать о том.
\rsbpar\vs 2Ez 6:23 Тогда царь Дарий приказал искать в книгохранилищах, находящихся в Вавилоне, и найдено в Екбатанах, в городе, находящемся в Мидийской области, одно место в памятной записи, где написано:
\vs 2Ez 6:24 в первый год царствования Кира, царь Кир повелел построить дом Господа в Иерусалиме, где приносят жертвы на огне неугасающем.
\vs 2Ez 6:25 Высота \bibemph{храма} шестьдесят локтей, ширина шестьдесят локтей, с тремя домами из тесаных камней и с одним новым из туземного дерева, а расходы производить из дома царя Кира,
\vs 2Ez 6:26 и священные сосуды дома Господня, золотые и серебряные, которые Навуходоносор вынес из дома Иерусалимского и перенес в Вавилон, возвратить в дом Иерусалимский, чтобы поставить их там, где они находились.
\vs 2Ez 6:27 Повелел также наблюдать Сисинни, правителю Сирии и Финикии, и Сафравузану и товарищам их и поставленным в Сирии и Финикии начальникам, чтобы они держали себя в стороне от сего места и оставили раба Господа Зоровавеля, князя Иудейского, и старейшин Иудейских строить этот дом Господа на его месте.
\vs 2Ez 6:28 Я повелел совершенно отстроить его и наблюдать, чтобы возвратившимся из плена Иудеям оказываемо было содействие к совершенному окончанию дома Господня
\vs 2Ez 6:29 и чтобы из податей Келе-Сирии и Финикии исправно давалось для этих людей, на жертвы Господу, князю Зоровавелю, на тельцов, овнов и агнцев.
\vs 2Ez 6:30 Равным образом, чтобы постоянно каждый год беспрекословно давалась пшеница, соль, вино и масло, как скажут находящиеся в Иерусалиме священники, сколько издерживается на каждый день;
\vs 2Ez 6:31 чтобы приносили Всевышнему Богу жертвы за царя и за детей его и молились о жизни их.
\vs 2Ez 6:32 Притом объявить, что если кто преступит или нарушит что-нибудь из написанного, то пусть взято будет дерево из его собственных, и он повешен будет на нем, а имущество его сделается царским.
\vs 2Ez 6:33 За это и Господь, Которого имя призывается там, да погубит всякого царя и народ, который прострет руку свою, чтобы воспрепятствовать или сделать какое-либо зло этому дому Господа в Иерусалиме.
\vs 2Ez 6:34 Я, царь Дарий, определил, чтобы в точности было по сему.
\vs 2Ez 7:1 Тогда Сисинни, правитель Келе-Сирии и Финикии, и Сафравузан и товарищи их, следуя повеленному от царя Дария,
\vs 2Ez 7:2 усердно принялись за святое дело, помогая старейшинам и священноначальникам Иудейским.
\vs 2Ez 7:3 И успешно шло святое дело, при пророчествах пророков Аггея и Захарии.
\vs 2Ez 7:4 И совершили всё по повелению Господа, Бога Израилева, и по воле Кира, Дария и Артаксеркса, царей Персидских.
\rsbpar\vs 2Ez 7:5 Окончен святый дом к двадцать третьему дню месяца Адара, на шестом году царя Дария.
\vs 2Ez 7:6 И сделали сыны Израиля, священники и левиты и прочие, возвратившиеся из плена, которые были приставлены, \bibemph{всё} по написанному в книге Моисея.
\vs 2Ez 7:7 И принесли \bibemph{в жертву} на обновление храма Господня сто волов, двести овнов, четыреста агнцев,
\vs 2Ez 7:8 двенадцать козлов за грехи всего Израиля, по числу двенадцати колен Израильских.
\vs 2Ez 7:9 И стояли священники и левиты по племенам, в облачении, при делах Господа Бога Израилева, согласно с книгою Моисеевою, и привратники при каждых воротах.
\vs 2Ez 7:10 И устроили возвратившиеся из плена сыны Израилевы пасху в четырнадцатый день первого месяца, когда очистились священники и левиты вместе,
\vs 2Ez 7:11 и все сыны пленения, потому что очистились, ибо левиты все вместе очистились.
\vs 2Ez 7:12 И закололи пасхальных агнцев для всех сынов плена, для братьев своих, священников, и для себя самих.
\vs 2Ez 7:13 И ели сыны Израилевы, возвратившиеся из плена, все, которые, удалившись от мерзостей народов земли, взыскали Господа.
\vs 2Ez 7:14 И праздновали праздник опресноков семь дней, радуясь пред Господом,
\vs 2Ez 7:15 что Он обратил к ним сердце царя Ассирийского, чтоб укрепить руки их на дела Господа Бога Израилева.
\vs 2Ez 8:1 После сих событий, в царствование Артаксеркса, царя Персидского, пришел Ездра, сын Азарии, Зехрия, Хелкия, Салима,
\vs 2Ez 8:2 Саддука, Ахитова, Амария, Озии, Мемерофа, Зарея, Сауя, Вокка, Ависая, Финееса, Елеазара, Аарона первосвященника.
\vs 2Ez 8:3 Сей Ездра пришел из Вавилона, как ученый, сведущий в законе Моисея, данном от Господа Бога Израилева,
\vs 2Ez 8:4 и оказал ему царь честь, ибо он снискал у него благоволение ко всем прошениям своим.
\vs 2Ez 8:5 И пришли с ним в Иерусалим некоторые из сынов Израиля, из священников и левитов, священнопевцов и привратников и служителей при храме,
\vs 2Ez 8:6 на седьмом году царствования Артаксеркса, в пятый месяц того же седьмого года царствования; ибо они, выйдя из Вавилона в новолуние первого месяца, пришли в Иерусалим, по данному им от Господа благопоспешению в пути, \bibemph{в новолуние пятого}.
\rsbpar\vs 2Ez 8:7 Ездра же прилагал великую заботу, чтобы ничего не опустить из закона Господня и заповедей, чтобы научить всего Израиля постановлениям и судам.
\vs 2Ez 8:8 Пришло и письменное повеление, данное от царя Артаксеркса Ездре, священнику и чтецу закона Господня, следующее:
\vs 2Ez 8:9 Царь Артаксеркс Ездре, священнику и чтецу закона Господня, радоваться.
\vs 2Ez 8:10 Рассудив человеколюбиво, я повелел, чтобы добровольно желающие из народа Иудейского и из священников и левитов, находящихся в нашем царстве, шли вместе с тобою в Иерусалим.
\vs 2Ez 8:11 Итак, кто только желает, пусть соберутся и идут, как рассудилось мне и моим семи ближайшим советникам;
\vs 2Ez 8:12 пусть увидят, что делается в Иудее и Иерусалиме согласно с законом Господним,
\vs 2Ez 8:13 и отнесут в Иерусалим дары Господу Израиля, которые обещал я и мои приближенные, и всякое золото и серебро, какое найдется в стране Вавилонской для Господа в Иерусалим, вместе с даяниями от народа на храм Господа Бога их, находящийся в Иерусалиме;
\vs 2Ez 8:14 золото же и серебро~--- на волов, овнов и агнцев и прочее к сему относящееся,
\vs 2Ez 8:15 чтобы возносить жертвы Господу на жертвеннике Господа Бога их в Иерусалиме.
\vs 2Ez 8:16 И все, что бы ни захотел ты с братьями твоими сделать на это золото и серебро, делай по воле Бога твоего.
\vs 2Ez 8:17 И священные сосуды Господни, данные тебе для употребления во храме Бога твоего в Иерусалиме, поставь пред Господом, Богом твоим.
\vs 2Ez 8:18 И прочее, что потребуется тебе на нужды храма Бога твоего, давай из царского казнохранилища.
\vs 2Ez 8:19 И вот я, царь Артаксеркс, повелел казнохранителям Сирии и Финикии, чтобы они всё, чего потребует Ездра, священник и чтец закона Всевышнего Бога, исправно давали ему, даже до ста талантов серебра,
\vs 2Ez 8:20 также пшеницы до ста к\acc{о}ров и вина до ста мер.
\vs 2Ez 8:21 И все другое по закону Божию тщательно да приносится Всевышнему Богу, чтобы не было гнева на царство царя и сынов его.
\vs 2Ez 8:22 И еще говорю вам, чтобы на всех священниках и левитах, и священнопевцах и привратниках, и служителях храма и на писцах сего храма не было никакой дани или другого налога и чтобы никто не имел власти налагать что-либо на них.
\vs 2Ez 8:23 А ты, Ездра, по мудрости Божией, поставь начальников и судей, чтобы они судили по всей Сирии и Финикии всех, знающих закон Бога твоего, а незнающих поучай:
\vs 2Ez 8:24 и все те, которые будут преступать закон Бога твоего или царский, пусть будут непременно наказываемы, смертью ли или телесным наказанием, денежною пенею или изгнанием.
\rsbpar\vs 2Ez 8:25 Тогда сказал ученый Ездра: благословен единый Господь Бог отцов моих, положивший на сердце царя прославить дом Его в Иерусалиме
\vs 2Ez 8:26 и почтивший меня пред царем и советниками и всеми приближенными и вельможами его.
\vs 2Ez 8:27 И я ободрился помощью Господа Бога моего, и собрал мужей Израильских, чтобы они шли со мною.
\rsbpar\vs 2Ez 8:28 И вот начальники по племенам их и по старейшинству, вышедшие со мною из Вавилона в царствование царя Артаксеркса:
\vs 2Ez 8:29 из сынов Финееса~--- Гирсон; из сынов Ифамара~--- Гамалиил; из сынов Давида~--- Латтус, сын Сехения;
\vs 2Ez 8:30 из сынов Фороса~--- Захария, и с ним записались сто пятьдесят человек;
\vs 2Ez 8:31 из сынов Фаафмоава~--- Елиаония, сын Зарея, и с ним двести человек;
\vs 2Ez 8:32 из сынов Зафоя~--- Сехения, сын Иезила, и с ним триста человек; из сынов Адина~--- Овиф, сын Ионафа, и с ним двести пятьдесят человек;
\vs 2Ez 8:33 из сынов Илама~--- Иесия, сын Гофолия, и с ним семьдесят человек;
\vs 2Ez 8:34 из сынов Сафатии~--- Зараия, сын Михаила, и с ним семьдесят человек;
\vs 2Ez 8:35 из сынов Иоава~--- Авадия, сын Иезила, и с ним двести двенадцать человек;
\vs 2Ez 8:36 из сынов Вания~--- Асалимоф, сын Иосафия, и с ним сто шестьдесят человек;
\vs 2Ez 8:37 из сынов Вавия~--- Захария, сын Вивая, и с ним двадцать восемь человек;
\vs 2Ez 8:38 из сынов Астафа~--- Иоанн, сын Акатана, и с ним сто десять человек;
\vs 2Ez 8:39 из сынов Адоникама~--- последние, и вот имена их: Елифала, сын Иеуила, и Самей, и с ними семьдесят человек;
\vs 2Ez 8:40 из сынов Вагоя~--- Уфий, сын Исталкура, и с ним семьдесят человек.
\vs 2Ez 8:41 И я собрал их при реке, называемой Феран, и мы пробыли там три дня, и я осмотрел их.
\vs 2Ez 8:42 И не найдя там \bibemph{никого} из священников и левитов,
\vs 2Ez 8:43 я послал к Елеазару и Идуилу, и Маасману, и Алнафану, и Мамею, и Самею, и Иоривону, Нафану, Еннатану, Захарии и Мосолламу, начальствующим и ученым,
\vs 2Ez 8:44 и сказал им, чтоб они пошли к Доддею, начальствующему в местности Касифье,
\vs 2Ez 8:45 приказав им сказать Доддею и братьям его и находящимся в той местности Касифье, чтобы они прислали нам священников для дома Господа Бога нашего.
\vs 2Ez 8:46 И они привели к нам мощною рукою Господа Бога нашего мужей сведущих из сынов Мооли, сына Левия, сына Израилева, Асевивея и сыновей его и братьев его, которых было восемнадцать человек;
\vs 2Ez 8:47 и Асевию и Аннуя и Осея брата из сыновей Ханунея, и сыновей их двадцать человек;
\vs 2Ez 8:48 и из служителей храма, которых дал Давид и начальники на служение левитам, двести двадцать служителей, с именным списком всех.
\vs 2Ez 8:49 И объявил я там пост пред Господом Богом нашим,
\vs 2Ez 8:50 чтоб испросить от Бога благополучного пути нам и спутникам нашим и детям нашим и скоту,
\vs 2Ez 8:51 ибо я постыдился просить у царя пеших и конных и проводников для безопасности от противников наших;
\vs 2Ez 8:52 потому что мы сказали царю, что сила Господа нашего будет с ищущими Его во всяком добром предприятии.
\rsbpar\vs 2Ez 8:53 Итак, мы снова помолились Господу, Богу нашему, обо всем этом и получили от Него великую милость.
\vs 2Ez 8:54 И отделил я из родоначальников и священников двенадцать человек, Есеревию и Самию, и с ними из братьев их десять человек.
\vs 2Ez 8:55 И свесил при них серебро и золото и священные сосуды дома Господа нашего, которые дал в дар царь и советники его и вельможи и все Израильтяне.
\vs 2Ez 8:56 И, свесив, передал им серебра шестьсот пятьдесят талантов и сосудов серебряных сто талантов, и золота сто талантов, сосудов золотых двадцать и сосудов медных из отличной меди, блистающих, как золото, двенадцать.
\vs 2Ez 8:57 И сказал им: и вы святы Господу, и сосуды сии святы, равно и золото и серебро, данное по обету Господу, Богу отцов наших.
\vs 2Ez 8:58 Бодрствуйте и берегите их, доколе не сдадите старшим священникам и левитам и родоначальникам Израильским в Иерусалиме, в сосудохранилища дома Бога нашего.
\vs 2Ez 8:59 И священники и левиты, приняв серебро и золото и сосуды для Иерусалима, внесли их в храм Господа.
\rsbpar\vs 2Ez 8:60 И, поднявшись от реки Феран в двенадцатый день первого месяца, мы шли в Иерусалим под мощною рукою Господа над нами, и Он избавлял нас с начала пути от всякого врага, и мы пришли в Иерусалим.
\vs 2Ez 8:61 И здесь, по прошествии трех дней, в день четвертый, взвешенное серебро и золото передано в дом Господа нашего Мармофе, сыну Урии, священнику.
\vs 2Ez 8:62 И был с ним Елеазар, сын Финееса; также были с ним Иосавдос, сын Иисуса, и Моеф, сын Саванна, левиты; и сдали всё числом и весом; и весь вес их записан в то же время.
\vs 2Ez 8:63 Тогда пришедшие из плена принесли жертвы Богу Израиля, двенадцать волов за всех Израильтян, девяносто шесть овнов, семьдесят два агнца, двенадцать козлов за спасение: все это~--- в жертву Господу,~---
\vs 2Ez 8:64 и передали царские повеления царским правителям и начальникам Келе-Сирии и Финикии, и они почтили народ и храм Господа.
\rsbpar\vs 2Ez 8:65 И когда это было окончено, приступили ко мне начальники и сказали:
\vs 2Ez 8:66 не отделился народ Израильский и начальники и священники и левиты от иноплеменных народов земли и от нечистот их, от народов Хананейских, и Хеттейских, и Ферезейских и Евусейских, и Моавитских и Египетских и Идумейских;
\vs 2Ez 8:67 ибо вступили в супружество с дочерями их, как сами, так и сыновья их, и смешалось семя святое с иноплеменными народами земли, и предводители их и вельможи сделались участниками в этом беззаконии с самого начала.
\vs 2Ez 8:68 Как скоро услышал я об этом, разодрал на себе одежды и священное облачение, и рвал волосы на голове и бороде, и сидел озабоченный и печальный.
\vs 2Ez 8:69 И когда я сетовал об этом беззаконии, собрались ко мне все, которые подвигнуты были словом Господа, Бога Израилева, и я сидел печальный до вечерней жертвы.
\rsbpar\vs 2Ez 8:70 Тогда, встав от поста моего, в разодранных одеждах и \bibemph{разодранном} священном облачении пал на колени и, простерши руки к Господу, я сказал:
\vs 2Ez 8:71 Господи! я стыжусь и смущаюсь пред лицем Твоим,
\vs 2Ez 8:72 ибо грехи наши поднялись выше голов наших, и безумия наши вознеслись до неба;
\vs 2Ez 8:73 еще от времен отцов наших и до сего дня мы находимся в великом грехе;
\vs 2Ez 8:74 и за грехи наши и отцов наших мы с братьями нашими и царями нашими и священниками нашими преданы были царям иноземным под меч, в плен и на разграбление с посрамлением до сего дня.
\vs 2Ez 8:75 Но теперь сколь великая оказана нам милость от Тебя, Господи Боже, что Ты оставил нам корень и имя на месте святыни Твоей,
\vs 2Ez 8:76 что открыл нам светильник в доме Господа, Бога нашего, дал нам пропитание во время порабощения нашего! И, когда мы находились в порабощении, не были оставлены Господом, Богом нашим;
\vs 2Ez 8:77 но Он поставил нас в благоволение у царей Персидских, чтобы они дали нам пропитание
\vs 2Ez 8:78 и прославили храм Господа нашего, и чтобы воздвигнут был опустошенный Сион, и нам дано было утверждение в Иудее и Иерусалиме.
\vs 2Ez 8:79 И ныне чт\acc{о} скажем мы, Господи, имея все сие? Мы преступили повеления Твои, которые Ты дал рукою рабов Твоих, пророков, говоря:
\vs 2Ez 8:80 земля, в которую вы входите, чтобы наследовать ее, осквернена сквернами иноплеменных земли, и они наполнили ее нечистотами своими.
\vs 2Ez 8:81 И теперь не отдавайте дочерей ваших в замужество за сыновей их, и их дочерей не берите за сыновей ваших,
\vs 2Ez 8:82 и не ищите мира с ними во все время, чтоб укрепиться вам и вкушать блага сей земли и оставить ее в наследие детям вашим навек.
\vs 2Ez 8:83 И все, что приключается нам, бывает за злые дела наши и за великие грехи наши. Ты, Господи, облегчил грехи наши
\vs 2Ez 8:84 и дал нам такой корень; но мы снова обратились к преступлению закона Твоего смешением с нечистотами народов земли.
\vs 2Ez 8:85 Не прогневался ли Ты на нас так, чтобы погубить нас и не оставить ни корня, ни семени, ни имени нашего?
\vs 2Ez 8:86 Ты истинен, Господи, Боже Израиля! ибо мы остались корнем до сего дня.
\vs 2Ez 8:87 Но вот ныне пред Тобою мы в беззакониях наших; и в них не надлежало бы стоять пред Тобою.
\rsbpar\vs 2Ez 8:88 И когда Ездра молился и исповедовался и плакал, распростершись на земле пред храмом, собралось к нему из Иерусалима весьма много народа: мужчины, женщины и дети; и был большой плач в народе.
\vs 2Ez 8:89 И, возгласив, Иехония, сын Иоиля, из сынов Израиля, сказал: Ездра! мы согрешили пред Господом, мы взяли иноплеменных жен из народов земли; и вот теперь здесь весь Израиль:
\vs 2Ez 8:90 да будет совершена нами клятва пред Господом в том, чтоб отвергнуть всех иноплеменных жен наших с детьми их, как рассудилось тебе и всем, которые повинуются закону Господа.
\vs 2Ez 8:91 Встав, соверши это! ибо твое это дело, и мы с тобою в силах будем сделать его.
\vs 2Ez 8:92 И, встав, Ездра заклял старших из священников и левитов всего Израиля поступить по сему, и они поклялись.
\vs 2Ez 9:1 И, встав, Ездра от притвора храма пошел в жилище Ионана, сына Елиасивова,
\vs 2Ez 9:2 и, пребывая там, не ел хлеба и не пил воды, скорбя о великих беззакониях народа.
\vs 2Ez 9:3 И было воззвание по всей Иудее и Иерусалиму ко всем, возвратившимся из плена, чтобы собрались в Иерусалим;
\vs 2Ez 9:4 а которые не явятся в течение двух или трех дней, у тех по суду председательствующих старейшин отнято будет имение, и сами они отчуждены будут от сонма бывших в плену.
\rsbpar\vs 2Ez 9:5 И в три дня собрались в Иерусалим все бывшие от колена Иудина и Вениаминова,~--- это было в девятый месяц, в двадцатый день сего месяца.
\vs 2Ez 9:6 И сидел весь народ во дворе храма, дрожа от наставшей зимы.
\vs 2Ez 9:7 Ездра, встав, сказал им: вы сделали беззаконие и живете с иноплеменными женами, прилагая грехи Израилю.
\vs 2Ez 9:8 Итак, воздайте теперь исповедание и славу Господу, Богу отцов наших,
\vs 2Ez 9:9 и сотворите волю Его, и отделитесь от народов земли и от жен иноплеменных!
\vs 2Ez 9:10 И возгласил весь сонм, и сказали громким голосом: как ты сказал, так мы и сделаем.
\vs 2Ez 9:11 Но сонм многочислен, и время зимнее, и мы не в силах стоять под открытым небом, а дело это для нас не одного дня и не двух дней, ибо весьма много мы согрешили в этом:
\vs 2Ez 9:12 посему пусть поставлены будут начальники над сонмом, и все те, которые из селений наших имеют иноплеменных жен, пусть в свое время приходят к ним
\vs 2Ez 9:13 со старейшинами и судьями каждого места, доколе не отвратится от нас гнев Божий за это дело.
\rsbpar\vs 2Ez 9:14 И приняли на себя это Ионафан, сын Асаила, и Езекия, сын Феоканы, а Месуллам и Левис и Савватей содействовали им.
\vs 2Ez 9:15 И исполнили по всему этому возвратившиеся из плена.
\vs 2Ez 9:16 И выбрал себе Ездра священник главных родоначальников всех поименно, и сошлись они в новолуние десятого месяца для исследования дела.
\vs 2Ez 9:17 И приведено к концу исследование о мужьях, державших при себе иноплеменных жен, к новолунию первого месяца.
\vs 2Ez 9:18 И нашлись из собравшихся священников, которые имели иноплеменных жен:
\vs 2Ez 9:19 из сынов Иисуса, сына Иоседекова, и из братьев его~--- Мафилас и Елеазар и Иорив и Иоадан,
\vs 2Ez 9:20 которые дали руки отвергнуть жен своих и принесли овнов в умилостивление за грех свой;
\vs 2Ez 9:21 и из сынов Еммира~--- Анания и Завдей, и Манис и Самей, и Иереил и Азария;
\vs 2Ez 9:22 и из сынов Фесура~--- Елионаис, Массия, Исмаил и Нафанаил и Окодил и Салоя;
\vs 2Ez 9:23 и из левитов~--- Иозавад и Семеис и Колий, он же Калита, и Пафей и Иуда и Иона;
\vs 2Ez 9:24 из священнопевцов~--- Елиасав, Вакхур;
\vs 2Ez 9:25 из привратников~--- Салум и Толван;
\vs 2Ez 9:26 из народа Израильского, из сынов Фороса~--- Иерма и Иезия, и Мелхия и Маил, и Елеазар и Асевия и Ванея;
\vs 2Ez 9:27 из сынов Ила~--- Матфания, Захария и Иезриил, и Иоавдий и Иеремоф и Аидия;
\vs 2Ez 9:28 из сынов Замофа~--- Елиада, Елеасим, Офония, Иаримоф и Сават и Зералия;
\vs 2Ez 9:29 и из сынов Виваия~--- Иоанн и Анания и Иозавад и Амафия;
\vs 2Ez 9:30 из сынов Мани~--- Олам, Мамух, Иедей, Иасув и Иасаил и Иеремоф;
\vs 2Ez 9:31 и из сынов Адди~--- Нааф и Моосия, Лаккун и Наид, Матфания и Сесфил и Валнуй и Манассия;
\vs 2Ez 9:32 и из сынов Анана~--- Елиона и Асаия, и Мелхия и Саввей, и Симон Хосамей;
\vs 2Ez 9:33 и из сынов Асома~--- Алтаней и Маттафия, и Саванней и Елифалат, и Манассия и Семей;
\vs 2Ez 9:34 и из сынов Ваания~--- Иеремия, Момдий, Исмаир, Иуил, Мафдай и Педия и Анос, Равасион и Енасив и Мамнитанем, Елиасис, Ваннус, Елиали, Сомей, Селемия, Нафания; и из сынов Езора~--- Сесис, Езрил, Азаил, Самат, Замри, Иосиф;
\vs 2Ez 9:35 и из сынов Ефма~--- Мазития, Завадей, Идей, Иуил, Ванея.
\vs 2Ez 9:36 Все сии жили с женами иноплеменными и отпустили их с детьми.
\rsbpar\vs 2Ez 9:37 И поселились в новолуние седьмого месяца священники и левиты и Израильтяне, бывшие в Иерусалиме и в области \bibemph{его}, и сыны Израиля в местах своих.
\vs 2Ez 9:38 И собрался единодушно весь народ на пространстве пред восточными воротами храма,
\vs 2Ez 9:39 и сказали Ездре, священнику и чтецу, чтобы он принес закон Моисея, данный от Господа, Бога Израилева.
\vs 2Ez 9:40 И вынес первосвященник Ездра закон ко всему народу~--- от мужчины до женщины, и ко всем священникам, чтобы слушали закон, в новолуние седьмого месяца,
\vs 2Ez 9:41 и читал им \bibemph{его} на пространстве пред воротами храма с утра до полудня пред мужчинами и женщинами, и весь народ внимал закону.
\vs 2Ez 9:42 И стал Ездра, священник и чтец, на приготовленном деревянном возвышении;
\vs 2Ez 9:43 и пред ним стояли с правой стороны Маттафия, Саммус, Анания, Азария, Урия, Езекия и Ваалсам,
\vs 2Ez 9:44 а с левой~--- Фалдей и Мисаил, Мелхия, Аофасув, Навария, Захария.
\vs 2Ez 9:45 И, взяв Ездра книгу закона пред народом, со славою сел пред всеми;
\vs 2Ez 9:46 и когда он объяснял закон, все стояли прямо; и благословил Ездра Господа Бога Всевышнего, Бога Саваофа, Вседержителя.
\vs 2Ez 9:47 И весь народ возгласил: аминь! и, подняв кверху руки и припав на землю, поклонились Господу.
\vs 2Ez 9:48 Также Иисус и Анниуф, и Саравия и Иадин и Иакув, Саватия, Автея, Меанна и Калита, Азария и Иозавд, и Анания и Фалия, левиты, поучали закону Господа и читали пред народом закон Господа, объясняя притом чтение.
\rsbpar\vs 2Ez 9:49 И сказал Атфарат Ездре, первосвященнику и чтецу, и левитам, которые поучали народ, ко всем:
\vs 2Ez 9:50 день сей свят Господу, и все плакали во время слушания закона;
\vs 2Ez 9:51 идите и ешьте тучное, и пейте сладкое, и пошлите подаяния неимущим,
\vs 2Ez 9:52 ибо день сей свят Господу, и потому не скорбите, ибо Господь прославит вас.
\vs 2Ez 9:53 Также и левиты внушали всему народу и говорили: день сей свят, не скорбите.
\vs 2Ez 9:54 И пошли все есть и пить и веселиться, и подавать подаяния неимущим, и веселились много,
\vs 2Ez 9:55 ибо они проникнуты были словами, которым поучаемы были в собрании.
\newbookpage
\bibbookdescr{Tob}{
  inline={\LARGE Книга\\\Huge Товита\fns{Переведена с греческого.}},
  toc={Товит*},
  bookmark={Товит},
  header={Товит},
  %headerleft={},
  %headerright={},
  abbr={Тов}
}
\vs Tob 1:1 Книга сказаний Товита, сына Товиилова, Ананиилова, Адуилова, Гаваилова, из племени Асиилова, из колена Неффалимова,
\vs Tob 1:2 который во дни Ассирийского царя Енемессара взят был в плен из Фисвы, находящейся по правую \bibemph{сторону} Кидия Неффалимова, в Галилее, выше Асира. Я, Товит, во все дни жизни моей ходил путями истины и правды
\vs Tob 1:3 и делал много благодеяний братьям моим и народу моему, пришедшим вместе со мною в страну Ассирийскую, в Ниневию.
\vs Tob 1:4 Когда я жил в стране моей, в земле Израиля, будучи еще юношею, тогда все колено Неффалима, отца моего, находилось в отпадении от дома Иерусалима, избранного от всех колен Израиля, чтобы всем им приносить \bibemph{там} жертвы, где освящен храм селения Всевышнего и утвержден во все роды навек.
\vs Tob 1:5 Как все отложившиеся колена приносили жертвы Ваалу, юнице, так и дом Неффалима, отца моего.
\vs Tob 1:6 Я же один часто ходил в Иерусалим на праздники, как предписано всему Израилю установлением вечным, с начатками и десятинами произведений \bibemph{земли} и начатками шерсти овец,
\vs Tob 1:7 и отдавал это священникам, сынам Аароновым, для жертвенника: десятину всех произведений давал сынам Левииным, служащим в Иерусалиме; другую десятину продавал, и каждый год ходил и издерживал ее в Иерусалиме;
\vs Tob 1:8 а третью давал, кому следовало, как заповедала мне Деввора, мать отца моего, когда я после отца моего остался сиротою.
\vs Tob 1:9 Достигнув мужеского возраста, я взял жену Анну из отеческого нашего рода и родил от нее Товию.
\vs Tob 1:10 Когда я отведен был в плен в Ниневию, все братья мои и одноплеменники мои ели от снедей языческих,
\vs Tob 1:11 а я соблюдал душу мою и не ел,
\vs Tob 1:12 ибо я помнил Бога всею душею моею.
\vs Tob 1:13 И даровал мне Всевышний милость и благоволение у Енемессара, и я был у него поставщиком;
\vs Tob 1:14 и ходил в Мидию, и отдал \bibemph{на сохранение} Гаваилу, брату Гаврия, в Рагах Мидийских, десять талантов серебра.
\vs Tob 1:15 Когда же умер Енемессар, вместо него воцарился сын его Сеннахирим, которого пути не были постоянны, и я уже не мог ходить в Мидию.
\vs Tob 1:16 Во дни Енемессара я делал много благодеяний братьям моим:
\vs Tob 1:17 алчущим давал хлеб мой, нагим одежды мои и, если кого из племени моего видел умершим и выброшенным за стену Ниневии, погребал его.
\vs Tob 1:18 Тайно погребал я и тех, которых убивал царь Сеннахирим, когда, обращенный в бегство, возвратился из Иудеи. А он многих умертвил в ярости своей. И отыскивал царь трупы, но их не находили.
\vs Tob 1:19 Один из Ниневитян пошел и донес царю, что я погребаю их; тогда я скрылся. Узнав же, что меня ищут убить, от страха убежал \bibemph{из города}.
\vs Tob 1:20 И было расхищено все имущество мое, и не осталось у меня ничего, кроме Анны, жены моей, и Товии, сына моего.
\vs Tob 1:21 Но не прошло пятидесяти дней, как два сына его убили его и убежали в горы Араратские. И воцарился вместо него сын его Сахердан, который поставил Ахиахара Анаила, сына брата моего, над всею счетною частью царства своего и над всем домоправлением.
\vs Tob 1:22 И ходатайствовал Ахиахар за меня, и я возвратился в Ниневию. Ахиахар же был и виночерпий и хранитель перстня, и домоправитель и казначей; и Сахердан поставил его вторым по себе; он был сын брата моего.
\vs Tob 2:1 Когда я возвратился в дом свой, и отданы мне были Анна, жена моя, и Товия, сын мой, в праздник пятидесятницы, в святую седмицу седмиц, приготовлен у меня был хороший обед, и я возлег есть.
\vs Tob 2:2 Увидев много снедей, я сказал сыну моему: пойди и приведи, кого найдешь, бедного из братьев наших, который помнит Господа, а я подожду тебя.
\vs Tob 2:3 И пришел он и сказал: отец \bibemph{мой}, один из племени нашего удавленный брошен на площади.
\vs Tob 2:4 Тогда я, прежде нежели стал есть, поспешно выйдя, убрал его в одно жилье до захождения солнца.
\vs Tob 2:5 Возвратившись, совершил омовение и ел хлеб мой в скорби.
\vs Tob 2:6 И вспомнил я пророчество Амоса, как он сказал: праздники ваши обратятся в скорбь, и все увеселения ваши~--- в плач.
\vs Tob 2:7 И я плакал. Когда же зашло солнце, я пошел и, выкопав \bibemph{могилу}, похоронил его.
\vs Tob 2:8 Соседи насмехались \bibemph{надо мною} и говорили: еще не боится он быть убитым за это дело; бегал уже, и вот опять погребает мертвых.
\vs Tob 2:9 В эту самую ночь, возвратившись после погребения и будучи нечистым, я лег спать за стеною двора, и лице мое не было покрыто.
\vs Tob 2:10 И не заметил я, что на стене были воробьи. Когда глаза мои были открыты, воробьи испустили теплое на глаза мои, и сделались на глазах моих бельма. И ходил я к врачам, но они не помогли мне. Ахиахар доставлял мне пропитание, доколе не отправился в Елимаиду.
\vs Tob 2:11 А потом жена моя Анна в женских отделениях пряла шерсть
\vs Tob 2:12 и посылала богатым людям, которые давали ей плату и однажды в придачу дали козленка.
\vs Tob 2:13 Когда принесли его ко мне, он начал блеять; и я спросил \bibemph{жену}: откуда этот козленок? не краденый ли? отдай его, кому он принадлежит! ибо непозволительно есть краденое.
\vs Tob 2:14 Она отвечала: это подарили мне сверх платы. Но я не верил ей и настаивал, чтобы отдала его, кому он принадлежит, и разгневался на нее. А она в ответ сказала мне: где же милостыни твои и праведные дела? вот как все они обнаружились на тебе!
\vs Tob 3:1 Опечалившись, я заплакал и молился со скорбью, говоря:
\vs Tob 3:2 праведен Ты, Господи, и все дела Твои и все пути Твои~--- милость и истина, и судом истинным и правым судишь Ты вовек!
\vs Tob 3:3 Воспомяни меня и призри на меня: не наказывай меня за грехи мои и заблуждения мои и отцов моих, которыми они согрешили пред Тобою!
\vs Tob 3:4 Ибо они не послушали заповедей Твоих, и Ты предал нас на расхищение и пленение и смерть, и в притчу поношения пред всеми народами, между которыми мы рассеяны.
\vs Tob 3:5 И, поистине, многи и праведны суды Твои~--- делать со мною по грехам моим и грехам отцов моих, потому что не исполняли заповедей Твоих и не поступали по правде пред Тобою.
\vs Tob 3:6 Итак, твори со мною, что Тебе благоугодно; повели взять дух мой, чтобы я разрешился и обратился в землю, ибо мне лучше умереть, нежели жить, так как я слышу лживые упреки, и глубока скорбь во мне! Повели освободить меня от этой тяготы в обитель вечную и не отврати лица Твоего от меня.
\vs Tob 3:7 В тот самый день случилось и Сарре, дочери Рагуиловой, в Екбатанах Мидийских терпеть укоризны от служанок отца своего
\vs Tob 3:8 за то, что она была отдаваема семи мужьям, но Асмодей, злой дух, умерщвлял их прежде, нежели они были с нею, как с женою. Они говорили ей: разве тебе не совестно, что ты задушила мужей твоих? Уже семерых ты имела, но не назвалась именем ни одного из них.
\vs Tob 3:9 Что нас бить за них? Они умерли: иди и ты за ними, чтобы нам не видеть твоего сына или дочери вовек!
\vs Tob 3:10 Услышав это, она весьма опечалилась, так что решилась было лишить себя жизни, но подумала: я одна у отца моего; если сделаю это, бесчестие ему будет, и я сведу старость его с печалью в преисподнюю.
\vs Tob 3:11 И стала она молиться у окна и говорила: благословен Ты, Господи Боже мой, и благословенно имя Твое святое и славное вовеки: да благословляют Тебя все творения Твои вовек!
\vs Tob 3:12 И ныне к Тебе, Господи, обращаю очи мои и лице мое;
\vs Tob 3:13 молю, возьми меня от земли сей и не дай мне слышать еще укоризны!
\vs Tob 3:14 Ты знаешь, Господи, что я чиста от всякого греха с мужем
\vs Tob 3:15 и не обесчестила имени моего, ни имени отца моего в земле плена моего; я единородная у отца моего, и нет у него сына, который мог бы наследовать ему, ни брата близкого, ни сына братнего, которому я могла бы сберечь себя в жену: уже семеро погибли у меня. Для чего же мне жить? А если не угодно Тебе умертвить меня, то благоволи призреть на меня и помиловать меня, чтобы мне не слышать более укоризны!
\vs Tob 3:16 И услышана была молитва обоих пред славою великого Бога, и послан был Рафаил исцелить обоих:
\vs Tob 3:17 снять бельма у Товита и Сарру, дочь Рагуилову, дать в жену Товии, сыну Товитову, связав Асмодея, злого духа; ибо Товии предназначено наследовать ее.~--- И в одно и то же время Товит, по возвращении, вошел в дом свой, а Сарра, дочь Рагуилова, сошла с горницы своей.
\vs Tob 4:1 В тот день вспомнил Товит о серебре, которое отдал на сохранение Гаваилу в Рагах Мидийских,
\vs Tob 4:2 и сказал сам себе: я просил смерти; что же не позову сына моего Товии, чтобы объявить ему об этом, пока я не умер?
\vs Tob 4:3 И, призвав его, сказал: сын \bibemph{мой}! когда я умру, похорони меня и не покидай матери своей; почитай ее во все дни жизни твоей, делай угодное ей и не причиняй ей огорчения.
\vs Tob 4:4 Помни, сын мой, что она много имела скорбей из-за тебя \bibemph{еще} во время чревоношения. Когда она умрет, похорони ее подле меня в одном гробе.
\vs Tob 4:5 Во все дни помни, сын \bibemph{мой}, Господа Бога нашего и не желай грешить и преступать заповеди Его. Во все дни жизни твоей делай правду и не ходи путями беззакония,
\vs Tob 4:6 ибо, если ты будешь поступать по истине, в делах твоих будет успех, как у всех поступающих по правде.
\vs Tob 4:7 Из имения твоего подавай милостыню, и да не жалеет глаз твой, когда будешь творить милостыню. Ни от какого нищего не отвращай лица твоего, тогда и от тебя не отвратится лице Божие.
\vs Tob 4:8 Когда у тебя будет много, твори из того милостыню, и когда у тебя будет мало, не бойся творить милостыню и понемногу;
\vs Tob 4:9 ты запасешь себе богатое сокровище на день нужды,
\vs Tob 4:10 ибо милостыня избавляет от смерти и не попускает сойти во тьму.
\vs Tob 4:11 Милостыня есть богатый дар для всех, кто творит ее пред Всевышним.
\vs Tob 4:12 Берегись, сын \bibemph{мой}, всякого \bibemph{вида} распутства. Возьми себе жену из племени отцов твоих, но не бери жены иноземной, которая не из колена отца твоего, ибо мы сыны пророков. Издревле отцы наши~--- Ной, Авраам, Исаак и Иаков. Помни, сын \bibemph{мой}, что все они брали жен из \bibemph{среды} братьев своих и были благословенны в детях своих, и потомство их наследует землю.
\vs Tob 4:13 Итак, сын \bibemph{мой}, люби братьев твоих и не превозносись сердцем пред братьями твоими и пред сынами и дочерями народа твоего, чтобы не от них взять тебе жену, потому что от гордости~--- погибель и великое неустройство, а от непотребства~--- оскудение и разорение: непотребство есть мать голода.
\vs Tob 4:14 Плата наемника, который будет работать у тебя, да не переночует у тебя, а отдавай ее тотчас: и тебе воздастся, если будешь служить Богу. Будь осторожен, сын \bibemph{мой}, во всех поступках твоих и будь благоразумен во всем поведении твоем.
\vs Tob 4:15 Что ненавистно тебе самому, того не делай никому. Вина до опьянения не пей, и пьянство да не ходит с тобою в пути твоем.
\vs Tob 4:16 Давай алчущему от хлеба твоего и нагим от одежд твоих; от всего, в чем у тебя избыток, твори милостыни, и да не жалеет глаз твой, когда будешь творить милостыню.
\vs Tob 4:17 Раздавай хлебы твои при гробе праведных, но не давай грешникам.
\vs Tob 4:18 У всякого благоразумного проси совета, и не пренебрегай советом полезным.
\vs Tob 4:19 Благословляй Господа Бога во всякое время и проси у Него, чтобы пути твои были правы и все дела и намерения твои благоуспешны, ибо ни один народ не властен в \bibemph{успехе} начинаний, но Сам Господь ниспосылает все благое и, кого хочет, уничижает по Своей воле. Помни же, сын \bibemph{мой}, заповеди мои, и да не изгладятся они из сердца твоего!
\vs Tob 4:20 Теперь я открою тебе, что я отдал десять талантов серебра на сохранение Гаваилу, сыну Гавриеву, в Рагах Мидийских.
\vs Tob 4:21 Не бойся, сын \bibemph{мой}, что мы обнищали: у тебя много, если ты будешь бояться Господа и, удаляясь от всякого греха, делать угодное пред Ним.
\vs Tob 5:1 И сказал Товия в ответ ему: отец \bibemph{мой}, я исполню все, что ты завещаешь мне;
\vs Tob 5:2 но как я могу получить серебро, не зная того \bibemph{человека}?
\vs Tob 5:3 Тогда \bibemph{отец} дал ему расписку и сказал: найди себе человека, который сопутствовал бы тебе; я дам ему плату, пока еще жив, и ступайте за серебром.
\rsbpar\vs Tob 5:4 И пошел он искать человека и встретил Рафаила. Это был Ангел, но он не знал
\vs Tob 5:5 и сказал ему: можешь ли ты идти со мною в Раги Мидийские и знаешь ли эти места?
\vs Tob 5:6 Ангел отвечал: могу идти с тобою и дорогу знаю; я уже останавливался у Гаваила, брата нашего.
\vs Tob 5:7 И сказал ему Товия: подожди меня, я скажу отцу моему.
\vs Tob 5:8 Тот сказал: ступай, только не медли.
\vs Tob 5:9 Он, придя, сказал отцу: вот я нашел себе спутника. \bibemph{Отец} сказал: пригласи его ко мне; я узнаю, из какого он колена и надежный ли спутник тебе.
\vs Tob 5:10 И позвал его, и он вошел, и приветствовали друг друга.
\vs Tob 5:11 Товит спросил: скажи мне, брат, из какого ты колена и из какого рода?
\vs Tob 5:12 Он отвечал: колена и рода ты ищешь или наемника, который пошел бы с сыном твоим? И сказал ему Товит: брат, мне хочется знать род твой и имя твое.
\vs Tob 5:13 Он сказал: я Азария, \bibemph{из рода} Анании великого, из братьев твоих.
\vs Tob 5:14 Тогда \bibemph{Товит} сказал ему: брат, иди благополучно, и не гневайся на меня за то, что я спросил о колене и роде твоем. Ты доводишься брат мне, из честного и доброго рода. Я знал Ананию и Ионафана, сыновей Семея великого; мы вместе ходили в Иерусалим на поклонение, с первородными и десятинами \bibemph{земных} произведений, ибо не увлекались заблуждением братьев наших: ты, брат, от хорошего корня!
\vs Tob 5:15 Но скажи мне: какую плату я должен буду дать тебе? Я дам тебе драхму на день и все необходимое для тебя и для сына моего,
\vs Tob 5:16 и еще прибавлю тебе сверх этой платы, если благополучно возвратитесь.
\vs Tob 5:17 Так и условились. Тогда он сказал Товии: будь готов в путь, и отправляйтесь благополучно. И приготовил сын его нужное для пути. И сказал ему отец: иди с этим человеком; живущий же на небесах Бог да благоустроит путь ваш, и Ангел Его да сопутствует вам!~--- И отправились оба, и собака юноши с ними.
\rsbpar\vs Tob 5:18 Анна, мать его, заплакала и сказала Товиту: зачем отпустил ты сына нашего? Не он ли был опорою рук наших, когда входил и выходил пред нами?
\vs Tob 5:19 Не предпочитай серебра серебру; пусть оно будет как сор \bibemph{в сравнении} с сыном нашим!
\vs Tob 5:20 Ибо, сколько Господом определено нам жить, на это у нас довольно есть.
\vs Tob 5:21 Товит сказал ей: не печалься, сестра; он придет здоровым, и глаза твои увидят его,
\vs Tob 5:22 ибо ему будет сопутствовать добрый Ангел; путь его будет благоуспешен, и он возвратится здоровым.
\vs Tob 6:1 И перестала она плакать.
\vs Tob 6:2 А путники вечером пришли к реке Тигру и остановились там на ночь.
\vs Tob 6:3 Юноша пошел помыться, но из реки показалась рыба и хотела поглотить юношу.
\vs Tob 6:4 Тогда Ангел сказал ему: возьми эту рыбу. И юноша схватил рыбу и вытащил на землю.
\vs Tob 6:5 И сказал ему Ангел: разрежь рыбу, возьми сердце, печень и желчь, и сбереги \bibemph{их}.
\vs Tob 6:6 Юноша так и сделал, как сказал ему Ангел; рыбу же испекли и съели; и пошли дальше и дошли до Екбатан.
\vs Tob 6:7 И сказал юноша Ангелу: брат Азария, к чему эта печень и сердце и желчь из рыбы?
\vs Tob 6:8 Он отвечал: если кого мучит демон или злой дух, то сердцем и печенью должно курить пред \bibemph{таким} мужчиною или женщиною, и более уже не будет мучиться;
\vs Tob 6:9 а желчью помазать человека, который имеет бельма на глазах, и он исцелится.
\vs Tob 6:10 Когда же приближались к Раге,
\vs Tob 6:11 Ангел сказал юноше: брат, ныне мы переночуем у Рагуила, твоего родственника, у которого есть дочь, по имени Сарра.
\vs Tob 6:12 Я поговорю о ней, чтобы дали ее тебе в жену, ибо тебе предназначено наследство ее, так как ты один из рода ее; а девица прекрасная и умная.
\vs Tob 6:13 Так послушайся меня; я поговорю с ее отцом и, когда мы возвратимся из Раг, совершим брак. Я знаю Рагуила: он никак не даст ее мужу чужому вопреки закону Моисееву; иначе повинен будет смерти, так как наследство следует получить тебе, а не другому кому.
\vs Tob 6:14 Тогда юноша сказал Ангелу: брат Азария, я слышал, что эту девицу отдавали семи мужам, но все они погибли в брачной комнате;
\vs Tob 6:15 а я один у отца и боюсь, как бы, войдя \bibemph{к ней}, не умереть подобно прежним; ее любит демон, который никому не вредит, кроме приближающихся к ней. И потому я боюсь, как бы мне не умереть и не свести жизнь отца моего и матери моей печалью обо мне во гроб их; а другого сына, который похоронил бы их, нет у них.
\vs Tob 6:16 Ангел сказал ему: разве ты забыл слова, которые заповедал тебе отец твой, чтобы ты взял жену из рода твоего? Послушай же меня, брат: ей следует быть твоею женою, а о демоне не беспокойся; в эту же ночь отдадут тебе ее в жену.
\vs Tob 6:17 Только, когда ты войдешь в брачную комнату, возьми курильницу, вложи в нее с\acc{е}рдца и печени рыбы и покури;
\vs Tob 6:18 и демон ощутит запах и удалится, и не возвратится никогда. Когда же тебе надобно будет приблизиться к ней, встаньте оба, воззовите к милосердому Богу, и Он спасет и помилует вас. Не бойся; ибо она предназначена тебе от века, и ты спасешь ее, и она пойдет с тобою, и я знаю, что у тебя будут от нее дети. Выслушав это, Товия полюбил ее, и душа его крепко прилепилась к ней. И пришли они в Екбатаны.
\vs Tob 7:1 И подошли к дому Рагуила. Сарра встретила и приветствовала их, и они ее, и ввела их в дом.
\vs Tob 7:2 И сказал Рагуил Едне, жене своей: как похож этот юноша на Товита, сына брата моего!
\vs Tob 7:3 И спросил их Рагуил: откуда вы, братья? Они отвечали ему: мы из сынов Неффалима, плененных в Ниневию.
\vs Tob 7:4 Еще спросил их: знаете ли брата нашего Товита? Они отвечали: знаем. Потом спросил: здравствует ли он? Они отвечали: жив и здоров.
\vs Tob 7:5 А Товия сказал: это мой отец.
\vs Tob 7:6 И бросился к нему Рагуил и целовал его и плакал.
\vs Tob 7:7 И благословил его и сказал: ты сын честного и доброго человека. Но, услышав, что Товит потерял зрение, опечалился и плакал;
\vs Tob 7:8 плакали и Една, жена его, и Сарра, дочь его. И приняли их весьма радушно,
\vs Tob 7:9 и закололи овна, и предложили обильные снеди. Товия же сказал Рафаилу: брат Азария, переговори, о чем ты говорил на пути; пусть устроится это дело!
\vs Tob 7:10 И он передал эту речь Рагуилу, а Рагуил сказал Товии: ешь, пей и веселись, ибо тебе надлежит взять мою дочь. Впрочем, скажу тебе правду:
\vs Tob 7:11 я отдавал свою дочь семи мужам, и, когда они входили к ней, в ту же ночь умирали. Но ты ныне будь весел! И сказал Товия: я ничего не буду здесь есть до тех пор, пока не сговоритесь и не условитесь со мною. Рагуил сказал: возьми ее теперь же по праву; ты брат ее, и она твоя. Милосердый Бог да устроит вас наилучшим образом!
\vs Tob 7:12 И призвал Сарру, дочь свою, и, взяв руку ее, отдал ее Товии в жену и сказал: вот, по закону Моисееву, возьми ее и веди к отцу твоему. И благословил их.
\vs Tob 7:13 И призвал Едну, жену свою, и, взяв свиток, написал договор и запечатал.
\vs Tob 7:14 И начали есть.
\vs Tob 7:15 И призвал Рагуил Едну, жену свою, и сказал ей: приготовь, сестра, другую спальню и введи ее.
\vs Tob 7:16 И сделала, как он сказал; и ввела ее туда, и заплакала, и приняла взаимно слезы дочери своей, и сказала ей:
\vs Tob 7:17 успокойся, дочь; Господь неба и земли даст тебе радость вместо печали твоей. Успокойся, дочь \bibemph{моя}!
\vs Tob 8:1 Когда окончили ужин, ввели к ней Товию.
\vs Tob 8:2 Он же, идя, вспомнил слова Рафаила, и взял курильницу, и положил сердце и печень рыбы, и курил.
\vs Tob 8:3 Демон, ощутив этот запах, убежал в верхние страны Египта, и связал его Ангел.
\rsbpar\vs Tob 8:4 Когда они остались в комнате вдвоем, Товия встал с постели и сказал: встань, сестра, и помолимся, чтобы Господь помиловал нас.
\vs Tob 8:5 И начал Товия говорить: благословен Ты, Боже отцов наших, и благословенно имя Твое святое и славное вовеки! Да благословляют Тебя небеса и все творения Твои!
\vs Tob 8:6 Ты сотворил Адама и дал ему помощницею Еву, подпорою~--- жену его. От них произошел род человеческий. Ты сказал: нехорошо быть человеку одному, сотворим помощника, подобного ему.
\vs Tob 8:7 И ныне, Господи, я беру сию сестру мою не для удовлетворения похоти, но поистине \bibemph{как жену}: благоволи же помиловать меня и \bibemph{дай} мне состариться с нею!
\vs Tob 8:8 И она сказала с ним: аминь.
\vs Tob 8:9 И оба спокойно спали в эту ночь. Между тем Рагуил, встав, пошел и выкопал могилу,
\vs Tob 8:10 говоря: не умер ли и этот?
\vs Tob 8:11 И пришел Рагуил в дом свой
\vs Tob 8:12 и сказал Едне, жене своей: пошли одну из служанок посмотреть, жив ли он; если нет, похороним его, и никто не будет знать.
\vs Tob 8:13 Служанка, отворив дверь, вошла и увидела, что оба они спят.
\vs Tob 8:14 И, выйдя, объявила им, что он жив.
\rsbpar\vs Tob 8:15 И благословил Рагуил Бога, говоря: благословен Ты, Боже, всяким благословением чистым и святым! Да благословляют Тебя святые Твои, и все создания Твои, и все Ангелы Твои, и все избранные Твои, да благословляют Тебя вовеки!
\vs Tob 8:16 Благословен Ты, что возвеселил меня, и не случилось со мною так, как я думал, но сотворил с нами по великой Твоей милости!
\vs Tob 8:17 Благословен Ты, что помиловал двух единородных! Доверши, Владыка, милость над ними: дай им окончить жизнь во здравии, с весельем и милостью!
\vs Tob 8:18 И приказал рабам своим зарыть могилу.
\vs Tob 8:19 И сделал для них брачный пир на четырнадцать дней.
\vs Tob 8:20 И сказал ему Рагуил с клятвою прежде исполнения дней брачного пира: не уходи, доколе не исполнятся эти четырнадцать дней брачного пира;
\vs Tob 8:21 а тогда, взяв половину имения, благополучно отправляйся к отцу твоему: остальное же \bibemph{получишь}, когда умру я и жена моя.
\vs Tob 9:1 И позвал Товия Рафаила и сказал ему:
\vs Tob 9:2 брат Азария, возьми с собою раба и двух верблюдов и сходи в Раги Мидийские к Гаваилу; принеси мне серебро и самого его приведи ко мне на брак;
\vs Tob 9:3 ибо Рагуил обязал меня клятвою, чтоб я не уходил;
\vs Tob 9:4 между тем отец мой считает дни и, если я много замедлю, он будет очень скорбеть.
\vs Tob 9:5 И пошел Рафаил и остановился у Гаваила и отдал ему расписку; а тот принес мешки за печатями и передал ему.
\vs Tob 9:6 И на утро рано встали они вместе и пришли на брак. И благословил Товия жену свою.
\vs Tob 10:1 Товит, отец его, считал каждый день. И когда исполнились дни путешествия, а он не приходил,
\vs Tob 10:2 Товит сказал: не задержали ли их? или не умер ли Гаваил, и некому отдать им серебра?
\vs Tob 10:3 И очень печалился.
\vs Tob 10:4 Жена же его сказала ему: погиб сын наш, потому и не приходит. И начала плакать по нем и говорила:
\vs Tob 10:5 ничто не занимает меня, сын мой, потому что я отпустила тебя, свет очей моих!
\vs Tob 10:6 Товит говорит ей: молчи, не тревожься, он здоров.
\vs Tob 10:7 А она сказала ему: молчи ты, не обманывай меня; погибло детище мое.~--- И ежедневно ходила за город на дорогу, по которой они отправились; днем не ела хлеба, а по ночам не переставала плакать о сыне своем Товии, пока не окончились четырнадцать дней брачного пира, которые Рагуил заклял его провести там. Тогда Товия сказал Рагуилу: отпусти меня, потому что отец мой и мать моя не надеются уже видеть меня.
\vs Tob 10:8 Тесть же сказал ему: побудь у меня; я пошлю к отцу твоему, и известят его о тебе.
\vs Tob 10:9 А Товия говорит: нет, отпусти меня к отцу моему.
\vs Tob 10:10 И встал Рагуил и отдал ему Сарру, жену его, и половину имения, рабов и скота и серебро,
\vs Tob 10:11 и, благословив их, отпустил и сказал: дети! да благопоспешит вам Бог Небесный, прежде нежели я умру.
\vs Tob 10:12 Потом сказал дочери своей: почитай твоего свекра и свекровь; теперь они~--- родители твои; желаю слышать добрый слух о тебе. И поцеловал ее. И Една сказала Товии: возлюбленный брат, да восставит тебя Господь Небесный и дарует мне видеть детей от Сарры, дочери моей, дабы я возрадовалась пред Господом. И вот, отдаю тебе дочь мою на сохранение; не огорчай ее.
\rsbpar\vs Tob 10:13 После того отправился Товия, благословляя Бога, что Он благоустроил путь его, и благословлял Рагуила и Едну, жену его. И продолжал путь, и приблизились они к Ниневии.
\vs Tob 11:1 И сказал Рафаил Товии: ты знаешь, брат, \bibemph{в} каком \bibemph{положении} ты оставил отца твоего;
\vs Tob 11:2 пойдем вперед, прежде жены твоей, и приготовим помещение;
\vs Tob 11:3 а ты возьми в руку и желчь рыбью. И пошли; за ними побежала и собака.
\rsbpar\vs Tob 11:4 Между тем Анна сидела, высматривая на дороге сына своего,
\vs Tob 11:5 и, заметив, что он идет, сказала отцу его: вот, идет сын твой и человек, отправившийся с ним.
\vs Tob 11:6 Рафаил сказал: я знаю, Товия, что у отца твоего откроются глаза;
\vs Tob 11:7 ты только помажь желчью глаза его, и он, ощутив едкость, оботрет \bibemph{их}, и спадут бельма, и он увидит тебя.
\vs Tob 11:8 Анна, подбежав, бросилась на шею к сыну своему и сказала ему: увидела я тебя, дитя \bibemph{мое},~--- теперь мне хотя умереть. И оба заплакали.
\vs Tob 11:9 А Товит пошел к дверям и споткнулся, но сын его поспешил к нему, и поддержал отца своего,
\vs Tob 11:10 и приложил желчь к глазам отца своего, и сказал: ободрись, отец \bibemph{мой}!
\vs Tob 11:11 Глаза его заело, и он отер их,
\vs Tob 11:12 и снялись с краев глаз его бельма. Увидев сына своего, он пал на шею к нему
\vs Tob 11:13 и заплакал и сказал: благословен Ты, Боже, и благословенно имя Твое вовеки, и благословенны все святые Ангелы Твои!
\vs Tob 11:14 Потому что Ты наказал и помиловал меня. Вот, я вижу Товию, сына моего.~--- И вошел сын его радостно и рассказал отцу своему о чудных \bibemph{делах}, бывших с ним в Мидии.
\vs Tob 11:15 И вышел Товит навстречу невестке своей к воротам Ниневии, радуясь и благословляя Бога. Видевшие, что он идет, удивлялись, как он прозрел.
\vs Tob 11:16 И Товит исповедал пред ними, что Бог помиловал его. Когда подошел Товит к Сарре, невестке своей, благословил ее и сказал: здравствуй, дочь \bibemph{моя}! Благословен Бог, Который привел тебя к нам, и \bibemph{благословенны} отец твой и мать твоя! Обрадовались и все братья его в Ниневии.
\vs Tob 11:17 И пришел Ахиахар и Насвас, племянник его,
\vs Tob 11:18 и весело праздновали брак Товии семь дней.
\vs Tob 12:1 И призвал Товит сына своего Товию и сказал ему: приготовь, сын \bibemph{мой}, плату человеку, который ходил с тобою; ему надобно еще прибавить.
\vs Tob 12:2 Он отвечал: отец \bibemph{мой}, я не буду в убытке, если отдам ему половину всего, что принес;
\vs Tob 12:3 потому что он привел меня к тебе здоровым и жену мою уврачевал, и серебро мое принес, и тебя также исцелил.
\vs Tob 12:4 Старец сказал: так и следует ему.
\vs Tob 12:5 И призвал Ангела и сказал ему: возьми половину всего, что вы принесли, и иди с миром.
\rsbpar\vs Tob 12:6 Тогда, отозвав обоих особо, \bibemph{Ангел} сказал им: благословляйте Бога, прославляйте Его, признавайте величие Его и исповедуйте пред всеми живущими, что Он сделал для вас. Доброе \bibemph{дело}~--- благословлять Бога, превозносить имя Его и благоговейно проповедовать о делах Божиих; и вы не ленитесь прославлять Его.
\vs Tob 12:7 Тайну цареву прилично хранить, а о делах Божиих объявлять похвально. Делайте добро, и зло не постигнет вас.
\vs Tob 12:8 Доброе \bibemph{дело}~--- молитва с постом и милостынею и справедливостью. Лучше малое со справедливостью, нежели многое с неправдою; лучше творить милостыню, нежели собирать золото,
\vs Tob 12:9 ибо милостыня от смерти избавляет и может очищать всякий грех. Творящие милостыни и \bibemph{дела} правды будут долгоденствовать.
\vs Tob 12:10 Грешники же суть враги своей жизни.
\vs Tob 12:11 Не скрою от вас ничего; я сказал уже: тайну цареву прилично хранить, а о делах Божиих объявлять похвально.
\vs Tob 12:12 Когда молился ты и невестка твоя Сарра, я возносил память молитвы вашей пред Святаго, и когда ты хоронил мертвых, я также был с тобою.
\vs Tob 12:13 И когда ты не обленился встать и оставить обед свой, чтобы пойти и убрать мертвого, твоя благотворительность не утаилась от меня, но я был с тобою.
\vs Tob 12:14 И ныне Бог послал меня уврачевать тебя и невестку твою Сарру.
\vs Tob 12:15 Я~--- Рафаил, один из семи святых Ангелов, которые возносят молитвы святых и восходят пред славу Святаго.
\rsbpar\vs Tob 12:16 Тогда оба смутились и пали лицем \bibemph{на землю}, потому что были в страхе.
\vs Tob 12:17 Но он сказал им: не бойтесь, мир будет вам. Благословляйте Бога вовек.
\vs Tob 12:18 Ибо я пришел не по своему произволению, а по воле Бога нашего; потому и благословляйте Его вовек.
\vs Tob 12:19 Все дни я был видим вами, но я не ел и не пил,~--- только взорам вашим представлялось \bibemph{это}.
\vs Tob 12:20 Итак, прославляйте теперь Бога, потому что я восхожу к Пославшему меня, и напишите все совершившееся в книгу.
\vs Tob 12:21 И встали они и более уже не видели его.
\vs Tob 12:22 И стали рассказывать о великих и чудных делах Божиих, и как явился им Ангел Господень.
\vs Tob 13:1 В радости Товит написал молитву \bibemph{в сих} словах: благословен Бог, вечно живущий, и \bibemph{благословенно} царство Его!
\vs Tob 13:2 Ибо Он наказует и милует, низводит до ада и возводит, и нет никого, кто избежал бы от руки Его.
\vs Tob 13:3 Сыны Израилевы! прославляйте Его пред язычниками, ибо Он рассеял нас между ними.
\vs Tob 13:4 Там возвещайте величие Его, превозносите Его пред всем живущим, ибо Он Господь наш и Бог, Отец наш во все веки:
\vs Tob 13:5 накажет нас за неправды наши, и опять помилует и соберет нас из всех народов, где бы вы ни были рассеяны между ними.
\vs Tob 13:6 Если вы будете обращаться к Нему всем сердцем вашим и всею душею вашею, чтобы поступать пред Ним по истине, тогда Он обратится к вам и не скроет от вас лица Своего. Увидите, чт\acc{о} Он сделает с вами. Прославляйте Его всеми \bibemph{глаголами} уст ваших и благословляйте Господа правды и превозносите Царя веков. В земле плена моего я прославляю Его и проповедую силу и величие Его народу грешников. Обратитесь, грешники, и делайте правду пред Ним. Кто знает, может быть, Он возблаговолит о вас и окажет вам милость?
\vs Tob 13:7 Превозношу я Бога моего, и душа моя~--- Небесного Царя, и радуется о величии Его.
\vs Tob 13:8 Пусть все возвещают о Нем и прославляют Его в Иерусалиме.
\vs Tob 13:9 Иерусалим, город святый! Он накажет \bibemph{тебя} за дела сынов твоих и опять помилует сынов праведных.
\vs Tob 13:10 Славь Господа усердно и благословляй Царя веков, чтобы снова сооружена была скиния Его в тебе с радостью, чтобы Он возвеселил среди тебя пленных и возлюбил в тебе несчастных во все роды века.
\vs Tob 13:11 Многие народы издалека придут к имени Господа Бога с дарами в руках, с дарами Царю Небесному; роды родов восхвалят тебя с восклицаниями радостными.
\vs Tob 13:12 Прокляты все ненавидящие тебя, благословенны будут вовек все любящие тебя!
\vs Tob 13:13 Радуйся и веселись о сынах праведных, ибо они соберутся и будут благословлять Господа праведных.
\vs Tob 13:14 О, блаженны любящие тебя! они возрадуются о мире твоем. Блаженны скорбевшие о всех бедствиях твоих, ибо они возрадуются о тебе, когда увидят всю славу твою, и будут веселиться вечно.
\vs Tob 13:15 Да благословляет душа моя Бога, Царя великого,
\vs Tob 13:16 ибо Иерусалим отстроен будет из сапфира и смарагда и из дорогих камней; стены твои, башни и укрепления~--- из чистого золота;
\vs Tob 13:17 и площади Иерусалимские выстланы будут бериллом, анфраксом и камнем из Офира.
\vs Tob 13:18 На всех улицах его будет раздаваться: аллилуия,~--- и будут славословить, говоря: благословен Бог, Который превознес \bibemph{Иерусалим}, на все веки!
\vs Tob 14:1 И окончил Товит славословие.
\rsbpar\vs Tob 14:2 Он был восьмидесяти восьми лет, когда потерял зрение, и чрез восемь лет прозрел. И творил милостыни, и продолжал быть благоговейным пред Господом Богом и прославлять Его.
\vs Tob 14:3 Наконец он очень состарился, и призвал сына своего и шесть сыновей его, и сказал ему: сын \bibemph{мой}, возьми сыновей твоих; вот я состарился и уже на исходе жизни моей.
\vs Tob 14:4 Отправься в Мидию, сын \bibemph{мой}, ибо я уверен, что Ниневия будет разорена, как говорил пророк Иона; а в Мидии будет спокойнее до времени. Братья наши, находящиеся в \bibemph{отечественной} земле, будут рассеяны из сей доброй земли; Иерусалим будет пустынею, и дом Божий в нем будет сожжен и до времени останется пуст.
\vs Tob 14:5 Но опять Бог помилует их и возвратит их в землю; и воздвигнут дом \bibemph{Божий}, не такой, как прежний, доколе не исполнятся времена века. И после того возвратятся из плена и построят Иерусалим великолепно, и дом Божий восстановлен будет в нем на все роды века,~--- здание величественное, как говорили о нем пророки.
\vs Tob 14:6 И все народы обратятся и будут истинно благоговеть пред Господом Богом и ниспровергнут идолов своих;
\vs Tob 14:7 и все народы будут благословлять Господа. И Его народ будет прославлять Бога, и Господь вознесет народ Свой; и все, истинно и праведно любящие Господа Бога, будут радоваться, оказывая милость братьям нашим.
\vs Tob 14:8 Итак, сын \bibemph{мой}, выйди из Ниневии, ибо непременно исполнится то, что говорил пророк Иона.
\vs Tob 14:9 Ты же соблюдай закон и повеления и будь любомилостив и справедлив, чтобы хорошо было тебе.
\vs Tob 14:10 Похорони меня прилично, и мать твою со мною, и потом не оставайтесь в Ниневии.~--- Сын \bibemph{мой}, смотри, чт\acc{о} сделал Аман с Ахиахаром, который воспитал его: как он из света привел его в тьму, и как воздано ему. Ахиахар спасен, а тот получил достойное возмездие~--- сошел во тьму. Манассия творил милостыню, и спасен от смертной сети, которую расставили ему; Аман же пал в сеть и погиб.
\vs Tob 14:11 Итак, дети, знайте, чт\acc{о} делает милостыня и как спасает справедливость.~--- Когда он это сказал, душа его оставила его на ложе; было же ему сто пятьдесят восемь лет, и сын с честью похоронил его.
\rsbpar\vs Tob 14:12 Когда умерла Анна, он похоронил и ее с отцом своим. После того Товия с женою своею и детьми своими отправился в Екбатаны к Рагуилу, тестю своему,
\vs Tob 14:13 и достиг честной старости, и похоронил прилично тестя и тещу своих, и получил в наследство имение их и Товита, отца своего.
\vs Tob 14:14 И умер ста двадцати семи лет в Екбатанах Мидийских.
\vs Tob 14:15 Но прежде нежели умер, он слышал о погибели Ниневии, которую пленил Навуходоносор и Асуир, и возрадовался пред смертью о Ниневии.

\bibbookdescr{Jdt}{
  inline={\LARGE Книга\\\Huge Иудифи\fns{Переведена с греческого.}},
  toc={Иудифь*},
  bookmark={Иудифь},
  header={Иудифь},
  %headerleft={},
  %headerright={},
  abbr={Иудифь}
}
\vs Jdt 1:1 В двенадцатый год царствования Навуходоносора, царствовавшего над Ассириянами в великом городе Ниневии,~--- во дни Арфаксада, который царствовал над Мидянами в Екбатанах
\vs Jdt 1:2 и построил вокруг Екбатан стены из тесаных камней, шириною в три локтя, а длиною в шесть локтей; и сделал высоту стены в семьдесят, а ширину в пятьдесят локтей,
\vs Jdt 1:3 и поставил над воротами башни во сто локтей, имевшие в основании до шестидесяти локтей ширины;
\vs Jdt 1:4 а ворота, построенные им для выхода сильных войск его и для строев пехоты его, поднимались в высоту на семьдесят локтей, а в ширину имели сорок локтей:
\vs Jdt 1:5 в те дни царь Навуходоносор предпринял войну против царя Арфаксада на великой равнине, которая в пределах Рагава.
\vs Jdt 1:6 К нему собрались все живущие в нагорной стране, и все живущие при Евфрате, Тигре и Идасписе, и с равнины Ариох, царь Елимейский, и сошлись очень многие народы в ополчение сынов Хелеуда.
\vs Jdt 1:7 И послал Навуходоносор, царь Ассирийский, ко всем живущим в Персии и ко всем живущим на западе, к живущим в Киликии и Дамаске, Ливане и Антиливане, и ко всем живущим на передней стороне приморья,
\vs Jdt 1:8 и между народами Кармила и Галаада и в верхней Галилее и на великой равнине Ездрилон,
\vs Jdt 1:9 и ко всем живущим в Самарии и городах ее, и за Иорданом до Иерусалима, и Ветани и Хела, и Кадиса и реки Египетской, и Тафны и Рамессы, и во всей земле Гесемской
\vs Jdt 1:10 до входа в верхний Танис и Мемфис, и ко всем живущим в Египте до входа в пределы Ефиопии.
\vs Jdt 1:11 Но все обитавшие во всей этой земле презрели слово Ассирийского царя Навуходоносора и не собрались к нему на войну, потому что они не боялись его, но он был для них как один из них: они отослали от себя его послов ни с чем, в бесчестии.
\vs Jdt 1:12 Навуходоносор весьма разгневался на всю эту землю и поклялся престолом и царством своим отмстить всем пределам Киликии, Дамаска и Сирии, и мечом своим умертвить всех, живущих в земле Моава, и сынов Аммона и всю Иудею, и всех, обитающих в Египте до входа в пределы двух морей.
\rsbpar\vs Jdt 1:13 И в семнадцатый год он ополчился со своим войском против царя Арфаксада и одолел его в сражении и обратил в бегство все войско Арфаксада, всю конницу его и все колесницы его,
\vs Jdt 1:14 и овладел городами его, дошел до Екбатан, занял укрепления, опустошил улицы \bibemph{города} и красоту его обратил в позор.
\vs Jdt 1:15 А Арфаксада схватил на горах Рагава и, пронзив его копьем своим, в тот же день погубил его.
\vs Jdt 1:16 Потом пошел назад со своими в Ниневию,~--- он и все союзники его~--- весьма многое множество ратных мужей; там он отдыхал, и пировал с войском своим сто двадцать дней.
\vs Jdt 2:1 В восемнадцатом году, в двадцать второй день первого месяца, последовало в доме Навуходоносора, царя Ассирийского, повеление~--- совершить, как он сказал, отмщение всей земле.
\vs Jdt 2:2 Созвав всех служителей и всех сановников своих, он открыл им тайну своего намерения и своими устами определил всякое зло той земле.
\vs Jdt 2:3 И они решили погубить всех, кто не повиновался слову уст его.
\vs Jdt 2:4 По окончании своего совещания, Навуходоносор, царь Ассирийский, призвал главного вождя войска своего, Олоферна, который был вторым по нем, и сказал ему:
\vs Jdt 2:5 так говорит великий царь, господин всей земли: вот, ты пойдешь от лица моего и возьмешь с собою мужей, уверенных в своей силе,~--- пеших сто двадцать тысяч и множество коней с двенадцатью тысячами всадников,~---
\vs Jdt 2:6 и выйдешь против всей земли на западе за то, что не повиновались слову уст моих.
\vs Jdt 2:7 И объявишь им, чтобы они приготовляли землю и воду, потому что я с гневом выйду на них, покрою все лице земли \bibemph{их} ногами войска моего и предам ему их на расхищение.
\vs Jdt 2:8 Долы и потоки наполнятся их ранеными, и река, запруженная трупами их, переполнится;
\vs Jdt 2:9 а пленных их я рассею по концам всей земли.
\vs Jdt 2:10 Ты же отправившись завладей для меня всеми пределами их: которые сами сдадутся тебе, тех ты сохрани до дня обличения их;
\vs Jdt 2:11 а непокорных да не пощадит глаз твой: предавай их смерти и разграблению по всей земле твоей.
\vs Jdt 2:12 Ибо жив я,~--- и крепко царство мое: что сказал, то сделаю моею рукою.
\vs Jdt 2:13 Не преступи же ни в чем слов господина твоего, но непременно исполни, как я приказал тебе, и не медли исполнением.
\rsbpar\vs Jdt 2:14 Олоферн, выйдя от лица господина своего, пригласил к себе всех сановников, полководцев и начальников войска Ассирийского,
\vs Jdt 2:15 отсчитал для сражения отборных мужей, как повелел ему господин его, сто двадцать тысяч, и конных стрелков двенадцать тысяч,
\vs Jdt 2:16 и привел их в такой порядок, каким строится войско, идущее на сражение.
\vs Jdt 2:17 Он взял весьма много верблюдов, ослов и мулов для обоза их, а овец, волов и коз для продовольствия их~--- без числа,
\vs Jdt 2:18 и много пищи для всех, и очень много золота и серебра из царского дома.
\vs Jdt 2:19 И выступил в поход со всем войском своим, чтобы предварить царя Навуходоносора и покрыть все лице земли на западе колесницами, конницею и отборною пехотою своею.
\vs Jdt 2:20 И с ним вышли союзники в таком множестве, как саранча и как песок земной, потому что от множества не было и счета им.
\vs Jdt 2:21 Пройдя путь трех дней от Ниневии до передней стороны равнины Вектелеф, они поворотили от Вектелефа, близ горы, лежащей по левую сторону верхней Киликии.
\vs Jdt 2:22 Оттуда, взяв все войско свое, пеших и конных и колесницы свои, он отправился в нагорную страну;
\vs Jdt 2:23 разбил Фудян и Лудян и разграбил всех сынов Рассиса и сынов Исмаила, живших в пустыне на юг к земле Хеллеонской.
\vs Jdt 2:24 \bibemph{Потом}, переправившись чрез Евфрат, он прошел Месопотамию и разрушил все высокие города при потоке Авроне до входа в море.
\vs Jdt 2:25 Заняв пределы Киликии, он избил всех, противоставших ему, и, пройдя до пределов Иафета, лежащих к югу на передней стороне Аравии,
\vs Jdt 2:26 обошел кругом всех сынов Мадиама, выжег жилища их и разграбил стада их.
\vs Jdt 2:27 Потом спустился на равнину Дамаска, во время жатвы пшеницы, выжег все нивы их, отдал на истребление стада овец и волов, разграбил города их, опустошил их поля и избил всех юношей их острием меча.
\vs Jdt 2:28 Страх и ужас напал на жителей приморской страны, обитавших в Сидоне и Тире, на жителей Сура и Окины и на всех жителей Иемнаана,~--- и все обитатели Азота и Аскалона сильно испугались его.
\vs Jdt 3:1 И послали к нему вестников с таким мирным предложением:
\vs Jdt 3:2 вот мы, рабы великого царя Навуходоносора, повергаемся перед тобою: делай с нами, что тебе угодно.
\vs Jdt 3:3 Вот перед тобою: и селения наши, и все места наши, и все нивы с пшеницею, и стада овец и волов, и все строения наших жилищ: употребляй их, как пожелаешь.
\vs Jdt 3:4 Вот и города наши и обитающие в них~--- рабы твои: иди и поступай с ними, как будет глазам твоим угодно.
\rsbpar\vs Jdt 3:5 И пришли к Олоферну мужи и передали ему эти слова.
\vs Jdt 3:6 \bibemph{Тогда} он пришел в приморскую страну с войском своим, окружил высокие города стражею и взял из них отборных мужей в соратники себе.
\vs Jdt 3:7 А они и вся окрестность их приняли его с венками, ликами и тимпанами.
\vs Jdt 3:8 Он же разорил все высоты их и вырубил рощи их: ему приказано было истребить всех богов той земли, чтобы все народы служили одному Навуходоносору, и все языки и все племена их призывали его, как Бога.
\vs Jdt 3:9 Придя к Ездрилону близ Дотеи, лежащей против великой теснины Иудейской,
\vs Jdt 3:10 он расположился лагерем между Гаваем и городом Скифов и оставался там целый месяц, чтобы собрать весь обоз своего войска.
\vs Jdt 4:1 Сыны Израиля, жившие в Иудее, услышав обо всем, что сделал с народами Олоферн, военачальник Ассирийского царя Навуходоносора, и как разграбил он все святилища их и отдал их на уничтожение,
\vs Jdt 4:2 очень, очень испугались его и трепетали за Иерусалим и храм Господа Бога своего;
\vs Jdt 4:3 потому что недавно возвратились они из плена, недавно весь народ Иудейский собрался, и освящены от осквернения сосуды, жертвенник и дом \bibemph{Господень}.
\vs Jdt 4:4 Они послали во все пределы Самарии и Конии, и Ветерона и Вельмена, и Иерихона, и в Хову и Эсору, и в равнину Салимскую,
\vs Jdt 4:5 заняли все вершины высоких гор, оградили стенами находящиеся на них селения и отложили запасы хлеба на случай войны, так как нивы их недавно были сжаты,
\vs Jdt 4:6 а великий священник Иоаким, бывший в те дни в Иерусалиме, написал жителям Ветилуи и Ветомесфема, лежащего против Ездрилона, на передней стороне равнины, близкой к Дофаиму,
\vs Jdt 4:7 чтобы они заняли восходы в нагорную страну, потому что чрез них был вход в Иудею, и легко было им воспрепятствовать приходящим, так как тесен был проход даже для двух человек.
\rsbpar\vs Jdt 4:8 Сыны Израиля поступили так, как велел им великий священник Иоаким и старейшины всего народа Израильского, пребывавшие в Иерусалиме.
\vs Jdt 4:9 И с великим усердием возопили к Богу все мужи Израиля и смирили души свои с великим усердием:
\vs Jdt 4:10 они и жены их, и дети их, и скот их; и всякий пришлец, и наемник, и купленный за серебро наложили вретища на чресла свои.
\vs Jdt 4:11 И всякий муж Израильский и \bibemph{всякая} жена, и дети, и жители Иерусалима пали пред храмом, посыпали пеплом свои головы, разостлали пред Господом свои вретища,
\vs Jdt 4:12 облекли жертвенник во вретище и прилежно и единодушно взывали к Богу Израилеву, чтобы Он, на радость язычникам, не предал детей их на расхищение, жен их в добычу, городов наследия их на разорение, святынь их на осквернение и поругание.
\vs Jdt 4:13 И Господь услышал голос их и призрел на скорбь их; и во всей Иудее и Иерусалиме народ много дней постился пред святилищем Господа Вседержителя.
\vs Jdt 4:14 А Иоаким, великий священник, и все предстоящие пред Господом священники, служители Его, препоясав вретищем чресла свои, приносили непрестанные всесожжения, обеты и доброхотные дары народа.
\vs Jdt 4:15 На кидарах их был пепел, и они от всей силы взывали к Господу, чтобы Он посетил милостью весь дом Израиля.
\vs Jdt 5:1 Между тем Олоферну, военачальнику войска Ассирийского, дано было знать, что сыны Израиля приготовились к войне: заложили входы в нагорную страну и укрепили стенами всякую вершину высокой горы, а на равнинах устроили преграды.
\vs Jdt 5:2 Он весьма разгневался и, призвав всех начальников Моава и вождей Аммона и всех правителей приморской страны, сказал им:
\vs Jdt 5:3 скажите мне, сыны Ханаана, что это за народ, живущий в нагорной стране, какие обитаемые ими города, много ли у них войска, в чем их крепость и сила, кто поставлен над ними царем, предводителем войска их,
\vs Jdt 5:4 и почему они больше всех, живущих на западе, упорствуют выйти мне навстречу?
\vs Jdt 5:5 Ахиор, предводитель всех сынов Аммона, сказал ему: выслушай, господин мой, слово из уст раба твоего; я скажу тебе истину об этом народе, живущем близ тебя в этой нагорной стране, и не выйдет лжи из уст раба твоего.
\vs Jdt 5:6 Этот народ происходит от Халдеев.
\vs Jdt 5:7 Прежде они поселились в Месопотамии, потому что не хотели служить богам отцов своих, которые были в земле Халдейской,
\vs Jdt 5:8 и уклонились от пути предков своих и начали поклоняться Богу неба, Богу, Которого они познали; и \bibemph{Халдеи} выгнали их от лица богов своих,~--- и они бежали в Месопотамию и долго там обитали.
\vs Jdt 5:9 Но Бог их сказал, чтобы они вышли из места переселения и шли в землю Ханаанскую; они поселились там и весьма обогатились золотом, серебром и множеством скота.
\vs Jdt 5:10 \bibemph{Отсюда} перешли они в Египет, так как голод накрыл лице земли Ханаанской, и там оставались, пока находили пропитание, и умножились там до того, что не было и числа роду их.
\vs Jdt 5:11 И восстал на них царь Египетский, употребил против них хитрость, обременяя их трудом и деланьем кирпича, и сделал их рабами.
\vs Jdt 5:12 Тогда они воззвали к Богу своему,~--- и Он поразил всю землю Египетскую неисцельными язвами,~--- и Египтяне прогнали их от себя.
\vs Jdt 5:13 Бог иссушил перед ними Чермное море
\vs Jdt 5:14 и вел их путем Сины и Кадис-Варни; они выгнали всех обитавших в этой пустыне;
\vs Jdt 5:15 поселились в земле Аморреев, своею силою истребили всех Есевонитян, перешли Иордан, наследовали всю нагорную страну
\vs Jdt 5:16 и, прогнав от себя Хананея, Ферезея, Иевусея, Сихема и всех Гергесеян, жили в ней много дней.
\vs Jdt 5:17 И доколе не согрешили пред Богом своим, счастье было с ними, потому что с ними Бог, ненавидящий неправду.
\vs Jdt 5:18 Но когда уклонились от пути, который Он завещал им, то во многих войнах они потерпели весьма сильные поражения, отведены в плен, в чужую землю, храм Бога их разрушен, и города их взяты неприятелями.
\vs Jdt 5:19 Ныне же, обратившись к Богу своему, они возвратились из рассеяния, в котором были, овладели Иерусалимом, в котором святилище их, и поселились в нагорной стране, так как она была пуста.
\vs Jdt 5:20 И теперь, повелитель-господин, если есть заблуждение в этом народе, и они грешат пред Богом своим, и мы заметим, что в них есть это преткновение, то мы пойдем и победим их.
\vs Jdt 5:21 А если нет в этом народе беззакония, то пусть удалится господин мой, чтобы Господь не защитил их, и Бог их \bibemph{не был} за них,~--- и тогда мы для всей земли будем предметом поношения.
\rsbpar\vs Jdt 5:22 Когда Ахиор окончил эту речь, весь народ, стоявший вокруг шатра, возроптал, а вельможи Олоферна и все, населявшие приморье и землю Моава, заговорили: тотчас надобно убить его;
\vs Jdt 5:23 потому что мы не побоимся сынов Израиля: это~--- народ, у которого нет ни войска, ни силы для крепкого ополчения.
\vs Jdt 5:24 Итак, пойдем, повелитель Олоферн,~--- и они сделаются добычею всего войска твоего.
\vs Jdt 6:1 Когда утих шум вокруг собрания, Олоферн, военачальник войска Ассирийского, сказал Ахиору пред всем народом иноплеменных и всем сынам Моава:
\vs Jdt 6:2 кто ты такой, Ахиор, с наемниками Ефрема, что напророчил нам сегодня и сказал, чтобы мы не воевали с родом Израильским, потому что Бог их защищает? Кто же Бог, как не Навуходоносор? Он пошлет свою силу и сотрет их с лица земли,~--- и Бог их не избавит их.
\vs Jdt 6:3 Но мы, рабы его, поразим их, как одного человека, и не устоять им против силы наших коней.
\vs Jdt 6:4 Мы растопчем их; горы их упьются их кровью, равнины их наполнятся их трупами, и не станет стопа ног их против нашего лица, но гибелью погибнут они, говорит царь Навуходоносор, господин всей земли. Ибо он сказал,~--- и не напрасны будут слова повелений его.
\vs Jdt 6:5 А ты, Ахиор, наемник Аммона, высказавший слова эти в день неправды твоей, от сего дня не увидишь больше лица моего, доколе я не отомщу этому народу, \bibemph{пришедшему} из Египта.
\vs Jdt 6:6 Когда же я возвращусь, меч войска моего и толпа слуг моих пройдет по ребрам твоим,~--- и ты падешь между ранеными их.
\vs Jdt 6:7 Рабы мои отведут тебя в нагорную страну и оставят в одном из городов на высотах,
\vs Jdt 6:8 и ты не умрешь там, доколе не будешь с ними истреблен.
\vs Jdt 6:9 Если же ты надеешься в сердце твоем, что они не будут взяты, то да не спадает лице твое. Я сказал, и ни одно из слов моих не пропадет.
\vs Jdt 6:10 И приказал Олоферн рабам своим, предстоявшим в шатре его, взять Ахиора, отвести его в Ветилую и предать в руки сынов Израиля.
\vs Jdt 6:11 Рабы его схватили и вывели его за стан на поле, а со среды равнины поднялись в нагорную страну и пришли к источникам, бывшим под Ветилуею.
\vs Jdt 6:12 Когда увидели их жители города на вершине горы, то взялись за оружия свои и, выйдя за город на вершину горы, все мужи-пращники охраняли восход свой и бросали в них каменьями.
\vs Jdt 6:13 А они, подойдя под гору, связали Ахиора и, оставив его брошенным при подошве горы, ушли к своему господину.
\vs Jdt 6:14 Сыны же Израиля, вышедшие из своего города, остановились над ним и, развязав его, привели в Ветилую, и представили его начальникам своего города,
\vs Jdt 6:15 которыми были в те дни Озия, сын Михи из колена Симеонова, Хаврий, сын Гофониила, и Хармий, сын Мелхиила.
\vs Jdt 6:16 Они созвали всех старейшин города, и сбежались в собрание все юноши их и жены, и поставили Ахиора среди всего народа своего, и Озия спросил его о случившемся.
\vs Jdt 6:17 Он в ответ пересказал им слова собрания Олофернова и все слова, которые он высказал среди начальников сынов Ассура, и все высокомерные речи Олоферна о доме Израиля.
\vs Jdt 6:18 \bibemph{Тогда} народ пал, поклонился Богу и воззвал:
\vs Jdt 6:19 Господи, Боже Небесный! воззри на их гордыню и помилуй смирение рода нашего, и призри на лице освященных Тебе в этот день.
\vs Jdt 6:20 И утешили Ахиора и расхвалили его.
\vs Jdt 6:21 Потом Озия взял его из собрания в свой дом и сделал пир для старейшин,~--- и целую ночь ту они призывали Бога Израилева на помощь.
\vs Jdt 7:1 На другой день Олоферн приказал всему войску своему и всему народу своему, пришедшему к нему на помощь, подступить к Ветилуе, занять высоты нагорной страны и начать войну против сынов Израилевых.
\vs Jdt 7:2 И в тот же день поднялись все сильные мужи их: войско их \bibemph{состояло} из ста семидесяти тысяч ратников, воинов пеших, и из двенадцати тысяч конных, кроме обоза и пеших людей, бывших при них,~--- а и этих было многое множество.
\vs Jdt 7:3 Остановившись на долине близ Ветилуи при источнике, они протянулись в ширину от Дофаима до Велфема, а в длину от Ветилуи до Киамона, лежащего против Ездрилона.
\vs Jdt 7:4 Сыны же Израиля, увидев множество их, очень смутились, и каждый говорил ближнему своему: теперь они опустошат всю землю, и ни высокие горы, ни долины, ни холмы не выдержат их тяжести.
\vs Jdt 7:5 И, взяв каждый свое боевое оружие и зажегши огни на башнях своих, они всю эту ночь провели на страже.
\rsbpar\vs Jdt 7:6 На другой день Олоферн вывел всю свою конницу пред лице сынов Израилевых, бывших в Ветилуе,
\vs Jdt 7:7 осмотрел восходы города их, обошел и занял источники вод их и, оцепив их ратными мужами, возвратился к своему народу.
\vs Jdt 7:8 Но пришли к нему все начальники сынов Исава, и все вожди народа Моавитского, и все военачальники приморья и сказали:
\vs Jdt 7:9 выслушай, господин наш, слово, чтобы не было потери в войске твоем.
\vs Jdt 7:10 Этот народ сынов Израиля надеется не на копья свои, но на высоты гор своих, на которых живут, потому что неудобно восходить на вершины их гор.
\vs Jdt 7:11 Итак, господин, не воюй с ним так, как бывает обыкновенная война,~--- и ни один муж не падет из народа твоего.
\vs Jdt 7:12 Ты останься в своем лагере, чтобы сберечь каждого мужа в войске твоем, а рабы твои пусть овладеют источником воды, который вытекает из подошвы горы;
\vs Jdt 7:13 потому что оттуда берут воду все жители Ветилуи,~--- и погубит их жажда, и они сдадут свой город; а мы с нашим народом взойдем на ближние вершины гор и расположимся на них для стражи, чтобы ни один человек не вышел из города.
\vs Jdt 7:14 И будут томиться они голодом, и жены их и дети их, и прежде, нежели коснется их меч, падут на улицах обиталища своего;
\vs Jdt 7:15 и ты воздашь им злом за то, что они возмутились и не встретили тебя с миром.
\vs Jdt 7:16 Понравились эти слова их Олоферну и всем слугам его, и он решил поступить так, как они сказали.
\vs Jdt 7:17 И двинулся полк сынов Аммона и с ними пять тысяч сынов Ассура и, расположившись в долине, овладели водами и источниками вод сынов Израиля.
\vs Jdt 7:18 А сыны Исава и сыны Аммона взошли и заняли нагорную область против Дофаима, и отправили \bibemph{часть} их на юг и на восток против Екревиля, что близ Хуса, стоящего при потоке Мохмур; остальное же Ассирийское войско расположилось на равнине и покрыло все лице земли: шатры и обозы их с множеством народа растянулись на весьма большом пространстве.
\rsbpar\vs Jdt 7:19 Сыны Израиля воззвали к Господу Богу своему, потому что они пришли в уныние, так как все враги их окружили их, и им нельзя было бежать от них.
\vs Jdt 7:20 Вокруг них стояло все войско Ассирийское,~--- пешие, колесницы и конница их,~--- тридцать четыре дня; у всех жителей Ветилуи истощились все сосуды с водою,
\vs Jdt 7:21 опустели водоемы, и ни в один день они не могли пить воды досыта, потому что давали им пить мерою.
\vs Jdt 7:22 И уныли дети их и жены их и юноши, и в изнеможении от жажды падали на улицах города и в проходах ворот, и уже не было в них крепости.
\vs Jdt 7:23 \bibemph{Тогда} весь народ собрался к Озии и к начальникам города,~--- юноши, жены и дети,~--- и с громким воплем говорили всем старейшинам:
\vs Jdt 7:24 суди Бог между нами и вами; вы сделали нам великую неправду, потому что не предложили мира сынам Ассура;
\vs Jdt 7:25 и теперь нет нам помощника: Бог предал нас в их руки, чтобы погубить нас жаждою и великою погибелью.
\vs Jdt 7:26 Пригласите же их теперь и отдайте весь город на разграбление народу Олоферна и всему войску его,
\vs Jdt 7:27 ибо лучше для нас достаться им на расхищение: хотя мы будем рабами их, зато жива будет душа наша, и глаза наши не увидят смерти младенцев наших и жен и детей наших, расстающихся с душами своими.
\vs Jdt 7:28 Призываем пред вами во свидетели небо и землю, Бога нашего и Господа отцов наших, Который наказывает нас за грехи наши и за грехи отцов наших, да соделает по словам сим в нынешний день.
\vs Jdt 7:29 И подняли они единодушно великий плач среди собрания и громко взывали к Господу Богу.
\vs Jdt 7:30 Озия сказал им: не унывайте, братья! потерпим еще пять дней, в которые Господь Бог наш обратит милость Свою на нас, ибо Он не оставит нас вконец.
\vs Jdt 7:31 Если же они пройдут, и помощь к нам не придет,~--- я сделаю по вашим словам.
\vs Jdt 7:32 И отпустил народ в свой стан, и они пошли на стены и башни своего города, а жен и детей отослал по домам их; и в великой скорби оставались они в городе.
\vs Jdt 8:1 В эти дни услышала Иудифь, дочь Мерарии, сына Окса, сына Иосифа, сына Озиила, сына Елкия, сына Анании, сына Гедеона, сына Рафаина, сына Акифона, сына Илия, сына Елиава, сына Нафанаила, сына Саламиила, сына Саласадая, сына Иеиля.
\vs Jdt 8:2 Муж ее Манассия, из одного с нею колена и племени, умер во время жатвы ячменя;
\vs Jdt 8:3 потому что, когда он стоял в поле близ вязавших снопы, зной пал на его голову,~--- и он слег в постель и умер в своем городе Ветилуе; его похоронили с отцами его на поле между Дофаимом и Валамоном.
\vs Jdt 8:4 И вдовствовала Иудифь в своем доме три года и четыре месяца.
\vs Jdt 8:5 Она сделала для себя на кровле дома своего шатер, возложила на чресла свои вретище, и были на ней одежды вдовства ее.
\vs Jdt 8:6 Она постилась все дни вдовства своего, кроме дней пред субботами и суббот, дней пред новомесячиями и новомесячий, и праздников и торжеств дома Израилева.
\vs Jdt 8:7 Она была красива видом и весьма привлекательна взором; муж ее Манассия оставил ей золото и серебро, слуг и служанок, скот и поля, чем она и владела.
\vs Jdt 8:8 И никто не укорял ее злым словом, потому что она была очень богобоязненна.
\vs Jdt 8:9 Услышала она о дурных речах народа против начальника, потому что они малодушествовали по причине оскудения воды, услышала Иудифь и о всех словах, которые сказал им Озия, как он поклялся им чрез пять дней сдать город Ассириянам,
\vs Jdt 8:10 и послала она служанку свою, распоряжавшуюся всем ее имуществом, пригласить Озию, Хаврина и Хармина, старейшин ее города.
\rsbpar\vs Jdt 8:11 Они пришли,~--- и она сказала им: выслушайте меня, начальники жителей Ветилуи! неправо слово ваше, которое вы сегодня сказали перед народом, и положили клятву, которую изрекли между Богом и вами, и сказали, что сдадите город нашим врагам, если на этих \bibemph{днях} Господь не поможет нам.
\vs Jdt 8:12 Кто же вы, искушавшие сегодня Бога и ставшие вместо Бога посреди сынов человеческих?
\vs Jdt 8:13 Вот, вы теперь испытуете Господа Вседержителя, но никогда ничего не узнаете;
\vs Jdt 8:14 потому что вам не постигнуть глубины сердца у человека и не понять слов мысли его: как же испытаете вы Бога, сотворившего все это, и познаете ум Его, и поймете мысль Его? Нет, братья, не прогневляйте Господа, Бога нашего!
\vs Jdt 8:15 Ибо если Он не захочет помочь нам в эти пять дней, то Он имеет власть защитить нас в какие угодно Ему дни, или поразить нас пред лицем врагов наших.
\vs Jdt 8:16 Не отдавайте же в залог советов Господа Бога нашего: Богу нельзя грозить, как человеку, нельзя и указывать Ему, как сыну человеческому.
\vs Jdt 8:17 Посему, ожидая от Него спасения, будем призывать Его к себе на помощь, и Он услышит голос наш, если это Ему будет угодно.
\vs Jdt 8:18 Ибо не было в родах наших, и нет в настоящее время ни колена, ни племени, ни народа, ни города у нас, которые кланялись бы богам рукотворенным, как было в прежние дни,
\vs Jdt 8:19 за что отцы наши преданы были мечу и расхищению и пали великим падением пред нашими врагами.
\vs Jdt 8:20 Но мы не знаем другого Бога, кроме Его, а потому и надеемся, что Он не презрит нас и никого из нашего рода.
\vs Jdt 8:21 Ибо с пленением нас падет и вся Иудея, и святыни наши будут разграблены, и Он взыщет осквернение их от уст наших,
\vs Jdt 8:22 и убиение братьев наших и пленение земли и опустошение наследия нашего обратит на нашу голову среди народов, которым мы будем порабощены, и будем в соблазн и поношение у тех, которые овладеют нами;
\vs Jdt 8:23 потому что рабство не послужит нам в честь, но Господь, Бог наш, вменит его в бесчестие.
\vs Jdt 8:24 Итак, братья, покажем братьям нашим, что от нас зависит жизнь их, и на нас утверждаются и святыни, и дом \bibemph{Господень}, и жертвенник.
\vs Jdt 8:25 За все это возблагодарим Господа, Бога нашего, Который испытует нас, как и отцов наших.
\vs Jdt 8:26 Вспомните, чт\acc{о} Он сделал с Авраамом, чем искушал Исаака, чт\acc{о} было с Иаковом в Сирской Месопотамии, когда он пас овец Лавана, брата матери своей:
\vs Jdt 8:27 как их искушал Он не для истязания сердца их, так и нам не мстит, а только для вразумления наказывает Господь приближающихся к Нему.
\vs Jdt 8:28 Озия сказал ей: все, что ты сказала, сказала от доброго сердца, и никто не будет противиться словам твоим,
\vs Jdt 8:29 ибо не с настоящего только дня известна мудрость твоя, но от начала дней твоих весь народ знает разум твой и доброе расположение твоего сердца.
\vs Jdt 8:30 Но народ истомился от жажды и принудил нас поступить так, как мы сказали им, и обязал нас клятвою, которой мы не нарушим.
\vs Jdt 8:31 Помолись же о нас, ибо ты жена благочестивая, и Господь пошлет дождь для наполнения водохранилищ наших, и мы больше не будем изнемогать \bibemph{от жажды}.
\vs Jdt 8:32 Иудифь сказала им: послушайте меня,~--- и я совершу дело, которое пронесется сынами рода нашего в роды родов.
\vs Jdt 8:33 Станьте в эту ночь у ворот,~--- а я выйду с моею служанкою, и в продолжение дней, после которых вы решили отдать город нашим врагам, Господь посетит Израиля моею рукою.
\vs Jdt 8:34 Только не расспрашивайте о моем предприятии, потому что я не скажу вам, доколе не совершится то, что я намерена сделать.
\vs Jdt 8:35 И сказал ей Озия и начальники: ступай с миром, и Господь Бог пред тобою на отмщение врагам нашим!
\vs Jdt 8:36 И вышли из шатра ее и пошли к полкам своим.
\vs Jdt 9:1 А Иудифь пала на лице, посыпала голову свою пеплом и сбросила с себя вретище, в которое была одета; и только что воскурили в Иерусалиме, в доме Господнем, вечерний фимиам, Иудифь громким голосом воззвала к Господу и сказала:
\vs Jdt 9:2 Господи Боже отца моего Симеона, которому Ты дал в руку меч на отмщение иноплеменным, которые открыли ложесна девы для оскорбления, обнажили бедро для позора и осквернили ложесна для посрамления! Ты сказал: да не будет сего, а они сделали.
\vs Jdt 9:3 И за то Ты предал князей их на убиение, постель их, которая видела обольщение их, обагрил кровью и поразил рабов подле владетелей и владетелей на тронах их,
\vs Jdt 9:4 и отдал жен их в расхищение, дочерей их в плен и всю добычу в раздел сынам, возлюбленным Тобою, которые возревновали Твоею ревностью, возгнушались осквернением крови их, и призвали Тебя на помощь. Боже, Боже мой, услышь меня вдову!
\vs Jdt 9:5 Ты сотворил прежде сего бывшее, и сие и последующее за сим, и содержал в уме настоящее и грядущее, и, что помыслил Ты, то и совершилось;
\vs Jdt 9:6 что определил, то и явилось и сказало: вот я. Ибо все пути Твои готовы, и суд Твой \bibemph{Тобою} предвиден.
\vs Jdt 9:7 Вот, Ассирияне умножились в силе своей, гордятся конем и всадником, тщеславятся мышцею пеших, надеются и на щит и на копье и на лук и на пращу, а не знают того, что Ты~--- Господь, сокрушающий брани.
\vs Jdt 9:8 Господь~--- имя Тебе; сокруши же их крепость силою Твоею, и уничтожь их силу гневом Твоим, ибо они замыслили осквернить святилище Твое, поругаться над мирным селением имени славы Твоей и железом сокрушить рог Твоего жертвенника.
\vs Jdt 9:9 Воззри на превозношение их, пошли гнев Твой на главы их, дай вдовьей руке моей крепость на то, что задумала я.
\vs Jdt 9:10 Устами хитрости моей порази раба перед вождем, и вождя~--- перед рабом его, \bibemph{и} сокруши гордыню их рукою женскою;
\vs Jdt 9:11 ибо не во множестве сила Твоя и не в могучих могущество Твое; но Ты~--- Бог смиренных, Ты~--- помощник умаленных, заступник немощных, покровитель упавших духом, спаситель безнадежных.
\vs Jdt 9:12 Так, так, Боже отца моего и Боже наследия Израилева, Владыка неба и земли, Творец вод, Царь всякого создания Твоего! Услышь молитву мою,
\vs Jdt 9:13 сделай слово мое и хитрость мою раною и язвою для тех, которые задумали жестокое против завета Твоего, святаго дома Твоего, высоты Сиона и дома наследия сынов Твоих.
\vs Jdt 9:14 Вразуми весь народ Твой и всякое племя, чтобы видели они, что Ты~--- Бог, Бог всякой крепости и силы, и нет другого защитника рода Израилева, кроме Тебя.
\vs Jdt 10:1 Когда она перестала взывать к Богу Израилеву и окончила все эти слова
\vs Jdt 10:2 то поднялась на ноги, позвала служанку свою и вошла в дом, в котором она проводила субботние дни и праздники свои.
\vs Jdt 10:3 Здесь она сняла с себя вретище, которое надевала, сняла и одежды вдовства своего, омыла тело водою и намастилась драгоценным миром, причесала волосы и надела на голову повязку, оделась в одежды веселия своего, в которые она наряжалась во дни жизни мужа своего Манассии;
\vs Jdt 10:4 обула ноги свои в сандалии, и возложила на себя цепочки, запястья, кольца, серьги и все свои наряды, и разукрасила себя, чтобы прельстить глаза мужчин, которые увидят ее.
\vs Jdt 10:5 И дала служанке своей мех вина и сосуд масла, наполнила мешок мукою и сушеными плодами и чистыми хлебами и, обвернув все эти припасы свои, возложила их на нее.
\rsbpar\vs Jdt 10:6 Выйдя к воротам города Ветилуи, они нашли стоявшими при них Озию и старейшин города, Хаврина и Хармина.
\vs Jdt 10:7 Когда они увидели ее и перемену в ее лице и одежде, очень много дивились красоте ее и сказали ей:
\vs Jdt 10:8 Бог, Бог отцов наших, да даст тебе благодать и да совершит твои намерения на радость сынов Израиля и на возвеличение Иерусалима. Она поклонилась Богу
\vs Jdt 10:9 и сказала им: велите отворить для меня ворота города; я выйду для исполнения дела, о котором вы говорили со мною. И велели юношам отворить для нее, как она сказала.
\vs Jdt 10:10 Они исполнили это. И вышла Иудифь и служанка ее с нею; а мужи городские смотрели вслед за нею, пока она сходила с горы, пока проходила долиной и пока не скрылась от их глаз.
\vs Jdt 10:11 Они шли прямо долиною, и встретила \bibemph{Иудифь} передовая стража Ассириян,
\vs Jdt 10:12 и взяли ее и спросили: чья ты, откуда идешь и куда отправляешься? Она сказала: я дочь Евреев и бегу от них, потому что они будут преданы вам на истребление.
\vs Jdt 10:13 Я иду к Олоферну, вождю вашего войска, возвестить слова истины и указать ему путь, которым он пойдет и овладеет всею нагорною страною, так что не погибнет из мужей его ни один человек и ни одна живая душа.
\vs Jdt 10:14 Когда эти люди слушали слова ее и всматривались в лице ее,~--- она показалась им чудом по красоте, и они сказали ей:
\vs Jdt 10:15 ты спасла душу твою, поспешив прийти к господину нашему; ступай же к шатру его, а наши проводят тебя, пока не передадут тебя ему на руки.
\vs Jdt 10:16 Когда ты станешь перед ним,~--- не бойся сердцем твоим, но выскажи слова твои, и он тебя облагодетельствует.
\vs Jdt 10:17 И, выбрав из среды своей сто человек, приставили их к ней и к служанке ее, и они повели их к шатру Олоферна.
\vs Jdt 10:18 Во всем стане произошло движение, потому что весть о приходе ее разнеслась по шатрам: сбежавшиеся окружили ее, так как она стояла вне шатра Олоферна, пока не возвестили ему о ней;
\vs Jdt 10:19 и дивились красоте ее, а из-за нее дивились и сынам Израиля, и говорили каждый ближнему своему: кто пренебрежет таким народом, который имеет таких жен у себя! Неблагоразумно оставить из них ни одного мужа, потому что оставшиеся будут в состоянии перехитрить всю землю.
\vs Jdt 10:20 \bibemph{Между тем} спавшие при Олоферне и все служители его вышли и ввели ее в шатер.
\vs Jdt 10:21 Олоферн отдыхал на своей постели за занавесом, украшенным пурпуром, золотом, изумрудом и драгоценными камнями.
\vs Jdt 10:22 \bibemph{Когда} ему доложили о ней, он вышел в переднее отделение шатра, и перед ним несли серебряные лампады.
\vs Jdt 10:23 Когда Иудифь представилась ему и служителям его, все удивились красоте лица ее. Она, пав на лице, поклонилась ему, и служители его подняли ее.
\vs Jdt 11:1 Олоферн сказал ей: ободрись, жена; не бойся сердцем твоим, потому что я не сделал зла никому, кто добровольно решился служить Навуходоносору, царю всей земли.
\vs Jdt 11:2 И теперь, если бы народ твой, живущий в нагорной стране, не пренебрег мною, я не поднял бы на них копья моего; но они сами это сделали для себя.
\vs Jdt 11:3 Скажи же мне: почему ты бежала от них и пришла к нам? Ты найдешь себе \bibemph{здесь} спасение; не бойся: ты будешь жива в эту ночь и после,
\vs Jdt 11:4 потому что тебя никто не обидит, напротив, всякий будет благодетельствовать тебе, как бывает с рабами господина моего, царя Навуходоносора.
\vs Jdt 11:5 Иудифь сказала ему: выслушай слова рабы твоей; пусть раба говорит пред лицем твоим: я не скажу лжи господину моему в эту ночь.
\vs Jdt 11:6 И если ты последуешь словам рабы твоей, то Бог чрез тебя совершит дело, и господин мой не ошибется в своих предприятиях.
\vs Jdt 11:7 Да живет Навуходоносор, царь всей земли, и да живет держава его, пославшего тебя для исправления всякой души, потому что не только люди чрез тебя будут служить ему, но и звери полевые, и скот, и птицы небесные чрез твою силу будут жить под властью Навуходоносора и всего дома его.
\vs Jdt 11:8 Ибо мы слышали о твоей мудрости и хитрости ума твоего, и всей земле известно, что ты один добр во всем царстве, силен в знании и дивен в воинских подвигах.
\vs Jdt 11:9 А что говорил Ахиор в собрании твоем, мы слышали слова его, потому что мужи Ветилуи оставили его в живых, и он рассказал им все, о чем говорил тебе.
\vs Jdt 11:10 Посему, владыка-господин, не оставляй без внимания сл\acc{о}ва его, но сложи его в сердце твоем, потому что оно истинно: род наш не наказывается, меч не имеет силы над нами, если они не грешат пред Богом своим.
\vs Jdt 11:11 Итак, чтобы господин мой не был отражен и безуспешен и чтобы их постигла смерть,~--- овладел ими грех, которым они прогневляют Бога своего, делая то, чего не следует;
\vs Jdt 11:12 потому что у них оказался недостаток в пище и вся вода истощилась,~--- и \bibemph{вот}, они решились броситься на скот свой и думают питаться всем, что Бог строго запретил в законе Своем употреблять в пищу.
\vs Jdt 11:13 Даже начатки пшеницы и десятины вина и масла, которые, по освящении, хранятся для священников, предстоящих пред лицем Бога нашего в Иерусалиме, они решились употребить, тогда как и руками касаться их не следовало никому из народа.
\vs Jdt 11:14 Они послали в Иерусалим, так как и тамошние жители делали это, принести к ним разрешение на то от собрания старейшин.
\vs Jdt 11:15 И как скоро им дано будет известие, и они сделают это, то в тот же день будут преданы тебе на погубление.
\vs Jdt 11:16 Вот почему я, раба твоя, узнав обо всем этом, бежала от них, и Бог послал меня сделать вместе с тобою такие дела, которым изумится вся земля, где только услышат о них,
\vs Jdt 11:17 ибо раба твоя благочестива и день и ночь служит Богу Небесному. Теперь, господин мой, я останусь у тебя; только пусть раба твоя по ночам выходит на долину молиться Богу,~--- и Он откроет мне, когда они сделают свое преступление.
\vs Jdt 11:18 Я приду и объявлю тебе, и ты выходи \bibemph{тогда} со всем твоим войском,~--- и никто из них не противостанет тебе.
\vs Jdt 11:19 Я поведу тебя чрез Иудею, доколе не дойдем до Иерусалима; поставлю среди его седалище твое, и ты погонишь их, как овец, не имеющих пастуха,~--- и пес не пошевелит против тебя языком своим. Это сказано мне по откровению и объявлено мне, и я послана возвестить тебе.
\vs Jdt 11:20 Понравились слова ее Олоферну и всем слугам его. Они дивились мудрости ее и говорили:
\vs Jdt 11:21 от края до края земли нет такой жены по красоте лица и по разумным речам.
\vs Jdt 11:22 Олоферн сказал ей: хорошо Бог сделал, что вперед этого народа послал тебя, чтобы в руках наших была сила, а среди презревших господина моего~--- гибель.
\vs Jdt 11:23 Прекрасна ты лицем, и добры речи твои. Если ты сделаешь, как сказала, то твой Бог будет моим Богом; ты будешь жить в доме царя Навуходоносора и будешь именита во всей земле.
\vs Jdt 12:1 И приказал ввести ее \bibemph{туда}, где хранились серебряные сосуды его, и велел ей пользоваться пищею от стола его и пить вино его.
\vs Jdt 12:2 Но Иудифь сказала: не буду есть этого, чтобы не было соблазна, но пусть подают мне то, что принесено со мною.
\vs Jdt 12:3 Олоферн сказал ей: а когда истощится то, что с тобою, откуда мы возьмем, чтобы подавать тебе подобное этому? Ибо среди нас нет никого из рода твоего.
\vs Jdt 12:4 Иудифь отвечала ему: да живет душа твоя, господин мой; раба твоя не издержит того, что со мною, прежде, нежели Господь совершит моею рукою то, что Он определил.
\vs Jdt 12:5 И ввели ее слуги Олоферна в шатер, и спала она до полночи; а пред утреннею стражею встала
\vs Jdt 12:6 и послала сказать Олоферну: да даст господин мой повеление, чтобы рабе твоей дозволили выходить на молитву.
\vs Jdt 12:7 Олоферн приказал своим телохранителям не препятствовать ей. И пробыла она в лагере три дня, а по ночам выходила в долину Ветилуи, омывалась при источнике воды у лагеря.
\vs Jdt 12:8 И, выходя, молилась Господу, Богу Израилеву, чтоб Он направил путь ее к избавлению сынов Его народа.
\vs Jdt 12:9 По возвращении она пребывала в шатре чистою, а к вечеру приносили ей пищу.
\vs Jdt 12:10 В четвертый день Олоферн сделал пир для одних слуг своих и не пригласил к услужению никого из приставленных к службам.
\vs Jdt 12:11 И сказал евнуху Вагою, управлявшему всем, что у него было: ступай и убеди Еврейскую женщину, которая у тебя, прийти к нам и есть и пить с нами:
\vs Jdt 12:12 стыдно нам оставить такую жену, не побеседовав с нею; она осмеет нас, если мы не пригласим ее.
\vs Jdt 12:13 Вагой, выйдя от Олоферна, пришел к ней и сказал: не откажись, прекрасная молодая женщина, прийти к господину моему, чтобы принять честь пред лицем его и пить с нами вино в веселие и быть в этот день как одною из дочерей сынов Ассура, которые предстоят в доме Навуходоносора.
\vs Jdt 12:14 Иудифь сказала ему: кто я, чтобы прекословить господину моему? поспешу исполнить все, что будет угодно господину моему, и это будет служить мне утешением до дня смерти моей.
\vs Jdt 12:15 Она встала и нарядилась в одежду и во все женское украшение; а служанка ее пришла и разостлала для нее по земле пред Олоферном ковры, которые она получила от Вагоя для всегдашнего употребления, чтобы есть, возлежа на них.
\vs Jdt 12:16 Затем Иудифь пришла и возлегла. Подвиглось сердце Олоферна к ней, и душа его взволновалась: он сильно желал сойтись с нею и искал случая обольстить ее с того самого дня, как увидел ее.
\vs Jdt 12:17 И сказал ей Олоферн: пей же и веселись с нами.
\vs Jdt 12:18 А Иудифь сказала: буду пить, господин, потому что сегодня жизнь моя возвеличилась во мне больше, нежели во все дни от рождения моего.
\vs Jdt 12:19 И она брала, ела и пила пред ним, что приготовила служанка ее.
\vs Jdt 12:20 А Олоферн любовался на нее и пил вина весьма много, сколько не пил никогда, ни в один день от рождения.
\vs Jdt 13:1 Когда поздно стало, рабы его поспешили удалиться, а Вагой, отпустив предстоявших пред лицем его господина, затворил шатер снаружи, и они пошли к постелям своим, так как все были утомлены продолжительностью пира.
\vs Jdt 13:2 В шатре осталась одна Иудифь с Олоферном, упавшим на ложе свое, потому что был переполнен вином.
\vs Jdt 13:3 Иудифь велела служанке своей стать вне спальни ее и ожидать ее выхода, как было каждый день, сказав, что она выйдет на молитву. То же самое сказала она и Вагою.
\vs Jdt 13:4 \bibemph{Когда} все от нее ушли и никого в спальне не осталось, ни малого, ни большого, Иудифь, став у постели \bibemph{Олоферна}, сказала в сердце своем: Господи, Боже всякой силы! призри в час сей на дела рук моих к возвышению Иерусалима,
\vs Jdt 13:5 ибо теперь время защитить наследие Твое и исполнить мое намерение, поразить врагов, восставших на нас.
\vs Jdt 13:6 \bibemph{Потом}, подойдя к столбику постели, стоявшему в головах у Олоферна, она сняла с него меч его
\vs Jdt 13:7 и, приблизившись к постели, схватила волосы головы его и сказала: Господи, Боже Израиля! укрепи меня в этот день.
\vs Jdt 13:8 И изо всей силы дважды ударила по шее \bibemph{Олоферна} и сняла с него голову
\vs Jdt 13:9 и, сбросив с постели тело его, взяла со столбов занавес. Спустя немного она вышла и отдала служанке своей голову Олоферна,
\vs Jdt 13:10 а эта положила ее в мешок со съестными припасами, и обе вместе вышли, по обычаю своему, на молитву. Пройдя стан, они обошли кругом ущелье, поднялись на гору Ветилуи и пошли к воротам ее.
\vs Jdt 13:11 Иудифь издали кричала сторожившим при воротах: отворите, отворите ворота! с нами Бог, Бог наш, чтобы даровать еще силу Израилю и победу над врагами, как даровал Он и сегодня.
\vs Jdt 13:12 Как только услышали городские мужи голос ее, поспешили прийти к городским воротам и созвали старейшин города.
\vs Jdt 13:13 И сбежались все, от малого до большого, так как приход ее был для них сверх ожидания, и, отворив ворота, приняли их, и, зажегши для освещения огонь, окружили их.
\vs Jdt 13:14 Она же сказала им громким голосом: хвалите Господа, хвалите, хвалите Господа, что Он не удалил милости Своей от дома Израилева, но в эту ночь сокрушил врагов наших моею рукою.
\vs Jdt 13:15 И, вынув голову из мешка, показала ее и сказала им: вот голова Олоферна, вождя Ассирийского войска, и вот занавес его, за которым он лежал от опьянения,~--- и Господь поразил его рукою женщины.
\vs Jdt 13:16 Жив Господь, сохранивший меня в пути, которым я шла! ибо лице мое прельстило \bibemph{Олоферна} на погибель его, но он не сделал со мною скверного и постыдного греха.
\vs Jdt 13:17 Весь народ чрезвычайно изумился; пали, поклонились Богу и единодушно сказали: благословен Ты, Боже наш, уничиживший сегодня врагов народа Твоего!
\vs Jdt 13:18 А Озия сказал ей: благословенна ты, дочь, Всевышним Богом более всех жен на земле, и благословен Господь Бог, создавший небеса и землю и наставивший тебя на поражение головы начальника наших врагов;
\vs Jdt 13:19 ибо надежда твоя не отступит от сердца людей, помнящих силу Божию, до века.
\vs Jdt 13:20 Да вменит тебе это Бог в вечную славу и да наградит тебя благами за то, что ты жизни твоей не пощадила при унижении рода нашего, но выступила вперед, когда мы падали, ты, право ходившая пред Богом нашим. И весь народ сказал: да будет, да будет!
\vs Jdt 14:1 Иудифь сказала им: послушайте же меня, братья, возьмите эту голову и повесьте на зубцах вашей стены.
\vs Jdt 14:2 Когда же настанет утро и солнце взойдет над землею, возьмите каждый боевое свое оружие, идите все сильные за город и дайте им вождя, как будто намереваясь сойти на равнину против передовой стражи сынов Ассура, но не сходите.
\vs Jdt 14:3 Тогда они, взяв все свое оружие, пойдут в свой стан, разбудят вождей войска Ассирийского, и сбегутся к шатру \bibemph{Олоферна}, но не найдут его; оттого нападет на них страх, и они побегут от вас.
\vs Jdt 14:4 А вы и все живущие во всяком пределе Израильском, преследуя их, поражайте их на пути.
\vs Jdt 14:5 Но прежде, чем сделаете это, пригласите ко мне Ахиора Аммонитянина: пусть увидит и узнает он того, кто уничижал дом Израиля и прислал его к нам будто на смерть.
\rsbpar\vs Jdt 14:6 И призвали Ахиора из дома Озии. Когда он пришел и увидел голову Олоферна в руке одного мужа среди собрания народа, то пал на лице свое и ослабел духом.
\vs Jdt 14:7 Когда же подняли его, он припал к ногам Иудифи, поклонился ей и сказал: благословенна ты во всяком селении Иуды и во всяком народе, которые, услышав об имени твоем, изумятся.
\vs Jdt 14:8 Расскажи же мне теперь, что ты делала в эти дни? И Иудифь среди народа рассказала ему все, что она сделала с того дня, как вышла, до того дня, в который говорила с ними.
\vs Jdt 14:9 Когда она перестала говорить, народ громко воскликнул, и радостный крик его раздался в городе.
\vs Jdt 14:10 Ахиор же, видя все, что сделал Бог Израилев, искренно уверовал в Бога, обрезал крайнюю плоть свою и присоединился к дому Израилеву, даже до сего дня.
\rsbpar\vs Jdt 14:11 Когда настало утро, повесили голову Олоферна на стену; каждый муж взял свое оружие, и вышли отрядами на всходы горы.
\vs Jdt 14:12 Сыны Ассура, увидев их, послали к своим начальникам, а они пошли к вождям, к тысяченачальникам и ко всякому предводителю своему.
\vs Jdt 14:13 Придя к шатру Олоферна, они сказали управлявшему всем имением его: разбуди нашего господина, потому что эти рабы осмелились выйти на сражение с нами, чтобы быть совершенно истребленными.
\vs Jdt 14:14 Вагой вошел и постучался в дверь шатра, ибо думал, что он спит с Иудифью.
\vs Jdt 14:15 Когда же никто не отзывался ему, то, отворив, вошел в спальню и нашел, что \bibemph{Олоферн} мертвый лежит у порога и голова его снята с него.
\vs Jdt 14:16 И он громко воскликнул с плачем, стоном и крепким воплем, и разорвал свои одежды.
\vs Jdt 14:17 Потом вошел в шатер, в котором пребывала Иудифь, и не нашел ее. Тогда он выскочил к народу и закричал:
\vs Jdt 14:18 рабы поступили вероломно; одна Еврейская жена опозорила дом царя Навуходоносора, ибо вот Олоферн на полу и головы нет на нем.
\vs Jdt 14:19 Когда услышали эти слова начальники войска Ассирийского, то разорвали одежды свои, и душа их сильно смутилась, и раздался у них крик и весьма великий вопль среди стана.
\vs Jdt 15:1 Когда бывшие в шатрах услышали о том, что случилось, то смутились,
\vs Jdt 15:2 и напал на них страх и трепет, и ни один из них не остался в глазах ближнего, но все они бросившись бежали по всем дорогам равнины и нагорной страны.
\vs Jdt 15:3 И расположившиеся лагерем в нагорной стране около Ветилуи также обратились в бегство. Тогда сыны Израиля, каждый из них воинственный муж, погнались за ними.
\vs Jdt 15:4 Озия послал в Ветомасфем, Виваю, Ховаю и Холу и во все пределы Израильские, чтобы известить о совершившемся и чтобы все погнались за неприятелями для истребления их.
\vs Jdt 15:5 Как скоро услышали об этом сыны Израиля, все дружно напали на них и поражали их до Ховы; равно и пришедшие из Иерусалима и из всей нагорной страны, так как им возвещено было о том, чт\acc{о} случилось в стане врагов их, и из Галаада и Галилеи, со всех сторон наносили им большое поражение, доколе они не прошли за Дамаск и за пределы его.
\vs Jdt 15:6 Прочие жители Ветилуи напали на стан Ассирийский, разграбили его и весьма обогатились.
\vs Jdt 15:7 А сыны Израиля, возвратившиеся от поражения, овладели остальным; и села и деревни в нагорной стране и на равнине получили большую добычу, потому что ее было весьма многое множество.
\vs Jdt 15:8 Великий священник Иоаким и старейшины сынов Израилевых, жившие в Иерусалиме, пришли посмотреть, какое благо сотворил Господь для Израиля, и видеть Иудифь и приветствовать ее.
\vs Jdt 15:9 Как только они вошли к ней, то все единодушно благословили ее и сказали ей: ты величие Израиля, ты великая радость Израиля, ты великая слава нашего рода.
\vs Jdt 15:10 Все это ты сделала твоею рукою; ты сделала добро Израилю, и да благоволит к нему Бог; будь \bibemph{же} благословенна от Господа Вседержителя на вечное время. И весь народ сказал: да будет!
\rsbpar\vs Jdt 15:11 Народ расхищал лагерь в продолжение тридцати дней, и Иудифи отдали шатер Олоферна и все серебряные сосуды и постели и чаши и всю утварь его. Она взяла, возложила на мула своего, запрягла колесницы свои и сложила это на них.
\vs Jdt 15:12 И сбежались все жены Израильские видеть ее, и благословляли ее и составили из себя для нее хор; а она взяла в свои руки обвитые виноградными листьями жезлы и дала женщинам, бывшим с нею,
\vs Jdt 15:13 и возложили на себя масличные венки~--- она и бывшие с нею. Она шла впереди всего народа в хоре и вела за собою всех жен; за нею следовали все мужи Израильские, вооруженные, с венками и с торжественными песнями в своих устах.
\vs Jdt 15:14 Иудифь начала пред всем Израилем благодарственную песнь, и весь народ подпевал эту песнь.
\vs Jdt 16:1 И сказала Иудифь: начните Богу моему на тимпанах, пойте Господу моему на кимвалах, стройно воспевайте Ему новую песнь, возносите и призывайте имя Его;
\vs Jdt 16:2 потому что Он есть Бог Господь, сокрушающий брани, потому что Он ополчился за меня среди народа и исторг меня из руки моих преследователей.
\vs Jdt 16:3 Пришел Ассур с гор севера, пришел с мириадами войска своего, и множество их запрудило воду в источниках, и конница их покрыла холмы.
\vs Jdt 16:4 Он сказал, что пределы мои сожжет, юношей моих мечом истребит, грудных младенцев бросит о землю, малых детей моих отдаст на расхищение, дев моих пленит.
\vs Jdt 16:5 Но Господь Вседержитель низложил их рукою жены.
\vs Jdt 16:6 Не от юношей пал сильный их, не сыны титанов поразили его, и не рослые исполины налегли на него, но Иудифь, дочь Мерарии, красотою лица своего погубила его;
\vs Jdt 16:7 потому что она для возвышения бедствовавших в Израиле сняла с себя одежды вдовства своего, помазала лице свое благовонною мастью,
\vs Jdt 16:8 украсила волосы свои головным убором, надела для прельщения его льняную одежду.
\vs Jdt 16:9 Ее сандалии восхитили взор его, и красота ее пленила душу его; меч прошел по шее его.
\vs Jdt 16:10 Персы ужаснулись отваги ее, и М\acc{и}дяне растерялись от смелости ее.
\vs Jdt 16:11 Тогда воскликнули смиренные мои,~--- и они испугались; немощные мои,~--- и они пришли в смущение; возвысили голос свой,~--- и они обратились в бегство.
\vs Jdt 16:12 Сыновья молодых жен кололи их и, как детям беглых рабов, наносили им раны; они погибли от ополчения Господа моего.
\vs Jdt 16:13 Воспою Господу моему песнь новую. Велик Ты, Господи, и славен, дивен силою и непобедим!
\vs Jdt 16:14 Да работает Тебе всякое создание Твое: ибо Ты сказал,~--- и совершилось; Ты послал Духа Твоего,~--- и устроилось,~--- и нет \bibemph{никого}, кто противостал бы гласу Твоему.
\vs Jdt 16:15 Горы с водами подвигнутся с оснований, и камни, как воск, растают от лица Твоего, но к боящимся Тебя Ты благомилостив.
\vs Jdt 16:16 Мала всякая жертва для вон\acc{и} благоухания, и всякий тук ничтожен для всесожжения Тебе, но боящийся Господа всегда велик.
\vs Jdt 16:17 Горе народам, восстающим на род мой: Господь Вседержитель отмстит им в день суда, пошлет огонь и червей на их тела,~--- и они будут чувствовать \bibemph{боль} и плакать вечно.
\rsbpar\vs Jdt 16:18 Когда пришли в Иерусалим, они поклонились Богу, и, когда народ очистился, вознесли всесожжения свои и доброхотные \bibemph{жертвы} свои и дары свои.
\vs Jdt 16:19 Иудифь же принесла все сосуды Олоферна, которые отдал ей народ, и занавес, который она взяла из спальни его, отдала в жертву Господу.
\vs Jdt 16:20 Народ веселился в Иерусалиме пред святилищем три месяца, и Иудифь пребывала с ними.
\vs Jdt 16:21 Но после сих дней каждый возвратился в удел свой, а Иудифь отправилась в Ветилую, \bibemph{где} оставалась в имении своем, и была в свое время славною во всей земле.
\vs Jdt 16:22 Многие желали ее, но мужчина не познал ее во все дни ее жизни с того дня, как муж ее Манассия умер и приложился к народу своему.
\vs Jdt 16:23 Она приобрела великую славу и состарилась в доме мужа своего, \bibemph{прожив} до ста пяти лет, и отпустила служанку свою на свободу. Она умерла в Ветилуе, и похоронили ее в пещере мужа ее Манассии.
\vs Jdt 16:24 Дом Израиля оплакивал ее семь дней. Имение же свое прежде смерти своей она разделила между родственниками Манассии, мужа своего, и между близкими из рода своего.
\vs Jdt 16:25 И никто более не устрашал сынов Израиля во дни Иудифи и много дней по смерти ее.
\newbookpage
\bibbookdescr{Est}{
  inline={\LARGE Книга\\\Huge Есфирь},
  toc={Есфирь},
  bookmark={Есфирь},
  header={Есфирь},
  %headerleft={},
  %headerright={},
  abbr={Есф}
}
\vs Est 0:0 [Во второй год царствования Артаксеркса великого, в первый день месяца Нисана, сон видел Мардохей, сын Иаиров, Семеев, Кисеев, из колена Вениаминова, Иудеянин, живший в городе Сузах, человек великий, служивший при царском дворце. Он был из пленников, которых Навуходоносор, царь Вавилонский, взял в плен из Иерусалима с Иехониею, царем Иудейским. Сон же его такой: вот ужасный шум, гром и землетрясение и смятение на земле; и вот, вышли два больших змея, готовые драться друг с другом; и велик был вой их, и по вою их все народы приготовились к войне, чтобы поразить народ праведных; и вот~--- день тьмы и мрака, скорбь и стеснение, страдание и смятение великое на земле; и смутился весь народ праведных, опасаясь бед себе, и приготовились они погибнуть и стали взывать к Господу; от вопля их произошла, как бы от малого источника, великая река с множеством воды; и воссиял свет и солнце, и вознеслись смиренные и истребили тщеславных.~--- Мардохей, пробудившись после этого сновидения, \bibemph{изображавшего}, чт\acc{о} Бог хотел совершить, содержал этот сон в сердце и желал уразуметь его во всех частях его, до ночи. И пребывал Мардохей во дворце вместе с Гавафою и Фаррою, двумя царскими евнухами, оберегавшими дворец, и услышал разговоры их и разведал замыслы их и узнал, что они готовятся наложить руки на царя Артаксеркса, и донес о них царю; а царь пытал этих двух евнухов, и, когда они сознались, были казнены. Царь записал это событие на память, и Мардохей записал об этом событии. И приказал царь Мардохею служить во дворце и дал ему подарки за это. При царе же был \bibemph{тогда} знатен Аман, сын Амадафов, Вугеянин, и старался он причинить зло Мардохею и народу его за двух евнухов царских.]
\rsbpar\vs Est 1:1 И было [после сего] во дни Артаксеркса,~--- этот Артаксеркс царствовал над ста двадцатью семью областями от Индии и до Ефиопии,~---
\vs Est 1:2 в то время, как царь Артаксеркс сел на царский престол свой, что в Сузах, городе престольном,
\vs Est 1:3 в третий год своего царствования он сделал пир для всех князей своих и для служащих при нем, для главных начальников войска Персидского и Мидийского и для правителей областей своих,
\vs Est 1:4 показывая великое богатство царства своего и отличный блеск величия своего \bibemph{в течение} многих дней, ста восьмидесяти дней.
\vs Est 1:5 По окончании сих дней, сделал царь для народа своего, находившегося в престольном городе Сузах, от большого до малого, пир семидневный на садовом дворе дома царского.
\vs Est 1:6 Белые, бумажные и яхонтового цвета шерстяные ткани, прикрепленные виссонными и пурпуровыми шнурами, \bibemph{висели} на серебряных кольцах и мраморных столбах.
\vs Est 1:7 Золотые и серебряные ложа \bibemph{были} на помосте, устланном камнями зеленого цвета и мрамором, и перламутром, и камнями черного цвета.
\vs Est 1:8 Напитки подаваемы \bibemph{были} в золотых сосудах и сосудах разнообразных, ценою в тридцать тысяч талантов; и вина царского было множество, по богатству царя. Питье \bibemph{шло} чинно, никто не принуждал, потому что царь дал такое приказание всем управляющим в доме его, чтобы делали по воле каждого.
\vs Est 1:9 И царица Астинь сделала также пир для женщин в царском доме царя Артаксеркса.
\vs Est 1:10 В седьмой день, когда развеселилось сердце царя от вина, он сказал Мегуману, Бизфе, Харбоне, Бигфе и Авагфе, Зефару и Каркасу~--- семи евнухам, служившим пред лицем царя Артаксеркса,
\vs Est 1:11 чтобы они привели царицу Астинь пред лице царя в венце царском для того, чтобы показать народам и князьям красоту ее; потому что она была очень красива.
\vs Est 1:12 Но царица Астинь не захотела прийти по приказанию царя, \bibemph{объявленному} чрез евнухов.
\vs Est 1:13 И разгневался царь сильно, и ярость его загорелась в нем. И сказал царь мудрецам, знающим \bibemph{прежние} времена,~--- ибо дела царя \bibemph{делались} пред всеми знающими закон и прав\acc{а},~---
\vs Est 1:14 приближенными же к нему \bibemph{тогда были}: Каршена, Шефар, Адмафа, Фарсис, Мерес, Марсена, Мемухан~--- семь князей Персидских и Мидийских, которые могли видеть лице царя \bibemph{и} сидели первыми в царстве:
\vs Est 1:15 как поступить по закону с царицею Астинь за то, что она не сделала по слову царя Артаксеркса, \bibemph{объявленному} чрез евнухов?
\vs Est 1:16 И сказал Мемухан пред лицем царя и князей: не пред царем одним виновна царица Астинь, а пред всеми князьями и пред всеми народами, которые по всем областям царя Артаксеркса;
\vs Est 1:17 потому что поступок царицы дойдет до всех жен, и они будут пренебрегать мужьями своими и говорить: царь Артаксеркс велел привести царицу Астинь пред лице свое, а она не пошла.
\vs Est 1:18 Теперь княгини Персидские и Мидийские, которые услышат о поступке царицы, будут \bibemph{то же} говорить всем князьям царя; и пренебрежения и огорчения будет довольно.
\vs Est 1:19 Если благоугодно царю, пусть выйдет от него царское постановление и впишется в законы Персидские и Мидийские и не отменяется, о том, что Астинь не будет входить пред лице царя Артаксеркса, а царское достоинство ее царь передаст другой, которая лучше ее.
\vs Est 1:20 Когда услышат о сем постановлении царя, которое разойдется по всему царству его, как оно ни велико, тогда все жены будут почитать мужей своих, от большого до малого.
\vs Est 1:21 И угодно было слово сие в глазах царя и князей; и сделал царь по слову Мемухана.
\vs Est 1:22 И послал во все области царя письма, писанные в каждую область письменами ее и к каждому народу на языке его, чтобы всякий муж был господином в доме своем, и чтобы это было объявлено каждому на природном языке его.
\vs Est 2:1 После сего, когда утих гнев царя Артаксеркса, он вспомнил об Астинь и о том, что она сделала, и что было определено о ней.
\vs Est 2:2 И сказали отроки царя, служившие при нем: пусть бы поискали царю молодых красивых девиц,
\vs Est 2:3 и пусть бы назначил царь наблюдателей во все области своего царства, которые собрали бы всех молодых девиц, красивых видом, в престольный город Сузы, в дом жен под надзор Гегая, царского евнуха, стража жен, и пусть бы выдавали им притиранья [и прочее, что нужно];
\vs Est 2:4 и девица, которая понравится глазам царя, пусть будет царицею вместо Астинь. И угодно было слово это в глазах царя, и он так и сделал.
\rsbpar\vs Est 2:5 Был в Сузах, городе престольном, один Иудеянин, имя его Мардохей, сын Иаира, сын Семея, сын Киса, из колена Вениаминова.
\vs Est 2:6 Он был переселен из Иерусалима вместе с пленниками, выведенными с Иехониею, царем Иудейским, которых переселил Навуходоносор, царь Вавилонский.
\vs Est 2:7 И был он воспитателем Гадассы,~--- она же Есфирь,~--- дочери дяди его, так как не было у нее ни отца, ни матери. Девица эта была красива станом и пригожа лицем. И по смерти отца ее и матери ее, Мардохей взял ее к себе вместо дочери.
\rsbpar\vs Est 2:8 Когда объявлено было повеление царя и указ его, и когда собраны были многие девицы в престольный город Сузы под надзор Гегая, тогда взята была и Есфирь в царский дом под надзор Гегая, стража жен.
\vs Est 2:9 И понравилась эта девица глазам его и приобрела у него благоволение, и он поспешил выдать ей притиранья и \bibemph{все, назначенное на} часть ее, и приставить к ней семь девиц, достойных быть при ней, из дома царского, и переместил ее и девиц ее в лучшее отделение женского дома.
\vs Est 2:10 Не сказывала Есфирь ни о народе своем, ни о родстве своем, потому что Мардохей дал ей приказание, чтобы она не сказывала.
\vs Est 2:11 И всякий день Мардохей приходил ко двору женского дома, чтобы наведываться о здоровье Есфири и о том, что делается с нею.
\rsbpar\vs Est 2:12 Когда наступало время каждой девице входить к царю Артаксерксу, после того, как в течение двенадцати месяцев выполнено было над нею все, определенное женщинам,~--- ибо столько времени продолжались дни притиранья их: шесть месяцев мирровым маслом и шесть месяцев ароматами и другими притираньями женскими,~---
\vs Est 2:13 тогда девица входила к царю. Чего бы она ни потребовала, ей давали всё для выхода из женского дома в дом царя.
\vs Est 2:14 Вечером она входила и утром возвращалась в другой дом женский под надзор Шаазгаза, царского евнуха, стража наложниц; и уже не входила к царю, разве только царь пожелал бы ее, и она призывалась бы по имени.
\rsbpar\vs Est 2:15 Когда настало время Есфири, дочери Аминадава, дяди Мардохея, который взял ее к себе вместо дочери,~--- идти к царю, тогда она не просила ничего, кроме того, о чем сказал ей Гегай, евнух царский, страж жен. И приобрела Есфирь расположение \bibemph{к себе} в глазах всех, видевших ее.
\vs Est 2:16 И взята была Есфирь к царю Артаксерксу, в царский дом его, в десятом месяце, то есть в месяце Тебефе, в седьмой год его царствования.
\vs Est 2:17 И полюбил царь Есфирь более всех жен, и она приобрела его благоволение и благорасположение более всех девиц; и он возложил царский венец на голову ее и сделал ее царицею на место Астинь.
\vs Est 2:18 И сделал царь большой пир для всех князей своих и для служащих при нем,~--- пир ради Есфири, и сделал льготу областям и раздал дары с царственною щедростью.
\vs Est 2:19 И когда во второй раз собраны были девицы, и Мардохей сидел у ворот царских,
\vs Est 2:20 Есфирь все еще не сказывала о родстве своем и о народе своем, как приказал ей Мардохей; а слово Мардохея Есфирь выполняла \bibemph{и теперь} так же, как тогда, когда была у него на воспитании.
\vs Est 2:21 В это время, как Мардохей сидел у ворот царских, два царских евнуха, Гавафа и Фарра, оберегавшие порог, озлобились [за то, что предпочтен был Мардохей], и замышляли наложить руку на царя Артаксеркса.
\vs Est 2:22 Узнав о том, Мардохей сообщил царице Есфири, а Есфирь сказала царю от имени Мардохея.
\vs Est 2:23 Дело было исследовано и найдено \bibemph{верным}, и их обоих повесили на дереве. И было вписано о благодеянии Мардохея в книгу дневных записей у царя.
\vs Est 3:1 После сего возвеличил царь Артаксеркс Амана, сына Амадафа, Вугеянина, и вознес его, и поставил седалище его выше всех князей, которые у него;
\vs Est 3:2 и все служащие при царе, которые \bibemph{были} у царских ворот, кланялись и падали ниц пред Аманом, ибо так приказал царь. А Мардохей не кланялся и не падал ниц.
\vs Est 3:3 И говорили служащие при царе, которые у царских ворот, Мардохею: зачем ты преступаешь повеление царское?
\vs Est 3:4 И как они говорили ему каждый день, а он не слушал их, то они донесли Аману, чтобы посмотреть, устоит ли в слове \bibemph{своем} Мардохей, ибо он сообщил им, что он Иудеянин.
\rsbpar\vs Est 3:5 И когда увидел Аман, что Мардохей не кланяется и не падает ниц пред ним, то исполнился гнева Аман.
\vs Est 3:6 И показалось ему ничтожным наложить руку на одного Мардохея; но так как сказали ему, из какого народа Мардохей, то задумал Аман истребить всех Иудеев, которые \bibemph{были} во всем царстве Артаксеркса, \bibemph{как} народ Мардохеев.
\vs Est 3:7 [И сделал совет] в первый месяц, который есть месяц Нисан, в двенадцатый год царя Артаксеркса, и бросали пур, то есть жребий, пред лицем Амана изо дня в день и из месяца в месяц, [чтобы в один день погубить народ Мардохеев, и пал жребий] на двенадцатый \bibemph{месяц}, то есть на месяц Адар.
\vs Est 3:8 И сказал Аман царю Артаксерксу: есть один народ, разбросанный и рассеянный между народами по всем областям царства твоего; и законы их отличны от \bibemph{законов} всех народов, и законов царя они не выполняют; и царю не следует \bibemph{так} оставлять их.
\vs Est 3:9 Если царю благоугодно, то пусть будет предписано истребить их, и десять тысяч талантов серебра я отвешу в руки приставников, чтобы внести в казну царскую.
\vs Est 3:10 Тогда снял царь перстень свой с руки своей и отдал его Аману, сыну Амадафа, Вугеянину, чтобы скрепить указ против Иудеев.
\vs Est 3:11 И сказал царь Аману: отдаю тебе \bibemph{это} серебро и народ; поступи с ним, как тебе угодно.
\rsbpar\vs Est 3:12 И призваны были писцы царские в первый месяц, в тринадцатый день его, и написано было, как приказал Аман, к сатрапам царским и к начальствующим над каждою областью [от области Индийской до Ефиопии, над ста двадцатью семью областями], и к князьям у каждого народа, в каждую область письменами ее и к каждому народу на языке его: \bibemph{все} было написано от имени царя Артаксеркса и скреплено царским перстнем.
\vs Est 3:13 И посланы были письма через гонцов во все области царя, чтобы убить, погубить и истребить всех Иудеев, малого и старого, детей и женщин в один день, в тринадцатый день двенадцатого месяца, то есть месяца Адара, и имение их разграбить. [Вот список с этого письма: великий царь Артаксеркс начальствующим от Индии до Ефиопии над ста двадцатью семью областями и подчиненным им наместникам. Царствуя над многими народами и властвуя над всею вселенною, я хотел, не превозносясь гордостью власти, но управляя всегда кротко и тихо, сделать жизнь подданных постоянно безмятежною и, соблюдая царство свое мирным и удобопроходимым до пределов \bibemph{его}, восстановить желаемый для всех людей мир. Когда же я спросил советников, каким бы образом привести это в исполнение, то отличающийся у нас мудростью и \bibemph{пользующийся} неизменным благоволением, и доказавший твердую верность, и получивший вторую честь по царе, Аман объяснил нам, что во всех племенах вселенной замешался один враждебный народ, по законам \bibemph{своим} противный всякому народу, постоянно пренебрегающий царскими повелениями, дабы не благоустроялось безукоризненно совершаемое нами соуправление. Итак, узнав, что один только этот народ всегда противится всякому человеку, ведет образ жизни, чуждый законам, и, противясь нашим действиям, совершает величайшие злодеяния, чтобы царство \bibemph{наше} не достигло благосостояния, мы повелели указанных вам в грамотах Амана, поставленного над делами и второго отца нашего, всех с женами и детьми всецело истребить вражескими мечами, без всякого сожаления и пощады, в тринадцатый день двенадцатого месяца Адара настоящего года, чтобы эти и прежде и теперь враждебные \bibemph{люди}, быв в один день насильно низвергнуты в преисподнюю, не препятствовали нам в последующее время проводить жизнь мирно и безмятежно до конца.]
\vs Est 3:14 Список с указа отдать в каждую область \bibemph{как} закон, объявляемый для всех народов, чтобы они были готовы к тому дню.
\vs Est 3:15 Гонцы отправились быстро с царским повелением. Объявлен был указ и в Сузах, престольном городе; и царь и Аман сидели и пили, а город Сузы \bibemph{был} в смятении.
\vs Est 4:1 Когда Мардохей узнал все, что делалось, разодрал одежды свои и возложил на себя вретище и пепел, и вышел на средину города и взывал с воплем великим и горьким: [истребляется народ ни в чем не повинный!]
\vs Est 4:2 И дошел до царских ворот [и остановился,] так как нельзя было входить в царские ворота во вретище [и с пеплом].
\vs Est 4:3 Равно и во всякой области и месте, куда \bibemph{только} доходило повеление царя и указ его, было большое сетование у Иудеев, и пост, и плач, и вопль; вретище и пепел служили постелью для многих.
\rsbpar\vs Est 4:4 И пришли служанки Есфири и евнухи ее и рассказали ей, и сильно встревожилась царица. И послала одежды, чтобы Мардохей надел их и снял с себя вретище свое. Но он не принял.
\vs Est 4:5 Тогда позвала Есфирь Гафаха, одного из евнухов царя, которого он приставил к ней, и послала его к Мардохею узнать: что это и отчего это?
\vs Est 4:6 И пошел Гафах к Мардохею на городскую площадь, которая пред царскими воротами.
\vs Est 4:7 И рассказал ему Мардохей обо всем, что с ним случилось, и об определенном числе серебра, которое обещал Аман отвесить в казну царскую за Иудеев, чтобы истребить их;
\vs Est 4:8 и вручил ему список с указа, обнародованного в Сузах, об истреблении их, чтобы показать Есфири и дать ей знать \bibemph{обо всем}; притом наказывал ей, чтобы она пошла к царю и молила его о помиловании и просила его за народ свой, [вспомнив дни смирения своего, когда она воспитывалась под рукою моею, потому что Аман, второй по царе, осудил нас на смерть, и чтобы призвала Господа и сказала о нас царю, да избавит нас от смерти].
\vs Est 4:9 И пришел Гафах и пересказал Есфири слова Мардохея.
\vs Est 4:10 И сказала Есфирь Гафаху и послала его \bibemph{сказать} Мардохею:
\vs Est 4:11 все служащие при царе и народы в областях царских знают, что всякому, и мужчине и женщине, кто войдет к царю во внутренний двор, не быв позван, один суд~--- смерть; только тот, к кому прострет царь свой золотой скипетр, останется жив. А я не звана к царю вот уже тридцать дней.
\vs Est 4:12 И пересказали Мардохею слова Есфири.
\vs Est 4:13 И сказал Мардохей в ответ Есфири: не думай, что ты \bibemph{одна} спасешься в доме царском из всех Иудеев.
\vs Est 4:14 Если ты промолчишь в это время, то свобода и избавление придет для Иудеев из другого места, а ты и дом отца твоего погибнете. И кто знает, не для такого ли времени ты и достигла достоинства царского?
\vs Est 4:15 И сказала Есфирь в ответ Мардохею:
\vs Est 4:16 пойди, собери всех Иудеев, находящихся в Сузах, и поститесь ради меня, и не ешьте и не пейте три дня, ни днем, ни ночью, и я с служанками моими буду также поститься и потом пойду к царю, хотя это против закона, и если погибнуть~--- погибну.
\rsbpar\vs Est 4:17 И пошел Мардохей и сделал, как приказала ему Есфирь. [И молился он Господу, воспоминая все дела Господни, и говорил: Господи, Господи, Царю, Вседержителю! Все в Твоей власти, и нет противящегося Тебе, когда Ты захочешь спасти Израиля; Ты сотворил небо и землю и все дивное в поднебесной; Ты~--- Господь всех, и нет \bibemph{такого}, кто воспротивился бы Тебе, Господу. Ты знаешь всё; Ты знаешь, Господи, что не для обиды и не по гордости и не по тщеславию я делал это, что не поклонялся тщеславному Аману, ибо я охотно стал бы лобызать следы ног его для спасения Израиля; но я делал это для того, чтобы не воздать славы человеку выше славы Божией и не поклоняться никому, кроме Тебя, Господа моего, и я не стану делать этого по гордости. И ныне, Господи Боже, Царю, Боже Авраамов, пощади народ Твой; ибо замышляют нам погибель и хотят истребить изначальное наследие Твое; не презри достояния Твоего, которое Ты избавил для Себя из земли Египетской; услышь молитву мою и умилосердись над наследием Твоим и обрати сетование наше в веселие, дабы мы, живя, воспевали имя Твое, Господи, и не погуби уст, прославляющих Тебя, Господи. И все Израильтяне взывали \bibemph{всеми} силами своими, потому что смерть их \bibemph{была} пред глазами их. И царица Есфирь прибегла к Господу, объятая смертною горестью, и, сняв одежды славы своей, облеклась в одежды скорби и сетования, и, вместо многоценных мастей, пеплом и прахом посыпала голову свою, и весьма изнурила тело свое, и всякое место, украшаемое в веселии ее, покрыла распущенными волосами своими, и молилась Господу Богу Израилеву, говоря: Господи мой! Ты один Царь наш; помоги мне, одинокой и не имеющей помощника, кроме Тебя; ибо беда моя близ меня. Я слышала, Господи, от отца моего, в родном колене моем, что Ты, Господи, избрал себе Израиля из всех народов и отцов наших из всех предков их в наследие вечное, и сделал для них то, о чем говорил им. И ныне мы согрешили пред Тобою, и предал Ты нас в руки врагов наших за то, что мы славили богов их: праведен Ты, Господи! А ныне они не удовольствовались горьким рабством нашим, но положили руки свои в руки идолов своих, чтобы ниспровергнуть заповедь уст Твоих, и истребить наследие Твое, и заградить уста воспевающих Тебя, и погасить славу \bibemph{храма} Твоего и жертвенника Твоего, и отверзть уста народов на прославление тщетных \bibemph{богов}, и царю плотскому величаться вовек. Не предай, Господи, скипетра Твоего \bibemph{богам} несуществующим, и пусть не радуются падению нашему, но обрати замысел их на них самих: наветника же против нас предай позору. Помяни, Господи, яви Себя нам во время скорби нашей и дай мне мужество. Царь богов и Владыка всякого начальства! даруй устам моим слово благоприятное пред этим львом и исполни сердце его ненавистью к преследующему нас, на погибель ему и единомышленникам его; нас же избавь рукою Твоею и помоги мне, одинокой и не имеющей помощника, кроме Тебя, Господи. Ты имеешь ведение всего и знаешь, что я ненавижу славу беззаконных и гнушаюсь ложа необрезанных и всякого иноплеменника; Ты знаешь необходимость мою, что я гнушаюсь знака гордости моей, который бывает на голове моей во дни появления моего, гнушаюсь его, как одежды, оскверненной кровью, и не ношу его во дни уединения моего. И не вкушала раба Твоя от трапезы Амана и не дорожила пиром царским, и не пила вина идоложертвенного, и не веселилась раба Твоя со дня перемены \bibemph{судьбы} моей доныне, кроме как о Тебе, Господи Боже Авраамов. Боже, имеющий силу над всеми! услышь голос безнадежных, и спаси нас от руки злоумышляющих, и избавь меня от страха моего.]
\vs Est 5:1 На третий день Есфирь [перестав молиться, сняла одежды сетования и] оделась по-царски, [и сделавшись великолепною, призывая всевидца Бога и Спасителя, взяла двух служанок, и на одну опиралась, как бы предавшись неге, а другая следовала \bibemph{за нею}, поддерживая одеяние ее. Она была прекрасна во цвете красоты своей, и лице ее радостно, как бы исполненное любви, но сердце ее было стеснено от страха]. И стала она на внутреннем дворе царского дома, перед домом царя; царь же сидел \bibemph{тогда} на царском престоле своем, в царском доме, прямо против входа в дом, [облеченный во все одеяние величия своего, весь в золоте и драгоценных камнях, и был весьма страшен]. Когда царь увидел царицу Есфирь, стоящую на дворе, она нашла милость в глазах его. [Обратив лице свое, пламеневшее славою, он взглянул с сильным гневом; и царица упала \bibemph{духом} и изменилась в лице своем от ослабления и склонилась на голову служанки, которая сопровождала ее. И изменил Бог дух царя на кротость, и поспешно встал он с престола своего и принял ее в объятия свои, пока она не пришла в себя. Потом он утешил ее ласковыми словами, сказав ей: что \bibemph{тебе}, Есфирь? Я~--- брат твой; ободрись, не умрешь, ибо наше владычество общее; подойди.]
\vs Est 5:2 И простер царь к Есфири золотой скипетр, который был в руке его, и подошла Есфирь и коснулась конца скипетра, [и положил \bibemph{царь} скипетр на шею ее и поцеловал ее и сказал: говори мне. И сказала она: я видела в тебе, господин, как бы Ангела Божия, и смутилось сердце мое от страха пред славою твоею, ибо дивен ты, господин, и лице твое исполнено благодати.~--- Но во время беседы она упала от ослабления; и царь смутился, и все слуги его утешали ее].
\vs Est 5:3 И сказал ей царь: что тебе, царица Есфирь, и какая просьба твоя? Даже до полуцарства будет дано тебе.
\vs Est 5:4 И сказала Есфирь: [ныне у меня день праздничный;] если царю благоугодно, пусть придет царь с Аманом сегодня на пир, который я приготовила ему.
\vs Est 5:5 И сказал царь: сходите скорее за Аманом, чтобы сделать по слову Есфири. И пришел царь с Аманом на пир, который приготовила Есфирь.
\vs Est 5:6 И сказал царь Есфири при питье вина: какое желание твое? оно будет удовлетворено; и какая просьба твоя? \bibemph{хотя бы} до полуцарства, она будет исполнена.
\vs Est 5:7 И отвечала Есфирь, и сказала: \bibemph{вот} мое желание и моя просьба:
\vs Est 5:8 если я нашла благоволение в очах царя, и если царю благоугодно удовлетворить желание мое и исполнить просьбу мою, то пусть царь с Аманом придет [еще завтра] на пир, который я приготовлю для них, и завтра я исполню слово царя.
\vs Est 5:9 И вышел Аман в тот день веселый и благодушный. Но когда увидел Аман Мардохея у ворот царских, и тот не встал и с места не тронулся пред ним, тогда исполнился Аман гневом на Мардохея.
\vs Est 5:10 Однако же скрепился Аман. А когда пришел в дом свой, то послал позвать друзей своих и Зерешь, жену свою.
\vs Est 5:11 И рассказывал им Аман о великом богатстве своем и о множестве сыновей своих и обо всем том, как возвеличил его царь и как вознес его над князьями и слугами царскими.
\vs Est 5:12 И сказал Аман: да и царица Есфирь никого не позвала с царем на пир, который она приготовила, кроме меня; так и на завтра я зван к ней с царем.
\vs Est 5:13 Но всего этого не довольно для меня, доколе я вижу Мардохея Иудеянина сидящим у ворот царских.
\vs Est 5:14 И сказала ему Зерешь, жена его, и все друзья его: пусть приготовят дерево вышиною в пятьдесят локтей, и утром скажи царю, чтобы повесили Мардохея на нем, и тогда весело иди на пир с царем. И понравилось это слово Аману, и он приготовил дерево.
\vs Est 6:1 В ту ночь Господь отнял сон от царя, и он велел [слуге] принести памятную книгу дневных записей; и читали их пред царем,
\vs Est 6:2 и найдено записанным \bibemph{там}, как донес Мардохей на Гавафу и Фарру, двух евнухов царских, оберегавших порог, которые замышляли наложить руку на царя Артаксеркса.
\vs Est 6:3 И сказал царь: какая дана почесть и отличие Мардохею за это? И сказали отроки царя, служившие при нем: ничего не сделано ему.
\vs Est 6:4 [Когда царь расспрашивал о благодеянии Мардохея, пришел на двор Аман,] и сказал царь: кто на дворе? Аман же пришел \bibemph{тогда} на внешний двор царского дома поговорить с царем, чтобы повесили Мардохея на дереве, которое он приготовил для него.
\vs Est 6:5 И сказали отроки царю: вот, Аман стоит на дворе. И сказал царь: пусть войдет.
\vs Est 6:6 И вошел Аман. И сказал ему царь: что сделать бы тому человеку, которого царь хочет отличить почестью? Аман подумал в сердце своем: кому \bibemph{другому} захочет царь оказать почесть, кроме меня?
\vs Est 6:7 И сказал Аман царю: тому человеку, которого царь хочет отличить почестью,
\vs Est 6:8 пусть принесут одеяние царское, в которое одевается царь, и \bibemph{приведут} коня, на котором ездит царь, возложат царский венец на голову его,
\vs Est 6:9 и пусть подадут одеяние и коня в руки одному из первых князей царских,~--- и облекут того человека, которого царь хочет отличить почестью, и выведут его на коне на городскую площадь, и провозгласят пред ним: так делается тому человеку, которого царь хочет отличить почестью!
\vs Est 6:10 И сказал царь Аману: [хорошо ты сказал;] тотчас же возьми одеяние и коня, как ты сказал, и сделай это Мардохею Иудеянину, сидящему у царских ворот; ничего не опусти из всего, что ты говорил.
\vs Est 6:11 И взял Аман одеяние и коня и облек Мардохея, и вывел его на коне на городскую площадь и провозгласил пред ним: так делается тому человеку, которого царь хочет отличить почестью!
\vs Est 6:12 И возвратился Мардохей к царским воротам. Аман же поспешил в дом свой, печальный и закрыв голову.
\vs Est 6:13 И пересказал Аман Зереши, жене своей, и всем друзьям своим все, что случилось с ним. И сказали ему мудрецы его и Зерешь, жена его: если из племени Иудеев Мардохей, из-за которого ты начал падать, то не пересилишь его, а наверно падешь пред ним, [ибо с ним Бог живый].
\vs Est 6:14 Они еще разговаривали с ним, \bibemph{как} пришли евнухи царя и стали торопить Амана идти на пир, который приготовила Есфирь.
\vs Est 7:1 И пришел царь с Аманом пировать у Есфири царицы.
\vs Est 7:2 И сказал царь Есфири также и в \bibemph{этот} второй день во время пира: какое желание твое, царица Есфирь? оно будет удовлетворено; и какая просьба твоя? \bibemph{хотя бы} до полуцарства, она будет исполнена.
\vs Est 7:3 И отвечала царица Есфирь и сказала: если я нашла благоволение в очах твоих, царь, и если царю благоугодно, то да будут дарованы мне жизнь моя, по желанию моему, и народ мой, по просьбе моей!
\vs Est 7:4 Ибо проданы мы, я и народ мой, на истребление, убиение и погибель. Если бы мы проданы были в рабы и рабыни, я молчала бы, хотя враг не вознаградил бы ущерба царя.
\vs Est 7:5 И отвечал царь Артаксеркс и сказал царице Есфири: кто это такой, и где тот, который отважился в сердце своем сделать так?
\vs Est 7:6 И сказала Есфирь: враг и неприятель~--- этот злобный Аман! И Аман затрепетал пред царем и царицею.
\vs Est 7:7 И царь встал во гневе своем с пира \bibemph{и пошел} в сад при дворце; Аман же остался умолять о жизни своей царицу Есфирь, ибо видел, что определена ему злая участь от царя.
\vs Est 7:8 Когда царь возвратился из сада при дворце в дом пира, Аман был припавшим к ложу, на котором находилась Есфирь. И сказал царь: даже и насиловать царицу \bibemph{хочет} в доме у меня! Слово вышло из уст царя,~--- и накрыли лице Аману.
\vs Est 7:9 И сказал Харбона, один из евнухов при царе: вот и дерево, которое приготовил Аман для Мардохея, говорившего доброе для царя, стоит у дома Амана, вышиною в пятьдесят локтей. И сказал царь: повесьте его на нем.
\vs Est 7:10 И повесили Амана на дереве, которое он приготовил для Мардохея. И гнев царя утих.
\vs Est 8:1 В тот день царь Артаксеркс отдал царице Есфири дом Амана, врага Иудеев; а Мардохей вошел пред лице царя, ибо Есфирь объявила, чт\acc{о} он для нее.
\vs Est 8:2 И снял царь перстень свой, который он отнял у Амана, и отдал его Мардохею; Есфирь же поставила Мардохея смотрителем над домом Амана.
\vs Est 8:3 И продолжала Есфирь говорить пред царем и пала к ногам его, и плакала и умоляла его отвратить злобу Амана Вугеянина и замысел его, который он замыслил против Иудеев.
\vs Est 8:4 И простер царь к Есфири золотой скипетр; и поднялась Есфирь, и стала пред лицем царя,
\vs Est 8:5 и сказала: если царю благоугодно, и если я нашла благоволение пред лицем его, и справедливо дело сие пред лицем царя, и нравлюсь я очам его, то пусть было бы написано, чтобы возвращены были письма по замыслу Амана, сына Амадафа, Вугеянина, писанные им об истреблении Иудеев во всех областях царя;
\vs Est 8:6 ибо, как я могу видеть бедствие, которое постигнет народ мой, и как я могу видеть погибель родных моих?
\vs Est 8:7 И сказал царь Артаксеркс царице Есфири и Мардохею Иудеянину: вот, я дом Амана отдал Есфири, и его самого повесили на дереве за то, что он налагал руку свою на Иудеев;
\vs Est 8:8 напишите и вы о Иудеях, что вам угодно, от имени царя и скрепите царским перстнем, ибо письма, написанного от имени царя и скрепленного перстнем царским, нельзя изменить.
\rsbpar\vs Est 8:9 И позваны были тогда царские писцы в третий месяц, то есть в месяц Сиван, в двадцать третий день его, и написано было все так, как приказал Мардохей, к Иудеям, и к сатрапам, и областеначальникам, и правителям областей от Индии до Ефиопии, ста двадцати семи областей, в каждую область письменами ее и к каждому народу на языке его, и к Иудеям письменами их и на языке их.
\vs Est 8:10 И написал он от имени царя Артаксеркса, и скрепил царским перстнем, и послал письма чрез гонцов на конях, на дромадерах и мулах царских,
\vs Est 8:11 о том, что царь позволяет Иудеям, находящимся во всяком городе, собраться и стать на защиту жизни своей, истребить, убить и погубить всех сильных в народе и в области, которые во вражде с ними, детей и жен, и имение их разграбить,
\vs Est 8:12 в один день по всем областям царя Артаксеркса, в тринадцатый день двенадцатого месяца, то есть месяца Адара. [Список с этого указа следующий: великий царь Артаксеркс начальствующим от Индии до Ефиопии над ста двадцатью семью областями и властителям, доброжелательствующим нам, радоваться. Многие, по чрезвычайной доброте благодетелей щедро награждаемые почестями, чрезмерно возгордились и не только подданным нашим ищут причинить зло, но, не могши насытить гордость, покушаются строить козни самим благодетелям своим, не только теряют чувство человеческой признательности, но, кичась надменностью безумных, преступно думают избежать суда всё и всегда видящего Бога. Но часто и многие, будучи облечены властью, чтоб устроять дела доверивших им друзей, своим убеждением делают их виновниками \bibemph{пролития} невинной крови и подвергают неисправимым бедствиям, хитросплетением коварной лжи обманывая непорочное благомыслие державных. \bibemph{Это} можно видеть не столько из древних историй, как мы сказали, сколько из дел, преступно совершаемых пред вами злобою недостойно властвующих. Посему нужно озаботиться на последующее время, чтобы нам устроить царство безмятежным для всех людей в мире, не допуская изменений, но представляющиеся дела обсуждая с надлежащей предусмотрительностью. Так Аман Амадафов, Македонянин, поистине чуждый персидской крови и весьма далекий от нашей благости, быв принят у нас гостем, удостоился благосклонности, которую мы имеем ко всякому народу, настолько, что был провозглашен нашим отцом и почитаем всеми, представляя второе лицо при царском престоле; но, не умерив гордости, замышлял лишить нас власти и души, а нашего спасителя и всегдашнего благодетеля Мардохея и непорочную общницу царства Есфирь, со всем народом их, домогался разнообразными коварными мерами погубить. Таким образом он думал сделать нас безлюдными, а державу Персидскую передать Македонянам. Мы же находим Иудеев, осужденных этим злодеем на истребление, не зловредными, а живущими по справедливейшим законам, сынами Вышнего, величайшего живаго Бога, даровавшего нам и предкам нашим царство в самом лучшем состоянии. Посему вы хорошо сделаете, не приводя в исполнение грамот, посланных Аманом Амадафовым; ибо он, совершивший это, при воротах Сузских повешен со всем домом, \bibemph{по воле} владычествующего всем Бога, воздавшего ему скоро достойный суд. Список же с этого указа выставив на всяком месте открыто, оставьте Иудеев пользоваться своими законами и содействуйте им, чтобы восстававшим на них во время скорби они могли отмстить в тринадцатый день двенадцатого месяца Адара, в самый тот день. Ибо владычествующий над всем Бог, вместо погибели избранного рода, устроил им такую радость. И вы, в числе именитых праздников ваших, проводите сей знаменитый день со всем весельем, дабы и ныне и после памятно было спасение для нас и для благорасположенных \bibemph{к нам} Персов и погубление строивших нам козни. Всякий город или область вообще, которая не исполнит сего, нещадно опустошится мечом и огнем и сделается не только необитаемою для людей, но и для зверей и птиц навсегда отвратительною.]
\vs Est 8:13 Список с сего указа отдать в каждую область, \bibemph{как} закон, объявляемый для всех народов, чтоб Иудеи готовы были к тому дню мстить врагам своим.
\vs Est 8:14 Гонцы, поехавшие верхом на быстрых конях царских, погнали скоро и поспешно, с царским повелением. Объявлен был указ и в Сузах, престольном городе.
\vs Est 8:15 И Мардохей вышел от царя в царском одеянии яхонтового и белого цвета и в большом золотом венце, и в мантии виссонной и пурпуровой. И город Сузы возвеселился и возрадовался.
\vs Est 8:16 А у Иудеев было \bibemph{тогда} освещение и радость, и веселье, и торжество.
\vs Est 8:17 И во всякой области и во всяком городе, во \bibemph{всяком} месте, куда \bibemph{только} доходило повеление царя и указ его, была радость у Иудеев и веселье, пиршество и праздничный день. И многие из народов страны сделались Иудеями, потому что напал на них страх пред Иудеями.
\vs Est 9:1 В двенадцатый месяц, то есть в месяц Адар, в тринадцатый день его, в который пришло время исполниться повелению царя и указу его, в тот день, когда надеялись неприятели Иудеев взять власть над ними, а вышло наоборот, что сами Иудеи взяли власть над врагами своими,~---
\vs Est 9:2 собрались Иудеи в городах своих по всем областям царя Артаксеркса, чтобы наложить руку на зложелателей своих; и никто не мог устоять пред лицем их, потому что страх пред ними напал на все народы.
\vs Est 9:3 И все князья в областях и сатрапы, и областеначальники, и исполнители дел царских поддерживали Иудеев, потому что напал на них страх пред Мардохеем.
\vs Est 9:4 Ибо велик был Мардохей в доме у царя, и слава о нем ходила по всем областям, так как сей человек, Мардохей, поднимался выше и выше.
\rsbpar\vs Est 9:5 И избивали Иудеи всех врагов своих, побивая мечом, умерщвляя и истребляя, и поступали с неприятелями своими по своей воле.
\vs Est 9:6 В Сузах, городе престольном, умертвили Иудеи и погубили пятьсот человек;
\vs Est 9:7 и Паршандафу и Далфона и Асфафу,
\vs Est 9:8 и Порафу и Адалью и Аридафу,
\vs Est 9:9 и Пармашфу и Арисая и Аридая и Ваиезафу,~---
\vs Est 9:10 десятерых сыновей Амана, сына Амадафа, врага Иудеев, умертвили они, а на грабеж не простерли руки своей.
\vs Est 9:11 В тот же день донесли царю о числе умерщвленных в Сузах, престольном городе.
\vs Est 9:12 И сказал царь царице Есфири: в Сузах, городе престольном, умертвили Иудеи и погубили пятьсот человек и десятерых сыновей Амана; что же сделали они в прочих областях царя? Какое желание твое? и оно будет удовлетворено. И какая еще просьба твоя? она будет исполнена.
\vs Est 9:13 И сказала Есфирь: если царю благоугодно, то пусть бы позволено было Иудеям, которые в Сузах, делать то же и завтра, что сегодня, и десятерых сыновей Амановых пусть бы повесили на дереве.
\vs Est 9:14 И приказал царь сделать так; и дан \bibemph{на это} указ в Сузах, и десятерых сыновей Амановых повесили.
\rsbpar\vs Est 9:15 И собрались Иудеи, которые в Сузах, также и в четырнадцатый день месяца Адара и умертвили в Сузах триста человек, а на грабеж не простерли руки своей.
\vs Est 9:16 И прочие Иудеи, находившиеся в царских областях, собрались, чтобы стать на защиту жизни своей и быть покойными от врагов своих, и умертвили из неприятелей своих семьдесят пять тысяч, а на грабеж не простерли руки своей.
\vs Est 9:17 \bibemph{Это было} в тринадцатый день месяца Адара; а в четырнадцатый день сего же месяца они успокоились и сделали его днем пиршества и веселья.
\vs Est 9:18 Иудеи же, которые в Сузах, собирались в тринадцатый день его и в четырнадцатый день его, а в пятнадцатый день его успокоились и сделали его днем пиршества и веселья.
\vs Est 9:19 Поэтому Иудеи сельские, живущие в селениях открытых, проводят четырнадцатый день месяца Адара в веселье и пиршестве, как день праздничный, посылая подарки друг ко другу; [живущие же в митрополиях и пятнадцатый день Адара проводят в добром веселье, посылая подарки ближним].
\rsbpar\vs Est 9:20 И описал Мардохей эти происшествия и послал письма ко всем Иудеям, которые в областях царя Артаксеркса, к близким и к дальним,
\vs Est 9:21 \bibemph{о том}, чтобы они установили каждогодно празднование у себя четырнадцатого дня месяца Адара и пятнадцатого дня его,
\vs Est 9:22 как таких дней, в которые Иудеи сделались покойны от врагов своих, и \bibemph{как} такого месяца, в который превратилась у них печаль в радость, и сетование~--- в день праздничный,~--- чтобы сделали их днями пиршества и веселья, посылая подарки друг другу и подаяния бедным.
\vs Est 9:23 И приняли Иудеи то, что уже сами начали делать, и о чем Мардохей написал к ним,
\vs Est 9:24 как Аман, сын Амадафа, Вугеянин, враг всех Иудеев, думал погубить Иудеев и бросал пур, \bibemph{жребий}, об истреблении и погублении их,
\vs Est 9:25 и как Есфирь дошла до царя, и как царь приказал новым письмом, чтобы злой замысл Амана, который он задумал на Иудеев, обратился на голову его, и чтобы повесили его и сыновей его на дереве.
\vs Est 9:26 Потому и назвали эти дни Пурим, от имени: пур [\bibemph{жребий}, ибо на языке их жребии называются пурим]. Поэтому, согласно со всеми словами сего письма и с тем, что сами видели и до чего доходило у них,
\vs Est 9:27 постановили Иудеи и приняли на себя и на детей своих и на всех, присоединяющихся к ним, неотменно, чтобы праздновать эти два дня, по предписанному о них и в свое для них время, каждый год;
\vs Est 9:28 и чтобы дни эти были памятны и празднуемы во все роды в каждом племени, в каждой области и в каждом городе; и чтобы дни эти Пурим не отменялись у Иудеев, и память о них не исчезла у детей их.
\rsbpar\vs Est 9:29 Написала также царица Есфирь, дочь Абихаила, и Мардохей Иудеянин, со всею настойчивостью, чтобы исполняли это новое письмо о Пуриме;
\vs Est 9:30 и послали письма ко всем Иудеям в сто двадцать семь областей царства Артаксерксова со словами мира и правды,
\vs Est 9:31 чтобы они твердо наблюдали эти дни Пурим в свое время, какое уставил о них Мардохей Иудеянин и царица Есфирь, и как они сами назначали их для себя и для детей своих в дни пощения и воплей.
\vs Est 9:32 Так повеление Есфири подтвердило это слово о Пуриме, и оно вписано в книгу.
\vs Est 10:1 Потом наложил царь Артаксеркс подать на землю и на острова морские.
\vs Est 10:2 Впрочем, все дела силы его и могущества его и обстоятельное показание о величии Мардохея, которым возвеличил его царь, записаны в книге дневных записей царей Мидийских и Персидских,
\vs Est 10:3 \bibemph{равно как и то}, что Мардохей Иудеянин \bibemph{был} вторым по царе Артаксерксе и великим у Иудеев и любимым у множества братьев своих, \bibemph{ибо} искал добра народу своему и говорил во благо всего племени своего. [И сказал Мардохей: от Бога было это, ибо я вспомнил сон, который я видел о сих событиях; не осталось в нем ничего неисполнившимся. Малый источник сделался рекою, и был свет и солнце и множество воды: эта река есть Есфирь, которую взял себе в жену царь и сделал царицею. А два змея~--- это я и Аман; народы~--- это собравшиеся истребить имя Иудеев; а народ мой~--- это Израильтяне, воззвавшие к Богу и спасенные. И спас Господь народ Свой, и избавил нас Господь от всех сих зол, и совершил Бог знамения и чудеса великие, какие не бывали между язычниками. Так устроил Бог два жребия: один для народа Божия, а другой для всех язычников, и вышли эти два жребия в час и время и в день суда пред Богом и всеми язычниками. И вспомнил Господь о народе Своем и оправдал наследие Свое. И будут праздноваться эти дни месяца Адара, в четырнадцатый и пятнадцатый день этого месяца, с торжеством и радостью и весельем пред Богом, в роды вечные, в народе Его Израиле. В четвертый год царствования Птоломея и Клеопатры Досифей, который, говорят, был священником и левитом, и Птоломей, сын его, принесли \bibemph{в Александрию} это послание о Пуриме, которое, говорят, истолковал Лисимах, \bibemph{сын} Птоломея, бывший в Иерусалиме.]

\bibbookdescr{Job}{
  inline={\LARGE Книга\\\Huge Иова},
  toc={Иов},
  bookmark={Иов},
  header={Иов},
  %headerleft={},
  %headerright={},
  abbr={Иов}
}
\vs Job 1:1 Был человек в земле Уц, имя его Иов; и был человек этот непорочен, справедлив и богобоязнен и удалялся от зла.
\vs Job 1:2 И родились у него семь сыновей и три дочери.
\vs Job 1:3 Имения у него было: семь тысяч мелкого скота, три тысячи верблюдов, пятьсот пар волов и пятьсот ослиц и весьма много прислуги; и был человек этот знаменитее всех сынов Востока.
\vs Job 1:4 Сыновья его сходились, делая пиры каждый в своем доме в свой день, и посылали и приглашали трех сестер своих есть и пить с ними.
\vs Job 1:5 Когда круг пиршественных дней совершался, Иов посылал \bibemph{за ними} и освящал их и, вставая рано утром, возносил всесожжения по числу всех их [и одного тельца за грех о душах их]. Ибо говорил Иов: может быть, сыновья мои согрешили и похулили Бога в сердце своем. Так делал Иов во все \bibemph{такие} дни.
\rsbpar\vs Job 1:6 И был день, когда пришли сыны Божии предстать пред Господа; между ними пришел и сатана.
\vs Job 1:7 И сказал Господь сатане: откуда ты пришел? И отвечал сатана Господу и сказал: я ходил по земле и обошел ее.
\vs Job 1:8 И сказал Господь сатане: обратил ли ты внимание твое на раба Моего Иова? ибо нет такого, как он, на земле: человек непорочный, справедливый, богобоязненный и удаляющийся от зла.
\vs Job 1:9 И отвечал сатана Господу и сказал: разве даром богобоязнен Иов?
\vs Job 1:10 Не Ты ли кругом оградил его и дом его и все, что у него? Дело рук его Ты благословил, и стада его распространяются по земле;
\vs Job 1:11 но простри руку Твою и коснись всего, что у него,~--- благословит ли он Тебя?
\vs Job 1:12 И сказал Господь сатане: вот, все, что у него, в руке твоей; только на него не простирай руки твоей. И отошел сатана от лица Господня.
\rsbpar\vs Job 1:13 И был день, когда сыновья его и дочери его ели и вино пили в доме первородного брата своего.
\vs Job 1:14 И \bibemph{вот}, приходит вестник к Иову и говорит:
\vs Job 1:15 волы орали, и ослицы паслись подле них, как напали Савеяне и взяли их, а отроков поразили острием меча; и спасся только я один, чтобы возвестить тебе.
\vs Job 1:16 Еще он говорил, как приходит другой и сказывает: огонь Божий упал с неба и опалил овец и отроков и пожрал их; и спасся только я один, чтобы возвестить тебе.
\vs Job 1:17 Еще он говорил, как приходит другой и сказывает: Халдеи расположились тремя отрядами и бросились на верблюдов и взяли их, а отроков поразили острием меча; и спасся только я один, чтобы возвестить тебе.
\vs Job 1:18 Еще этот говорил, приходит другой и сказывает: сыновья твои и дочери твои ели и вино пили в доме первородного брата своего;
\vs Job 1:19 и вот, большой ветер пришел от пустыни и охватил четыре угла дома, и дом упал на отроков, и они умерли; и спасся только я один, чтобы возвестить тебе.
\rsbpar\vs Job 1:20 Тогда Иов встал и разодрал верхнюю одежду свою, остриг голову свою и пал на землю и поклонился
\vs Job 1:21 и сказал: наг я вышел из чрева матери моей, наг и возвращусь. Господь дал, Господь и взял; [как угодно было Господу, так и сделалось;] да будет имя Господне благословенно!
\vs Job 1:22 Во всем этом не согрешил Иов и не произнес ничего неразумного о Боге.
\vs Job 2:1 Был день, когда пришли сыны Божии предстать пред Господа; между ними пришел и сатана предстать пред Господа.
\vs Job 2:2 И сказал Господь сатане: откуда ты пришел? И отвечал сатана Господу и сказал: я ходил по земле и обошел ее.
\vs Job 2:3 И сказал Господь сатане: обратил ли ты внимание твое на раба Моего Иова? ибо нет такого, как он, на земле: человек непорочный, справедливый, богобоязненный и удаляющийся от зла, и доселе тверд в своей непорочности; а ты возбуждал Меня против него, чтобы погубить его безвинно.
\vs Job 2:4 И отвечал сатана Господу и сказал: кожу за кожу, а за жизнь свою отдаст человек все, что есть у него;
\vs Job 2:5 но простри руку Твою и коснись кости его и плоти его,~--- благословит ли он Тебя?
\vs Job 2:6 И сказал Господь сатане: вот, он в руке твоей, только душу его сбереги.
\rsbpar\vs Job 2:7 И отошел сатана от лица Господня и поразил Иова проказою лютою от подошвы ноги его по самое темя его.
\vs Job 2:8 И взял он себе черепицу, чтобы скоблить себя ею, и сел в пепел [вне селения].
\vs Job 2:9 И сказала ему жена его: ты все еще тверд в непорочности твоей! похули Бога и умри.\fns{Этот стих по переводу 70-ти: По многом времени сказала ему жена его: доколе ты будешь терпеть? Вот, подожду еще немного в надежде спасения моего. Ибо погибли с земли память твоя, сыновья и дочери, болезни чрева моего и труды, которыми напрасно трудилась. Сам ты сидишь в смраде червей, проводя ночь без покрова, а я скитаюсь и служу, перехожу с места на место, из дома в дом, ожидая, когда зайдет солнце, чтобы успокоиться от трудов моих и болезней, которые ныне удручают меня. Но скажи некое слово к Богу и умри.}
\vs Job 2:10 Но он сказал ей: ты говоришь как одна из безумных: неужели доброе мы будем принимать от Бога, а злого не будем принимать? Во всем этом не согрешил Иов устами своими.
\vs Job 2:11 И услышали трое друзей Иова о всех этих несчастьях, постигших его, и пошли каждый из своего места: Елифаз Феманитянин, Вилдад Савхеянин и Софар Наамитянин, и сошлись, чтобы идти вместе сетовать с ним и утешать его.
\vs Job 2:12 И подняв глаза свои издали, они не узнали его; и возвысили голос свой и зарыдали; и разодрал каждый верхнюю одежду свою, и бросали пыль над головами своими к небу.
\vs Job 2:13 И сидели с ним на земле семь дней и семь ночей; и никто не говорил ему ни слова, ибо видели, что страдание его весьма велико.
\vs Job 3:1 После того открыл Иов уста свои и проклял день свой.
\vs Job 3:2 И начал Иов и сказал:
\vs Job 3:3 погибни день, в который я родился, и ночь, в которую сказано: зачался человек!
\vs Job 3:4 День тот да будет тьмою; да не взыщет его Бог свыше, и да не воссияет над ним свет!
\vs Job 3:5 Да омрачит его тьма и тень смертная, да обложит его туча, да страшатся его, как палящего зноя!
\vs Job 3:6 Ночь та,~--- да обладает ею мрак, да не сочтется она в днях года, да не войдет в число месяцев!
\vs Job 3:7 О! ночь та~--- да будет она безлюдна; да не войдет в нее веселье!
\vs Job 3:8 Да проклянут ее проклинающие день, способные разбудить левиафана!
\vs Job 3:9 Да померкнут звезды рассвета ее: пусть ждет она света, и он не приходит, и да не увидит она ресниц денницы
\vs Job 3:10 за то, что не затворила дверей чрева \bibemph{матери} моей и не сокрыла горести от очей моих!
\vs Job 3:11 Для чего не умер я, выходя из утробы, и не скончался, когда вышел из чрева?
\vs Job 3:12 Зачем приняли меня колени? зачем было мне сосать сосцы?
\vs Job 3:13 Теперь бы лежал я и почивал; спал бы, и мне было бы покойно
\vs Job 3:14 с царями и советниками земли, которые застраивали для себя пустыни,
\vs Job 3:15 или с князьями, у которых было золото, и которые наполняли домы свои серебром;
\vs Job 3:16 или, как выкидыш сокрытый, я не существовал бы, как младенцы, не увидевшие света.
\vs Job 3:17 Там беззаконные перестают наводить страх, и там отдыхают истощившиеся в силах.
\vs Job 3:18 Там узники вместе наслаждаются покоем и не слышат криков приставника.
\vs Job 3:19 Малый и великий там равны, и раб свободен от господина своего.
\vs Job 3:20 На что дан страдальцу свет, и жизнь огорченным душею,
\vs Job 3:21 которые ждут смерти, и нет ее, которые вырыли бы ее охотнее, нежели клад,
\vs Job 3:22 обрадовались бы до восторга, восхитились бы, что нашли гроб?
\vs Job 3:23 \bibemph{На что дан свет} человеку, которого путь закрыт, и которого Бог окружил мраком?
\vs Job 3:24 Вздохи мои предупреждают хлеб мой, и стоны мои льются, как вода,
\vs Job 3:25 ибо ужасное, чего я ужасался, то и постигло меня; и чего я боялся, то и пришло ко мне.
\vs Job 3:26 Нет мне мира, нет покоя, нет отрады: постигло несчастье.
\vs Job 4:1 И отвечал Елифаз Феманитянин и сказал:
\vs Job 4:2 \bibemph{если} попытаемся мы \bibemph{сказать} к тебе слово,~--- не тяжело ли будет тебе? Впрочем кто может возбранить слову!
\vs Job 4:3 Вот, ты наставлял многих и опустившиеся руки поддерживал,
\vs Job 4:4 падающего восставляли слова твои, и гнущиеся колени ты укреплял.
\vs Job 4:5 А теперь дошло до тебя, и ты изнемог; коснулось тебя, и ты упал духом.
\vs Job 4:6 Богобоязненность твоя не должна ли быть твоею надеждою, и непорочность путей твоих~--- упованием твоим?
\vs Job 4:7 Вспомни же, погибал ли кто невинный, и где праведные бывали искореняемы?
\vs Job 4:8 Как я видал, то оравшие нечестие и сеявшие зло пожинают его;
\vs Job 4:9 от дуновения Божия погибают и от духа гнева Его исчезают.
\vs Job 4:10 Рев льва и голос рыкающего \bibemph{умолкает}, и зубы скимнов сокрушаются;
\vs Job 4:11 могучий лев погибает без добычи, и дети львицы рассеиваются.
\vs Job 4:12 И вот, ко мне тайно принеслось слово, и ухо мое приняло нечто от него.
\vs Job 4:13 Среди размышлений о ночных видениях, когда сон находит на людей,
\vs Job 4:14 объял меня ужас и трепет и потряс все кости мои.
\vs Job 4:15 И дух прошел надо мною; дыбом стали волосы на мне.
\vs Job 4:16 Он стал,~--- но я не распознал вида его,~--- только облик был пред глазами моими; тихое веяние,~--- и я слышу голос:
\vs Job 4:17 человек праведнее ли Бога? и муж чище ли Творца своего?
\vs Job 4:18 Вот, Он и слугам Своим не доверяет и в Ангелах Своих усматривает недостатки:
\vs Job 4:19 тем более~--- в обитающих в храминах из брения, которых основание прах, которые истребляются скорее моли.
\vs Job 4:20 Между утром и вечером они распадаются; не увидишь, как они вовсе исчезнут.
\vs Job 4:21 Не погибают ли с ними и достоинства их? Они умирают, не достигнув мудрости.
\vs Job 5:1 Взывай, если есть отвечающий тебе. И к кому из святых обратишься ты?
\vs Job 5:2 Так, глупца убивает гневливость, и несмысленного губит раздражительность.
\vs Job 5:3 Видел я, как глупец укореняется, и тотчас проклял дом его.
\vs Job 5:4 Дети его далеки от счастья, их будут бить у ворот, и не будет заступника.
\vs Job 5:5 Жатву его съест голодный и из-за терна возьмет ее, и жаждущие поглотят имущество его.
\vs Job 5:6 Так, не из праха выходит горе, и не из земли вырастает беда;
\vs Job 5:7 но человек рождается на страдание, \bibemph{как} искры, чтобы устремляться вверх.
\vs Job 5:8 Но я к Богу обратился бы, предал бы дело мое Богу,
\vs Job 5:9 Который творит дела великие и неисследимые, чудные без числа,
\vs Job 5:10 дает дождь на лице земли и посылает воды на лице полей;
\vs Job 5:11 униженных поставляет на высоту, и сетующие возносятся во спасение.
\vs Job 5:12 Он разрушает замыслы коварных, и руки их не довершают предприятия.
\vs Job 5:13 Он уловляет мудрецов их же лукавством, и совет хитрых становится тщетным:
\vs Job 5:14 днем они встречают тьму и в полдень ходят ощупью, как ночью.
\vs Job 5:15 Он спасает бедного от меча, от уст их и от руки сильного.
\vs Job 5:16 И есть несчастному надежда, и неправда затворяет уста свои.
\vs Job 5:17 Блажен человек, которого вразумляет Бог, и потому наказания Вседержителева не отвергай,
\vs Job 5:18 ибо Он причиняет раны и Сам обвязывает их; Он поражает, и Его же руки врачуют.
\vs Job 5:19 В шести бедах спасет тебя, и в седьмой не коснется тебя зло.
\vs Job 5:20 Во время голода избавит тебя от смерти, и на войне~--- от руки меча.
\vs Job 5:21 От бича языка укроешь себя и не убоишься опустошения, когда оно придет.
\vs Job 5:22 Опустошению и голоду посмеешься и зверей земли не убоишься,
\vs Job 5:23 ибо с камнями полевыми у тебя союз, и звери полевые в мире с тобою.
\vs Job 5:24 И узн\acc{а}ешь, что шатер твой в безопасности, и будешь смотреть за домом твоим, и не согрешишь.
\vs Job 5:25 И увидишь, что семя твое многочисленно, и отрасли твои, как трава на земле.
\vs Job 5:26 Войдешь во гроб в зрелости, как укладываются снопы пшеницы в свое время.
\vs Job 5:27 Вот, что мы дознали; так оно и есть; выслушай это и заметь для себя.
\vs Job 6:1 И отвечал Иов и сказал:
\vs Job 6:2 о, если бы верно взвешены были вопли мои, и вместе с ними положили на весы страдание мое!
\vs Job 6:3 Оно верно перетянуло бы песок морей! Оттого слова мои неистовы.
\vs Job 6:4 Ибо стрелы Вседержителя во мне; яд их пьет дух мой; ужасы Божии ополчились против меня.
\vs Job 6:5 Ревет ли дикий осел на траве? мычит ли бык у месива своего?
\vs Job 6:6 Едят ли безвкусное без соли, и есть ли вкус в яичном белке?
\vs Job 6:7 До чего не хотела коснуться душа моя, то составляет отвратительную пищу мою.
\vs Job 6:8 О, когда бы сбылось желание мое и чаяние мое исполнил Бог!
\vs Job 6:9 О, если бы благоволил Бог сокрушить меня, простер руку Свою и сразил меня!
\vs Job 6:10 Это было бы еще отрадою мне, и я крепился бы в моей беспощадной болезни, ибо я не отвергся изречений Святаго.
\vs Job 6:11 Что за сила у меня, чтобы надеяться мне? и какой конец, чтобы длить мне жизнь мою?
\vs Job 6:12 Твердость ли камней твердость моя? и медь ли плоть моя?
\vs Job 6:13 Есть ли во мне помощь для меня, и есть ли для меня какая опора?
\vs Job 6:14 К страждущему должно быть сожаление от друга его, если только он не оставил страха к Вседержителю.
\vs Job 6:15 Но братья мои неверны, как поток, как быстро текущие ручьи,
\vs Job 6:16 которые черны от льда и в которых скрывается снег.
\vs Job 6:17 Когда становится тепло, они умаляются, а во время жары исчезают с мест своих.
\vs Job 6:18 Уклоняют они направление путей своих, заходят в пустыню и теряются;
\vs Job 6:19 смотрят на них дороги Фемайские, надеются на них пути Савейские,
\vs Job 6:20 но остаются пристыженными в своей надежде; приходят туда и от стыда краснеют.
\vs Job 6:21 Так и вы теперь ничто: увидели страшное и испугались.
\vs Job 6:22 Говорил ли я: дайте мне, или от достатка вашего заплатите за меня;
\vs Job 6:23 и избавьте меня от руки врага, и от руки мучителей выкупите меня?
\vs Job 6:24 Науч\acc{и}те меня, и я замолчу; укажите, в чем я погрешил.
\vs Job 6:25 Как сильны слова правды! Но что доказывают обличения ваши?
\vs Job 6:26 Вы придумываете речи для обличения? На ветер пускаете слова ваши.
\vs Job 6:27 Вы нападаете на сироту и роете яму другу вашему.
\vs Job 6:28 Но прошу вас, взгляните на меня; буду ли я говорить ложь пред лицем вашим?
\vs Job 6:29 Пересмотрите, есть ли неправда? пересмотрите,~--- правда моя.
\vs Job 6:30 Есть ли на языке моем неправда? Неужели гортань моя не может различить горечи?
\vs Job 7:1 Не определено ли человеку время на земле, и дни его не то же ли, что дни наемника?
\vs Job 7:2 Как раб жаждет тени, и как наемник ждет окончания работы своей,
\vs Job 7:3 так я получил в удел месяцы суетные, и ночи горестные отчислены мне.
\vs Job 7:4 Когда ложусь, то говорю: <<когда-то встану?>>, а вечер длится, и я ворочаюсь досыта до самого рассвета.
\vs Job 7:5 Тело мое одето червями и пыльными струпами; кожа моя лопается и гноится.
\vs Job 7:6 Дни мои бегут скорее челнока и кончаются без надежды.
\vs Job 7:7 Вспомни, что жизнь моя дуновение, что око мое не возвратится видеть доброе.
\vs Job 7:8 Не увидит меня око видевшего меня; очи Твои на меня,~--- и нет меня.
\vs Job 7:9 Редеет облако и уходит; так нисшедший в преисподнюю не выйдет,
\vs Job 7:10 не возвратится более в дом свой, и место его не будет уже знать его.
\vs Job 7:11 Не буду же я удерживать уст моих; буду говорить в стеснении духа моего; буду жаловаться в горести души моей.
\vs Job 7:12 Разве я море или морское чудовище, что Ты поставил надо мною стражу?
\vs Job 7:13 Когда подумаю: утешит меня постель моя, унесет горесть мою ложе мое,
\vs Job 7:14 Ты страшишь меня снами и видениями пугаешь меня;
\vs Job 7:15 и душа моя желает лучше прекращения дыхания, лучше смерти, нежели \bibemph{сбережения} костей моих.
\vs Job 7:16 Опротивела мне жизнь. Не вечно жить мне. Отступи от меня, ибо дни мои суета.
\vs Job 7:17 Что такое человек, что Ты столько ценишь его и обращаешь на него внимание Твое,
\vs Job 7:18 посещаешь его каждое утро, каждое мгновение испытываешь его?
\vs Job 7:19 Доколе же Ты не оставишь, доколе не отойдешь от меня, доколе не дашь мне проглотить слюну мою?
\vs Job 7:20 Если я согрешил, то что я сделаю Тебе, страж человеков! Зачем Ты поставил меня противником Себе, так что я стал самому себе в тягость?
\vs Job 7:21 И зачем бы не простить мне греха и не снять с меня беззакония моего? ибо, вот, я лягу в прахе; завтра поищешь меня, и меня нет.
\vs Job 8:1 И отвечал Вилдад Савхеянин и сказал:
\vs Job 8:2 долго ли ты будешь говорить так?~--- слов\acc{а} уст твоих бурный ветер!
\vs Job 8:3 Неужели Бог извращает суд, и Вседержитель превращает правду?
\vs Job 8:4 Если сыновья твои согрешили пред Ним, то Он и предал их в руку беззакония их.
\vs Job 8:5 Если же ты взыщешь Бога и помолишься Вседержителю,
\vs Job 8:6 и если ты чист и прав, то Он ныне же встанет над тобою и умиротворит жилище правды твоей.
\vs Job 8:7 И если вначале у тебя было мало, то впоследствии будет весьма много.
\vs Job 8:8 Ибо спроси у прежних родов и вникни в наблюдения отцов их;
\vs Job 8:9 а мы~--- вчерашние и ничего не знаем, потому что наши дни на земле тень.
\vs Job 8:10 Вот, они научат тебя, скажут тебе и от сердца своего произнесут слова:
\vs Job 8:11 поднимается ли тростник без влаги? растет ли камыш без воды?
\vs Job 8:12 Еще он в свежести своей и не срезан, а прежде всякой травы засыхает.
\vs Job 8:13 Таковы пути всех забывающих Бога, и надежда лицемера погибнет;
\vs Job 8:14 упование его подсечено, и уверенность его~--- дом паука.
\vs Job 8:15 Обопрется о дом свой и не устоит; ухватится за него и не удержится.
\vs Job 8:16 Зеленеет он пред солнцем, за сад простираются ветви его;
\vs Job 8:17 в кучу \bibemph{камней} вплетаются корни его, между камнями врезываются.
\vs Job 8:18 Но когда вырвут его с места его, оно откажется от него: <<я не видало тебя!>>
\vs Job 8:19 Вот радость пути его! а из земли вырастают другие.
\vs Job 8:20 Видишь, Бог не отвергает непорочного и не поддерживает рук\acc{и} злодеев.
\vs Job 8:21 Он еще наполнит смехом уста твои и губы твои радостным восклицанием.
\vs Job 8:22 Ненавидящие тебя облекутся в стыд, и шатра нечестивых не станет.
\vs Job 9:1 И отвечал Иов и сказал:
\vs Job 9:2 правда! знаю, что так; но как оправдается человек пред Богом?
\vs Job 9:3 Если захочет вступить в прение с Ним, то не ответит Ему ни на одно из тысячи.
\vs Job 9:4 Премудр сердцем и могущ силою; кто восставал против Него и оставался в покое?
\vs Job 9:5 Он передвигает горы, и не узна\acc{ю}т их: Он превращает их в гневе Своем;
\vs Job 9:6 сдвигает землю с места ее, и столбы ее дрожат;
\vs Job 9:7 скажет солнцу,~--- и не взойдет, и на звезды налагает печать.
\vs Job 9:8 Он один распростирает небеса и ходит по высотам моря;
\vs Job 9:9 сотворил Ас, Кесиль и Хима\fns{Созвездия, соответствующие нынешним названиям: Медведицы, Ориона и Плеяд.} и тайники юга;
\vs Job 9:10 делает великое, неисследимое и чудное без числа!
\vs Job 9:11 Вот, Он пройдет предо мною, и не увижу Его; пронесется, и не замечу Его.
\vs Job 9:12 Возьмет, и кто возбранит Ему? кто скажет Ему: что Ты делаешь?
\vs Job 9:13 Бог не отвратит гнева Своего; пред Ним падут поборники гордыни.
\vs Job 9:14 Тем более могу ли я отвечать Ему и приискивать себе слова пред Ним?
\vs Job 9:15 Хотя бы я и прав был, но не буду отвечать, а буду умолять Судию моего.
\vs Job 9:16 Если бы я воззвал, и Он ответил мне,~--- я не поверил бы, что голос мой услышал Тот,
\vs Job 9:17 Кто в вихре разит меня и умножает безвинно мои раны,
\vs Job 9:18 не дает мне перевести духа, но пресыщает меня горестями.
\vs Job 9:19 Если \bibemph{действовать} силою, то Он могуществен; если судом, кто сведет меня с Ним?
\vs Job 9:20 Если я буду оправдываться, то мои же уста обвинят меня; \bibemph{если} я невинен, то Он призн\acc{а}ет меня виновным.
\vs Job 9:21 Невинен я; не хочу знать души моей, презираю жизнь мою.
\vs Job 9:22 Все одно; поэтому я сказал, что Он губит и непорочного и виновного.
\vs Job 9:23 Если этого поражает Он бичом вдруг, то пытке невинных посмевается.
\vs Job 9:24 Земля отдана в руки нечестивых; лица судей ее Он закрывает. Если не Он, то кто же?
\vs Job 9:25 Дни мои быстрее гонца,~--- бегут, не видят добра,
\vs Job 9:26 несутся, как легкие ладьи, как орел стремится на добычу.
\vs Job 9:27 Если сказать мне: забуду я жалобы мои, отложу мрачный вид свой и ободрюсь;
\vs Job 9:28 то трепещу всех страданий моих, зная, что Ты не объявишь меня невинным.
\vs Job 9:29 Если же я виновен, то для чего напрасно томлюсь?
\vs Job 9:30 Хотя бы я омылся и снежною водою и совершенно очистил руки мои,
\vs Job 9:31 то и тогда Ты погрузишь меня в грязь, и возгнушаются мною одежды мои.
\vs Job 9:32 Ибо Он не человек, как я, чтоб я мог отвечать Ему и идти вместе с Ним на суд!
\vs Job 9:33 Нет между нами посредника, который положил бы руку свою на обоих нас.
\vs Job 9:34 Да отстранит Он от меня жезл Свой, и страх Его да не ужасает меня,~---
\vs Job 9:35 и тогда я буду говорить и не убоюсь Его, ибо я не таков сам в себе.
\vs Job 10:1 Опротивела душе моей жизнь моя; предамся печали моей; буду говорить в горести души моей.
\vs Job 10:2 Скажу Богу: не обвиняй меня; объяви мне, за что Ты со мною борешься?
\vs Job 10:3 Хорошо ли для Тебя, что Ты угнетаешь, что презираешь дело рук Твоих, а на совет нечестивых посылаешь свет?
\vs Job 10:4 Разве у Тебя плотские очи, и Ты смотришь, как смотрит человек?
\vs Job 10:5 Разве дни Твои, как дни человека, или лета Твои, как дни мужа,
\vs Job 10:6 что Ты ищешь порока во мне и допытываешься греха во мне,
\vs Job 10:7 хотя знаешь, что я не беззаконник, и что некому избавить меня от руки Твоей?
\vs Job 10:8 Твои руки трудились надо мною и образовали всего меня кругом,~--- и Ты губишь меня?
\vs Job 10:9 Вспомни, что Ты, как глину, обделал меня, и в прах обращаешь меня?
\vs Job 10:10 Не Ты ли вылил меня, как молоко, и, как творог, сгустил меня,
\vs Job 10:11 кожею и плотью одел меня, костями и жилами скрепил меня,
\vs Job 10:12 жизнь и милость даровал мне, и попечение Твое хранило дух мой?
\vs Job 10:13 Но и то скрывал Ты в сердце Своем,~--- знаю, что это было у Тебя,~---
\vs Job 10:14 что если я согрешу, Ты заметишь и не оставишь греха моего без наказания.
\vs Job 10:15 Если я виновен, горе мне! если и прав, то не осмелюсь поднять головы моей. Я пресыщен унижением; взгляни на бедствие мое:
\vs Job 10:16 оно увеличивается. Ты гонишься за мною, как лев, и снова нападаешь на меня и чудным являешься во мне.
\vs Job 10:17 Выводишь новых свидетелей Твоих против меня; усиливаешь гнев Твой на меня; и беды, одни за другими, ополчаются против меня.
\vs Job 10:18 И зачем Ты вывел меня из чрева? пусть бы я умер, когда еще ничей глаз не видел меня;
\vs Job 10:19 пусть бы я, как небывший, из чрева перенесен был во гроб!
\vs Job 10:20 Не малы ли дни мои? Оставь, отступи от меня, чтобы я немного ободрился,
\vs Job 10:21 прежде нежели отойду,~--- и уже не возвращусь,~--- в страну тьмы и сени смертной,
\vs Job 10:22 в страну мрака, каков есть мрак тени смертной, где нет устройства, \bibemph{где} темно, как самая тьма.
\vs Job 11:1 И отвечал Софар Наамитянин и сказал:
\vs Job 11:2 разве на множество слов нельзя дать ответа, и разве человек многоречивый прав?
\vs Job 11:3 Пустословие твое заставит ли молчать мужей, чтобы ты глумился, и некому было постыдить тебя?
\vs Job 11:4 Ты сказал: суждение мое верно, и чист я в очах Твоих.
\vs Job 11:5 Но если бы Бог возглаголал и отверз уста Свои к тебе
\vs Job 11:6 и открыл тебе тайны премудрости, что тебе вдвое больше следовало бы понести! Итак знай, что Бог для тебя некоторые из беззаконий твоих предал забвению.
\vs Job 11:7 Можешь ли ты исследованием найти Бога? Можешь ли совершенно постигнуть Вседержителя?
\vs Job 11:8 Он превыше небес,~--- что можешь сделать? глубже преисподней,~--- что можешь узнать?
\vs Job 11:9 Длиннее земли мера Его и шире моря.
\vs Job 11:10 Если Он пройдет и заключит кого в оковы и представит на суд, то кто отклонит Его?
\vs Job 11:11 Ибо Он знает людей лживых и видит беззаконие, и оставит ли его без внимания?
\vs Job 11:12 Но пустой человек мудрствует, хотя человек рождается подобно дикому осленку.
\vs Job 11:13 Если ты управишь сердце твое и прострешь к Нему руки твои,
\vs Job 11:14 и если есть порок в руке твоей, а ты удалишь его и не дашь беззаконию обитать в шатрах твоих,
\vs Job 11:15 то поднимешь незапятнанное лице твое и будешь тверд и не будешь бояться.
\vs Job 11:16 Тогда забудешь горе: как о воде протекшей, будешь вспоминать о нем.
\vs Job 11:17 И яснее полдня пойдет жизнь твоя; просветлеешь, как утро.
\vs Job 11:18 И будешь спокоен, ибо есть надежда; ты огражден, и можешь спать безопасно.
\vs Job 11:19 Будешь лежать, и не будет устрашающего, и многие будут заискивать у тебя.
\vs Job 11:20 А глаза беззаконных истают, и убежище пропадет у них, и надежда их исчезнет.
\vs Job 12:1 И отвечал Иов и сказал:
\vs Job 12:2 подлинно, \bibemph{только} вы люди, и с вами умрет мудрость!
\vs Job 12:3 И у меня \bibemph{есть} сердце, как у вас; не ниже я вас; и кто не знает того же?
\vs Job 12:4 Посмешищем стал я для друга своего, я, который взывал к Богу, и которому Он отвечал, посмешищем~--- \bibemph{человек} праведный, непорочный.
\vs Job 12:5 Так презрен по мыслям сидящего в покое факел, приготовленный для спотыкающихся ногами.
\vs Job 12:6 Покойны шатры у грабителей и безопасны у раздражающих Бога, которые как бы Бога носят в руках своих.
\vs Job 12:7 И подлинно: спроси у скота, и научит тебя, у птицы небесной, и возвестит тебе;
\vs Job 12:8 или побеседуй с землею, и наставит тебя, и скажут тебе рыбы морские.
\vs Job 12:9 Кто во всем этом не узнает, что рука Господа сотворила сие?
\vs Job 12:10 В Его руке душа всего живущего и дух всякой человеческой плоти.
\vs Job 12:11 Не ухо ли разбирает слова, и не язык ли распознает вкус пищи?
\vs Job 12:12 В старцах~--- мудрость, и в долголетних~--- разум.
\vs Job 12:13 У Него премудрость и сила; Его совет и разум.
\vs Job 12:14 Что Он разрушит, то не построится; кого Он заключит, тот не высвободится.
\vs Job 12:15 Остановит воды, и все высохнет; пустит их, и превратят землю.
\vs Job 12:16 У Него могущество и премудрость, пред Ним заблуждающийся и вводящий в заблуждение.
\vs Job 12:17 Он приводит советников в необдуманность и судей делает глупыми.
\vs Job 12:18 Он лишает перевязей царей и поясом обвязывает чресла их;
\vs Job 12:19 князей лишает достоинства и низвергает храбрых;
\vs Job 12:20 отнимает язык у велеречивых и старцев лишает смысла;
\vs Job 12:21 покрывает стыдом знаменитых и силу могучих ослабляет;
\vs Job 12:22 открывает глубокое из среды тьмы и выводит на свет тень смертную;
\vs Job 12:23 умножает народы и истребляет их; рассевает народы и собирает их;
\vs Job 12:24 отнимает ум у глав народа земли и оставляет их блуждать в пустыне, где нет пути:
\vs Job 12:25 ощупью ходят они во тьме без света и шатаются, как пьяные.
\vs Job 13:1 Вот, все \bibemph{это} видело око мое, слышало ухо мое и заметило для себя.
\vs Job 13:2 Сколько знаете вы, знаю и я: не ниже я вас.
\vs Job 13:3 Но я к Вседержителю хотел бы говорить и желал бы состязаться с Богом.
\vs Job 13:4 А вы сплетчики лжи; все вы бесполезные врачи.
\vs Job 13:5 О, если бы вы только молчали! это было бы \bibemph{вменено} вам в мудрость.
\vs Job 13:6 Выслушайте же рассуждения мои и вникните в возражение уст моих.
\vs Job 13:7 Надлежало ли вам ради Бога говорить неправду и для Него говорить ложь?
\vs Job 13:8 Надлежало ли вам быть лицеприятными к Нему и за Бога так препираться?
\vs Job 13:9 Хорошо ли будет, когда Он испытает вас? Обманете ли Его, как обманывают человека?
\vs Job 13:10 Строго накажет Он вас, хотя вы и скрытно лицемерите.
\vs Job 13:11 Неужели величие Его не устрашает вас, и страх Его не нападает на вас?
\vs Job 13:12 Напоминания ваши подобны пеплу; оплоты ваши~--- оплоты глиняные.
\vs Job 13:13 Замолчите предо мною, и я буду говорить, что бы ни постигло меня.
\vs Job 13:14 Для чего мне терзать тело мое зубами моими и душу мою полагать в руку мою?
\vs Job 13:15 Вот, Он убивает меня, но я буду надеяться; я желал бы только отстоять пути мои пред лицем Его!
\vs Job 13:16 И это уже в оправдание мне, потому что лицемер не пойдет пред лице Его!
\vs Job 13:17 Выслушайте внимательно слово мое и объяснение мое ушами вашими.
\vs Job 13:18 Вот, я завел судебное дело: знаю, что буду прав.
\vs Job 13:19 Кто в состоянии оспорить меня? Ибо я скоро умолкну и испущу дух.
\vs Job 13:20 Двух только \bibemph{вещей} не делай со мною, и тогда я не буду укрываться от лица Твоего:
\vs Job 13:21 удали от меня руку Твою, и ужас Твой да не потрясает меня.
\vs Job 13:22 Тогда зови, и я буду отвечать, или буду говорить я, а Ты отвечай мне.
\vs Job 13:23 Сколько у меня пороков и грехов? покажи мне беззаконие мое и грех мой.
\vs Job 13:24 Для чего скрываешь лице Твое и считаешь меня врагом Тебе?
\vs Job 13:25 Не сорванный ли листок Ты сокрушаешь и не сухую ли соломинку преследуешь?
\vs Job 13:26 Ибо Ты пишешь на меня горькое и вменяешь мне грехи юности моей,
\vs Job 13:27 и ставишь в колоду ноги мои и подстерегаешь все стези мои,~--- гонишься по следам ног моих.
\vs Job 13:28 А он, как гниль, распадается, как одежда, изъеденная молью.
\vs Job 14:1 Человек, рожденный женою, краткодневен и пресыщен печалями:
\vs Job 14:2 как цветок, он выходит и опадает; убегает, как тень, и не останавливается.
\vs Job 14:3 И на него-то Ты отверзаешь очи Твои, и меня ведешь на суд с Тобою?
\vs Job 14:4 Кто родится чистым от нечистого? Ни один.
\vs Job 14:5 Если дни ему определены, и число месяцев его у Тебя, если Ты положил ему предел, которого он не перейдет,
\vs Job 14:6 то уклонись от него: пусть он отдохнет, доколе не окончит, как наемник, дня своего.
\vs Job 14:7 Для дерева есть надежда, что оно, если и будет срублено, снова оживет, и отрасли от него \bibemph{выходить} не перестанут:
\vs Job 14:8 если и устарел в земле корень его, и пень его замер в пыли,
\vs Job 14:9 но, лишь почуяло воду, оно дает отпрыски и пускает ветви, как бы вновь посаженное.
\vs Job 14:10 А человек умирает и распадается; отошел, и где он?
\vs Job 14:11 Уходят воды из озера, и река иссякает и высыхает:
\vs Job 14:12 так человек ляжет и не встанет; до скончания неба он не пробудится и не воспрянет от сна своего.
\vs Job 14:13 О, если бы Ты в преисподней сокрыл меня и укрывал меня, пока пройдет гнев Твой, положил мне срок и потом вспомнил обо мне!
\vs Job 14:14 Когда умрет человек, то будет ли он опять жить? Во все дни определенного мне времени я ожидал бы, пока придет мне смена.
\vs Job 14:15 Воззвал бы Ты, и я дал бы Тебе ответ, и Ты явил бы благоволение творению рук Твоих;
\vs Job 14:16 ибо тогда Ты исчислял бы шаги мои и не подстерегал бы греха моего;
\vs Job 14:17 в свитке было бы запечатано беззаконие мое, и Ты закрыл бы вину мою.
\vs Job 14:18 Но гора падая разрушается, и скала сходит с места своего;
\vs Job 14:19 вода стирает камни; разлив ее смывает земную пыль: так и надежду человека Ты уничтожаешь.
\vs Job 14:20 Теснишь его до конца, и он уходит; изменяешь ему лице и отсылаешь его.
\vs Job 14:21 В чести ли дети его~--- он не знает, унижены ли~--- он не замечает;
\vs Job 14:22 но плоть его на нем болит, и душа его в нем страдает.
\vs Job 15:1 И отвечал Елифаз Феманитянин и сказал:
\vs Job 15:2 станет ли мудрый отвечать знанием пустым и наполнять чрево свое ветром палящим,
\vs Job 15:3 оправдываться словами бесполезными и речью, не имеющею никакой силы?
\vs Job 15:4 Да ты отложил и страх и за малость считаешь речь к Богу.
\vs Job 15:5 Нечестие твое настроило так уста твои, и ты избрал язык лукавых.
\vs Job 15:6 Тебя обвиняют уста твои, а не я, и твой язык говорит против тебя.
\vs Job 15:7 Разве ты первым человеком родился и прежде холмов создан?
\vs Job 15:8 Разве совет Божий ты слышал и привлек к себе премудрость?
\vs Job 15:9 Что знаешь ты, чего бы не знали мы? что разумеешь ты, чего не было бы и у нас?
\vs Job 15:10 И седовласый и старец есть между нами, днями превышающий отца твоего.
\vs Job 15:11 Разве малость для тебя утешения Божии? И это неизвестно тебе?
\vs Job 15:12 К чему порывает тебя сердце твое, и к чему так гордо смотришь?
\vs Job 15:13 Что устремляешь против Бога дух твой и устами твоими произносишь такие речи?
\vs Job 15:14 Что такое человек, чтоб быть ему чистым, и чтобы рожденному женщиною быть праведным?
\vs Job 15:15 Вот, Он и святым Своим не доверяет, и небеса нечисты в очах Его:
\vs Job 15:16 тем больше нечист и растлен человек, пьющий беззаконие, как воду.
\vs Job 15:17 Я буду говорить тебе, слушай меня; я расскажу тебе, что видел,
\vs Job 15:18 что слышали мудрые и не скрыли слышанного от отцов своих,
\vs Job 15:19 которым одним отдана была земля, и среди которых чужой не ходил.
\vs Job 15:20 Нечестивый мучит себя во все дни свои, и число лет закрыто от притеснителя;
\vs Job 15:21 звук ужасов в ушах его; среди мира идет на него губитель.
\vs Job 15:22 Он не надеется спастись от тьмы; видит пред собою меч.
\vs Job 15:23 Он скитается за куском хлеба повсюду; знает, что уже готов, в руках у него день тьмы.
\vs Job 15:24 Устрашает его нужда и теснота; одолевает его, как царь, приготовившийся к битве,
\vs Job 15:25 за то, что он простирал против Бога руку свою и противился Вседержителю,
\vs Job 15:26 устремлялся против Него с \bibemph{гордою} выею, под толстыми щитами своими;
\vs Job 15:27 потому что он покрыл лице свое жиром своим и обложил туком лядвеи свои.
\vs Job 15:28 И он селится в городах разоренных, в домах, в которых не живут, которые обречены на развалины.
\vs Job 15:29 Не пребудет он богатым, и не уцелеет имущество его, и не распрострется по земле приобретение его.
\vs Job 15:30 Не уйдет от тьмы; отрасли его иссушит пламя и дуновением уст своих увлечет его.
\vs Job 15:31 Пусть не доверяет суете заблудший, ибо суета будет и воздаянием ему.
\vs Job 15:32 Не в свой день он скончается, и ветви его не будут зеленеть.
\vs Job 15:33 Сбросит он, как виноградная лоза, недозрелую ягоду свою и, как маслина, стряхнет цвет свой.
\vs Job 15:34 Так опустеет дом нечестивого, и огонь пожрет шатры мздоимства.
\vs Job 15:35 Он зачал зло и родил ложь, и утроба его приготовляет обман.
\vs Job 16:1 И отвечал Иов и сказал:
\vs Job 16:2 слышал я много такого; жалкие утешители все вы!
\vs Job 16:3 Будет ли конец ветреным словам? и что побудило тебя так отвечать?
\vs Job 16:4 И я мог бы так же говорить, как вы, если бы душа ваша была на месте души моей; ополчался бы на вас словами и кивал бы на вас головою моею;
\vs Job 16:5 подкреплял бы вас языком моим и движением губ утешал бы.
\vs Job 16:6 Говорю ли я, не утоляется скорбь моя; перестаю ли, что отходит от меня?
\vs Job 16:7 Но ныне Он изнурил меня. Ты разрушил всю семью мою.
\vs Job 16:8 Ты покрыл меня морщинами во свидетельство против меня; восстает на меня изможденность моя, в лицо укоряет меня.
\vs Job 16:9 Гнев Его терзает и враждует против меня, скрежещет на меня зубами своими; неприятель мой острит на меня глаза свои.
\vs Job 16:10 Разинули на меня пасть свою; ругаясь бьют меня по щекам: все сговорились против меня.
\vs Job 16:11 Предал меня Бог беззаконнику и в руки нечестивым бросил меня.
\vs Job 16:12 Я был спокоен, но Он потряс меня; взял меня за шею и избил меня и поставил меня целью для Себя.
\vs Job 16:13 Окружили меня стрельцы Его; Он рассекает внутренности мои и не щадит, пролил на землю желчь мою,
\vs Job 16:14 пробивает во мне пролом за проломом, бежит на меня, как ратоборец.
\vs Job 16:15 Вретище сшил я на кожу мою и в прах положил голову мою.
\vs Job 16:16 Лицо мое побагровело от плача, и на веждах моих тень смерти,
\vs Job 16:17 при всем том, что нет хищения в руках моих, и молитва моя чиста.
\vs Job 16:18 Земля! не закрой моей крови, и да не будет места воплю моему.
\vs Job 16:19 И ныне вот на небесах Свидетель мой, и Заступник мой в вышних!
\vs Job 16:20 Многоречивые друзья мои! К Богу слезит око мое.
\vs Job 16:21 О, если бы человек мог иметь состязание с Богом, как сын человеческий с ближним своим!
\vs Job 16:22 Ибо летам моим приходит конец, и я отхожу в путь невозвратный.
\vs Job 17:1 Дыхание мое ослабело; дни мои угасают; гробы предо мною.
\vs Job 17:2 Если бы не насмешки их, то и среди споров их око мое пребывало бы спокойно.
\vs Job 17:3 Заступись, поручись \bibemph{Сам} за меня пред Собою! иначе кто поручится за меня?
\vs Job 17:4 Ибо Ты закрыл сердце их от разумения, и потому не дашь восторжествовать \bibemph{им}.
\vs Job 17:5 Кто обрекает друзей своих в добычу, у детей того глаза истают.
\vs Job 17:6 Он поставил меня притчею для народа и посмешищем для него.
\vs Job 17:7 Помутилось от горести око мое, и все члены мои, как тень.
\vs Job 17:8 Изумятся о сем праведные, и невинный вознегодует на лицемера.
\vs Job 17:9 Но праведник будет крепко держаться пути своего, и чистый руками будет больше и больше утверждаться.
\vs Job 17:10 Выступайте, все вы, и подойдите; не найду я мудрого между вами.
\vs Job 17:11 Дни мои прошли; думы мои~--- достояние сердца моего~--- разбиты.
\vs Job 17:12 А они ночь \bibemph{хотят} превратить в день, свет приблизить к лицу тьмы.
\vs Job 17:13 Если бы я и ожидать стал, то преисподняя~--- дом мой; во тьме постелю я постель мою;
\vs Job 17:14 гробу скажу: ты отец мой, червю: ты мать моя и сестра моя.
\vs Job 17:15 Где же после этого надежда моя? и ожидаемое мною кто увидит?
\vs Job 17:16 В преисподнюю сойдет она и будет покоиться со мною в прахе.
\vs Job 18:1 И отвечал Вилдад Савхеянин и сказал:
\vs Job 18:2 когда же положите вы конец таким речам? обдумайте, и потом будем говорить.
\vs Job 18:3 Зачем считаться нам за животных и быть униженными в собственных глазах ваших?
\vs Job 18:4 \bibemph{О ты}, раздирающий душу твою в гневе твоем! Неужели для тебя опустеть земле, и скале сдвинуться с места своего?
\vs Job 18:5 Да, свет у беззаконного потухнет, и не останется искры от огня его.
\vs Job 18:6 Померкнет свет в шатре его, и светильник его угаснет над ним.
\vs Job 18:7 Сократятся шаги могущества его, и низложит его собственный замысл его,
\vs Job 18:8 ибо он попадет в сеть своими ногами и по тенетам ходить будет.
\vs Job 18:9 Петля зацепит за ногу его, и грабитель уловит его.
\vs Job 18:10 Скрытно разложены по земле силки для него и западни на дороге.
\vs Job 18:11 Со всех сторон будут страшить его ужасы и заставят его бросаться туда и сюда.
\vs Job 18:12 Истощится от голода сила его, и гибель готова, сбоку у него.
\vs Job 18:13 Съест члены тела его, съест члены его первенец смерти.
\vs Job 18:14 Изгнана будет из шатра его надежда его, и это низведет его к царю ужасов.
\vs Job 18:15 Поселятся в шатре его, потому что он уже не его; жилище его посыпано будет серою.
\vs Job 18:16 Снизу подсохнут корни его, и сверху увянут ветви его.
\vs Job 18:17 Память о нем исчезнет с земли, и имени его не будет на площади.
\vs Job 18:18 Изгонят его из света во тьму и сотрут его с лица земли.
\vs Job 18:19 Ни сына его, ни внука не будет в народе его, и никого не останется в жилищах его.
\vs Job 18:20 О дне его ужаснутся потомки, и современники будут объяты трепетом.
\vs Job 18:21 Таковы жилища беззаконного, и таково место того, кто не знает Бога.
\vs Job 19:1 И отвечал Иов и сказал:
\vs Job 19:2 доколе будете мучить душу мою и терзать меня речами?
\vs Job 19:3 Вот, уже раз десять вы срамили меня и не стыдитесь теснить меня.
\vs Job 19:4 Если я и действительно погрешил, то погрешность моя при мне остается.
\vs Job 19:5 Если же вы хотите повеличаться надо мною и упрекнуть меня позором моим,
\vs Job 19:6 то знайте, что Бог ниспроверг меня и обложил меня Своею сетью.
\vs Job 19:7 Вот, я кричу: обида! и никто не слушает; вопию, и нет суда.
\vs Job 19:8 Он преградил мне дорогу, и не могу пройти, и на стези мои положил тьму.
\vs Job 19:9 Совлек с меня славу мою и снял венец с головы моей.
\vs Job 19:10 Кругом разорил меня, и я отхожу; и, как дерево, Он исторг надежду мою.
\vs Job 19:11 Воспылал на меня гневом Своим и считает меня между врагами Своими.
\vs Job 19:12 Полки Его пришли вместе и направили путь свой ко мне и расположились вокруг шатра моего.
\vs Job 19:13 Братьев моих Он удалил от меня, и знающие меня чуждаются меня.
\vs Job 19:14 Покинули меня близкие мои, и знакомые мои забыли меня.
\vs Job 19:15 Пришлые в доме моем и служанки мои чужим считают меня; посторонним стал я в глазах их.
\vs Job 19:16 Зову слугу моего, и он не откликается; устами моими я должен умолять его.
\vs Job 19:17 Дыхание мое опротивело жене моей, и я должен умолять ее ради детей чрева моего.
\vs Job 19:18 Даже малые дети презирают меня: поднимаюсь, и они издеваются надо мною.
\vs Job 19:19 Гнушаются мною все наперсники мои, и те, которых я любил, обратились против меня.
\vs Job 19:20 Кости мои прилипли к коже моей и плоти моей, и я остался только с кожею около зубов моих.
\vs Job 19:21 Помилуйте меня, помилуйте меня вы, друзья мои, ибо рука Божия коснулась меня.
\vs Job 19:22 Зачем и вы преследуете меня, как Бог, и плотью моею не можете насытиться?
\vs Job 19:23 О, если бы записаны были слова мои! Если бы начертаны были они в книге
\vs Job 19:24 резцом железным с оловом,~--- на вечное время на камне вырезаны были!
\vs Job 19:25 А я знаю, Искупитель мой жив, и Он в последний день восставит из праха распадающуюся кожу мою сию,
\vs Job 19:26 и я во плоти моей узрю Бога.
\vs Job 19:27 Я узрю Его сам; мои глаза, не глаза другого, увидят Его. Истаевает сердце мое в груди моей!
\vs Job 19:28 Вам надлежало бы сказать: зачем мы преследуем его? Как будто корень зла найден во мне.
\vs Job 19:29 Убойтесь меча, ибо меч есть отмститель неправды, и знайте, что есть суд.
\vs Job 20:1 И отвечал Софар Наамитянин и сказал:
\vs Job 20:2 размышления мои побуждают меня отвечать, и я поспешаю выразить их.
\vs Job 20:3 Упрек, позорный для меня, выслушал я, и дух разумения моего ответит за меня.
\vs Job 20:4 Разве не знаешь ты, что от века,~--- с того времени, как поставлен человек на земле,~---
\vs Job 20:5 веселье беззаконных кратковременно, и радость лицемера мгновенна?
\vs Job 20:6 Хотя бы возросло до небес величие его, и голова его касалась облаков,~---
\vs Job 20:7 как помет его, на веки пропадает он; видевшие его скажут: где он?
\vs Job 20:8 Как сон, улетит, и не найдут его; и, как ночное видение, исчезнет.
\vs Job 20:9 Глаз, видевший его, больше не увидит его, и уже не усмотрит его место его.
\vs Job 20:10 Сыновья его будут заискивать у нищих, и руки его возвратят похищенное им.
\vs Job 20:11 Кости его наполнены грехами юности его, и с ним лягут они в прах.
\vs Job 20:12 Если сладко во рту его зло, и он таит его под языком своим,
\vs Job 20:13 бережет и не бросает его, а держит его в устах своих,
\vs Job 20:14 то эта пища его в утробе его превратится в желчь аспидов внутри его.
\vs Job 20:15 Имение, которое он глотал, изблюет: Бог исторгнет его из чрева его.
\vs Job 20:16 Змеиный яд он сосет; умертвит его язык ехидны.
\vs Job 20:17 Не видать ему ручьев, рек, текущих медом и молоком!
\vs Job 20:18 Нажитое трудом возвратит, не проглотит; по мере имения его будет и расплата его, а он не порадуется.
\vs Job 20:19 Ибо он угнетал, отсылал бедных; захватывал домы, которых не строил;
\vs Job 20:20 не знал сытости во чреве своем и в жадности своей не щадил ничего.
\vs Job 20:21 Ничего не спаслось от обжорства его, зато не устоит счастье его.
\vs Job 20:22 В полноте изобилия будет тесно ему; всякая рука обиженного поднимется на него.
\vs Job 20:23 Когда будет чем наполнить утробу его, Он пошлет на него ярость гнева Своего и одождит на него болезни в плоти его.
\vs Job 20:24 Убежит ли он от оружия железного,~--- пронзит его лук медный;
\vs Job 20:25 станет вынимать \bibemph{стрелу},~--- и она выйдет из тела, выйдет, сверкая сквозь желчь его; ужасы смерти найдут на него!
\vs Job 20:26 Все мрачное сокрыто внутри его; будет пожирать его огонь, никем не раздуваемый; зло постигнет и оставшееся в шатре его.
\vs Job 20:27 Небо откроет беззаконие его, и земля восстанет против него.
\vs Job 20:28 Исчезнет стяжание дома его; все расплывется в день гнева Его.
\vs Job 20:29 Вот удел человеку беззаконному от Бога и наследие, определенное ему Вседержителем!
\vs Job 21:1 И отвечал Иов и сказал:
\vs Job 21:2 выслушайте внимательно речь мою, и это будет мне утешением от вас.
\vs Job 21:3 Потерпите меня, и я буду говорить; а после того, как поговорю, насмехайся.
\vs Job 21:4 Разве к человеку речь моя? как же мне и не малодушествовать?
\vs Job 21:5 Посмотрите на меня и ужаснитесь, и положите перст на уста.
\vs Job 21:6 Лишь только я вспомню,~--- содрогаюсь, и трепет объемлет тело мое.
\vs Job 21:7 Почему беззаконные живут, достигают старости, да и силами крепки?
\vs Job 21:8 Дети их с ними перед лицем их, и внуки их перед глазами их.
\vs Job 21:9 Домы их безопасны от страха, и нет жезла Божия на них.
\vs Job 21:10 Вол их оплодотворяет и не извергает, корова их зачинает и не выкидывает.
\vs Job 21:11 Как стадо, выпускают они малюток своих, и дети их прыгают.
\vs Job 21:12 Восклицают под \bibemph{голос} тимпана и цитры и веселятся при \bibemph{звуках} свирели;
\vs Job 21:13 проводят дни свои в счастьи и мгновенно нисходят в преисподнюю.
\vs Job 21:14 А между тем они говорят Богу: отойди от нас, не хотим мы знать путей Твоих!
\vs Job 21:15 Что Вседержитель, чтобы нам служить Ему? и что пользы прибегать к Нему?
\vs Job 21:16 Видишь, счастье их не от их рук.~--- Совет нечестивых будь далек от меня!
\vs Job 21:17 Часто ли угасает светильник у беззаконных, и находит на них беда, и Он дает им в удел страдания во гневе Своем?
\vs Job 21:18 Они должны быть, как соломинка пред ветром и как плева, уносимая вихрем.
\vs Job 21:19 \bibemph{Скажешь}: Бог бережет для детей его несчастье его.~--- Пусть воздаст Он ему самому, чтобы он это знал.
\vs Job 21:20 Пусть его глаза увидят несчастье его, и пусть он сам пьет от гнева Вседержителева.
\vs Job 21:21 Ибо какая ему забота до дома своего после него, когда число месяцев его кончится?
\vs Job 21:22 Но Бога ли учить мудрости, когда Он судит и горних?
\vs Job 21:23 Один умирает в самой полноте сил своих, совершенно спокойный и мирный;
\vs Job 21:24 внутренности его полны жира, и кости его напоены мозгом.
\vs Job 21:25 А другой умирает с душею огорченною, не вкусив добра.
\vs Job 21:26 И они вместе будут лежать во прахе, и червь покроет их.
\vs Job 21:27 Знаю я ваши мысли и ухищрения, какие вы против меня сплетаете.
\vs Job 21:28 Вы скажете: где дом князя, и где шатер, в котором жили беззаконные?
\vs Job 21:29 Разве вы не спрашивали у путешественников и незнакомы с их наблюдениями,
\vs Job 21:30 что в день погибели пощажен бывает злодей, в день гнева отводится в сторону?
\vs Job 21:31 Кто представит ему пред лице путь его, и кто воздаст ему за то, что он делал?
\vs Job 21:32 Его провожают ко гробам и на его могиле ставят стражу.
\vs Job 21:33 Сладки для него глыбы долины, и за ним идет толпа людей, а идущим перед ним нет числа.
\vs Job 21:34 Как же вы хотите утешать меня пустым? В ваших ответах остается \bibemph{одна} ложь.
\vs Job 22:1 И отвечал Елифаз Феманитянин и сказал:
\vs Job 22:2 разве может человек доставлять пользу Богу? Разумный доставляет пользу себе самому.
\vs Job 22:3 Что за удовольствие Вседержителю, что ты праведен? И будет ли Ему выгода от того, что ты содержишь пути твои в непорочности?
\vs Job 22:4 Неужели Он, боясь тебя, вступит с тобою в состязание, пойдет судиться с тобою?
\vs Job 22:5 Верно, злоба твоя велика, и беззакониям твоим нет конца.
\vs Job 22:6 Верно, ты брал залоги от братьев твоих ни за что и с полунагих снимал одежду.
\vs Job 22:7 Утомленному жаждою не подавал воды напиться и голодному отказывал в хлебе;
\vs Job 22:8 а человеку сильному ты \bibemph{давал} землю, и сановитый селился на ней.
\vs Job 22:9 Вдов ты отсылал ни с чем и сирот оставлял с пустыми руками.
\vs Job 22:10 За то вокруг тебя петли, и возмутил тебя неожиданный ужас,
\vs Job 22:11 или тьма, в которой ты ничего не видишь, и множество вод покрыло тебя.
\vs Job 22:12 Не превыше ли небес Бог? посмотри вверх на звезды, как они высоко!
\vs Job 22:13 И ты говоришь: что знает Бог? может ли Он судить сквозь мрак?
\vs Job 22:14 Облака~--- завеса Его, так что Он не видит, а ходит \bibemph{только} по небесному кругу.
\vs Job 22:15 Неужели ты держишься пути древних, по которому шли люди беззаконные,
\vs Job 22:16 которые преждевременно были истреблены, когда вода разлилась под основание их?
\vs Job 22:17 Они говорили Богу: отойди от нас! и что сделает им Вседержитель?
\vs Job 22:18 А Он наполнял домы их добром. Но совет нечестивых будь далек от меня!
\vs Job 22:19 Видели праведники и радовались, и непорочный смеялся им:
\vs Job 22:20 враг наш истреблен, а оставшееся после них пожрал огонь.
\vs Job 22:21 Сблизься же с Ним~--- и будешь спокоен; чрез это придет к тебе добро.
\vs Job 22:22 Прими из уст Его закон и положи слова Его в сердце твое.
\vs Job 22:23 Если ты обратишься к Вседержителю, то вновь устроишься, удалишь беззаконие от шатра твоего
\vs Job 22:24 и будешь вменять в прах блестящий металл, и в камни потоков~--- \bibemph{золото} Офирское.
\vs Job 22:25 И будет Вседержитель твоим золотом и блестящим серебром у тебя,
\vs Job 22:26 ибо тогда будешь радоваться о Вседержителе и поднимешь к Богу лице твое.
\vs Job 22:27 Помолишься Ему, и Он услышит тебя, и ты исполнишь обеты твои.
\vs Job 22:28 Положишь намерение, и оно состоится у тебя, и над путями твоими будет сиять свет.
\vs Job 22:29 Когда кто уничижен будет, ты скажешь: возвышение! и Он спасет поникшего лицем,
\vs Job 22:30 избавит и небезвинного, и он спасется чистотою рук твоих.
\vs Job 23:1 И отвечал Иов и сказал:
\vs Job 23:2 еще и ныне горька речь моя: страдания мои тяжелее стонов моих.
\vs Job 23:3 О, если бы я знал, где найти Его, и мог подойти к престолу Его!
\vs Job 23:4 Я изложил бы пред Ним дело мое и уста мои наполнил бы оправданиями;
\vs Job 23:5 узнал бы слова, какими Он ответит мне, и понял бы, что Он скажет мне.
\vs Job 23:6 Неужели Он в полном могуществе стал бы состязаться со мною? О, нет! Пусть Он только обратил бы внимание на меня.
\vs Job 23:7 Тогда праведник мог бы состязаться с Ним,~--- и я навсегда получил бы свободу от Судии моего.
\vs Job 23:8 Но вот, я иду вперед~--- и нет Его, назад~--- и не нахожу Его;
\vs Job 23:9 делает ли Он что на левой стороне, я не вижу; скрывается ли на правой, не усматриваю.
\vs Job 23:10 Но Он знает путь мой; пусть испытает меня,~--- выйду, как золото.
\vs Job 23:11 Нога моя твердо держится стези Его; пути Его я хранил и не уклонялся.
\vs Job 23:12 От заповеди уст Его не отступал; глаголы уст Его хранил больше, нежели мои правила.
\vs Job 23:13 Но Он тверд; и кто отклонит Его? Он делает, чего хочет душа Его.
\vs Job 23:14 Так, Он выполнит положенное мне, и подобного этому много у Него.
\vs Job 23:15 Поэтому я трепещу пред лицем Его; размышляю~--- и страшусь Его.
\vs Job 23:16 Бог расслабил сердце мое, и Вседержитель устрашил меня.
\vs Job 23:17 Зачем я не уничтожен прежде этой тьмы, и Он не сокрыл мрака от лица моего!
\vs Job 24:1 Почему не сокрыты от Вседержителя времена, и знающие Его не видят дней Его?
\vs Job 24:2 Межи передвигают, угоняют стада и пасут \bibemph{у себя}.
\vs Job 24:3 У сирот уводят осла, у вдовы берут в залог вола;
\vs Job 24:4 бедных сталкивают с дороги, все уничиженные земли принуждены скрываться.
\vs Job 24:5 Вот они, \bibemph{как} дикие ослы в пустыне, выходят на дело свое, вставая рано на добычу; степь \bibemph{дает} хлеб для них и для детей их;
\vs Job 24:6 жнут они на поле не своем и собирают виноград у нечестивца;
\vs Job 24:7 нагие ночуют без покрова и без одеяния на стуже;
\vs Job 24:8 мокнут от горных дождей и, не имея убежища, жмутся к скале;
\vs Job 24:9 отторгают от сосцов сироту и с нищего берут залог;
\vs Job 24:10 заставляют ходить нагими, без одеяния, и голодных кормят колосьями;
\vs Job 24:11 между стенами выжимают масло оливковое, топчут в точилах и жаждут.
\vs Job 24:12 В городе люди стонут, и душа убиваемых вопит, и Бог не воспрещает того.
\vs Job 24:13 Есть из них враги света, не знают путей его и не ходят по стезям его.
\vs Job 24:14 С рассветом встает убийца, умерщвляет бедного и нищего, а ночью бывает вором.
\vs Job 24:15 И око прелюбодея ждет сумерков, говоря: ничей глаз не увидит меня,~--- и закрывает лице.
\vs Job 24:16 В темноте подкапываются под домы, которые днем они заметили для себя; не знают света.
\vs Job 24:17 Ибо для них утро~--- смертная тень, так как они знакомы с ужасами смертной тени.
\vs Job 24:18 Легок такой на поверхности воды, проклята часть его на земле, и не смотрит он на дорогу садов виноградных.
\vs Job 24:19 Засуха и жара поглощают снежную воду: так преисподняя~--- грешников.
\vs Job 24:20 Пусть забудет его утроба \bibemph{матери}; пусть лакомится им червь; пусть не остается о нем память; как дерево, пусть сломится беззаконник,
\vs Job 24:21 который угнетает бездетную, не рождавшую, и вдове не делает добра.
\vs Job 24:22 Он и сильных увлекает своею силою; он встает, и никто не уверен за жизнь свою.
\vs Job 24:23 А Он дает ему \bibemph{все} для безопасности, и он \bibemph{на то} опирается, и очи Его видят пути их.
\vs Job 24:24 Поднялись высоко,~--- и вот, нет их; падают и умирают, как и все, и, как верхушки колосьев, срезываются.
\vs Job 24:25 Если это не так,~--- кто обличит меня во лжи и в ничто обратит речь мою?
\vs Job 25:1 И отвечал Вилдад Савхеянин и сказал:
\vs Job 25:2 держава и страх у Него; Он творит мир на высотах Своих!
\vs Job 25:3 Есть ли счет воинствам Его? и над кем не восходит свет Его?
\vs Job 25:4 И как человеку быть правым пред Богом, и как быть чистым рожденному женщиною?
\vs Job 25:5 Вот даже луна, и та несветла, и звезды нечисты пред очами Его.
\vs Job 25:6 Тем менее человек, \bibemph{который} есть червь, и сын человеческий, \bibemph{который} есть моль.
\vs Job 26:1 И отвечал Иов и сказал:
\vs Job 26:2 как ты помог бессильному, поддержал мышцу немощного!
\vs Job 26:3 Какой совет подал ты немудрому и как во всей полноте объяснил дело!
\vs Job 26:4 Кому ты говорил эти слова, и чей дух исходил из тебя?
\vs Job 26:5 Рефаимы трепещут под водами, и живущие в них.
\vs Job 26:6 Преисподняя обнажена пред Ним, и нет покрывала Аваддону.
\vs Job 26:7 Он распростер север над пустотою, повесил землю ни на чем.
\vs Job 26:8 Он заключает воды в облаках Своих, и облако не расседается под ними.
\vs Job 26:9 Он поставил престол Свой, распростер над ним облако Свое.
\vs Job 26:10 Черту провел над поверхностью воды, до границ света со тьмою.
\vs Job 26:11 Столпы небес дрожат и ужасаются от грозы Его.
\vs Job 26:12 Силою Своею волнует море и разумом Своим сражает его дерзость.
\vs Job 26:13 От духа Его~--- великолепие неба; рука Его образовала быстрого скорпиона.
\vs Job 26:14 Вот, это части путей Его; и как мало мы слышали о Нем! А гром могущества Его кто может уразуметь?
\vs Job 27:1 И продолжал Иов возвышенную речь свою и сказал:
\vs Job 27:2 жив Бог, лишивший \bibemph{меня} суда, и Вседержитель, огорчивший душу мою,
\vs Job 27:3 что, доколе еще дыхание мое во мне и дух Божий в ноздрях моих,
\vs Job 27:4 не скажут уста мои неправды, и язык мой не произнесет лжи!
\vs Job 27:5 Далек я от того, чтобы признать вас справедливыми; доколе не умру, не уступлю непорочности моей.
\vs Job 27:6 Крепко держал я правду мою и не опущу ее; не укорит меня сердце мое во все дни мои.
\vs Job 27:7 Враг мой будет, как нечестивец, и восстающий на меня, как беззаконник.
\vs Job 27:8 Ибо какая надежда лицемеру, когда возьмет, когда исторгнет Бог душу его?
\vs Job 27:9 Услышит ли Бог вопль его, когда придет на него беда?
\vs Job 27:10 Будет ли он утешаться Вседержителем и призывать Бога во всякое время?
\vs Job 27:11 Возвещу вам, чт\acc{о} в руке Божией; чт\acc{о} у Вседержителя, не скрою.
\vs Job 27:12 Вот, все вы и сами видели; и для чего вы столько пустословите?
\vs Job 27:13 Вот доля человеку беззаконному от Бога, и наследие, какое получают от Вседержителя притеснители.
\vs Job 27:14 Если умножаются сыновья его, то под меч; и потомки его не насытятся хлебом.
\vs Job 27:15 Оставшихся по нем смерть низведет во гроб, и вдовы их не будут плакать.
\vs Job 27:16 Если он наберет кучи серебра, как праха, и наготовит одежд, как брение,
\vs Job 27:17 то он наготовит, а одеваться будет праведник, и серебро получит себе на долю беспорочный.
\vs Job 27:18 Он строит, как моль, дом свой и, как сторож, делает себе шалаш;
\vs Job 27:19 ложится спать богачом и таким не встанет; открывает глаза свои, и он уже не тот.
\vs Job 27:20 Как в\acc{о}ды, постигнут его ужасы; в ночи похитит его буря.
\vs Job 27:21 Поднимет его восточный ветер и понесет, и он быстро побежит от него.
\vs Job 27:22 Устремится на него и не пощадит, как бы он ни силился убежать от руки его.
\vs Job 27:23 Всплеснут о нем руками и посвищут над ним с места его!
\vs Job 28:1 Так! у серебра есть источная жила, и у золота место, \bibemph{где его} плавят.
\vs Job 28:2 Железо получается из земли; из камня выплавляется медь.
\vs Job 28:3 \bibemph{Человек} полагает предел тьме и тщательно разыскивает камень во мраке и тени смертной.
\vs Job 28:4 Вырывают рудокопный колодезь в местах, забытых ногою, спускаются вглубь, висят \bibemph{и} зыблются вдали от людей.
\vs Job 28:5 Земля, на которой вырастает хлеб, внутри изрыта как бы огнем.
\vs Job 28:6 Камни ее~--- место сапфира, и в ней песчинки золота.
\vs Job 28:7 Стези \bibemph{туда} не знает хищная птица, и не видал ее глаз коршуна;
\vs Job 28:8 не попирали ее скимны, и не ходил по ней шакал.
\vs Job 28:9 На гранит налагает он руку свою, с корнем опрокидывает горы;
\vs Job 28:10 в скалах просекает каналы, и все драгоценное видит глаз его;
\vs Job 28:11 останавливает течение потоков и сокровенное выносит на свет.
\vs Job 28:12 Но где премудрость обретается? и где место разума?
\vs Job 28:13 Не знает человек цены ее, и она не обретается на земле живых.
\vs Job 28:14 Бездна говорит: не во мне она; и море говорит: не у меня.
\vs Job 28:15 Не дается она за золото и не приобретается она за вес серебра;
\vs Job 28:16 не оценивается она золотом Офирским, ни драгоценным ониксом, ни сапфиром;
\vs Job 28:17 не равняется с нею золото и кристалл, и не выменяешь ее на сосуды из чистого золота.
\vs Job 28:18 А о кораллах и жемчуге и упоминать нечего, и приобретение премудрости выше рубинов.
\vs Job 28:19 Не равняется с нею топаз Ефиопский; чистым золотом не оценивается она.
\vs Job 28:20 Откуда же исходит премудрость? и где место разума?
\vs Job 28:21 Сокрыта она от очей всего живущего и от птиц небесных утаена.
\vs Job 28:22 Аваддон и смерть говорят: ушами нашими слышали мы слух о ней.
\vs Job 28:23 Бог знает путь ее, и Он ведает место ее.
\vs Job 28:24 Ибо Он прозирает до концов земли и видит под всем небом.
\vs Job 28:25 Когда Он ветру полагал вес и располагал воду по мере,
\vs Job 28:26 когда назначал устав дождю и путь для молнии громоносной,
\vs Job 28:27 тогда Он видел ее и явил ее, приготовил ее и еще испытал ее
\vs Job 28:28 и сказал человеку: вот, страх Господень есть истинная премудрость, и удаление от зла~--- разум.
\vs Job 29:1 И продолжал Иов возвышенную речь свою и сказал:
\vs Job 29:2 о, если бы я был, как в прежние месяцы, как в те дни, когда Бог хранил меня,
\vs Job 29:3 когда светильник Его светил над головою моею, и я при свете Его ходил среди тьмы;
\vs Job 29:4 как был я во дни молодости моей, когда милость Божия \bibemph{была} над шатром моим,
\vs Job 29:5 когда еще Вседержитель \bibemph{был} со мною, и дети мои вокруг меня,
\vs Job 29:6 когда пути мои обливались молоком, и скала источала для меня ручьи елея!
\vs Job 29:7 когда я выходил к воротам города и на площади ставил седалище свое,~---
\vs Job 29:8 юноши, увидев меня, прятались, а старцы вставали и стояли;
\vs Job 29:9 князья удерживались от речи и персты полагали на уста свои;
\vs Job 29:10 голос знатных умолкал, и язык их прилипал к гортани их.
\vs Job 29:11 Ухо, слышавшее меня, ублажало меня; око видевшее восхваляло меня,
\vs Job 29:12 потому что я спасал страдальца вопиющего и сироту беспомощного.
\vs Job 29:13 Благословение погибавшего приходило на меня, и сердцу вдовы доставлял я радость.
\vs Job 29:14 Я облекался в правду, и суд мой одевал меня, как мантия и увясло.
\vs Job 29:15 Я был глазами слепому и ногами хромому;
\vs Job 29:16 отцом был я для нищих и тяжбу, которой я не знал, разбирал внимательно.
\vs Job 29:17 Сокрушал я беззаконному челюсти и из зубов его исторгал похищенное.
\vs Job 29:18 И говорил я: в гнезде моем скончаюсь, и дни \bibemph{мои} будут многи, как песок;
\vs Job 29:19 корень мой открыт для воды, и роса ночует на ветвях моих;
\vs Job 29:20 слава моя не стареет, лук мой крепок в руке моей.
\vs Job 29:21 Внимали мне и ожидали, и безмолвствовали при совете моем.
\vs Job 29:22 После слов моих уже не рассуждали; речь моя капала на них.
\vs Job 29:23 Ждали меня, как дождя, и, \bibemph{как} дождю позднему, открывали уста свои.
\vs Job 29:24 Бывало, улыбнусь им~--- они не верят; и света лица моего они не помрачали.
\vs Job 29:25 Я назначал пути им и сидел во главе и жил как царь в кругу воинов, как утешитель плачущих.
\vs Job 30:1 А ныне смеются надо мною младшие меня летами, те, которых отцов я не согласился бы поместить с псами стад моих.
\vs Job 30:2 И сила рук их к чему мне? Над ними уже прошло время.
\vs Job 30:3 Бедностью и голодом истощенные, они убегают в степь безводную, мрачную и опустевшую;
\vs Job 30:4 щиплют зелень подле кустов, и ягоды можжевельника~--- хлеб их.
\vs Job 30:5 Из общества изгоняют их, кричат на них, как на воров,
\vs Job 30:6 чтобы жили они в рытвинах потоков, в ущельях земли и утесов.
\vs Job 30:7 Ревут между кустами, жмутся под терном.
\vs Job 30:8 Люди отверженные, люди без имени, отребье земли!
\vs Job 30:9 Их-то сделался я ныне песнью и пищею разговора их.
\vs Job 30:10 Они гнушаются мною, удаляются от меня и не удерживаются плевать пред лицем моим.
\vs Job 30:11 Так как Он развязал повод мой и поразил меня, то они сбросили с себя узду пред лицем моим.
\vs Job 30:12 С правого боку встает это исчадие, сбивает меня с ног, направляет гибельные свои пути ко мне.
\vs Job 30:13 А мою стезю испортили: всё успели сделать к моей погибели, не имея помощника.
\vs Job 30:14 Они пришли ко мне, как сквозь широкий пролом; с шумом бросились на меня.
\vs Job 30:15 Ужасы устремились на меня; как ветер, развеялось величие мое, и счастье мое унеслось, как облако.
\vs Job 30:16 И ныне изливается душа моя во мне: дни скорби объяли меня.
\vs Job 30:17 Ночью ноют во мне кости мои, и жилы мои не имеют покоя.
\vs Job 30:18 С великим трудом снимается с меня одежда моя; края хитона моего жмут меня.
\vs Job 30:19 Он бросил меня в грязь, и я стал, как прах и пепел.
\vs Job 30:20 Я взываю к Тебе, и Ты не внимаешь мне,~--- стою, а Ты \bibemph{только} смотришь на меня.
\vs Job 30:21 Ты сделался жестоким ко мне, крепкою рукою враждуешь против меня.
\vs Job 30:22 Ты поднял меня и заставил меня носиться по ветру и сокрушаешь меня.
\vs Job 30:23 Так, я знаю, что Ты приведешь меня к смерти и в дом собрания всех живущих.
\vs Job 30:24 Верно, Он не прострет руки Своей на дом костей: будут ли они кричать при своем разрушении?
\vs Job 30:25 Не плакал ли я о том, кто был в горе? не скорбела ли душа моя о бедных?
\vs Job 30:26 Когда я чаял добра, пришло зло; когда ожидал света, пришла тьма.
\vs Job 30:27 Мои внутренности кипят и не перестают; встретили меня дни печали.
\vs Job 30:28 Я хожу почернелый, но не от солнца; встаю в собрании и кричу.
\vs Job 30:29 Я стал братом шакалам и другом страусам.
\vs Job 30:30 Моя кожа почернела на мне, и кости мои обгорели от жара.
\vs Job 30:31 И цитра моя сделалась унылою, и свирель моя~--- голосом плачевным.
\vs Job 31:1 Завет положил я с глазами моими, чтобы не помышлять мне о девице.
\vs Job 31:2 Какая же участь \bibemph{мне} от Бога свыше? И какое наследие от Вседержителя с небес?
\vs Job 31:3 Не для нечестивого ли гибель, и не для делающего ли зло напасть?
\vs Job 31:4 Не видел ли Он путей моих, и не считал ли всех моих шагов?
\vs Job 31:5 Если я ходил в суете, и если нога моя спешила на лукавство,~---
\vs Job 31:6 пусть взвесят меня на весах правды, и Бог узнает мою непорочность.
\vs Job 31:7 Если стопы мои уклонялись от пути и сердце мое следовало за глазами моими, и если что-либо \bibemph{нечистое} пристало к рукам моим,
\vs Job 31:8 то пусть я сею, а другой ест, и пусть отрасли мои искоренены будут.
\vs Job 31:9 Если сердце мое прельщалось женщиною и я строил ковы у дверей моего ближнего,~---
\vs Job 31:10 пусть моя жена мелет на другого, и пусть другие издеваются над нею,
\vs Job 31:11 потому что это~--- преступление, это~--- беззаконие, подлежащее суду;
\vs Job 31:12 это~--- огонь, поядающий до истребления, который искоренил бы все добро мое.
\vs Job 31:13 Если я пренебрегал правами слуги и служанки моей, когда они имели спор со мною,
\vs Job 31:14 то что стал бы я делать, когда бы Бог восстал? И когда бы Он взглянул на меня, что мог бы я отвечать Ему?
\vs Job 31:15 Не Он ли, Который создал меня во чреве, создал и его и равно образовал нас в утробе?
\vs Job 31:16 Отказывал ли я нуждающимся в их просьбе и томил ли глаза вдовы?
\vs Job 31:17 Один ли я съедал кусок мой, и не ел ли от него и сирота?
\vs Job 31:18 Ибо с детства он рос со мною, как с отцом, и от чрева матери моей я руководил \bibemph{вдову}.
\vs Job 31:19 Если я видел кого погибавшим без одежды и бедного без покрова,~---
\vs Job 31:20 не благословляли ли меня чресла его, и не был ли он согрет шерстью овец моих?
\vs Job 31:21 Если я поднимал руку мою на сироту, когда видел помощь себе у ворот,
\vs Job 31:22 то пусть плечо мое отпадет от спины, и рука моя пусть отломится от локтя,
\vs Job 31:23 ибо страшно для меня наказание от Бога: пред величием Его не устоял бы я.
\vs Job 31:24 Полагал ли я в золоте опору мою и говорил ли сокровищу: ты~--- надежда моя?
\vs Job 31:25 Радовался ли я, что богатство мое было велико, и что рука моя приобрела много?
\vs Job 31:26 Смотря на солнце, как оно сияет, и на луну, как она величественно шествует,
\vs Job 31:27 прельстился ли я в тайне сердца моего, и целовали ли уста мои руку мою?
\vs Job 31:28 Это также было бы преступление, подлежащее суду, потому что я отрекся бы \bibemph{тогда} от Бога Всевышнего.
\vs Job 31:29 Радовался ли я погибели врага моего и торжествовал ли, когда несчастье постигало его?
\vs Job 31:30 Не позволял я устам моим грешить проклятием души его.
\vs Job 31:31 Не говорили ли люди шатра моего: о, если бы мы от мяс его не насытились?
\vs Job 31:32 Странник не ночевал на улице; двери мои я отворял прохожему.
\vs Job 31:33 Если бы я скрывал проступки мои, как человек, утаивая в груди моей пороки мои,
\vs Job 31:34 то я боялся бы большого общества, и презрение одноплеменников страшило бы меня, и я молчал бы и не выходил бы за двери.
\vs Job 31:35 О, если бы кто выслушал меня! Вот мое желание, чтобы Вседержитель отвечал мне, и чтобы защитник мой составил запись.
\vs Job 31:36 Я носил бы ее на плечах моих и возлагал бы ее, как венец;
\vs Job 31:37 объявил бы ему число шагов моих, сблизился бы с ним, как с князем.
\vs Job 31:38 Если вопияла на меня земля моя и жаловались на меня борозды ее;
\vs Job 31:39 если я ел плоды ее без платы и отягощал жизнь земледельцев,
\vs Job 31:40 то пусть вместо пшеницы вырастает волчец и вместо ячменя куколь. Слова Иова кончились.
\vs Job 32:1 Когда те три мужа перестали отвечать Иову, потому что он был прав в глазах своих,
\vs Job 32:2 тогда воспылал гнев Елиуя, сына Варахиилова, Вузитянина из племени Рамова: воспылал гнев его на Иова за то, что он оправдывал себя больше, нежели Бога,
\vs Job 32:3 а на трех друзей его воспылал гнев его за то, что они не нашли, что отвечать, а между тем обвиняли Иова.
\vs Job 32:4 Елиуй ждал, пока Иов говорил, потому что они летами были старше его.
\vs Job 32:5 Когда же Елиуй увидел, что нет ответа в устах тех трех мужей, тогда воспылал гнев его.
\vs Job 32:6 И отвечал Елиуй, сын Варахиилов, Вузитянин, и сказал: я молод летами, а вы~--- старцы; поэтому я робел и боялся объявлять вам мое мнение.
\vs Job 32:7 Я говорил сам себе: пусть говорят дни, и многолетие поучает мудрости.
\vs Job 32:8 Но дух в человеке и дыхание Вседержителя дает ему разумение.
\vs Job 32:9 Не многолетние \bibemph{только} мудры, и не старики разумеют правду.
\vs Job 32:10 Поэтому я говорю: выслушайте меня, объявлю вам мое мнение и я.
\vs Job 32:11 Вот, я ожидал слов ваших,~--- вслушивался в суждения ваши, доколе вы придумывали, чт\acc{о} сказать.
\vs Job 32:12 Я пристально смотрел на вас, и вот никто из вас не обличает Иова и не отвечает на слова его.
\vs Job 32:13 Не скажите: мы нашли мудрость: Бог опровергнет его, а не человек.
\vs Job 32:14 Если бы он обращал слова свои ко мне, то я не вашими речами отвечал бы ему.
\vs Job 32:15 Испугались, не отвечают более; перестали говорить.
\vs Job 32:16 И как я ждал, а они не говорят, остановились и не отвечают более,
\vs Job 32:17 то и я отвечу с моей стороны, объявлю мое мнение и я,
\vs Job 32:18 ибо я полон речами, и дух во мне теснит меня.
\vs Job 32:19 Вот, утроба моя, как вино неоткрытое: она готова прорваться, подобно новым мехам.
\vs Job 32:20 Поговорю, и будет легче мне; открою уста мои и отвечу.
\vs Job 32:21 На лице человека смотреть не буду и никакому человеку льстить не стану,
\vs Job 32:22 потому что я не умею льстить: сейчас убей меня, Творец мой.
\vs Job 33:1 Итак слушай, Иов, речи мои и внимай всем словам моим.
\vs Job 33:2 Вот, я открываю уста мои, язык мой говорит в гортани моей.
\vs Job 33:3 Слова мои от искренности моего сердца, и уста мои произнесут знание чистое.
\vs Job 33:4 Дух Божий создал меня, и дыхание Вседержителя дало мне жизнь.
\vs Job 33:5 Если можешь, отвечай мне и стань передо мною.
\vs Job 33:6 Вот я, по желанию твоему, вместо Бога. Я образован также из брения;
\vs Job 33:7 поэтому страх передо мною не может смутить тебя, и рука моя не будет тяжела для тебя.
\vs Job 33:8 Ты говорил в уши мои, и я слышал звук слов:
\vs Job 33:9 чист я, без порока, невинен я, и нет во мне неправды;
\vs Job 33:10 а Он нашел обвинение против меня и считает меня Своим противником;
\vs Job 33:11 поставил ноги мои в колоду, наблюдает за всеми путями моими.
\vs Job 33:12 Вот в этом ты неправ, отвечаю тебе, потому что Бог выше человека.
\vs Job 33:13 Для чего тебе состязаться с Ним? Он не дает отчета ни в каких делах Своих.
\vs Job 33:14 Бог говорит однажды и, если того не заметят, в другой раз:
\vs Job 33:15 во сне, в ночном видении, когда сон находит на людей, во время дремоты на ложе.
\vs Job 33:16 Тогда Он открывает у человека ухо и запечатлевает Свое наставление,
\vs Job 33:17 чтобы отвести человека от какого-либо предприятия и удалить от него гордость,
\vs Job 33:18 чтобы отвести душу его от пропасти и жизнь его от поражения мечом.
\vs Job 33:19 Или он вразумляется болезнью на ложе своем и жестокою болью во всех костях своих,~---
\vs Job 33:20 и жизнь его отвращается от хлеба и душа его от любимой пищи.
\vs Job 33:21 Плоть на нем пропадает, так что ее не видно, и показываются кости его, которых не было видно.
\vs Job 33:22 И душа его приближается к могиле и жизнь его~--- к смерти.
\vs Job 33:23 Если есть у него Ангел-наставник, один из тысячи, чтобы показать человеку прямой \bibemph{путь} его,~---
\vs Job 33:24 \bibemph{Бог} умилосердится над ним и скажет: освободи его от могилы; Я нашел умилостивление.
\vs Job 33:25 Тогда тело его сделается свеж\acc{е}е, нежели в молодости; он возвратится к дням юности своей.
\vs Job 33:26 Будет молиться Богу, и Он~--- милостив к нему; с радостью взирает на лице его и возвращает человеку праведность его.
\vs Job 33:27 Он будет смотреть на людей и говорить: грешил я и превращал правду, и не воздано мне;
\vs Job 33:28 Он освободил душу мою от могилы, и жизнь моя видит свет.
\vs Job 33:29 Вот, все это делает Бог два-три раза с человеком,
\vs Job 33:30 чтобы отвести душу его от могилы и просветить его светом живых.
\vs Job 33:31 Внимай, Иов, слушай меня, молчи, и я буду говорить.
\vs Job 33:32 Если имеешь, что сказать, отвечай; говори, потому что я желал бы твоего оправдания;
\vs Job 33:33 если же нет, то слушай меня: молчи, и я научу тебя мудрости.
\vs Job 34:1 И продолжал Елиуй и сказал:
\vs Job 34:2 выслушайте, мудрые, речь мою, и приклоните ко мне ухо, рассудительные!
\vs Job 34:3 Ибо ухо разбирает слова, как гортань различает вкус в пище.
\vs Job 34:4 Установим между собою рассуждение и распознаем, что хорошо.
\vs Job 34:5 Вот, Иов сказал: я прав, но Бог лишил меня суда.
\vs Job 34:6 Должен ли я лгать на правду мою? Моя рана неисцелима без вины.
\vs Job 34:7 Есть ли такой человек, как Иов, который пьет глумление, как воду,
\vs Job 34:8 вступает в сообщество с делающими беззаконие и ходит с людьми нечестивыми?
\vs Job 34:9 Потому что он сказал: нет пользы для человека в благоугождении Богу.
\vs Job 34:10 Итак послушайте меня, мужи мудрые! Не может быть у Бога неправда или у Вседержителя неправосудие,
\vs Job 34:11 ибо Он по делам человека поступает с ним и по путям мужа воздает ему.
\vs Job 34:12 Истинно, Бог не делает неправды и Вседержитель не извращает суда.
\vs Job 34:13 Кто кроме Его промышляет о земле? И кто управляет всею вселенною?
\vs Job 34:14 Если бы Он обратил сердце Свое к Себе и взял к Себе дух ее и дыхание ее,~---
\vs Job 34:15 вдруг погибла бы всякая плоть, и человек возвратился бы в прах.
\vs Job 34:16 Итак, если ты имеешь разум, то слушай это и внимай словам моим.
\vs Job 34:17 Ненавидящий правду может ли владычествовать? И можешь ли ты обвинить Всеправедного?
\vs Job 34:18 Можно ли сказать царю: ты~--- нечестивец, и князьям: вы~--- беззаконники?
\vs Job 34:19 Но Он не смотрит и на лица князей и не предпочитает богатого бедному, потому что все они дело рук Его.
\vs Job 34:20 Внезапно они умирают; среди ночи народ возмутится, и они исчезают; и сильных изгоняют не силою.
\vs Job 34:21 Ибо очи Его над путями человека, и Он видит все шаги его.
\vs Job 34:22 Нет тьмы, ни тени смертной, где могли бы укрыться делающие беззаконие.
\vs Job 34:23 Потому Он уже не требует от человека, чтобы шел на суд с Богом.
\vs Job 34:24 Он сокрушает сильных без исследования и поставляет других на их места;
\vs Job 34:25 потому что Он делает известными дела их и низлагает их ночью, и они истребляются.
\vs Job 34:26 Он поражает их, как беззаконных людей, пред глазами других,
\vs Job 34:27 за то, что они отвратились от Него и не уразумели всех путей Его,
\vs Job 34:28 так что дошел до Него вопль бедных, и Он услышал стенание угнетенных.
\vs Job 34:29 Дарует ли Он тишину, кто может возмутить? скрывает ли Он лице Свое, кто может увидеть Его? Будет ли это для народа, или для одного человека,
\vs Job 34:30 чтобы не царствовал лицемер к соблазну народа.
\vs Job 34:31 К Богу должно говорить: я потерпел, больше не буду грешить.
\vs Job 34:32 А чего я не знаю, Ты научи меня; и если я сделал беззаконие, больше не буду.
\vs Job 34:33 По твоему ли \bibemph{рассуждению} Он должен воздавать? И как ты отвергаешь, то тебе следует избирать, а не мне; говори, что знаешь.
\vs Job 34:34 Люди разумные скажут мне, и муж мудрый, слушающий меня:
\vs Job 34:35 Иов не умно говорит, и слова его не со смыслом.
\vs Job 34:36 Я желал бы, чтобы Иов вполне был испытан, по ответам его, свойственным людям нечестивым.
\vs Job 34:37 Иначе он ко греху своему прибавит отступление, будет рукоплескать между нами и еще больше наговорит против Бога.
\vs Job 35:1 И продолжал Елиуй и сказал:
\vs Job 35:2 считаешь ли ты справедливым, что сказал: я правее Бога?
\vs Job 35:3 Ты сказал: что пользы мне? и какую прибыль я имел бы пред тем, как если бы я и грешил?
\vs Job 35:4 Я отвечу тебе и твоим друзьям с тобою:
\vs Job 35:5 взгляни на небо и смотри; воззри на облака, они выше тебя.
\vs Job 35:6 Если ты грешишь, что делаешь ты Ему? и если преступления твои умножаются, что причиняешь ты Ему?
\vs Job 35:7 Если ты праведен, что даешь Ему? или что получает Он от руки твоей?
\vs Job 35:8 Нечестие твое относится к человеку, как ты, и праведность твоя к сыну человеческому.
\vs Job 35:9 От множества притеснителей стонут притесняемые, и от руки сильных вопиют.
\vs Job 35:10 Но никто не говорит: где Бог, Творец мой, Который дает песни в ночи,
\vs Job 35:11 Который научает нас более, нежели скотов земных, и вразумляет нас более, нежели птиц небесных?
\vs Job 35:12 Там они вопиют, и Он не отвечает им, по причине гордости злых людей.
\vs Job 35:13 Но неправда, что Бог не слышит и Вседержитель не взирает на это.
\vs Job 35:14 Хотя ты сказал, что ты не видишь Его, но суд пред Ним, и~--- жди его.
\vs Job 35:15 Но ныне, потому что гнев Его не посетил его и он не познал его во всей строгости,
\vs Job 35:16 Иов и открыл легкомысленно уста свои и безрассудно расточает слова.
\vs Job 36:1 И продолжал Елиуй и сказал:
\vs Job 36:2 подожди меня немного, и я покажу тебе, что я имею еще что сказать за Бога.
\vs Job 36:3 Начну мои рассуждения издалека и воздам Создателю моему справедливость,
\vs Job 36:4 потому что слова мои точно не ложь: пред тобою~--- совершенный в познаниях.
\vs Job 36:5 Вот, Бог могуществен и не презирает сильного крепостью сердца;
\vs Job 36:6 Он не поддерживает нечестивых и воздает должное угнетенным;
\vs Job 36:7 Он не отвращает очей Своих от праведников, но с царями навсегда посаждает их на престоле, и они возвышаются.
\vs Job 36:8 Если же они окованы цепями и содержатся в узах бедствия,
\vs Job 36:9 то Он указывает им на дела их и на беззакония их, потому что умножились,
\vs Job 36:10 и открывает их ухо для вразумления и говорит им, чтоб они отстали от нечестия.
\vs Job 36:11 Если послушают и будут служить Ему, то проведут дни свои в благополучии и лета свои в радости;
\vs Job 36:12 если же не послушают, то погибнут от стрелы и умрут в неразумии.
\vs Job 36:13 Но лицемеры питают в сердце гнев и не взывают к Нему, когда Он заключает их в узы;
\vs Job 36:14 поэтому душа их умирает в молодости и жизнь их с блудниками.
\vs Job 36:15 Он спасает бедного от беды его и в угнетении открывает ухо его.
\vs Job 36:16 И тебя вывел бы Он из тесноты на простор, где нет стеснения, и поставляемое на стол твой было бы наполнено туком;
\vs Job 36:17 но ты преисполнен суждениями нечестивых: суждение и осуждение~--- близки.
\vs Job 36:18 Да не поразит тебя гнев \bibemph{Божий} наказанием! Большой выкуп не спасет тебя.
\vs Job 36:19 Даст ли Он какую цену твоему богатству? Нет,~--- ни золоту и никакому сокровищу.
\vs Job 36:20 Не желай той ночи, когда народы истребляются на своем месте.
\vs Job 36:21 Берегись, не склоняйся к нечестию, которое ты предпочел страданию.
\vs Job 36:22 Бог высок могуществом Своим, и кто такой, как Он, наставник?
\vs Job 36:23 Кто укажет Ему путь Его; кто может сказать: Ты поступаешь несправедливо?
\vs Job 36:24 Помни о том, чтобы превозносить дела его, которые люди видят.
\vs Job 36:25 Все люди могут видеть их; человек может усматривать их издали.
\vs Job 36:26 Вот, Бог велик, и мы не можем познать Его; число лет Его неисследимо.
\vs Job 36:27 Он собирает капли воды; они во множестве изливаются дождем:
\vs Job 36:28 из облаков каплют и изливаются обильно на людей.
\vs Job 36:29 Кто может также постигнуть протяжение облаков, треск шатра Его?
\vs Job 36:30 Вот, Он распространяет над ним свет Свой и покрывает дно моря.
\vs Job 36:31 Оттуда Он судит народы, дает пищу в изобилии.
\vs Job 36:32 Он сокрывает в дланях Своих молнию и повелевает ей, кого разить.
\vs Job 36:33 Треск ее дает знать о ней; скот также чувствует происходящее.
\vs Job 37:1 И от сего трепещет сердце мое и подвиглось с места своего.
\vs Job 37:2 Слушайте, слушайте голос Его и гром, исходящий из уст Его.
\vs Job 37:3 Под всем небом раскат его, и блистание его~--- до краев земли.
\vs Job 37:4 За ним гремит глас; гремит Он гласом величества Своего и не останавливает его, когда голос Его услышан.
\vs Job 37:5 Дивно гремит Бог гласом Своим, делает дела великие, для нас непостижимые.
\vs Job 37:6 Ибо снегу Он говорит: будь на земле; равно мелкий дождь и большой дождь в Его власти.
\vs Job 37:7 Он полагает печать на руку каждого человека, чтобы все люди знали дело Его.
\vs Job 37:8 Тогда зверь уходит в убежище и остается в своих логовищах.
\vs Job 37:9 От юга приходит буря, от севера~--- стужа.
\vs Job 37:10 От дуновения Божия происходит лед, и поверхность воды сжимается.
\vs Job 37:11 Также влагою Он наполняет тучи, и облака сыплют свет Его,
\vs Job 37:12 и они направляются по намерениям Его, чтоб исполнить то, что Он повелит им на лице обитаемой земли.
\vs Job 37:13 Он повелевает им идти или для наказания, или в благоволение, или для помилования.
\vs Job 37:14 Внимай сему, Иов; стой и разумевай чудные дела Божии.
\vs Job 37:15 Знаешь ли, как Бог располагает ими и повелевает свету блистать из облака Своего?
\vs Job 37:16 Разумеешь ли равновесие облаков, чудное дело Совершеннейшего в знании?
\vs Job 37:17 Как нагревается твоя одежда, когда Он успокаивает землю от юга?
\vs Job 37:18 Ты ли с Ним распростер небеса, твердые, как литое зеркало?
\vs Job 37:19 Научи нас, что сказать Ему? Мы в этой тьме ничего не можем сообразить.
\vs Job 37:20 Будет ли возвещено Ему, что я говорю? Сказал ли кто, что сказанное доносится Ему?
\vs Job 37:21 Теперь не видно яркого света в облаках, но пронесется ветер и расчистит их.
\vs Job 37:22 Светлая погода приходит от севера, и окрест Бога страшное великолепие.
\vs Job 37:23 Вседержитель! мы не постигаем Его. Он велик силою, судом и полнотою правосудия. Он \bibemph{никого} не угнетает.
\vs Job 37:24 Посему да благоговеют пред Ним люди, и да трепещут пред Ним все мудрые сердцем!
\vs Job 38:1 [Когда Елиуй перестал говорить,] Господь отвечал Иову из бури и сказал:
\vs Job 38:2 кто сей, омрачающий Провидение словами без смысла?
\vs Job 38:3 Препояшь ныне чресла твои, как муж: Я буду спрашивать тебя, и ты объясняй Мне:
\vs Job 38:4 где был ты, когда Я полагал основания земли? Скажи, если знаешь.
\vs Job 38:5 Кто положил меру ей, если знаешь? или кто протягивал по ней вервь?
\vs Job 38:6 На чем утверждены основания ее, или кто положил краеугольный камень ее,
\vs Job 38:7 при общем ликовании утренних звезд, когда все сыны Божии восклицали от радости?
\vs Job 38:8 Кто затворил море воротами, когда оно исторглось, вышло как бы из чрева,
\vs Job 38:9 когда Я облака сделал одеждою его и мглу пеленами его,
\vs Job 38:10 и утвердил ему Мое определение, и поставил запоры и ворота,
\vs Job 38:11 и сказал: доселе дойдешь и не перейдешь, и здесь предел надменным волнам твоим?
\vs Job 38:12 Давал ли ты когда в жизни своей приказания утру и указывал ли заре место ее,
\vs Job 38:13 чтобы она охватила края земли и стряхнула с нее нечестивых,
\vs Job 38:14 чтобы \bibemph{земля} изменилась, как глина под печатью, и стала, как разноцветная одежда,
\vs Job 38:15 и чтобы отнялся у нечестивых свет их и дерзкая рука их сокрушилась?
\vs Job 38:16 Нисходил ли ты во глубину моря и входил ли в исследование бездны?
\vs Job 38:17 Отворялись ли для тебя врата смерти, и видел ли ты врата тени смертной?
\vs Job 38:18 Обозрел ли ты широту земли? Объясни, если знаешь все это.
\vs Job 38:19 Где путь к жилищу света, и где место тьмы?
\vs Job 38:20 Ты, конечно, доходил до границ ее и знаешь стези к дому ее.
\vs Job 38:21 Ты знаешь это, потому что ты был уже тогда рожден, и число дней твоих очень велико.
\vs Job 38:22 Входил ли ты в хранилища снега и видел ли сокровищницы града,
\vs Job 38:23 которые берегу Я на время смутное, на день битвы и войны?
\vs Job 38:24 По какому пути разливается свет и разносится восточный ветер по земле?
\vs Job 38:25 Кто проводит протоки для излияния воды и путь для громоносной молнии,
\vs Job 38:26 чтобы шел дождь на землю безлюдную, на пустыню, где нет человека,
\vs Job 38:27 чтобы насыщать пустыню и степь и возбуждать травные зародыши к возрастанию?
\vs Job 38:28 Есть ли у дождя отец? или кто рождает капли росы?
\vs Job 38:29 Из чьего чрева выходит лед, и иней небесный,~--- кто рождает его?
\vs Job 38:30 Воды, как камень, крепнут, и поверхность бездны замерзает.
\vs Job 38:31 Можешь ли ты связать узел Хима и разрешить узы Кесиль?
\vs Job 38:32 Можешь ли выводить созвездия в свое время и вести Ас с ее детьми?
\vs Job 38:33 Знаешь ли ты уставы неба, можешь ли установить господство его на земле?
\vs Job 38:34 Можешь ли возвысить голос твой к облакам, чтобы вода в обилии покрыла тебя?
\vs Job 38:35 Можешь ли посылать молнии, и пойдут ли они и скажут ли тебе: вот мы?
\vs Job 38:36 Кто вложил мудрость в сердце, или кто дал смысл разуму?
\vs Job 38:37 Кто может расчислить облака своею мудростью и удержать сосуды неба,
\vs Job 38:38 когда пыль обращается в грязь и глыбы слипаются?
\vs Job 38:39 Ты ли ловишь добычу львице и насыщаешь молодых львов,
\vs Job 38:40 когда они лежат в берлогах или покоятся под тенью в засаде?
\vs Job 38:41 Кто приготовляет в\acc{о}рону корм его, когда птенцы его кричат к Богу, бродя без пищи?
\vs Job 39:1 Знаешь ли ты время, когда рождаются дикие козы на скалах, и замечал ли роды ланей?
\vs Job 39:2 можешь ли расчислить месяцы беременности их? и знаешь ли время родов их?
\vs Job 39:3 Они изгибаются, рождая детей своих, выбрасывая свои ноши;
\vs Job 39:4 дети их приходят в силу, растут на поле, уходят и не возвращаются к ним.
\vs Job 39:5 Кто пустил дикого осла на свободу, и кто разрешил узы онагру,
\vs Job 39:6 которому степь Я назначил домом и солончаки~--- жилищем?
\vs Job 39:7 Он посмевается городскому многолюдству и не слышит криков погонщика,
\vs Job 39:8 по горам ищет себе пищи и гоняется за всякою зеленью.
\vs Job 39:9 Захочет ли единорог служить тебе и переночует ли у яслей твоих?
\vs Job 39:10 Можешь ли веревкою привязать единорога к борозде, и станет ли он боронить за тобою поле?
\vs Job 39:11 Понадеешься ли на него, потому что у него сила велика, и предоставишь ли ему работу твою?
\vs Job 39:12 Поверишь ли ему, что он семена твои возвратит и сложит на гумно твое?
\vs Job 39:13 Ты ли дал красивые крылья павлину и перья и пух страусу?
\vs Job 39:14 Он оставляет яйца свои на земле, и на песке согревает их,
\vs Job 39:15 и забывает, что нога может раздавить их и полевой зверь может растоптать их;
\vs Job 39:16 он жесток к детям своим, как бы не своим, и не опасается, что труд его будет напрасен;
\vs Job 39:17 потому что Бог не дал ему мудрости и не уделил ему смысла;
\vs Job 39:18 а когда поднимется на высоту, посмевается коню и всаднику его.
\vs Job 39:19 Ты ли дал коню силу и облек шею его гривою?
\vs Job 39:20 Можешь ли ты испугать его, как саранчу? Храпение ноздрей его~--- ужас;
\vs Job 39:21 роет ногою землю и восхищается силою; идет навстречу оружию;
\vs Job 39:22 он смеется над опасностью и не робеет и не отворачивается от меча;
\vs Job 39:23 колчан звучит над ним, сверкает копье и дротик;
\vs Job 39:24 в порыве и ярости он глотает землю и не может стоять при звуке трубы;
\vs Job 39:25 при трубном звуке он издает голос: гу! гу! и издалека чует битву, громкие голоса вождей и крик.
\vs Job 39:26 Твоею ли мудростью летает ястреб и направляет крылья свои на полдень?
\vs Job 39:27 По твоему ли слову возносится орел и устрояет на высоте гнездо свое?
\vs Job 39:28 Он живет на скале и ночует на зубце утесов и на местах неприступных;
\vs Job 39:29 оттуда высматривает себе пищу: глаза его смотрят далеко;
\vs Job 39:30 птенцы его пьют кровь, и где труп, там и он.
\vs Job 39:31 И продолжал Господь и сказал Иову:
\vs Job 39:32 будет ли состязающийся со Вседержителем еще учить? Обличающий Бога пусть отвечает Ему.
\vs Job 39:33 И отвечал Иов Господу и сказал:
\vs Job 39:34 вот, я ничтожен; что буду я отвечать Тебе? Руку мою полагаю на уста мои.
\vs Job 39:35 Однажды я говорил,~--- теперь отвечать не буду, даже дважды, но более не буду.
\vs Job 40:1 И отвечал Господь Иову из бури и сказал:
\vs Job 40:2 препояшь, как муж, чресла твои: Я буду спрашивать тебя, а ты объясняй Мне.
\vs Job 40:3 Ты хочешь ниспровергнуть суд Мой, обвинить Меня, чтобы оправдать себя?
\vs Job 40:4 Такая ли у тебя мышца, как у Бога? И можешь ли возгреметь голосом, как Он?
\vs Job 40:5 Укрась же себя величием и славою, облекись в блеск и великолепие;
\vs Job 40:6 излей ярость гнева твоего, посмотри на все гордое и смири его;
\vs Job 40:7 взгляни на всех высокомерных и унизь их, и сокруши нечестивых на местах их;
\vs Job 40:8 зарой всех их в землю и лица их покрой тьмою.
\vs Job 40:9 Тогда и Я признаю, что десница твоя может спасать тебя.
\vs Job 40:10 Вот бегемот, которого Я создал, как и тебя; он ест траву, как вол;
\vs Job 40:11 вот, его сила в чреслах его и крепость его в мускулах чрева его;
\vs Job 40:12 поворачивает хвостом своим, как кедром; жилы же на бедрах его переплетены;
\vs Job 40:13 ноги у него, как медные трубы; кости у него, как железные прутья;
\vs Job 40:14 это~--- верх путей Божиих; только Сотворивший его может приблизить к нему меч Свой;
\vs Job 40:15 горы приносят ему пищу, и там все звери полевые играют;
\vs Job 40:16 он ложится под тенистыми деревьями, под кровом тростника и в болотах;
\vs Job 40:17 тенистые дерева покрывают его своею тенью; ивы при ручьях окружают его;
\vs Job 40:18 вот, он пьет из реки и не торопится; остается спокоен, хотя бы Иордан устремился ко рту его.
\vs Job 40:19 Возьмет ли кто его в глазах его и проколет ли ему нос багром?
\vs Job 40:20 Можешь ли ты удою вытащить левиафана и веревкою схватить за язык его?
\vs Job 40:21 вденешь ли кольцо в ноздри его? проколешь ли иглою челюсть его?
\vs Job 40:22 будет ли он много умолять тебя и будет ли говорить с тобою кротко?
\vs Job 40:23 сделает ли он договор с тобою, и возьмешь ли его навсегда себе в рабы?
\vs Job 40:24 станешь ли забавляться им, как птичкою, и свяжешь ли его для девочек твоих?
\vs Job 40:25 будут ли продавать его товарищи ловли, разделят ли его между Хананейскими купцами?
\vs Job 40:26 можешь ли пронзить кожу его копьем и голову его рыбачьею острогою?
\vs Job 40:27 Клади на него руку твою, и помни о борьбе: вперед не будешь.
\vs Job 41:1 Надежда тщетна: не упадешь ли от одного взгляда его?
\vs Job 41:2 Нет столь отважного, который осмелился бы потревожить его; кто же может устоять перед Моим лицем?
\vs Job 41:3 Кто предварил Меня, чтобы Мне воздавать ему? под всем небом все Мое.
\vs Job 41:4 Не умолчу о членах его, о силе и красивой соразмерности их.
\vs Job 41:5 Кто может открыть верх одежды его, кто подойдет к двойным челюстям его?
\vs Job 41:6 Кто может отворить двери лица его? круг зубов его~--- ужас;
\vs Job 41:7 крепкие щиты его~--- великолепие; они скреплены как бы твердою печатью;
\vs Job 41:8 один к другому прикасается близко, так что и воздух не проходит между ними;
\vs Job 41:9 один с другим лежат плотно, сцепились и не раздвигаются.
\vs Job 41:10 От его чихания показывается свет; глаза у него как ресницы зари;
\vs Job 41:11 из пасти его выходят пламенники, выскакивают огненные искры;
\vs Job 41:12 из ноздрей его выходит дым, как из кипящего горшка или котла.
\vs Job 41:13 Дыхание его раскаляет угли, и из пасти его выходит пламя.
\vs Job 41:14 На шее его обитает сила, и перед ним бежит ужас.
\vs Job 41:15 Мясистые части тела его сплочены между собою твердо, не дрогнут.
\vs Job 41:16 Сердце его твердо, как камень, и жестко, как нижний жернов.
\vs Job 41:17 Когда он поднимается, силачи в страхе, совсем теряются от ужаса.
\vs Job 41:18 Меч, коснувшийся его, не устоит, ни копье, ни дротик, ни латы.
\vs Job 41:19 Железо он считает за солому, медь~--- за гнилое дерево.
\vs Job 41:20 Дочь лука не обратит его в бегство; пращные камни обращаются для него в плеву.
\vs Job 41:21 Булава считается у него за соломину; свисту дротика он смеется.
\vs Job 41:22 Под ним острые камни, и он на острых камнях лежит в грязи.
\vs Job 41:23 Он кипятит пучину, как котел, и море претворяет в кипящую мазь;
\vs Job 41:24 оставляет за собою светящуюся стезю; бездна кажется сединою.
\vs Job 41:25 Нет на земле подобного ему; он сотворен бесстрашным;
\vs Job 41:26 на все высокое смотрит смело; он царь над всеми сынами гордости.
\vs Job 42:1 И отвечал Иов Господу и сказал:
\vs Job 42:2 знаю, что Ты все можешь, и что намерение Твое не может быть остановлено.
\vs Job 42:3 Кто сей, омрачающий Провидение, ничего не разумея?~--- Так, я говорил о том, чего не разумел, о делах чудных для меня, которых я не знал.
\vs Job 42:4 Выслушай, \bibemph{взывал я}, и я буду говорить, и что буду спрашивать у Тебя, объясни мне.
\vs Job 42:5 Я слышал о Тебе слухом уха; теперь же мои глаза видят Тебя;
\vs Job 42:6 поэтому я отрекаюсь и раскаиваюсь в прахе и пепле.
\rsbpar\vs Job 42:7 И было после того, как Господь сказал слова те Иову, сказал Господь Елифазу Феманитянину: горит гнев Мой на тебя и на двух друзей твоих за то, что вы говорили о Мне не так верно, как раб Мой Иов.
\vs Job 42:8 Итак возьмите себе семь тельцов и семь овнов и пойдите к рабу Моему Иову и принесите за себя жертву; и раб Мой Иов помолится за вас, ибо только лице его Я приму, дабы не отвергнуть вас за то, что вы говорили о Мне не так верно, как раб Мой Иов.
\vs Job 42:9 И пошли Елифаз Феманитянин и Вилдад Савхеянин и Софар Наамитянин, и сделали так, как Господь повелел им,~--- и Господь принял лице Иова.
\rsbpar\vs Job 42:10 И возвратил Господь потерю Иова, когда он помолился за друзей своих; и дал Господь Иову вдвое больше того, что он имел прежде.
\vs Job 42:11 Тогда пришли к нему все братья его и все сестры его и все прежние знакомые его, и ели с ним хлеб в доме его, и тужили с ним, и утешали его за все зло, которое Господь навел на него, и дали ему каждый по кесите и по золотому кольцу.
\rsbpar\vs Job 42:12 И благословил Бог последние дни Иова более, нежели прежние: у него было четырнадцать тысяч мелкого скота, шесть тысяч верблюдов, тысяча пар волов и тысяча ослиц.
\vs Job 42:13 И было у него семь сыновей и три дочери.
\vs Job 42:14 И нарек он имя первой Емима, имя второй~--- Кассия, а имя третьей~--- Керенгаппух.
\vs Job 42:15 И не было на всей земле таких прекрасных женщин, как дочери Иова, и дал им отец их наследство между братьями их.
\vs Job 42:16 После того Иов жил сто сорок лет, и видел сыновей своих и сыновей сыновних до четвертого рода;
\vs Job 42:17 и умер Иов в старости, насыщенный днями.\fns{В Славянской Библии к книге Иова имеется следующее добавление: <<Написано, что он опять восстанет с теми, коих воскресит Господь. О нем толкуется в Сирской книге, что жил он в земле Авситидийской на пределах Идумеи и Аравии: прежде же было имя ему Иовав. Взяв жену Аравитянку, родил сына, которому имя Еннон. Происходил он от отца Зарефа, сынов Исавовых сын, матери же Воссоры, так что был он пятым от Авраама. И сии цари, царствовавшие в Едоме, какою страною и он обладал: первый Валак, сын Веора, и имя городу его Деннава; после же Валака Иовав, называемый Иовом; после сего Ассом, игемон из Феманитской страны; после него Адад, сын Варада, поразивший Мадиама на поле Моава,~--- и имя городу его Гефем. Пришедшие же к нему друзья, Елифаз (сын Софана) от сынов Исавовых, царь Феманский, Валдад (сын Амнона Ховарского) савхейский властитель, Софар Минейский царь. (Феман сын Елифаза, игемон Идумеи. О нем говорится в книге Сирской, что жил в земле Авситидийской, около берегов Евфрата; прежде имя его было Иовав, отец же его был Зареф, от востока солнца.)>>.}

\bibbookdescr{Psa}{
  inline={Псалтирь\fns{У Евреев: <<Книга Хвалений>>.}},
  toc={Псалтирь},
  bookmark={Псалтирь},
  header={Псалтирь},
  %headerleft={},
  %headerright={},
  abbr={Пс}
}
\vs Psa 1:0 Псалом Давида.
\rsbpar\vs Psa 1:1 Блажен муж, который не ходит на совет нечестивых и не стоит на пути грешных и не сидит в собрании развратителей,
\vs Psa 1:2 но в законе Господа воля его, и о законе Его размышляет он день и ночь!
\vs Psa 1:3 И будет он как дерево, посаженное при потоках вод, которое приносит плод свой во время свое, и лист которого не вянет; и во всем, что он ни делает, успеет.
\vs Psa 1:4 Не так~--- нечестивые, [не так]: но они~--- как прах, возметаемый ветром [с лица земли].
\vs Psa 1:5 Потому не устоят\fns{В славянском переводе: Сего ради не воскреснут\dots} нечестивые на суде, и грешники~--- в собрании праведных.
\vs Psa 1:6 Ибо знает Господь путь праведных, а путь нечестивых погибнет.
\vs Psa 2:0 Псалом Давида.
\rsbpar\vs Psa 2:1 Зачем мятутся народы, и племена замышляют тщетное?
\vs Psa 2:2 Восстают цари земли, и князья совещаются вместе против Господа и против Помазанника Его.
\vs Psa 2:3 <<Расторгнем узы их, и свергнем с себя оковы их>>.
\vs Psa 2:4 Живущий на небесах посмеется, Господь поругается им.
\vs Psa 2:5 Тогда скажет им во гневе Своем и яростью Своею приведет их в смятение:
\vs Psa 2:6 <<Я помазал Царя Моего над Сионом, святою горою Моею\fns{6-й стих по переводу 70-ти: Я поставлен от Него Царем над Сионом, святою горою Его.};
\vs Psa 2:7 возвещу определение: Господь сказал Мне: Ты Сын Мой; Я ныне родил Тебя;
\vs Psa 2:8 проси у Меня, и дам народы в наследие Тебе и пределы земли во владение Тебе;
\vs Psa 2:9 Ты поразишь их жезлом железным; сокрушишь их, как сосуд горшечника>>.
\vs Psa 2:10 Итак вразумитесь, цари; научитесь, судьи земли!
\vs Psa 2:11 Служите Господу со страхом и радуйтесь [пред Ним] с трепетом.
\vs Psa 2:12 Почтите Сына, чтобы Он не прогневался, и чтобы вам не погибнуть в пути \bibemph{вашем}, ибо гнев Его возгорится вскоре. Блаженны все, уповающие на Него.
\vs Psa 3:1 Псалом Давида, когда он бежал от Авессалома, сына своего.
\rsbpar\vs Psa 3:2 Господи! как умножились враги мои! Многие восстают на меня;
\vs Psa 3:3 многие говорят душе моей: <<нет ему спасения в Боге>>.
\vs Psa 3:4 Но Ты, Господи, щит предо мною, слава моя, и Ты возносишь голову мою.
\vs Psa 3:5 Гласом моим взываю к Господу, и Он слышит меня со святой горы Своей.
\vs Psa 3:6 Ложусь я, сплю и встаю, ибо Господь защищает меня.
\vs Psa 3:7 Не убоюсь тем народа, которые со всех сторон ополчились на меня.
\vs Psa 3:8 Восстань, Господи! спаси меня, Боже мой! ибо Ты поражаешь в ланиту всех врагов моих; сокрушаешь зубы нечестивых.
\vs Psa 3:9 От Господа спасение. Над народом Твоим благословение Твое.
\vs Psa 4:1 Начальнику хора. На струнных \bibemph{орудиях}. Псалом Давида.
\rsbpar\vs Psa 4:2 Когда я взываю, услышь меня, Боже правды моей! В тесноте Ты давал мне простор. Помилуй меня и услышь молитву мою.
\vs Psa 4:3 Сыны мужей! доколе слава моя будет в поругании? доколе будете любить суету и искать лжи?
\vs Psa 4:4 Знайте, что Господь отделил для Себя святаго Своего; Господь слышит, когда я призываю Его.
\vs Psa 4:5 Гневаясь, не согрешайте: размыслите в сердцах ваших на ложах ваших, и утишитесь;
\vs Psa 4:6 приносите жертвы правды и уповайте на Господа.
\vs Psa 4:7 Многие говорят: <<кто покажет нам благо?>> Яви нам свет лица Твоего, Господи!
\vs Psa 4:8 Ты исполнил сердце мое веселием с того времени, как у них хлеб и вино [и елей] умножились.
\vs Psa 4:9 Спокойно ложусь я и сплю, ибо Ты, Господи, един даешь мне жить в безопасности.
\vs Psa 5:1 Начальнику хора. На духовых \bibemph{орудиях}. Псалом Давида.
\rsbpar\vs Psa 5:2 Услышь, Господи, слова мои, уразумей помышления мои.
\vs Psa 5:3 Внемли гласу вопля моего, Царь мой и Бог мой! ибо я к Тебе молюсь.
\vs Psa 5:4 Господи! рано услышь голос мой,~--- рано предстану пред Тобою, и буду ожидать,
\vs Psa 5:5 ибо Ты Бог, не любящий беззакония; у Тебя не водворится злой;
\vs Psa 5:6 нечестивые не пребудут пред очами Твоими: Ты ненавидишь всех, делающих беззаконие.
\vs Psa 5:7 Ты погубишь говорящих ложь; кровожадного и коварного гнушается Господь.
\vs Psa 5:8 А я, по множеству милости Твоей, войду в дом Твой, поклонюсь святому храму Твоему в страхе Твоем.
\vs Psa 5:9 Господи! путеводи меня в правде Твоей, ради врагов моих; уровняй предо мною путь Твой.
\vs Psa 5:10 Ибо нет в устах их истины: сердце их~--- пагуба, гортань их~--- открытый гроб, языком своим льстят.
\vs Psa 5:11 Осуди их, Боже, да падут они от замыслов своих; по множеству нечестия их, отвергни их, ибо они возмутились против Тебя.
\vs Psa 5:12 И возрадуются все уповающие на Тебя, вечно будут ликовать, и Ты будешь покровительствовать им; и будут хвалиться Тобою любящие имя Твое.
\vs Psa 5:13 Ибо Ты благословляешь праведника, Господи; благоволением, как щитом, венчаешь его.
\vs Psa 6:1 Начальнику хора. На восьмиструнном. Псалом Давида.
\rsbpar\vs Psa 6:2 Господи! не в ярости Твоей обличай меня и не во гневе Твоем наказывай меня.
\vs Psa 6:3 Помилуй меня, Господи, ибо я немощен; исцели меня, Господи, ибо кости мои потрясены;
\vs Psa 6:4 и душа моя сильно потрясена; Ты же, Господи, доколе?
\vs Psa 6:5 Обратись, Господи, избавь душу мою, спаси меня ради милости Твоей,
\vs Psa 6:6 ибо в смерти нет памятования о Тебе: во гробе кто будет славить Тебя?
\vs Psa 6:7 Утомлен я воздыханиями моими: каждую ночь омываю ложе мое, слезами моими омочаю постель мою.
\vs Psa 6:8 Иссохло от печали око мое, обветшало от всех врагов моих.
\vs Psa 6:9 Удалитесь от меня все, делающие беззаконие, ибо услышал Господь голос плача моего,
\vs Psa 6:10 услышал Господь моление мое; Господь примет молитву мою.
\vs Psa 6:11 Да будут постыжены и жестоко поражены все враги мои; да возвратятся и постыдятся мгновенно.
\vs Psa 7:1 Плачевная песнь, которую Давид воспел Господу по делу Хуса, из племени Вениаминова.
\rsbpar\vs Psa 7:2 Господи, Боже мой! на Тебя я уповаю; спаси меня от всех гонителей моих и избавь меня;
\vs Psa 7:3 да не исторгнет он, подобно льву, души моей, терзая, когда нет избавляющего [и спасающего].
\vs Psa 7:4 Господи, Боже мой! если я что сделал, если есть неправда в руках моих,
\vs Psa 7:5 если я платил злом тому, кто был со мною в мире,~--- я, который спасал даже того, кто без причины стал моим врагом,~---
\vs Psa 7:6 то пусть враг преследует душу мою и настигнет, пусть втопчет в землю жизнь мою, и славу мою повергнет в прах.
\vs Psa 7:7 Восстань, Господи, во гневе Твоем; подвигнись против неистовства врагов моих, пробудись для меня на суд, который Ты заповедал,~---
\vs Psa 7:8 сонм людей станет вокруг Тебя; над ним поднимись на высоту.
\vs Psa 7:9 Господь судит народы. Суди меня, Господи, по правде моей и по непорочности моей во мне.
\vs Psa 7:10 Да прекратится злоба нечестивых, а праведника подкрепи, ибо Ты испытуешь сердца и утробы, праведный Боже!
\vs Psa 7:11 Щит мой в Боге, спасающем правых сердцем.
\vs Psa 7:12 Бог~--- судия праведный, [крепкий и долготерпеливый,] и Бог, всякий день строго взыскивающий,
\vs Psa 7:13 если \bibemph{кто} не обращается. Он изощряет Свой меч, напрягает лук Свой и направляет его,
\vs Psa 7:14 приготовляет для него сосуды смерти, стрелы Свои делает палящими.
\vs Psa 7:15 Вот, \bibemph{нечестивый} зачал неправду, был чреват злобою и родил себе ложь;
\vs Psa 7:16 рыл ров, и выкопал его, и упал в яму, которую приготовил:
\vs Psa 7:17 злоба его обратится на его голову, и злодейство его упадет на его темя.
\vs Psa 7:18 Славлю Господа по правде Его и пою имени Господа Всевышнего.
\vs Psa 8:1 Начальнику хора. На Гефском \bibemph{орудии}. Псалом Давида.
\rsbpar\vs Psa 8:2 Господи, Боже наш! как величественно имя Твое по всей земле! Слава Твоя простирается превыше небес!
\vs Psa 8:3 Из уст младенцев и грудных детей Ты устроил хвалу, ради врагов Твоих, дабы сделать безмолвным врага и мстителя.
\vs Psa 8:4 Когда взираю я на небеса Твои~--- дело Твоих перстов, на луну и звезды, которые Ты поставил,
\vs Psa 8:5 то чт\acc{о} \bibemph{есть} человек, что Ты помнишь его, и сын человеческий, что Ты посещаешь его?
\vs Psa 8:6 Не много Ты умалил его пред Ангелами: славою и честью увенчал его;
\vs Psa 8:7 поставил его владыкою над делами рук Твоих; всё положил под ноги его:
\vs Psa 8:8 овец и волов всех, и также полевых зверей,
\vs Psa 8:9 птиц небесных и рыб морских, все, преходящее морскими стезями.
\vs Psa 8:10 Господи, Боже наш! Как величественно имя Твое по всей земле!
\vs Psa 9:1 Начальнику хора. По смерти Лабена. Псалом Давида.
\rsbpar\vs Psa 9:2 Буду славить [Тебя], Господи, всем сердцем моим, возвещать все чудеса Твои.
\vs Psa 9:3 Буду радоваться и торжествовать о Тебе, петь имени Твоему, Всевышний.
\vs Psa 9:4 Когда враги мои обращены назад, то преткнутся и погибнут пред лицем Твоим,
\vs Psa 9:5 ибо Ты производил мой суд и мою тяжбу; Ты воссел на престоле, Судия праведный.
\vs Psa 9:6 Ты вознегодовал на народы, погубил нечестивого, имя их изгладил на веки и веки.
\vs Psa 9:7 У врага совсем не стало оружия, и город\acc{а} Ты разрушил; погибла память их с ними.
\vs Psa 9:8 Но Господь пребывает вовек; Он приготовил для суда престол Свой,
\vs Psa 9:9 и Он будет судить вселенную по правде, совершит суд над народами по правоте.
\vs Psa 9:10 И будет Господь прибежищем угнетенному, прибежищем во времена скорби;
\vs Psa 9:11 и будут уповать на Тебя знающие имя Твое, потому что Ты не оставляешь ищущих Тебя, Господи.
\vs Psa 9:12 Пойте Господу, живущему на Сионе, возвещайте между народами дела Его,
\vs Psa 9:13 ибо Он взыскивает за кровь; помнит их, не забывает вопля угнетенных.
\vs Psa 9:14 Помилуй меня, Господи; воззри на страдание мое от ненавидящих меня,~--- Ты, Который возносишь меня от врат смерти,
\vs Psa 9:15 чтобы я возвещал все хвалы Твои во вратах дщери Сионовой: буду радоваться о спасении Твоем.
\vs Psa 9:16 Обрушились народы в яму, которую выкопали; в сети, которую скрыли они, запуталась нога их.
\vs Psa 9:17 Познан был Господь по суду, который Он совершил; нечестивый уловлен делами рук своих.
\vs Psa 9:18 Да обратятся нечестивые в ад,~--- все народы, забывающие Бога.
\vs Psa 9:19 Ибо не навсегда забыт будет нищий, и надежда бедных не до конца погибнет.
\vs Psa 9:20 Восстань, Господи, да не преобладает человек, да судятся народы пред лицем Твоим.
\vs Psa 9:21 Наведи, Господи, страх на них; да знают народы, что человеки они.
\vs Psa 9:22 Для чего, Господи, стоишь вдали, скрываешь Себя во время скорби?
\vs Psa 9:23 По гордости своей нечестивый преследует бедного: да уловятся они ухищрениями, которые сами вымышляют.
\vs Psa 9:24 Ибо нечестивый хвалится похотью души своей; корыстолюбец ублажает себя.
\vs Psa 9:25 В надмении своем нечестивый пренебрегает Господа: <<не взыщет>>; во всех помыслах его: <<нет Бога!>>
\vs Psa 9:26 Во всякое время пути его гибельны; суды Твои далеки для него; на всех врагов своих он смотрит с пренебрежением;
\vs Psa 9:27 говорит в сердце своем: <<не поколеблюсь; в род и род не приключится \bibemph{мне} зла>>;
\vs Psa 9:28 уста его полны проклятия, коварства и лжи; под языком~--- его мучение и пагуба;
\vs Psa 9:29 сидит в засаде за двором, в потаенных местах убивает невинного; глаза его подсматривают за бедным;
\vs Psa 9:30 подстерегает в потаенном месте, как лев в логовище; подстерегает в засаде, чтобы схватить бедного; хватает бедного, увлекая в сети свои;
\vs Psa 9:31 сгибается, прилегает,~--- и бедные падают в сильные когти его;
\vs Psa 9:32 говорит в сердце своем: <<забыл Бог, закрыл лице Свое, не увидит никогда>>.
\vs Psa 9:33 Восстань, Господи, Боже [мой], вознеси руку Твою, не забудь угнетенных [Твоих до конца].
\vs Psa 9:34 Зачем нечестивый пренебрегает Бога, говоря в сердце своем: <<Ты не взыщешь>>?
\vs Psa 9:35 Ты видишь, ибо Ты взираешь на обиды и притеснения, чтобы воздать Твоею рукою. Тебе предает себя бедный; сироте Ты помощник.
\vs Psa 9:36 Сокруши мышцу нечестивому и злому, так чтобы искать и не найти его нечестия.
\vs Psa 9:37 Господь~--- царь на веки, навсегда; исчезнут язычники с земли Его.
\vs Psa 9:38 Господи! Ты слышишь желания смиренных; укрепи сердце их; открой ухо Твое,
\vs Psa 9:39 чтобы дать суд сироте и угнетенному, да не устрашает более человек на земле.
\vs Psa 10:0 Начальнику хора. Псалом Давида.
\rsbpar\vs Psa 10:1 На Господа уповаю; как же вы говорите душе моей: <<улетай на гору вашу, \bibemph{как} птица>>?
\vs Psa 10:2 Ибо вот, нечестивые натянули лук, стрелу свою приложили к тетиве, чтобы во тьме стрелять в правых сердцем.
\vs Psa 10:3 Когда разрушены основания, что сделает праведник?
\vs Psa 10:4 Господь во святом храме Своем, Господь,~--- престол Его на небесах, очи Его зрят [на нищего]; вежды Его испытывают сынов человеческих.
\vs Psa 10:5 Господь испытывает праведного, а нечестивого и любящего насилие ненавидит душа Его.
\vs Psa 10:6 Дождем прольет Он на нечестивых горящие угли, огонь и серу; и палящий ветер~--- их доля из чаши;
\vs Psa 10:7 ибо Господь праведен, любит правду; лице Его видит праведника.
\vs Psa 11:1 Начальнику хора. На восьмиструнном. Псалом Давида.
\rsbpar\vs Psa 11:2 Спаси [меня], Господи, ибо не стало праведного, ибо нет верных между сынами человеческими.
\vs Psa 11:3 Ложь говорит каждый своему ближнему; уста льстивы, говорят от сердца притворного.
\vs Psa 11:4 Истребит Господь все уста льстивые, язык велеречивый,
\vs Psa 11:5 \bibemph{тех}, которые говорят: <<языком нашим пересилим, уста наши с нами; кто нам господин>>?
\vs Psa 11:6 Ради страдания нищих и воздыхания бедных ныне восстану, говорит Господь, поставлю в безопасности того, кого уловить хотят.
\vs Psa 11:7 Слова Господни~--- слова чистые, серебро, очищенное от земли в горниле, семь раз переплавленное.
\vs Psa 11:8 Ты, Господи, сохранишь их, соблюдешь от рода сего вовек.
\vs Psa 11:9 Повсюду ходят нечестивые, когда ничтожные из сынов человеческих возвысились.
\vs Psa 12:1 Начальнику хора. Псалом Давида.
\rsbpar\vs Psa 12:2 Доколе, Господи, будешь забывать меня вконец, доколе будешь скрывать лице Твое от меня?
\vs Psa 12:3 Доколе мне слагать советы в душе моей, скорбь в сердце моем день [и ночь]? Доколе врагу моему возноситься надо мною?
\vs Psa 12:4 Призри, услышь меня, Господи Боже мой! Просвети очи мои, да не усну я \bibemph{сном} смертным;
\vs Psa 12:5 да не скажет враг мой: <<я одолел его>>. Да не возрадуются гонители мои, если я поколеблюсь.
\vs Psa 12:6 Я же уповаю на милость Твою; сердце мое возрадуется о спасении Твоем; воспою Господу, облагодетельствовавшему меня, [и буду петь имени Господа Всевышнего].
\vs Psa 13:0 Начальнику хора. Псалом Давида.
\rsbpar\vs Psa 13:1 Сказал безумец в сердце своем: <<нет Бога>>. Они развратились, совершили гнусные дела; нет делающего добро.
\vs Psa 13:2 Господь с небес призрел на сынов человеческих, чтобы видеть, есть ли разумеющий, ищущий Бога.
\vs Psa 13:3 Все уклонились, сделались равно непотребными; нет делающего добро, нет ни одного.
\vs Psa 13:4 Неужели не вразумятся все, делающие беззаконие, съедающие народ мой, \bibemph{как} едят хлеб, и не призывающие Господа?
\vs Psa 13:5 Там убоятся они страха, [где нет страха,] ибо Бог в роде праведных.
\vs Psa 13:6 Вы посмеялись над мыслью нищего, что Господь упование его.
\vs Psa 13:7 <<Кто даст с Сиона спасение Израилю!>> Когда Господь возвратит пленение народа Своего, тогда возрадуется Иаков и возвеселится Израиль.
\vs Psa 14:0 Псалом Давида.
\rsbpar\vs Psa 14:1 Господи! кто может пребывать в жилище Твоем? кто может обитать на святой горе Твоей?
\vs Psa 14:2 Тот, кто ходит непорочно и делает правду, и говорит истину в сердце своем;
\vs Psa 14:3 кто не клевещет языком своим, не делает искреннему своему зла и не принимает поношения на ближнего своего;
\vs Psa 14:4 тот, в глазах которого презрен отверженный, но который боящихся Господа славит; кто клянется, \bibemph{хотя бы} злому, и не изменяет;
\vs Psa 14:5 кто серебра своего не отдает в рост и не принимает даров против невинного. Поступающий так не поколеблется вовек.
\vs Psa 15:0 Песнь Давида.
\rsbpar\vs Psa 15:1 Храни меня, Боже, ибо я на Тебя уповаю.
\vs Psa 15:2 Я сказал Господу: Ты~--- Господь мой; блага мои Тебе не нужны.
\vs Psa 15:3 К святым, которые на земле, и к дивным \bibemph{Твоим}~--- к ним все желание мое.
\vs Psa 15:4 Пусть умножаются скорби у тех, которые текут к \bibemph{богу} чужому; я не возлию кровавых возлияний их и не помяну имен их устами моими.
\vs Psa 15:5 Господь есть часть наследия моего и чаши моей. Ты держишь жребий мой.
\vs Psa 15:6 Межи мои прошли по прекрасным \bibemph{местам}, и наследие мое приятно для меня.
\vs Psa 15:7 Благословлю Господа, вразумившего меня; даже и ночью учит меня внутренность моя.
\vs Psa 15:8 Всегда видел я пред собою Господа, ибо Он одесную меня; не поколеблюсь.
\vs Psa 15:9 Оттого возрадовалось сердце мое и возвеселился язык мой; даже и плоть моя успокоится в уповании,
\vs Psa 15:10 ибо Ты не оставишь души моей в аде и не дашь святому Твоему увидеть тление,
\vs Psa 15:11 Ты укажешь мне путь жизни: полнота радостей пред лицем Твоим, блаженство в деснице Твоей вовек.
\vs Psa 16:0 Молитва Давида.
\rsbpar\vs Psa 16:1 Услышь, Господи, правду [мою], внемли воплю моему, прими мольбу из уст нелживых.
\vs Psa 16:2 От Твоего лица суд мне да изыдет; да воззрят очи Твои на правоту.
\vs Psa 16:3 Ты испытал сердце мое, посетил меня ночью, искусил меня и ничего не нашел; от мыслей моих не отступают уста мои.
\vs Psa 16:4 В делах человеческих, по слову уст Твоих, я охранял себя от путей притеснителя.
\vs Psa 16:5 Утверди шаги мои на путях Твоих, да не колеблются стопы мои.
\vs Psa 16:6 К Тебе взываю я, ибо Ты услышишь меня, Боже; приклони ухо Твое ко мне, услышь слова мои.
\vs Psa 16:7 Яви дивную милость Твою, Спаситель уповающих [на Тебя] от противящихся деснице Твоей.
\vs Psa 16:8 Храни меня, как зеницу ока; в тени крыл Твоих укрой меня
\vs Psa 16:9 от лица нечестивых, нападающих на меня,~--- от врагов души моей, окружающих меня:
\vs Psa 16:10 они заключились в туке своем, надменно говорят устами своими.
\vs Psa 16:11 На всяком шагу нашем ныне окружают нас; они устремили глаза свои, чтобы низложить \bibemph{меня} на землю;
\vs Psa 16:12 они подобны льву, жаждущему добычи, подобны скимну, сидящему в местах скрытных.
\vs Psa 16:13 Восстань, Господи, предупреди их, низложи их. Избавь душу мою от нечестивого мечом Твоим,
\vs Psa 16:14 от людей~--- рукою Твоею, Господи, от людей мира, которых удел в \bibemph{этой} жизни, которых чрево Ты наполняешь из сокровищниц Твоих; сыновья их сыты и оставят остаток детям своим.
\vs Psa 16:15 А я в правде буду взирать на лице Твое; пробудившись, буду насыщаться образом Твоим.
\vs Psa 17:1 Начальнику хора. Раба Господня Давида, который произнес слова песни сей к Господу, когда Господь избавил его от рук всех врагов его и от руки Саула. И он сказал:
\rsbpar\vs Psa 17:2 Возлюблю тебя, Господи, крепость моя!
\vs Psa 17:3 Господь~--- твердыня моя и прибежище мое, Избавитель мой, Бог мой,~--- скала моя; на Него я уповаю; щит мой, рог спасения моего и убежище мое.
\vs Psa 17:4 Призову достопоклоняемого Господа и от врагов моих спасусь.
\vs Psa 17:5 Объяли меня муки смертные, и потоки беззакония устрашили меня;
\vs Psa 17:6 цепи ада облегли меня, и сети смерти опутали меня.
\vs Psa 17:7 В тесноте моей я призвал Господа и к Богу моему воззвал. И Он услышал от [святаго] чертога Своего голос мой, и вопль мой дошел до слуха Его.
\vs Psa 17:8 Потряслась и всколебалась земля, дрогнули и подвиглись основания гор, ибо разгневался [Бог];
\vs Psa 17:9 поднялся дым от гнева Его и из уст Его огонь поядающий; горячие угли \bibemph{сыпались} от Него.
\vs Psa 17:10 Наклонил Он небеса и сошел,~--- и мрак под ногами Его.
\vs Psa 17:11 И воссел на Херувимов и полетел, и понесся на крыльях ветра.
\vs Psa 17:12 И мрак сделал покровом Своим, сению вокруг Себя мрак вод, облаков воздушных.
\vs Psa 17:13 От блистания пред Ним бежали облака Его, град и угли огненные.
\vs Psa 17:14 Возгремел на небесах Господь, и Всевышний дал глас Свой, град и угли огненные.
\vs Psa 17:15 Пустил стрелы Свои и рассеял их, множество молний, и рассыпал их.
\vs Psa 17:16 И явились источники вод, и открылись основания вселенной от грозного \bibemph{гласа} Твоего, Господи, от дуновения духа гнева Твоего.
\vs Psa 17:17 Он простер \bibemph{руку} с высоты и взял меня, и извлек меня из вод многих;
\vs Psa 17:18 избавил меня от врага моего сильного и от ненавидящих меня, которые были сильнее меня.
\vs Psa 17:19 Они восстали на меня в день бедствия моего, но Господь был мне опорою.
\vs Psa 17:20 Он вывел меня на пространное место и избавил меня, ибо Он благоволит ко мне.
\vs Psa 17:21 Воздал мне Господь по правде моей, по чистоте рук моих вознаградил меня,
\vs Psa 17:22 ибо я хранил пути Господни и не был нечестивым пред Богом моим;
\vs Psa 17:23 ибо все заповеди Его предо мною, и от уставов Его я не отступал.
\vs Psa 17:24 Я был непорочен пред Ним и остерегался, чтобы не согрешить мне;
\vs Psa 17:25 и воздал мне Господь по правде моей, по чистоте рук моих пред очами Его.
\vs Psa 17:26 С милостивым Ты поступаешь милостиво, с мужем искренним~--- искренно,
\vs Psa 17:27 с чистым~--- чисто, а с лукавым~--- по лукавству его,
\vs Psa 17:28 ибо Ты людей угнетенных спасаешь, а очи надменные унижаешь.
\vs Psa 17:29 Ты возжигаешь светильник мой, Господи; Бог мой просвещает тьму мою.
\vs Psa 17:30 С Тобою я поражаю войско, с Богом моим восхожу на стену.
\vs Psa 17:31 Бог!~--- Непорочен путь Его, чисто слово Господа; щит Он для всех, уповающих на Него.
\vs Psa 17:32 Ибо кто Бог, кроме Господа, и кто защита, кроме Бога нашего?
\vs Psa 17:33 Бог препоясывает меня силою и устрояет мне верный путь;
\vs Psa 17:34 делает ноги мои, как оленьи, и на высотах моих поставляет меня;
\vs Psa 17:35 научает руки мои брани, и мышцы мои сокрушают медный лук.
\vs Psa 17:36 Ты дал мне щит спасения Твоего, и десница Твоя поддерживает меня, и милость Твоя возвеличивает меня.
\vs Psa 17:37 Ты расширяешь шаг мой подо мною, и не колеблются ноги мои.
\vs Psa 17:38 Я преследую врагов моих и настигаю их, и не возвращаюсь, доколе не истреблю их;
\vs Psa 17:39 поражаю их, и они не могут встать, падают под ноги мои,
\vs Psa 17:40 ибо Ты препоясал меня силою для войны и низложил под ноги мои восставших на меня;
\vs Psa 17:41 Ты обратил ко мне тыл врагов моих, и я истребляю ненавидящих меня:
\vs Psa 17:42 они вопиют, но нет спасающего; ко Господу,~--- но Он не внемлет им;
\vs Psa 17:43 я рассеваю их, как прах пред лицем ветра, как уличную грязь попираю их.
\vs Psa 17:44 Ты избавил меня от мятежа народа, поставил меня главою иноплеменников; народ, которого я не знал, служит мне;
\vs Psa 17:45 по одному слуху о мне повинуются мне; иноплеменники ласкательствуют предо мною;
\vs Psa 17:46 иноплеменники бледнеют и трепещут в укреплениях своих.
\vs Psa 17:47 Жив Господь и благословен защитник мой! Да будет превознесен Бог спасения моего,
\vs Psa 17:48 Бог, мстящий за меня и покоряющий мне народы,
\vs Psa 17:49 и избавляющий меня от врагов моих! Ты вознес меня над восстающими против меня и от человека жестокого избавил меня.
\vs Psa 17:50 За то буду славить Тебя, Господи, между иноплеменниками и буду петь имени Твоему,
\vs Psa 17:51 величественно спасающий царя и творящий милость помазаннику Твоему Давиду и потомству его во веки.
\vs Psa 18:1 Начальнику хора. Псалом Давида.
\rsbpar\vs Psa 18:2 Небеса проповедуют славу Божию, и о делах рук Его вещает твердь.
\vs Psa 18:3 День дню передает речь, и ночь ночи открывает знание.
\vs Psa 18:4 Нет языка, и нет наречия, где не слышался бы голос их.
\vs Psa 18:5 По всей земле проходит звук их, и до пределов вселенной слов\acc{а} их. Он поставил в них жилище солнцу,
\vs Psa 18:6 и оно выходит, как жених из брачного чертога своего, радуется, как исполин, пробежать поприще:
\vs Psa 18:7 от края небес исход его, и шествие его до края их, и ничто не укрыто от теплоты его.
\vs Psa 18:8 Закон Господа совершен, укрепляет душу; откровение Господа верно, умудряет простых.
\vs Psa 18:9 Повеления Господа праведны, веселят сердце; заповедь Господа светла, просвещает очи.
\vs Psa 18:10 Страх Господень чист, пребывает вовек. Суды Господни истина, все праведны;
\vs Psa 18:11 они вожделеннее золота и даже множества золота чистого, слаще меда и капель сота;
\vs Psa 18:12 и раб Твой охраняется ими, в соблюдении их великая награда.
\vs Psa 18:13 Кто усмотрит погрешности свои? От тайных \bibemph{моих} очисти меня
\vs Psa 18:14 и от умышленных удержи раба Твоего, чтобы не возобладали мною. Тогда я буду непорочен и чист от великого развращения.
\vs Psa 18:15 Да будут слова уст моих и помышление сердца моего благоугодны пред Тобою, Господи, твердыня моя и Избавитель мой!
\vs Psa 19:1 Начальнику хора. Псалом Давида.
\rsbpar\vs Psa 19:2 Да услышит тебя Господь в день печали, да защитит тебя имя Бога Иаковлева.
\vs Psa 19:3 Да пошлет тебе помощь из Святилища и с Сиона да подкрепит тебя.
\vs Psa 19:4 Да воспомянет все жертвоприношения твои и всесожжение твое да соделает тучным.
\vs Psa 19:5 Да даст тебе [Господь] по сердцу твоему и все намерения твои да исполнит.
\vs Psa 19:6 Мы возрадуемся о спасении твоем и во имя Бога нашего поднимем знамя. Да исполнит Господь все прошения твои.
\vs Psa 19:7 Ныне познал я, что Господь спасает помазанника Своего, отвечает ему со святых небес Своих могуществом спасающей десницы Своей.
\vs Psa 19:8 Иные колесницами, иные конями, а мы именем Господа Бога нашего хвалимся:
\vs Psa 19:9 они поколебались и пали, а мы встали и стоим прямо.
\vs Psa 19:10 Господи! спаси царя и услышь нас, когда будем взывать [к Тебе].
\vs Psa 20:1 Начальнику хора. Псалом Давида.
\rsbpar\vs Psa 20:2 Господи! силою Твоею веселится царь и о спасении Твоем безмерно радуется.
\vs Psa 20:3 Ты дал ему, чего желало сердце его, и прошения уст его не отринул,
\vs Psa 20:4 ибо Ты встретил его благословениями благости, возложил на голову его венец из чистого золота.
\vs Psa 20:5 Он просил у Тебя жизни; Ты дал ему долгоденствие на век и век.
\vs Psa 20:6 Велика слава его в спасении Твоем; Ты возложил на него честь и величие.
\vs Psa 20:7 Ты положил на него благословения на веки, возвеселил его радостью лица Твоего,
\vs Psa 20:8 ибо царь уповает на Господа, и по благости Всевышнего не поколеблется.
\vs Psa 20:9 Рука Твоя найдет всех врагов Твоих, десница Твоя найдет [всех] ненавидящих Тебя.
\vs Psa 20:10 Во время гнева Твоего Ты сделаешь их, как печь огненную; во гневе Своем Господь погубит их, и пожрет их огонь.
\vs Psa 20:11 Ты истребишь плод их с земли и семя их~--- из среды сынов человеческих,
\vs Psa 20:12 ибо они предприняли против Тебя злое, составили замыслы, но не могли [выполнить их].
\vs Psa 20:13 Ты поставишь их целью, из луков Твоих пустишь стрелы в лице их.
\vs Psa 20:14 Вознесись, Господи, силою Твоею: мы будем воспевать и прославлять Твое могущество.
\vs Psa 21:1 Начальнику хора. При появлении зари. Псалом Давида.
\rsbpar\vs Psa 21:2 Боже мой! Боже мой! [внемли мне] для чего Ты оставил меня? Далеки от спасения моего слова вопля моего.
\vs Psa 21:3 Боже мой! я вопию днем,~--- и Ты не внемлешь мне, ночью,~--- и нет мне успокоения.
\vs Psa 21:4 Но Ты, Святый, живешь среди славословий Израиля.
\vs Psa 21:5 На Тебя уповали отцы наши; уповали, и Ты избавлял их;
\vs Psa 21:6 к Тебе взывали они, и были спасаемы; на Тебя уповали, и не оставались в стыде.
\vs Psa 21:7 Я же червь, а не человек, поношение у людей и презрение в народе.
\vs Psa 21:8 Все, видящие меня, ругаются надо мною, говорят устами, кивая головою:
\vs Psa 21:9 <<он уповал на Господа; пусть избавит его, пусть спасет, если он угоден Ему>>.
\vs Psa 21:10 Но Ты извел меня из чрева, вложил в меня упование у грудей матери моей.
\vs Psa 21:11 На Тебя оставлен я от утробы; от чрева матери моей Ты~--- Бог мой.
\vs Psa 21:12 Не удаляйся от меня, ибо скорбь близка, а помощника нет.
\vs Psa 21:13 Множество тельцов обступили меня; тучные Васанские окружили меня,
\vs Psa 21:14 раскрыли на меня пасть свою, \bibemph{как} лев, алчущий добычи и рыкающий.
\vs Psa 21:15 Я пролился, как вода; все кости мои рассыпались; сердце мое сделалось, как воск, растаяло посреди внутренности моей.
\vs Psa 21:16 Сила моя иссохла, как черепок; язык мой прильпнул к гортани моей, и Ты свел меня к персти смертной.
\vs Psa 21:17 Ибо псы окружили меня, скопище злых обступило меня, пронзили руки мои и ноги мои.
\vs Psa 21:18 Можно было бы перечесть все кости мои; а они смотрят и делают из меня зрелище;
\vs Psa 21:19 делят ризы мои между собою и об одежде моей бросают жребий.
\vs Psa 21:20 Но Ты, Господи, не удаляйся от меня; сила моя! поспеши на помощь мне;
\vs Psa 21:21 избавь от меча душу мою и от псов одинокую мою;
\vs Psa 21:22 спаси меня от пасти льва и от рогов единорогов, услышав, \bibemph{избавь} меня.
\vs Psa 21:23 Буду возвещать имя Твое братьям моим, посреди собрания восхвалять Тебя.
\vs Psa 21:24 Боящиеся Господа! восхвалите Его. Все семя Иакова! прославь Его. Да благоговеет пред Ним все семя Израиля,
\vs Psa 21:25 ибо Он не презрел и не пренебрег скорби страждущего, не скрыл от него лица Своего, но услышал его, когда сей воззвал к Нему.
\vs Psa 21:26 О Тебе хвала моя в собрании великом; воздам обеты мои пред боящимися Его.
\vs Psa 21:27 Да едят бедные и насыщаются; да восхвалят Господа ищущие Его; да живут сердца ваши во веки!
\vs Psa 21:28 Вспомнят, и обратятся к Господу все концы земли, и поклонятся пред Тобою все племена язычников,
\vs Psa 21:29 ибо Господне есть царство, и Он~--- Владыка над народами.
\vs Psa 21:30 Будут есть и поклоняться все тучные земли; преклонятся пред Ним все нисходящие в персть и не могущие сохранить жизни своей.
\vs Psa 21:31 Потомство [мое] будет служить Ему, и будет называться Господним вовек:
\vs Psa 21:32 придут и будут возвещать правду Его людям, которые родятся, чт\acc{о} сотворил Господь.
\vs Psa 22:0 Псалом Давида.
\rsbpar\vs Psa 22:1 Господь~--- Пастырь мой; я ни в чем не буду нуждаться:
\vs Psa 22:2 Он покоит меня на злачных пажитях и водит меня к водам тихим,
\vs Psa 22:3 подкрепляет душу мою, направляет меня на стези правды ради имени Своего.
\vs Psa 22:4 Если я пойду и долиною смертной тени, не убоюсь зла, потому что Ты со мной; Твой жезл и Твой посох~--- они успокаивают меня.
\vs Psa 22:5 Ты приготовил предо мною трапезу в виду врагов моих; умастил елеем голову мою; чаша моя преисполнена.
\vs Psa 22:6 Так, благость и милость [Твоя] да сопровождают меня во все дни жизни моей, и я пребуду в доме Господнем многие дни.
\vs Psa 23:0 Псалом Давида. [В первый день недели.]
\rsbpar\vs Psa 23:1 Господня земля и что наполняет ее, вселенная и все живущее в ней,
\vs Psa 23:2 ибо Он основал ее на морях и на реках утвердил ее.
\vs Psa 23:3 Кто взойдет на гору Господню, или кто станет на святом месте Его?
\vs Psa 23:4 Тот, у которого руки неповинны и сердце чисто, кто не клялся душею своею напрасно и не божился ложно [ближнему своему],~---
\vs Psa 23:5 \bibemph{тот} получит благословение от Господа и милость от Бога, Спасителя своего.
\vs Psa 23:6 Таков род ищущих Его, ищущих лица Твоего, Боже Иакова!
\vs Psa 23:7 Поднимите, врата, верхи ваши, и поднимитесь, двери вечные, и войдет Царь славы!
\vs Psa 23:8 Кто сей Царь славы?~--- Господь крепкий и сильный, Господь, сильный в брани.
\vs Psa 23:9 Поднимите, врата, верхи ваши, и поднимитесь, двери вечные, и войдет Царь славы!
\vs Psa 23:10 Кто сей Царь славы?~--- Господь сил, Он~--- Царь славы.
\vs Psa 24:0 Псалом Давида.
\rsbpar\vs Psa 24:1 К Тебе, Господи, возношу душу мою.
\vs Psa 24:2 Боже мой! на Тебя уповаю, да не постыжусь [вовек], да не восторжествуют надо мною враги мои,
\vs Psa 24:3 да не постыдятся и все надеющиеся на Тебя: да постыдятся беззаконнующие втуне.
\vs Psa 24:4 Укажи мне, Господи, пути Твои и научи меня стезям Твоим.
\vs Psa 24:5 Направь меня на истину Твою и научи меня, ибо Ты Бог спасения моего; на Тебя надеюсь всякий день.
\vs Psa 24:6 Вспомни щедроты Твои, Господи, и милости Твои, ибо они от века.
\vs Psa 24:7 Грехов юности моей и преступлений моих не вспоминай; по милости Твоей вспомни меня Ты, ради благости Твоей, Господи!
\vs Psa 24:8 Благ и праведен Господь, посему наставляет грешников на путь,
\vs Psa 24:9 направляет кротких к правде, и научает кротких путям Своим.
\vs Psa 24:10 Все пути Господни~--- милость и истина к хранящим завет Его и откровения Его.
\vs Psa 24:11 Ради имени Твоего, Господи, прости согрешение мое, ибо велико оно.
\vs Psa 24:12 Кто есть человек, боящийся Господа? Ему укажет Он путь, который избрать.
\vs Psa 24:13 Душа его пребудет во благе, и семя его наследует землю.
\vs Psa 24:14 Тайна Господня~--- боящимся Его, и завет Свой Он открывает им.
\vs Psa 24:15 Очи мои всегда к Господу, ибо Он извлекает из сети ноги мои.
\vs Psa 24:16 Призри на меня и помилуй меня, ибо я одинок и угнетен.
\vs Psa 24:17 Скорби сердца моего умножились; выведи меня из бед моих,
\vs Psa 24:18 призри на страдание мое и на изнеможение мое и прости все грехи мои.
\vs Psa 24:19 Посмотри на врагов моих, как много их, и \bibemph{какою} лютою ненавистью они ненавидят меня.
\vs Psa 24:20 Сохрани душу мою и избавь меня, да не постыжусь, что я на Тебя уповаю.
\vs Psa 24:21 Непорочность и правота да охраняют меня, ибо я на Тебя надеюсь.
\vs Psa 24:22 Избавь, Боже, Израиля от всех скорбей его.
\vs Psa 25:0 Псалом Давида.
\rsbpar\vs Psa 25:1 Рассуди меня, Господи, ибо я ходил в непорочности моей, и, уповая на Господа, не поколеблюсь.
\vs Psa 25:2 Искуси меня, Господи, и испытай меня; расплавь внутренности мои и сердце мое,
\vs Psa 25:3 ибо милость Твоя пред моими очами, и я ходил в истине Твоей,
\vs Psa 25:4 не сидел я с людьми лживыми, и с коварными не пойду;
\vs Psa 25:5 возненавидел я сборище злонамеренных, и с нечестивыми не сяду;
\vs Psa 25:6 буду омывать в невинности руки мои и обходить жертвенник Твой, Господи,
\vs Psa 25:7 чтобы возвещать гласом хвалы и поведать все чудеса Твои.
\vs Psa 25:8 Господи! возлюбил я обитель дома Твоего и место жилища славы Твоей.
\vs Psa 25:9 Не погуби души моей с грешниками и жизни моей с кровожадными,
\vs Psa 25:10 у которых в руках злодейство, и которых правая рука полна мздоимства.
\vs Psa 25:11 А я хожу в моей непорочности; избавь меня, [Господи,] и помилуй меня.
\vs Psa 25:12 Моя нога стоит на прямом \bibemph{пути}; в собраниях благословлю Господа.
\vs Psa 26:0 Псалом Давида. [Прежде помазания.]
\rsbpar\vs Psa 26:1 Господь~--- свет мой и спасение мое: кого мне бояться? Господь крепость жизни моей: кого мне страшиться?
\vs Psa 26:2 Если будут наступать на меня злодеи, противники и враги мои, чтобы пожрать плоть мою, то они сами преткнутся и падут.
\vs Psa 26:3 Если ополчится против меня полк, не убоится сердце мое; если восстанет на меня война, и тогда буду надеяться.
\vs Psa 26:4 Одного просил я у Господа, того только ищу, чтобы пребывать мне в доме Господнем во все дни жизни моей, созерцать красоту Господню и посещать [святый] храм Его,
\vs Psa 26:5 ибо Он укрыл бы меня в скинии Своей в день бедствия, скрыл бы меня в потаенном месте селения Своего, вознес бы меня на скалу.
\vs Psa 26:6 Тогда вознеслась бы голова моя над врагами, окружающими меня; и я принес бы в Его скинии жертвы славословия, стал бы петь и воспевать пред Господом.
\vs Psa 26:7 Услышь, Господи, голос мой, которым я взываю, помилуй меня и внемли мне.
\vs Psa 26:8 Сердце мое говорит от Тебя: <<ищите лица Моего>>; и я буду искать лица Твоего, Господи.
\vs Psa 26:9 Не скрой от меня лица Твоего; не отринь во гневе раба Твоего. Ты был помощником моим; не отвергни меня и не оставь меня, Боже, Спаситель мой!
\vs Psa 26:10 ибо отец мой и мать моя оставили меня, но Господь примет меня.
\vs Psa 26:11 Научи меня, Господи, пути Твоему и наставь меня на стезю правды, ради врагов моих;
\vs Psa 26:12 не предавай меня на произвол врагам моим, ибо восстали на меня свидетели лживые и дышат злобою.
\vs Psa 26:13 Но я верую, что увижу благость Господа на земле живых.
\vs Psa 26:14 Надейся на Господа, мужайся, и да укрепляется сердце твое, и надейся на Господа.
\vs Psa 27:0 Псалом Давида.
\rsbpar\vs Psa 27:1 К тебе, Господи, взываю: твердыня моя! не будь безмолвен для меня, чтобы при безмолвии Твоем я не уподобился нисходящим в могилу.
\vs Psa 27:2 Услышь голос молений моих, когда я взываю к Тебе, когда поднимаю руки мои к святому храму Твоему.
\vs Psa 27:3 Не погуби меня с нечестивыми и с делающими неправду, которые с ближними своими говорят о мире, а в сердце у них зло.
\vs Psa 27:4 Воздай им по делам их, по злым поступкам их; по делам рук их воздай им, отдай им заслуженное ими.
\vs Psa 27:5 За то, что они невнимательны к действиям Господа и к делу рук Его, Он разрушит их и не созиждет их.
\vs Psa 27:6 Благословен Господь, ибо Он услышал голос молений моих.
\vs Psa 27:7 Господь~--- крепость моя и щит мой; на Него уповало сердце мое, и Он помог мне, и возрадовалось сердце мое; и я прославлю Его песнью моею.
\vs Psa 27:8 Господь~--- крепость народа Своего и спасительная защита помазанника Своего.
\vs Psa 27:9 Спаси народ Твой и благослови наследие Твое; паси их и возвышай их во веки!
\vs Psa 28:0 Псалом Давида. [При окончании праздника кущей.]
\rsbpar\vs Psa 28:1 Воздайте Господу, сыны Божии, воздайте Господу славу и честь,
\vs Psa 28:2 воздайте Господу славу имени Его; поклонитесь Господу в благолепном святилище \bibemph{Его}.
\vs Psa 28:3 Глас Господень над водами; Бог славы возгремел, Господь над водами многими.
\vs Psa 28:4 Глас Господа силен, глас Господа величествен.
\vs Psa 28:5 Глас Господа сокрушает кедры; Господь сокрушает кедры Ливанские
\vs Psa 28:6 и заставляет их скакать подобно тельцу, Ливан и Сирион, подобно молодому единорогу.
\vs Psa 28:7 Глас Господа высекает пламень огня.
\vs Psa 28:8 Глас Господа потрясает пустыню; потрясает Господь пустыню Кадес.
\vs Psa 28:9 Глас Господа разрешает от бремени ланей и обнажает леса; и во храме Его все возвещает о \bibemph{Его} славе.
\vs Psa 28:10 Господь восседал над потопом, и будет восседать Господь царем вовек.
\vs Psa 28:11 Господь даст силу народу Своему, Господь благословит народ Свой миром.
\vs Psa 29:1 Псалом Давида; песнь при обновлении дома.
\rsbpar\vs Psa 29:2 Превознесу Тебя, Господи, что Ты поднял меня и не дал моим врагам восторжествовать надо мною.
\vs Psa 29:3 Господи, Боже мой! я воззвал к Тебе, и Ты исцелил меня.
\vs Psa 29:4 Господи! Ты вывел из ада душу мою и оживил меня, чтобы я не сошел в могилу.
\vs Psa 29:5 Пойте Господу, святые Его, славьте память святыни Его,
\vs Psa 29:6 ибо на мгновение гнев Его, на \bibemph{всю} жизнь благоволение Его: вечером водворяется плач, а на утро радость.
\vs Psa 29:7 И я говорил в благоденствии моем: <<не поколеблюсь вовек>>.
\vs Psa 29:8 По благоволению Твоему, Господи, Ты укрепил гору мою; но Ты сокрыл лице Твое, \bibemph{и} я смутился.
\vs Psa 29:9 \bibemph{Тогда} к Тебе, Господи, взывал я, и Господа [моего] умолял:
\vs Psa 29:10 <<что пользы в крови моей, когда я сойду в могилу? будет ли прах славить Тебя? будет ли возвещать истину Твою?
\vs Psa 29:11 услышь, Господи, и помилуй меня; Господи! будь мне помощником>>.
\vs Psa 29:12 И Ты обратил сетование мое в ликование, снял с меня вретище и препоясал меня веселием,
\vs Psa 29:13 да славит Тебя душа моя и да не умолкает. Господи, Боже мой! буду славить Тебя вечно.
\vs Psa 30:1 Начальнику хора. Псалом Давида. [Во время смятения.]
\rsbpar\vs Psa 30:2 На Тебя, Господи, уповаю, да не постыжусь вовек; по правде Твоей избавь меня;
\vs Psa 30:3 приклони ко мне ухо Твое, поспеши избавить меня. Будь мне каменною твердынею, домом прибежища, чтобы спасти меня,
\vs Psa 30:4 ибо Ты каменная гора моя и ограда моя; ради имени Твоего води меня и управляй мною.
\vs Psa 30:5 Выведи меня из сети, которую тайно поставили мне, ибо Ты крепость моя.
\vs Psa 30:6 В Твою руку предаю дух мой; Ты избавлял меня, Господи, Боже истины.
\vs Psa 30:7 Ненавижу почитателей суетных идолов, но на Господа уповаю.
\vs Psa 30:8 Буду радоваться и веселиться о милости Твоей, потому что Ты призрел на бедствие мое, узнал горесть души моей
\vs Psa 30:9 и не предал меня в руки врага; поставил ноги мои на пространном месте.
\vs Psa 30:10 Помилуй меня, Господи, ибо тесно мне; иссохло от горести око мое, душа моя и утроба моя.
\vs Psa 30:11 Истощилась в печали жизнь моя и лета мои в стенаниях; изнемогла от грехов моих сила моя, и кости мои иссохли.
\vs Psa 30:12 От всех врагов моих я сделался поношением даже у соседей моих и страшилищем для знакомых моих; видящие меня на улице бегут от меня.
\vs Psa 30:13 Я забыт в сердцах, как мертвый; я~--- как сосуд разбитый,
\vs Psa 30:14 ибо слышу злоречие многих; отвсюду ужас, когда они сговариваются против меня, умышляют исторгнуть душу мою.
\vs Psa 30:15 А я на Тебя, Господи, уповаю; я говорю: Ты~--- мой Бог.
\vs Psa 30:16 В Твоей руке дни мои; избавь меня от руки врагов моих и от гонителей моих.
\vs Psa 30:17 Яви светлое лице Твое рабу Твоему; спаси меня милостью Твоею.
\vs Psa 30:18 Господи! да не постыжусь, что я к Тебе взываю; нечестивые же да посрамятся, да умолкнут в аде.
\vs Psa 30:19 Да онемеют уста лживые, которые против праведника говорят злое с гордостью и презреньем.
\vs Psa 30:20 Как много у Тебя благ, которые Ты хранишь для боящихся Тебя и которые приготовил уповающим на Тебя пред сынами человеческими!
\vs Psa 30:21 Ты укрываешь их под покровом лица Твоего от мятежей людских, скрываешь их под сенью от пререкания языков.
\vs Psa 30:22 Благословен Господь, что явил мне дивную милость Свою в укрепленном городе!
\vs Psa 30:23 В смятении моем я думал: <<отвержен я от очей Твоих>>; но Ты услышал голос молитвы моей, когда я воззвал к Тебе.
\vs Psa 30:24 Любите Господа, все праведные Его; Господь хранит верных и поступающим надменно воздает с избытком.
\vs Psa 30:25 Мужайтесь, и да укрепляется сердце ваше, все надеющиеся на Господа!
\vs Psa 31:0 Псалом Давида. Учение.
\rsbpar\vs Psa 31:1 Блажен, кому отпущены беззакония, и чьи грехи покрыты!
\vs Psa 31:2 Блажен человек, которому Господь не вменит греха, и в чьем духе нет лукавства!
\vs Psa 31:3 Когда я молчал, обветшали кости мои от вседневного стенания моего,
\vs Psa 31:4 ибо день и ночь тяготела надо мною рука Твоя; свежесть моя исчезла, как в летнюю засуху.
\vs Psa 31:5 Но я открыл Тебе грех мой и не скрыл беззакония моего; я сказал: <<исповедаю Господу преступления мои>>, и Ты снял с меня вину греха моего.
\vs Psa 31:6 За то помолится Тебе каждый праведник во время благопотребное, и тогда разлитие многих вод не достигнет его.
\vs Psa 31:7 Ты покров мой: Ты охраняешь меня от скорби, окружаешь меня радостями избавления.
\vs Psa 31:8 <<Вразумлю тебя, наставлю тебя на путь, по которому тебе идти; буду руководить тебя, око Мое над тобою>>.
\vs Psa 31:9 <<Не будьте как конь, как лошак несмысленный, которых челюсти нужно обуздывать уздою и удилами, чтобы они покорялись тебе>>.
\vs Psa 31:10 Много скорбей нечестивому, а уповающего на Господа окружает милость.
\vs Psa 31:11 Веселитесь о Господе и радуйтесь, праведные; торжествуйте, все правые сердцем.
\vs Psa 32:0 [Псалом Давида.]
\rsbpar\vs Psa 32:1 Радуйтесь, праведные, о Господе: правым прилично славословить.
\vs Psa 32:2 Славьте Господа на гуслях, пойте Ему на десятиструнной псалтири;
\vs Psa 32:3 пойте Ему новую песнь; пойте Ему стройно, с восклицанием,
\vs Psa 32:4 ибо слово Господне право и все дела Его верны.
\vs Psa 32:5 Он любит правду и суд; милости Господней полна земля.
\vs Psa 32:6 Словом Господа сотворены небеса, и духом уст Его~--- все воинство их:
\vs Psa 32:7 Он собрал, будто груды, морские воды, положил бездны в хранилищах.
\vs Psa 32:8 Да боится Господа вся земля; да трепещут пред Ним все живущие во вселенной,
\vs Psa 32:9 ибо Он сказал,~--- и сделалось; Он повелел,~--- и явилось.
\vs Psa 32:10 Господь разрушает советы язычников, уничтожает замыслы народов, [уничтожает советы князей].
\vs Psa 32:11 Совет же Господень стоит вовек; помышления сердца Его~--- в род и род.
\vs Psa 32:12 Блажен народ, у которого Господь есть Бог,~--- племя, которое Он избрал в наследие Себе.
\vs Psa 32:13 С небес призирает Господь, видит всех сынов человеческих;
\vs Psa 32:14 с престола, на котором восседает, Он призирает на всех, живущих на земле:
\vs Psa 32:15 Он создал сердца всех их и вникает во все дела их.
\vs Psa 32:16 Не спасется царь множеством воинства; исполина не защитит великая сила.
\vs Psa 32:17 Ненадежен конь для спасения, не избавит великою силою своею.
\vs Psa 32:18 Вот, око Господне над боящимися Его и уповающими на милость Его,
\vs Psa 32:19 что Он душу их спасет от смерти и во время голода пропитает их.
\vs Psa 32:20 Душа наша уповает на Господа: Он~--- помощь наша и защита наша;
\vs Psa 32:21 о Нем веселится сердце наше, ибо на святое имя Его мы уповали.
\vs Psa 32:22 Да будет милость Твоя, Господи, над нами, как мы уповаем на Тебя.
\vs Psa 33:1 Псалом Давида, когда он притворился безумным пред Авимелехом и был изгнан от него и удалился.
\rsbpar\vs Psa 33:2 Благословлю Господа во всякое время; хвала Ему непрестанно в устах моих.
\vs Psa 33:3 Господом будет хвалиться душа моя; услышат кроткие и возвеселятся.
\vs Psa 33:4 Величайте Господа со мною, и превознесем имя Его вместе.
\vs Psa 33:5 Я взыскал Господа, и Он услышал меня, и от всех опасностей моих избавил меня.
\vs Psa 33:6 Кто обращал взор к Нему, те просвещались, и лица их не постыдятся.
\vs Psa 33:7 Сей нищий воззвал,~--- и Господь услышал и спас его от всех бед его.
\vs Psa 33:8 Ангел Господень ополчается вокруг боящихся Его и избавляет их.
\vs Psa 33:9 Вкусите, и увидите, как благ Господь! Блажен человек, который уповает на Него!
\vs Psa 33:10 Бойтесь Господа, [все] святые Его, ибо нет скудости у боящихся Его.
\vs Psa 33:11 Скимны бедствуют и терпят голод, а ищущие Господа не терпят нужды ни в каком благе.
\vs Psa 33:12 Придите, дети, послушайте меня: страху Господню научу вас.
\vs Psa 33:13 Хочет ли человек жить и любит ли долгоденствие, чтобы видеть благо?
\vs Psa 33:14 Удерживай язык свой от зла и уста свои от коварных слов.
\vs Psa 33:15 Уклоняйся от зла и делай добро; ищи мира и следуй за ним.
\vs Psa 33:16 Очи Господни \bibemph{обращены} на праведников, и уши Его~--- к воплю их.
\vs Psa 33:17 Но лице Господне против делающих зло, чтобы истребить с земли память о них.
\vs Psa 33:18 Взывают [праведные], и Господь слышит, и от всех скорбей их избавляет их.
\vs Psa 33:19 Близок Господь к сокрушенным сердцем и смиренных духом спасет.
\vs Psa 33:20 Много скорбей у праведного, и от всех их избавит его Господь.
\vs Psa 33:21 Он хранит все кости его; ни одна из них не сокрушится.
\vs Psa 33:22 Убьет грешника зло, и ненавидящие праведного погибнут.
\vs Psa 33:23 Избавит Господь душу рабов Своих, и никто из уповающих на Него не погибнет.
\vs Psa 34:0 Псалом Давида.
\rsbpar\vs Psa 34:1 Вступись, Господи, в тяжбу с тяжущимися со мною, побори борющихся со мною;
\vs Psa 34:2 возьми щит и латы и восстань на помощь мне;
\vs Psa 34:3 обнажи меч и прегради \bibemph{путь} преследующим меня; скажи душе моей: <<Я~--- спасение твое!>>
\vs Psa 34:4 Да постыдятся и посрамятся ищущие души моей; да обратятся назад и покроются бесчестием умышляющие мне зло;
\vs Psa 34:5 да будут они, как прах пред лицем ветра, и Ангел Господень да прогоняет \bibemph{их};
\vs Psa 34:6 да будет путь их темен и скользок, и Ангел Господень да преследует их,
\vs Psa 34:7 ибо они без вины скрыли для меня яму~--- сеть свою, без вины выкопали \bibemph{ее} для души моей.
\vs Psa 34:8 Да придет на него гибель неожиданная, и сеть его, которую он скрыл \bibemph{для меня}, да уловит его самого; да впадет в нее на погибель.
\vs Psa 34:9 А моя душа будет радоваться о Господе, будет веселиться о спасении от Него.
\vs Psa 34:10 Все кости мои скажут: <<Господи! кто подобен Тебе, избавляющему слабого от сильного, бедного и нищего от грабителя его?>>
\vs Psa 34:11 Восстали на меня свидетели неправедные: чего я не знаю, о том допрашивают меня;
\vs Psa 34:12 воздают мне злом за добро, сиротством душе моей.
\vs Psa 34:13 Я во время болезни их одевался во вретище, изнурял постом душу мою, и молитва моя возвращалась в недро мое.
\vs Psa 34:14 Я поступал, как бы это был друг мой, брат мой; я ходил скорбный, с поникшею головою, как бы оплакивающий мать.
\vs Psa 34:15 А когда я претыкался, они радовались и собирались; собирались ругатели против меня, не знаю за что, поносили и не переставали;
\vs Psa 34:16 с лицемерными насмешниками скрежетали на меня зубами своими.
\vs Psa 34:17 Господи! долго ли будешь смотреть \bibemph{на это}? Отведи душу мою от злодейств их, от львов~--- одинокую мою.
\vs Psa 34:18 Я прославлю Тебя в собрании великом, среди народа многочисленного восхвалю Тебя,
\vs Psa 34:19 чтобы не торжествовали надо мною враждующие против меня неправедно, и не перемигивались глазами ненавидящие меня безвинно;
\vs Psa 34:20 ибо не о мире говорят они, но против мирных земли составляют лукавые замыслы;
\vs Psa 34:21 расширяют на меня уста свои; говорят: <<хорошо! хорошо! видел глаз наш>>.
\vs Psa 34:22 Ты видел, Господи, не умолчи; Господи! не удаляйся от меня.
\vs Psa 34:23 Подвигнись, пробудись для суда моего, для тяжбы моей, Боже мой и Господи мой!
\vs Psa 34:24 Суди меня по правде Твоей, Господи, Боже мой, и да не торжествуют они надо мною;
\vs Psa 34:25 да не говорят в сердце своем: <<хорошо! [хорошо!] по душе нашей!>> Да не говорят: <<мы поглотили его>>.
\vs Psa 34:26 Да постыдятся и посрамятся все, радующиеся моему несчастью; да облекутся в стыд и позор величающиеся надо мною.
\vs Psa 34:27 Да радуются и веселятся желающие правоты моей и говорят непрестанно: <<да возвеличится Господь, желающий мира рабу Своему!>>
\vs Psa 34:28 И язык мой будет проповедовать правду Твою и хвалу Твою всякий день.
\vs Psa 35:1 Начальнику хора. Раба Господня Давида.
\rsbpar\vs Psa 35:2 Нечестие беззаконного говорит в сердце моем: нет страха Божия пред глазами его,
\vs Psa 35:3 ибо он льстит себе в глазах своих, будто отыскивает беззаконие свое, чтобы возненавидеть его;
\vs Psa 35:4 слова уст его~--- неправда и лукавство; не хочет он вразумиться, чтобы делать добро;
\vs Psa 35:5 на ложе своем замышляет беззаконие, становится на путь недобрый, не гнушается злом.
\vs Psa 35:6 Господи! милость Твоя до небес, истина Твоя до облаков!
\vs Psa 35:7 Правда Твоя, как горы Божии, и судьбы Твои~--- бездна великая! Человеков и скотов хранишь Ты, Господи!
\vs Psa 35:8 Как драгоценна милость Твоя, Боже! Сыны человеческие в тени крыл Твоих покойны:
\vs Psa 35:9 насыщаются от тука дома Твоего, и из потока сладостей Твоих Ты напояешь их,
\vs Psa 35:10 ибо у Тебя источник жизни; во свете Твоем мы видим свет.
\vs Psa 35:11 Продли милость Твою к знающим Тебя и правду Твою к правым сердцем,
\vs Psa 35:12 да не наступит на меня нога гордыни, и рука грешника да не изгонит меня:
\vs Psa 35:13 там пали делающие беззаконие, низринуты и не могут встать.
\vs Psa 36:0 Псалом Давида.
\rsbpar\vs Psa 36:1 Не ревнуй злодеям, не завидуй делающим беззаконие,
\vs Psa 36:2 ибо они, как трава, скоро будут подкошены и, как зеленеющий злак, увянут.
\vs Psa 36:3 Уповай на Господа и делай добро; живи на земле и храни истину.
\vs Psa 36:4 Утешайся Господом, и Он исполнит желания сердца твоего.
\vs Psa 36:5 Предай Господу путь твой и уповай на Него, и Он совершит,
\vs Psa 36:6 и выведет, как свет, правду твою и справедливость твою, как полдень.
\vs Psa 36:7 Покорись Господу и надейся на Него. Не ревнуй успевающему в пути своем, человеку лукавствующему.
\vs Psa 36:8 Перестань гневаться и оставь ярость; не ревнуй до того, чтобы делать зло,
\vs Psa 36:9 ибо делающие зло истребятся, уповающие же на Господа наследуют землю.
\vs Psa 36:10 Еще немного, и не станет нечестивого; посмотришь на его место, и нет его.
\vs Psa 36:11 А кроткие наследуют землю и насладятся множеством мира.
\vs Psa 36:12 Нечестивый злоумышляет против праведника и скрежещет на него зубами своими:
\vs Psa 36:13 Господь же посмевается над ним, ибо видит, что приходит день его.
\vs Psa 36:14 Нечестивые обнажают меч и натягивают лук свой, чтобы низложить бедного и нищего, чтобы пронзить \bibemph{идущих} прямым путем:
\vs Psa 36:15 меч их войдет в их же сердце, и луки их сокрушатся.
\vs Psa 36:16 Малое у праведника~--- лучше богатства многих нечестивых,
\vs Psa 36:17 ибо мышцы нечестивых сокрушатся, а праведников подкрепляет Господь.
\vs Psa 36:18 Господь знает дни непорочных, и достояние их пребудет вовек:
\vs Psa 36:19 не будут они постыжены во время лютое и во дни голода будут сыты;
\vs Psa 36:20 а нечестивые погибнут, и враги Господни, как тук агнцев, исчезнут, в дыме исчезнут.
\vs Psa 36:21 Нечестивый берет взаймы и не отдает, а праведник милует и дает,
\vs Psa 36:22 ибо благословенные Им наследуют землю, а проклятые Им истребятся.
\vs Psa 36:23 Господом утверждаются стопы \bibemph{такого} человека, и Он благоволит к пути его:
\vs Psa 36:24 когда он будет падать, не упадет, ибо Господь поддерживает его за руку.
\vs Psa 36:25 Я был молод и состарился, и не видал праведника оставленным и потомков его просящими хлеба:
\vs Psa 36:26 он всякий день милует и взаймы дает, и потомство его в благословение будет.
\vs Psa 36:27 Уклоняйся от зла, и делай добро, и будешь жить вовек:
\vs Psa 36:28 ибо Господь любит правду и не оставляет святых Своих; вовек сохранятся они; [а беззаконные будут извержены] и потомство нечестивых истребится.
\vs Psa 36:29 Праведники наследуют землю и будут жить на ней вовек.
\vs Psa 36:30 Уста праведника изрекают премудрость, и язык его произносит правду.
\vs Psa 36:31 Закон Бога его в сердце у него; не поколеблются стопы его.
\vs Psa 36:32 Нечестивый подсматривает за праведником и ищет умертвить его;
\vs Psa 36:33 но Господь не отдаст его в руки его и не допустит обвинить его, когда он будет судим.
\vs Psa 36:34 Уповай на Господа и держись пути Его: и Он вознесет тебя, чтобы ты наследовал землю; и когда будут истребляемы нечестивые, ты увидишь.
\vs Psa 36:35 Видел я нечестивца грозного, расширявшегося, подобно укоренившемуся многоветвистому дереву;
\vs Psa 36:36 но он прошел, и вот нет его; ищу его и не нахожу.
\vs Psa 36:37 Наблюдай за непорочным и смотри на праведного, ибо будущность \bibemph{такого} человека есть мир;
\vs Psa 36:38 а беззаконники все истребятся; будущность нечестивых погибнет.
\vs Psa 36:39 От Господа спасение праведникам, Он~--- защита их во время скорби;
\vs Psa 36:40 и поможет им Господь и избавит их; избавит их от нечестивых и спасет их, ибо они на Него уповают.
\vs Psa 37:1 Псалом Давида. В воспоминание [о субботе].
\rsbpar\vs Psa 37:2 Господи! не в ярости Твоей обличай меня и не во гневе Твоем наказывай меня,
\vs Psa 37:3 ибо стрелы Твои вонзились в меня, и рука Твоя тяготеет на мне.
\vs Psa 37:4 Нет целого места в плоти моей от гнева Твоего; нет мира в костях моих от грехов моих,
\vs Psa 37:5 ибо беззакония мои превысили голову мою, как тяжелое бремя отяготели на мне,
\vs Psa 37:6 смердят, гноятся раны мои от безумия моего.
\vs Psa 37:7 Я согбен и совсем поник, весь день сетуя хожу,
\vs Psa 37:8 ибо чресла мои полны воспалениями, и нет целого места в плоти моей.
\vs Psa 37:9 Я изнемог и сокрушен чрезмерно; кричу от терзания сердца моего.
\vs Psa 37:10 Господи! пред Тобою все желания мои, и воздыхание мое не сокрыто от Тебя.
\vs Psa 37:11 Сердце мое трепещет; оставила меня сила моя, и свет очей моих,~--- и того нет у меня.
\vs Psa 37:12 Друзья мои и искренние отступили от язвы моей, и ближние мои стоят вдали.
\vs Psa 37:13 Ищущие же души моей ставят сети, и желающие мне зла говорят о погибели \bibemph{моей} и замышляют всякий день козни;
\vs Psa 37:14 а я, как глухой, не слышу, и как немой, который не открывает уст своих;
\vs Psa 37:15 и стал я, как человек, который не слышит и не имеет в устах своих ответа,
\vs Psa 37:16 ибо на Тебя, Господи, уповаю я; Ты услышишь, Господи, Боже мой.
\vs Psa 37:17 И я сказал: да не восторжествуют надо мною [враги мои]; когда колеблется нога моя, они величаются надо мною.
\vs Psa 37:18 Я близок к падению, и скорбь моя всегда предо мною.
\vs Psa 37:19 Беззаконие мое я сознаю, сокрушаюсь о грехе моем.
\vs Psa 37:20 А враги мои живут и укрепляются, и умножаются ненавидящие меня безвинно;
\vs Psa 37:21 и воздающие мне злом за добро враждуют против меня за то, что я следую добру.
\vs Psa 37:22 Не оставь меня, Господи, Боже мой! Не удаляйся от меня;
\vs Psa 37:23 поспеши на помощь мне, Господи, Спаситель мой!
\vs Psa 38:1 Начальнику хора, Идифуму. Псалом Давида.
\rsbpar\vs Psa 38:2 Я сказал: буду я наблюдать за путями моими, чтобы не согрешать мне языком моим; буду обуздывать уста мои, доколе нечестивый предо мною.
\vs Psa 38:3 Я был нем и безгласен, и молчал \bibemph{даже} о добром; и скорбь моя подвиглась.
\vs Psa 38:4 Воспламенилось сердце мое во мне; в мыслях моих возгорелся огонь; я стал говорить языком моим:
\vs Psa 38:5 скажи мне, Господи, кончину мою и число дней моих, какое оно, дабы я знал, какой век мой.
\vs Psa 38:6 Вот, Ты дал мне дни, \bibemph{как} пяди, и век мой как ничто пред Тобою. Подлинно, совершенная суета~--- всякий человек живущий.
\vs Psa 38:7 Подлинно, человек ходит подобно призраку; напрасно он суетится, собирает и не знает, кому достанется то.
\vs Psa 38:8 И ныне чего ожидать мне, Господи? надежда моя~--- на Тебя.
\vs Psa 38:9 От всех беззаконий моих избавь меня, не предавай меня на поругание безумному.
\vs Psa 38:10 Я стал нем, не открываю уст моих; потому что Ты соделал это.
\vs Psa 38:11 Отклони от меня удары Твои; я исчезаю от поражающей руки Твоей.
\vs Psa 38:12 Если Ты обличениями будешь наказывать человека за преступления, то рассыплется, как от моли, краса его. Так, суетен всякий человек!
\vs Psa 38:13 Услышь, Господи, молитву мою и внемли воплю моему; не будь безмолвен к слезам моим, ибо странник я у Тебя \bibemph{и} пришлец, как и все отцы мои.
\vs Psa 38:14 Отступи от меня, чтобы я мог подкрепиться, прежде нежели отойду и не будет меня.
\vs Psa 39:1 Начальнику хора. Псалом Давида.
\rsbpar\vs Psa 39:2 Твердо уповал я на Господа, и Он приклонился ко мне и услышал вопль мой;
\vs Psa 39:3 извлек меня из страшного рва, из тинистого болота, и поставил на камне ноги мои и утвердил стопы мои;
\vs Psa 39:4 и вложил в уста мои новую песнь~--- хвалу Богу нашему. Увидят многие и убоятся и будут уповать на Господа.
\vs Psa 39:5 Блажен человек, который на Господа возлагает надежду свою и не обращается к гордым и к уклоняющимся ко лжи.
\vs Psa 39:6 Много соделал Ты, Господи, Боже мой: о чудесах и помышлениях Твоих о нас~--- кто уподобится Тебе!~--- хотел бы я проповедовать и говорить, но они превышают число.
\vs Psa 39:7 Жертвы и приношения Ты не восхотел; Ты открыл мне уши\fns{Открыл мне уши~--- по переводу 70-ти: уготовил мне тело.}; всесожжения и жертвы за грех Ты не потребовал.
\vs Psa 39:8 Тогда я сказал: вот, иду; в свитке книжном написано о мне:
\vs Psa 39:9 я желаю исполнить волю Твою, Боже мой, и закон Твой у меня в сердце.
\vs Psa 39:10 Я возвещал правду Твою в собрании великом; я не возбранял устам моим: Ты, Господи, знаешь.
\vs Psa 39:11 Правды Твоей не скрывал в сердце моем, возвещал верность Твою и спасение Твое, не утаивал милости Твоей и истины Твоей пред собранием великим.
\vs Psa 39:12 Не удерживай, Господи, щедрот Твоих от меня; милость Твоя и истина Твоя да охраняют меня непрестанно,
\vs Psa 39:13 ибо окружили меня беды неисчислимые; постигли меня беззакония мои, так что видеть не могу: их более, нежели волос на голове моей; сердце мое оставило меня.
\vs Psa 39:14 Благоволи, Господи, избавить меня; Господи! поспеши на помощь мне.
\vs Psa 39:15 Да постыдятся и посрамятся все, ищущие погибели душе моей! Да будут обращены назад и преданы посмеянию желающие мне зла!
\vs Psa 39:16 Да смятутся от посрамления своего говорящие мне: <<хорошо! хорошо!>>
\vs Psa 39:17 Да радуются и веселятся Тобою все ищущие Тебя, и любящие спасение Твое да говорят непрестанно: <<велик Господь!>>
\vs Psa 39:18 Я же беден и нищ, но Господь печется о мне. Ты~--- помощь моя и избавитель мой, Боже мой! не замедли.
\vs Psa 40:1 Начальнику хора. Псалом Давида.
\rsbpar\vs Psa 40:2 Блажен, кто помышляет о бедном [и нищем]! В день бедствия избавит его Господь.
\vs Psa 40:3 Господь сохранит его и сбережет ему жизнь; блажен будет он на земле. И Ты не отдашь его на волю врагов его.
\vs Psa 40:4 Господь укрепит его на одре болезни его. Ты изменишь все ложе его в болезни его.
\vs Psa 40:5 Я сказал: Господи! помилуй меня, исцели душу мою, ибо согрешил я пред Тобою.
\vs Psa 40:6 Враги мои говорят обо мне злое: <<когда он умрет и погибнет имя его?>>
\vs Psa 40:7 И если приходит кто видеть меня, говорит ложь; сердце его слагает в себе неправду, и он, выйдя вон, толкует.
\vs Psa 40:8 Все ненавидящие меня шепчут между собою против меня, замышляют на меня зло:
\vs Psa 40:9 <<слово велиала пришло на него; он слег; не встать ему более>>.
\vs Psa 40:10 Даже человек мирный со мною, на которого я полагался, который ел хлеб мой, поднял на меня пяту.
\vs Psa 40:11 Ты же, Господи, помилуй меня и восставь меня, и я воздам им.
\vs Psa 40:12 Из того узнаю, что Ты благоволишь ко мне, если враг мой не восторжествует надо мною,
\vs Psa 40:13 а меня сохранишь в целости моей и поставишь пред лицем Твоим на веки.
\vs Psa 40:14 Благословен Господь Бог Израилев от века и до века! Аминь, аминь!
\vs Psa 41:1 Начальнику хора. Учение. Сынов Кореевых.
\rsbpar\vs Psa 41:2 Как лань желает к потокам воды, так желает душа моя к Тебе, Боже!
\vs Psa 41:3 Жаждет душа моя к Богу крепкому, живому: когда приду и явлюсь пред лице Божие!
\vs Psa 41:4 Слезы мои были для меня хлебом день и ночь, когда говорили мне всякий день: <<где Бог твой?>>
\vs Psa 41:5 Вспоминая об этом, изливаю душу мою, потому что я ходил в многолюдстве, вступал с ними в дом Божий со гласом радости и славословия празднующего сонма.
\vs Psa 41:6 Что унываешь ты, душа моя, и что смущаешься? Уповай на Бога, ибо я буду еще славить Его, Спасителя моего и Бога моего.
\vs Psa 41:7 Унывает во мне душа моя; посему я воспоминаю о Тебе с земли Иорданской, с Ермона, с горы Цоар.
\vs Psa 41:8 Бездна бездну призывает голосом водопадов Твоих; все воды Твои и волны Твои прошли надо мною.
\vs Psa 41:9 Днем явит Господь милость Свою, и ночью песнь Ему у меня, молитва к Богу жизни моей.
\vs Psa 41:10 Скажу Богу, заступнику моему: для чего Ты забыл меня? Для чего я сетуя хожу от оскорблений врага?
\vs Psa 41:11 Как бы поражая кости мои, ругаются надо мною враги мои, когда говорят мне всякий день: <<где Бог твой?>>
\vs Psa 41:12 Что унываешь ты, душа моя, и что смущаешься? Уповай на Бога, ибо я буду еще славить Его, Спасителя моего и Бога моего.
\vs Psa 42:1 Суди меня, Боже, и вступись в тяжбу мою с народом недобрым. От человека лукавого и несправедливого избавь меня,
\vs Psa 42:2 ибо Ты Бог крепости моей. Для чего Ты отринул меня? для чего я сетуя хожу от оскорблений врага?
\vs Psa 42:3 Пошли свет Твой и истину Твою; да ведут они меня и приведут на святую гору Твою и в обители Твои.
\vs Psa 42:4 И подойду я к жертвеннику Божию, к Богу радости и веселия моего, и на гуслях буду славить Тебя, Боже, Боже мой!
\vs Psa 42:5 Что унываешь ты, душа моя, и что смущаешься? Уповай на Бога; ибо я буду еще славить Его, Спасителя моего и Бога моего.
\vs Psa 43:1 Начальнику хора. Учение. Сынов Кореевых.
\rsbpar\vs Psa 43:2 Боже, мы слышали ушами своими, отцы наши рассказывали нам о деле, какое Ты соделал во дни их, во дни древние:
\vs Psa 43:3 Ты рукою Твоею истребил народы, а их насадил; поразил племена и изгнал их;
\vs Psa 43:4 ибо они не мечом своим приобрели землю, и не их мышца спасла их, но Твоя десница и Твоя мышца и свет лица Твоего, ибо Ты благоволил к ним.
\vs Psa 43:5 Боже, Царь мой! Ты~--- тот же; даруй спасение Иакову.
\vs Psa 43:6 С Тобою избодаем рогами врагов наших; во имя Твое попрем ногами восстающих на нас:
\vs Psa 43:7 ибо не на лук мой уповаю, и не меч мой спасет меня;
\vs Psa 43:8 но Ты спасешь нас от врагов наших, и посрамишь ненавидящих нас.
\vs Psa 43:9 О Боге похвалимся всякий день, и имя Твое будем прославлять вовек.
\vs Psa 43:10 Но ныне Ты отринул и посрамил нас, и не выходишь с войсками нашими;
\vs Psa 43:11 обратил нас в бегство от врага, и ненавидящие нас грабят нас;
\vs Psa 43:12 Ты отдал нас, как овец, на съедение и рассеял нас между народами;
\vs Psa 43:13 без выгоды Ты продал народ Твой и не возвысил цены его;
\vs Psa 43:14 отдал нас на поношение соседям нашим, на посмеяние и поругание живущим вокруг нас;
\vs Psa 43:15 Ты сделал нас притчею между народами, покиванием головы между иноплеменниками.
\vs Psa 43:16 Всякий день посрамление мое предо мною, и стыд покрывает лице мое
\vs Psa 43:17 от голоса поносителя и клеветника, от взоров врага и мстителя:
\vs Psa 43:18 все это пришло на нас, но мы не забыли Тебя и не нарушили завета Твоего.
\vs Psa 43:19 Не отступило назад сердце наше, и стопы наши не уклонились от пути Твоего,
\vs Psa 43:20 когда Ты сокрушил нас в земле драконов и покрыл нас тенью смертною.
\vs Psa 43:21 Если бы мы забыли имя Бога нашего и простерли руки наши к богу чужому,
\vs Psa 43:22 то не взыскал ли бы сего Бог? Ибо Он знает тайны сердца.
\vs Psa 43:23 Но за Тебя умерщвляют нас всякий день, считают нас за овец, \bibemph{обреченных} на заклание.
\vs Psa 43:24 Восстань, что спишь, Господи! пробудись, не отринь навсегда.
\vs Psa 43:25 Для чего скрываешь лице Твое, забываешь скорбь нашу и угнетение наше?
\vs Psa 43:26 ибо душа наша унижена до праха, утроба наша прильнула к земле.
\vs Psa 43:27 Восстань на помощь нам и избавь нас ради милости Твоей.
\vs Psa 44:1 Начальнику хора. На \bibemph{музыкальном орудии} Шошан. Учение. Сынов Кореевых. Песнь любви.
\rsbpar\vs Psa 44:2 Излилось из сердца моего слово благое; я говорю: песнь моя о Царе; язык мой~--- трость скорописца.
\vs Psa 44:3 Ты прекраснее сынов человеческих; благодать излилась из уст Твоих; посему благословил Тебя Бог на веки.
\vs Psa 44:4 Препояшь Себя по бедру мечом Твоим, Сильный, славою Твоею и красотою Твоею,
\vs Psa 44:5 и в сем украшении Твоем поспеши, воссядь на колесницу ради истины и кротости и правды, и десница Твоя покажет Тебе дивные дела.
\vs Psa 44:6 Остры стрелы Твои, [Сильный],~--- народы падут пред Тобою,~--- они~--- в сердце врагов Царя.
\vs Psa 44:7 Престол Твой, Боже, вовек; жезл правоты~--- жезл царства Твоего.
\vs Psa 44:8 Ты возлюбил правду и возненавидел беззаконие, посему помазал Тебя, Боже, Бог Твой елеем радости более соучастников Твоих.
\vs Psa 44:9 Все одежды Твои, как смирна и алой и касия; из чертогов слоновой кости увеселяют Тебя.
\vs Psa 44:10 Дочери царей между почетными у Тебя; стала царица одесную Тебя в Офирском золоте.
\vs Psa 44:11 Слыши, дщерь, и смотри, и приклони ухо твое, и забудь народ твой и дом отца твоего.
\vs Psa 44:12 И возжелает Царь красоты твоей; ибо Он Господь твой, и ты поклонись Ему.
\vs Psa 44:13 И дочь Тира с дарами, и богатейшие из народа будут умолять лице Твое.
\vs Psa 44:14 Вся слава дщери Царя внутри; одежда ее шита золотом;
\vs Psa 44:15 в испещренной одежде ведется она к Царю; за нею ведутся к Тебе девы, подруги ее,
\vs Psa 44:16 приводятся с весельем и ликованьем, входят в чертог Царя.
\vs Psa 44:17 Вместо отцов Твоих, будут сыновья Твои; Ты поставишь их князьями по всей земле.
\vs Psa 44:18 Сделаю имя Твое памятным в род и род; посему народы будут славить Тебя во веки и веки.
\vs Psa 45:1 Начальнику хора. Сынов Кореевых. На \bibemph{музыкальном орудии} Аламоф. Песнь.
\rsbpar\vs Psa 45:2 Бог нам прибежище и сила, скорый помощник в бедах,
\vs Psa 45:3 посему не убоимся, хотя бы поколебалась земля, и горы двинулись в сердце морей.
\vs Psa 45:4 Пусть шумят, вздымаются воды их, трясутся горы от волнения их.
\vs Psa 45:5 Речные потоки веселят град Божий, святое жилище Всевышнего.
\vs Psa 45:6 Бог посреди его; он не поколеблется: Бог поможет ему с раннего утра.
\vs Psa 45:7 Восшумели народы; двинулись царства: [Всевышний] дал глас Свой, и растаяла земля.
\vs Psa 45:8 Господь сил с нами, Бог Иакова заступник наш.
\vs Psa 45:9 Придите и видите дела Господа,~--- какие произвел Он опустошения на земле:
\vs Psa 45:10 прекращая брани до края земли, сокрушил лук и переломил копье, колесницы сжег огнем.
\vs Psa 45:11 Остановитесь и познайте, что Я~--- Бог: буду превознесен в народах, превознесен на земле.
\vs Psa 45:12 Господь сил с нами, заступник наш Бог Иакова.
\vs Psa 46:1 Начальнику хора. Сынов Кореевых. Псалом.
\rsbpar\vs Psa 46:2 Восплещите руками все народы, воскликните Богу гласом радости;
\vs Psa 46:3 ибо Господь Всевышний страшен,~--- великий Царь над всею землею;
\vs Psa 46:4 покорил нам народы и племена под ноги наши;
\vs Psa 46:5 избрал нам наследие наше, красу Иакова, которого возлюбил.
\vs Psa 46:6 Восшел Бог при восклицаниях, Господь при звуке трубном.
\vs Psa 46:7 Пойте Богу нашему, пойте; пойте Царю нашему, пойте,
\vs Psa 46:8 ибо Бог~--- Царь всей земли; пойте все разумно.
\vs Psa 46:9 Бог воцарился над народами, Бог воссел на святом престоле Своем;
\vs Psa 46:10 князья народов собрались к народу Бога Авраамова, ибо щиты земли~--- Божии; Он превознесен \bibemph{над ними}.
\vs Psa 47:1 Песнь. Псалом. Сынов Кореевых.
\rsbpar\vs Psa 47:2 Велик Господь и всехвален во граде Бога нашего, на святой горе Его.
\vs Psa 47:3 Прекрасная возвышенность, радость всей земли гора Сион; на северной стороне \bibemph{ее} город великого Царя.
\vs Psa 47:4 Бог в жилищах его ведом, как заступник:
\vs Psa 47:5 ибо вот, сошлись цари и прошли все мимо;
\vs Psa 47:6 увидели и изумились, смутились и обратились в бегство;
\vs Psa 47:7 страх объял их там и мука, как у женщин в родах;
\vs Psa 47:8 восточным ветром Ты сокрушил Фарсийские корабли.
\vs Psa 47:9 Как слышали мы, так и увидели во граде Господа сил, во граде Бога нашего: Бог утвердит его на веки.
\vs Psa 47:10 Мы размышляли, Боже, о благости Твоей посреди храма Твоего.
\vs Psa 47:11 Как имя Твое, Боже, так и хвала Твоя до концов земли; десница Твоя полна правды.
\vs Psa 47:12 Да веселится гора Сион, [и] да радуются дщери Иудейские ради судов Твоих, [Господи].
\vs Psa 47:13 Пойдите вокруг Сиона и обойдите его, пересчитайте башни его;
\vs Psa 47:14 обратите сердце ваше к укреплениям его, рассмотрите домы его, чтобы пересказать грядущему роду,
\vs Psa 47:15 ибо сей Бог есть Бог наш на веки и веки: Он будет вождем нашим до самой смерти.
\vs Psa 48:1 Начальнику хора. Сынов Кореевых. Псалом.
\rsbpar\vs Psa 48:2 Слушайте сие, все народы; внимайте сему, все живущие во вселенной,~---
\vs Psa 48:3 и простые и знатные, богатый, равно как бедный.
\vs Psa 48:4 Уста мои изрекут премудрость, и размышления сердца моего~--- знание.
\vs Psa 48:5 Приклоню ухо мое к притче, на гуслях открою загадку мою:
\vs Psa 48:6 <<для чего бояться мне во дни бедствия, \bibemph{когда} беззаконие путей моих окружит меня?>>
\vs Psa 48:7 Надеющиеся на силы свои и хвалящиеся множеством богатства своего!
\vs Psa 48:8 человек никак не искупит брата своего и не даст Богу выкупа за него:
\vs Psa 48:9 дорог\acc{а} цена искупления души их, и не будет того вовек,
\vs Psa 48:10 чтобы остался \bibemph{кто} жить навсегда и не увидел могилы.
\vs Psa 48:11 Каждый видит, что и мудрые умирают, равно как и невежды и бессмысленные погибают и оставляют имущество свое другим.
\vs Psa 48:12 В мыслях у них, что домы их вечны, и что жилища их в род и род, и земли свои они называют своими именами.
\vs Psa 48:13 Но человек в чести не пребудет; он уподобится животным, которые погибают.
\vs Psa 48:14 Этот путь их есть безумие их, хотя последующие за ними одобряют мнение их.
\vs Psa 48:15 Как овец, заключат их в преисподнюю; смерть будет пасти их, и наутро праведники будут владычествовать над ними; сила их истощится; могила~--- жилище их.
\vs Psa 48:16 Но Бог избавит душу мою от власти преисподней, когда примет меня.
\vs Psa 48:17 Не бойся, когда богатеет человек, когда слава дома его умножается:
\vs Psa 48:18 ибо умирая не возьмет ничего; не пойдет за ним слава его;
\vs Psa 48:19 хотя при жизни он ублажает душу свою, и прославляют тебя, что ты удовлетворяешь себе,
\vs Psa 48:20 но он пойдет к роду отцов своих, которые никогда не увидят света.
\vs Psa 48:21 Человек, который в чести и неразумен, подобен животным, которые погибают.
\vs Psa 49:0 Псалом Асафа.
\rsbpar\vs Psa 49:1 Бог богов, Господь возглаголал и призывает землю, от восхода солнца до запада.
\vs Psa 49:2 С Сиона, который есть верх красоты, является Бог,
\vs Psa 49:3 грядет Бог наш, и не в безмолвии: пред Ним огонь поядающий, и вокруг Его сильная буря.
\vs Psa 49:4 Он призывает свыше небо и землю, судить народ Свой:
\vs Psa 49:5 <<соберите ко Мне святых Моих, вступивших в завет со Мною при жертве>>.
\vs Psa 49:6 И небеса провозгласят правду Его, ибо судия сей есть Бог.
\vs Psa 49:7 <<Слушай, народ Мой, Я буду говорить; Израиль! Я буду свидетельствовать против тебя: Я Бог, твой Бог.
\vs Psa 49:8 Не за жертвы твои Я буду укорять тебя; всесожжения твои всегда предо Мною;
\vs Psa 49:9 не приму тельца из дома твоего, ни козлов из дворов твоих,
\vs Psa 49:10 ибо Мои все звери в лесу, и скот на тысяче гор,
\vs Psa 49:11 знаю всех птиц на горах, и животные на полях предо Мною.
\vs Psa 49:12 Если бы Я взалкал, то не сказал бы тебе, ибо Моя вселенная и все, что наполняет ее.
\vs Psa 49:13 Ем ли Я мясо волов и пью ли кровь козлов?
\vs Psa 49:14 Принеси в жертву Богу хвалу и воздай Всевышнему обеты твои,
\vs Psa 49:15 и призови Меня в день скорби; Я избавлю тебя, и ты прославишь Меня>>.
\vs Psa 49:16 Грешнику же говорит Бог: <<что ты проповедуешь уставы Мои и берешь завет Мой в уста твои,
\vs Psa 49:17 а сам ненавидишь наставление Мое и слова Мои бросаешь за себя?
\vs Psa 49:18 когда видишь вора, сходишься с ним, и с прелюбодеями сообщаешься;
\vs Psa 49:19 уста твои открываешь на злословие, и язык твой сплетает коварство;
\vs Psa 49:20 сидишь и говоришь на брата твоего, на сына матери твоей клевещешь;
\vs Psa 49:21 ты это делал, и Я молчал; ты подумал, что Я такой же, как ты. Изобличу тебя и представлю пред глаза твои [грехи твои].
\vs Psa 49:22 Уразумейте это, забывающие Бога, дабы Я не восхитил,~--- и не будет избавляющего.
\vs Psa 49:23 Кто приносит в жертву хвалу, тот чтит Меня, и кто наблюдает за путем своим, тому явлю Я спасение Божие>>.
\vs Psa 50:1 Начальнику хора. Псалом Давида,
\vs Psa 50:2 когда приходил к нему пророк Нафан, после того, как Давид вошел к Вирсавии.
\rsbpar\vs Psa 50:3 Помилуй меня, Боже, по великой милости Твоей, и по множеству щедрот Твоих изгладь беззакония мои.
\vs Psa 50:4 Многократно омой меня от беззакония моего, и от греха моего очисти меня,
\vs Psa 50:5 ибо беззакония мои я сознаю, и грех мой всегда предо мною.
\vs Psa 50:6 Тебе, Тебе единому согрешил я и лукавое пред очами Твоими сделал, так что Ты праведен в приговоре Твоем и чист в суде Твоем.
\vs Psa 50:7 Вот, я в беззаконии зачат, и во грехе родила меня мать моя.
\vs Psa 50:8 Вот, Ты возлюбил истину в сердце и внутрь меня явил мне мудрость [Твою].
\vs Psa 50:9 Окропи меня иссопом, и буду чист; омой меня, и буду белее снега.
\vs Psa 50:10 Дай мне услышать радость и веселие, и возрадуются кости, Тобою сокрушенные.
\vs Psa 50:11 Отврати лице Твое от грехов моих и изгладь все беззакония мои.
\vs Psa 50:12 Сердце чистое сотвори во мне, Боже, и дух правый обнови внутри меня.
\vs Psa 50:13 Не отвергни меня от лица Твоего и Духа Твоего Святаго не отними от меня.
\vs Psa 50:14 Возврати мне радость спасения Твоего и Духом владычественным утверди меня.
\vs Psa 50:15 Научу беззаконных путям Твоим, и нечестивые к Тебе обратятся.
\vs Psa 50:16 Избавь меня от кровей, Боже, Боже спасения моего, и язык мой восхвалит правду Твою.
\vs Psa 50:17 Господи! отверзи уста мои, и уста мои возвестят хвалу Твою:
\vs Psa 50:18 ибо жертвы Ты не желаешь,~--- я дал бы ее; к всесожжению не благоволишь.
\vs Psa 50:19 Жертва Богу~--- дух сокрушенный; сердца сокрушенного и смиренного Ты не презришь, Боже.
\vs Psa 50:20 Облагодетельствуй, [Господи,] по благоволению Твоему Сион; воздвигни стены Иерусалима:
\vs Psa 50:21 тогда благоугодны будут Тебе жертвы правды, возношение и всесожжение; тогда возложат на алтарь Твой тельцов.
\vs Psa 51:1 Начальнику хора. Учение Давида,
\vs Psa 51:2 после того, как приходил Доик Идумеянин и донес Саулу и сказал ему, что Давид пришел в дом Ахимелеха.
\rsbpar\vs Psa 51:3 Что хвалишься злодейством, сильный? милость Божия всегда \bibemph{со мною};
\vs Psa 51:4 гибель вымышляет язык твой; как изощренная бритва, он \bibemph{у тебя}, коварный!
\vs Psa 51:5 ты любишь больше зло, нежели добро, больше ложь, нежели говорить правду;
\vs Psa 51:6 ты любишь всякие гибельные речи, язык коварный:
\vs Psa 51:7 за то Бог сокрушит тебя вконец, изринет тебя и исторгнет тебя из жилища [твоего] и корень твой из земли живых.
\vs Psa 51:8 Увидят праведники и убоятся, посмеются над ним [и скажут]:
\vs Psa 51:9 <<вот человек, который не в Боге полагал крепость свою, а надеялся на множество богатства своего, укреплялся в злодействе своем>>.
\vs Psa 51:10 А я, как зеленеющая маслина, в доме Божием, и уповаю на милость Божию во веки веков,
\vs Psa 51:11 вечно буду славить Тебя за то, что Ты соделал, и уповать на имя Твое, ибо оно благо пред святыми Твоими.
\vs Psa 52:1 Начальнику хора. На духовом \bibemph{орудии}. Учение Давида.
\rsbpar\vs Psa 52:2 Сказал безумец в сердце своем: <<нет Бога>>. Развратились они и совершили гнусные преступления; нет делающего добро.
\vs Psa 52:3 Бог с небес призрел на сынов человеческих, чтобы видеть, есть ли разумеющий, ищущий Бога.
\vs Psa 52:4 Все уклонились, сделались равно непотребными; нет делающего добро, нет ни одного.
\vs Psa 52:5 Неужели не вразумятся делающие беззаконие, съедающие народ мой, \bibemph{как} едят хлеб, и не призывающие Бога?
\vs Psa 52:6 Там убоятся они страха, где нет страха, ибо рассыплет Бог кости ополчающихся против тебя. Ты постыдишь их, потому что Бог отверг их.
\vs Psa 52:7 Кто даст с Сиона спасение Израилю! Когда Бог возвратит пленение народа Своего, тогда возрадуется Иаков и возвеселится Израиль.
\vs Psa 53:1 Начальнику хора. На струнных \bibemph{орудиях}. Учение Давида,
\vs Psa 53:2 когда пришли Зифеи и сказали Саулу: <<не у нас ли скрывается Давид?>>
\rsbpar\vs Psa 53:3 Боже! именем Твоим спаси меня, и силою Твоею суди меня.
\vs Psa 53:4 Боже! услышь молитву мою, внемли словам уст моих,
\vs Psa 53:5 ибо чужие восстали на меня, и сильные ищут души моей; они не имеют Бога пред собою.
\vs Psa 53:6 Вот, Бог помощник мой; Господь подкрепляет душу мою.
\vs Psa 53:7 Он воздаст за зло врагам моим; истиною Твоею истреби их.
\vs Psa 53:8 Я усердно принесу Тебе жертву, прославлю имя Твое, Господи, ибо оно благо,
\vs Psa 53:9 ибо Ты избавил меня от всех бед, и на врагов моих смотрело око мое.
\vs Psa 54:1 Начальнику хора. На струнных \bibemph{орудиях}. Учение Давида.
\rsbpar\vs Psa 54:2 Услышь, Боже, молитву мою и не скрывайся от моления моего;
\vs Psa 54:3 внемли мне и услышь меня; я стенаю в горести моей, и смущаюсь
\vs Psa 54:4 от голоса врага, от притеснения нечестивого, ибо они возводят на меня беззаконие и в гневе враждуют против меня.
\vs Psa 54:5 Сердце мое трепещет во мне, и смертные ужасы напали на меня;
\vs Psa 54:6 страх и трепет нашел на меня, и ужас объял меня.
\vs Psa 54:7 И я сказал: <<кто дал бы мне крылья, как у голубя? я улетел бы и успокоился бы;
\vs Psa 54:8 далеко удалился бы я, и оставался бы в пустыне;
\vs Psa 54:9 поспешил бы укрыться от вихря, от бури>>.
\vs Psa 54:10 Расстрой, Господи, и раздели языки их, ибо я вижу насилие и распри в городе;
\vs Psa 54:11 днем и ночью ходят они кругом по стенам его; злодеяния и бедствие посреди его;
\vs Psa 54:12 посреди его пагуба; обман и коварство не сходят с улиц его:
\vs Psa 54:13 ибо не враг поносит меня,~--- это я перенес бы; не ненавистник мой величается надо мною,~--- от него я укрылся бы;
\vs Psa 54:14 но ты, который был для меня то же, что я, друг мой и близкий мой,
\vs Psa 54:15 с которым мы разделяли искренние беседы и ходили вместе в дом Божий.
\vs Psa 54:16 Да найдет на них смерть; да сойдут они живыми в ад, ибо злодейство в жилищах их, посреди их.
\vs Psa 54:17 Я же воззову к Богу, и Господь спасет меня.
\vs Psa 54:18 Вечером и утром и в полдень буду умолять и вопиять, и Он услышит голос мой,
\vs Psa 54:19 избавит в мире душу мою от восстающих на меня, ибо их много у меня;
\vs Psa 54:20 услышит Бог, и смирит их от века Живущий, потому что нет в них перемены; они не боятся Бога,
\vs Psa 54:21 простерли руки свои на тех, которые с ними в мире, нарушили союз свой;
\vs Psa 54:22 уста их мягче масла, а в сердце их вражда; слова их нежнее елея, но они суть обнаженные мечи.
\vs Psa 54:23 Возложи на Господа заботы твои, и Он поддержит тебя. Никогда не даст Он поколебаться праведнику.
\vs Psa 54:24 Ты, Боже, низведешь их в ров погибели; кровожадные и коварные не доживут и до половины дней своих. А я на Тебя, [Господи,] уповаю.
\vs Psa 55:1 Начальнику хора. О голубице, безмолвствующей в удалении. Писание Давида, когда Филистимляне захватили его в Гефе.
\rsbpar\vs Psa 55:2 Помилуй меня, Боже! ибо человек хочет поглотить меня; нападая всякий день, теснит меня.
\vs Psa 55:3 Враги мои всякий день ищут поглотить меня, ибо много восстающих на меня, о, Всевышний!
\vs Psa 55:4 Когда я в страхе, на Тебя я уповаю.
\vs Psa 55:5 В Боге восхвалю я слово Его; на Бога уповаю, не боюсь; что сделает мне плоть?
\vs Psa 55:6 Всякий день извращают слова мои; все помышления их обо мне~--- на зло:
\vs Psa 55:7 собираются, притаиваются, наблюдают за моими пятами, чтобы уловить душу мою.
\vs Psa 55:8 Неужели они избегнут воздаяния за неправду \bibemph{свою}? Во гневе низложи, Боже, народы.
\vs Psa 55:9 У Тебя исчислены мои скитания; положи слезы мои в сосуд у Тебя,~--- не в книге ли они Твоей?
\vs Psa 55:10 Враги мои обращаются назад, когда я взываю к Тебе, из этого я узна\acc{ю}, что Бог за меня.
\vs Psa 55:11 В Боге восхвалю я слово \bibemph{Его}, в Господе восхвалю слово \bibemph{Его}.
\vs Psa 55:12 На Бога уповаю, не боюсь; что сделает мне человек?
\vs Psa 55:13 На мне, Боже, обеты Тебе; Тебе воздам хвалы,
\vs Psa 55:14 ибо Ты избавил душу мою от смерти, [очи мои от слез,] да и ноги мои от преткновения, чтобы я ходил пред лицем Божиим во свете живых.
\vs Psa 56:1 Начальнику хора. Не погуби. Писание Давида, когда он убежал от Саула в пещеру.
\rsbpar\vs Psa 56:2 Помилуй меня, Боже, помилуй меня, ибо на Тебя уповает душа моя, и в тени крыл Твоих я укроюсь, доколе не пройдут беды.
\vs Psa 56:3 Воззову к Богу Всевышнему, Богу, благодетельствующему мне;
\vs Psa 56:4 Он пошлет с небес и спасет меня; посрамит ищущего поглотить меня; пошлет Бог милость Свою и истину Свою.
\vs Psa 56:5 Душа моя среди львов; я лежу среди дышущих пламенем, среди сынов человеческих, у которых зубы~--- копья и стрелы, и у которых язык~--- острый меч.
\vs Psa 56:6 Будь превознесен выше небес, Боже, и над всею землею да будет слава Твоя!
\vs Psa 56:7 Приготовили сеть ногам моим; душа моя поникла; выкопали предо мною яму, и \bibemph{сами} упали в нее.
\vs Psa 56:8 Готово сердце мое, Боже, готово сердце мое: буду петь и славить.
\vs Psa 56:9 Воспрянь, слава моя, воспрянь, псалтирь и гусли! Я встану рано.
\vs Psa 56:10 Буду славить Тебя, Господи, между народами; буду воспевать Тебя среди племен,
\vs Psa 56:11 ибо до небес велика милость Твоя и до облаков истина Твоя.
\vs Psa 56:12 Будь превознесен выше небес, Боже, и над всею землею да будет слава Твоя!
\vs Psa 57:1 Начальнику хора. Не погуби. Писание Давида.
\rsbpar\vs Psa 57:2 Подлинно ли правду говорите вы, судьи, и справедливо судите, сыны человеческие?
\vs Psa 57:3 Беззаконие составляете в сердце, кладете на весы злодеяния рук ваших на земле.
\vs Psa 57:4 С самого рождения отступили нечестивые, от утробы \bibemph{матери} заблуждаются, говоря ложь.
\vs Psa 57:5 Яд у них~--- как яд змеи, как глухого аспида, который затыкает уши свои
\vs Psa 57:6 и не слышит голоса заклинателя, самого искусного в заклинаниях.
\vs Psa 57:7 Боже! сокруши зубы их в устах их; разбей, Господи, челюсти львов!
\vs Psa 57:8 Да исчезнут, как вода протекающая; когда напрягут стрелы, пусть они будут как переломленные.
\vs Psa 57:9 Да исчезнут, как распускающаяся улитка; да не видят солнца, как выкидыш женщины.
\vs Psa 57:10 Прежде нежели котлы ваши ощутят горящий терн, и свежее и обгоревшее да разнесет вихрь.
\vs Psa 57:11 Возрадуется праведник, когда увидит отмщение; омоет стопы свои в крови нечестивого.
\vs Psa 57:12 И скажет человек: <<подлинно есть плод праведнику! итак есть Бог, судящий на земле!>>
\vs Psa 58:1 Начальнику хора. Не погуби. Писание Давида, когда Саул послал стеречь дом его, чтобы умертвить его.
\rsbpar\vs Psa 58:2 Избавь меня от врагов моих, Боже мой! защити меня от восстающих на меня;
\vs Psa 58:3 избавь меня от делающих беззаконие; спаси от кровожадных,
\vs Psa 58:4 ибо вот, они подстерегают душу мою; собираются на меня сильные не за преступление мое и не за грех мой, Господи;
\vs Psa 58:5 без вины \bibemph{моей} сбегаются и вооружаются; подвигнись на помощь мне и воззри.
\vs Psa 58:6 Ты, Господи, Боже сил, Боже Израилев, восстань посетить все народы, не пощади ни одного из нечестивых беззаконников:
\vs Psa 58:7 вечером возвращаются они, воют, как псы, и ходят вокруг города;
\vs Psa 58:8 вот они изрыгают хулу языком своим; в устах их мечи: <<ибо>>, \bibemph{думают они}, <<кто слышит?>>
\vs Psa 58:9 Но Ты, Господи, посмеешься над ними; Ты посрамишь все народы.
\vs Psa 58:10 Сила~--- у них, но я к Тебе прибегаю, ибо Бог~--- заступник мой.
\vs Psa 58:11 Бог мой, милующий меня, предварит меня; Бог даст мне смотреть на врагов моих.
\vs Psa 58:12 Не умерщвляй их, чтобы не забыл народ мой; расточи их силою Твоею и низложи их, Господи, защитник наш.
\vs Psa 58:13 Слово языка их есть грех уст их, да уловятся они в гордости своей за клятву и ложь, которую произносят.
\vs Psa 58:14 Расточи их во гневе, расточи, чтобы их не было; и да познают, что Бог владычествует над Иаковом до пределов земли.
\vs Psa 58:15 Пусть возвращаются вечером, воют, как псы, и ходят вокруг города;
\vs Psa 58:16 пусть бродят, чтобы найти пищу, и несытые проводят ночи.
\vs Psa 58:17 А я буду воспевать силу Твою и с раннего утра провозглашать милость Твою, ибо Ты был мне защитою и убежищем в день бедствия моего.
\vs Psa 58:18 Сила моя! Тебя буду воспевать я, ибо Бог~--- заступник мой, Бог мой, милующий меня.
\vs Psa 59:1 Начальнику хора. На \bibemph{музыкальном орудии} Шушан-Эдуф. Писание Давида для изучения,
\vs Psa 59:2 когда он воевал с Сириею Месопотамскою и с Сириею Цованскою, и когда Иоав, возвращаясь, поразил двенадцать тысяч Идумеев в долине Соляной.
\rsbpar\vs Psa 59:3 Боже! Ты отринул нас, Ты сокрушил нас, Ты прогневался: обратись к нам.
\vs Psa 59:4 Ты потряс землю, разбил ее: исцели повреждения ее, ибо она колеблется.
\vs Psa 59:5 Ты дал испытать народу твоему жестокое, напоил нас вином изумления.
\vs Psa 59:6 Даруй боящимся Тебя знамя, чтобы они подняли его ради истины,
\vs Psa 59:7 чтобы избавились возлюбленные Твои; спаси десницею Твоею и услышь меня.
\vs Psa 59:8 Бог сказал во святилище Своем: <<восторжествую, разделю Сихем и долину Сокхоф размерю:
\vs Psa 59:9 Мой Галаад, Мой Манассия, Ефрем крепость главы Моей, Иуда скипетр Мой,
\vs Psa 59:10 Моав умывальная чаша Моя; на Едома простру сапог Мой. Восклицай Мне, земля Филистимская!>>
\vs Psa 59:11 Кто введет меня в укрепленный город? Кто доведет меня до Едома?
\vs Psa 59:12 Не Ты ли, Боже, \bibemph{Который} отринул нас, и не выходишь, Боже, с войсками нашими?
\vs Psa 59:13 Подай нам помощь в тесноте, ибо защита человеческая суетна.
\vs Psa 59:14 С Богом мы окажем силу, Он низложит врагов наших.
\vs Psa 60:1 Начальнику хора. На струнном \bibemph{орудии}. Псалом Давида.
\rsbpar\vs Psa 60:2 Услышь, Боже, вопль мой, внемли молитве моей!
\vs Psa 60:3 От конца земли взываю к Тебе в унынии сердца моего; возведи меня на скалу, для меня недосягаемую,
\vs Psa 60:4 ибо Ты прибежище мое, Ты крепкая защита от врага.
\vs Psa 60:5 Да живу я вечно в жилище Твоем и покоюсь под кровом крыл Твоих,
\vs Psa 60:6 ибо Ты, Боже, услышал обеты мои и дал \bibemph{мне} наследие боящихся имени Твоего.
\vs Psa 60:7 Приложи дни ко дням царя, лета его \bibemph{продли} в род и род,
\vs Psa 60:8 да пребудет он вечно пред Богом; заповедуй милости и истине охранять его.
\vs Psa 60:9 И я буду петь имени Твоему вовек, исполняя обеты мои всякий день.
\vs Psa 61:1 Начальнику хора Идифумова. Псалом Давида.
\rsbpar\vs Psa 61:2 Только в Боге успокаивается душа моя: от Него спасение мое.
\vs Psa 61:3 Только Он~--- твердыня моя, спасение мое, убежище мое: не поколеблюсь более.
\vs Psa 61:4 Доколе вы будете налегать на человека? Вы будете низринуты, все вы, как наклонившаяся стена, как ограда пошатнувшаяся.
\vs Psa 61:5 Они задумали свергнуть его с высоты, прибегли ко лжи; устами благословляют, а в сердце своем клянут.
\vs Psa 61:6 Только в Боге успокаивайся, душа моя! ибо на Него надежда моя.
\vs Psa 61:7 Только Он~--- твердыня моя и спасение мое, убежище мое: не поколеблюсь.
\vs Psa 61:8 В Боге спасение мое и слава моя; крепость силы моей и упование мое в Боге.
\vs Psa 61:9 Народ! надейтесь на Него во всякое время; изливайте пред Ним сердце ваше: Бог нам прибежище.
\vs Psa 61:10 Сыны человеческие~--- только суета; сыны мужей~--- ложь; если положить их на весы, все они вместе легче пустоты.
\vs Psa 61:11 Не надейтесь на грабительство и не тщеславьтесь хищением; когда богатство умножается, не прилагайте \bibemph{к нему} сердца.
\vs Psa 61:12 Однажды сказал Бог, и дважды слышал я это, что сила у Бога,
\vs Psa 61:13 и у Тебя, Господи, милость, ибо Ты воздаешь каждому по делам его.
\vs Psa 62:1 Псалом Давида, когда он был в пустыне Иудейской.
\rsbpar\vs Psa 62:2 Боже! Ты Бог мой, Тебя от ранней зари ищу я; Тебя жаждет душа моя, по Тебе томится плоть моя в земле пустой, иссохшей и безводной,
\vs Psa 62:3 чтобы видеть силу Твою и славу Твою, как я видел Тебя во святилище:
\vs Psa 62:4 ибо милость Твоя лучше, нежели жизнь. Уста мои восхвалят Тебя.
\vs Psa 62:5 Так благословлю Тебя в жизни моей; во имя Твое вознесу руки мои.
\vs Psa 62:6 Как туком и елеем насыщается душа моя, и радостным гласом восхваляют Тебя уста мои,
\vs Psa 62:7 когда я вспоминаю о Тебе на постели моей, размышляю о Тебе в \bibemph{ночные} стражи,
\vs Psa 62:8 ибо Ты помощь моя, и в тени крыл Твоих я возрадуюсь;
\vs Psa 62:9 к Тебе прилепилась душа моя; десница Твоя поддерживает меня.
\vs Psa 62:10 А те, которые ищут погибели душе моей, сойдут в преисподнюю земли;
\vs Psa 62:11 сразят их силою меча; достанутся они в добычу лисицам.
\vs Psa 62:12 Царь же возвеселится о Боге, восхвален будет всякий, клянущийся Им, ибо заградятся уста говорящих неправду.
\vs Psa 63:1 Начальнику хора. Псалом Давида.
\rsbpar\vs Psa 63:2 Услышь, Боже, голос мой в молитве моей, сохрани жизнь мою от страха врага;
\vs Psa 63:3 укрой меня от замысла коварных, от мятежа злодеев,
\vs Psa 63:4 которые изострили язык свой, как меч; напрягли лук свой~--- язвительное слово,
\vs Psa 63:5 чтобы втайне стрелять в непорочного; они внезапно стреляют в него и не боятся.
\vs Psa 63:6 Они утвердились в злом намерении, совещались скрыть сеть, говорили: кто их увидит?
\vs Psa 63:7 Изыскивают неправду, делают расследование за расследованием даже до внутренней жизни человека и до глубины сердца.
\vs Psa 63:8 Но поразит их Бог стрелою: внезапно будут они уязвлены;
\vs Psa 63:9 языком своим они поразят самих себя; все, видящие их, удалятся \bibemph{от них}.
\vs Psa 63:10 И убоятся все человеки, и возвестят дело Божие, и уразумеют, что это Его дело.
\vs Psa 63:11 А праведник возвеселится о Господе и будет уповать на Него; и похвалятся все правые сердцем.
\vs Psa 64:1 Начальнику хора. Псалом Давида для пения.
\rsbpar\vs Psa 64:2 Тебе, Боже, принадлежит хвала на Сионе, и Тебе воздастся обет [в Иерусалиме].
\vs Psa 64:3 Ты слышишь молитву; к Тебе прибегает всякая плоть.
\vs Psa 64:4 Дела беззаконий превозмогают меня; Ты очистишь преступления наши.
\vs Psa 64:5 Блажен, кого Ты избрал и приблизил, чтобы он жил во дворах Твоих. Насытимся благами дома Твоего, святаго храма Твоего.
\vs Psa 64:6 Страшный в правосудии, услышь нас, Боже, Спаситель наш, упование всех концов земли и находящихся в море далеко,
\vs Psa 64:7 поставивший горы силою Своею, препоясанный могуществом,
\vs Psa 64:8 укрощающий шум морей, шум волн их и мятеж народов!
\vs Psa 64:9 И убоятся знамений Твоих живущие на пределах \bibemph{земли}. Утро и вечер возбудишь к славе \bibemph{Твоей}.
\vs Psa 64:10 Ты посещаешь землю и утоляешь жажду ее, обильно обогащаешь ее: поток Божий полон воды; Ты приготовляешь хлеб, ибо так устроил ее;
\vs Psa 64:11 напояешь борозды ее, уравниваешь глыбы ее, размягчаешь ее каплями дождя, благословляешь произрастания ее;
\vs Psa 64:12 венчаешь лето благости Твоей, и стези Твои источают тук,
\vs Psa 64:13 источают на пустынные пажити, и холмы препоясываются радостью;
\vs Psa 64:14 луга одеваются стадами, и долины покрываются хлебом, восклицают и поют.
\vs Psa 65:0 Начальнику хора. Песнь.
\rsbpar\vs Psa 65:1 Воскликните Богу, вся земля.
\vs Psa 65:2 Пойте славу имени Его, воздайте славу, хвалу Ему.
\vs Psa 65:3 Скажите Богу: как страшен Ты в делах Твоих! По множеству силы Твоей, покорятся Тебе враги Твои.
\vs Psa 65:4 Вся земля да поклонится Тебе и поет Тебе, да поет имени Твоему, [Вышний]!
\vs Psa 65:5 Придите и воззрите на дела Бога, страшного в делах над сынами человеческими.
\vs Psa 65:6 Он превратил море в сушу; через реку перешли стопами, там веселились мы о Нем.
\vs Psa 65:7 Могуществом Своим владычествует Он вечно; очи Его зрят на народы, да не возносятся мятежники.
\vs Psa 65:8 Благословите, народы, Бога нашего и провозгласите хвалу Ему.
\vs Psa 65:9 Он сохранил душе нашей жизнь и ноге нашей не дал поколебаться.
\vs Psa 65:10 Ты испытал нас, Боже, переплавил нас, как переплавляют серебро.
\vs Psa 65:11 Ты ввел нас в сеть, положил оковы на чресла наши,
\vs Psa 65:12 посадил человека на главу нашу. Мы вошли в огонь и в воду, и Ты вывел нас на свободу.
\vs Psa 65:13 Войду в дом Твой со всесожжениями, воздам Тебе обеты мои,
\vs Psa 65:14 которые произнесли уста мои и изрек язык мой в скорби моей.
\vs Psa 65:15 Всесожжения тучные вознесу Тебе с воскурением тука овнов, принесу в жертву волов и козлов.
\vs Psa 65:16 Придите, послушайте, все боящиеся Бога, и я возвещу \bibemph{вам}, что сотворил Он для души моей.
\vs Psa 65:17 Я воззвал к Нему устами моими и превознес Его языком моим.
\vs Psa 65:18 Если бы я видел беззаконие в сердце моем, то не услышал бы меня Господь.
\vs Psa 65:19 Но Бог услышал, внял гласу моления моего.
\vs Psa 65:20 Благословен Бог, Который не отверг молитвы моей и не отвратил от меня милости Своей.
\vs Psa 66:1 Начальнику хора. На струнных \bibemph{орудиях}. Псалом. Песнь.
\rsbpar\vs Psa 66:2 Боже! будь милостив к нам и благослови нас, освети нас лицем Твоим,
\vs Psa 66:3 дабы познали на земле путь Твой, во всех народах спасение Твое.
\vs Psa 66:4 Да восхвалят Тебя народы, Боже; да восхвалят Тебя народы все.
\vs Psa 66:5 Да веселятся и радуются племена, ибо Ты судишь народы праведно и управляешь на земле племенами.
\vs Psa 66:6 Да восхвалят Тебя народы, Боже, да восхвалят Тебя народы все.
\vs Psa 66:7 Земля дала плод свой; да благословит нас Бог, Бог наш.
\vs Psa 66:8 Да благословит нас Бог, и да убоятся Его все пределы земли.
\vs Psa 67:1 Начальнику хора. Псалом Давида. Песнь.
\rsbpar\vs Psa 67:2 Да восстанет Бог\fns{В славянском переводе: Да воскреснет Бог\dots}, и расточатся враги Его, и да бегут от лица Его ненавидящие Его.
\vs Psa 67:3 Как рассеивается дым, Ты рассей их; как тает воск от огня, так нечестивые да погибнут от лица Божия.
\vs Psa 67:4 А праведники да возвеселятся, да возрадуются пред Богом и восторжествуют в радости.
\vs Psa 67:5 Пойте Богу нашему, пойте имени Его, превозносите Шествующего на небесах; имя Ему: Господь, и радуйтесь пред лицем Его.
\vs Psa 67:6 Отец сирот и судья вдов Бог во святом Своем жилище.
\vs Psa 67:7 Бог одиноких вводит в дом, освобождает узников от оков, а непокорные остаются в знойной пустыне.
\vs Psa 67:8 Боже! когда Ты выходил пред народом Твоим, когда Ты шествовал пустынею,
\vs Psa 67:9 земля тряслась, даже небеса таяли от лица Божия, и этот Синай~--- от лица Бога, Бога Израилева.
\vs Psa 67:10 Обильный дождь проливал Ты, Боже, на наследие Твое, и когда оно изнемогало от труда, Ты подкреплял его.
\vs Psa 67:11 Народ Твой обитал там; по благости Твоей, Боже, Ты готовил \bibemph{необходимое} для бедного.
\vs Psa 67:12 Господь даст слово: провозвестниц великое множество.
\vs Psa 67:13 Цари воинств бегут, бегут, а сидящая дома делит добычу.
\vs Psa 67:14 Расположившись в уделах [своих], вы стали, как голубица, которой крылья покрыты серебром, а перья чистым золотом:
\vs Psa 67:15 когда Всемогущий рассеял царей на сей \bibemph{земле}, она забелела, как снег на Селмоне.
\vs Psa 67:16 Гора Божия~--- гора Васанская! гора высокая~--- гора Васанская!
\vs Psa 67:17 что вы завистливо смотрите, горы высокие, на гору, на которой Бог благоволит обитать и будет Господь обитать вечно?
\vs Psa 67:18 Колесниц Божиих тьмы, тысячи тысяч; среди их Господь на Синае, во святилище.
\vs Psa 67:19 Ты восшел на высоту, пленил плен, принял дары для человеков, так чтоб и из противящихся могли обитать у Господа Бога.
\vs Psa 67:20 Благословен Господь всякий день. Бог возлагает на нас бремя, но Он же и спасает нас.
\vs Psa 67:21 Бог для нас~--- Бог во спасение; во власти Господа Вседержителя врата смерти.
\vs Psa 67:22 Но Бог сокрушит голову врагов Своих, волосатое темя закоснелого в своих беззакониях.
\vs Psa 67:23 Господь сказал: <<от Васана возвращу, выведу из глубины морской,
\vs Psa 67:24 чтобы ты погрузил ногу твою, как и псы твои язык свой, в крови врагов>>.
\vs Psa 67:25 Видели шествие Твое, Боже, шествие Бога моего, Царя моего во святыне:
\vs Psa 67:26 впереди шли поющие, позади играющие на орудиях, в средине девы с тимпанами:
\vs Psa 67:27 <<в собраниях благословите \bibemph{Бога Господа}, вы~--- от семени Израилева!>>
\vs Psa 67:28 Там Вениамин младший~--- князь их; князья Иудины~--- владыки их, князья Завулоновы, князья Неффалимовы.
\vs Psa 67:29 Бог твой предназначил тебе силу. Утверди, Боже, то, что Ты соделал для нас!
\vs Psa 67:30 Ради храма Твоего в Иерусалиме цари принесут Тебе дары.
\vs Psa 67:31 Укроти зверя в тростнике, стадо волов среди тельцов народов, хвалящихся слитками серебра; рассыпь народы, желающие браней.
\vs Psa 67:32 Придут вельможи из Египта; Ефиопия прострет руки свои к Богу.
\vs Psa 67:33 Царства земные! пойте Богу, воспевайте Господа,
\vs Psa 67:34 шествующего на небесах небес от века. Вот, Он дает гласу Своему глас силы.
\vs Psa 67:35 Воздайте славу Богу! величие Его~--- над Израилем, и могущество Его~--- на облаках.
\vs Psa 67:36 Страшен Ты, Боже, во святилище Твоем. Бог Израилев~--- Он дает силу и крепость народу [Своему]. Благословен Бог!
\vs Psa 68:1 Начальнику хора. На Шошанниме. Псалом Давида.
\rsbpar\vs Psa 68:2 Спаси меня, Боже, ибо воды дошли до души [моей].
\vs Psa 68:3 Я погряз в глубоком болоте, и не на чем стать; вошел во глубину вод, и быстрое течение их увлекает меня.
\vs Psa 68:4 Я изнемог от вопля, засохла гортань моя, истомились глаза мои от ожидания Бога [моего].
\vs Psa 68:5 Ненавидящих меня без вины больше, нежели волос на голове моей; враги мои, преследующие меня несправедливо, усилились; чего я не отнимал, то должен отдать.
\vs Psa 68:6 Боже! Ты знаешь безумие мое, и грехи мои не сокрыты от Тебя.
\vs Psa 68:7 Да не постыдятся во мне все, надеющиеся на Тебя, Господи, Боже сил. Да не посрамятся во мне ищущие Тебя, Боже Израилев,
\vs Psa 68:8 ибо ради Тебя несу я поношение, и бесчестием покрывают лице мое.
\vs Psa 68:9 Чужим стал я для братьев моих и посторонним для сынов матери моей,
\vs Psa 68:10 ибо ревность по доме Твоем снедает меня, и злословия злословящих Тебя падают на меня;
\vs Psa 68:11 и пл\acc{а}чу, постясь душею моею, и это ставят в поношение мне;
\vs Psa 68:12 и возлагаю на себя вместо одежды вретище,~--- и делаюсь для них притчею;
\vs Psa 68:13 о мне толкуют сидящие у ворот, и поют в песнях пьющие вино.
\vs Psa 68:14 А я с молитвою моею к Тебе, Господи; во время благоугодное, Боже, по великой благости Твоей услышь меня в истине спасения Твоего;
\vs Psa 68:15 извлеки меня из тины, чтобы не погрязнуть мне; да избавлюсь от ненавидящих меня и от глубоких вод;
\vs Psa 68:16 да не увлечет меня стремление вод, да не поглотит меня пучина, да не затворит надо мною пропасть зева своего.
\vs Psa 68:17 Услышь меня, Господи, ибо блага милость Твоя; по множеству щедрот Твоих призри на меня;
\vs Psa 68:18 не скрывай лица Твоего от раба Твоего, ибо я скорблю; скоро услышь меня;
\vs Psa 68:19 приблизься к душе моей, избавь ее; ради врагов моих спаси меня.
\vs Psa 68:20 Ты знаешь поношение мое, стыд мой и посрамление мое: враги мои все пред Тобою.
\vs Psa 68:21 Поношение сокрушило сердце мое, и я изнемог, ждал сострадания, но нет его,~--- утешителей, но не нахожу.
\vs Psa 68:22 И дали мне в пищу желчь, и в жажде моей напоили меня уксусом.
\vs Psa 68:23 Да будет трапеза их сетью им, и мирное пиршество их~--- западнею;
\vs Psa 68:24 да помрачатся глаза их, чтоб им не видеть, и чресла их расслабь навсегда;
\vs Psa 68:25 излей на них ярость Твою, и пламень гнева Твоего да обымет их;
\vs Psa 68:26 жилище их да будет пусто, и в шатрах их да не будет живущих,
\vs Psa 68:27 ибо, кого Ты поразил, они \bibemph{еще} преследуют, и страдания уязвленных Тобою умножают.
\vs Psa 68:28 Приложи беззаконие к беззаконию их, и да не войдут они в правду Твою;
\vs Psa 68:29 да изгладятся они из книги живых и с праведниками да не напишутся.
\vs Psa 68:30 А я беден и страдаю; помощь Твоя, Боже, да восставит меня.
\vs Psa 68:31 Я буду славить имя Бога [моего] в песни, буду превозносить Его в славословии,
\vs Psa 68:32 и будет это благоугоднее Господу, нежели вол, нежели телец с рогами и с копытами.
\vs Psa 68:33 Увидят \bibemph{это} страждущие и возрадуются. И оживет сердце ваше, ищущие Бога,
\vs Psa 68:34 ибо Господь внемлет нищим и не пренебрегает узников Своих.
\vs Psa 68:35 Да восхвалят Его небеса и земля, моря и все движущееся в них;
\vs Psa 68:36 ибо спасет Бог Сион, создаст города Иудины, и поселятся там и наследуют его,
\vs Psa 68:37 и потомство рабов Его утвердится в нем, и любящие имя Его будут поселяться на нем.
\vs Psa 69:1 Начальнику хора. Псалом Давида. В воспоминание.
\rsbpar\vs Psa 69:2 Поспеши, Боже, избавить меня, \bibemph{поспеши}, Господи, на помощь мне.
\vs Psa 69:3 Да постыдятся и посрамятся ищущие души моей! Да будут обращены назад и преданы посмеянию желающие мне зла!
\vs Psa 69:4 Да будут обращены назад за поношение меня говорящие [мне]: <<хорошо! хорошо!>>
\vs Psa 69:5 Да возрадуются и возвеселятся о Тебе все, ищущие Тебя, и любящие спасение Твое да говорят непрестанно: <<велик Бог!>>
\vs Psa 69:6 Я же беден и нищ; Боже, поспеши ко мне! Ты помощь моя и Избавитель мой; Господи! не замедли.
\vs Psa 70:1 На Тебя, Господи, уповаю, да не постыжусь вовек.
\vs Psa 70:2 По правде Твоей избавь меня и освободи меня; приклони ухо Твое ко мне и спаси меня.
\vs Psa 70:3 Будь мне твердым прибежищем, куда я всегда мог бы укрываться; Ты заповедал спасти меня, ибо твердыня моя и крепость моя~--- Ты.
\vs Psa 70:4 Боже мой! избавь меня из руки нечестивого, из руки беззаконника и притеснителя,
\vs Psa 70:5 ибо Ты~--- надежда моя, Господи Боже, упование мое от юности моей.
\vs Psa 70:6 На Тебе утверждался я от утробы; Ты извел меня из чрева матери моей; Тебе хвала моя не престанет.
\vs Psa 70:7 Для многих я был как бы дивом, но Ты твердая моя надежда.
\vs Psa 70:8 Да наполнятся уста мои хвалою, [чтобы мне воспевать славу Твою,] всякий день великолепие Твое.
\vs Psa 70:9 Не отвергни меня во время старости; когда будет оскудевать сила моя, не оставь меня,
\vs Psa 70:10 ибо враги мои говорят против меня, и подстерегающие душу мою советуются между собою,
\vs Psa 70:11 говоря: <<Бог оставил его; преследуйте и схватите его, ибо нет избавляющего>>.
\vs Psa 70:12 Боже! не удаляйся от меня; Боже мой! поспеши на помощь мне.
\vs Psa 70:13 Да постыдятся и исчезнут враждующие против души моей, да покроются стыдом и бесчестием ищущие мне зла!
\vs Psa 70:14 А я всегда буду уповать [на Тебя] и умножать всякую хвалу Тебе.
\vs Psa 70:15 Уста мои будут возвещать правду Твою, всякий день благодеяния Твои; ибо я не знаю им числа.
\vs Psa 70:16 Войду в \bibemph{размышление} о силах Господа Бога; воспомяну правду Твою~--- единственно Твою.
\vs Psa 70:17 Боже! Ты наставлял меня от юности моей, и доныне я возвещаю чудеса Твои.
\vs Psa 70:18 И до старости, и до седины не оставь меня, Боже, доколе не возвещу силы Твоей роду сему и всем грядущим могущества Твоего.
\vs Psa 70:19 Правда Твоя, Боже, до превыспренних; великие дела соделал Ты; Боже, кто подобен Тебе?
\vs Psa 70:20 Ты посылал на меня многие и лютые беды, но и опять оживлял меня и из бездн земли опять выводил меня.
\vs Psa 70:21 Ты возвышал меня и утешал меня, [и из бездн земли выводил меня].
\vs Psa 70:22 И я буду славить Тебя на псалтири, Твою истину, Боже мой; буду воспевать Тебя на гуслях, Святый Израилев!
\vs Psa 70:23 Радуются уста мои, когда я пою Тебе, и душа моя, которую Ты избавил;
\vs Psa 70:24 и язык мой всякий день будет возвещать правду Твою, ибо постыжены и посрамлены ищущие мне зла.
\vs Psa 71:0 О Соломоне. [Псалом Давида.]
\rsbpar\vs Psa 71:1 Боже! даруй царю Твой суд и сыну царя Твою правду,
\vs Psa 71:2 да судит праведно людей Твоих и нищих Твоих на суде;
\vs Psa 71:3 да принесут горы мир людям и холмы правду;
\vs Psa 71:4 да судит нищих народа, да спасет сынов убогого и смирит притеснителя,~---
\vs Psa 71:5 и будут бояться Тебя, доколе пребудут солнце и луна, в роды родов.
\vs Psa 71:6 Он сойдет, как дождь на скошенный луг, как капли, орошающие землю;
\vs Psa 71:7 во дни его процветет праведник, и будет обилие мира, доколе не престанет луна;
\vs Psa 71:8 он будет обладать от моря до моря и от реки\fns{Евфрат.} до концов земли;
\vs Psa 71:9 падут пред ним жители пустынь, и враги его будут лизать прах;
\vs Psa 71:10 цари Фарсиса и островов поднесут ему дань; цари Аравии и Савы принесут дары;
\vs Psa 71:11 и поклонятся ему все цари; все народы будут служить ему;
\vs Psa 71:12 ибо он избавит нищего, вопиющего и угнетенного, у которого нет помощника.
\vs Psa 71:13 Будет милосерд к нищему и убогому, и души убогих спасет;
\vs Psa 71:14 от коварства и насилия избавит души их, и драгоценна будет кровь их пред очами его;
\vs Psa 71:15 и будет жить, и будут давать ему от золота Аравии, и будут молиться о нем непрестанно, всякий день благословлять его;
\vs Psa 71:16 будет обилие хлеба на земле, наверху гор; плоды его будут волноваться, как \bibemph{лес} на Ливане, и в городах размножатся люди, как трава на земле;
\vs Psa 71:17 будет имя его [благословенно] вовек; доколе пребывает солнце, будет передаваться имя его\fns{В славянском переводе: Прежде солнца пребывает имя его.}; и благословятся в нем [все племена земные], все народы ублажат его.
\vs Psa 71:18 Благословен Господь Бог, Бог Израилев, един творящий чудеса,
\vs Psa 71:19 и благословенно имя славы Его вовек, и наполнится славою Его вся земля. Аминь и аминь.
\vs Psa 71:20 Кончились молитвы Давида, сына Иесеева.
\vs Psa 72:0 Псалом Асафа.
\rsbpar\vs Psa 72:1 Как благ Бог к Израилю, к чистым сердцем!
\vs Psa 72:2 А я~--- едва не пошатнулись ноги мои, едва не поскользнулись стопы мои,~---
\vs Psa 72:3 я позавидовал безумным, видя благоденствие нечестивых,
\vs Psa 72:4 ибо им нет страданий до смерти их, и крепки силы их;
\vs Psa 72:5 на работе человеческой нет их, и с \bibemph{прочими} людьми не подвергаются ударам.
\vs Psa 72:6 Оттого гордость, как ожерелье, обложила их, и дерзость, \bibemph{как} наряд, одевает их;
\vs Psa 72:7 выкатились от жира глаза их, бродят помыслы в сердце;
\vs Psa 72:8 над всем издеваются, злобно разглашают клевету, говорят свысока;
\vs Psa 72:9 поднимают к небесам уста свои, и язык их расхаживает по земле.
\vs Psa 72:10 Потому туда же обращается народ Его, и пьют воду полною чашею,
\vs Psa 72:11 и говорят: <<как узнает Бог? и есть ли ведение у Вышнего?>>
\vs Psa 72:12 И вот, эти нечестивые благоденствуют в веке сем, умножают богатство.
\vs Psa 72:13 [И я сказал:] так не напрасно ли я очищал сердце мое и омывал в невинности руки мои,
\vs Psa 72:14 и подвергал себя ранам всякий день и обличениям всякое утро?
\vs Psa 72:15 \bibemph{Но} если бы я сказал: <<буду рассуждать так>>,~--- то я виновен был бы пред родом сынов Твоих.
\vs Psa 72:16 И думал я, как бы уразуметь это, но это трудно было в глазах моих,
\vs Psa 72:17 доколе не вошел я во святилище Божие и не уразумел конца их.
\vs Psa 72:18 Так! на скользких путях поставил Ты их и низвергаешь их в пропасти.
\vs Psa 72:19 Как нечаянно пришли они в разорение, исчезли, погибли от ужасов!
\vs Psa 72:20 Как сновидение по пробуждении, так Ты, Господи, пробудив \bibemph{их}, уничтожишь мечты их.
\vs Psa 72:21 Когда кипело сердце мое, и терзалась внутренность моя,
\vs Psa 72:22 тогда я был невежда и не разумел; как скот был я пред Тобою.
\vs Psa 72:23 Но я всегда с Тобою: Ты держишь меня за правую руку;
\vs Psa 72:24 Ты руководишь меня советом Твоим и потом примешь меня в славу.
\vs Psa 72:25 Кто мне на небе? и с Тобою ничего не хочу на земле.
\vs Psa 72:26 Изнемогает плоть моя и сердце мое: Бог твердыня сердца моего и часть моя вовек.
\vs Psa 72:27 Ибо вот, удаляющие себя от Тебя гибнут; Ты истребляешь всякого отступающего от Тебя.
\vs Psa 72:28 А мне благо приближаться к Богу! На Господа Бога я возложил упование мое, чтобы возвещать все дела Твои [во вратах дщери Сионовой].
\vs Psa 73:0 Учение Асафа.
\rsbpar\vs Psa 73:1 Для чего, Боже, отринул нас навсегда? возгорелся гнев Твой на овец пажити Твоей?
\vs Psa 73:2 Вспомни сонм Твой, \bibemph{который} Ты стяжал издревле, искупил в жезл достояния Твоего,~--- эту гору Сион, на которой Ты вселился.
\vs Psa 73:3 Подвигни стопы Твои к вековым развалинам: все разрушил враг во святилище.
\vs Psa 73:4 Рыкают враги Твои среди собраний Твоих; поставили знаки свои вместо знамений \bibemph{наших};
\vs Psa 73:5 показывали себя подобными поднимающему вверх секиру на сплетшиеся ветви дерева;
\vs Psa 73:6 и ныне все резьбы в нем в один раз разрушили секирами и бердышами;
\vs Psa 73:7 предали огню святилище Твое; совсем осквернили жилище имени Твоего;
\vs Psa 73:8 сказали в сердце своем: <<разорим их совсем>>,~--- и сожгли все места собраний Божиих на земле.
\vs Psa 73:9 Знамений наших мы не видим, нет уже пророка, и нет с нами, кто знал бы, доколе \bibemph{это будет}.
\vs Psa 73:10 Доколе, Боже, будет поносить враг? вечно ли будет хулить противник имя Твое?
\vs Psa 73:11 Для чего отклоняешь руку Твою и десницу Твою? Из среды недра Твоего порази \bibemph{их}.
\vs Psa 73:12 Боже, Царь мой от века, устрояющий спасение посреди земли!
\vs Psa 73:13 Ты расторг силою Твоею море, Ты сокрушил головы змиев в воде;
\vs Psa 73:14 Ты сокрушил голову левиафана, отдал его в пищу людям пустыни, [Ефиопским];
\vs Psa 73:15 Ты иссек источник и поток, Ты иссушил сильные реки.
\vs Psa 73:16 Твой день и Твоя ночь: Ты уготовал светила и солнце;
\vs Psa 73:17 Ты установил все пределы земли, лето и зиму Ты учредил.
\vs Psa 73:18 Вспомни же: враг поносит Господа, и люди безумные хулят имя Твое.
\vs Psa 73:19 Не предай зверям душу горлицы Твоей; собрания убогих Твоих не забудь навсегда.
\vs Psa 73:20 Призри на завет Твой; ибо наполнились все мрачные места земли жилищами насилия.
\vs Psa 73:21 Да не возвратится угнетенный посрамленным; нищий и убогий да восхвалят имя Твое.
\vs Psa 73:22 Восстань, Боже, защити дело Твое, вспомни вседневное поношение Твое от безумного;
\vs Psa 73:23 не забудь крика врагов Твоих; шум восстающих против Тебя непрестанно поднимается.
\vs Psa 74:1 Начальнику хора. Не погуби. Псалом Асафа. Песнь.
\rsbpar\vs Psa 74:2 Славим Тебя, Боже, славим, ибо близко имя Твое; возвещают чудеса Твои.
\vs Psa 74:3 <<Когда изберу время, Я произведу суд по правде.
\vs Psa 74:4 Колеблется земля и все живущие на ней: Я утвержу столпы ее>>.
\vs Psa 74:5 Говорю безумствующим: <<не безумствуйте>>, и нечестивым: <<не поднимайте р\acc{о}га,
\vs Psa 74:6 не поднимайте высоко р\acc{о}га вашего, [не] говорите [на Бога] жестоковыйно>>,
\vs Psa 74:7 ибо не от востока и не от запада и не от пустыни возвышение,
\vs Psa 74:8 но Бог есть судия: одного унижает, а другого возносит;
\vs Psa 74:9 ибо чаша в руке Господа, вино кипит в ней, полное смешения, и Он наливает из нее. Даже дрожжи ее будут выжимать и пить все нечестивые земли.
\vs Psa 74:10 А я буду возвещать вечно, буду воспевать Бога Иаковлева,
\vs Psa 74:11 все роги нечестивых сломлю, и вознесутся роги праведника.
\vs Psa 75:1 Начальнику хора. На струнных \bibemph{орудиях}. Псалом Асафа. Песнь.
\rsbpar\vs Psa 75:2 Ведом в Иудее Бог; у Израиля велико имя Его.
\vs Psa 75:3 И было в Салиме жилище Его и пребывание Его на Сионе.
\vs Psa 75:4 Там сокрушил Он стрелы лука, щит и меч и брань.
\vs Psa 75:5 Ты славен, могущественнее гор хищнических.
\vs Psa 75:6 Крепкие сердцем стали добычею, уснули сном своим, и не нашли все мужи силы рук своих.
\vs Psa 75:7 От прещения Твоего, Боже Иакова, вздремали и колесница и конь.
\vs Psa 75:8 Ты страшен, и кто устоит пред лицем Твоим во время гнева Твоего?
\vs Psa 75:9 С небес Ты возвестил суд; земля убоялась и утихла,
\vs Psa 75:10 когда восстал Бог на суд, чтобы спасти всех угнетенных земли.
\vs Psa 75:11 И гнев человеческий обратится во славу Тебе: остаток гнева Ты укротишь.
\vs Psa 75:12 Делайте и воздавайте обеты Господу, Богу вашему; все, которые вокруг Него, да принесут дары Страшному:
\vs Psa 75:13 Он укрощает дух князей, Он страшен для царей земных.
\vs Psa 76:1 Начальнику хора Идифумова. Псалом Асафа.
\rsbpar\vs Psa 76:2 Глас мой к Богу, и я буду взывать; глас мой к Богу, и Он услышит меня.
\vs Psa 76:3 В день скорби моей ищу Господа; рука моя простерта ночью и не опускается; душа моя отказывается от утешения.
\vs Psa 76:4 Вспоминаю о Боге и трепещу; помышляю, и изнемогает дух мой.
\vs Psa 76:5 Ты не даешь мне сомкнуть очей моих; я потрясен и не могу говорить.
\vs Psa 76:6 Размышляю о днях древних, о летах веков \bibemph{минувших};
\vs Psa 76:7 припоминаю песни мои в ночи, беседую с сердцем моим, и дух мой испытывает:
\vs Psa 76:8 неужели навсегда отринул Господь, и не будет более благоволить?
\vs Psa 76:9 неужели навсегда престала милость Его, и пресеклось слово Его в род и род?
\vs Psa 76:10 неужели Бог забыл миловать? Неужели во гневе затворил щедроты Свои?
\vs Psa 76:11 И сказал я: <<вот мое горе~--- изменение десницы Всевышнего>>.
\vs Psa 76:12 Буду вспоминать о делах Господа; буду вспоминать о чудесах Твоих древних;
\vs Psa 76:13 буду вникать во все дела Твои, размышлять о великих Твоих деяниях.
\vs Psa 76:14 Боже! свят путь Твой. Кто Бог так великий, как Бог [наш]!
\vs Psa 76:15 Ты~--- Бог, творящий чудеса; Ты явил могущество Свое среди народов;
\vs Psa 76:16 Ты избавил мышцею народ Твой, сынов Иакова и Иосифа.
\vs Psa 76:17 Видели Тебя, Боже, воды, видели Тебя воды и убоялись, и вострепетали бездны.
\vs Psa 76:18 Облака изливали воды, тучи издавали гром, и стрелы Твои летали.
\vs Psa 76:19 Глас грома Твоего в круге небесном; молнии освещали вселенную; земля содрогалась и тряслась.
\vs Psa 76:20 Путь Твой в море, и стезя Твоя в водах великих, и следы Твои неведомы.
\vs Psa 76:21 Как стадо, вел Ты народ Твой рукою Моисея и Аарона.
\vs Psa 77:0 Учение Асафа.
\rsbpar\vs Psa 77:1 Внимай, народ мой, закону моему, приклоните ухо ваше к словам уст моих.
\vs Psa 77:2 Открою уста мои в притче и произнесу гадания из древности.
\vs Psa 77:3 Что слышали мы и узнали, и отцы наши рассказали нам,
\vs Psa 77:4 не скроем от детей их, возвещая роду грядущему славу Господа, и силу Его, и чудеса Его, которые Он сотворил.
\vs Psa 77:5 Он постановил устав в Иакове и положил закон в Израиле, который заповедал отцам нашим возвещать детям их,
\vs Psa 77:6 чтобы знал грядущий род, дети, которые родятся, и чтобы они в свое время возвещали своим детям,~---
\vs Psa 77:7 возлагать надежду свою на Бога и не забывать дел Божиих, и хранить заповеди Его,
\vs Psa 77:8 и не быть подобными отцам их, роду упорному и мятежному, неустроенному сердцем и неверному Богу духом своим.
\vs Psa 77:9 Сыны Ефремовы, вооруженные, стреляющие из луков, обратились назад в день брани:
\vs Psa 77:10 они не сохранили завета Божия и отреклись ходить в законе Его;
\vs Psa 77:11 забыли дела Его и чудеса, которые Он явил им.
\vs Psa 77:12 Он пред глазами отцов их сотворил чудеса в земле Египетской, на поле Цоан:
\vs Psa 77:13 разделил море, и провел их чрез него, и поставил воды стеною;
\vs Psa 77:14 и днем вел их облаком, а во всю ночь светом огня;
\vs Psa 77:15 рассек камень в пустыне и напоил их, как из великой бездны;
\vs Psa 77:16 из скалы извел потоки, и воды потекли, как реки.
\vs Psa 77:17 Но они продолжали грешить пред Ним и раздражать Всевышнего в пустыне:
\vs Psa 77:18 искушали Бога в сердце своем, требуя пищи по душе своей,
\vs Psa 77:19 и говорили против Бога и сказали: <<может ли Бог приготовить трапезу в пустыне?>>
\vs Psa 77:20 Вот, Он ударил в камень, и потекли воды, и полились ручьи. <<Может ли Он дать и хлеб, может ли приготовлять мясо народу Своему?>>
\vs Psa 77:21 Господь услышал и воспламенился гневом, и огонь возгорелся на Иакова, и гнев подвигнулся на Израиля
\vs Psa 77:22 за то, что не веровали в Бога и не уповали на спасение Его.
\vs Psa 77:23 Он повелел облакам свыше и отверз двери неба,
\vs Psa 77:24 и одождил на них манну в пищу, и хлеб небесный дал им.
\vs Psa 77:25 Хлеб ангельский ел человек; послал Он им пищу до сытости.
\vs Psa 77:26 Он возбудил на небе восточный ветер и навел южный силою Своею
\vs Psa 77:27 и, как пыль, одождил на них мясо и, как песок морской, птиц пернатых:
\vs Psa 77:28 поверг их среди стана их, около жилищ их,~---
\vs Psa 77:29 и они ели и пресытились; и желаемое ими дал им.
\vs Psa 77:30 Но еще не прошла прихоть их, еще пища была в устах их,
\vs Psa 77:31 гнев Божий пришел на них, убил тучных их и юношей Израилевых низложил.
\vs Psa 77:32 При всем этом они продолжали грешить и не верили чудесам Его.
\vs Psa 77:33 И погубил дни их в суете и лета их в смятении.
\vs Psa 77:34 Когда Он убивал их, они искали Его и обращались, и с раннего утра прибегали к Богу,
\vs Psa 77:35 и вспоминали, что Бог~--- их прибежище, и Бог Всевышний~--- Избавитель их,
\vs Psa 77:36 и льстили Ему устами своими и языком своим лгали пред Ним;
\vs Psa 77:37 сердце же их было неправо пред Ним, и они не были верны завету Его.
\vs Psa 77:38 Но Он, Милостивый, прощал грех и не истреблял их, многократно отвращал гнев Свой и не возбуждал всей ярости Своей:
\vs Psa 77:39 Он помнил, что они плоть, дыхание, которое уходит и не возвращается.
\vs Psa 77:40 Сколько раз они раздражали Его в пустыне и прогневляли Его в \bibemph{стране} необитаемой!
\vs Psa 77:41 и снова искушали Бога и оскорбляли Святаго Израилева,
\vs Psa 77:42 не помнили рук\acc{и} Его, дня, когда Он избавил их от угнетения,
\vs Psa 77:43 когда сотворил в Египте знамения Свои и чудеса Свои на поле Цоан;
\vs Psa 77:44 и превратил реки их и потоки их в кровь, чтобы они не могли пить;
\vs Psa 77:45 послал на них насекомых, чтобы жалили их, и жаб, чтобы губили их;
\vs Psa 77:46 земные произрастения их отдал гусенице и труд их~--- саранче;
\vs Psa 77:47 виноград их побил градом и сикоморы их~--- льдом;
\vs Psa 77:48 скот их предал граду и стада их~--- молниям;
\vs Psa 77:49 послал на них пламень гнева Своего, и негодование, и ярость и бедствие, посольство злых ангелов;
\vs Psa 77:50 уравнял стезю гневу Своему, не охранял души их от смерти, и скот их предал моровой язве;
\vs Psa 77:51 поразил всякого первенца в Египте, начатки сил в шатрах Хамовых;
\vs Psa 77:52 и повел народ Свой, как овец, и вел их, как стадо, пустынею;
\vs Psa 77:53 вел их безопасно, и они не страшились, а врагов их покрыло море;
\vs Psa 77:54 и привел их в область святую Свою, на гору сию, которую стяжала десница Его;
\vs Psa 77:55 прогнал от лица их народы и землю их разделил в наследие им, и колена Израилевы поселил в шатрах их.
\vs Psa 77:56 Но они еще искушали и огорчали Бога Всевышнего, и уставов Его не сохраняли;
\vs Psa 77:57 отступали и изменяли, как отцы их, обращались назад, как неверный лук;
\vs Psa 77:58 огорчали Его высотами своими и истуканами своими возбуждали ревность Его.
\vs Psa 77:59 Услышал Бог и воспламенился гневом и сильно вознегодовал на Израиля;
\vs Psa 77:60 отринул жилище в Силоме, скинию, в которой обитал Он между человеками;
\vs Psa 77:61 и отдал в плен крепость Свою и славу Свою в руки врага,
\vs Psa 77:62 и предал мечу народ Свой и прогневался на наследие Свое.
\vs Psa 77:63 Юношей его поедал огонь, и девицам его не пели брачных песен;
\vs Psa 77:64 священники его падали от меча, и вдовы его не плакали.
\vs Psa 77:65 Но, как бы от сна, воспрянул Господь, как бы исполин, побежденный вином,
\vs Psa 77:66 и поразил врагов его в тыл, вечному сраму предал их;
\vs Psa 77:67 и отверг шатер Иосифов и колена Ефремова не избрал,
\vs Psa 77:68 а избрал колено Иудино, гору Сион, которую возлюбил.
\vs Psa 77:69 И устроил, как небо, святилище Свое и, как землю, утвердил его навек,
\vs Psa 77:70 и избрал Давида, раба Своего, и взял его от дворов овчих
\vs Psa 77:71 и от доящих привел его пасти народ Свой, Иакова, и наследие Свое, Израиля.
\vs Psa 77:72 И он пас их в чистоте сердца своего и руками мудрыми водил их.
\vs Psa 78:0 Псалом Асафа.
\rsbpar\vs Psa 78:1 Боже! язычники пришли в наследие Твое, осквернили святый храм Твой, Иерусалим превратили в развалины;
\vs Psa 78:2 трупы рабов Твоих отдали на съедение птицам небесным, тела святых Твоих~--- зверям земным;
\vs Psa 78:3 пролили кровь их, как воду, вокруг Иерусалима, и некому было похоронить их.
\vs Psa 78:4 Мы сделались посмешищем у соседей наших, поруганием и посрамлением у окружающих нас.
\vs Psa 78:5 Доколе, Господи, будешь гневаться непрестанно, будет пылать ревность Твоя, как огонь?
\vs Psa 78:6 Пролей гнев Твой на народы, которые не знают Тебя, и на царства, которые имени Твоего не призывают,
\vs Psa 78:7 ибо они пожрали Иакова и жилище его опустошили.
\vs Psa 78:8 Не помяни нам грехов \bibemph{наших} предков; скоро да предварят нас щедроты Твои, ибо мы весьма истощены.
\vs Psa 78:9 Помоги нам, Боже, Спаситель наш, ради славы имени Твоего; избавь нас и прости нам грехи наши ради имени Твоего.
\vs Psa 78:10 Для чего язычникам говорить: <<где Бог их?>> Да сделается известным между язычниками пред глазами нашими отмщение за пролитую кровь рабов Твоих.
\vs Psa 78:11 Да придет пред лице Твое стенание узника; могуществом мышцы Твоей сохрани обреченных на смерть.
\vs Psa 78:12 Семикратно возврати соседям нашим в недро их поношение, которым они Тебя, Господи, поносили.
\vs Psa 78:13 А мы, народ Твой и Твоей пажити овцы, вечно будем славить Тебя и в род и род возвещать хвалу Тебе.
\vs Psa 79:1 Начальнику хора. На музыкальном \bibemph{орудии} Шошанним-Эдуф. Псалом Асафа.
\rsbpar\vs Psa 79:2 Пастырь Израиля! внемли; водящий, как овец, Иосифа, восседающий на Херувимах, яви Себя.
\vs Psa 79:3 Пред Ефремом и Вениамином и Манассиею воздвигни силу Твою, и приди спасти нас.
\vs Psa 79:4 Боже! восстанови нас; да воссияет лице Твое, и спасемся!
\vs Psa 79:5 Господи, Боже сил! доколе будешь гневен к молитвам народа Твоего?
\vs Psa 79:6 Ты напитал их хлебом слезным, и напоил их слезами в большой мере,
\vs Psa 79:7 положил нас в пререкание соседям нашим, и враги наши издеваются \bibemph{над нами}.
\vs Psa 79:8 Боже сил! восстанови нас; да воссияет лице Твое, и спасемся!
\vs Psa 79:9 Из Египта перенес Ты виноградную лозу, выгнал народы и посадил ее;
\vs Psa 79:10 очистил для нее место, и утвердил корни ее, и она наполнила землю.
\vs Psa 79:11 Горы покрылись тенью ее, и ветви ее как кедры Божии;
\vs Psa 79:12 она пустила ветви свои до моря и отрасли свои до реки.
\vs Psa 79:13 Для чего разрушил Ты ограды ее, так что обрывают ее все, проходящие по пути?
\vs Psa 79:14 Лесной вепрь подрывает ее, и полевой зверь объедает ее.
\vs Psa 79:15 Боже сил! обратись же, призри с неба, и воззри, и посети виноград сей;
\vs Psa 79:16 охрани то, что насадила десница Твоя, и отрасли, которые Ты укрепил Себе.
\vs Psa 79:17 Он пожжен огнем, обсечен; от прещения лица Твоего погибнут.
\vs Psa 79:18 Да будет рука Твоя над мужем десницы Твоей, над сыном человеческим, которого Ты укрепил Себе,
\vs Psa 79:19 и мы не отступим от Тебя; оживи нас, и мы будем призывать имя Твое.
\vs Psa 79:20 Господи, Боже сил! восстанови нас; да воссияет лице Твое, и спасемся!
\vs Psa 80:1 Начальнику хора. На Гефском орудии. Псалом Асафа.
\rsbpar\vs Psa 80:2 Радостно пойте Богу, твердыне нашей; восклицайте Богу Иакова;
\vs Psa 80:3 возьмите псалом, дайте тимпан, сладкозвучные гусли с псалтирью;
\vs Psa 80:4 трубите в новомесячие трубою, в определенное время, в день праздника нашего;
\vs Psa 80:5 ибо это закон для Израиля, устав от Бога Иаковлева.
\vs Psa 80:6 Он установил это во свидетельство для Иосифа, когда он вышел из земли Египетской, где услышал звуки языка, которого не знал:
\vs Psa 80:7 <<Я снял с рамен его тяжести, и руки его освободились от корзин.
\vs Psa 80:8 В бедствии ты призвал Меня, и Я избавил тебя; из среды грома Я услышал тебя, при водах Меривы испытал тебя.
\vs Psa 80:9 Слушай, народ Мой, и Я буду свидетельствовать тебе: Израиль! о, если бы ты послушал Меня!
\vs Psa 80:10 Да не будет у тебя иного бога, и не поклоняйся богу чужеземному.
\vs Psa 80:11 Я Господь, Бог твой, изведший тебя из земли Египетской; открой уста твои, и Я наполню их>>.
\vs Psa 80:12 Но народ Мой не слушал гласа Моего, и Израиль не покорялся Мне;
\vs Psa 80:13 потому Я оставил их упорству сердца их, пусть ходят по своим помыслам.
\vs Psa 80:14 О, если бы народ Мой слушал Меня и Израиль ходил Моими путями!
\vs Psa 80:15 Я скоро смирил бы врагов их и обратил бы руку Мою на притеснителей их:
\vs Psa 80:16 ненавидящие Господа раболепствовали бы им, а их благоденствие продолжалось бы навсегда;
\vs Psa 80:17 Я питал бы их туком пшеницы и насыщал бы их медом из скалы.
\vs Psa 81:0 Псалом Асафа.
\rsbpar\vs Psa 81:1 Бог стал в сонме богов; среди богов произнес суд:
\vs Psa 81:2 доколе будете вы судить неправедно и оказывать лицеприятие нечестивым?
\vs Psa 81:3 Давайте суд бедному и сироте; угнетенному и нищему оказывайте справедливость;
\vs Psa 81:4 избавляйте бедного и нищего; исторгайте \bibemph{его} из руки нечестивых.
\vs Psa 81:5 Не знают, не разумеют, во тьме ходят; все основания земли колеблются.
\vs Psa 81:6 Я сказал: вы~--- боги, и сыны Всевышнего~--- все вы;
\vs Psa 81:7 но вы умрете, как человеки, и падете, как всякий из князей.
\vs Psa 81:8 Восстань\fns{В славянском переводе: Воскресни\dots}, Боже, суди землю, ибо Ты наследуешь все народы.
\vs Psa 82:1 Песнь. Псалом Асафа.
\rsbpar\vs Psa 82:2 Боже! Не премолчи, не безмолвствуй и не оставайся в покое, Боже,
\vs Psa 82:3 ибо вот, враги Твои шумят, и ненавидящие Тебя подняли голову;
\vs Psa 82:4 против народа Твоего составили коварный умысел и совещаются против хранимых Тобою;
\vs Psa 82:5 сказали: <<пойдем и истребим их из народов, чтобы не вспоминалось более имя Израиля>>.
\vs Psa 82:6 Сговорились единодушно, заключили против Тебя союз:
\vs Psa 82:7 селения Едомовы и Измаильтяне, Моав и Агаряне,
\vs Psa 82:8 Гевал и Аммон и Амалик, Филистимляне с жителями Тира.
\vs Psa 82:9 И Ассур пристал к ним: они стали мышцею для сынов Лотовых.
\vs Psa 82:10 Сделай им то же, что Мадиаму, что Сисаре, что Иавину у потока Киссона,
\vs Psa 82:11 которые истреблены в Аендоре, сделались навозом для земли.
\vs Psa 82:12 Поступи с ними, с князьями их, как с Оривом и Зивом и со всеми вождями их, как с Зевеем и Салманом,
\vs Psa 82:13 которые говорили: <<возьмем себе во владение селения Божии>>.
\vs Psa 82:14 Боже мой! Да будут они, как пыль в вихре, как солома перед ветром.
\vs Psa 82:15 Как огонь сжигает лес, и как пламя опаляет горы,
\vs Psa 82:16 так погони их бурею Твоею и вихрем Твоим приведи их в смятение;
\vs Psa 82:17 исполни лица их бесчестием, чтобы они взыскали имя Твое, Господи!
\vs Psa 82:18 Да постыдятся и смятутся на веки, да посрамятся и погибнут,
\vs Psa 82:19 и да познают, что Ты, Которого одного имя Господь, Всевышний над всею землею.
\vs Psa 83:1 Начальнику хора. На Гефском \bibemph{орудии}. Кореевых сынов. Псалом.
\rsbpar\vs Psa 83:2 Как вожделенны жилища Твои, Господи сил!
\vs Psa 83:3 Истомилась душа моя, желая во дворы Господни; сердце мое и плоть моя восторгаются к Богу живому.
\vs Psa 83:4 И птичка находит себе жилье, и ласточка гнездо себе, где положить птенцов своих, у алтарей Твоих, Господи сил, Царь мой и Бог мой!
\vs Psa 83:5 Блаженны живущие в доме Твоем: они непрестанно будут восхвалять Тебя.
\vs Psa 83:6 Блажен человек, которого сила в Тебе и у которого в сердце стези направлены \bibemph{к Тебе}.
\vs Psa 83:7 Проходя долиною плача, они открывают в ней источники, и дождь покрывает ее благословением;
\vs Psa 83:8 приходят от силы в силу, являются пред Богом на Сионе.
\vs Psa 83:9 Господи, Боже сил! Услышь молитву мою, внемли, Боже Иаковлев!
\vs Psa 83:10 Боже, защитник наш! Приникни и призри на лице помазанника Твоего.
\vs Psa 83:11 Ибо один день во дворах Твоих лучше тысячи. Желаю лучше быть у порога в доме Божием, нежели жить в шатрах нечестия.
\vs Psa 83:12 Ибо Господь Бог есть солнце и щит, Господь дает благодать и славу; ходящих в непорочности Он не лишает благ.
\vs Psa 83:13 Господи сил! Блажен человек, уповающий на Тебя!
\vs Psa 84:1 Начальнику хора. Кореевых сынов. Псалом.
\rsbpar\vs Psa 84:2 Господи! Ты умилосердился к земле Твоей, возвратил плен Иакова;
\vs Psa 84:3 простил беззаконие народа Твоего, покрыл все грехи его,
\vs Psa 84:4 отъял всю ярость Твою, отвратил лютость гнева Твоего.
\vs Psa 84:5 Восстанови нас, Боже спасения нашего, и прекрати негодование Твое на нас.
\vs Psa 84:6 Неужели вечно будешь гневаться на нас, прострешь гнев Твой от рода в род?
\vs Psa 84:7 Неужели снова не оживишь нас, чтобы народ Твой возрадовался о Тебе?
\vs Psa 84:8 Яви нам, Господи, милость Твою, и спасение Твое даруй нам.
\vs Psa 84:9 Послушаю, что скажет Господь Бог. Он скажет мир народу Своему и избранным Своим, но да не впадут они снова в безрассудство.
\vs Psa 84:10 Так, близко к боящимся Его спасение Его, чтобы обитала слава в земле нашей!
\vs Psa 84:11 Милость и истина сретятся, правда и мир облобызаются;
\vs Psa 84:12 истина возникнет из земли, и правда приникнет с небес;
\vs Psa 84:13 и Господь даст благо, и земля наша даст плод свой;
\vs Psa 84:14 правда пойдет пред Ним и поставит на путь стопы свои.
\vs Psa 85:0 Молитва Давида.
\rsbpar\vs Psa 85:1 Приклони, Господи, ухо Твое и услышь меня, ибо я беден и нищ.
\vs Psa 85:2 Сохрани душу мою, ибо я благоговею пред Тобою; спаси, Боже мой, раба Твоего, уповающего на Тебя.
\vs Psa 85:3 Помилуй меня, Господи, ибо к Тебе взываю каждый день.
\vs Psa 85:4 Возвесели душу раба Твоего, ибо к Тебе, Господи, возношу душу мою,
\vs Psa 85:5 ибо Ты, Господи, благ и милосерд и многомилостив ко всем, призывающим Тебя.
\vs Psa 85:6 Услышь, Господи, молитву мою и внемли гласу моления моего.
\vs Psa 85:7 В день скорби моей взываю к Тебе, потому что Ты услышишь меня.
\vs Psa 85:8 Нет между богами, как Ты, Господи, и нет дел, как Твои.
\vs Psa 85:9 Все народы, Тобою сотворенные, приидут и поклонятся пред Тобою, Господи, и прославят имя Твое,
\vs Psa 85:10 ибо Ты велик и творишь чудеса,~--- Ты, Боже, един Ты.
\vs Psa 85:11 Наставь меня, Господи, на путь Твой, и буду ходить в истине Твоей; утверди сердце мое в страхе имени Твоего.
\vs Psa 85:12 Буду восхвалять Тебя, Господи, Боже мой, всем сердцем моим и славить имя Твое вечно,
\vs Psa 85:13 ибо велика милость Твоя ко мне: Ты избавил душу мою от ада преисподнего.
\vs Psa 85:14 Боже! гордые восстали на меня, и скопище мятежников ищет души моей: не представляют они Тебя пред собою.
\vs Psa 85:15 Но Ты, Господи, Боже щедрый и благосердный, долготерпеливый и многомилостивый и истинный,
\vs Psa 85:16 призри на меня и помилуй меня; даруй крепость Твою рабу Твоему, и спаси сына рабы Твоей;
\vs Psa 85:17 покажи на мне знамение во благо, да видят ненавидящие меня и устыдятся, потому что Ты, Господи, помог мне и утешил меня.
\vs Psa 86:1 Сынов Кореевых. Псалом. Песнь.
\rsbpar\vs Psa 86:2 Основание его\fns{Иерусалима.} на горах святых. Господь любит врата Сиона более всех селений Иакова.
\vs Psa 86:3 Славное возвещается о тебе, град Божий!
\vs Psa 86:4 Упомяну знающим меня о Рааве\fns{О Египте.} и Вавилоне; вот Филистимляне и Тир с Ефиопиею,~--- \bibemph{скажут}: <<такой-то родился там>>.
\vs Psa 86:5 О Сионе же будут говорить: <<такой-то и такой-то муж родился в нем, и Сам Всевышний укрепил его>>.
\vs Psa 86:6 Господь в переписи народов напишет: <<такой-то родился там>>.
\vs Psa 86:7 И поющие и играющие,~--- все источники мои в тебе.
\vs Psa 87:1 Песнь. Псалом, Сынов Кореевых. Начальнику хора на Махалаф, для пения. Учение Емана Езрахита.
\rsbpar\vs Psa 87:2 Господи, Боже спасения моего! днем вопию и ночью пред Тобою:
\vs Psa 87:3 да внидет пред лице Твое молитва моя; приклони ухо Твое к молению моему,
\vs Psa 87:4 ибо душа моя насытилась бедствиями, и жизнь моя приблизилась к преисподней.
\vs Psa 87:5 Я сравнялся с нисходящими в могилу; я стал, как человек без силы,
\vs Psa 87:6 между мертвыми брошенный,~--- как убитые, лежащие во гробе, о которых Ты уже не вспоминаешь и которые от руки Твоей отринуты.
\vs Psa 87:7 Ты положил меня в ров преисподний, во мрак, в бездну.
\vs Psa 87:8 Отяготела на мне ярость Твоя, и всеми волнами Твоими Ты поразил [меня].
\vs Psa 87:9 Ты удалил от меня знакомых моих, сделал меня отвратительным для них; я заключен, и не могу выйти.
\vs Psa 87:10 Око мое истомилось от горести: весь день я взывал к Тебе, Господи, простирал к Тебе руки мои.
\vs Psa 87:11 Разве над мертвыми Ты сотворишь чудо? Разве мертвые встанут и будут славить Тебя?
\vs Psa 87:12 или во гробе будет возвещаема милость Твоя, и истина Твоя~--- в месте тления?
\vs Psa 87:13 разве во мраке позн\acc{а}ют чудеса Твои, и в земле забвения~--- правду Твою?
\vs Psa 87:14 Но я к Тебе, Господи, взываю, и рано утром молитва моя предваряет Тебя.
\vs Psa 87:15 Для чего, Господи, отреваешь душу мою, скрываешь лице Твое от меня?
\vs Psa 87:16 Я несчастен и истаеваю с юности; несу ужасы Твои и изнемогаю.
\vs Psa 87:17 Надо мною прошла ярость Твоя, устрашения Твои сокрушили меня,
\vs Psa 87:18 всякий день окружают меня, как вода: облегают меня все вместе.
\vs Psa 87:19 Ты удалил от меня друга и искреннего; знакомых моих не видно.
\vs Psa 88:1 Учение Ефама Езрахита.
\rsbpar\vs Psa 88:2 Милости [Твои], Господи, буду петь вечно, в род и род возвещать истину Твою устами моими.
\vs Psa 88:3 Ибо говорю: навек основана милость, на небесах утвердил Ты истину Твою, \bibemph{когда сказал}:
\vs Psa 88:4 <<Я поставил завет с избранным Моим, клялся Давиду, рабу Моему:
\vs Psa 88:5 навек утвержу семя твое, в род и род устрою престол твой>>.
\vs Psa 88:6 И небеса прославят чудные дела Твои, Господи, и истину Твою в собрании святых.
\vs Psa 88:7 Ибо кто на небесах сравнится с Господом? кто между сынами Божиими уподобится Господу?
\vs Psa 88:8 Страшен Бог в великом сонме святых, страшен Он для всех окружающих Его.
\vs Psa 88:9 Господи, Боже сил! кто силен, как Ты, Господи? И истина Твоя окрест Тебя.
\vs Psa 88:10 Ты владычествуешь над яростью моря: когда воздымаются волны его, Ты укрощаешь их.
\vs Psa 88:11 Ты низложил Раава, как пораженного; крепкою мышцею Твоею рассеял врагов Твоих.
\vs Psa 88:12 Твои небеса и Твоя земля; вселенную и что наполняет ее, Ты основал.
\vs Psa 88:13 Север и юг Ты сотворил; Фавор и Ермон о имени Твоем радуются.
\vs Psa 88:14 Крепка мышца Твоя, сильна рука Твоя, высока десница Твоя!
\vs Psa 88:15 Правосудие и правота~--- основание престола Твоего; милость и истина предходят пред лицем Твоим.
\vs Psa 88:16 Блажен народ, знающий трубный зов! Они ходят во свете лица Твоего, Господи,
\vs Psa 88:17 о имени Твоем радуются весь день и правдою Твоею возносятся,
\vs Psa 88:18 ибо Ты украшение силы их, и благоволением Твоим возвышается рог наш.
\vs Psa 88:19 От Господа~--- щит наш, и от Святаго Израилева~--- царь наш.
\vs Psa 88:20 Некогда говорил Ты в видении святому Твоему, и сказал: <<Я оказал помощь мужественному, вознес избранного из народа.
\vs Psa 88:21 Я обрел Давида, раба Моего, святым елеем Моим помазал его.
\vs Psa 88:22 Рука Моя пребудет с ним, и мышца Моя укрепит его.
\vs Psa 88:23 Враг не превозможет его, и сын беззакония не притеснит его.
\vs Psa 88:24 Сокрушу пред ним врагов его и поражу ненавидящих его.
\vs Psa 88:25 И истина Моя и милость Моя с ним, и Моим именем возвысится рог его.
\vs Psa 88:26 И положу на море руку его, и на реки~--- десницу его.
\vs Psa 88:27 Он будет звать Меня: Ты отец мой, Бог мой и твердыня спасения моего.
\vs Psa 88:28 И Я сделаю его первенцем, превыше царей земли,
\vs Psa 88:29 вовек сохраню ему милость Мою, и завет Мой с ним будет верен.
\vs Psa 88:30 И продолжу вовек семя его, и престол его~--- как дни неба.
\vs Psa 88:31 Если сыновья его оставят закон Мой и не будут ходить по заповедям Моим;
\vs Psa 88:32 если нарушат уставы Мои и повелений Моих не сохранят:
\vs Psa 88:33 посещу жезлом беззаконие их, и ударами~--- неправду их;
\vs Psa 88:34 милости же Моей не отниму от него, и не изменю истины Моей.
\vs Psa 88:35 Не нарушу завета Моего, и не переменю того, что вышло из уст Моих.
\vs Psa 88:36 Однажды Я поклялся святостью Моею: солгу ли Давиду?
\vs Psa 88:37 Семя его пребудет вечно, и престол его, как солнце, предо Мною,
\vs Psa 88:38 вовек будет тверд, как луна, и верный свидетель на небесах>>.
\vs Psa 88:39 Но \bibemph{ныне} Ты отринул и презрел, прогневался на помазанника Твоего;
\vs Psa 88:40 пренебрег завет с рабом Твоим, поверг на землю венец его;
\vs Psa 88:41 разрушил все ограды его, превратил в развалины крепости его.
\vs Psa 88:42 Расхищают его все проходящие путем; он сделался посмешищем у соседей своих.
\vs Psa 88:43 Ты возвысил десницу противников его, обрадовал всех врагов его;
\vs Psa 88:44 Ты обратил назад острие меча его и не укрепил его на брани;
\vs Psa 88:45 отнял у него блеск и престол его поверг на землю;
\vs Psa 88:46 сократил дни юности его и покрыл его стыдом.
\vs Psa 88:47 Доколе, Господи, будешь скрываться непрестанно, будет пылать ярость Твоя, как огонь?
\vs Psa 88:48 Вспомни, какой мой век: на какую суету сотворил Ты всех сынов человеческих?
\vs Psa 88:49 Кто из людей жил~--- и не видел смерти, избавил душу свою от руки преисподней?
\vs Psa 88:50 Где прежние милости Твои, Господи? Ты клялся Давиду истиною Твоею.
\vs Psa 88:51 Вспомни, Господи, поругание рабов Твоих, которое я ношу в недре моем от всех сильных народов;
\vs Psa 88:52 как поносят враги Твои, Господи, как бесславят следы помазанника Твоего.
\vs Psa 88:53 Благословен Господь вовек! Аминь, аминь.
\vs Psa 89:1 Молитва Моисея, человека Божия.
\rsbpar\vs Psa 89:2 Господи! Ты нам прибежище в род и род.
\vs Psa 89:3 Прежде нежели родились горы, и Ты образовал землю и вселенную, и от века и до века Ты~--- Бог.
\vs Psa 89:4 Ты возвращаешь человека в тление и говоришь: <<возвратитесь, сыны человеческие!>>
\vs Psa 89:5 Ибо пред очами Твоими тысяча лет, как день вчерашний, когда он прошел, и \bibemph{как} стража в ночи.
\vs Psa 89:6 Ты \bibemph{как} наводнением уносишь их; они~--- \bibemph{как} сон, как трава, которая утром вырастает, утром цветет и зеленеет, вечером подсекается и засыхает;
\vs Psa 89:7 ибо мы исчезаем от гнева Твоего и от ярости Твоей мы в смятении.
\vs Psa 89:8 Ты положил беззакония наши пред Тобою и тайное наше пред светом лица Твоего.
\vs Psa 89:9 Все дни наши прошли во гневе Твоем; мы теряем лета наши, как звук.
\vs Psa 89:10 Дней лет наших~--- семьдесят лет, а при большей крепости~--- восемьдесят лет; и самая лучшая пора их~--- труд и болезнь, ибо проходят быстро, и мы летим.
\vs Psa 89:11 Кто знает силу гнева Твоего, и ярость Твою по мере страха Твоего?
\vs Psa 89:12 Научи нас так счислять дни наши, чтобы нам приобрести сердце мудрое.
\vs Psa 89:13 Обратись, Господи! Доколе? Умилосердись над рабами Твоими.
\vs Psa 89:14 Рано насыти нас милостью Твоею, и мы будем радоваться и веселиться во все дни наши.
\vs Psa 89:15 Возвесели нас за дни, \bibemph{в которые} Ты поражал нас, за лета, \bibemph{в которые} мы видели бедствие.
\vs Psa 89:16 Да явится на рабах Твоих дело Твое и на сынах их слава Твоя;
\vs Psa 89:17 и да будет благоволение Господа Бога нашего на нас, и в деле рук наших споспешествуй нам, в деле рук наших споспешествуй.
\vs Psa 90:0 [Хвалебная песнь Давида.]
\rsbpar\vs Psa 90:1 Живущий под кровом Всевышнего под сенью Всемогущего покоится,
\vs Psa 90:2 говорит Господу: <<прибежище мое и защита моя, Бог мой, на Которого я уповаю!>>
\vs Psa 90:3 Он избавит тебя от сети ловца, от гибельной язвы,
\vs Psa 90:4 перьями Своими осенит тебя, и под крыльями Его будешь безопасен; щит и ограждение~--- истина Его.
\vs Psa 90:5 Не убоишься ужасов в ночи, стрелы, летящей днем,
\vs Psa 90:6 язвы, ходящей во мраке, заразы, опустошающей в полдень.
\vs Psa 90:7 Падут подле тебя тысяча и десять тысяч одесную тебя; но к тебе не приблизится:
\vs Psa 90:8 только смотреть будешь очами твоими и видеть возмездие нечестивым.
\vs Psa 90:9 Ибо ты \bibemph{сказал}: <<Господь~--- упование мое>>; Всевышнего избрал ты прибежищем твоим;
\vs Psa 90:10 не приключится тебе зло, и язва не приблизится к жилищу твоему;
\vs Psa 90:11 ибо Ангелам Своим заповедает о тебе~--- охранять тебя на всех путях твоих:
\vs Psa 90:12 на руках понесут тебя, да не преткнешься о камень ногою твоею;
\vs Psa 90:13 на аспида и василиска наступишь; попирать будешь льва и дракона.
\vs Psa 90:14 <<За то, что он возлюбил Меня, избавлю его; защищу его, потому что он познал имя Мое.
\vs Psa 90:15 Воззовет ко Мне, и услышу его; с ним Я в скорби; избавлю его и прославлю его,
\vs Psa 90:16 долготою дней насыщу его, и явлю ему спасение Мое>>.
\vs Psa 91:1 Псалом. Песнь на день субботний.
\rsbpar\vs Psa 91:2 Благо есть славить Господа и петь имени Твоему, Всевышний,
\vs Psa 91:3 возвещать утром милость Твою и истину Твою в ночи,
\vs Psa 91:4 на десятиструнном и псалтири, с песнью на гуслях.
\vs Psa 91:5 Ибо Ты возвеселил меня, Господи, творением Твоим: я восхищаюсь делами рук Твоих.
\vs Psa 91:6 Как велики дела Твои, Господи! дивно глубоки помышления Твои!
\vs Psa 91:7 Человек несмысленный не знает, и невежда не разумеет того.
\vs Psa 91:8 Тогда как нечестивые возникают, как трава, и делающие беззаконие цветут, чтобы исчезнуть на веки,~---
\vs Psa 91:9 Ты, Господи, высок во веки!
\vs Psa 91:10 Ибо вот, враги Твои, Господи,~--- вот, враги Твои гибнут, и рассыпаются все делающие беззаконие;
\vs Psa 91:11 а мой рог Ты возносишь, как рог единорога, и я умащен свежим елеем;
\vs Psa 91:12 и око мое смотрит на врагов моих, и уши мои слышат о восстающих на меня злодеях.
\vs Psa 91:13 Праведник цветет, как пальма, возвышается подобно кедру на Ливане.
\vs Psa 91:14 Насажденные в доме Господнем, они цветут во дворах Бога нашего;
\vs Psa 91:15 они и в старости плодовиты, сочны и свежи,
\vs Psa 91:16 чтобы возвещать, что праведен Господь, твердыня моя, и нет неправды в Нем.
\vs Psa 92:0 [Хвалебная песнь Давида. В день предсубботний, когда населена земля.]
\rsbpar\vs Psa 92:1 Господь царствует; Он облечен величием, облечен Господь могуществом [и] препоясан: потому вселенная тверда, не подвигнется.
\vs Psa 92:2 Престол Твой утвержден искони: Ты~--- от века.
\vs Psa 92:3 Возвышают реки, Господи, возвышают реки голос свой, возвышают реки волны свои.
\vs Psa 92:4 Но паче шума вод многих, сильных волн морских, силен в вышних Господь.
\vs Psa 92:5 Откровения Твои несомненно верны. Дому Твоему, Господи, принадлежит святость на долгие дни.
\vs Psa 93:0 [Псалом Давида в четвертый день недели.]
\rsbpar\vs Psa 93:1 Боже отмщений, Господи, Боже отмщений, яви Себя!
\vs Psa 93:2 Восстань, Судия земли, воздай возмездие гордым.
\vs Psa 93:3 Доколе, Господи, нечестивые, доколе нечестивые торжествовать будут?
\vs Psa 93:4 Они изрыгают дерзкие речи; величаются все делающие беззаконие;
\vs Psa 93:5 попирают народ Твой, Господи, угнетают наследие Твое;
\vs Psa 93:6 вдову и пришельца убивают, и сирот умерщвляют
\vs Psa 93:7 и говорят: <<не увидит Господь, и не узнает Бог Иаковлев>>.
\vs Psa 93:8 Образумьтесь, бессмысленные люди! когда вы будете умны, невежды?
\vs Psa 93:9 Насадивший ухо не услышит ли? и образовавший глаз не увидит ли?
\vs Psa 93:10 Вразумляющий народы неужели не обличит,~--- Тот, Кто учит человека разумению?
\vs Psa 93:11 Господь знает мысли человеческие, что они суетны.
\vs Psa 93:12 Блажен человек, которого вразумляешь Ты, Господи, и наставляешь законом Твоим,
\vs Psa 93:13 чтобы дать ему покой в бедственные дни, доколе нечестивому выроется яма!
\vs Psa 93:14 Ибо не отринет Господь народа Своего и не оставит наследия Своего.
\vs Psa 93:15 Ибо суд возвратится к правде, и за ним \bibemph{последуют} все правые сердцем.
\vs Psa 93:16 Кто восстанет за меня против злодеев? кто станет за меня против делающих беззаконие?
\vs Psa 93:17 Если бы не Господь был мне помощником, вскоре вселилась бы душа моя в \bibemph{страну} молчания.
\vs Psa 93:18 Когда я говорил: <<колеблется нога моя>>,~--- милость Твоя, Господи, поддерживала меня.
\vs Psa 93:19 При умножении скорбей моих в сердце моем, утешения Твои услаждают душу мою.
\vs Psa 93:20 Станет ли близ Тебя седалище губителей, умышляющих насилие вопреки закону?
\vs Psa 93:21 Толпою устремляются они на душу праведника и осуждают кровь неповинную.
\vs Psa 93:22 Но Господь~--- защита моя, и Бог мой~--- твердыня убежища моего,
\vs Psa 93:23 и обратит на них беззаконие их, и злодейством их истребит их, истребит их Господь Бог наш.
\vs Psa 94:0 [Хвалебная песнь Давида.]
\rsbpar\vs Psa 94:1 Приидите, воспоем Господу, воскликнем [Богу], твердыне спасения нашего;
\vs Psa 94:2 предстанем лицу Его со славословием, в песнях воскликнем Ему,
\vs Psa 94:3 ибо Господь есть Бог великий и Царь великий над всеми богами.
\vs Psa 94:4 В Его руке глубины земли, и вершины гор~--- Его же;
\vs Psa 94:5 Его~--- море, и Он создал его, и сушу образовали руки Его.
\vs Psa 94:6 Приидите, поклонимся и припадем, преклоним колени пред лицем Господа, Творца нашего;
\vs Psa 94:7 ибо Он есть Бог наш, и мы~--- народ паствы Его и овцы руки Его. О, если бы вы ныне послушали гласа Его:
\vs Psa 94:8 <<не ожесточите сердца вашего, как в Мериве, как в день искушения в пустыне,
\vs Psa 94:9 где искушали Меня отцы ваши, испытывали Меня, и видели дело Мое.
\vs Psa 94:10 Сорок лет Я был раздражаем родом сим, и сказал: это народ, заблуждающийся сердцем; они не познали путей Моих,
\vs Psa 94:11 и потому Я поклялся во гневе Моем, что они не войдут в покой Мой>>.
\vs Psa 95:0 [Хвалебная песнь Давида. На построение дома.]
\rsbpar\vs Psa 95:1 Воспойте Господу песнь новую; воспойте Господу, вся земля;
\vs Psa 95:2 пойте Господу, благословляйте имя Его, благовествуйте со дня на день спасение Его;
\vs Psa 95:3 возвещайте в народах славу Его, во всех племенах чудеса Его;
\vs Psa 95:4 ибо велик Господь и достохвален, страшен Он паче всех богов.
\vs Psa 95:5 Ибо все боги народов~--- идолы, а Господь небеса сотворил.
\vs Psa 95:6 Слава и величие пред лицем Его, сила и великолепие во святилище Его.
\vs Psa 95:7 Воздайте Господу, племена народов, воздайте Господу славу и честь;
\vs Psa 95:8 воздайте Господу славу имени Его, несите дары и идите во дворы Его;
\vs Psa 95:9 поклонитесь Господу во благолепии святыни. Трепещи пред лицем Его, вся земля!
\vs Psa 95:10 Скажите народам: Господь царствует! потому тверда вселенная, не поколеблется. Он будет судить народы по правде.
\vs Psa 95:11 Да веселятся небеса и да торжествует земля; да шумит море и что наполняет его;
\vs Psa 95:12 да радуется поле и все, что на нем, и да ликуют все дерева дубравные
\vs Psa 95:13 пред лицем Господа; ибо идет, ибо идет судить землю. Он будет судить вселенную по правде, и народы~--- по истине Своей.
\vs Psa 96:0 [Псалом Давида, когда устроялась земля его.]
\rsbpar\vs Psa 96:1 Господь царствует: да радуется земля; да веселятся многочисленные острова.
\vs Psa 96:2 Облако и мрак окрест Его; правда и суд~--- основание престола Его.
\vs Psa 96:3 Пред Ним идет огонь и вокруг попаляет врагов Его.
\vs Psa 96:4 Молнии Его освещают вселенную; земля видит и трепещет.
\vs Psa 96:5 Горы, как воск, тают от лица Господа, от лица Господа всей земли.
\vs Psa 96:6 Небеса возвещают правду Его, и все народы видят славу Его.
\vs Psa 96:7 Да постыдятся все служащие истуканам, хвалящиеся идолами. Поклонитесь пред Ним, все боги\fns{По переводу 70-ти: все Ангелы Его.}.
\vs Psa 96:8 Слышит Сион и радуется, и веселятся дщери Иудины ради судов Твоих, Господи,
\vs Psa 96:9 ибо Ты, Господи, высок над всею землею, превознесен над всеми богами.
\vs Psa 96:10 Любящие Господа, ненавидьте зло! Он хранит души святых Своих; из руки нечестивых избавляет их.
\vs Psa 96:11 Свет сияет на праведника, и на правых сердцем~--- веселие.
\vs Psa 96:12 Радуйтесь, праведные, о Господе и славьте память святыни Его.
\vs Psa 97:0 Псалом [Давида].
\rsbpar\vs Psa 97:1 Воспойте Господу новую песнь, ибо Он сотворил чудеса. Его десница и святая мышца Его доставили Ему победу.
\vs Psa 97:2 Явил Господь спасение Свое, открыл пред очами народов правду Свою.
\vs Psa 97:3 Вспомнил Он милость Свою [к Иакову] и верность Свою к дому Израилеву. Все концы земли увидели спасение Бога нашего.
\vs Psa 97:4 Восклицайте Господу, вся земля; торжествуйте, веселитесь и пойте;
\vs Psa 97:5 пойте Господу с гуслями, с гуслями и с гласом псалмопения;
\vs Psa 97:6 при звуке труб и рога торжествуйте пред Царем Господом.
\vs Psa 97:7 Да шумит море и что наполняет его, вселенная и живущие в ней;
\vs Psa 97:8 да рукоплещут реки, да ликуют вместе горы
\vs Psa 97:9 пред лицем Господа, ибо Он идет судить землю. Он будет судить вселенную праведно и народы~--- верно.
\vs Psa 98:0 [Псалом Давида.]
\rsbpar\vs Psa 98:1 Господь царствует: да трепещут народы! Он восседает на Херувимах: да трясется земля!
\vs Psa 98:2 Господь на Сионе велик, и высок Он над всеми народами.
\vs Psa 98:3 Да славят великое и страшное имя Твое: свято оно!
\vs Psa 98:4 И могущество царя любит суд. Ты утвердил справедливость; суд и правду Ты совершил в Иакове.
\vs Psa 98:5 Превозносите Господа, Бога нашего, и поклоняйтесь подножию Его: свято оно!
\vs Psa 98:6 Моисей и Аарон между священниками и Самуил между призывающими имя Его взывали к Господу, и Он внимал им.
\vs Psa 98:7 В столпе облачном говорил Он к ним; они хранили Его заповеди и устав, который Он дал им.
\vs Psa 98:8 Господи, Боже наш! Ты внимал им; Ты был для них Богом прощающим и наказывающим за дела их.
\vs Psa 98:9 Превозносите Господа, Бога нашего, и поклоняйтесь на святой горе Его, ибо свят Господь, Бог наш.
\vs Psa 99:0 Псалом [Давида] хвалебный.
\rsbpar\vs Psa 99:1 Воскликните Господу, вся земля!
\vs Psa 99:2 Слу\-ж\acc{и}\-те Господу с веселием; идите пред лице Его с восклицанием!
\vs Psa 99:3 Познайте, что Господь есть Бог, что Он сотворил нас, и мы~--- Его, Его народ и овцы паствы Его.
\vs Psa 99:4 Входите во врата Его со славословием, во дворы Его~--- с хвалою. Славьте Его, благословляйте имя Его,
\vs Psa 99:5 ибо благ Господь: милость Его вовек, и истина Его в род и род.
\vs Psa 100:0 Псалом Давида.
\rsbpar\vs Psa 100:1 Милость и суд буду петь; Тебе, Господи, буду петь.
\vs Psa 100:2 Буду размышлять о пути непорочном: <<когда ты придешь ко мне?>> Буду ходить в непорочности моего сердца посреди дома моего.
\vs Psa 100:3 Не положу пред очами моими вещи непотребной; дело преступное я ненавижу: не прилепится оно ко мне.
\vs Psa 100:4 Сердце развращенное будет удалено от меня; злого я не буду знать.
\vs Psa 100:5 Тайно клевещущего на ближнего своего изгоню; гордого очами и надменного сердцем не потерплю.
\vs Psa 100:6 Глаза мои на верных земли, чтобы они пребывали при мне; кто ходит путем непорочности, тот будет служить мне.
\vs Psa 100:7 Не будет жить в доме моем поступающий коварно; говорящий ложь не останется пред глазами моими.
\vs Psa 100:8 С раннего утра буду истреблять всех нечестивцев земли, дабы искоренить из града Господня всех делающих беззаконие.
\vs Psa 101:1 Молитва страждущего, когда он унывает и изливает пред Господом печаль свою.
\rsbpar\vs Psa 101:2 Господи! услышь молитву мою, и вопль мой да придет к Тебе.
\vs Psa 101:3 Не скрывай лица Твоего от меня; в день скорби моей приклони ко мне ухо Твое; в день, [когда] воззову [к Тебе], скоро услышь меня;
\vs Psa 101:4 ибо исчезли, как дым, дни мои, и кости мои обожжены, как головня;
\vs Psa 101:5 сердце мое поражено, и иссохло, как трава, так что я забываю есть хлеб мой;
\vs Psa 101:6 от голоса стенания моего кости мои прильпнули к плоти моей.
\vs Psa 101:7 Я уподобился пеликану в пустыне; я стал как филин на развалинах;
\vs Psa 101:8 не сплю и сижу, как одинокая птица на кровле.
\vs Psa 101:9 Всякий день поносят меня враги мои, и злобствующие на меня клянут мною.
\vs Psa 101:10 Я ем пепел, как хлеб, и питье мое растворяю слезами,
\vs Psa 101:11 от гнева Твоего и негодования Твоего, ибо Ты вознес меня и низверг меня.
\vs Psa 101:12 Дни мои~--- как уклоняющаяся тень, и я иссох, как трава.
\vs Psa 101:13 Ты же, Господи, вовек пребываешь, и память о Тебе в род и род.
\vs Psa 101:14 Ты восстанешь, умилосердишься над Сионом, ибо время помиловать его,~--- ибо пришло время;
\vs Psa 101:15 ибо рабы Твои возлюбили и камни его, и о прахе его жалеют.
\vs Psa 101:16 И убоятся народы имени Господня, и все цари земные~--- славы Твоей.
\vs Psa 101:17 Ибо созиждет Господь Сион и явится во славе Своей;
\vs Psa 101:18 призрит на молитву беспомощных и не презрит моления их.
\vs Psa 101:19 Напишется о сем для рода последующего, и поколение грядущее восхвалит Господа,
\vs Psa 101:20 ибо Он приникнул со святой высоты Своей, с небес призрел Господь на землю,
\vs Psa 101:21 чтобы услышать стон узников, разрешить сынов смерти,
\vs Psa 101:22 дабы возвещали на Сионе имя Господне и хвалу Его~--- в Иерусалиме,
\vs Psa 101:23 когда соберутся народы вместе и царства для служения Господу.
\vs Psa 101:24 Изнурил Он на пути силы мои, сократил дни мои.
\vs Psa 101:25 Я сказал: Боже мой! не восхити меня в половине дней моих. Твои лета в роды родов.
\vs Psa 101:26 В начале Ты, [Господи,] основал землю, и небеса~--- дело Твоих рук;
\vs Psa 101:27 они погибнут, а Ты пребудешь; и все они, как риза, обветшают, и, как одежду, Ты переменишь их, и изменятся;
\vs Psa 101:28 но Ты~--- тот же, и лета Твои не кончатся.
\vs Psa 101:29 Сыны рабов Твоих будут жить, и семя их утвердится пред лицем Твоим.
\vs Psa 102:0 Псалом Давида.
\rsbpar\vs Psa 102:1 Благослови, душа моя, Господа, и вся внутренность моя~--- святое имя Его.
\vs Psa 102:2 Благослови, душа моя, Господа и не забывай всех благодеяний Его.
\vs Psa 102:3 Он прощает все беззакония твои, исцеляет все недуги твои;
\vs Psa 102:4 избавляет от могилы жизнь твою, венчает тебя милостью и щедротами;
\vs Psa 102:5 насыщает благами желание твое: обновляется, подобно орлу, юность твоя.
\vs Psa 102:6 Господь творит правду и суд всем обиженным.
\vs Psa 102:7 Он показал пути Свои Моисею, сынам Израилевым~--- дела Свои.
\vs Psa 102:8 Щедр и милостив Господь, долготерпелив и многомилостив:
\vs Psa 102:9 не до конца гневается, и не вовек негодует.
\vs Psa 102:10 Не по беззакониям нашим сотворил нам, и не по грехам нашим воздал нам:
\vs Psa 102:11 ибо как высоко небо над землею, так велика милость [Господа] к боящимся Его;
\vs Psa 102:12 как далеко восток от запада, так удалил Он от нас беззакония наши;
\vs Psa 102:13 как отец милует сынов, так милует Господь боящихся Его.
\vs Psa 102:14 Ибо Он знает состав наш, помнит, что мы~--- персть.
\vs Psa 102:15 Дни человека~--- как трава; как цвет полевой, так он цветет.
\vs Psa 102:16 Пройдет над ним ветер, и нет его, и место его уже не узнает его.
\vs Psa 102:17 Милость же Господня от века и до века к боящимся Его,
\vs Psa 102:18 и правда Его на сынах сынов, хранящих завет Его и помнящих заповеди Его, чтобы исполнять их.
\vs Psa 102:19 Господь на небесах поставил престол Свой, и царство Его всем обладает.
\vs Psa 102:20 Благословите Господа, [все] Ангелы Его, крепкие силою, исполняющие слово Его, повинуясь гласу слова Его;
\vs Psa 102:21 благословите Господа, все воинства Его, служители Его, исполняющие волю Его;
\vs Psa 102:22 благословите Господа, все дела Его, во всех местах владычества Его. Благослови, душа моя, Господа!
\vs Psa 103:0 [Псалом Давида о сотворении мира.]
\rsbpar\vs Psa 103:1 Благослови, душа моя, Господа! Господи, Боже мой! Ты дивно велик, Ты облечен славою и величием;
\vs Psa 103:2 Ты одеваешься светом, как ризою, простираешь небеса, как шатер;
\vs Psa 103:3 устрояешь над водами горние чертоги Твои, делаешь облака Твоею колесницею, шествуешь на крыльях ветра.
\vs Psa 103:4 Ты творишь ангелами Твоими духов, служителями Твоими~--- огонь пылающий.
\vs Psa 103:5 Ты поставил землю на твердых основах: не поколеблется она во веки и веки.
\vs Psa 103:6 Бездною, как одеянием, покрыл Ты ее, на горах стоят воды.
\vs Psa 103:7 От прещения Твоего бегут они, от гласа грома Твоего быстро уходят;
\vs Psa 103:8 восходят на горы, нисходят в долины, на место, которое Ты назначил для них.
\vs Psa 103:9 Ты положил предел, которого не перейдут, и не возвратятся покрыть землю.
\vs Psa 103:10 Ты послал источники в долины: между горами текут [воды],
\vs Psa 103:11 поят всех полевых зверей; дикие ослы утоляют жажду свою.
\vs Psa 103:12 При них обитают птицы небесные, из среды ветвей издают голос.
\vs Psa 103:13 Ты напояешь горы с высот Твоих, плодами дел Твоих насыщается земля.
\vs Psa 103:14 Ты произращаешь траву для скота, и зелень на пользу человека, чтобы произвести из земли пищу,
\vs Psa 103:15 и вино, которое веселит сердце человека, и елей, от которого блистает лице его, и хлеб, который укрепляет сердце человека.
\vs Psa 103:16 Насыщаются древа Господа, кедры Ливанские, которые Он насадил;
\vs Psa 103:17 на них гнездятся птицы: ели~--- жилище аисту,
\vs Psa 103:18 высокие горы~--- сернам; каменные утесы~--- убежище зайцам.
\vs Psa 103:19 Он сотворил луну для \bibemph{указания} времен, солнце знает свой запад.
\vs Psa 103:20 Ты простираешь тьму и бывает ночь: во время нее бродят все лесные звери;
\vs Psa 103:21 львы рыкают о добыче и просят у Бога пищу себе.
\vs Psa 103:22 Восходит солнце, [и] они собираются и ложатся в свои логовища;
\vs Psa 103:23 выходит человек на дело свое и на работу свою до вечера.
\vs Psa 103:24 Как многочисленны дела Твои, Господи! Все соделал Ты премудро; земля полна произведений Твоих.
\vs Psa 103:25 Это~--- море великое и пространное: там пресмыкающиеся, которым нет числа, животные малые с большими;
\vs Psa 103:26 там плавают корабли, там этот левиафан, которого Ты сотворил играть в нем.
\vs Psa 103:27 Все они от Тебя ожидают, чтобы Ты дал им пищу их в свое время.
\vs Psa 103:28 Даешь им~--- принимают, отверзаешь руку Твою~--- насыщаются благом;
\vs Psa 103:29 скроешь лице Твое~--- мятутся, отнимешь дух их~--- умирают и в персть свою возвращаются;
\vs Psa 103:30 пошлешь дух Твой~--- созидаются, и Ты обновляешь лице земли.
\vs Psa 103:31 Да будет Господу слава во веки; да веселится Господь о делах Своих!
\vs Psa 103:32 Призирает на землю, и она трясется; прикасается к горам, и дымятся.
\vs Psa 103:33 Буду петь Господу во \bibemph{всю} жизнь мою, буду петь Богу моему, доколе есмь.
\vs Psa 103:34 Да будет благоприятна Ему песнь моя; буду веселиться о Господе.
\vs Psa 103:35 Да исчезнут грешники с земли, и беззаконных да не будет более. Благослови, душа моя, Господа! Аллилуия!
\vs Psa 104:1 Славьте Господа; призывайте имя Его; возвещайте в народах дела Его;
\vs Psa 104:2 воспойте Ему и пойте Ему; поведайте о всех чудесах Его.
\vs Psa 104:3 Хвалитесь именем Его святым; да веселится сердце ищущих Господа.
\vs Psa 104:4 Ищите Господа и силы Его, ищите лица Его всегда.
\vs Psa 104:5 Воспоминайте чудеса Его, которые сотворил, знамения Его и суды уст Его,
\vs Psa 104:6 вы, семя Авраамово, рабы Его, сыны Иакова, избранные Его.
\vs Psa 104:7 Он Господь Бог наш: по всей земле суды Его.
\vs Psa 104:8 Вечно помнит завет Свой, слово, [которое] заповедал в тысячу родов,
\vs Psa 104:9 которое завещал Аврааму, и клятву Свою Исааку,
\vs Psa 104:10 и поставил то Иакову в закон и Израилю в завет вечный,
\vs Psa 104:11 говоря: <<тебе дам землю Ханаанскую в удел наследия вашего>>.
\vs Psa 104:12 Когда их было еще мало числом, очень мало, и они были пришельцами в ней
\vs Psa 104:13 и переходили от народа к народу, из царства к иному племени,
\vs Psa 104:14 никому не позволял обижать их и возбранял о них царям:
\vs Psa 104:15 <<не прикасайтесь к помазанным Моим, и пророкам Моим не делайте зла>>.
\vs Psa 104:16 И призвал голод на землю; всякий стебель хлебный истребил.
\vs Psa 104:17 Послал пред ними человека: в рабы продан был Иосиф.
\vs Psa 104:18 Стеснили оковами ноги его; в железо вошла душа его,
\vs Psa 104:19 доколе исполнилось слово Его: слово Господне испытало его.
\vs Psa 104:20 Послал царь, и разрешил его владетель народов и освободил его;
\vs Psa 104:21 поставил его господином над домом своим и правителем над всем владением своим,
\vs Psa 104:22 чтобы он наставлял вельмож его по своей душе и старейшин его учил мудрости.
\vs Psa 104:23 Тогда пришел Израиль в Египет, и переселился Иаков в землю Хамову.
\vs Psa 104:24 И весьма размножил \bibemph{Бог} народ Свой и сделал его сильнее врагов его.
\vs Psa 104:25 Возбудил в сердце их ненависть против народа Его и ухищрение против рабов Его.
\vs Psa 104:26 Послал Моисея, раба Своего, Аарона, которого избрал.
\vs Psa 104:27 Они показали между ними слова знамений Его и чудеса [Его] в земле Хамовой.
\vs Psa 104:28 Послал тьму и сделал мрак, и не воспротивились слову Его.
\vs Psa 104:29 Преложил воду их в кровь, и уморил рыбу их.
\vs Psa 104:30 Земля их произвела множество жаб \bibemph{даже} в спальне царей их.
\vs Psa 104:31 Он сказал, и пришли разные насекомые, скнипы во все пределы их.
\vs Psa 104:32 Вместо дождя послал на них град, палящий огонь на землю их,
\vs Psa 104:33 и побил виноград их и смоковницы их, и сокрушил дерева в пределах их.
\vs Psa 104:34 Сказал, и пришла саранча и гусеницы без числа;
\vs Psa 104:35 и съели всю траву на земле их, и съели плоды на полях их.
\vs Psa 104:36 И поразил всякого первенца в земле их, начатки всей силы их.
\vs Psa 104:37 И вывел \bibemph{Израильтян} с серебром и золотом, и не было в коленах их болящего.
\vs Psa 104:38 Обрадовался Египет исшествию их; ибо страх от них напал на него.
\vs Psa 104:39 Простер облако в покров [им] и огонь, чтобы светить [им] ночью.
\vs Psa 104:40 Просили, и Он послал перепелов, и хлебом небесным насыщал их.
\vs Psa 104:41 Разверз камень, и потекли воды, потекли рекою по местам сухим,
\vs Psa 104:42 ибо вспомнил Он святое слово Свое к Аврааму, рабу Своему,
\vs Psa 104:43 и вывел народ Свой в радости, избранных Своих в веселии,
\vs Psa 104:44 и дал им земли народов, и они наследовали труд иноплеменных,
\vs Psa 104:45 чтобы соблюдали уставы Его и хранили законы Его. Аллилуия!
\vs Psa 105:0 Аллилуия.
\rsbpar\vs Psa 105:1 Славьте Господа, ибо Он благ, ибо вовек милость Его.
\vs Psa 105:2 Кто изречет могущество Господа, возвестит все хвалы Его?
\vs Psa 105:3 Блаженны хранящие суд и творящие правду во всякое время!
\vs Psa 105:4 Вспомни о мне, Господи, в благоволении к народу Твоему; посети меня спасением Твоим,
\vs Psa 105:5 дабы мне видеть благоденствие избранных Твоих, веселиться веселием народа Твоего, хвалиться с наследием Твоим.
\vs Psa 105:6 Согрешили мы с отцами нашими, совершили беззаконие, соделали неправду.
\vs Psa 105:7 Отцы наши в Египте не уразумели чудес Твоих, не помнили множества милостей Твоих, и возмутились у моря, у Чермного моря.
\vs Psa 105:8 Но Он спас их ради имени Своего, дабы показать могущество Свое.
\vs Psa 105:9 Грозно рек морю Чермному, и оно иссохло; и провел их по безднам, как по суше;
\vs Psa 105:10 и спас их от руки ненавидящего и избавил их от руки врага.
\vs Psa 105:11 Воды покрыли врагов их, ни одного из них не осталось.
\vs Psa 105:12 И поверили они словам Его, [и] воспели хвалу Ему.
\vs Psa 105:13 \bibemph{Но} скоро забыли дела Его, не дождались Его изволения;
\vs Psa 105:14 увлеклись похотением в пустыне, и искусили Бога в необитаемой.
\vs Psa 105:15 И Он исполнил прошение их, \bibemph{но} послал язву на души их.
\vs Psa 105:16 И позавидовали в стане Моисею \bibemph{и} Аарону, святому Господню.
\vs Psa 105:17 Разверзлась земля, и поглотила Дафана и покрыла скопище Авирона.
\vs Psa 105:18 И возгорелся огонь в скопище их, пламень попалил нечестивых.
\vs Psa 105:19 Сделали тельца у Хорива и поклонились истукану;
\vs Psa 105:20 и променяли славу свою на изображение вола, ядущего траву.
\vs Psa 105:21 Забыли Бога, Спасителя своего, совершившего великое в Египте,
\vs Psa 105:22 дивное в земле Хамовой, страшное у Чермного моря.
\vs Psa 105:23 И хотел истребить их, если бы Моисей, избранный Его, не стал пред Ним в расселине, чтобы отвратить ярость Его, да не погубит [их].
\vs Psa 105:24 И презрели они землю желанную, не верили слову Его;
\vs Psa 105:25 и роптали в шатрах своих, не слушались гласа Господня.
\vs Psa 105:26 И поднял Он руку Свою на них, чтобы низложить их в пустыне,
\vs Psa 105:27 низложить племя их в народах и рассеять их по землям.
\vs Psa 105:28 Они прилепились к Ваалфегору и ели жертвы бездушным,
\vs Psa 105:29 и раздражали \bibemph{Бога} делами своими, и вторглась к ним язва.
\vs Psa 105:30 И восстал Финеес и произвел суд,~--- и остановилась язва.
\vs Psa 105:31 И \bibemph{это} вменено ему в праведность в роды и роды во веки.
\vs Psa 105:32 И прогневали \bibemph{Бога} у вод Меривы, и Моисей потерпел за них,
\vs Psa 105:33 ибо они огорчили дух его, и он погрешил устами своими.
\vs Psa 105:34 Не истребили народов, о которых сказал им Господь,
\vs Psa 105:35 но смешались с язычниками и научились делам их;
\vs Psa 105:36 служили истуканам их, \bibemph{которые} были для них сетью,
\vs Psa 105:37 и приносили сыновей своих и дочерей своих в жертву бесам;
\vs Psa 105:38 проливали кровь невинную, кровь сыновей своих и дочерей своих, которых приносили в жертву идолам Ханаанским,~--- и осквернилась земля кровью;
\vs Psa 105:39 оскверняли себя делами своими, блудодействовали поступками своими.
\vs Psa 105:40 И воспылал гнев Господа на народ Его, и возгнушался Он наследием Своим
\vs Psa 105:41 и предал их в руки язычников, и ненавидящие их стали обладать ими.
\vs Psa 105:42 Враги их утесняли их, и они смирялись под рукою их.
\vs Psa 105:43 Много раз Он избавлял их; они же раздражали [Его] упорством своим, и были уничижаемы за беззаконие свое.
\vs Psa 105:44 Но Он призирал на скорбь их, когда слышал вопль их,
\vs Psa 105:45 и вспоминал завет Свой с ними и раскаивался по множеству милости Своей;
\vs Psa 105:46 и возбуждал к ним сострадание во всех, пленявших их.
\vs Psa 105:47 Спаси нас, Господи, Боже наш, и собери нас от народов, дабы славить святое имя Твое, хвалиться Твоею славою.
\vs Psa 105:48 Благословен Господь, Бог Израилев, от века и до века! И да скажет весь народ: аминь! Аллилуия!
\vs Psa 106:0 [Аллилуия.]
\rsbpar\vs Psa 106:1 Славьте Господа, ибо Он благ, ибо вовек милость Его!
\vs Psa 106:2 Так да скажут избавленные Господом, которых избавил Он от руки врага,
\vs Psa 106:3 и собрал от стран, от востока и запада, от севера и моря.
\vs Psa 106:4 Они блуждали в пустыне по безлюдному пути и не находили населенного города;
\vs Psa 106:5 терпели голод и жажду, душа их истаевала в них.
\vs Psa 106:6 Но воззвали к Господу в скорби своей, и Он избавил их от бедствий их,
\vs Psa 106:7 и повел их прямым путем, чтобы они шли к населенному городу.
\vs Psa 106:8 Да славят Господа за милость Его и за чудные дела Его для сынов человеческих:
\vs Psa 106:9 ибо Он насытил душу жаждущую и душу алчущую исполнил благами.
\vs Psa 106:10 Они сидели во тьме и тени смертной, окованные скорбью и железом;
\vs Psa 106:11 ибо не покорялись словам Божиим и небрегли о воле Всевышнего.
\vs Psa 106:12 Он смирил сердце их работами; они преткнулись, и не было помогающего.
\vs Psa 106:13 Но воззвали к Господу в скорби своей, и Он спас их от бедствий их;
\vs Psa 106:14 вывел их из тьмы и тени смертной, и расторгнул узы их.
\vs Psa 106:15 Да славят Господа за милость Его и за чудные дела Его для сынов человеческих:
\vs Psa 106:16 ибо Он сокрушил врата медные и вереи железные сломил.
\vs Psa 106:17 Безрассудные страдали за беззаконные пути свои и за неправды свои;
\vs Psa 106:18 от всякой пищи отвращалась душа их, и они приближались ко вратам смерти.
\vs Psa 106:19 Но воззвали к Господу в скорби своей, и Он спас их от бедствий их;
\vs Psa 106:20 послал слово Свое и исцелил их, и избавил их от могил их.
\vs Psa 106:21 Да славят Господа за милость Его и за чудные дела Его для сынов человеческих!
\vs Psa 106:22 Да приносят Ему жертву хвалы и да возвещают о делах Его с пением!
\vs Psa 106:23 Отправляющиеся на кораблях в море, производящие дела на больших водах,
\vs Psa 106:24 видят дела Господа и чудеса Его в пучине:
\vs Psa 106:25 Он речет,~--- и восстает бурный ветер и высоко поднимает волны его:
\vs Psa 106:26 восходят до небес, нисходят до бездны; душа их истаевает в бедствии;
\vs Psa 106:27 они кружатся и шатаются, как пьяные, и вся мудрость их исчезает.
\vs Psa 106:28 Но воззвали к Господу в скорби своей, и Он вывел их из бедствия их.
\vs Psa 106:29 Он превращает бурю в тишину, и волны умолкают.
\vs Psa 106:30 И веселятся, что они утихли, и Он приводит их к желаемой пристани.
\vs Psa 106:31 Да славят Господа за милость Его и за чудные дела Его для сынов человеческих!
\vs Psa 106:32 Да превозносят Его в собрании народном и да славят Его в сонме старейшин!
\vs Psa 106:33 Он превращает реки в пустыню и источники вод~--- в сушу,
\vs Psa 106:34 землю плодородную~--- в солончатую, за нечестие живущих на ней.
\vs Psa 106:35 Он превращает пустыню в озеро, и землю иссохшую~--- в источники вод;
\vs Psa 106:36 и поселяет там алчущих, и они строят город для обитания;
\vs Psa 106:37 засевают поля, насаждают виноградники, которые приносят им обильные плоды.
\vs Psa 106:38 Он благословляет их, и они весьма размножаются, и скота их не умаляет.
\vs Psa 106:39 Уменьшились они и упали от угнетения, бедствия и скорби,~---
\vs Psa 106:40 Он изливает бесчестие на князей и оставляет их блуждать в пустыне, где нет путей.
\vs Psa 106:41 Бедного же извлекает из бедствия и умножает род его, как стада овец.
\vs Psa 106:42 Праведники видят сие и радуются, а всякое нечестие заграждает уста свои.
\vs Psa 106:43 Кто мудр, тот заметит сие и уразумеет милость Господа.
\vs Psa 107:1 Песнь. Псалом Давида.
\rsbpar\vs Psa 107:2 Готово сердце мое, Боже, [готово сердце мое]; буду петь и воспевать во славе моей.
\vs Psa 107:3 Воспрянь, псалтирь и гусли! Я встану рано.
\vs Psa 107:4 Буду славить Тебя, Господи, между народами; буду воспевать Тебя среди племен,
\vs Psa 107:5 ибо превыше небес милость Твоя и до облаков истина Твоя.
\vs Psa 107:6 Будь превознесен выше небес, Боже; над всею землею \bibemph{да будет} слава Твоя,
\vs Psa 107:7 дабы избавились возлюбленные Твои: спаси десницею Твоею и услышь меня.
\vs Psa 107:8 Бог сказал во святилище Своем: <<восторжествую, разделю Сихем и долину Сокхоф размерю;
\vs Psa 107:9 Мой Галаад, Мой Манассия, Ефрем~--- крепость главы Моей, Иуда~--- скипетр Мой,
\vs Psa 107:10 Моав~--- умывальная чаша Моя, на Едома простру сапог Мой, над землею Филистимскою восклицать буду>>.
\vs Psa 107:11 Кто введет меня в укрепленный город? Кто доведет меня до Едома?
\vs Psa 107:12 Не Ты ли, Боже, \bibemph{Который} отринул нас и не выходишь, Боже, с войсками нашими?
\vs Psa 107:13 Подай нам помощь в тесноте, ибо защита человеческая суетна.
\vs Psa 107:14 С Богом мы окажем силу: Он низложит врагов наших.
\vs Psa 108:0 Начальнику хора. Псалом Давида.
\rsbpar\vs Psa 108:1 Боже хвалы моей! не премолчи,
\vs Psa 108:2 ибо отверзлись на меня уста нечестивые и уста коварные; говорят со мною языком лживым;
\vs Psa 108:3 отвсюду окружают меня словами ненависти, вооружаются против меня без причины;
\vs Psa 108:4 за любовь мою они враждуют на меня, а я молюсь;
\vs Psa 108:5 воздают мне за добро злом, за любовь мою~--- ненавистью.
\vs Psa 108:6 Поставь над ним нечестивого, и диавол да станет одесную его.
\vs Psa 108:7 Когда будет судиться, да выйдет виновным, и молитва его да будет в грех;
\vs Psa 108:8 да будут дни его кратки, и достоинство его да возьмет другой;
\vs Psa 108:9 дети его да будут сиротами, и жена его~--- вдовою;
\vs Psa 108:10 да скитаются дети его и нищенствуют, и просят \bibemph{хлеба} из развалин своих;
\vs Psa 108:11 да захватит заимодавец все, что есть у него, и чужие да расхитят труд его;
\vs Psa 108:12 да не будет сострадающего ему, да не будет милующего сирот его;
\vs Psa 108:13 да будет потомство его на погибель, и да изгладится имя их в следующем роде;
\vs Psa 108:14 да будет воспомянуто пред Господом беззаконие отцов его, и грех матери его да не изгладится;
\vs Psa 108:15 да будут они всегда в очах Господа, и да истребит Он память их на земле,
\vs Psa 108:16 за то, что он не думал оказывать милость, но преследовал человека бедного и нищего и сокрушенного сердцем, чтобы умертвить его;
\vs Psa 108:17 возлюбил проклятие,~--- оно и придет на него; не восхотел благословения,~--- оно и удалится от него;
\vs Psa 108:18 да облечется проклятием, как ризою, и да войдет оно, как вода, во внутренность его и, как елей, в кости его;
\vs Psa 108:19 да будет оно ему, как одежда, в которую он одевается, и как пояс, которым всегда опоясывается.
\vs Psa 108:20 Таково воздаяние от Господа врагам моим и говорящим злое на душу мою!
\vs Psa 108:21 Со мною же, Господи, Господи, твори ради имени Твоего, ибо блага милость Твоя; спаси меня,
\vs Psa 108:22 ибо я беден и нищ, и сердце мое уязвлено во мне.
\vs Psa 108:23 Я исчезаю, как уклоняющаяся тень; гонят меня, как саранчу.
\vs Psa 108:24 Колени мои изнемогли от поста, и тело мое лишилось тука.
\vs Psa 108:25 Я стал для них посмешищем: увидев меня, кивают головами [своими].
\vs Psa 108:26 Помоги мне, Господи, Боже мой, спаси меня по милости Твоей,
\vs Psa 108:27 да познают, что это~--- Твоя рука, и что Ты, Господи, соделал это.
\vs Psa 108:28 Они проклинают, а Ты благослови; они восстают, но да будут постыжены; раб же Твой да возрадуется.
\vs Psa 108:29 Да облекутся противники мои бесчестьем и, как одеждою, покроются стыдом своим.
\vs Psa 108:30 И я громко буду устами моими славить Господа и среди множества прославлять Его,
\vs Psa 108:31 ибо Он стоит одесную бедного, чтобы спасти его от судящих душу его.
\vs Psa 109:0 Псалом Давида.
\rsbpar\vs Psa 109:1 Сказал Господь Господу моему: седи одесную Меня, доколе положу врагов Твоих в подножие ног Твоих.
\vs Psa 109:2 Жезл силы Твоей пошлет Господь с Сиона: господствуй среди врагов Твоих.
\vs Psa 109:3 В день силы Твоей народ Твой готов во благолепии святыни; из чрева прежде денницы подобно росе рождение Твое\fns{По переводу 70-ти: из чрева прежде денницы Я родил Тебя.}.
\vs Psa 109:4 Клялся Господь и не раскается: Ты священник вовек по чину Мелхиседека.
\vs Psa 109:5 Господь одесную Тебя. Он в день гнева Своего поразит царей;
\vs Psa 109:6 совершит суд над народами, наполнит \bibemph{землю} трупами, сокрушит голову в земле обширной.
\vs Psa 109:7 Из потока на пути будет пить, и потому вознесет главу.
\vs Psa 110:0 Аллилуия.
\rsbpar\vs Psa 110:1 Славлю [Тебя], Господи, всем сердцем [моим] в совете праведных и в собрании.
\vs Psa 110:2 Велики дела Господни, вожделенны для всех, любящих оные.
\vs Psa 110:3 Дело Его~--- слава и красота, и правда Его пребывает вовек.
\vs Psa 110:4 Памятными соделал Он чудеса Свои; милостив и щедр Господь.
\vs Psa 110:5 Пищу дает боящимся Его; вечно помнит завет Свой.
\vs Psa 110:6 Силу дел Своих явил Он народу Своему, чтобы дать ему наследие язычников.
\vs Psa 110:7 Дела рук Его~--- истина и суд; все заповеди Его верны,
\vs Psa 110:8 тверды на веки и веки, основаны на истине и правоте.
\vs Psa 110:9 Избавление послал Он народу Своему; заповедал на веки завет Свой. Свято и страшно имя Его!
\vs Psa 110:10 Начало мудрости~--- страх Господень; разум верный у всех, исполняющих \bibemph{заповеди Его}. Хвала Ему пребудет вовек.
\vs Psa 111:0 Аллилуия.
\rsbpar\vs Psa 111:1 Блажен муж, боящийся Господа и крепко любящий заповеди Его.
\vs Psa 111:2 Сильно будет на земле семя его; род правых благословится.
\vs Psa 111:3 Обилие и богатство в доме его, и правда его пребывает вовек.
\vs Psa 111:4 Во тьме восходит свет правым; благ он и милосерд и праведен.
\vs Psa 111:5 Добрый человек милует и взаймы дает; он даст твердость словам своим на суде.
\vs Psa 111:6 Он вовек не поколеблется; в вечной памяти будет праведник.
\vs Psa 111:7 Не убоится худой молвы: сердце его твердо, уповая на Господа.
\vs Psa 111:8 Утверждено сердце его: он не убоится, когда посмотрит на врагов своих.
\vs Psa 111:9 Он расточил, раздал нищим; правда его пребывает во веки; рог его вознесется во славе.
\vs Psa 111:10 Нечестивый увидит \bibemph{это} и будет досадовать, заскрежещет зубами своими и истает. Желание нечестивых погибнет.
\vs Psa 112:0 Аллилуия.
\rsbpar\vs Psa 112:1 Хвалите, рабы Господни, хвалите имя Господне.
\vs Psa 112:2 Да будет имя Господне благословенно отныне и вовек.
\vs Psa 112:3 От восхода солнца до запада \bibemph{да будет} прославляемо имя Господне.
\vs Psa 112:4 Высок над всеми народами Господь; над небесами слава Его.
\vs Psa 112:5 Кто, как Господь, Бог наш, Который, обитая на высоте,
\vs Psa 112:6 приклоняется, чтобы призирать на небо и на землю;
\vs Psa 112:7 из праха поднимает бедного, из брения возвышает нищего,
\vs Psa 112:8 чтобы посадить его с князьями, с князьями народа его;
\vs Psa 112:9 неплодную вселяет в дом матерью, радующеюся о детях? Аллилуия!
\vs Psa 113:0 [Аллилуия.]
\rsbpar\vs Psa 113:1 Когда вышел Израиль из Египта, дом Иакова~--- из народа иноплеменного,
\vs Psa 113:2 Иуда сделался святынею Его, Израиль~--- владением Его.
\vs Psa 113:3 Море увидело и побежало; Иордан обратился назад.
\vs Psa 113:4 Горы прыгали, как овны, и холмы, как агнцы.
\vs Psa 113:5 Что с тобою, море, что ты побежало, и [с тобою], Иордан, что ты обратился назад?
\vs Psa 113:6 Что вы прыгаете, горы, как овны, и вы, холмы, как агнцы?
\vs Psa 113:7 Пред лицем Господа трепещи, земля, пред лицем Бога Иаковлева,
\vs Psa 113:8 превращающего скалу в озеро воды и камень в источник вод.
\vs Psa 113:9 Не нам, Господи, не нам, но имени Твоему дай славу, ради милости Твоей, ради истины Твоей.
\vs Psa 113:10 Для чего язычникам говорить: <<где же Бог их>>?
\vs Psa 113:11 Бог наш на небесах [и на земле]; творит все, что хочет.
\vs Psa 113:12 А их идолы~--- серебро и золото, дело рук человеческих.
\vs Psa 113:13 Есть у них уста, но не говорят; есть у них глаза, но не видят;
\vs Psa 113:14 есть у них уши, но не слышат; есть у них ноздри, но не обоняют;
\vs Psa 113:15 есть у них руки, но не осязают; есть у них ноги, но не ходят; и они не издают голоса гортанью своею.
\vs Psa 113:16 Подобны им да будут делающие их и все, надеющиеся на них.
\vs Psa 113:17 [Дом] Израилев! уповай на Господа: Он наша помощь и щит.
\vs Psa 113:18 Дом Ааронов! уповай на Господа: Он наша помощь и щит.
\vs Psa 113:19 Боящиеся Господа! уповайте на Господа: Он наша помощь и щит.
\vs Psa 113:20 Господь помнит нас, благословляет [нас], благословляет дом Израилев, благословляет дом Ааронов;
\vs Psa 113:21 благословляет боящихся Господа, малых с великими.
\vs Psa 113:22 Да приложит вам Господь более и более, вам и детям вашим.
\vs Psa 113:23 Благословенны вы Господом, сотворившим небо и землю.
\vs Psa 113:24 Небо~--- небо Господу, а землю Он дал сынам человеческим.
\vs Psa 113:25 Ни мертвые восхвалят Господа, ни все нисходящие в могилу;
\vs Psa 113:26 но мы [живые] будем благословлять Господа отныне и вовек. Аллилуия.
\vs Psa 114:0 [Аллилуия.]
\rsbpar\vs Psa 114:1 Я радуюсь, что Господь услышал голос мой, моление мое;
\vs Psa 114:2 приклонил ко мне ухо Свое, и потому буду призывать Его во \bibemph{все} дни мои.
\vs Psa 114:3 Объяли меня болезни смертные, муки адские постигли меня; я встретил тесноту и скорбь.
\vs Psa 114:4 Тогда призвал я имя Господне: Господи! избавь душу мою.
\vs Psa 114:5 Милостив Господь и праведен, и милосерд Бог наш.
\vs Psa 114:6 Хранит Господь простодушных: я изнемог, и Он помог мне.
\vs Psa 114:7 Возвратись, душа моя, в покой твой, ибо Господь облагодетельствовал тебя.
\vs Psa 114:8 Ты избавил душу мою от смерти, очи мои от слез и ноги мои от преткновения. Буду ходить пред лицем Господним на земле живых.
\vs Psa 115:0 [Аллилуия.]
\rsbpar\vs Psa 115:1 Я веровал, и потому говорил: я сильно сокрушен.
\vs Psa 115:2 Я сказал в опрометчивости моей: всякий человек ложь.
\vs Psa 115:3 Что воздам Господу за все благодеяния Его ко мне?
\vs Psa 115:4 Чашу спасения прииму и имя Господне призову.
\vs Psa 115:5 Обеты мои воздам Господу пред всем народом Его.
\vs Psa 115:6 Дорог\acc{а} в очах Господних смерть святых Его!
\vs Psa 115:7 О, Господи! я раб Твой, я раб Твой и сын рабы Твоей; Ты разрешил узы мои.
\vs Psa 115:8 Тебе принесу жертву хвалы, и имя Господне призову.
\vs Psa 115:9 Обеты мои воздам Господу пред всем народом Его,
\vs Psa 115:10 во дворах дома Господня, посреди тебя, Иерусалим! Аллилуия.
\vs Psa 116:0 [Аллилуия.]
\rsbpar\vs Psa 116:1 Хвалите Господа, все народы, прославляйте Его, все племена;
\vs Psa 116:2 ибо велика милость Его к нам, и истина Господня [пребывает] вовек. Аллилуия.
\vs Psa 117:0 [Аллилуия.]
\rsbpar\vs Psa 117:1 Славьте Господа, ибо Он благ, ибо вовек милость Его.
\vs Psa 117:2 Да скажет ныне [дом] Израилев: [Он благ,] ибо вовек милость Его.
\vs Psa 117:3 Да скажет ныне дом Ааронов: [Он благ,] ибо вовек милость Его.
\vs Psa 117:4 Да скажут ныне боящиеся Господа: [Он благ,] ибо вовек милость Его.
\vs Psa 117:5 Из тесноты воззвал я к Господу,~--- и услышал меня, и на пространное место \bibemph{вывел меня} Господь.
\vs Psa 117:6 Господь за меня~--- не устрашусь: что сделает мне человек?
\vs Psa 117:7 Господь мне помощник: буду смотреть на врагов моих.
\vs Psa 117:8 Лучше уповать на Господа, нежели надеяться на человека.
\vs Psa 117:9 Лучше уповать на Господа, нежели надеяться на князей.
\vs Psa 117:10 Все народы окружили меня, но именем Господним я низложил их;
\vs Psa 117:11 обступили меня, окружили меня, но именем Господним я низложил их;
\vs Psa 117:12 окружили меня, как пчелы [сот], и угасли, как огонь в терне: именем Господним я низложил их.
\vs Psa 117:13 Сильно толкнули меня, чтобы я упал, но Господь поддержал меня.
\vs Psa 117:14 Господь~--- сила моя и песнь; Он соделался моим спасением.
\vs Psa 117:15 Глас радости и спасения в жилищах праведников: десница Господня творит силу!
\vs Psa 117:16 Десница Господня высока, десница Господня творит силу!
\vs Psa 117:17 Не умру, но буду жить и возвещать дела Господни.
\vs Psa 117:18 Строго наказал меня Господь, но смерти не предал меня.
\vs Psa 117:19 Отворите мне врата правды; войду в них, прославлю Господа.
\vs Psa 117:20 Вот врата Господа; праведные войдут в них.
\vs Psa 117:21 Славлю Тебя, что Ты услышал меня и соделался моим спасением.
\vs Psa 117:22 Камень, который отвергли строители, соделался главою угла:
\vs Psa 117:23 это~--- от Господа, и есть дивно в очах наших.
\vs Psa 117:24 Сей день сотворил Господь: возрадуемся и возвеселимся в оный!
\vs Psa 117:25 О, Господи, спаси же! О, Господи, споспешествуй же!
\vs Psa 117:26 Благословен грядущий во имя Господне! Благословляем вас из дома Господня.
\vs Psa 117:27 Бог~--- Господь, и осиял нас; вяжите вервями жертву, \bibemph{ведите} к рогам жертвенника.
\vs Psa 117:28 Ты Бог мой: буду славить Тебя; Ты Бог мой: буду превозносить Тебя, [буду славить Тебя, ибо Ты услышал меня и соделался моим спасением].
\vs Psa 117:29 Славьте Господа, ибо Он благ, ибо вовек милость Его.
\vs Psa 118:0 [Аллилуия.]
\rsbpar\vs Psa 118:1 Блаженны непорочные в пути, ходящие в законе Господнем.
\vs Psa 118:2 Блаженны хранящие откровения Его, всем сердцем ищущие Его.
\vs Psa 118:3 Они не делают беззакония, ходят путями Его.
\vs Psa 118:4 Ты заповедал повеления Твои хранить твердо.
\vs Psa 118:5 О, если бы направлялись пути мои к соблюдению уставов Твоих!
\vs Psa 118:6 Тогда я не постыдился бы, взирая на все заповеди Твои:
\vs Psa 118:7 я славил бы Тебя в правоте сердца, поучаясь судам правды Твоей.
\vs Psa 118:8 Буду хранить уставы Твои; не оставляй меня совсем.
\vs Psa 118:9 Как юноше содержать в чистоте путь свой?~--- Хранением себя по слову Твоему.
\vs Psa 118:10 Всем сердцем моим ищу Тебя; не дай мне уклониться от заповедей Твоих.
\vs Psa 118:11 В сердце моем сокрыл я слово Твое, чтобы не грешить пред Тобою.
\vs Psa 118:12 Благословен Ты, Господи! научи меня уставам Твоим.
\vs Psa 118:13 Устами моими возвещал я все суды уст Твоих.
\vs Psa 118:14 На пути откровений Твоих я радуюсь, как во всяком богатстве.
\vs Psa 118:15 О заповедях Твоих размышляю, и взираю на пути Твои.
\vs Psa 118:16 Уставами Твоими утешаюсь, не забываю слова Твоего.
\vs Psa 118:17 Яви милость рабу Твоему, и буду жить и хранить слово Твое.
\vs Psa 118:18 Открой очи мои, и увижу чудеса закона Твоего.
\vs Psa 118:19 Странник я на земле; не скрывай от меня заповедей Твоих.
\vs Psa 118:20 Истомилась душа моя желанием судов Твоих во всякое время.
\vs Psa 118:21 Ты укротил гордых, проклятых, уклоняющихся от заповедей Твоих.
\vs Psa 118:22 Сними с меня поношение и посрамление, ибо я храню откровения Твои.
\vs Psa 118:23 Князья сидят и сговариваются против меня, а раб Твой размышляет об уставах Твоих.
\vs Psa 118:24 Откровения Твои~--- утешение мое, [и уставы Твои]~--- советники мои.
\vs Psa 118:25 Душа моя повержена в прах; оживи меня по слову Твоему.
\vs Psa 118:26 Объявил я пути мои, и Ты услышал меня; научи меня уставам Твоим.
\vs Psa 118:27 Дай мне уразуметь путь повелений Твоих, и буду размышлять о чудесах Твоих.
\vs Psa 118:28 Душа моя истаевает от скорби: укрепи меня по слову Твоему.
\vs Psa 118:29 Удали от меня путь лжи, и закон Твой даруй мне.
\vs Psa 118:30 Я избрал путь истины, поставил пред собою суды Твои.
\vs Psa 118:31 Я прилепился к откровениям Твоим, Господи; не постыди меня.
\vs Psa 118:32 Потеку путем заповедей Твоих, когда Ты расширишь сердце мое.
\vs Psa 118:33 Укажи мне, Господи, путь уставов Твоих, и я буду держаться его до конца.
\vs Psa 118:34 Вразуми меня, и буду соблюдать закон Твой и хранить его всем сердцем.
\vs Psa 118:35 Поставь меня на стезю заповедей Твоих, ибо я возжелал ее.
\vs Psa 118:36 Приклони сердце мое к откровениям Твоим, а не к корысти.
\vs Psa 118:37 Отврати очи мои, чтобы не видеть суеты; животвори меня на пути Твоем.
\vs Psa 118:38 Утверди слово Твое рабу Твоему, ради благоговения пред Тобою.
\vs Psa 118:39 Отврати поношение мое, которого я страшусь, ибо суды Твои благи.
\vs Psa 118:40 Вот, я возжелал повелений Твоих; животвори меня правдою Твоею.
\vs Psa 118:41 Да придут ко мне милости Твои, Господи, спасение Твое по слову Твоему,~---
\vs Psa 118:42 и я дам ответ поносящему меня, ибо уповаю на слово Твое.
\vs Psa 118:43 Не отнимай совсем от уст моих слова истины, ибо я уповаю на суды Твои
\vs Psa 118:44 и буду хранить закон Твой всегда, во веки и веки;
\vs Psa 118:45 буду ходить свободно, ибо я взыскал повелений Твоих;
\vs Psa 118:46 буду говорить об откровениях Твоих пред царями и не постыжусь;
\vs Psa 118:47 буду утешаться заповедями Твоими, которые возлюбил;
\vs Psa 118:48 руки мои буду простирать к заповедям Твоим, которые возлюбил, и размышлять об уставах Твоих.
\vs Psa 118:49 Вспомни слово [Твое] к рабу Твоему, на которое Ты повелел мне уповать:
\vs Psa 118:50 это~--- утешение в бедствии моем, что слово Твое оживляет меня.
\vs Psa 118:51 Гордые крайне ругались надо мною, но я не уклонился от закона Твоего.
\vs Psa 118:52 Вспоминал суды Твои, Господи, от века, и утешался.
\vs Psa 118:53 Ужас овладевает мною при виде нечестивых, оставляющих закон Твой.
\vs Psa 118:54 Уставы Твои были песнями моими на месте странствований моих.
\vs Psa 118:55 Ночью вспоминал я имя Твое, Господи, и хранил закон Твой.
\vs Psa 118:56 Он стал моим, ибо повеления Твои храню.
\vs Psa 118:57 Удел мой, Господи, сказал я, соблюдать слова Твои.
\vs Psa 118:58 Молился я Тебе всем сердцем: помилуй меня по слову Твоему.
\vs Psa 118:59 Размышлял о путях моих и обращал стопы мои к откровениям Твоим.
\vs Psa 118:60 Спешил и не медлил соблюдать заповеди Твои.
\vs Psa 118:61 Сети нечестивых окружили меня, но я не забывал закона Твоего.
\vs Psa 118:62 В полночь вставал славословить Тебя за праведные суды Твои.
\vs Psa 118:63 Общник я всем боящимся Тебя и хранящим повеления Твои.
\vs Psa 118:64 Милости Твоей, Господи, полна земля; научи меня уставам Твоим.
\vs Psa 118:65 Благо сотворил Ты рабу Твоему, Господи, по слову Твоему.
\vs Psa 118:66 Доброму разумению и ведению научи меня, ибо заповедям Твоим я верую.
\vs Psa 118:67 Прежде страдания моего я заблуждался; а ныне слово Твое храню.
\vs Psa 118:68 Благ и благодетелен Ты, [Господи]; научи меня уставам Твоим.
\vs Psa 118:69 Гордые сплетают на меня ложь; я же всем сердцем буду хранить повеления Твои.
\vs Psa 118:70 Ожирело сердце их, как тук; я же законом Твоим утешаюсь.
\vs Psa 118:71 Благо мне, что я пострадал, дабы научиться уставам Твоим.
\vs Psa 118:72 Закон уст Твоих для меня лучше тысяч золота и серебра.
\rsbpar\vs Psa 118:73 Руки Твои сотворили меня и устроили меня; вразуми меня, и научусь заповедям Твоим.
\vs Psa 118:74 Боящиеся Тебя увидят меня~--- и возрадуются, что я уповаю на слово Твое.
\vs Psa 118:75 Знаю, Господи, что суды Твои праведны и по справедливости Ты наказал меня.
\vs Psa 118:76 Да будет же милость Твоя утешением моим, по слову Твоему к рабу Твоему.
\vs Psa 118:77 Да придет ко мне милосердие Твое, и я буду жить; ибо закон Твой~--- утешение мое.
\vs Psa 118:78 Да будут постыжены гордые, ибо безвинно угнетают меня; я размышляю о повелениях Твоих.
\vs Psa 118:79 Да обратятся ко мне боящиеся Тебя и знающие откровения Твои.
\vs Psa 118:80 Да будет сердце мое непорочно в уставах Твоих, чтобы я не посрамился.
\vs Psa 118:81 Истаевает душа моя о спасении Твоем; уповаю на слово Твое.
\vs Psa 118:82 Истаевают очи мои о слове Твоем; я говорю: когда Ты утешишь меня?
\vs Psa 118:83 Я стал, как мех в дыму, \bibemph{но} уставов Твоих не забыл.
\vs Psa 118:84 Сколько дней раба Твоего? Когда произведешь суд над гонителями моими?
\vs Psa 118:85 Яму вырыли мне гордые, вопреки закону Твоему.
\vs Psa 118:86 Все заповеди Твои~--- истина; несправедливо преследуют меня: помоги мне;
\vs Psa 118:87 едва не погубили меня на земле, но я не оставил повелений Твоих.
\vs Psa 118:88 По милости Твоей оживляй меня, и буду хранить откровения уст Твоих.
\vs Psa 118:89 На веки, Господи, слово Твое утверждено на небесах;
\vs Psa 118:90 истина Твоя в род и род. Ты поставил землю, и она стоит.
\vs Psa 118:91 По определениям Твоим все стоит доныне, ибо все служит Тебе.
\vs Psa 118:92 Если бы не закон Твой был утешением моим, погиб бы я в бедствии моем.
\vs Psa 118:93 Вовек не забуду повелений Твоих, ибо ими Ты оживляешь меня.
\vs Psa 118:94 Твой я, спаси меня; ибо я взыскал повелений Твоих.
\vs Psa 118:95 Нечестивые подстерегают меня, чтобы погубить; \bibemph{а} я углубляюсь в откровения Твои.
\vs Psa 118:96 Я видел предел всякого совершенства, \bibemph{но} Твоя заповедь безмерно обширна.
\vs Psa 118:97 Как люблю я закон Твой! весь день размышляю о нем.
\vs Psa 118:98 Заповедью Твоею Ты соделал меня мудрее врагов моих, ибо она всегда со мною.
\vs Psa 118:99 Я стал разумнее всех учителей моих, ибо размышляю об откровениях Твоих.
\vs Psa 118:100 Я сведущ более старцев, ибо повеления Твои храню.
\vs Psa 118:101 От всякого злого пути удерживаю ноги мои, чтобы хранить слово Твое;
\vs Psa 118:102 от судов Твоих не уклоняюсь, ибо Ты научаешь меня.
\vs Psa 118:103 Как сладки гортани моей слова Твои! лучше меда устам моим.
\vs Psa 118:104 Повелениями Твоими я вразумлен; потому ненавижу всякий путь лжи.
\vs Psa 118:105 Слово Твое~--- светильник ноге моей и свет стезе моей.
\vs Psa 118:106 Я клялся хранить праведные суды Твои, и исполню.
\vs Psa 118:107 Сильно угнетен я, Господи; оживи меня по слову Твоему.
\vs Psa 118:108 Благоволи же, Господи, принять добровольную жертву уст моих, и судам Твоим научи меня.
\vs Psa 118:109 Душа моя непрестанно в руке моей, но закона Твоего не забываю.
\vs Psa 118:110 Нечестивые поставили для меня сеть, но я не уклонился от повелений Твоих.
\vs Psa 118:111 Откровения Твои я принял, как наследие на веки, ибо они веселие сердца моего.
\vs Psa 118:112 Я приклонил сердце мое к исполнению уставов Твоих навек, до конца.
\vs Psa 118:113 Вымыслы \bibemph{человеческие} ненавижу, а закон Твой люблю.
\vs Psa 118:114 Ты покров мой и щит мой; на слово Твое уповаю.
\vs Psa 118:115 Удалитесь от меня, беззаконные, и буду хранить заповеди Бога моего.
\vs Psa 118:116 Укрепи меня по слову Твоему, и буду жить; не посрами меня в надежде моей;
\vs Psa 118:117 поддержи меня, и спасусь; и в уставы Твои буду вникать непрестанно.
\vs Psa 118:118 Всех, отступающих от уставов Твоих, Ты низлагаешь, ибо ухищрения их~--- ложь.
\vs Psa 118:119 \bibemph{Как} изгарь, отметаешь Ты всех нечестивых земли; потому я возлюбил откровения Твои.
\vs Psa 118:120 Трепещет от страха Твоего плоть моя, и судов Твоих я боюсь.
\vs Psa 118:121 Я совершал суд и правду; не предай меня гонителям моим.
\vs Psa 118:122 Заступи раба Твоего ко благу \bibemph{его}, чтобы не угнетали меня гордые.
\vs Psa 118:123 Истаевают очи мои, ожидая спасения Твоего и слова правды Твоей.
\vs Psa 118:124 Сотвори с рабом Твоим по милости Твоей, и уставам Твоим научи меня.
\vs Psa 118:125 Я раб Твой: вразуми меня, и познаю откровения Твои.
\vs Psa 118:126 Время Господу действовать: закон Твой разорили.
\vs Psa 118:127 А я люблю заповеди Твои более золота, и золота чистого.
\vs Psa 118:128 Все повеления Твои~--- все призна\acc{ю} справедливыми; всякий путь лжи ненавижу.
\vs Psa 118:129 Дивны откровения Твои; потому хранит их душа моя.
\vs Psa 118:130 Откровение слов Твоих просвещает, вразумляет простых.
\vs Psa 118:131 Открываю уста мои и вздыхаю, ибо заповедей Твоих жажду.
\rsbpar\vs Psa 118:132 Призри на меня и помилуй меня, как поступаешь с любящими имя Твое.
\vs Psa 118:133 Утверди стопы мои в слове Твоем и не дай овладеть мною никакому беззаконию;
\vs Psa 118:134 избавь меня от угнетения человеческого, и буду хранить повеления Твои;
\vs Psa 118:135 осияй раба Твоего светом лица Твоего и научи меня уставам Твоим.
\vs Psa 118:136 Из глаз моих текут потоки вод оттого, что не хранят закона Твоего.
\vs Psa 118:137 Праведен Ты, Господи, и справедливы суды Твои.
\vs Psa 118:138 Откровения Твои, которые Ты заповедал,~--- правда и совершенная истина.
\vs Psa 118:139 Ревность моя снедает меня, потому что мои враги забыли слова Твои.
\vs Psa 118:140 Слово Твое весьма чисто, и раб Твой возлюбил его.
\vs Psa 118:141 Мал я и презрен, \bibemph{но} повелений Твоих не забываю.
\vs Psa 118:142 Правда Твоя~--- правда вечная, и закон Твой~--- истина.
\vs Psa 118:143 Скорбь и горесть постигли меня; заповеди Твои~--- утешение мое.
\vs Psa 118:144 Правда откровений Твоих вечна: вразуми меня, и буду жить.
\vs Psa 118:145 Взываю всем сердцем [моим]: услышь меня, Господи,~--- и сохраню уставы Твои.
\vs Psa 118:146 Призываю Тебя: спаси меня, и буду хранить откровения Твои.
\vs Psa 118:147 Предваряю рассвет и взываю; на слово Твое уповаю.
\vs Psa 118:148 Очи мои предваряют \bibemph{утреннюю} стражу, чтобы мне углубляться в слово Твое.
\vs Psa 118:149 Услышь голос мой по милости Твоей, Господи; по суду Твоему оживи меня.
\vs Psa 118:150 Приблизились замышляющие лукавство; далеки они от закона Твоего.
\vs Psa 118:151 Близок Ты, Господи, и все заповеди Твои~--- истина.
\vs Psa 118:152 Издавна узнал я об откровениях Твоих, что Ты утвердил их на веки.
\vs Psa 118:153 Воззри на бедствие мое и избавь меня, ибо я не забываю закона Твоего.
\vs Psa 118:154 Вступись в дело мое и защити меня; по слову Твоему оживи меня.
\vs Psa 118:155 Далеко от нечестивых спасение, ибо они уставов Твоих не ищут.
\vs Psa 118:156 Много щедрот Твоих, Господи; по суду Твоему оживи меня.
\vs Psa 118:157 Много у меня гонителей и врагов, \bibemph{но} от откровений Твоих я не удаляюсь.
\vs Psa 118:158 Вижу отступников, и сокрушаюсь, ибо они не хранят слова Твоего.
\vs Psa 118:159 Зри, как я люблю повеления Твои; по милости Твоей, Господи, оживи меня.
\vs Psa 118:160 Основание слова Твоего истинно, и вечен всякий суд правды Твоей.
\vs Psa 118:161 Князья гонят меня безвинно, но сердце мое боится слова Твоего.
\vs Psa 118:162 Радуюсь я слову Твоему, как получивший великую прибыль.
\vs Psa 118:163 Ненавижу ложь и гнушаюсь ею; закон же Твой люблю.
\vs Psa 118:164 Семикратно в день прославляю Тебя за суды правды Твоей.
\vs Psa 118:165 Велик мир у любящих закон Твой, и нет им преткновения.
\vs Psa 118:166 Уповаю на спасение Твое, Господи, и заповеди Твои исполняю.
\vs Psa 118:167 Душа моя хранит откровения Твои, и я люблю их крепко.
\vs Psa 118:168 Храню повеления Твои и откровения Твои, ибо все пути мои пред Тобою.
\vs Psa 118:169 Да приблизится вопль мой пред лице Твое, Господи; по слову Твоему вразуми меня.
\vs Psa 118:170 Да придет моление мое пред лице Твое; по слову Твоему избавь меня.
\vs Psa 118:171 Уста мои произнесут хвалу, когда Ты научишь меня уставам Твоим.
\vs Psa 118:172 Язык мой возгласит слово Твое, ибо все заповеди Твои праведны.
\vs Psa 118:173 Да будет рука Твоя в помощь мне, ибо я повеления Твои избрал.
\vs Psa 118:174 Жажду спасения Твоего, Господи, и закон Твой~--- утешение мое.
\vs Psa 118:175 Да живет душа моя и славит Тебя, и суды Твои да помогут мне.
\vs Psa 118:176 Я заблудился, как овца потерянная: взыщи раба Твоего, ибо я заповедей Твоих не забыл.
\vs Psa 119:0 Песнь восхождения.
\rsbpar\vs Psa 119:1 К Господу воззвал я в скорби моей, и Он услышал меня.
\vs Psa 119:2 Господи! избавь душу мою от уст лживых, от языка лукавого.
\vs Psa 119:3 Что даст тебе и что прибавит тебе язык лукавый?
\vs Psa 119:4 Изощренные стрелы сильного, с горящими углями дроковыми.
\vs Psa 119:5 Горе мне, что я пребываю у Мосоха, живу у шатров Кидарских.
\vs Psa 119:6 Долго жила душа моя с ненавидящими мир.
\vs Psa 119:7 Я мирен: но только заговорю, они~--- к войне.
\vs Psa 120:0 Песнь восхождения.
\rsbpar\vs Psa 120:1 Возвожу очи мои к горам, откуда придет помощь моя.
\vs Psa 120:2 Помощь моя от Господа, сотворившего небо и землю.
\vs Psa 120:3 Не даст Он поколебаться ноге твоей, не воздремлет хранящий тебя;
\vs Psa 120:4 не дремлет и не спит хранящий Израиля.
\vs Psa 120:5 Господь~--- хранитель твой; Господь~--- сень твоя с правой руки твоей.
\vs Psa 120:6 Днем солнце не поразит тебя, ни луна ночью.
\vs Psa 120:7 Господь сохранит тебя от всякого зла; сохранит душу твою [Господь].
\vs Psa 120:8 Господь будет охранять выхождение твое и вхождение твое отныне и вовек.
\vs Psa 121:0 Песнь восхождения. Давида.
\rsbpar\vs Psa 121:1 Возрадовался я, когда сказали мне: <<пойдем в дом Господень>>.
\vs Psa 121:2 Вот, стоят ноги наши во вратах твоих, Иерусалим,~---
\vs Psa 121:3 Иерусалим, устроенный как город, слитый в одно,
\vs Psa 121:4 куда восходят колена, колена Господни, по закону Израилеву, славить имя Господне.
\vs Psa 121:5 Там стоят престолы суда, престолы дома Давидова.
\vs Psa 121:6 Просите мира Иерусалиму: да благоденствуют любящие тебя!
\vs Psa 121:7 Да будет мир в стенах твоих, благоденствие~--- в чертогах твоих!
\vs Psa 121:8 Ради братьев моих и ближних моих говорю я: <<мир тебе!>>
\vs Psa 121:9 Ради дома Господа, Бога нашего, желаю блага тебе.
\vs Psa 122:0 Песнь восхождения.
\rsbpar\vs Psa 122:1 К Тебе возвожу очи мои, Живущий на небесах!
\vs Psa 122:2 Вот, как очи рабов \bibemph{обращены} на руку господ их, как очи рабы~--- на руку госпожи ее, так очи наши~--- к Господу, Богу нашему, доколе Он помилует нас.
\vs Psa 122:3 Помилуй нас, Господи, помилуй нас, ибо довольно мы насыщены презрением;
\vs Psa 122:4 довольно насыщена душа наша поношением от надменных и уничижением от гордых.
\vs Psa 123:0 Песнь восхождения. Давида.
\rsbpar\vs Psa 123:1 Если бы не Господь был с нами,~--- да скажет Израиль,~---
\vs Psa 123:2 если бы не Господь был с нами, когда восстали на нас люди,
\vs Psa 123:3 то живых они поглотили бы нас, когда возгорелась ярость их на нас;
\vs Psa 123:4 воды потопили бы нас, поток прошел бы над душею нашею;
\vs Psa 123:5 прошли бы над душею нашею воды бурные.
\vs Psa 123:6 Благословен Господь, Который не дал нас в добычу зубам их!
\vs Psa 123:7 Душа наша избавилась, как птица, из сети ловящих: сеть расторгнута, и мы избавились.
\vs Psa 123:8 Помощь наша~--- в имени Господа, сотворившего небо и землю.
\vs Psa 124:0 Песнь восхождения.
\rsbpar\vs Psa 124:1 Надеющийся на Господа, как гора Сион, не подвигнется: пребывает вовек.
\vs Psa 124:2 Горы окрест Иерусалима, а Господь окрест народа Своего отныне и вовек.
\vs Psa 124:3 Ибо не оставит [Господь] жезла нечестивых над жребием праведных, дабы праведные не простерли рук своих к беззаконию.
\vs Psa 124:4 Благотвори, Господи, добрым и правым в сердцах своих;
\vs Psa 124:5 а совращающихся на кривые пути свои да оставит Господь ходить с делающими беззаконие. Мир на Израиля!
\vs Psa 125:0 Песнь восхождения.
\rsbpar\vs Psa 125:1 Когда возвращал Господь плен Сиона, мы были как бы видящие во сне:
\vs Psa 125:2 тогда уста наши были полны веселья, и язык наш~--- пения; тогда между народами говорили: <<великое сотворил Господь над ними!>>
\vs Psa 125:3 Великое сотворил Господь над нами: мы радовались.
\vs Psa 125:4 Возврати, Господи, пленников наших, как потоки на полдень.
\vs Psa 125:5 Сеявшие со слезами будут пожинать с радостью.
\vs Psa 125:6 С плачем несущий семена возвратится с радостью, неся снопы свои.
\vs Psa 126:0 Песнь восхождения. Соломона.
\rsbpar\vs Psa 126:1 Если Господь не созиждет дома, напрасно трудятся строящие его; если Господь не охранит города, напрасно бодрствует страж.
\vs Psa 126:2 Напрасно вы рано встаете, поздно просиживаете, едите хлеб печали, тогда как возлюбленному Своему Он дает сон.
\vs Psa 126:3 Вот наследие от Господа: дети; награда от Него~--- плод чрева.
\vs Psa 126:4 Что стрелы в руке сильного, то сыновья молодые.
\vs Psa 126:5 Блажен человек, который наполнил ими колчан свой! Не останутся они в стыде, когда будут говорить с врагами в воротах.
\vs Psa 127:0 Песнь восхождения.
\rsbpar\vs Psa 127:1 Блажен всякий боящийся Господа, ходящий путями Его!
\vs Psa 127:2 Ты будешь есть от трудов рук твоих: блажен ты, и благо тебе!
\vs Psa 127:3 Жена твоя, как плодовитая лоза, в доме твоем; сыновья твои, как масличные ветви, вокруг трапезы твоей:
\vs Psa 127:4 так благословится человек, боящийся Господа!
\vs Psa 127:5 Благословит тебя Господь с Сиона, и увидишь благоденствие Иерусалима во все дни жизни твоей;
\vs Psa 127:6 увидишь сыновей у сыновей твоих. Мир на Израиля!
\vs Psa 128:0 Песнь восхождения.
\rsbpar\vs Psa 128:1 Много теснили меня от юности моей, да скажет Израиль:
\vs Psa 128:2 много теснили меня от юности моей, но не одолели меня.
\vs Psa 128:3 На хребте моем орали оратаи, проводили длинные борозды свои.
\vs Psa 128:4 Но Господь праведен: Он рассек узы нечестивых.
\vs Psa 128:5 Да постыдятся и обратятся назад все ненавидящие Сион!
\vs Psa 128:6 Да будут, как трава на кровлях, которая прежде, нежели будет исторгнута, засыхает,
\vs Psa 128:7 которою жнец не наполнит руки своей, и вяжущий снопы~--- горсти своей;
\vs Psa 128:8 и проходящие мимо не скажут: <<благословение Господне на вас; благословляем вас именем Господним!>>
\vs Psa 129:0 Песнь восхождения.
\rsbpar\vs Psa 129:1 Из глубины взываю к Тебе, Господи.
\vs Psa 129:2 Господи! услышь голос мой. Да будут уши Твои внимательны к голосу молений моих.
\vs Psa 129:3 Если Ты, Господи, будешь замечать беззакония,~--- Господи! кто устоит?
\vs Psa 129:4 Но у Тебя прощение, да благоговеют пред Тобою.
\vs Psa 129:5 Надеюсь на Господа, надеется душа моя; на слово Его уповаю.
\vs Psa 129:6 Душа моя ожидает Господа более, нежели стражи~--- утра, более, нежели стражи~--- утра.
\vs Psa 129:7 Да уповает Израиль на Господа, ибо у Господа милость и многое у Него избавление,
\vs Psa 129:8 и Он избавит Израиля от всех беззаконий его.
\vs Psa 130:0 Песнь восхождения. Давида.
\rsbpar\vs Psa 130:1 Господи! не надмевалось сердце мое и не возносились очи мои, и я не входил в великое и для меня недосягаемое.
\vs Psa 130:2 Не смирял ли я и не успокаивал ли души моей, как дитяти, отнятого от груди матери? душа моя была во мне, как дитя, отнятое от груди.
\vs Psa 130:3 Да уповает Израиль на Господа отныне и вовек.
\vs Psa 131:0 Песнь восхождения.
\rsbpar\vs Psa 131:1 Вспомни, Господи, Давида и все сокрушение его:
\vs Psa 131:2 как он клялся Господу, давал обет Сильному Иакова:
\vs Psa 131:3 <<не войду в шатер дома моего, не взойду на ложе мое;
\vs Psa 131:4 не дам сна очам моим и веждам моим~--- дремания,
\vs Psa 131:5 доколе не найду места Господу, жилища~--- Сильному Иакова>>.
\vs Psa 131:6 Вот, мы слышали о нем в Ефрафе, нашли его на полях Иарима.
\vs Psa 131:7 Пойдем к жилищу Его, поклонимся подножию ног Его.
\vs Psa 131:8 Стань, Господи, на \bibemph{место} покоя Твоего,~--- Ты и ковчег могущества Твоего.
\vs Psa 131:9 Священники Твои облекутся правдою, и святые Твои возрадуются.
\vs Psa 131:10 Ради Давида, раба Твоего, не отврати лица помазанника Твоего.
\vs Psa 131:11 Клялся Господь Давиду в истине, и не отречется ее: <<от плода чрева твоего посажу на престоле твоем.
\vs Psa 131:12 Если сыновья твои будут сохранять завет Мой и откровения Мои, которым Я научу их, то и их сыновья во веки будут сидеть на престоле твоем>>.
\vs Psa 131:13 Ибо избрал Господь Сион, возжелал [его] в жилище Себе.
\vs Psa 131:14 <<Это покой Мой на веки: здесь вселюсь, ибо Я возжелал его.
\vs Psa 131:15 Пищу его благословляя благословлю, нищих его насыщу хлебом;
\vs Psa 131:16 священников его облеку во спасение, и святые его радостью возрадуются.
\vs Psa 131:17 Там возращу рог Давиду, поставлю светильник помазаннику Моему.
\vs Psa 131:18 Врагов его облеку стыдом, а на нем будет сиять венец его>>.
\vs Psa 132:0 Песнь восхождения. Давида.
\rsbpar\vs Psa 132:1 Как хорошо и как приятно жить братьям вместе!
\vs Psa 132:2 \bibemph{Это}~--- как драгоценный елей на голове, стекающий на бороду, бороду Ааронову, стекающий на края одежды его;
\vs Psa 132:3 как роса Ермонская, сходящая на горы Сионские, ибо там заповедал Господь благословение и жизнь на веки.
\vs Psa 133:0 Песнь восхождения.
\rsbpar\vs Psa 133:1 Благословите ныне Господа, все рабы Господни, стоящие в доме Господнем, [во дворах дома Бога нашего,] во время ночи.
\vs Psa 133:2 Воздвигните руки ваши к святилищу, и благословите Господа.
\vs Psa 133:3 Благословит тебя Господь с Сиона, сотворивший небо и землю.
\vs Psa 134:0 Аллилуия.
\rsbpar\vs Psa 134:1 Хвалите имя Господне, хвалите, рабы Господни,
\vs Psa 134:2 стоящие в доме Господнем, во дворах дома Бога нашего.
\vs Psa 134:3 Хвалите Господа, ибо Господь благ; пойте имени Его, ибо это сладостно,
\vs Psa 134:4 ибо Господь избрал Себе Иакова, Израиля в собственность Свою.
\vs Psa 134:5 Я познал, что велик Господь, и Господь наш превыше всех богов.
\vs Psa 134:6 Господь творит все, что хочет, на небесах и на земле, на морях и во всех безднах;
\vs Psa 134:7 возводит облака от края земли, творит молнии при дожде, изводит ветер из хранилищ Своих.
\vs Psa 134:8 Он поразил первенцев Египта, от человека до скота,
\vs Psa 134:9 послал знамения и чудеса среди тебя, Египет, на фараона и на всех рабов его,
\vs Psa 134:10 поразил народы многие и истребил царей сильных:
\vs Psa 134:11 Сигона, царя Аморрейского, и Ога, царя Васанского, и все царства Ханаанские;
\vs Psa 134:12 и отдал землю их в наследие, в наследие Израилю, народу Своему.
\vs Psa 134:13 Господи! имя Твое вовек; Господи! память о Тебе в род и род.
\vs Psa 134:14 Ибо Господь будет судить народ Свой и над рабами Своими умилосердится.
\vs Psa 134:15 Идолы язычников~--- серебро и золото, дело рук человеческих:
\vs Psa 134:16 есть у них уста, но не говорят; есть у них глаза, но не видят;
\vs Psa 134:17 есть у них уши, но не слышат, и нет дыхания в устах их.
\vs Psa 134:18 Подобны им будут делающие их и всякий, кто надеется на них.
\vs Psa 134:19 Дом Израилев! благословите Господа. Дом Ааронов! благословите Господа.
\vs Psa 134:20 Дом Левиин! благословите Господа. Боящиеся Господа! благословите Господа.
\vs Psa 134:21 Благословен Господь от Сиона, живущий в Иерусалиме! Аллилуия!
\vs Psa 135:0 [Аллилуия.]
\rsbpar\vs Psa 135:1 Славьте Господа, ибо Он благ, ибо вовек милость Его.
\vs Psa 135:2 Славьте Бога богов, ибо вовек милость Его.
\vs Psa 135:3 Славьте Господа господствующих, ибо вовек милость Его;
\vs Psa 135:4 Того, Который один творит чудеса великие, ибо вовек милость Его;
\vs Psa 135:5 Который сотворил небеса премудро, ибо вовек милость Его;
\vs Psa 135:6 утвердил землю на водах, ибо вовек милость Его;
\vs Psa 135:7 сотворил светила великие, ибо вовек милость Его;
\vs Psa 135:8 солнце~--- для управления днем, ибо вовек милость Его;
\vs Psa 135:9 луну и звезды~--- для управления ночью, ибо вовек милость Его;
\vs Psa 135:10 поразил Египет в первенцах его, ибо вовек милость Его;
\vs Psa 135:11 и вывел Израиля из среды его, ибо вовек милость Его;
\vs Psa 135:12 рукою крепкою и мышцею простертою, ибо вовек милость Его;
\vs Psa 135:13 разделил Чермное море, ибо вовек милость Его;
\vs Psa 135:14 и провел Израиля посреди его, ибо вовек милость Его;
\vs Psa 135:15 и низверг фараона и войско его в море Чермное, ибо вовек милость Его;
\vs Psa 135:16 провел народ Свой чрез пустыню, ибо вовек милость Его;
\vs Psa 135:17 поразил царей великих, ибо вовек милость Его;
\vs Psa 135:18 и убил царей сильных, ибо вовек милость Его;
\vs Psa 135:19 Сигона, царя Аморрейского, ибо вовек милость Его;
\vs Psa 135:20 и Ога, царя Васанского, ибо вовек милость Его;
\vs Psa 135:21 и отдал землю их в наследие, ибо вовек милость Его;
\vs Psa 135:22 в наследие Израилю, рабу Своему, ибо вовек милость Его;
\vs Psa 135:23 вспомнил нас в унижении нашем, ибо вовек милость Его;
\vs Psa 135:24 и избавил нас от врагов наших, ибо вовек милость Его;
\vs Psa 135:25 дает пищу всякой плоти, ибо вовек милость Его.
\vs Psa 135:26 Славьте Бога небес, ибо вовек милость Его.
\vs Psa 136:0 [Давида.]
\rsbpar\vs Psa 136:1 При реках Вавилона, там сидели мы и плакали, когда вспоминали о Сионе;
\vs Psa 136:2 на вербах, посреди его, повесили мы наши арфы.
\vs Psa 136:3 Там пленившие нас требовали от нас слов песней, и притеснители наши~--- веселья: <<пропойте нам из песней Сионских>>.
\vs Psa 136:4 Как нам петь песнь Господню на земле чужой?
\vs Psa 136:5 Если я забуду тебя, Иерусалим,~--- забудь меня десница моя;
\vs Psa 136:6 прилипни язык мой к гортани моей, если не буду помнить тебя, если не поставлю Иерусалима во главе веселия моего.
\vs Psa 136:7 Припомни, Господи, сынам Едомовым день Иерусалима, когда они говорили: <<разрушайте, разрушайте до основания его>>.
\vs Psa 136:8 Дочь Вавилона, опустошительница! блажен, кто воздаст тебе за то, что ты сделала нам!
\vs Psa 136:9 Блажен, кто возьмет и разобьет младенцев твоих о камень!
\vs Psa 137:0 Давида.
\rsbpar\vs Psa 137:1 Славлю Тебя всем сердцем моим, пред богами\fns{В переводе 70-ти: пред Ангелами.} пою Тебе, [что Ты услышал все слова уст моих].
\vs Psa 137:2 Поклоняюсь пред святым храмом Твоим и славлю имя Твое за милость Твою и за истину Твою, ибо Ты возвеличил слово Твое превыше всякого имени Твоего.
\vs Psa 137:3 В день, когда я воззвал, Ты услышал меня, вселил в душу мою бодрость.
\vs Psa 137:4 Прославят Тебя, Господи, все цари земные, когда услышат слова уст Твоих
\vs Psa 137:5 и воспоют пути Господни, ибо велика слава Господня.
\vs Psa 137:6 Высок Господь: и смиренного видит, и гордого узнает издали.
\vs Psa 137:7 Если я пойду посреди напастей, Ты оживишь меня, прострешь на ярость врагов моих руку Твою, и спасет меня десница Твоя.
\vs Psa 137:8 Господь совершит за меня! Милость Твоя, Господи, вовек: дело рук Твоих не оставляй.
\vs Psa 138:0 Начальнику хора. Псалом Давида.
\rsbpar\vs Psa 138:1 Господи! Ты испытал меня и знаешь.
\vs Psa 138:2 Ты знаешь, когда я сажусь и когда встаю; Ты разумеешь помышления мои издали.
\vs Psa 138:3 Иду ли я, отдыхаю ли~--- Ты окружаешь меня, и все пути мои известны Тебе.
\vs Psa 138:4 Еще нет слова на языке моем,~--- Ты, Господи, уже знаешь его совершенно.
\vs Psa 138:5 Сзади и спереди Ты объемлешь меня, и полагаешь на мне руку Твою.
\vs Psa 138:6 Дивно для меня ведение [Твое],~--- высоко, не могу постигнуть его!
\vs Psa 138:7 Куда пойду от Духа Твоего, и от лица Твоего куда убегу?
\vs Psa 138:8 Взойду ли на небо~--- Ты там; сойду ли в преисподнюю~--- и там Ты.
\vs Psa 138:9 Возьму ли крылья зари и переселюсь на край моря,~---
\vs Psa 138:10 и там рука Твоя поведет меня, и удержит меня десница Твоя.
\vs Psa 138:11 Скажу ли: <<может быть, тьма скроет меня, и свет вокруг меня \bibemph{сделается} ночью>>;
\vs Psa 138:12 но и тьма не затмит от Тебя, и ночь светла, как день: как тьма, так и свет.
\vs Psa 138:13 Ибо Ты устроил внутренности мои и соткал меня во чреве матери моей.
\vs Psa 138:14 Славлю Тебя, потому что я дивно устроен. Дивны дела Твои, и душа моя вполне сознает это.
\vs Psa 138:15 Не сокрыты были от Тебя кости мои, когда я созидаем был в тайне, образуем был во глубине утробы.
\vs Psa 138:16 Зародыш мой видели очи Твои; в Твоей книге записаны все дни, для меня назначенные, когда ни одного из них еще не было.
\vs Psa 138:17 Как возвышенны для меня помышления Твои, Боже, и как велико число их!
\vs Psa 138:18 Стану ли исчислять их, но они многочисленнее песка; когда я пробуждаюсь, я все еще с Тобою.
\vs Psa 138:19 О, если бы Ты, Боже, поразил нечестивого! Удалитесь от меня, кровожадные!
\vs Psa 138:20 Они говорят против Тебя нечестиво; суетное замышляют враги Твои.
\vs Psa 138:21 Мне ли не возненавидеть ненавидящих Тебя, Господи, и не возгнушаться восстающими на Тебя?
\vs Psa 138:22 Полною ненавистью ненавижу их: враги они мне.
\vs Psa 138:23 Испытай меня, Боже, и узнай сердце мое; испытай меня и узнай помышления мои;
\vs Psa 138:24 и зри, не на опасном ли я пути, и направь меня на путь вечный.
\vs Psa 139:0 Псалом.
\vs Psa 139:1 Начальнику хора. Псалом Давида.
\rsbpar\vs Psa 139:2 Избавь меня, Господи, от человека злого; сохрани меня от притеснителя:
\vs Psa 139:3 они злое мыслят в сердце, всякий день ополчаются на брань,
\vs Psa 139:4 изощряют язык свой, как змея; яд аспида под устами их.
\vs Psa 139:5 Соблюди меня, Господи, от рук нечестивого, сохрани меня от притеснителей, которые замыслили поколебать стопы мои.
\vs Psa 139:6 Гордые скрыли силки для меня и петли, раскинули сеть по дороге, тенета разложили для меня.
\vs Psa 139:7 Я сказал Господу: Ты Бог мой; услышь, Господи, голос молений моих!
\vs Psa 139:8 Господи, Господи, сила спасения моего! Ты покрыл голову мою в день брани.
\vs Psa 139:9 Не дай, Господи, желаемого нечестивому; не дай успеха злому замыслу его: они возгордятся.
\vs Psa 139:10 Да покроет головы окружающих меня зло собственных уст их.
\vs Psa 139:11 Да падут на них горящие угли; да будут они повержены в огонь, в пропасти, так, чтобы не встали.
\vs Psa 139:12 Человек злоязычный не утвердится на земле; зло увлечет притеснителя в погибель.
\vs Psa 139:13 Знаю, что Господь сотворит суд угнетенным и справедливость бедным.
\vs Psa 139:14 Так! праведные будут славить имя Твое; непорочные будут обитать пред лицем Твоим.
\vs Psa 140:0 Псалом Давида.
\rsbpar\vs Psa 140:1 Господи! к Тебе взываю: поспеши ко мне, внемли голосу моления моего, когда взываю к Тебе.
\vs Psa 140:2 Да направится молитва моя, как фимиам, пред лице Твое, воздеяние рук моих~--- как жертва вечерняя.
\vs Psa 140:3 Положи, Господи, охрану устам моим, и огради двери уст моих;
\vs Psa 140:4 не дай уклониться сердцу моему к словам лукавым для извинения дел греховных вместе с людьми, делающими беззаконие, и да не вкушу я от сластей их.
\vs Psa 140:5 Пусть наказывает меня праведник: это милость; пусть обличает меня: это лучший елей, который не повредит голове моей; но мольбы мои~--- против злодейств их.
\vs Psa 140:6 Вожди их рассыпались по утесам и слышат слова мои, что они кротки.
\vs Psa 140:7 Как будто землю рассекают и дробят нас; сыплются кости наши в челюсти преисподней.
\vs Psa 140:8 Но к Тебе, Господи, Господи, очи мои; на Тебя уповаю, не отринь души моей!
\vs Psa 140:9 Сохрани меня от силков, поставленных для меня, от тенет беззаконников.
\vs Psa 140:10 Падут нечестивые в сети свои, а я перейду.
\vs Psa 141:0 Учение Давида. Молитва его, когда он был в пещере.
\rsbpar\vs Psa 141:1 Голосом моим к Господу воззвал я, голосом моим к Господу помолился;
\vs Psa 141:2 излил пред Ним моление мое; печаль мою открыл Ему.
\vs Psa 141:3 Когда изнемогал во мне дух мой, Ты знал стезю мою. На пути, которым я ходил, они скрытно поставили сети для меня.
\vs Psa 141:4 Смотрю на правую сторону, и вижу, что никто не признаёт меня: не стало для меня убежища, никто не заботится о душе моей.
\vs Psa 141:5 Я воззвал к Тебе, Господи, я сказал: Ты прибежище мое и часть моя на земле живых.
\vs Psa 141:6 Внемли воплю моему, ибо я очень изнемог; избавь меня от гонителей моих, ибо они сильнее меня.
\vs Psa 141:7 Выведи из темницы душу мою, чтобы мне славить имя Твое. Вокруг меня соберутся праведные, когда Ты явишь мне благодеяние.
\vs Psa 142:0 Псалом Давида, [когда он преследуем был сыном своим Авессаломом].
\rsbpar\vs Psa 142:1 Господи! услышь молитву мою, внемли молению моему по истине Твоей; услышь меня по правде Твоей
\vs Psa 142:2 и не входи в суд с рабом Твоим, потому что не оправдается пред Тобой ни один из живущих.
\vs Psa 142:3 Враг преследует душу мою, втоптал в землю жизнь мою, принудил меня жить во тьме, как давно умерших,~---
\vs Psa 142:4 и уныл во мне дух мой, онемело во мне сердце мое.
\vs Psa 142:5 Вспоминаю дни древние, размышляю о всех делах Твоих, рассуждаю о делах рук Твоих.
\vs Psa 142:6 Простираю к Тебе руки мои; душа моя~--- к Тебе, как жаждущая земля.
\vs Psa 142:7 Скоро услышь меня, Господи: дух мой изнемогает; не скрывай лица Твоего от меня, чтобы я не уподобился нисходящим в могилу.
\vs Psa 142:8 Даруй мне рано услышать милость Твою, ибо я на Тебя уповаю. Укажи мне, [Господи,] путь, по которому мне идти, ибо к Тебе возношу я душу мою.
\vs Psa 142:9 Избавь меня, Господи, от врагов моих; к Тебе прибегаю.
\vs Psa 142:10 Научи меня исполнять волю Твою, потому что Ты Бог мой; Дух Твой благий да ведет меня в землю правды.
\vs Psa 142:11 Ради имени Твоего, Господи, оживи меня; ради правды Твоей выведи из напасти душу мою.
\vs Psa 142:12 И по милости Твоей истреби врагов моих и погуби всех, угнетающих душу мою, ибо я Твой раб.
\vs Psa 143:0 Давида. [Против Голиафа.]
\rsbpar\vs Psa 143:1 Благословен Господь, твердыня моя, научающий руки мои битве и персты мои брани,
\vs Psa 143:2 милость моя и ограждение мое, прибежище мое и Избавитель мой, щит мой,~--- и я на Него уповаю; Он подчиняет мне народ мой.
\vs Psa 143:3 Господи! что есть человек, что Ты знаешь о нем, и сын человеческий, что обращаешь на него внимание?
\vs Psa 143:4 Человек подобен дуновению; дни его~--- как уклоняющаяся тень.
\vs Psa 143:5 Господи! Приклони небеса Твои и сойди; коснись гор, и воздымятся;
\vs Psa 143:6 блесни молниею и рассей их; пусти стрелы Твои и расстрой их;
\vs Psa 143:7 простри с высоты руку Твою, избавь меня и спаси меня от вод многих, от руки сынов иноплеменных,
\vs Psa 143:8 которых уста говорят суетное и которых десница~--- десница лжи.
\vs Psa 143:9 Боже! новую песнь воспою Тебе, на десятиструнной псалтири воспою Тебе,
\vs Psa 143:10 дарующему спасение царям и избавляющему Давида, раба Твоего, от лютого меча.
\vs Psa 143:11 Избавь меня и спаси меня от руки сынов иноплеменных, которых уста говорят суетное и которых десница~--- десница лжи.
\vs Psa 143:12 Да будут сыновья наши, как разросшиеся растения в их молодости; дочери наши~--- как искусно изваянные столпы в чертогах.
\vs Psa 143:13 Да будут житницы наши полны, обильны всяким хлебом; да плодятся овцы наши тысячами и тьмами на пажитях наших;
\vs Psa 143:14 \bibemph{да будут} волы наши тучны; да не будет ни расхищения, ни пропажи, ни воплей на улицах наших.
\vs Psa 143:15 Блажен народ, у которого это есть. Блажен народ, у которого Господь есть Бог.
\vs Psa 144:0 Хвала Давида.
\rsbpar\vs Psa 144:1 Буду превозносить Тебя, Боже мой, Царь [мой], и благословлять имя Твое во веки и веки.
\vs Psa 144:2 Всякий день буду благословлять Тебя и восхвалять имя Твое во веки и веки.
\vs Psa 144:3 Велик Господь и достохвален, и величие Его неисследимо.
\vs Psa 144:4 Род роду будет восхвалять дела Твои и возвещать о могуществе Твоем.
\vs Psa 144:5 А я буду размышлять о высокой славе величия Твоего и о дивных делах Твоих.
\vs Psa 144:6 Будут говорить о могуществе страшных дел Твоих, и я буду возвещать о величии Твоем.
\vs Psa 144:7 Будут провозглашать память великой благости Твоей и воспевать правду Твою.
\vs Psa 144:8 Щедр и милостив Господь, долготерпелив и многомилостив.
\vs Psa 144:9 Благ Господь ко всем, и щедроты Его на всех делах Его.
\vs Psa 144:10 Да славят Тебя, Господи, все дела Твои, и да благословляют Тебя святые Твои;
\vs Psa 144:11 да проповедуют славу царства Твоего, и да повествуют о могуществе Твоем,
\vs Psa 144:12 чтобы дать знать сынам человеческим о могуществе Твоем и о славном величии царства Твоего.
\vs Psa 144:13 Царство Твое~--- царство всех веков, и владычество Твое во все роды. [Верен Господь во всех словах Своих и свят во всех делах Своих.]
\vs Psa 144:14 Господь поддерживает всех падающих и восставляет всех низверженных.
\vs Psa 144:15 Очи всех уповают на Тебя, и Ты даешь им пищу их в свое время;
\vs Psa 144:16 открываешь руку Твою и насыщаешь все живущее по благоволению.
\vs Psa 144:17 Праведен Господь во всех путях Своих и благ во всех делах Своих.
\vs Psa 144:18 Близок Господь ко всем призывающим Его, ко всем призывающим Его в истине.
\vs Psa 144:19 Желание боящихся Его Он исполняет, вопль их слышит и спасает их.
\vs Psa 144:20 Хранит Господь всех любящих Его, а всех нечестивых истребит.
\vs Psa 144:21 Уста мои изрекут хвалу Господню, и да благословляет всякая плоть святое имя Его во веки и веки.
\vs Psa 145:0 [Аллилуия. \bibemph{Аггея и Захарии}.]
\rsbpar\vs Psa 145:1 Хвали, душа моя, Господа.
\vs Psa 145:2 Буду восхвалять Господа, доколе жив; буду петь Богу моему, доколе есмь.
\vs Psa 145:3 Не надейтесь на князей, на сына человеческого, в котором нет спасения.
\vs Psa 145:4 Выходит дух его, и он возвращается в землю свою: в тот день исчезают [все] помышления его.
\vs Psa 145:5 Блажен, кому помощник Бог Иаковлев, у кого надежда на Господа Бога его,
\vs Psa 145:6 сотворившего небо и землю, море и все, что в них, вечно хранящего верность,
\vs Psa 145:7 творящего суд обиженным, дающего хлеб алчущим. Господь разрешает узников,
\vs Psa 145:8 Господь отверзает очи слепым, Господь восставляет согбенных, Господь любит праведных.
\vs Psa 145:9 Господь хранит пришельцев, поддерживает сироту и вдову, а путь нечестивых извращает.
\vs Psa 145:10 Господь будет царствовать во веки, Бог твой, Сион, в род и род. Аллилуия.
\vs Psa 146:0 [Аллилуия.]
\rsbpar\vs Psa 146:1 Хвалите Господа, ибо благо петь Богу нашему, ибо это сладостно,~--- хвала подобающая.
\vs Psa 146:2 Господь созидает Иерусалим, собирает изгнанников Израиля.
\vs Psa 146:3 Он исцеляет сокрушенных сердцем и врачует скорби их;
\vs Psa 146:4 исчисляет количество звезд; всех их называет именами их.
\vs Psa 146:5 Велик Господь наш и велика крепость [Его], и разум Его неизмерим.
\vs Psa 146:6 Смиренных возвышает Господь, а нечестивых унижает до земли.
\vs Psa 146:7 Пойте поочередно славословие Господу; пойте Богу нашему на гуслях.
\vs Psa 146:8 Он покрывает небо облаками, приготовляет для земли дождь, произращает на горах траву [и злак на пользу человеку];
\vs Psa 146:9 дает скоту пищу его и птенцам ворона, взывающим \bibemph{к Нему}.
\vs Psa 146:10 Не на силу коня смотрит Он, не к \bibemph{быстроте} ног человеческих благоволит,~---
\vs Psa 146:11 благоволит Господь к боящимся Его, к уповающим на милость Его.
\vs Psa 147:0 [Аллилуия.]
\rsbpar\vs Psa 147:1 Хвали, Иерусалим, Господа; хвали, Сион, Бога твоего,
\vs Psa 147:2 ибо Он укрепляет вереи ворот твоих, благословляет сынов твоих среди тебя;
\vs Psa 147:3 утверждает в пределах твоих мир; туком пшеницы насыщает тебя;
\vs Psa 147:4 посылает слово Свое на землю; быстро течет слово Его;
\vs Psa 147:5 дает снег, как в\acc{о}лну; сыплет иней, как пепел;
\vs Psa 147:6 бросает град Свой кусками; перед морозом Его кто устоит?
\vs Psa 147:7 Пошлет слово Свое, и все растает; подует ветром Своим, и потекут воды.
\vs Psa 147:8 Он возвестил слово Свое Иакову, уставы Свои и суды Свои Израилю.
\vs Psa 147:9 Не сделал Он того никакому \bibemph{другому} народу, и судов Его они не знают. Аллилуия.
\vs Psa 148:0 [Аллилуия.]
\rsbpar\vs Psa 148:1 Хвалите Господа с небес, хвалите Его в вышних.
\vs Psa 148:2 Хвалите Его, все Ангелы Его, хвалите Его, все воинства Его.
\vs Psa 148:3 Хвалите Его, солнце и луна, хвалите Его, все звезды света.
\vs Psa 148:4 Хвалите Его, небеса небес и воды, которые превыше небес.
\vs Psa 148:5 Да хвалят имя Господа, ибо Он [сказал, и они сделались,] повелел, и сотворились;
\vs Psa 148:6 поставил их на веки и веки; дал устав, который не прейдет.
\vs Psa 148:7 Хвалите Господа от земли, великие рыбы и все бездны,
\vs Psa 148:8 огонь и град, снег и туман, бурный ветер, исполняющий слово Его,
\vs Psa 148:9 горы и все холмы, дерева плодоносные и все кедры,
\vs Psa 148:10 звери и всякий скот, пресмыкающиеся и птицы крылатые,
\vs Psa 148:11 цари земные и все народы, князья и все судьи земные,
\vs Psa 148:12 юноши и девицы, старцы и отроки
\vs Psa 148:13 да хвалят имя Господа, ибо имя Его единого превознесенно, слава Его на земле и на небесах.
\vs Psa 148:14 Он возвысил рог народа Своего, славу всех святых Своих, сынов Израилевых, народа, близкого к Нему. Аллилуия.
\vs Psa 149:0 [Аллилуия.]
\rsbpar\vs Psa 149:1 Пойте Господу песнь новую; хвала Ему в собрании святых.
\vs Psa 149:2 Да веселится Израиль о Создателе своем; сыны Сиона да радуются о Царе своем.
\vs Psa 149:3 да хвалят имя Его с ликами, на тимпане и гуслях да поют Ему,
\vs Psa 149:4 ибо благоволит Господь к народу Своему, прославляет смиренных спасением.
\vs Psa 149:5 Да торжествуют святые во славе, да радуются на ложах своих.
\vs Psa 149:6 Да будут славословия Богу в устах их, и меч обоюдоострый в руке их,
\vs Psa 149:7 для того, чтобы совершать мщение над народами, наказание над племенами,
\vs Psa 149:8 заключать царей их в узы и вельмож их в оковы железные,
\vs Psa 149:9 производить над ними суд писанный. Честь сия~--- всем святым Его. Аллилуия.
\vs Psa 150:0 [Аллилуия.]
\rsbpar\vs Psa 150:1 Хвалите Бога во святыне Его, хвалите Его на тверди силы Его.
\vs Psa 150:2 Хвалите Его по могуществу Его, хвалите Его по множеству величия Его.
\vs Psa 150:3 Хвалите Его со звуком трубным, хвалите Его на псалтири и гуслях.
\vs Psa 150:4 Хвалите Его с тимпаном и ликами, хвалите Его на струнах и органе.
\vs Psa 150:5 Хвалите Его на звучных кимвалах, хвалите Его на кимвалах громогласных.
\vs Psa 150:6 Все дышащее да хвалит Господа! Аллилуия.
\vs Psa 151:0 [\bibemph{Псалом Давида на единоборство с Голиафом}\fns{У Евреев этого псалма нет: он переведен с греческого.}.
\rsbpar\vs Psa 151:1 Я был меньший между братьями моими и юнейший в доме отца моего; пас овец отца моего.
\vs Psa 151:2 Руки мои сделали орган, персты мои настраивали псалтирь.
\vs Psa 151:3 И кто возвестил бы Господу моему?~--- Сам Господь, Сам услышал меня.
\vs Psa 151:4 Он послал вестника Своего и взял меня от овец отца моего, и помазал меня елеем помазания Своего.
\vs Psa 151:5 Братья мои прекрасны и велики, но Господь не благоволил избрать из них.
\vs Psa 151:6 Я вышел навстречу иноплеменнику, и он проклял меня идолами своими.
\vs Psa 151:7 Но я, исторгнув у него меч, обезглавил его и избавил сынов Израилевых от поношения.]

\bibbookdescr{Pro}{
  inline={\LARGE Книга\\\Huge Притчей Соломоновых},
  toc={Притчи},
  bookmark={Притчи},
  header={Притчи},
  %headerleft={},
  %headerright={},
  abbr={Притч}
}
\vs Pro 1:1 Притчи Соломона, сына Давидова, царя Израильского,
\vs Pro 1:2 чтобы познать мудрость и наставление, понять изречения разума;
\vs Pro 1:3 усвоить правила благоразумия, правосудия, суда и правоты;
\vs Pro 1:4 простым дать смышленость, юноше~--- знание и рассудительность;
\vs Pro 1:5 послушает мудрый~--- и умножит познания, и разумный найдет мудрые советы;
\vs Pro 1:6 чтобы разуметь притчу и замысловатую речь, слова мудрецов и загадки их.
\rsbpar\vs Pro 1:7 Начало мудрости~--- страх Господень; [доброе разумение у всех, водящихся им; а благоговение к Богу~--- начало разумения;] глупцы только презирают мудрость и наставление.
\vs Pro 1:8 Слушай, сын мой, наставление отца твоего и не отвергай завета матери твоей,
\vs Pro 1:9 потому что это~--- прекрасный венок для головы твоей и украшение для шеи твоей.
\vs Pro 1:10 Сын мой! если будут склонять тебя грешники, не соглашайся;
\vs Pro 1:11 если будут говорить: <<иди с нами, сделаем засаду для убийства, подстережем непорочного без вины,
\vs Pro 1:12 живых проглотим их, как преисподняя, и~--- целых, как нисходящих в могилу;
\vs Pro 1:13 наберем всякого драгоценного имущества, наполним домы наши добычею;
\vs Pro 1:14 жребий твой ты будешь бросать вместе с нами, склад один будет у всех нас>>,~---
\vs Pro 1:15 сын мой! не ходи в путь с ними, удержи ногу твою от стези их,
\vs Pro 1:16 потому что ноги их бегут ко злу и спешат на пролитие крови.
\vs Pro 1:17 В глазах всех птиц напрасно расставляется сеть,
\vs Pro 1:18 а делают засаду для их крови и подстерегают их души.
\vs Pro 1:19 Таковы пути всякого, кто алчет чужого добра: оно отнимает жизнь у завладевшего им.
\rsbpar\vs Pro 1:20 Премудрость возглашает на улице, на площадях возвышает голос свой,
\vs Pro 1:21 в главных местах собраний проповедует, при входах в городские ворота говорит речь свою:
\vs Pro 1:22 <<доколе, невежды, будете любить невежество? \bibemph{доколе} буйные будут услаждаться буйством? доколе глупцы будут ненавидеть знание?
\vs Pro 1:23 Обратитесь к моему обличению: вот, я изолью на вас дух мой, возвещу вам слова мои.
\vs Pro 1:24 Я звала, и вы не послушались; простирала руку мою, и не было внимающего;
\vs Pro 1:25 и вы отвергли все мои советы, и обличений моих не приняли.
\vs Pro 1:26 За то и я посмеюсь вашей погибели; порадуюсь, когда придет на вас ужас;
\vs Pro 1:27 когда придет на вас ужас, как буря, и беда, как вихрь, принесется на вас; когда постигнет вас скорбь и теснота.
\vs Pro 1:28 Тогда будут звать меня, и я не услышу; с утра будут искать меня, и не найдут меня.
\vs Pro 1:29 За то, что они возненавидели знание и не избрали \bibemph{для себя} страха Господня,
\vs Pro 1:30 не приняли совета моего, презрели все обличения мои;
\vs Pro 1:31 за то и будут они вкушать от плодов путей своих и насыщаться от помыслов их.
\vs Pro 1:32 Потому что упорство невежд убьет их, и беспечность глупцов погубит их,
\vs Pro 1:33 а слушающий меня будет жить безопасно и спокойно, не страшась зла>>.
\vs Pro 2:1 Сын мой! если ты примешь слова мои и сохранишь при себе заповеди мои,
\vs Pro 2:2 так что ухо твое сделаешь внимательным к мудрости и наклонишь сердце твое к размышлению;
\vs Pro 2:3 если будешь призывать знание и взывать к разуму;
\vs Pro 2:4 если будешь искать его, как серебра, и отыскивать его, как сокровище,
\vs Pro 2:5 то уразумеешь страх Господень и найдешь познание о Боге.
\vs Pro 2:6 Ибо Господь дает мудрость; из уст Его~--- знание и разум;
\vs Pro 2:7 Он сохраняет для праведных спасение; Он~--- щит для ходящих непорочно;
\vs Pro 2:8 Он охраняет пути правды и оберегает стезю святых Своих.
\vs Pro 2:9 Тогда ты уразумеешь правду и правосудие и прямоту, всякую добрую стезю.
\vs Pro 2:10 Когда мудрость войдет в сердце твое, и знание будет приятно душе твоей,
\vs Pro 2:11 тогда рассудительность будет оберегать тебя, разум будет охранять тебя,
\vs Pro 2:12 дабы спасти тебя от пути злого, от человека, говорящего ложь,
\vs Pro 2:13 от тех, которые оставляют стези прямые, чтобы ходить путями тьмы;
\vs Pro 2:14 от тех, которые радуются, делая зло, восхищаются злым развратом,
\vs Pro 2:15 которых пути кривы, и которые блуждают на стезях своих;
\vs Pro 2:16 дабы спасти тебя от жены другого, от чужой, которая умягчает речи свои,
\vs Pro 2:17 которая оставила руководителя юности своей и забыла завет Бога своего.
\vs Pro 2:18 Дом ее ведет к смерти, и стези ее~--- к мертвецам;
\vs Pro 2:19 никто из вошедших к ней не возвращается и не вступает на путь жизни.
\vs Pro 2:20 Посему ходи путем добрых и держись стезей праведников,
\vs Pro 2:21 потому что праведные будут жить на земле, и непорочные пребудут на ней;
\vs Pro 2:22 а беззаконные будут истреблены с земли, и вероломные искоренены из нее.
\vs Pro 3:1 Сын мой! наставления моего не забывай, и заповеди мои да хранит сердце твое;
\vs Pro 3:2 ибо долготы дней, лет жизни и мира они приложат тебе.
\vs Pro 3:3 Милость и истина да не оставляют тебя: обвяжи ими шею твою, напиши их на скрижали сердца твоего,
\vs Pro 3:4 и обретешь милость и благоволение в очах Бога и людей.
\vs Pro 3:5 Надейся на Господа всем сердцем твоим, и не полагайся на разум твой.
\vs Pro 3:6 Во всех путях твоих познавай Его, и Он направит стези твои.
\vs Pro 3:7 Не будь мудрецом в глазах твоих; бойся Господа и удаляйся от зла:
\vs Pro 3:8 это будет здравием для тела твоего и питанием для костей твоих.
\vs Pro 3:9 Чти Господа от имения твоего и от начатков всех прибытков твоих,
\vs Pro 3:10 и наполнятся житницы твои до избытка, и точила твои будут переливаться новым вином.
\rsbpar\vs Pro 3:11 Наказания Господня, сын мой, не отвергай, и не тяготись обличением Его;
\vs Pro 3:12 ибо кого любит Господь, того наказывает и благоволит к тому, как отец к сыну своему.
\vs Pro 3:13 Блажен человек, который снискал мудрость, и человек, который приобрел разум,~---
\vs Pro 3:14 потому что приобретение ее лучше приобретения серебра, и прибыли от нее больше, нежели от золота:
\vs Pro 3:15 она дороже драгоценных камней; [никакое зло не может противиться ей; она хорошо известна всем, приближающимся к ней,] и ничто из желаемого тобою не сравнится с нею.
\vs Pro 3:16 Долгоденствие~--- в правой руке ее, а в левой у нее~--- богатство и слава; [из уст ее выходит правда; закон и милость она на языке носит;]
\vs Pro 3:17 пути ее~--- пути приятные, и все стези ее~--- мирные.
\vs Pro 3:18 Она~--- древо жизни для тех, которые приобретают ее,~--- и блаженны, которые сохраняют ее!
\rsbpar\vs Pro 3:19 Господь премудростью основал землю, небеса утвердил разумом;
\vs Pro 3:20 Его премудростью разверзлись бездны, и облака кропят росою.
\vs Pro 3:21 Сын мой! не упускай их из глаз твоих; храни здравомыслие и рассудительность,
\vs Pro 3:22 и они будут жизнью для души твоей и украшением для шеи твоей.
\vs Pro 3:23 Тогда безопасно пойдешь по пути твоему, и нога твоя не споткнется.
\vs Pro 3:24 Когда ляжешь спать,~--- не будешь бояться; и когда уснешь,~--- сон твой приятен будет.
\vs Pro 3:25 Не убоишься внезапного страха и пагубы от нечестивых, когда она придет;
\vs Pro 3:26 потому что Господь будет упованием твоим и сохранит ногу твою от уловления.
\rsbpar\vs Pro 3:27 Не отказывай в благодеянии нуждающемуся, когда рука твоя в силе сделать его.
\vs Pro 3:28 Не говори другу твоему: <<пойди и приди опять, и завтра я дам>>, когда ты имеешь при себе. [Ибо ты не знаешь, чт\acc{о} родит грядущий день.]
\vs Pro 3:29 Не замышляй против ближнего твоего зла, когда он без опасения живет с тобою.
\vs Pro 3:30 Не ссорься с человеком без причины, когда он не сделал зла тебе.
\vs Pro 3:31 Не соревнуй человеку, поступающему насильственно, и не избирай ни одного из путей его;
\vs Pro 3:32 потому что мерзость пред Господом развратный, а с праведными у Него общение.
\vs Pro 3:33 Проклятие Господне на доме нечестивого, а жилище благочестивых Он благословляет.
\vs Pro 3:34 Если над кощунниками Он посмевается, то смиренным дает благодать.
\vs Pro 3:35 Мудрые наследуют славу, а глупые~--- бесславие.
\vs Pro 4:1 Слушайте, дети, наставление отца, и внимайте, чтобы научиться разуму,
\vs Pro 4:2 потому что я преподал вам доброе учение. Не оставляйте заповеди моей.
\vs Pro 4:3 Ибо и я был сын у отца моего, нежно любимый и единственный у матери моей,
\vs Pro 4:4 и он учил меня и говорил мне: да удержит сердце твое слова мои; храни заповеди мои, и живи.
\vs Pro 4:5 Приобретай мудрость, приобретай разум: не забывай этого и не уклоняйся от слов уст моих.
\vs Pro 4:6 Не оставляй ее, и она будет охранять тебя; люби ее, и она будет оберегать тебя.
\vs Pro 4:7 Главное~--- мудрость: приобретай мудрость, и всем имением твоим приобретай разум.
\vs Pro 4:8 Высоко цени ее, и она возвысит тебя; она прославит тебя, если ты прилепишься к ней;
\vs Pro 4:9 возложит на голову твою прекрасный венок, доставит тебе великолепный венец.
\rsbpar\vs Pro 4:10 Слушай, сын мой, и прими слова мои,~--- и умножатся тебе лета жизни.
\vs Pro 4:11 Я указываю тебе путь мудрости, веду тебя по стезям прямым.
\vs Pro 4:12 Когда пойдешь, не будет стеснен ход твой, и когда побежишь, не споткнешься.
\vs Pro 4:13 Крепко держись наставления, не оставляй, храни его, потому что оно~--- жизнь твоя.
\vs Pro 4:14 Не вступай на стезю нечестивых и не ходи по пути злых;
\vs Pro 4:15 оставь его, не ходи по нему, уклонись от него и пройди мимо;
\vs Pro 4:16 потому что они не заснут, если не сделают зла; пропадает сон у них, если они не доведут кого до падения;
\vs Pro 4:17 ибо они едят хлеб беззакония и пьют вино хищения.
\vs Pro 4:18 Стезя праведных~--- как светило лучезарное, которое более и более светлеет до полного дня.
\vs Pro 4:19 Путь же беззаконных~--- как тьма; они не знают, обо что споткнутся.
\rsbpar\vs Pro 4:20 Сын мой! словам моим внимай, и к речам моим приклони ухо твое;
\vs Pro 4:21 да не отходят они от глаз твоих; храни их внутри сердца твоего:
\vs Pro 4:22 потому что они жизнь для того, кто нашел их, и здравие для всего тела его.
\vs Pro 4:23 Больше всего хранимого храни сердце твое, потому что из него источники жизни.
\vs Pro 4:24 Отвергни от себя лживость уст, и лукавство языка удали от себя.
\vs Pro 4:25 Глаза твои пусть прямо смотрят, и ресницы твои да направлены будут прямо пред тобою.
\vs Pro 4:26 Обдумай стезю для ноги твоей, и все пути твои да будут тверды.
\vs Pro 4:27 Не уклоняйся ни направо, ни налево; удали ногу твою от зла,
\vs Pro 4:28 [потому что пути правые наблюдает Господь, а левые~--- испорчены.
\vs Pro 4:29 Он же прямыми сделает пути твои, и шествия твои в мире устроит.]
\vs Pro 5:1 Сын мой! внимай мудрости моей, и приклони ухо твое к разуму моему,
\vs Pro 5:2 чтобы соблюсти рассудительность, и чтобы уста твои сохранили знание. [Не внимай льстивой женщине;]
\vs Pro 5:3 ибо мед источают уста чужой жены, и мягче елея речь ее;
\vs Pro 5:4 но последствия от нее горьки, как полынь, остры, как меч обоюдоострый;
\vs Pro 5:5 ноги ее нисходят к смерти, стопы ее достигают преисподней.
\vs Pro 5:6 Если бы ты захотел постигнуть стезю жизни ее, то пути ее непостоянны, и ты не узнаешь их.
\vs Pro 5:7 Итак, дети, слушайте меня и не отступайте от слов уст моих.
\vs Pro 5:8 Держи дальше от нее путь твой и не подходи близко к дверям дома ее,
\vs Pro 5:9 чтобы здоровья твоего не отдать другим и лет твоих мучителю;
\vs Pro 5:10 чтобы не насыщались силою твоею чужие, и труды твои не были для чужого дома.
\vs Pro 5:11 И ты будешь стонать после, когда плоть твоя и тело твое будут истощены,~---
\vs Pro 5:12 и скажешь: <<зачем я ненавидел наставление, и сердце мое пренебрегало обличением,
\vs Pro 5:13 и я не слушал голоса учителей моих, не приклонял уха моего к наставникам моим:
\vs Pro 5:14 едва не впал я во всякое зло среди собрания и общества!>>
\rsbpar\vs Pro 5:15 Пей воду из твоего водоема и текущую из твоего колодезя.
\vs Pro 5:16 Пусть [не] разливаются источники твои по улице, потоки вод~--- по площадям;
\vs Pro 5:17 пусть они будут принадлежать тебе одному, а не чужим с тобою.
\vs Pro 5:18 Источник твой да будет благословен; и утешайся женою юности твоей,
\vs Pro 5:19 любезною ланью и прекрасною серною: груди ее да упоявают тебя во всякое время, любовью ее услаждайся постоянно.
\vs Pro 5:20 И для чего тебе, сын мой, увлекаться постороннею и обнимать груди чужой?
\vs Pro 5:21 Ибо пред очами Господа пути человека, и Он измеряет все стези его.
\vs Pro 5:22 Беззаконного уловляют собственные беззакония его, и в узах греха своего он содержится:
\vs Pro 5:23 он умирает без наставления, и от множества безумия своего теряется.
\vs Pro 6:1 Сын мой! если ты поручился за ближнего твоего и дал руку твою за другого,~---
\vs Pro 6:2 ты опутал себя словами уст твоих, пойман словами уст твоих.
\vs Pro 6:3 Сделай же, сын мой, вот что, и избавь себя, так как ты попался в руки ближнего твоего: пойди, пади к ногам и умоляй ближнего твоего;
\vs Pro 6:4 не давай сна глазам твоим и дремания веждам твоим;
\vs Pro 6:5 спасайся, как серна из руки и как птица из руки птицелова.
\rsbpar\vs Pro 6:6 Пойди к муравью, ленивец, посмотри на действия его, и будь мудрым.
\vs Pro 6:7 Нет у него ни начальника, ни приставника, ни повелителя;
\vs Pro 6:8 но он заготовляет летом хлеб свой, собирает во время жатвы пищу свою. [Или пойди к пчеле и познай, как она трудолюбива, какую почтенную работу она производит; ее труды употребляют во здравие и цари и простолюдины; любима же она всеми и славна; хотя силою она слаба, но мудростью почтена.]
\vs Pro 6:9 Доколе ты, ленивец, будешь спать? когда ты встанешь от сна твоего?
\vs Pro 6:10 Немного поспишь, немного подремлешь, немного, сложив руки, полежишь:
\vs Pro 6:11 и придет, как прохожий, бедность твоя, и нужда твоя, как разбойник. [Если же будешь не ленив, то, как источник, придет жатва твоя; скудость же далеко убежит от тебя.]
\rsbpar\vs Pro 6:12 Человек лукавый, человек нечестивый ходит со лживыми устами,
\vs Pro 6:13 мигает глазами своими, говорит ногами своими, дает знаки пальцами своими;
\vs Pro 6:14 коварство в сердце его: он умышляет зло во всякое время, сеет раздоры.
\vs Pro 6:15 Зато внезапно придет погибель его, вдруг будет разбит~--- без исцеления.
\vs Pro 6:16 Вот шесть, чт\acc{о} ненавидит Господь, даже семь, чт\acc{о} мерзость душе Его:
\vs Pro 6:17 глаза гордые, язык лживый и руки, проливающие кровь невинную,
\vs Pro 6:18 сердце, кующее злые замыслы, ноги, быстро бегущие к злодейству,
\vs Pro 6:19 лжесвидетель, наговаривающий ложь и сеющий раздор между братьями.
\rsbpar\vs Pro 6:20 Сын мой! храни заповедь отца твоего и не отвергай наставления матери твоей;
\vs Pro 6:21 навяжи их навсегда на сердце твое, обвяжи ими шею твою.
\vs Pro 6:22 Когда ты пойдешь, они будут руководить тебя; когда ляжешь спать, будут охранять тебя; когда пробудишься, будут беседовать с тобою:
\vs Pro 6:23 ибо заповедь есть светильник, и наставление~--- свет, и назидательные поучения~--- путь к жизни,
\vs Pro 6:24 чтобы остерегать тебя от негодной женщины, от льстивого языка чужой.
\vs Pro 6:25 Не пожелай красоты ее в сердце твоем, [да не уловлен будешь очами твоими,] и да не увлечет она тебя ресницами своими;
\vs Pro 6:26 потому что из-за жены блудной \bibemph{обнищевают} до куска хлеба, а замужняя жена уловляет дорогую душу.
\vs Pro 6:27 Может ли кто взять себе огонь в пазуху, чтобы не прогорело платье его?
\vs Pro 6:28 Может ли кто ходить по горящим угольям, чтобы не обжечь ног своих?
\vs Pro 6:29 То же бывает и с тем, кто входит к жене ближнего своего: кто прикоснется к ней, не останется без вины.
\vs Pro 6:30 Не спускают вору, если он крадет, чтобы насытить душу свою, когда он голоден;
\vs Pro 6:31 но, будучи пойман, он заплатит всемеро, отдаст все имущество дома своего.
\vs Pro 6:32 Кто же прелюбодействует с женщиною, у того нет ума; тот губит душу свою, кто делает это:
\vs Pro 6:33 побои и позор найдет он, и бесчестие его не изгладится,
\vs Pro 6:34 потому что ревность~--- ярость мужа, и не пощадит он в день мщения,
\vs Pro 6:35 не примет никакого выкупа и не удовольствуется, сколько бы ты ни умножал даров.
\vs Pro 7:1 Сын мой! храни слова мои и заповеди мои сокрой у себя. [Сын мой! чти Господа,~--- и укрепишься, и кроме Его не бойся никого.]
\vs Pro 7:2 Храни заповеди мои и живи, и учение мое, как зрачок глаз твоих.
\vs Pro 7:3 Навяжи их на персты твои, напиши их на скрижали сердца твоего.
\vs Pro 7:4 Скажи мудрости: <<ты сестра моя!>> и разум назови родным твоим,
\vs Pro 7:5 чтобы они охраняли тебя от жены другого, от чужой, которая умягчает слова свои.
\vs Pro 7:6 Вот, однажды смотрел я в окно дома моего, сквозь решетку мою,
\vs Pro 7:7 и увидел среди неопытных, заметил между молодыми людьми неразумного юношу,
\vs Pro 7:8 переходившего площадь близ угла ее и шедшего по дороге к дому ее,
\vs Pro 7:9 в сумерки в вечер дня, в ночной темноте и во мраке.
\vs Pro 7:10 И вот~--- навстречу к нему женщина, в наряде блудницы, с коварным сердцем,
\vs Pro 7:11 шумливая и необузданная; ноги ее не живут в доме ее:
\vs Pro 7:12 то на улице, то на площадях, и у каждого угла строит она ковы.
\vs Pro 7:13 Она схватила его, целовала его, и с бесстыдным лицом говорила ему:
\vs Pro 7:14 <<мирная жертва у меня: сегодня я совершила обеты мои;
\vs Pro 7:15 поэтому и вышла навстречу тебе, чтобы отыскать тебя, и~--- нашла тебя;
\vs Pro 7:16 коврами я убрала постель мою, разноцветными тканями Египетскими;
\vs Pro 7:17 спальню мою надушила смирною, алоем и корицею;
\vs Pro 7:18 зайди, будем упиваться нежностями до утра, насладимся любовью,
\vs Pro 7:19 потому что мужа нет дома: он отправился в дальнюю дорогу;
\vs Pro 7:20 кошелек серебра взял с собою; придет домой ко дню полнолуния>>.
\vs Pro 7:21 Множеством ласковых слов она увлекла его, мягкостью уст своих овладела им.
\vs Pro 7:22 Тотчас он пошел за нею, как вол идет на убой, [и как пес~--- на цепь,] и как олень~--- на выстрел,
\vs Pro 7:23 доколе стрела не пронзит печени его; как птичка кидается в силки, и не знает, что они~--- на погибель ее.
\vs Pro 7:24 Итак, дети, слушайте меня и внимайте словам уст моих.
\vs Pro 7:25 Да не уклоняется сердце твое на пути ее, не блуждай по стезям ее,
\vs Pro 7:26 потому что многих повергла она ранеными, и много сильных убиты ею:
\vs Pro 7:27 дом ее~--- пути в преисподнюю, нисходящие во внутренние жилища смерти.
\vs Pro 8:1 Не премудрость ли взывает? и не разум ли возвышает голос свой?
\vs Pro 8:2 Она становится на возвышенных местах, при дороге, на распутиях;
\vs Pro 8:3 она взывает у ворот при входе в город, при входе в двери:
\vs Pro 8:4 <<к вам, люди, взываю я, и к сынам человеческим голос мой!
\vs Pro 8:5 Научитесь, неразумные, благоразумию, и глупые~--- разуму.
\vs Pro 8:6 Слушайте, потому что я буду говорить важное, и изречение уст моих~--- правда;
\vs Pro 8:7 ибо истину произнесет язык мой, и нечестие~--- мерзость для уст моих;
\vs Pro 8:8 все слова уст моих справедливы; нет в них коварства и лукавства;
\vs Pro 8:9 все они ясны для разумного и справедливы для приобретших знание.
\vs Pro 8:10 Примите учение мое, а не серебро; лучше знание, нежели отборное золото;
\vs Pro 8:11 потому что мудрость лучше жемчуга, и ничто из желаемого не сравнится с нею.
\vs Pro 8:12 Я, премудрость, обитаю с разумом и ищу рассудительного знания.
\vs Pro 8:13 Страх Господень~--- ненавидеть зло; гордость и высокомерие и злой путь и коварные уста я ненавижу.
\vs Pro 8:14 У меня совет и правда; я разум, у меня сила.
\vs Pro 8:15 Мною цари царствуют и повелители узаконяют правду;
\vs Pro 8:16 мною начальствуют начальники и вельможи и все судьи земли.
\vs Pro 8:17 Любящих меня я люблю, и ищущие меня найдут меня;
\vs Pro 8:18 богатство и слава у меня, сокровище непогибающее и правда;
\vs Pro 8:19 плоды мои лучше золота, и золота самого чистого, и пользы от меня больше, нежели от отборного серебра.
\vs Pro 8:20 Я хожу по пути правды, по стезям правосудия,
\vs Pro 8:21 чтобы доставить любящим меня существенное благо, и сокровищницы их я наполняю. [Когда я возвещу то, что бывает ежедневно, то не забуду исчислить то, что от века.]
\rsbpar\vs Pro 8:22 Господь имел меня началом пути Своего, прежде созданий Своих, искони;
\vs Pro 8:23 от века я помазана, от начала, прежде бытия земли.
\vs Pro 8:24 Я родилась, когда еще не существовали бездны, когда еще не было источников, обильных водою.
\vs Pro 8:25 Я родилась прежде, нежели водружены были горы, прежде холмов,
\vs Pro 8:26 когда еще Он не сотворил ни земли, ни полей, ни начальных пылинок вселенной.
\vs Pro 8:27 Когда Он уготовлял небеса, \bibemph{я была} там. Когда Он проводил круговую черту по лицу бездны,
\vs Pro 8:28 когда утверждал вверху облака, когда укреплял источники бездны,
\vs Pro 8:29 когда давал морю устав, чтобы воды не переступали пределов его, когда полагал основания земли:
\vs Pro 8:30 тогда я была при Нем художницею, и была радостью всякий день, веселясь пред лицем Его во все время,
\vs Pro 8:31 веселясь на земном кругу Его, и радость моя \bibemph{была} с сынами человеческими.
\rsbpar\vs Pro 8:32 Итак, дети, послушайте меня; и блаженны те, которые хранят пути мои!
\vs Pro 8:33 Послушайте наставления и будьте мудры, и не отступайте \bibemph{от него}.
\vs Pro 8:34 Блажен человек, который слушает меня, бодрствуя каждый день у ворот моих и стоя на страже у дверей моих!
\vs Pro 8:35 потому что, кто нашел меня, тот нашел жизнь, и получит благодать от Господа;
\vs Pro 8:36 а согрешающий против меня наносит вред душе своей: все ненавидящие меня любят смерть>>.
\vs Pro 9:1 Премудрость построила себе дом, вытесала семь столбов его,
\vs Pro 9:2 заколола жертву, растворила вино свое и приготовила у себя трапезу;
\vs Pro 9:3 послала слуг своих провозгласить с возвышенностей городских:
\vs Pro 9:4 <<кто неразумен, обратись сюда!>> И скудоумному она сказала:
\vs Pro 9:5 <<идите, ешьте хлеб мой и пейте вино, мною растворенное;
\vs Pro 9:6 оставьте неразумие, и живите, и ходите путем разума>>.
\vs Pro 9:7 Поучающий кощунника наживет себе бесславие, и обличающий нечестивого~--- пятно себе.
\vs Pro 9:8 Не обличай кощунника, чтобы он не возненавидел тебя; обличай мудрого, и он возлюбит тебя;
\vs Pro 9:9 дай \bibemph{наставление} мудрому, и он будет еще мудрее; научи правдивого, и он приумножит знание.
\vs Pro 9:10 Начало мудрости~--- страх Господень, и познание Святаго~--- разум;
\vs Pro 9:11 потому что чрез меня умножатся дни твои, и прибавится тебе лет жизни.
\vs Pro 9:12 [Сын мой!] если ты мудр, то мудр для себя [и для ближних твоих]; и если буен, то один потерпишь. [Кто утверждается на лжи, тот пасет ветры, тот гоняется за птицами летающими: ибо он оставил пути своего виноградника и блуждает по тропинкам поля своего; проходит чрез безводную пустыню и землю, обреченную на жажду; собирает руками бесплодие.]
\rsbpar\vs Pro 9:13 Женщина безрассудная, шумливая, глупая и ничего не знающая
\vs Pro 9:14 садится у дверей дома своего на стуле, на возвышенных местах города,
\vs Pro 9:15 чтобы звать проходящих дорогою, идущих прямо своими путями:
\vs Pro 9:16 <<кто глуп, обратись сюда!>> и скудоумному сказала она:
\vs Pro 9:17 <<воды краденые сладки, и утаенный хлеб приятен>>.
\vs Pro 9:18 И он не знает, что мертвецы там, и что в глубине преисподней зазванные ею. [Но ты отскочи, не медли на месте, не останавливай взгляда твоего на ней; ибо таким образом ты пройдешь воду чужую. От воды чужой удаляйся, и из источника чужого не пей, чтобы пожить многое время, и чтобы прибавились тебе лета жизни.]
\vs Pro 10:1 Притчи Соломона. Сын мудрый радует отца, а сын глупый~--- огорчение для его матери.
\vs Pro 10:2 Не доставляют пользы сокровища неправедные, правда же избавляет от смерти.
\vs Pro 10:3 Не допустит Господь терпеть голод душе праведного, стяжание же нечестивых исторгнет.
\vs Pro 10:4 Ленивая рука делает бедным, а рука прилежных обогащает.
\vs Pro 10:5 Собирающий во время лета~--- сын разумный, спящий же во время жатвы~--- сын беспутный.
\vs Pro 10:6 Благословения~--- на голове праведника, уста же беззаконных заградит насилие.
\vs Pro 10:7 Память праведника пребудет благословенна, а имя нечестивых омерзеет.
\vs Pro 10:8 Мудрый сердцем принимает заповеди, а глупый устами преткнется.
\vs Pro 10:9 Кто ходит в непорочности, тот ходит безопасно; а кто превращает пути свои, тот будет наказан.
\vs Pro 10:10 Кто мигает глазами, тот причиняет досаду, а глупый устами преткнется.
\vs Pro 10:11 Уста праведника~--- источник жизни, уста же беззаконных заградит насилие.
\vs Pro 10:12 Ненависть возбуждает раздоры, но любовь покрывает все грехи.
\vs Pro 10:13 В устах разумного находится мудрость, но на теле глупого~--- розга.
\vs Pro 10:14 Мудрые сберегают знание, но уста глупого~--- близкая погибель.
\vs Pro 10:15 Имущество богатого~--- крепкий город его, беда для бедных~--- скудость их.
\vs Pro 10:16 Труды праведного~--- к жизни, успех нечестивого~--- ко греху.
\vs Pro 10:17 Кто хранит наставление, тот на пути к жизни; а отвергающий обличение~--- блуждает.
\vs Pro 10:18 Кто скрывает ненависть, у того уста лживые; и кто разглашает клевету, тот глуп.
\vs Pro 10:19 При многословии не миновать греха, а сдерживающий уста свои~--- разумен.
\vs Pro 10:20 Отборное серебро~--- язык праведного, сердце же нечестивых~--- ничтожество.
\vs Pro 10:21 Уста праведного пасут многих, а глупые умирают от недостатка разума.
\vs Pro 10:22 Благословение Господне~--- оно обогащает и печали с собою не приносит.
\vs Pro 10:23 Для глупого преступное деяние как бы забава, а человеку разумному свойственна мудрость.
\vs Pro 10:24 Чего страшится нечестивый, то и постигнет его, а желание праведников исполнится.
\vs Pro 10:25 Как проносится вихрь, \bibemph{так} нет более нечестивого; а праведник~--- на вечном основании.
\vs Pro 10:26 Что уксус для зубов и дым для глаз, то ленивый для посылающих его.
\vs Pro 10:27 Страх Господень прибавляет дней, лета же нечестивых сократятся.
\vs Pro 10:28 Ожидание праведников~--- радость, а надежда нечестивых погибнет.
\vs Pro 10:29 Путь Господень~--- твердыня для непорочного и страх для делающих беззаконие.
\vs Pro 10:30 Праведник во веки не поколеблется, нечестивые же не поживут на земле.
\vs Pro 10:31 Уста праведника источают мудрость, а язык зловредный отсечется.
\vs Pro 10:32 Уста праведного знают благоприятное, а уста нечестивых~--- развращенное.
\vs Pro 11:1 Неверные весы~--- мерзость пред Господом, но правильный вес угоден Ему.
\vs Pro 11:2 Придет гордость, придет и посрамление; но со смиренными~--- мудрость. [Праведник, умирая, оставляет сожаление; но внезапна и радостна бывает погибель нечестивых.]
\vs Pro 11:3 Непорочность прямодушных будет руководить их, а лукавство коварных погубит их.
\vs Pro 11:4 Не поможет богатство в день гнева, правда же спасет от смерти.
\vs Pro 11:5 Правда непорочного уравнивает путь его, а нечестивый падет от нечестия своего.
\vs Pro 11:6 Правда прямодушных спасет их, а беззаконники будут уловлены беззаконием своим.
\vs Pro 11:7 Со смертью человека нечестивого исчезает надежда, и ожидание беззаконных погибает.
\vs Pro 11:8 Праведник спасается от беды, а вместо него попадает \bibemph{в нее} нечестивый.
\vs Pro 11:9 Устами лицемер губит ближнего своего, но праведники прозорливостью спасаются.
\vs Pro 11:10 При благоденствии праведников веселится город, и при погибели нечестивых \bibemph{бывает} торжество.
\vs Pro 11:11 Благословением праведных возвышается город, а устами нечестивых разрушается.
\vs Pro 11:12 Скудоумный высказывает презрение к ближнему своему; но разумный человек молчит.
\vs Pro 11:13 Кто ходит переносчиком, тот открывает тайну; но верный человек таит дело.
\vs Pro 11:14 При недостатке попечения падает народ, а при многих советниках благоденствует.
\vs Pro 11:15 Зло причиняет себе, кто ручается за постороннего; а кто ненавидит ручательство, тот безопасен.
\vs Pro 11:16 Благонравная жена приобретает славу [мужу, а жена, ненавидящая правду, есть верх бесчестия. Ленивцы бывают скудны], а трудолюбивые приобретают богатство.
\vs Pro 11:17 Человек милосердый благотворит душе своей, а жестокосердый разрушает плоть свою.
\vs Pro 11:18 Нечестивый делает дело ненадежное, а сеющему правду~--- награда верная.
\vs Pro 11:19 Праведность \bibemph{ведет} к жизни, а стремящийся к злу \bibemph{стремится} к смерти своей.
\vs Pro 11:20 Мерзость пред Господом~--- коварные сердцем; но благоугодны Ему непорочные в пути.
\vs Pro 11:21 Можно поручиться, что порочный не останется ненаказанным; семя же праведных спасется.
\vs Pro 11:22 Что золотое кольцо в носу у свиньи, то женщина красивая и~--- безрассудная.
\vs Pro 11:23 Желание праведных \bibemph{есть} одно добро, ожидание нечестивых~--- гнев.
\vs Pro 11:24 Иной сыплет щедро, и \bibemph{ему} еще прибавляется; а другой сверх меры бережлив, и однако же беднеет.
\vs Pro 11:25 Благотворительная душа будет насыщена, и кто напояет \bibemph{других}, тот и сам напоен будет.
\vs Pro 11:26 Кто удерживает у себя хлеб, того клянет народ; а на голове продающего~--- благословение.
\vs Pro 11:27 Кто стремится к добру, тот ищет благоволения; а кто ищет зла, к тому оно и приходит.
\vs Pro 11:28 Надеющийся на богатство свое упадет; а праведники, как лист, будут зеленеть.
\vs Pro 11:29 Расстроивающий дом свой получит в удел ветер, и глупый будет рабом мудрого сердцем.
\vs Pro 11:30 Плод праведника~--- древо жизни, и мудрый привлекает души.
\vs Pro 11:31 Так праведнику воздается на земле, тем паче нечестивому и грешнику.
\vs Pro 12:1 Кто любит наставление, тот любит знание; а кто ненавидит обличение, тот невежда.
\vs Pro 12:2 Добрый приобретает благоволение от Господа; а человека коварного Он осудит.
\vs Pro 12:3 Не утвердит себя человек беззаконием; корень же праведников неподвижен.
\vs Pro 12:4 Добродетельная жена~--- венец для мужа своего; а позорная~--- как гниль в костях его.
\vs Pro 12:5 Помышления праведных~--- правда, а замыслы нечестивых~--- коварство.
\vs Pro 12:6 Речи нечестивых~--- засада для пролития крови, уста же праведных спасают их.
\vs Pro 12:7 Коснись нечестивых несчастие~--- и нет их, а дом праведных стоит.
\vs Pro 12:8 Хвалят человека по мере разума его, а развращенный сердцем будет в презрении.
\vs Pro 12:9 Лучше простой, но работающий на себя, нежели выдающий себя за знатного, но нуждающийся в хлебе.
\vs Pro 12:10 Праведный печется и о жизни скота своего, сердце же нечестивых жестоко.
\vs Pro 12:11 Кто возделывает землю свою, тот будет насыщаться хлебом; а кто идет по следам празднолюбцев, тот скудоумен. [Кто находит удовольствие в трате времени за вином, тот в своем доме оставит бесславие.]
\vs Pro 12:12 Нечестивый желает уловить в сеть зла; но корень праведных тверд.
\vs Pro 12:13 Нечестивый уловляется грехами уст своих; но праведник выйдет из беды. [Смотрящий кротко помилован будет, а встречающийся в воротах стеснит других.]
\vs Pro 12:14 От плода уст \bibemph{своих} человек насыщается добром, и воздаяние человеку~--- по делам рук его.
\vs Pro 12:15 Путь глупого прямой в его глазах; но кто слушает совета, тот мудр.
\vs Pro 12:16 У глупого тотчас же выкажется гнев его, а благоразумный скрывает оскорбление.
\vs Pro 12:17 Кто говорит то, что знает, тот говорит правду; а у свидетеля ложного~--- обман.
\vs Pro 12:18 Иной пустослов уязвляет как мечом, а язык мудрых~--- врачует.
\vs Pro 12:19 Уста правдивые вечно пребывают, а лживый язык~--- только на мгновение.
\vs Pro 12:20 Коварство~--- в сердце злоумышленников, радость~--- у миротворцев.
\vs Pro 12:21 Не приключится праведнику никакого зла, нечестивые же будут преисполнены зол.
\vs Pro 12:22 Мерзость пред Господом~--- уста лживые, а говорящие истину благоугодны Ему.
\vs Pro 12:23 Человек рассудительный скрывает знание, а сердце глупых высказывает глупость.
\vs Pro 12:24 Рука прилежных будет господствовать, а ленивая будет под данью.
\vs Pro 12:25 Тоска на сердце человека подавляет его, а доброе слово развеселяет его.
\vs Pro 12:26 Праведник указывает ближнему своему путь, а путь нечестивых вводит их в заблуждение.
\vs Pro 12:27 Ленивый не жарит своей дичи; а имущество человека прилежного многоценно.
\vs Pro 12:28 На пути правды~--- жизнь, и на стезе ее нет смерти.
\vs Pro 13:1 Мудрый сын \bibemph{слушает} наставление отца, а буйный не слушает обличения.
\vs Pro 13:2 От плода уст \bibemph{своих} человек вкусит добро, душа же законопреступников~--- зло.
\vs Pro 13:3 Кто хранит уста свои, тот бережет душу свою; а кто широко раскрывает свой рот, тому беда.
\vs Pro 13:4 Душа ленивого желает, но тщетно; а душа прилежных насытится.
\vs Pro 13:5 Праведник ненавидит ложное слово, а нечестивый срамит и бесчестит \bibemph{себя}.
\vs Pro 13:6 Правда хранит непорочного в пути, а нечестие губит грешника.
\vs Pro 13:7 Иной выдает себя за богатого, а у него ничего нет; другой выдает себя за бедного, а у него богатства много.
\vs Pro 13:8 Богатством своим человек выкупает жизнь \bibemph{свою}, а бедный и угрозы не слышит.
\vs Pro 13:9 Свет праведных весело горит, светильник же нечестивых угасает. [Души коварные блуждают в грехах, а праведники сострадают и милуют.]
\vs Pro 13:10 От высокомерия происходит раздор, а у советующихся~--- мудрость.
\vs Pro 13:11 Богатство от суетности истощается, а собирающий трудами умножает его.
\vs Pro 13:12 Надежда, долго не сбывающаяся, томит сердце, а исполнившееся желание~--- \bibemph{как} древо жизни.
\vs Pro 13:13 Кто пренебрегает словом, тот причиняет вред себе; а кто боится заповеди, тому воздается.
\vs Pro 13:14 [У сына лукавого ничего нет доброго, а у разумного раба дела благоуспешны, и путь его прямой.]
\vs Pro 13:15 Учение мудрого~--- источник жизни, удаляющий от сетей смерти.
\vs Pro 13:16 Добрый разум доставляет приятность, путь же беззаконных жесток.
\vs Pro 13:17 Всякий благоразумный действует с знанием, а глупый выставляет напоказ глупость.
\vs Pro 13:18 Худой посол попадает в беду, а верный посланник~--- спасение.
\vs Pro 13:19 Нищета и посрамление отвергающему учение; а кто соблюдает наставление, будет в чести.
\vs Pro 13:20 Желание исполнившееся~--- приятно для души; но несносно для глупых уклоняться от зла.
\vs Pro 13:21 Общающийся с мудрыми будет мудр, а кто дружит с глупыми, развратится.
\vs Pro 13:22 Грешников преследует зло, а праведникам воздается добром.
\vs Pro 13:23 Добрый оставляет наследство \bibemph{и} внукам, а богатство грешника сберегается для праведного.
\vs Pro 13:24 Много хлеба \bibemph{бывает} и на ниве бедных; но некоторые гибнут от беспорядка.
\vs Pro 13:25 Кто жалеет розги своей, тот ненавидит сына; а кто любит, тот с детства наказывает его.
\vs Pro 13:26 Праведник ест до сытости, а чрево беззаконных терпит лишение.
\vs Pro 14:1 Мудрая жена устроит дом свой, а глупая разрушит его своими руками.
\vs Pro 14:2 Идущий прямым путем боится Господа; но чьи пути кривы, тот небрежет о Нем.
\vs Pro 14:3 В устах глупого~--- бич гордости; уста же мудрых охраняют их.
\vs Pro 14:4 Где нет волов, \bibemph{там} ясли пусты; а много прибыли от силы волов.
\vs Pro 14:5 Верный свидетель не лжет, а свидетель ложный наговорит много лжи.
\vs Pro 14:6 Распутный ищет мудрости, и не находит; а для разумного знание легко.
\vs Pro 14:7 Отойди от человека глупого, у которого ты не замечаешь разумных уст.
\vs Pro 14:8 Мудрость разумного~--- знание пути своего, глупость же безрассудных~--- заблуждение.
\vs Pro 14:9 Глупые смеются над грехом, а посреди праведных~--- благоволение.
\vs Pro 14:10 Сердце знает горе души своей, и в радость его не вмешается чужой.
\vs Pro 14:11 Дом беззаконных разорится, а жилище праведных процветет.
\vs Pro 14:12 Есть пути, которые кажутся человеку прямыми; но конец их~--- путь к смерти.
\vs Pro 14:13 И при смехе \bibemph{иногда} болит сердце, и концом радости бывает печаль.
\vs Pro 14:14 Человек с развращенным сердцем насытится от путей своих, и добрый~--- от своих.
\vs Pro 14:15 Глупый верит всякому слову, благоразумный же внимателен к путям своим.
\vs Pro 14:16 Мудрый боится и удаляется от зла, а глупый раздражителен и самонадеян.
\vs Pro 14:17 Вспыльчивый может сделать глупость; но человек, умышленно делающий зло, ненавистен.
\vs Pro 14:18 Невежды получают в удел себе глупость, а благоразумные увенчаются знанием.
\vs Pro 14:19 Преклонятся злые пред добрыми и нечестивые~--- у ворот праведника.
\vs Pro 14:20 Бедный ненавидим бывает даже близким своим, а у богатого много друзей.
\vs Pro 14:21 Кто презирает ближнего своего, тот грешит; а кто милосерд к бедным, тот блажен.
\vs Pro 14:22 Не заблуждаются ли умышляющие зло? [не знают милости и верности делающие зло;] но милость и верность у благомыслящих.
\vs Pro 14:23 От всякого труда есть прибыль, а от пустословия только ущерб.
\vs Pro 14:24 Венец мудрых~--- богатство их, а глупость невежд глупость \bibemph{и есть}.
\vs Pro 14:25 Верный свидетель спасает души, а лживый наговорит много лжи.
\vs Pro 14:26 В страхе пред Господом~--- надежда твердая, и сынам Своим Он прибежище.
\vs Pro 14:27 Страх Господень~--- источник жизни, удаляющий от сетей смерти.
\vs Pro 14:28 Во множестве народа~--- величие царя, а при малолюдстве народа беда государю.
\vs Pro 14:29 У терпеливого человека много разума, а раздражительный выказывает глупость.
\vs Pro 14:30 Кроткое сердце~--- жизнь для тела, а зависть~--- гниль для костей.
\vs Pro 14:31 Кто теснит бедного, тот хулит Творца его; чтущий же Его благотворит нуждающемуся.
\vs Pro 14:32 За зло свое нечестивый будет отвергнут, а праведный и при смерти своей имеет надежду.
\vs Pro 14:33 Мудрость почиет в сердце разумного, и среди глупых дает знать о себе.
\vs Pro 14:34 Праведность возвышает народ, а беззаконие~--- бесчестие народов.
\vs Pro 14:35 Благоволение царя~--- к рабу разумному, а гнев его~--- против того, кто позорит его.
\vs Pro 15:1 [Гнев губит и разумных.] Кроткий ответ отвращает гнев, а оскорбительное слово возбуждает ярость.
\vs Pro 15:2 Язык мудрых сообщает добрые знания, а уста глупых изрыгают глупость.
\vs Pro 15:3 На всяком месте очи Господни: они видят злых и добрых.
\vs Pro 15:4 Кроткий язык~--- древо жизни, но необузданный~--- сокрушение духа.
\vs Pro 15:5 Глупый пренебрегает наставлением отца своего; а кто внимает обличениям, тот благоразумен. [В обилии правды великая сила, а нечестивые искоренятся из земли.]
\vs Pro 15:6 В доме праведника~--- обилие сокровищ, а в прибытке нечестивого~--- расстройство.
\vs Pro 15:7 Уста мудрых распространяют знание, а сердце глупых не так.
\vs Pro 15:8 Жертва нечестивых~--- мерзость пред Господом, а молитва праведных благоугодна Ему.
\vs Pro 15:9 Мерзость пред Господом~--- путь нечестивого, а идущего путем правды Он любит.
\vs Pro 15:10 Злое наказание~--- уклоняющемуся от пути, и ненавидящий обличение погибнет.
\vs Pro 15:11 Преисподняя и Аваддон \bibemph{открыты} пред Господом, тем более сердца сынов человеческих.
\vs Pro 15:12 Не любит распутный обличающих его, и к мудрым не пойдет.
\vs Pro 15:13 Веселое сердце делает лице веселым, а при сердечной скорби дух унывает.
\vs Pro 15:14 Сердце разумного ищет знания, уста же глупых питаются глупостью.
\vs Pro 15:15 Все дни несчастного печальны; а у кого сердце весело, у того всегда пир.
\vs Pro 15:16 Лучше немногое при страхе Господнем, нежели большое сокровище, и при нем тревога.
\vs Pro 15:17 Лучше блюдо зелени, и при нем любовь, нежели откормленный бык, и при нем ненависть.
\vs Pro 15:18 Вспыльчивый человек возбуждает раздор, а терпеливый утишает распрю.
\vs Pro 15:19 Путь ленивого~--- как терновый плетень, а путь праведных~--- гладкий.
\vs Pro 15:20 Мудрый сын радует отца, а глупый человек пренебрегает мать свою.
\vs Pro 15:21 Глупость~--- радость для малоумного, а человек разумный идет прямою дорогою.
\vs Pro 15:22 Без совета предприятия расстроятся, а при множестве советников они состоятся.
\vs Pro 15:23 Радость человеку в ответе уст его, и как хорошо слово вовремя!
\vs Pro 15:24 Путь жизни мудрого вверх, чтобы уклониться от преисподней внизу.
\vs Pro 15:25 Дом надменных разорит Господь, а межу вдовы укрепит.
\vs Pro 15:26 Мерзость пред Господом~--- помышления злых, слова же непорочных угодны Ему.
\vs Pro 15:27 Корыстолюбивый расстроит дом свой, а ненавидящий подарки будет жить.
\vs Pro 15:28 Сердце праведного обдумывает ответ, а уста нечестивых изрыгают зло. [Приятны пред Господом пути праведных; чрез них и враги делаются друзьями.]
\vs Pro 15:29 Далек Господь от нечестивых, а молитву праведников слышит.
\vs Pro 15:30 Светлый взгляд радует сердце, добрая весть утучняет кости.
\vs Pro 15:31 Ухо, внимательное к учению жизни, пребывает между мудрыми.
\vs Pro 15:32 Отвергающий наставление не радеет о своей душе; а кто внимает обличению, тот приобретает разум.
\vs Pro 15:33 Страх Господень научает мудрости, и славе предшествует смирение.
\vs Pro 16:1 Человеку \bibemph{принадлежат} предположения сердца, но от Господа ответ языка.
\vs Pro 16:2 Все пути человека чисты в его глазах, но Господь взвешивает души.
\vs Pro 16:3 Предай Господу дела твои, и предприятия твои совершатся.
\vs Pro 16:4 Все сделал Господь ради Себя; и даже нечестивого \bibemph{блюдет} на день бедствия.
\vs Pro 16:5 Мерзость пред Господом всякий надменный сердцем; можно поручиться, что он не останется ненаказанным. [Начало доброго пути~--- делать правду; это угоднее пред Богом, нежели приносить жертвы. Ищущий Господа найдет знание с правдою; истинно ищущие Его найдут мир.]
\vs Pro 16:6 Милосердием и правдою очищается грех, и страх Господень отводит от зла.
\vs Pro 16:7 Когда Господу угодны пути человека, Он и врагов его примиряет с ним.
\vs Pro 16:8 Лучше немногое с правдою, нежели множество прибытков с неправдою.
\vs Pro 16:9 Сердце человека обдумывает свой путь, но Господь управляет шествием его.
\vs Pro 16:10 В устах царя~--- слово вдохновенное; уста его не должны погрешать на суде.
\vs Pro 16:11 Верные весы и весовые чаши~--- от Господа; от Него же все гири в суме.
\vs Pro 16:12 Мерзость для царей~--- дело беззаконное, потому что правдою утверждается престол.
\vs Pro 16:13 Приятны царю уста правдивые, и говорящего истину он любит.
\vs Pro 16:14 Царский гнев~--- вестник смерти; но мудрый человек умилостивит его.
\vs Pro 16:15 В светлом взоре царя~--- жизнь, и благоволение его~--- как облако с поздним дождем.
\vs Pro 16:16 Приобретение мудрости гораздо лучше золота, и приобретение разума предпочтительнее отборного серебра.
\vs Pro 16:17 Путь праведных~--- уклонение от зла: тот бережет душу свою, кто хранит путь свой.
\vs Pro 16:18 Погибели предшествует гордость, и падению~--- надменность.
\vs Pro 16:19 Лучше смиряться духом с кроткими, нежели разделять добычу с гордыми.
\vs Pro 16:20 Кто ведет дело разумно, тот найдет благо, и кто надеется на Господа, тот блажен.
\vs Pro 16:21 Мудрый сердцем прозовется благоразумным, и сладкая речь прибавит к учению.
\vs Pro 16:22 Разум для имеющих его~--- источник жизни, а ученость глупых~--- глупость.
\vs Pro 16:23 Сердце мудрого делает язык его мудрым и умножает знание в устах его.
\vs Pro 16:24 Приятная речь~--- сотовый мед, сладка для души и целебна для костей.
\vs Pro 16:25 Есть пути, которые кажутся человеку прямыми, но конец их путь к смерти.
\vs Pro 16:26 Трудящийся трудится для себя, потому что понуждает его \bibemph{к тому} рот его.
\vs Pro 16:27 Человек лукавый замышляет зло, и на устах его как бы огонь палящий.
\vs Pro 16:28 Человек коварный сеет раздор, и наушник разлучает друзей.
\vs Pro 16:29 Человек неблагонамеренный развращает ближнего своего и ведет его на путь недобрый;
\vs Pro 16:30 прищуривает глаза свои, чтобы придумать коварство; закусывая себе губы, совершает злодейство; [он~--- печь злобы].
\vs Pro 16:31 Венец славы~--- седина, которая находится на пути правды.
\vs Pro 16:32 Долготерпеливый лучше храброго, и владеющий собою \bibemph{лучше} завоевателя города.
\vs Pro 16:33 В полу бросается жребий, но все решение его~--- от Господа.
\vs Pro 17:1 Лучше кусок сухого хлеба, и с ним мир, нежели дом, полный заколотого скота, с раздором.
\vs Pro 17:2 Разумный раб господствует над беспутным сыном и между братьями разделит наследство.
\vs Pro 17:3 Плавильня~--- для серебра, и горнило~--- для золота, а сердца испытывает Господь.
\vs Pro 17:4 Злодей внимает устам беззаконным, лжец слушается языка пагубного.
\vs Pro 17:5 Кто ругается над нищим, тот хулит Творца его; кто радуется несчастью, тот не останется ненаказанным [а милосердый помилован будет].
\vs Pro 17:6 Венец стариков~--- сыновья сыновей, и слава детей~--- родители их. [У верного целый мир богатства, а у неверного~--- ни обола.]
\vs Pro 17:7 Неприлична глупому важная речь, тем паче знатному~--- уста лживые.
\vs Pro 17:8 Подарок~--- драгоценный камень в глазах владеющего им: куда ни обратится он, успеет.
\vs Pro 17:9 Прикрывающий проступок ищет любви; а кто снова напоминает о нем, тот удаляет друга.
\vs Pro 17:10 На разумного сильнее действует выговор, нежели на глупого сто ударов.
\vs Pro 17:11 Возмутитель ищет только зла; поэтому жестокий ангел будет послан против него.
\vs Pro 17:12 Лучше встретить человеку медведицу, лишенную детей, нежели глупца с его глупостью.
\vs Pro 17:13 Кто за добро воздает злом, от дома того не отойдет зло.
\vs Pro 17:14 Начало ссоры~--- как прорыв воды; оставь ссору прежде, нежели разгорелась она.
\vs Pro 17:15 Оправдывающий нечестивого и обвиняющий праведного~--- оба мерзость пред Господом.
\vs Pro 17:16 К чему сокровище в руках глупца? Для приобретения мудрости \bibemph{у него} нет разума. [Кто высоким делает свой дом, тот ищет разбиться; а уклоняющийся от учения впадет в беды.]
\vs Pro 17:17 Друг любит во всякое время и, как брат, явится во время несчастья.
\vs Pro 17:18 Человек малоумный дает руку и ручается за ближнего своего.
\vs Pro 17:19 Кто любит ссоры, любит грех, и кто высоко поднимает ворота свои, тот ищет падения.
\vs Pro 17:20 Коварное сердце не найдет добра, и лукавый язык попадет в беду.
\vs Pro 17:21 Родил кто глупого,~--- себе на г\acc{о}ре, и отец глупого не порадуется.
\vs Pro 17:22 Веселое сердце благотворно, как врачевство, а унылый дух сушит кости.
\vs Pro 17:23 Нечестивый берет подарок из пазухи, чтобы извратить пути правосудия.
\vs Pro 17:24 Мудрость~--- пред лицем у разумного, а глаза глупца~--- на конце земли.
\vs Pro 17:25 Глупый сын~--- досада отцу своему и огорчение для матери своей.
\vs Pro 17:26 Нехорошо и обвинять правого, \bibemph{и} бить вельмож за правду.
\vs Pro 17:27 Разумный воздержан в словах своих, и благоразумный хладнокровен.
\vs Pro 17:28 И глупец, когда молчит, может показаться мудрым, и затворяющий уста свои~--- благоразумным.
\vs Pro 18:1 Прихоти ищет своенравный, восстает против всего умного.
\vs Pro 18:2 Глупый не любит знания, а только бы выказать свой ум.
\vs Pro 18:3 С приходом нечестивого приходит и презрение, а с бесславием~--- поношение.
\vs Pro 18:4 Слова уст человеческих~--- глубокие воды; источник мудрости~--- струящийся поток.
\vs Pro 18:5 Нехорошо быть лицеприятным к нечестивому, чтобы ниспровергнуть праведного на суде.
\vs Pro 18:6 Уста глупого идут в ссору, и слова его вызывают побои.
\vs Pro 18:7 Язык глупого~--- гибель для него, и уста его~--- сеть для души его.
\vs Pro 18:8 [Ленивого низлагает страх, а души женоподобные будут голодать.]
\vs Pro 18:9 Слова наушника~--- как лакомства, и они входят во внутренность чрева.
\vs Pro 18:10 Нерадивый в работе своей~--- брат расточителю.
\vs Pro 18:11 Имя Господа~--- крепкая башня: убегает в нее праведник~--- и безопасен.
\vs Pro 18:12 Имение богатого~--- крепкий город его, и как высокая ограда в его воображении.
\vs Pro 18:13 Перед падением возносится сердце человека, а смирение предшествует славе.
\vs Pro 18:14 Кто дает ответ не выслушав, тот глуп, и стыд ему.
\vs Pro 18:15 Дух человека переносит его немощи; а пораженный дух~--- кто может подкрепить его?
\vs Pro 18:16 Сердце разумного приобретает знание, и ухо мудрых ищет знания.
\vs Pro 18:17 Подарок у человека дает ему простор и до вельмож доведет его.
\vs Pro 18:18 Первый в тяжбе своей прав, но приходит соперник его и исследует его.
\vs Pro 18:19 Жребий прекращает споры и решает между сильными.
\vs Pro 18:20 Озлобившийся брат \bibemph{неприступнее} крепкого города, и ссоры подобны запорам з\acc{а}мка.
\vs Pro 18:21 От плода уст человека наполняется чрево его; произведением уст своих он насыщается.
\vs Pro 18:22 Смерть и жизнь~--- во власти языка, и любящие его вкусят от плодов его.
\vs Pro 18:23 Кто нашел [добрую] жену, тот нашел благо и получил благодать от Господа. [Кто изгоняет добрую жену, тот изгоняет счастье, а содержащий прелюбодейку~--- безумен и нечестив.]
\vs Pro 18:24 С мольбою говорит нищий, а богатый отвечает грубо.
\vs Pro 18:25 Кто хочет иметь друзей, тот и сам должен быть дружелюбным; и бывает друг, более привязанный, нежели брат.
\vs Pro 19:1 Лучше бедный, ходящий в своей непорочности, нежели [богатый] со лживыми устами, и притом глупый.
\vs Pro 19:2 Нехорошо душе без знания, и торопливый ногами оступится.
\vs Pro 19:3 Глупость человека извращает путь его, а сердце его негодует на Господа.
\vs Pro 19:4 Богатство прибавляет много друзей, а бедный оставляется и другом своим.
\vs Pro 19:5 Лжесвидетель не останется ненаказанным, и кто говорит ложь, не спасется.
\vs Pro 19:6 Многие заискивают у знатных, и всякий~--- друг человеку, делающему подарки.
\vs Pro 19:7 Бедного ненавидят все братья его, тем паче друзья его удаляются от него: гонится за ними, чтобы поговорить, но и этого нет.
\vs Pro 19:8 Кто приобретает разум, тот любит душу свою; кто наблюдает благоразумие, тот находит благо.
\vs Pro 19:9 Лжесвидетель не останется ненаказанным, и кто говорит ложь, погибнет.
\vs Pro 19:10 Неприлична глупцу пышность, тем паче рабу господство над князьями.
\vs Pro 19:11 Благоразумие делает человека медленным на гнев, и слава для него~--- быть снисходительным к проступкам.
\vs Pro 19:12 Гнев царя~--- как рев льва, а благоволение его~--- как роса на траву.
\vs Pro 19:13 Глупый сын~--- сокрушение для отца своего, и сварливая жена~--- сточная труба.
\vs Pro 19:14 Дом и имение~--- наследство от родителей, а разумная жена~--- от Господа.
\vs Pro 19:15 Леность погружает в сонливость, и нерадивая душа будет терпеть голод.
\vs Pro 19:16 Хранящий заповедь хранит душу свою, а нерадящий о путях своих погибнет.
\vs Pro 19:17 Благотворящий бедному дает взаймы Господу, и Он воздаст ему за благодеяние его.
\vs Pro 19:18 Наказывай сына своего, доколе есть надежда, и не возмущайся криком его.
\vs Pro 19:19 Гневливый пусть терпит наказание, потому что, если пощадишь \bibemph{его}, придется тебе еще больше наказывать его.
\vs Pro 19:20 Слушайся совета и принимай обличение, чтобы сделаться тебе впоследствии мудрым.
\vs Pro 19:21 Много замыслов в сердце человека, но состоится только определенное Господом.
\vs Pro 19:22 Радость человеку~--- благотворительность его, и бедный человек лучше, нежели лживый.
\vs Pro 19:23 Страх Господень \bibemph{ведет} к жизни, и \bibemph{кто имеет его}, всегда будет доволен, и зло не постигнет его.
\vs Pro 19:24 Ленивый опускает руку свою в чашу, и не хочет донести ее до рта своего.
\vs Pro 19:25 Если ты накажешь кощунника, то и простой сделается благоразумным; и \bibemph{если} обличишь разумного, то он поймет наставление.
\vs Pro 19:26 Разоряющий отца и выгоняющий мать~--- сын срамной и бесчестный.
\vs Pro 19:27 Перестань, сын мой, слушать внушения об уклонении от изречений разума.
\vs Pro 19:28 Лукавый свидетель издевается над судом, и уста беззаконных глотают неправду.
\vs Pro 19:29 Готовы для кощунствующих суды, и побои~--- на тело глупых.
\vs Pro 20:1 Вино~--- глумливо, сикера~--- буйна; и всякий, увлекающийся ими, неразумен.
\vs Pro 20:2 Гроза царя~--- как бы рев льва: кто раздражает его, тот грешит против самого себя.
\vs Pro 20:3 Честь для человека~--- отстать от ссоры; а всякий глупец задорен.
\vs Pro 20:4 Ленивец зимою не пашет: поищет летом~--- и нет ничего.
\vs Pro 20:5 Помыслы в сердце человека~--- глубокие воды, но человек разумный вычерпывает их.
\vs Pro 20:6 Многие хвалят человека за милосердие, но правдивого человека кто находит?
\vs Pro 20:7 Праведник ходит в своей непорочности: блаженны дети его после него!
\vs Pro 20:8 Царь, сидящий на престоле суда, разгоняет очами своими все злое.
\vs Pro 20:9 Кто может сказать: <<я очистил мое сердце, я чист от греха моего?>>
\vs Pro 20:10 Неодинаковые весы, неодинаковая мера, то и другое~--- мерзость пред Господом.
\vs Pro 20:11 Можно узнать даже отрока по занятиям его, чисто ли и правильно ли будет поведение его.
\vs Pro 20:12 Ухо слышащее и глаз видящий~--- и то и другое создал Господь.
\vs Pro 20:13 Не люби спать, чтобы тебе не обеднеть; держи открытыми глаза твои, и будешь досыта есть хлеб.
\vs Pro 20:14 <<Дурно, дурно>>, говорит покупатель, а когда отойдет, хвалится.
\vs Pro 20:15 Есть золото и много жемчуга, но драгоценная утварь~--- уста разумные.
\vs Pro 20:16 Возьми платье его, так как он поручился за чужого; и за стороннего возьми от него залог.
\vs Pro 20:17 Сладок для человека хлеб, \bibemph{приобретенный} неправдою; но после рот его наполнится дресвою.
\vs Pro 20:18 Предприятия получают твердость чрез совещание, и по совещании веди войну.
\vs Pro 20:19 Кто ходит переносчиком, тот открывает тайну; и кто широко раскрывает рот, с тем не сообщайся.
\vs Pro 20:20 Кто злословит отца своего и свою мать, того светильник погаснет среди глубокой тьмы.
\vs Pro 20:21 Наследство, поспешно захваченное вначале, не благословится впоследствии.
\vs Pro 20:22 Не говори: <<я отплачу за зло>>; предоставь Господу, и Он сохранит тебя.
\vs Pro 20:23 Мерзость пред Господом~--- неодинаковые гири, и неверные весы~--- не добро.
\vs Pro 20:24 От Господа направляются шаги человека; человеку же как узнать путь свой?
\vs Pro 20:25 Сеть для человека~--- поспешно давать обет, и после обета обдумывать.
\vs Pro 20:26 Мудрый царь вывеет нечестивых и обратит на них колесо.
\vs Pro 20:27 Светильник Господень~--- дух человека, испытывающий все глубины сердца.
\vs Pro 20:28 Милость и истина охраняют царя, и милостью он поддерживает престол свой.
\vs Pro 20:29 Слава юношей~--- сила их, а украшение стариков~--- седина.
\vs Pro 20:30 Раны от побоев~--- врачевство против зла, и удары, проникающие во внутренности чрева.
\vs Pro 21:1 Сердце царя~--- в руке Господа, как потоки вод: куда захочет, Он направляет его.
\vs Pro 21:2 Всякий путь человека прям в глазах его; но Господь взвешивает сердца.
\vs Pro 21:3 Соблюдение правды и правосудия более угодно Господу, нежели жертва.
\vs Pro 21:4 Гордость очей и надменность сердца, отличающие нечестивых,~--- грех.
\vs Pro 21:5 Помышления прилежного стремятся к изобилию, а всякий торопливый терпит лишение.
\vs Pro 21:6 Приобретение сокровища лживым языком~--- мимолетное дуновение ищущих смерти.
\vs Pro 21:7 Насилие нечестивых обрушится на них, потому что они отреклись соблюдать правду.
\vs Pro 21:8 Превратен путь человека развращенного; а кто чист, того действие прямо.
\vs Pro 21:9 Лучше жить в углу на кровле, нежели со сварливою женою в пространном доме.
\vs Pro 21:10 Душа нечестивого желает зла: не найдет милости в глазах его и друг его.
\vs Pro 21:11 Когда наказывается кощунник, простой делается мудрым; и когда вразумляется мудрый, то он приобретает знание.
\vs Pro 21:12 Праведник наблюдает за домом нечестивого: как повергаются нечестивые в несчастие.
\vs Pro 21:13 Кто затыкает ухо свое от вопля бедного, тот и сам будет вопить,~--- и не будет услышан.
\vs Pro 21:14 Подарок тайный тушит гнев, и дар в пазуху~--- сильную ярость.
\vs Pro 21:15 Соблюдение правосудия~--- радость для праведника и страх для делающих зло.
\vs Pro 21:16 Человек, сбившийся с пути разума, водворится в собрании мертвецов.
\vs Pro 21:17 Кто любит веселье, обеднеет; а кто любит вино и тук, не разбогатеет.
\vs Pro 21:18 Выкупом будет за праведного нечестивый и за прямодушного~--- лукавый.
\vs Pro 21:19 Лучше жить в земле пустынной, нежели с женою сварливою и сердитою.
\vs Pro 21:20 Вожделенное сокровище и тук~--- в доме мудрого; а глупый человек расточает их.
\vs Pro 21:21 Соблюдающий правду и милость найдет жизнь, правду и славу.
\vs Pro 21:22 Мудрый входит в город сильных и ниспровергает крепость, на которую они надеялись.
\vs Pro 21:23 Кто хранит уста свои и язык свой, тот хранит от бед душу свою.
\vs Pro 21:24 Надменный злодей~--- кощунник имя ему~--- действует в пылу гордости.
\vs Pro 21:25 Алчба ленивца убьет его, потому что руки его отказываются работать;
\vs Pro 21:26 всякий день он сильно алчет, а праведник дает и не жалеет.
\vs Pro 21:27 Жертва нечестивых~--- мерзость, особенно когда с лукавством приносят ее.
\vs Pro 21:28 Лжесвидетель погибнет; а человек, который говорит, что знает, будет говорить всегда.
\vs Pro 21:29 Человек нечестивый дерзок лицом своим, а праведный держит прямо путь свой.
\vs Pro 21:30 Нет мудрости, и нет разума, и нет совета вопреки Господу.
\vs Pro 21:31 Коня приготовляют на день битвы, но победа~--- от Господа.
\vs Pro 22:1 Доброе имя лучше большого богатства, и добрая слава лучше серебра и золота.
\vs Pro 22:2 Богатый и бедный встречаются друг с другом: того и другого создал Господь.
\vs Pro 22:3 Благоразумный видит беду, и укрывается; а неопытные идут вперед, и наказываются.
\vs Pro 22:4 За смирением следует страх Господень, богатство и слава и жизнь.
\vs Pro 22:5 Терны и сети на пути коварного; кто бережет душу свою, удались от них.
\vs Pro 22:6 Наставь юношу при начале пути его: он не уклонится от него, когда и состарится.
\vs Pro 22:7 Богатый господствует над бедным, и должник \bibemph{делается} рабом заимодавца.
\vs Pro 22:8 Сеющий неправду пожнет беду, и трости гнева его не станет. [Человека, доброхотно дающего, любит Бог, и недостаток дел его восполнит.]
\vs Pro 22:9 Милосердый будет благословляем, потому что дает бедному от хлеба своего. [Победу и честь приобретает дающий дары, и даже овладевает душею получающих оные.]
\vs Pro 22:10 Прогони кощунника, и удалится раздор, и прекратятся ссора и брань.
\vs Pro 22:11 Кто любит чистоту сердца, у того приятность на устах, тому царь~--- друг.
\vs Pro 22:12 Очи Господа охраняют знание, а слова законопреступника Он ниспровергает.
\vs Pro 22:13 Ленивец говорит: <<лев на улице! посреди площади убьют меня!>>
\vs Pro 22:14 Глубокая пропасть~--- уста блудниц: на кого прогневается Господь, тот упадет туда.
\vs Pro 22:15 Глупость привязалась к сердцу юноши, но исправительная розга удалит ее от него.
\vs Pro 22:16 Кто обижает бедного, чтобы умножить свое богатство, и кто дает богатому, тот обеднеет.
\rsbpar\vs Pro 22:17 Приклони ухо твое, и слушай слова мудрых, и сердце твое обрати к моему знанию;
\vs Pro 22:18 потому что утешительно будет, если ты будешь хранить их в сердце твоем, и они будут также в устах твоих.
\vs Pro 22:19 Чтобы упование твое было на Господа, я учу тебя и сегодня, и ты \bibemph{помни}.
\vs Pro 22:20 Не писал ли я тебе трижды в советах и наставлении,
\vs Pro 22:21 чтобы научить тебя точным словам истины, дабы ты мог передавать слова истины посылающим тебя?
\rsbpar\vs Pro 22:22 Не будь грабителем бедного, потому что он беден, и не притесняй несчастного у ворот,
\vs Pro 22:23 потому что Господь вступится в дело их и исхитит душу у грабителей их.
\vs Pro 22:24 Не дружись с гневливым и не сообщайся с человеком вспыльчивым,
\vs Pro 22:25 чтобы не научиться путям его и не навлечь петли на душу твою.
\vs Pro 22:26 Не будь из тех, которые дают руки и поручаются за долги:
\vs Pro 22:27 если тебе нечем заплатить, то для чего доводить себя, чтобы взяли постель твою из-под тебя?
\vs Pro 22:28 Не передвигай межи давней, которую провели отцы твои.
\vs Pro 22:29 Видел ли ты человека проворного в своем деле? Он будет стоять перед царями, он не будет стоять перед простыми.
\vs Pro 23:1 Когда сядешь вкушать пищу с властелином, то тщательно наблюдай, что перед тобою,
\vs Pro 23:2 и поставь преграду в гортани твоей, если ты алчен.
\vs Pro 23:3 Не прельщайся лакомыми яствами его; это~--- обманчивая пища.
\vs Pro 23:4 Не заботься о том, чтобы нажить богатство; оставь такие мысли твои.
\vs Pro 23:5 Устремишь глаза твои на него, и~--- его уже нет; потому что оно сделает себе крылья и, как орел, улетит к небу.
\vs Pro 23:6 Не вкушай пищи у человека завистливого и не прельщайся лакомыми яствами его;
\vs Pro 23:7 потому что, каковы мысли в душе его, таков и он; <<ешь и пей>>, говорит он тебе, а сердце его не с тобою.
\vs Pro 23:8 Кусок, который ты съел, изблюешь, и добрые слова твои ты потратишь напрасно.
\vs Pro 23:9 В уши глупого не говори, потому что он презрит разумные слова твои.
\vs Pro 23:10 Не передвигай межи давней и на поля сирот не заходи,
\vs Pro 23:11 потому что Защитник их силен; Он вступится в дело их с тобою.
\vs Pro 23:12 Приложи сердце твое к учению и уши твои~--- к умным словам.
\vs Pro 23:13 Не оставляй юноши без наказания: если накажешь его розгою, он не умрет;
\vs Pro 23:14 ты накажешь его розгою и спасешь душу его от преисподней.
\rsbpar\vs Pro 23:15 Сын мой! если сердце твое будет мудро, то порадуется и мое сердце;
\vs Pro 23:16 и внутренности мои будут радоваться, когда уста твои будут говорить правое.
\vs Pro 23:17 Да не завидует сердце твое грешникам, но да пребудет оно во все дни в страхе Господнем;
\vs Pro 23:18 потому что есть будущность, и надежда твоя не потеряна.
\vs Pro 23:19 Слушай, сын мой, и будь мудр, и направляй сердце твое на прямой путь.
\vs Pro 23:20 Не будь между упивающимися вином, между пресыщающимися мясом:
\vs Pro 23:21 потому что пьяница и пресыщающийся обеднеют, и сонливость оденет в рубище.
\vs Pro 23:22 Слушайся отца твоего: он родил тебя; и не пренебрегай матери твоей, когда она и состарится.
\vs Pro 23:23 Купи истину и не продавай мудрости и учения и разума.
\vs Pro 23:24 Торжествует отец праведника, и родивший мудрого радуется о нем.
\vs Pro 23:25 Да веселится отец твой и да торжествует мать твоя, родившая тебя.
\rsbpar\vs Pro 23:26 Сын мой! отдай сердце твое мне, и глаза твои да наблюдают пути мои,
\vs Pro 23:27 потому что блудница~--- глубокая пропасть, и чужая жена~--- тесный колодезь;
\vs Pro 23:28 она, как разбойник, сидит в засаде и умножает между людьми законопреступников.
\vs Pro 23:29 У кого вой? у кого стон? у кого ссоры? у кого горе? у кого раны без причины? у кого багровые глаза?
\vs Pro 23:30 У тех, которые долго сидят за вином, которые приходят отыскивать \bibemph{вина} приправленного.
\vs Pro 23:31 Не смотри на вино, как оно краснеет, как оно искрится в чаше, как оно ухаживается ровно:
\vs Pro 23:32 впоследствии, как змей, оно укусит, и ужалит, как аспид;
\vs Pro 23:33 глаза твои будут смотреть на чужих жен, и сердце твое заговорит развратное,
\vs Pro 23:34 и ты будешь, как спящий среди моря и как спящий на верху мачты.
\vs Pro 23:35 [И скажешь:] <<били меня, мне не было больно; толкали меня, я не чувствовал. Когда проснусь, опять буду искать того же>>.
\vs Pro 24:1 Не ревнуй злым людям и не желай быть с ними,
\vs Pro 24:2 потому что о насилии помышляет сердце их, и о злом говорят уста их.
\vs Pro 24:3 Мудростью устрояется дом и разумом утверждается,
\vs Pro 24:4 и с уменьем внутренности его наполняются всяким драгоценным и прекрасным имуществом.
\vs Pro 24:5 Человек мудрый силен, и человек разумный укрепляет силу свою.
\vs Pro 24:6 Поэтому с обдуманностью веди войну твою, и успех \bibemph{будет} при множестве совещаний.
\vs Pro 24:7 Для глупого слишком высока мудрость; у ворот не откроет он уст своих.
\vs Pro 24:8 Кто замышляет сделать зло, того называют злоумышленником.
\vs Pro 24:9 Помысл глупости~--- грех, и кощунник~--- мерзость для людей.
\vs Pro 24:10 Если ты в день бедствия оказался слабым, то бедна сила твоя.
\vs Pro 24:11 Спасай взятых на смерть, и неужели откажешься от обреченных на убиение?
\vs Pro 24:12 Скажешь ли: <<вот, мы не знали этого>>? А Испытующий сердц\acc{а} разве не знает? Наблюдающий над душею твоею знает это, и воздаст человеку по делам его.
\vs Pro 24:13 Ешь, сын мой, мед, потому что он приятен, и сот, который сладок для гортани твоей:
\vs Pro 24:14 таково и познание мудрости для души твоей. Если ты нашел \bibemph{ее}, то есть будущность, и надежда твоя не потеряна.
\vs Pro 24:15 Не злоумышляй, нечестивый, против жилища праведника, не опустошай места покоя его,
\vs Pro 24:16 ибо семь раз упадет праведник, и встанет; а нечестивые впадут в погибель.
\vs Pro 24:17 Не радуйся, когда упадет враг твой, и да не веселится сердце твое, когда он споткнется.
\vs Pro 24:18 Иначе, увидит Господь, и неугодно будет это в очах Его, и Он отвратит от него гнев Свой.
\vs Pro 24:19 Не негодуй на злодеев и не завидуй нечестивым,
\vs Pro 24:20 потому что злой не имеет будущности,~--- светильник нечестивых угаснет.
\vs Pro 24:21 Бойся, сын мой, Господа и царя; с мятежниками не сообщайся,
\vs Pro 24:22 потому что внезапно придет погибель от них, и беду от них обоих кто предузнает?
\vs Pro 24:23 Сказано также мудрыми: иметь лицеприятие на суде~--- нехорошо.
\vs Pro 24:24 Кто говорит виновному: <<ты прав>>, того будут проклинать народы, того будут ненавидеть племена;
\vs Pro 24:25 а обличающие будут любимы, и на них придет благословение.
\vs Pro 24:26 В уста целует, кто отвечает словами верными.
\vs Pro 24:27 Соверши дела твои вне дома, окончи их на поле твоем, и потом устрояй и дом твой.
\vs Pro 24:28 Не будь лжесвидетелем на ближнего твоего: к чему тебе обманывать устами твоими?
\vs Pro 24:29 Не говори: <<как он поступил со мною, так и я поступлю с ним, воздам человеку по делам его>>.
\vs Pro 24:30 Проходил я мимо поля человека ленивого и мимо виноградника человека скудоумного:
\vs Pro 24:31 и вот, все это заросло терном, поверхность его покрылась крапивою, и каменная ограда его обрушилась.
\vs Pro 24:32 И посмотрел я, и обратил сердце мое, и посмотрел и получил урок:
\vs Pro 24:33 <<немного поспишь, немного подремлешь, немного, сложив руки, полежишь,~---
\vs Pro 24:34 и придет, \bibemph{как} прохожий, бедность твоя, и нужда твоя~--- как человек вооруженный>>.
\vs Pro 25:1 И это притчи Соломона, которые собрали мужи Езекии, царя Иудейского.
\vs Pro 25:2 Слава Божия~--- облекать тайною дело, а слава царей~--- исследовать дело.
\vs Pro 25:3 Как небо в высоте и земля в глубине, так сердце царей~--- неисследимо.
\vs Pro 25:4 Отдели примесь от серебра, и выйдет у серебряника сосуд:
\vs Pro 25:5 удали неправедного от царя, и престол его утвердится правдою.
\vs Pro 25:6 Не величайся пред лицем царя, и на месте великих не становись;
\vs Pro 25:7 потому что лучше, когда скажут тебе: <<пойди сюда повыше>>, нежели когда понизят тебя пред знатным, которого видели глаза твои.
\vs Pro 25:8 Не вступай поспешно в тяжбу: иначе что будешь делать при окончании, когда соперник твой осрамит тебя?
\vs Pro 25:9 Веди тяжбу с соперником твоим, но тайны другого не открывай,
\vs Pro 25:10 дабы не укорил тебя услышавший это, и тогда бесчестие твое не отойдет от тебя. [Любовь и дружба освобождают: сбереги их для себя, чтобы не сделаться тебе достойным поношения; сохрани пути твои благоустроенными.]
\vs Pro 25:11 Золотые яблоки в серебряных прозрачных сосудах~--- слово, сказанное прилично.
\vs Pro 25:12 Золотая серьга и украшение из чистого золота~--- мудрый обличитель для внимательного уха.
\vs Pro 25:13 Что прохлада от снега во время жатвы, то верный посол для посылающего его: он доставляет душе господина своего отраду.
\vs Pro 25:14 Что тучи и ветры без дождя, то человек, хвастающий ложными подарками.
\vs Pro 25:15 Кротостью склоняется к милости вельможа, и мягкий язык переламывает кость.
\vs Pro 25:16 Нашел ты мед,~--- ешь, сколько тебе потребно, чтобы не пресытиться им и не изблевать его.
\vs Pro 25:17 Не учащай входить в дом друга твоего, чтобы он не наскучил тобою и не возненавидел тебя.
\vs Pro 25:18 Что молот и меч и острая стрела, то человек, произносящий ложное свидетельство против ближнего своего.
\vs Pro 25:19 Что сломанный зуб и расслабленная нога, то надежда на ненадежного [человека] в день бедствия.
\vs Pro 25:20 Что снимающий с себя одежду в холодный день, что уксус на рану, то поющий песни печальному сердцу. [Как моль одежде и червь дереву, так печаль вредит сердцу человека.]
\vs Pro 25:21 Если голоден враг твой, накорми его хлебом; и если он жаждет, напой его водою:
\vs Pro 25:22 ибо, [делая сие,] ты собираешь горящие угли на голову его, и Господь воздаст тебе.
\vs Pro 25:23 Северный ветер производит дождь, а тайный язык~--- недовольные лица.
\vs Pro 25:24 Лучше жить в углу на кровле, нежели со сварливою женою в пространном доме.
\vs Pro 25:25 Что холодная вода для истомленной жаждой души, то добрая весть из дальней страны.
\vs Pro 25:26 Что возмущенный источник и поврежденный родник, то праведник, падающий пред нечестивым.
\vs Pro 25:27 Как нехорошо есть много меду, так домогаться славы не есть слава.
\vs Pro 25:28 Что город разрушенный, без стен, то человек, не владеющий духом своим.
\vs Pro 26:1 Как снег летом и дождь во время жатвы, так честь неприлична глупому.
\vs Pro 26:2 Как воробей вспорхнет, как ласточка улетит, так незаслуженное проклятие не сбудется.
\vs Pro 26:3 Бич для коня, узда для осла, а палка для глупых.
\vs Pro 26:4 Не отвечай глупому по глупости его, чтобы и тебе не сделаться подобным ему;
\vs Pro 26:5 но отвечай глупому по глупости его, чтобы он не стал мудрецом в глазах своих.
\vs Pro 26:6 Подрезывает себе ноги, терпит неприятность тот, кто дает словесное поручение глупцу.
\vs Pro 26:7 Неровно поднимаются ноги у хромого,~--- и притча в устах глупцов.
\vs Pro 26:8 Что влагающий драгоценный камень в пращу, то воздающий глупому честь.
\vs Pro 26:9 Что \bibemph{колючий} терн в руке пьяного, то притча в устах глупцов.
\vs Pro 26:10 Сильный делает все произвольно: и глупого награждает, и всякого прохожего награждает.
\vs Pro 26:11 Как пес возвращается на блевотину свою, так глупый повторяет глупость свою.
\vs Pro 26:12 Видал ли ты человека, мудрого в глазах его? На глупого больше надежды, нежели на него.
\vs Pro 26:13 Ленивец говорит: <<лев на дороге! лев на площадях!>>
\vs Pro 26:14 Дверь ворочается на крючьях своих, а ленивец на постели своей.
\vs Pro 26:15 Ленивец опускает руку свою в чашу, и ему тяжело донести ее до рта своего.
\vs Pro 26:16 Ленивец в глазах своих мудрее семерых, отвечающих обдуманно.
\vs Pro 26:17 Хватает пса за уши, кто, проходя мимо, вмешивается в чужую ссору.
\vs Pro 26:18 Как притворяющийся помешанным бросает огонь, стрелы и смерть,
\vs Pro 26:19 так~--- человек, который коварно вредит другу своему и потом говорит: <<я только пошутил>>.
\vs Pro 26:20 Где нет больше дров, огонь погасает, и где нет наушника, раздор утихает.
\vs Pro 26:21 Уголь~--- для жара и дрова~--- для огня, а человек сварливый~--- для разжжения ссоры.
\vs Pro 26:22 Слова наушника~--- как лакомства, и они входят во внутренность чрева.
\vs Pro 26:23 Что нечистым серебром обложенный глиняный сосуд, то пламенные уста и сердце злобное.
\vs Pro 26:24 Устами своими притворяется враг, а в сердце своем замышляет коварство.
\vs Pro 26:25 Если он говорит и нежным голосом, не верь ему, потому что семь мерзостей в сердце его.
\vs Pro 26:26 Если ненависть прикрывается наедине, то откроется злоба его в народном собрании.
\vs Pro 26:27 Кто роет яму, тот упадет в нее, и кто покатит вверх камень, к тому он воротится.
\vs Pro 26:28 Лживый язык ненавидит уязвляемых им, и льстивые уста готовят падение.
\vs Pro 27:1 Не хвались завтрашним днем, потому что не знаешь, чт\acc{о} родит тот день.
\vs Pro 27:2 Пусть хвалит тебя другой, а не уста твои,~--- чужой, а не язык твой.
\vs Pro 27:3 Тяжел камень, весок и песок; но гнев глупца тяжелее их обоих.
\vs Pro 27:4 Жесток гнев, неукротима ярость; но кто устоит против ревности?
\vs Pro 27:5 Лучше открытое обличение, нежели скрытая любовь.
\vs Pro 27:6 Искренни укоризны от любящего, и лживы поцелуи ненавидящего.
\vs Pro 27:7 Сытая душа попирает и сот, а голодной душе все горькое сладко.
\vs Pro 27:8 Как птица, покинувшая гнездо свое, так человек, покинувший место свое.
\vs Pro 27:9 Масть и курение радуют сердце; так сладок \bibemph{всякому} друг сердечным советом своим.
\vs Pro 27:10 Не покидай друга твоего и друга отца твоего, и в дом брата твоего не ходи в день несчастья твоего: лучше сосед вблизи, нежели брат вдали.
\rsbpar\vs Pro 27:11 Будь мудр, сын мой, и радуй сердце мое; и я буду иметь, что отвечать злословящему меня.
\vs Pro 27:12 Благоразумный видит беду и укрывается; а неопытные идут вперед \bibemph{и} наказываются.
\vs Pro 27:13 Возьми у него платье его, потому что он поручился за чужого, и за стороннего возьми от него залог.
\vs Pro 27:14 Кто громко хвалит друга своего с раннего утра, того сочтут за злословящего.
\vs Pro 27:15 Непрестанная капель в дождливый день и сварливая жена~--- равны:
\vs Pro 27:16 кто хочет скрыть ее, тот хочет скрыть ветер и масть в правой руке своей, дающую знать о себе.
\vs Pro 27:17 Железо железо острит, и человек изощряет взгляд друга своего.
\vs Pro 27:18 Кто стережет смоковницу, тот будет есть плоды ее; и кто бережет господина своего, тот будет в чести.
\vs Pro 27:19 Как в воде лицо~--- к лицу, так сердце человека~--- к человеку.
\vs Pro 27:20 Преисподняя и Аваддон~--- ненасытимы; так ненасытимы и глаза человеческие. [Мерзость пред Господом дерзко поднимающий глаза, и неразумны невоздержанные языком.]
\vs Pro 27:21 Что плавильня~--- для серебра, горнило~--- для золота, то для человека уста, которые хвалят его. [Сердце беззаконника ищет зла, сердце же правое ищет знания.]
\vs Pro 27:22 Толк\acc{и} глупого в ступе пестом вместе с зерном, не отделится от него глупость его.
\vs Pro 27:23 Хорошо наблюдай за скотом твоим, имей попечение о стадах;
\vs Pro 27:24 потому что \bibemph{богатство} не навек, да и власть разве из рода в род?
\vs Pro 27:25 Прозябает трава, и является зелень, и собирают горные травы.
\vs Pro 27:26 Овцы~--- на одежду тебе, и козлы~--- на покупку поля.
\vs Pro 27:27 И довольно козьего молока в пищу тебе, в пищу домашним твоим и на продовольствие служанкам твоим.
\vs Pro 28:1 Нечестивый бежит, когда никто не гонится \bibemph{за ним}; а праведник смел, как лев.
\vs Pro 28:2 Когда страна отступит от закона, тогда много в ней начальников; а при разумном и знающем муже она долговечна.
\vs Pro 28:3 Человек бедный и притесняющий слабых \bibemph{то же, что} проливной дождь, смывающий хлеб.
\vs Pro 28:4 Отступники от закона хвалят нечестивых, а соблюдающие закон негодуют на них.
\vs Pro 28:5 Злые люди не разумеют справедливости, а ищущие Господа разумеют всё.
\vs Pro 28:6 Лучше бедный, ходящий в своей непорочности, нежели тот, кто извращает пути свои, хотя он и богат.
\vs Pro 28:7 Хранящий закон~--- сын разумный, а знающийся с расточителями срамит отца своего.
\vs Pro 28:8 Умножающий имение свое ростом и лихвою соберет его для благотворителя бедных.
\vs Pro 28:9 Кто отклоняет ухо свое от слушания закона, того и молитва~--- мерзость.
\vs Pro 28:10 Совращающий праведных на путь зла сам упадет в свою яму, а непорочные наследуют добро.
\vs Pro 28:11 Человек богатый~--- мудрец в глазах своих, но умный бедняк обличит его.
\vs Pro 28:12 Когда торжествуют праведники, великая слава, но когда возвышаются нечестивые, люди укрываются.
\vs Pro 28:13 Скрывающий свои преступления не будет иметь успеха; а кто сознается и оставляет их, тот будет помилован.
\rsbpar\vs Pro 28:14 Блажен человек, который всегда пребывает в благоговении; а кто ожесточает сердце свое, тот попадет в беду.
\vs Pro 28:15 Как рыкающий лев и голодный медведь, так нечестивый властелин над бедным народом.
\vs Pro 28:16 Неразумный правитель много делает притеснений, а ненавидящий корысть продолжит дни.
\vs Pro 28:17 Человек, виновный в пролитии человеческой крови, будет бегать до могилы, чтобы кто не схватил его.
\vs Pro 28:18 Кто ходит непорочно, тот будет невредим; а ходящий кривыми путями упадет на одном из них.
\vs Pro 28:19 Кто возделывает землю свою, тот будет насыщаться хлебом, а кто подражает праздным, тот насытится нищетою.
\vs Pro 28:20 Верный человек богат благословениями, а кто спешит разбогатеть, тот не останется ненаказанным.
\vs Pro 28:21 Быть лицеприятным~--- нехорошо: такой человек и за кусок хлеба сделает неправду.
\vs Pro 28:22 Спешит к богатству завистливый человек, и не думает, что нищета постигнет его.
\vs Pro 28:23 Обличающий человека найдет после б\acc{о}льшую приязнь, нежели тот, кто льстит языком.
\vs Pro 28:24 Кто обкрадывает отца своего и мать свою и говорит: <<это не грех>>, тот~--- сообщник грабителям.
\vs Pro 28:25 Надменный разжигает ссору, а надеющийся на Господа будет благоденствовать.
\vs Pro 28:26 Кто надеется на себя, тот глуп; а кто ходит в мудрости, тот будет цел.
\vs Pro 28:27 Дающий нищему не обеднеет; а кто закрывает глаза свои от него, на том много проклятий.
\vs Pro 28:28 Когда возвышаются нечестивые, люди укрываются, а когда они падают, умножаются праведники.
\vs Pro 29:1 Человек, который, будучи обличаем, ожесточает выю свою, внезапно сокрушится, и не будет \bibemph{ему} исцеления.
\vs Pro 29:2 Когда умножаются праведники, веселится народ, а когда господствует нечестивый, народ стенает.
\vs Pro 29:3 Человек, любящий мудрость, радует отца своего; а кто знается с блудницами, тот расточает имение.
\vs Pro 29:4 Царь правосудием утверждает землю, а любящий подарки разоряет ее.
\vs Pro 29:5 Человек, льстящий другу своему, расстилает сеть ногам его.
\vs Pro 29:6 В грехе злого человека~--- сеть \bibemph{для него}, а праведник веселится и радуется.
\vs Pro 29:7 Праведник тщательно вникает в тяжбу бедных, а нечестивый не разбирает дела.
\vs Pro 29:8 Люди развратные возмущают город, а мудрые утишают мятеж.
\vs Pro 29:9 Умный человек, судясь с человеком глупым, сердится ли, смеется ли,~--- не имеет покоя.
\vs Pro 29:10 Кровожадные люди ненавидят непорочного, а праведные заботятся о его жизни.
\vs Pro 29:11 Глупый весь гнев свой изливает, а мудрый сдерживает его.
\vs Pro 29:12 Если правитель слушает ложные речи, то и все служащие у него нечестивы.
\vs Pro 29:13 Бедный и лихоимец встречаются друг с другом; но свет глазам того и другого дает Господь.
\vs Pro 29:14 Если царь судит бедных по правде, то престол его навсегда утвердится.
\vs Pro 29:15 Розга и обличение дают мудрость; но отрок, оставленный в небрежении, делает стыд своей матери.
\vs Pro 29:16 При умножении нечестивых умножается беззаконие; но праведники увидят падение их.
\vs Pro 29:17 Наказывай сына твоего, и он даст тебе покой, и доставит радость душе твоей.
\rsbpar\vs Pro 29:18 Без откровения свыше народ необуздан, а соблюдающий закон блажен.
\vs Pro 29:19 Словами не научится раб, потому что, хотя он понимает \bibemph{их}, но не слушается.
\vs Pro 29:20 Видал ли ты человека опрометчивого в словах своих? на глупого больше надежды, нежели на него.
\vs Pro 29:21 Если с детства воспитывать раба в неге, то впоследствии он захочет быть сыном.
\vs Pro 29:22 Человек гневливый заводит ссору, и вспыльчивый много грешит.
\vs Pro 29:23 Гордость человека унижает его, а смиренный духом приобретает честь.
\vs Pro 29:24 Кто делится с вором, тот ненавидит душу свою; слышит он проклятие, но не объявляет о том.
\vs Pro 29:25 Боязнь пред людьми ставит сеть; а надеющийся на Господа будет безопасен.
\vs Pro 29:26 Многие ищут \bibemph{благосклонного} лица правителя, но судьба человека~--- от Господа.
\vs Pro 29:27 Мерзость для праведников~--- человек неправедный, и мерзость для нечестивого~--- идущий прямым путем.
\vs Pro 30:1 Слова Агура, сына Иакеева. Вдохновенные изречения, \bibemph{которые} сказал этот человек Ифиилу, Ифиилу и Укалу:
\vs Pro 30:2 подлинно, я более невежда, нежели кто-либо из людей, и разума человеческого нет у меня,
\vs Pro 30:3 и не научился я мудрости, и познания святых не имею.
\vs Pro 30:4 Кто восходил на небо и нисходил? кто собрал ветер в пригоршни свои? кто завязал воду в одежду? кто поставил все пределы земли? какое имя ему? и какое имя сыну его? знаешь ли?
\rsbpar\vs Pro 30:5 Всякое слово Бога чисто; Он~--- щит уповающим на Него.
\vs Pro 30:6 Не прибавляй к словам Его, чтобы Он не обличил тебя, и ты не оказался лжецом.
\rsbpar\vs Pro 30:7 Двух вещей я прошу у Тебя, не откажи мне, прежде нежели я умру:
\vs Pro 30:8 суету и ложь удали от меня, нищеты и богатства не давай мне, питай меня насущным хлебом,
\vs Pro 30:9 дабы, пресытившись, я не отрекся \bibemph{Тебя} и не сказал: <<кто Господь?>> и чтобы, обеднев, не стал красть и употреблять имя Бога моего всуе.
\vs Pro 30:10 Не злословь раба пред господином его, чтобы он не проклял тебя, и ты не остался виноватым.
\vs Pro 30:11 Есть род, который проклинает отца своего и не благословляет матери своей.
\vs Pro 30:12 Есть род, который чист в глазах своих, тогда как не омыт от нечистот своих.
\vs Pro 30:13 Есть род~--- о, как высокомерны глаза его, и как подняты ресницы его!
\vs Pro 30:14 Есть род, у которого зубы~--- мечи, и челюсти~--- ножи, чтобы пожирать бедных на земле и нищих между людьми.
\vs Pro 30:15 У ненасытимости две дочери: <<давай, давай!>> Вот три ненасытимых, и четыре, которые не скажут: <<довольно!>>
\vs Pro 30:16 Преисподняя и утроба бесплодная, земля, которая не насыщается водою, и огонь, который не говорит: <<довольно!>>
\vs Pro 30:17 Глаз, насмехающийся над отцом и пренебрегающий покорностью к матери, выклюют в\acc{о}роны дольные, и сожрут птенцы орлиные!
\vs Pro 30:18 Три вещи непостижимы для меня, и четырех я не понимаю:
\vs Pro 30:19 пути орла на небе, пути змея на скале, пути корабля среди моря и пути мужчины к девице.
\vs Pro 30:20 Таков путь и жены прелюбодейной; поела и обтерла рот свой, и говорит: <<я ничего худого не сделала>>.
\vs Pro 30:21 От трех трясется земля, четырех она не может носить:
\vs Pro 30:22 раба, когда он делается царем; глупого, когда он досыта ест хлеб;
\vs Pro 30:23 позорную женщину, когда она выходит замуж, и служанку, когда она занимает место госпожи своей.
\vs Pro 30:24 Вот четыре малых на земле, но они мудрее мудрых:
\vs Pro 30:25 муравьи~--- народ не сильный, но летом заготовляют пищу свою;
\vs Pro 30:26 горные мыши~--- народ слабый, но ставят домы свои на скале;
\vs Pro 30:27 у саранчи нет царя, но выступает вся она стройно;
\vs Pro 30:28 паук лапками цепляется, но бывает в царских чертогах.
\vs Pro 30:29 Вот трое имеют стройную походку, и четверо стройно выступают:
\vs Pro 30:30 лев, силач между зверями, не посторонится ни перед кем;
\vs Pro 30:31 конь и козел, [предводитель стада,] и царь среди народа своего.
\vs Pro 30:32 Если ты в заносчивости своей сделал глупость и помыслил злое, то \bibemph{положи} руку на уста;
\vs Pro 30:33 потому что, как сбивание молока производит масло, толчок в нос производит кровь, так и возбуждение гнева производит ссору.
\vs Pro 31:1 Слова Лемуила царя. Наставление, которое преподала ему мать его:
\vs Pro 31:2 что, сын мой? что, сын чрева моего? что, сын обетов моих?
\vs Pro 31:3 Не отдавай женщинам сил твоих, ни путей твоих губительницам царей.
\vs Pro 31:4 Не царям, Лемуил, не царям пить вино, и не князьям~--- сикеру,
\vs Pro 31:5 чтобы, напившись, они не забыли закона и не превратили суда всех угнетаемых.
\vs Pro 31:6 Дайте сикеру погибающему и вино огорченному душею;
\vs Pro 31:7 пусть он выпьет и забудет бедность свою и не вспомнит больше о своем страдании.
\vs Pro 31:8 Открывай уста твои за безгласного и для защиты всех сирот.
\vs Pro 31:9 Открывай уста твои для правосудия и для дела бедного и нищего.
\rsbpar\vs Pro 31:10 Кто найдет добродетельную жену? цена ее выше жемчугов;
\vs Pro 31:11 уверено в ней сердце мужа ее, и он не останется без прибытка;
\vs Pro 31:12 она воздает ему добром, а не злом, во все дни жизни своей.
\vs Pro 31:13 Добывает шерсть и лен, и с охотою работает своими руками.
\vs Pro 31:14 Она, как купеческие корабли, издалека добывает хлеб свой.
\vs Pro 31:15 Она встает еще ночью и раздает пищу в доме своем и урочное служанкам своим.
\vs Pro 31:16 Задумает она о поле, и приобретает его; от плодов рук своих насаждает виноградник.
\vs Pro 31:17 Препоясывает силою чресла свои и укрепляет мышцы свои.
\vs Pro 31:18 Она чувствует, что занятие ее хорошо, и~--- светильник ее не гаснет и ночью.
\vs Pro 31:19 Протягивает руки свои к прялке, и персты ее берутся за веретено.
\vs Pro 31:20 Длань свою она открывает бедному, и руку свою подает нуждающемуся.
\vs Pro 31:21 Не боится стужи для семьи своей, потому что вся семья ее одета в двойные одежды.
\vs Pro 31:22 Она делает себе ковры; виссон и пурпур~--- одежда ее.
\vs Pro 31:23 Муж ее известен у ворот, когда сидит со старейшинами земли.
\vs Pro 31:24 Она делает покрывала и продает, и поясы доставляет купцам Финикийским.
\vs Pro 31:25 Крепость и красота~--- одежда ее, и весело смотрит она на будущее.
\vs Pro 31:26 Уста свои открывает с мудростью, и кроткое наставление на языке ее.
\vs Pro 31:27 Она наблюдает за хозяйством в доме своем и не ест хлеба праздности.
\vs Pro 31:28 Встают дети и ублажают ее,~--- муж, и хвалит ее:
\vs Pro 31:29 <<много было жен добродетельных, но ты превзошла всех их>>.
\vs Pro 31:30 Миловидность обманчива и красота суетна; но жена, боящаяся Господа, достойна хвалы.
\vs Pro 31:31 Дайте ей от плода рук ее, и да прославят ее у ворот дел\acc{а} ее!

\bibbookdescr{Ecc}{
  inline={\LARGE Книга\\\Huge Екклесиаста\\или Проповедника},
  toc={Екклесиаст},
  bookmark={Екклесиаст},
  header={Екклесиаст},
  %headerleft={},
  %headerright={},
  abbr={Еккл}
}
\vs Ecc 1:1 Слова Екклесиаста, сына Давидова, царя в Иерусалиме.
\rsbpar\vs Ecc 1:2 Суета сует, сказал Екклесиаст, суета сует,~--- всё суета!
\vs Ecc 1:3 Что пользы человеку от всех трудов его, которыми трудится он под солнцем?
\vs Ecc 1:4 Род проходит, и род приходит, а земля пребывает во веки.
\vs Ecc 1:5 Восходит солнце, и заходит солнце, и спешит к месту своему, где оно восходит.
\vs Ecc 1:6 Идет ветер к югу, и переходит к северу, кружится, кружится на ходу своем, и возвращается ветер на круги свои.
\vs Ecc 1:7 Все реки текут в море, но море не переполняется: к тому месту, откуда реки текут, они возвращаются, чтобы опять течь.
\vs Ecc 1:8 Все вещи~--- в труде: не может человек пересказать всего; не насытится око зрением, не наполнится ухо слушанием.
\vs Ecc 1:9 Что было, то и будет; и что делалось, то и будет делаться, и нет ничего нового под солнцем.
\vs Ecc 1:10 Бывает нечто, о чем говорят: <<смотри, вот это новое>>; но \bibemph{это} было уже в веках, бывших прежде нас.
\vs Ecc 1:11 Нет памяти о прежнем; да и о том, что будет, не останется памяти у тех, которые будут после.
\rsbpar\vs Ecc 1:12 Я, Екклесиаст, был царем над Израилем в Иерусалиме;
\vs Ecc 1:13 и предал я сердце мое тому, чтобы исследовать и испытать мудростью все, что делается под небом: это тяжелое занятие дал Бог сынам человеческим, чтобы они упражнялись в нем.
\vs Ecc 1:14 Видел я все дела, какие делаются под солнцем, и вот, всё~--- суета и томление духа!
\vs Ecc 1:15 Кривое не может сделаться прямым, и чего нет, того нельзя считать.
\vs Ecc 1:16 Говорил я с сердцем моим так: вот, я возвеличился и приобрел мудрости больше всех, которые были прежде меня над Иерусалимом, и сердце мое видело много мудрости и знания.
\vs Ecc 1:17 И предал я сердце мое тому, чтобы познать мудрость и познать безумие и глупость: узнал, что и это~--- томление духа;
\vs Ecc 1:18 потому что во многой мудрости много печали; и кто умножает познания, умножает скорбь.
\vs Ecc 2:1 Сказал я в сердце моем: <<дай, испытаю я тебя весельем, и насладись добром>>; но и это~--- суета!
\vs Ecc 2:2 О смехе сказал я: <<глупость!>>, а о веселье: <<что оно делает?>>
\vs Ecc 2:3 Вздумал я в сердце моем услаждать вином тело мое и, между тем, как сердце мое руководилось мудростью, придержаться и глупости, доколе не увижу, что хорошо для сынов человеческих, что должны были бы они делать под небом в немногие дни жизни своей.
\rsbpar\vs Ecc 2:4 Я предпринял большие дела: построил себе домы, посадил себе виноградники,
\vs Ecc 2:5 устроил себе сады и рощи и насадил в них всякие плодовитые дерева;
\vs Ecc 2:6 сделал себе водоемы для орошения из них рощей, произращающих деревья;
\vs Ecc 2:7 приобрел себе слуг и служанок, и домочадцы были у меня; также крупного и мелкого скота было у меня больше, нежели у всех, бывших прежде меня в Иерусалиме;
\vs Ecc 2:8 собрал себе серебра и золота и драгоценностей от царей и областей; завел у себя певцов и певиц и услаждения сынов человеческих~--- разные музыкальные орудия.
\vs Ecc 2:9 И сделался я великим и богатым больше всех, бывших прежде меня в Иерусалиме; и мудрость моя пребыла со мною.
\vs Ecc 2:10 Чего бы глаза мои ни пожелали, я не отказывал им, не возбранял сердцу моему никакого веселья, потому что сердце мое радовалось во всех трудах моих, и это было моею долею от всех трудов моих.
\vs Ecc 2:11 И оглянулся я на все дела мои, которые сделали руки мои, и на труд, которым трудился я, делая \bibemph{их}: и вот, всё~--- суета и томление духа, и нет \bibemph{от них} пользы под солнцем!
\rsbpar\vs Ecc 2:12 И обратился я, чтобы взглянуть на мудрость и безумие и глупость: ибо что \bibemph{может сделать} человек после царя \bibemph{сверх того}, что уже сделано?
\vs Ecc 2:13 И увидел я, что преимущество мудрости перед глупостью такое же, как преимущество света перед тьмою:
\vs Ecc 2:14 у мудрого глаза его~--- в голове его, а глупый ходит во тьме; но узнал я, что одна участь постигает их всех.
\vs Ecc 2:15 И сказал я в сердце моем: <<и меня постигнет та же участь, как и глупого: к чему же я сделался очень мудрым?>> И сказал я в сердце моем, что и это~--- суета;
\vs Ecc 2:16 потому что мудрого не будут помнить вечно, как и глупого; в грядущие дни все будет забыто, и увы! мудрый умирает наравне с глупым.
\vs Ecc 2:17 И возненавидел я жизнь, потому что противны стали мне дела, которые делаются под солнцем; ибо всё~--- суета и томление духа!
\vs Ecc 2:18 И возненавидел я весь труд мой, которым трудился под солнцем, потому что должен оставить его человеку, который будет после меня.
\vs Ecc 2:19 И кто знает: мудрый ли будет он, или глупый? А он будет распоряжаться всем трудом моим, которым я трудился и которым показал себя мудрым под солнцем. И это~--- суета!
\vs Ecc 2:20 И обратился я, чтобы внушить сердцу моему отречься от всего труда, которым я трудился под солнцем,
\vs Ecc 2:21 потому что иной человек трудится мудро, с знанием и успехом, и должен отдать всё человеку, не трудившемуся в том, как бы часть его. И это~--- суета и зло великое!
\vs Ecc 2:22 Ибо что будет иметь человек от всего труда своего и заботы сердца своего, что трудится он под солнцем?
\vs Ecc 2:23 Потому что все дни его~--- скорби, и его труды~--- беспокойство; даже и ночью сердце его не знает покоя. И это~--- суета!
\rsbpar\vs Ecc 2:24 Не во власти человека и то благо, чтобы есть и пить и услаждать душу свою от труда своего. Я увидел, что и это~--- от руки Божией;
\vs Ecc 2:25 потому что кто может есть и кто может наслаждаться без Него?
\vs Ecc 2:26 Ибо человеку, который добр пред лицем Его, Он дает мудрость и знание и радость; а грешнику дает заботу собирать и копить, чтобы \bibemph{после} отдать доброму пред лицем Божиим. И это~--- суета и томление духа!
\vs Ecc 3:1 Всему свое время, и время всякой вещи под небом:
\vs Ecc 3:2 время рождаться, и время умирать; время насаждать, и время вырывать посаженное;
\vs Ecc 3:3 время убивать, и время врачевать; время разрушать, и время строить;
\vs Ecc 3:4 время плакать, и время смеяться; время сетовать, и время плясать;
\vs Ecc 3:5 время разбрасывать камни, и время собирать камни; время обнимать, и время уклоняться от объятий;
\vs Ecc 3:6 время искать, и время терять; время сберегать, и время бросать;
\vs Ecc 3:7 время раздирать, и время сшивать; время молчать, и время говорить;
\vs Ecc 3:8 время любить, и время ненавидеть; время войне, и время миру.
\rsbpar\vs Ecc 3:9 Что пользы работающему от того, над чем он трудится?
\vs Ecc 3:10 Видел я эту заботу, которую дал Бог сынам человеческим, чтобы они упражнялись в том.
\vs Ecc 3:11 Всё соделал Он прекрасным в свое время, и вложил мир в сердце их, хотя человек не может постигнуть дел, которые Бог делает, от начала до конца.
\vs Ecc 3:12 Познал я, что нет для них ничего лучшего, как веселиться и делать доброе в жизни своей.
\vs Ecc 3:13 И если какой человек ест и пьет, и видит доброе во всяком труде своем, то это~--- дар Божий.
\vs Ecc 3:14 Познал я, что всё, что делает Бог, пребывает вовек: к тому нечего прибавлять и от того нечего убавить,~--- и Бог делает так, чтобы благоговели пред лицем Его.
\vs Ecc 3:15 Что было, то и теперь есть, и что будет, то уже было,~--- и Бог воззовет прошедшее.
\rsbpar\vs Ecc 3:16 Еще видел я под солнцем: место суда, а там беззаконие; место правды, а там неправда.
\vs Ecc 3:17 И сказал я в сердце своем: <<праведного и нечестивого будет судить Бог; потому что время для всякой вещи и \bibemph{суд} над всяким делом там>>.
\rsbpar\vs Ecc 3:18 Сказал я в сердце своем о сынах человеческих, чтобы испытал их Бог, и чтобы они видели, что они сами по себе животные;
\vs Ecc 3:19 потому что участь сынов человеческих и участь животных~--- участь одна: как те умирают, так умирают и эти, и одно дыхание у всех, и нет у человека преимущества перед скотом, потому что всё~--- суета!
\vs Ecc 3:20 Все идет в одно место: все произошло из праха и все возвратится в прах.
\vs Ecc 3:21 Кто знает: дух сынов человеческих восходит ли вверх, и дух животных сходит ли вниз, в землю?
\rsbpar\vs Ecc 3:22 Итак увидел я, что нет ничего лучше, как наслаждаться человеку делами своими: потому что это~--- доля его; ибо кто приведет его посмотреть на то, что будет после него?
\vs Ecc 4:1 И обратился я и увидел всякие угнетения, какие делаются под солнцем: и вот слезы угнетенных, а утешителя у них нет; и в руке угнетающих их~--- сила, а утешителя у них нет.
\vs Ecc 4:2 И ублажил я мертвых, которые давно умерли, более живых, которые живут доселе;
\vs Ecc 4:3 а блаженнее их обоих тот, кто еще не существовал, кто не видал злых дел, какие делаются под солнцем.
\rsbpar\vs Ecc 4:4 Видел я также, что всякий труд и всякий успех в делах производят взаимную между людьми зависть. И это~--- суета и томление духа!
\vs Ecc 4:5 Глупый \bibemph{сидит}, сложив свои руки, и съедает плоть свою.
\vs Ecc 4:6 Лучше горсть с покоем, нежели пригоршни с трудом и томлением духа.
\rsbpar\vs Ecc 4:7 И обратился я и увидел еще суету под солнцем;
\vs Ecc 4:8 \bibemph{человек} одинокий, и другого нет; ни сына, ни брата нет у него; а всем трудам его нет конца, и глаз его не насыщается богатством. <<Для кого же я тружусь и лишаю душу мою блага?>> И это~--- суета и недоброе дело!
\rsbpar\vs Ecc 4:9 Двоим лучше, нежели одному; потому что у них есть доброе вознаграждение в труде их:
\vs Ecc 4:10 ибо если упадет один, то другой поднимет товарища своего. Но горе одному, когда упадет, а другого нет, который поднял бы его.
\vs Ecc 4:11 Также, если лежат двое, то тепло им; а одному как согреться?
\vs Ecc 4:12 И если станет преодолевать кто-либо одного, то двое устоят против него: и нитка, втрое скрученная, нескоро порвется.
\rsbpar\vs Ecc 4:13 Лучше бедный, но умный юноша, нежели старый, но неразумный царь, который не умеет принимать советы;
\vs Ecc 4:14 ибо тот из темницы выйдет на царство, хотя родился в царстве своем бедным.
\vs Ecc 4:15 Видел я всех живущих, которые ходят под солнцем, с этим другим юношею, который займет место того.
\vs Ecc 4:16 Не было числа всему народу, который был перед ним, хотя позднейшие не порадуются им. И это~--- суета и томление духа!
\rsbpar\vs Ecc 4:17 Наблюдай за ногою твоею, когда идешь в дом Божий, и будь готов более к слушанию, нежели к жертвоприношению; ибо они не думают, что худо делают.
\vs Ecc 5:1 Не торопись языком твоим, и сердце твое да не спешит произнести слово пред Богом; потому что Бог на небе, а ты на земле; поэтому слова твои да будут немноги.
\vs Ecc 5:2 Ибо, как сновидения бывают при множестве забот, так голос глупого познается при множестве слов.
\rsbpar\vs Ecc 5:3 Когда даешь обет Богу, то не медли исполнить его, потому что Он не благоволит к глупым: что обещал, исполни.
\vs Ecc 5:4 Лучше тебе не обещать, нежели обещать и не исполнить.
\vs Ecc 5:5 Не дозволяй устам твоим вводить в грех плоть твою, и не говори пред Ангелом [Божиим]: <<это~--- ошибка!>> Для чего тебе \bibemph{делать}, чтобы Бог прогневался на слово твое и разрушил дело рук твоих?
\vs Ecc 5:6 Ибо во множестве сновидений, как и во множестве слов,~--- много суеты; но ты бойся Бога.
\rsbpar\vs Ecc 5:7 Если ты увидишь в какой области притеснение бедному и нарушение суда и правды, то не удивляйся этому: потому что над высоким наблюдает высший, а над ними еще высший;
\vs Ecc 5:8 превосходство же страны в целом есть царь, заботящийся о стране.
\vs Ecc 5:9 Кто любит серебро, тот не насытится серебром, и кто любит богатство, тому нет пользы от того. И это~--- суета!
\vs Ecc 5:10 Умножается имущество, умножаются и потребляющие его; и какое благо для владеющего им: разве только смотреть своими глазами?
\vs Ecc 5:11 Сладок сон трудящегося, мало ли, много ли он съест; но пресыщение богатого не дает ему уснуть.
\vs Ecc 5:12 Есть мучительный недуг, который видел я под солнцем: богатство, сберегаемое владетелем его во вред ему.
\vs Ecc 5:13 И гибнет богатство это от несчастных случаев: родил он сына, и ничего нет в руках у него.
\vs Ecc 5:14 Как вышел он нагим из утробы матери своей, таким и отходит, каким пришел, и ничего не возьмет от труда своего, что мог бы он понести в руке своей.
\vs Ecc 5:15 И это тяжкий недуг: каким пришел он, таким и отходит. Какая же польза ему, что он трудился на ветер?
\vs Ecc 5:16 А он во все дни свои ел впотьмах, в большом раздражении, в огорчении и досаде.
\rsbpar\vs Ecc 5:17 Вот еще, что я нашел доброго и приятного: есть и пить и наслаждаться добром во всех трудах своих, какими кто трудится под солнцем во все дни жизни своей, которые дал ему Бог; потому что это его доля.
\vs Ecc 5:18 И если какому человеку Бог дал богатство и имущество, и дал ему власть пользоваться от них и брать свою долю и наслаждаться от трудов своих, то это дар Божий.
\vs Ecc 5:19 Недолго будут у него в памяти дни жизни его; поэтому Бог и вознаграждает его радостью сердца его.
\vs Ecc 6:1 Есть зло, которое видел я под солнцем, и оно часто бывает между людьми:
\vs Ecc 6:2 Бог дает человеку богатство и имущество и славу, и нет для души его недостатка ни в чем, чего не пожелал бы он; но не дает ему Бог пользоваться этим, а пользуется тем чужой человек: это~--- суета и тяжкий недуг!
\vs Ecc 6:3 Если бы какой человек родил сто \bibemph{детей}, и прожил многие годы, и еще умножились дни жизни его, но душа его не наслаждалась бы добром и не было бы ему и погребения, то я сказал бы: выкидыш счастливее его,
\vs Ecc 6:4 потому что он напрасно пришел и отошел во тьму, и его имя покрыто мраком.
\vs Ecc 6:5 Он даже не видал и не знал солнца: ему покойнее, нежели тому.
\vs Ecc 6:6 А тот, хотя бы прожил две тысячи лет и не наслаждался добром, не все ли пойдет в одно место?
\vs Ecc 6:7 Все труды человека~--- для рта его, а душа его не насыщается.
\vs Ecc 6:8 Какое же преимущество мудрого перед глупым, какое~--- бедняка, умеющего ходить перед живущими?
\vs Ecc 6:9 Лучше видеть глазами, нежели бродить душею. И это~--- также суета и томление духа!
\vs Ecc 6:10 Что существует, тому уже наречено имя, и известно, что это~--- человек, и что он не может препираться с тем, кто сильнее его.
\vs Ecc 6:11 Много таких вещей, которые умножают суету: что же для человека лучше?
\vs Ecc 6:12 Ибо кто знает, что хорошо для человека в жизни, во все дни суетной жизни его, которые он проводит как тень? И кто скажет человеку, что будет после него под солнцем?
\vs Ecc 7:1 Доброе имя лучше дорогой масти, и день смерти~--- дня рождения.
\vs Ecc 7:2 Лучше ходить в дом плача об умершем, нежели ходить в дом пира; ибо таков конец всякого человека, и живой приложит \bibemph{это} к своему сердцу.
\vs Ecc 7:3 Сетование лучше смеха; потому что при печали лица сердце делается лучше.
\vs Ecc 7:4 Сердце мудрых~--- в доме плача, а сердце глупых~--- в доме веселья.
\vs Ecc 7:5 Лучше слушать обличения от мудрого, нежели слушать песни глупых;
\vs Ecc 7:6 потому что смех глупых то же, что треск тернового хвороста под котлом. И это~--- суета!
\rsbpar\vs Ecc 7:7 Притесняя других, мудрый делается глупым, и подарки портят сердце.
\vs Ecc 7:8 Конец дела лучше начала его; терпеливый лучше высокомерного.
\vs Ecc 7:9 Не будь духом твоим поспешен на гнев, потому что гнев гнездится в сердце глупых.
\vs Ecc 7:10 Не говори: <<отчего это прежние дни были лучше нынешних?>>, потому что не от мудрости ты спрашиваешь об этом.
\vs Ecc 7:11 Хороша мудрость с наследством, и особенно для видящих солнце:
\vs Ecc 7:12 потому что под сенью ее \bibemph{то же, что} под сенью серебра; но превосходство знания в \bibemph{том, что} мудрость дает жизнь владеющему ею.
\vs Ecc 7:13 Смотри на действование Божие: ибо кто может выпрямить то, что Он сделал кривым?
\vs Ecc 7:14 Во дни благополучия пользуйся благом, а во дни несчастья размышляй: то и другое соделал Бог для того, чтобы человек ничего не мог сказать против Него.
\rsbpar\vs Ecc 7:15 Всего насмотрелся я в суетные дни мои: праведник гибнет в праведности своей; нечестивый живет долго в нечестии своем.
\vs Ecc 7:16 Не будь слишком строг, и не выставляй себя слишком мудрым; зачем тебе губить себя?
\vs Ecc 7:17 Не предавайся греху, и не будь безумен: зачем тебе умирать не в свое время?
\vs Ecc 7:18 Хорошо, если ты будешь держаться одного и не отнимать руки от другого; потому что кто боится Бога, тот избежит всего того.
\vs Ecc 7:19 Мудрость делает мудрого сильнее десяти властителей, которые в городе.
\rsbpar\vs Ecc 7:20 Нет человека праведного на земле, который делал бы добро и не грешил бы;
\vs Ecc 7:21 поэтому не на всякое слово, которое говорят, обращай внимание, чтобы не услышать тебе раба твоего, когда он злословит тебя;
\vs Ecc 7:22 ибо сердце твое знает много случаев, когда и сам ты злословил других.
\rsbpar\vs Ecc 7:23 Все это испытал я мудростью; я сказал: <<буду я мудрым>>; но мудрость далека от меня.
\vs Ecc 7:24 Далеко то, что было, и глубоко~--- глубоко: кто постигнет его?
\vs Ecc 7:25 Обратился я сердцем моим к тому, чтобы узнать, исследовать и изыскать мудрость и разум, и познать нечестие глупости, невежества и безумия,~---
\vs Ecc 7:26 и нашел я, что горче смерти женщина, потому что она~--- сеть, и сердце ее~--- силки, руки ее~--- оковы; добрый пред Богом спасется от нее, а грешник уловлен будет ею.
\vs Ecc 7:27 Вот это нашел я, сказал Екклесиаст, испытывая одно за другим.
\vs Ecc 7:28 Чего еще искала душа моя, и я не нашел?~--- Мужчину одного из тысячи я нашел, а женщины между всеми ими не нашел.
\vs Ecc 7:29 Только это я нашел, что Бог сотворил человека правым, а люди пустились во многие помыслы.
\vs Ecc 8:1 Кто~--- как мудрый, и кто понимает значение вещей? Мудрость человека просветляет лице его, и суровость лица его изменяется.
\vs Ecc 8:2 \bibemph{Я говорю}: слово царское храни, и \bibemph{это} ради клятвы пред Богом.
\vs Ecc 8:3 Не спеши уходить от лица его, и не упорствуй в худом деле; потому что он, что захочет, все может сделать.
\vs Ecc 8:4 Где слово царя, там власть; и кто скажет ему: <<что ты делаешь?>>
\rsbpar\vs Ecc 8:5 Соблюдающий заповедь не испытает никакого зла: сердце мудрого знает и время и устав;
\vs Ecc 8:6 потому что для всякой вещи есть свое время и устав; а человеку великое зло оттого,
\vs Ecc 8:7 что он не знает, что будет; и как это будет~--- кто скажет ему?
\rsbpar\vs Ecc 8:8 Человек не властен над духом, чтобы удержать дух, и нет власти у него над днем смерти, и нет избавления в этой борьбе, и не спасет нечестие нечестивого.
\vs Ecc 8:9 Все это я видел, и обращал сердце мое на всякое дело, какое делается под солнцем. Бывает время, когда человек властвует над человеком во вред ему.
\vs Ecc 8:10 Видел я тогда, что хоронили нечестивых, и приходили и отходили от святого места, и они забываемы были в городе, где они так поступали. И это~--- суета!
\vs Ecc 8:11 Не скоро совершается суд над худыми делами; от этого и не страшится сердце сынов человеческих делать зло.
\vs Ecc 8:12 Хотя грешник сто раз делает зло и коснеет в нем, но я знаю, что благо будет боящимся Бога, которые благоговеют пред лицем Его;
\vs Ecc 8:13 а нечестивому не будет добра, и, подобно тени, недолго продержится тот, кто не благоговеет пред Богом.
\vs Ecc 8:14 Есть и такая суета на земле: праведников постигает то, чего заслуживали бы дела нечестивых, а с нечестивыми бывает то, чего заслуживали бы дела праведников. И сказал я: и это~--- суета!
\vs Ecc 8:15 И похвалил я веселье; потому что нет лучшего для человека под солнцем, как есть, пить и веселиться: это сопровождает его в трудах во дни жизни его, которые дал ему Бог под солнцем.
\rsbpar\vs Ecc 8:16 Когда я обратил сердце мое на то, чтобы постигнуть мудрость и обозреть дела, которые делаются на земле, и среди которых \bibemph{человек} ни днем, ни ночью не знает сна,~---
\vs Ecc 8:17 тогда я увидел все дела Божии и \bibemph{нашел}, что человек не может постигнуть дел, которые делаются под солнцем. Сколько бы человек ни трудился в исследовании, он все-таки не постигнет этого; и если бы какой мудрец сказал, что он знает, он не может постигнуть \bibemph{этого}.
\vs Ecc 9:1 На все это я обратил сердце мое для исследования, что праведные и мудрые и деяния их~--- в руке Божией, и что человек ни любви, ни ненависти не знает во всем том, что перед ним.
\vs Ecc 9:2 Всему и всем~--- одно: одна участь праведнику и нечестивому, доброму и [злому], чистому и нечистому, приносящему жертву и не приносящему жертвы; как добродетельному, так и грешнику; как клянущемуся, так и боящемуся клятвы.
\vs Ecc 9:3 Это-то и худо во всем, что делается под солнцем, что одна участь всем, и сердце сынов человеческих исполнено зла, и безумие в сердце их, в жизни их; а после того они \bibemph{отходят} к умершим.
\vs Ecc 9:4 Кто находится между живыми, тому есть еще надежда, так как и псу живому лучше, нежели мертвому льву.
\vs Ecc 9:5 Живые знают, что умрут, а мертвые ничего не знают, и уже нет им воздаяния, потому что и память о них предана забвению,
\vs Ecc 9:6 и любовь их и ненависть их и ревность их уже исчезли, и нет им более части во веки ни в чем, что делается под солнцем.
\vs Ecc 9:7 \bibemph{Итак} иди, ешь с весельем хлеб твой, и пей в радости сердца вино твое, когда Бог благоволит к делам твоим.
\vs Ecc 9:8 Да будут во всякое время одежды твои светлы, и да не оскудевает елей на голове твоей.
\vs Ecc 9:9 Наслаждайся жизнью с женою, которую любишь, во все дни суетной жизни твоей, и которую дал тебе Бог под солнцем на все суетные дни твои; потому что это~--- доля твоя в жизни и в трудах твоих, какими ты трудишься под солнцем.
\vs Ecc 9:10 Все, что может рука твоя делать, по силам делай; потому что в могиле, куда ты пойдешь, нет ни работы, ни размышления, ни знания, ни мудрости.
\rsbpar\vs Ecc 9:11 И обратился я, и видел под солнцем, что не проворным достается успешный бег, не храбрым~--- победа, не мудрым~--- хлеб, и не у разумных~--- богатство, и не искусным~--- благорасположение, но время и случай для всех их.
\vs Ecc 9:12 Ибо человек не знает своего времени. Как рыбы попадаются в пагубную сеть, и как птицы запутываются в силках, так сыны человеческие уловляются в бедственное время, когда оно неожиданно находит на них.
\rsbpar\vs Ecc 9:13 Вот еще какую мудрость видел я под солнцем, и она показалась мне важною:
\vs Ecc 9:14 город небольшой, и людей в нем немного; к нему подступил великий царь и обложил его и произвел против него большие осадные работы;
\vs Ecc 9:15 но в нем нашелся мудрый бедняк, и он спас своею мудростью этот город; и однако же никто не вспоминал об этом бедном человеке.
\vs Ecc 9:16 И сказал я: мудрость лучше силы, и однако же мудрость бедняка пренебрегается, и слов его не слушают.
\vs Ecc 9:17 Слова мудрых, \bibemph{высказанные} спокойно, выслушиваются \bibemph{лучше}, нежели крик властелина между глупыми.
\vs Ecc 9:18 Мудрость лучше воинских орудий; но один погрешивший погубит много доброго.
\vs Ecc 10:1 Мертвые мухи портят и делают зловонною благовонную масть мироварника: то же делает небольшая глупость уважаемого человека с его мудростью и честью.
\vs Ecc 10:2 Сердце мудрого~--- на правую сторону, а сердце глупого~--- на левую.
\vs Ecc 10:3 По какой бы дороге ни шел глупый, у него \bibemph{всегда} недостает смысла, и всякому он выскажет, что он глуп.
\vs Ecc 10:4 Если гнев начальника вспыхнет на тебя, то не оставляй места твоего; потому что кротость покрывает и большие проступки.
\rsbpar\vs Ecc 10:5 Есть зло, которое видел я под солнцем, это~--- как бы погрешность, происходящая от властелина:
\vs Ecc 10:6 невежество поставляется на большой высоте, а богатые сидят низко.
\vs Ecc 10:7 Видел я рабов на конях, а князей ходящих, подобно рабам, пешком.
\vs Ecc 10:8 Кто копает яму, тот упадет в нее, и кто разрушает ограду, того ужалит змей.
\vs Ecc 10:9 Кто передвигает камни, тот может надсадить себя, и кто колет дрова, тот может подвергнуться опасности от них.
\vs Ecc 10:10 Если притупится топор, и если лезвие его не будет отточено, то надобно будет напрягать силы; мудрость умеет это исправить.
\vs Ecc 10:11 Если змей ужалит без заговаривания, то не лучше его и злоязычный.
\vs Ecc 10:12 Слова из уст мудрого~--- благодать, а уста глупого губят его же:
\vs Ecc 10:13 начало слов из уст его~--- глупость, \bibemph{а} конец речи из уст его~--- безумие.
\vs Ecc 10:14 Глупый наговорит много, \bibemph{хотя} человек не знает, что будет, и кто скажет ему, что будет после него?
\vs Ecc 10:15 Труд глупого утомляет его, потому что не знает \bibemph{даже} дороги в город.
\vs Ecc 10:16 Горе тебе, земля, когда царь твой отрок, и когда князья твои едят рано!
\vs Ecc 10:17 Благо тебе, земля, когда царь у тебя из благородного рода, и князья твои едят вовремя, для подкрепления, а не для пресыщения!
\vs Ecc 10:18 От лености обвиснет потолок, и когда опустятся руки, то протечет дом.
\vs Ecc 10:19 Пиры устраиваются для удовольствия, и вино веселит жизнь; а за все отвечает серебро.
\vs Ecc 10:20 Даже и в мыслях твоих не злословь царя, и в спальной комнате твоей не злословь богатого; потому что птица небесная может перенести слово \bibemph{твое}, и крылатая~--- пересказать речь \bibemph{твою}.
\vs Ecc 11:1 Отпускай хлеб твой по водам, потому что по прошествии многих дней опять найдешь его.
\vs Ecc 11:2 Давай часть семи и даже восьми, потому что не знаешь, какая беда будет на земле.
\vs Ecc 11:3 Когда облака будут полны, то они прольют на землю дождь; и если упадет дерево на юг или на север, то оно там и останется, куда упадет.
\vs Ecc 11:4 Кто наблюдает ветер, тому не сеять; и кто смотрит на облака, тому не жать.
\vs Ecc 11:5 Как ты не знаешь путей ветра и того, как \bibemph{образуются} кости во чреве беременной, так не можешь знать дело Бога, Который делает все.
\vs Ecc 11:6 Утром сей семя твое, и вечером не давай отдыха руке твоей, потому что ты не знаешь, то или другое будет удачнее, или то и другое равно хорошо будет.
\vs Ecc 11:7 Сладок свет, и приятно для глаз видеть солнце.
\vs Ecc 11:8 Если человек проживет \bibemph{и} много лет, то пусть веселится он в продолжение всех их, и пусть помнит о днях темных, которых будет много: все, что будет,~--- суета!
\vs Ecc 11:9 Веселись, юноша, в юности твоей, и да вкушает сердце твое радости во дни юности твоей, и ходи по путям сердца твоего и по видению очей твоих; только знай, что за все это Бог приведет тебя на суд.
\vs Ecc 11:10 И удаляй печаль от сердца твоего, и уклоняй злое от тела твоего, потому что детство и юность~--- суета.
\vs Ecc 12:1 И помни Создателя твоего в дни юности твоей, доколе не пришли тяжелые дни и не наступили годы, о которых ты будешь говорить: <<нет мне удовольствия в них!>>
\vs Ecc 12:2 доколе не померкли солнце и свет и луна и звезды, и не нашли новые тучи вслед за дождем.
\vs Ecc 12:3 В тот день, когда задрожат стерегущие дом и согнутся мужи силы; и перестанут молоть мелющие, потому что их немного осталось; и помрачатся смотрящие в окно;
\vs Ecc 12:4 и запираться будут двери на улицу; когда замолкнет звук жернова, и будет вставать \bibemph{человек} по крику петуха и замолкнут дщери пения;
\vs Ecc 12:5 и высоты будут им страшны, и на дороге ужасы; и зацветет миндаль, и отяжелеет кузнечик, и рассыплется каперс. Ибо отходит человек в вечный дом свой, и готовы окружить его по улице плакальщицы;~---
\vs Ecc 12:6 доколе не порвалась серебряная цепочка, и не разорвалась золотая повязка, и не разбился кувшин у источника, и не обрушилось колесо над колодезем.
\vs Ecc 12:7 И возвратится прах в землю, чем он и был; а дух возвратится к Богу, Который дал его.
\vs Ecc 12:8 Суета сует, сказал Екклесиаст, всё~--- суета!
\rsbpar\vs Ecc 12:9 Кроме того, что Екклесиаст был мудр, он учил еще народ знанию. Он \bibemph{все} испытывал, исследовал, \bibemph{и} составил много притчей.
\vs Ecc 12:10 Старался Екклесиаст приискивать изящные изречения, и слова истины написаны \bibemph{им} верно.
\vs Ecc 12:11 Слова мудрых~--- как иглы и как вбитые гвозди, и составители их~--- от Единого Пастыря.
\vs Ecc 12:12 А что сверх всего этого, сын мой, того берегись: составлять много книг~--- конца не будет, и много читать~--- утомительно для тела.
\rsbpar\vs Ecc 12:13 Выслушаем сущность всего: бойся Бога и заповеди Его соблюдай, потому что в этом всё для человека;
\vs Ecc 12:14 ибо всякое дело Бог приведет на суд, и все тайное, хорошо ли оно, или худо.

\bibbookdescr{Sol}{
  inline={\LARGE Книга\\\Huge Песни Песней Соломона},
  toc={Песнь Песней},
  bookmark={Песнь Песней},
  header={Песнь Песней},
  %headerleft={},
  %headerright={},
  abbr={Песн}
}
\vs Sol 1:1 Да лобзает он меня лобзанием уст своих! Ибо ласки твои лучше вина.
\vs Sol 1:2 От благовония мастей твоих имя твое~--- как разлитое миро; поэтому девицы любят тебя.
\vs Sol 1:3 Влеки меня, мы побежим за тобою;~--- царь ввел меня в чертоги свои,~--- будем восхищаться и радоваться тобою, превозносить ласки твои больше, нежели вино; достойно любят тебя!
\rsbpar\vs Sol 1:4 Дщери Иерусалимские! черна я, но красива, как шатры Кидарские, как завесы Соломоновы.
\vs Sol 1:5 Не смотрите на меня, что я смугла, ибо солнце опалило меня: сыновья матери моей разгневались на меня, поставили меня стеречь виноградники,~--- моего собственного виноградника я не стерегла.
\rsbpar\vs Sol 1:6 Скажи мне, ты, которого любит душа моя: где пасешь ты? где отдыхаешь в полдень? к чему мне быть скиталицею возле стад товарищей твоих?
\vs Sol 1:7 Если ты не знаешь этого, прекраснейшая из женщин, то иди себе по следам овец и паси козлят твоих подле шатров пастушеских.
\vs Sol 1:8 Кобылице моей в колеснице фараоновой я уподобил тебя, возлюбленная моя.
\vs Sol 1:9 Прекрасны ланиты твои под подвесками, шея твоя в ожерельях;
\vs Sol 1:10 золотые подвески мы сделаем тебе с серебряными блестками.
\vs Sol 1:11 Доколе царь был за столом своим, нард мой издавал благовоние свое.
\vs Sol 1:12 Мирровый пучок~--- возлюбленный мой у меня, у грудей моих пребывает.
\vs Sol 1:13 Как кисть кипера, возлюбленный мой у меня в виноградниках Енгедских.
\vs Sol 1:14 О, ты прекрасна, возлюбленная моя, ты прекрасна! глаза твои голубиные.
\vs Sol 1:15 О, ты прекрасен, возлюбленный мой, и любезен! и ложе у нас~--- зелень;
\vs Sol 1:16 кровли домов наших~--- кедры, потолки наши~--- кипарисы.
\vs Sol 2:1 Я нарцисс Саронский, лилия долин!
\vs Sol 2:2 Что лилия между тернами, то возлюбленная моя между девицами.
\vs Sol 2:3 Что яблоня между лесными деревьями, то возлюбленный мой между юношами. В тени ее люблю я сидеть, и плоды ее сладки для гортани моей.
\rsbpar\vs Sol 2:4 Он ввел меня в дом пира, и знамя его надо мною~--- любовь.
\vs Sol 2:5 Подкрепите меня вином, освежите меня яблоками, ибо я изнемогаю от любви.
\vs Sol 2:6 Левая рука его у меня под головою, а правая обнимает меня.
\vs Sol 2:7 Заклинаю вас, дщери Иерусалимские, сернами или полевыми ланями: не будите и не тревожьте возлюбленной, доколе ей угодно.
\rsbpar\vs Sol 2:8 Голос возлюбленного моего! вот, он идет, скачет по горам, прыгает по холмам.
\vs Sol 2:9 Друг мой похож на серну или на молодого оленя. Вот, он стоит у нас за стеною, заглядывает в окно, мелькает сквозь решетку.
\vs Sol 2:10 Возлюбленный мой начал говорить мне: встань, возлюбленная моя, прекрасная моя, выйди!
\vs Sol 2:11 Вот, зима уже прошла; дождь миновал, перестал;
\vs Sol 2:12 цветы показались на земле; время пения настало, и голос горлицы слышен в стране нашей;
\vs Sol 2:13 смоковницы распустили свои почки, и виноградные лозы, расцветая, издают благовоние. Встань, возлюбленная моя, прекрасная моя, выйди!
\vs Sol 2:14 Голубица моя в ущелье скалы под кровом утеса! покажи мне лице твое, дай мне услышать голос твой, потому что голос твой сладок и лице твое приятно.
\vs Sol 2:15 Ловите нам лисиц, лисенят, которые портят виноградники, а виноградники наши в цвете.
\rsbpar\vs Sol 2:16 Возлюбленный мой принадлежит мне, а я ему; он пасет между лилиями.
\vs Sol 2:17 Доколе день дышит \bibemph{прохладою}, и убегают тени, возвратись, будь подобен серне или молодому оленю на расселинах гор.
\vs Sol 3:1 На ложе моем ночью искала я того, которого любит душа моя, искала его и не нашла его.
\vs Sol 3:2 Встану же я, пойду по городу, по улицам и площадям, и буду искать того, которого любит душа моя; искала я его и не нашла его.
\vs Sol 3:3 Встретили меня стражи, обходящие город: <<не видали ли вы того, которого любит душа моя?>>
\vs Sol 3:4 Но едва я отошла от них, как нашла того, которого любит душа моя, ухватилась за него, и не отпустила его, доколе не привела его в дом матери моей и во внутренние комнаты родительницы моей.
\rsbpar\vs Sol 3:5 Заклинаю вас, дщери Иерусалимские, сернами или полевыми ланями: не будите и не тревожьте возлюбленной, доколе ей угодно.
\vs Sol 3:6 Кто эта, восходящая от пустыни как бы столбы дыма, окуриваемая миррою и фимиамом, всякими порошками мироварника?
\rsbpar\vs Sol 3:7 Вот одр его~--- Соломона: шестьдесят сильных вокруг него, из сильных Израилевых.
\vs Sol 3:8 Все они держат по мечу, опытны в бою; у каждого меч при бедре его ради страха ночного.
\vs Sol 3:9 Носильный одр сделал себе царь Соломон из дерев Ливанских;
\vs Sol 3:10 столпцы его сделал из серебра, локотники его из золота, седалище его из пурпуровой ткани; внутренность его убрана с любовью дщерями Иерусалимскими.
\vs Sol 3:11 Пойдите и посмотрите, дщери Сионские, на царя Соломона в венце, которым увенчала его мать его в день бракосочетания его, в день, радостный для сердца его.
\vs Sol 4:1 О, ты прекрасна, возлюбленная моя, ты прекрасна! глаза твои голубиные под кудрями твоими; волосы твои~--- как стадо коз, сходящих с горы Галаадской;
\vs Sol 4:2 зубы твои~--- как стадо выстриженных овец, выходящих из купальни, из которых у каждой пара ягнят, и бесплодной нет между ними;
\vs Sol 4:3 как лента алая губы твои, и уста твои любезны; как половинки гранатового яблока~--- ланиты твои под кудрями твоими;
\vs Sol 4:4 шея твоя~--- как столп Давидов, сооруженный для оружий, тысяча щитов висит на нем~--- все щиты сильных;
\vs Sol 4:5 два сосца твои~--- как двойни молодой серны, пасущиеся между лилиями.
\vs Sol 4:6 Доколе день дышит \bibemph{прохладою}, и убегают тени, пойду я на гору мирровую и на холм фимиама.
\rsbpar\vs Sol 4:7 Вся ты прекрасна, возлюбленная моя, и пятна нет на тебе!
\vs Sol 4:8 Со мною с Ливана, невеста! со мною иди с Ливана! спеши с вершины Аманы, с вершины Сенира и Ермона, от логовищ львиных, от гор барсовых!
\vs Sol 4:9 Пленила ты сердце мое, сестра моя, невеста! пленила ты сердце мое одним взглядом очей твоих, одним ожерельем на шее твоей.
\vs Sol 4:10 О, как любезны ласки твои, сестра моя, невеста! о, как много ласки твои лучше вина, и благовоние мастей твоих лучше всех ароматов!
\vs Sol 4:11 Сотовый мед каплет из уст твоих, невеста; мед и молоко под языком твоим, и благоухание одежды твоей подобно благоуханию Ливана!
\vs Sol 4:12 Запертый сад~--- сестра моя, невеста, заключенный колодезь, запечатанный источник:
\vs Sol 4:13 рассадники твои~--- сад с гранатовыми яблоками, с превосходными плодами, киперы с нардами,
\vs Sol 4:14 нард и шафран, аир и корица со всякими благовонными деревами, мирра и алой со всякими лучшими ароматами;
\vs Sol 4:15 садовый источник~--- колодезь живых вод и потоки с Ливана.
\vs Sol 4:16 Поднимись \bibemph{ветер} с севера и принесись с юга, повей на сад мой,~--- и польются ароматы его!~--- Пусть придет возлюбленный мой в сад свой и вкушает сладкие плоды его.
\vs Sol 5:1 Пришел я в сад мой, сестра моя, невеста; набрал мирры моей с ароматами моими, поел сотов моих с медом моим, напился вина моего с молоком моим. Ешьте, друзья, пейте и насыщайтесь, возлюбленные!
\rsbpar\vs Sol 5:2 Я сплю, а сердце мое бодрствует; \bibemph{вот}, голос моего возлюбленного, который стучится: <<отвори мне, сестра моя, возлюбленная моя, голубица моя, чистая моя! потому что голова моя вся покрыта росою, кудри мои~--- ночною влагою>>.
\vs Sol 5:3 Я скинула хитон мой; как же мне опять надевать его? Я вымыла ноги мои; как же мне марать их?
\vs Sol 5:4 Возлюбленный мой протянул руку свою сквозь скважину, и внутренность моя взволновалась от него.
\vs Sol 5:5 Я встала, чтобы отпереть возлюбленному моему, и с рук моих капала мирра, и с перстов моих мирра капала на ручки замка.
\vs Sol 5:6 Отперла я возлюбленному моему, а возлюбленный мой повернулся и ушел. Души во мне не стало, когда он говорил; я искала его и не находила его; звала его, и он не отзывался мне.
\vs Sol 5:7 Встретили меня стражи, обходящие город, избили меня, изранили меня; сняли с меня покрывало стерегущие стены.
\vs Sol 5:8 Заклинаю вас, дщери Иерусалимские: если вы встретите возлюбленного моего, что скажете вы ему? что я изнемогаю от любви.
\vs Sol 5:9 <<Чем возлюбленный твой лучше других возлюбленных, прекраснейшая из женщин? Чем возлюбленный твой лучше других, что ты так заклинаешь нас?>>
\vs Sol 5:10 Возлюбленный мой бел и румян, лучше десяти тысяч других:
\vs Sol 5:11 голова его~--- чистое золото; кудри его волнистые, черные, как ворон;
\vs Sol 5:12 глаза его~--- как голуби при потоках вод, купающиеся в молоке, сидящие в довольстве;
\vs Sol 5:13 щеки его~--- цветник ароматный, гряды благовонных растений; губы его~--- лилии, источают текучую мирру;
\vs Sol 5:14 руки его~--- золотые кругляки, усаженные топазами; живот его~--- как изваяние из слоновой кости, обложенное сапфирами;
\vs Sol 5:15 голени его~--- мраморные столбы, поставленные на золотых подножиях; вид его подобен Ливану, величествен, как кедры;
\vs Sol 5:16 уста его~--- сладость, и весь он~--- любезность. Вот кто возлюбленный мой, и вот кто друг мой, дщери Иерусалимские!
\vs Sol 6:1 <<Куда пошел возлюбленный твой, прекраснейшая из женщин? куда обратился возлюбленный твой? мы поищем его с тобою>>.
\vs Sol 6:2 Мой возлюбленный пошел в сад свой, в цветники ароматные, чтобы пасти в садах и собирать лилии.
\vs Sol 6:3 Я принадлежу возлюбленному моему, а возлюбленный мой~--- мне; он пасет между лилиями.
\rsbpar\vs Sol 6:4 Прекрасна ты, возлюбленная моя, как Фирца, любезна, как Иерусалим, грозна, как полки со знаменами.
\vs Sol 6:5 Уклони очи твои от меня, потому что они волнуют меня.
\vs Sol 6:6 Волосы твои~--- как стадо коз, сходящих с Галаада; зубы твои~--- как стадо овец, выходящих из купальни, из которых у каждой пара ягнят, и бесплодной нет между ними;
\vs Sol 6:7 как половинки гранатового яблока~--- ланиты твои под кудрями твоими.
\vs Sol 6:8 Есть шестьдесят цариц и восемьдесят наложниц и девиц без числа,
\vs Sol 6:9 но единственная~--- она, голубица моя, чистая моя; единственная она у матери своей, отличенная у родительницы своей. Увидели ее девицы, и~--- превознесли ее, царицы и наложницы, и~--- восхвалили ее.
\vs Sol 6:10 Кто эта, блистающая, как заря, прекрасная, как луна, светлая, как солнце, грозная, как полки со знаменами?
\vs Sol 6:11 Я сошла в ореховый сад посмотреть на зелень долины, поглядеть, распустилась ли виноградная лоза, расцвели ли гранатовые яблоки?
\vs Sol 6:12 Не знаю, как душа моя влекла меня к колесницам знатных народа моего.
\vs Sol 7:1 <<Оглянись, оглянись, Суламита! оглянись, оглянись,~--- и мы посмотрим на тебя>>. Что вам смотреть на Суламиту, как на хоровод Манаимский?
\vs Sol 7:2 О, как прекрасны ноги твои в сандалиях, дщерь именитая! Округление бедр твоих, как ожерелье, дело рук искусного художника;
\vs Sol 7:3 живот твой~--- круглая чаша, \bibemph{в которой} не истощается ароматное вино; чрево твое~--- ворох пшеницы, обставленный лилиями;
\vs Sol 7:4 два сосца твои~--- как два козленка, двойни серны;
\vs Sol 7:5 шея твоя~--- как столп из слоновой кости; глаза твои~--- озерки Есевонские, что у ворот Батраббима; нос твой~--- башня Ливанская, обращенная к Дамаску;
\vs Sol 7:6 голова твоя на тебе, как Кармил, и волосы на голове твоей, как пурпур; царь увлечен \bibemph{твоими} кудрями.
\vs Sol 7:7 Как ты прекрасна, как привлекательна, возлюбленная, твоею миловидностью!
\vs Sol 7:8 Этот стан твой похож на пальму, и груди твои на виноградные кисти.
\vs Sol 7:9 Подумал я: влез бы я на пальму, ухватился бы за ветви ее; и груди твои были бы вместо кистей винограда, и запах от ноздрей твоих, как от яблоков;
\vs Sol 7:10 уста твои~--- как отличное вино. Оно течет прямо к другу моему, услаждает уста утомленных.
\vs Sol 7:11 Я принадлежу другу моему, и ко мне \bibemph{обращено} желание его.
\vs Sol 7:12 Приди, возлюбленный мой, выйдем в поле, побудем в селах;
\vs Sol 7:13 поутру пойдем в виноградники, посмотрим, распустилась ли виноградная лоза, раскрылись ли почки, расцвели ли гранатовые яблоки; там я окажу ласки мои тебе.
\vs Sol 7:14 Мандрагоры уже пустили благовоние, и у дверей наших всякие превосходные плоды, новые и старые: \bibemph{это} сберегла я для тебя, мой возлюбленный!
\vs Sol 8:1 О, если бы ты был мне брат, сосавший груди матери моей! тогда я, встретив тебя на улице, целовала бы тебя, и меня не осуждали бы.
\vs Sol 8:2 Повела бы я тебя, привела бы тебя в дом матери моей. Ты учил бы меня, а я поила бы тебя ароматным вином, соком гранатовых яблоков моих.
\vs Sol 8:3 Левая рука его у меня под головою, а правая обнимает меня.
\vs Sol 8:4 Заклинаю вас, дщери Иерусалимские,~--- не будите и не тревожьте возлюбленной, доколе ей угодно.
\rsbpar\vs Sol 8:5 Кто это восходит от пустыни, опираясь на своего возлюбленного? Под яблоней разбудила я тебя: там родила тебя мать твоя, там родила тебя родительница твоя.
\vs Sol 8:6 Положи меня, как печать, на сердце твое, как перстень, на руку твою: ибо крепка, как смерть, любовь; люта, как преисподняя, ревность; стрелы ее~--- стрелы огненные; она пламень весьма сильный.
\vs Sol 8:7 Большие воды не могут потушить любви, и реки не зальют ее. Если бы кто давал все богатство дома своего за любовь, то он был бы отвергнут с презреньем.
\rsbpar\vs Sol 8:8 Есть у нас сестра, которая еще мала, и сосцов нет у нее; что нам будет делать с сестрою нашею, когда будут свататься за нее?
\vs Sol 8:9 Если бы она была стена, то мы построили бы на ней палаты из серебра; если бы она была дверь, то мы обложили бы ее кедровыми досками.
\vs Sol 8:10 Я~--- стена, и сосцы у меня, как башни; потому я буду в глазах его, как достигшая полноты.
\rsbpar\vs Sol 8:11 Виноградник был у Соломона в Ваал-Гамоне; он отдал этот виноградник сторожам; каждый должен был доставлять за плоды его тысячу сребреников.
\vs Sol 8:12 А мой виноградник у меня при себе. Тысяча пусть тебе, Соломон, а двести~--- стерегущим плоды его.
\vs Sol 8:13 Жительница садов! товарищи внимают голосу твоему, дай и мне послушать его.
\rsbpar\vs Sol 8:14 Беги, возлюбленный мой; будь подобен серне или молодому оленю на горах бальзамических!

\bibbookdescr{Wis}{
  inline={\LARGE Книга\\\Huge Премудрости Соломона\fns{Переведена с греческого.}},
  toc={Премудрость Соломона*},
  bookmark={Премудрость Соломона},
  header={Премудрость Соломона},
  %headerleft={},
  %headerright={},
  abbr={Прем}
}
\vs Wis 1:1 Любите справедливость, судьи земли, право мыслите о Господе, и в простоте сердца ищите Его,
\vs Wis 1:2 ибо Он обретается неискушающими Его и является не неверующим Ему.
\vs Wis 1:3 Ибо неправые умствования отдаляют от Бога, и испытание силы Его обличит безумных.
\vs Wis 1:4 В лукавую душу не войдет премудрость и не будет обитать в теле, порабощенном греху,
\vs Wis 1:5 ибо святый Дух премудрости удалится от лукавства и уклонится от неразумных умствований, и устыдится приближающейся неправды.
\vs Wis 1:6 Человеколюбивый дух~--- премудрость, но не оставит безнаказанным богохульствующего устами, потому что Бог есть свидетель внутренних чувств его и истинный зритель сердца его, и слышатель языка его.
\vs Wis 1:7 Дух Господа наполняет вселенную и, как все объемлющий, знает \bibemph{всякое} слово.
\vs Wis 1:8 Посему никто, говорящий неправду, не утаится, и не минет его обличающий суд.
\vs Wis 1:9 Ибо будет испытание помыслов нечестивого, и слов\acc{а} его взойдут к Господу в обличение беззаконий его;
\vs Wis 1:10 потому что ухо ревности слышит все, и ропот не скроется.
\vs Wis 1:11 Итак, хранитесь от бесполезного ропота и берегитесь от злоречия языка, ибо и тайное слово не пройдет даром, а клевещущие уста убивают душу.
\vs Wis 1:12 Не ускоряйте смерти заблуждениями вашей жизни и не привлекайте к себе погибели делами рук ваших.
\vs Wis 1:13 Бог не сотворил смерти и не радуется погибели живущих,
\vs Wis 1:14 ибо Он создал все для бытия, и все в мире спасительно, и нет пагубного яда, нет и царства ада на земле.
\vs Wis 1:15 Праведность бессмертна, а неправда причиняет смерть:
\vs Wis 1:16 нечестивые привлекли ее и руками и словами, сочли ее другом и исчахли, и заключили союз с нею, ибо они достойны быть ее жребием.
\vs Wis 2:1 Неправо умствующие говорили сами в себе: <<коротка и прискорбна наша жизнь, и нет человеку спасения от смерти, и не знают, чтобы кто освободил из ада.
\vs Wis 2:2 Случайно мы рождены и после будем как небывшие: дыхание в ноздрях наших~--- дым, и слово~--- искра в движении нашего сердца.
\vs Wis 2:3 Когда она угаснет, тело обратится в прах, и дух рассеется, как жидкий воздух;
\vs Wis 2:4 и имя наше забудется со временем, и никто не вспомнит о делах наших; и жизнь наша пройдет, как след облака, и рассеется, как туман, разогнанный лучами солнца и отягченный теплотою его.
\vs Wis 2:5 Ибо жизнь наша~--- прохождение тени, и нет нам возврата от смерти: ибо положена печать, и никто не возвращается.
\vs Wis 2:6 Будем же наслаждаться настоящими благами и спешить пользоваться миром, как юностью;
\vs Wis 2:7 преисполнимся дорогим вином и благовониями, и да не пройдет мимо нас весенний цвет жизни;
\vs Wis 2:8 увенчаемся цветами роз прежде, нежели они увяли;
\vs Wis 2:9 никто из нас не лишай себя участия в нашем наслаждении; везде оставим следы веселья, ибо это наша доля и наш жребий.
\vs Wis 2:10 Будем притеснять бедняка праведника, не пощадим вдовы и не постыдимся многолетних седин старца.
\vs Wis 2:11 Сила наша да будет законом правды, ибо бессилие оказывается бесполезным.
\vs Wis 2:12 Устроим ковы праведнику, ибо он в тягость нам и противится делам нашим, укоряет нас в грехах против закона и поносит нас за грехи нашего воспитания;
\vs Wis 2:13 объявляет себя имеющим познание о Боге и называет себя сыном Господа;
\vs Wis 2:14 он пред нами~--- обличение помыслов наших.
\vs Wis 2:15 Тяжело нам и смотреть на него, ибо жизнь его не похожа на жизнь других, и отличны пути его:
\vs Wis 2:16 он считает нас мерзостью и удаляется от путей наших, как от нечистот, ублажает кончину праведных и тщеславно называет отцом своим Бога.
\vs Wis 2:17 Увидим, истинны ли слова его, и испытаем, какой будет исход его;
\vs Wis 2:18 ибо если этот праведник есть сын Божий, то \bibemph{Бог} защитит его и избавит его от руки врагов.
\vs Wis 2:19 Испытаем его оскорблением и мучением, дабы узнать смирение его и видеть незлобие его;
\vs Wis 2:20 осудим его на бесчестную смерть, ибо, по словам его, о нем попечение будет>>.
\vs Wis 2:21 Так они умствовали, и ошиблись; ибо злоба их ослепила их,
\vs Wis 2:22 и они не познали тайн Божиих, не ожидали воздаяния за святость и не считали достойными награды душ непорочных.
\vs Wis 2:23 Бог создал человека для нетления и соделал его образом вечного бытия Своего;
\vs Wis 2:24 но завистью диавола вошла в мир смерть, и испытывают ее принадлежащие к уделу его.
\vs Wis 3:1 А души праведных в руке Божией, и мучение не коснется их.
\vs Wis 3:2 В глазах неразумных они казались умершими, и исход их считался погибелью,
\vs Wis 3:3 и отшествие от нас~--- уничтожением; но они пребывают в мире.
\vs Wis 3:4 Ибо, хотя они в глазах людей и наказываются, но надежда их полна бессмертия.
\vs Wis 3:5 И немного наказанные, они будут много облагодетельствованы, потому что Бог испытал их и нашел их достойными Его.
\vs Wis 3:6 Он испытал их как золото в горниле и принял их как жертву всесовершенную.
\vs Wis 3:7 Во время воздаяния им они воссияют как искры, бегущие по стеблю.
\vs Wis 3:8 Будут судить племена и владычествовать над народами, а над ними будет Господь царствовать во веки.
\vs Wis 3:9 Надеющиеся на Него познают истину, и верные в любви пребудут у Него; ибо благодать и милость со святыми Его и промышление об избранных Его.
\vs Wis 3:10 Нечестивые же, как умствовали, так и понесут наказание за то, что презрели праведного и отступили от Господа.
\vs Wis 3:11 Ибо презирающий мудрость и наставление несчастен, и надежда их суетна, и труды бесплодны, и дела их непотребны.
\vs Wis 3:12 Жены их несмысленны, и дети их злы, проклят род их.
\vs Wis 3:13 Блаженна неплодная неосквернившаяся, которая не познала беззаконного ложа; она получит плод при воздаянии святых душ.
\vs Wis 3:14 \bibemph{Блажен} и евнух, не сделавший беззакония рукою и не помысливший лукавого против Господа, ибо дастся ему особенная благодать веры и приятнейший жребий в храме Господнем.
\vs Wis 3:15 Плод добрых трудов славен, и корень мудрости неподвижен.
\vs Wis 3:16 Дети прелюбодеев будут несовершенны, и семя беззаконного ложа исчезнет.
\vs Wis 3:17 Если и будут они долгожизненны, но будут почитаться за ничто, и поздняя старость их будет без почета.
\vs Wis 3:18 А если скоро умрут, не будут иметь надежды и утешения в день суда;
\vs Wis 3:19 ибо ужасен конец неправедного рода.
\vs Wis 4:1 Лучше бездетность с добродетелью, ибо память о ней бессмертна: она признается и у Бога и у людей.
\vs Wis 4:2 Когда она присуща, ей подражают, а когда отойдет, стремятся к ней: и в вечности увенчанная она торжествует, как одержавшая победу непорочными подвигами.
\vs Wis 4:3 А плодородное множество нечестивых не принесет пользы, и прелюбодейные отрасли не дадут корней в глубину и не достигнут незыблемого основания;
\vs Wis 4:4 и хотя на время позеленеют в ветвях, но, не имея твердости, поколеблются от ветра и порывом ветров искоренятся;
\vs Wis 4:5 некрепкие ветви переломятся, и плод их \bibemph{будет} бесполезен, незрел для пищи и ни к чему не годен;
\vs Wis 4:6 ибо дети, рождаемые от беззаконных сожитий, суть свидетели разврата против родителей при допросе их.
\vs Wis 4:7 А праведник, если и рановременно умрет, будет в покое,
\vs Wis 4:8 ибо не в долговечности честная старость и не числом лет измеряется:
\vs Wis 4:9 мудрость есть седина для людей, и беспорочная жизнь~--- возраст старости.
\vs Wis 4:10 Как благоугодивший Богу, он возлюблен, и, как живший посреди грешников, преставлен,
\vs Wis 4:11 восхищен, чтобы злоба не изменила разума его, или коварство не прельстило души его.
\vs Wis 4:12 Ибо упражнение в нечестии помрачает доброе, и волнение похоти развращает ум незлобивый.
\vs Wis 4:13 Достигнув совершенства в короткое время, он исполнил долгие лета;
\vs Wis 4:14 ибо душа его была угодна Господу, потому и ускорил он из среды нечестия. А люди видели это и не поняли, даже и не подумали о том,
\vs Wis 4:15 что благодать и милость со святыми Его и промышление об избранных Его.
\vs Wis 4:16 Праведник, умирая, осудит живых нечестивых, и скоро достигшая совершенства юность~--- долголетнюю старость неправедного;
\vs Wis 4:17 ибо они увидят кончину мудрого и не поймут, что Господь определил о нем и для чего поставил его в безопасность;
\vs Wis 4:18 они увидят и уничтожат его, но Господь посмеется им;
\vs Wis 4:19 и после сего будут они бесчестным трупом и позором между умершими навек, ибо Он повергнет их ниц безгласными и сдвинет их с оснований, и они вконец запустеют и будут в скорби, и память их погибнет;
\vs Wis 4:20 в сознании грехов своих они предстанут со страхом, и беззакония их осудят их в лице их.
\vs Wis 5:1 Тогда праведник с великим дерзновением станет пред лицем тех, которые оскорбляли его и презирали подвиги его;
\vs Wis 5:2 они же, увидев, смутятся великим страхом и изумятся неожиданности спасения его
\vs Wis 5:3 и, раскаиваясь и воздыхая от стеснения духа, будут говорить сами в себе: <<это тот самый, который был у нас некогда в посмеянии и притчею поругания.
\vs Wis 5:4 Безумные, мы почитали жизнь его сумасшествием и кончину его бесчестною!
\vs Wis 5:5 Как же он причислен к сынам Божиим, и жребий его~--- со святыми?
\vs Wis 5:6 Итак, мы заблудились от пути истины, и свет правды не светил нам, и солнце не озаряло нас.
\vs Wis 5:7 Мы преисполнились делами беззакония и погибели и ходили по непроходимым пустыням, а пути Господня не познали.
\vs Wis 5:8 Какую пользу принесло нам высокомерие, и что доставило нам богатство с тщеславием?
\vs Wis 5:9 Все это прошло как тень и как молва быстротечная.
\vs Wis 5:10 Как после прохождения корабля, идущего по волнующейся воде, невозможно найти следа, ни стези дна его в волнах;
\vs Wis 5:11 или как от птицы, пролетающей по воздуху, никакого не остается знака ее пути, но легкий воздух, ударяемый крыльями и рассекаемый быстротою движения, пройден движущимися крыльями, и после того не осталось никакого знака прохождения по нему;
\vs Wis 5:12 или как от стрелы, пущенной в цель, разделенный воздух тотчас опять сходится, так что нельзя узнать, где прошла она;
\vs Wis 5:13 так и мы родились и умерли, и не могли показать никакого знака добродетели, но истощились в беззаконии нашем>>.
\vs Wis 5:14 Ибо надежда нечестивого исчезает, как прах, уносимый ветром, и как тонкий иней, разносимый бурею, и как дым, рассеиваемый ветром, и проходит, как память об однодневном госте.
\vs Wis 5:15 А праведники живут во веки; награда их~--- в Господе, и попечение о них~--- у Вышнего.
\vs Wis 5:16 Посему они получат царство славы и венец красоты от руки Господа, ибо Он покроет их десницею и защитит их мышцею.
\vs Wis 5:17 Он возьмет всеоружие~--- ревность Свою, и тварь вооружит к отмщению врагам;
\vs Wis 5:18 облечется в броню~--- в правду, и возложит на Себя шлем~--- нелицеприятный суд;
\vs Wis 5:19 возьмет непобедимый щит~--- святость;
\vs Wis 5:20 строгий гнев Он изострит, как меч, и мир ополчится с Ним против безумцев.
\vs Wis 5:21 Понесутся меткие стрелы молний и из облаков, как из туго натянутого лука, полетят в цель.
\vs Wis 5:22 И, как из каменометного орудия, с яростью посыплется град; вознегодует на них вода морская и реки свирепо потопят их;
\vs Wis 5:23 восстанет против них дух силы и, как вихрь, развеет их.
\vs Wis 5:24 Так беззаконие опустошит всю землю, и злодеяние ниспровергнет престолы сильных.
\vs Wis 6:1 Итак, слушайте, цари, и разумейте, научитесь, судьи концов земли!
\vs Wis 6:2 Внимайте, обладатели множества и гордящиеся пред народами!
\vs Wis 6:3 От Господа дана вам держава, и сила~--- от Вышнего, Который исследует ваши дела и испытает намерения.
\vs Wis 6:4 Ибо вы, будучи служителями Его царства, не судили справедливо, не соблюдали закона и не поступали по воле Божией.
\vs Wis 6:5 Страшно и скоро Он явится вам,~--- и строг суд над начальствующими,
\vs Wis 6:6 ибо меньший заслуживает помилование, а сильные сильно будут истязаны.
\vs Wis 6:7 Господь всех не убоится лица и не устрашится величия, ибо Он сотворил и малого и великого и одинаково промышляет о всех;
\vs Wis 6:8 но начальствующим предстоит строгое испытание.
\vs Wis 6:9 Итак, к вам, цари, слова мои, чтобы вы научились премудрости и не падали.
\vs Wis 6:10 Ибо свято хранящие святое освятятся, и научившиеся тому найдут оправдание.
\vs Wis 6:11 Итак, возжелайте слов моих, полюбите и научитесь.
\vs Wis 6:12 Премудрость светла и неувядающа, и легко созерцается любящими ее, и обретается ищущими ее;
\vs Wis 6:13 она \bibemph{даже} упреждает желающих познать ее.
\vs Wis 6:14 С раннего утра ищущий ее не утомится, ибо найдет ее сидящею у дверей своих.
\vs Wis 6:15 Помышлять о ней есть уже совершенство разума, и бодрствующий ради нее скоро освободится от забот,
\vs Wis 6:16 ибо она сама обходит и ищет достойных ее, и благосклонно является им на путях, и при всякой мысли встречается с ними.
\vs Wis 6:17 Начало ее есть искреннейшее желание учения,
\vs Wis 6:18 а забота об учении~--- любовь, любовь же~--- хранение законов ее, а наблюдение законов~--- залог бессмертия,
\vs Wis 6:19 а бессмертие приближает к Богу;
\vs Wis 6:20 поэтому желание премудрости возводит к царству.
\vs Wis 6:21 Итак, властители народов, если вы услаждаетесь престолами и скипетрами, то почтите премудрость, чтобы вам царствовать во веки.
\vs Wis 6:22 Что же есть премудрость, и как она произошла, я возвещу,
\vs Wis 6:23 и не скрою от вас тайн, но исследую от начала рождения,
\vs Wis 6:24 и открою познание ее, и не миную истины;
\vs Wis 6:25 и не пойду вместе с истаевающим от зависти, ибо таковой не будет причастником премудрости.
\vs Wis 6:26 Множество мудрых~--- спасение миру, и царь разумный~--- благосостояние народа.
\vs Wis 6:27 Итак учитесь от слов моих, и получите пользу.
\vs Wis 7:1 И я человек смертный, подобный всем, потомок первозданного земнородного.
\vs Wis 7:2 И я в утробе матерней образовался в плоть в десятимесячное время, сгустившись в крови от семени мужа и услаждения, соединенного со сном,
\vs Wis 7:3 и я, родившись, начал дышать общим воздухом и ниспал на ту же землю, первый голос обнаружил плачем одинаково со всеми,
\vs Wis 7:4 вскормлен в пеленах и заботах;
\vs Wis 7:5 ибо ни один царь не имел иного начала рождения:
\vs Wis 7:6 один для всех вход в жизнь и одинаковый исход.
\vs Wis 7:7 Посему я молился, и дарован мне разум; я взывал, и сошел на меня дух премудрости.
\vs Wis 7:8 Я предпочел ее скипетрам и престолам и богатство почитал за ничто в сравнении с нею;
\vs Wis 7:9 драгоценного камня я не сравнил с нею, потому что перед нею все золото~--- ничтожный песок, а серебро~--- грязь в сравнении с нею.
\vs Wis 7:10 Я полюбил ее более здоровья и красоты и избрал ее предпочтительно перед светом, ибо свет ее неугасим.
\vs Wis 7:11 А вместе с нею пришли ко мне все блага и несметное богатство через руки ее;
\vs Wis 7:12 я радовался всему, потому что премудрость руководствовала ими, но я не знал, что она~--- виновница их.
\vs Wis 7:13 Без хитрости я научился, и без зависти преподаю, не скрываю богатства ее,
\vs Wis 7:14 ибо она есть неистощимое сокровище для людей; пользуясь ею, они входят в содружество с Богом, посредством даров учения.
\vs Wis 7:15 Только дал бы мне Бог говорить по разумению и достойно мыслить о дарованном, ибо Он есть руководитель к мудрости и исправитель мудрых.
\vs Wis 7:16 Ибо в руке Его и мы и слова наши, и всякое разумение и искусство делания.
\vs Wis 7:17 Сам Он даровал мне неложное познание существующего, чтобы познать устройство мира и действие стихий,
\vs Wis 7:18 начало, конец и средину времен, смены поворотов и перемены времен,
\vs Wis 7:19 круги годов и положение звезд,
\vs Wis 7:20 природу животных и свойства зверей, стремления ветров и мысли людей, различия растений и силы корней.
\vs Wis 7:21 Познал я все, и сокровенное и явное, ибо научила меня Премудрость, художница всего.
\vs Wis 7:22 Она есть дух разумный, святый, единородный, многочастный, тонкий, удобоподвижный, светлый, чистый, ясный, невредительный, благолюбивый, скорый, неудержимый,
\vs Wis 7:23 благодетельный, человеколюбивый, твердый, непоколебимый, спокойный, беспечальный, всевидящий и проникающий все умные, чистые, тончайшие духи.
\vs Wis 7:24 Ибо премудрость подвижнее всякого движения, и по чистоте своей сквозь все проходит и проникает.
\vs Wis 7:25 Она есть дыхание силы Божией и чистое излияние славы Вседержителя: посему ничто оскверненное не войдет в нее.
\vs Wis 7:26 Она есть отблеск вечного света и чистое зеркало действия Божия и образ благости Его.
\vs Wis 7:27 Она~--- одна, но может все, и, пребывая в самой себе, все обновляет, и, переходя из рода в род в святые души, приготовляет друзей Божиих и пророков;
\vs Wis 7:28 ибо Бог никого не любит, кроме живущего с премудростью.
\vs Wis 7:29 Она прекраснее солнца и превосходнее сонма звезд; в сравнении со светом она выше;
\vs Wis 7:30 ибо свет сменяется ночью, а премудрости не превозмогает злоба.
\vs Wis 8:1 Она быстро распростирается от одного конца до другого и все устрояет на пользу.
\vs Wis 8:2 Я полюбил ее и взыскал от юности моей, и пожелал взять ее в невесту себе, и стал любителем красоты ее.
\vs Wis 8:3 Она возвышает \bibemph{свое} благородство тем, что имеет сожитие с Богом, и Владыка всех возлюбил ее:
\vs Wis 8:4 она таинница ума Божия и избирательница дел Его.
\vs Wis 8:5 Если богатство есть вожделенное приобретение в жизни, то что богаче премудрости, которая все делает?
\vs Wis 8:6 Если же благоразумие делает \bibemph{многое}, то какой художник лучше ее?
\vs Wis 8:7 Если кто любит праведность,~--- плоды ее суть добродетели: она научает целомудрию и рассудительности, справедливости и мужеству, полезнее которых ничего нет для людей в жизни.
\vs Wis 8:8 Если кто желает большой опытности, мудрость знает давнопрошедшее и угадывает будущее, знает тонкости слов и разрешение загадок, предузнает знамения и чудеса и последствия лет и времен.
\vs Wis 8:9 Посему я рассудил принять ее в сожитие с собою, зная, что она будет мне советницею на доброе и утешеньем в заботах и печали.
\vs Wis 8:10 Через нее я буду иметь славу в народе и честь перед старейшими, будучи юношею;
\vs Wis 8:11 окажусь проницательным в суде, и в глазах сильных заслужу удивление.
\vs Wis 8:12 Когда я буду молчать, они будут ожидать, и когда начну говорить, будут внимать, и когда продлю беседу, положат руку на уста свои.
\vs Wis 8:13 Чрез нее я достигну бессмертия и оставлю вечную память будущим после меня.
\vs Wis 8:14 Я буду управлять народами, и племена покорятся мне;
\vs Wis 8:15 убоятся меня, когда услышат обо мне страшные тираны; в народе явлюсь добрым и на войне мужественным.
\vs Wis 8:16 Войдя в дом свой, я успокоюсь ею, ибо в обращении ее нет суровости, ни в сожитии с нею скорби, но веселие и радость.
\vs Wis 8:17 Размышляя о сем сам в себе и обдумывая в сердце своем, что в родстве с премудростью~--- бессмертие,
\vs Wis 8:18 и в дружестве с нею~--- благое наслаждение, и в трудах рук ее~--- богатство неоскудевающее, и в собеседовании с нею~--- разум, и в общении слов ее~--- добрая слава,~--- я ходил и искал, как бы мне взять ее себе.
\vs Wis 8:19 Я был отрок даровитый и душу получил добрую;
\vs Wis 8:20 притом, будучи добрым, я вошел и в тело чистое.
\vs Wis 8:21 Познав же, что иначе не могу овладеть ею, как если дарует Бог,~--- и что уже было делом разума, чтобы познать, чей этот дар,~--- я обратился к Господу и молился Ему, и говорил от всего сердца моего:
\vs Wis 9:1 Боже отцов и Господи милости, сотворивший все словом Твоим
\vs Wis 9:2 и премудростию Твоею устроивший человека, чтобы он владычествовал над созданными Тобою тварями
\vs Wis 9:3 и управлял миром свято и справедливо, и в правоте души производил суд!
\vs Wis 9:4 Даруй мне приседящую престолу Твоему премудрость и не отринь меня от отроков Твоих,
\vs Wis 9:5 ибо я раб Твой и сын рабы Твоей, человек немощный и кратковременный и слабый в разумении суда и законов.
\vs Wis 9:6 Да хотя бы кто и совершен был между сынами человеческими, без Твоей премудрости он будет признан за ничто.
\vs Wis 9:7 Ты избрал меня царем народа Твоего и судьею сынов Твоих и дщерей;
\vs Wis 9:8 Ты сказал, чтобы я построил храм на святой горе Твоей и алтарь в городе обитания Твоего, по подобию святой скинии, которую Ты предуготовил от начала.
\vs Wis 9:9 С Тобою премудрость, которая знает дела Твои и присуща была, когда Ты творил мир, и ведает, что угодно пред очами Твоими и что право по заповедям Твоим:
\vs Wis 9:10 ниспошли ее от святых небес и от престола славы Твоей ниспошли ее, чтобы она споспешествовала мне в трудах моих, и чтобы я знал, что благоугодно пред Тобою;
\vs Wis 9:11 ибо она все знает и разумеет, и мудро будет руководить меня в делах моих, и сохранит меня в своей славе;
\vs Wis 9:12 и дела мои будут благоприятны, и буду судить народ Твой справедливо, и буду достойным престола отца моего.
\vs Wis 9:13 Ибо какой человек в состоянии познать совет Божий? или кто может уразуметь, что угодно Господу?
\vs Wis 9:14 Помышления смертных нетверды, и мысли наши ошибочны,
\vs Wis 9:15 ибо тленное тело отягощает душу, и эта земная храмина подавляет многозаботливый ум.
\vs Wis 9:16 Мы едва можем постигать и то, что на земле, и с трудом понимаем то, что под руками, а что на небесах~--- кто исследовал?
\vs Wis 9:17 Волю же Твою кто познал бы, если бы Ты не даровал премудрости и не ниспослал свыше святаго Твоего Духа?
\vs Wis 9:18 И так исправились пути живущих на земле, и люди научились тому, что угодно Тебе,
\vs Wis 9:19 и спаслись премудростью.
\vs Wis 10:1 Она сохраняла первозданного отца мира, который сотворен был один, и спасала его от собственного его падения:
\vs Wis 10:2 она дала ему силу владычествовать над всем.
\vs Wis 10:3 А отступивший от нее неправедный во гневе своем погиб от братоубийственной ярости.
\vs Wis 10:4 Ради него потопляемую землю опять премудрость спасла, сохранив праведника посредством малого дерева.
\vs Wis 10:5 Она же между народами, смешанными в единомыслии зла, нашла праведника и соблюла его неукоризненным пред Богом, и сохранила мужественным в жалости к сыну.
\vs Wis 10:6 Она во время погибели нечестивых спасла праведного, который избежал огня, нисшедшего на пять городов,
\vs Wis 10:7 от которых во свидетельство нечестия осталась дымящаяся пустая земля и растения, не в свое время приносящие плоды, и памятником неверной души~--- стоящий соляной столб.
\vs Wis 10:8 Ибо они, презрев премудрость, не только повредили себе тем, что не познали добра, но и оставили живущим память о своем безумии, дабы не могли скрыть того, в чем заблудились.
\vs Wis 10:9 Премудрость же спасла от бед служащих ей.
\vs Wis 10:10 Праведного, бежавшего от братнего гнева, она наставляла на правые пути, показала ему царство Божие и даровала ему познание святых, помогала ему в огорчениях и обильно вознаградила труды его.
\vs Wis 10:11 Когда из корыстолюбия обижали его, она предстала и обогатила его,
\vs Wis 10:12 сохранила его от врагов, и обезопасила от коварствовавших против него, и в крепкой борьбе доставила ему победу, дабы он знал, что благочестие всего сильнее.
\vs Wis 10:13 Она не оставила проданного праведника, но спасла его от греха:
\vs Wis 10:14 она нисходила с ним в ров и не оставляла его в узах, и потом принесла ему скипетр царства и власть над угнетавшими его, показала лжецами обвинявших его и даровала ему вечную славу.
\vs Wis 10:15 Она освободила святой народ и непорочное семя от народа угнетавших \bibemph{его},
\vs Wis 10:16 вошла в душу служителя Господня и противостала страшным царям чудесами и знамениями.
\vs Wis 10:17 Она воздала святым награду за труды их, вела их путем дивным; и днем была им покровом, а ночью~--- звездным светом.
\vs Wis 10:18 Она перевела их чрез Чермное море и провела их сквозь большую воду,
\vs Wis 10:19 а врагов их потопила и извергла их из глубины бездны.
\vs Wis 10:20 Итак, праведные завладели доспехами нечестивых и воспели святое имя Твое, Господи, и единодушно прославили поборающую руку Твою;
\vs Wis 10:21 ибо премудрость отверзла уста немых и сделала внятными языки младенцев.
\vs Wis 11:1 Она благоустроила дела их рукою святого пророка:
\vs Wis 11:2 они прошли по необитаемой пустыне, и на непроходных \bibemph{местах} поставили шатры;
\vs Wis 11:3 противостали неприятелям и отмстили врагам;
\vs Wis 11:4 томились жаждою и воззвали к Тебе, и дана им была вода из утесистой скалы и утоление жажды~--- из твердого камня.
\vs Wis 11:5 Ибо, чем наказаны были враги их,
\vs Wis 11:6 тем они, находясь в затруднении, были облагодетельствованы:
\vs Wis 11:7 вместо источника постоянно текущей реки, смрадною кровью возмущенной,
\vs Wis 11:8 в обличение их детоубийственного повеления, Ты неожиданно дал им обильную воду,
\vs Wis 11:9 показав тогда чрез жажду, как Ты наказал их противников.
\vs Wis 11:10 Ибо, когда они были испытываемы, подвергаясь, впрочем, милостивому вразумлению, тогда познали, как мучились во гневе судимые нечестивые;
\vs Wis 11:11 потому что их Ты испытывал, как отец, поучая, а тех, как гневный царь, осуждая, истязал.
\vs Wis 11:12 И отсутствовавшие и присутствовавшие одинаково пострадали:
\vs Wis 11:13 их постигла сугубая скорбь и стенание от воспоминания о прошедшем.
\vs Wis 11:14 Они, когда услышали, что чрез их наказания те были облагодетельствованы, познали Господа.
\vs Wis 11:15 Кого они прежде, как отверженного, отреклись с ругательством, Тому в последствие событий удивлялись, потерпев неодинаковую с праведными жажду.
\vs Wis 11:16 А за неразумные помышления их неправды, по которым они в заблуждении служили бессловесным пресмыкающимся и презренным чудовищам, Ты в наказание наслал на них множество бессловесных животных,
\vs Wis 11:17 чтобы они познали, что, чем кто согрешает, тем и наказывается.
\vs Wis 11:18 Не невозможно было бы для всемогущей руки Твоей, создавшей мир из необразного вещества, наслать на них множество медведей или свирепых львов,
\vs Wis 11:19 или неизвестных новосозданных лютых зверей, или дышащих огненным дыханием, или извергающих клубы дыма, или бросающих из глаз ужасные искры,
\vs Wis 11:20 которые не только повреждением могли истребить их, но и ужасающим видом погубить.
\vs Wis 11:21 Да и без этого они могли погибнуть от одного дуновения, преследуемые правосудием и рассеваемые духом силы Твоей; но Ты все расположил мерою, числом и весом.
\vs Wis 11:22 Ибо великая сила всегда присуща Тебе, и кто противостанет силе мышцы Твоей?
\vs Wis 11:23 Весь мир пред Тобою, как колебание чашки весов, или как капля утренней росы, сходящей на землю.
\vs Wis 11:24 Ты всех милуешь, потому что все можешь, и покрываешь грехи людей ради покаяния.
\vs Wis 11:25 Ты любишь все существующее, и ничем не гнушаешься, что сотворил, ибо не создал бы, если бы что ненавидел.
\vs Wis 11:26 И как могло бы пребывать что-либо, если бы Ты не восхотел? Или как сохранилось бы то, что не было призвано Тобою?
\vs Wis 11:27 Но Ты все щадишь, потому что все Твое, душелюбивый Господи.
\vs Wis 12:1 Нетленный Твой дух пребывает во всем.
\vs Wis 12:2 Посему заблуждающихся Ты мало-помалу обличаешь и, напоминая \bibemph{им}, в чем они согрешают, вразумляешь, чтобы они, отступив от зла, уверовали в Тебя, Господи.
\vs Wis 12:3 Так, возгнушавшись древними обитателями святой земли Твоей,
\vs Wis 12:4 совершавшими ненавистные дела волхвований и нечестивые жертвоприношения,
\vs Wis 12:5 и безжалостными убийцами детей, и на жертвенных пирах пожиравшими внутренности человеческой плоти и крови в тайных собраниях,
\vs Wis 12:6 и родителями, убивавшими беспомощные души,~--- Ты восхотел погубить \bibemph{их} руками отцов наших,
\vs Wis 12:7 дабы земля, драгоценнейшая всех у Тебя, приняла достойное население чад Божиих.
\vs Wis 12:8 Но и их, как людей, Ты щадил, послав предтечами воинства Твоего шершней, дабы они мало-помалу истребляли их.
\vs Wis 12:9 Хотя не невозможно было Тебе войною покорить нечестивых праведным, или истребить их страшными зверями, или грозным словом в один раз;
\vs Wis 12:10 но Ты, мало-помалу наказывая \bibemph{их}, давал место покаянию, зная, однако, что племя их негодное и зло их врожденное, и помышление их не изменится во веки.
\vs Wis 12:11 Ибо семя их было проклятое от начала, и не из опасения перед кем-либо Ты допускал безнаказанность грехов их.
\vs Wis 12:12 Ибо кто скажет: <<что Ты сделал?>> или кто противостанет суду Твоему? и кто обвинит Тебя в погублении народов, которых Ты сотворил? Или какой защитник придет к Тебе с ходатайством за неправедных людей?
\vs Wis 12:13 Ибо кроме Тебя нет Бога, который имеет попечение о всех, чтобы доказывать Тебе, что Ты несправедливо судил.
\vs Wis 12:14 Ни царь, ни властелин не в состоянии явиться к Тебе на глаза за тех, которых Ты погубил.
\vs Wis 12:15 Будучи праведен, Ты всем управляешь праведно, почитая не свойственным Твоей силе осудить того, кто не заслуживает наказания.
\vs Wis 12:16 Ибо сила Твоя есть начало правды, и то самое, что Ты господствуешь над всеми, располагает Тебя щадить всех.
\vs Wis 12:17 Силу Твою Ты показываешь не верующим всемогуществу Твоему и в не признающих Тебя обличаешь дерзость;
\vs Wis 12:18 но, обладая силою, Ты судишь снисходительно и управляешь нами с великою милостью, ибо могущество Твое всегда в Твоей воле.
\vs Wis 12:19 Но такими делами Ты поучал народ Твой, что праведному должно быть человеколюбивым, и внушал сынам Твоим благую надежду, что Ты даешь время покаянию во грехах.
\vs Wis 12:20 Ибо, если врагов сынам Твоим и повинных смерти Ты наказывал с таким снисхождением и пощадою, давая \bibemph{им} время и побуждение освободиться от зла,
\vs Wis 12:21 то с каким вниманием Ты судил сынов Твоих, которых отцам Ты дал клятвы и заветы благих обетований!
\vs Wis 12:22 Итак, вразумляя нас, Ты наказываешь врагов наших тысячекратно, дабы мы, когда судим, помышляли о Твоей благости и, когда бываем судимы, ожидали помилования.
\vs Wis 12:23 Посему-то и тех нечестивых, которые проводили жизнь в неразумии, Ты истязал собственными их мерзостями,
\vs Wis 12:24 ибо они очень далеко уклонились на путях заблуждения, обманываясь подобно неразумным детям и почитая за богов тех из животных, которые и у врагов были презренными.
\vs Wis 12:25 Посему, как неразумным детям, в посмеяние послал Ты им и наказание.
\vs Wis 12:26 Но, не вразумившись обличительным посмеянием, они испытывали заслуженный суд Божий.
\vs Wis 12:27 Ибо, что они сами терпели с досадою, то же увидев на тех, которых считали богами и чрез которых были наказываемы, они познали Бога истинного, Которого прежде отрекались знать;
\vs Wis 12:28 посему и пришло на них окончательное осуждение.
\vs Wis 13:1 Подлинно суетны по природе все люди, у которых не было ведения о Боге, которые из видимых совершенств не могли познать Сущего и, взирая на дела, не познали Виновника,
\vs Wis 13:2 а почитали за богов, правящих миром, или огонь, или ветер, или движущийся воздух, или звездный круг, или бурную воду, или небесные светила.
\vs Wis 13:3 Если, пленяясь их красотою, они почитали их за богов, то должны были бы познать, сколько лучше их Господь, ибо Он, Виновник красоты, создал их.
\vs Wis 13:4 А если удивлялись силе и действию их, то должны были бы узнать из них, сколько могущественнее Тот, Кто сотворил их;
\vs Wis 13:5 ибо от величия красоты созданий сравнительно познается Виновник бытия их.
\vs Wis 13:6 Впрочем, они меньше заслуживают порицания, ибо заблуждаются, может быть, ища Бога и желая найти Его:
\vs Wis 13:7 потому что, обращаясь к делам Его, они исследуют и убеждаются зрением, что все видимое прекрасно.
\vs Wis 13:8 Но и они неизвинительны:
\vs Wis 13:9 если они столько могли разуметь, что в состоянии были исследовать временный мир, то почему они тотчас не обрели Господа его?
\vs Wis 13:10 Но более жалки те, и надежды их~--- на бездушных, которые называют богами дела рук человеческих, золото и серебро, изделия художества, изображения животных, или негодный камень, дело давней руки.
\vs Wis 13:11 Или какой-либо древодел, вырубив годное дерево, искусно снял с него всю кору и, обделав красиво, устроил из него сосуд, полезный к употреблению в жизни,
\vs Wis 13:12 а обрезки от работы употребил на приготовление пищи и насытился;
\vs Wis 13:13 один же из обрезков, ни к чему не годный, дерево кривое и сучковатое, взяв, старательно округлил на досуге и, с опытностью знатока обделав его, уподобил его образу человека,
\vs Wis 13:14 или сделал подобным какому-нибудь низкому животному, намазал суриком и покрыл краскою поверхность его, и закрасил в нем всякий недостаток,
\vs Wis 13:15 и, устроив для него достойное его место, повесил его на стене, укрепив железом.
\vs Wis 13:16 Итак, чтобы \bibemph{произведение} его не упало, он наперед озаботился, зная, что оно само себе помочь не может, ибо это кумир и имеет нужду в помощи.
\vs Wis 13:17 Молясь же \bibemph{пред ним} о своих стяжаниях, о браке и о детях, он не стыдится говорить бездушному,
\vs Wis 13:18 и о здоровье взывает к немощному, о жизни просит мертвое, о помощи умоляет совершенно неспособное, о путешествии~--- не могущее ступить,
\vs Wis 13:19 о прибытке, о ремесле и об успехе рук~--- совсем не могущее делать руками, о силе просит самое бессильное.
\vs Wis 14:1 Еще: иной, собираясь плыть и переплывать свирепые волны, призывает на помощь дерево, слабейшее носящего его корабля;
\vs Wis 14:2 ибо стремление к приобретениям выдумало оный, а художник искусно устроил,
\vs Wis 14:3 но промысл Твой, Отец, управляет кораблем, ибо Ты дал и путь \bibemph{в море} и безопасную стезю в волнах,
\vs Wis 14:4 показывая, что Ты можешь от всего спасать, хотя бы кто отправлялся \bibemph{в море} и без искусства.
\vs Wis 14:5 Ты хочешь, чтобы не тщетны были дела Твоей премудрости; поэтому люди вверяют свою жизнь малейшему дереву и спасаются, проходя по волнам на ладье.
\vs Wis 14:6 Ибо и вначале, когда погубляемы были гордые исполины, надежда мира, управленная Твоею рукою, прибегнув к кораблю, оставила миру семя рода.
\vs Wis 14:7 Благословенно дерево, чрез которое бывает правда!
\vs Wis 14:8 А это рукотворенное проклято и само, и сделавший его~--- за то, что сделал; а это тленное названо богом.
\vs Wis 14:9 Ибо равно ненавистны Богу и нечестивец и нечестие его;
\vs Wis 14:10 и сделанное вместе со сделавшим будет наказано.
\vs Wis 14:11 Посему и на идолов языческих будет суд, так как они среди создания Божия сделались мерзостью, соблазном душ человеческих и сетью ногам неразумных.
\vs Wis 14:12 Ибо вымысл идолов~--- начало блуда, и изобретение их~--- растление жизни.
\vs Wis 14:13 Не было их вначале, и не во веки они будут.
\vs Wis 14:14 Они вошли в мир по человеческому тщеславию, и потому близкий сужден им конец.
\vs Wis 14:15 Отец, терзающийся горькою скорбью о рано умершем сыне, сделав изображение его, как уже мертвого человека, затем стал почитать его, как бога, и передал подвластным тайны и жертвоприношения.
\vs Wis 14:16 Потом утвердившийся временем этот нечестивый обычай соблюдаем был, как закон, и по повелениям властителей изваяние почитаемо было, как божество.
\vs Wis 14:17 Кого в лицо люди не могли почитать по отдаленности жительства, того отдаленное лицо они изображали: делали видимый образ почитаемого царя, дабы этим усердием польстить отсутствующему, как бы присутствующему.
\vs Wis 14:18 К усилению же почитания и от незнающих поощряло тщание художника,
\vs Wis 14:19 ибо он, желая, может быть, угодить властителю, постарался искусством сделать подобие покрасивее;
\vs Wis 14:20 а народ, увлеченный красотою отделки, незадолго пред тем почитаемого, как человека, признал теперь божеством.
\vs Wis 14:21 И это было соблазном для людей, потому что они, покоряясь или несчастью, или тиранству, несообщимое Имя прилагали к камням и деревам.
\vs Wis 14:22 Потом не довольно было для них заблуждаться в познании о Боге, но они, живя в великой борьбе невежества, такое великое зло называют миром.
\vs Wis 14:23 Совершая или детоубийственные жертвы, или скрытные тайны, или \bibemph{заимствованные} от чужих обычаев неистовые пиршества,
\vs Wis 14:24 они не берегут ни жизни, ни чистых браков, но один другого или коварством убивает, или прелюбодейством обижает.
\vs Wis 14:25 Всеми же без различия обладают кровь и убийство, хищение и коварство, растление, вероломство, мятеж, клятвопреступление, расхищение имуществ,
\vs Wis 14:26 забвение благодарности, осквернение душ, превращение полов, бесчиние браков, прелюбодеяние и распутство.
\vs Wis 14:27 Служение идолам, недостойным именования, есть начало и причина, и конец всякого зла,
\vs Wis 14:28 ибо они или веселясь неистовствуют, или прорицают ложь, или живут беззаконно, или скоро нарушают клятву.
\vs Wis 14:29 Надеясь на бездушных идолов, они не думают быть наказанными за то, что несправедливо клянутся.
\vs Wis 14:30 Но за то и другое придет на них осуждение, \bibemph{и за то}, что нечестиво мыслили о Боге, обращаясь к идолам, и \bibemph{за то}, что ложно клялись, коварно презирая святое.
\vs Wis 14:31 Ибо не сила тех, которыми они клянутся, но суд над согрешающими следует всегда за преступлением неправедных.
\vs Wis 15:1 Но Ты, Бог наш, благ и истинен, долготерпелив и управляешь всем милостиво.
\vs Wis 15:2 Если мы и согрешаем, мы~--- Твои, признающие власть Твою; но мы не будем грешить, зная, что мы признаны Твоими.
\vs Wis 15:3 Знать Тебя есть полная праведность, и признавать власть Твою~--- корень бессмертия.
\vs Wis 15:4 Не обольщает нас лукавое человеческое изобретение, ни бесплодный труд художников~--- изображения, испещренные различными красками,
\vs Wis 15:5 взгляд на которые возбуждает в безумных похотение и вожделение к бездушному виду мертвого образа.
\vs Wis 15:6 И делающие, и похотствующие, и чествующие суть любители зла, достойные таких надежд.
\vs Wis 15:7 Горшечник мнет мягкую землю, заботливо лепит всякий \bibemph{сосуд} на службу нашу; из одной и той же глины выделывает сосуды, потребные и для чистых дел и для нечистых~--- все одинаково; но какое каждого из них употребление, судья~--- тот же горшечник.
\vs Wis 15:8 И суетный труженик из той же глины лепит суетного бога, тогда как сам недавно родился из земли и вскоре пойдет туда же, откуда он взят, и взыщется с него долг души его.
\vs Wis 15:9 Но у него забота не о том, что он должен много трудиться, и не о том, что жизнь его кратка; но он соревнует художникам золотых и серебряных изделий, и подражает медникам, и вменяет себе в славу, что делает мерзости.
\vs Wis 15:10 Сердце его~--- пепел, и надежда его ничтожнее земли, и жизнь его презреннее грязи;
\vs Wis 15:11 ибо он не познал Сотворившего его и вдунувшего в него деятельную душу и вдохнувшего в него дух жизни.
\vs Wis 15:12 Они считают жизнь нашу забавою и житие прибыльною торговлею, ибо говорят, что должно же откуда-либо извлекать прибыль, хотя бы и из зла.
\vs Wis 15:13 Впрочем такой более всех знает, что он грешит, делая из земляного вещества бренные сосуды и изваяния.
\vs Wis 15:14 Самые же неразумные из всех и беднее умом самых младенцев~--- враги народа Твоего, угнетающие его,
\vs Wis 15:15 потому что они почитают богами всех идолов языческих, у которых нет употребления ни глаз для зрения, ни ноздрей для привлечения воздуха, ни ушей для слышания, ни перстов рук для осязания и которых ноги негодны для хождения.
\vs Wis 15:16 Хотя человек сделал их, и заимствовавший дух образовал их, но никакой человек не может образовать бога, как он сам.
\vs Wis 15:17 Будучи смертным, он делает нечестивыми руками мертвое, поэтому он превосходнее божеств своих, ибо он жил, а те~--- никогда.
\vs Wis 15:18 Притом они почитают животных самых отвратительных, которые по бессмыслию сравнительно хуже всех.
\vs Wis 15:19 Они даже некрасивы по виду, как \bibemph{другие} животные, чтобы могли привлекать к себе, но лишены и одобрения Божия и благословения Его.
\vs Wis 16:1 Посему они достойно были наказаны чрез подобных \bibemph{животных} и терзаемы множеством чудовищ.
\vs Wis 16:2 Вместо такого наказания Ты благодетельствовал народу Твоему: в удовлетворение прихоти их Ты приготовил им в насыщение необычайную пищу~--- перепелов,
\vs Wis 16:3 дабы те, мучимые голодом, по отвратительному виду насланных \bibemph{гадов}, отказывали и необходимому позыву на пищу, а эти, кратковременно потерпев недостаток, вкусили необычайной пищи.
\vs Wis 16:4 Ибо тех притеснителей должен был постигнуть неотвратимый недостаток, а этим только нужно было показать, как мучились враги их.
\vs Wis 16:5 И тогда, как постигла их ужасная ярость зверей и они были истребляемы угрызениями коварных змиев, гнев Твой не продолжился до конца.
\vs Wis 16:6 Но они были смущены на краткое время для вразумления, получив знамение спасения на воспоминание о заповеди закона Твоего,
\vs Wis 16:7 ибо обращавшийся исцелялся не тем, на что взирал, но Тобою, Спасителем всех.
\vs Wis 16:8 И этим Ты показал врагам нашим, что Ты~--- избавляющий от всякого зла:
\vs Wis 16:9 ибо их убивали уязвления саранчи и мух, и не нашлось врачевства для души их, потому что они достойны были мучения от сих.
\vs Wis 16:10 А сынов Твоих не одолели и зубы ядовитых змиев, ибо милость Твоя пришла на помощь и исцелила их.
\vs Wis 16:11 Хотя они и были уязвляемы в напоминание им слов Твоих, но скоро были и исцеляемы, дабы, впав в глубокое забвение \bibemph{оных}, не лишились Твоего благодеяния.
\vs Wis 16:12 Не трава и не пластырь врачевали их, но Твое, Господи, всеисцеляющее слово.
\vs Wis 16:13 Ты имеешь власть жизни и смерти и низводишь до врат ада и возводишь.
\vs Wis 16:14 Человек по злобе своей убивает, но не может возвратить исшедшего духа и не может призвать взятой души.
\vs Wis 16:15 А Твоей руки невозможно избежать,
\vs Wis 16:16 ибо нечестивые, отрекшиеся познать Тебя, наказаны силою мышцы Твоей, быв преследуемы необыкновенными дождями, градами и неотвратимыми бурями и истребляемы огнем.
\vs Wis 16:17 Но самое чудное было то, что огонь сильнее оказывал действие в воде, все погашающей, ибо самый мир есть поборник за праведных.
\vs Wis 16:18 Иногда пламя укрощалось, чтобы не сжечь животных, посланных на нечестивых, и чтобы они, видя это, познали, что преследуются судом Божиим.
\vs Wis 16:19 А иногда и среди воды жгло сильнее огня, дабы истребить произведения земли неправедной.
\vs Wis 16:20 Вместо того народ Твой Ты питал пищею ангельскою и послал им, нетрудящимся, с неба готовый хлеб, имевший всякую приятность по вкусу каждого.
\vs Wis 16:21 Ибо свойство пищи Твоей показывало Твою любовь к детям и в удовлетворение желания вкушающего изменялось по вкусу каждого.
\vs Wis 16:22 А снег и лед выдерживали огонь и не таяли, дабы они знали, что огонь, горящий в граде и блистающий в дождях, истреблял плоды врагов.
\vs Wis 16:23 Но тот же огонь, дабы напитались праведные, терял свою силу.
\vs Wis 16:24 Ибо тварь, служа Тебе, Творцу, устремляется к наказанию нечестивых и утихает для благодеяния верующим в Тебя.
\vs Wis 16:25 Посему и тогда она, изменяясь во всё, повиновалась Твоей благодати, питающей всех, по желанию нуждающихся,
\vs Wis 16:26 дабы сыны Твои, которых Ты, Господи, возлюбил, познали, что не роды плодов питают человека, но слово Твое сохраняет верующих в Тебя.
\vs Wis 16:27 Ибо неповреждаемое огнем, будучи согреваемо слабым солнечным лучом, тотчас растаявало,
\vs Wis 16:28 дабы известно было, что должно предупреждать солнце благодарением Тебе и обращаться к Тебе на восток света.
\vs Wis 16:29 Ибо надежда неблагодарного растает, как зимний иней, и выльется, как негодная вода.
\vs Wis 17:1 Велики и непостижимы суды Твои, посему ненаученные души впали в заблуждение.
\vs Wis 17:2 Ибо беззаконные, которые задумали угнетать святой народ, узники тьмы и пленники долгой ночи, затворившись в домах, скрывались от вечного Промысла.
\vs Wis 17:3 Думая укрыться в тайных грехах, они, под темным покровом забвения, рассеялись, сильно устрашаемые и смущаемые призраками,
\vs Wis 17:4 ибо и самое потаенное место, заключавшее их, не спасало их от страха, но страшные звуки вокруг них приводили их в смущение, и являлись свирепые чудовища со страшными лицами.
\vs Wis 17:5 И никакая сила огня не могла озарить, ни яркий блеск звезд не в состоянии был осветить этой мрачной ночи.
\vs Wis 17:6 Являлись им только сами собою горящие костры, полные ужаса, и они, страшась невидимого~--- призрака, представляли себе видимое еще худшим.
\vs Wis 17:7 Пали обольщения волшебного искусства, и хвастовство мудростью подверглось посмеянию,
\vs Wis 17:8 ибо обещавшиеся отогнать от страдавшей души ужасы и страхи, сами страдали позорною боязливостью.
\vs Wis 17:9 И хотя никакие устрашения не тревожили их, но, преследуемые брожениями ядовитых зверей и свистами пресмыкающихся, они исчезали от страха, боясь взглянуть даже на воздух, от которого никуда нельзя убежать,
\vs Wis 17:10 ибо осуждаемое собственным свидетельством нечестие боязливо и, преследуемое совестью, всегда придумывает ужасы.
\vs Wis 17:11 Страх есть не что иное, как лишение помощи от рассудка.
\vs Wis 17:12 Чем меньше надежды внутри, тем больше представляется неизвестность причины, производящей мучение.
\vs Wis 17:13 И они в эту истинно невыносимую и из глубин нестерпимого ада исшедшую ночь, располагаясь заснуть обыкновенным сном,
\vs Wis 17:14 то были тревожимы страшными призраками, то расслабляемы душевным унынием, ибо находил на них внезапный и неожиданный страх.
\vs Wis 17:15 Итак, где кто тогда был застигнут, делался пленником и заключаем был в эту темницу без оков.
\vs Wis 17:16 Был ли то земледелец или пастух, или занимающийся работами в пустыне, всякий, быв застигнут, подвергался этой неизбежной судьбе,
\vs Wis 17:17 ибо все были связаны одними неразрешимыми узами тьмы. Свищущий ли ветер, или среди густых ветвей сладкозвучный голос птиц, или сила быстро текущей воды, или сильный треск низвергающихся камней,
\vs Wis 17:18 или незримое бегание скачущих животных, или голос ревущих свирепейших зверей, или отдающееся из горных углублений эхо, \bibemph{все это}, ужасая их, повергало в расслабление.
\vs Wis 17:19 Ибо весь мир был освещаем ясным светом и занимался беспрепятственно делами;
\vs Wis 17:20 а над ними одними была распростерта тяжелая ночь, образ тьмы, имевшей некогда объять их; но сами для себя они были тягостнее тьмы.
\vs Wis 18:1 А для святых Твоих был величайший свет. И те, слыша голос их, а образа не видя, называли их блаженными, потому что они не страдали.
\vs Wis 18:2 А за то, что, быв прежде обижаемы ими, не мстили им, благодарили и просили прощения в том, что заставляли переносить их.
\vs Wis 18:3 Вместо того, Ты дал им указателем на незнакомом пути огнесветлый столп, а для благополучного странствования~--- безвредное солнце.
\vs Wis 18:4 Ибо те достойны были лишения света и заключения во тьме, потому что держали в заключении сынов Твоих, чрез которых имел быть дан миру нетленный свет закона.
\vs Wis 18:5 Когда определили они избить детей святых, хотя одного сына покинутого и спасли, в наказание за то Ты отнял множество их детей и самих всех погубил в сильной воде.
\vs Wis 18:6 Та ночь была предвозвещена отцам нашим, дабы они, твердо зная обетования, каким верили, были благодушны.
\vs Wis 18:7 И народ Твой ожидал как спасения праведных, так и погибели врагов,
\vs Wis 18:8 ибо, чем Ты наказывал врагов, тем самым возвеличил нас, которых Ты призвал.
\vs Wis 18:9 Святые дети добрых тайно совершали жертвоприношение и единомысленно постановили божественным законом, чтобы святые равно участвовали в одних и тех же благах и опасностях, когда отцы уже воспевали хвалы.
\vs Wis 18:10 С противной же стороны отдавался нестройный крик врагов, и разносился жалобный вопль над оплакиваемыми детьми.
\vs Wis 18:11 Одинаковым судом был наказан раб с господином, и простолюдин терпел одно и то же с царем:
\vs Wis 18:12 все вообще имели бесчисленных мертвецов, \bibemph{умерших} одинаковою смертью; и живых недоставало для погребения, так как в одно мгновение погублено было \bibemph{все} драгоценнейшее их поколение.
\vs Wis 18:13 И не верившие ничему ради чародейства, при погублении первенцев, признали, что \bibemph{этот} народ есть сын Божий,
\vs Wis 18:14 ибо, когда все окружало тихое безмолвие и ночь в своем течении достигла средины,
\vs Wis 18:15 сошло с небес от царственных престолов на средину погибельной земли всемогущее слово Твое, как грозный воин.
\vs Wis 18:16 Оно несло острый меч~--- неизменное Твое повеление и, став, наполнило все смертью: оно касалось неба и ходило по земле.
\vs Wis 18:17 Тогда вдруг сильно встревожили их мечты сновидений, и наступили неожиданные ужасы;
\vs Wis 18:18 и, будучи поражаем~--- один там, другой тут, полумертвый объявлял причину, по которой он умирал:
\vs Wis 18:19 ибо встревожившие их сновидения предварительно показали \bibemph{им} это, чтобы они не погибли, не зная того, за что терпят зло.
\vs Wis 18:20 Хотя искушение смерти коснулось и праведных, и много их погибло в пустыне, но недолго продолжался этот гнев,
\vs Wis 18:21 ибо непорочный муж поспешил защитить их; принеся оружие своего служения, молитву и умилостивление кадильное, он противостал гневу и положил конец бедствию, показав тем, что он слуга Твой.
\vs Wis 18:22 Он победил истребителя не силою телесною и не действием оружия, но словом покорил наказывавшего, воспомянув клятвы и заветы отцов.
\vs Wis 18:23 Ибо, когда уже грудами лежали мертвые одни на других, он, став в средине, остановил гнев и пресек \bibemph{ему} путь к живым.
\vs Wis 18:24 На подире его был целый мир, и славные \bibemph{имена} отцов были вырезаны на камнях в четыре ряда, и величие Твое~--- на диадиме головы его.
\vs Wis 18:25 Этому уступил истребитель, и этого убоялся: ибо довольно было одного этого испытания гневного.
\vs Wis 19:1 А над нечестивыми до конца тяготел немилостивый гнев, ибо Он предвидел и будущие их \bibemph{дела},
\vs Wis 19:2 что они, позволив им отправиться и с поспешностью выслав их, раскаются и погонятся за ними,
\vs Wis 19:3 ибо, еще имея в руках печали и рыдая над гробами мертвых, они возымели другой безумный помысл, и тех, кого с мольбою высылали, преследовали, как беглецов.
\vs Wis 19:4 Влекла же их к тому концу судьба, которой они были достойны, и она навела забвение о случившемся, дабы они восполнили наказание, недостававшее к их мучениям,
\vs Wis 19:5 и дабы народ Твой совершил славное путешествие, а они нашли себе необычайную смерть.
\vs Wis 19:6 Ибо вся тварь снова свыше преобразовалась в своей природе, повинуясь особым повелениям, дабы сыны Твои сохранились невредимыми.
\vs Wis 19:7 Явилось облако, осеняющее стан, а где стояла прежде вода, показалась сухая земля, из Чермного моря~--- беспрепятственный путь, и из бурной пучины~--- зеленая долина.
\vs Wis 19:8 Покрываемые Твоею рукою, они прошли по ней всем народом, видя дивные чудеса.
\vs Wis 19:9 Они паслись как кони и играли как агнцы, славя Тебя, Господи, Избавителя их,
\vs Wis 19:10 ибо они еще помнили о том, что случилось во время пребывания их там, как земля вместо рождения \bibemph{других} животных произвела скнипов и река вместо рыб извергла множество жаб.
\vs Wis 19:11 А после они увидели и новый род птиц, когда, увлекшись пожеланием, просили приятной пищи,
\vs Wis 19:12 ибо в утешение им налетели с моря перепелы, а грешных постигли наказания не без знамений, бывших силою молний. Они справедливо страдали за свою злобу,
\vs Wis 19:13 ибо они более сильную питали ненависть к чужеземцам: иные не принимали незнаемых странников, а эти порабощали благодетельных пришельцев.
\vs Wis 19:14 И мало этого, но еще будет суд на них за то, что те враждебно принимали чужих,
\vs Wis 19:15 а эти, с радостью приняв, потом уже пользовавшихся одинаковыми правами стали угнетать ужасными работами.
\vs Wis 19:16 Посему они поражены были слепотою, как те \bibemph{некогда} при дверях праведника, когда, будучи объяты густою тьмою, искали каждый входа в его двери.
\vs Wis 19:17 Самые стихии изменились, как в арфе звуки изменяют свой характер, всегда оставаясь теми же звуками; это можно усмотреть чрез тщательное наблюдение бывшего.
\vs Wis 19:18 Ибо земные \bibemph{животные} переменялись в водяные, а плавающие в водах выходили на землю.
\vs Wis 19:19 Огонь в воде удерживал свою силу, а вода теряла угашающее свое свойство;
\vs Wis 19:20 пламя, наоборот, не вредило телам бродящих удоборазрушимых животных, и не таял легко растаявающий снеговидный род небесной пищи.
\vs Wis 19:21 Так, Господи, Ты во всем возвеличил и прославил народ Твой, и не оставлял его, но во всякое время и на всяком месте пребывал с ним.

\bibbookdescr{Sir}{
  inline={\LARGE Книга\\\Huge Премудрости Иисуса,\\сына Сирахова\fns{Переведена с греческого.}},
  toc={Сирах*},
  bookmark={Сирах},
  header={Сирах},
  %headerleft={},
  %headerright={},
  abbr={Сир}
}
\chhdr{Предисловие\fns{Предисловие к греческому переводу, имеющееся у 70-ти и содержащееся в Славянской Библии.}}
\vs Sir 0:0 Многое и великое дано нам через закон, пророков и прочих \bibemph{писателей}, следовавших за ними, за что должно прославлять \bibemph{народ} Израильский за образованность и мудрость; и не только сами изучающие должны делаться разумными, но и находящимся вне [Палестины] усердно занимающиеся [писанием] могут приносить пользу словом и писанием. Поэтому дед мой Иисус, больше других предаваясь изучению закона, пророков и других отеческих книг и приобретя достаточный в них навык, решился и сам написать нечто, относящееся к образованию и мудрости, чтобы любители учения, вникая и в эту [книгу], еще более преуспевали в жизни по закону. Итак, прошу вас, читайте [эту книгу] благосклонно и внимательно и имейте снисхождение к тому, что в некоторых местах мы, может быть, погрешили, трудясь над переводом: ибо неодинаковый смысл имеет то, что читается по-еврейски, когда переведено будет на другой язык,~--- и не только эта [книга], но даже закон, пророчества и остальные книги имеют немалую разницу в смысле, если читать их в подлиннике. Прибыв в Египет в тридцать восьмом году при царе Евергете [Птоломее] и пробыв там, я нашел немалую разницу в образовании [между палестинскими и египетскими евреями], и счел крайне необходимым и самому приложить усердие к тому, чтобы перевести эту книгу. Много бессонного труда и знаний положил я в это время, чтобы довести книгу до конца и сделать ее доступною и тем, которые, находясь на чужбине, желают учиться и приспособляют свои нравы к тому, чтобы жить по закону.
\vs Sir 1:1 Всякая премудрость~--- от Господа и с Ним пребывает вовек.
\vs Sir 1:2 Песок морей и капли дождя и дни вечности кто исчислит?
\vs Sir 1:3 Высоту неба и широту земли, и бездну и премудрость кто исследует?
\vs Sir 1:4 Прежде всего произошла Премудрость, и разумение мудрости~--- от века.
\vs Sir 1:5 Источник премудрости~--- слово Бога Всевышнего, и шествие ее~--- вечные заповеди.
\vs Sir 1:6 Кому открыт корень премудрости? и кто познал искусство ее?
\vs Sir 1:7 Один есть премудрый, весьма страшный, сидящий на престоле Своем, Господь.
\vs Sir 1:8 Он произвел ее и видел и измерил ее
\vs Sir 1:9 и излил ее на все дела Свои
\vs Sir 1:10 и на всякую плоть по дару Своему, и особенно наделил ею любящих Его.
\vs Sir 1:11 Страх Господень~--- слава и честь, и веселие и венец радости.
\vs Sir 1:12 Страх Господень усладит сердце и даст веселие и радость и долгоденствие.
\vs Sir 1:13 Боящемуся Господа благо будет напоследок, и в день смерти своей он получит благословение. Страх Господень~--- дар от Господа и поставляет на стезях любви.
\vs Sir 1:14 Любовь к Господу~--- славная премудрость, и кому благоволит Он, разделяет ее по Своему усмотрению.
\vs Sir 1:15 Начало премудрости~--- бояться Бога, и с верными она образуется вместе во чреве. Среди людей она утвердила себе вечное основание и семени их вверится.
\vs Sir 1:16 Полнота премудрости~--- бояться Господа; она напояет их от плодов своих:
\vs Sir 1:17 весь дом их она наполнит всем, чего желают, и кладовые их~--- произведениями своими.
\vs Sir 1:18 Венец премудрости~--- страх Господень, произращающий мир и невредимое здравие; но то и другое~--- дары Бога, Который распространяет славу любящих Его.
\vs Sir 1:19 Он видел ее и измерил, пролил как дождь в\acc{е}дение и разумное знание и возвысил славу обладающих ею.
\vs Sir 1:20 Корень премудрости~--- бояться Господа, а ветви ее~--- долгоденствие.
\rsbpar\vs Sir 1:21 Страх Господень отгоняет грехи; не имеющий же страха не может оправдаться.
\vs Sir 1:22 Не может быть оправдан несправедливый гнев, ибо \bibemph{самое} движение гнева есть падение для человека.
\vs Sir 1:23 Терпеливый до времени удержится и после вознаграждается веселием.
\vs Sir 1:24 До времени он скроет слова свои, и уста верных расскажут о благоразумии его.
\vs Sir 1:25 В сокровищницах премудрости~--- притчи разума, грешнику же страх Господень ненавистен.
\vs Sir 1:26 Если желаешь премудрости, соблюдай заповеди, и Господь подаст ее тебе,
\vs Sir 1:27 ибо премудрость и знание есть страх пред Господом, и благоугождение Ему~--- вера и кротость.
\vs Sir 1:28 Не будь недоверчивым к страху пред Господом и не приступай к Нему с раздвоенным сердцем.
\vs Sir 1:29 Не лицемерь пред устами других и будь внимателен к устам твоим.
\vs Sir 1:30 Не возноси себя, чтобы не упасть и не навлечь бесчестия на душу твою, ибо Господь откроет тайны твои и уничижит тебя среди собрания за то, что ты не приступил искренно к страху Господню, и сердце твое полно лукавства.
\vs Sir 2:1 Сын мой! если ты приступаешь служить Господу Богу, то приготовь душу твою к искушению:
\vs Sir 2:2 управь сердце твое и будь тверд, и не смущайся во время посещения;
\vs Sir 2:3 прилепись к Нему и не отступай, дабы возвеличиться тебе напоследок.
\vs Sir 2:4 Все, что ни приключится тебе, принимай охотно, и в превратностях твоего уничижения будь долготерпелив,
\vs Sir 2:5 ибо золото испытывается в огне, а люди, угодные Богу,~--- в горниле уничижения.
\vs Sir 2:6 Веруй Ему, и Он защитит тебя; управь пути твои и надейся на Него.
\vs Sir 2:7 Боящиеся Господа! ожидайте милости Его и не уклоняйтесь \bibemph{от Него}, чтобы не упасть.
\vs Sir 2:8 Боящиеся Господа! веруйте Ему, и не погибнет награда ваша.
\vs Sir 2:9 Боящиеся Господа! надейтесь на благое, на радость вечную и милости.
\vs Sir 2:10 Взгляните на древние роды и посмотрите: кто верил Господу~--- и был постыжен? или кто пребывал в страхе Его~--- и был оставлен? или кто взывал к Нему, и Он презрел его?
\vs Sir 2:11 Ибо Господь сострадателен и милостив и прощает грехи, и спасает во время скорби.
\vs Sir 2:12 Горе сердцам боязливым и рукам ослабленным и грешнику, ходящему по двум стезям!
\vs Sir 2:13 Горе сердцу расслабленному! ибо оно не верует, и за то не будет защищено.
\vs Sir 2:14 Горе вам, потерявшим терпение! что будете вы делать, когда Господь посетит?
\vs Sir 2:15 Боящиеся Господа не будут недоверчивы к словам Его, и любящие Его сохранят пути Его.
\vs Sir 2:16 Боящиеся Господа будут искать благоволения Его, и любящие Его насытятся законом.
\vs Sir 2:17 Боящиеся Господа уготовят сердца свои и смирят пред Ним души свои, говоря:
\vs Sir 2:18 впадем в руки Господа, а не в руки людей; ибо, каково величие Его, такова и милость Его.
\vs Sir 3:1 Дети, послушайте меня, отца, и поступайте так, чтобы вам спастись,
\vs Sir 3:2 ибо Господь возвысил отца над детьми и утвердил суд матери над сыновьями.
\vs Sir 3:3 Почитающий отца очистится от грехов,
\vs Sir 3:4 и уважающий мать свою~--- как приобретающий сокровища.
\vs Sir 3:5 Почитающий отца будет иметь радость от детей своих и в день молитвы своей будет услышан.
\vs Sir 3:6 Уважающий отца будет долгоденствовать, и послушный Господу успокоит мать свою.
\vs Sir 3:7 Боящийся Господа почтит отца и, как владыкам, послужит родившим его.
\vs Sir 3:8 Делом и словом почитай отца твоего и мать, чтобы пришло на тебя благословение от них,
\vs Sir 3:9 ибо благословение отца утверждает домы детей, а клятва матери разрушает до основания.
\vs Sir 3:10 Не ищи славы в бесчестии отца твоего, ибо не слава тебе бесчестие отца.
\vs Sir 3:11 Слава человека~--- от чести отца его, и позор детям~--- мать в бесславии.
\vs Sir 3:12 Сын! прими отца твоего в старости \bibemph{его} и не огорчай его в жизни его.
\vs Sir 3:13 Хотя бы он и оскудел разумом, имей снисхождение и не пренебрегай им при полноте силы твоей,
\vs Sir 3:14 ибо милосердие к отцу не будет забыто; несмотря на грехи твои, благосостояние твое умножится.
\vs Sir 3:15 В день скорби твоей воспомянется о тебе: как лед от теплоты, разрешатся грехи твои.
\vs Sir 3:16 Оставляющий отца~--- то же, что богохульник, и проклят от Господа раздражающий мать свою.
\rsbpar\vs Sir 3:17 Сын мой! веди дела твои с кротостью, и будешь любим богоугодным человеком.
\vs Sir 3:18 Сколько ты велик, столько смиряйся, и найдешь благодать у Господа.
\vs Sir 3:19 Много высоких и славных, но тайны открываются смиренным,
\vs Sir 3:20 ибо велико могущество Господа, и Он смиренными прославляется.
\vs Sir 3:21 Чрез меру трудного для тебя не ищи, и, что свыше сил твоих, того не испытывай.
\vs Sir 3:22 Что заповедано тебе, о том размышляй; ибо не нужно тебе, что сокрыто.
\vs Sir 3:23 При многих занятиях твоих, о лишнем не заботься: тебе открыто очень много из человеческого знания;
\vs Sir 3:24 ибо многих ввели в заблуждение их предположения, и лукавые мечты поколебали ум их.
\vs Sir 3:25 Кто любит опасность, тот впадет в нее;
\vs Sir 3:26 упорное сердце напоследок потерпит зло:
\vs Sir 3:27 упорное сердце будет обременено скорбями, и грешник приложит грехи ко грехам.
\vs Sir 3:28 Испытания не служат врачевством для гордого, потому что злое растение укоренилось в нем.
\vs Sir 3:29 Сердце разумного обдумает притчу, и внимательное ухо есть желание мудрого.
\vs Sir 3:30 Вода угасит пламень огня, и милостыня очистит грехи.
\vs Sir 3:31 Кто воздает за благодеяния, тот помышляет о будущем и во время падения найдет опору.
\vs Sir 4:1 Сын мой! не отказывай в пропитании нищему и не утомляй ожиданием очей нуждающихся;
\vs Sir 4:2 не опечаль души алчущей и не огорчай человека в его скудости;
\vs Sir 4:3 не смущай сердца уже огорченного и не откладывай подавать нуждающемуся;
\vs Sir 4:4 не отказывай угнетенному, умоляющему о помощи, и не отвращай лица твоего от нищего;
\vs Sir 4:5 не отвращай очей от просящего и не давай человеку повода проклинать тебя;
\vs Sir 4:6 ибо, когда он в горести души своей будет проклинать тебя, Сотворивший его услышит моление его.
\vs Sir 4:7 В собрании старайся быть приятным и пред высшим наклоняй твою голову;
\vs Sir 4:8 приклоняй ухо твое к нищему и отвечай ему ласково, с кротостью;
\vs Sir 4:9 спасай обижаемого от руки обижающего и не будь малодушен, когда судишь;
\vs Sir 4:10 сиротам будь как отец и матери их~--- вместо мужа:
\vs Sir 4:11 и будешь как сын Вышнего, и Он возлюбит тебя более, нежели мать твоя.
\rsbpar\vs Sir 4:12 Премудрость возвышает сынов своих и поддерживает ищущих ее:
\vs Sir 4:13 любящий ее любит жизнь, и ищущие ее с раннего утра исполнятся радости:
\vs Sir 4:14 обладающий ею наследует славу, и, куда бы ни пошел, Господь благословит его;
\vs Sir 4:15 служащие ей служат Святому, и любящих ее любит Господь;
\vs Sir 4:16 послушный ей будет судить народы, и внимающий ей будет жить надежно;
\vs Sir 4:17 кто вверится ей, тот наследует ее, и потомки его будут обладать ею:
\vs Sir 4:18 ибо сначала она пойдет с ним путями извилистыми, наведет на него страх и боязнь
\vs Sir 4:19 и будет мучить его своим водительством, доколе не уверится в душе его и не искусит его своими уставами;
\vs Sir 4:20 но потом она выйдет к нему на прямом пути и обрадует его
\vs Sir 4:21 и откроет ему тайны свои.
\vs Sir 4:22 Если он совратится с пути, она оставляет его и отдает его в руки падения его.
\rsbpar\vs Sir 4:23 Наблюдай время и храни себя от зла~---
\vs Sir 4:24 и не постыдишься за душу твою:
\vs Sir 4:25 есть стыд, ведущий ко греху, и есть стыд~--- слава и благодать.
\vs Sir 4:26 Не будь лицеприятен против души твоей и не стыдись ко вреду твоему.
\vs Sir 4:27 Не удерживай слова, когда оно может помочь:
\vs Sir 4:28 ибо в слове познается мудрость и в речи языка~--- знание.
\vs Sir 4:29 Не противоречь истине и стыдись твоего невежества.
\vs Sir 4:30 Не стыдись исповедовать грехи твои и не удерживай течения реки.
\vs Sir 4:31 Не подчиняйся человеку глупому и не смотри на сильного.
\vs Sir 4:32 Подвизайся за истину до смерти, и Господь Бог поборет за тебя.
\vs Sir 4:33 Не будь скор языком твоим и ленив и нерадив в делах твоих.
\vs Sir 4:34 Не будь, как лев, в доме твоем и подозрителен к домочадцам твоим.
\vs Sir 4:35 Да не будет рука твоя распростертою к принятию и сжатою при отдании.
\vs Sir 5:1 Не полагайся на имущества твои и не говори: <<станет на жизнь мою>>.
\vs Sir 5:2 Не следуй влечению души твоей и крепости твоей, чтобы ходить в похотях сердца твоего,
\vs Sir 5:3 и не говори: <<кто властен в делах моих?>>, ибо Господь непременно отмстит за дерзость твою.
\vs Sir 5:4 Не говори: <<я грешил, и что мне было?>>, ибо Господь долготерпелив.
\vs Sir 5:5 При мысли об умилостивлении не будь бесстрашен, чтобы прилагать грех ко грехам
\vs Sir 5:6 и не говори: <<милосердие Его велико, Он простит множество грехов моих>>;
\vs Sir 5:7 ибо милосердие и гнев у Него, и на грешниках пребывает ярость Его.
\vs Sir 5:8 Не медли обратиться к Господу и не откладывай со дня на день:
\vs Sir 5:9 ибо внезапно найдет гнев Господа, и ты погибнешь во время отмщения.
\vs Sir 5:10 Не полагайся на имущества неправедные, ибо они не принесут тебе пользы в день посещения.
\vs Sir 5:11 Не вей при всяком ветре и не ходи всякою стезею: таков двоязычный грешник.
\vs Sir 5:12 Будь тверд в твоем убеждении, и одно да будет твое слово.
\vs Sir 5:13 Будь скор к слушанию, и обдуманно давай ответ.
\vs Sir 5:14 Если имеешь знание, то отвечай ближнему, а если нет, то рука твоя да будет на устах твоих.
\vs Sir 5:15 В речах~--- слава и бесчестие, и язык человека бывает падением ему.
\vs Sir 5:16 Не прослыви наушником, и не коварствуй языком твоим:
\vs Sir 5:17 ибо на воре~--- стыд, и на двоязычном~--- злое порицание.
\vs Sir 5:18 Не будь неразумным ни в большом ни в малом.
\vs Sir 6:1 И не делайся врагом из друга, ибо худое имя получает в удел стыд и позор; так~--- и грешник двоязычный.
\vs Sir 6:2 Не возноси себя в помыслах души твоей, чтобы душа твоя не была растерзана, как вол:
\vs Sir 6:3 листья твои ты истребишь и плоды твои погубишь, и останешься, как сухое дерево.
\vs Sir 6:4 Душа лукавая погубит своего обладателя и сделает его посмешищем врагов.
\rsbpar\vs Sir 6:5 Сладкие уста умножат друзей, и доброречивый язык умножит приязнь.
\vs Sir 6:6 Живущих с тобою в мире да будет много, а советником твоим~--- один из тысячи.
\vs Sir 6:7 Если хочешь приобрести друга, приобретай его по испытании и не скоро вверяйся ему.
\vs Sir 6:8 Бывает друг в нужное для него время, и не останется с тобой в день скорби твоей;
\vs Sir 6:9 и бывает друг, который превращается во врага и откроет ссору к поношению твоему.
\vs Sir 6:10 Бывает другом участник в трапезе, и не останется с тобою в день скорби твоей.
\vs Sir 6:11 В имении твоем он будет как ты, и дерзко будет обращаться с домочадцами твоими;
\vs Sir 6:12 но если ты будешь унижен, он будет против тебя и скроется от лица твоего.
\vs Sir 6:13 Отдаляйся от врагов твоих и будь осмотрителен с друзьями твоими.
\vs Sir 6:14 Верный друг~--- крепкая защита: кто нашел его, нашел сокровище.
\vs Sir 6:15 Верному другу нет цены, и нет меры доброте его.
\vs Sir 6:16 Верный друг~--- врачевство для жизни, и боящиеся Господа найдут его.
\vs Sir 6:17 Боящийся Господа направляет дружбу свою так, что, каков он сам, таким делается и друг его.
\rsbpar\vs Sir 6:18 Сын мой! от юности твоей предайся учению, и до седин твоих найдешь мудрость.
\vs Sir 6:19 Приступай к ней как пашущий и сеющий и ожидай добрых плодов ее:
\vs Sir 6:20 ибо малое время потрудишься в возделывании ее, и скоро будешь есть плоды ее.
\vs Sir 6:21 Для невежд она очень сурова, и неразумный не останется с нею:
\vs Sir 6:22 она будет на нем как тяжелый камень испытания, и он не замедлит сбросить ее.
\vs Sir 6:23 Премудрость соответствует имени своему, и немногим открывается.
\vs Sir 6:24 Послушай, сын мой, и прими мнение мое, и не отвергни совета моего.
\vs Sir 6:25 Наложи на ноги твои путы ее и на шею твою цепь ее.
\vs Sir 6:26 Подставь ей плечо твое, и носи ее и не тяготись узами ее.
\vs Sir 6:27 Приблизься к ней всею душею твоею, и всею силою твоею соблюдай пути ее.
\vs Sir 6:28 Исследуй и ищи, и она будет познана тобою и, сделавшись обладателем ее, не покидай ее;
\vs Sir 6:29 ибо наконец ты найдешь в ней успокоение, и она обратится в радость тебе.
\vs Sir 6:30 Путы ее будут тебе крепкою защитою, и цепи ее~--- славным одеянием;
\vs Sir 6:31 ибо на ней украшение золотое, и узы ее~--- гиацинтовые нити.
\vs Sir 6:32 Как одеждою славы ты облечешься ею, и возложишь ее на себя как венец радости.
\vs Sir 6:33 Сын мой! если ты пожелаешь ее, то научишься, и если предашься ей душею твоею, то будешь ко всему способен.
\vs Sir 6:34 Если с любовью будешь слушать \bibemph{ее}, то поймешь ее, и если приклонишь ухо твое, то будешь мудр.
\vs Sir 6:35 Бывай в собрании старцев, и кто мудр, прилепись к тому; люби слушать всякую священную повесть, и притчи разумные да не ускользают от тебя.
\vs Sir 6:36 Если увидишь разумного, ходи к нему с раннего утра, и пусть нога твоя истирает пороги дверей его.
\vs Sir 6:37 Размышляй о повелениях Господа и всегда поучайся в заповедях Его: Он укрепит твое сердце, и желание премудрости дастся тебе.
\vs Sir 7:1 Не делай зла, и тебя не постигнет зло;
\vs Sir 7:2 удаляйся от неправды, и она уклонится от тебя.
\vs Sir 7:3 Сын мой! не сей на бороздах неправды, и не будешь в семь раз более пожинать с них.
\vs Sir 7:4 Не проси у Господа власти, и у царя~--- почетного места.
\vs Sir 7:5 Не оправдывай себя пред Господом, и не мудрствуй пред царем.
\vs Sir 7:6 Не домогайся сделаться судьею, чтобы не оказаться тебе бессильным сокрушить неправду, чтобы не убояться когда-либо лица сильного и не положить тени на правоту твою.
\vs Sir 7:7 Не греши против городского общества, и не роняй себя пред народом.
\vs Sir 7:8 Не прилагай греха ко греху, ибо и за один не останешься ненаказанным.
\vs Sir 7:9 Не говори: <<Он призрит на множество даров моих, и, когда я принесу их Богу Вышнему, Он примет>>.
\vs Sir 7:10 Не малодушествуй в молитве твоей и не пренебрегай подавать милостыню.
\vs Sir 7:11 Не насмехайся над человеком, находящимся в горести души его; ибо есть Смиряющий и Возвышающий.
\vs Sir 7:12 Не выдумывай лжи на брата твоего, и не делай того же против друга.
\vs Sir 7:13 Не желай говорить какую бы то ни было ложь; ибо повторение ее не послужит ко благу.
\vs Sir 7:14 Пред собранием старших не многословь, и не повторяй слова в прошении твоем.
\vs Sir 7:15 Не отвращайся от трудной работы и от земледелия, которое учреждено от Вышнего.
\vs Sir 7:16 Не прилагайся ко множеству грешников.
\vs Sir 7:17 Глубоко смири душу твою.
\vs Sir 7:18 Помни, что гнев не замедлит,
\vs Sir 7:19 что наказание нечестивому~--- огонь и червь.
\vs Sir 7:20 Не меняй друга на сокровище, и брата однокровного~--- на золото Офирское.
\vs Sir 7:21 Не оставляй умной и доброй жены, ибо достоинство ее драгоценнее золота.
\vs Sir 7:22 Не обижай раба, трудящегося усердно, ни наемника, преданного тебе душею.
\vs Sir 7:23 Разумного раба да любит душа твоя, и не откажи ему в свободе.
\vs Sir 7:24 Есть у тебя скот? наблюдай за ним, и если он полезен тебе, пусть остается у тебя.
\vs Sir 7:25 Есть у тебя сыновья? учи их и с юности нагибай шею их.
\vs Sir 7:26 Есть у тебя дочери? имей попечение о теле их и не показывай им веселого лица твоего.
\vs Sir 7:27 Выдай дочь в замужество, и сделаешь великое дело, и подари ее мужу разумному.
\vs Sir 7:28 Есть у тебя жена по душе? не отгоняй ее.
\vs Sir 7:29 Всем сердцем почитай отца твоего и не забывай родильных болезней матери твоей.
\vs Sir 7:30 Помни, что ты рожден от них: и что можешь ты воздать им, как они тебе?
\vs Sir 7:31 Всею душею твоею благоговей пред Господом и уважай священников Его.
\vs Sir 7:32 Всею силою люби Творца твоего, и не оставляй служителей Его.
\vs Sir 7:33 Бойся Господа, и почитай священника, и давай ему часть, как заповедано тебе:
\vs Sir 7:34 начатки, и за грех, и даяние плеч, и жертву освящения, и начатки святых.
\vs Sir 7:35 И к бедному простирай руку твою, дабы благословение твое было совершенно.
\vs Sir 7:36 Милость даяния да будет ко всякому живущему, но и умершего не лишай милости.
\vs Sir 7:37 Не устраняйся от плачущих, и с сетующими сетуй.
\vs Sir 7:38 Не ленись посещать больного, ибо за это ты будешь возлюблен.
\vs Sir 7:39 Во всех делах твоих помни о конце твоем, и вовек не согрешишь.
\vs Sir 8:1 Не ссорься с человеком сильным, чтобы когда-нибудь не впасть в его руки.
\vs Sir 8:2 Не заводи тяжбы с человеком богатым, чтобы он не имел перевеса над тобою;
\vs Sir 8:3 ибо золото многих погубило, и склоняло сердца царей.
\vs Sir 8:4 Не спорь с человеком, дерзким на язык, и не подкладывай дров на огонь его.
\vs Sir 8:5 Не шути с невеждою, чтобы не подверглись бесчестию твои предки.
\vs Sir 8:6 Не укоряй человека, обращающегося от греха: помни, что все мы находимся под эпитимиями.
\vs Sir 8:7 Не пренебрегай человека в старости его, ибо и мы стареем.
\vs Sir 8:8 Не радуйся смерти человека, хотя бы он был самый враждебный тебе: помни, что все мы умрем.
\vs Sir 8:9 Не пренебрегай повестью мудрых и упражняйся в притчах их;
\vs Sir 8:10 ибо от них научишься в\acc{е}дению и~--- как служить сильным.
\vs Sir 8:11 Не удаляйся от повести старцев, ибо и они научились от отцов своих,
\vs Sir 8:12 и ты научишься от них рассудительности и~--- какой в случае надобности дать ответ.
\vs Sir 8:13 Не разжигай углей грешника, чтобы не сгореть от пламени огня его,
\vs Sir 8:14 и не восставай против наглеца, чтобы он не засел засадою в устах твоих.
\vs Sir 8:15 Не давай взаймы человеку, который сильнее тебя; а если дашь, то считай себя потерявшим.
\vs Sir 8:16 Не поручайся сверх силы твоей; а если поручишься, заботься, как обязанный заплатить.
\vs Sir 8:17 Не судись с судьею, потому что его будут судить по его почету.
\vs Sir 8:18 С отважным не пускайся в путь, чтобы он не был тебе в тягость; ибо он будет поступать по своему произволу, и ты можешь погибнуть от его безрассудства.
\vs Sir 8:19 Не заводи ссоры со вспыльчивым и не проходи с ним чрез пустыню; потому что кровь~--- как ничто в глазах его, и где нет помощи, он поразит тебя.
\vs Sir 8:20 Не советуйся с глупым, ибо он не может умолчать о деле.
\vs Sir 8:21 При чужом не делай тайного, ибо не знаешь, что он сделает.
\vs Sir 8:22 Не открывай всякому человеку твоего сердца, чтобы он дурно не отблагодарил тебя.
\vs Sir 9:1 Не будь ревнив к жене сердца твоего и не подавай ей дурного урока против тебя самого.
\vs Sir 9:2 Не отдавай жене души твоей, чтобы она не восстала против власти твоей.
\vs Sir 9:3 Не выходи навстречу развратной женщине, чтобы как-нибудь не попасть в сети ее.
\vs Sir 9:4 Не оставайся долго с певицею, чтобы не плениться тебе искусством ее.
\vs Sir 9:5 Не засматривайся на девицу, чтобы не соблазниться прелестями ее.
\vs Sir 9:6 Не отдавай души твоей блудницам, чтобы не погубить наследства твоего.
\vs Sir 9:7 Не смотри по сторонам на улицах города и не броди по пустым местам его.
\vs Sir 9:8 Отвращай око твое от женщины благообразной и не засматривайся на чужую красоту:
\vs Sir 9:9 многие совратились с пути чрез красоту женскую; от нее, как огонь, загорается любовь.
\vs Sir 9:10 Отнюдь не сиди с женою замужнею и не оставайся с нею на пиру за вином,
\vs Sir 9:11 чтобы не склонилась к ней душа твоя и чтобы ты не поползнулся духом в погибель.
\vs Sir 9:12 Не оставляй старого друга, ибо новый не может сравниться с ним;
\vs Sir 9:13 друг новый~--- то же, что вино новое: когда оно сделается старым, с удовольствием будешь пить его.
\rsbpar\vs Sir 9:14 Не завидуй славе грешника, ибо не знаешь, какой будет конец его.
\vs Sir 9:15 Не одобряй того, что одобряют нечестивые: помни, что они до \bibemph{самого} ада не исправятся.
\vs Sir 9:16 Держи себя дальше от человека, имеющего власть умерщвлять, и ты не будешь смущаться страхом смерти;
\vs Sir 9:17 а если сближаешься с ним, не ошибись, чтобы он не лишил тебя жизни:
\vs Sir 9:18 знай, что ты посреди сетей идешь и по зубцам городских стен проходишь.
\vs Sir 9:19 По силе твоей узнавай ближних и советуйся с мудрыми.
\vs Sir 9:20 Рассуждение твое да будет с разумными, и всякая беседа твоя~--- в законе Вышнего.
\vs Sir 9:21 Да вечеряют с тобою мужи праведные, и слава твоя да будет в страхе Господнем.
\vs Sir 9:22 Изделие хвалится по руке художника, а правитель народа считается мудрым по словам его.
\vs Sir 9:23 Боятся в городе дерзкого на язык, и ненавидят опрометчивого в словах.
\vs Sir 10:1 Мудрый правитель научит народ свой, и правление разумного будет благоустроено.
\vs Sir 10:2 Каков правитель народа, таковы и служащие при нем; и каков начальствующий над городом, таковы и все живущие в нем.
\vs Sir 10:3 Царь ненаученный погубит народ свой, а при благоразумии сильных устроится город.
\vs Sir 10:4 В руке Господа власть над землею, и \bibemph{человека} потребного Он вовремя воздвигнет на ней.
\vs Sir 10:5 В руке Господа благоуспешность человека, и на лице книжника Он отпечатлеет славу Свою.
\vs Sir 10:6 Не гневайся за всякое оскорбление на ближнего, и никого не оскорбляй делом.
\vs Sir 10:7 Гордость ненавистна и Господу и людям и преступна против обоих.
\vs Sir 10:8 Владычество переходит от народа к народу по причине несправедливости, обид и любостяжания.
\vs Sir 10:9 Что гордится земля и пепел?
\vs Sir 10:10 И при жизни извергаются внутренности его.
\vs Sir 10:11 Продолжительною болезнью врач пренебрегает:
\vs Sir 10:12 и вот, ныне царь, а завтра умирает.
\vs Sir 10:13 Когда же человек умрет, то наследием его становятся пресмыкающиеся, звери и черви.
\vs Sir 10:14 Начало гордости~--- удаление человека от Господа и отступление сердца его от Творца его;
\vs Sir 10:15 ибо начало греха~--- гордость, и обладаемый ею изрыгает мерзость;
\vs Sir 10:16 и за это Господь посылает на него страшные наказания и вконец низлагает его.
\vs Sir 10:17 Господь низвергает престолы властителей и посаждает кротких на место их.
\vs Sir 10:18 Господь вырывает с корнем народы и насаждает, вместо них, смиренных.
\vs Sir 10:19 Господь опустошает страны народов и разрушает их до оснований земли.
\vs Sir 10:20 Он иссушает их, и погубляет \bibemph{людей} и истребляет от земли память их.
\vs Sir 10:21 Гордость не сотворена для людей, ни ярость гнева~--- для рождающихся от жен.
\rsbpar\vs Sir 10:22 Семя почтенное какое?~--- Семя человеческое. Семя почтенное какое?~--- Боящиеся Господа.
\vs Sir 10:23 Семя бесчестное какое?~--- Семя человеческое. Семя бесчестное какое?~--- Преступающие заповеди.
\vs Sir 10:24 Старший между братьями~--- в почтении у них, так и боящиеся Господа~--- в очах Его.
\vs Sir 10:25 Богат ли кто и славен, или беден, похвала их~--- страх Господень.
\vs Sir 10:26 Несправедливо~--- бесчестить разумного бедного, и не должно прославлять мужа грешного.
\vs Sir 10:27 Почтенны вельможа, судья и властелин, но нет из них больше боящегося Господа.
\vs Sir 10:28 Рабу мудрому будут служить свободные, и разумный человек, будучи наставляем им, не будет роптать.
\vs Sir 10:29 Не умничай много, чтобы делать дело твое, и не хвались во время нужды.
\vs Sir 10:30 Лучше тот, кто трудится и имеет во всем достаток, нежели кто праздно ходит и хвалится, но нуждается в хлебе.
\vs Sir 10:31 Сын мой! кротостью прославляй душу твою и воздавай ей честь по ее достоинству.
\vs Sir 10:32 Кто будет оправдывать согрешающего против души своей? И кто будет хвалить позорящего жизнь свою?
\vs Sir 10:33 Бедного почитают за познания его, а богатого~--- за его богатство:
\vs Sir 10:34 уважаемый же в бедности насколько больше будет уважаем в богатстве? А бесславный в богатстве насколько будет бесславнее в бедности?
\vs Sir 11:1 Мудрость смиренного вознесет голову его и посадит его среди вельмож.
\vs Sir 11:2 Не хвали человека за красоту его, и не имей отвращения к человеку за наружность его.
\vs Sir 11:3 Мала пчела между летающими, но плод ее~--- лучший из сластей.
\vs Sir 11:4 Не хвались пышностью одежд и не превозносись в день славы: ибо дивны дела Господа, и сокровенны дела Его между людьми.
\vs Sir 11:5 Многие из властелинов сидели на земле, тот же, о ком не думали, носил венец.
\vs Sir 11:6 Многие из сильных подверглись крайнему бесчестию, и славные преданы были в руки других.
\vs Sir 11:7 Прежде, нежели исследуешь, не порицай; узнай прежде, и тогда упрекай.
\vs Sir 11:8 Прежде, нежели выслушаешь, не отвечай, и среди речи не перебивай.
\vs Sir 11:9 Не спорь о деле, для тебя ненужном, и не сиди на суде грешников.
\rsbpar\vs Sir 11:10 Сын мой! не берись за множество дел: при множестве дел не останешься без вины. И если будешь гнаться за ними, не достигнешь, и, убегая, не уйдешь.
\vs Sir 11:11 Иной трудится, напрягает силы, поспешает, и тем более отстает.
\vs Sir 11:12 Иной вял, нуждается в помощи, слабосилен и изобилует нищетою;
\vs Sir 11:13 но очи Господа призрели на него во благо ему, и Он восставил его из унижения его и вознес голову его, и многие изумлялись, смотря на него.
\vs Sir 11:14 Доброе и худое, жизнь и смерть, бедность и богатство~--- от Господа.
\vs Sir 11:15 Даяние Господа предоставлено благочестивым, и благоволение Его будет благопоспешно для них вовек.
\vs Sir 11:16 Иной делается богатым от осмотрительности и бережливости своей, и это часть награды его,
\vs Sir 11:17 когда он скажет: <<я нашел покой и теперь наслаждаюсь моими благами>>.
\vs Sir 11:18 И не знает он, сколько пройдет времени до того, когда он оставит их другим и умрет.
\vs Sir 11:19 Твердо стой в завете твоем и пребывай в нем и состарься в деле твоем.
\vs Sir 11:20 Не удивляйся делам грешника, веруй Господу, и пребывай в труде твоем:
\vs Sir 11:21 ибо легко в очах Господа~--- скоро и внезапно обогатить бедного.
\vs Sir 11:22 Благословение Господа~--- награда благочестивого, и в скором времени процветает он благословением Его.
\vs Sir 11:23 Не говори: <<что мне еще нужно? и какие отныне могу иметь еще блага?>>
\vs Sir 11:24 Не говори: <<довольно у меня, и какое отныне могу я потерпеть зло?>>
\vs Sir 11:25 Во дни счастья бывает забвение о несчастье, и во дни несчастья не вспомнится о счастье.
\vs Sir 11:26 Легко для Господа~--- в день смерти воздать человеку по делам его.
\vs Sir 11:27 Минутное страдание производит забвение утех, и при кончине человека открываются дела его.
\vs Sir 11:28 Прежде смерти не называй никого блаженным; человек познается в детях своих.
\rsbpar\vs Sir 11:29 Не всякого человека вводи в дом твой, ибо много козней у коварного.
\vs Sir 11:30 Как охотничья птица в западне, таково сердце надменного: он, как лазутчик, подсматривает падение;
\vs Sir 11:31 превращая добро во зло, он строит козни и на людей избранных кладет пятно.
\vs Sir 11:32 От искры огня умножаются угли, и человек грешный строит козни на кровь.
\vs Sir 11:33 Остерегайся злодея,~--- ибо он строит зло,~--- чтобы он когда-нибудь не положил на тебе пятна навек.
\vs Sir 11:34 Посели в доме твоем чужого, и он расстроит тебя смутами и сделает тебя чужим для твоих.
\vs Sir 12:1 Если ты делаешь добро, знай, кому делаешь, и будет благодарность за твои благодеяния.
\vs Sir 12:2 Делай добро благочестивому, и получишь воздаяние, и если не от него, то от Всевышнего.
\vs Sir 12:3 Нет добра для того, кто постоянно занимается злом и кто не подает милостыни.
\vs Sir 12:4 Давай благочестивому, и не помогай грешнику.
\vs Sir 12:5 Делай добро смиренному, и не давай нечестивому: запирай от него хлеб и не давай ему, чтобы он чрез то не превозмог тебя;
\vs Sir 12:6 ибо ты получил бы сугубое зло за все добро, которое сделал бы ему; ибо и Всевышний ненавидит грешников и нечестивым воздает отмщением.
\vs Sir 12:7 Давай доброму, и не помогай грешнику.
\vs Sir 12:8 Друг не познается в счастье, и враг не скроется в несчастье.
\vs Sir 12:9 При счастье человека враги его в печали, а в несчастье его и друг разойдется с ним.
\vs Sir 12:10 Не верь врагу твоему вовек, ибо, как ржавеет медь, так и злоба его:
\vs Sir 12:11 хотя бы он смирился и ходил согнувшись, будь внимателен душею твоею и остерегайся его, и будешь пред ним, как чистое зеркало, и узнаешь, что он не до конца очистился от ржавчины;
\vs Sir 12:12 не ставь его подле себя, чтобы он, низринув тебя, не стал на твое место; не сажай его по правую сторону себя, чтобы он когда-нибудь не стал домогаться твоего седалища, и ты наконец поймешь слова мои и со скорбью вспомнишь о наставлениях моих.
\vs Sir 12:13 Кто пожалеет об ужаленном заклинателе змей и обо всех, приближающихся к диким зверям? Так и о сближающемся с грешником и приобщающемся грехам его:
\vs Sir 12:14 на время он останется с тобою, но если ты поколеблешься, он не устоит.
\vs Sir 12:15 Устами своими враг усладит \bibemph{тебя}, но в сердце своем замышляет ввергнуть тебя в яму: глазами своими враг будет плакать, а когда найдет случай, не насытится кровью.
\vs Sir 12:16 Если встретится с тобою несчастье, ты найдешь его там прежде себя,
\vs Sir 12:17 и он, как будто желая помочь, подставит тебе ногу:
\vs Sir 12:18 будет кивать головою и хлопать руками, многое будет шептать, и изменит лицо свое.
\vs Sir 13:1 Кто прикасается к смоле, тот очернится, и кто входит в общение с гордым, сделается подобным ему.
\vs Sir 13:2 Не поднимай тяжести свыше твоей силы, и не входи в общение с тем, кто сильнее и богаче тебя.
\vs Sir 13:3 Какое общение у горшка с котлом? Этот толкнет его, и он разобьется.
\vs Sir 13:4 Богач обидел, и сам же грозит; бедняк обижен, и сам же упрашивает.
\vs Sir 13:5 Если ты выгоден для него, он употребит тебя; а если обеднеешь, он оставит тебя.
\vs Sir 13:6 Если ты достаточен, он будет жить с тобою и истощит тебя, а сам не поболезнует.
\vs Sir 13:7 Возымел он в тебе нужду,~--- будет льстить тебе, будет улыбаться тебе и обнадеживать тебя, ласково будет говорить с тобою и скажет: <<не нужно ли тебе чего?>>
\vs Sir 13:8 Своими угощениями он будет пристыжать тебя, доколе, два или три раза ограбив тебя, не насмеется наконец над тобою.
\vs Sir 13:9 После того он, увидев тебя, уклонится от тебя и будет кивать головою при встрече с тобою.
\vs Sir 13:10 Наблюдай, чтобы тебе не быть обманутым
\vs Sir 13:11 и не быть униженным в твоем веселье.
\vs Sir 13:12 Когда сильный будет приглашать тебя, уклоняйся, и тем более он будет приглашать тебя.
\vs Sir 13:13 Не будь навязчив, чтобы не оттолкнули тебя, и не слишком удаляйся, чтобы не забыли о тебе.
\vs Sir 13:14 Не дозволяй себе говорить с ним, как с равным тебе, и не верь слишком многим словам его; ибо долгим разговором он будет искушать тебя и, как бы шутя, изведывать тебя.
\vs Sir 13:15 Немилостив к себе, кто не удерживает себя в словах своих, и он не убережет себя от оскорбления и от уз.
\vs Sir 13:16 Будь осторожен и весьма внимателен, ибо ты ходишь с падением твоим.
\vs Sir 13:17 Услышав это во сне твоем, не засыпай.
\vs Sir 13:18 Во всю жизнь люби Господа и взывай к Нему о спасении твоем.
\vs Sir 13:19 Всякое животное любит подобное себе, и всякий человек~--- ближнего своего.
\vs Sir 13:20 Всякая плоть соединяется по роду своему, и человек прилепляется к подобному себе.
\vs Sir 13:21 Какое общение у волка с ягненком? Так и у грешника~--- с благочестивым.
\vs Sir 13:22 Какой мир у гиены с собакою? И какой мир у богатого с бедным?
\vs Sir 13:23 Ловля у львов~--- дикие ослы в пустыне, так пастбища богатых~--- бедные.
\vs Sir 13:24 Отвратительно для гордого смирение: так отвратителен для богатого бедный.
\vs Sir 13:25 Когда пошатнется богатый, он поддерживается друзьями; а когда упадет бедный, то отталкивается и друзьями.
\vs Sir 13:26 Когда подвергнется несчастью богатый, у него много помощников; сказал нелепость, и оправдали его.
\vs Sir 13:27 Подвергся несчастью бедняк, и еще бранят его; сказал разумно, и его не слушают.
\vs Sir 13:28 Заговорил богатый,~--- и все замолчали и превознесли речь его до облаков;
\vs Sir 13:29 заговорил бедный, и говорят: <<это кто такой?>> И если он споткнется, то совсем низвергнут его.
\rsbpar\vs Sir 13:30 Хорошо богатство, в котором нет греха, и зла бедность в устах нечестивого.
\vs Sir 13:31 Сердце человека изменяет лицо его или на хорошее, или на худое.
\vs Sir 13:32 Признак сердца в счастье~--- лицо веселое, а изобретение притчей соединено с напряженным размышлением.
\vs Sir 14:1 Блажен человек, который не погрешал устами своими и не уязвлен был печалью греха.
\vs Sir 14:2 Блажен, кого не зазирает душа его и кто не потерял надежды своей.
\vs Sir 14:3 Не добро богатство человеку скупому. И на что имение человеку недоброжелательному?
\vs Sir 14:4 Кто собирает, отнимая у души своей, тот собирает для других, и благами его будут пресыщаться другие.
\vs Sir 14:5 Кто зол для себя, для кого будет добр? И не будет он иметь радости от имения своего.
\vs Sir 14:6 Нет хуже человека, который недоброжелателен к самому себе, и это~--- воздаяние за злобу его.
\vs Sir 14:7 Если он и делает добро, то делает в забывчивости, и после обнаруживает зло свое.
\vs Sir 14:8 Зол, кто имеет завистливые глаза, отвращает лицо и презирает души.
\vs Sir 14:9 Глаза любостяжательного не насыщаются какою-либо частью, и неправда злого иссушает душу.
\vs Sir 14:10 Злой глаз завистлив даже на хлеб и в столе своем терпит скудость.
\vs Sir 14:11 Сын мой! по состоянию твоему делай добро себе и приношения Господу достойно приноси.
\vs Sir 14:12 Помни, что смерть не медлит, и завет ада не открыт тебе:
\vs Sir 14:13 прежде, нежели умрешь, делай добро другу, и по силе твоей простирай твою руку и давай ему.
\vs Sir 14:14 Не лишай себя доброго дня, и часть доброго желания да не пройдет мимо тебя.
\vs Sir 14:15 Не другим ли оставишь ты стяжания твои и плоды усилий твоих для раздела по жребию?
\vs Sir 14:16 Давай и принимай, и утешай душу твою,
\vs Sir 14:17 ибо в аде нельзя найти утех.
\vs Sir 14:18 Всякая плоть, как одежда, ветшает; ибо от века~--- определение: <<смертью умрешь>>.
\vs Sir 14:19 Как зеленеющие листья на густом дереве~--- одни спадают, а другие вырастают: так и род от плоти и крови~--- один умирает, а другой рождается.
\vs Sir 14:20 Всякая вещь, подверженная тлению, исчезает, и сделавший ее умирает с нею.
\rsbpar\vs Sir 14:21 Блажен человек, который упражняется в мудрости и в разуме своем поучается святому.
\vs Sir 14:22 Кто размышляет в сердце своем о путях ее, тот получит разумение и в тайнах ее.
\vs Sir 14:23 Выходи за нею, как ловчий, и строй засаду на путях ее.
\vs Sir 14:24 Кто приклоняется к окнам ее, тот послушает и при дверях ее.
\vs Sir 14:25 Кто обращается вблизи дома ее, тот вобьет гвоздь и в стенах ее, поставит палатку свою подле нее и будет обитать в жилище благ.
\vs Sir 14:26 Он положит детей своих под кровом ее и будет иметь ночлег под сенью ее.
\vs Sir 14:27 Он прикроется ею от зноя и будет жить в славе ее.
\vs Sir 15:1 Боящийся Господа будет поступать так, и твердый в законе овладеет ею.
\vs Sir 15:2 И она встретит его, как мать, и примет его к себе, как целомудренная супруга;
\vs Sir 15:3 напитает его хлебом разума, и водою мудрости напоит его.
\vs Sir 15:4 Он утвердится на ней и не поколеблется; прилепится к ней и не постыдится.
\vs Sir 15:5 И она вознесет его над ближними его, и среди собрания откроет уста его.
\vs Sir 15:6 Веселье и венец радости и вечное имя наследует он.
\vs Sir 15:7 Не постигнут ее люди неразумные, и грешники не увидят ее.
\vs Sir 15:8 Далека она от гордости, и люди лживые не подумают о ней.
\vs Sir 15:9 Неприятна похвала в устах грешника, ибо не от Господа послана она.
\vs Sir 15:10 Будет похвала произнесена мудростью, и Господь благопоспешит ей.
\rsbpar\vs Sir 15:11 Не говори: <<ради Господа я отступил>>; ибо, что Он ненавидит, того ты не должен делать.
\vs Sir 15:12 Не говори: <<Он ввел меня в заблуждение>>, ибо Он не имеет надобности в муже грешном.
\vs Sir 15:13 Всякую мерзость Господь ненавидит, и неприятна она боящимся Его.
\vs Sir 15:14 Он от начала сотворил человека и оставил его в руке произволения его.
\vs Sir 15:15 Если хочешь, соблюдешь заповеди и сохранишь благоугодную верность.
\vs Sir 15:16 Он предложил тебе огонь и воду: на что хочешь, прострешь руку твою.
\vs Sir 15:17 Пред человеком жизнь и смерть, и чего он пожелает, то и дастся ему.
\vs Sir 15:18 Велика премудрость Господа, крепок Он могуществом и видит всё.
\vs Sir 15:19 Очи Его~--- на боящихся Его, и Он знает всякое дело человека.
\vs Sir 15:20 Никому не заповедал Он поступать нечестиво и никому не дал позволения грешить.
\vs Sir 16:1 Не желай множества негодных детей и не радуйся о сыновьях нечестивых. Когда они умножаются, не радуйся о них, если нет в них страха Господня.
\vs Sir 16:2 Не надейся на их жизнь и не опирайся на их множество.
\vs Sir 16:3 Лучше один праведник, нежели тысяча \bibemph{грешников},
\vs Sir 16:4 и лучше умереть бездетным, нежели иметь детей нечестивых,
\vs Sir 16:5 ибо от одного разумного населится город, а племя беззаконных опустеет.
\vs Sir 16:6 Много такого видело око мое, и еще более того слышало ухо мое.
\vs Sir 16:7 В сборище грешников возгорится огонь, как и в народе непокорном возгорался гнев.
\vs Sir 16:8 Не умилостивился Он над древними исполинами, которые в надежде на силу свою сделались отступниками;
\vs Sir 16:9 не пощадил и живших в одном месте с Лотом, которыми возгнушался за их гордость;
\vs Sir 16:10 не помиловал народа погибельного, который надмевался грехами своими,
\vs Sir 16:11 равно как и шестисот тысяч человек, соединившихся в жестокосердии своем. И хотя бы и один был непокорный, было бы удивительно, если б он остался ненаказанным;
\vs Sir 16:12 ибо и милость и гнев~--- во власти Его: силен Он помиловать и излить гнев.
\vs Sir 16:13 Как велика милость Его, так велико и обличение Его. Он судит человека по делам его.
\vs Sir 16:14 Не убежит от Него грешник с хищением, и терпение благочестивого не останется тщетным.
\vs Sir 16:15 Всякой милостыне Он даст место, каждый получит по делам своим.
\vs Sir 16:16 Не говори: <<я скроюсь от Господа; неужели с высоты кто вспомнит обо мне?
\vs Sir 16:17 Во множестве народа меня не заметят; ибо что душа моя в неизмеримом создании?
\vs Sir 16:18 Вот, небо и небо небес~--- Божие, бездна и земля колеблются от посещения Его.
\vs Sir 16:19 Равно сотрясаются от страха горы и основания земли, когда Он взирает.
\vs Sir 16:20 И этого не может понять сердце;
\vs Sir 16:21 а пути Его кто постигнет? Как ветер, которого человек не может видеть, так и большая часть дел Его сокрыты.
\vs Sir 16:22 Кто возвестит о делах правосудия Его? или кто будет ожидать их? ибо далеко это определение>>.
\vs Sir 16:23 Скудный умом думает так, и человек неразумный и заблуждающийся размышляет так глупо.
\rsbpar\vs Sir 16:24 Слушай меня, сын мой, и учись знанию, и внимай сердцем твоим словам моим.
\vs Sir 16:25 Я показываю тебе учение обдуманное и передаю знание точное.
\vs Sir 16:26 По определению Господа дела Его от начала, и от сотворения их Он разделил части их.
\vs Sir 16:27 Навек устроил Он дела Свои, и начала их~--- в роды их. Они не алчут, не утомляются и не прекращают своих действий.
\vs Sir 16:28 Ни одно не стесняет близкого ему,
\vs Sir 16:29 и до века не воспротивятся они слову Его.
\vs Sir 16:30 И потом воззрел Господь на землю и наполнил ее Своими благами.
\vs Sir 16:31 Душа всего живущего покрыла лице ее, и в нее все возвратится.
\vs Sir 17:1 Господь создал человека из земли и опять возвращает его в нее.
\vs Sir 17:2 Определенное число дней и время дал Он им, и дал им власть над всем, что на ней.
\vs Sir 17:3 По природе их, облек их силою и сотворил их по образу Своему,
\vs Sir 17:4 и вложил страх к ним во всякую плоть, чтобы господствовать им над зверями и птицами.
\vs Sir 17:5 Он дал им смысл, язык и глаза, и уши и сердце для рассуждения,
\vs Sir 17:6 исполнил их проницательностью разума и показал им добро и зло.
\vs Sir 17:7 Он положил око Свое на сердца их, чтобы показать им величие дел Своих,
\vs Sir 17:8 да прославляют они святое имя Его и возвещают о величии дел Его.
\vs Sir 17:9 Он приложил им знание и дал им в наследство закон жизни;
\vs Sir 17:10 вечный завет поставил с ними и показал им суды Свои.
\vs Sir 17:11 Величие славы видели глаза их, и славу голоса Его слышало ухо их.
\vs Sir 17:12 И сказал Он им: <<остерегайтесь всякой неправды>>; и заповедал каждому из них обязанность к ближнему.
\vs Sir 17:13 Пути их всегда пред Ним, не скроются от очей Его.
\vs Sir 17:14 Каждому народу поставил Он вождя,
\vs Sir 17:15 а Израиль есть удел Господа.
\vs Sir 17:16 Все дела их~--- как солнце пред Ним, и очи Его всегда на путях их.
\vs Sir 17:17 Не утаились от Него неправды их, и все грехи их~--- пред Господом.
\rsbpar\vs Sir 17:18 Милостыня человека~--- как печать у Него, и благодеяние человека сохранит Он, как зеницу ока.
\vs Sir 17:19 Потом Он восстанет и воздаст им, и даяние их на голову их возвратит.
\vs Sir 17:20 Но кающимся Он давал обращение и ободрял ослабевавших в терпении.
\vs Sir 17:21 Обратись к Господу и оставь грехи;
\vs Sir 17:22 молись пред Ним и уменьши твои преткновения.
\vs Sir 17:23 Возвратись ко Всевышнему, и отвратись от неправды, и сильно возненавидь мерзость.
\vs Sir 17:24 Кто будет восхвалять Всевышнего в аде, вместо живущих и прославляющих Его?
\vs Sir 17:25 От мертвого, как от несуществующего, нет прославления:
\vs Sir 17:26 живый и здоровый восхвалит Господа.
\vs Sir 17:27 Как велико милосердие Господа и примирение с обращающимися к Нему!
\vs Sir 17:28 Не может быть всего в человеке,
\vs Sir 17:29 потому что не бессмертен сын человеческий.
\vs Sir 17:30 Что светлее солнца? но и оно затмевается. И о злом будет помышлять плоть и кровь.
\vs Sir 17:31 За силами высоких небес Он Сам наблюдает, а люди все~--- земля и пепел.
\vs Sir 18:1 Все вообще создал Живущий во веки; Господь один праведен.
\vs Sir 18:2 Никому не предоставил Он изъяснять дел\acc{а} Его.
\vs Sir 18:3 И кто может исследовать великие дела Его?
\vs Sir 18:4 Кто может измерить силу величия Его? и кто может также изречь милости Его?
\vs Sir 18:5 Невозможно ни умалить, ни увеличить, и невозможно исследовать дивных дел Господа.
\vs Sir 18:6 Когда человек окончил бы, тогда он только начинает, и когда перестанет, придет в изумление.
\vs Sir 18:7 Что есть человек и что польза его? что благо его и что зло его?
\vs Sir 18:8 Число дней человека~--- много, если сто лет: как капля воды из моря или крупинка песка, так малы лета его в дне вечности.
\vs Sir 18:9 Посему Господь долготерпелив к \bibemph{людям} и изливает на них милость Свою.
\vs Sir 18:10 Он видит и знает, что конец их очень бедствен,
\vs Sir 18:11 и потому умножает милости Свои.
\vs Sir 18:12 Милость человека~--- к ближнему его, а милость Господа~--- на всякую плоть.
\vs Sir 18:13 Он обличает и вразумляет, и поучает и обращает, как пастырь стадо свое.
\vs Sir 18:14 Он милует принимающих вразумление и усердно обращающихся к закону Его.
\rsbpar\vs Sir 18:15 Сын мой! при благотворениях не делай упреков, и при всяком даре не оскорбляй словами.
\vs Sir 18:16 Роса не охлаждает ли зноя? так слово~--- лучше, нежели даяние.
\vs Sir 18:17 Поэтому не выше ли доброго даяния слово? а у человека доброжелательного и то и другое.
\vs Sir 18:18 Глупый немилосердно укоряет, и подаяние неблагорасположенного иссушает глаза.
\vs Sir 18:19 Прежде, нежели начнешь говорить, обдумывай, и прежде болезни заботься о себе.
\vs Sir 18:20 Испытывай себя прежде суда, и во время посещения найдешь милость.
\vs Sir 18:21 Прежде, нежели почувствуешь слабость, смиряйся, и во время грехов покажи обращение.
\vs Sir 18:22 Ничто да не препятствует тебе исполнить обет благовременно, и не откладывай оправдания до смерти.
\vs Sir 18:23 Прежде, нежели начнешь молиться, приготовь себя, и не будь как человек, искушающий Господа.
\vs Sir 18:24 Припоминай о гневе в день смерти и о времени отмщения, когда Господь отвратит лице Свое.
\vs Sir 18:25 Во время сытости вспоминай о времени голода и во дни богатства~--- о бедности и нужде.
\vs Sir 18:26 От утра до вечера изменяется время, и все скоротечно пред Господом.
\vs Sir 18:27 Человек мудрый во всем будет осторожен и во дни грехов удержится от беспечности.
\vs Sir 18:28 Всякий разумный познает премудрость и нашедшему ее воздаст хвалу.
\vs Sir 18:29 Рассудительные в словах и сами умудряются, и источают основательные притчи.
\rsbpar\vs Sir 18:30 Не ходи вслед похотей твоих и воздерживайся от пожеланий твоих.
\vs Sir 18:31 Если будешь доставлять душе твоей приятное для вожделений, то она сделает тебя потехою для врагов твоих.
\vs Sir 18:32 Не ищи увеселения в большой роскоши и не привязывайся к пиршествам.
\vs Sir 18:33 Не сделайся нищим, пиршествуя на занятые деньги, когда ничего нет у тебя в кошельке.
\vs Sir 19:1 Работник, склонный к пьянству, не обогатится, и ни во что ставящий малое мало-помалу придет в упадок.
\vs Sir 19:2 Вино и женщины развратят разумных, а связывающийся с блудницами сделается еще наглее;
\vs Sir 19:3 гниль и черви наследуют его, и дерзкая душа истребится.
\vs Sir 19:4 Кто скоро доверяет, тот легкомыслен, и согрешающий грешит против души своей.
\vs Sir 19:5 Преданный сердцем удовольствиям будет осужден, а сопротивляющийся вожделениям увенчает жизнь свою.
\vs Sir 19:6 Обуздывающий язык будет жить мирно, и ненавидящий болтливость уменьшит зло.
\vs Sir 19:7 Никогда не повторяй слова, и ничего у тебя не убудет.
\vs Sir 19:8 Ни другу ни недругу не рассказывай и, если это тебе не грех, не открывай;
\vs Sir 19:9 ибо он выслушает тебя, и будет остерегаться тебя, и по времени возненавидит тебя.
\vs Sir 19:10 Выслушал ты слово, пусть умрет оно с тобою: не бойся, не расторгнет оно тебя.
\vs Sir 19:11 Глупый от слова терпит такую же муку, как рождающая~--- от младенца.
\vs Sir 19:12 Что стрела, вонзенная в бедро, то слово в сердце глупого.
\vs Sir 19:13 Расспроси друга \bibemph{твоего}, может быть, не сделал он того; и если сделал, то пусть вперед не делает.
\vs Sir 19:14 Расспроси друга, может быть, не говорил он того; и если сказал, то пусть не повторит того.
\vs Sir 19:15 Расспроси друга, ибо часто бывает клевета.
\vs Sir 19:16 Не всякому слову верь.
\vs Sir 19:17 Иной погрешает \bibemph{словом}, но не от души; и кто не погрешал языком своим?
\vs Sir 19:18 Расспроси ближнего твоего прежде, нежели грозить ему, и дай место закону Всевышнего.\rsbpar Всякая мудрость~--- страх Господень, и во всякой мудрости~--- исполнение закона.
\vs Sir 19:19 И не есть мудрость знание худого. И нет разума, где совет грешников.
\vs Sir 19:20 Есть лукавство, и это мерзость; и есть неразумный, скудный мудростью.
\vs Sir 19:21 Лучше скудный знанием, но богобоязненный, нежели богатый знанием~--- и преступающий закон.
\vs Sir 19:22 Есть хитрость изысканная, но она беззаконна, и есть превращающий \bibemph{суд}, чтобы произнести приговор.
\vs Sir 19:23 Есть лукавый, который ходит согнувшись, в унынии, но внутри он полон коварства.
\vs Sir 19:24 Он поник лицом и притворяется глухим, но он предварит тебя там, где и не думаешь.
\vs Sir 19:25 И если недостаток силы воспрепятствует ему повредить тебе, то он сделает тебе зло, когда найдет случай.
\vs Sir 19:26 По виду узнается человек, и по выражению лица при встрече познается разумный.
\vs Sir 19:27 Одежда и осклабление зубов и походка человека показывают свойство его.
\vs Sir 19:28 Бывает обличение, но не вовремя, и бывает, что иной молчит~--- и он благоразумен.
\vs Sir 20:1 Гораздо лучше обличить, нежели сердиться тайно; и обличаемый наедине предостережется от вреда.
\vs Sir 20:2 Как хорошо обличенному показать раскаяние!
\vs Sir 20:3 Ибо он избежит вольного греха.
\vs Sir 20:4 Что~--- пожелание евнуха растлить девицу, то~--- производящий суд с натяжкою.
\vs Sir 20:5 Иной молчит~--- и оказывается мудрым; а иной бывает ненавистным за многую болтливость.
\vs Sir 20:6 Иной молчит, потому что не имеет, что отвечать; а иной молчит, потому что знает время.
\vs Sir 20:7 Мудрый человек будет молчать до времени; а тщеславный и безрассудный не будет ждать времени.
\vs Sir 20:8 Многоречивый опротивеет, и кто восхищает себе право говорить, будет возненавиден.
\vs Sir 20:9 Бывает успех человеку ко злу, а находка~--- в потерю.
\vs Sir 20:10 Есть даяние, которое не будет тебе на пользу, и есть даяние, за которое бывает сугубое воздаяние.
\vs Sir 20:11 Бывает унижение для славы, а иной от унижения поднимает голову.
\vs Sir 20:12 Иной малым покупает многое и заплатит за то в семь раз больше.
\vs Sir 20:13 Мудрый в слове делается любезным, любезности же глупых останутся напрасными.
\vs Sir 20:14 Даяние безумного не будет тебе на пользу; ибо у него вместо одного много глаз для принятия.
\vs Sir 20:15 Немного даст он, а попрекать будет много, и раскроет уста свои, как глашатай. Ныне он взаем дает, а завтра потребует назад: ненавистен такой человек Господу и людям.
\vs Sir 20:16 Глупый говорит: <<нет у меня друга, и нет благодарности за мои благодеяния. Съедающие хлеб мой льстивы языком>>.
\vs Sir 20:17 Как часто и сколь многие будут насмехаться над ним!
\vs Sir 20:18 Преткновение от земли лучше, нежели от языка. Итак, скоро придет падение злых.
\vs Sir 20:19 Неприятный человек~--- безвременная басня; она всегда будет на устах невежд.
\vs Sir 20:20 Притча из уст глупого отвратительна, ибо он не скажет ее в свое время.
\vs Sir 20:21 Иной удерживается от греха скудостью, и в этом воздержании он не будет сокрушаться.
\vs Sir 20:22 Иной губит душу свою по робости, и губит ее из лицеприятия к безумному.
\vs Sir 20:23 Иной из-за стыда дает обещания другу, и без причины наживает в нем себе врага.
\vs Sir 20:24 Злой порок в человеке~--- ложь; в устах невежд она~--- всегда.
\vs Sir 20:25 Лучше вор, нежели постоянно говорящий ложь; но оба они наследуют погибель.
\vs Sir 20:26 Поведение лживого человека~--- бесчестно, и позор его всегда с ним.
\vs Sir 20:27 Мудрый в словах возвысит себя, и человек разумный понравится вельможам.
\vs Sir 20:28 Возделывающий землю увеличит свой стог, и угождающий вельможам получит помилование в случае неправды.
\vs Sir 20:29 Угощения и подарки ослепляют глаза мудрых и, как бы узда в устах, отвращают обличения.
\vs Sir 20:30 Скрытая мудрость и утаенное сокровище~--- какая польза от обоих?
\vs Sir 20:31 Лучше человек, скрывающий свою глупость, нежели человек, скрывающий свою мудрость.
\vs Sir 21:1 Сын мой! если ты согрешил, не прилагай более грехов и о прежних молись.
\vs Sir 21:2 Беги от греха, как от лица змея; ибо, если подойдешь к нему, он ужалит тебя.
\vs Sir 21:3 Зубы его~--- зубы львиные, которые умерщвляют души людей.
\vs Sir 21:4 Всякое беззаконие как обоюдоострый меч: ране от него нет исцеления.
\vs Sir 21:5 Устрашения и насилия опустошат богатство: так опустеет и дом высокомерного.
\vs Sir 21:6 Моление из уст нищего~--- \bibemph{только} до ушей его; но суд над ним поспешно приближается.
\vs Sir 21:7 Ненавидящий обличение идет по следам грешника, а боящийся Господа обратится сердцем.
\vs Sir 21:8 Издалека узнается сильный языком; но разумный видит, где тот спотыкается.
\vs Sir 21:9 Строящий дом свой на чужие деньги~--- то же, что собирающий камни для своей могилы.
\vs Sir 21:10 Сборище беззаконных~--- куча пакли, и конец их~--- пламень огненный.
\vs Sir 21:11 Путь грешников вымощен камнями, но на конце его~--- пропасть ада.
\vs Sir 21:12 Соблюдающий закон обладает своими мыслями,
\vs Sir 21:13 и совершение страха Господня~--- мудрость.
\vs Sir 21:14 Не научится тот, кто неспособен;
\vs Sir 21:15 но есть способность, умножающая горечь.
\vs Sir 21:16 Знание мудрого увеличивается подобно наводнению, и совет его,~--- как источник жизни.
\vs Sir 21:17 Сердце глупого подобно разбитому сосуду и не удержит в себе никакого знания.
\vs Sir 21:18 Если мудрое слово услышит разумный, то он похвалит его и приложит к себе. Услышал его легкомысленный, и оно не понравилось ему, и он бросил его за себя.
\vs Sir 21:19 Речь глупого~--- как бремя в пути, в устах же разумного находят приятность.
\vs Sir 21:20 Речей разумного будут искать в собрании, и о словах его будут размышлять в сердце.
\vs Sir 21:21 Как разрушенный дом, так мудрость глупому, и знание неразумного~--- бессмысленные слова.
\vs Sir 21:22 Наставление для безумных~--- оковы на ногах и как цепи на правой руке.
\vs Sir 21:23 Глупый в смехе возвышает голос свой, а муж благоразумный едва тихо улыбнется.
\vs Sir 21:24 Как золотой наряд~--- наставление для разумного, и как драгоценное украшение на правой руке.
\vs Sir 21:25 Нога глупого спешит в чужой дом, но человек многоопытный постыдится людей;
\vs Sir 21:26 неразумный сквозь дверь заглядывает в дом, а человек благовоспитанный остановится вне;
\vs Sir 21:27 невежество человека~--- подслушивать у дверей, благоразумный же огорчится таким бесстыдством.
\vs Sir 21:28 Уста многоречивых рассказывают чужое, а слова благоразумных взвешиваются на весах.
\vs Sir 21:29 В устах глупых~--- сердце их, уста же мудрых~--- в сердце их.
\vs Sir 21:30 Когда нечестивый проклинает сатану, то проклинает свою душу.
\vs Sir 21:31 Наушник оскверняет свою душу и будет ненавидим везде, где только жить будет.
\vs Sir 22:1 Грязному камню подобен ленивый: всякий освищет бесславие его.
\vs Sir 22:2 Воловьему помету подобен ленивый: всякий, поднявший его, отряхнет руку.
\vs Sir 22:3 Стыд отцу рождение невоспитанного сына, дочь же \bibemph{невоспитанная} рождается на унижение.
\vs Sir 22:4 Разумная дочь приобретет себе мужа, а бесстыдная~--- печаль родившему.
\vs Sir 22:5 Наглая позорит отца и мужа, и у обоих будет в презрении.
\vs Sir 22:6 Не вовремя рассказ~--- то же, что музыка во время печали; наказание же и учение мудрости прилично всякому времени.
\vs Sir 22:7 Поучающий глупого~--- то же, что склеивающий черепки или пробуждающий спящего от глубокого сна.
\vs Sir 22:8 Рассказывающий что-либо глупому~--- то же, что рассказывающий дремлющему, который по окончании спрашивает: <<что?>>
\vs Sir 22:9 Плачь над умершим, ибо свет исчез для него; плачь и над глупым, ибо разум исчез для него.
\vs Sir 22:10 Меньше плачь над умершим, потому что он успокоился, а злая жизнь глупого~--- хуже смерти.
\vs Sir 22:11 Плачь об умершем~--- семь дней, а о глупом и нечестивом~--- все дни жизни его.
\vs Sir 22:12 С безрассудным много не говори, и к неразумному не ходи;
\vs Sir 22:13 берегись от него, чтобы не иметь неприятности и не замарать себя столкновением с ним;
\vs Sir 22:14 уклонись от него и найдешь покой и не будешь огорчен безумием его.
\vs Sir 22:15 Что тяжелее свинца? и какое имя ему, как не глупый?
\vs Sir 22:16 Легче понести песок и соль и глыбу железа, нежели человека бессмысленного.
\vs Sir 22:17 Как деревянная связь в доме, крепко устроенная, не дает ему распадаться при сотрясении, так сердце, утвержденное на обдуманном совете, не поколеблется во время страха.
\vs Sir 22:18 Сердце, утвержденное на разумном размышлении,~--- как лепное украшение на вытесанной стене.
\vs Sir 22:19 Подпорка, поставленная на высоте, не устоит против ветра:
\vs Sir 22:20 так боязливое сердце, при глупом размышлении, не устоит против страха.
\rsbpar\vs Sir 22:21 Наносящий удар глазу вызывает слезы, а наносящий удар сердцу возбуждает чувство болезненное.
\vs Sir 22:22 Бросающий камень в птиц отгонит их; а поносящий друга расторгнет дружбу.
\vs Sir 22:23 Если ты на друга извлек меч, не отчаивайся, ибо возможно возвращение дружбы.
\vs Sir 22:24 Если ты открыл уста против друга, не бойся, ибо возможно примирение.
\vs Sir 22:25 Только поношение, гордость, обнаружение тайны и коварное злодейство могут отогнать всякого друга.
\vs Sir 22:26 Приобретай доверенность ближнего в нищете его, чтобы радоваться вместе с ним при богатстве его;
\vs Sir 22:27 оставайся с ним во время скорби, чтобы иметь участие в его наследии.
\vs Sir 22:28 Прежде пламени бывает в печи пар и дым: так прежде кровопролития~--- ссоры.
\vs Sir 22:29 Защищать друга я не постыжусь и не скроюсь от лица его;
\vs Sir 22:30 а если приключится мне чрез него зло, то всякий, кто услышит, будет остерегаться его.
\rsbpar\vs Sir 22:31 Кто даст мне стражу к устам моим и печать благоразумия на уста мои, чтобы мне не пасть чрез них и чтобы язык мой не погубил меня!
\vs Sir 23:1 Господи, Отче и Владыко жизни моей! Не оставь меня на волю их и не допусти меня пасть чрез них.
\vs Sir 23:2 Кто приставит бич к помышлениям моим и к сердцу моему наставника в мудрости, чтобы они не щадили проступков моих и не потворствовали заблуждениям их;
\vs Sir 23:3 чтобы не умножались проступки мои и не увеличивались заблуждения мои; чтобы не упасть мне пред противниками, и чтобы не порадовался надо мною враг мой?
\vs Sir 23:4 Господи, Отче и Боже жизни моей! Не дай мне возношения очей и вожделение отврати от меня.
\vs Sir 23:5 Пожелания чрева и сладострастие да не овладеют мною, и не предай меня бесстыдной душе.
\rsbpar\vs Sir 23:6 Выслушайте, дети, наставление для уст: соблюдающий его не будет уловлен своими устами.
\vs Sir 23:7 Уловлен будет ими грешник, и злоречивый и надменный преткнутся чрез них.
\vs Sir 23:8 Не приучай уст твоих к клятве
\vs Sir 23:9 и не обращай в привычку употреблять в клятве имя Святаго.
\vs Sir 23:10 Ибо, как раб, постоянно подвергающийся наказанию, не избавляется от ран, так и клянущийся непрестанно именем Святаго не очистится от греха.
\vs Sir 23:11 Человек, часто клянущийся, исполнится беззакония, и не отступит от дома его бич.
\vs Sir 23:12 Если он согрешит, грех его на нем; и если он вознерадел, то сугубо согрешит;
\vs Sir 23:13 и если он клялся напрасно, то не оправдается, и дом его наполнится несчастьями.
\vs Sir 23:14 Есть речь, облеченная смертью: да не найдется она в наследии Иакова!
\vs Sir 23:15 Ибо от благочестивых все это будет удалено, и они не запутаются во грехах.
\vs Sir 23:16 Не приучай твоих уст к грубой невежливости, ибо при ней бывают греховные слова.
\vs Sir 23:17 Помни об отце и о матери твоей, когда сидишь среди вельмож,
\vs Sir 23:18 чтобы тебе не забыться пред ними и по привычке не сделать глупости, и не пожелать, что лучше бы ты не родился, и не проклясть дня рождения твоего.
\vs Sir 23:19 Человек, привыкающий к бранным словам, во все дни свои не научится.
\vs Sir 23:20 Два качества умножают грехи, а третье навлекает гнев:
\vs Sir 23:21 душа горячая, как пылающий огонь, не угаснет, пока не истощится;
\vs Sir 23:22 человек, блудодействующий в теле плоти своей, не перестанет, пока не прогорит огонь.
\vs Sir 23:23 Блуднику сладок всякий хлеб: он не перестанет, доколе не умрет.
\vs Sir 23:24 Человек, который согрешает против своего ложа, говорит в душе своей: <<кто видит меня?
\vs Sir 23:25 Вокруг меня тьма, и стены закрывают меня, и никто не видит меня: чего мне бояться? Всевышний не воспомянет грехов моих>>.
\vs Sir 23:26 Страх его~--- только глаза человеческие,
\vs Sir 23:27 и не знает он того, что очи Господа в десять тысяч крат светлее солнца
\vs Sir 23:28 и взирают на все пути человеческие, и проникают в места сокровенные.
\vs Sir 23:29 Ему известно было все прежде, нежели сотворено было, равно как и по совершении.
\vs Sir 23:30 Такой \bibemph{человек} будет наказан на улицах города и будет застигнут там, где не думал.
\vs Sir 23:31 Так и жена, оставившая мужа и произведшая наследника от чужого:
\vs Sir 23:32 ибо, во-первых, она не покорилась закону Всевышнего, во-вторых, согрешила против своего мужа и, в-третьих, в блуде прелюбодействовала и произвела детей от чужого мужа.
\vs Sir 23:33 Она будет выведена пред собрание, и о детях ее будет исследование.
\vs Sir 23:34 Дети ее не укоренятся, и ветви ее не дадут плода.
\vs Sir 23:35 Она оставит память о себе на проклятие, и позор ее не изгладится.
\vs Sir 23:36 Оставшиеся познают, что нет ничего лучше страха Господня и нет ничего сладостнее, как внимать заповедям Господним.
\vs Sir 23:37 Великая слава~--- следовать Господу, а быть тебе принятым от Него~--- долгоденствие.
\vs Sir 24:1 Премудрость прославит себя и среди народа своего будет восхвалена.
\vs Sir 24:2 В церкви Всевышнего она откроет уста свои, и пред воинством Его будет прославлять себя:
\vs Sir 24:3 <<я вышла из уст Всевышнего и подобно облаку покрыла землю;
\vs Sir 24:4 я поставила скинию на высоте, и престол мой~--- в столпе облачном;
\vs Sir 24:5 я одна обошла круг небесный и ходила во глубине бездны;
\vs Sir 24:6 в волнах моря и по всей земле и во всяком народе и племени имела я владение:
\vs Sir 24:7 между всеми ими я искала успокоения, и в чьем наследии водвориться мне.
\vs Sir 24:8 Тогда Создатель всех повелел мне, и Произведший меня указал мне покойное жилище и сказал:
\vs Sir 24:9 поселись в Иакове и прими наследие в Израиле.
\vs Sir 24:10 Прежде века от начала Он произвел меня, и я не скончаюсь во веки.
\vs Sir 24:11 Я служила пред Ним во святой скинии и так утвердилась в Сионе.
\vs Sir 24:12 Он дал мне также покой в возлюбленном городе, и в Иерусалиме~--- власть моя.
\vs Sir 24:13 И укоренилась я в прославленном народе, в наследственном уделе Господа.
\vs Sir 24:14 Я возвысилась, как кедр на Ливане и как кипарис на горах Ермонских;
\vs Sir 24:15 я возвысилась, как пальма в Енгадди и как розовые кусты в Иерихоне;
\vs Sir 24:16 я, как красивая маслина в долине и как платан, возвысилась.
\vs Sir 24:17 Как корица и аспалаф, я издала ароматный запах и, как отличная смирна, распространила благоухание,
\vs Sir 24:18 как халвани, оникс и стакти и как благоухание ладана в скинии.
\vs Sir 24:19 Я распростерла свои ветви, как теревинф, и ветви мои~--- ветви славы и благодати.
\vs Sir 24:20 Я~--- как виноградная лоза, произращающая благодать, и цветы мои~--- плод славы и богатства.
\vs Sir 24:21 Приступите ко мне, желающие меня, и насыщайтесь плодами моими;
\vs Sir 24:22 ибо воспоминание обо мне слаще меда и обладание мною приятнее медового сота.
\vs Sir 24:23 Ядущие меня еще будут алкать, и пьющие меня еще будут жаждать.
\vs Sir 24:24 Слушающий меня не постыдится, и трудящиеся со мною не погрешат.
\vs Sir 24:25 Все это~--- книга завета Бога Всевышнего,
\vs Sir 24:26 закон, который заповедал Моисей как наследие сонмам Иаковлевым.
\vs Sir 24:27 Он насыщает мудростью, как Фисон и как Тигр во дни новин;
\vs Sir 24:28 он наполняет разумом, как Евфрат и как Иордан во дни жатвы;
\vs Sir 24:29 он разливает учение, как свет и как Гион во время собирания винограда.
\vs Sir 24:30 Первый человек не достиг полного познания ее; не исследует ее также и последний;
\vs Sir 24:31 ибо мысли ее полнее моря, и намерения ее глубже великой бездны.
\vs Sir 24:32 И я, как канал из реки и как водопровод, вышла в рай.
\vs Sir 24:33 Я сказала: полью мой сад и напою мои гряды.
\vs Sir 24:34 И вот, канал мой сделался рекою, и река моя сделалась морем.
\vs Sir 24:35 И буду я сиять учением, как утренним светом, и далеко проявлю его;
\vs Sir 24:36 и буду я изливать учение, как пророчество, и оставлю его в роды вечные>>.
\vs Sir 24:37 Видите, что я трудился не для себя одного, но для всех, ищущих \bibemph{премудрости}.
\vs Sir 25:1 Тремя я украсилась и стала прекрасною пред Господом и людьми:
\vs Sir 25:2 это~--- единомыслие между братьями и любовь между ближними, и жена и муж, согласно живущие между собою.
\vs Sir 25:3 И три рода \bibemph{людей} возненавидела душа моя, и очень отвратительна для меня жизнь их:
\vs Sir 25:4 надменного нищего, лживого богача и старика-прелюбодея, ослабевающего в рассудке.
\vs Sir 25:5 Чего не собрал ты в юности,~--- как же можешь приобрести в старости твоей?
\vs Sir 25:6 Как прилично сединам судить, и старцам~--- уметь давать совет!
\vs Sir 25:7 Как прекрасна мудрость старцев и как приличны людям почтенным рассудительность и совет!
\vs Sir 25:8 Венец старцев~--- многосторонняя опытность, и хвала их~--- страх Господень.
\vs Sir 25:9 Девять помышлений похвалил я в сердце, а десятое выскажу языком:
\vs Sir 25:10 \bibemph{это} человек, радующийся о детях и при жизни видящий падение врагов.
\vs Sir 25:11 Блажен, кто живет с женою разумною, кто не погрешает языком и не служит недостойному себя.
\vs Sir 25:12 Блажен, кто приобрел мудрость и передает ее в уши слушающих.
\vs Sir 25:13 Как велик тот, кто нашел премудрость! но он не выше того, кто боится Господа.
\vs Sir 25:14 Страх Господень все превосходит, и имеющий его с кем может быть сравнен?
\rsbpar\vs Sir 25:15 \bibemph{Можно перенести} всякую рану, только не рану сердечную, и всякую злость, только не злость женскую,
\vs Sir 25:16 всякое нападение, только не нападение от ненавидящих, и всякое мщение, только не мщение врагов;
\vs Sir 25:17 нет головы ядовитее головы змеиной, и нет ярости сильнее ярости врага.
\vs Sir 25:18 Соглашусь лучше жить со львом и драконом, нежели жить со злою женою.
\vs Sir 25:19 Злость жены изменяет взгляд ее и делает лице ее мрачным, как у медведя.
\vs Sir 25:20 Сядет муж ее среди друзей своих и, услышав \bibemph{о ней}, горько вздохнет.
\vs Sir 25:21 Всякая злость мала в сравнении со злостью жены; жребий грешника да падет на нее.
\vs Sir 25:22 Что восхождение по песку для ног старика, то сварливая жена для тихого мужа.
\vs Sir 25:23 Не засматривайся на красоту женскую и не похотствуй на жену.
\vs Sir 25:24 Досада, стыд и большой срам, когда жена будет преобладать над своим мужем.
\vs Sir 25:25 Сердце унылое и лице печальное и рана сердечная~--- злая жена.
\vs Sir 25:26 Опущенные руки и расслабленные колени~--- жена, которая не счастливит своего мужа.
\vs Sir 25:27 От жены начало греха, и чрез нее все мы умираем.
\vs Sir 25:28 Не давай воде выхода, ни злой жене~--- власти;
\vs Sir 25:29 если она не ходит под рукою твоею, то отсеки ее от плоти твоей.
\vs Sir 26:1 Счастлив муж доброй жены, и число дней его~--- сугубое.
\vs Sir 26:2 Жена добродетельная радует своего мужа и лета его исполнит миром;
\vs Sir 26:3 добрая жена~--- счастливая доля: она дается в удел боящимся Господа;
\vs Sir 26:4 с нею у богатого и бедного~--- сердце довольное и лице во всякое время веселое.
\vs Sir 26:5 Трех страшится сердце мое, а при четвертом я молюсь:
\vs Sir 26:6 городского злословия, возмущения черни и оболгания на смерть,~--- всё это ужасно.
\vs Sir 26:7 Болезнь сердца и печаль~--- жена, ревнивая к \bibemph{другой} жене,
\vs Sir 26:8 и бич языка ее, ко всем приражающийся.
\vs Sir 26:9 Движущееся туда и сюда воловье ярмо~--- злая жена; берущий ее~--- то же, что хватающий скорпиона.
\vs Sir 26:10 Большая досада~--- жена, преданная пьянству, и она не скроет своего срама.
\vs Sir 26:11 Наклонность женщины к блуду узнается по поднятию глаз и век ее.
\vs Sir 26:12 Над бесстыдною дочерью поставь крепкую стражу, чтобы она, улучив послабление, не злоупотребила собою.
\vs Sir 26:13 Берегись бесстыдного глаза, и не удивляйся, если он согрешит против тебя:
\vs Sir 26:14 как томимый жаждою путник открывает уста и пьет всякую близкую воду,
\vs Sir 26:15 так она сядет напротив всякого шатра и пред стрелою откроет колчан.
\vs Sir 26:16 Любезность жены усладит ее мужа, и благоразумие ее утучнит кости его.
\vs Sir 26:17 Кроткая жена~--- дар Господа, и нет цены благовоспитанной душе.
\vs Sir 26:18 Благодать на благодать~--- жена стыдливая,
\vs Sir 26:19 и нет достойной меры для воздержной души.
\vs Sir 26:20 Что солнце, восходящее на высотах Господних,
\vs Sir 26:21 то красота доброй жены в убранстве дома ее;
\vs Sir 26:22 что светильник, сияющий на святом свещнике, то красота лица ее в зрелом возрасте;
\vs Sir 26:23 что золотые столбы на серебряном основании, то прекрасные ноги ее на твердых пятах.
\rsbpar\vs Sir 26:24 От двух скорбело сердце мое, а при третьем возбуждалось во мне негодование:
\vs Sir 26:25 если воин терпит от бедности, и разумные мужи бывают в пренебрежении;
\vs Sir 26:26 и если кто обращается от праведности ко греху, Господь уготовит того на меч.
\vs Sir 26:27 Купец едва может избежать погрешности, а корчемник не спасется от греха.
\vs Sir 27:1 Многие погрешали ради маловажных вещей, и ищущий богатства отвращает глаза.
\vs Sir 27:2 Посреди скреплений камней вбивается гвоздь: так посреди продажи и купли вторгается грех.
\vs Sir 27:3 Если кто не удерживается тщательно в страхе Господнем, то скоро разорится дом его.
\vs Sir 27:4 При трясении решета остается сор: так нечистота человека~--- при рассуждении его.
\vs Sir 27:5 Глиняные сосуды испытываются в печи, а испытание человека~--- в разговоре его.
\vs Sir 27:6 Уход за деревом открывается в плоде его: т\acc{а}к в слове~--- помышления сердца человеческого.
\vs Sir 27:7 Прежде беседы не хвали человека, ибо она есть испытание людей.
\vs Sir 27:8 Если ты усердно будешь искать правды, то найдешь ее и облечешься ею, как подиром славы.
\vs Sir 27:9 Птицы слетаются к подобным себе, и истина обращается к тем, которые упражняются в ней.
\vs Sir 27:10 Как лев подстерегает добычу, так и грехи~--- делающих неправду.
\rsbpar\vs Sir 27:11 Беседа благочестивого~--- всегда мудрость, а безумный изменяется, как луна.
\vs Sir 27:12 Среди неразумных не трать времени, а проводи его постоянно среди благоразумных.
\vs Sir 27:13 Беседа глупых отвратительна, и смех их~--- в забаве грехом.
\vs Sir 27:14 Пустословие много клянущихся поднимет дыбом волосы, а спор их заткнет уши.
\vs Sir 27:15 Ссора надменных~--- кровопролитие, и брань их несносна для слуха.
\vs Sir 27:16 Открывающий тайны потерял доверие и не найдет друга по душе своей.
\vs Sir 27:17 Люби друга и будь верен ему;
\vs Sir 27:18 а если откроешь тайны его, не гонись больше за ним:
\vs Sir 27:19 ибо как человек убивает своего врага, так ты убил дружбу ближнего;
\vs Sir 27:20 и как ты выпустил бы из рук своих птицу, так ты упустил друга и не поймаешь его;
\vs Sir 27:21 не гонись за ним, ибо он далеко ушел и убежал, как серна из сети.
\vs Sir 27:22 Рану можно перевязать, и после ссоры возможно примирение;
\vs Sir 27:23 но кто открыл тайны, тот потерял надежду \bibemph{на примирение}.
\vs Sir 27:24 Кто мигает глазом, тот строит козни, и никто не удержит его от того;
\vs Sir 27:25 пред глазами твоими он будет говорить сладко и будет удивляться словам твоим,
\vs Sir 27:26 а после извратит уста свои и в словах твоих откроет соблазн;
\vs Sir 27:27 многое я ненавижу, но не столько, как его; и Господь возненавидит его.
\rsbpar\vs Sir 27:28 Кто бросает камень вверх, бросает его на свою голову, и коварный удар разделит раны.
\vs Sir 27:29 Кто роет яму, сам упадет в нее, и кто ставит сеть, сам будет уловлен ею.
\vs Sir 27:30 Кто делает зло, на того обратится оно, и он не узн\acc{а}ет, откуда оно пришло к нему;
\vs Sir 27:31 посмеяние и поношение от гордых и мщение, как лев, подстерегут его.
\vs Sir 27:32 Уловлены будут сетью радующиеся о падении благочестивых, и скорбь измождит их прежде смерти их.
\vs Sir 27:33 Злоба и гнев~--- тоже мерзости, и муж грешный будет обладаем ими.
\vs Sir 28:1 Мстительный получит отмщение от Господа, Который не забудет грехов его.
\vs Sir 28:2 Прости ближнему твоему обиду, и тогда по молитве твоей отпустятся грехи твои.
\vs Sir 28:3 Человек питает гнев к человеку, а у Господа просит прощения;
\vs Sir 28:4 к подобному себе человеку не имеет милосердия, и молится о грехах своих;
\vs Sir 28:5 сам, будучи плотию, питает злобу: кто очистит грехи его?
\vs Sir 28:6 Помни последнее и перестань враждовать; помни истление и смерть и соблюдай заповеди;
\vs Sir 28:7 помни заповеди и не злобствуй на ближнего;
\vs Sir 28:8 помни завет Всевышнего и презирай невежество.
\vs Sir 28:9 Удерживайся от ссоры~--- и ты уменьшишь грехи;
\vs Sir 28:10 ибо раздражительный человек возжжет ссору; человек грешник смутит друзей и поселит раздор между живущими в мире.
\vs Sir 28:11 Каково вещество огня, так он и возгорится;
\vs Sir 28:12 и какова сила человека, таков будет и гнев его, и по мере богатства усилится ярость его.
\vs Sir 28:13 Жаркий спор возжигает огонь, а жаркая ссора проливает кровь.
\vs Sir 28:14 Если подуешь на искру, она разгорится, а если плюнешь на нее, угаснет: то и другое выходит из уст твоих.
\rsbpar\vs Sir 28:15 Наушник и двоязычный да будут прокляты, ибо они погубили многих, живших в тишине;
\vs Sir 28:16 язык третий многих поколебал и изгонял их от народа к народу,
\vs Sir 28:17 и разорял укрепленные города и ниспровергал домы вельмож;
\vs Sir 28:18 язык третий изгнал доблестных жен и лишил их трудов их;
\vs Sir 28:19 внимающий ему не найдет покоя и не будет жить в тишине.
\vs Sir 28:20 Удар бича делает рубцы, а удар языка сокрушит кости;
\vs Sir 28:21 многие пали от острия меча, но не столько, сколько павших от языка;
\vs Sir 28:22 счастлив, кто укрылся от него, кто не испытал ярости его, кто не влачил ярма его и не связан был узами его;
\vs Sir 28:23 ибо ярмо его~--- ярмо железное, и узы его~--- узы медные,
\vs Sir 28:24 смерть лютая~--- смерть его, и самый ад лучше его.
\vs Sir 28:25 Не овладеет он благочестивыми, и не сгорят они в пламени его;
\vs Sir 28:26 оставляющие Господа впадут в него; в них возгорится он и не угаснет: он будет послан на них, как лев, и, как барс, будет истреблять их.
\vs Sir 28:27 Смотри, огради владение твое терновником,
\vs Sir 28:28 свяжи серебро твое и золото,
\vs Sir 28:29 и для слов твоих сделай вес и меру, и для уст твоих~--- дверь и запор.
\vs Sir 28:30 Берегись, чтобы не споткнуться ими и не пасть пред злоумышляющим.
\vs Sir 29:1 Кто оказывает милость, тот дает взаем ближнему, и кто поддерживает его своею рукою, тот соблюдает заповеди.
\vs Sir 29:2 Давай взаймы ближнему во время нужды его и сам в свое время возвращай ближнему.
\vs Sir 29:3 Твердо держи слово и будь верен ему~--- и ты во всякое время найдешь нужное для тебя.
\vs Sir 29:4 Многие считали заем находкою и причинили огорчение тем, которые помогли им.
\vs Sir 29:5 Доколе не получит, он будет целовать руку его и из-за денег ближнего смирит голос;
\vs Sir 29:6 а в срок отдачи он будет протягивать время и будет отвечать уныло и жаловаться на время.
\vs Sir 29:7 Если он будет в состоянии, то едва половину принесет~--- и это вменит ему в находку;
\vs Sir 29:8 а если будет не в состоянии, то заимодавец лишился своих денег и без причины приобрел себе врага в нем:
\vs Sir 29:9 он воздаст ему проклятиями и бранью и вместо почтения воздаст бесчестием.
\vs Sir 29:10 Многие по причине такого лукавства уклоняются \bibemph{от ссуды}, опасаясь напрасно потерпеть утрату.
\vs Sir 29:11 Но к бедному ты будь снисходителен и милостынею ему не медли;
\vs Sir 29:12 ради заповеди помоги бедному и в нужде его не отпускай его ни с чем.
\vs Sir 29:13 Трать серебро для брата и друга и не давай ему заржаветь под камнем на погибель;
\vs Sir 29:14 располагай сокровищем твоим по заповедям Всевышнего, и оно принесет тебе более пользы, нежели золото;
\vs Sir 29:15 заключи в кладовых твоих милостыню, и она избавит тебя от всякого несчастья:
\vs Sir 29:16 лучше крепкого щита и твердого копья она защитит тебя против врага.
\vs Sir 29:17 Добрый человек поручится за ближнего, а потерявший стыд оставит его.
\vs Sir 29:18 Не забывай благодеяний поручителя; ибо он дал душу свою за тебя.
\vs Sir 29:19 Грешник расстроит состояние поручителя, и неблагодарный в душе оставит своего избавителя.
\vs Sir 29:20 Поручительство привело в разорение многих достаточных людей и пошатнуло их, как волна морская;
\vs Sir 29:21 мужей могущественных изгнало из домов, и они блуждали между чужими народами.
\vs Sir 29:22 Грешник, принимающий на себя поручительство и ищущий корысти, впадет в тяжбу.
\vs Sir 29:23 Помогай ближнему по силе твоей и берегись, чтобы тебе не впасть \bibemph{в то же}.
\vs Sir 29:24 Главная потребность для жизни~--- вода и хлеб, и одежда и дом, прикрывающий наготу.
\vs Sir 29:25 Лучше жизнь бедного под дощатым кровом, нежели роскошные пиршества в чужих \bibemph{домах}.
\vs Sir 29:26 Будь доволен малым, как и многим.
\vs Sir 29:27 Худая жизнь~--- \bibemph{скитаться} из дома в дом, и где водворишься, не посмеешь и рта открыть;
\vs Sir 29:28 будешь подавать пищу и питье без благодарности, да и сверх того еще услышишь горькое:
\vs Sir 29:29 <<пойди сюда, пришлец, приготовь стол и, если есть что у тебя, накорми меня>>;
\vs Sir 29:30 <<удались, пришлец, ради почетного лица: брат пришел ко мне в гости, дом нужен>>.
\vs Sir 29:31 Тяжел для человека с чувством упрек за приют в доме и порицание за одолжение.
\vs Sir 30:1 Кто любит своего сына, тот пусть чаще наказывает его, чтобы впоследствии утешаться им.
\vs Sir 30:2 Кто наставляет своего сына, тот будет иметь помощь от него и среди знакомых будет хвалиться им.
\vs Sir 30:3 Кто учит своего сына, тот возбуждает зависть во враге, а пред друзьями будет радоваться о нем.
\vs Sir 30:4 Умер отец его~--- и как будто не умирал, ибо оставил по себе подобного себе;
\vs Sir 30:5 при жизни своей он смотрел на него и утешался, и при смерти своей не опечалился;
\vs Sir 30:6 для врагов он оставил в нем мстителя, а для друзей~--- воздающего благодарность.
\vs Sir 30:7 Поблажающий сыну будет перевязывать раны его, и при всяком крике его будет тревожиться сердце его.
\vs Sir 30:8 Необъезженный конь бывает упрям, а сын, оставленный на свою волю, делается дерзким.
\vs Sir 30:9 Лелей дитя, и оно устрашит тебя; играй с ним, и оно опечалит тебя.
\vs Sir 30:10 Не смейся с ним, чтобы не горевать с ним и после не скрежетать зубами своими.
\vs Sir 30:11 Не давай ему воли в юности и не потворствуй неразумию его.
\vs Sir 30:12 Нагибай выю его в юности и сокрушай рёбра его, доколе оно молодо, дабы, сделавшись упорным, оно не вышло из повиновения тебе.
\vs Sir 30:13 Учи сына твоего и трудись над ним, чтобы не иметь тебе огорчения от непристойных поступков его.
\rsbpar\vs Sir 30:14 Лучше бедняк здоровый и крепкий силами, нежели богач с изможденным телом;
\vs Sir 30:15 здоровье и благосостояние тела дороже всякого золота, и крепкое тело лучше несметного богатства;
\vs Sir 30:16 нет богатства лучше телесного здоровья, и нет радости выше радости сердечной;
\vs Sir 30:17 лучше смерть, нежели горестная жизнь или постоянно продолжающаяся болезнь.
\vs Sir 30:18 Сласти, поднесенные к сомкнутым устам, то же, что снеди, поставленные на могиле.
\vs Sir 30:19 Какая польза идолу от жертвы? он ни есть, ни обонять не может:
\vs Sir 30:20 так преследуемый от Господа,
\vs Sir 30:21 смотря глазами и стеная, подобен евнуху, который обнимает девицу и вздыхает.
\vs Sir 30:22 Не предавайся печали душею твоею и не мучь себя своею мнительностью;
\vs Sir 30:23 веселье сердца~--- жизнь человека, и радость мужа~--- долгоденствие;
\vs Sir 30:24 люби душу твою и утешай сердце твое и удаляй от себя печаль,
\vs Sir 30:25 ибо печаль многих убила, а пользы в ней нет.
\vs Sir 30:26 Ревность и гнев сокращают дни, а забота~--- прежде времени приводит старость.
\vs Sir 30:27 Открытое и доброе сердце заботится и о снедях своих.
\vs Sir 31:1 Бдительность над богатством изнуряет тело, и забота о нем отгоняет сон.
\vs Sir 31:2 Бдительная забота не дает дремать, и тяжкая болезнь отнимает сон.
\vs Sir 31:3 Потрудился богатый при умножении имуществ~--- и в покое насыщается своими благами.
\vs Sir 31:4 Потрудился бедный при недостатках в жизни~--- и в покое остается скудным.
\vs Sir 31:5 Любящий золото не будет прав, и кто гоняется за тлением, наполнится им.
\vs Sir 31:6 Многие ради золота подверглись падению, и погибель их была пред лицем их;
\vs Sir 31:7 оно~--- дерево преткновения для приносящих ему жертвы, и всякий несмысленный будет уловлен им.
\vs Sir 31:8 Счастлив богач, который оказался безукоризненным и который не гонялся за золотом.
\vs Sir 31:9 Кто он? и мы прославим его; ибо он сделал чудо в народе своем.
\vs Sir 31:10 Кто был искушаем \bibemph{золотом}~--- и остался непорочным? Да будет это в похвалу ему.
\vs Sir 31:11 Кто мог погрешить~--- и не погрешил, сделать зло~--- и не сделал?
\vs Sir 31:12 Прочно будет богатство его, и о милостынях его будет возвещать собрание.
\rsbpar\vs Sir 31:13 Когда ты сядешь за богатый стол, не раскрывай на него гортани твоей
\vs Sir 31:14 и не говори: <<много же на нем!>> Помни, что алчный глаз~--- злая вещь.
\vs Sir 31:15 Что из сотворенного завистливее глаза? Потому он плачет о всем, что видит.
\vs Sir 31:16 Куда он посмотрит, не протягивай руки, и не сталкивайся с ним в блюде.
\vs Sir 31:17 Суди о ближнем по себе и о всяком действии рассуждай.
\vs Sir 31:18 Ешь, как человек, что тебе предложено, и не пресыщайся, чтобы не возненавидели тебя;
\vs Sir 31:19 переставай \bibemph{есть} первый из вежливости и не будь алчен, чтобы не послужить соблазном;
\vs Sir 31:20 и если ты сядешь посреди многих, то не протягивай руки твоей прежде них.
\vs Sir 31:21 Немногим довольствуется человек благовоспитанный, и потому он не страдает одышкою на своем ложе.
\vs Sir 31:22 Здоровый сон бывает при умеренности желудка: встал рано, и душа его с ним;
\vs Sir 31:23 страдание бессонницею и холера и резь в животе бывают у человека ненасытного.
\vs Sir 31:24 Если ты обременил себя яствами, то встань из-за стола и отдохни.
\vs Sir 31:25 Послушай меня, сын мой, и не пренебреги мною, и впоследствии ты поймешь слова мои.
\vs Sir 31:26 Во всех делах твоих будь осмотрителен, и никакая болезнь не приключится тебе.
\vs Sir 31:27 Щедрого на хлебы будут благословлять уста, и свидетельство о доброте его~--- верно;
\vs Sir 31:28 против скупого на хлеб будет роптать город, и свидетельство о скупости его~--- справедливо.
\vs Sir 31:29 Против вина не показывай себя храбрым, ибо многих погубило вино.
\vs Sir 31:30 Печь испытывает крепость лезвия закалкою; так вино испытывает сердца гордых~--- пьянством.
\vs Sir 31:31 Вино полезно для жизни человека, если будешь пить его умеренно.
\vs Sir 31:32 Что за жизнь без вина? оно сотворено на веселие людям.
\vs Sir 31:33 Отрада сердцу и утешение душе~--- вино, умеренно употребляемое вовремя;
\vs Sir 31:34 горесть для души~--- вино, когда пьют его много, при раздражении и ссоре.
\vs Sir 31:35 Излишнее употребление вина увеличивает ярость неразумного до преткновения, умаляя крепость его и причиняя раны.
\vs Sir 31:36 На пиру за вином не упрекай ближнего и не унижай его во время его веселья;
\vs Sir 31:37 не говори ему оскорбительных слов и не обременяй его требованиями.
\vs Sir 32:1 Если поставили тебя старшим \bibemph{на пиру}, не возносись; будь между другими как один из них:
\vs Sir 32:2 позаботься о них и потом садись. И когда всё твое дело исполнишь, тогда займи твое место,
\vs Sir 32:3 чтобы порадоваться на них и за хорошее распоряжение получить венок.
\vs Sir 32:4 Разговор веди ты, старший,~--- ибо это прилично тебе,~---
\vs Sir 32:5 с основательным знанием, и не возбраняй музыки.
\vs Sir 32:6 Когда слушают, не размножай разговора и безвременно не мудрствуй.
\vs Sir 32:7 Что рубиновая печать в золотом украшении, то благозвучие музыки в пиру за вином;
\vs Sir 32:8 что смарагдовая печать в золотой оправе, то приятность песней за вкусным вином.
\vs Sir 32:9 Говори, юноша, если нужно тебе, едва слова два, когда будешь спрошен,
\vs Sir 32:10 говори главное, многое в немногих словах. Будь как знающий и, вместе, как умеющий молчать.
\vs Sir 32:11 Среди вельмож не равняйся с ними, и, когда говорит другой, ты много не говори.
\vs Sir 32:12 Грому предшествует молния, а стыдливого предваряет благорасположение.
\vs Sir 32:13 Вставай вовремя и не будь последним; поспешай домой и не останавливайся.
\vs Sir 32:14 Там забавляйся и делай, что тебе нравится; но не согрешай гордым словом.
\vs Sir 32:15 И за это благословляй Сотворившего тебя и Насыщающего тебя Своими благами.
\rsbpar\vs Sir 32:16 Боящийся Господа примет наставление, и с раннего утра обращающиеся к Нему приобретут благоволение Его.
\vs Sir 32:17 Ищущий закона насытится им, а лицемер преткнется в нем.
\vs Sir 32:18 Боящиеся Господа найдут суд и, как свет, возжгут правосудие.
\vs Sir 32:19 Человек грешный уклоняется от обличения и находит извинение, согласно желанию своему.
\vs Sir 32:20 Человек рассудительный не пренебрегает размышлением, а безрассудный и гордый не содрогается от страха и после того, как сделал что-либо без размышления.
\vs Sir 32:21 Без рассуждения не делай ничего, и когда сделаешь, не раскаивайся.
\vs Sir 32:22 Не ходи по пути, где развалины, чтобы не споткнуться о камень;
\vs Sir 32:23 не полагайся и на ровный путь; остерегайся даже детей твоих.
\vs Sir 32:24 Во всяком деле верь душе твоей: и это есть соблюдение заповедей.
\vs Sir 32:25 Верующий закону внимателен к заповедям, и надеющийся на Господа не потерпит вреда.
\vs Sir 33:1 Боящемуся Господа не приключится зла, но и в искушении Он избавит его.
\vs Sir 33:2 Мудрый муж не возненавидит закона, а притворно держащийся его~--- как корабль в бурю.
\vs Sir 33:3 Разумный человек верит закону, и закон для него верен, как ответ урима.
\vs Sir 33:4 Приготовь слово~--- и будешь выслушан; собери наставления~--- и отвечай.
\vs Sir 33:5 Колесо в колеснице~--- сердце глупого, и как вертящаяся ось~--- мысль его.
\vs Sir 33:6 Насмешливый друг то же, что ярый конь, который под всяким седоком ржет.
\rsbpar\vs Sir 33:7 Почему один день лучше другого, тогда как каждый дневной свет в году \bibemph{исходит} от солнца?
\vs Sir 33:8 Они разделены премудростью Господа; Он отличил времена и празднества:
\vs Sir 33:9 некоторые из них Он возвысил и освятил, а прочие положил в числе обыкновенных дней.
\vs Sir 33:10 И все люди из праха, и Адам был создан из земли;
\vs Sir 33:11 но по всеведению Своему Господь положил различие между ними и назначил им разные пути:
\vs Sir 33:12 одних из них благословил и возвысил, других освятил и приблизил к Себе, а иных проклял и унизил и сдвинул с места их.
\vs Sir 33:13 Как глина у горшечника в руке его и все судьбы ее в его произволе, так люди~--- в руке Сотворившего их, и Он воздает им по суду Своему.
\vs Sir 33:14 Как напротив зла~--- добро и напротив смерти~--- жизнь, так напротив благочестивого~--- грешник. Так смотри и на все дела Всевышнего: их по два, одно напротив другого.
\vs Sir 33:15 И я последний бодрственно потрудился, как подбиравший позади собирателей винограда,
\vs Sir 33:16 и по благословению Господа успел и наполнил точило, как собиратель винограда.
\vs Sir 33:17 Поймите, что я трудился не для себя одного, но для всех ищущих наставления.
\rsbpar\vs Sir 33:18 Послушайте меня, князья народа, и внимайте, начальники собрания:
\vs Sir 33:19 ни сыну, ни жене, ни брату, ни другу не давай власти над тобою при жизни твоей;
\vs Sir 33:20 и не отдавай другому имения твоего, чтобы, раскаявшись, не умолять о нем.
\vs Sir 33:21 Доколе ты жив и дыхание в тебе, не заменяй себя никем;
\vs Sir 33:22 ибо лучше, чтобы дети просили тебя, нежели тебе смотреть в руки сыновей твоих.
\vs Sir 33:23 Во всех делах твоих будь главным, и не клади пятна на честь твою.
\vs Sir 33:24 При скончании дней жизни твоей и при смерти передай наследство.
\vs Sir 33:25 Корм, палка и бремя~--- для осла; хлеб, наказание и дело~--- для раба.
\vs Sir 33:26 Занимай раба работою~--- и будешь иметь покой; ослабь руки ему~--- и он будет искать свободы.
\vs Sir 33:27 Ярмо и ремень согнут выю \bibemph{вола}, а для лукавого раба~--- узы и раны;
\vs Sir 33:28 употребляй его на работу, чтобы он не оставался в праздности, ибо праздность научила многому худому;
\vs Sir 33:29 приставь его к делу, как ему следует, и если он не будет повиноваться, наложи на него тяжкие оковы.
\vs Sir 33:30 Но ни на кого не налагай лишнего и ничего не делай без рассуждения.
\vs Sir 33:31 Если есть у тебя раб, то да будет он как ты, ибо ты приобрел его кровью;
\vs Sir 33:32 если есть у тебя раб, то поступай с ним, как с братом, ибо ты будешь нуждаться в нем, как в душе твоей;
\vs Sir 33:33 если ты будешь обижать его, и он встанет и убежит от тебя, то на какой дороге ты будешь искать его?
\vs Sir 34:1 Пустые и ложные надежды~--- у человека безрассудного, и сонные грезы окрыляют глупых.
\vs Sir 34:2 Как обнимающий тень или гонящийся за ветром, так верящий сновидениям.
\vs Sir 34:3 Сновидения совершенно то же, что подобие лица против лица.
\vs Sir 34:4 От нечистого что может быть чистого, и от ложного что может быть истинного?
\vs Sir 34:5 Гадания и приметы и сновидения~--- суета, и сердце наполняется мечтами, как у рождающей.
\vs Sir 34:6 Если они не будут посланы от Всевышнего для вразумления, не прилагай к ним сердца твоего.
\vs Sir 34:7 Сновидения ввели многих в заблуждение, и надеявшиеся на них подверглись падению.
\vs Sir 34:8 Закон исполняется без обмана, и мудрость в устах верных совершается.
\vs Sir 34:9 Человек ученый знает много, и многоопытный выскажет знание.
\vs Sir 34:10 Кто не имел опытов, тот мало знает; а кто странствовал, тот умножил знание.
\vs Sir 34:11 Многое я видел в моем странствовании, и я знаю больше, нежели сколько говорю.
\vs Sir 34:12 Много раз был я в опасности смерти, и спасался при помощи \bibemph{опыта}.
\vs Sir 34:13 Дух боящихся Господа поживет, ибо надежда их~--- на Спасающего их.
\vs Sir 34:14 Боящийся Господа ничего не устрашится и не убоится, ибо Он~--- надежда его.
\vs Sir 34:15 Блаженна душа боящегося Господа! кем он держится, и кто опора его?
\vs Sir 34:16 Очи Господа~--- на любящих Его. Он~--- могущественная защита и крепкая опора, покров от зноя и покров от полуденного жара, охранение от преткновения и защита от падения;
\vs Sir 34:17 Он возвышает душу и просвещает очи, дает врачевство, жизнь и благословение.
\rsbpar\vs Sir 34:18 Кто приносит жертву от неправедного \bibemph{стяжания}, того приношение насмешливое, и дары беззаконных неблагоугодны;
\vs Sir 34:19 не благоволит Всевышний к приношениям нечестивых и множеством жертв не умилостивляется о грехах их.
\vs Sir 34:20 Что заколающий на жертву сына пред отцем его, то приносящий жертву из имения бедных.
\vs Sir 34:21 Хлеб нуждающихся есть жизнь бедных: отнимающий его есть кровопийца.
\vs Sir 34:22 Убивает ближнего, кто отнимает у него пропитание, и проливает кровь, кто лишает наемника платы.
\vs Sir 34:23 Когда один строит, а другой разрушает, то что они получат для себя кроме утомления?
\vs Sir 34:24 Когда один молится, а другой проклинает, чей голос услышит Владыка?
\vs Sir 34:25 Когда кто омывается от осквернения мертвым и опять прикасается к нему, какая польза от его омовения?
\vs Sir 34:26 Так человек, который постится за грехи свои и опять идет и делает то же самое: кто услышит молитву его? и какую пользу получит он оттого, что смирялся?
\vs Sir 35:1 Кто соблюдает закон, тот умножает приношения; кто держится заповедей, тот приносит жертву спасения.
\vs Sir 35:2 Кто воздает благодарность, тот приносит семидал; а подающий милостыню приносит жертву хвалы.
\vs Sir 35:3 Благоугождение Господу~--- отступление от зла, и умилостивление \bibemph{Его}~--- уклонение от неправды.
\vs Sir 35:4 Не являйся пред лице Господа с пустыми руками, ибо всё это~--- по заповеди.
\vs Sir 35:5 Приношение праведного утучняет алтарь, и благоухание его~--- пред Всевышним;
\vs Sir 35:6 жертва праведного мужа благоприятна, и память о ней незабвенна будет.
\vs Sir 35:7 С веселым оком прославляй Господа и не умаляй начатков трудов твоих;
\vs Sir 35:8 при всяком даре имей лице веселое и в радости посвящай десятину.
\vs Sir 35:9 Давай Всевышнему по даянию Его, и с веселым оком~--- по мере приобретения рукою твоею,
\vs Sir 35:10 ибо Господь есть воздаятель и воздаст тебе всемеро.
\vs Sir 35:11 Не уменьшай даров, ибо Он не примет их: и не надейся на неправедную жертву,
\vs Sir 35:12 ибо Господь есть судия, и нет у Него лицеприятия:
\vs Sir 35:13 Он не уважит лица пред бедным и молитву обиженного услышит;
\vs Sir 35:14 Он не презрит моления сироты, ни вдовы, когда она будет изливать прошение \bibemph{свое}.
\vs Sir 35:15 Не слезы ли вдовы льются по щекам, и не вопиет ли она против того, кто вынуждает их?
\vs Sir 35:16 Служащий \bibemph{Богу} будет принят с благоволением, и молитва его дойдет до облаков.
\vs Sir 35:17 Молитва смиренного проникнет сквозь облака, и он не утешится, доколе она не приблизится \bibemph{к Богу},
\vs Sir 35:18 и не отступит, доколе Всевышний не призрит и не рассудит справедливо и не произнесет решения.
\vs Sir 35:19 И Господь не замедлит и не потерпит, доколе не сокрушит чресл немилосердых;
\vs Sir 35:20 Он будет воздавать отмщение и народам, доколе не истребит сонма притеснителей и не сокрушит скипетров неправедных,
\vs Sir 35:21 доколе не воздаст человеку по делам его, и за дела людей~--- по намерениям их,
\vs Sir 35:22 доколе не совершит суда над народом Своим и не обрадует их Своею милостью.
\vs Sir 35:23 Благовременна милость во время скорби, как дождевые облака во время засухи.
\vs Sir 36:1 Помилуй нас, Владыко, Боже всех, и призри,
\vs Sir 36:2 и наведи на все народы страх Твой.
\vs Sir 36:3 Воздвигни руку Твою на чужие народы, и да позн\acc{а}ют они могущество Твое.
\vs Sir 36:4 Как пред ними Ты явил святость Твою в нас, так пред нами яви величие Твое в них,~---
\vs Sir 36:5 и да познают они Тебя, как мы познали, что нет Бога, кроме Тебя, Господи.
\vs Sir 36:6 Возобнови знамения и сотвори новые чудеса;
\vs Sir 36:7 прославь руку и правую мышцу \bibemph{Твою}; воздвигни ярость и пролей гнев;
\vs Sir 36:8 истреби противника и уничтожь врага;
\vs Sir 36:9 ускори время и вспомни клятву, и да возвестят о великих делах Твоих.
\vs Sir 36:10 Яростью огня да будет истреблен убегающий \bibemph{от меча}, и угнетающие народ Твой да найдут погибель.
\vs Sir 36:11 Сокруши головы начальников вражеских, которые говорят: <<никого нет, кроме нас!>>
\vs Sir 36:12 Собери все колена Иакова и соделай их наследием Твоим, как было сначала.
\vs Sir 36:13 Помилуй, Господи, народ, названный по имени Твоему, и Израиля, которого Ты нарек первенцем.
\vs Sir 36:14 Умилосердись над городом святыни Твоей, над Иерусалимом, местом покоя Твоего.
\vs Sir 36:15 Наполни Сион хвалою обетований Твоих, и Твоею славою~--- народ Твой.
\vs Sir 36:16 Даруй свидетельство тем, которые от начала были достоянием Твоим, и воздвигни пророчества от имени Твоего.
\vs Sir 36:17 Даруй награду надеющимся на Тебя, и да веруют пророкам Твоим.
\vs Sir 36:18 Услышь, Господи, молитву рабов Твоих, по благословению Аарона, о народе Твоем,~---
\vs Sir 36:19 и познают все живущие на земле, что Ты~--- Господь, Бог веков.
\rsbpar\vs Sir 36:20 Желудок принимает в себя всякую пищу, но пища пищи лучше:
\vs Sir 36:21 гортань отличает пищу из дичи, так разумное сердце~--- слова ложные.
\vs Sir 36:22 Лукавое сердце причинит печаль, но человек многоопытный воздаст ему.
\vs Sir 36:23 Женщина примет всякого мужа, но девица девицы лучше:
\vs Sir 36:24 красота жены веселит лице и всего вожделеннее для мужа;
\vs Sir 36:25 если есть на языке ее приветливость и кротость, то муж ее выходит из ряда сынов человеческих.
\vs Sir 36:26 Приобретающий жену полагает начало стяжанию, приобретает соответственно ему помощника, опору спокойствия его.
\vs Sir 36:27 Где нет ограды, \bibemph{там} расхитится имение; а у кого нет жены, тот будет вздыхать скитаясь:
\vs Sir 36:28 ибо кто поверит вооруженному разбойнику, скитающемуся из города в город?
\vs Sir 36:29 Так и человеку, не имеющему оседлости и останавливающемуся для ночлега там, где он запоздает.
\vs Sir 37:1 Всякий друг может сказать: <<и я подружился с ним>>. Но бывает друг по имени только другом.
\vs Sir 37:2 Не есть ли это скорбь до смерти, когда приятель и друг обращается во врага?
\vs Sir 37:3 О, злая мысль! откуда вторглась ты, чтобы покрыть землю коварством?
\vs Sir 37:4 Приятель радуется при веселии друга, а во время скорби его будет против него.
\vs Sir 37:5 Приятель помогает другу в трудах его ради чрева, а в случае войны возьмется за щит.
\vs Sir 37:6 Не забывай друга в душе твоей и не забывай его в имении твоем.
\vs Sir 37:7 Всякий советник хвалит \bibemph{свой} совет, но иной советует в свою пользу;
\vs Sir 37:8 от советника охраняй душу твою и наперед узнай, что ему нужно; ибо, может быть, он будет советовать для самого себя;
\vs Sir 37:9 может быть, он бросит на тебя жребий и скажет тебе: <<путь твой хорош>>; а сам станет напротив тебя, чтобы посмотреть, что случится с тобою.
\vs Sir 37:10 Не советуйся с недоброжелателем твоим и от завистников твоих скрывай намерения.
\vs Sir 37:11 Не советуйся с женою о сопернице ее и с боязливым~--- о войне, с продавцом~--- о мене, с покупщиком~--- о продаже, с завистливым~--- о благодарности,
\vs Sir 37:12 с немилосердым~--- о благотворительности, с ленивым~--- о всяком деле,
\vs Sir 37:13 с годовым наемником~--- об окончании работы, с ленивым рабом~--- о большой работе:
\vs Sir 37:14 не полагайся на таких ни при каком совещании,
\vs Sir 37:15 но обращайся всегда только с мужем благочестивым, о котором узн\acc{а}ешь, что он соблюдает заповеди Господни,
\vs Sir 37:16 который своею душею~--- по душе тебе и, в случае падения твоего, поскорбит вместе с тобою.
\vs Sir 37:17 Держись совета сердца твоего, ибо нет никого для тебя вернее его;
\vs Sir 37:18 душа человека иногда более скажет, нежели семь наблюдателей, сидящих на высоком месте для наблюдения.
\vs Sir 37:19 Но при всем этом молись Всевышнему, чтобы Он управил путь твой в истине.
\rsbpar\vs Sir 37:20 Начало всякого дела~--- размышление, а прежде всякого действия~--- совет.
\vs Sir 37:21 Выражение сердечного изменения~--- лице. Четыре состояния выражаются на нем: добро и зло, жизнь и смерть, а господствует всегда язык.
\vs Sir 37:22 Иной человек искусен и многих учит, а для своей души бесполезен.
\vs Sir 37:23 Иной ухищряется в речах, а \bibemph{бывает} ненавистен,~--- такой останется без всякого пропитания;
\vs Sir 37:24 ибо не дана ему от Господа благодать, и он лишен всякой мудрости.
\vs Sir 37:25 Иной мудр для души своей, и плоды знания на устах его верны.
\vs Sir 37:26 Мудрый муж поучает народ свой, и плоды знания его верны.
\vs Sir 37:27 Мудрый муж будет изобиловать благословением, и все видящие его будут называть его блаженным.
\vs Sir 37:28 Жизнь человека определяется числом дней, а дни Израиля бесчисленны.
\vs Sir 37:29 Мудрый приобретет доверие у своего народа, и имя его будет жить вовек.
\rsbpar\vs Sir 37:30 Сын мой! в продолжение жизни испытывай твою душу и наблюдай, что для нее вредно, и не давай ей того;
\vs Sir 37:31 ибо не всё полезно для всех, и не всякая душа ко всему расположена.
\vs Sir 37:32 Не пресыщайся всякою сластью и не бросайся на разные снеди,
\vs Sir 37:33 ибо от многоядения бывает болезнь, и пресыщение доводит до холеры;
\vs Sir 37:34 от пресыщения многие умерли, а воздержный прибавит себе жизни.
\vs Sir 38:1 Почитай врача честью по надобности в нем, ибо Господь создал его,
\vs Sir 38:2 и от Вышнего~--- врачевание, и от царя получает он дар.
\vs Sir 38:3 Знание врача возвысит его голову, и между вельможами он будет в почете.
\vs Sir 38:4 Господь создал из земли врачевства, и благоразумный человек не будет пренебрегать ими.
\vs Sir 38:5 Не от дерева ли вода сделалась сладкою, чтобы познана была сила Его?
\vs Sir 38:6 Для того Он и дал людям знание, чтобы прославляли Его в чудных делах Его:
\vs Sir 38:7 ими он врачует \bibemph{человека} и уничтожает болезнь его.
\vs Sir 38:8 Приготовляющий лекарства делает из них смесь, и занятия его не оканчиваются, и чрез него бывает благо на лице земли.
\vs Sir 38:9 Сын мой! в болезни твоей не будь небрежен, но молись Господу, и Он исцелит тебя.
\vs Sir 38:10 Оставь греховную жизнь и исправь руки твои, и от всякого греха очисти сердце.
\vs Sir 38:11 Вознеси благоухание и из семидала памятную жертву и сделай приношение тучное, как бы уже умирающий;
\vs Sir 38:12 и дай место врачу, ибо и его создал Господь, и да не удаляется он от тебя, ибо он нужен.
\vs Sir 38:13 В иное время и в их руках бывает успех;
\vs Sir 38:14 ибо и они молятся Господу, чтобы Он помог им подать \bibemph{больному} облегчение и исцеление к продолжению жизни.
\vs Sir 38:15 Но кто согрешает пред Сотворившим его, да впадет в руки врача!
\vs Sir 38:16 Сын мой! над умершим пролей слезы и, как бы подвергшийся жестокому несчастию, начни плач; прилично облеки тело его и не пренебреги погребением его;
\vs Sir 38:17 горький да будет плач и рыдание теплое, и продолжи сетование о нем, по достоинству его, день или два, для избежания осуждения, и тогда утешься от печали;
\vs Sir 38:18 ибо от печали бывает смерть, и печаль сердечная истощит силу.
\vs Sir 38:19 С несчастьем пребывает и печаль, и жизнь нищего тяжела для сердца.
\vs Sir 38:20 Не предавай сердца твоего печали; отдаляй ее \bibemph{от себя}, вспоминая о конце.
\vs Sir 38:21 Не забывай о сем, ибо нет возвращения; и ему ты не принесешь пользы, а себе повредишь.
\vs Sir 38:22 <<Вспоминай о приговоре надо мною, потому что он также и над тобою; мне вчера, а тебе сегодня>>.
\vs Sir 38:23 С упокоением умершего успокой и память о нем, и утешься о нем по исходе души его.
\rsbpar\vs Sir 38:24 Мудрость книжная приобретается в благоприятное время досуга, и кто мало имеет своих занятий, может приобрести мудрость.
\vs Sir 38:25 Как может сделаться мудрым тот, кто правит плугом и хвалится бичом, гоняет волов и занят работами их, и которого разговор \bibemph{только} о молодых волах?
\vs Sir 38:26 Сердце его занято тем, чтобы проводить борозды, и забота его~--- о корме для телиц.
\vs Sir 38:27 Так и всякий плотник и зодчий, который проводит ночь, как день: кто занимается резьбою, того прилежание в том, чтобы оразнообразить форму;
\vs Sir 38:28 сердце свое он устремляет на то, чтобы изображение было похоже, и забота его~--- о том, чтоб окончить дело в совершенстве.
\vs Sir 38:29 Так и ковач, который сидит у наковальни и думает об изделии из железа: дым от огня изнуряет его тело, и с жаром от печи борется он;
\vs Sir 38:30 звук молота оглушает его слух, и глаза его устремлены на модель сосуда;
\vs Sir 38:31 сердце его устремлено на окончание дела, и попечение его~--- о том, чтобы отделать его в совершенстве.
\vs Sir 38:32 Так и горшечник, который сидит над своим делом и ногами своими вертит колесо,
\vs Sir 38:33 который постоянно в заботе о деле своем и у которого исчислена вся работа его:
\vs Sir 38:34 рукою своею он дает форму глине, а ногами умягчает ее жесткость;
\vs Sir 38:35 он устремляет сердце к тому, чтобы хорошо окончить сосуд, и забота его~--- о том, чтоб очистить печь.
\vs Sir 38:36 Все они надеются на свои руки, и каждый умудряется в своем деле;
\vs Sir 38:37 без них ни город не построится, ни жители не населятся и не будут жить в нем;
\vs Sir 38:38 и однако ж они в собрание не приглашаются, на судейском седалище не сидят и не рассуждают о судебных постановлениях, не произносят оправдания и осуждения и не занимаются притчами;
\vs Sir 38:39 но поддерживают быт житейский, и молитва их~--- об успехе художества их.
\vs Sir 39:1 Только тот, кто посвящает свою душу размышлению о законе Всевышнего, будет искать мудрости всех древних и упражняться в пророчествах:
\vs Sir 39:2 он будет замечать сказания мужей именитых и углубляться в тонкие обороты притчей;
\vs Sir 39:3 будет исследовать сокровенный смысл изречений и заниматься загадками притчей.
\vs Sir 39:4 Он будет проходить служение среди вельмож и являться пред правителем;
\vs Sir 39:5 будет путешествовать по земле чужих народов, ибо испытал доброе и злое между людьми.
\vs Sir 39:6 Сердце свое он направит к тому, чтобы с раннего утра обращаться к Господу, сотворившему его, и будет молиться пред Всевышним; откроет в молитве уста свои и будет молиться о грехах своих.
\vs Sir 39:7 Если Господу великому угодно будет, он исполнится духом разума,
\vs Sir 39:8 будет источать слова мудрости своей и в молитве прославлять Господа;
\vs Sir 39:9 благоуправит свою волю и ум и будет размышлять о тайнах Господа;
\vs Sir 39:10 он покажет мудрость своего учения и будет хвалиться законом завета Господня.
\vs Sir 39:11 Многие будут прославлять знание его, и он не будет забыт вовек;
\vs Sir 39:12 память о нем не погибнет, и имя его будет жить в роды родов.
\vs Sir 39:13 Народы будут прославлять его мудрость, и общество будет возвещать хвалу его;
\vs Sir 39:14 доколе будет жить, он приобретет б\acc{о}льшую славу, нежели тысячи; а когда почиет, увеличит ее.
\rsbpar\vs Sir 39:15 Еще размыслив, расскажу, ибо я полон, как луна в полноте своей.
\vs Sir 39:16 Выслушайте меня, благочестивые дети, и растите, как роза, растущая на поле при потоке;
\vs Sir 39:17 издавайте благоухание, как ливан;
\vs Sir 39:18 цветите, как лилия, распространяйте благовоние и пойте песнь;
\vs Sir 39:19 благословляйте Господа во всех делах; величайте имя Его и прославляйте Его хвалою Его,
\vs Sir 39:20 песнями уст и гуслями и, прославляя, говорите так:
\vs Sir 39:21 все дела Господа весьма благотворны, и всякое повеление Его в свое время исполнится;
\vs Sir 39:22 и нельзя сказать: <<что это? для чего это?>>, ибо все в свое время откроется.
\vs Sir 39:23 По слову Его стала вода, как стог, и по изречению уст Его \bibemph{явились} вместилища вод.
\vs Sir 39:24 В повелениях Его~--- всё Его благоволение, и никто не может умалить спасительность их.
\vs Sir 39:25 Пред Ним дела всякой плоти, и невозможно укрыться от очей Его.
\vs Sir 39:26 Он прозирает из века в век, и ничего нет дивного пред Ним.
\vs Sir 39:27 Нельзя сказать: <<что это? для чего это?>>, ибо все создано для своего употребления.
\vs Sir 39:28 Благословение Его покрывает, как река, и, как потоп, напояет сушу.
\vs Sir 39:29 Но и гнев Его испытывают народы, как некогда Он превратил воды в солончаки.
\vs Sir 39:30 Пути Его для святых прямы, а для беззаконных они~--- преткновения.
\vs Sir 39:31 От начала для добрых создано доброе, как для грешников~--- злое.
\vs Sir 39:32 Главное из всех потребностей для жизни человека~--- вода, огонь, железо, соль, пшеничная мука, мед, молоко, виноградный сок, масло и одежда:
\vs Sir 39:33 все это благочестивым служит в пользу, а грешникам может обратиться во вред.
\vs Sir 39:34 Есть ветры, которые созданы для отмщения и в ярости своей усиливают удары свои,
\vs Sir 39:35 во время устремления своего изливают силу и удовлетворяют ярости Сотворившего их.
\vs Sir 39:36 Огонь и град, голод и смерть~--- все это создано для отмщения;
\vs Sir 39:37 зубы зверей, и скорпионы, и змеи, и меч, мстящий нечестивым погибелью,~---
\vs Sir 39:38 обрадуются повелению Его и готовы будут на земле, когда потребуются, и в свое время не преступят слова Его.
\vs Sir 39:39 Посему я с самого начала решил, обдумал и оставил в писании,
\vs Sir 39:40 что все дела Господа прекрасны, и Он дарует все потребное в свое время;
\vs Sir 39:41 и нельзя сказать: <<это хуже того>>, ибо все в свое время признано будет хорошим.
\vs Sir 39:42 Итак, всем сердцем и устами пойте и благословляйте имя Господа.
\vs Sir 40:1 Много трудов предназначено каждому человеку, и тяжело иго на сынах Адама со дня исхода из чрева матери их до дня возвращения к матери всех.
\vs Sir 40:2 Мысль об ожидаемом и день смерти производит в них размышления и страх сердца.
\vs Sir 40:3 От сидящего на славном престоле и до поверженного на земле и во прахе,
\vs Sir 40:4 от носящего порфиру и венец и до одетого в рубище,~---
\vs Sir 40:5 \bibemph{у всякого} досада и ревность, и смущение, и беспокойство, и страх смерти, и негодование, и распря, и во время успокоения на ложе ночной сон расстраивает ум его.
\vs Sir 40:6 Мало, почти совсем не имеет он покоя, и потому и во сне он, как днем, на страже:
\vs Sir 40:7 будучи смущен сердечными своими мечтами, как бежавший с поля брани, во время безопасности своей он пробуждается и не может надивиться, что ничего не было страшного.
\vs Sir 40:8 Хотя \bibemph{это бывает} со всякою плотью, от человека до скота, но у грешников в семь крат более сего.
\vs Sir 40:9 Смерть, убийство, ссора, меч, бедствия, голод, сокрушение и удары,~---
\vs Sir 40:10 все это~--- для беззаконных; и потоп был для них.
\vs Sir 40:11 Все, что от земли, обращается в землю, и что из воды, возвращается в море.
\vs Sir 40:12 Всякий подарок и несправедливость будут истреблены, а верность будет стоять вовек.
\vs Sir 40:13 Имения неправедных, как поток, иссохнут и, как сильный гром при проливном дожде, прогремят.
\vs Sir 40:14 Кто открывает руку, тот бывает весел; а преступники вконец погибнут.
\vs Sir 40:15 Потомки нечестивых не умножат ветвей, и нечистые корни~--- на утесистой скале:
\vs Sir 40:16 осока при всякой воде и на берегу реки скашивается прежде всякой другой травы.
\vs Sir 40:17 Благотворительность, как рай, полна благословений, и милостыня пребывает вовек.
\vs Sir 40:18 Жизнь довольного своею участью \bibemph{и} труженика сладостна; но превосходит обоих тот, кто находит сокровище.
\vs Sir 40:19 Дети и построение города увековечивают имя, но превосходнее того и другого считается безукоризненная жена.
\vs Sir 40:20 Вино и музыка веселят сердце, но лучше того и другого~--- любовь к мудрости.
\vs Sir 40:21 Свирель и гусли делают приятным пение, но лучше их~--- приятный язык.
\vs Sir 40:22 Приятность и красота вожделенны для очей твоих, но более той и другой~--- зелень посева.
\vs Sir 40:23 Друг и приятель сходятся по временам, но жена с мужем~--- всегда.
\vs Sir 40:24 Братья и покровители~--- во время скорби, но вернее тех и других спасает милостыня.
\vs Sir 40:25 Золото и серебро утверждают стопы, но надежнее того и другого признаётся \bibemph{добрый} совет.
\vs Sir 40:26 Богатство и сила возвышают сердце, но выше того~--- страх Господень:
\vs Sir 40:27 в страхе Господнем нет недостатка, и нет надобности искать при нем помощи;
\vs Sir 40:28 страх Господень~--- как благословенный рай, и облекает его всякою славою.
\rsbpar\vs Sir 40:29 Сын мой! не живи жизнью нищенскою: лучше умереть, нежели просить милостыни.
\vs Sir 40:30 Кто засматривается на чужой стол, того жизнь~--- не жизнь: он унижает душу свою чужими яствами;
\vs Sir 40:31 но человек разумный и благовоспитанный предостережет себя от того.
\vs Sir 40:32 В устах бесстыдного сладким покажется прошение милостыни, но в утробе его огонь возгорится.
\vs Sir 41:1 О, смерть! как горько воспоминание о тебе для человека, который спокойно живет в своих владениях,
\vs Sir 41:2 для человека, который ничем не озабочен и во всем счастлив и еще в силах принимать пищу.
\vs Sir 41:3 О, смерть! отраден твой приговор для человека, нуждающегося и изнемогающего в силах,
\vs Sir 41:4 для престарелого и обремененного заботами обо всем, для не имеющего надежды и потерявшего терпение.
\vs Sir 41:5 Не бойся смертного приговора: вспомни о предках твоих и потомках. Это приговор от Господа над всякою плотью.
\vs Sir 41:6 Итак, для чего ты отвращаешься от того, что благоугодно Всевышнему? десять ли, сто ли, или тысяча лет,~---
\vs Sir 41:7 в аде нет исследования о \bibemph{времени} жизни.
\vs Sir 41:8 Дети грешников бывают дети отвратительные и общаются с нечестивыми.
\vs Sir 41:9 Наследие детей грешников погибнет, и вместе с племенем их будет распространяться бесславие.
\vs Sir 41:10 Нечестивого отца будут укорять дети, потому что за него они терпят бесславие.
\vs Sir 41:11 Горе вам, люди нечестивые, которые оставили закон Бога Всевышнего!
\vs Sir 41:12 Когда вы рождаетесь, то рождаетесь на проклятие; и когда умираете, то получаете в удел свой проклятие.
\vs Sir 41:13 Все, что из земли, возвратится в землю: так нечестивые~--- от проклятия в погибель.
\vs Sir 41:14 Плач людей бывает о телах их, но грешников и имя недоброе изгладится.
\vs Sir 41:15 Заботься об имени, ибо оно пребудет с тобою долее, нежели многие тысячи золота:
\vs Sir 41:16 дням доброй жизни есть число, но доброе имя пребывает вовек.
\rsbpar\vs Sir 41:17 Соблюдайте, дети, наставление в мире; а сокрытая мудрость и сокровище невидимое~--- какая в них польза?
\vs Sir 41:18 Лучше человек, скрывающий свою глупость, нежели человек, скрывающий свою мудрость.
\vs Sir 41:19 Итак, стыдитесь того, о чем я скажу,
\vs Sir 41:20 ибо не всякую стыдливость хорошо соблюдать и не всё всеми одобряется по истине.
\vs Sir 41:21 Стыдитесь пред отцом и матерью блуда, пред начальником и властелином~--- лжи;
\vs Sir 41:22 пред судьею и князем~--- преступления, пред собранием и народом~--- беззакония;
\vs Sir 41:23 пред товарищем и другом~--- неправды, пред соседями~--- кражи:
\vs Sir 41:24 стыдитесь сего и пред истиною Бога и завета Его. Стыдись и облокачивания на стол, обмана при займе и отдаче;
\vs Sir 41:25 стыдись молчания пред приветствующими, смотрения на распутную женщину, отвращения лица от родственника,
\vs Sir 41:26 отнятия доли и дара, помысла на замужнюю женщину, ухаживания за своею служанкою,
\vs Sir 41:27 и не подходи к постели ее;
\vs Sir 41:28 пред друзьями стыдись слов укорительных,~--- и после того, как ты дал, не попрекай,~---
\vs Sir 41:29 повторения слухов и разглашения слов тайных. И будешь истинно стыдлив и приобретешь благорасположение всякого человека.
\vs Sir 42:1 Не стыдись вот чего, и из лицеприятия не греши:
\vs Sir 42:2 не стыдись \bibemph{точного исполнения} закона Всевышнего и завета, и суда, чтобы оказать правосудие нечестивому,
\vs Sir 42:3 спора между товарищем и посторонними и предоставления наследства друзьям,
\vs Sir 42:4 точности в весах и мерах,~--- много ли, мало ли приобретаешь,~---
\vs Sir 42:5 беспристрастия в купле и продаже и строгого воспитания детей, и~--- окровавить ребро худому рабу.
\vs Sir 42:6 При худой жене хорошо иметь печать, и, где много рук, там запирай.
\vs Sir 42:7 Если что выдаешь, \bibemph{выдавай} счетом и весом и делай всякую выдачу и прием по записи.
\vs Sir 42:8 Не стыдись вразумлять неразумного и глупого, и престарелого, состязающегося с молодыми: и будешь истинно благовоспитанным и заслужишь одобрение от всякого человека.
\rsbpar\vs Sir 42:9 Дочь для отца~--- тайная постоянная забота, и попечение о ней отгоняет сон: в юности ее~--- как бы не отцвела, а в замужестве~--- как бы не опротивела;
\vs Sir 42:10 в девстве~--- как бы не осквернилась и не сделалась беременною в отцовском доме, в замужестве~--- чтобы не нарушила супружеской верности и в сожительстве с мужем не осталась бесплодною.
\vs Sir 42:11 Над бесстыдною дочерью усиль надзор, чтобы она не сделала тебя посмешищем для врагов, притчею в городе и упреком в народе и не осрамила тебя пред обществом.
\vs Sir 42:12 Не смотри на красоту человека и не сиди среди женщин:
\vs Sir 42:13 ибо как из одежд выходит моль, так от женщины~--- лукавство женское.
\vs Sir 42:14 Лучше злой мужчина, нежели ласковая женщина,~--- женщина, которая стыдит до поношения.
\rsbpar\vs Sir 42:15 Воспомяну теперь о делах Господа и расскажу о том, что я видел. По слову Господа \bibemph{явились} дела Его:
\vs Sir 42:16 сияющее солнце смотрит на все, и все дело его полно славы Господней.
\vs Sir 42:17 И святым не предоставил Господь провозвестить о всех чудесах Его, которые утвердил Господь Вседержитель, чтобы вселенная стояла твердо во славу Его.
\vs Sir 42:18 Он проникает бездну и сердце и видит все изгибы их; ибо Господь знает всякое в\acc{е}дение и прозирает в знамения века,
\vs Sir 42:19 возвещая прошедшее и будущее и открывая следы сокровенного;
\vs Sir 42:20 не минует Его никакое помышление и не утаится от Него ни одно слово.
\vs Sir 42:21 Он устроил великие дела Своей премудрости и пребывает прежде века и вовек;
\vs Sir 42:22 Он не увеличился и не умалился и не требовал никакого советника.
\vs Sir 42:23 Как вожделенны все дела Его, хотя мы можем видеть их как только искры!
\vs Sir 42:24 Все они живут и пребывают вовек для всяких потребностей, и все повинуются \bibemph{Ему}.
\vs Sir 42:25 Все они~--- вдвойне, одно напротив другого, и ничего не сотворил Он несовершенным:
\vs Sir 42:26 одно поддерживает благо другого,~--- и кто насытится зрением славы Его?
\vs Sir 43:1 Величие высоты, твердь чистоты, вид неба в славном явлении!
\vs Sir 43:2 Солнце, когда оно является, возвещает о них при восходе: чудное создание, дело Всевышнего!
\vs Sir 43:3 В полдень свой оно иссушает землю, и пред жаром его кто устоит?
\vs Sir 43:4 Распаляют горн для работ плавильных, но втрое сильнее солнце палит горы: дыша пламенем огня и блистая лучами, оно ослепляет глаза.
\vs Sir 43:5 Велик Господь, Который сотворил его, и по слову Его оно поспешно пробегает путь свой.
\vs Sir 43:6 И луна всем в свое время служит указанием времен и знамением века:
\vs Sir 43:7 от луны~--- указание праздника; свет ее умаляется по достижении ею полноты;
\vs Sir 43:8 месяц называется по имени ее; она дивно возрастает в своем изменении;
\vs Sir 43:9 это~--- глава вышних строев; она сияет на тверди небесной;
\vs Sir 43:10 красота неба, слава звезд, блестящее украшение, владыка на высотах!
\vs Sir 43:11 По слову Святаго \bibemph{звезды} стоят по чину и не устают на страже своей.
\vs Sir 43:12 Взгляни на радугу, и прославь Сотворившего ее: прекрасна она в сиянии своем!
\vs Sir 43:13 Величественным кругом своим она обнимает небо; руки Всевышнего распростерли ее.
\vs Sir 43:14 Повелением Его скоро сыплется снег, и быстро сверкают молнии суда Его.
\vs Sir 43:15 Отверзаются сокровищницы и вылетают из них облака, как птицы.
\vs Sir 43:16 Могуществом Своим Он укрепляет облака, и разбиваются камни града;
\vs Sir 43:17 от взора Его потрясаются горы, и по изволению Его веет южный ветер.
\vs Sir 43:18 Голос грома Его приводит в трепет землю, и северная буря и вихрь.
\vs Sir 43:19 Он сыплет снег подобно летящим вниз крылатым, и ниспадение его~--- как опускающаяся саранча;
\vs Sir 43:20 красоте белизны его удивляется глаз, и ниспадению его изумляется сердце.
\vs Sir 43:21 И как соль, рассыпает Он по земле иней, который, замерзая, делается остроконечным.
\vs Sir 43:22 Подует северный холодный ветер,~--- и из воды делается лед: он расстилается на всяком вместилище вод, и вода облекается как бы в латы;
\vs Sir 43:23 поядает горы, и пожигает пустыню, и, как огонь, опаляет траву.
\vs Sir 43:24 Но скорым исцелением всему служит туман; появляющаяся роса прохлаждает от зноя.
\vs Sir 43:25 Повелением Своим Господь укрощает бездну и насаждает на ней острова.
\vs Sir 43:26 Плавающие по морю рассказывают об опасностях на нем, и мы дивимся тому, что слышим ушами нашими:
\vs Sir 43:27 ибо там необычайные и чудные дела, разнообразие всяких животных, роды чудовищ.
\vs Sir 43:28 Чрез Него все успешно достигает своего назначения, и все держится словом Его.
\vs Sir 43:29 Многое можем мы сказать, и, однако же, не постигнем Его, и конец слов: Он есть всё.
\vs Sir 43:30 Где возьмем силу, чтобы прославить Его? ибо Он превыше всех дел Своих.
\vs Sir 43:31 Страшен Господь и весьма велик, и дивно могущество Его!
\vs Sir 43:32 Прославляя Господа, превозносите Его, сколько можете, но и затем Он будет превосходнее;
\vs Sir 43:33 и, величая Его, прибавьте силы: но не труд\acc{и}тесь, ибо не постигнете.
\vs Sir 43:34 Кто видел Его, и объяснит? и кто прославит Его, как Он есть?
\vs Sir 43:35 Много сокрыто, что гораздо больше сего; ибо мы видим малую часть дел Его.
\vs Sir 43:36 Всё сотворил Господь, и благочестивым даровал мудрость.
\vs Sir 44:1 Теперь восхвалим славных мужей и отцов нашего рода:
\vs Sir 44:2 много славного Господь являл \bibemph{чрез них}, величие Свое от века;
\vs Sir 44:3 это были господствующие в царствах своих и мужи, именитые силою; они давали разумные советы, возвещали в пророчествах;
\vs Sir 44:4 они были руководителями народа при совещаниях и в книжном обучении.
\vs Sir 44:5 Мудрые слова были в учении их; они изобрели музыкальные строи и гимны предали писанию;
\vs Sir 44:6 люди богатые, одаренные силою, они мирно обитали в жилищах своих.
\vs Sir 44:7 Все они были уважаемы между племенами своими и во дни свои были славою.
\vs Sir 44:8 Есть между ними такие, которые оставили по себе имя для возвещения хвалы их,~--- и есть такие, о которых не осталось памяти, которые исчезли, как будто не существовали, и сделались как бы небывшими, и дети их после них.
\vs Sir 44:9 Но те были мужи милости, которых праведные дела не забываются;
\vs Sir 44:10 в семени их пребывает доброе наследство; потомки их~--- в заветах;
\vs Sir 44:11 семя их будет твердо, и дети их~--- ради них;
\vs Sir 44:12 семя их пребудет до века, и слава их не истребится;
\vs Sir 44:13 тела их погребены в мире, и имена их живут в роды;
\vs Sir 44:14 народы будут рассказывать о их мудрости, а церковь будет возвещать их хвалу.
\rsbpar\vs Sir 44:15 Енох угодил Господу и был взят на небо,~--- образ покаяния для \bibemph{всех} родов.
\vs Sir 44:16 Ной оказался совершенным, праведным; во время гнева он был умилостивлением;
\vs Sir 44:17 посему сделался остатком на земле, когда был потоп;
\vs Sir 44:18 с ним заключен был вечный завет, что никакая плоть не истребится более потопом.
\vs Sir 44:19 Авраам~--- великий отец множества народов, и не было подобного ему в славе;
\vs Sir 44:20 он сохранил закон Всевышнего и был в завете с Ним,
\vs Sir 44:21 и на своей плоти утвердил завет и в испытании оказался верным;
\vs Sir 44:22 поэтому Господь с клятвою обещал ему, что в семени его благословятся все народы;
\vs Sir 44:23 обещал умножить его, как прах земли, и возвысить семя его, как звезды, и дать им наследство от моря до моря и от реки до края земли.
\vs Sir 44:24 И Исааку ради Авраама, отца его, Он также подтвердил благословение всех людей и завет;
\vs Sir 44:25 и оно же почило на голове Иакова:
\vs Sir 44:26 Он ущедрил его Своими благословениями, и дал ему в наследие \bibemph{землю}, и отделил участки ее, и разделил между двенадцатью коленами.
\rsbpar\vs Sir 44:27 И произвел от него мужа милости, который приобрел любовь в глазах всякой плоти,
\vs Sir 45:1 возлюбленного Богом и людьми Моисея, которого память благословенна.
\vs Sir 45:2 Он сравнял его в славе со святыми и возвеличил его делами на страх врагам;
\vs Sir 45:3 Он его словом прекращал чудесные знамения, прославил его пред лицем царей, давал чрез него повеления к народу его и показал ему от славы Своей.
\vs Sir 45:4 За верность и кротость его Он освятил его, избрал Себе из всех людей,
\vs Sir 45:5 сподобил его слышать голос Его, ввел его во мглу
\vs Sir 45:6 и дал ему лицем к лицу заповеди, закон жизни и в\acc{е}дения, чтобы он научил Иакова завету и Израиля~--- постановлениям Его.
\vs Sir 45:7 Он возвысил Аарона, подобного ему святого, брата его из колена Левиина,~---
\vs Sir 45:8 постановил с ним вечный завет и дал ему священство в народе; Он благословил его особым украшением и опоясал его поясом славы;
\vs Sir 45:9 Он облек его высшим украшением и облачил его в богатые одежды:
\vs Sir 45:10 в исподнюю одежду, в подир и ефод;
\vs Sir 45:11 и окружил его золотыми яблоками и весьма многими позвонками, чтобы при хождении его они издавали звук, чтобы сделать слышным в храме звон для напоминания сынам народа Его;
\vs Sir 45:12 облек его одеждою святою из золота и гиацинтовой шерсти и крученого виссона художественной работы, словом суда, уримом и туммимом,
\vs Sir 45:13 червленым тканьем искусной работы, многоценными камнями, вырезанными как на печати, в золотой оправе гранильной работы, с вырезанными на память начертаниями \bibemph{имен} по числу колен Израилевых;
\vs Sir 45:14 на кидаре его~--- золотой венец, знамение святыни, слава достоинства: величественное украшение, дело искусства, вожделенное для глаз.
\vs Sir 45:15 Прежде него не было сего от века:
\vs Sir 45:16 непринадлежащий к его племени не одевался так, только сыновья его и потомки его во все времена.
\vs Sir 45:17 Жертвы их приносятся каждый день, всегда по два раза.
\vs Sir 45:18 Моисей наполнил руки его и помазал его святым елеем:
\vs Sir 45:19 ему постановлено в вечный завет и семени его на дни неба, чтобы они служили Ему и вместе священнодействовали и благословляли народ Его именем Его;
\vs Sir 45:20 Он избрал его из всех живущих, чтобы приносить Господу жертву, курение и благоухание в память умилостивления о народе своем;
\vs Sir 45:21 Он дал ему Свои заповеди и власть в постановлениях судебных, чтобы учить Иакова откровениям и наставлять Израиля в законе Его.
\vs Sir 45:22 Восстали против него чужие, и позавидовали ему в пустыне люди, приставшие к Дафану и Авирону, и скопище Корея в ярости и гневе;
\vs Sir 45:23 Господь увидел, и Ему неугодно было это,~--- и они погибли от ярости гнева.
\vs Sir 45:24 Он сотворил над ними чудо, истребив их пламенем огня Своего.
\vs Sir 45:25 И умножил славу Аарона и дал ему наследие~--- отделил им начатки плодов:
\vs Sir 45:26 прежде всего уготовил им хлеб в насыщение, ибо они едят и жертвы Господни, которые Он дал ему и семени его;
\vs Sir 45:27 но он не должен иметь наследия в земле народа и нет ему участка между народом, ибо Он Сам удел и наследие его.
\vs Sir 45:28 Также и Финеес, сын Елеазара, третий по славе, потому что он ревновал о страхе Господнем и, при отпадении народа, устоял в добром расположении души своей и умилостивил Господа к Израилю;
\vs Sir 45:29 посему постановлен с ним завет мира, чтобы быть ему предстоятелем святых и народа своего, чтобы ему и семени его принадлежало достоинство священства навеки.
\vs Sir 45:30 Как по завету с Давидом, сыном Иессея из колена Иудина, царское наследие переходило от сына к сыну, так наследие священства \bibemph{принадлежало} Аарону и семени его.
\vs Sir 45:31 Да даст нам Бог мудрость в нашем сердце~--- судить народ Его справедливо, дабы не погибли блага их и слава их пребыла в роды их.
\vs Sir 46:1 Силен был в бранях Иисус Навин и был преемником Моисея в пророчествах.
\vs Sir 46:2 Соответственно имени своему, он был велик в спасении избранных Божиих, когда мстил восставшим врагам, чтобы ввести Израиля в наследие \bibemph{его}.
\vs Sir 46:3 Как он прославился, когда поднял руки свои и простер меч на города!
\vs Sir 46:4 Кто прежде него так стоял? Ибо он вел брани Господни.
\vs Sir 46:5 Не его ли рукою остановлено было солнце, и один день был как бы два?
\vs Sir 46:6 Он воззвал ко Всевышнему Владыке, когда со всех сторон стеснили его враги, и великий Господь услышал его:
\vs Sir 46:7 камнями града с могущественною силою бросил Он на враждебный народ и погубил противников на склоне горы,
\vs Sir 46:8 дабы язычники познали всеоружие \bibemph{его}, что война его была пред Господом, а он \bibemph{только} следовал за Всемогущим.
\vs Sir 46:9 И во дни Моисея он оказал благодеяние, он и Халев, сын Иефоннии,~--- тем, что они противостояли враждующим, удерживали народ от греха и утишали злой ропот.
\vs Sir 46:10 И они только двое из шестисот тысяч путешествовавших были спасены, чтобы ввести \bibemph{народ} в наследие~--- в землю, текущую молоком и медом.
\vs Sir 46:11 И дал Господь Халеву крепость, которая сохранилась в нем до старости, взойти на высоту земли, и семя его получило наследие,
\vs Sir 46:12 дабы видели все сыны Израилевы, что благо следовать Господу.
\vs Sir 46:13 Также и судии, каждый по своему имени, которых сердце не заблуждалось и которые не отвращались от Господа,~--- да будет память их во благословениях!
\vs Sir 46:14 Да процветут кости их от места своего,
\vs Sir 46:15 и имя их да перейдет к сынам их в прославлении их!
\vs Sir 46:16 Возлюбленный Господом своим Самуил, пророк Господень, учредил царство и помазал царей народу своему;
\vs Sir 46:17 он судил народ по закону Господню, и Господь призирал на Иакова;
\vs Sir 46:18 по вере своей он был истинным пророком, и в словах его дознана верность видения.
\vs Sir 46:19 Он воззвал ко Всемогущему Господу, когда отвсюду теснили его враги, и принес в жертву молодого агнца,~---
\vs Sir 46:20 и Господь возгремел с неба и в сильном шуме слышным сделал голос Свой,
\vs Sir 46:21 и истребил вождей Тирских и всех князей Филистимских.
\vs Sir 46:22 Еще прежде времени вечного успокоения своего он свидетельствовался пред Господом и помазанником \bibemph{Его}: <<имущества, ни даже обуви, я не брал ни от кого>>, и никто не укорил его.
\vs Sir 46:23 Он пророчествовал и по смерти своей, и предсказал царю смерть его, и в пророчестве возвысил из земли голос свой, что беззаконный народ истребится.
\vs Sir 47:1 После сего явился Нафан, чтобы пророчествовать во дни Давида.
\vs Sir 47:2 Как тук, отделенный от мирной жертвы, так Давид от сынов Израилевых.
\vs Sir 47:3 Он играл со львами, как с козлятами, и с медведями, как с ягнятами.
\vs Sir 47:4 В юности своей не убил ли он исполина, не снял ли поношение с народа,
\vs Sir 47:5 когда поднял руку с пращным камнем и низложил гордыню Голиафа?
\vs Sir 47:6 Ибо он воззвал к Господу Всевышнему, и Он дал крепость правой руке его~--- поразить человека, сильного в войне, и возвысить рог народа своего.
\vs Sir 47:7 Так прославил народ его тьмами и восхвалил его в благословениях Господа, как достойного венца славы,
\vs Sir 47:8 ибо он истребил окрестных врагов и смирил враждебных Филистимлян,~--- даже доныне сокрушил рог их.
\vs Sir 47:9 После каждого дела своего он приносил благодарение Святому Всевышнему словом хвалы;
\vs Sir 47:10 от всего сердца он воспевал и любил Создателя своего.
\vs Sir 47:11 И поставил пред жертвенником песнопевцев, чтобы голосом их услаждать песнопение.
\vs Sir 47:12 Он дал праздникам благолепие и с точностью определил времена, чтобы они хвалили святое имя Его и с раннего утра оглашали святилище.
\vs Sir 47:13 И Господь отпустил ему грехи и навеки вознес рог его и даровал ему завет царственный и престол славы в Израиле.
\vs Sir 47:14 После него восстал мудрый сын его и ради \bibemph{отца} жил счастливо.
\vs Sir 47:15 Соломон царствовал в мирные дни, потому что Бог успокоил его со всех сторон, дабы он построил дом во имя Его и приготовил святилище навеки.
\vs Sir 47:16 Как мудр был ты в юности твоей и, подобно реке, полон разума!
\vs Sir 47:17 Душа твоя покрыла землю, и ты наполнил ее загадочными притчами;
\vs Sir 47:18 имя твое пронеслось до отдаленных островов, и ты был любим за мир твой;
\vs Sir 47:19 за песни и изречения, за притчи и изъяснения тебе удивлялись страны.
\vs Sir 47:20 Во имя Господа Бога, наименованного Богом Израиля,
\vs Sir 47:21 ты собрал золото, как медь, и умножил серебро, как свинец.
\vs Sir 47:22 Но ты наклонил чресла твои к женщинам и поработился им телом твоим;
\vs Sir 47:23 ты положил пятно на славу твою и осквернил семя твое так, что навел гнев на детей твоих,~--- и они горько оплакивали твое безумие,~--- что власть разделилась надвое, и от Ефрема произошло непокорное царство.
\vs Sir 47:24 Но Господь не оставит Своей милости и не разрушит ни одного из дел Своих, не истребит потомков избранного Своего и не искоренит семени возлюбившего Его.
\vs Sir 47:25 И Он дал Иакову остаток, и Давиду~--- корень от него.
\rsbpar\vs Sir 47:26 И почил Соломон с отцами своими,
\vs Sir 47:27 и оставил по себе от семени своего безумие народу,
\vs Sir 47:28 скудного разумом Ровоама, который отвратил от себя народ чрез свое совещание,
\vs Sir 47:29 и Иеровоама, сына Наватова, который ввел в грех Израиля и Ефрему указал путь греха.
\vs Sir 47:30 И весьма умножились грехи их, так что они изгнаны были из земли своей;
\vs Sir 47:31 и посягали они на всякое зло, доколе не пришло на них мщение.
\vs Sir 48:1 И восстал Илия пророк, как огонь, и слово его горело, как светильник.
\vs Sir 48:2 Он навел на них голод и ревностью своею умалил \bibemph{число} их;
\vs Sir 48:3 словом Господним он заключил небо и три раза низводил огонь.
\vs Sir 48:4 Как прославился ты, Илия, чудесами твоими, и кто может сравниться с тобою в славе!
\vs Sir 48:5 Ты воздвиг мертвого от смерти и из ада словом Всевышнего;
\vs Sir 48:6 ты низводил в погибель царей и знатных с ложа их;
\vs Sir 48:7 ты слышал на Синае обличение \bibemph{на них} и на Хориве суды мщения;
\vs Sir 48:8 ты помазал царей на воздаяние и пророков~--- в преемники себе;
\vs Sir 48:9 ты восх\acc{и}щен был огненным вихрем на колеснице с огненными конями;
\vs Sir 48:10 ты предназначен был на обличения в свои времена, чтобы утишить гнев, прежде нежели обратится он в ярость,~--- обратить сердце отца к сыну и восстановить колена Иакова.
\vs Sir 48:11 Блаженны видевшие тебя и украшенные любовью,~--- и мы жизнью поживем.
\vs Sir 48:12 Илия сокрыт был вихрем,~--- и Елисей исполнился духом его
\vs Sir 48:13 и во дни свои не трепетал пред князем, и никто не превозмог его;
\vs Sir 48:14 ничто не одолело его, и по успении его пророчествовало тело его.
\vs Sir 48:15 И при жизни своей совершал он чудеса, и по смерти дивны были дела его.
\rsbpar\vs Sir 48:16 При всем том народ не покаялся, и не отступили от грехов своих, доколе не были пленены из земли своей и рассеяны по всей земле.
\vs Sir 48:17 И осталось весьма мало народа и князь из дома Давидова.
\vs Sir 48:18 Некоторые из них делали угодное Богу, а некоторые умножали грехи.
\vs Sir 48:19 Езекия укрепил город свой и провел внутрь его воду, пробил железом скалу и устроил хранилища для воды.
\vs Sir 48:20 Во дни его сделал нашествие Сеннахирим и послал к нему Рабсака, который поднял руку свою на Сион и много величался в гордости своей.
\vs Sir 48:21 Тогда затрепетали сердца и руки их, и они мучились, как родильницы;
\vs Sir 48:22 и воззвали они к Господу милосердому, простерши к Нему руки свои,
\vs Sir 48:23 и Святый скоро услышал их с неба и избавил их рукою Исаии;
\vs Sir 48:24 Он поразил войско Ассириян, и Ангел Его истребил их,
\vs Sir 48:25 ибо Езекия делал угодное Господу и крепко держался путей Давида, отца своего, как заповедал пророк Исаия, великий и верный в видениях своих.
\vs Sir 48:26 В его дни солнце отступило назад, и он прибавил жизни царю.
\vs Sir 48:27 Великим духом своим он провидел отдаленное будущее и утешал сетующих в Сионе;
\vs Sir 48:28 до века возвещал он будущее и сокровенное, прежде нежели оно исполнилось.
\vs Sir 49:1 Память Иосии~--- как состав фимиама, приготовленный искусством мироварника:
\vs Sir 49:2 во всяких устах она будет сладка, как мед и как музыка при угощении вином.
\vs Sir 49:3 Он успешно действовал в обращении народа и истребил мерзости беззакония;
\vs Sir 49:4 он направил к Господу сердце свое и во дни беззаконных утвердил благочестие.
\vs Sir 49:5 Кроме Давида, Езекии и Иосии, все тяжко согрешили,
\vs Sir 49:6 ибо оставили закон Всевышнего; цари Иудейские престали,
\vs Sir 49:7 ибо предали рог свой другим и славу свою~--- чужому народу.
\vs Sir 49:8 Избранный город святыни сожжен, и улицы его опустошены, как предсказал Иеремия,
\vs Sir 49:9 которого они оскорбляли, хотя он еще во чреве освящен был в пророка, чтобы искоренять, поражать и погублять, равно как строить и насаждать.
\rsbpar\vs Sir 49:10 Иезекииль видел явление славы, которую \bibemph{Бог} показал ему в херувимской колеснице;
\vs Sir 49:11 он напоминал о врагах под образом дождя и возвещал доброе тем, которые исправляли пути свои.
\vs Sir 49:12 И двенадцать пророков~--- да процветут кости их от места своего!~--- утешали Иакова и спасали их верною надеждою.
\vs Sir 49:13 Как возвеличим Зоровавеля? И он~--- как перстень на правой руке;
\vs Sir 49:14 также Иисус, сын Иоседека: они во дни свои построили дом и восстановили святый храм Господу, предназначенный к вечной славе.
\vs Sir 49:15 Велика память и Неемии, который воздвиг нам павшие стены, поставил ворота и запоры и возобновил разрушенные домы наши.
\vs Sir 49:16 Не было на земле никого из сотворенных, подобного Еноху,~--- ибо он был восх\acc{и}щен от земли,~---
\vs Sir 49:17 и не родился такой муж, как Иосиф, глава братьев, опора народа,~--- и кости его были почтены.
\vs Sir 49:18 Прославились между людьми Сим и Сиф, но выше всего живущего в творении~--- Адам.
\vs Sir 50:1 Симон, сын Онии, великий священник, при жизни своей исправил дом и во дни свои укрепил храм:
\vs Sir 50:2 им положено основание двойного возвышения~--- возведение высокой ограды храма;
\vs Sir 50:3 во дни его уменьшено водохранилище, окружность медного моря;
\vs Sir 50:4 чтобы предохранить народ свой от бедствия, он укрепил город против осады.
\vs Sir 50:5 Как величествен был он среди народа, при выходе из завесы храма!
\vs Sir 50:6 Как утренняя звезда среди облаков, как луна полная во днях,
\vs Sir 50:7 как солнце, сияющее над храмом Всевышнего, и как радуга, сияющая в величественных облаках,
\vs Sir 50:8 как цвет роз в весенние дни, как лилии при источниках вод, как ветвь ливана в летние дни,
\vs Sir 50:9 как огонь с ладаном в кадильнице,
\vs Sir 50:10 как кованый золотой сосуд, украшенный всякими драгоценными камнями,
\vs Sir 50:11 как маслина с плодами и как возвышающийся до облаков кипарис.
\vs Sir 50:12 Когда он принимал великолепную одежду и облекался во все величественное украшение, то, при восхождении к святому жертвеннику, освещал блеском окружность святилища.
\vs Sir 50:13 Также, когда он принимал \bibemph{жертвенные} части из рук священников, стоя у огня жертвенника,~---
\vs Sir 50:14 вокруг него был венец братьев, как отрасли кедра на Ливане, и они окружали его как финиковые ветви,
\vs Sir 50:15 и все сыны Аарона в славе своей, и приношение Господу в руках их пред всем собранием Израиля.
\vs Sir 50:16 В довершение служб на алтаре, чтобы увенчать приношение Всевышнему Вседержителю,
\vs Sir 50:17 он простирал свою руку к жертвенной чаше, лил в нее из винограда кровь и выливал ее к подножию жертвенника в вон\acc{ю} благоухания Вышнему Всецарю.
\vs Sir 50:18 Тогда сыны Аароновы восклицали, трубили коваными трубами и издавали громкий голос в напоминание пред Всевышним.
\vs Sir 50:19 Тогда весь народ вместе спешил падать лицем на землю, чтобы поклониться Господу своему, Вседержителю, Богу Вышнему;
\vs Sir 50:20 а песнопевцы восхваляли Его своими голосами; в пространном храме раздавалось сладостное пение,
\vs Sir 50:21 и народ молился Господу Всевышнему молитвою пред Милосердым, доколе совершалось славословие Господа,~--- и так оканчивали они службу Ему.
\vs Sir 50:22 Тогда он, сойдя, поднимал руки свои на все собрание сынов Израилевых, чтобы устами своими преподать благословение Господа и похвалиться именем Его;
\vs Sir 50:23 народ повторял поклонение, чтобы принять благословение от Всевышнего.
\vs Sir 50:24 И ныне все благословляйте Бога, Который везде совершает великие дела, Который продлил дни наши от утробы и поступает с нами по милости Своей:
\vs Sir 50:25 да даст Он нам веселие сердца, и да будет во дни наши мир в Израиле до дней века;
\vs Sir 50:26 да сохранит милость Свою к нам и в свое время да избавит нас!
\vs Sir 50:27 Двумя народами гнушается душа моя, а третий не есть народ:
\vs Sir 50:28 \bibemph{это} сидящие на горе Сеир, Филистимляне и глупый народ, живущий в Сикимах.
\vs Sir 50:29 Учение мудрости и благоразумия начертал в книге сей я, Иисус, сын Сирахов, Иерусалимлянин, который излил мудрость от сердца своего.
\vs Sir 50:30 Блажен, кто будет упражняться в сих \bibemph{наставлениях},~--- и кто положит их на сердце, тот сделается мудрым;
\vs Sir 50:31 а если будет исполнять, то все возможет; ибо свет Господень~--- путь его.
\chhdr{Молитва Иисуса, сына Сирахова.}
\vs Sir 51:1 Прославлю Тебя, Господи Царю, и восхвалю Тебя, Бога, Спасителя моего; прославляю имя Твое,
\vs Sir 51:2 ибо Ты был мне покровителем и помощником
\vs Sir 51:3 и избавил тело мое от погибели и от сети клеветнического языка, от уст сплетающих ложь; и против восставших на меня Ты был мне помощником
\vs Sir 51:4 и избавил меня, по множеству милости и ради имени Твоего, от скрежета зубов, готовых пожрать меня,
\vs Sir 51:5 от руки искавших души моей, от многих скорбей, которые я имел,
\vs Sir 51:6 от удушающего со всех сторон огня и из среды пламени, в котором я не сгорел,
\vs Sir 51:7 из глубины чрева адова, от языка нечистого и слова ложного, от клеветы пред царем языка неправедного.
\vs Sir 51:8 Душа моя близка была к смерти,
\vs Sir 51:9 и жизнь моя была близ ада преисподнего:
\vs Sir 51:10 со всех сторон окружали меня, и не было помогающего; искал я глазами заступления от людей,~--- и не было его.
\vs Sir 51:11 И вспомнил я о Твоей, Господи, милости и о делах Твоих от века,
\vs Sir 51:12 что Ты избавляешь надеющихся на Тебя и спасаешь их от руки врагов.
\vs Sir 51:13 И я вознес от земли моление мое и молился о избавлении от смерти:
\vs Sir 51:14 воззвал я к Господу, Отцу Господа моего, чтобы Он не оставил меня во дни скорби, когда не было помощи от людей надменных.
\vs Sir 51:15 Буду хвалить имя Твое непрестанно и воспевать в славословии, ибо молитва моя была услышана;
\vs Sir 51:16 Ты спас меня от погибели и избавил меня от злого времени.
\vs Sir 51:17 За это я буду прославлять и хвалить Тебя и благословлять имя Господа.
\rsbpar\vs Sir 51:18 Будучи еще юношею, прежде нежели пошел я странствовать, открыто искал я мудрости в молитве моей:
\vs Sir 51:19 пред храмом я молился о ней, и до конца буду искать ее; как бы от цвета зреющего винограда,
\vs Sir 51:20 сердце мое радуется о ней; нога моя шла прямым путем, я следил за нею от юности моей.
\vs Sir 51:21 Понемногу наклонял я ухо мое и принимал ее, и находил в ней много наставлений для себя:
\vs Sir 51:22 мне был успех в ней.
\vs Sir 51:23 Воздам славу Дающему мне мудрость.
\vs Sir 51:24 Я решился следовать ей, ревновал о добром, и не постыжусь.
\vs Sir 51:25 Душа моя подвизалась ради нее, и в делах моих я был точен;
\vs Sir 51:26 простирал руки мои к высоте и сознавал мое невежество.
\vs Sir 51:27 Я направил к ней душу мою, и сердце мое предал ей с самого начала~---
\vs Sir 51:28 и при чистоте достиг ее; посему не буду оставлен ею.
\vs Sir 51:29 И подвиглась внутренность моя, чтобы искать ее; посему я приобрел доброе приобретение.
\vs Sir 51:30 В награду мне Бог дал язык, и им я буду хвалить Его.
\vs Sir 51:31 Приблизьтесь ко мне, ненаученные, и водворитесь в доме учения,
\vs Sir 51:32 ибо вы нуждаетесь в этом и души ваши сильно жаждут.
\vs Sir 51:33 Я отверзаю уста мои и говорю: приобретайте ее себе без серебра;
\vs Sir 51:34 подклоните выю вашу под иго ее, и пусть душа ваша принимает учение; его можно найти близко.
\vs Sir 51:35 Видите своими глазами: я немного потрудился~--- и нашел себе великое успокоение.
\vs Sir 51:36 Приобретайте учение и за большое количество серебра,~--- и вы приобретете много золота.
\vs Sir 51:37 Да радуется душа ваша о милости Его, и не стыдитесь хвалить Его;
\vs Sir 51:38 делайте свое дело заблаговременно, и Он в свое время отдаст вашу награду.

\bibbookdescr{Isa}{
  inline={\LARGE Книга\\\Huge Пророка Исаии},
  toc={Исаия},
  bookmark={Исаия},
  header={Исаия},
  %headerleft={},
  %headerright={},
  abbr={Ис}
}
\vs Isa 1:1 Видение Исаии, сына Амосова, которое он видел о Иудее и Иерусалиме, во дни Озии, Иоафама, Ахаза, Езекии~--- царей Иудейских.
\rsbpar\vs Isa 1:2 Слушайте, небеса, и внимай, земля, потому что Господь говорит: Я воспитал и возвысил сыновей, а они возмутились против Меня.
\vs Isa 1:3 Вол знает владетеля своего, и осел~--- ясли господина своего; а Израиль не знает [Меня], народ Мой не разумеет.
\vs Isa 1:4 Увы, народ грешный, народ обремененный беззакониями, племя злодеев, сыны погибельные! Оставили Господа, презрели Святаго Израилева,~--- повернулись назад.
\vs Isa 1:5 Во что вас бить еще, продолжающие свое упорство? Вся голова в язвах, и все сердце исчахло.
\vs Isa 1:6 От подошвы ноги до темени головы нет у него здорового места: язвы, пятна, гноящиеся раны, неочищенные и необвязанные и не смягченные елеем.
\vs Isa 1:7 Земля ваша опустошена; города ваши сожжены огнем; поля ваши в ваших глазах съедают чужие; все опустело, как после разорения чужими.
\vs Isa 1:8 И осталась дщерь Сиона, как шатер в винограднике, как шалаш в огороде, как осажденный город.
\vs Isa 1:9 Если бы Господь Саваоф не оставил нам небольшого остатка, то мы были бы то же, что Содом, уподобились бы Гоморре.
\rsbpar\vs Isa 1:10 Слушайте слово Господне, князья Содомские; внимай закону Бога нашего, народ Гоморрский!
\vs Isa 1:11 К чему Мне множество жертв ваших? говорит Господь. Я пресыщен всесожжениями овнов и туком откормленного скота, и крови тельцов и агнцев и козлов не хочу.
\vs Isa 1:12 Когда вы приходите являться пред лице Мое, кто требует от вас, чтобы вы топтали дворы Мои?
\vs Isa 1:13 Не нос\acc{и}те больше даров тщетных: курение отвратительно для Меня; новомесячий и суббот, праздничных собраний не могу терпеть: беззаконие~--- и празднование!
\vs Isa 1:14 Новомесячия ваши и праздники ваши ненавидит душа Моя: они бремя для Меня; Мне тяжело нести их.
\vs Isa 1:15 И когда вы простираете руки ваши, Я закрываю от вас очи Мои; и когда вы умножаете моления ваши, Я не слышу: ваши руки полны крови.
\rsbpar\vs Isa 1:16 Омойтесь, очиститесь; удалите злые деяния ваши от очей Моих; перестаньте делать зло;
\vs Isa 1:17 научитесь делать добро, ищите правды, спасайте угнетенного, защищайте сироту, вступайтесь за вдову.
\vs Isa 1:18 Тогда придите~--- и рассудим, говорит Господь. Если будут грехи ваши, как багряное,~--- как снег убелю; если будут красны, как пурпур,~--- как в\acc{о}лну убелю.
\vs Isa 1:19 Если захотите и послушаетесь, то будете вкушать блага земли;
\vs Isa 1:20 если же отречетесь и будете упорствовать, то меч пожрет вас: ибо уста Господни говорят.
\vs Isa 1:21 Как сделалась блудницею верная столица, исполненная правосудия! Правда обитала в ней, а теперь~--- убийцы.
\vs Isa 1:22 Серебро твое стало изгарью, вино твое испорчено водою;
\vs Isa 1:23 князья твои~--- законопреступники и сообщники воров; все они любят подарки и гоняются за мздою; не защищают сироты, и дело вдовы не доходит до них.
\rsbpar\vs Isa 1:24 Посему говорит Господь, Господь Саваоф, Сильный Израилев: о, удовлетворю Я Себя над противниками Моими и отмщу врагам Моим!
\vs Isa 1:25 И обращу на тебя руку Мою и, как в щелочи, очищу с тебя примесь, и отделю от тебя все свинцовое;
\vs Isa 1:26 и опять буду поставлять тебе судей, как прежде, и советников, как вначале; тогда будут говорить о тебе: <<город правды, столица верная>>.
\vs Isa 1:27 Сион спасется правосудием, и обратившиеся \bibemph{сыны} его~--- правдою;
\vs Isa 1:28 всем же отступникам и грешникам~--- погибель, и оставившие Господа истребятся.
\vs Isa 1:29 Они будут постыжены за дубравы, которые столь вожделенны для вас, и посрамлены за сады, которые вы избрали себе;
\vs Isa 1:30 ибо вы будете, как дуб, \bibemph{которого} лист опал, и как сад, в котором нет воды.
\vs Isa 1:31 И сильный будет отрепьем, и дело его~--- искрою; и будут гореть вместе,~--- и никто не потушит.
\vs Isa 2:1 Слово, которое было в видении к Исаии, сыну Амосову, о Иудее и Иерусалиме.
\vs Isa 2:2 И будет в последние дни, гора дома Господня будет поставлена во главу гор и возвысится над холмами, и потекут к ней все народы.
\vs Isa 2:3 И пойдут многие народы и скажут: придите, и взойдем на гору Господню, в дом Бога Иаковлева, и научит Он нас Своим путям и будем ходить по стезям Его; ибо от Сиона выйдет закон, и слово Господне~--- из Иерусалима.
\vs Isa 2:4 И будет Он судить народы, и обличит многие племена; и перекуют мечи свои на орала, и копья свои~--- на серпы: не поднимет народ на народ меча, и не будут более учиться воевать.
\rsbpar\vs Isa 2:5 О, дом Иакова! Придите, и будем ходить во свете Господнем.
\vs Isa 2:6 Но Ты отринул народ Твой, дом Иакова, потому что они многое переняли от востока: и чародеи \bibemph{у них}, как у Филистимлян, и с сынами чужих они в общении.
\vs Isa 2:7 И наполнилась земля его серебром и золотом, и нет числа сокровищам его; и наполнилась земля его конями, и нет числа колесницам его;
\vs Isa 2:8 и наполнилась земля его идолами: они поклоняются делу рук своих, тому, что сделали персты их.
\vs Isa 2:9 И преклонился человек, и унизился муж,~--- и Ты не простишь их.
\vs Isa 2:10 Иди в скалу и сокройся в землю от страха Господа и от славы величия Его.
\vs Isa 2:11 Поникнут гордые взгляды человека, и высокое людское унизится; и один Господь будет высок в тот день.
\vs Isa 2:12 Ибо \bibemph{грядет} день Господа Саваофа на все гордое и высокомерное и на все превознесенное,~--- и оно будет унижено,~---
\vs Isa 2:13 и на все кедры Ливанские, высокие и превозносящиеся, и на все дубы Васанские,
\vs Isa 2:14 и на все высокие горы, и на все возвышающиеся холмы,
\vs Isa 2:15 и на всякую высокую башню, и на всякую крепкую стену,
\vs Isa 2:16 и на все корабли Фарсисские, и на все вожделенные украшения их.
\vs Isa 2:17 И падет величие человеческое, и высокое людское унизится; и один Господь будет высок в тот день,
\vs Isa 2:18 и идолы совсем исчезнут.
\vs Isa 2:19 И войдут \bibemph{люди} в расселины скал и в пропасти земли от страха Господа и от славы величия Его, когда Он восстанет сокрушить землю.
\vs Isa 2:20 В тот день человек бросит кротам и летучим мышам серебряных своих идолов и золотых своих идолов, которых сделал себе для поклонения им,
\vs Isa 2:21 чтобы войти в ущелья скал и в расселины гор от страха Господа и от славы величия Его, когда Он восстанет сокрушить землю.
\vs Isa 2:22 Перестаньте вы надеяться на человека, которого дыхание в ноздрях его, ибо что он значит?
\vs Isa 3:1 Вот, Господь, Господь Саваоф, отнимет у Иерусалима и у Иуды посох и трость, всякое подкрепление хлебом и всякое подкрепление водою,
\vs Isa 3:2 храброго вождя и воина, судью и пророка, и прозорливца и старца,
\vs Isa 3:3 пятидесятника и вельможу и советника, и мудрого художника и искусного в слове.
\vs Isa 3:4 И дам им отроков в начальники, и дети будут господствовать над ними.
\vs Isa 3:5 И в народе один будет угнетаем другим, и каждый~--- ближним своим; юноша будет нагло превозноситься над старцем, и простолюдин над вельможею.
\vs Isa 3:6 Тогда ухватится человек за брата своего, в семействе отца своего, \bibemph{и скажет}: у тебя \bibemph{есть} одежда, будь нашим вождем, и да будут эти развалины под рукою твоею.
\vs Isa 3:7 А \bibemph{он} с клятвою скажет: не могу исцелить \bibemph{ран общества}; и в моем доме нет ни хлеба, ни одежды; не делайте меня вождем народа.
\vs Isa 3:8 Так рушился Иерусалим, и пал Иуда, потому что язык их и дела их~--- против Господа, оскорбительны для очей славы Его.
\vs Isa 3:9 Выражение лиц их свидетельствует против них, и о грехе своем они рассказывают открыто, как Содомляне, не скрывают: горе душе их! ибо сами на себя навлекают зло.
\vs Isa 3:10 Скажите праведнику, что благо \bibemph{ему}, ибо он будет вкушать плоды дел своих;
\vs Isa 3:11 а беззаконнику~--- горе, ибо будет ему возмездие за \bibemph{дела} рук его.
\vs Isa 3:12 Притеснители народа Моего~--- дети, и женщины господствуют над ним. Народ Мой! вожди твои вводят тебя в заблуждение и путь стезей твоих испортили.
\rsbpar\vs Isa 3:13 Восстал Господь на суд~--- и стоит, чтобы судить народы.
\vs Isa 3:14 Господь вступает в суд со старейшинами народа Своего и с князьями его: вы опустошили виноградник; награбленное у бедного~--- в ваших домах;
\vs Isa 3:15 что вы тесните народ Мой и угнетаете бедных? говорит Господь, Господь Саваоф.
\rsbpar\vs Isa 3:16 И сказал Господь: за то, что дочери Сиона надменны и ходят, подняв шею и обольщая взорами, и выступают величавою поступью и гремят цепочками на ногах,~---
\vs Isa 3:17 оголит Господь темя дочерей Сиона и обнажит Господь срамоту их;
\vs Isa 3:18 в тот день отнимет Господь красивые цепочки на ногах и звездочки, и луночки,
\vs Isa 3:19 серьги, и ожерелья, и опахала, увясла и запястья, и пояса, и сосудцы с духами, и привески волшебные,
\vs Isa 3:20 перстни и кольца в носу,
\vs Isa 3:21 верхнюю одежду и нижнюю, и платки, и кошельки,
\vs Isa 3:22 светлые тонкие епанчи и повязки, и покрывала.
\vs Isa 3:23 И будет вместо благовония зловоние, и вместо пояса будет веревка, и вместо завитых волос~--- плешь, и вместо широкой епанчи~--- узкое вретище, вместо красоты~--- клеймо.
\vs Isa 3:24 Мужи твои падут от меча, и храбрые твои~--- на войне.
\vs Isa 3:25 И будут воздыхать и плакать ворота \bibemph{столицы}, и будет она сидеть на земле опустошенная.
\vs Isa 4:1 И ухватятся семь женщин за одного мужчину в тот день, и скажут: <<свой хлеб будем есть и свою одежду будем носить, только пусть будем называться твоим именем,~--- сними с нас позор>>.
\rsbpar\vs Isa 4:2 В тот день отрасль Господа явится в красоте и чести, и плод земли~--- в величии и славе, для уцелевших \bibemph{сынов} Израиля.
\vs Isa 4:3 Тогда оставшиеся на Сионе и уцелевшие в Иерусалиме будут именоваться святыми, все вписанные в книгу для житья в Иерусалиме,
\vs Isa 4:4 когда Господь омоет скверну дочерей Сиона и очистит кровь Иерусалима из среды его духом суда и духом огня.
\vs Isa 4:5 И сотворит Господь над всяким местом гор\acc{ы} Сиона и над собраниями ее облако и дым во время дня и блистание пылающего огня во время ночи; ибо над всем чтимым будет покров.
\vs Isa 4:6 И будет шатер для осенения днем от зноя и для убежища и защиты от непогод и дождя.
\vs Isa 5:1 Воспою Возлюбленному моему песнь Возлюбленного моего о винограднике Его. У Возлюбленного моего был виноградник на вершине утучненной горы,
\vs Isa 5:2 и Он обнес его оградою, и очистил его от камней, и насадил в нем отборные виноградные лозы, и построил башню посреди его, и выкопал в нем точило, и ожидал, что он принесет добрые грозды, а он принес дикие ягоды.
\vs Isa 5:3 И ныне, жители Иерусалима и мужи Иуды, рассудите Меня с виноградником Моим.
\vs Isa 5:4 Что еще надлежало бы сделать для виноградника Моего, чего Я не сделал ему? Почему, когда Я ожидал, что он принесет добрые грозды, он принес дикие ягоды?
\vs Isa 5:5 Итак Я скажу вам, что сделаю с виноградником Моим: отниму у него ограду, и будет он опустошаем; разрушу стены его, и будет попираем,
\vs Isa 5:6 и оставлю его в запустении: не будут ни обрезывать, ни вскапывать его,~--- и зарастет он тернами и волчцами, и повелю облакам не проливать на него дождя.
\vs Isa 5:7 Виноградник Господа Саваофа есть дом Израилев, и мужи Иуды~--- любимое насаждение Его. И ждал Он правосудия, но вот~--- кровопролитие; \bibemph{ждал} правды, и вот~--- вопль.
\rsbpar\vs Isa 5:8 Горе вам, прибавляющие дом к дому, присоединяющие поле к полю, так что \bibemph{другим} не остается места, как будто вы одни поселены на земле.
\vs Isa 5:9 В уши мои \bibemph{сказал} Господь Саваоф: многочисленные домы эти будут пусты, большие и красивые~--- без жителей;
\vs Isa 5:10 десять участков в винограднике дадут один бат, и хомер посеянного зерна едва принесет ефу.
\vs Isa 5:11 Горе тем, которые с раннего утра ищут сикеры и до позднего вечера разгорячают себя вином;
\vs Isa 5:12 и цитра и гусли, тимпан и свирель и вино на пиршествах их; а на дела Господа они не взирают и о деяниях рук Его не помышляют.
\vs Isa 5:13 За то народ мой пойдет в плен непредвиденно, и вельможи его будут голодать, и богачи его будут томиться жаждою.
\vs Isa 5:14 За то преисподняя расширилась и без меры раскрыла пасть свою: и сойдет \bibemph{туда} слава их и богатство их, и шум их и \bibemph{всё}, что веселит их.
\vs Isa 5:15 И преклонится человек, и смирится муж, и глаза гордых поникнут;
\vs Isa 5:16 а Господь Саваоф превознесется в суде, и Бог Святый явит святость Свою в правде.
\vs Isa 5:17 И будут пастись овцы по своей воле, и чужие будут питаться оставленными жирными пажитями богатых.
\vs Isa 5:18 Горе тем, которые влекут на себя беззаконие вервями суетности, и грех~--- как бы ремнями колесничными;
\vs Isa 5:19 которые говорят: <<пусть Он поспешит и ускорит дело Свое, чтобы мы видели, и пусть приблизится и придет в исполнение совет Святаго Израилева, чтобы мы узнали!>>
\vs Isa 5:20 Горе тем, которые зло называют добром, и добро~--- злом, тьму почитают светом, и свет~--- тьмою, горькое почитают сладким, и сладкое~--- горьким!
\vs Isa 5:21 Горе тем, которые мудры в своих глазах и разумны пред самими собою!
\vs Isa 5:22 Горе тем, которые храбры пить вино и сильны приготовлять крепкий напиток,
\vs Isa 5:23 которые за подарки оправдывают виновного и правых лишают законного!
\vs Isa 5:24 За то, как огонь съедает солому, и пламя истребляет сено, так истлеет корень их, и цвет их разнесется, как прах; потому что они отвергли закон Господа Саваофа и презрели слово Святаго Израилева.
\vs Isa 5:25 За то возгорится гнев Господа на народ Его, и прострет Он руку Свою на него и поразит его, так что содрогнутся горы, и трупы их будут как помет на улицах. И при всем этом гнев Его не отвратится, и рука Его еще будет простерта.
\vs Isa 5:26 И поднимет знамя народам дальним, и даст знак живущему на краю земли,~--- и вот, он легко и скоро придет;
\vs Isa 5:27 не будет у него ни усталого, ни изнемогающего; ни один не задремлет и не заснет, и не снимется пояс с чресл его, и не разорвется ремень у обуви его;
\vs Isa 5:28 стрелы его заострены, и все луки его натянуты; копыта коней его подобны кремню, и колеса его~--- как вихрь;
\vs Isa 5:29 рев его~--- как рев львицы; он рыкает подобно скимнам, и заревет, и схватит добычу и унесет, и никто не отнимет.
\vs Isa 5:30 И заревет на него в тот день как бы рев \bibemph{разъяренного} моря; и взглянет он на землю, и вот~--- тьма, горе, и свет померк в облаках.
\vs Isa 6:1 В год смерти царя Озии видел я Господа, сидящего на престоле высоком и превознесенном, и края риз Его наполняли весь храм.
\vs Isa 6:2 Вокруг Него стояли Серафимы; у каждого из них по шести крыл: двумя закрывал каждый лице свое, и двумя закрывал ноги свои, и двумя летал.
\vs Isa 6:3 И взывали они друг ко другу и говорили: Свят, Свят, Свят Господь Саваоф! вся земля полна славы Его!
\vs Isa 6:4 И поколебались верхи врат от гласа восклицающих, и дом наполнился курениями.
\vs Isa 6:5 И сказал я: горе мне! погиб я! ибо я человек с нечистыми устами, и живу среди народа также с нечистыми устами,~--- и глаза мои видели Царя, Господа Саваофа.
\vs Isa 6:6 Тогда прилетел ко мне один из Серафимов, и в руке у него горящий уголь, который он взял клещами с жертвенника,
\vs Isa 6:7 и коснулся уст моих и сказал: вот, это коснулось уст твоих, и беззаконие твое удалено от тебя, и грех твой очищен.
\rsbpar\vs Isa 6:8 И услышал я голос Господа, говорящего: кого Мне послать? и кто пойдет для Нас? И я сказал: вот я, пошли меня.
\vs Isa 6:9 И сказал Он: пойди и скажи этому народу: слухом услышите~--- и не уразумеете, и очами смотреть будете~--- и не увидите.
\vs Isa 6:10 Ибо огрубело сердце народа сего, и ушами с трудом слышат, и очи свои сомкнули, да не узрят очами, и не услышат ушами, и не уразумеют сердцем, и не обратятся, чтобы Я исцелил их.
\vs Isa 6:11 И сказал я: надолго ли, Господи? Он сказал: доколе не опустеют города, и останутся без жителей, и домы без людей, и доколе земля эта совсем не опустеет.
\vs Isa 6:12 И удалит Господь людей, и великое запустение будет на этой земле.
\vs Isa 6:13 И если еще останется десятая часть на ней и возвратится, и она опять будет разорена; \bibemph{но} как от теревинфа и как от дуба, когда они и срублены, \bibemph{остается} корень их, так святое семя \bibemph{будет} корнем ее.
\vs Isa 7:1 И было во дни Ахаза, сына Иоафамова, сына Озии, царя Иудейского, Рецин, царь Сирийский, и Факей, сын Ремалиин, царь Израильский, пошли против Иерусалима, чтобы завоевать его, но не могли завоевать.
\vs Isa 7:2 И было возвещено дому Давидову и сказано: Сирияне расположились в земле Ефремовой; и всколебалось сердце его и сердце народа его, как колеблются от ветра дерева в лесу.
\rsbpar\vs Isa 7:3 И сказал Господь Исаии: выйди ты и сын твой Шеар-ясув навстречу Ахазу, к концу водопровода верхнего пруда, на дорогу к полю белильничьему,
\vs Isa 7:4 и скажи ему: наблюдай и будь спокоен; не страшись и да не унывает сердце твое от двух концов этих дымящихся головней, от разгоревшегося гнева Рецина и Сириян и сына Ремалиина.
\vs Isa 7:5 Сирия, Ефрем и сын Ремалиин умышляют против тебя зло, говоря:
\vs Isa 7:6 пойдем на Иудею и возмутим ее, и овладеем ею и поставим в ней царем сына Тавеилова.
\vs Isa 7:7 Но Господь Бог так говорит: это не состоится и не сбудется;
\vs Isa 7:8 ибо глава Сирии~--- Дамаск, и глава Дамаска~--- Рецин; а чрез шестьдесят пять лет Ефрем перестанет быть народом;
\vs Isa 7:9 и глава Ефрема~--- Самария, и глава Самарии~--- сын Ремалиин. Если вы не верите, то потому, что вы не удостоверены.
\vs Isa 7:10 И продолжал Господь говорить к Ахазу, и сказал:
\vs Isa 7:11 проси себе знамения у Господа Бога твоего: проси или в глубине, или на высоте.
\vs Isa 7:12 И сказал Ахаз: не буду просить и не буду искушать Господа.
\vs Isa 7:13 Тогда сказал \bibemph{Исаия}: слушайте же, дом Давидов! разве мало для вас затруднять людей, что вы хотите затруднять и Бога моего?
\vs Isa 7:14 Итак Сам Господь даст вам знамение: се, Дева во чреве приимет и родит Сына, и нарекут имя Ему: Еммануил.
\vs Isa 7:15 Он будет питаться молоком и медом, доколе не будет разуметь отвергать худое и избирать доброе;
\vs Isa 7:16 ибо прежде нежели Этот Младенец будет разуметь отвергать худое и избирать доброе, земля та, которой ты страшишься, будет оставлена обоими царями ее.
\vs Isa 7:17 Но наведет Господь на тебя и на народ твой и на дом отца твоего дни, какие не приходили со времени отпадения Ефрема от Иуды, наведет царя Ассирийского.
\vs Isa 7:18 И будет в тот день: даст знак Господь мухе, которая при устье реки Египетской, и пчеле, которая в земле Ассирийской,~---
\vs Isa 7:19 и прилетят и усядутся все они по долинам опустелым и по расселинам скал, и по всем колючим кустарникам, и по всем деревам.
\vs Isa 7:20 В тот день обреет Господь бритвою, нанятою по ту сторону реки, царем Ассирийским, голову и волоса на ногах, и даже отнимет бороду.
\vs Isa 7:21 И будет в тот день: кто будет содержать корову и двух овец,
\vs Isa 7:22 по изобилию молока, которое они дадут, будет есть масло; маслом и медом будут питаться все, оставшиеся в этой земле.
\vs Isa 7:23 И будет в тот день: на всяком месте, где росла тысяча виноградных лоз на тысячу сребреников, будет терновник и колючий кустарник.
\vs Isa 7:24 Со стрелами и луками будут ходить туда, ибо вся земля будет терновником и колючим кустарником.
\vs Isa 7:25 И ни на одну из гор, которые расчищались бороздниками, не пойдешь, боясь терновника и колючего кустарника: туда будут выгонять волов, и мелкий скот будет топтать их.
\vs Isa 8:1 И сказал мне Господь: возьми себе большой свиток и начертай на нем человеческим письмом: Магер-шелал-хаш-баз\fns{Спешит грабеж, ускоряет добыча.}.
\vs Isa 8:2 И я взял себе верных свидетелей: Урию священника и Захарию, сына Варахиина,~---
\vs Isa 8:3 и приступил я к пророчице, и она зачала и родила сына. И сказал мне Господь: нареки ему имя: Магер-шелал-хаш-баз,
\vs Isa 8:4 ибо прежде нежели дитя будет уметь выговорить: отец мой, мать моя,~--- богатства Дамаска и добычи Самарийские понесут перед царем Ассирийским.
\rsbpar\vs Isa 8:5 И продолжал Господь говорить ко мне и сказал еще:
\vs Isa 8:6 за то, что этот народ пренебрегает водами Силоама, текущими тихо, и восхищается Рецином и сыном Ремалииным,
\vs Isa 8:7 наведет на него Господь воды реки бурные и большие~--- царя Ассирийского со всею славою его; и поднимется она во всех протоках своих и выступит из всех берегов своих;
\vs Isa 8:8 и пойдет по Иудее, наводнит ее и высоко поднимется~--- дойдет до шеи; и распростертие крыльев ее будет во всю широту земли Твоей, Еммануил!
\vs Isa 8:9 Враждуйте, народы, но трепещите, и внимайте, все отдаленные земли! Вооружайтесь, но трепещите; вооружайтесь, но трепещите!
\vs Isa 8:10 Замышляйте замыслы, но они рушатся; говорите слово, но оно не состоится: ибо с нами Бог!
\vs Isa 8:11 Ибо так говорил мне Господь, \bibemph{держа на мне} крепкую руку и внушая мне не ходить путем сего народа, и сказал:
\vs Isa 8:12 <<Не называйте заговором всего того, что народ сей называет заговором; и не бойтесь того, чего он боится, и не страшитесь.
\vs Isa 8:13 Господа Саваофа~--- Его чтите свято, и Он~--- страх ваш, и Он~--- трепет ваш!
\vs Isa 8:14 И будет Он освящением и камнем преткновения, и скалою соблазна для обоих домов Израиля, петлею и сетью для жителей Иерусалима.
\vs Isa 8:15 И многие из них преткнутся и упадут, и разобьются, и запутаются в сети, и будут уловлены.
\vs Isa 8:16 Завяжи свидетельство, и запечатай откровение при учениках Моих>>.
\rsbpar\vs Isa 8:17 Итак я надеюсь на Господа, сокрывшего лице Свое от дома Иаковлева, и уповаю на Него.
\vs Isa 8:18 Вот я и дети, которых дал мне Господь, как указания и предзнаменования в Израиле от Господа Саваофа, живущего на горе Сионе.
\vs Isa 8:19 И когда скажут вам: обратитесь к вызывателям умерших и к чародеям, к шептунам и чревовещателям,~--- тогда отвечайте: не должен ли народ обращаться к своему Богу? спрашивают ли мертвых о живых?
\vs Isa 8:20 \bibemph{Обращайтесь} к закону и откровению. Если они не говорят, как это слово, то нет в них света.
\vs Isa 8:21 И будут они бродить по земле, жестоко угнетенные и голодные; и во время голода будут злиться, хулить царя своего и Бога своего.
\vs Isa 8:22 И взглянут вверх, и посмотрят на землю; и вот~--- горе и мрак, густая тьма, и будут повержены во тьму. Но не всегда будет мрак там, где теперь он огустел.
\vs Isa 9:1 Прежнее время умалило землю Завулонову и землю Неффалимову; но последующее возвеличит приморский путь, Заиорданскую страну, Галилею языческую.
\vs Isa 9:2 Народ, ходящий во тьме, увидит свет великий; на живущих в стране тени смертной свет воссияет.
\vs Isa 9:3 Ты умножишь народ, увеличишь радость его. Он будет веселиться пред Тобою, как веселятся во время жатвы, как радуются при разделе добычи.
\vs Isa 9:4 Ибо ярмо, тяготившее его, и жезл, поражавший его, и трость притеснителя его Ты сокрушишь, как в день Мадиама.
\vs Isa 9:5 Ибо всякая обувь воина во время брани и одежда, обагренная кровью, будут отданы на сожжение, в пищу огню.
\vs Isa 9:6 Ибо Младенец родился нам~--- Сын дан нам; владычество на раменах Его, и нарекут имя Ему: Чудный, Советник, Бог крепкий, Отец вечности, Князь мира\fns{В Славянской Библии этот стих читается так: <<Ибо Младенец родился нам, Сын, и дан нам; владычество Его на раменах Его, и нарекут имя Ему: Великого Совета Ангел, Чудный, Советник, Бог крепкий, Властелин, Князь мира, Отец будущего века; ибо приведу мир князьям, мир и здравие Его>>.}.
\vs Isa 9:7 Умножению владычества Его и мира нет предела на престоле Давида и в царстве его, чтобы Ему утвердить его и укрепить его судом и правдою отныне и до века. Ревность Господа Саваофа соделает это.
\rsbpar\vs Isa 9:8 Слово посылает Господь на Иакова, и оно нисходит на Израиля,
\vs Isa 9:9 чтобы знал весь народ, Ефрем и жители Самарии, которые с гордостью и надменным сердцем говорят:
\vs Isa 9:10 кирпичи пали~--- построим из тесаного камня; сикоморы вырублены~--- заменим их кедрами.
\vs Isa 9:11 И воздвигнет Господь против него врагов Рецина, и неприятелей его вооружит:
\vs Isa 9:12 Сириян с востока, а Филистимлян с запада; и будут они пожирать Израиля полным ртом. При всем этом не отвратится гнев Его, и рука Его еще простерта.
\vs Isa 9:13 Но народ не обращается к Биющему его, и к Господу Саваофу не прибегает.
\vs Isa 9:14 И отсечет Господь у Израиля голову и хвост, пальму и трость, в один день:
\vs Isa 9:15 старец и знатный,~--- это голова; а пророк-лжеучитель есть хвост.
\vs Isa 9:16 И вожди сего народа введут его в заблуждение, и водимые ими погибнут.
\vs Isa 9:17 Поэтому о юношах его не порадуется Господь, и сирот его и вдов его не помилует: ибо все они~--- лицемеры и злодеи, и уста всех говорят нечестиво. При всем этом не отвратится гнев Его, и рука Его еще простерта.
\vs Isa 9:18 Ибо беззаконие, как огонь, разгорелось, пожирает терновник и колючий кустарник и пылает в чащах леса, и поднимаются столбы дыма.
\vs Isa 9:19 Ярость Господа Саваофа опалит землю, и народ сделается как бы пищею огня; не пощадит человек брата своего.
\vs Isa 9:20 И будут резать по правую сторону, и останутся голодны; и будут есть по левую, и не будут сыты; каждый будет пожирать плоть мышцы своей:
\vs Isa 9:21 Манассия~--- Ефрема, и Ефрем~--- Манассию, оба вместе~--- Иуду. При всем этом не отвратится гнев Его, и рука Его еще простерта.
\vs Isa 10:1 Горе тем, которые постановляют несправедливые законы и пишут жестокие решения,
\vs Isa 10:2 чтобы устранить бедных от правосудия и похитить права у малосильных из народа Моего, чтобы вдов сделать добычею своею и ограбить сирот.
\vs Isa 10:3 И что вы будете делать в день посещения, когда придет гибель издалека? К кому прибегнете за помощью? И где оставите богатство ваше?
\vs Isa 10:4 Без Меня согнутся между узниками и падут между убитыми. При всем этом не отвратится гнев Его, и рука Его еще простерта.
\vs Isa 10:5 О, Ассур, жезл гнева Моего! и бич в руке его~--- Мое негодование!
\vs Isa 10:6 Я пошлю его против народа нечестивого и против народа гнева Моего, дам ему повеление ограбить грабежом и добыть добычу и попирать его, как грязь на улицах.
\vs Isa 10:7 Но он не так подумает и не так помыслит сердце его; у него будет на сердце~--- разорить и истребить немало народов.
\vs Isa 10:8 Ибо он скажет: <<не все ли цари князья мои?
\vs Isa 10:9 Халне не то же ли, что Кархемис? Емаф не то же ли, что Арпад? Самария не то же ли, что Дамаск?
\vs Isa 10:10 Так как рука моя овладела царствами идольскими, в которых кумиров более, нежели в Иерусалиме и Самарии,~---
\vs Isa 10:11 то не сделаю ли того же с Иерусалимом и изваяниями его, что сделал с Самариею и идолами ее?>>
\vs Isa 10:12 И будет, когда Господь совершит все Свое дело на горе Сионе и в Иерусалиме, скажет: посмотрю на успех надменного сердца царя Ассирийского и на тщеславие высоко поднятых глаз его.
\vs Isa 10:13 Он говорит: <<силою руки моей и моею мудростью я сделал это, потому что я умен: и переставляю пределы народов, и расхищаю сокровища их, и низвергаю с престолов, как исполин;
\vs Isa 10:14 и рука моя захватила богатство народов, как гнезда; и как забирают оставленные в них яйца, так забрал я всю землю, и никто не пошевелил крылом, и не открыл рта, и не пискнул>>.
\vs Isa 10:15 Величается ли секира пред тем, кто рубит ею? Пила гордится ли пред тем, кто двигает ее? Как будто жезл восстает против того, кто поднимает его; как будто палка поднимается на того, кто не дерево!
\vs Isa 10:16 За то Господь, Господь Саваоф, пошлет чахлость на тучных его, и между знаменитыми его возжет пламя, как пламя огня.
\vs Isa 10:17 Свет Израиля будет огнем, и Святый его~--- пламенем, которое сожжет и пожрет терны его и волчцы его в один день;
\vs Isa 10:18 и славный лес его и сад его, от души до тела, истребит; и он будет, как чахлый умирающий.
\vs Isa 10:19 И остаток дерев леса его так будет малочислен, что дитя в состоянии будет сделать опись.
\rsbpar\vs Isa 10:20 И будет в тот день: остаток Израиля и спасшиеся из дома Иакова не будут более полагаться на того, кто поразил их, но возложат упование на Господа, Святаго Израилева, чистосердечно.
\vs Isa 10:21 Остаток обратится, остаток Иакова~--- к Богу сильному.
\vs Isa 10:22 Ибо, хотя бы народа у тебя, Израиль, \bibemph{было} столько, сколько песку морского, только остаток его обратится; истребление определено изобилующею правдою;
\vs Isa 10:23 ибо определенное истребление совершит Господь, Господь Саваоф, во всей земле.
\vs Isa 10:24 Посему так говорит Господь, Господь Саваоф: народ Мой, живущий на Сионе! не бойся Ассура. Он поразит тебя жезлом и трость свою поднимет на тебя, как Египет.
\vs Isa 10:25 Еще немного, очень немного, и пройдет Мое негодование, и ярость Моя \bibemph{обратится} на истребление их.
\vs Isa 10:26 И поднимет Господь Саваоф бич на него, как во время поражения Мадиама у скалы Орива, или как \bibemph{простер} на море жезл, и поднимет его, как на Египет.
\vs Isa 10:27 И будет в тот день: снимется с рамен твоих бремя его, и ярмо его~--- с шеи твоей; и распадется ярмо от тука.
\vs Isa 10:28 Он идет на Аиаф, проходит Мигрон, в Михмасе складывает свои запасы.
\vs Isa 10:29 Проходят теснины; в Геве ночлег их; Рама трясется; Гива Саулова разбежалась.
\vs Isa 10:30 Вой голосом твоим, дочь Галима; пусть услышит тебя Лаис, бедный Анафоф!
\vs Isa 10:31 Мадмена разбежалась, жители Гевима спешат уходить.
\vs Isa 10:32 Еще день простоит он в Нове; грозит рукою своею горе Сиону, холму Иерусалимскому.
\vs Isa 10:33 Вот, Господь, Господь Саваоф, страшною силою сорвет ветви дерев, и величающиеся ростом будут срублены, высокие~--- повержены на землю.
\vs Isa 10:34 И посечет чащу леса железом, и Ливан падет от Всемогущего.
\vs Isa 11:1 И произойдет отрасль от корня Иессеева, и ветвь произрастет от корня его;
\vs Isa 11:2 и почиет на Нем Дух Господень, дух премудрости и разума, дух совета и крепости, дух в\acc{е}дения и благочестия;
\vs Isa 11:3 и страхом Господним исполнится, и будет судить не по взгляду очей Своих и не по слуху ушей Своих решать дела.
\vs Isa 11:4 Он будет судить бедных по правде, и дела страдальцев земли решать по истине; и жезлом уст Своих поразит землю, и духом уст Своих убьет нечестивого.
\vs Isa 11:5 И будет препоясанием чресл Его правда, и препоясанием бедр Его~--- истина.
\vs Isa 11:6 Тогда волк будет жить вместе с ягненком, и барс будет лежать вместе с козленком; и теленок, и молодой лев, и вол будут вместе, и малое дитя будет водить их.
\vs Isa 11:7 И корова будет пастись с медведицею, и детеныши их будут лежать вместе, и лев, как вол, будет есть солому.
\vs Isa 11:8 И младенец будет играть над норою аспида, и дитя протянет руку свою на гнездо змеи.
\vs Isa 11:9 Не будут делать зла и вреда на всей святой горе Моей, ибо земля будет наполнена в\acc{е}дением Господа, как воды наполняют море.
\rsbpar\vs Isa 11:10 И будет в тот день: к корню Иессееву, который станет, как знамя для народов, обратятся язычники,~--- и покой его будет слава.
\vs Isa 11:11 И будет в тот день: Господь снова прострет руку Свою, чтобы возвратить Себе остаток народа Своего, какой останется у Ассура, и в Египте, и в Патросе, и у Хуса, и у Елама, и в Сеннааре, и в Емафе, и на островах моря.
\vs Isa 11:12 И поднимет знамя язычникам, и соберет изгнанников Израиля, и рассеянных Иудеев созовет от четырех концов земли.
\vs Isa 11:13 И прекратится зависть Ефрема, и враждующие против Иуды будут истреблены. Ефрем не будет завидовать Иуде, и Иуда не будет притеснять Ефрема.
\vs Isa 11:14 И полетят на плеча Филистимлян к западу, ограбят всех детей Востока; на Едома и Моава наложат руку свою, и дети Аммона будут подданными им.
\vs Isa 11:15 И иссушит Господь залив моря Египетского, и прострет руку Свою на реку в сильном ветре Своем, и разобьет ее на семь ручьев, так что в сандалиях могут переходить ее.
\vs Isa 11:16 Тогда для остатка народа Его, который останется у Ассура, будет большая дорога, как это было для Израиля, когда он выходил из земли Египетской.
\vs Isa 12:1 И скажешь в тот день: славлю Тебя, Господи; Ты гневался на меня, но отвратил гнев Твой и утешил меня.
\vs Isa 12:2 Вот, Бог~--- спасение мое: уповаю на Него и не боюсь; ибо Господь~--- сила моя, и пение мое~--- Господь; и Он был мне во спасение.
\vs Isa 12:3 И в радости будете почерпать воду из источников спасения,
\vs Isa 12:4 и скажете в тот день: славьте Господа, призывайте имя Его; возвещайте в народах дела Его; напоминайте, что велико имя Его;
\vs Isa 12:5 пойте Господу, ибо Он соделал великое,~--- да знают это по всей земле.
\vs Isa 12:6 Веселись и радуйся, жительница Сиона, ибо велик посреди тебя Святый Израилев.
\vs Isa 13:1 Пророчество о Вавилоне, которое изрек Исаия, сын Амосов.
\vs Isa 13:2 Поднимите знамя на открытой горе, возвысьте голос; махните им рукою, чтобы шли в ворота властелинов.
\vs Isa 13:3 Я дал повеление избранным Моим и призвал для \bibemph{совершения} гнева Моего сильных Моих, торжествующих в величии Моем.
\vs Isa 13:4 Большой шум на горах, как бы от многолюдного народа, мятежный шум царств и народов, собравшихся вместе: Господь Саваоф обозревает боевое войско.
\vs Isa 13:5 Идут из отдаленной страны, от края неба, Господь и орудия гнева Его, чтобы сокрушить всю землю.
\vs Isa 13:6 Рыдайте, ибо день Господа близок, идет как разрушительная сила от Всемогущего.
\vs Isa 13:7 Оттого руки у всех опустились, и сердце у каждого человека растаяло.
\vs Isa 13:8 Ужаснулись, судороги и боли схватили их; мучатся, как рождающая, с изумлением смотрят друг на друга, лица у них разгорелись.
\vs Isa 13:9 Вот, приходит день Господа лютый, с гневом и пылающею яростью, чтобы сделать землю пустынею и истребить с нее грешников ее.
\vs Isa 13:10 Звезды небесные и светила не дают от себя света; солнце меркнет при восходе своем, и луна не сияет светом своим.
\vs Isa 13:11 Я накажу мир за зло, и нечестивых~--- за беззакония их, и положу конец высокоумию гордых, и уничижу надменность притеснителей;
\vs Isa 13:12 сделаю то, что люди будут дороже чистого золота, и мужи~--- дороже золота Офирского.
\vs Isa 13:13 Для сего потрясу небо, и земля сдвинется с места своего от ярости Господа Саваофа, в день пылающего гнева Его.
\vs Isa 13:14 Тогда каждый, как преследуемая серна и как покинутые овцы, обратится к народу своему, и каждый побежит в свою землю.
\vs Isa 13:15 Но кто попадется, будет пронзен, и кого схватят, тот падет от меча.
\vs Isa 13:16 И младенцы их будут разбиты пред глазами их; домы их будут разграблены и жены их обесчещены.
\vs Isa 13:17 Вот, Я подниму против них М\acc{и}дян, которые не ценят серебра и не пристрастны к золоту.
\vs Isa 13:18 Л\acc{у}ки их сразят юношей и не пощадят плода чрева: глаз их не сжалится над детьми.
\vs Isa 13:19 И Вавилон, краса царств, гордость Халдеев, будет ниспровержен Богом, как Содом и Гоморра,
\vs Isa 13:20 не заселится никогда, и в роды родов не будет жителей в нем; не раскинет Аравитянин шатра своего, и пастухи со стадами не будут отдыхать там.
\vs Isa 13:21 Но будут обитать в нем звери пустыни, и домы наполнятся филинами; и страусы поселятся, и косматые будут скакать там.
\vs Isa 13:22 Шакалы будут выть в чертогах их, и гиены~--- в увеселительных домах.
\vs Isa 14:1 Близко время его, и не замедлят дни его, ибо помилует Господь Иакова и снова возлюбит Израиля; и поселит их на земле их, и присоединятся к ним иноземцы и прилепятся к дому Иакова.
\vs Isa 14:2 И возьмут их народы, и приведут на место их, и дом Израиля усвоит их себе на земле Господней рабами и рабынями, и возьмет в плен пленивших его, и будет господствовать над угнетателями своими.
\rsbpar\vs Isa 14:3 И будет в тот день: когда Господь устроит тебя от скорби твоей и от страха и от тяжкого рабства, которому ты порабощен был,
\vs Isa 14:4 ты произнесешь победную песнь на царя Вавилонского и скажешь: как не стало мучителя, пресеклось грабительство!
\vs Isa 14:5 Сокрушил Господь жезл нечестивых, скипетр владык,
\vs Isa 14:6 поражавший народы в ярости ударами неотвратимыми, во гневе господствовавший над племенами с неудержимым преследованием.
\vs Isa 14:7 Вся земля отдыхает, покоится, восклицает от радости;
\vs Isa 14:8 и кипарисы радуются о тебе, и кедры ливанские, \bibemph{говоря}: <<с тех пор, как ты заснул, никто не приходит рубить нас>>.
\vs Isa 14:9 Ад преисподний пришел в движение ради тебя, чтобы встретить тебя при входе твоем; пробудил для тебя Рефаимов, всех вождей земли; поднял всех царей языческих с престолов их.
\vs Isa 14:10 Все они будут говорить тебе: и ты сделался бессильным, как мы! и ты стал подобен нам!
\vs Isa 14:11 В преисподнюю низвержена гордыня твоя со всем шумом твоим; под тобою подстилается червь, и черви~--- покров твой.
\vs Isa 14:12 Как упал ты с неба, денница, сын зари! разбился о землю, попиравший народы.
\vs Isa 14:13 А говорил в сердце своем: <<взойду на небо, выше звезд Божиих вознесу престол мой и сяду на гор\acc{е} в сонме богов, на краю севера;
\vs Isa 14:14 взойду на высоты облачные, буду подобен Всевышнему>>.
\vs Isa 14:15 Но ты низвержен в ад, в глубины преисподней.
\vs Isa 14:16 Видящие тебя всматриваются в тебя, размышляют о тебе: <<тот ли это человек, который колебал землю, потрясал царства,
\vs Isa 14:17 вселенную сделал пустынею и разрушал города ее, пленников своих не отпускал домой?>>.
\vs Isa 14:18 Все цари народов, все лежат с честью, каждый в своей усыпальнице;
\vs Isa 14:19 а ты повержен вне гробницы своей, как презренная ветвь, как одежда убитых, сраженных мечом, которых опускают в каменные рвы,~--- ты, как попираемый труп,
\vs Isa 14:20 не соединишься с ними в могиле; ибо ты разорил землю твою, убил народ твой: во веки не помянется племя злодеев.
\vs Isa 14:21 Готовьте заклание сыновьям его за беззаконие отца их, чтобы не восстали и не завладели землею и не наполнили вселенной неприятелями.
\vs Isa 14:22 И восстану на них, говорит Господь Саваоф, и истреблю имя Вавилона и весь остаток, и сына и внука, говорит Господь.
\vs Isa 14:23 И сделаю его владением ежей и болотом, и вымету его метлою истребительною, говорит Господь Саваоф.
\vs Isa 14:24 С клятвою говорит Господь Саваоф: как Я помыслил, так и будет; как Я определил, так и состоится,
\vs Isa 14:25 чтобы сокрушить Ассура в земле Моей и растоптать его на горах Моих; и спадет с них ярмо его, и снимется бремя его с рамен их.
\vs Isa 14:26 Таково определение, постановленное о всей земле, и вот рука, простертая на все народы,
\vs Isa 14:27 ибо Господь Саваоф определил, и кто может отменить это? рука Его простерта,~--- и кто отвратит ее?
\rsbpar\vs Isa 14:28 В год смерти царя Ахаза было такое пророческое слово:
\vs Isa 14:29 не радуйся, земля Филистимская, что сокрушен жезл, который поражал тебя, ибо из корня змеиного выйдет аспид, и плодом его будет летучий дракон.
\vs Isa 14:30 Тогда беднейшие будут накормлены, и нищие будут покоиться в безопасности; а твой корень уморю голодом, и он убьет остаток твой.
\vs Isa 14:31 Рыдайте, ворота! вой голосом, город! Распадешься ты, вся земля Филистимская, ибо от севера дым идет, и нет отсталого в полчищах их.
\vs Isa 14:32 Что же скажут вестники народа?~--- То, что Господь утвердил Сион, и в нем найдут убежище бедные из народа Его.
\vs Isa 15:1 Пророчество о Моаве.~--- Так! ночью будет разорен Ар-Моав и уничтожен; так! ночью будет разорен Кир-Моав и уничтожен!
\vs Isa 15:2 Он восходит к Баиту и Дивону, восходит на высоты, чтобы плакать; Моав рыдает над Нев\acc{о} и Медевою; у всех их острижены головы, у всех обриты бороды.
\vs Isa 15:3 На улицах его препоясываются вретищем; на кровлях его и площадях его всё рыдает, утопает в слезах.
\vs Isa 15:4 И вопит Есевон и Елеала; голос их слышится до самой Иаацы; за ними и воины Моава рыдают; душа его возмущена в нем.
\vs Isa 15:5 Рыдает сердце мое о Моаве; бегут из него к Сигору, до третьей Эглы; восходят на Лухит с плачем; по дороге Хоронаимской поднимают страшный крик;
\vs Isa 15:6 потому что воды Нимрима иссякли, луга засохли, трава выгорела, не стало зелени.
\vs Isa 15:7 Поэтому они остатки стяжания и, что сбережено ими, переносят за реку Аравийскую.
\vs Isa 15:8 Ибо вопль по всем пределам Моава, до Эглаима плач его и до Беэр-Елима плач его;
\vs Isa 15:9 потому что воды Димона наполнились кровью, и Я наведу на Димон еще новое~--- львов на убежавших из Моава и на оставшихся в стране.
\vs Isa 16:1 Посылайте агнцев владетелю земли из Селы в пустыне на гору дочери Сиона;
\vs Isa 16:2 ибо блуждающей птице, выброшенной из гнезда, будут подобны дочери Моава у бродов Арнонских.
\vs Isa 16:3 <<Составь совет, постанови решение; осени нас среди полудня, как ночью, тенью твоею, укрой изгнанных, не выдай скитающихся.
\vs Isa 16:4 Пусть поживут у тебя мои изгнанные Моавитяне; будь им покровом от грабителя: ибо притеснителя не станет, грабеж прекратится, попирающие исчезнут с земли.
\vs Isa 16:5 И утвердится престол милостью, и воссядет на нем в истине, в шатре Давидовом, судия, ищущий правды и стремящийся к правосудию>>.
\vs Isa 16:6 <<Слыхали мы о гордости Моава, гордости чрезмерной, о надменности его и высокомерии и неистовстве его: неискренна речь его>>.
\vs Isa 16:7 Поэтому возрыдает Моав о Моаве,~--- все будут рыдать; стенайте о твердынях Кирхарешета: они совершенно разрушены.
\vs Isa 16:8 Поля Есевонские оскудели, также и виноградник Севамский; властители народов истребили лучшие лозы его, которые достигали до Иазера, расстилались по пустыне; побеги их расширялись, переходили за море.
\vs Isa 16:9 Посему \bibemph{и} я буду плакать о лозе Севамской плачем Иазера, буду обливать тебя слезами моими, Есевон и Елеала; ибо во время собирания винограда твоего и во время жатвы твоей нет более шумной радости.
\vs Isa 16:10 Исчезло с плодоносной земли веселье и ликование, и в виноградниках не поют, не ликуют; виноградарь не топчет винограда в точилах: Я прекратил ликование.
\vs Isa 16:11 Оттого внутренность моя стонет о Моаве, как гусли, и сердце мое~--- о Кирхарешете.
\vs Isa 16:12 Хотя и явится Моав, и будет до утомления \bibemph{подвизаться} на высотах, и придет к святилищу своему помолиться, но ничто не поможет.
\rsbpar\vs Isa 16:13 Вот слово, которое изрек Господь о Моаве издавна.
\vs Isa 16:14 Ныне же так говорит Господь: чрез три года, считая годами наемничьими, величие Моава будет унижено со всем великим многолюдством, и остаток \bibemph{будет} очень малый и незначительный.
\vs Isa 17:1 Пророчество о Дамаске.~--- Вот, Дамаск исключается из \bibemph{числа} городов и будет грудою развалин.
\vs Isa 17:2 Города Ароерские будут покинуты,~--- останутся для стад, которые будут отдыхать там, и некому будет пугать их.
\vs Isa 17:3 Не станет твердыни Ефремовой и царства Дамасского с остальною Сириею; с ними будет то же, что со славою сынов Израиля, говорит Господь Саваоф.
\rsbpar\vs Isa 17:4 И будет в тот день: умалится слава Иакова, и тучное тело его сделается тощим.
\vs Isa 17:5 То же будет, что по собрании хлеба жнецом, когда рука его пожнет колосья, и когда соберут колосья в долине Рефаимской.
\vs Isa 17:6 И останутся у него, как бывает при обивании маслин, две-три ягоды на самой вершине, или четыре-пять на плодоносных ветвях, говорит Господь, Бог Израилев.
\vs Isa 17:7 В тот день обратит человек взор свой к Творцу своему, и глаза его будут устремлены к Святому Израилеву;
\vs Isa 17:8 и не взглянет на жертвенники, на дело рук своих, и не посмотрит на то, что сделали персты его, на кумиры Астарты и Ваала.
\vs Isa 17:9 В тот день укрепленные города его будут, как развалины в лесах и на вершинах гор, оставленные пред сынами Израиля,~--- и будет пусто.
\vs Isa 17:10 Ибо ты забыл Бога спасения твоего, и не воспоминал о скале прибежища твоего; оттого развел увеселительные сады и насадил черенки от чужой лозы.
\vs Isa 17:11 В день насаждения твоего ты заботился, чтобы оно росло и чтобы посеянное тобою рано расцвело; но в день собирания не куча жатвы будет, но скорбь жестокая.
\vs Isa 17:12 Увы! шум народов многих! шумят они, как шумит море. Рев племен! они ревут, как ревут сильные воды.
\vs Isa 17:13 Ревут народы, как ревут сильные воды; но Он погрозил им и они далеко побежали, и были гонимы, как прах по горам от ветра и как пыль от вихря.
\vs Isa 17:14 Вечер~--- и вот ужас! и прежде утра уже нет его. Такова участь грабителей наших, жребий разорителей наших.
\vs Isa 18:1 Горе земле, осеняющей крыльями по ту сторону рек Ефиопских,
\vs Isa 18:2 посылающей послов по морю, и в папировых суднах по водам! Идите, быстрые послы, к народу крепкому и бодрому, к народу страшному от начала и доныне, к народу рослому и \bibemph{всё} попирающему, которого землю разрезывают реки.
\vs Isa 18:3 Все вы, населяющие вселенную и живущие на земле! смотрите, когда знамя поднимется на горах, и, когда загремит труба, слушайте!
\vs Isa 18:4 Ибо так Господь сказал мне: Я спокойно смотрю из жилища Моего, как светлая теплота после дождя, как облако росы во время жатвенного зноя.
\vs Isa 18:5 Ибо прежде собирания винограда, когда он отцветет, и грозд начнет созревать, Он отрежет ножом ветви и отнимет, и отрубит отрасли.
\vs Isa 18:6 И оставят всё хищным птицам на горах и зверям полевым; и птицы будут проводить там лето, а все звери полевые будут зимовать там.
\vs Isa 18:7 В то время будет принесен дар Господу Саваофу от народа крепкого и бодрого, от народа страшного от начала и доныне, от народа рослого и \bibemph{всё} попирающего, которого землю разрезывают реки,~--- к месту имени Господа Саваофа, на гору Сион.
\vs Isa 19:1 Пророчество о Египте.~--- Вот, Господь восседит на облаке легком и грядет в Египет. И потрясутся от лица Его идолы Египетские, и сердце Египта растает в нем.
\vs Isa 19:2 Я вооружу Египтян против Египтян; и будут сражаться брат против брата и друг против друга, город с городом, царство с царством.
\vs Isa 19:3 И дух Египта изнеможет в нем, и разрушу совет его, и прибегнут они к идолам и к чародеям, и к вызывающим мертвых и к гадателям.
\vs Isa 19:4 И предам Египтян в руки властителя жестокого, и свирепый царь будет господствовать над ними, говорит Господь, Господь Саваоф.
\vs Isa 19:5 И истощатся воды в море и река иссякнет и высохнет;
\vs Isa 19:6 и оскудеют реки, и каналы Египетские обмелеют и высохнут; камыш и тростник завянут.
\vs Isa 19:7 Поля при реке, по берегам реки, и все, посеянное при реке, засохнет, развеется и исчезнет.
\vs Isa 19:8 И восплачут рыбаки, и возрыдают все, бросающие уду в реку, и ставящие сети в воде впадут в уныние;
\vs Isa 19:9 и будут в смущении обрабатывающие лен и ткачи белых полотен;
\vs Isa 19:10 и будут сокрушены сети, и все, которые содержат садки для живой рыбы, упадут в духе.
\vs Isa 19:11 Так! обезумели князья Цоанские; совет мудрых советников фараоновых стал бессмысленным. Как скажете вы фараону: <<я сын мудрецов, сын царей древних?>>
\vs Isa 19:12 Где они? где твои мудрецы? пусть они теперь скажут тебе; пусть узн\acc{а}ют, чт\acc{о} Господь Саваоф определил о Египте.
\vs Isa 19:13 Обезумели князья Цоанские; обманулись князья Мемфисские, и совратил Египет с пути гл\acc{а}вы племен его.
\vs Isa 19:14 Господь послал в него дух опьянения; и они ввели Египет в заблуждение во всех делах его, подобно тому, как пьяный бродит по блевотине своей.
\vs Isa 19:15 И не будет в Египте такого дела, которое совершить умели бы голова и хвост, пальма и трость.
\vs Isa 19:16 В тот день Египтяне будут подобны женщинам, и вострепещут и убоятся движения руки Господа Саваофа, которую Он поднимет на них.
\vs Isa 19:17 Земля Иудина сделается ужасом для Египта; кто вспомнит о ней, тот затрепещет от определения Господа Саваофа, которое Он постановил о нем.
\vs Isa 19:18 В тот день пять городов в земле Египетской будут говорить языком Ханаанским и клясться Господом Саваофом; один назовется городом солнца.
\vs Isa 19:19 В тот день жертвенник Господу будет посреди земли Египетской, и памятник Господу~--- у пределов ее.
\vs Isa 19:20 И будет он знамением и свидетельством о Господе Саваофе в земле Египетской, потому что они воззовут к Господу по причине притеснителей, и Он пошлет им спасителя и заступника, и избавит их.
\vs Isa 19:21 И Господь явит Себя в Египте; и Египтяне в тот день познают Господа и принесут жертвы и дары, и дадут обеты Господу, и исполнят.
\vs Isa 19:22 И поразит Господь Египет; поразит и исцелит; они обратятся к Господу, и Он услышит их, и исцелит их.
\vs Isa 19:23 В тот день из Египта в Ассирию будет большая дорога, и будет приходить Ассур в Египет, и Египтяне~--- в Ассирию; и Египтяне вместе с Ассириянами будут служить Господу.
\vs Isa 19:24 В тот день Израиль будет третьим с Египтом и Ассириею; благословение будет посреди земли,
\vs Isa 19:25 которую благословит Господь Саваоф, говоря: благословен народ Мой~--- Египтяне, и дело рук Моих~--- Ассирияне, и наследие Мое~--- Израиль.
\vs Isa 20:1 В год, когда Тартан пришел к Азоту, быв послан от Саргона, царя Ассирийского, и воевал против Азота, и взял его,
\vs Isa 20:2 в то самое время Господь сказал Исаии, сыну Амосову, так: пойди и сними вретище с чресл твоих и сбрось сандалии твои с ног твоих. Он так и сделал: ходил нагой и босой.
\vs Isa 20:3 И сказал Господь: как раб Мой Исаия ходил нагой и босой три года, в указание и предзнаменование о Египте и Ефиопии,
\vs Isa 20:4 так поведет царь Ассирийский пленников из Египта и переселенцев из Ефиопии, молодых и старых, нагими и босыми и с обнаженными чреслами, в посрамление Египту.
\vs Isa 20:5 Тогда ужаснутся и устыдятся из-за Ефиопии, надежды своей, и из-за Египта, которым хвалились.
\vs Isa 20:6 И скажут в тот день жители этой страны: вот каковы те, на которых мы надеялись и к которым прибегали за помощью, чтобы спастись от царя Ассирийского! и как спаслись бы мы?
\vs Isa 21:1 Пророчество о пустыне приморской.~--- Как бури на юге носятся, идет он от пустыни, из земли страшной.
\vs Isa 21:2 Грозное видение показано мне: грабитель грабит, опустошитель опустошает; восходи, Елам, осаждай, Мид! всем стенаниям я положу конец.
\vs Isa 21:3 От этого чресла мои трясутся; муки схватили меня, как муки рождающей. Я взволнован от того, что слышу; я смущен от того, что вижу.
\vs Isa 21:4 Сердце мое трепещет; дрожь бьет меня; отрадная ночь моя превратилась в ужас для меня.
\vs Isa 21:5 Приготовляют стол, расстилают покрывала,~--- едят, пьют. <<Вставайте, князья, мажьте щиты!>>
\vs Isa 21:6 Ибо так сказал мне Господь: пойди, поставь сторожа; пусть он сказывает, что увидит.
\vs Isa 21:7 И увидел он едущих попарно всадников на конях, всадников на ослах, всадников на верблюдах; и вслушивался он прилежно, с большим вниманием,~---
\vs Isa 21:8 и закричал, \bibemph{как} лев: господин мой! на страже стоял я весь день, и на месте моем оставался целые ночи:
\vs Isa 21:9 и вот, едут люди, всадники на конях попарно. Потом он возгласил и сказал: пал, пал Вавилон, и все идолы богов его лежат на земле разбитые.
\vs Isa 21:10 О, измолоченный мой и сын гумна моего! Что слышал я от Господа Саваофа, Бога Израилева, то и возвестил вам.
\rsbpar\vs Isa 21:11 Пророчество о Думе.~--- Кричат мне с Сеира: сторож! сколько ночи? сторож! сколько ночи?
\vs Isa 21:12 Сторож отвечает: приближается утро, но еще ночь. Если вы настоятельно спрашиваете, то обратитесь и приходите.
\rsbpar\vs Isa 21:13 Пророчество об Аравии.~--- В лесу Аравийском ночуйте, караваны Деданские!
\vs Isa 21:14 Живущие в земле Фемайской! несите вод\acc{ы} навстречу жаждущим; с хлебом встречайте бегущих,
\vs Isa 21:15 ибо они от мечей бегут, от меча обнаженного и от лука натянутого, и от лютости войны.
\vs Isa 21:16 Ибо так сказал мне Господь: еще год, равный году наемничьему, и вся слава Кидарова исчезнет,
\vs Isa 21:17 и луков у храбрых сынов Кидара останется немного: так сказал Господь, Бог Израилев.
\vs Isa 22:1 Пророчество о долине видения.~--- Что с тобою, что ты весь взошел на кровли?
\vs Isa 22:2 Город шумный, волнующийся, город ликующий! Пораженные твои не мечом убиты и не в битве умерли;
\vs Isa 22:3 все вожди твои бежали вместе, но были связаны стрелками; все найденные у тебя связаны вместе, как ни далеко бежали.
\vs Isa 22:4 Потому говорю: оставьте меня, я буду плакать горько; не усиливайтесь утешать меня в разорении дочери народа моего.
\vs Isa 22:5 Ибо день смятения и попрания и замешательства в долине видения от Господа, Бога Саваофа. Ломают стену, и крик восходит на горы.
\vs Isa 22:6 И Елам несет колчан; люди на колесницах \bibemph{и} всадники, и Кир обнажает щит.
\vs Isa 22:7 И вот, лучшие долины твои полны колесницами, и всадники выстроились против ворот,
\vs Isa 22:8 и снимают покров с Иудеи; и ты в тот день обращаешь взор на запас оружия в доме кедровом.
\vs Isa 22:9 Но вы видите, что много проломов в стене города Давидова, и собираете в\acc{о}ды в нижнем пруде;
\vs Isa 22:10 и отмечаете домы в Иерусалиме, и разрушаете домы, чтобы укрепить стену;
\vs Isa 22:11 и устрояете между двумя стенами хранилище для вод старого пруда. А на Того, Кто это делает, не взираете, и не смотрите на Того, Кто издавна определил это.
\vs Isa 22:12 И Господь, Господь Саваоф, призывает вас в этот день плакать и сетовать, и остричь волоса и препоясаться вретищем.
\vs Isa 22:13 Но вот, веселье и радость! Убивают волов, и режут овец; едят мясо, и пьют вино: <<будем есть и пить, ибо завтра умрем!>>
\vs Isa 22:14 И открыл мне в уши Господь Саваоф: не будет прощено вам это нечестие, доколе не умрете, сказал Господь, Господь Саваоф.
\rsbpar\vs Isa 22:15 Так сказал Господь, Господь Саваоф: ступай, пойди к этому царедворцу, к Севне, начальнику дворца [и скажи ему]:
\vs Isa 22:16 что у тебя здесь, и кто здесь у тебя, что ты здесь высекаешь себе гробницу?~--- Он высекает себе гробницу на возвышенности, вырубает в скале жилище себе.
\vs Isa 22:17 Вот, Господь перебросит тебя, как бросает сильный человек, и сожмет тебя в ком;
\vs Isa 22:18 свернув тебя в сверток, бросит тебя, как меч, в землю обширную; там ты умрешь, и там великолепные колесницы твои будут поношением для дома господина твоего.
\vs Isa 22:19 И столкну тебя с места твоего, и свергну тебя со степени твоей.
\vs Isa 22:20 И будет в тот день, призову раба Моего Елиакима, сына Хелкиина,
\vs Isa 22:21 и одену его в одежду твою, и поясом твоим опояшу его, и власть твою передам в руки его; и будет он отцом для жителей Иерусалима и для дома Иудина.
\vs Isa 22:22 И ключ дома Давидова возложу на рамена его; отворит он, и никто не запрет; запрет он, и никто не отворит.
\vs Isa 22:23 И укреплю его как гвоздь в твердом месте; и будет он как седалище славы для дома отца своего.
\vs Isa 22:24 И будет висеть на нем вся слава дома отца его, детей и внуков, всей домашней утвари до последних музыкальных орудий.
\vs Isa 22:25 В тот день, говорит Господь Саваоф, пошатнется гвоздь, укрепленный в твердом месте, и будет выбит, и упадет, и распадется вся тяжесть, которая на нем: ибо Господь говорит.
\vs Isa 23:1 Пророчество о Тире.~--- Рыдайте, корабли Фарсиса, ибо он разрушен; нет домов, и некому входить в домы. Так им возвещено из земли Киттийской.
\vs Isa 23:2 Умолкните, обитатели острова, который наполняли купцы Сидонские, плавающие по морю.
\vs Isa 23:3 По великим водам привозились в него семена Сихора, жатва \bibemph{большой} реки, и был он торжищем народов.
\vs Isa 23:4 Устыдись, Сидон; ибо \bibemph{вот что} говорит море, крепость морская: <<как бы ни мучилась я родами и ни рождала, и ни воспитывала юношей, ни возращала девиц>>.
\vs Isa 23:5 Когда весть дойдет до Египтян, содрогнутся они, услышав о Тире.
\vs Isa 23:6 Переселяйтесь в Фарсис, рыдайте, обитатели острова!
\vs Isa 23:7 Это ли ваш ликующий город, которого начало от дней древних? Ноги его несут его скитаться в стране далекой.
\vs Isa 23:8 Кто определил это Тиру, который раздавал венцы, которого купцы \bibemph{были} князья, торговцы~--- знаменитости земли?
\vs Isa 23:9 Господь Саваоф определил это, чтобы посрамить надменность всякой славы, чтобы унизить все знаменитости земли.
\vs Isa 23:10 Ходи по земле твоей, дочь Фарсиса, как река: нет более препоны.
\vs Isa 23:11 Он простер руку Свою на море, потряс царства; Господь дал повеление о Ханаане разрушить крепости его
\vs Isa 23:12 и сказал: ты не будешь более ликовать, посрамленная девица, дочь Сидона! Вставай, иди в Киттим, \bibemph{но} и там не будет тебе покоя.
\vs Isa 23:13 Вот земля Халдеев. Этого народа прежде не было; Ассур положил ему начало из обитателей пустынь. Они ставят башни свои, разрушают чертоги его, превращают его в развалины.
\vs Isa 23:14 Рыдайте, корабли Фарсисские! Ибо твердыня ваша разорена.
\rsbpar\vs Isa 23:15 И будет в тот день, забудут Тир на семьдесят лет, в мере дней одного царя. По окончании же семидесяти лет с Тиром будет то же, что поют о блуднице:
\vs Isa 23:16 <<возьми цитру, ходи по городу, забытая блудница! Играй складно, пой много песен, чтобы вспомнили о тебе>>.
\vs Isa 23:17 И будет, по истечении семидесяти лет, Господь посетит Тир; и он снова начнет получать прибыль свою и будет блудодействовать со всеми царствами земными по всей вселенной.
\vs Isa 23:18 Но торговля его и прибыль его будут посвящаемы Господу; не будут заперты и уложены в кладовые, ибо к живущим пред лицем Господа будет переходить прибыль от торговли его, чтобы они ели до сытости и имели одежду прочную.
\vs Isa 24:1 Вот, Господь опустошает землю и делает ее бесплодною; изменяет вид ее и рассевает живущих на ней.
\vs Isa 24:2 И что будет с народом, то и со священником; что со слугою, то и с господином его; что со служанкою, то и с госпожею ее; что с покупающим, то и с продающим; что с заемщиком, то и с заимодавцем; что с ростовщиком, то и с дающим рост.
\vs Isa 24:3 Земля опустошена вконец и совершенно разграблена, ибо Господь изрек слово сие.
\vs Isa 24:4 Сетует, уныла земля; поникла, уныла вселенная; поникли возвышавшиеся над народом земл\acc{и}.
\vs Isa 24:5 И земля осквернена под живущими на ней, ибо они преступили законы, изменили устав, нарушили вечный завет.
\vs Isa 24:6 За то проклятие поедает землю, и несут наказание живущие на ней; за то сожжены обитатели земли, и немного осталось людей.
\vs Isa 24:7 Плачет сок грозда; болит виноградная лоза; воздыхают все веселившиеся сердцем.
\vs Isa 24:8 Прекратилось веселье с тимпанами; умолк шум веселящихся; затихли звуки гуслей;
\vs Isa 24:9 уже не пьют вина с песнями; горька сикера для пьющих ее.
\vs Isa 24:10 Разрушен опустевший город, все домы заперты, нельзя войти.
\vs Isa 24:11 Плачут о вине на улицах; помрачилась всякая радость; изгнано всякое веселье земли.
\vs Isa 24:12 В городе осталось запустение, и ворота развалились.
\vs Isa 24:13 А посреди земли, между народами, будет то же, что бывает при обивании маслин, при обирании \bibemph{винограда}, когда кончена уборка.
\vs Isa 24:14 Они возвысят голос свой, восторжествуют в величии Господа, громко будут восклицать с моря.
\vs Isa 24:15 Итак славьте Господа на востоке, на островах морских~--- имя Господа, Бога Израилева.
\vs Isa 24:16 От края земли мы слышим песнь: <<Слава Праведному!>> И сказал я: беда мне, беда мне! увы мне! злодеи злодействуют, и злодействуют злодеи злодейски.
\vs Isa 24:17 Ужас и яма и петля для тебя, житель земли!
\vs Isa 24:18 Тогда побежавший от крика ужаса упадет в яму; и кто выйдет из ямы, попадет в петлю; ибо окна с \bibemph{небесной} высоты растворятся, и основания земли потрясутся.
\vs Isa 24:19 Земля сокрушается, земля распадается, земля сильно потрясена;
\vs Isa 24:20 шатается земля, как пьяный, и качается, как колыбель, и беззаконие ее тяготеет на ней; она упадет, и уже не встанет.
\rsbpar\vs Isa 24:21 И будет в тот день: посетит Господь воинство выспреннее на высоте и царей земных на земле.
\vs Isa 24:22 И будут собраны вместе, как узники, в ров, и будут заключены в темницу, и после многих дней будут наказаны.
\vs Isa 24:23 И покраснеет луна, и устыдится солнце, когда Господь Саваоф воцарится на горе Сионе и в Иерусалиме, и пред старейшинами его \bibemph{будет} слава.
\vs Isa 25:1 Господи! Ты Бог мой; превознесу Тебя, восхвалю имя Твое, ибо Ты совершил дивное; предопределения древние истинны, аминь.
\vs Isa 25:2 Ты превратил город в груду камней, твердую крепость в развалины; чертогов иноплеменников уже не стало в городе; вовек не будет он восстановлен.
\vs Isa 25:3 Посему будут прославлять Тебя народы сильные; города страшных племен будут бояться Тебя,
\vs Isa 25:4 ибо Ты был убежищем бедного, убежищем нищего в тесное для него время, защитою от бури, тенью от зноя; ибо гневное дыхание тиранов было подобно буре против стены.
\vs Isa 25:5 Как зной в месте безводном, Ты укротил буйство врагов; \bibemph{как} зной тенью облака, подавлено ликование притеснителей.
\vs Isa 25:6 И сделает Господь Саваоф на горе сей для всех народов трапезу из тучных яств, трапезу из чистых вин, из тука костей и самых чистых вин;
\vs Isa 25:7 и уничтожит на горе сей покрывало, покрывающее все народы, покрывало, лежащее на всех племенах.
\vs Isa 25:8 Поглощена будет смерть навеки, и отрет Господь Бог слезы со всех лиц, и снимет поношение с народа Своего по всей земле; ибо так говорит Господь.
\vs Isa 25:9 И скажут в тот день: вот Он, Бог наш! на Него мы уповали, и Он спас нас! Сей есть Господь; на Него уповали мы; возрадуемся и возвеселимся во спасении Его!
\vs Isa 25:10 Ибо рука Господа почиет на горе сей, и Моав будет попран на месте своем, как попирается солома в навозе.
\vs Isa 25:11 И хотя он распрострет посреди его руки свои, как плавающий распростирает их для плавания; \bibemph{но Бог} унизит гордость его вместе с лукавством рук его.
\vs Isa 25:12 И твердыню высоких стен твоих обрушит, низвергнет, повергнет на землю, в прах.
\vs Isa 26:1 В тот день будет воспета песнь сия в земле Иудиной: город крепкий у нас; спасение дал Он вместо стены и вала.
\vs Isa 26:2 Отворите ворота; да войдет народ праведный, хранящий истину.
\vs Isa 26:3 Твердого духом Ты хранишь в совершенном мире, ибо на Тебя уповает он.
\vs Isa 26:4 Уповайте на Господа вовеки, ибо Господь Бог есть твердыня вечная:
\vs Isa 26:5 Он ниспроверг живших на высоте, высоко стоявший город; поверг его, поверг на землю, бросил его в прах.
\vs Isa 26:6 Нога попирает его, ноги бедного, стопы нищих.
\vs Isa 26:7 Путь праведника прям; Ты уравниваешь стезю праведника.
\vs Isa 26:8 И на пути судов Твоих, Господи, мы уповали на Тебя; к имени Твоему и к воспоминанию о Тебе стремилась душа наша.
\vs Isa 26:9 Душею моею я стремился к Тебе ночью, и духом моим я буду искать Тебя во внутренности моей с раннего утра: ибо когда суды Твои \bibemph{совершаются} на земле, тогда живущие в мире научаются правде.
\vs Isa 26:10 Если нечестивый будет помилован, то не научится он правде,~--- будет злодействовать в земле правых и не будет взирать на величие Господа.
\vs Isa 26:11 Господи! рука Твоя была высоко поднята, но они не видали ее; увидят и устыдятся ненавидящие народ Твой; огонь пожрет врагов Твоих.
\vs Isa 26:12 Господи! Ты даруешь нам мир; ибо и все дела наши Ты устрояешь для нас.
\vs Isa 26:13 Господи Боже наш! другие владыки кроме Тебя господствовали над нами; но чрез Тебя только мы славим имя Твое.
\vs Isa 26:14 Мертвые не оживут; рефаимы не встанут, потому что Ты посетил и истребил их, и уничтожил всякую память о них.
\vs Isa 26:15 Ты умножил народ, Господи, умножил народ,~--- прославил Себя, распространил все пределы земли.
\vs Isa 26:16 Господи! в бедствии он искал Тебя; изливал тихие моления, когда наказание Твое постигало его.
\vs Isa 26:17 Как беременная женщина, при наступлении родов, мучится, вопит от болей своих, так были мы пред Тобою, Господи.
\vs Isa 26:18 Были беременны, мучились,~--- и рождали как бы ветер; спасения не доставили земле, и прочие жители вселенной не пали.
\vs Isa 26:19 Оживут мертвецы Твои, восстанут мертвые тела! Воспрян\acc{и}те и торжествуйте, поверженные в прахе: ибо роса Твоя~--- роса растений, и земля извергнет мертвецов.
\vs Isa 26:20 Пойди, народ мой, войди в покои твои и запри за собой двери твои, укройся на мгновение, доколе не пройдет гнев;
\vs Isa 26:21 ибо вот, Господь выходит из жилища Своего наказать обитателей земли за их беззаконие, и земля откроет поглощенную ею кровь и уже не скроет убитых своих.
\vs Isa 27:1 В тот день поразит Господь мечом Своим тяжелым, и большим и крепким, левиафана, змея прямо бегущего, и левиафана, змея изгибающегося, и убьет чудовище морское.
\vs Isa 27:2 В тот день воспойте о нем~--- о возлюбленном винограднике:
\vs Isa 27:3 Я, Господь, хранитель его, в каждое мгновение напояю его; ночью и днем стерегу его, чтобы кто не ворвался в него.
\vs Isa 27:4 Гнева нет во Мне. Но если бы кто противопоставил Мне \bibemph{в нем} волчцы и терны, Я войною пойду против него, выжгу его совсем.
\vs Isa 27:5 Разве прибегнет к защите Моей и заключит мир со Мною? тогда пусть заключит мир со Мною.
\vs Isa 27:6 В грядущие \bibemph{дни} укоренится Иаков, даст отпрыск и расцветет Израиль; и наполнится плодами вселенная.
\vs Isa 27:7 Так ли Он поражал его, как поражал поражавших его? Так ли убивал его, как убиты убивавшие его?
\vs Isa 27:8 Мерою Ты наказывал его, когда отвергал его; выбросил его сильным дуновением Своим как бы в день восточного ветра.
\vs Isa 27:9 И чрез это загладится беззаконие Иакова; и плодом сего будет снятие греха с него, когда все камни жертвенников он обратит в куски извести, и не будут уже стоять дубравы и истуканы солнца.
\vs Isa 27:10 Ибо укрепленный город опустеет, жилища \bibemph{будут} покинуты и заброшены, как пустыня. Там будет пастись теленок, и там он будет покоиться и объедать ветви его.
\vs Isa 27:11 Когда ветви его засохнут, их обломают; женщины придут и сожгут их. Так как это народ безрассудный, то не сжалится над ним Творец его, и не помилует его Создатель его.
\vs Isa 27:12 Но будет в тот день: Господь потрясет всё от великой реки до потока Египетского, и вы, сыны Израиля, будете собраны один к другому;
\vs Isa 27:13 и будет в тот день: вострубит великая труба, и придут затерявшиеся в Ассирийской земле и изгнанные в землю Египетскую и поклонятся Господу на горе святой в Иерусалиме.
\vs Isa 28:1 Горе венку гордости пьяных Ефремлян, увядшему цветку красивого убранства его, который на вершине тучной долины сраженных вином!
\vs Isa 28:2 Вот, крепкий и сильный у Господа, как ливень с градом и губительный вихрь, как разлившееся наводнение бурных вод, с силою повергает его на землю.
\vs Isa 28:3 Ногами попирается венок гордости пьяных Ефремлян.
\vs Isa 28:4 И с увядшим цветком красивого убранства его, который на вершине тучной долины, делается то же, что бывает с созревшею прежде времени смоквою, которую, как скоро кто увидит, тотчас берет в руку и проглатывает ее.
\rsbpar\vs Isa 28:5 В тот день Господь Саваоф будет великолепным венцом и славною диадемою для остатка народа Своего,
\vs Isa 28:6 и духом правосудия для сидящего в судилище и мужеством для отражающих неприятеля до ворот.
\vs Isa 28:7 Но и эти шатаются от вина и сбиваются с пути от сикеры; священник и пророк спотыкаются от крепких напитков; побеждены вином, обезумели от сикеры, в видении ошибаются, в суждении спотыкаются.
\vs Isa 28:8 Ибо все столы наполнены отвратительною блевотиною, нет \bibemph{чистого} места.~---
\vs Isa 28:9 А \bibemph{говорят}: <<кого хочет он учить в\acc{е}дению? и кого вразумлять проповедью? отнятых от грудного молока, отлученных от сосцов \bibemph{матери}?
\vs Isa 28:10 Ибо всё заповедь на заповедь, заповедь на заповедь, правило на правило, правило на правило, тут немного и там немного>>.
\vs Isa 28:11 За то лепечущими устами и на чужом языке будут говорить к этому народу.
\vs Isa 28:12 Им говорили: <<вот~--- покой, дайте покой утружденному, и вот~--- успокоение>>. Но они не хотели слушать.
\vs Isa 28:13 И стало у них словом Господа: заповедь на заповедь, заповедь на заповедь, правило на правило, правило на правило, тут немного, там немного,~--- так что они пойдут, и упадут навзничь, и разобьются, и попадут в сеть и будут уловлены.
\rsbpar\vs Isa 28:14 Итак слушайте слово Господне, хульники, правители народа сего, который в Иерусалиме.
\vs Isa 28:15 Так как вы говорите: <<мы заключили союз со смертью и с преисподнею сделали договор: когда всепоражающий бич будет проходить, он не дойдет до нас,~--- потому что ложь сделали мы убежищем для себя, и обманом прикроем себя>>.
\rsbpar\vs Isa 28:16 Посему так говорит Господь Бог: вот, Я полагаю в основание на Сионе камень,~--- камень испытанный, краеугольный, драгоценный, крепко утвержденный: верующий в него не постыдится.
\vs Isa 28:17 И поставлю суд мерилом и правду весами; и градом истребится убежище лжи, и воды потопят место укрывательства.
\vs Isa 28:18 И союз ваш со смертью рушится, и договор ваш с преисподнею не устоит. Когда пойдет всепоражающий бич, вы будете попраны.
\vs Isa 28:19 Как скоро он пойдет, схватит вас; ходить же будет каждое утро, день и ночь, и один слух о нем будет внушать ужас.
\vs Isa 28:20 Слишком коротка будет постель, чтобы протянуться; слишком узко и одеяло, чтобы завернуться в него.
\vs Isa 28:21 Ибо восстанет Господь, как на горе Перациме; разгневается, как на долине Гаваонской, чтобы сделать дело Свое, необычайное дело, и совершить действие Свое, чудное Свое действие.
\rsbpar\vs Isa 28:22 Итак не кощунствуйте, чтобы узы ваши не стали крепче; ибо я слышал от Господа, Бога Саваофа, что истребление определено для всей земли.
\vs Isa 28:23 Приклоните ухо, и послушайте моего голоса; будьте внимательны, и выслушайте речь мою.
\vs Isa 28:24 Всегда ли земледелец пашет для посева, бороздит и боронит землю свою?
\vs Isa 28:25 Нет; когда уровняет поверхность ее, он сеет чернуху, или рассыпает тмин, или разбрасывает пшеницу рядами, и ячмень в определенном месте, и полбу рядом с ним.
\vs Isa 28:26 И такому порядку учит его Бог его; Он наставляет его.
\vs Isa 28:27 Ибо не молотят чернухи катком зубчатым, и колес молотильных не катают по тмину; но палкою выколачивают чернуху, и тмин~--- палкою.
\vs Isa 28:28 Зерновой хлеб вымолачивают, но не разбивают его; и водят по нему молотильные колеса с конями их, но не растирают его.
\vs Isa 28:29 И это происходит от Господа Саваофа: дивны судьбы Его, велика премудрость Его!
\vs Isa 29:1 Горе Ариилу, Ариилу, городу, в котором жил Давид! прилож\acc{и}те год к году; пусть заколают жертвы.
\vs Isa 29:2 Но Я стесню Ариил, и будет плач и сетование; и он останется у Меня, как Ариил.
\vs Isa 29:3 Я расположусь станом вокруг тебя и стесню тебя стражею наблюдательною, и воздвигну против тебя укрепления.
\vs Isa 29:4 И будешь унижен, с земли будешь говорить, и глуха будет речь твоя из-под праха, и голос твой будет, как голос чревовещателя, и из-под праха шептать будет речь твоя.
\vs Isa 29:5 Множество врагов твоих будет, как мелкая пыль, и полчище лютых, как разлетающаяся плева; и это совершится внезапно, в одно мгновение.
\vs Isa 29:6 Господь Саваоф посетит тебя громом и землетрясением, и сильным гласом, бурею и вихрем, и пламенем всепожирающего огня.
\vs Isa 29:7 И как сон, как ночное сновидение, будет множество всех народов, воюющих против Ариила, и всех выступивших против него и укреплений его и стеснивших его.
\vs Isa 29:8 И как голодному снится, будто он ест, но пробуждается, и душа его тоща; и как жаждущему снится, будто он пьет, но пробуждается, и вот он томится, и душа его жаждет: то же будет и множеству всех народов, воюющих против горы Сиона.
\vs Isa 29:9 Изумляйтесь и дивитесь: они ослепили других, и сами ослепли; они пьяны, но не от вина,~--- шатаются, но не от сикеры;
\vs Isa 29:10 ибо навел на вас Господь дух усыпления и сомкнул глаза ваши, пророки, и закрыл ваши головы, прозорливцы.
\vs Isa 29:11 И всякое пророчество для вас то же, что слова в запечатанной книге, которую подают умеющему читать книгу и говорят: <<прочитай ее>>; и тот отвечает: <<не могу, потому что она запечатана>>.
\vs Isa 29:12 И передают книгу тому, кто читать не умеет, и говорят: <<прочитай ее>>; и тот отвечает: <<я не умею читать>>.
\rsbpar\vs Isa 29:13 И сказал Господь: так как этот народ приближается ко Мне устами своими, и языком своим чтит Меня, сердце же его далеко отстоит от Меня, и благоговение их предо Мною есть изучение заповедей человеческих;
\vs Isa 29:14 то вот, Я еще необычайно поступлю с этим народом, чудно и дивно, так что мудрость мудрецов его погибнет, и разума у разумных его не станет.
\vs Isa 29:15 Горе тем, которые думают скрыться в глубину, чтобы замысл свой утаить от Господа, которые делают дела свои во мраке и говорят: <<кто увидит нас? и кто узнает нас?>>
\vs Isa 29:16 Какое безрассудство! Разве можно считать горшечника, как глину? Скажет ли изделие о сделавшем его: <<не он сделал меня>>? и скажет ли произведение о художнике своем: <<он не разумеет>>?
\vs Isa 29:17 Еще немного, очень немного, и Ливан не превратится ли в сад, а сад не будут ли почитать, как лес?
\vs Isa 29:18 И в тот день глухие услышат слова книги, и прозрят из тьмы и мрака глаза слепых.
\vs Isa 29:19 И страждущие более и более будут радоваться о Господе, и бедные люди будут торжествовать о Святом Израиля,
\vs Isa 29:20 потому что не будет более обидчика, и хульник исчезнет, и будут истреблены все поборники неправды,
\vs Isa 29:21 которые запутывают человека в словах, и требующему суда у ворот расставляют сети, и отталкивают правого.
\vs Isa 29:22 Посему так говорит о доме Иакова Господь, Который искупил Авраама: тогда Иаков не будет в стыде, и лице его более не побледнеет.
\vs Isa 29:23 Ибо когда увидит у себя детей своих, дело рук Моих, то они свято будут чтить имя Мое и свято чтить Святаго Иаковлева, и благоговеть пред Богом Израилевым.
\vs Isa 29:24 Тогда блуждающие духом познают мудрость, и непокорные научатся послушанию.
\vs Isa 30:1 Горе непокорным сынам, говорит Господь, которые делают совещания, но без Меня, и заключают союзы, но не по духу Моему, чтобы прилагать грех ко греху:
\vs Isa 30:2 не вопросив уст Моих, идут в Египет, чтобы подкрепить себя силою фараона и укрыться под тенью Египта.
\vs Isa 30:3 Но сила фараона будет для вас стыдом, и убежище под тенью Египта~--- бесчестием;
\vs Isa 30:4 потому что князья его\fns{Иерусалима.} уже в Цоане, и послы его дошли до Ханеса.
\vs Isa 30:5 Все они будут постыжены из-за народа, \bibemph{который} бесполезен для них; не будет от него ни помощи, ни пользы, но~--- стыд и срам.
\vs Isa 30:6 Тяжести на животных, \bibemph{идущих} на юг, по земле угнетения и тесноты, откуда \bibemph{выходят} львицы и львы, аспиды и летучие змеи; они несут на хребтах ослов богатства свои и на горбах верблюдов сокровища свои к народу, который не принесет им пользы.
\vs Isa 30:7 Ибо помощь Египта будет тщетна и напрасна; потому Я сказал им: сила их~--- сидеть спокойно.
\vs Isa 30:8 Теперь пойди, начертай это на доске у них, и впиши это в книгу, чтобы осталось на будущее время, навсегда, навеки.
\vs Isa 30:9 Ибо это народ мятежный, дети лживые, дети, которые не хотят слушать закона Господня,
\vs Isa 30:10 которые провидящим говорят: <<перестаньте провидеть>>, и пророкам: <<не пророчествуйте нам правды, говорите нам лестное, предсказывайте приятное;
\vs Isa 30:11 сойдите с дороги, уклонитесь от пути; устраните от глаз наших Святаго Израилева>>.
\rsbpar\vs Isa 30:12 Посему так говорит Святый Израилев: так как вы отвергаете слово сие, а надеетесь на обман и неправду, и опираетесь на то:
\vs Isa 30:13 то беззаконие это будет для вас, как угрожающая падением трещина, обнаружившаяся в высокой стене, которой разрушение настанет внезапно, в одно мгновение.
\vs Isa 30:14 И Он разрушит ее, как сокрушают глиняный сосуд, разбивая его без пощады, так что в обломках его не найдется и черепка, чтобы взять огня с очага или зачерпнуть воды из водоема;
\vs Isa 30:15 ибо так говорит Господь Бог, Святый Израилев: оставаясь на месте и в покое, вы спаслись бы; в тишине и уповании крепость ваша; но вы не хотели
\vs Isa 30:16 и говорили: <<нет, мы на конях убежим>>,~--- за то и побежите; <<мы на быстрых ускачем>>,~--- за то и преследующие вас будут быстры.
\vs Isa 30:17 От угрозы одного \bibemph{побежит} тысяча, от угрозы пятерых побежите так, что остаток ваш будет как веха на вершине горы и как знамя на холме.
\vs Isa 30:18 И потому Господь медлит, чтобы помиловать вас, и потому еще удерживается, чтобы сжалиться над вами; ибо Господь есть Бог правды: блаженны все уповающие на Него!
\vs Isa 30:19 Народ будет жить на Сионе в Иерусалиме; ты не будешь много плакать,~--- Он помилует тебя, по голосу вопля твоего, и как только услышит его, ответит тебе.
\vs Isa 30:20 И даст вам Господь хлеб в горести и воду в нужде; и учители твои уже не будут скрываться, и глаза твои будут видеть учителей твоих;
\vs Isa 30:21 и уши твои будут слышать слово, говорящее позади тебя: <<вот путь, идите по нему>>, если бы вы уклонились направо и если бы вы уклонились налево.
\vs Isa 30:22 Тогда вы будете считать скверною оклад идолов из серебра твоего и оклад истуканов из золота твоего; ты бросишь их, как нечистоту; ты скажешь им: прочь отсюда.
\vs Isa 30:23 И Он даст дождь на семя твое, которым засеешь поле, и хлеб, плод земли, и он будет обилен и сочен; стада твои в тот день будут пастись на обширных пастбищах.
\vs Isa 30:24 И волы и ослы, возделывающие поле, будут есть корм соленый, очищенный лопатою и веялом.
\vs Isa 30:25 И на всякой горе высокой и на всяком холме возвышенном потекут ручьи, потоки вод, в день великого поражения, когда упадут башни.
\vs Isa 30:26 И свет луны будет, как свет солнца, а свет солнца будет светлее всемеро, как свет семи дней, в тот день, когда Господь обвяжет рану народа Своего и исцелит нанесенные ему язвы.
\rsbpar\vs Isa 30:27 Вот, имя Господа идет издали, горит гнев Его, и пламя его сильно, уста Его исполнены негодования, и язык Его, как огонь поедающий,
\vs Isa 30:28 и дыхание Его, как разлившийся поток, который поднимается даже до шеи, чтобы развеять народы до истощания; и будет в челюстях народов узда, направляющая к заблуждению.
\vs Isa 30:29 А у вас будут песни, как в ночь священного праздника, и веселье сердца, как у идущего со свирелью на гору Господню, к твердыне Израилевой.
\vs Isa 30:30 И возгремит Господь величественным гласом Своим и явит тяготеющую мышцу Свою в сильном гневе и в пламени поедающего огня, в буре и в наводнении и в каменном граде.
\vs Isa 30:31 Ибо от гласа Господа содрогнется Ассур, жезлом поражаемый.
\vs Isa 30:32 И всякое движение определенного ему жезла, который Господь направит на него, будет с тимпанами и цитрами, и Он пойдет против него войною опустошительною.
\vs Isa 30:33 Ибо Тофет давно уже устроен; он приготовлен и для царя, глубок и широк; в костре его много огня и дров; дуновение Господа, как поток серы, зажжет его.
\vs Isa 31:1 Горе тем, которые идут в Египет за помощью, надеются на коней и полагаются на колесницы, потому что их много, и на всадников, потому что они весьма сильны, а на Святаго Израилева не взирают и к Господу не прибегают!
\vs Isa 31:2 Но премудр Он; и наведет бедствие, и не отменит слов Своих; восстанет против дома нечестивых и против помощи делающих беззаконие.
\vs Isa 31:3 И Египтяне~--- люди, а не Бог; и кони их~--- плоть, а не дух. И прострет руку Свою Господь, и споткнется защитник, и упадет защищаемый, и все вместе погибнут.
\vs Isa 31:4 Ибо так сказал мне Господь: как лев, как скимен, ревущий над своею добычею, хотя бы множество пастухов кричало на него, от крика их не содрогнется и множеству их не уступит,~--- так Господь Саваоф сойдет сразиться за гору Сион и за холм его.
\vs Isa 31:5 Как птицы~--- птенцов, так Господь Саваоф покроет Иерусалим, защитит и избавит, пощадит и спасет.
\vs Isa 31:6 Обратитесь к Тому, от Которого вы столько отпали, сыны Израиля!
\vs Isa 31:7 В тот день отбросит каждый человек своих серебряных идолов и золотых своих идолов, которых руки ваши сделали вам на грех.
\vs Isa 31:8 И Ассур падет не от человеческого меча, и не человеческий меч потребит его,~--- он избежит от меча, и юноши его будут податью.
\vs Isa 31:9 И от страха пробежит мимо крепости своей; и князья его будут пугаться знамени, говорит Господь, Которого огонь на Сионе и горнило в Иерусалиме.
\vs Isa 32:1 Вот, Царь будет царствовать по правде, и князья будут править по закону;
\vs Isa 32:2 и каждый из них будет как защита от ветра и покров от непогоды, как источники вод в степи, как тень от высокой скалы в земле жаждущей.
\vs Isa 32:3 И очи видящих не будут закрываемы, и уши слышащих будут внимать.
\vs Isa 32:4 И сердце легкомысленных будет уметь рассуждать; и косноязычные будут говорить ясно.
\vs Isa 32:5 Невежду уже не будут называть почтенным, и о коварном не скажут, что он честный.
\vs Isa 32:6 Ибо невежда говорит глупое, и сердце его помышляет о беззаконном, чтобы действовать лицемерно и произносить хулу на Господа, душу голодного лишать хлеба и отнимать питье у жаждущего.
\vs Isa 32:7 У коварного и действования гибельные: он замышляет ковы, чтобы погубить бедного словами лжи, хотя бы бедный был и прав.
\vs Isa 32:8 А честный и мыслит о честном и твердо стоит во всем, что честно.
\rsbpar\vs Isa 32:9 Женщины беспечные! встаньте, послушайте голоса моего; дочери беззаботные! приклоните слух к моим словам.
\vs Isa 32:10 Еще несколько дней сверх года, и ужаснетесь, беспечные! ибо не будет обирания винограда, и время жатвы не настанет.
\vs Isa 32:11 Содрогнитесь, беззаботные! ужаснитесь, беспечные! сбросьте одежды, обнажитесь и препояшьте чресла.
\vs Isa 32:12 Будут бить себя в грудь о прекрасных полях, о виноградной лозе плодовитой.
\vs Isa 32:13 На земле народа моего будут расти терны и волчцы, равно и на всех домах веселья в ликующем городе;
\vs Isa 32:14 ибо чертоги будут оставлены; шумный город будет покинут; Офел и башня навсегда будут служить, вместо пещер, убежищем диких ослов и пасущихся стад,
\vs Isa 32:15 доколе не излиется на нас Дух свыше, и пустыня не сделается садом, а сад не будут считать лесом.
\vs Isa 32:16 Тогда суд водворится в этой пустыне, и правосудие будет пребывать на плодоносном поле.
\vs Isa 32:17 И делом правды будет мир, и плодом правосудия~--- спокойствие и безопасность вовеки.
\vs Isa 32:18 Тогда народ мой будет жить в обители мира и в селениях безопасных, и в покоищах блаженных.
\vs Isa 32:19 И град будет падать на лес, и город спустится в долину.
\vs Isa 32:20 Блаженны вы, сеющие при всех водах и посылающие туда вола и осла.
\vs Isa 33:1 Горе тебе, опустошитель, который не был опустошаем, и грабитель, которого не грабили! Когда кончишь опустошение, будешь опустошен и ты; когда прекратишь грабительства, разграбят и тебя.
\vs Isa 33:2 Господи! помилуй нас; на Тебя уповаем мы; будь нашею мышцею с раннего утра и спасением нашим во время тесное.
\vs Isa 33:3 От грозного гласа \bibemph{Твоего} побегут народы; когда восстанешь, рассеются племена,
\vs Isa 33:4 и будут собирать добычу вашу, как собирает гусеница; бросятся на нее, как бросается саранча.
\vs Isa 33:5 Высок Господь, живущий в вышних; Он наполнит Сион судом и правдою.
\vs Isa 33:6 И настанут безопасные времена твои, изобилие спасения, мудрости и в\acc{е}дения; страх Господень будет сокровищем твоим.
\vs Isa 33:7 Вот, сильные их кричат на улицах; послы для мира горько плачут.
\vs Isa 33:8 Опустели дороги; не стало путешествующих; он нарушил договор, разрушил города,~--- ни во что ставит людей.
\vs Isa 33:9 Земля сетует, сохнет; Ливан постыжен, увял; Сарон похож стал на пустыню, и обнажены от листьев своих Васан и Кармил.
\rsbpar\vs Isa 33:10 Ныне Я восстану, говорит Господь, ныне поднимусь, ныне вознесусь.
\vs Isa 33:11 Вы беременны сеном, разродитесь соломою; дыхание ваше~--- огонь, который пожрет вас.
\vs Isa 33:12 И будут народы, \bibemph{как} горящая известь, \bibemph{как} срубленный терновник, будут сожжены в огне.
\rsbpar\vs Isa 33:13 Слушайте, дальние, что сделаю Я; и вы, ближние, познайте могущество Мое.
\vs Isa 33:14 Устрашились грешники на Сионе; трепет овладел нечестивыми: <<кто из нас может жить при огне пожирающем? кто из нас может жить при вечном пламени?>>~---
\vs Isa 33:15 Тот, кто ходит в правде и говорит истину; кто презирает корысть от притеснения, удерживает руки свои от взяток, затыкает уши свои, чтобы не слышать о кровопролитии, и закрывает глаза свои, чтобы не видеть зла;
\vs Isa 33:16 тот будет обитать на высотах; убежище его~--- неприступные скалы; хлеб будет дан ему; вода у него не иссякнет.
\vs Isa 33:17 Глаза твои увидят Царя в красоте Его, узрят землю отдаленную;
\vs Isa 33:18 сердце твое будет \bibemph{только} вспоминать об ужасах: <<где делавший перепись? где весивший \bibemph{дань}? где осматривающий башни?>>
\vs Isa 33:19 Не увидишь более народа свирепого, народа с глухою, невнятною речью, с языком странным, непонятным.
\vs Isa 33:20 Взгляни на Сион, город праздничных собраний наших; глаза твои увидят Иерусалим, жилище мирное, непоколебимую скинию; столпы ее никогда не исторгнутся, и ни одна вервь ее не порвется.
\vs Isa 33:21 Там у нас великий Господь будет вместо рек, вместо широких каналов; туда не войдет ни одно весельное судно, и не пройдет большой корабль.
\vs Isa 33:22 Ибо Господь~--- судия наш, Господь~--- законодатель наш, Господь~--- царь наш; Он спасет нас.
\vs Isa 33:23 Ослабли веревки твои, не могут удержать мачты и натянуть паруса. Тогда будет большой раздел добычи, так что и хромые пойдут на грабеж.
\vs Isa 33:24 И ни один из жителей не скажет: <<я болен>>; народу, живущему там, будут отпущены согрешения.
\vs Isa 34:1 Приступите, народы, слушайте и внимайте, племена! да слышит земля и всё, что наполняет ее, вселенная и всё рождающееся в ней!
\vs Isa 34:2 Ибо гнев Господа на все народы, и ярость Его на все воинство их. Он предал их заклятию, отдал их на заклание.
\vs Isa 34:3 И убитые их будут разбросаны, и от трупов их поднимется смрад, и горы размокнут от крови их.
\vs Isa 34:4 И истлеет все небесное воинство\fns{Звезды.}; и небеса свернутся, как свиток книжный; и все воинство их падет, как спадает лист с виноградной лозы, и как увядший лист~--- со смоковницы.
\vs Isa 34:5 Ибо упился меч Мой на небесах: вот, для суда нисходит он на Едом и на народ, преданный Мною заклятию.
\vs Isa 34:6 Меч Господа наполнится кровью, утучнеет от тука, от крови агнцев и козлов, от тука с почек овнов: ибо жертва у Господа в Восоре и большое заклание в земле Едома.
\vs Isa 34:7 И буйволы падут с ними и тельцы вместе с волами, и упьется земля их кровью, и прах их утучнеет от тука.
\vs Isa 34:8 Ибо день мщения у Господа, год возмездия за Сион.
\vs Isa 34:9 И превратятся реки его в смолу, и прах его~--- в серу, и будет земля его горящею смолою:
\vs Isa 34:10 не будет гаснуть ни днем, ни ночью; вечно будет восходить дым ее; будет от рода в род оставаться опустелою; во веки веков никто не пройдет по ней;
\vs Isa 34:11 и завладеют ею пеликан и еж; и филин и ворон поселятся в ней; и протянут по ней вервь разорения и отвес уничтожения.
\vs Isa 34:12 Никого не останется там из знатных ее, кого можно было бы призвать на царство, и все князья ее будут ничто.
\vs Isa 34:13 И зарастут дворцы ее колючими растениями, крапивою и репейником~--- твердыни ее; и будет она жилищем шакалов, пристанищем страусов.
\vs Isa 34:14 И звери пустыни будут встречаться с дикими кошками, и лешие будут перекликаться один с другим; там будет отдыхать ночное привидение и находить себе покой.
\vs Isa 34:15 Там угнездится летучий змей, будет класть яйца и выводить детей и собирать их под тень свою; там и коршуны будут собираться один к другому.
\vs Isa 34:16 Отыщите в книге Господней и прочитайте; ни одно из сих не преминет прийти, и одно другим не заменится. Ибо сами уста Его повелели, и сам дух Его соберет их.
\vs Isa 34:17 И Сам Он бросил им жребий, и Его рука разделила им ее мерою; во веки будут они владеть ею, из рода в род будут жить на ней.
\vs Isa 35:1 Возвеселится пустыня и сухая земля, и возрадуется страна необитаемая и расцветет как нарцисс;
\vs Isa 35:2 великолепно будет цвести и радоваться, будет торжествовать и ликовать; слава Ливана дастся ей, великолепие Кармила и Сарона; они увидят славу Господа, величие Бога нашего.
\vs Isa 35:3 Укрепите ослабевшие руки и утвердите колени дрожащие;
\vs Isa 35:4 скажите робким душею: будьте тверды, не бойтесь; вот Бог ваш, придет отмщение, воздаяние Божие; Он придет и спасет вас.
\vs Isa 35:5 Тогда откроются глаза слепых, и уши глухих отверзутся.
\vs Isa 35:6 Тогда хромой вскочит, как олень, и язык немого будет петь; ибо пробьются воды в пустыне, и в степи~--- потоки.
\vs Isa 35:7 И превратится призрак вод в озеро, и жаждущая земля~--- в источники вод; в жилище шакалов, где они покоятся, будет место для тростника и камыша.
\vs Isa 35:8 И будет там большая дорога, и путь по ней назовется путем святым: нечистый не будет ходить по нему; но он будет для них \bibemph{одних}; идущие этим путем, даже и неопытные, не заблудятся.
\vs Isa 35:9 Льва не будет там, и хищный зверь не взойдет на него; его не найдется там, а будут ходить искупленные.
\vs Isa 35:10 И возвратятся избавленные Господом, придут на Сион с радостным восклицанием; и радость вечная будет над головою их; они найдут радость и веселье, а печаль и воздыхание удалятся.
\vs Isa 36:1 И было в четырнадцатый год царя Езекии, пошел Сеннахирим, царь Ассирийский, против всех укрепленных городов Иудеи и взял их.
\vs Isa 36:2 И послал царь Ассирийский из Лахиса в Иерусалим к царю Езекии Рабсака с большим войском; и он остановился у водопровода верхнего пруда на дороге поля белильничьего.
\vs Isa 36:3 И вышел к нему Елиаким, сын Хелкиин, начальник дворца, и Севна писец, и Иоах, сын Асафов, дееписатель.
\vs Isa 36:4 И сказал им Рабсак: скажите Езекии: так говорит царь великий, царь Ассирийский: что это за упование, на которое ты уповаешь?
\vs Isa 36:5 Я думаю, \bibemph{что} это одни пустые слова, \bibemph{а} для войны нужны совет и сила: итак на кого ты уповаешь, что отложился от меня?
\vs Isa 36:6 Вот, ты думаешь опереться на Египет, на эту трость надломленную, которая, если кто опрется на нее, войдет тому в руку и проколет ее! Таков фараон, царь Египетский, для всех уповающих на него.
\vs Isa 36:7 А если скажешь мне: <<на Господа, Бога нашего мы уповаем>>, то на Того ли, Которого высоты и жертвенники отменил Езекия и сказал Иуде и Иерусалиму: <<пред сим только жертвенником поклоняйтесь>>?
\vs Isa 36:8 Итак вступи в союз с господином моим, царем Ассирийским; я дам тебе две тысячи коней; можешь ли достать себе всадников на них?
\vs Isa 36:9 И как ты хочешь заставить отступить вождя, одного из малейших рабов господина моего, надеясь на Египет, ради колесниц и коней?
\vs Isa 36:10 Да разве я без воли Господней пошел на землю сию, чтобы разорить ее? Господь сказал мне: пойди на землю сию и разори ее.
\rsbpar\vs Isa 36:11 И сказал Елиаким и Севна и Иоах Рабсаку: говори рабам твоим по-арамейски, потому что мы понимаем, а не говори с нами по-иудейски, вслух народа, который на стене.
\vs Isa 36:12 И сказал Рабсак: разве \bibemph{только} к господину твоему и к тебе послал меня господин мой сказать слова сии? Нет, \bibemph{также} и к людям, которые сидят на стене, чтобы есть помет свой и пить мочу свою с вами.
\vs Isa 36:13 И встал Рабсак, и возгласил громким голосом по-иудейски, и сказал: слушайте слово царя великого, царя Ассирийского!
\vs Isa 36:14 Так говорит царь: пусть не обольщает вас Езекия, ибо он не может спасти вас;
\vs Isa 36:15 и пусть не обнадеживает вас Езекия Господом, говоря: <<спасет нас Господь; не будет город сей отдан в руки царя Ассирийского>>.
\vs Isa 36:16 Не слушайте Езекии, ибо так говорит царь Ассирийский: примиритесь со мною и выйдите ко мне, и пусть каждый ест плоды виноградной лозы своей и смоковницы своей, и пусть каждый пьет воду из своего колодезя,
\vs Isa 36:17 доколе я не приду и не возьму вас в землю такую же, как и ваша земля, в землю хлеба и вина, в землю плодов и виноградников.
\vs Isa 36:18 \bibemph{Итак} да не обольщает вас Езекия, говоря: <<Господь спасет нас>>. Спасли ли боги народов, каждый свою землю, от руки царя Ассирийского?
\vs Isa 36:19 Где боги Емафа и Арпада? Где боги Сепарваима? Спасли ли они Самарию от руки моей?
\vs Isa 36:20 Который из всех богов земель сих спас землю свою от руки моей? Так неужели спасет Господь Иерусалим от руки моей?
\vs Isa 36:21 Но они молчали и не отвечали ему ни слова, потому что от царя дано было приказание: не отвечайте ему.
\vs Isa 36:22 И пришел Елиаким, сын Хелкиин, начальник дворца, и Севна писец, и Иоах, сын Асафов, дееписатель, к Езекии в разодранных одеждах и пересказали ему слова Рабсака.
\vs Isa 37:1 Когда услышал это царь Езекия, то разодрал одежды свои и покрылся вретищем, и пошел в дом Господень;
\vs Isa 37:2 и послал Елиакима, начальника дворца, и Севну писца, и старших священников, покрытых вретищами, к пророку Исаии, сыну Амосову.
\vs Isa 37:3 И они сказали ему: так говорит Езекия: день скорби и наказания и посрамления день сей, ибо младенцы дошли до отверстия утробы матерней, а силы нет родить.
\vs Isa 37:4 Может быть, услышит Господь Бог твой слова Рабсака, которого послал царь Ассирийский, господин его, хулить Бога живаго и поносить словами, какие слышал Господь, Бог твой; вознеси же молитву об оставшихся, которые находятся еще в живых.
\rsbpar\vs Isa 37:5 И пришли слуги царя Езекии к Исаии.
\vs Isa 37:6 И сказал им Исаия: так скажите господину вашему: так говорит Господь: не бойся слов, которые слышал ты, которыми поносили Меня слуги царя Ассирийского.
\vs Isa 37:7 Вот, Я пошлю в него дух, и он услышит весть, и возвратится в землю свою, и Я поражу его мечом в земле его.
\rsbpar\vs Isa 37:8 И возвратился Рабсак и нашел царя Ассирийского воюющим против Ливны; ибо он слышал, что тот отошел от Лахиса.
\vs Isa 37:9 И услышал он о Тиргаке, царе Ефиопском; \bibemph{ему} сказали: вот, он вышел сразиться с тобою. Услышав это, он послал послов к Езекии, сказав:
\vs Isa 37:10 так скажите Езекии, царю Иудейскому: пусть не обманывает тебя Бог твой, на Которого ты уповаешь, думая: <<не будет отдан Иерусалим в руки царя Ассирийского>>.
\vs Isa 37:11 Вот, ты слышал, что сделали цари Ассирийские со всеми землями, положив на них заклятие; ты ли уцелеешь?
\vs Isa 37:12 Боги народов, которых разорили отцы мои, спасли ли их, \bibemph{спасли ли} Гозан и Харан, и Рецеф, и сынов Едена, что в Фалассаре?
\vs Isa 37:13 Где царь Емафа и царь Арпада, и царь города Сепарваима, Ены и Иввы?
\rsbpar\vs Isa 37:14 И взял Езекия письмо из руки послов и прочитал его, и пошел в дом Господень, и развернул его Езекия пред лицем Господним;
\vs Isa 37:15 и молился Езекия пред лицем Господним и говорил:
\vs Isa 37:16 Господи Саваоф, Боже Израилев, сидящий на Херувимах! Ты один Бог всех царств земли; Ты сотворил небо и землю.
\vs Isa 37:17 Приклони, Господи, ухо Твое и услышь; открой, Господи, очи Твои и воззри, и услышь слова Сеннахирима, который послал поносить Тебя, Бога живаго.
\vs Isa 37:18 Правда, о, Господи! цари Ассирийские опустошили все страны и земли их
\vs Isa 37:19 и побросали богов их в огонь; но это были не боги, а изделие рук человеческих, дерево и камень, потому и истребили их.
\vs Isa 37:20 И ныне, Господи Боже наш, спаси нас от руки его; и узнают все царства земли, что Ты, Господи, Бог один.
\rsbpar\vs Isa 37:21 И послал Исаия, сын Амосов, к Езекии сказать: так говорит Господь, Бог Израилев: о чем ты молился Мне против Сеннахирима, царя Ассирийского,~---
\vs Isa 37:22 вот слово, которое Господь изрек о нем: презрит тебя, посмеется над тобою девствующая дочь Сиона, покачает вслед тебя головою дочь Иерусалима.
\vs Isa 37:23 Кого ты порицал и поносил? и на кого возвысил голос и поднял так высоко глаза твои? на Святаго Израилева.
\vs Isa 37:24 Чрез рабов твоих ты порицал Господа и сказал: <<со множеством колесниц моих я взошел на высоту гор, на ребра Ливана, и срубил рослые кедры его, отличные кипарисы его, и пришел на самую вершину его, в рощу сада его;
\vs Isa 37:25 и откапывал я, и пил воду; и осушу ступнями ног моих все реки Египетские>>.
\vs Isa 37:26 Разве не слышал ты, что Я издавна сделал это, в древние дни предначертал это, а ныне выполнил тем, что ты опустошаешь крепкие города, \bibemph{превращая} их в груды развалин?
\vs Isa 37:27 И жители их сделались маломощны, трепещут и остаются в стыде; они стали как трава на поле и нежная зелень, как порост на кровлях и опаленный хлеб, прежде нежели выколосился.
\vs Isa 37:28 Сядешь ли ты, выйдешь ли, войдешь ли, Я знаю \bibemph{всё, знаю} и дерзость твою против Меня.
\vs Isa 37:29 За твою дерзость против Меня и за то, что надмение твое дошло до ушей Моих, Я вложу кольцо Мое в ноздри твои и удила Мои в рот твой, и возвращу тебя назад тою же дорогою, которою ты пришел.
\vs Isa 37:30 И вот, тебе, Езекия, знамение: ешьте в этот год выросшее от упавшего зерна, и на другой год~--- самородное; а на третий год сейте и жните, и садите виноградные сады, и ешьте плоды их.
\vs Isa 37:31 И уцелевший в доме Иудином остаток пустит опять корень внизу и принесет плод вверху,
\vs Isa 37:32 ибо из Иерусалима произойдет остаток, и спасенное~--- от горы Сиона. Ревность Господа Саваофа соделает это.
\vs Isa 37:33 Посему так говорит Господь о царе Ассирийском: <<не войдет он в этот город и не бросит туда стрел\acc{ы}, и не приступит к нему со щитом, и не насыплет против него вала.
\vs Isa 37:34 По той же дороге, по которой пришел, возвратится, а в город сей не войдет, говорит Господь.
\vs Isa 37:35 Я буду охранять город сей, чтобы спасти его ради Себя и ради Давида, раба Моего>>.
\vs Isa 37:36 И вышел Ангел Господень и поразил в стане Ассирийском сто восемьдесят пять тысяч \bibemph{человек}. И встали поутру, и вот, всё тела мертвые.
\vs Isa 37:37 И отступил, и пошел, и возвратился Сеннахирим, царь Ассирийский, и жил в Ниневии.
\vs Isa 37:38 И когда он поклонялся в доме Нисроха, бога своего, Адрамелех и Шарецер, сыновья его, убили его мечом, а сами убежали в землю Араратскую. И воцарился Асардан, сын его, вместо него.
\vs Isa 38:1 В те дни Езекия заболел смертельно. И пришел к нему пророк Исаия, сын Амосов, и сказал ему: так говорит Господь: сделай завещание для дома твоего, ибо ты умрешь, не выздоровеешь.
\vs Isa 38:2 Тогда Езекия отворотился лицем к стене и молился Господу, говоря:
\vs Isa 38:3 <<о, Господи! вспомни, что я ходил пред лицем Твоим верно и с преданным \bibemph{Тебе} сердцем и делал угодное в очах Твоих>>. И заплакал Езекия сильно.
\rsbpar\vs Isa 38:4 И было слово Господне к Исаии, и сказано:
\vs Isa 38:5 пойди и скажи Езекии: так говорит Господь, Бог Давида, отца твоего: Я услышал молитву твою, увидел слезы твои, и вот, Я прибавлю к дням твоим пятнадцать лет,
\vs Isa 38:6 и от руки царя Ассирийского спасу тебя и город сей и защищу город сей.
\vs Isa 38:7 И вот тебе знамение от Господа, что Господь исполнит слово, которое Он изрек.
\vs Isa 38:8 Вот, я возвращу назад на десять ступеней солнечную тень, которая прошла по ступеням Ахазовым. И возвратилось солнце на десять ступеней по ступеням, по которым оно сходило.
\rsbpar\vs Isa 38:9 Молитва Езекии, царя Иудейского, когда он болен был и выздоровел от болезни:
\vs Isa 38:10 <<Я сказал в себе: в преполовение дней моих должен я идти во врата преисподней; я лишен остатка лет моих.
\vs Isa 38:11 Я говорил: не увижу я Господа, Господа на земле живых; не увижу больше человека между живущими в мире;
\vs Isa 38:12 жилище мое снимается с места и уносится от меня, как шалаш пастушеский; я должен отрезать подобно ткачу жизнь мою; Он отрежет меня от основы; день и ночь я ждал, что Ты пошлешь мне кончину.
\vs Isa 38:13 Я ждал до утра; подобно льву, Он сокрушал все кости мои; день и ночь я ждал, что Ты пошлешь мне кончину.
\vs Isa 38:14 Как журавль, как ласточка издавал я звуки, тосковал как голубь; уныло смотрели глаза мои к небу: Господи! тесно мне; спаси меня.
\vs Isa 38:15 Что скажу я? Он сказал мне, Он и сделал. Тихо буду проводить все годы жизни моей, помня горесть души моей.
\vs Isa 38:16 Господи! так живут, и во всем этом жизнь моего духа; Ты исцелишь меня, даруешь мне жизнь.
\vs Isa 38:17 Вот, во благо мне была сильная горесть, и Ты избавил душу мою от рва погибели, бросил все грехи мои за хребет Свой.
\vs Isa 38:18 Ибо не преисподняя славит Тебя, не смерть восхваляет Тебя, не нисшедшие в могилу уповают на истину Твою.
\vs Isa 38:19 Живой, только живой прославит Тебя, как я ныне: отец возвестит детям истину Твою.
\vs Isa 38:20 Господь спасет меня; и мы во все дни жизни нашей \bibemph{со звуками} струн моих будем воспевать песни в доме Господнем>>.
\vs Isa 38:21 И сказал Исаия: пусть принесут пласт смокв и обложат им нарыв; и он выздоровеет.
\vs Isa 38:22 А Езекия сказал: какое знамение, что я буду ходить в дом Господень?
\vs Isa 39:1 В то время Меродах Валадан, сын Валадана, царь Вавилонский, прислал к Езекии письмо и дары, ибо слышал, что он был болен и выздоровел.
\vs Isa 39:2 И обрадовался посланным Езекия, и показал им дом сокровищ своих, серебро и золото, и ароматы, и драгоценные масти, весь оружейный свой дом и все, что находилось в сокровищницах его; ничего не осталось, чего не показал бы им Езекия в доме своем и во всем владении своем.
\rsbpar\vs Isa 39:3 И пришел пророк Исаия к царю Езекии и сказал ему: что говорили эти люди? и откуда они приходили к тебе? Езекия сказал: из далекой земли приходили они ко мне, из Вавилона.
\vs Isa 39:4 И сказал \bibemph{Исаия}: что видели они в доме твоем? Езекия сказал: видели всё, что есть в доме моем; ничего не осталось в сокровищницах моих, чего я не показал бы им.
\vs Isa 39:5 И сказал Исаия Езекии: выслушай слово Господа Саваофа:
\vs Isa 39:6 вот, придут дни, и всё, что есть в доме твоем и что собрали отцы твои до сего дня, будет унесено в Вавилон; ничего не останется, говорит Господь.
\vs Isa 39:7 И возьмут из сыновей твоих, которые произойдут от тебя, которых ты родишь,~--- и они будут евнухами во дворце царя Вавилонского.
\vs Isa 39:8 И сказал Езекия Исаии: благо слово Господне, которое ты изрек; потому что, присовокупил он, мир и благосостояние пребудут во дни мои.
\vs Isa 40:1 Утешайте, утешайте народ Мой, говорит Бог ваш;
\vs Isa 40:2 говорите к сердцу Иерусалима и возвещайте ему, что исполнилось время борьбы его, что за неправды его сделано удовлетворение, ибо он от руки Господней принял вдвое за все грехи свои.
\rsbpar\vs Isa 40:3 Глас вопиющего в пустыне: приготовьте путь Господу, прямыми сделайте в степи стези Богу нашему;
\vs Isa 40:4 всякий дол да наполнится, и всякая гора и холм да понизятся, кривизны выпрямятся и неровные пути сделаются гладкими;
\vs Isa 40:5 и явится слава Господня, и узрит всякая плоть [спасение Божие]; ибо уста Господни изрекли это.
\vs Isa 40:6 Голос говорит: возвещай! И сказал: что мне возвещать? Всякая плоть~--- трава, и вся красота ее~--- как цвет полевой.
\vs Isa 40:7 Засыхает трава, увядает цвет, когда дунет на него дуновение Господа: так и народ~--- трава.
\vs Isa 40:8 Трава засыхает, цвет увядает, а слово Бога нашего пребудет вечно.
\vs Isa 40:9 Взойди на высокую гору, благовествующий Сион! возвысь с силою голос твой, благовествующий Иерусалим! возвысь, не бойся; скажи городам Иудиным: вот Бог ваш!
\vs Isa 40:10 Вот, Господь Бог грядет с силою, и мышца Его со властью. Вот, награда Его с Ним и воздаяние Его пред лицем Его.
\vs Isa 40:11 Как пастырь Он будет пасти стадо Свое; агнцев будет брать на руки и носить на груди Своей, и водить дойных.
\vs Isa 40:12 Кто исчерпал воды горстью своею и пядью измерил небеса, и вместил в меру прах земли, и взвесил на весах горы и на чашах весовых холмы?
\vs Isa 40:13 Кто уразумел дух Господа, и был советником у Него и учил Его?
\vs Isa 40:14 С кем советуется Он, и кто вразумляет Его и наставляет Его на путь правды, и учит Его знанию, и указывает Ему путь мудрости?
\vs Isa 40:15 Вот народы~--- как капля из ведра, и считаются как пылинка на весах. Вот, острова как порошинку поднимает Он.
\vs Isa 40:16 И Ливана недостаточно для жертвенного огня, и животных на нем~--- для всесожжения.
\vs Isa 40:17 Все народы пред Ним как ничто,~--- менее ничтожества и пустоты считаются у Него.
\vs Isa 40:18 Итак кому уподобите вы Бога? И какое подобие найдете Ему?
\vs Isa 40:19 Идола выливает художник, и золотильщик покрывает его золотом и приделывает серебряные цепочки.
\vs Isa 40:20 А кто беден для такого приношения, выбирает негниющее дерево, приискивает себе искусного художника, чтобы сделать идола, который стоял бы твердо.
\vs Isa 40:21 Разве не знаете? разве вы не слышали? разве вам не говорено было от начала? разве вы не уразумели из оснований земли?
\vs Isa 40:22 Он есть Тот, Который восседает над кругом земли, и живущие на ней~--- как саранча \bibemph{пред Ним}; Он распростер небеса, как тонкую ткань, и раскинул их, как шатер для жилья.
\vs Isa 40:23 Он обращает князей в ничто, делает чем-то пустым судей земли.
\vs Isa 40:24 Едва они посажены, едва посеяны, едва укоренился в земле ствол их, и как только Он дохнул на них, они высохли, и вихрь унес их, как солому.
\vs Isa 40:25 Кому же вы уподобите Меня и с кем сравните? говорит Святый.
\vs Isa 40:26 Поднимите глаза ваши на высоту \bibemph{небес} и посмотрите, кто сотворил их? Кто выводит воинство их счетом? Он всех их называет по имени: по множеству могущества и великой силе у Него ничто не выбывает.
\vs Isa 40:27 Как же говоришь ты, Иаков, и высказываешь, Израиль: <<путь мой сокрыт от Господа, и дело мое забыто у Бога моего>>?
\vs Isa 40:28 Разве ты не знаешь? разве ты не слышал, что вечный Господь Бог, сотворивший концы земли, не утомляется и не изнемогает? разум Его неисследим.
\vs Isa 40:29 Он дает утомленному силу, и изнемогшему дарует крепость.
\vs Isa 40:30 Утомляются и юноши и ослабевают, и молодые люди падают,
\vs Isa 40:31 а надеющиеся на Господа обновятся в силе: поднимут крылья, как орлы, потекут~--- и не устанут, пойдут~--- и не утомятся.
\vs Isa 41:1 Умолкните предо Мною, острова, и народы да обновят свои силы; пусть они приблизятся и скажут: <<станем вместе на суд>>.
\vs Isa 41:2 Кто воздвиг от востока мужа правды, призвал его следовать за собою, предал ему народы и покорил царей? Он обратил их мечом его в прах, луком его в солому, разносимую ветром.
\vs Isa 41:3 Он гонит их, идет спокойно дорогою, по которой никогда не ходил ногами своими.
\vs Isa 41:4 Кто сделал и совершил это? Тот, Кто от начала вызывает роды; Я~--- Господь первый, и в последних~--- Я Тот же.
\vs Isa 41:5 Увидели острова и ужаснулись, концы земли затрепетали. Они сблизились и сошлись;
\vs Isa 41:6 каждый помогает своему товарищу и говорит своему брату: <<крепись!>>
\vs Isa 41:7 Кузнец ободряет плавильщика, разглаживающий листы молотом~--- кующего на наковальне, говоря о спайке: <<хороша>>; и укрепляет гвоздями, чтобы было твердо.
\vs Isa 41:8 А ты, Израиль, раб Мой, Иаков, которого Я избрал, семя Авраама, друга Моего,~---
\vs Isa 41:9 ты, которого Я взял от концов земли и призвал от краев ее, и сказал тебе: <<ты Мой раб, Я избрал тебя и не отвергну тебя>>:
\vs Isa 41:10 не бойся, ибо Я с тобою; не смущайся, ибо Я Бог твой; Я укреплю тебя, и помогу тебе, и поддержу тебя десницею правды Моей.
\vs Isa 41:11 Вот, в стыде и посрамлении останутся все, раздраженные против тебя; будут как ничто и погибнут препирающиеся с тобою.
\vs Isa 41:12 Будешь искать их, и не найдешь их, враждующих против тебя; борющиеся с тобою будут как ничто, совершенно ничто;
\vs Isa 41:13 ибо Я Господь, Бог твой; держу тебя за правую руку твою, говорю тебе: <<не бойся, Я помогаю тебе>>.
\vs Isa 41:14 Не бойся, червь Иаков, малолюдный Израиль,~--- Я помогаю тебе, говорит Господь и Искупитель твой, Святый Израилев.
\vs Isa 41:15 Вот, Я сделал тебя острым молотилом, новым, зубчатым; ты будешь молотить и растирать горы, и холмы сделаешь, как мякину.
\vs Isa 41:16 Ты будешь веять их, и ветер разнесет их, и вихрь развеет их; а ты возрадуешься о Господе, будешь хвалиться Святым Израилевым.
\vs Isa 41:17 Бедные и нищие ищут воды, и нет \bibemph{ее}; язык их сохнет от жажды: Я, Господь, услышу их, Я, Бог Израилев, не оставлю их.
\vs Isa 41:18 Открою на горах реки и среди долин источники; пустыню сделаю озером и сухую землю~--- источниками воды;
\vs Isa 41:19 посажу в пустыне кедр, ситтим и мирту и маслину; насажу в степи кипарис, явор и бук вместе,
\vs Isa 41:20 чтобы увидели и познали, и рассмотрели и уразумели, что рука Господня соделала это, и Святый Израилев сотворил сие.
\rsbpar\vs Isa 41:21 Представьте дело ваше, говорит Господь; приведите ваши доказательства, говорит Царь Иакова.
\vs Isa 41:22 Пусть они представят и скажут нам, что произойдет; пусть возвестят что-либо прежде, нежели оно произошло, и мы вникнем умом своим и узнаем, как оно кончилось, или пусть предвозвестят нам о будущем.
\vs Isa 41:23 Скажите, что произойдет в будущем, и мы будем знать, что вы боги, или сделайте что-нибудь, доброе ли, худое ли, чтобы мы изумились и вместе с вами увидели.
\vs Isa 41:24 Но вы ничто, и дело ваше ничтожно; мерзость тот, кто избирает вас.
\vs Isa 41:25 Я воздвиг его от севера, и он придет; от восхода солнца будет призывать имя Мое и попирать владык, как грязь, и топтать, как горшечник глину.
\vs Isa 41:26 Кто возвестил об этом изначала, чтобы нам знать, и задолго пред тем, чтобы нам можно было сказать: <<правда>>? Но никто не сказал, никто не возвестил, никто не слыхал слов ваших.
\vs Isa 41:27 Я первый \bibemph{сказал} Сиону: <<вот оно!>> и дал Иерусалиму благовестника.
\vs Isa 41:28 Итак Я смотрел, и не было никого, и между ними не нашлось советника, чтоб Я мог спросить их, и они дали ответ.
\vs Isa 41:29 Вот, все они ничто, ничтожны и дела их; ветер и пустота истуканы их.
\vs Isa 42:1 Вот, Отрок Мой, Которого Я держу за руку, избранный Мой, к Которому благоволит душа Моя. Положу дух Мой на Него, и возвестит народам суд;
\vs Isa 42:2 не возопиет и не возвысит голоса Своего, и не даст услышать его на улицах;
\vs Isa 42:3 трости надломленной не переломит, и льна курящегося не угасит; будет производить суд по истине;
\vs Isa 42:4 не ослабеет и не изнеможет, доколе на земле не утвердит суда, и на закон Его будут уповать острова\fns{По переводу 70-ти: на имя Его будут уповать народы.}.
\rsbpar\vs Isa 42:5 Так говорит Господь Бог, сотворивший небеса и пространство их, распростерший землю с произведениями ее, дающий дыхание народу на ней и дух ходящим по ней.
\vs Isa 42:6 Я, Господь, призвал Тебя в правду, и буду держать Тебя за руку и хранить Тебя, и поставлю Тебя в завет для народа, во свет для язычников,
\vs Isa 42:7 чтобы открыть глаза слепых, чтобы узников вывести из заключения и сидящих во тьме~--- из темницы.
\vs Isa 42:8 Я Господь, это~--- Мое имя, и не дам славы Моей иному и хвалы Моей истуканам.
\vs Isa 42:9 Вот, \bibemph{предсказанное} прежде сбылось, и новое Я возвещу; прежде нежели оно произойдет, Я возвещу вам.
\vs Isa 42:10 Пойте Господу новую песнь, хвалу Ему от концов земли, вы, плавающие по морю, и всё, наполняющее его, острова и живущие на них.
\vs Isa 42:11 Да возвысит голос пустыня и города ее, селения, где обитает Кидар; да торжествуют живущие на скалах, да возглашают с вершин гор.
\vs Isa 42:12 Да воздадут Господу славу, и хвалу Его да возвестят на островах.
\vs Isa 42:13 Господь выйдет, как исполин, как муж браней возбудит ревность; воззовет и поднимет воинский крик, и покажет Себя сильным против врагов Своих.
\vs Isa 42:14 Долго молчал Я, терпел, удерживался; теперь буду кричать, как рождающая, буду разрушать и поглощать всё;
\vs Isa 42:15 опустошу горы и холмы, и всю траву их иссушу; и реки сделаю островами, и осушу озера;
\vs Isa 42:16 и поведу слепых дорогою, которой они не знают, неизвестными путями буду вести их; мрак сделаю светом пред ними, и кривые пути~--- прямыми: вот что Я сделаю для них и не оставлю их.
\vs Isa 42:17 Тогда обратятся вспять и великим стыдом покроются надеющиеся на идолов, говорящие истуканам: <<вы наши боги>>.
\rsbpar\vs Isa 42:18 Слушайте, глухие, и смотрите, слепые, чтобы видеть.
\vs Isa 42:19 Кто так слеп, как раб Мой, и глух, как вестник Мой, Мною посланный? Кто так слеп, как возлюбленный, так слеп, как раб Господа?
\vs Isa 42:20 Ты видел многое, но не замечал; уши были открыты, но не слышал.
\vs Isa 42:21 Господу угодно было, ради правды Своей, возвеличить и прославить закон.
\vs Isa 42:22 Но это народ разоренный и разграбленный; все они связаны в подземельях и сокрыты в темницах; сделались добычею, и нет избавителя; ограблены, и никто не говорит: <<отдай назад!>>
\vs Isa 42:23 Кто из вас приклонил к этому ухо, вникнул и выслушал это для будущего?
\vs Isa 42:24 Кто предал Иакова на разорение и Израиля грабителям? не Господь ли, против Которого мы грешили? Не хотели они ходить путями Его и не слушали закона Его.
\vs Isa 42:25 И Он излил на них ярость гнева Своего и лютость войны: она окружила их пламенем со всех сторон, но они не примечали; и горела у них, но они не уразумели этого сердцем.
\vs Isa 43:1 Ныне же так говорит Господь, сотворивший тебя, Иаков, и устроивший тебя, Израиль: не бойся, ибо Я искупил тебя, назвал тебя по имени твоему; ты Мой.
\vs Isa 43:2 Будешь ли переходить через воды, Я с тобою,~--- через реки ли, они не потопят тебя; пойдешь ли через огонь, не обожжешься, и пламя не опалит тебя.
\vs Isa 43:3 Ибо Я Господь, Бог твой, Святый Израилев, Спаситель твой; в выкуп за тебя отдал Египет, Ефиопию и Савею за тебя.
\vs Isa 43:4 Так как ты дорог в очах Моих, многоценен, и Я возлюбил тебя, то отдам \bibemph{других} людей за тебя, и народы за душу твою.
\vs Isa 43:5 Не бойся, ибо Я с тобою; от востока приведу племя твое и от запада соберу тебя.
\vs Isa 43:6 Северу скажу: <<отдай>>; и югу: <<не удерживай; веди сыновей Моих издалека и дочерей Моих от концов земли,
\vs Isa 43:7 каждого кто называется Моим именем, кого Я сотворил для славы Моей, образовал и устроил.
\vs Isa 43:8 Выведи народ слепой, хотя у него есть глаза, и глухой, хотя у него есть уши>>.
\vs Isa 43:9 Пусть все народы соберутся вместе, и совокупятся племена. Кто между ними предсказал это? пусть возвестят, что было от начала; пусть представят свидетелей от себя и оправдаются, чтобы можно было услышать и сказать: <<правда!>>
\vs Isa 43:10 А Мои свидетели, говорит Господь, вы и раб Мой, которого Я избрал, чтобы вы знали и верили Мне, и разумели, что это Я: прежде Меня не было Бога и после Меня не будет.
\rsbpar\vs Isa 43:11 Я, Я Господь, и нет Спасителя кроме Меня.
\vs Isa 43:12 Я предрек и спас, и возвестил; а иного нет у вас, и вы~--- свидетели Мои, говорит Господь, что Я Бог;
\vs Isa 43:13 от \bibemph{начала} дней Я Тот же, и никто не спасет от руки Моей; Я сделаю, и кто отменит это?
\rsbpar\vs Isa 43:14 Так говорит Господь, Искупитель ваш, Святый Израилев: ради вас Я послал в Вавилон и сокрушил все запоры и Халдеев, величавшихся кораблями.
\vs Isa 43:15 Я Господь, Святый ваш, Творец Израиля, Царь ваш.
\rsbpar\vs Isa 43:16 Так говорит Господь, открывший в море дорогу, в сильных водах стезю,
\vs Isa 43:17 выведший колесницы и коней, войско и силу; все легли вместе, не встали; потухли как светильня, погасли.
\vs Isa 43:18 Но вы не вспоминаете прежнего и о древнем не помышляете.
\vs Isa 43:19 Вот, Я делаю новое; ныне же оно явится; неужели вы и этого не хотите знать? Я проложу дорогу в степи, реки в пустыне.
\vs Isa 43:20 Полевые звери прославят Меня, шакалы и страусы, потому что Я в пустынях дам воду, реки в сухой степи, чтобы поить избранный народ Мой.
\vs Isa 43:21 Этот народ Я образовал для Себя; он будет возвещать славу Мою.
\vs Isa 43:22 А ты, Иаков, не взывал ко Мне; ты, Израиль, не трудился для Меня.
\vs Isa 43:23 Ты не приносил Мне агнцев твоих во всесожжение и жертвами твоими не чтил Меня. Я не заставлял тебя служить Мне хлебным приношением и не отягощал тебя фимиамом.
\vs Isa 43:24 Ты не покупал Мне благовонной трости за серебро и туком жертв твоих не насыщал Меня; но ты грехами твоими затруднял Меня, беззакониями твоими отягощал Меня.
\vs Isa 43:25 Я, Я Сам изглаживаю преступления твои ради Себя Самого и грехов твоих не помяну:
\vs Isa 43:26 припомни Мне; станем судиться; говори ты, чтоб оправдаться.
\vs Isa 43:27 Праотец твой согрешил, и ходатаи твои отступили от Меня.
\vs Isa 43:28 За то Я предстоятелей святилища лишил священства и Иакова предал на заклятие и Израиля на поругание.
\vs Isa 44:1 А ныне слушай, Иаков, раб Мой, и Израиль, которого Я избрал.
\vs Isa 44:2 Так говорит Господь, создавший тебя и образовавший тебя, помогающий тебе от утробы матерней: не бойся, раб Мой, Иаков, и возлюбленный [Израиль], которого Я избрал;
\vs Isa 44:3 ибо Я изолью воды на жаждущее и потоки на иссохшее; излию дух Мой на племя твое и благословение Мое на потомков твоих.
\vs Isa 44:4 И будут расти между травою, как ивы при потоках вод.
\vs Isa 44:5 Один скажет: <<я Господень>>, другой назовется именем Иакова; а иной напишет рукою своею: <<я Господень>>, и прозовется именем Израиля.
\rsbpar\vs Isa 44:6 Так говорит Господь, Царь Израиля, и Искупитель его, Господь Саваоф: Я первый и Я последний, и кроме Меня нет Бога,
\vs Isa 44:7 ибо кто как Я? Пусть он расскажет, возвестит и в порядке представит Мне всё с того времени, как Я устроил народ древний, или пусть возвестят наступающее и будущее.
\vs Isa 44:8 Не бойтесь и не страшитесь: не издавна ли Я возвестил тебе и предсказал? И вы Мои свидетели. Есть ли Бог кроме Меня? нет другой твердыни, никакой не знаю.
\vs Isa 44:9 Делающие идолов все ничтожны, и вожделеннейшие их не приносят никакой пользы, и они сами себе свидетели в том. Они не видят и не разумеют, и потому будут посрамлены.
\vs Isa 44:10 Кто сделал бога и вылил идола, не приносящего никакой пользы?
\vs Isa 44:11 Все участвующие в этом будут постыжены, ибо и художники сами из людей же; пусть все они соберутся и станут; они устрашатся, и все будут постыжены.
\vs Isa 44:12 Кузнец делает из железа топор и работает на угольях, молотами обделывает его и трудится над ним сильною рукою своею до того, что становится голоден и бессилен, не пьет воды и изнемогает.
\vs Isa 44:13 Плотник [выбрав дерево], протягивает по нему линию, остроконечным орудием делает на нем очертание, потом обделывает его резцом и округляет его, и выделывает из него образ человека красивого вида, чтобы поставить его в доме.
\vs Isa 44:14 Он рубит себе кедры, берет сосну и дуб, которые выберет между деревьями в лесу, садит ясень, а дождь возращает его.
\vs Isa 44:15 И это служит человеку топливом, и \bibemph{часть} из этого употребляет он на то, чтобы ему было тепло, и разводит огонь, и печет хлеб. И из того же делает бога, и поклоняется ему, делает идола, и повергается перед ним.
\vs Isa 44:16 Часть дерева сожигает в огне, другою частью варит мясо в пищу, жарит жаркое и ест досыта, а также греется и говорит: <<хорошо, я согрелся; почувствовал огонь>>.
\vs Isa 44:17 А из остатков от того делает бога, идола своего, поклоняется ему, повергается перед ним и молится ему, и говорит: <<спаси меня, ибо ты бог мой>>.
\vs Isa 44:18 Не знают и не разумеют они: Он закрыл глаза их, чтобы не видели, \bibemph{и} сердца их, чтобы не разумели.
\vs Isa 44:19 И не возьмет он этого к своему сердцу, и нет у него столько знания и смысла, чтобы сказать: <<половину его я сжег в огне и на угольях его испек хлеб, изжарил мясо и съел; а из остатка его сделаю ли я мерзость? буду ли поклоняться куску дерева?>>
\vs Isa 44:20 Он гоняется за пылью; обманутое сердце ввело его в заблуждение, и он не может освободить души своей и сказать: <<не обман ли в правой руке моей?>>
\vs Isa 44:21 Помни это, Иаков и Израиль, ибо ты раб Мой; Я образовал тебя: раб Мой ты, Израиль, не забывай Меня.
\vs Isa 44:22 Изглажу беззакония твои, как туман, и грехи твои, как облако; обратись ко Мне, ибо Я искупил тебя.
\vs Isa 44:23 Торжествуйте, небеса, ибо Господь соделал это. Восклицайте, глубины земли; шумите от радости, горы, лес и все деревья в нем; ибо искупил Господь Иакова и прославится в Израиле.
\rsbpar\vs Isa 44:24 Так говорит Господь, искупивший тебя и образовавший тебя от утробы матерней: Я Господь, Который сотворил все, один распростер небеса и Своею силою разостлал землю,
\vs Isa 44:25 Который делает ничтожными знамения лжепророков и обнаруживает безумие волшебников, мудрецов прогоняет назад и знание их делает глупостью,
\vs Isa 44:26 Который утверждает слово раба Своего и приводит в исполнение изречение Своих посланников, Который говорит Иерусалиму: <<ты будешь населен>>, и городам Иудиным: <<вы будете построены, и развалины его Я восстановлю>>,
\vs Isa 44:27 Который бездне говорит: <<иссохни!>> и реки твои Я иссушу,
\vs Isa 44:28 Который говорит о Кире: пастырь Мой, и он исполнит всю волю Мою и скажет Иерусалиму: <<ты будешь построен!>> и храму: <<ты будешь основан!>>
\vs Isa 45:1 Так говорит Господь помазаннику Своему Киру: Я держу тебя за правую руку, чтобы покорить тебе народы, и сниму поясы с чресл царей, чтоб отворялись для тебя двери, и ворота не затворялись;
\vs Isa 45:2 Я пойду пред тобою и горы уровняю, медные двери сокрушу и запоры железные сломаю;
\vs Isa 45:3 и отдам тебе хранимые во тьме сокровища и сокрытые богатства, дабы ты познал, что Я Господь, называющий тебя по имени, Бог Израилев.
\vs Isa 45:4 Ради Иакова, раба Моего, и Израиля, избранного Моего, Я назвал тебя по имени, почтил тебя, хотя ты не знал Меня.
\vs Isa 45:5 Я Господь, и нет иного; нет Бога кроме Меня; Я препоясал тебя, хотя ты не знал Меня,
\vs Isa 45:6 дабы узнали от восхода солнца и от запада, что нет кроме Меня; Я Господь, и нет иного.
\vs Isa 45:7 Я образую свет и творю тьму, делаю мир и произвожу бедствия; Я, Господь, делаю все это.
\vs Isa 45:8 Кропите, небеса, свыше, и облака да проливают правду; да раскроется земля и приносит спасение, и да произрастает вместе правда. Я, Господь, творю это.
\vs Isa 45:9 Горе тому, кто препирается с Создателем своим, черепок из черепков земных! Скажет ли глина горшечнику: <<что ты делаешь?>> и твое дело \bibemph{скажет ли о тебе}: <<у него нет рук>>?
\vs Isa 45:10 Горе тому, кто говорит отцу: <<зачем ты произвел \bibemph{меня} на свет?>>, а матери: <<зачем ты родила \bibemph{меня}?>>
\vs Isa 45:11 Так говорит Господь, Святый Израиля и Создатель его: вы спрашиваете Меня о будущем сыновей Моих и хотите Мне указывать в деле рук Моих?
\vs Isa 45:12 Я создал землю и сотворил на ней человека; Я~--- Мои руки распростерли небеса, и всему воинству их дал закон Я.
\vs Isa 45:13 Я воздвиг его в правде и уровняю все пути его. Он построит город Мой и отпустит пленных Моих, не за выкуп и не за дары, говорит Господь Саваоф.
\rsbpar\vs Isa 45:14 Так говорит Господь: труды Египтян и торговля Ефиоплян, и Савейцы, люди рослые, к тебе перейдут и будут твоими; они последуют за тобою, в цепях придут и повергнутся пред тобою, и будут умолять тебя, \bibemph{говоря}: у тебя только Бог, и нет иного Бога.
\vs Isa 45:15 Истинно Ты Бог сокровенный, Бог Израилев, Спаситель.
\vs Isa 45:16 Все они будут постыжены и посрамлены; вместе с ними со стыдом пойдут и все, делающие идолов.
\vs Isa 45:17 Израиль же будет спасен спасением вечным в Господе; вы не будете постыжены и посрамлены во веки веков.
\vs Isa 45:18 Ибо так говорит Господь, сотворивший небеса, Он, Бог, образовавший землю и создавший ее; Он утвердил ее, не напрасно сотворил ее; Он образовал ее для жительства: Я Господь, и нет иного.
\vs Isa 45:19 Не тайно Я говорил, не в темном месте земли; не говорил Я племени Иакова: <<напрасно ищете Меня>>. Я Господь, изрекающий правду, открывающий истину.
\vs Isa 45:20 Соберитесь и придите, приблизьтесь все, уцелевшие из народов. Невежды те, которые носят деревянного своего идола и молятся богу, который не спасает.
\vs Isa 45:21 Объявите и скажите, посоветовавшись между собою: кто возвестил это из древних времен, наперед сказал это? Не Я ли, Господь? и нет иного Бога кроме Меня, Бога праведного и спасающего нет кроме Меня.
\vs Isa 45:22 Ко Мне обратитесь, и будете спасены, все концы земли, ибо я Бог, и нет иного.
\vs Isa 45:23 Мною клянусь: из уст Моих исходит правда, слово неизменное, что предо Мною преклонится всякое колено, Мною будет клясться всякий язык.
\vs Isa 45:24 Только у Господа, будут говорить о Мне, правда и сила; к Нему придут и устыдятся все, враждовавшие против Него.
\vs Isa 45:25 Господом будет оправдано и прославлено все племя Израилево.
\vs Isa 46:1 Пал Вил, низвергся Нев\acc{о}; истуканы их~--- на скоте и вьючных животных; ваша ноша сделалась бременем для усталых животных.
\vs Isa 46:2 Низверглись, пали вместе; не могли защитить носивших, и сами пошли в плен.
\rsbpar\vs Isa 46:3 Послушайте меня, дом Иаковлев и весь остаток дома Израилева, принятые \bibemph{Мною} от чрева, носимые Мною от утробы \bibemph{матерней}:
\vs Isa 46:4 и до старости вашей Я Тот же буду, и до седины вашей Я же буду носить \bibemph{вас}; Я создал и буду носить, поддерживать и охранять вас.
\vs Isa 46:5 Кому уподобите Меня, и \bibemph{с кем} сравните, и с кем сличите, чтобы мы были сходны?
\vs Isa 46:6 Высыпают золото из кошелька и весят серебро на весах, и нанимают серебряника, чтобы он сделал из него бога; кланяются ему и повергаются перед ним;
\vs Isa 46:7 поднимают его на плечи, несут его и ставят его на свое место; он стоит, с места своего не двигается; кричат к нему,~--- он не отвечает, не спасает от беды.
\vs Isa 46:8 Вспомните это и покажите себя мужами; примите это, отступники, к сердцу;
\vs Isa 46:9 вспомните прежде бывшее, от \bibemph{начала} века, ибо Я Бог, и нет иного Бога, и нет подобного Мне.
\vs Isa 46:10 Я возвещаю от начала, что будет в конце, и от древних времен то, что еще не сделалось, говорю: Мой совет состоится, и все, что Мне угодно, Я сделаю.
\vs Isa 46:11 Я воззвал орла от востока, из дальней страны, исполнителя определения Моего. Я сказал, и приведу это в исполнение; предначертал, и сделаю.
\vs Isa 46:12 Послушайте Меня, жестокие сердцем, далекие от правды:
\vs Isa 46:13 Я приблизил правду Мою, она не далеко, и спасение Мое не замедлит; и дам Сиону спасение, Израилю славу Мою.
\vs Isa 47:1 Сойди и сядь на прах, девица, дочь Вавилона; сиди на земле: престола нет, дочь Халдеев, и вперед не будут называть тебя нежною и роскошною.
\vs Isa 47:2 Возьми жернова и мели муку; сними покрывало твое, подбери подол, открой голени, переходи через реки:
\vs Isa 47:3 откроется нагота твоя, и даже виден будет стыд твой. Совершу мщение и не пощажу никого.
\vs Isa 47:4 Искупитель наш~--- Господь Саваоф имя Ему, Святый Израилев.
\vs Isa 47:5 Сиди молча и уйди в темноту, дочь Халдеев: ибо вперед не будут называть тебя госпожею царств.
\vs Isa 47:6 Я прогневался на народ Мой, уничижил наследие Мое и предал их в руки твои; \bibemph{а} ты не оказала им милосердия, на старца налагала крайне тяжкое иго твое.
\vs Isa 47:7 И ты говорила: <<вечно буду госпожею>>, а не представляла того в уме твоем, не помышляла, что будет после.
\vs Isa 47:8 Но ныне выслушай это, изнеженная, живущая беспечно, говорящая в сердце своем: <<я,~--- и другой подобной мне нет; не буду сидеть вдовою и не буду знать потери детей>>.
\vs Isa 47:9 Но внезапно, в один день, придет к тебе то и другое, потеря детей и вдовство; в полной мере придут они на тебя, несмотря на множество чародейств твоих и на великую силу волшебств твоих.
\vs Isa 47:10 Ибо ты надеялась на злодейство твое, говорила: <<никто не видит меня>>. Мудрость твоя и знание твое~--- они сбили тебя с пути; и ты говорила в сердце твоем: <<я, и никто кроме меня>>.
\vs Isa 47:11 И придет на тебя бедствие: ты не узнаешь, откуда оно поднимется; и нападет на тебя беда, которой ты не в силах будешь отвратить, и внезапно придет на тебя пагуба, о которой ты и не думаешь.
\vs Isa 47:12 Оставайся же с твоими волшебствами и со множеством чародейств твоих, которыми ты занималась от юности твоей: может быть, пособишь себе, может быть, устоишь.
\vs Isa 47:13 Ты утомлена множеством советов твоих; пусть же выступят наблюдатели небес и звездочеты и предвещатели по новолуниям, и спасут тебя от того, что должно приключиться тебе.
\vs Isa 47:14 Вот они, как солома: огонь сожег их,~--- не избавили души своей от пламени; не осталось угля, чтобы погреться, ни огня, чтобы посидеть перед ним.
\vs Isa 47:15 Такими стали для тебя те, с которыми ты трудилась, с которыми вела торговлю от юности твоей. Каждый побрел в свою сторону; никто не спасает тебя.
\vs Isa 48:1 Слушайте это, дом Иакова, называющиеся именем Израиля и происшедшие от источника Иудина, клянущиеся именем Господа и исповедающие Бога Израилева, хотя не по истине и не по правде.
\vs Isa 48:2 Ибо они называют себя \bibemph{происходящими} от святого города и опираются на Бога Израилева; Господь Саваоф~--- имя Ему.
\vs Isa 48:3 Прежнее Я задолго объявлял; из Моих уст выходило оно, и Я возвещал это и внезапно делал, и все сбывалось.
\vs Isa 48:4 Я знал, что ты упорен, и что в шее твоей жилы железные, и лоб твой~--- медный;
\vs Isa 48:5 поэтому и объявлял тебе задолго, прежде нежели это приходило, и предъявлял тебе, чтобы ты не сказал: <<идол мой сделал это, и истукан мой и изваянный мой повелел этому быть>>.
\vs Isa 48:6 Ты слышал,~--- посмотри на все это! и неужели вы не признаёте этого? А ныне Я возвещаю тебе новое и сокровенное, и ты не знал этого.
\vs Isa 48:7 Оно произошло ныне, а не задолго и не за день, и ты не слыхал о том, чтобы ты не сказал: <<вот! я знал это>>.
\vs Isa 48:8 Ты и не слыхал и не знал об этом, и ухо твое не было прежде открыто; ибо Я знал, что ты поступишь вероломно, и от самого чрева \bibemph{матернего} ты прозван отступником.
\vs Isa 48:9 Ради имени Моего отлагал гнев Мой, и ради славы Моей удерживал Себя от истребления тебя.
\vs Isa 48:10 Вот, Я расплавил тебя, но не как серебро; испытал тебя в горниле страдания.
\vs Isa 48:11 Ради Себя, ради Себя Самого делаю это,~--- ибо какое было бы нарекание \bibemph{на имя Мое}! славы Моей не дам иному.
\rsbpar\vs Isa 48:12 Послушай Меня, Иаков и Израиль, призванный Мой: Я Тот же, Я первый и Я последний.
\vs Isa 48:13 Моя рука основала землю, и Моя десница распростерла небеса; призову их, и они предстанут вместе.
\vs Isa 48:14 Соберитесь все и слушайте: кто между ними предсказал это? Господь возлюбил его, и он исполнит волю Его над Вавилоном и явит мышцу Его над Халдеями.
\vs Isa 48:15 Я, Я сказал, и призвал его; Я привел его, и путь его будет благоуспешен.
\vs Isa 48:16 Приступите ко Мне, слушайте это: Я и сначала говорил не тайно; с того времени, как это происходит, Я был там; и ныне послал Меня Господь Бог и Дух Его.
\rsbpar\vs Isa 48:17 Так говорит Господь, Искупитель твой, Святый Израилев: Я Господь, Бог твой, научающий тебя полезному, ведущий тебя по тому пути, по которому должно тебе идти.
\vs Isa 48:18 О, если бы ты внимал заповедям Моим! тогда мир твой был бы как река, и правда твоя~--- как волны морские.
\vs Isa 48:19 И семя твое было бы как песок, и происходящие из чресл твоих~--- как песчинки: не изгладилось бы, не истребилось бы имя его предо Мною.
\vs Isa 48:20 Выходите из Вавилона, бегите от Халдеев, со гласом радости возвещайте и проповедуйте это, распространяйте эту весть до пределов земли; говорите: <<Господь искупил раба Своего Иакова>>.
\vs Isa 48:21 И не жаждут они в пустынях, чрез которые Он ведет их: Он источает им воду из камня; рассекает скалу, и льются воды.
\vs Isa 48:22 Нечестивым же нет мира, говорит Господь.
\vs Isa 49:1 Слушайте Меня, острова, и внимайте, народы дальние: Господь призвал Меня от чрева, от утробы матери Моей называл имя Мое;
\vs Isa 49:2 и соделал уста Мои как острый меч; тенью руки Своей покрывал Меня, и соделал Меня стрелою изостренною; в колчане Своем хранил Меня;
\vs Isa 49:3 и сказал Мне: Ты раб Мой, Израиль, в Тебе Я прославлюсь.
\vs Isa 49:4 А Я сказал: напрасно Я трудился, ни на что и вотще истощал силу Свою. Но Мое право у Господа, и награда Моя у Бога Моего.
\vs Isa 49:5 И ныне говорит Господь, образовавший Меня от чрева в раба Себе, чтобы обратить к Нему Иакова и чтобы Израиль собрался к Нему; Я почтен в очах Господа, и Бог Мой~--- сила Моя.
\vs Isa 49:6 И Он сказал: мало того, что Ты будешь рабом Моим для восстановления колен Иаковлевых и для возвращения остатков Израиля, но Я сделаю Тебя светом народов, чтобы спасение Мое простерлось до концов земли.
\vs Isa 49:7 Так говорит Господь, Искупитель Израиля, Святый Его, презираемому всеми, поносимому народом, рабу властелинов: цари увидят, и встанут; князья поклонятся ради Господа, Который верен, ради Святаго Израилева, Который избрал Тебя.
\rsbpar\vs Isa 49:8 Так говорит Господь: во время благоприятное Я услышал Тебя, и в день спасения помог Тебе; и Я буду охранять Тебя, и сделаю Тебя заветом народа, чтобы восстановить землю, чтобы возвратить наследникам наследия опустошенные,
\vs Isa 49:9 сказать узникам: <<выходите>>, и тем, которые во тьме: <<покажитесь>>. Они при дорогах будут пасти, и по всем холмам будут пажити их;
\vs Isa 49:10 не будут терпеть голода и жажды, и не поразит их зной и солнце; ибо Милующий их будет вести их и приведет их к источникам вод.
\vs Isa 49:11 И все горы Мои сделаю путем, и дороги Мои будут подняты.
\vs Isa 49:12 Вот, одни придут издалека; и вот, одни от севера и моря, а другие из земли Синим.
\vs Isa 49:13 Радуйтесь, небеса, и веселись, земля, и восклицайте, горы, от радости; ибо утешил Господь народ Свой и помиловал страдальцев Своих.
\vs Isa 49:14 А Сион говорил: <<оставил меня Господь, и Бог мой забыл меня!>>
\vs Isa 49:15 Забудет ли женщина грудное дитя свое, чтобы не пожалеть сына чрева своего? но если бы и она забыла, то Я не забуду тебя.
\vs Isa 49:16 Вот, Я начертал тебя на дланях \bibemph{Моих}; стены твои всегда предо Мною.
\vs Isa 49:17 Сыновья твои поспешат \bibemph{к тебе}, а разорители и опустошители твои уйдут от тебя.
\vs Isa 49:18 Возведи очи твои и посмотри вокруг,~--- все они собираются, идут к тебе. Живу Я! говорит Господь,~--- всеми ими ты облечешься, как убранством, и нарядишься ими, как невеста.
\vs Isa 49:19 Ибо развалины твои и пустыни твои, и разоренная земля твоя будут теперь слишком тесны для жителей, и поглощавшие тебя удалятся от тебя.
\vs Isa 49:20 Дети, которые будут у тебя после потери прежних, будут говорить вслух тебе: <<тесно для меня место; уступи мне, чтобы я мог жить>>.
\vs Isa 49:21 И ты скажешь в сердце твоем: кто мне родил их? я была бездетна и бесплодна, отведена в плен и удалена; кто же возрастил их? вот, я оставалась одинокою; где же они были?
\rsbpar\vs Isa 49:22 Так говорит Господь Бог: вот, Я подниму руку Мою к народам, и выставлю знамя Мое племенам, и принесут сыновей твоих на руках и дочерей твоих на плечах.
\vs Isa 49:23 И будут цари питателями твоими, и царицы их кормилицами твоими; лицом до земли будут кланяться тебе и лизать прах ног твоих, и узнаешь, что Я Господь, что надеющиеся на Меня не постыдятся.
\vs Isa 49:24 Может ли быть отнята у сильного добыча, и могут ли быть отняты у победителя взятые в плен?
\vs Isa 49:25 Да! так говорит Господь: и плененные сильным будут отняты, и добыча тирана будет избавлена; потому что Я буду состязаться с противниками твоими и сыновей твоих Я спасу;
\vs Isa 49:26 и притеснителей твоих накормлю собственною их плотью, и они будут упоены кровью своею, как молодым вином; и всякая плоть узнает, что Я Господь, Спаситель твой и Искупитель твой, Сильный Иаковлев.
\vs Isa 50:1 Так говорит Господь: где разводное письмо вашей матери, с которым Я отпустил ее? или которому из Моих заимодавцев Я продал вас? Вот, вы проданы за грехи ваши, и за преступления ваши отпущена мать ваша.
\vs Isa 50:2 Почему, когда Я приходил, никого не было, и когда Я звал, никто не отвечал? Разве рука Моя коротка стала для того, чтобы избавлять, или нет силы во Мне, чтобы спасать? Вот, прещением Моим Я иссушаю море, превращаю реки в пустыню; рыбы в них гниют от недостатка воды и умирают от жажды.
\vs Isa 50:3 Я облекаю небеса мраком, и вретище делаю покровом их.
\vs Isa 50:4 Господь Бог дал Мне язык мудрых, чтобы Я мог словом подкреплять изнемогающего; каждое утро Он пробуждает, пробуждает ухо Мое, чтобы Я слушал, подобно учащимся.
\vs Isa 50:5 Господь Бог открыл Мне ухо, и Я не воспротивился, не отступил назад.
\vs Isa 50:6 Я предал хребет Мой биющим и ланиты Мои поражающим; лица Моего не закрывал от поруганий и оплевания.
\vs Isa 50:7 И Господь Бог помогает Мне: поэтому Я не стыжусь, поэтому Я держу лице Мое, как кремень, и знаю, что не останусь в стыде.
\vs Isa 50:8 Близок оправдывающий Меня: кто хочет состязаться со Мною? станем вместе. Кто хочет судиться со Мною? пусть подойдет ко Мне.
\vs Isa 50:9 Вот, Господь Бог помогает Мне: кто осудит Меня? Вот, все они, как одежда, обветшают; моль съест их.
\vs Isa 50:10 Кто из вас боится Господа, слушается гласа Раба Его? Кто ходит во мраке, без света, да уповает на имя Господа и да утверждается в Боге своем.
\vs Isa 50:11 Вот, все вы, которые возжигаете огонь, вооруженные зажигательными стрелами,~--- идите в пламень огня вашего и стрел, раскаленных вами! Это будет вам от руки Моей; в мучении умрете.
\vs Isa 51:1 Послушайте Меня, стремящиеся к правде, ищущие Господа! Взгляните на скалу, из которой вы иссечены, в глубину рва, из которого вы извлечены.
\vs Isa 51:2 Посмотрите на Авраама, отца вашего, и на Сарру, родившую вас: ибо Я призвал его одного и благословил его, и размножил его.
\vs Isa 51:3 Так, Господь утешит Сион, утешит все развалины его и сделает пустыни его, как рай, и степь его, как сад Господа; радость и веселье будет в нем, славословие и песнопение.
\vs Isa 51:4 Послушайте Меня, народ Мой, и племя Мое, приклоните ухо ко Мне! ибо от Меня произойдет закон, и суд Мой поставлю во свет для народов.
\vs Isa 51:5 Правда Моя близка; спасение Мое восходит, и мышца Моя будет судить народы; острова будут уповать на Меня и надеяться на мышцу Мою.
\vs Isa 51:6 Поднимите глаза ваши к небесам, и посмотрите на землю вниз: ибо небеса исчезнут, как дым, и земля обветшает, как одежда, и жители ее также вымрут; а Мое спасение пребудет вечным, и правда Моя не престанет.
\vs Isa 51:7 Послушайте Меня, знающие правду, народ, у которого в сердце закон Мой! Не бойтесь поношения от людей, и злословия их не страшитесь.
\vs Isa 51:8 Ибо, как одежду, съест их моль и, как в\acc{о}лну, съест их червь; а правда Моя пребудет вовек, и спасение Мое~--- в роды родов.
\vs Isa 51:9 Восстань, восстань, облекись крепостью, мышца Господня! Восстань, как в дни древние, в роды давние! Не ты ли сразила Раава, поразила крокодила?
\vs Isa 51:10 Не ты ли иссушила море, в\acc{о}ды великой бездны, превратила глубины моря в дорогу, чтобы прошли искупленные?
\vs Isa 51:11 И возвратятся избавленные Господом и придут на Сион с пением, и радость вечная над головою их; они найдут радость и веселье: печаль и вздохи удалятся.
\vs Isa 51:12 Я, Я Сам~--- Утешитель ваш. Кто ты, что боишься человека, который умирает, и сына человеческого, который то же, что трава,
\vs Isa 51:13 и забываешь Господа, Творца своего, распростершего небеса и основавшего землю; и непрестанно, всякий день страшишься ярости притеснителя, как бы он готов был истребить? Но где ярость притеснителя?
\vs Isa 51:14 Скоро освобожден будет пленный, и не умрет в яме и не будет нуждаться в хлебе.
\vs Isa 51:15 Я Господь, Бог твой, возмущающий море, так что волны его ревут: Господь Саваоф~--- имя Его.
\vs Isa 51:16 И Я вложу слова Мои в уста твои, и тенью руки Моей покрою тебя, чтобы устроить небеса и утвердить землю и сказать Сиону: <<ты Мой народ>>.
\vs Isa 51:17 Воспряни, воспряни, восстань, Иерусалим, ты, который из руки Господа выпил чашу ярости Его, выпил до дна чашу опьянения, осушил.
\vs Isa 51:18 Некому было вести его из всех сыновей, рожденных им, и некому было поддержать его за руку из всех сыновей, \bibemph{которых} он возрастил.
\vs Isa 51:19 Тебя постигли два \bibemph{бедствия}, кто пожалеет о тебе?~--- опустошение и истребление, голод и меч: кем я утешу тебя?
\vs Isa 51:20 Сыновья твои изнемогли, лежат по углам всех улиц, как серна в тенетах, исполненные гнева Господа, прещения Бога твоего.
\vs Isa 51:21 Итак выслушай это, страдалец и опьяневший, но не от вина.
\vs Isa 51:22 Так говорит Господь твой, Господь и Бог твой, отмщающий за Свой народ: вот, Я беру из руки твоей чашу опьянения, дрожжи из чаши ярости Моей: ты не будешь уже пить их,
\vs Isa 51:23 и подам ее в руки мучителям твоим, которые говорили тебе: <<пади ниц, чтобы нам пройти по тебе>>; и ты хребет твой делал как бы землею и улицею для проходящих.
\vs Isa 52:1 Восстань, восстань, облекись в силу твою, Сион! Облекись в одежды величия твоего, Иерусалим, город святый! ибо уже не будет более входить в тебя необрезанный и нечистый.
\vs Isa 52:2 Отряси с себя прах; встань, пленный Иерусалим! сними цепи с шеи твоей, пленная дочь Сиона!
\vs Isa 52:3 ибо так говорит Господь: за ничто были вы проданы, и без серебра будете выкуплены;
\vs Isa 52:4 ибо так говорит Господь Бог: народ Мой ходил прежде в Египет, чтобы там пожить, и Ассур теснил его ни за что.
\vs Isa 52:5 И теперь что у Меня здесь? говорит Господь; народ Мой взят даром, властители их неистовствуют, говорит Господь, и постоянно, всякий день имя Мое бесславится.
\vs Isa 52:6 Поэтому народ Мой узн\acc{а}ет имя Мое; поэтому \bibemph{узн\acc{а}ет} в тот день, что Я Тот же, Который сказал: <<вот Я!>>
\vs Isa 52:7 Как прекрасны на горах ноги благовестника, возвещающего мир, благовествующего радость, проповедующего спасение, говорящего Сиону: <<воцарился Бог твой!>>
\vs Isa 52:8 Голос сторожей твоих~--- они возвысили голос, и все вместе ликуют, ибо своими глазами видят, что Господь возвращается в Сион.
\vs Isa 52:9 Торжествуйте, пойте вместе, развалины Иерусалима, ибо утешил Господь народ Свой, искупил Иерусалим.
\vs Isa 52:10 Обнажил Господь святую мышцу Свою пред глазами всех народов; и все концы земли увидят спасение Бога нашего.
\vs Isa 52:11 Идите, идите, выходите оттуда; не касайтесь нечистого; выходите из среды его, очистите себя, носящие сосуды Господни!
\vs Isa 52:12 ибо вы выйдете неторопливо, и не побежите; потому что впереди вас пойдет Господь, и Бог Израилев будет стражем позади вас.
\vs Isa 52:13 Вот, раб Мой будет благоуспешен, возвысится и вознесется, и возвеличится.
\vs Isa 52:14 Как многие изумлялись, \bibemph{смотря} на Тебя,~--- столько был обезображен паче всякого человека лик Его, и вид Его~--- паче сынов человеческих!
\vs Isa 52:15 Так многие народы приведет Он в изумление; цари закроют пред Ним уста свои, ибо они увидят то, о чем не было говорено им, и узнают то, чего не слыхали.
\vs Isa 53:1 [Господи!] кто поверил слышанному от нас, и кому открылась мышца Господня?
\vs Isa 53:2 Ибо Он взошел пред Ним, как отпрыск и как росток из сухой земли; нет в Нем ни вида, ни величия; и мы видели Его, и не было в Нем вида, который привлекал бы нас к Нему.
\vs Isa 53:3 Он был презрен и умален пред людьми, муж скорбей и изведавший болезни, и мы отвращали от Него лице свое; Он был презираем, и мы ни во что ставили Его.
\vs Isa 53:4 Но Он взял на Себя наши немощи и понес наши болезни; а мы думали, \bibemph{что} Он был поражаем, наказуем и уничижен Богом.
\vs Isa 53:5 Но Он изъязвлен был за грехи наши и мучим за беззакония наши; наказание мира нашего \bibemph{было} на Нем, и ранами Его мы исцелились.
\vs Isa 53:6 Все мы блуждали, как овцы, совратились каждый на свою дорогу: и Господь возложил на Него грехи всех нас.
\vs Isa 53:7 Он истязуем был, но страдал добровольно и не открывал уст Своих; как овца, веден был Он на заклание, и как агнец пред стригущим его безгласен, так Он не отверзал уст Своих.
\vs Isa 53:8 От уз и суда Он был взят; но род Его кто изъяснит? ибо Он отторгнут от земли живых; за преступления народа Моего претерпел казнь.
\vs Isa 53:9 Ему назначали гроб со злодеями, но Он погребен у богатого, потому что не сделал греха, и не было лжи в устах Его.
\vs Isa 53:10 Но Господу угодно было поразить Его, и Он предал Его мучению; когда же душа Его принесет жертву умилостивления, Он узрит потомство долговечное, и воля Господня благоуспешно будет исполняться рукою Его.
\vs Isa 53:11 На подвиг души Своей Он будет смотреть с довольством; чрез познание Его Он, Праведник, Раб Мой, оправдает многих и грехи их на Себе понесет.
\vs Isa 53:12 Посему Я дам Ему часть между великими, и с сильными будет делить добычу, за то, что предал душу Свою на смерть, и к злодеям причтен был, тогда как Он понес на Себе грех многих и за преступников сделался ходатаем.
\vs Isa 54:1 Возвеселись, неплодная, нерождающая; воскликни и возгласи, немучившаяся родами; потому что у оставленной гораздо более детей, нежели у имеющей мужа, говорит Господь.
\vs Isa 54:2 Распространи место шатра твоего, расширь покровы жилищ твоих; не стесняйся, пусти длиннее верви твои и утверди колья твои;
\vs Isa 54:3 ибо ты распространишься направо и налево, и потомство твое завладеет народами и населит опустошенные города.
\vs Isa 54:4 Не бойся, ибо не будешь постыжена; не смущайся, ибо не будешь в поругании: ты забудешь посрамление юности твоей и не будешь более вспоминать о бесславии вдовства твоего.
\vs Isa 54:5 Ибо твой Творец есть супруг твой; Господь Саваоф~--- имя Его; и Искупитель твой~--- Святый Израилев: Богом всей земли назовется Он.
\vs Isa 54:6 Ибо как жену, оставленную и скорбящую духом, призывает тебя Господь, и \bibemph{как} жену юности, которая была отвержена, говорит Бог твой.
\vs Isa 54:7 На малое время Я оставил тебя, но с великою милостью восприму тебя.
\vs Isa 54:8 В жару гнева Я сокрыл от тебя лице Мое на время, но вечною милостью помилую тебя, говорит Искупитель твой, Господь.
\vs Isa 54:9 Ибо это для Меня, как воды Ноя: как Я поклялся, что воды Ноя не придут более на землю, так поклялся не гневаться на тебя и не укорять тебя.
\vs Isa 54:10 Горы сдвинутся и холмы поколеблются,~--- а милость Моя не отступит от тебя, и завет мира Моего не поколеблется, говорит милующий тебя Господь.
\vs Isa 54:11 Бедная, бросаемая бурею, безутешная! Вот, Я положу камни твои на рубине и сделаю основание твое из сапфиров;
\vs Isa 54:12 и сделаю окна твои из рубинов и ворота твои~--- из жемчужин, и всю ограду твою~--- из драгоценных камней.
\vs Isa 54:13 И все сыновья твои будут научены Господом, и великий мир будет у сыновей твоих.
\vs Isa 54:14 Ты утвердишься правдою, будешь далека от угнетения, ибо тебе бояться нечего, и от ужаса, ибо он не приблизится к тебе.
\vs Isa 54:15 Вот, будут вооружаться \bibemph{против тебя}, но не от Меня; кто бы ни вооружился против тебя, падет.
\vs Isa 54:16 Вот, Я сотворил кузнеца, который раздувает угли в огне и производит орудие для своего дела,~--- и Я творю губителя для истребления.
\vs Isa 54:17 Ни одно орудие, сделанное против тебя, не будет успешно; и всякий язык, который будет состязаться с тобою на суде,~--- ты обвинишь. Это есть наследие рабов Господа, оправдание их от Меня, говорит Господь.
\vs Isa 55:1 Жаждущие! идите все к водам; даже и вы, у которых нет серебра, идите, покупайте и ешьте; идите, покупайте без серебра и без платы вино и молоко.
\vs Isa 55:2 Для чего вам отвешивать серебро за то, что не хлеб, и трудовое свое за то, что не насыщает? Послушайте Меня внимательно и вкушайте благо, и душа ваша да насладится туком.
\vs Isa 55:3 Приклоните ухо ваше и придите ко Мне: послушайте, и жива будет душа ваша,~--- и дам вам завет вечный, неизменные милости, \bibemph{обещанные} Давиду.
\vs Isa 55:4 Вот, Я дал Его свидетелем для народов, вождем и наставником народам.
\vs Isa 55:5 Вот, ты призовешь народ, которого ты не знал, и народы, которые тебя не знали, поспешат к тебе ради Господа Бога твоего и ради Святаго Израилева, ибо Он прославил тебя.
\rsbpar\vs Isa 55:6 Ищите Господа, когда можно найти Его; призывайте Его, когда Он близко.
\vs Isa 55:7 Да оставит нечестивый путь свой и беззаконник~--- помыслы свои, и да обратится к Господу, и Он помилует его, и к Богу нашему, ибо Он многомилостив.
\vs Isa 55:8 Мои мысли~--- не ваши мысли, ни ваши пути~--- пути Мои, говорит Господь.
\vs Isa 55:9 Но как небо выше земли, так пути Мои выше путей ваших, и мысли Мои выше мыслей ваших.
\vs Isa 55:10 Как дождь и снег нисходит с неба и туда не возвращается, но напояет землю и делает ее способною рождать и произращать, чтобы она давала семя тому, кто сеет, и хлеб тому, кто ест,~---
\vs Isa 55:11 так и слово Мое, которое исходит из уст Моих,~--- оно не возвращается ко Мне тщетным, но исполняет то, что Мне угодно, и совершает то, для чего Я послал его.
\vs Isa 55:12 Итак вы выйдете с весельем и будете провожаемы с миром; горы и холмы будут петь пред вами песнь, и все дерева в поле рукоплескать вам.
\vs Isa 55:13 Вместо терновника вырастет кипарис; вместо крапивы возрастет мирт; и это будет во славу Господа, в знамение вечное, несокрушимое.
\vs Isa 56:1 Так говорит Господь: сохраняйте суд и делайте правду; ибо близко спасение Мое и откровение правды Моей.
\vs Isa 56:2 Блажен муж, который делает это, и сын человеческий, который крепко держится этого, который хранит субботу от осквернения и оберегает руку свою, чтобы не сделать никакого зла.
\vs Isa 56:3 Да не говорит сын иноплеменника, присоединившийся к Господу: <<Господь совсем отделил меня от Своего народа>>, и да не говорит евнух: <<вот я сухое дерево>>.
\vs Isa 56:4 Ибо Господь так говорит об евнухах: которые хранят Мои субботы и избирают угодное Мне, и крепко держатся завета Моего,~---
\vs Isa 56:5 тем дам Я в доме Моем и в стенах Моих место и имя лучшее, нежели сыновьям и дочерям; дам им вечное имя, которое не истребится.
\vs Isa 56:6 И сыновей иноплеменников, присоединившихся к Господу, чтобы служить Ему и любить имя Господа, быть рабами Его, всех, хранящих субботу от осквернения ее и твердо держащихся завета Моего,
\vs Isa 56:7 Я приведу на святую гору Мою и обрадую их в Моем доме молитвы; всесожжения их и жертвы их \bibemph{будут} благоприятны на жертвеннике Моем, ибо дом Мой назовется домом молитвы для всех народов.
\rsbpar\vs Isa 56:8 Господь Бог, собирающий рассеянных Израильтян, говорит: к собранным у него Я буду еще собирать других.
\vs Isa 56:9 Все звери полевые, все звери лесные! идите есть.
\vs Isa 56:10 Стражи их слепы все и невежды: все они немые псы, не могущие лаять, бредящие лежа, любящие спать.
\vs Isa 56:11 И это псы, жадные душею, не знающие сытости; и это пастыри бессмысленные: все смотрят на свою дорогу, каждый до последнего, на свою корысть;
\vs Isa 56:12 приходите, \bibemph{говорят}, я достану вина, и мы напьемся сикеры; и завтра то же будет, что сегодня, да еще и больше.
\vs Isa 57:1 Праведник умирает, и никто не принимает этого к сердцу; и мужи благочестивые восхищаются \bibemph{от земли}, и никто не помыслит, что праведник восхищается от зла.
\vs Isa 57:2 Он отходит к миру; ходящие прямым путем будут покоиться на ложах своих.
\vs Isa 57:3 Но приблизьтесь сюда вы, сыновья чародейки, семя прелюбодея и блудницы!
\vs Isa 57:4 Над кем вы глумитесь? против кого расширяете рот, высовываете язык? не дети ли вы преступления, семя лжи,
\vs Isa 57:5 разжигаемые похотью к идолам под каждым ветвистым деревом, заколающие детей при ручьях, между расселинами скал?
\vs Isa 57:6 В гладких камнях ручьев доля твоя; они, они жребий твой; им ты делаешь возлияние и приносишь жертвы: могу ли Я быть доволен этим?
\vs Isa 57:7 На высокой и выдающейся горе ты ставишь ложе твое и туда восходишь приносить жертву.
\vs Isa 57:8 За дверью также и за косяками ставишь памяти твои; ибо, отвратившись от Меня, ты обнажаешься и восходишь; распространяешь ложе твое и договариваешься с теми из них, с которыми любишь лежать, высматриваешь место.
\vs Isa 57:9 Ты ходила также к царю с благовонною мастью и умножила масти твои, и далеко посылала послов твоих, и унижалась до преисподней.
\vs Isa 57:10 От долгого пути твоего утомлялась, но не говорила: <<надежда потеряна!>>; все еще находила живость в руке твоей, и потому не чувствовала ослабления.
\vs Isa 57:11 Кого же ты испугалась и устрашилась, что сделалась неверною и Меня перестала помнить и хранить в твоем сердце? не оттого ли, что Я молчал, и притом долго, ты перестала бояться Меня?
\vs Isa 57:12 Я покажу правду твою и дела твои,~--- и они будут не в пользу тебе.
\vs Isa 57:13 Когда ты будешь вопить, спасет ли тебя сборище твое?~--- всех их унесет ветер, развеет дуновение; а надеющийся на Меня наследует землю и будет владеть святою горою Моею.
\rsbpar\vs Isa 57:14 И сказал: поднимайте, поднимайте, ровняйте путь, убирайте преграду с пути народа Моего.
\vs Isa 57:15 Ибо так говорит Высокий и Превознесенный, вечно Живущий,~--- Святый имя Его: Я живу на высоте \bibemph{небес} и во святилище, и также с сокрушенными и смиренными духом, чтобы оживлять дух смиренных и оживлять сердца сокрушенных.
\vs Isa 57:16 Ибо не вечно буду Я вести тяжбу и не до конца гневаться; иначе изнеможет предо Мною дух и всякое дыхание, Мною сотворенное.
\vs Isa 57:17 За грех корыстолюбия его Я гневался и поражал его, скрывал лице и негодовал; но он, отвратившись, пошел по пути своего сердца.
\vs Isa 57:18 Я видел пути его, и исцелю его, и буду водить его и утешать его и сетующих его.
\vs Isa 57:19 Я исполню слово: мир, мир дальнему и ближнему, говорит Господь, и исцелю его.
\vs Isa 57:20 А нечестивые~--- как море взволнованное, которое не может успокоиться и которого в\acc{о}ды выбрасывают ил и грязь.
\vs Isa 57:21 Нет мира нечестивым, говорит Бог мой.
\vs Isa 58:1 Взывай громко, не удерживайся; возвысь голос твой, подобно трубе, и укажи народу Моему на беззакония его, и дому Иаковлеву~--- на грехи его.
\vs Isa 58:2 Они каждый день ищут Меня и хотят знать пути Мои, как бы народ, поступающий праведно и не оставляющий законов Бога своего; они вопрошают Меня о судах правды, желают приближения к Богу:
\vs Isa 58:3 <<Почему мы постимся, а Ты не видишь? смиряем души свои, а Ты не знаешь?>>~--- Вот, в день поста вашего вы исполняете волю вашу и требуете тяжких трудов от других.
\vs Isa 58:4 Вот, вы поститесь для ссор и распрей и для того, чтобы дерзкою рукою бить других; вы не поститесь в это время так, чтобы голос ваш был услышан на высоте.
\vs Isa 58:5 Таков ли тот пост, который Я избрал, день, в который томит человек душу свою, когда гнет голову свою, как тростник, и подстилает под себя рубище и пепел? Это ли назовешь постом и днем, угодным Господу?
\vs Isa 58:6 Вот пост, который Я избрал: разреши оковы неправды, развяжи узы ярма, и угнетенных отпусти на свободу, и расторгни всякое ярмо;
\vs Isa 58:7 раздели с голодным хлеб твой, и скитающихся бедных введи в дом; когда увидишь нагого, одень его, и от единокровного твоего не укрывайся.
\vs Isa 58:8 Тогда откроется, как заря, свет твой, и исцеление твое скоро возрастет, и правда твоя пойдет пред тобою, и слава Господня будет сопровождать тебя.
\vs Isa 58:9 Тогда ты воззовешь, и Господь услышит; возопиешь, и Он скажет: <<вот Я!>> Когда ты удалишь из среды твоей ярмо, перестанешь поднимать перст и говорить оскорбительное,
\vs Isa 58:10 и отдашь голодному душу твою и напитаешь душу страдальца: тогда свет твой взойдет во тьме, и мрак твой \bibemph{будет} как полдень;
\vs Isa 58:11 и будет Господь вождем твоим всегда, и во время засухи будет насыщать душу твою и утучнять кости твои, и ты будешь, как напоенный водою сад и как источник, которого воды никогда не иссякают.
\vs Isa 58:12 И застроятся \bibemph{потомками} твоими пустыни вековые: ты восстановишь основания многих поколений, и будут называть тебя восстановителем развалин, возобновителем путей для населения.
\vs Isa 58:13 Если ты удержишь ногу твою ради субботы от исполнения прихотей твоих во святый день Мой, и будешь называть субботу отрадою, святым днем Господним, чествуемым, и почтишь ее тем, что не будешь заниматься обычными твоими делами, угождать твоей прихоти и пустословить,~---
\vs Isa 58:14 то будешь иметь радость в Господе, и Я возведу тебя на высоты земли и дам вкусить тебе наследие Иакова, отца твоего: уста Господни изрекли это.
\vs Isa 59:1 Вот, рука Господа не сократилась на то, чтобы спасать, и ухо Его не отяжелело для того, чтобы слышать.
\vs Isa 59:2 Но беззакония ваши произвели разделение между вами и Богом вашим, и грехи ваши отвращают лице \bibemph{Его} от вас, чтобы не слышать.
\vs Isa 59:3 Ибо руки ваши осквернены кровью и персты ваши~--- беззаконием; уста ваши говорят ложь, язык ваш произносит неправду.
\vs Isa 59:4 Никто не возвышает голоса за правду, и никто не вступается за истину; надеются на пустое и говорят ложь, зачинают зло и рождают злодейство;
\vs Isa 59:5 высиживают змеиные яйца и ткут паутину; кто поест яиц их,~--- умрет, а если раздавит,~--- выползет ехидна.
\vs Isa 59:6 Паутины их для одежды негодны, и они не покроются своим произведением; дела их~--- дела неправедные, и насилие в руках их.
\vs Isa 59:7 Ноги их бегут ко злу, и они спешат на пролитие невинной крови; мысли их~--- мысли нечестивые; опустошение и гибель на стезях их.
\vs Isa 59:8 Пути мира они не знают, и нет суда на стезях их; пути их искривлены, и никто, идущий по ним, не знает мира.
\vs Isa 59:9 Потому-то и далек от нас суд, и правосудие не достигает до нас; ждем света, и вот тьма,~--- озарения, и ходим во мраке.
\vs Isa 59:10 Осязаем, как слепые стену, и, как без глаз, ходим ощупью; спотыкаемся в полдень, как в сумерки, между живыми~--- как мертвые.
\vs Isa 59:11 Все мы ревем, как медведи, и стонем, как голуби; ожидаем суда, и нет \bibemph{его},~--- спасения, но оно далеко от нас.
\vs Isa 59:12 Ибо преступления наши многочисленны пред Тобою, и грехи наши свидетельствуют против нас; ибо преступления наши с нами, и беззакония наши мы знаем.
\vs Isa 59:13 Мы изменили и солгали пред Господом, и отступили от Бога нашего; говорили клевету и измену, зачинали и рождали из сердца лживые слова.
\vs Isa 59:14 И суд отступил назад, и правда стала вдали, ибо истина преткнулась на площади, и честность не может войти.
\vs Isa 59:15 И не стало истины, и удаляющийся от зла подвергается оскорблению. И Господь увидел это, и противно было очам Его, что нет суда.
\vs Isa 59:16 И видел, что нет человека, и дивился, что нет заступника; и помогла Ему мышца Его, и правда Его поддержала Его.
\vs Isa 59:17 И Он возложил на Себя правду, как броню, и шлем спасения на главу Свою; и облекся в ризу мщения, как в одежду, и покрыл Себя ревностью, как плащом.
\vs Isa 59:18 По мере возмездия, по этой мере Он воздаст противникам Своим~--- яростью, врагам Своим~--- местью, островам воздаст должное.
\vs Isa 59:19 И убоятся имени Господа на западе и славы Его~--- на восходе солнца. Если враг придет как река, дуновение Господа прогонит его.
\vs Isa 59:20 И придет Искупитель Сиона и \bibemph{сынов} Иакова, обратившихся от нечестия, говорит Господь.
\vs Isa 59:21 И вот завет Мой с ними, говорит Господь: Дух Мой, Который на тебе, и слова Мои, которые вложил Я в уста твои, не отступят от уст твоих и от уст потомства твоего, и от уст потомков потомства твоего, говорит Господь, отныне и до века.
\vs Isa 60:1 Восстань, светись, [Иерусалим], ибо пришел свет твой, и слава Господня взошла над тобою.
\vs Isa 60:2 Ибо вот, тьма покроет землю, и мрак~--- народы; а над тобою воссияет Господь, и слава Его явится над тобою.
\vs Isa 60:3 И придут народы к свету твоему, и цари~--- к восходящему над тобою сиянию.
\vs Isa 60:4 Возведи очи твои и посмотри вокруг: все они собираются, идут к тебе; сыновья твои издалека идут и дочерей твоих на руках несут.
\vs Isa 60:5 Тогда увидишь, и возрадуешься, и затрепещет и расширится сердце твое, потому что богатство моря обратится к тебе, достояние народов придет к тебе.
\vs Isa 60:6 Множество верблюдов покроет тебя~--- дромадеры из Мадиама и Ефы; все они из Савы придут, принесут золото и ладан и возвестят славу Господа.
\vs Isa 60:7 Все овцы Кидарские будут собраны к тебе; овны Неваиофские послужат тебе: взойдут на алтарь Мой жертвою благоугодною, и Я прославлю дом славы Моей.
\vs Isa 60:8 Кто это летят, как облака, и как голуби~--- к голубятням своим?
\vs Isa 60:9 Так, Меня ждут острова и впереди их~--- корабли Фарсисские, чтобы перевезти сынов твоих издалека и с ними серебро их и золото их, во имя Господа Бога твоего и Святаго Израилева, потому что Он прославил тебя.
\vs Isa 60:10 Тогда сыновья иноземцев будут строить стены твои, и цари их~--- служить тебе; ибо во гневе Моем Я поражал тебя, но в благоволении Моем буду милостив к тебе.
\vs Isa 60:11 И будут всегда отверсты врата твои, не будут затворяться ни днем ни ночью, чтобы приносимо было к тебе достояние народов и приводимы были цари их.
\vs Isa 60:12 Ибо народ и царства, которые не захотят служить тебе,~--- погибнут, и такие народы совершенно истребятся.
\vs Isa 60:13 Слава Ливана придет к тебе, кипарис и певг и вместе кедр, чтобы украсить место святилища Моего, и Я прославлю подножие ног Моих.
\vs Isa 60:14 И придут к тебе с покорностью сыновья угнетавших тебя, и падут к стопам ног твоих все, презиравшие тебя, и назовут тебя городом Господа, Сионом Святаго Израилева.
\vs Isa 60:15 Вместо того, что ты был оставлен и ненавидим, так что никто не проходил чрез \bibemph{тебя}, Я соделаю тебя величием навеки, радостью в роды родов.
\vs Isa 60:16 Ты будешь насыщаться молоком народов, и груди царские сосать будешь, и узнаешь, что Я Господь~--- Спаситель твой и Искупитель твой, Сильный Иаковлев.
\vs Isa 60:17 Вместо меди буду доставлять тебе золото, и вместо железа серебро, и вместо дерева медь, и вместо камней железо; и поставлю правителем твоим мир и надзирателями твоими~--- правду.
\vs Isa 60:18 Не слышно будет более насилия в земле твоей, опустошения и разорения~--- в пределах твоих; и будешь называть стены твои спасением и ворота твои~--- славою.
\vs Isa 60:19 Не будет уже солнце служить тебе светом дневным, и сияние луны~--- светить тебе; но Господь будет тебе вечным светом, и Бог твой~--- славою твоею.
\vs Isa 60:20 Не зайдет уже солнце твое, и луна твоя не сокроется, ибо Господь будет для тебя вечным светом, и окончатся дни сетования твоего.
\vs Isa 60:21 И народ твой весь будет праведный, на веки наследует землю,~--- отрасль насаждения Моего, дело рук Моих, к прославлению Моему.
\vs Isa 60:22 От малого произойдет тысяча, и от самого слабого~--- сильный народ. Я, Господь, ускорю совершить это в свое время.
\vs Isa 61:1 Дух Господа Бога на Мне, ибо Господь помазал Меня благовествовать нищим, послал Меня исцелять сокрушенных сердцем, проповедовать пленным освобождение и узникам открытие темницы\fns{По переводу 70-ти: слепым прозрение.},
\vs Isa 61:2 проповедовать лето Господне благоприятное и день мщения Бога нашего, утешить всех сетующих,
\vs Isa 61:3 возвестить сетующим на Сионе, что им вместо пепла дастся украшение, вместо плача~--- елей радости, вместо унылого духа~--- славная одежда, и назовут их сильными правдою, насаждением Господа во славу Его.
\vs Isa 61:4 И застроят пустыни вековые, восстановят древние развалины и возобновят города разоренные, остававшиеся в запустении с давних родов.
\vs Isa 61:5 И придут иноземцы и будут пасти стада ваши; и сыновья чужестранцев \bibemph{будут} вашими земледельцами и вашими виноградарями.
\vs Isa 61:6 А вы будете называться священниками Господа, служителями Бога нашего будут именовать вас; будете пользоваться достоянием народов и славиться славою их.
\vs Isa 61:7 За посрамление вам будет вдвое; за поношение они будут радоваться своей доле, потому что в земле своей вдвое получат; веселие вечное будет у них.
\vs Isa 61:8 Ибо Я, Господь, люблю правосудие, ненавижу грабительство с насилием, и воздам награду им по истине, и завет вечный поставлю с ними;
\vs Isa 61:9 и будет известно между народами семя их, и потомство их~--- среди племен; все видящие их познают, что они семя, благословенное Господом.
\vs Isa 61:10 Радостью буду радоваться о Господе, возвеселится душа моя о Боге моем; ибо Он облек меня в ризы спасения, одеждою правды одел меня, как на жениха возложил венец и, как невесту, украсил убранством.
\vs Isa 61:11 Ибо, как земля производит растения свои, и как сад произращает посеянное в нем, так Господь Бог проявит правду и славу пред всеми народами.
\vs Isa 62:1 Не умолкну ради Сиона, и ради Иерусалима не успокоюсь, доколе не взойдет, как свет, правда его и спасение его~--- как горящий светильник.
\vs Isa 62:2 И увидят народы правду твою и все цари~--- славу твою, и назовут тебя новым именем, которое нарекут уста Господа.
\vs Isa 62:3 И будешь венцом славы в руке Господа и царскою диадемою на длани Бога твоего.
\vs Isa 62:4 Не будут уже называть тебя <<оставленным>>, и землю твою не будут более называть <<пустынею>>, но будут называть тебя: <<Мое благоволение к нему>>, а землю твою~--- <<замужнею>>, ибо Господь благоволит к тебе, и земля твоя сочетается.
\vs Isa 62:5 Как юноша сочетается с девою, так сочетаются с тобою сыновья твои; и \bibemph{как} жених радуется о невесте, так будет радоваться о тебе Бог твой.
\vs Isa 62:6 На стенах твоих, Иерусалим, Я поставил сторожей, \bibemph{которые} не будут умолкать ни днем, ни ночью. О, вы, напоминающие о Господе! не умолкайте,~---
\vs Isa 62:7 не умолкайте пред Ним, доколе Он не восстановит и доколе не сделает Иерусалима славою на земле.
\vs Isa 62:8 Господь поклялся десницею Своею и крепкою мышцею Своею: не дам зерна твоего более в пищу врагам твоим, и сыновья чужих не будут пить вина твоего, над которым ты трудился;
\vs Isa 62:9 но собирающие его будут есть его и славить Господа, и обирающие виноград будут пить \bibemph{вино} его во дворах святилища Моего.
\vs Isa 62:10 Проход\acc{и}те, проход\acc{и}те в ворота, приготовляйте путь народу! Ровняйте, ровняйте дорогу, убирайте камни, поднимите знамя для народов!
\vs Isa 62:11 Вот, Господь объявляет до конца земли: скажите дщери Сиона: грядет Спаситель твой; награда Его с Ним и воздаяние Его пред Ним.
\vs Isa 62:12 И назовут их народом святым, искупленным от Господа, а тебя назовут взысканным городом, неоставленным.
\vs Isa 63:1 Кто это идет от Едома, в червленых ризах от Восора, столь величественный в Своей одежде, выступающий в полноте силы Своей? <<Я~--- изрекающий правду, сильный, чтобы спасать>>.
\vs Isa 63:2 Отчего же одеяние Твое красно, и ризы у Тебя, как у топтавшего в точиле?
\vs Isa 63:3 <<Я топтал точило один, и из народов никого не было со Мною; и Я топтал их во гневе Моем и попирал их в ярости Моей; кровь их брызгала на ризы Мои, и Я запятнал все одеяние Свое;
\vs Isa 63:4 ибо день мщения~--- в сердце Моем, и год Моих искупленных настал.
\vs Isa 63:5 Я смотрел, и не было помощника; дивился, что не было поддерживающего; но помогла Мне мышца Моя, и ярость Моя~--- она поддержала Меня:
\vs Isa 63:6 и попрал Я народы во гневе Моем, и сокрушил их в ярости Моей, и вылил на землю кровь их>>.
\vs Isa 63:7 Воспомяну милости Господни и славу Господню за все, что Господь даровал нам, и великую благость \bibemph{Его} к дому Израилеву, какую оказал Он ему по милосердию Своему и по множеству щедрот Своих.
\rsbpar\vs Isa 63:8 Он сказал: <<подлинно они народ Мой, дети, которые не солгут>>, и Он был для них Спасителем.
\vs Isa 63:9 Во всякой скорби их Он не оставлял их, и Ангел лица Его спасал их; по любви Своей и благосердию Своему Он искупил их, взял и носил их во все дни древние.
\vs Isa 63:10 Но они возмутились и огорчили Святаго Духа Его; поэтому Он обратился в неприятеля их: Сам воевал против них.
\vs Isa 63:11 Тогда народ Его вспомнил древние дни, Моисеевы: где Тот, Который вывел их из моря с пастырем овец Своих? где Тот, Который вложил в сердце его Святаго Духа Своего,
\vs Isa 63:12 Который вел Моисея за правую руку величественною мышцею Своею, разделил пред ними воды, чтобы сделать Себе вечное имя,
\vs Isa 63:13 Который вел их чрез бездны, как коня по степи, \bibemph{и} они не спотыкались?
\vs Isa 63:14 Как стадо сходит в долину, Дух Господень вел их к покою. Так вел Ты народ Твой, чтобы сделать Себе славное имя.
\vs Isa 63:15 Призри с небес и посмотри из жилища святыни Твоей и славы Твоей: где ревность Твоя и могущество Твое?~--- благоутробие Твое и милости Твои ко мне удержаны.
\vs Isa 63:16 Только Ты~--- Отец наш; ибо Авраам не узнаёт нас, и Израиль не признаёт нас своими; Ты, Господи, Отец наш, от века имя Твое: <<Искупитель наш>>.
\vs Isa 63:17 Для чего, Господи, Ты попустил нам совратиться с путей Твоих, ожесточиться сердцу нашему, чтобы не бояться Тебя? обратись ради рабов Твоих, ради колен наследия Твоего.
\vs Isa 63:18 Короткое время владел им народ святыни Твоей: враги наши попрали святилище Твое.
\vs Isa 63:19 Мы сделались такими, над которыми Ты как бы никогда не владычествовал и над которыми не именовалось имя Твое.
\vs Isa 64:1 О, если бы Ты расторг небеса \bibemph{и} сошел! горы растаяли бы от лица Твоего,
\vs Isa 64:2 как от плавящего огня, как от кипятящего воду, чтобы имя Твое сделать известным врагам Твоим; от лица Твоего содрогнулись бы народы.
\vs Isa 64:3 Когда Ты совершал страшные дела, нами неожиданные, и нисходил,~--- горы таяли от лица Твоего.
\vs Isa 64:4 Ибо от века не слыхали, не внимали ухом, и никакой глаз не видал другого бога, кроме Тебя, который столько сделал бы для надеющихся на него.
\vs Isa 64:5 Ты милостиво встречал радующегося и делающего правду, поминающего Тебя на путях Твоих. Но вот, Ты прогневался, потому что мы издавна грешили; и как же мы будем спасены?
\vs Isa 64:6 Все мы сделались~--- как нечистый, и вся праведность наша~--- как запачканная одежда; и все мы поблекли, как лист, и беззакония наши, как ветер, уносят нас.
\vs Isa 64:7 И нет призывающего имя Твое, который положил бы крепко держаться за Тебя; поэтому Ты сокрыл от нас лице Твое и оставил нас погибать от беззаконий наших.
\vs Isa 64:8 Но ныне, Господи, Ты~--- Отец наш; мы~--- глина, а Ты~--- образователь наш, и все мы~--- дело руки Твоей.
\vs Isa 64:9 Не гневайся, Господи, без меры, и не вечно помни беззаконие. Воззри же: мы все народ Твой.
\vs Isa 64:10 Города святыни Твоей сделались пустынею; пустынею стал Сион; Иерусалим опустошен.
\vs Isa 64:11 Дом освящения нашего и славы нашей, где отцы наши прославляли Тебя, сожжен огнем, и все драгоценности наши разграблены.
\vs Isa 64:12 После этого будешь ли еще удерживаться, Господи, будешь ли молчать и карать нас без меры?
\vs Isa 65:1 Я открылся не вопрошавшим обо Мне; Меня нашли не искавшие Меня. <<Вот Я! вот Я!>>~--- говорил Я народу, не именовавшемуся именем Моим.
\vs Isa 65:2 Всякий день простирал Я руки Мои к народу непокорному, ходившему путем недобрым, по своим помышлениям,~---
\vs Isa 65:3 к народу, который постоянно оскорбляет Меня в лице, приносит жертвы в рощах и сожигает фимиам на черепках,
\vs Isa 65:4 сидит в гробах и ночует в пещерах; ест свиное мясо, и мерзкое варево в сосудах у него;
\vs Isa 65:5 который говорит: <<остановись, не подходи ко мне, потому что я свят для тебя>>. Они~--- дым для обоняния Моего, огонь, горящий всякий день.
\vs Isa 65:6 Вот что написано пред лицем Моим: не умолчу, но воздам, воздам в недро их
\vs Isa 65:7 беззакония ваши, говорит Господь, и вместе беззакония отцов ваших, которые воскуряли фимиам на горах, и на холмах поносили Меня; и отмерю в недра их прежние деяния их.
\rsbpar\vs Isa 65:8 Так говорит Господь: когда в виноградной кисти находится сок, тогда говорят: <<не повреди ее, ибо в ней благословение>>; то же сделаю Я и ради рабов Моих, чтобы не всех погубить.
\vs Isa 65:9 И произведу от Иакова семя, и от Иуды наследника гор Моих, и наследуют это избранные Мои, и рабы Мои будут жить там.
\vs Isa 65:10 И будет Сарон пастбищем для овец и долина Ахор~--- местом отдыха для волов народа Моего, который взыскал Меня.
\vs Isa 65:11 А вас, которые оставили Господа, забыли святую гору Мою, приготовляете трапезу для Гада и растворяете полную чашу для Мени\fns{Гад (или Ваал-Гад) и Мени (Мануфи)~--- имена божеств солнца и луны.},~---
\vs Isa 65:12 вас обрекаю Я мечу, и все вы прекл\acc{о}нитесь на заклание: потому что Я звал, и вы не отвечали; говорил, и вы не слушали, но делали злое в очах Моих и избирали то, что было неугодно Мне.
\vs Isa 65:13 Посему так говорит Господь Бог: вот, рабы Мои будут есть, а вы будете голодать; рабы Мои будут пить, а вы будете томиться жаждою;
\vs Isa 65:14 рабы Мои будут веселиться, а вы будете в стыде; рабы Мои будут петь от сердечной радости, а вы будете кричать от сердечной скорби и рыдать от сокрушения духа.
\vs Isa 65:15 И оставите имя ваше избранным Моим для проклятия; и убьет тебя Господь Бог, а рабов Своих назовет иным именем,
\vs Isa 65:16 которым кто будет благословлять себя на земле, будет благословляться Богом истины; и кто будет клясться на земле, будет клясться Богом истины,~--- потому что прежние скорби будут забыты и сокрыты от очей Моих.
\vs Isa 65:17 Ибо вот, Я творю новое небо и новую землю, и прежние уже не будут воспоминаемы и не придут на сердце.
\vs Isa 65:18 А вы будете веселиться и радоваться вовеки о том, что Я творю: ибо вот, Я творю Иерусалим весельем и народ его радостью.
\vs Isa 65:19 И буду радоваться о Иерусалиме и веселиться о народе Моем; и не услышится в нем более голос плача и голос вопля.
\vs Isa 65:20 Там не будет более малолетнего и старца, который не достигал бы полноты дней своих; ибо столетний будет умирать юношею, но столетний грешник будет проклинаем.
\vs Isa 65:21 И будут строить домы и жить в них, и насаждать виноградники и есть плоды их.
\vs Isa 65:22 Не будут строить, чтобы другой жил, не будут насаждать, чтобы другой ел; ибо дни народа Моего будут, как дни дерева, и избранные Мои долго будут пользоваться изделием рук своих.
\vs Isa 65:23 Не будут трудиться напрасно и рождать детей на г\acc{о}ре; ибо будут семенем, благословенным от Господа, и потомки их с ними.
\vs Isa 65:24 И будет, прежде нежели они воззовут, Я отвечу; они еще будут говорить, и Я уже услышу.
\vs Isa 65:25 Волк и ягненок будут пастись вместе, и лев, как вол, будет есть солому, а для змея прах будет пищею: они не будут причинять зла и вреда на всей святой горе Моей, говорит Господь.
\vs Isa 66:1 Так говорит Господь: небо~--- престол Мой, а земля~--- подножие ног Моих; где же построите вы дом для Меня, и где место покоя Моего?
\vs Isa 66:2 Ибо все это соделала рука Моя, и все сие было, говорит Господь. А вот на кого Я призрю: на смиренного и сокрушенного духом и на трепещущего пред словом Моим.
\vs Isa 66:3 [Беззаконник же,] заколающий вола~--- то же, что убивающий человека; приносящий агнца в жертву~--- то же, что задушающий пса; приносящий семидал~--- то же, что приносящий свиную кровь; воскуряющий фимиам [в память]~--- то же, что молящийся идолу; и как они избрали собственные свои пути, и душа их находит удовольствие в мерзостях их,~---
\vs Isa 66:4 так и Я употреблю их обольщение и наведу на них ужасное для них: потому что Я звал, и не было отвечающего, говорил, и они не слушали, а делали злое в очах Моих и избирали то, что неугодно Мне.
\rsbpar\vs Isa 66:5 Выслушайте слово Господа, трепещущие пред словом Его: ваши братья, ненавидящие вас и изгоняющие вас за имя Мое, говорят: <<пусть явит Себя в славе Господь, и мы посмотрим на веселье ваше>>. Но они будут постыжены.
\vs Isa 66:6 Вот, шум из города, голос из храма, голос Господа, воздающего возмездие врагам Своим.
\vs Isa 66:7 Еще не мучилась родами, а родила; прежде нежели наступили боли ее, разрешилась сыном.
\vs Isa 66:8 Кто слыхал таковое? кто видал подобное этому? возникала ли страна в один день? рождался ли народ в один раз, как Сион, едва начал родами мучиться, родил сынов своих?
\vs Isa 66:9 Доведу ли Я до родов, и не дам родить? говорит Господь. Или, давая силу родить, заключу ли \bibemph{утробу}? говорит Бог твой.
\vs Isa 66:10 Возвеселитесь с Иерусалимом и радуйтесь о нем, все любящие его! возрадуйтесь с ним радостью, все сетовавшие о нем,
\vs Isa 66:11 чтобы вам питаться и насыщаться от сосцов утешений его, упиваться и наслаждаться преизбытком славы его.
\vs Isa 66:12 Ибо так говорит Господь: вот, Я направляю к нему мир как реку, и богатство народов~--- как разливающийся поток для наслаждения вашего; на руках будут носить вас и на коленях ласкать.
\vs Isa 66:13 Как утешает кого-либо мать его, так утешу Я вас, и вы будете утешены в Иерусалиме.
\vs Isa 66:14 И увидите это, и возрадуется сердце ваше, и кости ваши расцветут, как молодая зелень, и откроется рука Господа рабам Его, а на врагов Своих Он разгневается.
\vs Isa 66:15 Ибо вот, придет Господь в огне, и колесницы Его~--- как вихрь, чтобы излить гнев Свой с яростью и прещение Свое с пылающим огнем.
\vs Isa 66:16 Ибо Господь с огнем и мечом Своим произведет суд над всякою плотью, и много будет пораженных Господом.
\vs Isa 66:17 Те, которые освящают и очищают себя в рощах, один за другим, едят свиное мясо и мерзость и мышей,~--- все погибнут, говорит Господь.
\vs Isa 66:18 Ибо Я \bibemph{знаю} деяния их и мысли их; и вот, приду собрать все народы и языки, и они придут и увидят славу Мою.
\vs Isa 66:19 И положу на них знамение, и пошлю из спасенных от них к народам: в Фарсис, к Пулу и Луду, к натягивающим лук, к Тубалу и Явану, на дальние острова, которые не слышали обо Мне и не видели славы Моей: и они возвестят народам славу Мою
\vs Isa 66:20 и представят всех братьев ваших от всех народов в дар Господу на конях и колесницах, и на носилках, и на мулах, и на быстрых верблюдах, на святую гору Мою, в Иерусалим, говорит Господь,~--- подобно тому, как сыны Израилевы приносят дар в дом Господа в чистом сосуде.
\vs Isa 66:21 Из них буду брать также в священники и левиты, говорит Господь.
\vs Isa 66:22 Ибо, как новое небо и новая земля, которые Я сотворю, всегда будут пред лицем Моим, говорит Господь, так будет и семя ваше и имя ваше.
\vs Isa 66:23 Тогда из месяца в месяц и из субботы в субботу будет приходить всякая плоть пред лице Мое на поклонение, говорит Господь.
\vs Isa 66:24 И будут выходить и увидят трупы людей, отступивших от Меня: ибо червь их не умрет, и огонь их не угаснет; и будут они мерзостью для всякой плоти.

\bibbookdescr{Jer}{
  inline={\LARGE Книга\\\Huge Пророка Иеремии},
  toc={Иеремия},
  bookmark={Иеремия},
  header={Иеремия},
  %headerleft={},
  %headerright={},
  abbr={Иер}
}
\vs Jer 1:1 Слова Иеремии, сына Хелкиина, из священников в Анафофе, в земле Вениаминовой,
\vs Jer 1:2 к которому было слово Господне во дни Иосии, сына Амонова, царя Иудейского, в тринадцатый год царствования его,
\vs Jer 1:3 и также во дни Иоакима, сына Иосиина, царя Иудейского, до конца одиннадцатого года Седекии, сына Иосиина, царя Иудейского, до переселения Иерусалима в пятом месяце.
\rsbpar\vs Jer 1:4 И было ко мне слово Господне:
\vs Jer 1:5 прежде нежели Я образовал тебя во чреве, Я познал тебя, и прежде нежели ты вышел из утробы, Я освятил тебя: пророком для народов поставил тебя.
\vs Jer 1:6 А я сказал: о, Господи Боже! я не умею говорить, ибо я еще молод.
\vs Jer 1:7 Но Господь сказал мне: не говори: <<я молод>>; ибо ко всем, к кому пошлю тебя, пойдешь, и все, что повелю тебе, скажешь.
\vs Jer 1:8 Не бойся их; ибо Я с тобою, чтобы избавлять тебя, сказал Господь.
\vs Jer 1:9 И простер Господь руку Свою, и коснулся уст моих, и сказал мне Господь: вот, Я вложил слова Мои в уста твои.
\vs Jer 1:10 Смотри, Я поставил тебя в сей день над народами и царствами, чтобы искоренять и разорять, губить и разрушать, созидать и насаждать.
\rsbpar\vs Jer 1:11 И было слово Господне ко мне: что видишь ты, Иеремия? Я сказал: вижу жезл миндального дерева.
\vs Jer 1:12 Господь сказал мне: ты верно видишь; ибо Я бодрствую над словом Моим, чтоб оно скоро исполнилось.
\rsbpar\vs Jer 1:13 И было слово Господне ко мне в другой раз: что видишь ты? Я сказал: вижу поддуваемый ветром кипящий котел, и лицо его со стороны севера.
\vs Jer 1:14 И сказал мне Господь: от севера откроется бедствие на всех обитателей сей земли.
\vs Jer 1:15 Ибо вот, Я призову все племена царств северных, говорит Господь, и придут они, и поставят каждый престол свой при входе в ворота Иерусалима, и вокруг всех стен его, и во всех городах Иудейских.
\vs Jer 1:16 И произнесу над ними суды Мои за все беззакония их, за то, что они оставили Меня, и воскуряли фимиам чужеземным богам и поклонялись делам рук своих.
\vs Jer 1:17 А ты препояшь чресла твои, и встань, и скажи им все, что Я повелю тебе; не малодушествуй пред ними, чтобы Я не поразил тебя в глазах их.
\vs Jer 1:18 И вот, Я поставил тебя ныне укрепленным городом и железным столбом и медною стеною на всей этой земле, против царей Иуды, против князей его, против священников его и против народа земли сей.
\vs Jer 1:19 Они будут ратовать против тебя, но не превозмогут тебя; ибо Я с тобою, говорит Господь, чтобы избавлять тебя.
\vs Jer 2:1 И было слово Господне ко мне:
\vs Jer 2:2 иди и возгласи в уши \bibemph{дщери} Иерусалима: так говорит Господь: Я вспоминаю о дружестве юности твоей, о любви твоей, когда ты была невестою, когда последовала за Мною в пустыню, в землю незасеянную.
\vs Jer 2:3 Израиль \bibemph{был} святынею Господа, начатком плодов Его; все поедавшие его были осуждаемы, бедствие постигало их, говорит Господь.
\rsbpar\vs Jer 2:4 Выслушайте слово Господне, дом Иаковлев и все роды дома Израилева!
\vs Jer 2:5 Так говорит Господь: какую неправду нашли во Мне отцы ваши, что удалились от Меня и пошли за суетою, и осуетились,
\vs Jer 2:6 и не сказали: <<где Господь, Который вывел нас из земли Египетской, вел нас по пустыне, по земле пустой и необитаемой, по земле сухой, по земле тени смертной, по которой никто не ходил и где не обитал человек?>>
\vs Jer 2:7 И Я ввел вас в землю плодоносную, чтобы вы питались плодами ее и добром ее; а вы вошли и осквернили землю Мою, и достояние Мое сделали мерзостью.
\vs Jer 2:8 Священники не говорили: <<где Господь?>>, и учители закона не знали Меня, и пастыри отпали от Меня, и пророки пророчествовали во имя Ваала и ходили во след тех, которые не помогают.
\vs Jer 2:9 Поэтому Я еще буду судиться с вами, говорит Господь, и с сыновьями сыновей ваших буду судиться.
\vs Jer 2:10 Ибо пойдите на острова Хиттимские и посмотрите, и пошлите в Кидар и разведайте прилежно, и рассмотрите: было ли \bibemph{там} что-нибудь подобное сему?
\vs Jer 2:11 переменил ли какой народ богов \bibemph{своих}, хотя они и не боги? а Мой народ променял славу свою на то, что не помогает.
\vs Jer 2:12 Подивитесь сему, небеса, и содрогнитесь, и ужаснитесь, говорит Господь.
\vs Jer 2:13 Ибо два зла сделал народ Мой: Меня, источник воды живой, оставили, и высекли себе водоемы разбитые, которые не могут держать воды.
\vs Jer 2:14 Разве Израиль раб? или он домочадец? почему он сделался добычею?
\vs Jer 2:15 Зарыкали на него молодые львы, подали голос свой и сделали землю его пустынею; города его сожжены, без жителей.
\vs Jer 2:16 И сыновья Мемфиса и Тафны объели темя твое.
\vs Jer 2:17 Не причинил ли ты себе это тем, что оставил Господа Бога твоего в то время, когда Он путеводил тебя?
\vs Jer 2:18 И ныне для чего тебе путь в Египет, чтобы пить воду из Нила? и для чего тебе путь в Ассирию, чтобы пить воду из реки ее?
\vs Jer 2:19 Накажет тебя нечестие твое, и отступничество твое обличит тебя; итак познай и размысли, как худо и горько то, что ты оставил Господа Бога твоего и страха Моего нет в тебе, говорит Господь Бог Саваоф.
\vs Jer 2:20 Ибо издавна Я сокрушил ярмо твое, разорвал узы твои, и ты говорил: <<не буду служить \bibemph{идолам>>}, а между тем на всяком высоком холме и под всяким ветвистым деревом ты блудодействовал.
\vs Jer 2:21 Я насадил тебя \bibemph{как} благородную лозу,~--- самое чистое семя; как же ты превратилась у Меня в дикую отрасль чужой лозы?
\vs Jer 2:22 Посему, хотя бы ты умылся мылом и много употребил на себя щелоку, нечестие твое отмечено предо Мною, говорит Господь Бог.
\vs Jer 2:23 Как можешь ты сказать: <<я не осквернил себя, я не ходил во след Ваала?>> Посмотри на поведение твое в долине, познай, что делала ты, резвая верблюдица, рыщущая по путям твоим?
\vs Jer 2:24 Привыкшую к пустыне дикую ослицу, в страсти души своей глотающую воздух, кто может удержать? Все, ищущие ее, не утомятся: в ее месяце они найдут ее.
\vs Jer 2:25 Не давай ногам твоим истаптывать обувь, и гортани твоей~--- томиться жаждою. Но ты сказал: <<не надейся, нет! ибо люблю чужих и буду ходить во след их>>.
\vs Jer 2:26 Как вор, когда поймают его, бывает осрамлен, так осрамил себя дом Израилев: они, цари их, князья их, и священники их, и пророки их,~---
\vs Jer 2:27 говоря дереву: <<ты мой отец>>, и камню: <<ты родил меня>>; ибо они оборотили ко Мне спину, а не лице; а во время бедствия своего будут говорить: <<встань и спаси нас!>>
\vs Jer 2:28 Где же боги твои, которых ты сделал себе?~--- пусть они встанут, если могут спасти тебя во время бедствия твоего; ибо сколько у тебя городов, столько и богов у тебя, Иуда.
\vs Jer 2:29 Для чего вам состязаться со Мною?~--- все вы [нечестиво поступали и] согрешали против Меня, говорит Господь.
\vs Jer 2:30 Вотще поражал Я детей ваших: они не приняли вразумления; пророков ваших поядал меч ваш, как истребляющий лев [, и вы не убоялись].
\vs Jer 2:31 О, род! внемлите вы слову Господню: был ли Я пустынею для Израиля? был ли Я страною мрака? Зачем же народ Мой говорит: <<мы сами себе господа; мы уже не придем к Тебе>>?
\vs Jer 2:32 Забывает ли девица украшение свое и невеста~--- наряд свой? а народ Мой забыл Меня,~--- нет числа дням.
\vs Jer 2:33 Как искусно направляешь ты пути твои, чтобы снискать любовь! и для того даже к преступлениям приспособляла ты пути твои.
\vs Jer 2:34 Даже на полах одежды твоей находится кровь людей бедных, невинных, которых ты не застала при взломе, и, несмотря на все это,
\vs Jer 2:35 говоришь: <<так как я невинна, то верно гнев Его отвратится от меня>>. Вот, Я буду судиться с тобою за то, что говоришь: <<я не согрешила>>.
\vs Jer 2:36 Зачем ты так много бродишь, меняя путь твой? Ты так же будешь посрамлена и Египтом, как была посрамлена Ассириею;
\vs Jer 2:37 и от него ты выйдешь, положив руки на голову, потому что отверг Господь надежды твои, и не будешь иметь с ними успеха.
\vs Jer 3:1 Говорят: <<если муж отпустит жену свою, и она отойдет от него и сделается женою другого мужа, то может ли она возвратиться к нему? Не осквернилась ли бы этим страна та?>> А ты со многими любовниками блудодействовала,~--- и однако же возвратись ко Мне, говорит Господь.
\vs Jer 3:2 Подними глаза твои на высоты и посмотри, где не блудодействовали с тобою? У дороги сидела ты для них, как Аравитянин в пустыне, и осквернила землю блудом твоим и лукавством твоим.
\vs Jer 3:3 За то были удержаны дожди, и не было дождя позднего; но у тебя был лоб блудницы, ты отбросила стыд.
\vs Jer 3:4 Не будешь ли ты отныне взывать ко Мне: <<Отец мой! Ты был путеводителем юности моей!
\vs Jer 3:5 Неужели всегда будет Он во гневе? и неужели вечно будет удерживать его в Себе?>> Вот, что говоришь ты, а делаешь зло и преуспеваешь в нем.
\rsbpar\vs Jer 3:6 Господь сказал мне во дни Иосии царя: видел ли ты, что делала отступница, дочь Израиля? Она ходила на всякую высокую гору и под всякое ветвистое дерево и там блудодействовала.
\vs Jer 3:7 И после того, как она все это делала, Я говорил: <<возвратись ко Мне>>; но она не возвратилась; и видела \bibemph{это} вероломная сестра ее Иудея.
\vs Jer 3:8 И Я видел, что, когда за все прелюбодейные действия отступницы, дочери Израиля, Я отпустил ее и дал ей разводное письмо, вероломная сестра ее Иудея не убоялась, а пошла и сама блудодействовала.
\vs Jer 3:9 И явным блудодейством она осквернила землю, и прелюбодействовала с камнем и деревом.
\vs Jer 3:10 Но при всем этом вероломная сестра ее Иудея не обратилась ко Мне всем сердцем своим, а только притворно, говорит Господь.
\vs Jer 3:11 И сказал мне Господь: отступница, \bibemph{дочь} Израилева, оказалась правее, нежели вероломная Иудея.
\vs Jer 3:12 Иди и провозгласи слова сии к северу, и скажи: возвратись, отступница, \bibemph{дочь} Израилева, говорит Господь. Я не изолью на вас гнева Моего; ибо Я милостив, говорит Господь,~--- не вечно буду негодовать.
\vs Jer 3:13 Признай только вину твою: ибо ты отступила от Господа Бога твоего и распутствовала с чужими под всяким ветвистым деревом, а гласа Моего вы не слушали, говорит Господь.
\vs Jer 3:14 Возвратитесь, дети-отступники, говорит Господь, потому что Я сочетался с вами, и возьму вас по одному из города, по два из племени, и приведу вас на Сион.
\vs Jer 3:15 И дам вам пастырей по сердцу Моему, которые будут пасти вас с знанием и благоразумием.
\vs Jer 3:16 И будет, когда вы размножитесь и сделаетесь многоплодными на земле, в те дни, говорит Господь, не будут говорить более: <<ковчег завета Господня>>; он и на ум не придет, и не вспомнят о нем, и не будут приходить к нему, и его уже не будет.
\vs Jer 3:17 В то время назовут Иерусалим престолом Господа; и все народы ради имени Господа соберутся в Иерусалим и не будут более поступать по упорству злого сердца своего.
\vs Jer 3:18 В те дни придет дом Иудин к дому Израилеву, и пойдут вместе из земли северной в землю, которую Я дал в наследие отцам вашим.
\vs Jer 3:19 И говорил Я: как поставлю тебя в число детей и дам тебе вожделенную землю, прекраснейшее наследие множества народов? И сказал: ты будешь называть Меня отцом твоим и не отступишь от Меня.
\vs Jer 3:20 Но поистине, как жена вероломно изменяет другу своему, так вероломно поступили со Мною вы, дом Израилев, говорит Господь.
\vs Jer 3:21 Голос слышен на высотах, жалобный плач сынов Израиля о том, что они извратили путь свой, забыли Господа Бога своего.
\vs Jer 3:22 Возвратитесь, мятежные дети: Я исцелю вашу непокорность.~--- Вот, мы идем к Тебе, ибо Ты~--- Господь Бог наш.
\vs Jer 3:23 Поистине, напрасно надеялись мы на холмы и на множество гор; поистине, в Господе Боге нашем спасение Израилево!
\vs Jer 3:24 От юности нашей эта мерзость пожирала труды отцов наших, овец их и волов их, сыновей их и дочерей их.
\vs Jer 3:25 Мы лежим в стыде своем, и срам наш покрывает нас, потому что мы грешили пред Господом Богом нашим,~--- мы и отцы наши, от юности нашей и до сего дня, и не слушались голоса Господа Бога нашего.
\vs Jer 4:1 Если хочешь обратиться, Израиль, говорит Господь, ко Мне обратись; и если удалишь мерзости твои от лица Моего, то не будешь скитаться.
\vs Jer 4:2 И будешь клясться: <<жив Господь!>> в истине, суде и правде; и народы Им будут благословляться и Им хвалиться.
\vs Jer 4:3 Ибо так говорит Господь к мужам Иуды и Иерусалима: распашите себе новые нивы и не сейте между тернами.
\vs Jer 4:4 Обрежьте себя для Господа, и снимите крайнюю плоть с сердца вашего, мужи Иуды и жители Иерусалима, чтобы гнев Мой не открылся, как огонь, и не воспылал неугасимо по причине злых наклонностей ваших.
\vs Jer 4:5 Объявите в Иудее и разгласите в Иерусалиме, и говорите, и трубите трубою по земле; взывайте громко и говорите: <<соберитесь, и пойдем в укрепленные города>>.
\vs Jer 4:6 Выставьте знамя к Сиону, бегите, не останавливайтесь, ибо Я приведу от севера бедствие и великую гибель.
\vs Jer 4:7 Выходит лев из своей чащи, и выступает истребитель народов: он выходит из своего места, чтобы землю твою сделать пустынею; города твои будут разорены, \bibemph{останутся} без жителей.
\vs Jer 4:8 Посему препояшьтесь вретищем, плачьте и рыдайте, ибо ярость гнева Господня не отвратится от нас.
\rsbpar\vs Jer 4:9 И будет в тот день, говорит Господь, замрет сердце у царя и сердце у князей; и ужаснутся священники, и изумятся пророки.
\vs Jer 4:10 И сказал я: о, Господи Боже! Неужели Ты обольщал только народ сей и Иерусалим, говоря: <<мир будет у вас>>; а между тем меч доходит до души?
\vs Jer 4:11 В то время сказано будет народу сему и Иерусалиму: жгучий ветер несется с высот пустынных на путь дочери народа Моего, не для веяния и не для очищения;
\vs Jer 4:12 и придет ко Мне оттуда ветер сильнее сего, и Я произнесу суд над ними.
\vs Jer 4:13 Вот, поднимается он подобно облакам, и колесницы его~--- как вихрь, кони его быстрее орлов; горе нам! ибо мы будем разорены.
\vs Jer 4:14 Смой злое с сердца твоего, Иерусалим, чтобы спастись тебе: доколе будут гнездиться в тебе злочестивые мысли?
\vs Jer 4:15 Ибо уже несется голос от Дана и гибельная весть с горы Ефремовой:
\vs Jer 4:16 объявите народам, известите Иерусалим, что идут из дальней страны осаждающие и криками своими оглашают города Иудеи.
\vs Jer 4:17 Как сторожа полей, они обступают его кругом, ибо он возмутился против Меня, говорит Господь.
\vs Jer 4:18 Пути твои и деяния твои причинили тебе это; от твоего нечестия тебе так горько, что доходит до сердца твоего.
\vs Jer 4:19 Утроба моя! утроба моя! скорблю во глубине сердца моего, волнуется во мне сердце мое, не могу молчать; ибо ты слышишь, душа моя, звук трубы, тревогу брани.
\vs Jer 4:20 Беда за бедою: вся земля опустошается, внезапно разорены шатры мои, мгновенно~--- палатки мои.
\vs Jer 4:21 Долго ли мне видеть знамя, слушать звук трубы?
\vs Jer 4:22 Это оттого, что народ Мой глуп, не знает Меня: неразумные они дети, и нет у них смысла; они умны на зло, но добра делать не умеют.
\vs Jer 4:23 Смотрю на землю, и вот, она разорена и пуста,~--- на небеса, и нет на них света.
\vs Jer 4:24 Смотрю на горы, и вот, они дрожат, и все холмы колеблются.
\vs Jer 4:25 Смотрю, и вот, нет человека, и все птицы небесные разлетелись.
\vs Jer 4:26 Смотрю, и вот, Кармил~--- пустыня, и все города его разрушены от лица Господа, от ярости гнева Его.
\vs Jer 4:27 Ибо так сказал Господь: вся земля будет опустошена, но совершенного истребления не сделаю.
\vs Jer 4:28 Восплачет о сем земля, и небеса помрачатся вверху, потому что Я сказал, Я определил, и не раскаюсь в том, и не отступлю от того.
\vs Jer 4:29 От шума всадников и стрелков разбегутся все города: они уйдут в густые леса и влезут на скалы; все города будут оставлены, и не будет в них ни одного жителя.
\vs Jer 4:30 А ты, опустошенная, что станешь делать? Хотя ты одеваешься в пурпур, хотя украшаешь себя золотыми нарядами, обрисовываешь глаза твои красками, но напрасно украшаешь себя: презрели тебя любовники, они ищут души твоей.
\vs Jer 4:31 Ибо Я слышу голос как бы женщины в родах, стон как бы рождающей в первый раз, голос дочери Сиона; она стонет, простирая руки свои: <<о, горе мне! душа моя изнывает пред убийцами>>.
\vs Jer 5:1 Поход\acc{и}те по улицам Иерусалима, и посмотр\acc{и}те, и разведайте, и поищите на площадях его, не найдете ли человека, нет ли соблюдающего правду, ищущего истины? Я пощадил бы \bibemph{Иерусалим}.
\vs Jer 5:2 Хотя и говорят они: <<жив Господь!>>, но клянутся ложно.
\vs Jer 5:3 О, Господи! очи Твои не к истине ли \bibemph{обращены}? Ты поражаешь их, а они не чувствуют боли; Ты истребляешь их, а они не хотят принять вразумления; лица свои сделали они крепче камня, не хотят обратиться.
\vs Jer 5:4 И сказал я \bibemph{сам в себе}: это, может быть, бедняки; они глупы, потому что не знают пути Господня, закона Бога своего;
\vs Jer 5:5 пойду я к знатным и поговорю с ними, ибо они знают путь Господень, закон Бога своего. Но и они все сокрушили ярмо, расторгли узы.
\vs Jer 5:6 За то поразит их лев из леса, волк пустынный опустошит их, барс будет подстерегать у городов их: кто выйдет из них, будет растерзан; ибо умножились преступления их, усилились отступничества их.
\vs Jer 5:7 Как же Мне простить тебя за это? Сыновья твои оставили Меня и клянутся теми, которые не боги. Я насыщал их, а они прелюбодействовали и толпами ходили в домы блудниц.
\vs Jer 5:8 Это откормленные кони: каждый из них ржет на жену другого.
\vs Jer 5:9 Неужели Я не накажу за это? говорит Господь; и не отмстит ли душа Моя такому народу, как этот?
\vs Jer 5:10 Восход\acc{и}те на стены его и разрушайте, но не до конца; уничтожьте зубцы их, потому что они не Господни;
\vs Jer 5:11 ибо дом Израилев и дом Иудин поступили со Мною очень вероломно, говорит Господь:
\vs Jer 5:12 они солгали на Господа и сказали: <<нет Его, и беда не придет на нас, и мы не увидим ни меча, ни голода.
\vs Jer 5:13 И пророки станут ветром, и сл\acc{о}ва [Господня] нет в них; над ними самими пусть это будет>>.
\rsbpar\vs Jer 5:14 Посему так говорит Господь Бог Саваоф: за то, что вы говорите такие слова, вот, Я сделаю слова Мои в устах твоих огнем, а этот народ~--- дровами, и этот \bibemph{огонь} пожрет их.
\vs Jer 5:15 Вот, Я приведу на вас, дом Израилев, народ издалека, говорит Господь, народ сильный, народ древний, народ, которого языка ты не знаешь, и не будешь понимать, что он говорит.
\vs Jer 5:16 Колчан его~--- как открытый гроб; все они люди храбрые.
\vs Jer 5:17 И съедят они жатву твою и хлеб твой, съедят сыновей твоих и дочерей твоих, съедят овец твоих и волов твоих, съедят виноград твой и смоквы твои; разрушат мечом укрепленные города твои, на которые ты надеешься.
\vs Jer 5:18 Но и в те дни, говорит Господь, не истреблю вас до конца.
\vs Jer 5:19 И если вы скажете: <<за что Господь, Бог наш, делает нам все это?>>, то отвечай: так как вы оставили Меня и служили чужим богам в земле своей, то будете служить чужим в земле не вашей.
\rsbpar\vs Jer 5:20 Объявите это в доме Иакова и возвестите в Иудее, говоря:
\vs Jer 5:21 выслушай это, народ глупый и неразумный, у которого есть глаза, а не видит, у которого есть уши, а не слышит:
\vs Jer 5:22 Меня ли вы не боитесь, говорит Господь, предо Мною ли не трепещете? Я положил песок границею морю, вечным пределом, которого не перейдет; и хотя волны его устремляются, но превозмочь не могут; хотя они бушуют, но переступить его не могут.
\vs Jer 5:23 А у народа сего сердце буйное и мятежное; они отступили и пошли;
\vs Jer 5:24 и не сказали в сердце своем: <<убоимся Господа Бога нашего, Который дает нам дождь ранний и поздний в свое время, хранит для нас седмицы, назначенные для жатвы>>.
\vs Jer 5:25 Беззакония ваши отвратили это, и грехи ваши удалили от вас это доброе.
\vs Jer 5:26 Ибо между народом Моим находятся нечестивые: сторожат, как птицеловы, припадают к земле, ставят ловушки и уловляют людей.
\vs Jer 5:27 Как клетка, наполненная птицами, домы их полны обмана; чрез это они и возвысились и разбогатели,
\vs Jer 5:28 сделались тучны, жирны, переступили даже всякую меру во зле, не разбирают судебных дел, дел сирот; благоденствуют, и справедливому делу нищих не дают суда.
\vs Jer 5:29 Неужели Я не накажу за это? говорит Господь; и не отмстит ли душа Моя такому народу, как этот?
\vs Jer 5:30 Изумительное и ужасное совершается в сей земле:
\vs Jer 5:31 пророки пророчествуют ложь, и священники господствуют при посредстве их, и народ Мой любит это. Что же вы будете делать после всего этого?
\vs Jer 6:1 Бегите, дети Вениаминовы, из среды Иерусалима, и в Фекое трубите трубою и дайте знать огнем в Бефкареме, ибо от севера появляется беда и великая гибель.
\vs Jer 6:2 Разорю Я дочь Сиона, красивую и изнеженную.
\vs Jer 6:3 Пастухи со своими стадами придут к ней, раскинут палатки вокруг нее; каждый будет пасти свой участок.
\vs Jer 6:4 Приготовляйте против нее войну; вставайте и пойдем в полдень. Горе нам! день уже склоняется, распростираются вечерние тени.
\vs Jer 6:5 Вставайте, пойдем и ночью, и разорим чертоги ее!
\vs Jer 6:6 Ибо так говорит Господь Саваоф: рубите дерева и делайте насыпь против Иерусалима: этот город должен быть наказан; в нем всякое угнетение.
\vs Jer 6:7 Как источник извергает из себя воду, так он источает из себя зло: в нем слышно насилие и грабительство, пред лицем Моим всегда обиды и раны.
\vs Jer 6:8 Вразумись, Иерусалим, чтобы душа Моя не удалилась от тебя, чтоб Я не сделал тебя пустынею, землею необитаемою.
\rsbpar\vs Jer 6:9 Так говорит Господь Саваоф: до конца доберут остаток Израиля, как виноград; работай рукою твоею, как обиратель винограда, наполняя корзины.
\vs Jer 6:10 К кому мне говорить и кого увещевать, чтобы слушали? Вот, ухо у них необрезанное, и они не могут слушать; вот, слово Господне у них в посмеянии; оно неприятно им.
\vs Jer 6:11 Поэтому я преисполнен яростью Господнею, не могу держать ее в себе; изолью ее на детей на улице и на собрание юношей; взяты будут муж с женою, пожилой с отжившим лета.
\vs Jer 6:12 И домы их перейдут к другим, равно поля и жены; потому что Я простру руку Мою на обитателей сей земли, говорит Господь.
\vs Jer 6:13 Ибо от малого до большого, каждый из них предан корысти, и от пророка до священника~--- все действуют лживо;
\vs Jer 6:14 врачуют раны народа Моего легкомысленно, говоря: <<мир! мир!>>, а мира нет.
\vs Jer 6:15 Стыдятся ли они, делая мерзости? нет, нисколько не стыдятся и не краснеют. За то падут между падшими, и во время посещения Моего будут повержены, говорит Господь.
\rsbpar\vs Jer 6:16 Так говорит Господь: остановитесь на путях ваших и рассмотрите, и расспросите о путях древних, где путь добрый, и идите по нему, и найдете покой душам вашим. Но они сказали: <<не пойдем>>.
\vs Jer 6:17 И поставил Я стражей над вами, \bibemph{сказав}: <<слушайте звука трубы>>. Но они сказали: <<не будем слушать>>.
\vs Jer 6:18 Итак слушайте, народы, и знай, собрание, что с ними будет.
\vs Jer 6:19 Слушай, земля: вот, Я приведу на народ сей пагубу, плод помыслов их; ибо они слов Моих не слушали и закон Мой отвергли.
\vs Jer 6:20 Для чего Мне ладан, который идет из Савы, и благовонный тростник из дальней страны? Всесожжения ваши неугодны, и жертвы ваши неприятны Мне.
\vs Jer 6:21 Посему так говорит Господь: вот, Я полагаю пред народом сим преткновения, и преткнутся о них отцы и дети вместе, сосед и друг его, и погибнут.
\rsbpar\vs Jer 6:22 Так говорит Господь: вот, идет народ от страны северной, и народ великий поднимается от краев земли;
\vs Jer 6:23 держат в руках лук и копье; они жестоки и немилосерды, голос их шумит, как море, и несутся на конях, выстроены, как один человек, чтобы сразиться с тобою, дочь Сиона.
\vs Jer 6:24 Мы услышали весть о них, и руки у нас опустились, скорбь объяла нас, муки, как женщину в родах.
\vs Jer 6:25 Не выходите в поле и не ходите по дороге, ибо меч неприятелей, ужас со всех сторон.
\vs Jer 6:26 Дочь народа моего! опояшь себя вретищем и посыпь себя пеплом; сокрушайся, как бы о смерти единственного сына, горько плачь; ибо внезапно придет на нас губитель.
\vs Jer 6:27 Башнею поставил Я тебя среди народа Моего, столпом, чтобы ты знал и следил путь их.
\vs Jer 6:28 Все они~--- упорные отступники, живут клеветою; это медь и железо,~--- все они развратители.
\vs Jer 6:29 Раздувальный мех обгорел, свинец истлел от огня: плавильщик плавил напрасно, ибо злые не отделились;
\vs Jer 6:30 отверженным серебром назовут их, ибо Господь отверг их.
\vs Jer 7:1 Слово, которое было к Иеремии от Господа:
\vs Jer 7:2 стань во вратах дома Господня и провозгласи там слово сие и скажи: слушайте слово Господне, все Иудеи, входящие сими вратами на поклонение Господу.
\vs Jer 7:3 Так говорит Господь Саваоф, Бог Израилев: исправьте пути ваши и деяния ваши, и Я оставлю вас жить на сем месте.
\vs Jer 7:4 Не надейтесь на обманчивые слова: <<здесь храм Господень, храм Господень, храм Господень>>.
\vs Jer 7:5 Но если совсем исправите пути ваши и деяния ваши, если будете верно производить суд между человеком и соперником его,
\vs Jer 7:6 не будете притеснять иноземца, сироты и вдовы, и проливать невинной крови на месте сем, и не пойдете во след иных богов на беду себе,~---
\vs Jer 7:7 то Я оставлю вас жить на месте сем, на этой земле, которую дал отцам вашим в роды родов.
\vs Jer 7:8 Вот, вы надеетесь на обманчивые слова, которые не принесут вам пользы.
\vs Jer 7:9 Как! вы крадете, убиваете и прелюбодействуете, и клянетесь во лжи и кадите Ваалу, и ходите во след иных богов, которых вы не знаете,
\vs Jer 7:10 и потом приходите и становитесь пред лицем Моим в доме сем, над которым наречено имя Мое, и говорите: <<мы спасены>>, чтобы впредь делать все эти мерзости.
\vs Jer 7:11 Не соделался ли вертепом разбойников в глазах ваших дом сей, над которым наречено имя Мое? Вот, Я видел это, говорит Господь.
\vs Jer 7:12 Пойдите же на место Мое в Силом, где Я прежде назначил пребывать имени Моему, и посмотрите, что сделал Я с ним за нечестие народа Моего Израиля.
\vs Jer 7:13 И ныне, так как вы делаете все эти дела, говорит Господь, и Я говорил вам с раннего утра, а вы не слушали, и звал вас, а вы не отвечали,~---
\vs Jer 7:14 то Я так же поступлю с домом \bibemph{сим}, над которым наречено имя Мое, на который вы надеетесь, и с местом, которое Я дал вам и отцам вашим, как поступил с Силомом.
\vs Jer 7:15 И отвергну вас от лица Моего, как отверг всех братьев ваших, все семя Ефремово.
\vs Jer 7:16 Ты же не проси за этот народ и не возноси за них молитвы и прошения, и не ходатайствуй предо Мною, ибо Я не услышу тебя.
\vs Jer 7:17 Не видишь ли, что они делают в городах Иудеи и на улицах Иерусалима?
\vs Jer 7:18 Дети собирают дрова, а отцы разводят огонь, и женщины месят тесто, чтобы делать пирожки для богини неба и совершать возлияния иным богам, чтобы огорчать Меня.
\vs Jer 7:19 Но Меня ли огорчают они? говорит Господь; не себя ли самих к стыду своему?
\vs Jer 7:20 Посему так говорит Господь Бог: вот, изливается гнев Мой и ярость Моя на место сие, на людей и на скот, и на дерева полевые и на плоды земли, и возгорится и не погаснет.
\rsbpar\vs Jer 7:21 Так говорит Господь Саваоф, Бог Израилев: всесожжения ваши прилагайте к жертвам вашим и ешьте мясо;
\vs Jer 7:22 ибо отцам вашим Я не говорил и не давал им заповеди в тот день, в который Я вывел их из земли Египетской, о всесожжении и жертве;
\vs Jer 7:23 но такую заповедь дал им: <<слушайтесь гласа Моего, и Я буду вашим Богом, а вы будете Моим народом, и ходите по всякому пути, который Я заповедаю вам, чтобы вам было хорошо>>.
\vs Jer 7:24 Но они не послушали и не приклонили уха своего, и жили по внушению и упорству злого сердца своего, и стали ко Мне спиною, а не лицом.
\vs Jer 7:25 С того дня, как отцы ваши вышли из земли Египетской, до сего дня Я посылал к вам всех рабов Моих~--- пророков, посылал всякий день с раннего утра;
\vs Jer 7:26 но они не слушались Меня и не приклонили уха своего, а ожесточили выю свою, поступали хуже отцов своих.
\vs Jer 7:27 И когда ты будешь говорить им все эти слова, они тебя не послушают; и когда будешь звать их, они тебе не ответят.
\vs Jer 7:28 Тогда скажи им: вот народ, который не слушает гласа Господа Бога своего и не принимает наставления! Не стало у них истины, она отнята от уст их.
\vs Jer 7:29 Остриги волоса твои и брось, и подними плач на горах, ибо отверг Господь и оставил род, \bibemph{навлекший} гнев Его.
\vs Jer 7:30 Ибо сыновья Иуды делают злое пред очами Моими, говорит Господь; поставили мерзости свои в доме, над которым наречено имя Мое, чтобы осквернить его;
\vs Jer 7:31 и устроили высоты Тофета в долине сыновей Енномовых, чтобы сожигать сыновей своих и дочерей своих в огне, чего Я не повелевал и что Мне на сердце не приходило.
\vs Jer 7:32 За то вот, приходят дни, говорит Господь, когда не будут более называть \bibemph{место сие} Тофетом и долиною сыновей Енномовых, но долиною убийства, и в Тофете будут хоронить по недостатку места.
\vs Jer 7:33 И будут трупы народа сего пищею птицам небесным и зверям земным, и некому будет отгонять их.
\vs Jer 7:34 И прекращу в городах Иудеи и на улицах Иерусалима голос торжества и голос веселия, голос жениха и голос невесты; потому что земля эта будет пустынею.
\vs Jer 8:1 В то время, говорит Господь, выбросят кости царей Иуды, и кости князей его, и кости священников, и кости пророков, и кости жителей Иерусалима из гробов их;
\vs Jer 8:2 и раскидают их пред солнцем и луною и пред всем воинством небесным\fns{Пред звездами.}, которых они любили и которым служили и в след которых ходили, которых искали и которым поклонялись; не уберут их и не похоронят: они будут навозом на земле.
\vs Jer 8:3 И будут смерть предпочитать жизни все остальные, которые останутся от этого злого племени во всех местах, куда Я изгоню их, говорит Господь Саваоф.
\vs Jer 8:4 И скажи им: так говорит Господь: разве, упав, не встают и, совратившись с дороги, не возвращаются?
\vs Jer 8:5 Для чего этот народ, Иерусалим, находится в упорном отступничестве? они крепко держатся обмана и не хотят обратиться.
\vs Jer 8:6 Я наблюдал и слушал: не говорят они правды, никто не раскаивается в своем нечестии, никто не говорит: <<что я сделал?>>; каждый обращается на свой путь, как конь, бросающийся в сражение.
\vs Jer 8:7 И аист под небом знает свои определенные времена, и горлица, и ласточка, и журавль наблюдают время, когда им прилететь; а народ Мой не знает определения Господня.
\vs Jer 8:8 Как вы говорите: <<мы мудры, и закон Господень у нас>>? А вот, лживая трость книжников \bibemph{и его} превращает в ложь.
\vs Jer 8:9 Посрамились мудрецы, смутились и запутались в сеть: вот, они отвергли слово Господне; в чем же мудрость их?
\vs Jer 8:10 За то жен их отдам другим, поля их~--- иным владетелям; потому что все они, от малого до большого, предались корыстолюбию; от пророка до священника~--- все действуют лживо.
\vs Jer 8:11 И врачуют рану дочери народа Моего легкомысленно, говоря: <<мир, мир!>>, а мира нет.
\vs Jer 8:12 Стыдятся ли они, делая мерзости? нет, они нисколько не стыдятся и не краснеют. За то падут они между падшими; во время посещения их будут повержены, говорит Господь.
\vs Jer 8:13 До конца оберу их, говорит Господь, не останется ни одной виноградины на лозе, ни смоквы на смоковнице, и лист опадет, и что Я дал им, отойдет от них.
\vs Jer 8:14 <<Что мы сидим? собирайтесь, пойдем в укрепленные города, и там погибнем; ибо Господь Бог наш определил нас на погибель и дает нам пить воду с желчью за то, что мы грешили пред Господом>>.
\vs Jer 8:15 Ждем мира, а ничего доброго нет,~--- времени исцеления, и вот ужасы.
\vs Jer 8:16 От Дана слышен храп коней его, от громкого ржания жеребцов его дрожит вся земля; и придут и истребят землю и всё, что на ней, город и живущих в нем.
\vs Jer 8:17 Ибо вот, Я пошлю на вас змеев, василисков, против которых нет заговариванья, и они будут уязвлять вас, говорит Господь.
\vs Jer 8:18 Когда утешусь я в горести моей! сердце мое изныло во мне.
\vs Jer 8:19 Вот, слышу вопль дщери народа Моего из дальней страны: разве нет Господа на Сионе? разве нет Царя его на нем?~--- Зачем они подвигли Меня на гнев своими идолами, чужеземными, ничтожными?
\vs Jer 8:20 Прошла жатва, кончилось лето, а мы не спасены.
\vs Jer 8:21 О сокрушении дщери народа моего я сокрушаюсь, хожу мрачен, ужас объял меня.
\vs Jer 8:22 Разве нет бальзама в Галааде? разве нет там врача? Отчего же нет исцеления дщери народа моего?
\vs Jer 9:1 О, кто даст голове моей воду и глазам моим~--- источник слез! я плакал бы день и ночь о пораженных дщери народа моего.
\vs Jer 9:2 О, кто дал бы мне в пустыне пристанище путников! оставил бы я народ мой и ушел бы от них: ибо все они прелюбодеи, скопище вероломных.
\vs Jer 9:3 Как лук, напрягают язык свой для лжи, усиливаются на земле неправдою; ибо переходят от одного зла к другому, и Меня не знают, говорит Господь.
\vs Jer 9:4 Берегитесь каждый своего друга, и не доверяйте ни одному из своих братьев; ибо всякий брат ставит преткновение другому, и всякий друг разносит клеветы.
\vs Jer 9:5 Каждый обманывает своего друга, и правды не говорят: приучили язык свой говорить ложь, лукавствуют до усталости.
\vs Jer 9:6 Ты живешь среди коварства; по коварству они отрекаются знать Меня, говорит Господь.
\vs Jer 9:7 Посему так говорит Господь Саваоф: вот, Я расплавлю и испытаю их; ибо как иначе Мне поступать со дщерью народа Моего?
\vs Jer 9:8 Язык их~--- убийственная стрела, говорит коварно; устами своими говорят с ближним своим дружелюбно, а в сердце своем строят ему ковы.
\vs Jer 9:9 Неужели Я не накажу их за это? говорит Господь; не отмстит ли душа Моя такому народу, как этот?
\vs Jer 9:10 О горах подниму плач и вопль, и о степных пастбищах~--- рыдание, потому что они выжжены, так что никто там не проходит, и не слышно блеяния стад: от птиц небесных до скота~--- \bibemph{все} рассеялись, ушли.
\vs Jer 9:11 И сделаю Иерусалим грудою камней, жилищем шакалов, и города Иудеи сделаю пустынею, без жителей.
\vs Jer 9:12 Есть ли такой мудрец, который понял бы это? И к кому говорят уста Господни~--- объяснил бы, за что погибла страна и выжжена, как пустыня, так что никто не проходит \bibemph{по ней}?
\rsbpar\vs Jer 9:13 И сказал Господь: за то, что они оставили закон Мой, который Я постановил для них, и не слушали гласа Моего и не поступали по нему;
\vs Jer 9:14 а ходили по упорству сердца своего и во след Ваалов, как научили их отцы их.
\vs Jer 9:15 Посему так говорит Господь Саваоф, Бог Израилев: вот, Я накормлю их, этот народ, полынью, и напою их водою с желчью;
\vs Jer 9:16 и рассею их между народами, которых не знали ни они, ни отцы их, и пошлю вслед их меч, доколе не истреблю их.
\vs Jer 9:17 Так говорит Господь Саваоф: подумайте, и позовите плакальщиц, чтобы они пришли; пошлите за искусницами \bibemph{в этом деле}, чтобы они пришли.
\vs Jer 9:18 Пусть они поспешат и поднимут плач о нас, чтобы из глаз наших лились слезы, и с ресниц наших текла вода.
\vs Jer 9:19 Ибо голос плача слышен с Сиона: <<как мы ограблены! мы жестоко посрамлены, ибо оставляем землю, потому что разрушили жилища наши>>.
\vs Jer 9:20 Итак слушайте, женщины, слово Господа, и да внимает ухо ваше слову уст Его; и учите дочерей ваших плачу, и одна другую~--- плачевным песням.
\vs Jer 9:21 Ибо смерть входит в наши окна, вторгается в чертоги наши, чтобы истребить детей с улицы, юношей с площадей.
\vs Jer 9:22 Скажи: так говорит Господь: и будут повержены трупы людей, как навоз на поле и как снопы позади жнеца, и некому будет собрать их.
\rsbpar\vs Jer 9:23 Так говорит Господь: да не хвалится мудрый мудростью своею, да не хвалится сильный силою своею, да не хвалится богатый богатством своим.
\vs Jer 9:24 Но хвалящийся хвались тем, что разумеет и знает Меня, что Я~--- Господь, творящий милость, суд и правду на земле; ибо только это благоугодно Мне, говорит Господь.
\vs Jer 9:25 Вот, приходят дни, говорит Господь, когда Я посещу всех обрезанных и необрезанных:
\vs Jer 9:26 Египет и Иудею, и Едома и сыновей Аммоновых, и Моава и всех стригущих волосы на висках, обитающих в пустыне; ибо все эти народы необрезаны, а весь дом Израилев с необрезанным сердцем.
\vs Jer 10:1 Слушайте слово, которое Господь говорит вам, дом Израилев.
\vs Jer 10:2 Так говорит Господь: не учитесь путям язычников и не страшитесь знамений небесных, которых язычники страшатся.
\vs Jer 10:3 Ибо уставы народов~--- пустота: вырубают дерево в лесу, обделывают его руками плотника при помощи топора,
\vs Jer 10:4 покрывают серебром и золотом, прикрепляют гвоздями и молотом, чтобы не шаталось.
\vs Jer 10:5 Они~--- как обточенный столп, и не говорят; их носят, потому что ходить не могут. Не бойтесь их, ибо они не могут причинить зла, но и добра делать не в силах.
\vs Jer 10:6 Нет подобного Тебе, Господи! Ты велик, и имя Твое велико могуществом.
\vs Jer 10:7 Кто не убоится Тебя, Царь народов? ибо Тебе \bibemph{единому} принадлежит это; потому что между всеми мудрецами народов и во всех царствах их нет подобного Тебе.
\vs Jer 10:8 Все до одного они бессмысленны и глупы; пустое учение~--- это дерево.
\vs Jer 10:9 Разбитое в листы серебро привезено из Фарсиса, золото~--- из Уфаза, дело художника и рук плавильщика; одежда на них~--- гиацинт и пурпур: все это~--- дело людей искусных.
\vs Jer 10:10 А Господь Бог есть истина; Он есть Бог живый и Царь вечный. От гнева Его дрожит земля, и народы не могут выдержать негодования Его.
\vs Jer 10:11 Так говорите им: боги, которые не сотворили неба и земли, исчезнут с земли и из-под небес.
\vs Jer 10:12 Он сотворил землю силою Своею, утвердил вселенную мудростью Своею и разумом Своим распростер небеса.
\vs Jer 10:13 По гласу Его шумят воды на небесах, и Он возводит облака от краев земли, творит молнии среди дождя и изводит ветер из хранилищ Своих.
\vs Jer 10:14 Безумствует всякий человек в своем знании, срамит себя всякий плавильщик истуканом \bibemph{своим}, ибо выплавленное им есть ложь, и нет в нем духа.
\vs Jer 10:15 Это совершенная пустота, дело заблуждения; во время посещения их они исчезнут.
\vs Jer 10:16 Не такова, как их, доля Иакова; ибо \bibemph{Бог его} есть Творец всего, и Израиль есть жезл наследия Его; имя Его~--- Господь Саваоф.
\vs Jer 10:17 Убирай с земли имущество твое, имеющая сидеть в осаде;
\vs Jer 10:18 ибо так говорит Господь: вот, Я выброшу жителей сей земли на сей раз и загоню их в тесное место, чтобы схватили их.
\vs Jer 10:19 Горе мне в моем сокрушении; мучительна рана моя, но я говорю \bibemph{сам в себе}: <<подлинно, это моя скорбь, и я буду нести ее;
\vs Jer 10:20 шатер мой опустошен, и все веревки мои порваны; дети мои ушли от меня, и нет их: некому уже раскинуть шатра моего и развесить ковров моих,
\vs Jer 10:21 ибо пастыри сделались бессмысленными и не искали Господа, а потому они и поступали безрассудно, и все стадо их рассеяно>>.
\vs Jer 10:22 Несется слух: вот он идет, и большой шум от страны северной, чтобы города Иудеи сделать пустынею, жилищем шакалов.
\vs Jer 10:23 Знаю, Господи, что не в воле человека путь его, что не во власти идущего давать направление стопам своим.
\vs Jer 10:24 Наказывай меня, Господи, но по правде, не во гневе Твоем, чтобы не умалить меня.
\vs Jer 10:25 Излей ярость Твою на народы, которые не знают Тебя, и на племена, которые не призывают имени Твоего; ибо они съели Иакова, пожрали его и истребили его, и жилище его опустошили.
\vs Jer 11:1 Слово, которое было к Иеремии от Господа:
\vs Jer 11:2 слушайте слова завета сего и скажите мужам Иуды и жителям Иерусалима;
\vs Jer 11:3 и скажи им: так говорит Господь, Бог Израилев: проклят человек, который не послушает слов завета сего,
\vs Jer 11:4 который Я заповедал отцам вашим, когда вывел их из земли Египетской, из железной печи, сказав: <<слушайтесь гласа Моего и делайте все, что Я заповедаю вам,~--- и будете Моим народом, и Я буду вашим Богом,
\vs Jer 11:5 чтобы исполнить клятву, которою Я клялся отцам вашим~--- дать им землю, текущую молоком и медом, как это ныне>>. И отвечал я, сказав: аминь, Господи!
\vs Jer 11:6 И сказал мне Господь: провозгласи все сии слова в городах Иуды и на улицах Иерусалима и скажи: слушайте слов\acc{а} завета сего и исполняйте их.
\vs Jer 11:7 Ибо отцов ваших Я увещевал постоянно с того дня, как вывел их из земли Египетской, до сего дня; увещевал их с раннего утра, говоря: <<слушайтесь гласа Моего>>.
\vs Jer 11:8 Но они не слушались и не приклоняли уха своего, а ходили каждый по упорству злого сердца своего: поэтому Я навел на них все сказанное в завете сем, который Я заповедал им исполнять, а они не исполняли.
\rsbpar\vs Jer 11:9 И сказал мне Господь: есть заговор между мужами Иуды и жителями Иерусалима:
\vs Jer 11:10 они опять обратились к беззакониям праотцев своих, которые отреклись слушаться слов Моих и пошли вослед чужих богов, служа им. Дом Израиля и дом Иуды нарушили завет Мой, который Я заключил с отцами их.
\vs Jer 11:11 Посему так говорит Господь: вот, Я наведу на них бедствие, от которого они не могут избавиться, и когда воззовут ко Мне, не услышу их.
\vs Jer 11:12 Тогда город\acc{а} Иуды и жители Иерусалима пойдут и воззовут к богам, которым они кадят; но они нисколько не помогут им во время бедствия их.
\vs Jer 11:13 Ибо сколько у тебя городов, столько и богов у тебя, Иуда, и сколько улиц в Иерусалиме, столько вы наставили жертвенников постыдному, жертвенников для каждения Ваалу.
\vs Jer 11:14 Ты же не проси за этот народ и не возноси за них молитвы и прошений; ибо Я не услышу, когда они будут взывать ко Мне в бедствии своем.
\vs Jer 11:15 Что возлюбленному Моему в доме Моем, когда в нем совершаются многие непотребства? и священные мяса\fns{Жертвы.} не помогут тебе, когда, делая зло, ты радуешься.
\vs Jer 11:16 Зеленеющею маслиною, красующеюся приятными плодами, именовал тебя Господь. А ныне, при шуме сильного смятения, Он воспламенил огонь вокруг нее, и сокрушились ветви ее.
\vs Jer 11:17 Господь Саваоф, Который насадил тебя, изрек на тебя злое за зло дома Израилева и дома Иудина, которое они причинили себе тем, что подвигли Меня на гнев каждением Ваалу.
\vs Jer 11:18 Господь открыл мне, и я знаю; Ты показал мне деяния их.
\vs Jer 11:19 А я, как кроткий агнец, ведомый на заклание, и не знал, что они составляют замыслы против меня, \bibemph{говоря}: <<положим \bibemph{ядовитое} дерево в пищу его и отторгнем его от земли живых, чтобы и имя его более не упоминалось>>.
\vs Jer 11:20 Но, Господи Саваоф, Судия праведный, испытующий сердца и утробы! дай увидеть мне мщение Твое над ними, ибо Тебе вверил я дело мое.
\vs Jer 11:21 Посему так говорит Господь о мужах Анафофа, ищущих души твоей и говорящих: <<не пророчествуй во имя Господа, чтобы не умереть тебе от рук наших>>;
\vs Jer 11:22 посему так говорит Господь Саваоф: вот, Я посещу их: юноши \bibemph{их} умрут от меча; сыновья их и дочери их умрут от голода.
\vs Jer 11:23 И остатка не будет от них; ибо Я наведу бедствие на мужей Анафофа в год посещения их.
\vs Jer 12:1 Праведен будешь Ты, Господи, если я стану судиться с Тобою; и однако же буду говорить с Тобою о правосудии: почему путь нечестивых благоуспешен, и все вероломные благоденствуют?
\vs Jer 12:2 Ты насадил их, и они укоренились, выросли и приносят плод. В устах их Ты близок, но далек от сердца их.
\vs Jer 12:3 А меня, Господи, Ты знаешь, видишь меня и испытываешь сердце мое, каково оно к Тебе. Отдели их, как овец на заклание, и приготовь их на день убиения.
\vs Jer 12:4 Долго ли будет сетовать земля, и трава на всех полях~--- сохнуть? скот и птицы гибнут за нечестие жителей ее, ибо они говорят: <<Он не увидит, что с нами будет>>.
\vs Jer 12:5 Если ты с пешими бежал, и они утомили тебя, как же тебе состязаться с конями? и если в стране мирной ты был безопасен, то что будешь делать в наводнение Иордана?
\vs Jer 12:6 Ибо и братья твои и дом отца твоего, и они вероломно поступают с тобою, и они кричат вслед тебя громким голосом. Не верь им, когда они говорят тебе и доброе.
\vs Jer 12:7 Я оставил дом Мой; покинул удел Мой; самое любезное для души Моей отдал в руки врагов его.
\vs Jer 12:8 Удел Мой сделался для Меня как лев в лесу; возвысил на Меня голос свой: за то Я возненавидел его.
\vs Jer 12:9 Удел Мой стал у Меня, как разноцветная птица, на которую со всех сторон напали другие хищные птицы. Идите, собирайтесь, все полевые звери: идите пожирать его.
\vs Jer 12:10 Множество пастухов испортили Мой виноградник, истоптали ногами участок Мой; любимый участок Мой сделали пустою степью;
\vs Jer 12:11 сделали его пустынею, и в запустении он плачет предо Мною; вся земля опустошена, потому что ни один человек не прилагает этого к сердцу.
\vs Jer 12:12 На все горы в пустыне пришли опустошители; ибо меч Господа пожирает \bibemph{всё} от одного края земли до другого: нет мира ни для какой плоти.
\vs Jer 12:13 Они сеяли пшеницу, а пожали терны; измучились, и не получили никакой пользы; постыдитесь же таких прибытков ваших по причине пламенного гнева Господа.
\vs Jer 12:14 Так говорит Господь обо всех злых Моих соседях, нападающих на удел, который Я дал в наследие народу Моему, Израилю: вот, Я исторгну их из земли их, и дом Иудин исторгну из среды их.
\vs Jer 12:15 Но после того, как Я исторгну их, снова возвращу и помилую их, и приведу каждого в удел его и каждого в землю его.
\vs Jer 12:16 И если они научатся путям народа Моего, чтобы клясться именем Моим: <<жив Господь!>>, как они научили народ Мой клясться Ваалом, то водворятся среди народа Моего.
\vs Jer 12:17 Если же не послушаются, то Я искореню и совершенно истреблю такой народ, говорит Господь.
\vs Jer 13:1 Так сказал мне Господь: пойди, купи себе льняной пояс и положи его на чресла твои, но в воду не клади его.
\vs Jer 13:2 И я купил пояс, по слову Господню, и положил его на чресла мои.
\vs Jer 13:3 И было ко мне слово Господне в другой раз, и сказано:
\vs Jer 13:4 возьми пояс, который ты купил, который на чреслах твоих, и встань, пойди к Евфрату и спрячь его там в расселине скалы.
\vs Jer 13:5 Я пошел и спрятал его у Евфрата, как повелел мне Господь.
\vs Jer 13:6 По прошествии же многих дней сказал мне Господь: встань, пойди к Евфрату и возьми оттуда пояс, который Я велел тебе спрятать там.
\vs Jer 13:7 И я пришел к Евфрату, выкопал и взял пояс из того места, где спрятал его, и вот, пояс был испорчен, ни к чему стал не годен.
\rsbpar\vs Jer 13:8 И было ко мне слово Господне:
\vs Jer 13:9 так говорит Господь: так сокрушу Я гордость Иуды и великую гордость Иерусалима.
\vs Jer 13:10 Этот негодный народ, который не хочет слушать слов Моих, живет по упорству сердца своего и ходит во след иных богов, чтобы служить им и поклоняться им, будет как этот пояс, который ни к чему не годен.
\vs Jer 13:11 Ибо, как пояс близко лежит к чреслам человека, так Я приблизил к Себе весь дом Израилев и весь дом Иудин, говорит Господь, чтобы они были Моим народом и Моею славою, хвалою и украшением; но они не послушались.
\vs Jer 13:12 Посему скажи им слово сие: так говорит Господь, Бог Израилев: всякий винный мех наполняется вином. Они скажут тебе: <<разве мы не знаем, что всякий винный мех наполняется вином?>>
\vs Jer 13:13 А ты скажи им: так говорит Господь: вот, Я наполню вином до опьянения всех жителей сей земли и царей, сидящих на престоле Давида, и священников, и пророков и всех жителей Иерусалима,
\vs Jer 13:14 и сокрушу их друг о друга, и отцов и сыновей вместе, говорит Господь; не пощажу и не помилую, и не пожалею истребить их.
\vs Jer 13:15 Слушайте и внимайте; не будьте горды, ибо Господь говорит.
\vs Jer 13:16 Воздайте славу Господу Богу вашему, доколе Он еще не навел темноты, и доколе еще ноги ваши не спотыкаются на горах мрака: тогда вы будете ожидать света, а Он обратит его в тень смерти и сделает тьмою.
\vs Jer 13:17 Если же вы не послушаете сего, то душа моя в сокровенных местах будет оплакивать гордость вашу, будет плакать горько, и глаза мои будут изливаться в слезах; потому что стадо Господне отведено будет в плен.
\vs Jer 13:18 Скажи царю и царице: смиритесь, сядьте пониже, ибо упал с головы вашей венец славы вашей.
\vs Jer 13:19 Южные города заперты, и некому отворять их; Иуда весь отводится в плен, отводится в плен весь совершенно.
\vs Jer 13:20 Поднимите глаза ваши и посмотрите на идущих от севера: где стадо, которое дано было тебе, прекрасное стадо твое?
\vs Jer 13:21 Что скажешь, \bibemph{дочь Сиона}, когда Он посетит тебя? Ты сама приучила их начальствовать над тобою; не схватят ли тебя боли, как рождающую женщину?
\vs Jer 13:22 И если скажешь в сердце твоем: <<за что постигло меня это?>>~--- За множество беззаконий твоих открыт подол у тебя, обнажены пяты твои.
\vs Jer 13:23 Может ли Ефиоплянин переменить кожу свою и барс~--- пятна свои? так и вы можете ли делать доброе, привыкнув делать злое?
\vs Jer 13:24 Поэтому развею их, как прах, разносимый ветром пустынным.
\vs Jer 13:25 Вот жребий твой, отмеренная тебе от Меня часть, говорит Господь, потому что ты забыла Меня и надеялась на ложь.
\vs Jer 13:26 За то будет поднят подол твой на лице твое, чтобы открылся срам твой.
\vs Jer 13:27 Видел Я прелюбодейство твое и неистовые похотения твои, твои непотребства и твои мерзости на холмах в поле. Горе тебе, Иерусалим! ты и после сего не очистишься. Доколе же?
\vs Jer 14:1 Слово Господа, которое было к Иеремии по случаю бездождия.
\vs Jer 14:2 Плачет Иуда, ворота его распались, почернели на земле, и вопль поднимается в Иерусалиме.
\vs Jer 14:3 Вельможи посылают слуг своих за водою; они приходят к колодезям и не находят воды; возвращаются с пустыми сосудами; пристыженные и смущенные, они покрывают свои головы.
\vs Jer 14:4 Так как почва растрескалась оттого, что не было дождя на землю, то и земледельцы в смущении и покрывают свои головы.
\vs Jer 14:5 Даже и лань рождает на поле и оставляет \bibemph{детей}, потому что нет травы.
\vs Jer 14:6 И дикие ослы стоят на возвышенных местах и глотают, подобно шакалам, воздух; глаза их потускли, потому что нет травы.
\vs Jer 14:7 Хотя беззакония наши свидетельствуют против нас, но Ты, Господи, твори с нами ради имени Твоего; отступничество наше велико, согрешили мы пред Тобою.
\vs Jer 14:8 Надежда Израиля, Спаситель его во время скорби! Для чего Ты~--- как чужой в этой земле, как прохожий, который зашел переночевать?
\vs Jer 14:9 Для чего Ты~--- как человек изумленный, как сильный, не имеющий силы спасти? И однако же Ты, Господи, посреди нас, и Твое имя наречено над нами; не оставляй нас.
\vs Jer 14:10 Так говорит Господь народу сему: за то, что они любят бродить, не удерживают ног своих, за то Господь не благоволит к ним, припоминает ныне беззакония их и наказывает грехи их.
\rsbpar\vs Jer 14:11 И сказал мне Господь: ты не молись о народе сем во благо ему.
\vs Jer 14:12 Если они будут поститься, Я не услышу вопля их; и если вознесут всесожжение и дар, не приму их; но мечом и голодом, и моровою язвою истреблю их.
\vs Jer 14:13 Тогда сказал я: Господи Боже! вот, пророки говорят им: <<не увидите меча, и голода не будет у вас, но постоянный мир дам вам на сем месте>>.
\vs Jer 14:14 И сказал мне Господь: пророки пророчествуют ложное именем Моим; Я не посылал их и не давал им повеления, и не говорил им; они возвещают вам видения ложные и гадания, и пустое и мечты сердца своего.
\vs Jer 14:15 Поэтому так говорит Господь о пророках: они пророчествуют именем Моим, а Я не посылал их; они говорят: <<меча и голода не будет на сей земле>>: мечом и голодом будут истреблены эти пророки,
\vs Jer 14:16 и народ, которому они пророчествуют, разбросан будет по улицам Иерусалима от голода и меча, и некому будет хоронить их,~--- они и жены их, и сыновья их, и дочери их; и Я изолью на них зло их.
\vs Jer 14:17 И скажи им слово сие: да льются из глаз моих слезы ночь и день, и да не перестают; ибо великим поражением поражена дева, дочь народа моего, тяжким ударом.
\vs Jer 14:18 Выхожу я на поле,~--- и вот, убитые мечом; вхожу в город,~--- и вот истаевающие от голода; даже и пророк и священник бродят по земле бессознательно.
\vs Jer 14:19 Разве Ты совсем отверг Иуду? Разве душе Твоей опротивел Сион? Для чего поразил нас так, что нет нам исцеления? Ждем мира, и ничего доброго нет; ждем времени исцеления, и вот ужасы.
\vs Jer 14:20 Сознаем, Господи, нечестие наше, беззаконие отцов наших; ибо согрешили мы пред Тобою.
\vs Jer 14:21 Не отрини \bibemph{нас} ради имени Твоего; не унижай престола славы Твоей: вспомни, не разрушай завета Твоего с нами.
\vs Jer 14:22 Есть ли между суетными \bibemph{богами} языческими производящие дождь? или может ли небо \bibemph{само собою} подавать ливень? не Ты ли это, Господи, Боже наш? На Тебя надеемся мы; ибо Ты творишь все это.
\vs Jer 15:1 И сказал мне Господь: хотя бы предстали пред лице Мое Моисей и Самуил, душа Моя не \bibemph{приклонится} к народу сему; отгони \bibemph{их} от лица Моего, пусть они отойдут.
\vs Jer 15:2 Если же скажут тебе: <<куда нам идти?>>, то скажи им: так говорит Господь: кто \bibemph{обречен} на смерть, иди на смерть; и кто под меч,~--- под меч; и кто на голод,~--- на голод; и кто в плен,~--- в плен.
\vs Jer 15:3 И пошлю на них четыре рода \bibemph{казней}, говорит Господь: меч, чтобы убивать, и псов, чтобы терзать, и птиц небесных и зверей полевых, чтобы пожирать и истреблять;
\vs Jer 15:4 и отдам их на озлобление всем царствам земли за Манассию, сына Езекии, царя Иудейского, за то, что он сделал в Иерусалиме.
\vs Jer 15:5 Ибо кто пожалеет о тебе, Иерусалим? и кто окажет сострадание к тебе? и кто зайдет к тебе спросить о твоем благосостоянии?
\vs Jer 15:6 Ты оставил Меня, говорит Господь, отступил назад; поэтому Я простру на тебя руку Мою и погублю тебя: Я устал миловать.
\vs Jer 15:7 Я развеваю их веялом за ворота земли; лишаю их детей, гублю народ Мой; но они не возвращаются с путей своих.
\vs Jer 15:8 Вдов их у Меня более, нежели песку в море; наведу на них, на мать юношей, опустошителя в полдень; нападет на них внезапно страх и ужас.
\vs Jer 15:9 Лежит в изнеможении родившая семерых, испускает дыхание свое; еще днем закатилось солнце ее, она постыжена и посрамлена. И остаток их предам мечу пред глазами врагов их, говорит Господь.
\vs Jer 15:10 <<Горе мне, мать моя, что ты родила меня человеком, который спорит и ссорится со всею землею! никому не давал я в рост, и мне никто не давал в рост, \bibemph{а} все проклинают меня>>.
\rsbpar\vs Jer 15:11 Господь сказал: конец твой будет хорош, и Я заставлю врага поступать с тобою хорошо во время бедствия и во время скорби.
\vs Jer 15:12 Может ли железо сокрушить железо северное и медь?
\vs Jer 15:13 Имущество твое и сокровища твои отдам на расхищение, без платы, за все грехи твои, во всех пределах твоих;
\vs Jer 15:14 и отправлю с врагами твоими в землю, которой ты не знаешь; ибо огонь возгорелся в гневе Моем,~--- будет пылать на вас.
\vs Jer 15:15 О, Господи! Ты знаешь \bibemph{всё}; вспомни обо мне и посети меня, и отмсти за меня гонителям моим; не погуби меня по долготерпению Твоему; Ты знаешь, что ради Тебя несу я поругание.
\vs Jer 15:16 Обретены слова Твои, и я съел их; и было слово Твое мне в радость и в веселье сердца моего; ибо имя Твое наречено на мне, Господи, Боже Саваоф.
\vs Jer 15:17 Не сидел я в собрании смеющихся и не веселился: под тяготеющею на мне рукою Твоею я сидел одиноко, ибо Ты исполнил меня негодования.
\vs Jer 15:18 За что так упорна болезнь моя, и рана моя так неисцельна, что отвергает врачевание? Неужели Ты будешь для меня как бы обманчивым источником, неверною водою?
\vs Jer 15:19 На сие так сказал Господь: если ты обратишься, то Я восставлю тебя, и будешь предстоять пред лицем Моим; и если извлечешь драгоценное из ничтожного, то будешь как Мои уста. Они сами будут обращаться к тебе, а не ты будешь обращаться к ним.
\vs Jer 15:20 И сделаю тебя для этого народа крепкою медною стеною; они будут ратовать против тебя, но не одолеют тебя, ибо Я с тобою, чтобы спасать и избавлять тебя, говорит Господь.
\vs Jer 15:21 И спасу тебя от руки злых и избавлю тебя от руки притеснителей.
\vs Jer 16:1 И было ко мне слово Господне:
\vs Jer 16:2 не бери себе жены, и пусть не будет у тебя ни сыновей, ни дочерей на месте сем.
\vs Jer 16:3 Ибо так говорит Господь о сыновьях и дочерях, которые родятся на месте сем, и о матерях их, которые родят их, и об отцах их, которые произведут их на сей земле:
\vs Jer 16:4 тяжкими смертями умрут они и не будут ни оплаканы, ни похоронены; будут навозом на поверхности земли; мечом и голодом будут истреблены, и трупы их будут пищею птицам небесным и зверям земным.
\vs Jer 16:5 Ибо так говорит Господь: не входи в дом сетующих и не ходи плакать и жалеть с ними; ибо Я отнял от этого народа, говорит Господь, мир Мой и милость и сожаление.
\vs Jer 16:6 И умрут великие и малые на земле сей; и не будут погребены, и не будут оплакивать их, ни терзать себя, ни стричься ради них.
\vs Jer 16:7 И не будут преломлять для них хлеб в печали, в утешение об умершем; и не подадут им чаши утешения, чтобы пить по отце их и матери их.
\vs Jer 16:8 Не ходи также и в дом пиршества, чтобы сидеть с ними, есть и пить;
\vs Jer 16:9 ибо так говорит Господь Саваоф, Бог Израилев: вот, Я прекращу на месте сем в глазах ваших и во дни ваши голос радости и голос веселья, голос жениха и голос невесты.
\vs Jer 16:10 Когда ты перескажешь народу сему все эти слова, и они скажут тебе: <<за что изрек на нас Господь все это великое бедствие, и какая наша неправда, и какой наш грех, которым согрешили мы пред Господом Богом нашим?>>~---
\vs Jer 16:11 тогда скажи им: за то, что отцы ваши оставили Меня, говорит Господь, и пошли вослед иных богов, и служили им, и поклонялись им, а Меня оставили, и закона Моего не хранили.
\vs Jer 16:12 А вы поступаете еще хуже отцов ваших и живете каждый по упорству злого сердца своего, чтобы не слушать Меня.
\vs Jer 16:13 За это выброшу вас из земли сей в землю, которой не знали ни вы, ни отцы ваши, и там будете служить иным богам день и ночь; ибо Я не окажу вам милосердия.
\vs Jer 16:14 Посему вот, приходят дни, говорит Господь, когда не будут уже говорить: <<жив Господь, Который вывел сынов Израилевых из земли Египетской>>;
\vs Jer 16:15 но: <<жив Господь, Который вывел сынов Израилевых из земли северной и из всех земель, в которые изгнал их>>: ибо возвращу их в землю их, которую Я дал отцам их.
\vs Jer 16:16 Вот, Я пошлю множество рыболовов, говорит Господь, и будут ловить их; а потом пошлю множество охотников, и они погонят их со всякой горы, и со всякого холма, и из ущелий скал.
\vs Jer 16:17 Ибо очи Мои на всех путях их; они не скрыты от лица Моего, и неправда их не сокрыта от очей Моих.
\vs Jer 16:18 И воздам им прежде всего за неправду их и за сугубый грех их, потому что осквернили землю Мою, трупами гнусных своих и мерзостями своими наполнили наследие Мое.
\vs Jer 16:19 Господи, сила моя и крепость моя и прибежище мое в день скорби! к Тебе придут народы от краев земли и скажут: <<только ложь наследовали наши отцы, пустоту и то, в чем никакой нет пользы>>.
\vs Jer 16:20 Может ли человек сделать себе богов, которые впрочем не боги?
\vs Jer 16:21 Посему, вот Я покажу им ныне, покажу им руку Мою и могущество Мое, и узнают, что имя Мое~--- Господь.
\vs Jer 17:1 Грех Иуды написан железным резцом, алмазным острием начертан на скрижали сердца их и на рогах жертвенников их.
\vs Jer 17:2 Как о сыновьях своих, воспоминают они о жертвенниках своих и дубравах своих у зеленых дерев, на высоких холмах.
\vs Jer 17:3 Гору Мою в поле, имущество твое и все сокровища твои отдам на расхищение, и все высоты твои~--- за грехи во всех пределах твоих.
\vs Jer 17:4 И ты чрез себя лишишься наследия твоего, которое Я дал тебе, и отдам тебя в рабство врагам твоим, в землю, которой ты не знаешь, потому что вы воспламенили огонь гнева Моего; он будет гореть во веки.
\rsbpar\vs Jer 17:5 Так говорит Господь: проклят человек, который надеется на человека и плоть делает своею опорою, и которого сердце удаляется от Господа.
\vs Jer 17:6 Он будет как вереск в пустыне и не увидит, когда придет доброе, и поселится в местах знойных в степи, на земле бесплодной, необитаемой.
\vs Jer 17:7 Благословен человек, который надеется на Господа, и которого упование~--- Господь.
\vs Jer 17:8 Ибо он будет как дерево, посаженное при водах и пускающее корни свои у потока; не знает оно, когда приходит зной; лист его зелен, и во время засухи оно не боится и не перестает приносить плод.
\vs Jer 17:9 Лукаво сердце \bibemph{человеческое} более всего и крайне испорчено; кто узнает его?
\vs Jer 17:10 Я, Господь, проникаю сердце и испытываю внутренности, чтобы воздать каждому по пути его и по плодам дел его.
\vs Jer 17:11 Куропатка садится на яйца, которых не несла; таков приобретающий богатство неправдою: он оставит его на половине дней своих, и глупцом останется при конце своем.
\vs Jer 17:12 Престол славы, возвышенный от начала, есть место освящения нашего.
\vs Jer 17:13 Ты, Господи, надежда Израилева; все, оставляющие Тебя, посрамятся. <<Отступающие от Меня будут написаны на прахе, потому что оставили Господа, источник воды живой>>.
\vs Jer 17:14 Исцели меня, Господи, и исцелен буду; спаси меня, и спасен буду; ибо Ты хвала моя.
\vs Jer 17:15 Вот, они говорят мне: <<где слово Господне? пусть оно придет!>>
\vs Jer 17:16 Я не спешил быть пастырем у Тебя и не желал бедственного дня, Ты это знаешь; что вышло из уст моих, открыто пред лицем Твоим.
\vs Jer 17:17 Не будь страшен для меня, Ты~--- надежда моя в день бедствия.
\vs Jer 17:18 Пусть постыдятся гонители мои, а я не буду постыжен; пусть они вострепещут, а я буду бестрепетен; наведи на них день бедствия и сокруши их сугубым сокрушением.
\rsbpar\vs Jer 17:19 Так сказал мне Господь: пойди и стань в воротах сынов народа, которыми входят цари Иудейские и которыми они выходят, и во всех воротах Иерусалимских,
\vs Jer 17:20 и говори им: слушайте слово Господне, цари Иудейские, и вся Иудея, и все жители Иерусалима, входящие сими воротами.
\vs Jer 17:21 Так говорит Господь: берегите души свои и не носите нош в день субботний и не вносите их воротами Иерусалимскими,
\vs Jer 17:22 и не выносите нош из домов ваших в день субботний, и не занимайтесь никакою работою, но святите день субботний так, как Я заповедал отцам вашим,
\vs Jer 17:23 которые впрочем не послушались и не приклонили уха своего, но сделались жестоковыйными, чтобы не слушать и не принимать наставления.
\vs Jer 17:24 И если вы послушаете Меня в том, говорит Господь, чтобы не носить нош воротами сего города в день субботний и чтобы святить субботу, не занимаясь в этот день никакою работою,
\vs Jer 17:25 то воротами сего города будут входить цари и князья, сидящие на престоле Давида, ездящие на колесницах и на конях, они и князья их, Иудеи и жители Иерусалима, и город сей будет обитаем вечно.
\vs Jer 17:26 И будут приходить из городов Иудейских, и из окрестностей Иерусалима, и из земли Вениаминовой, и с равнины и с гор и с юга, и приносить всесожжение и жертву, и хлебное приношение, и ливан, и благодарственные жертвы в дом Господень.
\vs Jer 17:27 А если не послушаете Меня в том, чтобы святить день субботний и не носить нош, входя в ворота Иерусалима в день субботний, то возжгу огонь в воротах его, и он пожрет чертоги Иерусалима и не погаснет.
\vs Jer 18:1 Слово, которое было к Иеремии от Господа:
\vs Jer 18:2 встань и сойди в дом горшечника, и там Я возвещу тебе слова Мои.
\vs Jer 18:3 И сошел я в дом горшечника, и вот, он работал свою работу на кружале.
\vs Jer 18:4 И сосуд, который горшечник делал из глины, развалился в руке его; и он снова сделал из него другой сосуд, какой горшечнику вздумалось сделать.
\vs Jer 18:5 И было слово Господне ко мне:
\vs Jer 18:6 не могу ли Я поступить с вами, дом Израилев, подобно горшечнику сему? говорит Господь. Вот, что глина в руке горшечника, то вы в Моей руке, дом Израилев.
\vs Jer 18:7 Иногда Я скажу о каком-либо народе и царстве, что искореню, сокрушу и погублю его;
\vs Jer 18:8 но если народ этот, на который Я это изрек, обратится от своих злых дел, Я отлагаю то зло, которое помыслил сделать ему.
\vs Jer 18:9 А иногда скажу о каком-либо народе и царстве, что устрою и утвержу его;
\vs Jer 18:10 но если он будет делать злое пред очами Моими и не слушаться гласа Моего, Я отменю то добро, которым хотел облагодетельствовать его.
\vs Jer 18:11 Итак скажи мужам Иуды и жителям Иерусалима: так говорит Господь: вот, Я готовлю вам зло и замышляю против вас; итак обратитесь каждый от злого пути своего и исправьте пути ваши и поступки ваши.
\vs Jer 18:12 Но они говорят: <<не надейся; мы будем жить по своим помыслам и будем поступать каждый по упорству злого своего сердца>>.
\vs Jer 18:13 Посему так говорит Господь: спросите между народами, слыхал ли кто подобное сему? крайне гнусные дела совершила дева Израилева.
\vs Jer 18:14 Оставляет ли снег Ливанский скалу горы? и иссякают ли из других мест текущие холодные воды?
\vs Jer 18:15 А народ Мой оставил Меня; они кадят суетным, споткнулись на путях своих, оставили пути древние, чтобы ходить по стезям пути непроложенного,
\vs Jer 18:16 чтобы сделать землю свою ужасом, всегдашним посмеянием, так что каждый, проходящий по ней, изумится и покачает головою своею.
\vs Jer 18:17 Как восточным ветром развею их пред лицем врага; спиною, а не лицем обращусь к ним в день бедствия их.
\vs Jer 18:18 А они сказали: <<придите, составим замысел против Иеремии; ибо не исчез же закон у священника и совет у мудрого, и слово у пророка; придите, сразим его языком и не будем внимать словам его>>.
\vs Jer 18:19 Внемли мне, Господи, и услышь голос моих противников.
\vs Jer 18:20 Должно ли воздавать злом за добро? а они роют яму душе моей. Вспомни, что я стою пред лицем Твоим, чтобы говорить за них доброе, чтобы отвратить от них гнев Твой.
\vs Jer 18:21 Итак предай сыновей их голоду и подвергни их мечу; да будут жены их бездетными и вдовами, и мужья их да будут поражены смертью, и юноши их умерщвлены мечом на войне.
\vs Jer 18:22 Да будет слышен вопль из домов их, когда приведешь на них полки внезапно; ибо они роют яму, чтобы поймать меня, и тайно расставили сети для ног моих.
\vs Jer 18:23 Но Ты, Господи, знаешь все замыслы их против меня, чтобы умертвить меня; не прости неправды их и греха их не изгладь пред лицем Твоим; да будут они низвержены пред Тобою; поступи с ними во время гнева Твоего.
\vs Jer 19:1 Так сказал Господь: пойди и купи глиняный кувшин у горшечника; и возьми с собою старейших из народа и из старейшин священнических,
\vs Jer 19:2 и выйди в долину сыновей Енномовых, которая у ворот Харшиф, и провозгласи там слова, которые скажу тебе,
\vs Jer 19:3 и скажи: слушайте слово Господне, цари Иудейские и жители Иерусалима! так говорит Господь Саваоф, Бог Израилев: вот, Я наведу бедствие на место сие,~--- о котором кто услышит, у того зазвенит в ушах,
\vs Jer 19:4 за то, что они оставили Меня и чужим сделали место сие и кадят на нем иным богам, которых не знали ни они, ни отцы их, ни цари Иудейские; наполнили место сие кровью невинных
\vs Jer 19:5 и устроили высоты Ваалу, чтобы сожигать сыновей своих огнем во всесожжение Ваалу, чего Я не повелевал и не говорил, и что на мысль не приходило Мне;
\vs Jer 19:6 за то вот, приходят дни, говорит Господь, когда место сие не будет более называться Тофетом или долиною сыновей Енномовых, но долиною убиения.
\vs Jer 19:7 И уничтожу совет Иуды и Иерусалима на месте сем и сражу их мечом пред лицем врагов их и рукою ищущих души их, и отдам трупы их в пищу птицам небесным и зверям земным.
\vs Jer 19:8 И сделаю город сей ужасом и посмеянием; каждый, проходящий через него, изумится и посвищет, смотря на все язвы его.
\vs Jer 19:9 И накормлю их плотью сыновей их и плотью дочерей их; и будет каждый есть плоть своего ближнего, находясь в осаде и тесноте, когда стеснят их враги их и ищущие души их.
\vs Jer 19:10 И разбей кувшин пред глазами тех мужей, которые придут с тобою,
\vs Jer 19:11 и скажи им: так говорит Господь Саваоф: так сокрушу Я народ сей и город сей, как сокрушен горшечников сосуд, который уже не может быть восстановлен, и будут хоронить их в Тофете, по недостатку места для погребения.
\vs Jer 19:12 Так поступлю с местом сим, говорит Господь, и с жителями его; и город сей сделаю подобным Тофету.
\vs Jer 19:13 И домы Иерусалима и домы царей Иудейских будут, как место Тофет, нечистыми, потому что на кровлях всех домов кадят всему воинству небесному и совершают возлияния богам чужим.
\rsbpar\vs Jer 19:14 И пришел Иеремия с Тофета, куда Господь посылал его пророчествовать, и стал на дворе дома Господня и сказал всему народу:
\vs Jer 19:15 так говорит Господь Саваоф, Бог Израилев: вот, Я наведу на город сей и на все города его все то бедствие, которое изрек на него, потому что они жестоковыйны и не слушают слов Моих.
\vs Jer 20:1 Когда Пасхор, сын Еммеров, священник, он же и надзиратель в доме Господнем, услышал, что Иеремия пророчески произнес слова сии,
\vs Jer 20:2 то ударил Пасхор Иеремию пророка и посадил его в колоду, которая была у верхних ворот Вениаминовых при доме Господнем.
\vs Jer 20:3 Но на другой день Пасхор выпустил Иеремию из колоды, и Иеремия сказал ему: не <<Пасхор>>\fns{Мир вокруг.} нарек Господь имя тебе, но <<Магор Миссавив>>\fns{Ужас вокруг.}.
\vs Jer 20:4 Ибо так говорит Господь: вот, Я сделаю тебя ужасом для тебя самого и для всех друзей твоих, и падут они от меча врагов своих, и твои глаза увидят это. И всего Иуду предам в руки царя Вавилонского, и отведет их в Вавилон и поразит их мечом.
\vs Jer 20:5 И предам все богатство этого города и все стяжание его, и все драгоценности его; и все сокровища царей Иудейских отдам в руки врагов их, и разграбят их и возьмут, и отправят их в Вавилон.
\vs Jer 20:6 И ты, Пасхор, и все живущие в доме твоем, пойдете в плен; и придешь в Вавилон, и там умрешь, и там будешь похоронен, ты и все друзья твои, которым ты пророчествовал ложно.
\vs Jer 20:7 Ты влек меня, Господи,~--- и я увлечен; Ты сильнее меня~--- и превозмог, и я каждый день в посмеянии, всякий издевается надо мною.
\vs Jer 20:8 Ибо лишь только начну говорить я,~--- кричу о насилии, вопию о разорении, потому что слово Господне обратилось в поношение мне и в повседневное посмеяние.
\vs Jer 20:9 И подумал я: <<не буду я напоминать о Нем и не буду более говорить во имя Его>>; но было в сердце моем, как бы горящий огонь, заключенный в костях моих, и я истомился, удерживая его, и не мог.
\vs Jer 20:10 Ибо я слышал толки многих: угрозы вокруг; <<заявите, \bibemph{говорили они}, и мы сделаем донос>>. Все, жившие со мною в мире, сторожат за мною, не споткнусь ли я: <<может быть, \bibemph{говорят}, он попадется, и мы одолеем его и отмстим ему>>.
\vs Jer 20:11 Но со мною Господь, как сильный ратоборец; поэтому гонители мои споткнутся и не одолеют; сильно посрамятся, потому что поступали неразумно; посрамление будет вечное, никогда не забудется.
\vs Jer 20:12 Господи сил! Ты испытываешь праведного и видишь внутренность и сердце. Да увижу я мщение Твое над ними, ибо Тебе вверил я дело мое.
\vs Jer 20:13 Пойте Господу, хвалите Господа, ибо Он спасает душу бедного от руки злодеев.~---
\vs Jer 20:14 Проклят день, в который я родился! день, в который родила меня мать моя, да не будет благословен!
\vs Jer 20:15 Проклят человек, который принес весть отцу моему и сказал: <<у тебя родился сын>>, \bibemph{и} тем очень обрадовал его.
\vs Jer 20:16 И да будет с тем человеком, что с городами, которые разрушил Господь и не пожалел; да слышит он утром вопль и в полдень рыдание
\vs Jer 20:17 за то, что он не убил меня в самой утробе~--- так, чтобы мать моя была мне гробом, и чрево ее оставалось вечно беременным.
\vs Jer 20:18 Для чего вышел я из утробы, чтобы видеть труды и скорби, и чтобы дни мои исчезали в бесславии?
\vs Jer 21:1 Слово, которое было к Иеремии от Господа, когда царь Седекия прислал к нему Пасхора, сына Молхиина, и Софонию, сына Маасеи священника, сказать \bibemph{ему}:
\vs Jer 21:2 <<вопроси о нас Господа, ибо Навуходоносор, царь Вавилонский, воюет против нас; может быть, Господь сотворит с нами что-либо такое, как все чудеса Его, чтобы тот отступил от нас>>.
\rsbpar\vs Jer 21:3 И сказал им Иеремия: так скажите Седекии:
\vs Jer 21:4 так говорит Господь, Бог Израилев: вот, Я обращу назад воинские орудия, которые в руках ваших, которыми вы сражаетесь с царем Вавилонским и с Халдеями, осаждающими вас вне стены, и соберу оные посреди города сего;
\vs Jer 21:5 и Сам буду воевать против вас рукою простертою и мышцею крепкою, во гневе и в ярости и в великом негодовании;
\vs Jer 21:6 и поражу живущих в сем городе~--- и людей и скот; от великой язвы умрут они.
\vs Jer 21:7 А после того, говорит Господь, Седекию, царя Иудейского, слуг его и народ, и оставшихся в городе сем от моровой язвы, меча и голода, предам в руки Навуходоносора, царя Вавилонского, и в руки врагов их и в руки ищущих души их; и он поразит их острием меча и не пощадит их, и не пожалеет и не помилует.
\vs Jer 21:8 И народу сему скажи: так говорит Господь: вот, Я предлагаю вам путь жизни и путь смерти:
\vs Jer 21:9 кто останется в этом городе, тот умрет от меча и голода и моровой язвы; а кто выйдет и предастся Халдеям, осаждающим вас, тот будет жив, и душа его будет ему вместо добычи;
\vs Jer 21:10 ибо Я обратил лице Мое против города сего, говорит Господь, на зло, а не на добро; он будет предан в руки царя Вавилонского, и тот сожжет его огнем.
\rsbpar\vs Jer 21:11 И дому царя Иудейского \bibemph{скажи}: слушайте слово Господне:
\vs Jer 21:12 дом Давидов! так говорит Господь: с раннего утра производите суд и спасайте обижаемого от руки обидчика, чтобы ярость Моя не вышла, как огонь, и не разгорелась по причине злых дел ваших до того, что никто не погасит.
\vs Jer 21:13 Вот, Я~--- против тебя, жительница долины, скала равнины, говорит Господь,~--- против вас, которые говорите: <<кто выступит против нас и кто войдет в жилища наши?>>
\vs Jer 21:14 Но Я посещу вас по плодам дел ваших, говорит Господь, и зажгу огонь в лесу вашем, и пожрет все вокруг него.
\vs Jer 22:1 Так сказал Господь: сойди в дом царя Иудейского и произнеси слово сие
\vs Jer 22:2 и скажи: выслушай слово Господне, царь Иудейский, сидящий на престоле Давидовом, ты, и слуги твои, и народ твой, входящие сими воротами.
\vs Jer 22:3 Так говорит Господь: производите суд и правду и спасайте обижаемого от руки притеснителя, не обижайте и не тесните пришельца, сироты и вдовы, и невинной крови не проливайте на месте сем.
\vs Jer 22:4 Ибо если вы будете исполнять слово сие, то будут входить воротами дома сего цари, сидящие вместо Давида на престоле его, ездящие на колеснице и на конях, сами и слуги их и народ их.
\vs Jer 22:5 А если не послушаете слов сих, то Мною клянусь, говорит Господь, что дом сей сделается пустым.
\vs Jer 22:6 Ибо так говорит Господь дому царя Иудейского: Галаад ты у Меня, вершина Ливана; но Я сделаю тебя пустынею и города необитаемыми
\vs Jer 22:7 и приготовлю против тебя истребителей, каждого со своими орудиями, и срубят лучшие кедры твои и бросят в огонь.
\vs Jer 22:8 И многие народы будут проходить через город сей и говорить друг другу: <<за что Господь так поступил с этим великим городом?>>
\vs Jer 22:9 И скажут в ответ: <<за то, что они оставили завет Господа Бога своего и поклонялись иным богам и служили им>>.
\vs Jer 22:10 Не плачьте об умершем и не жалейте о нем; но горько плачьте об отходящем в плен, ибо он уже не возвратится и не увидит родной страны своей.
\vs Jer 22:11 Ибо так говорит Господь о Саллуме, сыне Иосии, царе Иудейском, который царствовал после отца своего, Иосии, и который вышел из сего места: он уже не возвратится сюда,
\vs Jer 22:12 но умрет в том месте, куда отвели его пленным, и более не увидит земли сей.
\vs Jer 22:13 Горе тому, кто строит дом свой неправдою и горницы свои беззаконием, кто заставляет ближнего своего работать даром и не отдает ему платы его,
\vs Jer 22:14 кто говорит: <<построю себе дом обширный и горницы просторные>>,~--- и прорубает себе окна, и обшивает кедром, и красит красною краскою.
\vs Jer 22:15 Думаешь ли ты быть царем, потому что заключил себя в кедр? отец твой ел и пил, но производил суд и правду, и потому ему было хорошо.
\vs Jer 22:16 Он разбирал дело бедного и нищего, и потому ему хорошо было. Не это ли значит знать Меня? говорит Господь.
\vs Jer 22:17 Но твои глаза и твое сердце обращены только к твоей корысти и к пролитию невинной крови, к тому, чтобы делать притеснение и насилие.
\vs Jer 22:18 Посему так говорит Господь о Иоакиме, сыне Иосии, царе Иудейском: не будут оплакивать его: <<увы, брат мой!>> и: <<увы, сестра!>> Не будут оплакивать его: <<увы, государь!>> и: <<увы, его величие!>>
\vs Jer 22:19 Ослиным погребением будет он погребен; вытащат его и бросят далеко за ворота Иерусалима.
\vs Jer 22:20 Взойди на Ливан и кричи, и на Васане возвысь голос твой и кричи с Аварима, ибо сокрушены все друзья твои.
\vs Jer 22:21 Я говорил тебе во время благоденствия твоего; но ты сказал: <<не послушаю>>. Таково было поведение твое с самой юности твоей, что ты не слушал гласа Моего.
\vs Jer 22:22 Всех пастырей твоих унесет ветер, и друзья твои пойдут в плен; и тогда ты будешь постыжен и посрамлен за все злодеяния твои.
\vs Jer 22:23 Живущий на Ливане, гнездящийся на кедрах! как жалок будешь ты, когда постигнут тебя муки, как боли женщины в родах!
\vs Jer 22:24 Живу Я, сказал Господь: если бы Иехония, сын Иоакима, царь Иудейский, был перстнем на правой руке Моей, то и отсюда Я сорву тебя
\vs Jer 22:25 и отдам тебя в руки ищущих души твоей и в руки тех, которых ты боишься, в руки Навуходоносора, царя Вавилонского, и в руки Халдеев,
\vs Jer 22:26 и выброшу тебя и твою мать, которая родила тебя, в чужую страну, где вы не родились, и там умрете;
\vs Jer 22:27 а в землю, куда душа их будет желать возвратиться, туда не возвратятся.
\vs Jer 22:28 <<Неужели этот человек, Иехония, есть создание презренное, отверженное? или он~--- сосуд непотребный? за что они выброшены~--- он и племя его, и брошены в страну, которой не знали?>>
\vs Jer 22:29 О, земля, земля, земля! слушай слово Господне.
\vs Jer 22:30 Так говорит Господь: запишите человека сего лишенным детей, человеком злополучным во дни свои, потому что никто уже из племени его не будет сидеть на престоле Давидовом и владычествовать в Иудее.
\vs Jer 23:1 Горе пастырям, которые губят и разгоняют овец паствы Моей! говорит Господь.
\vs Jer 23:2 Посему так говорит Господь, Бог Израилев, к пастырям, пасущим народ Мой: вы рассеяли овец Моих, и разогнали их, и не смотрели за ними; вот, Я накажу вас за злые деяния ваши, говорит Господь.
\vs Jer 23:3 И соберу остаток стада Моего из всех стран, куда Я изгнал их, и возвращу их во дворы их; и будут плодиться и размножаться.
\vs Jer 23:4 И поставлю над ними пастырей, которые будут пасти их, и они уже не будут бояться и пугаться, и не будут теряться, говорит Господь.
\rsbpar\vs Jer 23:5 Вот, наступают дни, говорит Господь, и восставлю Давиду Отрасль праведную, и воцарится Царь, и будет поступать мудро, и будет производить суд и правду на земле.
\vs Jer 23:6 Во дни Его Иуда спасется и Израиль будет жить безопасно; и вот имя Его, которым будут называть Его: <<Господь оправдание наше!>>
\vs Jer 23:7 Посему, вот наступают дни, говорит Господь, когда уже не будут говорить: <<жив Господь, Который вывел сынов Израилевых из земли Египетской>>,
\vs Jer 23:8 но: <<жив Господь, Который вывел и Который привел племя дома Израилева из земли северной и из всех земель, куда Я изгнал их>>, и будут жить на земле своей.
\rsbpar\vs Jer 23:9 О пророках. Сердце мое во мне раздирается, все кости мои сотрясаются; я~--- как пьяный, как человек, которого одолело вино, ради Господа и ради святых слов Его,
\vs Jer 23:10 потому что земля наполнена прелюбодеями, потому что плачет земля от проклятия; засохли пастбища пустыни, и стремление их~--- зло, и сила их~--- неправда,
\vs Jer 23:11 ибо и пророк и священник~--- лицемеры; даже в доме Моем Я нашел нечестие их, говорит Господь.
\vs Jer 23:12 За то путь их будет для них, как скользкие места в темноте: их толкнут, и они упадут там; ибо Я наведу на них бедствие, год посещения их, говорит Господь.
\vs Jer 23:13 И в пророках Самарии Я видел безумие; они пророчествовали именем Ваала, и ввели в заблуждение народ Мой, Израиля.
\vs Jer 23:14 Но в пророках Иерусалима вижу ужасное: они прелюбодействуют и ходят во лжи, поддерживают руки злодеев, чтобы никто не обращался от своего нечестия; все они предо Мною~--- как Содом, и жители его~--- как Гоморра.
\vs Jer 23:15 Посему так говорит Господь Саваоф о пророках: вот, Я накормлю их полынью и напою их водою с желчью, ибо от пророков Иерусалимских нечестие распространилось на всю землю.
\vs Jer 23:16 Так говорит Господь Саваоф: не слушайте слов пророков, пророчествующих вам: они обманывают вас, рассказывают мечты сердца своего, \bibemph{а} не от уст Господних.
\vs Jer 23:17 Они постоянно говорят пренебрегающим Меня: <<Господь сказал: мир будет у вас>>. И всякому, поступающему по упорству своего сердца, говорят: <<не придет на вас беда>>.
\vs Jer 23:18 Ибо кто стоял в совете Господа и видел и слышал слово Его? Кто внимал слову Его и услышал?
\vs Jer 23:19 Вот, идет буря Господня с яростью, буря грозная, и падет на главу нечестивых.
\vs Jer 23:20 Гнев Господа не отвратится, доколе Он не совершит и доколе не выполнит намерений сердца Своего; в последующие дни вы ясно уразумеете это.
\vs Jer 23:21 Я не посылал пророков сих, а они сами побежали; Я не говорил им, а они пророчествовали.
\vs Jer 23:22 Если бы они стояли в Моем совете, то объявили бы народу Моему слова Мои и отводили бы их от злого пути их и от злых дел их.
\vs Jer 23:23 Разве Я~--- Бог \bibemph{только} вблизи, говорит Господь, а не Бог и вдали?
\vs Jer 23:24 Может ли человек скрыться в тайное место, где Я не видел бы его? говорит Господь. Не наполняю ли Я небо и землю? говорит Господь.
\vs Jer 23:25 Я слышал, что говорят пророки, Моим именем пророчествующие ложь. Они говорят: <<мне снилось, мне снилось>>.
\vs Jer 23:26 Долго ли это будет в сердце пророков, пророчествующих ложь, пророчествующих обман своего сердца?
\vs Jer 23:27 Думают ли они довести народ Мой до забвения имени Моего посредством снов своих, которые они пересказывают друг другу, как отцы их забыли имя Мое из-за Ваала?
\vs Jer 23:28 Пророк, который видел сон, пусть и рассказывает его как сон; а у которого Мое слово, тот пусть говорит слово Мое верно. Что общего у мякины с чистым зерном? говорит Господь.
\vs Jer 23:29 Слово Мое не подобно ли огню, говорит Господь, и не подобно ли молоту, разбивающему скалу?
\vs Jer 23:30 Посему, вот Я~--- на пророков, говорит Господь, которые крадут слова Мои друг у друга.
\vs Jer 23:31 Вот, Я~--- на пророков, говорит Господь, которые действуют своим языком, а говорят: <<Он сказал>>.
\vs Jer 23:32 Вот, Я~--- на пророков ложных снов, говорит Господь, которые рассказывают их и вводят народ Мой в заблуждение своими обманами и обольщением, тогда как Я не посылал их и не повелевал им, и они никакой пользы не приносят народу сему, говорит Господь.
\vs Jer 23:33 Если спросит у тебя народ сей, или пророк, или священник: <<какое бремя от Господа?>>, то скажи им: <<какое бремя? Я покину вас, говорит Господь>>.
\vs Jer 23:34 Если пророк, или священник, или народ скажет: <<бремя от Господа>>, Я накажу того человека и дом его.
\vs Jer 23:35 Так говорите друг другу и брат брату: <<что ответил Господь?>> или: <<что сказал Господь?>>
\vs Jer 23:36 А этого слова: <<бремя от Господа>>, впредь не употребляйте: ибо бременем будет \bibemph{такому} человеку слово его, потому что вы извращаете слова живаго Бога, Господа Саваофа Бога нашего.
\vs Jer 23:37 Так говори пророку: <<что ответил тебе Господь?>> или: <<что сказал Господь?>>
\vs Jer 23:38 А если вы еще будете говорить: <<бремя от Господа>>, то так говорит Господь: за то, что вы говорите слово сие: <<бремя от Господа>>, тогда как Я послал сказать вам: <<не говорите: бремя от Господа>>,~---
\vs Jer 23:39 за то, вот, Я забуду вас вовсе и оставлю вас, и город сей, который Я дал вам и отцам вашим, отвергну от лица Моего
\vs Jer 23:40 и положу на вас поношение вечное и бесславие вечное, которое не забудется.
\vs Jer 24:1 Господь показал мне: и вот, две корзины со смоквами поставлены пред храмом Господним, после того, как Навуходоносор, царь Вавилонский, вывел из Иерусалима пленными Иехонию, сына Иоакимова, царя Иудейского, и князей Иудейских с плотниками и кузнецами и привел их в Вавилон:
\vs Jer 24:2 одна корзина была со смоквами весьма хорошими, каковы бывают смоквы ранние, а другая корзина~--- со смоквами весьма худыми, которых по негодности \bibemph{их} нельзя есть.
\vs Jer 24:3 И сказал мне Господь: что видишь ты, Иеремия? Я сказал: смоквы, смоквы хорошие~--- весьма хороши, а худые~--- весьма худы, так что их нельзя есть, потому что они очень нехороши.
\rsbpar\vs Jer 24:4 И было ко мне слово Господне:
\vs Jer 24:5 так говорит Господь, Бог Израилев: подобно этим смоквам хорошим Я призн\acc{а}ю хорошими переселенцев Иудейских, которых Я послал из сего места в землю Халдейскую;
\vs Jer 24:6 и обращу на них очи Мои во благо им и возвращу их в землю сию, и устрою их, а не разорю, и насажду их, а не искореню;
\vs Jer 24:7 и дам им сердце, чтобы знать Меня, что Я Господь, и они будут Моим народом, а Я буду их Богом; ибо они обратятся ко Мне всем сердцем своим.
\vs Jer 24:8 А о худых смоквах, которых и есть нельзя по негодности \bibemph{их}, так говорит Господь: таким Я сделаю Седекию, царя Иудейского, и князей его и прочих Иерусалимлян, остающихся в земле сей и живущих в земле Египетской;
\vs Jer 24:9 и отдам их на озлобление и на злострадание во всех царствах земных, в поругание, в притчу, в посмеяние и проклятие во всех местах, куда Я изгоню их.
\vs Jer 24:10 И пошлю на них меч, голод и моровую язву, доколе не истреблю их с земли, которую Я дал им и отцам их.
\vs Jer 25:1 Слово, которое было к Иеремии о всем народе Иудейском, в четвертый год Иоакима, сына Иосии, царя Иудейского,~--- это был первый год Навуходоносора, царя Вавилонского,~---
\vs Jer 25:2 и которое пророк Иеремия произнес ко всему народу Иудейскому и ко всем жителям Иерусалима и сказал:
\vs Jer 25:3 от тринадцатого года Иосии, сына Амонова, царя Иудейского, до сего дня, вот уже двадцать три года, было ко мне слово Господне, и я с раннего утра говорил вам,~--- и вы не слушали.
\vs Jer 25:4 Господь посылал к вам всех рабов Своих, пророков, с раннего утра посылал,~--- и вы не слушали и не приклоняли уха своего, чтобы слушать.
\vs Jer 25:5 Вам говорили: <<обратитесь каждый от злого пути своего и от злых дел своих и живите на земле, которую Господь дал вам и отцам вашим из века в век;
\vs Jer 25:6 и не ходите во след иных богов, чтобы служить им и поклоняться им, и не прогневляйте Меня делами рук своих, и не сделаю вам зла>>.
\vs Jer 25:7 Но вы не слушали Меня, говорит Господь, прогневляя Меня делами рук своих, на зло себе.
\vs Jer 25:8 Посему так говорит Господь Саваоф: за то, что вы не слушали слов Моих,
\vs Jer 25:9 вот, Я пошлю и возьму все племена северные, говорит Господь, и пошлю к Навуходоносору, царю Вавилонскому, рабу Моему, и приведу их на землю сию и на жителей ее и на все окрестные народы; и совершенно истреблю их и сделаю их ужасом и посмеянием и вечным запустением.
\vs Jer 25:10 И прекращу у них голос радости и голос веселья, голос жениха и голос невесты, звук жерновов и свет светильника.
\vs Jer 25:11 И вся земля эта будет пустынею и ужасом; и народы сии будут служить царю Вавилонскому семьдесят лет.
\vs Jer 25:12 И будет: когда исполнится семьдесят лет, накажу царя Вавилонского и тот народ, говорит Господь, за их нечестие, и землю Халдейскую, и сделаю ее вечною пустынею.
\vs Jer 25:13 И совершу над тою землею все слова Мои, которые Я произнес на нее, все написанное в сей книге, что Иеремия пророчески изрек на все народы.
\vs Jer 25:14 Ибо и их поработят многочисленные народы и цари великие; и Я воздам им по их поступкам и по делам рук их.
\vs Jer 25:15 Ибо так сказал мне Господь, Бог Израилев: возьми из руки Моей чашу сию с вином ярости и напой из нее все народы, к которым Я посылаю тебя.
\vs Jer 25:16 И они выпьют, и будут шататься и обезумеют при виде меча, который Я пошлю на них.
\vs Jer 25:17 И взял я чашу из руки Господней и напоил из нее все народы, к которым послал меня Господь:
\vs Jer 25:18 Иерусалим и города Иудейские, и царей его и князей его, чтоб опустошить их и сделать ужасом, посмеянием и проклятием, как и видно ныне,
\vs Jer 25:19 фараона, царя Египетского, и слуг его, и князей его и весь народ его,
\vs Jer 25:20 и весь смешанный народ, и всех царей земли Уца, и всех царей земли Филистимской, и Аскалон, и Газу, и Екрон, и остатки Азота,
\vs Jer 25:21 Едома, и Моава, и сыновей Аммоновых,
\vs Jer 25:22 и всех царей Тира, и всех царей Сидона, и царей островов, которые за морем,
\vs Jer 25:23 Дедана, и Фему, и Буза, и всех, стригущих волосы на висках,
\vs Jer 25:24 и всех царей Аравии, и всех царей народов разноплеменных, живущих в пустыне,
\vs Jer 25:25 всех царей Зимврии, и всех царей Елама, и всех царей Мидии,
\vs Jer 25:26 и всех царей севера, близких друг к другу и дальних, и все царства земные, которые на лице земли, а царь Сесаха выпьет после них.
\vs Jer 25:27 И скажи им: так говорит Господь Саваоф, Бог Израилев: пейте и опьянейте, и изрыгните и падите, и не вставайте при виде меча, который Я пошлю на вас.
\vs Jer 25:28 Если же они будут отказываться брать чашу из руки твоей, чтобы пить, то скажи им: так говорит Господь Саваоф: вы непременно будете пить.
\vs Jer 25:29 Ибо вот на город сей, на котором наречено имя Мое, Я начинаю наводить бедствие; и вы ли останетесь ненаказанными? Нет, не останетесь ненаказанными; ибо Я призываю меч на всех живущих на земле, говорит Господь Саваоф.
\vs Jer 25:30 Посему прореки на них все слова сии и скажи им: Господь возгремит с высоты и из жилища святыни Своей подаст глас Свой; страшно возгремит на селение Свое; как топчущие в точиле, воскликнет на всех живущих на земле.
\vs Jer 25:31 Шум дойдет до концов земли, ибо у Господа состязание с народами: Он будет судиться со всякою плотью, нечестивых Он предаст мечу, говорит Господь.
\rsbpar\vs Jer 25:32 Так говорит Господь Саваоф: вот, бедствие пойдет от народа к народу, и большой вихрь поднимется от краев земли.
\vs Jer 25:33 И будут пораженные Господом в тот день от конца земли до конца земли, не будут оплаканы и не будут прибраны и похоронены, навозом будут на лице земли.
\vs Jer 25:34 Рыдайте, пастыри, и стенайте, и посыпайте себя прахом, вожди стада; ибо исполнились дни ваши для заклания и рассеяния вашего, и падете, как дорогой сосуд.
\vs Jer 25:35 И не будет убежища пастырям и спасения вождям стада.
\vs Jer 25:36 Слышен вопль пастырей и рыдание вождей стада, ибо опустошил Господь пажить их.
\vs Jer 25:37 Истребляются мирные селения от ярости гнева Господня.
\vs Jer 25:38 Он оставил жилище Свое, как лев; и земля их сделалась пустынею от ярости опустошителя и от пламенного гнева Его.
\vs Jer 26:1 В начале царствования Иоакима, сына Иосии, царя Иудейского, было такое слово от Господа:
\vs Jer 26:2 так говорит Господь: стань на дворе дома Господня и скажи ко всем городам Иудеи, приходящим на поклонение в дом Господень, все те слова, какие повелю тебе сказать им; не убавь ни слова.
\vs Jer 26:3 Может быть, они послушают и обратятся каждый от злого пути своего, и тогда Я отменю то бедствие, которое думаю сделать им за злые деяния их.
\vs Jer 26:4 И скажи им: так говорит Господь: если вы не послушаетесь Меня в том, чтобы поступать по закону Моему, который Я дал вам,
\vs Jer 26:5 чтобы внимать словам рабов Моих, пророков, которых Я посылаю к вам, посылаю с раннего утра, и которых вы не слушаете,~---
\vs Jer 26:6 то с домом сим Я сделаю то же, что с Силомом, и город сей предам на проклятие всем народам земли.
\rsbpar\vs Jer 26:7 Священники и пророки и весь народ слушали Иеремию, когда он говорил сии слова в доме Господнем.
\vs Jer 26:8 И когда Иеремия сказал все, что Господь повелел ему сказать всему народу, тогда схватили его священники и пророки и весь народ, и сказали: <<ты должен умереть;
\vs Jer 26:9 зачем ты пророчествуешь именем Господа и говоришь: дом сей будет как Силом, и город сей опустеет, \bibemph{останется} без жителей?>> И собрался весь народ против Иеремии в доме Господнем.
\vs Jer 26:10 Когда услышали об этом князья Иудейские, то пришли из дома царя к дому Господню и сели у входа в новые ворота \bibemph{дома} Господня.
\vs Jer 26:11 Тогда священники и пророки так сказали князьям и всему народу: <<смертный приговор этому человеку! потому что он пророчествует против города сего, как вы слышали своими ушами>>.
\vs Jer 26:12 И сказал Иеремия всем князьям и всему народу: <<Господь послал меня пророчествовать против дома сего и против города сего все те слова, которые вы слышали;
\vs Jer 26:13 итак исправьте пути ваши и деяния ваши и послушайтесь гласа Господа Бога вашего, и Господь отменит бедствие, которое изрек на вас;
\vs Jer 26:14 а что до меня, вот~--- я в ваших руках; делайте со мною, что в глазах ваших покажется хорошим и справедливым;
\vs Jer 26:15 только твердо знайте, что если вы умертвите меня, то невинную кровь возложите на себя и на город сей и на жителей его; ибо истинно Господь послал меня к вам сказать все те слова в уши ваши>>.
\vs Jer 26:16 Тогда князья и весь народ сказали священникам и пророкам: <<этот человек не подлежит смертному приговору, потому что он говорил нам именем Господа Бога нашего>>.
\vs Jer 26:17 И из старейшин земли встали некоторые и сказали всему народному собранию:
\vs Jer 26:18 <<Михей Морасфитянин пророчествовал во дни Езекии, царя Иудейского, и сказал всему народу Иудейскому: так говорит Господь Саваоф: Сион будет вспахан, как поле, и Иерусалим сделается грудою развалин, и гора дома сего~--- лесистым холмом.
\vs Jer 26:19 Умертвили ли его за это Езекия, царь Иудейский, и весь Иуда? Не убоялся ли он Господа и не умолял ли Господа? и Господь отменил бедствие, которое изрек на них; а мы хотим сделать большое зло душам нашим?
\vs Jer 26:20 Пророчествовал также именем Господа некто Урия, сын Шемаии, из Кариаф-Иарима,~--- и пророчествовал против города сего и против земли сей точно такими же словами, как Иеремия.
\vs Jer 26:21 Когда услышал слова его царь Иоаким и все вельможи его и все князья, то искал царь умертвить его. Услышав об этом, Урия убоялся и убежал, и удалился в Египет.
\vs Jer 26:22 Но царь Иоаким и в Египет послал людей: Елнафана, сына Ахборова, и других с ним.
\vs Jer 26:23 И вывели Урию из Египта и привели его к царю Иоакиму, и он умертвил его мечом и бросил труп его, где были простонародные гробницы.
\vs Jer 26:24 Но рука Ахикама, сына Сафанова, была за Иеремию, чтобы не отдавать его в руки народа на убиение>>.
\vs Jer 27:1 В начале царствования Иоакима\fns{Седекии.}, сына Иосии, царя Иудейского, было слово сие к Иеремии от Господа:
\vs Jer 27:2 так сказал мне Господь: сделай себе узы и ярмо и возложи их себе на выю;
\vs Jer 27:3 и пошли такие же к царю Идумейскому, и к царю Моавитскому, и к царю сыновей Аммоновых, и к царю Тира, и к царю Сидона, через послов, пришедших в Иерусалим к Седекии, царю Иудейскому;
\vs Jer 27:4 и накажи им сказать государям их: так говорит Господь Саваоф, Бог Израилев: так скажите государям вашим:
\vs Jer 27:5 Я сотворил землю, человека и животных, которые на лице земли, великим могуществом Моим и простертою мышцею Моею, и отдал ее, кому Мне благоугодно было.
\vs Jer 27:6 И ныне Я отдаю все земли сии в руку Навуходоносора, царя Вавилонского, раба Моего, и даже зверей полевых отдаю ему на служение.
\vs Jer 27:7 И все народы будут служить ему и сыну его и сыну сына его, доколе не придет время и его земле и ему самому; и будут служить ему народы многие и цари великие.
\vs Jer 27:8 И если какой народ и царство не захочет служить ему, Навуходоносору, царю Вавилонскому, и не подклонит выи своей под ярмо царя Вавилонского,~--- этот народ Я накажу мечом, голодом и моровою язвою, говорит Господь, доколе не истреблю их рукою его.
\vs Jer 27:9 И вы не слушайте своих пророков и своих гадателей, и своих сновидцев, и своих волшебников, и своих звездочетов, которые говорят вам: <<не будете служить царю Вавилонскому>>.
\vs Jer 27:10 Ибо они пророчествуют вам ложь, чтобы удалить вас из земли вашей, и чтобы Я изгнал вас и вы погибли.
\vs Jer 27:11 Народ же, который подклонит выю свою под ярмо царя Вавилонского и станет служить ему, Я оставлю на земле своей, говорит Господь, и он будет возделывать ее и жить на ней.
\vs Jer 27:12 И Седекии, царю Иудейскому, я говорил всеми сими словами и сказал: подклоните выю свою под ярмо царя Вавилонского и служите ему и народу его, и будете живы.
\vs Jer 27:13 Зачем умирать тебе и народу твоему от меча, голода и моровой язвы, как изрек Господь о том народе, который не будет служить царю Вавилонскому?
\vs Jer 27:14 И не слушайте слов пророков, которые говорят вам: <<не будете служить царю Вавилонскому>>; ибо они пророчествуют вам ложь.
\vs Jer 27:15 Я не посылал их, говорит Господь; и они ложно пророчествуют именем Моим, чтоб Я изгнал вас и чтобы вы погибли,~--- вы и пророки ваши, пророчествующие вам.
\vs Jer 27:16 И священникам и всему народу сему я говорил: так говорит Господь: не слушайте слов пророков ваших, которые пророчествуют вам и говорят: <<вот, скоро возвращены будут из Вавилона сосуды дома Господня>>; ибо они пророчествуют вам ложь.
\vs Jer 27:17 Не слушайте их, служите царю Вавилонскому и живите; зачем доводить город сей до опустошения?
\vs Jer 27:18 А если они пророки, и если у них есть слово Господне, то пусть ходатайствуют пред Господом Саваофом, чтобы сосуды, остающиеся в доме Господнем и в доме царя Иудейского и в Иерусалиме, не перешли в Вавилон.
\vs Jer 27:19 Ибо так говорит Господь Саваоф о столбах и о \bibemph{медном} море и о подножиях и о прочих вещах, оставшихся в этом городе,
\vs Jer 27:20 которых Навуходоносор, царь Вавилонский, не взял, когда Иехонию, сына Иоакима, царя Иудейского, и всех знатных Иудеев и Иерусалимлян вывел из Иерусалима в Вавилон,
\vs Jer 27:21 ибо так говорит Господь Саваоф, Бог Израилев, о сосудах, оставшихся в доме Господнем и в доме царя Иудейского и в Иерусалиме:
\vs Jer 27:22 они будут отнесены в Вавилон и там останутся до того дня, когда Я посещу их, говорит Господь, и выведу их и возвращу их на место сие.
\vs Jer 28:1 В тот же год, в начале царствования Седекии, царя Иудейского, в четвертый год, в пятый месяц, Анания, сын Азура, пророк из Гаваона, говорил мне в доме Господнем пред глазами священников и всего народа и сказал:
\vs Jer 28:2 так говорит Господь Саваоф, Бог Израилев: сокрушу ярмо царя Вавилонского;
\vs Jer 28:3 через два года Я возвращу на место сие все сосуды дома Господня, которые Навуходоносор, царь Вавилонский, взял из сего места и перенес их в Вавилон;
\vs Jer 28:4 и Иехонию, сына Иоакима, царя Иудейского, и всех пленных Иудеев, пришедших в Вавилон, Я возвращу на место сие, говорит Господь; ибо сокрушу ярмо царя Вавилонского.
\vs Jer 28:5 И сказал Иеремия пророк пророку Анании пред глазами священников и пред глазами всего народа, стоявших в доме Господнем,~---
\vs Jer 28:6 и сказал Иеремия пророк: да будет так, да сотворит сие Господь! да исполнит Господь слова твои, какие ты произнес о возвращении из Вавилона сосудов дома Господня и всех пленников на место сие!
\vs Jer 28:7 Только выслушай слово сие, которое я скажу вслух тебе и вслух всего народа:
\vs Jer 28:8 пророки, которые издавна были прежде меня и прежде тебя, предсказывали многим землям и великим царствам войну и бедствие и мор.
\vs Jer 28:9 Если какой пророк предсказывал мир, то тогда только он признаваем был за пророка, которого истинно послал Господь, когда сбывалось слово того пророка.
\vs Jer 28:10 Тогда пророк Анания взял ярмо с выи Иеремии пророка и сокрушил его.
\vs Jer 28:11 И сказал Анания пред глазами всего народа сии слова: так говорит Господь: так сокрушу ярмо Навуходоносора, царя Вавилонского, через два года, \bibemph{сняв его} с выи всех народов. И пошел Иеремия своею дорогою.
\rsbpar\vs Jer 28:12 И было слово Господне к Иеремии после того, как пророк Анания сокрушил ярмо с выи пророка Иеремии:
\vs Jer 28:13 иди и скажи Анании: так говорит Господь: ты сокрушил ярмо деревянное, и сделаешь вместо него ярмо железное.
\vs Jer 28:14 Ибо так говорит Господь Саваоф, Бог Израилев: железное ярмо возложу на выю всех этих народов, чтобы они работали Навуходоносору, царю Вавилонскому, и они будут служить ему, и даже зверей полевых Я отдал ему.
\vs Jer 28:15 И сказал пророк Иеремия пророку Анании: послушай, Анания: Господь тебя не посылал, и ты обнадеживаешь народ сей ложно.
\vs Jer 28:16 Посему так говорит Господь: вот, Я сброшу тебя с лица земли; в этом же году ты умрешь, потому что ты говорил вопреки Господу.
\vs Jer 28:17 И умер пророк Анания в том же году, в седьмом месяце.
\vs Jer 29:1 И вот слова письма, которое пророк Иеремия послал из Иерусалима к остатку старейшин между переселенцами и к священникам, и к пророкам, и ко всему народу, которых Навуходоносор вывел из Иерусалима в Вавилон,~---
\vs Jer 29:2 после того, как вышли из Иерусалима царь Иехония и царица и евнухи, князья Иудеи и Иерусалима, и плотники и кузнецы,~---
\vs Jer 29:3 через Елеасу, сына Сафанова, и Гемарию, сына Хелкиина, которых Седекия, царь Иудейский, посылал в Вавилон к Навуходоносору, царю Вавилонскому:
\vs Jer 29:4 так говорит Господь Саваоф, Бог Израилев, всем пленникам, которых Я переселил из Иерусалима в Вавилон:
\vs Jer 29:5 стройте домы и живите \bibemph{в них}, и разводите сады и ешьте плоды их;
\vs Jer 29:6 берите жен и рождайте сыновей и дочерей; и сыновьям своим берите жен и дочерей своих отдавайте в замужество, чтобы они рождали сыновей и дочерей, и размножайтесь там, а не умаляйтесь;
\vs Jer 29:7 и заботьтесь о благосостоянии города, в который Я переселил вас, и молитесь за него Господу; ибо при благосостоянии его и вам будет мир.
\vs Jer 29:8 Ибо так говорит Господь Саваоф, Бог Израилев: да не обольщают вас пророки ваши, которые среди вас, и гадатели ваши; и не слушайте снов ваших, которые вам снятся;
\vs Jer 29:9 ложно пророчествуют они вам именем Моим; Я не посылал их, говорит Господь.
\vs Jer 29:10 Ибо так говорит Господь: когда исполнится вам в Вавилоне семьдесят лет, тогда Я посещу вас и исполню доброе слово Мое о вас, чтобы возвратить вас на место сие.
\vs Jer 29:11 Ибо \bibemph{только} Я знаю намерения, какие имею о вас, говорит Господь, намерения во благо, а не на зло, чтобы дать вам будущность и надежду.
\vs Jer 29:12 И воззовете ко Мне, и пойдете и помолитесь Мне, и Я услышу вас;
\vs Jer 29:13 и взыщете Меня и найдете, если взыщете Меня всем сердцем вашим.
\vs Jer 29:14 И буду Я найден вами, говорит Господь, и возвращу вас из плена и соберу вас из всех народов и из всех мест, куда Я изгнал вас, говорит Господь, и возвращу вас в то место, откуда переселил вас.
\vs Jer 29:15 Вы говорите: <<Господь воздвиг нам пророков и в Вавилоне>>.
\rsbpar\vs Jer 29:16 Так говорит Господь о царе, сидящем на престоле Давидовом, и о всем народе, живущем в городе сем, о братьях ваших, которые не отведены с вами в плен,~---
\vs Jer 29:17 так говорит \bibemph{о них} Господь Саваоф: вот, Я пошлю на них меч, голод и моровую язву, и сделаю их такими, как негодные смоквы, которых нельзя есть по негодности \bibemph{их};
\vs Jer 29:18 и буду преследовать их мечом, голодом и моровою язвою, и предам их на озлобление всем царствам земли, на проклятие и ужас, на посмеяние и поругание между всеми народами, куда Я изгоню их,
\vs Jer 29:19 за то, что они не слушали слов Моих, говорит Господь, с которыми Я посылал к ним рабов Моих, пророков, посылал с раннего утра, но они не слушали, говорит Господь.
\rsbpar\vs Jer 29:20 А вы, все переселенцы, которых Я послал из Иерусалима в Вавилон, слушайте слово Господне:
\vs Jer 29:21 так говорит Господь Саваоф, Бог Израилев, об Ахаве, сыне Колии, и о Седекии, сыне Маасеи, которые пророчествуют вам именем Моим ложь: вот, Я предам их в руки Навуходоносора, царя Вавилонского, и он умертвит их пред вашими глазами.
\vs Jer 29:22 И принято будет от них всеми переселенцами Иудейскими, которые в Вавилоне, проклинать так: <<да соделает тебе Господь то же, что Седекии и Ахаву>>, которых царь Вавилонский изжарил на огне
\vs Jer 29:23 за то, что они делали гнусное в Израиле: прелюбодействовали с женами ближних своих и именем Моим говорили ложь, чего Я не повелевал им; Я знаю это, и Я свидетель, говорит Господь.
\vs Jer 29:24 И Шемаии Нехеламитянину скажи:
\vs Jer 29:25 так говорит Господь Саваоф, Бог Израилев: за то, что ты посылал письма от имени своего ко всему народу, который в Иерусалиме, и к священнику Софонии, сыну Маасеи, и ко всем священникам, и писал:
\vs Jer 29:26 <<Господь поставил тебя священником вместо священника Иодая, чтобы ты был между блюстителями в доме Господнем за всяким человеком, неистовствующим и пророчествующим, и чтобы ты сажал такого в темницу и в колоду:
\vs Jer 29:27 почему же ты не запретишь Иеремии Анафофскому пророчествовать у вас?
\vs Jer 29:28 Ибо он и к нам в Вавилон прислал сказать: плен будет продолжителен: стройте домы и живите в них; разводите сады и ешьте плоды их>>.
\vs Jer 29:29 Когда Софония священник прочитал это письмо вслух пророка Иеремии,
\vs Jer 29:30 тогда было слово Господне к Иеремии:
\vs Jer 29:31 пошли ко всем переселенцам сказать: так говорит Господь о Шемаии Нехеламитянине: за то, что Шемаия у вас пророчествует, а Я не посылал его, и обнадеживает вас ложно,~---
\vs Jer 29:32 за то, так говорит Господь: вот, Я накажу Шемаию Нехеламитянина и племя его; не будет от него человека, живущего среди народа сего, и не увидит он того добра, которое Я сделаю народу Моему, говорит Господь; ибо он говорил вопреки Господу.
\vs Jer 30:1 Слово, которое было к Иеремии от Господа:
\vs Jer 30:2 так говорит Господь, Бог Израилев: напиши себе все слова, которые Я говорил тебе, в книгу.
\vs Jer 30:3 Ибо вот, наступают дни, говорит Господь, когда Я возвращу из плена народ Мой, Израиля и Иуду, говорит Господь; и приведу их опять в ту землю, которую дал отцам их, и они будут владеть ею.
\vs Jer 30:4 И вот те слова, которые сказал Господь об Израиле и Иуде.
\vs Jer 30:5 Так сказал Господь: голос смятения и ужаса слышим мы, а не мира.
\vs Jer 30:6 Спрос\acc{и}те и рассуд\acc{и}те: рождает ли мужчина? Почему же Я вижу у каждого мужчины руки на чреслах его, как у женщины в родах, и лица у всех бледные?
\vs Jer 30:7 О, горе! велик тот день, не было подобного ему; это~--- бедственное время для Иакова, но он будет спасен от него.
\vs Jer 30:8 И будет в тот день, говорит Господь Саваоф: сокрушу ярмо его, которое на вые твоей, и узы твои разорву; и не будут уже служить чужеземцам,
\vs Jer 30:9 но будут служить Господу Богу своему и Давиду, царю своему, которого Я восстановлю им.
\vs Jer 30:10 И ты, раб Мой Иаков, не бойся, говорит Господь, и не страшись, Израиль; ибо вот, Я спасу тебя из далекой страны и племя твое из земли пленения их; и возвратится Иаков и будет жить спокойно и мирно, и никто не будет устрашать его,
\vs Jer 30:11 ибо Я с тобою, говорит Господь, чтобы спасать тебя: Я совершенно истреблю все народы, среди которых рассеял тебя, а тебя не истреблю; Я буду наказывать тебя в мере, но ненаказанным не оставлю тебя.
\vs Jer 30:12 Ибо так говорит Господь: рана твоя неисцельна, язва твоя жестока;
\vs Jer 30:13 никто не заботится о деле твоем, чтобы заживить рану твою; целебного врачевства нет для тебя;
\vs Jer 30:14 все друзья твои забыли тебя, не ищут тебя; ибо Я поразил тебя ударами неприятельскими, жестоким наказанием за множество беззаконий твоих, потому что грехи твои умножились.
\vs Jer 30:15 Что вопиешь ты о ранах твоих, о жестокости болезни твоей? по множеству беззаконий твоих Я сделал тебе это, потому что грехи твои умножились.
\vs Jer 30:16 Но все пожирающие тебя будут пожраны; и все враги твои, все сами пойдут в плен, и опустошители твои будут опустошены, и всех грабителей твоих предам грабежу.
\vs Jer 30:17 Я обложу тебя пластырем и исцелю тебя от ран твоих, говорит Господь. Тебя называли отверженным, говоря: <<вот Сион, о котором никто не спрашивает>>;
\vs Jer 30:18 так говорит Господь: вот, возвращу плен шатров Иакова и селения его помилую; и город опять будет построен на холме своем, и храм устроится по-прежнему.
\vs Jer 30:19 И вознесутся из них благодарение и голос веселящихся; и Я умножу их, и не будут умаляться, и прославлю их, и не будут унижены.
\vs Jer 30:20 И сыновья его будут, как прежде, и сонм его будет предстоять предо Мною, и накажу всех притеснителей его.
\vs Jer 30:21 И будет вождь его из него самого, и владыка его произойдет из среды его; и Я приближу его, и он приступит ко Мне; ибо кто отважится сам собою приблизиться ко Мне? говорит Господь.
\vs Jer 30:22 И вы будете Моим народом, и Я буду вам Богом.
\vs Jer 30:23 Вот, яростный вихрь идет от Господа, вихрь грозный; он падет на голову нечестивых.
\vs Jer 30:24 Пламенный гнев Господа не отвратится, доколе Он не совершит и не выполнит намерений сердца Своего. В последние дни уразумеете это.
\vs Jer 31:1 В то время, говорит Господь, Я буду Богом всем племенам Израилевым, а они будут Моим народом.
\vs Jer 31:2 Так говорит Господь: народ, уцелевший от меча, нашел милость в пустыне; иду успокоить Израиля.
\vs Jer 31:3 Издали явился мне Господь и сказал: любовью вечною Я возлюбил тебя и потому простер к тебе благоволение.
\vs Jer 31:4 Я снова устрою тебя, и ты будешь устроена, дева Израилева, снова будешь украшаться тимпанами твоими и выходить в хороводе веселящихся;
\vs Jer 31:5 снова разведешь виноградники на горах Самарии; виноградари, которые будут разводить их, сами будут и пользоваться ими.
\vs Jer 31:6 Ибо будет день, когда стражи на горе Ефремовой провозгласят: <<вставайте, и взойдем на Сион к Господу Богу нашему>>.
\vs Jer 31:7 Ибо так говорит Господь: радостно пойте об Иакове и восклицайте пред главою народов: провозглашайте, славьте и говорите: <<спаси, Господи, народ твой, остаток Израиля!>>
\vs Jer 31:8 Вот, Я приведу их из страны северной и соберу их с краев земли; слепой и хромой, беременная и родильница вместе с ними,~--- великий сонм возвратится сюда.
\vs Jer 31:9 Они пошли со слезами, а Я поведу их с утешением; поведу их близ потоков вод дорогою ровною, на которой не споткнутся; ибо Я~--- отец Израилю, и Ефрем~--- первенец Мой.
\vs Jer 31:10 Слушайте слово Господне, народы, и возвестите островам отдаленным и скажите: <<Кто рассеял Израиля, Тот и соберет его, и будет охранять его, как пастырь стадо свое>>;
\vs Jer 31:11 ибо искупит Господь Иакова и избавит его от руки того, кто был сильнее его.
\vs Jer 31:12 И придут они, и будут торжествовать на высотах Сиона; и стекутся к благостыне Господа, к пшенице и вину и елею, к агнцам и волам; и душа их будет как напоенный водою сад, и они не будут уже более томиться.
\vs Jer 31:13 Тогда девица будет веселиться в хороводе, и юноши и старцы вместе; и изменю печаль их на радость и утешу их, и обрадую их после скорби их.
\vs Jer 31:14 И напитаю душу священников туком, и народ Мой насытится благами Моими, говорит Господь.
\rsbpar\vs Jer 31:15 Так говорит Господь: голос слышен в Раме, вопль и горькое рыдание; Рахиль плачет о детях своих и не хочет утешиться о детях своих, ибо их нет.
\vs Jer 31:16 Так говорит Господь: удержи голос твой от рыдания и глаза твои от слез, ибо есть награда за труд твой, говорит Господь, и возвратятся они из земли неприятельской.
\vs Jer 31:17 И есть надежда для будущности твоей, говорит Господь, и возвратятся сыновья твои в пределы свои.
\vs Jer 31:18 Слышу Ефрема плачущего: <<Ты наказал меня, и я наказан, как телец неукротимый; обрати меня, и обращусь, ибо Ты Господь Бог мой.
\vs Jer 31:19 Когда я был обращен, я каялся, и когда был вразумлен, бил себя по бедрам; я был постыжен, я был смущен, потому что нес бесславие юности моей>>.
\vs Jer 31:20 Не дорог\acc{о}й ли у Меня сын Ефрем? не любимое ли дитя? ибо, как только заговорю о нем, всегда с любовью воспоминаю о нем; внутренность Моя возмущается за него; умилосержусь над ним, говорит Господь.
\vs Jer 31:21 Поставь себе путевые знаки, поставь себе столбы, обрати сердце твое на дорогу, на путь, по которому ты шла; возвращайся, дева Израилева, возвращайся в сии города твои.
\vs Jer 31:22 Долго ли тебе скитаться, отпадшая дочь? Ибо Господь сотворит на земле нечто новое: жена спасет мужа.
\rsbpar\vs Jer 31:23 Так говорит Господь Саваоф, Бог Израилев: впредь, когда Я возвращу плен их, будут говорить на земле Иуды и в городах его сие слово: <<да благословит тебя Господь, жилище правды, гора святая!>>
\vs Jer 31:24 И поселится на ней Иуда и все города его вместе, земледельцы и ходящие со стадами.
\vs Jer 31:25 Ибо Я напою душу утомленную и насыщу всякую душу скорбящую.
\vs Jer 31:26 При этом я пробудился и посмотрел, и сон мой был приятен мне.
\vs Jer 31:27 Вот, наступают дни, говорит Господь, когда Я засею дом Израилев и дом Иудин семенем человека и семенем скота.
\vs Jer 31:28 И как Я наблюдал за ними, искореняя и сокрушая, и разрушая и погубляя, и повреждая, так буду наблюдать за ними, созидая и насаждая, говорит Господь.
\vs Jer 31:29 В те дни уже не будут говорить: <<отцы ели кислый виноград, а у детей на зубах оскомина>>,
\vs Jer 31:30 но каждый будет умирать за свое собственное беззаконие; кто будет есть кислый виноград, у того на зубах и оскомина будет.
\rsbpar\vs Jer 31:31 Вот наступают дни, говорит Господь, когда Я заключу с домом Израиля и с домом Иуды новый завет,
\vs Jer 31:32 не такой завет, какой Я заключил с отцами их в тот день, когда взял их за руку, чтобы вывести их из земли Египетской; тот завет Мой они нарушили, хотя Я оставался в союзе с ними, говорит Господь.
\vs Jer 31:33 Но вот завет, который Я заключу с домом Израилевым после тех дней, говорит Господь: вложу закон Мой во внутренность их и на сердцах их напишу его, и буду им Богом, а они будут Моим народом.
\vs Jer 31:34 И уже не будут учить друг друга, брат брата, и говорить: <<познайте Господа>>, ибо все сами будут знать Меня, от малого до большого, говорит Господь, потому что Я прощу беззакония их и грехов их уже не воспомяну более.
\vs Jer 31:35 Так говорит Господь, Который дал солнце для освещения днем, уставы луне и звездам для освещения ночью, Который возмущает море, так что волны его ревут; Господь Саваоф~--- имя Ему.
\vs Jer 31:36 Если сии уставы перестанут действовать предо Мною, говорит Господь, то и племя Израилево перестанет быть народом предо Мною навсегда.
\vs Jer 31:37 Так говорит Господь: если небо может быть измерено вверху, и основания земли исследованы внизу, то и Я отвергну все племя Израилево за все то, что они делали, говорит Господь.
\vs Jer 31:38 Вот, наступают дни, говорит Господь, когда город устроен будет во славу Господа от башни Анамеила до ворот уг\acc{о}льных,
\vs Jer 31:39 и землемерная вервь пойдет далее прямо до холма Гарива и обойдет Гоаф.
\vs Jer 31:40 И вся долина трупов и пепла, и все поле до потока Кедрона, до угла конских ворот к востоку, будет святынею Господа; не разрушится и не распадется вовеки.
\vs Jer 32:1 Слово, которое было от Господа к Иеремии в десятый год Седекии, царя Иудейского; этот год был восемнадцатым годом Навуходоносора.
\vs Jer 32:2 Тогда войско царя Вавилонского осаждало Иерусалим, и Иеремия пророк был заключен во дворе стражи, который был при доме царя Иудейского.
\vs Jer 32:3 Седекия, царь Иудейский, заключил его туда, сказав: <<зачем ты пророчествуешь и говоришь: так говорит Господь: вот, Я отдаю город сей в руки царя Вавилонского, и он возьмет его;
\vs Jer 32:4 и Седекия, царь Иудейский, не избегнет от рук Халдеев, но непременно предан будет в руки царя Вавилонского, и будет говорить с ним устами к устам, и глаза его увидят глаза его;
\vs Jer 32:5 и он отведет Седекию в Вавилон, где он и будет, доколе не посещу его, говорит Господь. Если вы будете воевать с Халдеями, то не будете иметь успеха?>>
\rsbpar\vs Jer 32:6 И сказал Иеремия: таково было ко мне слово Господне:
\vs Jer 32:7 вот Анамеил, сын Саллума, дяди твоего, идет к тебе сказать: <<купи себе поле мое, которое в Анафофе, потому что по праву родства тебе надлежит купить его>>.
\vs Jer 32:8 И Анамеил, сын дяди моего, пришел ко мне, по слову Господню, во двор стражи и сказал мне: <<купи поле мое, которое в Анафофе, в земле Вениаминовой, ибо право наследства твое и право выкупа твое; купи себе>>. Тогда я узнал, что это было слово Господне.
\vs Jer 32:9 И купил я поле у Анамеила, сына дяди моего, которое в Анафофе, и отвесил ему семь сиклей серебра и десять сребреников;
\vs Jer 32:10 и записал в книгу и запечатал ее, и пригласил к тому свидетелей и отвесил серебро на весах.
\vs Jer 32:11 И взял я купчую запись, как запечатанную по закону и уставу, так и открытую;
\vs Jer 32:12 и отдал эту купчую запись Варуху, сыну Нирии, сына Маасеи, в глазах Анамеила, сына дяди моего, и в глазах свидетелей, подписавших эту купчую запись, в глазах всех Иудеев, сидевших на дворе стражи;
\vs Jer 32:13 и заповедал Варуху в присутствии их:
\vs Jer 32:14 так говорит Господь Саваоф, Бог Израилев: возьми сии записи, эту купчую запись, которая запечатана, и эту запись открытую, и положи их в глиняный сосуд, чтобы они оставались там многие дни.
\vs Jer 32:15 Ибо так говорит Господь Саваоф, Бог Израилев: домы и поля и виноградники будут снова покупаемы в земле сей.
\vs Jer 32:16 И, передав купчую запись Варуху, сыну Нирии, я помолился Господу:
\vs Jer 32:17 <<о, Господи Боже! Ты сотворил небо и землю великою силою Твоею и простертою мышцею; для Тебя ничего нет невозможного;
\vs Jer 32:18 Ты являешь милость тысячам и за беззаконие отцов воздаешь в недро детям их после них: Боже великий, сильный, Которому имя Господь Саваоф!
\vs Jer 32:19 Великий в совете и сильный в делах, Которого очи отверсты на все пути сынов человеческих, чтобы воздавать каждому по путям его и по плодам дел его,
\vs Jer 32:20 Который совершил чудеса и знамения в земле Египетской, \bibemph{и совершаешь} до сего дня и в Израиле и между всеми людьми, и соделал Себе имя, как в сей день,
\vs Jer 32:21 и вывел народ Твой Израиля из земли Египетской знамениями и чудесами, и рукою сильною и мышцею простертою, при великом ужасе,
\vs Jer 32:22 и дал им землю сию, которую дать им клятвенно обещал отцам их, землю, текущую молоком и медом.
\vs Jer 32:23 Они вошли и завладели ею, но не стали слушать гласа Твоего и поступать по закону Твоему, не стали делать того, что Ты заповедал им делать, и за то Ты навел на них все это бедствие.
\vs Jer 32:24 Вот, насыпи достигают до города, чтобы взять его; и город от меча и голода и моровой язвы отдается в руки Халдеев, воюющих против него; что Ты говорил, то и исполняется, и вот, Ты видишь это.
\vs Jer 32:25 А Ты, Господи Боже, сказал мне: <<купи себе поле за серебро и пригласи свидетелей>>, тогда как город отдается в руки Халдеев>>.
\rsbpar\vs Jer 32:26 И было слово Господне к Иеремии:
\vs Jer 32:27 вот, Я Господь, Бог всякой плоти; есть ли что невозможное для Меня?
\vs Jer 32:28 Посему так говорит Господь: вот, Я отдаю город сей в руки Халдеев и в руки Навуходоносора, царя Вавилонского, и он возьмет его,
\vs Jer 32:29 и войдут Халдеи, осаждающие сей город, зажгут город огнем и сожгут его и домы, на кровлях которых возносились курения Ваалу и возливаемы были возлияния чужим богам, чтобы прогневлять Меня.
\vs Jer 32:30 Ибо сыновья Израилевы и сыновья Иудины только зло делали пред очами Моими от юности своей; сыновья Израилевы только прогневляли Меня делами рук своих, говорит Господь.
\vs Jer 32:31 И как бы для гнева Моего и ярости Моей существовал город сей с самого дня построения его до сего дня, чтобы Я отверг его от лица Моего
\vs Jer 32:32 за все зло сыновей Израиля и сыновей Иуды, какое они к прогневлению Меня делали, они, цари их, князья их, священники их и пророки их, и мужи Иуды и жители Иерусалима.
\vs Jer 32:33 Они оборотились ко Мне спиною, а не лицем; и когда Я учил их, с раннего утра учил, они не хотели принять наставления,
\vs Jer 32:34 и в доме, над которым наречено имя Мое, поставили мерзости свои, оскверняя его.
\vs Jer 32:35 Устроили капища Ваалу в долине сыновей Енномовых, чтобы проводить через огонь сыновей своих и дочерей своих в честь Молоху, чего Я не повелевал им, и Мне на ум не приходило, чтобы они делали эту мерзость, вводя в грех Иуду.
\vs Jer 32:36 И однако же ныне так говорит Господь, Бог Израилев, об этом городе, о котором вы говорите: <<он предается в руки царя Вавилонского мечом и голодом и моровою язвою>>,~---
\vs Jer 32:37 вот, Я соберу их из всех стран, в которые изгнал их во гневе Моем и в ярости Моей и в великом негодовании, и возвращу их на место сие и дам им безопасное житие.
\vs Jer 32:38 Они будут Моим народом, а Я буду им Богом.
\vs Jer 32:39 И дам им одно сердце и один путь, чтобы боялись Меня во все дни \bibemph{жизни}, ко благу своему и благу детей своих после них.
\vs Jer 32:40 И заключу с ними вечный завет, по которому Я не отвращусь от них, чтобы благотворить им, и страх Мой вложу в сердца их, чтобы они не отступали от Меня.
\vs Jer 32:41 И буду радоваться о них, благотворя им, и насажду их на земле сей твердо, от всего сердца Моего и от всей души Моей.
\vs Jer 32:42 Ибо так говорит Господь: как Я навел на народ сей все это великое зло, так наведу на них все благо, какое Я изрек о них.
\vs Jer 32:43 И будут покупать поля в земле сей, о которой вы говорите: <<это пустыня, без людей и без скота; она отдана в руки Халдеям>>;
\vs Jer 32:44 будут покупать поля за серебро и вносить в записи, и запечатывать и приглашать свидетелей~--- в земле Вениаминовой и в окрестностях Иерусалима, и в городах Иуды и в городах нагорных, и в городах низменных и в городах южных; ибо возвращу плен их, говорит Господь.
\vs Jer 33:1 И было слово Господне к Иеремии вторично, когда он еще содержался во дворе стражи:
\vs Jer 33:2 Так говорит Господь, Который сотворил [землю], Господь, Который устроил и утвердил ее,~--- Господь имя Ему:
\vs Jer 33:3 воззови ко Мне~--- и Я отвечу тебе, покажу тебе великое и недоступное, чего ты не знаешь.
\vs Jer 33:4 Ибо так говорит Господь, Бог Израилев, о домах города сего и о домах царей Иудейских, которые разрушаются для завалов и для сражения
\vs Jer 33:5 пришедшими воевать с Халдеями, чтобы наполнить домы трупами людей, которых Я поражу во гневе Моем и в ярости Моей, и за все беззакония которых Я сокрыл лице Мое от города сего.
\vs Jer 33:6 Вот, Я приложу ему пластырь и целебные средства, и уврачую их, и открою им обилие мира и истины,
\vs Jer 33:7 и возвращу плен Иуды и плен Израиля и устрою их, как вначале,
\vs Jer 33:8 и очищу их от всего нечестия их, которым они грешили предо Мною, и прощу все беззакония их, которыми они грешили предо Мною и отпали от Меня.
\vs Jer 33:9 И будет для меня \bibemph{Иерусалим} радостным именем, похвалою и честью пред всеми народами земли, которые услышат о всех благах, какие Я сделаю ему, и изумятся и затрепещут от всех благодеяний и всего благоденствия, которое Я доставлю ему.
\rsbpar\vs Jer 33:10 Так говорит Господь: на этом месте, о котором вы говорите: <<оно пусто, без людей и без скота>>,~--- в городах Иудейских и на улицах Иерусалима, которые пусты, без людей, без жителей, без скота,
\vs Jer 33:11 опять будет слышен голос радости и голос веселья, голос жениха и голос невесты, голос говорящих: <<славьте Господа Саваофа, ибо благ Господь, ибо вовек милость Его>>, и голос приносящих жертву благодарения в доме Господнем; ибо Я возвращу плененных сей земли в прежнее состояние, говорит Господь.
\vs Jer 33:12 Так говорит Господь Саваоф: на этом месте, которое пусто, без людей, без скота, и во всех городах его опять будут жилища пастухов, которые будут покоить стада.
\vs Jer 33:13 В городах нагорных, в городах низменных и в городах южных, и в земле Вениаминовой, и в окрестностях Иерусалима, и в городах Иуды опять будут проходить стада под рукою считающего, говорит Господь.
\vs Jer 33:14 Вот, наступят дни, говорит Господь, когда Я выполню то доброе слово, которое изрек о доме Израилевом и о доме Иудином.
\vs Jer 33:15 В те дни и в то время возращу Давиду Отрасль праведную, и будет производить суд и правду на земле.
\vs Jer 33:16 В те дни Иуда будет спасен и Иерусалим будет жить безопасно, и нарекут имя Ему: <<Господь оправдание наше!>>
\vs Jer 33:17 Ибо так говорит Господь: не прекратится у Давида муж, сидящий на престоле дома Израилева,
\vs Jer 33:18 и у священников-левитов не будет недостатка в муже пред лицем Моим, во все дни возносящем всесожжение и сожигающем приношения и совершающем жертвы.
\rsbpar\vs Jer 33:19 И было слово Господне к Иеремии:
\vs Jer 33:20 так говорит Господь: если можете разрушить завет Мой о дне и завет Мой о ночи, чтобы день и ночь не приходили в свое время,
\vs Jer 33:21 то может быть разрушен и завет Мой с рабом Моим Давидом, так что не будет у него сына, царствующего на престоле его, и также с левитами-священниками, служителями Моими.
\vs Jer 33:22 Как неисчислимо небесное воинство и неизмерим песок морской, так размножу племя Давида, раба Моего, и левитов, служащих Мне.
\vs Jer 33:23 И было слово Господне к Иеремии:
\vs Jer 33:24 не видишь ли, что народ этот говорит: <<те два племени, которые избрал Господь, Он отверг?>> и чрез это они презирают народ Мой, как бы он уже не был народом в глазах их.
\vs Jer 33:25 Так говорит Господь: если завета Моего о дне и ночи и уставов неба и земли Я не утвердил,
\vs Jer 33:26 то и племя Иакова и Давида, раба Моего, отвергну, чтобы не брать более владык из его племени для племени Авраама, Исаака и Иакова; ибо возвращу плен их и помилую их.
\vs Jer 34:1 Слово, которое было к Иеремии от Господа, когда Навуходоносор, царь Вавилонский, и все войско его и все царства земли, подвластные руке его, и все народы воевали против Иерусалима и против всех городов его:
\vs Jer 34:2 так говорит Господь, Бог Израилев: иди и скажи Седекии, царю Иудейскому, и скажи ему: так говорит Господь: вот, Я отдаю город сей в руки царя Вавилонского, и он сожжет его огнем;
\vs Jer 34:3 и ты не избежишь от руки его, но непременно будешь взят и предан в руки его, и глаза твои увидят глаза царя Вавилонского, и уста его будут говорить твоим устам, и пойдешь в Вавилон.
\vs Jer 34:4 Впрочем слушай слово Господне, Седекия, царь Иудейский! так говорит Господь о тебе: ты не умрешь от меча;
\vs Jer 34:5 ты умрешь в мире, и как для отцов твоих, прежних царей, которые были прежде тебя, сожигали \bibemph{при погребении благовония}, так сожгут и для тебя и оплачут тебя: <<увы, государь!>>, ибо Я изрек это слово, говорит Господь.
\vs Jer 34:6 Иеремия пророк все слова сии пересказал Седекии, царю Иудейскому, в Иерусалиме.
\vs Jer 34:7 Между тем войско царя Вавилонского воевало против Иерусалима и против всех городов Иудейских, которые еще оставались, против Лахиса и Азеки; ибо из городов Иудейских сии только оставались, как города укрепленные.
\vs Jer 34:8 Слово, которое было к Иеремии от Господа после того, как царь Седекия заключил завет со всем народом, бывшим в Иерусалиме, чтобы объявить свободу,
\vs Jer 34:9 чтобы каждый отпустил на волю раба своего и рабу свою, Еврея и Евреянку, чтобы никто из них не держал в рабстве Иудея, брата своего.
\vs Jer 34:10 И послушались все князья и весь народ, которые вступили в завет, чтобы отпустить каждому раба своего и каждому рабу свою на волю, чтобы не держать их впредь в рабах,~--- и послушались и отпустили;
\vs Jer 34:11 но после того, раздумавши, стали брать назад рабов и рабынь, которых отпустили на волю, и принудили их быть рабами и рабынями.
\rsbpar\vs Jer 34:12 И было слово Господне к Иеремии от Господа:
\vs Jer 34:13 так говорит Господь, Бог Израилев: Я заключил завет с отцами вашими, когда вывел их из земли Египетской, из дома рабства, и сказал:
\vs Jer 34:14 <<в конце седьмого года отпускайте каждый брата своего, Еврея, который продал себя тебе; пусть он работает тебе шесть лет, а потом отпусти его от себя на волю>>; но отцы ваши не послушали Меня и не приклонили уха своего.
\vs Jer 34:15 Вы ныне обратились и поступили справедливо пред очами Моими, объявив каждый свободу ближнему своему, и заключили предо Мною завет в доме, над которым наречено имя Мое;
\vs Jer 34:16 но потом раздумали и обесславили имя Мое, и возвратили к себе каждый раба своего и каждый рабу свою, которых отпустили на волю, куда душе их угодно, и принуждаете их быть у вас рабами и рабынями.
\vs Jer 34:17 Посему так говорит Господь: вы не послушались Меня в том, чтобы каждый объявил свободу брату своему и ближнему своему; за то вот Я, говорит Господь, объявляю вам свободу подвергнуться мечу, моровой язве и голоду, и отдам вас на озлобление во все царства земли;
\vs Jer 34:18 и отдам преступивших завет Мой и не устоявших в словах завета, который они заключили пред лицем Моим, рассекши тельца надвое и пройдя между рассеченными частями его,
\vs Jer 34:19 князей Иудейских и князей Иерусалимских, евнухов и священников и весь народ земли, проходивший между рассеченными частями тельца,~---
\vs Jer 34:20 отдам их в руки врагов их и в руки ищущих душ\acc{и} их, и трупы их будут пищею птицам небесным и зверям земным.
\vs Jer 34:21 И Седекию, царя Иудейского, и князей его отдам в руки врагов их и в руки ищущих души их и в руки войска царя Вавилонского, которое отступило от вас.
\vs Jer 34:22 Вот, Я дам повеление, говорит Господь, и возвращу их к этому городу, и они нападут на него, и возьмут его, и сожгут его огнем, и города Иудеи сделаю пустынею необитаемою.
\vs Jer 35:1 Слово, которое было к Иеремии от Господа во дни Иоакима, сына Иосии, царя Иудейского:
\vs Jer 35:2 иди в дом Рехавитов и поговори с ними, и приведи их в дом Господень, в одну из комнат, и дай им пить вина.
\vs Jer 35:3 Я взял Иазанию, сына Иеремии, сына Авацинии, и братьев его, и всех сыновей его и весь дом Рехавитов,
\vs Jer 35:4 и привел их в дом Господень, в комнату сынов Анана, сына Годолии, человека Божия, которая подле комнаты князей, над комнатою Маасеи, сына Селлумова, стража у входа;
\vs Jer 35:5 и поставил перед сынами дома Рехавитов полные чаши вина и стаканы и сказал им: пейте вино.
\vs Jer 35:6 Но они сказали: мы вина не пьем; потому что Ионадав, сын Рехава, отец наш, дал нам заповедь, сказав: <<не пейте вина ни вы, ни дети ваши, вовеки;
\vs Jer 35:7 и домов не стройте, и семян не сейте, и виноградников не разводите, и не имейте их, но живите в шатрах во все дни \bibemph{жизни} вашей, чтобы вам долгое время прожить на той земле, где вы странниками>>.
\vs Jer 35:8 И мы послушались голоса Ионадава, сына Рехавова, отца нашего, во всем, что он завещал нам, чтобы не пить вина во все дни наши,~--- мы и жены наши, и сыновья наши и дочери наши,~---
\vs Jer 35:9 и чтобы не строить домов для жительства нашего; и у нас нет ни виноградников, ни полей, ни посева;
\vs Jer 35:10 а живем в шатрах и во всем слушаемся и делаем все, что заповедал нам Ионадав, отец наш.
\vs Jer 35:11 Когда же Навуходоносор, царь Вавилонский, пришел в землю сию, мы сказали: <<пойдем, уйдем в Иерусалим от войска Халдеев и от войска Арамеев>>, и вот, мы живем в Иерусалиме.
\rsbpar\vs Jer 35:12 И было слово Господне к Иеремии:
\vs Jer 35:13 так говорит Господь Саваоф, Бог Израилев: иди и скажи мужам Иуды и жителям Иерусалима: неужели вы не возьмете из этого наставление для себя, чтобы слушаться слов Моих? говорит Господь.
\vs Jer 35:14 Слова Ионадава, сына Рехавова, который завещал сыновьям своим не пить вина, выполняются, и они не пьют до сего дня, потому что слушаются завещания отца своего; а Я непрестанно говорил вам, говорил с раннего утра, и вы не послушались Меня.
\vs Jer 35:15 Я посылал к вам всех рабов Моих, пророков, посылал с раннего утра, и говорил: <<обратитесь каждый от злого пути своего и исправьте поведение ваше, и не ходите во след иных богов, чтобы служить им; и будете жить на этой земле, которую Я дал вам и отцам вашим>>; но вы не приклонили уха своего и не послушались Меня.
\vs Jer 35:16 Так как сыновья Ионадава, сына Рехавова, выполняют заповедь отца своего, которую он заповедал им, а народ сей не слушает Меня,
\vs Jer 35:17 посему так говорит Господь Бог Саваоф, Бог Израилев: вот, Я наведу на Иудею и на всех жителей Иерусалима все то зло, которое Я изрек на них, потому что Я говорил им, а они не слушались, звал их, а они не отвечали.
\vs Jer 35:18 А дому Рехавитов сказал Иеремия: так говорит Господь Саваоф, Бог Израилев: за то, что вы послушались завещания Ионадава, отца вашего, и храните все заповеди его и во всем поступаете, как он завещал вам,~---
\vs Jer 35:19 за то, так говорит Господь Саваоф, Бог Израилев: не отнимется у Ионадава, сына Рехавова, муж, предстоящий пред лицем Моим во все дни.
\vs Jer 36:1 В четвертый год Иоакима, сына Иосии, царя Иудейского, было такое слово к Иеремии от Господа:
\vs Jer 36:2 возьми себе книжный свиток и напиши в нем все слова, которые Я говорил тебе об Израиле и об Иуде и о всех народах с того дня, как Я начал говорить тебе, от дней Иосии до сего дня;
\vs Jer 36:3 может быть, дом Иудин услышит о всех бедствиях, какие Я помышляю сделать им, чтобы они обратились каждый от злого пути своего, чтобы Я простил неправду их и грех их.
\vs Jer 36:4 И призвал Иеремия Варуха, сына Нирии, и написал Варух в книжный свиток из уст Иеремии все слова Господа, которые Он говорил ему.
\vs Jer 36:5 И приказал Иеремия Варуху и сказал: я заключен и не могу идти в дом Господень;
\vs Jer 36:6 итак иди ты и прочитай написанные тобою в свитке с уст моих слова Господни вслух народа в доме Господнем в день поста, также и вслух всех Иудеев, пришедших из городов своих, прочитай их;
\vs Jer 36:7 может быть, они вознесут смиренное моление пред лице Господа и обратятся каждый от злого пути своего; ибо велик гнев и негодование, которое объявил Господь на народ сей.
\vs Jer 36:8 Варух, сын Нирии, сделал все, что приказал ему пророк Иеремия, чтобы слова Господни, написанные в свитке, прочитать в доме Господнем.
\rsbpar\vs Jer 36:9 В пятый год Иоакима, сына Иосии, царя Иудейского, в девятом месяце объявили пост пред лицем Господа всему народу в Иерусалиме и всему народу, пришедшему в Иерусалим из городов Иудейских.
\vs Jer 36:10 И прочитал Варух написанные в свитке слова Иеремии в доме Господнем, в комнате Гемарии, сына Сафанова, писца, на верхнем дворе, у входа в новые ворота дома Господня, вслух всего народа.
\vs Jer 36:11 Михей, сын Гемарии, сына Сафанова, слышал все слова Господни, \bibemph{написанные} в свитке,
\vs Jer 36:12 и сошел в дом царя, в комнату царского писца, и вот, там сидели все князья: Елисам, царский писец, и Делаия, сын Семаия, и Елнафан, сын Ахбора, и Гемария, сын Сафана, и Седекия, сын Анании, и все князья;
\vs Jer 36:13 и пересказал им Михей все слова, которые он слышал, когда Варух читал свиток вслух народа.
\vs Jer 36:14 Тогда все князья послали к Варуху Иегудия, сына Нафании, сына Селемии, сына Хусии, сказать ему: свиток, который ты читал вслух народа, возьми в руку твою и приди. И взял Варух, сын Нирии, свиток в руку свою и пришел к ним.
\vs Jer 36:15 Они сказали ему: сядь, и прочитай нам вслух. И прочитал Варух вслух им.
\vs Jer 36:16 Когда они выслушали все слова, то с ужасом посмотрели друг на друга и сказали Варуху: мы непременно перескажем все сии слова царю.
\vs Jer 36:17 И спросили Варуха: скажи же нам, как ты написал все слова сии из уст его?
\vs Jer 36:18 И сказал им Варух: он произносил мне устами своими все сии слова, а я чернилами писал их в этот свиток.
\vs Jer 36:19 Тогда сказали князья Варуху: пойди, скройся, ты и Иеремия, чтобы никто не знал, где вы.
\vs Jer 36:20 И пошли они к царю во дворец, а свиток оставили в комнате Елисама, царского писца, и пересказали вслух царя все слова сии.
\vs Jer 36:21 Царь послал Иегудия принести свиток, и он взял его из комнаты Елисама, царского писца; и читал его Иегудий вслух царя и вслух всех князей, стоявших подле царя.
\vs Jer 36:22 Царь в то время, в девятом месяце, сидел в зимнем доме, и перед ним горела жаровня.
\vs Jer 36:23 Когда Иегудий прочитывал три или четыре столбца, \bibemph{царь} отрезывал их писцовым ножичком и бросал на огонь в жаровне, доколе не уничтожен был весь свиток на огне, который был в жаровне.
\vs Jer 36:24 И не убоялись, и не разодрали одежд своих ни царь, ни все слуги его, слышавшие все слова сии.
\vs Jer 36:25 Хотя Елнафан и Делаия и Гемария упрашивали царя не сожигать свитка, но он не послушал их.
\vs Jer 36:26 И приказал царь Иерамеилу, сыну царя, и Сераии, сыну Азриилову, и Селемии, сыну Авдиилову, взять Варуха писца и Иеремию пророка; но Господь сокрыл их.
\rsbpar\vs Jer 36:27 И было слово Господне к Иеремии, после того как царь сожег свиток и слова, которые Варух написал из уст Иеремии, и сказано ему:
\vs Jer 36:28 возьми себе опять другой свиток и напиши в нем все прежние слова, какие были в первом свитке, который сожег Иоаким, царь Иудейский;
\vs Jer 36:29 а царю Иудейскому Иоакиму скажи: так говорит Господь: ты сожег свиток сей, сказав: <<зачем ты написал в нем: непременно придет царь Вавилонский и разорит землю сию, и истребит на ней людей и скот?>>
\vs Jer 36:30 за это, так говорит Господь об Иоакиме, царе Иудейском: не будет от него сидящего на престоле Давидовом, и труп его будет брошен на зной дневной и на холод ночной;
\vs Jer 36:31 и посещу его и племя его и слуг его за неправду их, и наведу на них и на жителей Иерусалима и на мужей Иуды все зло, которое Я изрек на них, а они не слушали.
\vs Jer 36:32 И взял Иеремия другой свиток и отдал его Варуху писцу, сыну Нирии, и он написал в нем из уст Иеремии все слова того свитка, который сожег Иоаким, царь Иудейский, на огне; и еще прибавлено к ним много подобных тем слов.
\vs Jer 37:1 Вместо Иехонии, сына Иоакима, царствовал Седекия, сын Иосии, которого Навуходоносор, царь Вавилонский, поставил царем в земле Иудейской.
\vs Jer 37:2 Ни он, ни слуги его, ни народ страны не слушали слов Господа, которые говорил Он чрез Иеремию пророка.
\vs Jer 37:3 Царь Седекия послал Иегухала, сына Селемии, и Софонию, сына Маасеи, священника, к Иеремии пророку сказать: помолись о нас Господу Богу нашему.
\vs Jer 37:4 Иеремия тогда еще свободно входил и выходил среди народа, потому что не заключили его в дом темничный.
\vs Jer 37:5 Между тем войско фараоново выступило из Египта, и Халдеи, осаждавшие Иерусалим, услышав весть о том, отступили от Иерусалима.
\vs Jer 37:6 И было слово Господне к Иеремии пророку:
\vs Jer 37:7 так говорит Господь, Бог Израилев: так скажите царю Иудейскому, пославшему вас ко Мне вопросить Меня: вот, войско фараоново, которое шло к вам на помощь, возвратится в землю свою, в Египет;
\vs Jer 37:8 а Халдеи снова придут и будут воевать против города сего, и возьмут его и сожгут его огнем.
\vs Jer 37:9 Так говорит Господь: не обманывайте себя, говоря: <<непременно отойдут от нас Халдеи>>, ибо они не отойдут;
\vs Jer 37:10 если бы вы даже разбили все войско Халдеев, воюющих против вас, и остались бы у них только раненые, то и те встали бы, каждый из палатки своей, и сожгли бы город сей огнем.
\rsbpar\vs Jer 37:11 В то время, как войско Халдейское отступило от Иерусалима, по причине войска фараонова,
\vs Jer 37:12 Иеремия пошел из Иерусалима, чтобы уйти в землю Вениаминову, скрываясь оттуда среди народа.
\vs Jer 37:13 Но когда он был в воротах Вениаминовых, бывший там начальник стражи, по имени Иреия, сын Селемии, сына Анании, задержал Иеремию пророка, сказав: ты хочешь перебежать к Халдеям?
\vs Jer 37:14 Иеремия сказал: это ложь; я не хочу перебежать к Халдеям. Но он не послушал его, и взял Иреия Иеремию и привел его к князьям.
\vs Jer 37:15 Князья озлобились на Иеремию и били его, и заключили его в темницу, в дом Ионафана писца, потому что сделали его темницею.
\vs Jer 37:16 Когда Иеремия вошел в темницу и подвал, и пробыл там Иеремия много дней,~---
\vs Jer 37:17 царь Седекия послал и взял его. И спрашивал его царь в доме своем тайно и сказал: нет ли слова от Господа? Иеремия сказал: есть; и сказал: ты будешь предан в руки царя Вавилонского.
\vs Jer 37:18 И сказал Иеремия царю Седекии: чем я согрешил перед тобою и перед слугами твоими, и перед народом сим, что вы посадили меня в темницу?
\vs Jer 37:19 и где ваши пророки, которые пророчествовали вам, говоря: <<царь Вавилонский не пойдет против вас и против земли сей>>?
\vs Jer 37:20 И ныне послушай, государь мой царь, да падет прошение мое пред лице твое; не возвращай меня в дом Ионафана писца, чтобы мне не умереть там.
\vs Jer 37:21 И дал повеление царь Седекия, чтобы заключили Иеремию во дворе стражи и давали ему по куску хлеба на день из улицы хлебопеков, доколе не истощился весь хлеб в городе; и так оставался Иеремия во дворе стражи.
\vs Jer 38:1 И услышали Сафатия, сын Матфана, и Годолия, сын Пасхора, и Юхал, сын Селемии, и Пасхор, сын Малхии, слова, которые Иеремия произнес ко всему народу, говоря:
\vs Jer 38:2 так говорит Господь: кто останется в этом городе, умрет от меча, голода и моровой язвы; а кто выйдет к Халдеям, будет жив, и душа его будет ему вместо добычи, и он останется жив.
\vs Jer 38:3 Так говорит Господь: непременно предан будет город сей в руки войска царя Вавилонского, и он возьмет его.
\vs Jer 38:4 Тогда князья сказали царю: да будет этот человек предан смерти, потому что он ослабляет руки воинов, которые остаются в этом городе, и руки всего народа, говоря к ним такие слова; ибо этот человек не благоденствия желает народу сему, а бедствия.
\vs Jer 38:5 И сказал царь Седекия: вот, он в ваших руках, потому что царь ничего не может делать вопреки вам.
\vs Jer 38:6 Тогда взяли Иеремию и бросили его в яму Малхии, сына царя, которая была во дворе стражи, и опустили Иеремию на веревках; в яме той не было воды, а только грязь, и погрузился Иеремия в грязь.
\vs Jer 38:7 И услышал Авдемелех Ефиоплянин, один из евнухов, находившихся в царском доме, что Иеремию посадили в яму; а царь сидел тогда у ворот Вениаминовых.
\vs Jer 38:8 И вышел Авдемелех из дома царского и сказал царю:
\vs Jer 38:9 государь мой царь! худо сделали эти люди, так поступив с Иеремиею пророком, которого бросили в яму; он умрет там от голода, потому что нет более хлеба в городе.
\vs Jer 38:10 Царь дал приказание Авдемелеху Ефиоплянину, сказав: возьми с собою отсюда тридцать человек и вытащи Иеремию пророка из ямы, доколе он не умер.
\vs Jer 38:11 Авдемелех взял людей с собою и вошел в дом царский под кладовую, и взял оттуда старых негодных тряпок и старых негодных лоскутьев и опустил их на веревках в яму к Иеремии.
\vs Jer 38:12 И сказал Авдемелех Ефиоплянин Иеремии: подложи эти старые брошенные тряпки и лоскутья под мышки рук твоих, под веревки. И сделал так Иеремия.
\vs Jer 38:13 И потащили Иеремию на веревках и вытащили его из ямы; и оставался Иеремия во дворе стражи.
\rsbpar\vs Jer 38:14 Тогда царь Седекия послал и призвал Иеремию пророка к себе, при третьем входе в дом Господень, и сказал царь Иеремии: я у тебя спрошу нечто; не скрой от меня ничего.
\vs Jer 38:15 И сказал Иеремия Седекии: если я открою тебе, не предашь ли ты меня смерти? и если дам тебе совет, ты не послушаешь меня.
\vs Jer 38:16 И клялся царь Седекия Иеремии тайно, говоря: жив Господь, Который сотворил нам душу сию, не предам тебя смерти и не отдам в руки этих людей, которые ищут души твоей.
\vs Jer 38:17 Тогда Иеремия сказал Седекии: так говорит Господь Бог Саваоф, Бог Израилев: если ты выйдешь к князьям царя Вавилонского, то жива будет душа твоя, и этот город не будет сожжен огнем, и ты будешь жив, и дом твой;
\vs Jer 38:18 а если не выйдешь к князьям царя Вавилонского, то этот город будет предан в руки Халдеев, и они сожгут его огнем, и ты не избежишь от рук их.
\vs Jer 38:19 И сказал царь Седекия Иеремии: я боюсь Иудеев, которые перешли к Халдеям, чтобы \bibemph{Халдеи} не предали меня в руки их, и чтобы те не надругались надо мною.
\vs Jer 38:20 И сказал Иеремия: не предадут; послушай гласа Господа в том, что я говорю тебе, и хорошо тебе будет, и жива будет душа твоя.
\vs Jer 38:21 А если ты не захочешь выйти, то вот слово, которое открыл мне Господь:
\vs Jer 38:22 вот, все жены, которые остались в доме царя Иудейского, отведены будут к князьям царя Вавилонского, и скажут они: <<тебя обольстили и превозмогли друзья твои; ноги твои погрузились в грязь, и они удалились от тебя>>.
\vs Jer 38:23 И всех жен твоих и детей твоих отведут к Халдеям, и ты не избежишь от рук их; но будешь взят рукою царя Вавилонского, и сделаешь то, что город сей будет сожжен огнем.
\vs Jer 38:24 И сказал Седекия Иеремии: никто не должен знать этих слов, и тогда ты не умрешь;
\vs Jer 38:25 и если услышат князья, что я разговаривал с тобою, и придут к тебе, и скажут тебе: <<скажи нам, что говорил ты царю, не скрой от нас, и мы не предадим тебя смерти,~--- и также что говорил тебе царь>>,
\vs Jer 38:26 то скажи им: <<я повергнул пред лице царя прошение мое, чтобы не возвращать меня в дом Ионафана, чтобы не умереть там>>.
\vs Jer 38:27 И пришли все князья к Иеремии и спрашивали его, и он сказал им согласно со всеми словами, какие царь велел \bibemph{сказать}, и они молча оставили его, потому что не узнали сказанного царю.
\vs Jer 38:28 И оставался Иеремия во дворе стражи до того дня, в который был взят Иерусалим. И Иерусалим был взят.
\vs Jer 39:1 В девятый год Седекии, царя Иудейского, в десятый месяц, пришел Навуходоносор, царь Вавилонский, со всем войском своим к Иерусалиму, и обложили его.
\vs Jer 39:2 А в одиннадцатый год Седекии, в четвертый месяц, в девятый день месяца город был взят.
\vs Jer 39:3 И вошли \bibemph{в него} все князья царя Вавилонского, и расположились в средних воротах, Нергал-Шарецер, Самгар-Нево, Сарсехим, начальник евнухов, Нергал-Шарецер, начальник магов, и все остальные князья царя Вавилонского.
\vs Jer 39:4 Когда Седекия, царь Иудейский, и все военные люди увидели их,~--- побежали, и ночью вышли из города через царский сад в ворота между двумя стенами и пошли по дороге равнины.
\vs Jer 39:5 Но войско Халдейское погналось за ними; и настигли Седекию на равнинах Иерихонских; и взяли его и отвели к Навуходоносору, царю Вавилонскому, в Ривлу, в землю Емаф, где он произнес суд над ним.
\vs Jer 39:6 И заколол царь Вавилонский сыновей Седекии в Ривле перед его глазами, и всех вельмож Иудейских заколол царь Вавилонский;
\vs Jer 39:7 а Седекии выколол глаза и заковал его в оковы, чтобы отвести его в Вавилон.
\vs Jer 39:8 Дом царя и домы народа сожгли Халдеи огнем, и стены Иерусалима разрушили.
\vs Jer 39:9 А остаток народа, остававшийся в городе, и перебежчиков, которые перешли к нему, и прочий оставшийся народ Навузардан, начальник телохранителей, переселил в Вавилон.
\vs Jer 39:10 Бедных же из народа, которые ничего не имели, Навузардан, начальник телохранителей, оставил в Иудейской земле и дал им тогда же виноградники и поля.
\vs Jer 39:11 А о Иеремии Навуходоносор, царь Вавилонский, дал такое повеление Навузардану, начальнику телохранителей:
\vs Jer 39:12 возьми его и имей его во внимании, и не делай ему ничего худого, но поступай с ним так, как он скажет тебе.
\vs Jer 39:13 И послал Навузардан, начальник телохранителей, и Навузазван, начальник евнухов, и Нергал-Шарецер, начальник магов, и все князья царя Вавилонского
\vs Jer 39:14 послали и взяли Иеремию со двора стражи, и поручили его Годолии, сыну Ахикама, сына Сафанова, отвести его домой. И он остался жить среди народа.
\rsbpar\vs Jer 39:15 К Иеремии, когда он еще содержался во дворе темничном, было слово Господне:
\vs Jer 39:16 иди, скажи Авдемелеху Ефиоплянину: так говорит Господь Саваоф, Бог Израилев: вот, Я исполню слова Мои о городе сем во зло, а не в добро ему, и они сбудутся в тот день перед глазами твоими;
\vs Jer 39:17 но тебя Я избавлю в тот день, говорит Господь, и не будешь предан в руки людей, которых ты боишься.
\vs Jer 39:18 Я избавлю тебя, и ты не падешь от меча, и душа твоя останется у тебя вместо добычи, потому что ты на Меня возложил упование, сказал Господь.
\vs Jer 40:1 Слово, которое было к Иеремии от Господа, после того как Навузардан, начальник телохранителей, отпустил его из Рамы, где он взял его скованного цепями среди прочих пленных Иерусалимлян и Иудеев, переселяемых в Вавилон.
\vs Jer 40:2 Начальник телохранителей взял Иеремию и сказал ему: Господь Бог твой изрек это бедствие на место сие,
\vs Jer 40:3 и навел его Господь и сделал то, что сказал; потому что вы согрешили пред Господом и не слушались гласа Его, за то и постигло вас это.
\vs Jer 40:4 Итак вот, я освобождаю тебя сегодня от цепей, которые на руках твоих: если тебе угодно идти со мною в Вавилон, иди, и я буду иметь попечение о тебе; а если не угодно тебе идти со мною в Вавилон, оставайся. Вот, вся земля перед тобою; куда тебе угодно, и куда нравится идти, туда и иди.
\vs Jer 40:5 Когда он еще не отошел, сказал \bibemph{Навузардан}: пойди к Годолии, сыну Ахикама, сына Сафанова, которого царь Вавилонский поставил начальником над городами Иудейскими, и оставайся с ним среди народа; или иди, куда нравится тебе идти. И дал ему начальник телохранителей продовольствие и подарок и отпустил его.
\vs Jer 40:6 И пришел Иеремия к Годолии, сыну Ахикама, в Массифу, и жил с ним среди народа, остававшегося в стране.
\vs Jer 40:7 Когда все военачальники, бывшие в поле, они и люди их, услышали, что царь Вавилонский поставил Годолию, сына Ахикама, начальником над страною и поручил ему мужчин и женщин, и детей, и тех из бедных страны, которые не были переселены в Вавилон;
\vs Jer 40:8 тогда пришли к Годолии в Массифу и Исмаил, сын Нафании, и Иоанан и Ионафан, сыновья Карея, и Сераия, сын Фанасмефа, и сыновья Офи из Нетофафы, и Иезония, сын Махафы, они и дружина их.
\vs Jer 40:9 Годолия, сын Ахикама, сына Сафанова, клялся им и людям их, говоря: не бойтесь служить Халдеям, оставайтесь на земле и служите царю Вавилонскому, и будет вам хорошо;
\vs Jer 40:10 а я останусь в Массифе, чтобы предстательствовать пред лицем Халдеев, которые будут приходить к нам; вы же собирайте вино и летние плоды, и масло и убирайте в сосуды ваши, и живите в городах ваших, которые заняли.
\vs Jer 40:11 Также все Иудеи, которые находились в земле Моавитской и между сыновьями Аммона и в Идумее, и во всех странах, услышали, что царь Вавилонский оставил часть Иудеев и поставил над ними Годолию, сына Ахикама, сына Сафана:
\vs Jer 40:12 и возвратились все сии Иудеи из всех мест, куда были изгнаны, и пришли в землю Иудейскую к Годолии в Массифу, и собрали вина и летних плодов очень много.
\vs Jer 40:13 Между тем Иоанан, сын Карея, и все военные начальники, бывшие в поле, пришли к Годолии в Массифу
\vs Jer 40:14 и сказали ему: знаешь ли ты, что Ваалис, царь сыновей Аммоновых, прислал Исмаила, сына Нафании, чтобы убить тебя? Но Годолия, сын Ахикама, не поверил им.
\vs Jer 40:15 Тогда Иоанан, сын Карея, сказал Годолии тайно в Массифе: позволь мне, я пойду и убью Исмаила, сына Нафании, и никто не узнает; зачем допускать, чтобы он убил тебя, и чтобы все Иудеи, собравшиеся к тебе, рассеялись, и чтобы погиб остаток Иуды?
\vs Jer 40:16 Но Годолия, сын Ахикама, сказал Иоанану, сыну Карея: не делай этого, ибо ты неправду говоришь об Исмаиле.
\vs Jer 41:1 И было в седьмой месяц, Исмаил, сын Нафании, сына Елисама из племени царского, и вельможи царя и десять человек с ним пришли к Годолии, сыну Ахикама, в Массифу, и там они ели вместе хлеб в Массифе.
\vs Jer 41:2 И встал Исмаил, сын Нафании, и десять человек, которые были с ним, и поразили Годолию, сына Ахикама, сына Сафанова, мечом и умертвили того, которого царь Вавилонский поставил начальником над страною.
\vs Jer 41:3 Также убил Исмаил и всех Иудеев, которые были с ним, с Годолиею, в Массифе, и находившихся там Халдеев, людей военных.
\vs Jer 41:4 На другой день по убиении Годолии, когда никто не знал об этом,
\vs Jer 41:5 пришли из Сихема, Силома и Самарии восемьдесят человек с обритыми бородами и в разодранных одеждах, и изранив себя, с дарами и ливаном в руках для принесения их в дом Господень.
\vs Jer 41:6 Исмаил, сын Нафании, вышел из Массифы навстречу им, идя и плача, и, встретившись с ними, сказал им: идите к Годолии, сыну Ахикама.
\vs Jer 41:7 И как только они вошли в средину города, Исмаил, сын Нафании, убил их и \bibemph{бросил} в ров, он и бывшие с ним люди.
\vs Jer 41:8 Но нашлись между ними десять человек, которые сказали Исмаилу: не умерщвляй нас, ибо у нас есть в поле скрытые кладовые с пшеницею и ячменем, и маслом и медом. И он удержался и не умертвил их с другими братьями их.
\vs Jer 41:9 Ров же, куда бросил Исмаил все трупы людей, которых он убил из-за Годолии, был тот самый, который сделал царь Аса, боясь Ваасы, царя Израильского; его наполнил Исмаил, сын Нафании, убитыми.
\vs Jer 41:10 И захватил Исмаил весь остаток народа, бывшего в Массифе, дочерей царя и весь остававшийся в Массифе народ, который Навузардан, начальник телохранителей, поручил Годолии, сыну Ахикама, и захватил их Исмаил, сын Нафании, и отправился к сыновьям Аммоновым.
\vs Jer 41:11 Но когда Иоанан, сын Карея, и все бывшие с ним военные начальники услышали о всех злодеяниях, какие совершил Исмаил, сын Нафании,
\vs Jer 41:12 взяли всех людей и пошли сразиться с Исмаилом, сыном Нафании, и настигли его у больших вод, в Гаваоне.
\vs Jer 41:13 И когда весь народ, бывший у Исмаила, увидел Иоанана, сына Карея, и всех бывших с ним военных начальников, обрадовался;
\vs Jer 41:14 и отворотился весь народ, который Исмаил увел в плен из Массифы, и обратился и пошел к Иоанану, сыну Карея;
\vs Jer 41:15 а Исмаил, сын Нафании, убежал от Иоанана с восемью человеками и ушел к сыновьям Аммоновым.
\vs Jer 41:16 Тогда Иоанан, сын Карея, и все бывшие с ним военные начальники взяли из Массифы весь оставшийся народ, который он освободил от Исмаила, сына Нафании, после того как тот убил Годолию, сына Ахикама, мужчин, военных людей, и жен, и детей, и евнухов, которых он вывел из Гаваона;
\vs Jer 41:17 и пошли, и остановились в селении Химам, близ Вифлеема, чтобы уйти в Египет
\vs Jer 41:18 от Халдеев, ибо они боялись их, потому что Исмаил, сын Нафании, убил Годолию, сына Ахикама, которого царь Вавилонский поставил начальником над страною.
\vs Jer 42:1 И приступили все военные начальники, и Иоанан, сын Карея, и Иезания, сын Гошаии, и весь народ от малого до большого,
\vs Jer 42:2 и сказали Иеремии пророку: да падет пред лице твое прошение наше, помолись о нас Господу Богу твоему обо всем этом остатке, ибо из многого осталось нас мало, как глаза твои видят нас,
\vs Jer 42:3 чтобы Господь, Бог твой, указал нам путь, по которому нам идти, и то, что нам делать.
\vs Jer 42:4 И сказал им Иеремия пророк: слышу, помолюсь Господу Богу вашему по словам вашим, и все, что ответит вам Господь, объявлю вам, не скрою от вас ни слова.
\vs Jer 42:5 Они сказали Иеремии: Господь да будет между нами свидетелем верным и истинным в том, что мы точно выполним все то, с чем пришлет тебя к нам Господь Бог твой:
\vs Jer 42:6 хорошо ли, худо ли то будет, но гласа Господа Бога нашего, к Которому посылаем тебя, послушаемся, чтобы нам было хорошо, когда будем послушны гласу Господа Бога нашего.
\rsbpar\vs Jer 42:7 По прошествии десяти дней было слово Господне к Иеремии.
\vs Jer 42:8 Он позвал к себе Иоанана, сына Карея, и всех бывших с ним военных начальников и весь народ, от малого и до большого,
\vs Jer 42:9 и сказал им: так говорит Господь, Бог Израилев, к Которому вы посылали меня, чтобы повергнуть пред Ним моление ваше:
\vs Jer 42:10 если останетесь на земле сей, то Я устрою вас и не разорю, насажду вас и не искореню, ибо Я сожалею о том бедствии, какое сделал вам.
\vs Jer 42:11 Не бойтесь царя Вавилонского, которого вы боитесь; не бойтесь его, говорит Господь, ибо Я с вами, чтобы спасать вас и избавлять вас от руки его.
\vs Jer 42:12 И явлю к вам милость, и он умилостивится к вам и возвратит вас в землю вашу.
\vs Jer 42:13 Если же вы скажете: <<не хотим жить в этой земле>>, и не послушаетесь гласа Господа Бога вашего, говоря:
\vs Jer 42:14 <<нет, мы пойдем в землю Египетскую, где войны не увидим и трубного голоса не услышим, и голодать не будем, и там будем жить>>;
\vs Jer 42:15 то выслушайте ныне слово Господне, вы, остаток Иуды: так говорит Господь Саваоф, Бог Израилев: если вы решительно обратите лица ваши, чтобы идти в Египет, и пойдете, чтобы жить там,
\vs Jer 42:16 то меч, которого вы боитесь, настигнет вас там, в земле Египетской, и голод, которого вы страшитесь, будет всегда следовать за вами там, в Египте, и там умрете.
\vs Jer 42:17 И все, которые обратят лице свое, чтобы идти в Египет и там жить, умрут от меча, голода и моровой язвы, и ни один из них не останется и не избежит того бедствия, которое Я наведу на них.
\vs Jer 42:18 Ибо так говорит Господь Саваоф, Бог Израилев: как излился гнев Мой и ярость Моя на жителей Иерусалима, так изольется ярость Моя на вас, когда войдете в Египет, и вы будете проклятием и ужасом, и поруганием и поношением, и не увидите более места сего.
\vs Jer 42:19 К вам, остаток Иуды, изрек Господь: <<не ходите в Египет>>; твердо знайте, что я ныне предостерегал вас,
\vs Jer 42:20 ибо вы погрешили против себя самих: вы послали меня к Господу Богу нашему сказав: <<помолись о нас Господу Богу нашему и все, что скажет Господь Бог наш, объяви нам, и мы сделаем>>.
\vs Jer 42:21 Я объявил вам ныне; но вы не послушали гласа Господа Бога нашего и всего того, с чем Он послал меня к вам.
\vs Jer 42:22 Итак знайте, что вы умрете от меча, голода и моровой язвы в том месте, куда хотите идти, чтобы жить там.
\vs Jer 43:1 Когда Иеремия передал всему народу все слова Господа Бога их, все те слова, с которыми Господь, Бог их, послал его к ним,
\vs Jer 43:2 тогда сказал Азария, сын Осаии, и Иоанан, сын Карея, и все дерзкие люди сказали Иеремии: неправду ты говоришь, не посылал тебя Господь Бог наш сказать: <<не ходите в Египет, чтобы жить там>>;
\vs Jer 43:3 а Варух, сын Нирии, возбуждает тебя против нас, чтобы предать нас в руки Халдеев, чтобы они умертвили нас или отвели нас пленными в Вавилон.
\vs Jer 43:4 И не послушал Иоанан, сын Карея, и все военные начальники и весь народ гласа Господа, чтобы остаться в земле Иудейской.
\vs Jer 43:5 И взял Иоанан, сын Карея, и все военные начальники весь остаток Иудеев, которые возвратились из всех народов, куда они были изгнаны, чтобы жить в земле Иудейской,
\vs Jer 43:6 мужей и жен, и детей, и дочерей царя, и всех тех, которых Навузардан, начальник телохранителей, оставил с Годолиею, сыном Ахикама, сына Сафанова, и Иеремию пророка, и Варуха, сына Нирии;
\vs Jer 43:7 и пошли в землю Египетскую, ибо не послушали гласа Господня, и дошли до Тафниса.
\rsbpar\vs Jer 43:8 И было слово Господне к Иеремии в Тафнисе:
\vs Jer 43:9 возьми в руки свои большие камни и скрой их в смятой глине при входе в дом фараона в Тафнисе, пред глазами Иудеев,
\vs Jer 43:10 и скажи им: так говорит Господь Саваоф, Бог Израилев: вот, Я пошлю и возьму Навуходоносора, царя Вавилонского, раба Моего, и поставлю престол его на этих камнях, скрытых Мною, и раскинет он над ним великолепный шатер свой
\vs Jer 43:11 и придет, и поразит землю Египетскую: кто \bibemph{обречен} на смерть, тот \bibemph{предан будет} смерти; и кто в плен, \bibemph{пойдет} в плен; и кто под меч, под меч.
\vs Jer 43:12 И зажгу огонь в капищах богов Египтян; и он сожжет оные, а их пленит, и оденется в землю Египетскую, как пастух надевает на себя одежду свою, и выйдет оттуда спокойно,
\vs Jer 43:13 и сокрушит статуи в Бефсамисе, что в земле Египетской, и капища богов Египетских сожжет огнем.
\vs Jer 44:1 Слово, которое было к Иеремии о всех Иудеях, живущих в земле Египетской, поселившихся в Магдоле и Тафнисе, и в Нофе, и в земле Пафрос:
\vs Jer 44:2 так говорит Господь Саваоф, Бог Израилев: вы видели все бедствие, какое Я навел на Иерусалим и на все города Иудейские; вот, они теперь пусты, и никто не живет в них,
\vs Jer 44:3 за нечестие их, которое они делали, прогневляя Меня, ходя кадить и служить иным богам, которых не знали ни они, ни вы, ни отцы ваши.
\vs Jer 44:4 Я посылал к вам всех рабов Моих, пророков, посылал с раннего утра, чтобы сказать: <<не делайте этого мерзкого дела, которое Я ненавижу>>.
\vs Jer 44:5 Но они не слушали и не приклонили уха своего, чтобы обратиться от своего нечестия, не кадить иным богам.
\vs Jer 44:6 И излилась ярость Моя и гнев Мой и разгорелась в городах Иудеи и на улицах Иерусалима; и они сделались развалинами и пустынею, как видите ныне.
\vs Jer 44:7 И ныне так говорит Господь Бог Саваоф, Бог Израилев: зачем вы делаете это великое зло душам вашим, истребляя у себя мужей и жен, взрослых детей и младенцев из среды Иудеи, чтобы не оставить у себя остатка,
\vs Jer 44:8 прогневляя Меня изделием рук своих, каждением иным богам в земле Египетской, куда вы пришли жить, чтобы погубить себя и сделаться проклятием и поношением у всех народов земли?
\vs Jer 44:9 Разве вы забыли нечестие отцов ваших и нечестие царей Иудейских, ваше собственное нечестие и нечестие жен ваших, какое они делали в земле Иудейской и на улицах Иерусалима?
\vs Jer 44:10 Не смирились они и до сего дня, и не боятся и не поступают по закону Моему и по уставам Моим, которые Я дал вам и отцам вашим.
\vs Jer 44:11 Посему так говорит Господь Саваоф, Бог Израилев: вот, Я обращу против вас лице Мое на погибель и на истребление всей Иудеи
\vs Jer 44:12 и возьму оставшихся Иудеев, которые обратили лице свое, чтобы идти в землю Египетскую и жить там, и все они будут истреблены, падут в земле Египетской; мечом и голодом будут истреблены; от малого и до большого умрут от меча и голода, и будут проклятием и ужасом, поруганием и поношением.
\vs Jer 44:13 Посещу живущих в земле Египетской, как Я посетил Иерусалим, мечом, голодом и моровою язвою,
\vs Jer 44:14 и никто не избежит и не уцелеет из остатка Иудеев, пришедших в землю Египетскую, чтобы пожить там и потом возвратиться в землю Иудейскую, куда они всею душею желают возвратиться, чтобы жить там; никто не возвратится, кроме тех, которые убегут оттуда.
\rsbpar\vs Jer 44:15 И отвечали Иеремии все мужья, знавшие, что жены их кадят иным богам, и все жены, стоявшие \bibemph{там} в большом множестве, и весь народ, живший в земле Египетской, в Пафросе, и сказали:
\vs Jer 44:16 сл\acc{о}ва, которое ты говорил нам именем Господа, мы не слушаем от тебя;
\vs Jer 44:17 но непременно будем делать все то, что вышло из уст наших, чтобы кадить богине неба и возливать ей возлияния, как мы делали, мы и отцы наши, цари наши и князья наши, в городах Иудеи и на улицах Иерусалима, потому что тогда мы были сыты и счастливы и беды не видели.
\vs Jer 44:18 А с того времени, как перестали мы кадить богине неба и возливать ей возлияния, терпим во всем недостаток и гибнем от меча и голода.
\vs Jer 44:19 И когда мы кадили богине неба и возливали ей возлияния, то разве без ведома мужей наших делали мы ей пирожки с изображением ее и возливали ей возлияния?
\vs Jer 44:20 Тогда сказал Иеремия всему народу, мужьям и женам, и всему народу, который так отвечал ему:
\vs Jer 44:21 не это ли каждение, которое совершали вы в городах Иудейских и на улицах Иерусалима, вы и отцы ваши, цари ваши и князья ваши, и народ страны, воспомянул Господь? И не оно ли взошло Ему на сердце?
\vs Jer 44:22 Господь не мог более терпеть злых дел ваших и мерзостей, какие вы делали; поэтому и сделалась земля ваша пустынею и ужасом, и проклятием, без жителей, как видите ныне.
\vs Jer 44:23 Так как вы, совершая то курение, грешили пред Господом и не слушали гласа Господа, и не поступали по закону Его и по установлениям Его, и по повелениям Его, то и постигло вас это бедствие, как видите ныне.
\vs Jer 44:24 И сказал Иеремия всему народу и всем женам: слушайте слово Господне, все Иудеи, которые в земле Египетской:
\vs Jer 44:25 так говорит Господь Саваоф, Бог Израилев: вы и жены ваши, что устами своими говорили, то и руками своими делали; вы говорите: <<станем выполнять обеты наши, какие мы обещали, чтобы кадить богине неба и возливать ей возлияние>>,~--- твердо держитесь обетов ваших и в точности исполняйте обеты ваши.
\vs Jer 44:26 За то выслушайте слово Господне, все Иудеи, живущие в земле Египетской: вот, Я поклялся великим именем Моим, говорит Господь, что не будет уже на всей земле Египетской произносимо имя Мое устами какого-либо Иудея, говорящего: <<жив Господь Бог!>>
\vs Jer 44:27 Вот, Я буду наблюдать над вами к погибели, а не к добру; и все Иудеи, которые в земле Египетской, будут погибать от меча и голода, доколе совсем не истребятся.
\vs Jer 44:28 Только малое число избежавших от меча возвратится из земли Египетской в землю Иудейскую, и узнают все оставшиеся Иудеи, которые пришли в землю Египетскую, чтобы пожить там, чье слово сбудется: Мое или их.
\vs Jer 44:29 И вот вам знамение, говорит Господь, что Я посещу вас на сем месте, чтобы вы знали, что сбудутся слова Мои о вас на погибель вам.
\vs Jer 44:30 Так говорит Господь: вот, Я отдам фараона Вафрия, царя Египетского, в руки врагов его и в руки ищущих души его, как отдал Седекию, царя Иудейского, в руки Навуходоносора, царя Вавилонского, врага его и искавшего души его.
\vs Jer 45:1 Слово, которое пророк Иеремия сказал Варуху, сыну Нирии, когда он написал слова сии из уст Иеремии в книгу, в четвертый год Иоакима, сына Иосии, царя Иудейского:
\vs Jer 45:2 так говорит Господь, Бог Израилев, к тебе, Варух:
\vs Jer 45:3 ты говоришь: <<горе мне! ибо Господь приложил скорбь к болезни моей; я изнемог от вздохов моих, и не нахожу покоя>>.
\vs Jer 45:4 Так скажи ему: так говорит Господь: вот, что Я построил, разрушу, и что насадил, искореню,~--- всю эту землю.
\vs Jer 45:5 А ты просишь себе великого: не проси; ибо вот, Я наведу бедствие на всякую плоть, говорит Господь, а тебе вместо добычи оставлю душу твою во всех местах, куда ни пойдешь.
\vs Jer 46:1 Слово Господне, которое было к Иеремии пророку о народах \bibemph{языческих}:
\vs Jer 46:2 о Египте, о войске фараона Нехао, царя Египетского, которое было при реке Евфрате в Кархемисе, и которое поразил Навуходоносор, царь Вавилонский, в четвертый год Иоакима, сына Иосии, царя Иудейского.
\vs Jer 46:3 Готовьте щиты и копья, и вступайте в сражение:
\vs Jer 46:4 седлайте коней и садитесь, всадники, и становитесь в шлемах; точите копья, облекайтесь в брони.
\vs Jer 46:5 Почему же, вижу Я, они оробели и обратились назад? и сильные их поражены, и бегут не оглядываясь; отвсюду ужас, говорит Господь.
\vs Jer 46:6 Не убежит быстроногий, и не спасется сильный; на севере, у реки Евфрата, они споткнутся и падут.
\vs Jer 46:7 Кто это поднимается, как река, и, как потоки, волнуются воды его?
\vs Jer 46:8 Египет поднимается, как река, и, как потоки, взволновались воды его, и говорит: <<поднимусь и покрою землю, погублю город и жителей его>>.
\vs Jer 46:9 Садитесь на коней, и мчитесь, колесницы, и выступайте, сильные Ефиопляне и Ливияне, вооруженные щитом, и Лидяне, держащие луки и натягивающие их;
\vs Jer 46:10 ибо день сей у Господа Бога Саваофа есть день отмщения, чтобы отмстить врагам Его; и меч будет пожирать, и насытится и упьется кровью их; ибо это Господу Богу Саваофу будет жертвоприношение в земле северной, при реке Евфрате.
\vs Jer 46:11 Пойди в Галаад и возьми бальзама, дева, дочь Египта; напрасно ты будешь умножать врачевства, нет для тебя исцеления.
\vs Jer 46:12 Услышали народы о посрамлении твоем, и вопль твой наполнил землю; ибо сильный столкнулся с сильным, и оба вместе пали.
\rsbpar\vs Jer 46:13 Слово, которое сказал Господь пророку Иеремии о нашествии Навуходоносора, царя Вавилонского, чтобы поразить землю Египетскую:
\vs Jer 46:14 возвестите в Египте и дайте знать в Магдоле, и дайте знать в Нофе и Тафнисе; скажите: <<становись и готовься, ибо меч пожирает окрестности твои>>.
\vs Jer 46:15 Отчего сильный твой опрокинут?~--- Не устоял, потому что Господь погнал его.
\vs Jer 46:16 Он умножил падающих, даже падали один на другого и говорили: <<вставай и возвратимся к народу нашему в родную нашу землю от губительного меча>>.
\vs Jer 46:17 А там кричат: <<фараон, царь Египта, смутился; он пропустил условленное время>>.
\vs Jer 46:18 Живу Я, говорит Царь, Которого имя Господь Саваоф: как Фавор среди гор и как Кармил при море, \bibemph{так верно} придет он.
\vs Jer 46:19 Готовь себе нужное для переселения, дочь~--- жительница Египта, ибо Ноф будет опустошен, разорен, останется без жителя.
\vs Jer 46:20 Египет~--- прекрасная телица; но погибель от севера идет, идет.
\vs Jer 46:21 И наемники его среди него, как откормленные тельцы,~--- и сами обратились назад, побежали все, не устояли, потому что пришел на них день погибели их, время посещения их.
\vs Jer 46:22 Голос его несется, как змеиный; они идут с войском, придут на него с топорами, как дровосеки;
\vs Jer 46:23 вырубят лес его, говорит Господь, ибо они несметны; их более, нежели саранчи, и нет числа им.
\vs Jer 46:24 Посрамлена дочь Египта, предана в руки народа северного.
\vs Jer 46:25 Господь Саваоф, Бог Израилев, говорит: вот, Я посещу Аммона, который в Но, и фараона и Египет, и богов его и царей его, фараона и надеющихся на него;
\vs Jer 46:26 и предам их в руки ищущих души их и в руки Навуходоносора, царя Вавилонского, и в руки рабов его; но после того будет он населен, как в прежние дни, говорит Господь.
\vs Jer 46:27 Ты же не бойся, раб мой Иаков, и не страшись, Израиль: ибо вот, Я спасу тебя из далекой страны и семя твое из земли плена их; и возвратится Иаков, и будет жить спокойно и мирно, и никто не будет устрашать его.
\vs Jer 46:28 Не бойся, раб Мой Иаков, говорит Господь: ибо Я с тобою; Я истреблю все народы, к которым Я изгнал тебя, а тебя не истреблю, а только накажу тебя в мере; ненаказанным же не оставлю тебя.
\vs Jer 47:1 Слово Господа, которое было к пророку Иеремии о Филистимлянах, прежде нежели фараон поразил Газу.
\vs Jer 47:2 Так говорит Господь: вот, поднимаются воды с севера и сделаются наводняющим потоком, и потопят землю и все, что наполняет ее, город и живущих в нем; тогда возопиют люди, и зарыдают все обитатели страны.
\vs Jer 47:3 От шумного топота копыт сильных коней его, от стука колесниц его, от звука колес его, отцы не оглянутся на детей своих, потому что руки у них опустятся
\vs Jer 47:4 от того дня, который придет истребить всех Филистимлян, отнять у Тира и Сидона всех остальных помощников, ибо Господь разорит Филистимлян, остаток острова Кафтора.
\vs Jer 47:5 Оплешивела Газа, гибнет Аскалон, остаток долины их.
\vs Jer 47:6 Доколе будешь посекать, о, меч Господень! доколе ты не успокоишься? возвратись в ножны твои, перестань и успокойся.
\vs Jer 47:7 Но как тебе успокоиться, когда Господь дал повеление против Аскалона и против берега морского? туда Он направил его.
\vs Jer 48:1 О Моаве так говорит Господь Саваоф, Бог Израилев: горе Нев\acc{о}! он опустошен; Кариафаим посрамлен и взят; Мизгав посрамлен и сокрушен.
\vs Jer 48:2 Нет более славы Моава; в Есевоне замышляют против него зло: <<пойдем, истребим его из числа народов>>. И ты, Мадмена, погибнешь; меч следует за тобою.
\vs Jer 48:3 Слышен вопль от Оронаима, опустошение и разрушение великое.
\vs Jer 48:4 Сокрушен Моав; вопль подняли дети его.
\vs Jer 48:5 На восхождении в Лухит плач за плачем поднимается; и на спуске с Оронаима неприятель слышит вопль о разорении.
\vs Jer 48:6 Бегите, спасайте жизнь свою, и будьте подобны обнаженному дереву в пустыне.
\vs Jer 48:7 Так как ты надеялся на дела твои и на сокровища твои, то и ты будешь взят, и Хамос пойдет в плен вместе со своими священниками и своими князьями.
\vs Jer 48:8 И придет опустошитель на всякий город, и город не уцелеет; и погибнет долина, и опустеет равнина, как сказал Господь.
\vs Jer 48:9 Дайте крылья Моаву, чтобы он мог улететь; города его будут пустынею, потому что некому будет жить в них.
\vs Jer 48:10 Проклят, кто дело Господне делает небрежно, и проклят, кто удерживает меч Его от крови!
\vs Jer 48:11 Моав от юности своей был в покое, сидел на дрожжах своих и не был переливаем из сосуда с сосуд, и в плен не ходил; оттого оставался в нем вкус его, и запах его не изменялся.
\vs Jer 48:12 Посему вот, приходят дни, говорит Господь, когда Я пришлю к нему переливателей, которые перельют его и опорожнят сосуды его, и разобьют кувшины его.
\vs Jer 48:13 И постыжен будет Моав ради Хамоса, как дом Израилев постыжен был ради Вефиля, надежды своей.
\vs Jer 48:14 Как вы говорите: <<мы люди храбрые и крепкие для войны>>?
\vs Jer 48:15 Опустошен Моав, и города его горят, и отборные юноши его пошли на заклание, говорит Царь,~--- Господь Саваоф имя Его.
\vs Jer 48:16 Близка погибель Моава, и сильно спешит бедствие его.
\vs Jer 48:17 Пожалейте о нем все соседи его и все, знающие имя его, скажите: <<как сокрушен жезл силы, посох славы!>>
\vs Jer 48:18 Сойди с высоты величия и сиди в жажде, дочь~--- обитательница Дивона, ибо опустошитель Моава придет к тебе и разорит укрепления твои.
\vs Jer 48:19 Стань у дороги и смотри, обитательница Ароера, спрашивай бегущего и спасающегося: <<что сделалось?>>
\vs Jer 48:20 Посрамлен Моав, ибо сокрушен; рыдайте и вопите, объявите в Арноне, что опустошен Моав.
\vs Jer 48:21 И суд пришел на равнины, на Халон и на Иаацу, и на Мофаф,
\vs Jer 48:22 и на Дивон и на Нев\acc{о}, и на Бет-Дивлафаим,
\vs Jer 48:23 и на Кариафаим и на Бет-Гамул, и на Бет-Маон,
\vs Jer 48:24 и на Кериоф и на Восор, и на все города земли Моавитской, дальние и ближние.
\vs Jer 48:25 Отсечен рог Моава, и мышца его сокрушена, говорит Господь.
\vs Jer 48:26 Напойте его пьяным, ибо он вознесся против Господа; и пусть Моав валяется в блевотине своей, и сам будет посмеянием.
\vs Jer 48:27 Не был ли в посмеянии у тебя Израиль? разве он между ворами был пойман, что ты, бывало, лишь только заговоришь о нем, качаешь головою?
\vs Jer 48:28 Оставьте города и живите на скалах, жители Моава, и будьте как голуби, которые делают гнезда во входе в пещеру.
\vs Jer 48:29 Слыхали мы о гордости Моава, гордости чрезмерной, о его высокомерии и его надменности, и кичливости его и превозношении сердца его.
\vs Jer 48:30 Знаю Я дерзость его, говорит Господь, но это ненадежно; пустые слова его: не так сделают.
\vs Jer 48:31 Поэтому буду рыдать о Моаве и вопить о всем Моаве; будут воздыхать о мужах Кирхареса.
\vs Jer 48:32 Буду плакать о тебе, виноградник Севамский, плачем Иазера; отрасли твои простирались за море, достигали до озера Иазера; опустошитель напал на летние плоды твои и на зрелый виноград.
\vs Jer 48:33 Радость и веселье отнято от Кармила и от земли Моава. Я положу конец вину в точилах; не будут более топтать в них с песнями; крик брани будет, а не крик радости.
\vs Jer 48:34 От вопля Есевона до Елеалы и до Иаацы они поднимут голос свой от Сигора до Оронаима, до третьей Эглы, ибо и воды Нимрима иссякнут.
\vs Jer 48:35 Истреблю у Моава, говорит Господь, приносящих жертвы на высотах и кадящих богам его.
\vs Jer 48:36 Оттого сердце мое стонет о Моаве, как свирель; о жителях Кирхареса стонет сердце мое, как свирель, ибо богатства, ими приобретенные, погибли:
\vs Jer 48:37 у каждого голова гола и у каждого борода умалена; у всех на руках царапины и на чреслах вретище.
\vs Jer 48:38 На всех кровлях Моава и на улицах его общий плач, ибо Я сокрушил Моава, как непотребный сосуд, говорит Господь.
\vs Jer 48:39 <<Как сокрушен он!>> будут говорить рыдая; <<как Моав покрылся стыдом, обратив тыл!>>. И будет Моав посмеянием и ужасом для всех окружающих его,
\vs Jer 48:40 ибо так говорит Господь: вот, как орел, налетит он и распрострет крылья свои над Моавом.
\vs Jer 48:41 Города будут взяты, и крепости завоеваны, и сердце храбрых Моавитян будет в тот день, как сердце женщины, мучимой родами.
\vs Jer 48:42 И истреблен будет Моав из числа народов, потому что он восстал против Господа.
\vs Jer 48:43 Ужас и яма и петля~--- для тебя, житель Моава, сказал Господь.
\vs Jer 48:44 Кто убежит от ужаса, упадет в яму; а кто выйдет из ямы, попадет в петлю, ибо Я наведу на него, на Моава, годину посещения их, говорит Господь.
\vs Jer 48:45 Под тенью Есевона остановились бегущие, обессилев; но огонь вышел из Есевона и пламя из среды Сигона, и пожрет бок Моава и темя сыновей мятежных.
\vs Jer 48:46 Горе тебе, Моав! погиб народ Хамоса, ибо сыновья твои взяты в плен, и дочери твои~--- в пленение.
\vs Jer 48:47 Но в последние дни возвращу плен Моава, говорит Господь. Доселе суд на Моава.
\vs Jer 49:1 О сыновьях Аммоновых так говорит Господь: разве нет сыновей у Израиля? разве нет у него наследника? Почему же Малхом завладел Гадом, и народ его живет в городах его?
\vs Jer 49:2 Посему вот, наступают дни, говорит Господь, когда в Равве сыновей Аммоновых слышен будет крик брани, и сделается она грудою развалин, и города ее будут сожжены огнем, и овладеет Израиль теми, которые владели им, говорит Господь.
\vs Jer 49:3 Рыдай, Есевон, ибо опустошен Гай; кричите, дочери Раввы, опояшьтесь вретищем, плачьте и скитайтесь по огородам, ибо Малхом пойдет в плен вместе со священниками и князьями своими.
\vs Jer 49:4 Что хвалишься долинами? Потечет долина твоя кровью, вероломная дочь, надеющаяся на сокровища свои, \bibemph{говорящая}: <<кто придет ко мне?>>
\vs Jer 49:5 Вот, Я наведу на тебя ужас со всех окрестностей твоих, говорит Господь Бог Саваоф; разбежитесь, кто куда, и никто не соберет разбежавшихся.
\vs Jer 49:6 Но после того Я возвращу плен сыновей Аммоновых, говорит Господь.
\rsbpar\vs Jer 49:7 О Едоме так говорит Господь Саваоф: разве нет более мудрости в Фемане? \bibemph{разве} не стало совета у разумных? разве оскудела мудрость их?
\vs Jer 49:8 Бегите, обратив тыл, скрывайтесь в пещерах, жители Дедана, ибо погибель Исава Я наведу на него,~--- время посещения Моего.
\vs Jer 49:9 Если бы обиратели винограда пришли к тебе, то верно оставили бы несколько недобранных ягод. И если бы воры \bibemph{пришли} ночью, то они похитили бы, сколько им нужно.
\vs Jer 49:10 А Я донага оберу Исава, открою потаенные места его, и скрыться он не может. Истреблено будет племя его, и братья его и соседи его; и не будет его.
\vs Jer 49:11 Оставь сирот твоих, Я поддержу жизнь их, и вдовы твои пусть надеются на Меня.
\vs Jer 49:12 Ибо так говорит Господь: вот и те, которым не суждено было пить чашу, непременно будут пить ее, и ты ли останешься ненаказанным? Нет, не останешься ненаказанным, но непременно будешь пить \bibemph{чашу}.
\vs Jer 49:13 Ибо Мною клянусь, говорит Господь, что ужасом, посмеянием, пустынею и проклятием будет Восор, и все города его сделаются вечными пустынями.
\vs Jer 49:14 Я слышал слух от Господа, и посол послан к народам сказать: соберитесь и идите против него, и поднимайтесь на войну.
\vs Jer 49:15 Ибо вот, Я сделаю тебя малым между народами, презренным между людьми.
\vs Jer 49:16 Грозное положение твое и надменность сердца твоего обольстили тебя, живущего в расселинах скал и занимающего вершины холмов. Но, хотя бы ты, как орел, высоко свил гнездо твое, и оттуда низрину тебя, говорит Господь.
\vs Jer 49:17 И будет Едом ужасом; всякий, проходящий мимо, изумится и посвищет, \bibemph{смотря} на все язвы его.
\vs Jer 49:18 Как ниспровергнуты Содом и Гоморра и соседние города их, говорит Господь, так \bibemph{и} там ни один человек не будет жить, и сын человеческий не остановится в нем.
\vs Jer 49:19 Вот, восходит он, как лев, от возвышения Иордана на укрепленные жилища; но Я заставлю их поспешно уйти из \bibemph{Идумеи}, и кто избран, того поставлю над нею. Ибо кто подобен Мне? и кто потребует ответа от Меня? и какой пастырь противостанет Мне?
\vs Jer 49:20 Итак выслушайте определение Господа, какое Он поставил об Едоме, и намерения Его, какие Он имеет о жителях Фемана: истинно, самые малые из стад повлекут их и опустошат жилища их.
\vs Jer 49:21 От шума падения их потрясется земля, и отголосок крика их слышен будет у Чермного моря.
\vs Jer 49:22 Вот, как орел поднимется он, и полетит, и распустит крылья свои над Восором; и сердце храбрых Идумеян будет в тот день, как сердце женщины в родах.
\rsbpar\vs Jer 49:23 О Дамаске.~--- Посрамлены Емаф и Арпад, ибо, услышав скорбную весть, они уныли; тревога на море, успокоиться не могут.
\vs Jer 49:24 Оробел Дамаск и обратился в бегство; страх овладел им; боль и муки схватили его, как женщину в родах.
\vs Jer 49:25 Как не уцелел город славы, город радости моей?
\vs Jer 49:26 Итак падут юноши его на улицах его, и все воины погибнут в тот день, говорит Господь Саваоф.
\vs Jer 49:27 И зажгу огонь в стенах Дамаска, и истребит чертоги Венадада.
\rsbpar\vs Jer 49:28 О Кидаре и о царствах Асорских, которые поразил Навуходоносор, царь Вавилонский, так говорит Господь: вставайте, выступайте против Кидара, и опустошайте сыновей востока!
\vs Jer 49:29 Шатры их и овец их возьмут себе, и покровы их и всю утварь их, и верблюдов их возьмут, и будут кричать им: <<ужас отовсюду!>>
\vs Jer 49:30 Бегите, уходите скорее, сокройтесь в пропасти, жители Асора, говорит Господь, ибо Навуходоносор, царь Вавилонский, сделал решение о вас и составил против вас замысел.
\vs Jer 49:31 Вставайте, выступайте против народа мирного, живущего беспечно, говорит Господь; ни дверей, ни запоров нет у него, живут поодиночке.
\vs Jer 49:32 Верблюды их \bibemph{отданы} будут в добычу, и множество стад их~--- на расхищение; и рассею их по всем ветрам, этих стригущих волосы на висках, и со всех сторон их наведу на них гибель, говорит Господь.
\vs Jer 49:33 И будет Асор жилищем шакалов, вечною пустынею; человек не будет жить там, и сын человеческий не будет останавливаться в нем.
\rsbpar\vs Jer 49:34 Слово Господа, которое было к Иеремии пророку против Елама, в начале царствования Седекии, царя Иудейского:
\vs Jer 49:35 так говорит Господь Саваоф: вот, Я сокрушу лук Елама, главную силу их.
\vs Jer 49:36 И наведу на Елам четыре ветра от четырех краев неба и развею их по всем этим ветрам, и не будет народа, к которому не пришли бы изгнанные Еламиты.
\vs Jer 49:37 И поражу Еламитян страхом пред врагами их и пред ищущими души их; и наведу на них бедствие, гнев Мой, говорит Господь, и пошлю вслед их меч, доколе не истреблю их.
\vs Jer 49:38 И поставлю престол Мой в Еламе, и истреблю там царя и князей, говорит Господь.
\vs Jer 49:39 Но в последние дни возвращу плен Елама, говорит Господь.
\vs Jer 50:1 Слово, которое изрек Господь о Вавилоне и о земле Халдеев чрез Иеремию пророка:
\vs Jer 50:2 возвестите и разгласите между народами, и поднимите знамя, объявите, не скрывайте, говорите: <<Вавилон взят, Вил посрамлен, Меродах сокрушен, истуканы его посрамлены, идолы его сокрушены>>.
\vs Jer 50:3 Ибо от севера поднялся против него народ, который сделает землю его пустынею, и никто не будет жить там, от человека до скота, все двинутся и уйдут.
\vs Jer 50:4 В те дни и в то время, говорит Господь, придут сыновья Израилевы, они и сыновья Иудины вместе, будут ходить и плакать, и взыщут Господа Бога своего.
\vs Jer 50:5 Будут спрашивать о пути к Сиону, и, обращая к нему лица, \bibemph{будут говорить}: <<идите и присоединитесь к Господу союзом вечным, который не забудется>>.
\vs Jer 50:6 Народ Мой был как погибшие овцы; пастыри их совратили их с пути, разогнали их по горам; скитались они с горы на холм, забыли ложе свое.
\vs Jer 50:7 Все, которые находили их, пожирали их, и притеснители их говорили: <<мы не виноваты, потому что они согрешили пред Господом, пред жилищем правды и пред Господом, надеждою отцов их>>.
\vs Jer 50:8 Бегите из среды Вавилона, и уходите из Халдейской земли, и будьте как козлы впереди стада овец.
\vs Jer 50:9 Ибо вот, Я подниму и приведу на Вавилон сборище великих народов от земли северной, и расположатся против него, и он будет взят; стрелы у них, как у искусного воина, не возвращаются даром.
\vs Jer 50:10 И Халдея сделается добычею их; и опустошители ее насытятся, говорит Господь.
\vs Jer 50:11 Ибо вы веселились, вы торжествовали, расхитители наследия Моего; прыгали от радости, как телица на траве, и ржали, как боевые кони.
\vs Jer 50:12 В большом стыде будет мать ваша, покраснеет родившая вас; вот будущность тех народов~--- пустыня, сухая земля и степь.
\vs Jer 50:13 От гнева Господа она сделается необитаемою, и вся она будет пуста; всякий проходящий чрез Вавилон изумится и посвищет, смотря на все язвы его.
\vs Jer 50:14 Выстройтесь в боевой порядок вокруг Вавилона; все, натягивающие лук, стреляйте в него, не жалейте стрел, ибо он согрешил против Господа.
\vs Jer 50:15 Поднимите крик против него со всех сторон; он подал руку свою; пали твердыни его, рушились стены его, ибо это~--- возмездие Господа; отмщайте ему; как он поступал, так и вы поступайте с ним.
\vs Jer 50:16 Истребите в Вавилоне \bibemph{и} сеющего и действующего серпом во время жатвы; от страха губительного меча пусть каждый возвратится к народу своему, и каждый пусть бежит в землю свою.
\vs Jer 50:17 Израиль~--- рассеянное стадо; львы разогнали \bibemph{его}; прежде объедал его царь Ассирийский, а сей последний, Навуходоносор, царь Вавилонский, и кости его сокрушил.
\vs Jer 50:18 Посему так говорит Господь Саваоф, Бог Израилев: вот, Я посещу царя Вавилонского и землю его, как посетил царя Ассирийского.
\vs Jer 50:19 И возвращу Израиля на пажить его, и будет он пастись на Кармиле и Васане, и душа его насытится на горе Ефремовой и в Галааде.
\vs Jer 50:20 В те дни и в то время, говорит Господь, будут искать неправды Израилевой, и не будет ее, и грехов Иуды, и не найдется их; ибо прощу тех, которых оставлю \bibemph{в живых}.
\vs Jer 50:21 Иди на нее, на землю возмутительную, и накажи жителей ее; опустошай и истребляй всё за ними, говорит Господь, и сделай всё, что Я повелел тебе.
\vs Jer 50:22 Шум брани на земле и великое разрушение!
\vs Jer 50:23 Как разбит и сокрушен молот всей земли! Как Вавилон сделался ужасом между народами!
\vs Jer 50:24 Я расставил сети для тебя, и ты пойман, Вавилон, не предвидя того; ты найден и схвачен, потому что восстал против Господа.
\vs Jer 50:25 Господь открыл хранилище Свое и взял \bibemph{из него} сосуды гнева Своего, потому что у Господа Бога Саваофа есть дело в земле Халдейской.
\vs Jer 50:26 Идите на нее со всех краев, растворяйте житницы ее, топчите ее как снопы, совсем истребите ее, чтобы ничего от нее не осталось.
\vs Jer 50:27 Убивайте всех волов ее, пусть идут на заклание; горе им! ибо пришел день их, время посещения их.
\vs Jer 50:28 \bibemph{Слышен} голос бегущих и спасающихся из земли Вавилонской, чтобы возвестить на Сионе о мщении Господа Бога нашего, о мщении за храм Его.
\vs Jer 50:29 Созовите против Вавилона стрельцов; все, напрягающие лук, расположитесь станом вокруг него, чтобы никто не спасся из него; воздайте ему по делам его; как он поступал, так поступите и с ним, ибо он вознесся против Господа, против Святаго Израилева.
\vs Jer 50:30 За то падут юноши его на улицах его, и все воины его истреблены будут в тот день, говорит Господь.
\vs Jer 50:31 Вот, Я~--- на тебя, гордыня, говорит Господь Бог Саваоф; ибо пришел день твой, время посещения твоего.
\vs Jer 50:32 И споткнется гордыня, и упадет, и никто не поднимет его; и зажгу огонь в городах его, и пожрет все вокруг него.
\rsbpar\vs Jer 50:33 Так говорит Господь Саваоф: угнетены сыновья Израиля, как и сыновья Иуды, и все, пленившие их, крепко держат их и не хотят отпустить их.
\vs Jer 50:34 Но Искупитель их силен, Господь Саваоф имя Его; Он разберет дело их, чтобы успокоить землю и привести в трепет жителей Вавилона.
\vs Jer 50:35 Меч на Халдеев, говорит Господь, и на жителей Вавилона, и на князей его, и на мудрых его;
\vs Jer 50:36 меч на обаятелей, и они обезумеют; меч на воинов его, и они оробеют;
\vs Jer 50:37 меч на коней его и на колесницы его и на все разноплеменные народы среди него, и они будут как женщины; меч на сокровища его, и они будут расхищены;
\vs Jer 50:38 засуха на воды его, и они иссякнут; ибо это земля истуканов, и они обезумеют от идольских страшилищ.
\vs Jer 50:39 И поселятся там степные звери с шакалами, и будут жить на ней страусы, и не будет обитаема во веки и населяема в роды родов.
\vs Jer 50:40 Как ниспровержены Богом Содом и Гоморра и соседние города их, говорит Господь, так \bibemph{и} тут ни один человек не будет жить, и сын человеческий не будет останавливаться.
\vs Jer 50:41 Вот, идет народ от севера, и народ великий, и многие цари поднимаются от краев земли;
\vs Jer 50:42 держат в руках лук и копье; они жестоки и немилосерды; голос их шумен, как море; несутся на конях, выстроились как один человек, чтобы сразиться с тобою, дочь Вавилона.
\vs Jer 50:43 Услышал царь Вавилонский весть о них, и руки у него опустились; скорбь объяла его, муки, как женщину в родах.
\vs Jer 50:44 Вот, восходит он, как лев, от возвышения Иордана на укрепленные жилища; но Я заставлю их поспешно уйти из него, и, кто избран, тому вверю его. Ибо кто подобен Мне? и кто потребует от Меня ответа? И какой пастырь противостанет Мне?
\vs Jer 50:45 Итак выслушайте определение Господа, какое Он постановил о Вавилоне, и намерения Его, какие Он имеет о земле Халдейской: истинно, самые малые из стад повлекут их; истинно, он опустошит жилища их с ними.
\vs Jer 50:46 От шума взятия Вавилона потрясется земля, и вопль будет слышен между народами.
\vs Jer 51:1 Так говорит Господь: вот, Я подниму на Вавилон и на живущих среди него противников Моих.
\vs Jer 51:2 И пошлю на Вавилон веятелей, и развеют его, и опустошат землю его; ибо в день бедствия нападут на него со всех сторон.
\vs Jer 51:3 Пусть стрелец напрягает лук против напрягающего \bibemph{лук} и на величающегося бронею своею; и не щадите юношей его, истребите все войско его.
\vs Jer 51:4 Пораженные пусть падут на земле Халдейской, и пронзенные~--- на дорогах ее.
\vs Jer 51:5 Ибо не овдовел Израиль и Иуда от Бога Своего, Господа Саваофа; хотя земля их полна грехами пред Святым Израилевым.
\vs Jer 51:6 Бегите из среды Вавилона и спасайте каждый душу свою, чтобы не погибнуть от беззакония его, ибо это время отмщения у Господа, Он воздает ему воздаяние.
\vs Jer 51:7 Вавилон был золотою чашею в руке Господа, опьянявшею всю землю; народы пили из нее вино и безумствовали.
\vs Jer 51:8 Внезапно пал Вавилон и разбился; рыдайте о нем, возьмите бальзама для раны его: может быть, он исцелеет.
\vs Jer 51:9 Врачевали мы Вавилон, но не исцелился; оставьте его, и пойдем каждый в свою землю, потому что приговор о нем достиг до небес и поднялся до облаков.
\vs Jer 51:10 Господь вывел на свет правду нашу; пойдем и возвестим на Сионе дело Господа Бога нашего.
\vs Jer 51:11 Острите стрелы, наполняйте колчаны; Господь возбудил дух царей Мидийских, потому что у Него есть намерение против Вавилона, чтобы истребить его, ибо это есть отмщение Господа, отмщение за храм Его.
\vs Jer 51:12 Против стен Вавилона поднимите знамя, усильте надзор, расставьте сторожей, приготовьте засады, ибо, как Господь помыслил, так и сделает, что изрек на жителей Вавилона.
\vs Jer 51:13 О, ты, живущий при водах великих, изобилующий сокровищами! пришел конец твой, мера жадности твоей.
\vs Jer 51:14 Господь Саваоф поклялся Самим Собою: истинно говорю, что наполню тебя людьми, как саранчою, и поднимут крик против тебя.
\vs Jer 51:15 Он сотворил землю силою Своею, утвердил вселенную мудростью Своею и разумом Своим распростер небеса.
\vs Jer 51:16 По гласу Его шумят воды на небесах, и Он возводит облака от краев земли, творит молнии среди дождя и изводит ветер из хранилищ Своих.
\vs Jer 51:17 Безумствует всякий человек в своем знании, срамит себя всякий плавильщик истуканом своим, ибо истукан его есть ложь, и нет в нем духа.
\vs Jer 51:18 Это совершенная пустота, дело заблуждения; во время посещения их они исчезнут.
\vs Jer 51:19 Не такова, как их, доля Иакова, ибо \bibemph{Бог его} есть Творец всего, и \bibemph{Израиль} есть жезл наследия Его, имя Его~--- Господь Саваоф.
\vs Jer 51:20 Ты у Меня~--- молот, оружие воинское; тобою Я поражал народы и тобою разорял царства;
\vs Jer 51:21 тобою поражал коня и всадника его и тобою поражал колесницу и возницу ее;
\vs Jer 51:22 тобою поражал мужа и жену, тобою поражал и старого и молодого, тобою поражал и юношу и девицу;
\vs Jer 51:23 и тобою поражал пастуха и стадо его, тобою поражал и земледельца и рабочий скот его, тобою поражал и областеначальников и градоправителей.
\vs Jer 51:24 И воздам Вавилону и всем жителям Халдеи за все то зло, какое они делали на Сионе в глазах ваших, говорит Господь.
\vs Jer 51:25 Вот, Я~--- на тебя, гора губительная, говорит Господь, разоряющая всю землю, и простру на тебя руку Мою, и низрину тебя со скал, и сделаю тебя горою обгорелою.
\vs Jer 51:26 И не возьмут из тебя камня для углов и камня для основания, но вечно будешь запустением, говорит Господь.
\vs Jer 51:27 Поднимите знамя на земле, трубите трубою среди народов, вооружите против него народы, созовите на него царства Араратские, Минийские и Аскеназские, поставьте вождя против него, наведите коней, как страшную саранчу.
\vs Jer 51:28 Вооружите против него народы, царей Мидии, областеначальников ее и всех градоправителей ее, и всю землю, подвластную ей.
\vs Jer 51:29 Трясется земля и трепещет, ибо исполняются над Вавилоном намерения Господа сделать землю Вавилонскую пустынею, без жителей.
\vs Jer 51:30 Перестали сражаться сильные Вавилонские, сидят в укреплениях своих; истощилась сила их, сделались как женщины, жилища их сожжены, затворы их сокрушены.
\vs Jer 51:31 Гонец бежит навстречу гонцу, и вестник навстречу вестнику, чтобы возвестить царю Вавилонскому, что город его взят со всех концов,
\vs Jer 51:32 и броды захвачены, и ограды сожжены огнем, и воины поражены страхом.
\vs Jer 51:33 Ибо так говорит Господь Саваоф, Бог Израилев: дочь Вавилона подобна гумну во время молотьбы на нем; еще немного, и наступит время жатвы ее.
\vs Jer 51:34 Пожирал меня и грыз меня Навуходоносор, царь Вавилонский; сделал меня пустым сосудом; поглощал меня, как дракон; наполнял чрево свое сластями моими, извергал меня.
\vs Jer 51:35 Обида моя и плоть моя~--- на Вавилоне, скажет обитательница Сиона, и кровь моя~--- на жителях Халдеи, скажет Иерусалим.
\vs Jer 51:36 Посему так говорит Господь: вот, Я вступлюсь в твое дело и отмщу за тебя, и осушу море его, и иссушу каналы его.
\vs Jer 51:37 И Вавилон будет грудою развалин, жилищем шакалов, ужасом и посмеянием, без жителей.
\vs Jer 51:38 Как львы зарыкают все они, и заревут как щенки львиные.
\vs Jer 51:39 Во время разгорячения их сделаю им пир и упою их, чтобы они повеселились и заснули вечным сном, и не пробуждались, говорит Господь.
\vs Jer 51:40 Сведу их как ягнят на заклание, как овнов с козлами.
\vs Jer 51:41 Как взят Сесах, и завоевана слава всей земли! Как сделался Вавилон ужасом между народами!
\vs Jer 51:42 Устремилось на Вавилон море; он покрыт множеством волн его.
\vs Jer 51:43 Города его сделались пустыми, землею сухою, степью, землею, где не живет ни один человек и где не проходит сын человеческий.
\vs Jer 51:44 И посещу Вила в Вавилоне, и исторгну из уст его проглоченное им, и народы не будут более стекаться к нему, даже и стены Вавилонские падут.
\vs Jer 51:45 Выходи из среды его, народ Мой, и спасайте каждый душу свою от пламенного гнева Господа.
\vs Jer 51:46 Да не ослабевает сердце ваше, и не бойтесь слуха, который будет слышен на земле; слух придет в \bibemph{один} год, и потом в \bibemph{другой} год, и на земле \bibemph{будет} насилие, властелин \bibemph{восстанет} на властелина.
\vs Jer 51:47 Посему вот, приходят дни, когда Я посещу идолов Вавилона, и вся земля его будет посрамлена, и все пораженные его падут среди него.
\vs Jer 51:48 И восторжествуют над Вавилоном небо и земля и всё, что на них; ибо от севера придут к нему опустошители, говорит Господь.
\vs Jer 51:49 Как Вавилон повергал пораженных Израильтян, так в Вавилоне будут повержены пораженные всей страны.
\vs Jer 51:50 Спасшиеся от меча, уходите, не останавливайтесь, вспомните издали о Господе, и да взойдет Иерусалим на сердце ваше.
\vs Jer 51:51 Стыдно нам было, когда мы слышали ругательство: бесчестие покрывало лица наши, когда чужеземцы пришли во святилище дома Господня.
\vs Jer 51:52 За то вот, приходят дни, говорит Господь, когда Я посещу истуканов его, и по всей земле его будут стонать раненые.
\vs Jer 51:53 Хотя бы Вавилон возвысился до небес, и хотя бы он на высоте укрепил твердыню свою; \bibemph{но} от Меня придут к нему опустошители, говорит Господь.
\vs Jer 51:54 \bibemph{Пронесется} гул вопля от Вавилона и великое разрушение~--- от земли Халдейской,
\vs Jer 51:55 ибо Господь опустошит Вавилон и положит конец горделивому голосу в нем. Зашумят волны их как большие воды, раздастся шумный голос их.
\vs Jer 51:56 Ибо придет на него, на Вавилон, опустошитель, и взяты будут ратоборцы его, сокрушены будут луки их; ибо Господь, Бог воздаяний, воздаст воздаяние.
\vs Jer 51:57 И напою допьяна князей его и мудрецов его, областеначальников его, и градоправителей его, и воинов его, и заснут сном вечным, и не пробудятся, говорит Царь~--- Господь Саваоф имя Его.
\vs Jer 51:58 Так говорит Господь Саваоф: толстые стены Вавилона до основания будут разрушены, и высокие ворота его будут сожжены огнем; итак напрасно трудились народы, и племена мучили себя для огня.
\rsbpar\vs Jer 51:59 Слово, которое пророк Иеремия заповедал Сераии, сыну Нирии, сыну Маасеи, когда он отправлялся в Вавилон с Седекиею, царем Иудейским, в четвертый год его царствования; Сераия был главный постельничий.
\vs Jer 51:60 Иеремия вписал в одну книгу все бедствия, какие должны были прийти на Вавилон, все сии речи, написанные на Вавилон.
\vs Jer 51:61 И сказал Иеремия Сераии: когда ты придешь в Вавилон, то смотри, прочитай все сии речи,
\vs Jer 51:62 и скажи: <<Господи! Ты изрек о месте сем, что истребишь его так, что не останется в нем ни человека, ни скота, но оно будет вечною пустынею>>.
\vs Jer 51:63 И когда окончишь чтение сей книги, привяжи к ней камень и брось ее в средину Евфрата,
\vs Jer 51:64 и скажи: <<так погрузится Вавилон и не восстанет от того бедствия, которое Я наведу на него, и они совершенно изнемогут>>. Доселе речи Иеремии.
\vs Jer 52:1 Седекия был двадцати одного года, когда начал царствовать, и царствовал в Иерусалиме одиннадцать лет; имя матери его~--- Хамуталь, дочь Иеремии из Ливны.
\vs Jer 52:2 И он делал злое в очах Господа, все то, что делал Иоаким;
\vs Jer 52:3 посему гнев Господа был над Иерусалимом и Иудою до того, что Он отверг их от лица Своего; и Седекия отложился от царя Вавилонского.
\vs Jer 52:4 И было, в девятый год его царствования, в десятый месяц, в десятый день месяца, пришел Навуходоносор, царь Вавилонский, сам и все войско его, к Иерусалиму, и обложили его, и устроили вокруг него насыпи.
\vs Jer 52:5 И находился город в осаде до одиннадцатого года царя Седекии.
\vs Jer 52:6 В четвертом месяце, в девятый день месяца, голод в городе усилился, и не было хлеба у народа земли.
\vs Jer 52:7 Сделан был пролом в город, и побежали все военные, и вышли из города ночью воротами, находящимися между двумя стенами, подле царского сада, и пошли дорогою степи; Халдеи же были вокруг города.
\vs Jer 52:8 Войско Халдейское погналось за царем, и настигли Седекию на равнинах Иерихонских, и все войско его разбежалось от него.
\vs Jer 52:9 И взяли царя, и привели его к царю Вавилонскому, в Ривлу, в землю Емаф, где он произнес над ним суд.
\vs Jer 52:10 И заколол царь Вавилонский сыновей Седекии пред глазами его, и всех князей Иудейских заколол в Ривле.
\vs Jer 52:11 А Седекии выколол глаза и велел оковать его медными оковами; и отвел его царь Вавилонский в Вавилон и посадил его в дом стражи до дня смерти его.
\rsbpar\vs Jer 52:12 В пятый месяц, в десятый день месяца,~--- это был девятнадцатый год царя Навуходоносора, царя Вавилонского,~--- пришел Навузардан, начальник телохранителей, предстоявший пред царем Вавилонским, в Иерусалим
\vs Jer 52:13 и сожег дом Господень, и дом царя, и все домы в Иерусалиме, и все домы большие сожег огнем.
\vs Jer 52:14 И все войско Халдейское, бывшее с начальником телохранителей, разрушило все стены вокруг Иерусалима.
\vs Jer 52:15 Бедных из народа и прочий народ, остававшийся в городе, и переметчиков, которые передались царю Вавилонскому, и вообще остаток простого народа Навузардан, начальник телохранителей, выселил.
\vs Jer 52:16 Только несколько из бедного народа земли Навузардан, начальник телохранителей, оставил для виноградников и земледелия.
\vs Jer 52:17 И столбы медные, которые были в доме Господнем, и подставы, и медное море, которое в доме Господнем, изломали Халдеи и отнесли всю медь их в Вавилон.
\vs Jer 52:18 И тазы, и лопатки, и ножи, и чаши, и ложки, и все медные сосуды, которые употребляемы были при богослужении, взяли;
\vs Jer 52:19 и блюда, и щипцы, и чаши, и котлы, и лампады, и фимиамники, и кр\acc{у}жки, что было золотое~--- золотое, и что было серебряное~--- серебряное, взял начальник телохранителей;
\vs Jer 52:20 также два столба, одно море и двенадцать медных волов, которые служили подставами, которые царь Соломон сделал в доме Господнем,~--- меди во всех этих вещах невозможно было взвесить.
\vs Jer 52:21 Столбы сии были каждый столб в восемнадцать локтей вышины, и шнурок в двенадцать локтей обнимал его, а толщина стенок его внутри пустого, в четыре перста.
\vs Jer 52:22 И венец на нем медный, а высота венца пять локтей; и сетка и гранатовые яблоки вокруг были все медные; то же и на другом столбе с гранатовыми яблоками.
\vs Jer 52:23 Гранатовых яблоков было по всем сторонам девяносто шесть; всех яблоков вокруг сетки сто.
\vs Jer 52:24 Начальник телохранителей взял также Сераию первосвященника и Цефанию, второго священника, и трех сторожей порога.
\vs Jer 52:25 И из города взял одного евнуха, который был начальником над военными людьми, и семь человек предстоявших лицу царя, которые находились в городе, и главного писца в войске, записывавшего в войско народ земли, и шестьдесят человек из народа страны, найденных в городе.
\vs Jer 52:26 И взял их Навузардан, начальник телохранителей, и отвел их к царю Вавилонскому в Ривлу.
\vs Jer 52:27 И поразил их царь Вавилонский и умертвил их в Ривле, в земле Емаф; и выселен был Иуда из земли своей.
\vs Jer 52:28 Вот народ, который выселил Навуходоносор: в седьмой год три тысячи двадцать три Иудея;
\vs Jer 52:29 в восемнадцатый год Навуходоносора из Иерусалима \bibemph{выселено} восемьсот тридцать две души;
\vs Jer 52:30 в двадцать третий год Навуходоносора Навузардан, начальник телохранителей, выселил Иудеев семьсот сорок пять душ: всего четыре тысячи шестьсот душ.
\rsbpar\vs Jer 52:31 В тридцать седьмой год после переселения Иоакима\fns{Иехонии.}, царя Иудейского, в двенадцатый месяц, в двадцать пятый день месяца, Евильмеродах, царь Вавилонский, в первый год царствования своего, возвысил Иоакима, царя Иудейского, и вывел его из темничного дома.
\vs Jer 52:32 И беседовал с ним дружелюбно, и поставил престол его выше престола царей, которые были у него в Вавилоне;
\vs Jer 52:33 и переменил темничные одежды его, и он всегда у него обедал во все дни жизни своей.
\vs Jer 52:34 И содержание его, содержание постоянное, выдаваемо было ему от царя изо дня в день до дня смерти его, во все дни жизни его.

\bibbookdescr{Lam}{
  inline={\LARGE Книга\\\Huge Плач Иеремии},
  toc={Плач Иеремии},
  bookmark={Плач Иеремии},
  header={Плач Иеремии},
  %headerleft={},
  %headerright={},
  abbr={Плач}
}
\vs Lam 1:1 Как одиноко сидит город, некогда многолюдный! он стал, как вдова; великий между народами, князь над областями сделался данником.
\vs Lam 1:2 Горько плачет он ночью, и слезы его на ланитах его. Нет у него утешителя из всех, любивших его; все друзья его изменили ему, сделались врагами ему.
\vs Lam 1:3 Иуда переселился по причине бедствия и тяжкого рабства, поселился среди язычников, и не нашел покоя; все, преследовавшие его, настигли его в тесных местах.
\vs Lam 1:4 Пути Сиона сетуют, потому что нет идущих на праздник; все ворота его опустели; священники его вздыхают, девицы его печальны, горько и ему самому.
\vs Lam 1:5 Враги его стали во главе, неприятели его благоденствуют, потому что Господь наслал на него горе за множество беззаконий его; дети его пошли в плен впереди врага.
\vs Lam 1:6 И отошло от дщери Сиона все ее великолепие; князья ее~--- как олени, не находящие пажити; обессиленные они пошли вперед погонщика.
\vs Lam 1:7 Вспомнил Иерусалим, во дни бедствия своего и страданий своих, о всех драгоценностях своих, какие были у него в прежние дни, тогда как народ его пал от руки врага, и никто не помогает ему; неприятели смотрят на него и смеются над его субботами.
\vs Lam 1:8 Тяжко согрешил Иерусалим, за то и сделался отвратительным; все, прославлявшие его, смотрят на него с презрением, потому что увидели наготу его; и сам он вздыхает и отворачивается назад.
\vs Lam 1:9 На подоле у него была нечистота, но он не помышлял о будущности своей, и поэтому необыкновенно унизился, и нет у него утешителя. <<Воззри, Господи, на бедствие мое, ибо враг возвеличился!>>
\vs Lam 1:10 Враг простер руку свою на все самое драгоценное его; он видит, как язычники входят во святилище его, о котором Ты заповедал, чтобы они не вступали в собрание Твое.
\vs Lam 1:11 Весь народ его вздыхает, ища хлеба, отдает драгоценности свои за пищу, чтобы подкрепить душу. <<Воззри, Господи, и посмотри, как я унижен!>>
\vs Lam 1:12 Да не будет этого с вами, все проходящие путем! взгляните и посмотрите, есть ли болезнь, как моя болезнь, какая постигла меня, какую наслал на меня Господь в день пламенного гнева Своего?
\vs Lam 1:13 Свыше послал Он огонь в кости мои, и он овладел ими; раскинул сеть для ног моих, опрокинул меня, сделал меня бедным и томящимся всякий день.
\vs Lam 1:14 Ярмо беззаконий моих связано в руке Его; они сплетены и поднялись на шею мою; Он ослабил силы мои. Господь отдал меня в руки, из которых не могу подняться.
\vs Lam 1:15 Всех сильных моих Господь низложил среди меня, созвал против меня собрание, чтобы истребить юношей моих; как в точиле, истоптал Господь деву, дочь Иуды.
\vs Lam 1:16 Об этом плачу я; око мое, око мое изливает воды, ибо далеко от меня утешитель, который оживил бы душу мою; дети мои разорены, потому что враг превозмог.
\vs Lam 1:17 Сион простирает руки свои, но утешителя нет ему. Господь дал повеление о Иакове врагам его окружить его; Иерусалим сделался мерзостью среди них.
\vs Lam 1:18 Праведен Господь, ибо я непокорен был слову Его. Послушайте, все народы, и взгляните на болезнь мою: девы мои и юноши мои пошли в плен.
\vs Lam 1:19 Зову друзей моих, но они обманули меня; священники мои и старцы мои издыхают в городе, ища пищи себе, чтобы подкрепить душу свою.
\vs Lam 1:20 Воззри, Господи, ибо мне тесно, волнуется во мне внутренность, сердце мое перевернулось во мне за то, что я упорно противился Тебе; отвне обесчадил меня меч, а дома~--- как смерть.
\vs Lam 1:21 Услышали, что я стенаю, а утешителя у меня нет; услышали все враги мои о бедствии моем и обрадовались, что Ты соделал это: о, если бы Ты повелел наступить дню, предреченному Тобою, и они стали бы подобными мне!
\vs Lam 1:22 Да предстанет пред лице Твое вся злоба их; и поступи с ними так же, как Ты поступил со мною за все грехи мои, ибо тяжки стоны мои, и сердце мое изнемогает.
\vs Lam 2:1 Как помрачил Господь во гневе Своем дщерь Сиона! с небес поверг на землю красу Израиля и не вспомнил о подножии ног Своих в день гнева Своего.
\vs Lam 2:2 Погубил Господь все жилища Иакова, не пощадил, разрушил в ярости Своей укрепления дщери Иудиной, поверг на землю, отверг царство и князей его, как нечистых:
\vs Lam 2:3 в пылу гнева сломил все роги Израилевы, отвел десницу Свою от неприятеля и воспылал в Иакове, как палящий огонь, пожиравший все вокруг;
\vs Lam 2:4 натянул лук Свой, как неприятель, направил десницу Свою, как враг, и убил все, вожделенное для глаз; на скинию дщери Сиона излил ярость Свою, как огонь.
\vs Lam 2:5 Господь стал как неприятель, истребил Израиля, разорил все чертоги его, разрушил укрепления его и распространил у дщери Иудиной сетование и плач.
\vs Lam 2:6 И отнял ограду Свою, как у сада; разорил Свое место собраний, заставил Господь забыть на Сионе празднества и субботы; и в негодовании гнева Своего отверг царя и священника.
\vs Lam 2:7 Отверг Господь жертвенник Свой, отвратил сердце Свое от святилища Своего, предал в руки врагов стены чертогов его; в доме Господнем они шумели, как в праздничный день.
\vs Lam 2:8 Господь определил разрушить стену дщери Сиона, протянул вервь, не отклонил руки Своей от разорения; истребил внешние укрепления, и стены вместе разрушены.
\vs Lam 2:9 Ворота ее вдались в землю; Он разрушил и сокрушил запоры их; царь ее и князья ее~--- среди язычников; не стало закона, и пророки ее не сподобляются видений от Господа.
\vs Lam 2:10 Сидят на земле безмолвно старцы дщери Сионовой, посыпали пеплом свои головы, препоясались вретищем; опустили к земле головы свои девы Иерусалимские.
\vs Lam 2:11 Истощились от слез глаза мои, волнуется во мне внутренность моя, изливается на землю печень моя от гибели дщери народа моего, когда дети и грудные младенцы умирают от голода среди городских улиц.
\vs Lam 2:12 Матерям своим говорят они: <<где хлеб и вино?>>, умирая, подобно раненым, на улицах городских, изливая души свои в лоно матерей своих.
\vs Lam 2:13 Что мне сказать тебе, с чем сравнить тебя, дщерь Иерусалима? чему уподобить тебя, чтобы утешить тебя, дева, дщерь Сиона? ибо рана твоя велика, как море; кто может исцелить тебя?
\vs Lam 2:14 Пророки твои провещали тебе пустое и ложное и не раскрывали твоего беззакония, чтобы отвратить твое пленение, и изрекали тебе откровения ложные и приведшие тебя к изгнанию.
\vs Lam 2:15 Руками всплескивают о тебе все проходящие путем, свищут и качают головою своею о дщери Иерусалима, говоря: <<это ли город, который называли совершенством красоты, радостью всей земли?>>
\vs Lam 2:16 Разинули на тебя пасть свою все враги твои, свищут и скрежещут зубами, говорят: <<поглотили мы его, только этого дня и ждали мы, дождались, увидели!>>
\vs Lam 2:17 Совершил Господь, что определил, исполнил слово Свое, изреченное в древние дни, разорил без пощады и дал врагу порадоваться над тобою, вознес рог неприятелей твоих.
\vs Lam 2:18 Сердце их вопиет к Господу: стена дщери Сиона! лей ручьем слезы день и ночь, не давай себе покоя, не спускай зениц очей твоих.
\vs Lam 2:19 Вставай, взывай ночью, при начале каждой стражи; изливай, как воду, сердце твое пред лицем Господа; простирай к Нему руки твои о душе детей твоих, издыхающих от голода на углах всех улиц.
\vs Lam 2:20 <<Воззри, Господи, и посмотри: кому Ты сделал так, чтобы женщины ели плод свой, младенцев, вскормленных ими? чтобы убиваемы были в святилище Господнем священник и пророк?
\vs Lam 2:21 Дети и старцы лежат на земле по улицам; девы мои и юноши мои пали от меча; Ты убивал их в день гнева Твоего, заколал без пощады.
\vs Lam 2:22 Ты созвал отовсюду, как на праздник, ужасы мои, и в день гнева Господня никто не спасся, никто не уцелел; тех, которые были мною вскормлены и выращены, враг мой истребил>>.
\vs Lam 3:1 Я человек, испытавший горе от жезла гнева Его.
\vs Lam 3:2 Он повел меня и ввел во тьму, а не во свет.
\vs Lam 3:3 Так, Он обратился на меня и весь день обращает руку Свою;
\vs Lam 3:4 измождил плоть мою и кожу мою, сокрушил кости мои;
\vs Lam 3:5 огородил меня и обложил горечью и тяготою;
\vs Lam 3:6 посадил меня в темное место, как давно умерших;
\vs Lam 3:7 окружил меня стеною, чтобы я не вышел, отяготил оковы мои,
\vs Lam 3:8 и когда я взывал и вопиял, задерживал молитву мою;
\vs Lam 3:9 каменьями преградил дороги мои, извратил стези мои.
\vs Lam 3:10 Он стал для меня как бы медведь в засаде, \bibemph{как бы} лев в скрытном месте;
\vs Lam 3:11 извратил пути мои и растерзал меня, привел меня в ничто;
\vs Lam 3:12 натянул лук Свой и поставил меня как бы целью для стрел;
\vs Lam 3:13 послал в почки мои стрелы из колчана Своего.
\vs Lam 3:14 Я стал посмешищем для всего народа моего, вседневною песнью их.
\vs Lam 3:15 Он пресытил меня горечью, напоил меня полынью.
\vs Lam 3:16 Сокрушил камнями зубы мои, покрыл меня пеплом.
\vs Lam 3:17 И удалился мир от души моей; я забыл о благоденствии,
\vs Lam 3:18 и сказал я: погибла сила моя и надежда моя на Господа.
\vs Lam 3:19 Помысли о моем страдании и бедствии моем, о полыни и желчи.
\vs Lam 3:20 Твердо помнит это душа моя и падает во мне.
\vs Lam 3:21 Вот что я отвечаю сердцу моему и потому уповаю:
\vs Lam 3:22 по милости Господа мы не исчезли, ибо милосердие Его не истощилось.
\vs Lam 3:23 Оно обновляется каждое утро; велика верность Твоя!
\vs Lam 3:24 Господь часть моя, говорит душа моя, итак буду надеяться на Него.
\vs Lam 3:25 Благ Господь к надеющимся на Него, к душе, ищущей Его.
\vs Lam 3:26 Благо тому, кто терпеливо ожидает спасения от Господа.
\vs Lam 3:27 Благо человеку, когда он несет иго в юности своей;
\vs Lam 3:28 сидит уединенно и молчит, ибо Он наложил его на него;
\vs Lam 3:29 полагает уста свои в прах, \bibemph{помышляя}: <<может быть, еще есть надежда>>;
\vs Lam 3:30 подставляет ланиту свою биющему его, пресыщается поношением,
\vs Lam 3:31 ибо не навек оставляет Господь.
\vs Lam 3:32 Но послал горе, и помилует по великой благости Своей.
\vs Lam 3:33 Ибо Он не по изволению сердца Своего наказывает и огорчает сынов человеческих.
\vs Lam 3:34 Но, когда попирают ногами своими всех узников земли,
\vs Lam 3:35 когда неправедно судят человека пред лицем Всевышнего,
\vs Lam 3:36 когда притесняют человека в деле его: разве не видит Господь?
\vs Lam 3:37 Кто это говорит: <<и то бывает, чему Господь не повелел быть>>?
\vs Lam 3:38 Не от уст ли Всевышнего происходит бедствие и благополучие?
\vs Lam 3:39 Зачем сетует человек живущий? всякий сетуй на грехи свои.
\vs Lam 3:40 Испытаем и исследуем пути свои, и обратимся к Господу.
\vs Lam 3:41 Вознесем сердце наше и руки к Богу, \bibemph{сущему} на небесах:
\vs Lam 3:42 мы отпали и упорствовали; Ты не пощадил.
\vs Lam 3:43 Ты покрыл Себя гневом и преследовал нас, умерщвлял, не щадил;
\vs Lam 3:44 Ты закрыл Себя облаком, чтобы не доходила молитва наша;
\vs Lam 3:45 сором и мерзостью Ты сделал нас среди народов.
\vs Lam 3:46 Разинули на нас пасть свою все враги наши.
\vs Lam 3:47 Ужас и яма, опустошение и разорение~--- доля наша.
\vs Lam 3:48 Потоки вод изливает око мое о гибели дщери народа моего.
\vs Lam 3:49 Око мое изливается и не перестает, ибо нет облегчения,
\vs Lam 3:50 доколе не призрит и не увидит Господь с небес.
\vs Lam 3:51 Око мое опечаливает душу мою ради всех дщерей моего города.
\vs Lam 3:52 Всячески усиливались уловить меня, как птичку, враги мои, без всякой причины;
\vs Lam 3:53 повергли жизнь мою в яму и закидали меня камнями.
\vs Lam 3:54 Воды поднялись до головы моей; я сказал: <<погиб я>>.
\vs Lam 3:55 Я призывал имя Твое, Господи, из ямы глубокой.
\vs Lam 3:56 Ты слышал голос мой; не закрой уха Твоего от воздыхания моего, от вопля моего.
\vs Lam 3:57 Ты приближался, когда я взывал к Тебе, и говорил: <<не бойся>>.
\vs Lam 3:58 Ты защищал, Господи, дело души моей; искуплял жизнь мою.
\vs Lam 3:59 Ты видишь, Господи, обиду мою; рассуди дело мое.
\vs Lam 3:60 Ты видишь всю мстительность их, все замыслы их против меня.
\vs Lam 3:61 Ты слышишь, Господи, ругательство их, все замыслы их против меня,
\vs Lam 3:62 речи восстающих на меня и их ухищрения против меня всякий день.
\vs Lam 3:63 Воззри, сидят ли они, встают ли, я для них~--- песнь.
\vs Lam 3:64 Воздай им, Господи, по делам рук их;
\vs Lam 3:65 пошли им помрачение сердца и проклятие Твое на них;
\vs Lam 3:66 преследуй их, Господи, гневом, и истреби их из поднебесной.
\vs Lam 4:1 Как потускло золото, изменилось золото наилучшее! камни святилища раскиданы по всем перекресткам.
\vs Lam 4:2 Сыны Сиона драгоценные, равноценные чистейшему золоту, как они сравнены с глиняною посудою, изделием рук горшечника!
\vs Lam 4:3 И чудовища подают сосцы и кормят своих детенышей, а дщерь народа моего стала жестока подобно страусам в пустыне.
\vs Lam 4:4 Язык грудного младенца прилипает к гортани его от жажды; дети просят хлеба, и никто не подает им.
\vs Lam 4:5 Евшие сладкое истаевают на улицах; воспитанные на багрянице жмутся к навозу.
\vs Lam 4:6 Наказание нечестия дщери народа моего превышает казнь за грехи Содома: тот низринут мгновенно, и руки человеческие не касались его.
\vs Lam 4:7 Князья ее \bibemph{были} в ней чище снега, белее молока; они были телом краше коралла, вид их был, как сапфир;
\vs Lam 4:8 а теперь темнее всего черного лице их; не узна\acc{ю}т их на улицах; кожа их прилипла к костям их, стала суха, как дерево.
\vs Lam 4:9 Умерщвляемые мечом счастливее умерщвляемых голодом, потому что сии истаевают, поражаемые недостатком плодов полевых.
\vs Lam 4:10 Руки мягкосердых женщин варили детей своих, чтобы они были для них пищею во время гибели дщери народа моего.
\vs Lam 4:11 Совершил Господь гнев Свой, излил ярость гнева Своего и зажег на Сионе огонь, который пожрал основания его.
\vs Lam 4:12 Не верили цари земли и все живущие во вселенной, чтобы враг и неприятель вошел во врата Иерусалима.
\vs Lam 4:13 \bibemph{Все это}~--- за грехи лжепророков его, за беззакония священников его, которые среди него проливали кровь праведников;
\vs Lam 4:14 бродили как слепые по улицам, осквернялись кровью, так что невозможно было прикоснуться к одеждам их.
\vs Lam 4:15 <<Сторонитесь! нечистый!>> кричали им; <<сторонитесь, сторонитесь, не прикасайтесь>>; и они уходили в смущении; а между народом говорили: <<их более не будет!
\vs Lam 4:16 лице Господне рассеет их; Он уже не призрит на них>>, потому что они лиц\acc{а} священников не уважают, старцев не милуют.
\vs Lam 4:17 Наши глаза истомлены в напрасном ожидании помощи; со сторожевой башни нашей мы ожидали народ, который не мог спасти нас.
\vs Lam 4:18 А они подстерегали шаги наши, чтобы мы не могли ходить по улицам нашим; приблизился конец наш, дни наши исполнились; пришел конец наш.
\vs Lam 4:19 Преследовавшие нас были быстрее орлов небесных; гонялись за нами по горам, ставили засаду для нас в пустыне.
\vs Lam 4:20 Дыхание жизни нашей, помазанник Господень пойман в ямы их, тот, о котором мы говорили: <<под тенью его будем жить среди народов>>.
\vs Lam 4:21 Радуйся и веселись, дочь Едома, обитательница земли Уц! И до тебя дойдет чаша; напьешься допьяна и обнажишься.
\vs Lam 4:22 Дщерь Сиона! наказание за беззаконие твое кончилось; Он не будет более изгонять тебя; но твое беззаконие, дочь Едома, Он посетит и обнаружит грехи твои.
\vs Lam 5:1 Вспомни, Господи, что над нами совершилось; призри и посмотри на поругание наше.
\vs Lam 5:2 Наследие наше перешло к чужим, домы наши~--- к иноплеменным;
\vs Lam 5:3 мы сделались сиротами, без отца; матери наши~--- как вдовы.
\vs Lam 5:4 Воду свою пьем за серебро, дрова наши достаются нам за деньги.
\vs Lam 5:5 Нас погоняют в шею, мы работаем, \bibemph{и} не имеем отдыха.
\vs Lam 5:6 Протягиваем руку к Египтянам, к Ассириянам, чтобы насытиться хлебом.
\vs Lam 5:7 Отцы наши грешили: их уже нет, а мы несем наказание за беззакония их.
\vs Lam 5:8 Рабы господствуют над нами, и некому избавить от руки их.
\vs Lam 5:9 С опасностью жизни от меча, в пустыне достаем хлеб себе.
\vs Lam 5:10 Кожа наша почернела, как печь, от жгучего голода.
\vs Lam 5:11 Жен бесчестят на Сионе, девиц~--- в городах Иудейских.
\vs Lam 5:12 Князья повешены руками их, лица старцев не уважены.
\vs Lam 5:13 Юношей берут к жерновам, и отроки падают под ношами дров.
\vs Lam 5:14 Старцы уже не сидят у ворот; юноши не поют.
\vs Lam 5:15 Прекратилась радость сердца нашего; хороводы наши обратились в сетование.
\vs Lam 5:16 Упал венец с головы нашей; горе нам, что мы согрешили!
\vs Lam 5:17 От сего-то изнывает сердце наше; от сего померкли глаза наши.
\vs Lam 5:18 Оттого, что опустела гора Сион, лисицы ходят по ней.
\vs Lam 5:19 Ты, Господи, пребываешь во веки; престол Твой~--- в род и род.
\vs Lam 5:20 Для чего совсем забываешь нас, оставляешь нас на долгое время?
\vs Lam 5:21 Обрати нас к Тебе, Господи, и мы обратимся; обнови дни наши, как древле.
\vs Lam 5:22 Неужели Ты совсем отверг нас, прогневался на нас безмерно?

\bibbookdescr{Epj}{
  inline={\LARGE Послание\\\Huge Иеремии\fns{Переведена с греческого.}},
  toc={Послание Иеремии*},
  bookmark={Послание Иеремии},
  header={Послание Иеремии},
  %headerleft={},
  %headerright={},
  abbr={Посл~Иер}
}
\vs Epj 1:1 Список послания, которое послал Иеремия к пленникам, отводимым в Вавилон царем Вавилонским, чтобы возвестить им, чт\acc{о} повелено ему Богом.
\rsbpar\vs Epj 1:2 За грехи, которыми вы согрешили пред Богом, будете отведены пленниками в Вавилон Навуходоносором, царем Вавилонским.
\vs Epj 1:3 Войдя в Вавилон, вы пробудете там многие годы и долгое время, даже до семи родов; после же сего Я выведу вас оттуда с миром.
\vs Epj 1:4 Теперь вы увидите в Вавилоне богов серебряных и золотых и деревянных, носимых на плечах, внушающих страх язычникам.
\vs Epj 1:5 Берегитесь же, чтобы и вам не сделаться подобными иноплеменникам, и чтобы страх пред ними не овладел и вами. Видя толпу спереди и сзади их поклоняющеюся перед ними, скажите в уме: <<Тебе должно поклоняться, Владыко!>>
\vs Epj 1:6 Ибо Ангел Мой с вами, и он защитник душ ваших.
\vs Epj 1:7 Язык их выстроган художником, и сами они оправлены в золото и серебро; но они ложные, и не могут говорить.
\vs Epj 1:8 И как бы для девицы, любящей украшение, берут они золото, и приготовляют венцы на головы богов своих.
\vs Epj 1:9 Бывает также, что жрецы похищают у богов своих золото и серебро и употребляют его на себя самих;
\vs Epj 1:10 уделяют из того и блудницам под их кровом; украшают богов золотых и серебряных и деревянных одеждами, как людей.
\vs Epj 1:11 Но они не спасаются от ржавчины и моли, хотя облечены в пурпуровую одежду.
\vs Epj 1:12 Обтирают лице их от пыли в капище, которой на них очень много.
\vs Epj 1:13 Имеет и скипетр, как человек~--- судья страны, но он не может умертвить виновного пред ним.
\vs Epj 1:14 Имеет меч в правой руке и секиру, а себя самого от войска и разбойников не защитит: отсюда познается, что они не боги; итак, не бойтесь их.
\vs Epj 1:15 Ибо, как разбитый сосуд делается бесполезным для человека, так и боги их.
\vs Epj 1:16 После того, как они поставлены в капищах, глаза их полны пыли от ног входящих.
\vs Epj 1:17 И как у нанесшего оскорбление царю заграждаются входы в жилье, когда он отводится на смерть, \bibemph{так} капища их охраняют жрецы их дверями и замками и засовами, чтобы они не были ограблены разбойниками;
\vs Epj 1:18 зажигают для них светильники, и больше, нежели для себя самих, а они ни одного из них не могут видеть.
\vs Epj 1:19 Они как бревно в доме; сердца их, говорят, точат черви земляные, и съедают их самих и одежду их,~--- а они не чувствуют.
\vs Epj 1:20 Лица их черны от курения в капищах.
\vs Epj 1:21 На тело их и на головы их налетают летучие мыши и ласточки и другие птицы, \bibemph{лазают} также по ним и кошки.
\vs Epj 1:22 Из этого уразумеете, что это не боги; итак, не бойтесь их.
\rsbpar\vs Epj 1:23 Если кто не очистит от ржавчины золота, которым они обложены для красы, то они не будут блестеть; и когда выливали их, они не чувствовали.
\vs Epj 1:24 За большую цену они куплены, а духа нет в них.
\vs Epj 1:25 Безногие, они носятся на плечах, показывая чрез то свою ничтожность людям; посрамляются же и служащие им;
\vs Epj 1:26 потому что, в случае падения их на землю, сами собою они не могут встать; также, если бы кто поставил их прямо, не могут сами собою двигаться и, если бы кто наклонил их, не могут выпрямиться; но как перед мертвыми полагают перед ними дары.
\vs Epj 1:27 Жертвы их жрецы продают и злоупотребляют ими; равно и жены их часть из них солят, и ничего не уделяют ни нищему, ни больному.
\vs Epj 1:28 К жертвам их прикасаются женщины нечистые и родильницы. Итак, познав из сего, что они не боги, не бойтесь их.
\vs Epj 1:29 Как же назвать их богами? женщины приносят жертвы этим серебряным и золотым и деревянным богам.
\vs Epj 1:30 И в капищах их сидят жрецы в разодранных одеждах, с обритыми головами и бородами и с непокрытыми головами:
\vs Epj 1:31 ревут они с воплем пред своими богами, как иные на поминках по умершим.
\vs Epj 1:32 Некоторые из одежд их жрецы берут себе и одевают ими своих жен и детей.
\vs Epj 1:33 Если испытывают от кого-либо злое или доброе, не могут воздать; не могут поставить царя, ни низложить его.
\vs Epj 1:34 Равно ни богатства, ни даже мелкой медной монеты они не могут дать. Если кто, обещав им обет, не исполнил бы его, не взыщут.
\vs Epj 1:35 От смерти человека не избавят, ни слабейшего у сильного не отнимут;
\vs Epj 1:36 человеку слепому не возвратят зрения; человеку в нужде не помогут;
\vs Epj 1:37 вдове не окажут сострадания, и сироте не сделают добра.
\vs Epj 1:38 Камням из гор подобны \bibemph{эти боги} деревянные и оправленные в золото и серебро,~--- и служащие им посрамятся.
\vs Epj 1:39 Как же можно подумать или сказать, что они боги?
\vs Epj 1:40 К тому же сами Халдеи обращаются с ними непочтительно: они, когда увидят немого, не могущего говорить, приносят его к Ваалу и требуют, чтобы он говорил, как будто он может чувствовать.
\vs Epj 1:41 И не могут они, заметив это, оставить их, потому что не имеют смысла.
\vs Epj 1:42 Женщины, обвязавшись тростниковым поясом, сидят на улицах, сожигая курение из оливковых зерен.
\vs Epj 1:43 И когда какая-либо из них, увлеченная проходящим, переспит с ним,~--- попрекает своей подруге, что та не удостоена того же, как она, и что перевязь ее не разорвана.
\vs Epj 1:44 Все, совершающееся у них, ложно. Посему как можно думать или говорить, что они боги?
\vs Epj 1:45 Устроены они художниками и плавильщиками золота; не чем иным они не делаются, как тем, чем желали их сделать художники.
\vs Epj 1:46 И те, которые приготовляют их, не бывают долговечны;
\vs Epj 1:47 как же сделанные ими могут быть богами? Они оставили по себе ложь и срам своим потомкам.
\vs Epj 1:48 Когда постигают их война и бедствия, жрецы совещаются между собою, где бы им скрыться с ними.
\vs Epj 1:49 Как же не понять, что те не боги, которые самих себя не спасают ни от войн, ни от бедствий?
\vs Epj 1:50 Так как они деревянные и оправленные в золото и серебро, то можно познать, что они ложь; всем народам и царям сделается ясным, что это не боги, а дела рук человеческих, и в них нет никакого действия божественного.
\vs Epj 1:51 Кому же после сего не понятно, что они не боги?
\vs Epj 1:52 Царя стране они не поставят, дождя людям не дадут;
\vs Epj 1:53 суда не рассудят, обидимого не защитят, будучи бессильны,
\vs Epj 1:54 как вор\acc{о}ны, находящиеся между небом и землею. Ибо и в том случае, когда подверглось бы пожару капище богов деревянных или оправленных в золото и серебро, жрецы их убегут и спасутся,~--- а они сами, как бревна в средине, сгорят.
\vs Epj 1:55 Ни царю, ни врагам они не могут противостать. Как же можно принять или подумать, что они боги?
\vs Epj 1:56 Ни от воров, ни от грабителей не могут охранить самих себя эти боги, деревянные и оправленные в серебро и золото:
\vs Epj 1:57 превосходя их силою, они снимают золото и серебро и одежды, которые на них, и уходят с добычею, а эти себе самим не в силах помочь.
\vs Epj 1:58 Поэтому лучше царь, выказывающий мужество, или полезный в доме сосуд, который употребляет хозяин, нежели ложные боги; или \bibemph{лучше} дверь в доме, охраняющая в нем имущество, нежели ложные боги; или \bibemph{лучше} деревянный столп в царском дворце, нежели ложные боги.
\vs Epj 1:59 Солнце и луна и звезды, будучи светлы и посылаемы ради потребности, благопослушны.
\vs Epj 1:60 Также и молния каждый раз, как является, ясно видима; также ветер во всякой стране веет.
\vs Epj 1:61 И облака, когда повелит им Бог пройти над всею вселенною, исполняют повеление.
\vs Epj 1:62 Тоже огонь, свыше ниспосылаемый для истребления гор и лесов, делает, что назначено; а эти не подобны им ни видом, ни силами.
\vs Epj 1:63 Почему же можно подумать или сказать, что они боги, когда они несильны ни суда рассудить, ни добра делать людям?
\vs Epj 1:64 Итак, зная, что они не боги, не бойтесь их.
\vs Epj 1:65 Царей они ни проклянут, ни благословят;
\vs Epj 1:66 знамений не покажут на небе и пред народами; не осветят, как солнце, и не осияют, как луна.
\vs Epj 1:67 Звери лучше их: они, убегая под кров, могут помочь себе.
\vs Epj 1:68 Итак, ни из чего не видно нам, что они боги; посему не бойтесь их.
\vs Epj 1:69 Как пугало в огороде ничего не сбережет, так и их деревянные, оправленные в золото и серебро боги.
\vs Epj 1:70 Равным образом их деревянные, оправленные в золото и серебро боги подобны терновому кусту в саду, на который садятся всякие птицы, также и трупу, брошенному во тьме.
\vs Epj 1:71 Из пурпура и червленицы, которые истлевают на них, вы можете уразуметь, что они не боги; да и сами они будут наконец съедены и будут позором в стране.
\rsbpar\vs Epj 1:72 Итак, лучше человек праведный, не имеющий идолов, ибо он~--- далеко от позора.

\bibbookdescr{Bar}{
  inline={\LARGE Книга\\\Huge Пророка Варуха\fns{Переведена с греческого.}},
  toc={Варух*},
  bookmark={Варух},
  header={Варух},
  %headerleft={},
  %headerright={},
  abbr={Вар}
}
\vs Bar 1:1 Слова книги, которые написал Варух, сын Нирии, сына Маасея, сына Седекии, сына Асадия, сына Хелкии, в Вавилоне,
\vs Bar 1:2 в пятый год, в седьмой день месяца, в то время, когда Халдеи взяли Иерусалим и сожгли его огнем.
\vs Bar 1:3 И прочитал Варух слова сей книги вслух Иехонии, сына Иоакимова, царя Иудейского, и вслух всего народа, пришедшего к слушанию книги,
\vs Bar 1:4 и вслух вельмож и сыновей царских, и вслух старейшин, и вслух всего народа, от малого до большого, всех, живших в Вавилоне при реке Суд.
\vs Bar 1:5 И они плакали, и постились, и молились пред Господом,
\vs Bar 1:6 и собрали серебра, сколько было по силам каждого,
\vs Bar 1:7 и послали в Иерусалим к Иоакиму, сыну Хелкии, сына Саломова, первосвященнику, и к священникам и ко всему народу, находившемуся с ним в Иерусалиме,
\vs Bar 1:8 когда \bibemph{Варух} унесенные из храма сосуды дома Господня принял для возвращения их в землю Иудейскую, в десятый день месяца Сиуала, сосуды серебряные, которые сделал Седекия, сын Иосии, царь Иудейский,
\vs Bar 1:9 после того, как Навуходоносор, царь Вавилонский, переселил из Иерусалима Иехонию и князей, и узников и вельмож, и народ земли и привел его в Вавилон.
\rsbpar\vs Bar 1:10 И говорили они: вот, мы посылаем вам серебро, и купите на это серебро всесожжения и \bibemph{жертву} за грех и ладан, и приготовьте дар, и вознесите на жертвенник Господа Бога нашего,
\vs Bar 1:11 и молитесь о жизни Навуходоносора, царя Вавилонского, и о жизни Валтасара, сына его, чтобы дни их были, как дни неба, на земле.
\vs Bar 1:12 И даст нам Господь силу и просветит глаза наши, и мы будем жить под покровом Навуходоносора, царя Вавилонского, и под покровом Валтасара, сына его, и будем служить им много дней, и найдем милость у них.
\vs Bar 1:13 Молитесь и о нас Господу Богу нашему, так как мы согрешили пред Господом, Богом нашим, и не отвратилась от нас ярость Господа и гнев Его до сего дня;
\vs Bar 1:14 и прочитайте сию книгу, которую мы посылаем вам, чтобы обнародовать в доме Господнем в день праздничный и в дни нарочитые;
\vs Bar 1:15 и скажите: у Господа Бога нашего~--- правда, а у нас~--- стыд на лицах, как сегодня, у всякого Иудея и у живущих в Иерусалиме,
\vs Bar 1:16 и у царей наших, и у князей наших, и у священников наших, и у пророков наших, и у отцов наших,
\vs Bar 1:17 оттого, что мы согрешили пред Господом,
\vs Bar 1:18 и не покорялись Ему, и не слушали гласа Господа Бога нашего, чтобы ходить в повелениях Господа, которые Он дал пред лицем нашим.
\vs Bar 1:19 С того дня, в который Господь вывел отцов наших из земли Египетской, и до сего дня мы были непокорны пред Господом Богом нашим и небрегли о том, что не слушали гласа Его.
\vs Bar 1:20 Посему и постигли нас бедствия и клятва,~--- как сегодня,~--- которую определил Господь пред рабом Своим Моисеем в тот день, в который вывел отцов наших из земли Египетской, чтобы дать нам землю, текущую молоком и медом.
\vs Bar 1:21 И не слушали мы гласа Господа Бога нашего во всех словах пророков, которых Он посылал к нам,
\vs Bar 1:22 и ходили каждый по мыслям злого сердца своего, служа иным богам, совершая злые дела пред очами Господа Бога нашего.
\vs Bar 2:1 И исполнил Господь слово Свое, которое Он изрек против нас и против судей наших, судивших Израиля, и против царей наших, и против князей наших, и против всякого Израильтянина и Иудея,
\vs Bar 2:2 что Он наведет на нас великие бедствия, каких не бывало под всем небом, как сделал Он в Иерусалиме, по написанному в законе Моисеевом,
\vs Bar 2:3 что мы будем есть~--- один плоть сына своего, а другой~--- плоть дочери своей.
\vs Bar 2:4 И Он отдал их в подданство всем царствам, которые вокруг нас, на поругание и опустошение всем окрестным народам, между которыми рассеял их Господь.
\vs Bar 2:5 И мы оказались внизу, а не наверху, потому что мы согрешили пред Господом Богом нашим, не слушая гласа Его.
\vs Bar 2:6 У Господа Бога нашего~--- правда, а у нас и отцов наших~--- стыд на лицах, как сегодня.
\vs Bar 2:7 Все те бедствия, какие Господь изрек на нас, постигли нас.
\vs Bar 2:8 Мы не молились пред лицем Господа, чтобы Он отвратил каждого от помышлений злого сердца его.
\vs Bar 2:9 И Господь наблюдал над сими бедствиями, и навел \bibemph{их} Господь на нас, ибо Господь праведен во всем, что заповедал нам.
\vs Bar 2:10 Но мы не слушали гласа Его, чтобы ходить в повелениях Господних, которые Он дал пред лицем нашим.
\vs Bar 2:11 И ныне, Господи, Боже Израилев, Ты, Который вывел народ Твой из земли Египетской рукою крепкою, и знамениями, и чудесами, и силою великою, и мышцею высокою, и сотворил Себе имя, как сегодня:
\vs Bar 2:12 согрешили мы, поступали нечестиво, неправедно против всех уставов Твоих, Господи Боже наш!
\vs Bar 2:13 Да отвратится от нас ярость Твоя, ибо мало осталось нас среди народов, между которыми Ты рассеял нас.
\vs Bar 2:14 Услышь, Господи, молитву нашу и прошение наше, и избавь нас ради Тебя, и дай нам милость пред лицем тех, которые переселили нас,
\vs Bar 2:15 дабы вся земля познала, что Ты~--- Господь Бог наш, так как имя Твое наречено на Израиле и роде его.
\vs Bar 2:16 Призри, Господи, от святаго дома Твоего и воспомяни о нас, и приклони, Господи, ухо Твое, и услышь.
\vs Bar 2:17 Открой очи Твои, посмотри, потому что не мертвые в аде, которых дух взят из внутренностей их, воздадут славу и хвалу Господу;
\vs Bar 2:18 но человек, скорбящий о великости бедствия, который ходит поникши и уныло, и глаза потусклые и душа алчущая воздадут славу и правду Тебе, Господи.
\vs Bar 2:19 Не по правдам отцов наших и царей наших мы повергаем моление сие пред лицем Твоим, Господи Боже наш;
\vs Bar 2:20 ибо на нас Ты послал ярость Твою и гнев Твой, как говорил Ты чрез рабов Твоих, пророков.
\vs Bar 2:21 Так сказал Господь: <<склон\acc{и}те плечи ваши, чтобы работать царю Вавилонскому, и будете жить на земле, которую Я дал отцам вашим;
\vs Bar 2:22 а если не послушаете гласа Господа, чтобы служить царю Вавилонскому,
\vs Bar 2:23 Я сделаю то, что исчезнет в городах Иудейских и окрестностях Иерусалима голос веселья и голос радости, голос жениха и голос невесты, и не будет на всей этой земле следа обитающих>>.
\vs Bar 2:24 Но мы не послушали гласа Твоего, чтобы служить царю Вавилонскому, и Ты исполнил слова Твои, которые говорил чрез рабов Твоих, пророков, что вынесены будут кости царей наших и кости отцов наших из места своего.
\vs Bar 2:25 И вот, они выброшены на дневной зной и ночной холод, а умерли они от злых болезней, от голода, от меча и изгнания.
\vs Bar 2:26 Ты оставил дом, на котором наречено имя Твое, как сегодня, за нечестие дома Израилева и дома Иудина.
\vs Bar 2:27 И Ты, Господи Боже наш, поступил с нами по всему снисхождению Твоему и по всему великому милосердию Твоему,
\vs Bar 2:28 как сказал Ты чрез раба Твоего Моисея в тот день, в который повелел ему написать закон Твой пред сынами Израиля, говоря:
\vs Bar 2:29 <<если вы не послушаете гласа Моего, то это великое и многое множество \bibemph{народа} обратится в малое среди народов, между которыми Я рассею их.
\vs Bar 2:30 Я знаю, что они не послушают Меня, ибо они~--- народ упрямый; но они обратятся к сердцу своему в земле переселения своего,
\vs Bar 2:31 и познают, что Я~--- Господь Бог их. И Я дам им сердце~--- и уразумеют, и уши~--- и услышат.
\vs Bar 2:32 И будут прославлять Меня на земле переселения своего и вспоминать имя Мое,
\vs Bar 2:33 и отвратятся от упорства своего и от злых дел своих; ибо вспомнят путь отцов своих, согрешивших пред Господом.
\vs Bar 2:34 И Я возвращу их в землю, которую с клятвою обещал отцам их, Аврааму и Исааку и Иакову, и они будут владеть ею; и умножу их, и не уменьшатся.
\vs Bar 2:35 И поставлю с ними вечный завет в том, что Я буду их Богом, а они будут Моим народом, и более не изгоню народа Моего Израиля из земли, которую дал им>>.
\vs Bar 3:1 Господи Вседержителю, Боже Израиля! стесненная душа и унылый дух взывает к Тебе:
\vs Bar 3:2 услышь, Господи, и помилуй, ибо Ты Бог милосердый; помилуй, ибо мы согрешили пред Тобою;
\vs Bar 3:3 Ты~--- вечно пребывающий, а мы~--- вечно погибающие.
\vs Bar 3:4 Господи Вседержителю, Боже Израиля! услышь молитву умерших Израиля и сынов их, согрешивших пред Тобою, которые не послушали гласа Господа Бога своего, за то и постигли нас бедствия.
\vs Bar 3:5 Не вспоминай неправд отцов наших, но вспомни руку Твою и имя Твое в сие время,
\vs Bar 3:6 ибо Ты~--- Господь Бог наш, и мы прославим Тебя, Господи.
\vs Bar 3:7 Ты для того вселил страх Твой в сердце наше, чтобы мы призывали имя Твое; и мы будем прославлять Тебя в переселении нашем, ибо мы отринули от сердца нашего всякую неправду отцов наших, согрешивших пред Тобою.
\vs Bar 3:8 Вот, мы теперь в переселении нашем, куда Ты рассеял нас в поношение и в клятву и в возмездие за все неправды отцов наших, которые отступили от Господа Бога нашего.
\rsbpar\vs Bar 3:9 Слушай, Израиль, заповеди жизни, внимайте, чтобы уразуметь мудрость.
\vs Bar 3:10 Что это значит, Израиль, что ты находишься в земле врагов? Состарился ты в чужой земле, осквернился вместе с мертвыми,
\vs Bar 3:11 причислен к находящимся в аде,
\vs Bar 3:12 оставил источник премудрости.
\vs Bar 3:13 Если бы ты ходил путем Божиим, то жил бы в мире вовеки.
\vs Bar 3:14 Познай, где находится мудрость, где сила, где знание, чтобы вместе с тем узнать, где находится долгоденствие и жизнь, где находится свет очей и мир.
\vs Bar 3:15 Кто нашел место ее, и кто взошел в сокровищницы ее?
\vs Bar 3:16 Где князья народов и владевшие зверями земными, забавлявшиеся птицами небесными,
\vs Bar 3:17 и собиравшие серебро и золото, на которые надеются люди, и стяжаниям которых нет конца?
\vs Bar 3:18 \bibemph{Где те}, которые занимались серебряными изделиями, и которых изделиям нет числа?
\vs Bar 3:19 Они исчезли и сошли в ад, и вместо них восстали другие.
\vs Bar 3:20 Позднейшие видели свет и жили на земле, но пути мудрости не познали;
\vs Bar 3:21 не уразумели стезей ее, и не достигли ее сыновья их: они были далеко от пути ее.
\vs Bar 3:22 Не было слышно о ней в Ханаане, и не было видно ее в Фемане.
\vs Bar 3:23 Сыновья Агари искали земного знания, равно и купцы Мерры и Фемана, и баснословы и исследователи знания; но пути премудрости не познали и не заметили стезей ее.
\vs Bar 3:24 О, Израиль! как велик дом Божий, и как пространно место владычества его!
\vs Bar 3:25 Велик он и не имеет конца, высок и неизмерим.
\vs Bar 3:26 Там были изначала славные исполины, весьма великие, искусные в войне.
\vs Bar 3:27 Но не их избрал Бог, и не им открыл пути премудрости;
\vs Bar 3:28 и они погибли оттого, что не имели мудрости, погибли от неразумия своего.
\vs Bar 3:29 Кто взошел на небо, и взял ее, и снес с облаков?
\vs Bar 3:30 Кто перешел моря и нашел ее, и кто принесет ее, лучшую чистого золота?
\vs Bar 3:31 Нет никого, знающего путь ее, ни помышляющего о стезе ее.
\vs Bar 3:32 Но Знающий все знает ее; Он открыл ее Своим разумом, Тот, Который сотворил землю на вечные времена и наполнил ее четвероногими скотами,
\vs Bar 3:33 Который посылает свет, и он идет, призвал его, и он послушался Его с трепетом;
\vs Bar 3:34 и звезды воссияли на стражах своих, и возвеселились.
\vs Bar 3:35 Он призвал их, и они сказали: <<вот мы>>, и воссияли радостью пред Творцом своим.
\vs Bar 3:36 Сей есть Бог наш, и никто другой не сравнится с Ним.
\vs Bar 3:37 Он нашел все пути премудрости и даровал ее рабу Своему Иакову и возлюбленному Своему Израилю.
\vs Bar 3:38 После того Он явился на земле и обращался между людьми.
\vs Bar 4:1 Вот книга заповедей Божиих и закон, пребывающий вовек. Все, держащиеся ее, будут жить, а оставляющие ее умрут.
\vs Bar 4:2 Обратись, Иаков, и возьми ее, ходи при сиянии света ее.
\vs Bar 4:3 Не отдавай другому славы твоей, и полезного для тебя~--- чужому народу.
\vs Bar 4:4 Счастливы мы, Израиль, что мы знаем, что благоугодно Богу.
\vs Bar 4:5 Дерзай, народ мой, памятник Израиля!
\vs Bar 4:6 Вы преданы язычникам не на погибель, но за то, что вы прогневали Бога, вы преданы врагам;
\vs Bar 4:7 ибо раздражили Сотворившего вас, принося жертвы бесам, а не Богу.
\vs Bar 4:8 Вы забыли питающего вас вечного Бога, а также огорчили и воспитавший вас Иерусалим,
\vs Bar 4:9 ибо он видел пришедший на вас гнев от Бога и говорил: <<слушайте, сожители Сиона, Бог навел на меня великую скорбь,
\vs Bar 4:10 ибо я видел пленение сыновей моих и дочерей, которое навел на них Вечный.
\vs Bar 4:11 Я питал их с радостью, а отпустил с плачем и горестью.
\vs Bar 4:12 Никто не радуйся о мне, вдовствующем и оставленном многими; я опустел за грехи детей моих, ибо они уклонились от закона Божия;
\vs Bar 4:13 не познали уставов Его, не ходили путями заповедей Бога, и не вступили на стези учения в правде Его.
\vs Bar 4:14 Придите, сожители Сиона, и вспомните пленение сыновей моих и дочерей, которое навел на них Вечный.
\vs Bar 4:15 Ибо Он навел на них народ издалека, народ наглый и иноязычный, ибо не устыдились старца, и не сжалились над младенцем,
\vs Bar 4:16 и увели у вдовы \bibemph{сыновей} возлюбленных, и лишили одинокую дочерей.
\vs Bar 4:17 Я же чем могу помочь вам?
\vs Bar 4:18 Кто навел на вас сии бедствия, Тот и избавит вас от руки врагов ваших.
\vs Bar 4:19 Идите, дети, идите, ибо я остался пуст.
\vs Bar 4:20 Я снял с себя одежду мира и оделся вретищем моления моего; буду взывать к Вечному во дни мои.
\vs Bar 4:21 Дерзайте, дети, взывайте к Богу, и Он избавит вас от насилия, от руки врагов.
\vs Bar 4:22 Ибо от Вечного я ожидал спасения вашего, и мне пришла от Святаго радость о милости, которая скоро придет к вам от Вечного, Спасителя нашего.
\vs Bar 4:23 Я отпускал вас с печалью и горестью, но Бог возвратит мне вас с радостью и весельем навеки.
\vs Bar 4:24 Ибо, как ныне сожители Сиона видели пленение ваше, так увидят скоро спасение ваше от Бога, которое придет к вам с великою славою и величием Вечного.
\vs Bar 4:25 Дети! потерпите постигший вас от Бога гнев: преследовал тебя враг, но ты скоро увидишь погибель его, и наступишь ему на шею.
\vs Bar 4:26 Воспитанные у меня в неге пошли жесткими путями, схвачены, как стадо, расхищенное врагами.
\vs Bar 4:27 Дерзайте, дети, и взывайте к Богу, ибо о вас вспомнит Тот, Кто навел на вас это.
\vs Bar 4:28 Какова была решимость ваша, чтобы удалиться от Бога, увеличьте ее в десять раз, чтобы обратиться и искать Его,
\vs Bar 4:29 ибо Тот, Который навел на вас сии бедствия, наведет на вас вечное веселье со спасением>>.
\vs Bar 4:30 Дерзай, Иерусалим! Даровавший тебе имя утешит тебя.
\vs Bar 4:31 Несчастны те, которые оскорбляли тебя и радовались твоему падению.
\vs Bar 4:32 Несчастны города, которым служили дети твои, несчастна \bibemph{земля}, принявшая сыновей твоих,
\vs Bar 4:33 ибо, как она радовалась о твоем падении и веселилась о твоем поражении, так будет скорбеть о своем опустошении.
\vs Bar 4:34 Я отниму у нее радость о множестве \bibemph{ее} народа, и хвастовство ее \bibemph{будет} в печаль;
\vs Bar 4:35 ибо придет на нее огонь от Вечного на долгие дни, и весьма долгое время она будет обитаема бесами.
\vs Bar 4:36 Оглянись, Иерусалим, на восток, и посмотри на радость, грядущую к тебе от Бога.
\vs Bar 4:37 Вот, идут сыновья твои, которых ты отпустил, идут собранные от востока до запада словом Святаго, радуясь о славе Божией.
\vs Bar 5:1 Иерусалим! сними с себя одежду плача и озлобления твоего и оденься в благолепие славы от Бога навеки.
\vs Bar 5:2 Облекись в одежду правды от Бога, возложи на голову твою венец славы Вечного,
\vs Bar 5:3 ибо Бог покажет всей поднебесной славу твою.
\vs Bar 5:4 Навек наречется от Бога имя тебе: <<мир правды и слава благочестия>>.
\vs Bar 5:5 Встань, Иерусалим, и стань на высоте, и обратись на восток, и посмотри на детей твоих, собранных от запада солнца до востока словом Святаго, радующихся о Божием воспоминании о них.
\vs Bar 5:6 Они вышли от тебя пешие, будучи ведомы врагами, а приведет к тебе их Бог возносимых со славою, как царских сыновей;
\vs Bar 5:7 ибо Бог определил, чтобы всякая высокая гора и вечные холмы понизились, а долины наполнились, для уравнения земли, чтобы Израиль шел твердо, со славою Божиею,
\vs Bar 5:8 а леса и всякое благовонное дерево осеняли Израиля по повелению Божию.
\vs Bar 5:9 Бог будет с радостью предводить Израиля светом славы Своей, с милостью и правдою Своею.

\bibbookdescr{Eze}{
  inline={\LARGE Книга\\\Huge Пророка Иезекииля},
  toc={Иезекииль},
  bookmark={Иезекииль},
  header={Иезекииль},
  %headerleft={},
  %headerright={},
  abbr={Иез}
}
\vs Eze 1:1 И было в тридцатый год, в четвертый \bibemph{месяц}, в пятый \bibemph{день} месяца, когда я находился среди переселенцев при реке Ховаре, отверзлись небеса, и я видел видения Божии.
\vs Eze 1:2 В пятый \bibemph{день} месяца (это был пятый год от пленения царя Иоакима),
\vs Eze 1:3 было слово Господне к Иезекиилю, сыну Вузия, священнику, в земле Халдейской, при реке Ховаре; и была на нем там рука Господня.
\rsbpar\vs Eze 1:4 И я видел, и вот, бурный ветер шел от севера, великое облако и клубящийся огонь, и сияние вокруг него,
\vs Eze 1:5 а из средины его как бы свет пламени из средины огня; и из средины его видно было подобие четырех животных,~--- и таков был вид их: облик их был, как у человека;
\vs Eze 1:6 и у каждого четыре лица, и у каждого из них четыре крыла;
\vs Eze 1:7 а ноги их~--- ноги прямые, и ступни ног их~--- как ступня ноги у тельца, и сверкали, как блестящая медь, [и крылья их легкие].
\vs Eze 1:8 И руки человеческие были под крыльями их, на четырех сторонах их;
\vs Eze 1:9 и лица у них и крылья у них~--- у всех четырех; крылья их соприкасались одно к другому; во время шествия своего они не оборачивались, а шли каждое по направлению лица своего.
\vs Eze 1:10 Подобие лиц их~--- лице человека и лице льва с правой стороны у всех их четырех; а с левой стороны лице тельца у всех четырех и лице орла у всех четырех.
\vs Eze 1:11 И лица их и крылья их сверху были разделены, но у каждого два крыла соприкасались одно к другому, а два покрывали тела их.
\vs Eze 1:12 И шли они, каждое в ту сторону, которая пред лицем его; куда дух хотел идти, туда и шли; во время шествия своего не оборачивались.
\vs Eze 1:13 И вид этих животных был как вид горящих углей, как вид лампад; \bibemph{огонь} ходил между животными, и сияние от огня и молния исходила из огня.
\vs Eze 1:14 И животные быстро двигались туда и сюда, как сверкает молния.
\vs Eze 1:15 И смотрел я на животных, и вот, на земле подле этих животных по одному колесу перед четырьмя лицами их.
\vs Eze 1:16 Вид колес и устроение их~--- как вид топаза, и подобие у всех четырех одно; и по виду их и по устроению их казалось, будто колесо находилось в колесе.
\vs Eze 1:17 Когда они шли, шли на четыре свои стороны; во время шествия не оборачивались.
\vs Eze 1:18 А ободья их~--- высоки и страшны были они; ободья их у всех четырех вокруг полны были глаз.
\vs Eze 1:19 И когда шли животные, шли и колеса подле \bibemph{них}; а когда животные поднимались от земли, тогда поднимались и колеса.
\vs Eze 1:20 Куда дух хотел идти, туда шли и они; куда бы ни пошел дух, и колеса поднимались наравне с ними, ибо дух животных \bibemph{был} в колесах.
\vs Eze 1:21 Когда шли те, шли и они; и когда те стояли, стояли и они; и когда те поднимались от земли, тогда наравне с ними поднимались и колеса, ибо дух животных \bibemph{был} в колесах.
\vs Eze 1:22 Над головами животных было подобие свода, как вид изумительного кристалла, простертого сверху над головами их.
\vs Eze 1:23 А под сводом простирались крылья их прямо одно к другому, и у каждого были два крыла, которые покрывали их, у каждого два крыла покрывали тела их.
\vs Eze 1:24 И когда они шли, я слышал шум крыльев их, как бы шум многих вод, как бы глас Всемогущего, сильный шум, как бы шум в воинском стане; \bibemph{а} когда они останавливались, опускали крылья свои.
\vs Eze 1:25 И голос был со свода, который над головами их; когда они останавливались, тогда опускали крылья свои.
\vs Eze 1:26 А над сводом, который над головами их, \bibemph{было} подобие престола по виду как бы из камня сапфира; а над подобием престола было как бы подобие человека вверху на нем.
\vs Eze 1:27 И видел я как бы пылающий металл, как бы вид огня внутри него вокруг; от вида чресл его и выше и от вида чресл его и ниже я видел как бы некий огонь, и сияние \bibemph{было} вокруг него.
\vs Eze 1:28 В каком виде бывает радуга на облаках во время дождя, такой вид имело это сияние кругом.
\vs Eze 2:1 Такое было видение подобия славы Господней. Увидев это, я пал на лице свое, и слышал глас Глаголющего, и Он сказал мне: сын человеческий! стань на ноги твои, и Я буду говорить с тобою.
\vs Eze 2:2 И когда Он говорил мне, вошел в меня дух и поставил меня на ноги мои, и я слышал Говорящего мне.
\vs Eze 2:3 И Он сказал мне: сын человеческий! Я посылаю тебя к сынам Израилевым, к людям непокорным, которые возмутились против Меня; они и отцы их изменники предо Мною до сего самого дня.
\vs Eze 2:4 И эти сыны с огрубелым лицем и с жестоким сердцем; к ним Я посылаю тебя, и ты скажешь им: <<так говорит Господь Бог!>>
\vs Eze 2:5 Будут ли они слушать, или не будут, ибо они мятежный дом; но пусть знают, что был пророк среди них.
\vs Eze 2:6 А ты, сын человеческий, не бойся их и не бойся речей их, если они волчцами и тернами будут для тебя, и ты будешь жить у скорпионов; не бойся речей их и не страшись лица их, ибо они мятежный дом;
\vs Eze 2:7 и говори им слова Мои, будут ли они слушать, или не будут, ибо они упрямы.
\vs Eze 2:8 Ты же, сын человеческий, слушай, что Я буду говорить тебе; не будь упрям, как этот мятежный дом; открой уста твои и съешь, что Я дам тебе.
\vs Eze 2:9 И увидел я, и вот, рука простерта ко мне, и вот, в ней книжный свиток.
\vs Eze 2:10 И Он развернул его передо мною, и вот, свиток исписан был внутри и снаружи, и написано на нем: <<плач, и стон, и горе>>.
\vs Eze 3:1 И сказал мне: сын человеческий! съешь, что перед тобою, съешь этот свиток, и иди, говори дому Израилеву.
\vs Eze 3:2 Тогда я открыл уста мои, и Он дал мне съесть этот свиток;
\vs Eze 3:3 и сказал мне: сын человеческий! напитай чрево твое и наполни внутренность твою этим свитком, который Я даю тебе; и я съел, и было в устах моих сладко, как мед.
\vs Eze 3:4 И Он сказал мне: сын человеческий! встань и иди к дому Израилеву, и говори им Моими словами;
\vs Eze 3:5 ибо не к народу с речью невнятною и с непонятным языком ты посылаешься, но к дому Израилеву,
\vs Eze 3:6 не к народам многим с невнятною речью и с непонятным языком, которых слов ты не разумел бы; да если бы Я послал тебя и к ним, то они послушались бы тебя;
\vs Eze 3:7 а дом Израилев не захочет слушать тебя; ибо они не хотят слушать Меня, потому что весь дом Израилев с крепким лбом и жестоким сердцем.
\vs Eze 3:8 Вот, Я сделал и твое лице крепким против лиц их, и твое чело крепким против их лба.
\vs Eze 3:9 Как алмаз, который крепче камня, сделал Я чело твое; не бойся их и не страшись перед лицем их, ибо они мятежный дом.
\vs Eze 3:10 И сказал мне: сын человеческий! все слова Мои, которые буду говорить тебе, прими сердцем твоим и выслушай ушами твоими;
\vs Eze 3:11 встань и пойди к переселенным, к сынам народа твоего, и говори к ним, и скажи им: <<так говорит Господь Бог!>> будут ли они слушать, или не будут.
\vs Eze 3:12 И поднял меня дух; и я слышал позади себя великий громовой голос: <<благословенна слава Господа от места своего!>>
\vs Eze 3:13 и также шум крыльев животных, соприкасающихся одно к другому, и стук колес подле них, и звук сильного грома.
\vs Eze 3:14 И дух поднял меня, и взял меня. И шел я в огорчении, с встревоженным духом; и рука Господня была крепко на мне.
\vs Eze 3:15 И пришел я к переселенным в Тел-Авив, живущим при реке Ховаре, и остановился там, где они жили, и провел среди них семь дней в изумлении.
\rsbpar\vs Eze 3:16 По прошествии же семи дней было ко мне слово Господне:
\vs Eze 3:17 сын человеческий! Я поставил тебя стражем дому Израилеву, и ты будешь слушать слово из уст Моих, и будешь вразумлять их от Меня.
\vs Eze 3:18 Когда Я скажу беззаконнику: <<смертью умрешь!>>, а ты не будешь вразумлять его и говорить, чтобы остеречь беззаконника от беззаконного пути его, чтобы он жив был, то беззаконник тот умрет в беззаконии своем, и Я взыщу кровь его от рук твоих.
\vs Eze 3:19 Но если ты вразумлял беззаконника, а он не обратился от беззакония своего и от беззаконного пути своего, то он умрет в беззаконии своем, а ты спас душу твою.
\vs Eze 3:20 И если праведник отступит от правды своей и поступит беззаконно, когда Я положу пред ним преткновение, и он умрет, то, если ты не вразумлял его, он умрет за грех свой, и не припомнятся ему праведные дела его, какие делал он; и Я взыщу кровь его от рук твоих.
\vs Eze 3:21 Если же ты будешь вразумлять праведника, чтобы праведник не согрешил, и он не согрешит, то и он жив будет, потому что был вразумлен, и ты спас душу твою.
\rsbpar\vs Eze 3:22 И была на мне там рука Господа, и Он сказал мне: встань и выйди в поле, и Я буду говорить там с тобою.
\vs Eze 3:23 И встал я, и вышел в поле; и вот, там стояла слава Господня, как слава, которую видел я при реке Ховаре; и пал я на лице свое.
\vs Eze 3:24 И вошел в меня дух, и поставил меня на ноги мои, и Он говорил со мною, и сказал мне: иди и запрись в доме твоем.
\vs Eze 3:25 И ты, сын человеческий,~--- вот, возложат на тебя узы, и свяжут тебя ими, и не будешь ходить среди них.
\vs Eze 3:26 И язык твой Я прилеплю к гортани твоей, и ты онемеешь, и не будешь обличителем их, ибо они мятежный дом.
\vs Eze 3:27 А когда Я буду говорить с тобою, тогда открою уста твои, и ты будешь говорить им: <<так говорит Господь Бог!>> кто хочет слушать, слушай; а кто не хочет слушать, не слушай: ибо они мятежный дом.
\vs Eze 4:1 И ты, сын человеческий, возьми себе кирпич и положи его перед собою, и начертай на нем город Иерусалим;
\vs Eze 4:2 и устрой осаду против него, и сделай укрепление против него, и насыпь вал вокруг него, и расположи стан против него, и расставь кругом против него стенобитные машины;
\vs Eze 4:3 и возьми себе железную доску, и поставь ее \bibemph{как бы} железную стену между тобою и городом, и обрати на него лице твое, и он будет в осаде, и ты осаждай его. Это будет знамением дому Израилеву.
\vs Eze 4:4 Ты же ложись на левый бок твой и положи на него беззаконие дома Израилева: по числу дней, в которые будешь лежать на нем, ты будешь нести беззаконие их.
\vs Eze 4:5 И Я определил тебе годы беззакония их числом дней: триста девяносто дней ты будешь нести беззаконие дома Израилева.
\vs Eze 4:6 И когда исполнишь это, то вторично ложись уже на правый бок, и сорок дней неси на себе беззаконие дома Иудина, день за год, день за год Я определил тебе.
\vs Eze 4:7 И обрати лице твое и обнаженную правую руку твою на осаду Иерусалима, и пророчествуй против него.
\vs Eze 4:8 Вот, Я возложил на тебя узы, и ты не повернешься с одного бока на другой, доколе не исполнишь дней осады твоей.
\vs Eze 4:9 Возьми себе пшеницы и ячменя, и бобов, и чечевицы, и пшена, и полбы, и всыпь их в один сосуд, и сделай себе из них хлебы, по числу дней, в которые ты будешь лежать на боку твоем; триста девяносто дней ты будешь есть их.
\vs Eze 4:10 И пищу твою, которою будешь питаться, ешь весом по двадцати сиклей в день; от времени до времени ешь это.
\vs Eze 4:11 И воду пей мерою, по шестой части гина пей; от времени до времени пей так.
\vs Eze 4:12 И ешь, как ячменные лепешки, и пеки их при глазах их на человеческом кале.
\vs Eze 4:13 И сказал Господь: так сыны Израилевы будут есть нечистый хлеб свой среди тех народов, к которым Я изгоню их.
\vs Eze 4:14 Тогда сказал я: о, Господи Боже! душа моя никогда не осквернялась, и мертвечины и растерзанного зверем я не ел от юности моей доныне; и никакое нечистое мясо не входило в уста мои.
\vs Eze 4:15 И сказал Он мне: вот, Я дозволяю тебе, вместо человеческого кала, коровий помет, и на нем приготовляй хлеб твой.
\vs Eze 4:16 И сказал мне: сын человеческий! вот, Я сокрушу в Иерусалиме опору хлебную, и будут есть хлеб весом и в печали, и воду будут пить мерою и в унынии,
\vs Eze 4:17 потому что у них будет недостаток в хлебе и воде; и они с ужасом будут смотреть друг на друга, и исчахнут в беззаконии своем.
\vs Eze 5:1 А ты, сын человеческий, возьми себе острый нож, бритву брадобреев возьми себе, и води ею по голове твоей и по бороде твоей, и возьми себе весы, и раздели волосы на части.
\vs Eze 5:2 Третью часть сожги огнем посреди города, когда исполнятся дни осады; третью часть возьми и изруби ножом в окрестностях его; и третью часть развей по ветру; а Я обнажу меч вслед за ними.
\vs Eze 5:3 И возьми из этого небольшое число, и завяжи их у себя в полы.
\vs Eze 5:4 Но и из этого еще возьми, и брось в огонь, и сожги это в огне. Оттуда выйдет огонь на весь дом Израилев.
\rsbpar\vs Eze 5:5 Так говорит Господь Бог: это Иерусалим! Я поставил его среди народов, и вокруг него~--- земли.
\vs Eze 5:6 А он поступил против постановлений Моих нечестивее язычников, и против уставов Моих~--- хуже, нежели земли вокруг него; ибо они отвергли постановления Мои и по уставам Моим не поступают.
\vs Eze 5:7 Посему так говорит Господь Бог: за то, что вы умножили беззакония ваши более, нежели язычники, которые вокруг вас, по уставам Моим не поступаете и постановлений Моих не исполняете, и даже не поступаете и по постановлениям язычников, которые вокруг вас,~---
\vs Eze 5:8 посему так говорит Господь Бог: вот и Я против тебя, Я Сам, и произведу среди тебя суд перед глазами язычников.
\vs Eze 5:9 И сделаю над тобою то, чего Я никогда не делал и чему подобного впредь не буду делать, за все твои мерзости.
\vs Eze 5:10 За то отцы будут есть сыновей среди тебя, и сыновья будут есть отцов своих; и произведу над тобою суд, и весь остаток твой развею по всем ветрам.
\vs Eze 5:11 Посему,~--- живу Я, говорит Господь Бог,~--- за то, что ты осквернил святилище Мое всеми мерзостями твоими и всеми гнусностями твоими, Я умалю тебя, и не пожалеет око Мое, и Я не помилую тебя.
\vs Eze 5:12 Третья часть у тебя умрет от язвы и погибнет от голода среди тебя; третья часть падет от меча в окрестностях твоих; а третью часть развею по всем ветрам, и обнажу меч вслед за ними.
\vs Eze 5:13 И совершится гнев Мой, и утолю ярость Мою над ними, и удовлетворюсь; и узнают, что Я, Господь, говорил в ревности Моей, когда совершится над ними ярость Моя.
\vs Eze 5:14 И сделаю тебя пустынею и поруганием среди народов, которые вокруг тебя, перед глазами всякого мимоходящего.
\vs Eze 5:15 И будешь посмеянием и поруганием, примером и ужасом у народов, которые вокруг тебя, когда Я произведу над тобою суд во гневе и ярости, и в яростных казнях;~--- Я, Господь, изрек сие;~---
\vs Eze 5:16 и когда пошлю на них лютые стрелы голода, которые будут губить, когда пошлю их на погибель вашу, и усилю голод между вами, и сокрушу хлебную опору у вас,
\vs Eze 5:17 и пошлю на вас голод и лютых зверей, и обесчадят тебя; и язва и кровь пройдет по тебе, и меч наведу на тебя; Я, Господь, изрек сие.
\vs Eze 6:1 И было ко мне слово Господне:
\vs Eze 6:2 сын человеческий! обрати лице твое к горам Израилевым и прореки на них,
\vs Eze 6:3 и скажи: горы Израилевы! слушайте слово Господа Бога. Так говорит Господь Бог горам и холмам, долинам и лощинам: вот, Я наведу на вас меч, и разрушу высоты ваши;
\vs Eze 6:4 и жертвенники ваши будут опустошены, столбы ваши в честь солнца будут разбиты, и повергну убитых ваших перед идолами вашими;
\vs Eze 6:5 и положу трупы сынов Израилевых перед идолами их, и рассыплю кости ваши вокруг жертвенников ваших.
\vs Eze 6:6 Во всех местах вашего жительства города будут опустошены и высоты разрушены, для того, чтобы опустошены и разрушены были жертвенники ваши, чтобы сокрушены и уничтожены были идолы ваши, и разбиты солнечные столбы ваши, и изгладились произведения ваши.
\vs Eze 6:7 И будут падать среди вас убитые, и узнаете, что Я Господь.
\vs Eze 6:8 Но Я сберегу остаток, так что будут у вас среди народов уцелевшие от меча, когда вы будете рассеяны по землям.
\vs Eze 6:9 И вспомнят о Мне уцелевшие ваши среди народов, куда будут отведены в плен, когда Я приведу в сокрушение блудное сердце их, отпавшее от Меня, и глаза их, блудившие вслед идолов; и они к самим себе почувствуют отвращение за то зло, какое они делали во всех мерзостях своих;
\vs Eze 6:10 и узнают, что Я Господь; не напрасно говорил Я, что наведу на них такое бедствие.
\vs Eze 6:11 Так говорит Господь Бог: всплесни руками твоими и топни ногою твоею, и скажи: горе за все гнусные злодеяния дома Израилева! падут они от меча, голода и моровой язвы.
\vs Eze 6:12 Кто вдали, тот умрет от моровой язвы; а кто близко, тот падет от меча; а оставшийся и уцелевший умрет от голода; так совершу над ними гнев Мой.
\vs Eze 6:13 И узнаете, что Я Господь, когда пораженные будут \bibemph{лежать} между идолами своими вокруг жертвенников их, на всяком высоком холме, на всех вершинах гор и под всяким зеленеющим деревом, и под всяким ветвистым дубом, на том месте, где они приносили благовонные курения всем идолам своим.
\vs Eze 6:14 И простру на них руку Мою, и сделаю землю пустынею и степью, от пустыни Дивлаф, во всех местах жительства их, и узнают, что Я Господь.
\vs Eze 7:1 И было ко мне слово Господне:
\vs Eze 7:2 и ты, сын человеческий, [скажи]: так говорит Господь Бог; земле Израилевой конец,~--- конец пришел на четыре края земли.
\vs Eze 7:3 Вот конец тебе; и пошлю на тебя гнев Мой, и буду судить тебя по путям твоим, и возложу на тебя все мерзости твои.
\vs Eze 7:4 И не пощадит тебя око Мое, и не помилую, и воздам тебе по путям твоим, и мерзости твои с тобою будут, и узнаете, что Я Господь.
\vs Eze 7:5 Так говорит Господь Бог: беда единственная, вот, идет беда.
\vs Eze 7:6 Конец пришел, пришел конец, встал на тебя; вот дошла,
\vs Eze 7:7 дошла напасть до тебя, житель земли! приходит время, приближается день смятения, а не веселых восклицаний на горах.
\vs Eze 7:8 Вот, скоро изолью на тебя ярость Мою и совершу над тобою гнев Мой, и буду судить тебя по путям твоим, и возложу на тебя все мерзости твои.
\vs Eze 7:9 И не пощадит тебя око Мое, и не помилую. По путям твоим воздам тебе, и мерзости твои с тобою будут; и узнаете, что Я Господь каратель.
\vs Eze 7:10 Вот день! вот пришла, наступила напасть! жезл вырос, гордость разрослась.
\vs Eze 7:11 Восстает сила на жезл нечестия; ничего \bibemph{не останется} от них, и от богатства их, и от шума их, и от пышности их.
\vs Eze 7:12 Пришло время, наступил день; купивший не радуйся, и продавший не плачь; ибо гнев над всем множеством их.
\vs Eze 7:13 Ибо продавший не возвратится к проданному, хотя бы и остались они в живых; ибо пророческое видение о всем множестве их не отменится, и никто своим беззаконием не укрепит своей жизни.
\vs Eze 7:14 Затрубят в трубу, и все готовится, но никто не идет на войну: ибо гнев Мой над всем множеством их.
\vs Eze 7:15 Вне дома меч, а в доме мор и голод. Кто в поле, тот умрет от меча; а кто в городе, того пожрут голод и моровая язва.
\vs Eze 7:16 А уцелевшие из них убегут и будут на горах, как голуби долин; все они будут стонать, каждый за свое беззаконие.
\vs Eze 7:17 У всех руки опустятся, и у всех колени задрожат, \bibemph{как} вода.
\vs Eze 7:18 Тогда они препояшутся вретищем, и обоймет их трепет; и у всех на лицах будет стыд, и у всех на головах плешь.
\vs Eze 7:19 Серебро свое они выбросят на улицы, и золото у них будет в пренебрежении. Серебро их и золото их не сильно будет спасти их в день ярости Господа. Они не насытят ими душ своих и не наполнят утроб своих; ибо оно было поводом к беззаконию их.
\vs Eze 7:20 И в красных нарядах своих они превращали его в гордость, и делали из него изображения гнусных своих истуканов; за то и сделаю его нечистым для них;
\vs Eze 7:21 и отдам его в руки чужим в добычу и беззаконникам земли на расхищение, и они осквернят его.
\vs Eze 7:22 И отвращу от них лице Мое, и осквернят сокровенное Мое; и придут туда грабители, и осквернят его.
\vs Eze 7:23 Сделай цепь, ибо земля эта наполнена кровавыми злодеяниями, и город полон насилий.
\vs Eze 7:24 Я приведу злейших из народов, и завладеют домами их. И положу конец надменности сильных, и будут осквернены святыни их.
\vs Eze 7:25 Идет пагуба; будут искать мира, и не найдут.
\vs Eze 7:26 Беда пойдет за бедою и весть за вестью; и будут просить у пророка видения, и не станет учения у священника и совета у старцев.
\vs Eze 7:27 Царь будет сетовать, и князь облечется в ужас, и у народа земли будут дрожать руки. Поступлю с ними по путям их, и по судам их буду судить их; и узнают, что Я Господь.
\vs Eze 8:1 И было в шестом году, в шестом \bibemph{месяце}, в пятый день месяца, сидел я в доме моем, и старейшины Иудейские сидели перед лицем моим, и низошла на меня там рука Господа Бога.
\vs Eze 8:2 И увидел я: и вот подобие [мужа], как бы огненное, и от чресл его и ниже~--- огонь, и от чресл его и выше~--- как бы сияние, как бы свет пламени.
\vs Eze 8:3 И простер Он как бы руку, и взял меня за волоса головы моей, и поднял меня дух между землею и небом, и принес меня в видениях Божиих в Иерусалим ко входу внутренних ворот, обращенных к северу, где поставлен был идол ревности, возбуждающий ревнование.
\vs Eze 8:4 И вот, там была слава Бога Израилева, подобная той, какую я видел на поле.
\vs Eze 8:5 И сказал мне: сын человеческий! подними глаза твои к северу. И я поднял глаза мои к северу, и вот, с северной стороны у ворот жертвенника~--- тот идол ревности при входе.
\vs Eze 8:6 И сказал Он мне: сын человеческий! видишь ли ты, что они делают? великие мерзости, какие делает дом Израилев здесь, чтобы Я удалился от святилища Моего? но обратись, и ты увидишь еще б\acc{о}льшие мерзости.
\vs Eze 8:7 И привел меня ко входу во двор, и я взглянул, и вот в стене скважина.
\vs Eze 8:8 И сказал мне: сын человеческий! прокопай стену; и я прокопал стену, и вот какая-то дверь.
\vs Eze 8:9 И сказал мне: войди и посмотри на отвратительные мерзости, какие они делают здесь.
\vs Eze 8:10 И вошел я, и вижу, и вот всякие изображения пресмыкающихся и нечистых животных и всякие идолы дома Израилева, написанные по стенам кругом.
\vs Eze 8:11 И семьдесят мужей из старейшин дома Израилева стоят перед ними, и Иезания, сын Сафанов, среди них; и у каждого в руке свое кадило, и густое облако курений возносится кверху.
\vs Eze 8:12 И сказал мне: видишь ли, сын человеческий, что делают старейшины дома Израилева в темноте, каждый в расписанной своей комнате? ибо говорят: <<не видит нас Господь, оставил Господь землю сию>>.
\vs Eze 8:13 И сказал мне: обратись, и увидишь еще б\acc{о}льшие мерзости, какие они делают.
\vs Eze 8:14 И привел меня ко входу в ворота дома Господня, которые к северу, и вот, там сидят женщины, плачущие по Фаммузе,
\vs Eze 8:15 и сказал мне: видишь ли, сын человеческий? обратись, и еще увидишь б\acc{о}льшие мерзости.
\vs Eze 8:16 И ввел меня во внутренний двор дома Господня, и вот у дверей храма Господня, между притвором и жертвенником, около двадцати пяти мужей \bibemph{стоят} спинами своими ко храму Господню, а лицами своими на восток, и кланяются на восток солнцу.
\vs Eze 8:17 И сказал мне: видишь ли, сын человеческий? мало ли дому Иудину, чтобы делать такие мерзости, какие они делают здесь? но они еще землю наполнили нечестием, и сугубо прогневляют Меня; и вот, они ветви подносят к носам своим.
\vs Eze 8:18 За то и Я стану действовать с яростью; не пожалеет око Мое, и не помилую; и хотя бы они взывали в уши Мои громким голосом, не услышу их.
\vs Eze 9:1 И возгласил в уши мои великим гласом, говоря: пусть приблизятся каратели города, каждый со своим губительным орудием в руке своей.
\vs Eze 9:2 И вот, шесть человек идут от верхних ворот, обращенных к северу, и у каждого в руке губительное орудие его, и между ними один, одетый в льняную одежду, у которого при поясе его прибор писца. И пришли и стали подле медного жертвенника.
\vs Eze 9:3 И слава Бога Израилева сошла с Херувима, на котором была, к порогу дома. И призвал Он человека, одетого в льняную одежду, у которого при поясе прибор писца.
\vs Eze 9:4 И сказал ему Господь: пройди посреди города, посреди Иерусалима, и на челах людей скорбящих, воздыхающих о всех мерзостях, совершающихся среди него, сделай знак.
\vs Eze 9:5 А тем сказал в слух мой: идите за ним по городу и поражайте; пусть не жалеет око ваше, и не щадите;
\vs Eze 9:6 старика, юношу и девицу, и младенца и жен бейте до смерти, но не троньте ни одного человека, на котором знак, и начните от святилища Моего. И начали они с тех старейшин, которые были перед домом.
\vs Eze 9:7 И сказал им: оскверните дом, и наполните дворы убитыми, и выйдите. И вышли, и стали убивать в городе.
\vs Eze 9:8 И когда они их убили, а я остался, тогда я пал на лице свое и возопил, и сказал: о, Господи Боже! неужели Ты погубишь весь остаток Израиля, изливая гнев Твой на Иерусалим?
\vs Eze 9:9 И сказал Он мне: нечестие дома Израилева и Иудина велико, весьма велико; и земля сия полна крови, и город исполнен неправды; ибо они говорят: <<оставил Господь землю сию, и не видит Господь>>.
\vs Eze 9:10 За то и Мое око не пощадит, и не помилую; обращу поведение их на их голову.
\vs Eze 9:11 И вот человек, одетый в льняную одежду, у которого при поясе прибор писца, дал ответ и сказал: я сделал, как Ты повелел мне.
\vs Eze 10:1 И видел я, и вот на своде, который над главами Херувимов, как бы камень сапфир, как бы нечто, похожее на престол, видимо было над ними.
\vs Eze 10:2 И говорил Он человеку, одетому в льняную одежду, и сказал: войди между колесами под Херувимов и возьми полные пригоршни горящих угольев между Херувимами, и брось на город; и он вошел в моих глазах.
\vs Eze 10:3 Херувимы же стояли по правую сторону дома, когда вошел тот человек, и облако наполняло внутренний двор.
\vs Eze 10:4 И поднялась слава Господня с Херувима к порогу дома, и дом наполнился облаком, и двор наполнился сиянием славы Господа.
\vs Eze 10:5 И шум от крыльев Херувимов слышен был даже на внешнем дворе, как бы глас Бога Всемогущего, когда Он говорит.
\vs Eze 10:6 И когда Он дал повеление человеку, одетому в льняную одежду, сказав: <<возьми огня между колесами, между Херувимами>>, и когда он вошел и стал у колеса,~---
\vs Eze 10:7 тогда из среды Херувимов один Херувим простер руку свою к огню, который между Херувимами, и взял и дал в пригоршни одетому в льняную одежду. Он взял и вышел.
\vs Eze 10:8 И видно было у Херувимов подобие рук человеческих под крыльями их.
\vs Eze 10:9 И видел я: и вот четыре колеса подле Херувимов, по одному колесу подле каждого Херувима, и колеса по виду как бы из камня топаза.
\vs Eze 10:10 И по виду все четыре сходны, как будто бы колесо находилось в колесе.
\vs Eze 10:11 Когда шли они, то шли на четыре свои стороны; во время шествия своего не оборачивались, но к тому месту, куда обращена была голова, и они туда шли; во время шествия своего не оборачивались.
\vs Eze 10:12 И все тело их, и спина их, и руки их, и крылья их, и колеса кругом были полны очей, все четыре колеса их.
\vs Eze 10:13 К колесам сим, как я слышал, сказано было: <<галгал>>\fns{Вихрь.}.
\vs Eze 10:14 И у каждого \bibemph{из} животных четыре лица: первое лице~--- лице херувимово, второе лице~--- лице человеческое, третье лице львиное и четвертое лице орлиное.
\vs Eze 10:15 Херувимы поднялись. Это были те же животные, которых видел я при реке Ховаре.
\vs Eze 10:16 И когда шли Херувимы, тогда шли подле них и колеса; и когда Херувимы поднимали крылья свои, чтобы подняться от земли, и колеса не отделялись, но были при них.
\vs Eze 10:17 Когда те стояли, стояли и они; когда те поднимались, поднимались и они; ибо в них \bibemph{был} дух животных.
\vs Eze 10:18 И отошла слава Господня от порога дома и стала над Херувимами.
\vs Eze 10:19 И подняли Херувимы крылья свои, и поднялись в глазах моих от земли; когда они уходили, то и колеса подле них; и стали у входа в восточные врата дома Господня, и слава Бога Израилева вверху над ними.
\vs Eze 10:20 Это были те же животные, которых видел я в подножии Бога Израилева при реке Ховаре. И я узнал, что это Херувимы.
\vs Eze 10:21 У каждого по четыре лица, и у каждого по четыре крыла, и под крыльями их подобие рук человеческих.
\vs Eze 10:22 А подобие лиц их то же, какие лица видел я при реке Ховаре,~--- и вид их, и сами они. Каждый шел прямо в ту сторону, которая была перед лицем его.
\vs Eze 11:1 И поднял меня дух, и привел меня к восточным воротам дома Господня, которые обращены к востоку. И вот, у входа в ворота двадцать пять человек; и между ними я видел Иазанию, сына Азурова, и Фалтию, сына Ванеева, князей народа.
\vs Eze 11:2 И Он сказал мне: сын человеческий! вот люди, у которых на уме беззаконие и которые дают худой совет в городе сем,
\vs Eze 11:3 говоря: <<еще не близко; будем строить домы; он\fns{Город.} котел, а мы мясо>>.
\vs Eze 11:4 Посему изреки на них пророчество, пророчествуй, сын человеческий.
\vs Eze 11:5 И нисшел на меня Дух Господень и сказал мне: скажи, так говорит Господь: что говорите вы, дом Израилев, и что на ум вам приходит, это Я знаю.
\vs Eze 11:6 Много убитых ваших вы положили в сем городе и улицы его наполнили трупами.
\vs Eze 11:7 Посему так говорит Господь Бог: убитые ваши, которых вы положили среди него, суть мясо, а он~--- котел; но вас Я выведу из него.
\vs Eze 11:8 Вы боитесь меча, и Я наведу на вас меч, говорит Господь Бог.
\vs Eze 11:9 И выведу вас из него, и отдам вас в руку чужих, и произведу над вами суд.
\vs Eze 11:10 От меча падете; на пределах Израилевых будут судить вас, и узнаете, что Я Господь.
\vs Eze 11:11 Он не будет для вас котлом, и вы не будете мясом в нем; на пределах Израилевых буду судить вас.
\vs Eze 11:12 И узнаете, что Я Господь; ибо по заповедям Моим вы не ходили и уставов Моих не выполняли, а поступали по уставам народов, окружающих вас.
\vs Eze 11:13 И было, когда я пророчествовал, Фалтия, сын Ванеев, умер. И пал я на лице, и возопил громким голосом, и сказал: о, Господи Боже! неужели Ты хочешь до конца истребить остаток Израиля?
\rsbpar\vs Eze 11:14 И было ко мне слово Господне:
\vs Eze 11:15 сын человеческий! твоим братьям, твоим братьям, твоим единокровным и всему дому Израилеву, всем им говорят живущие в Иерусалиме: <<живите вдали от Господа; нам во владение отдана эта земля>>.
\vs Eze 11:16 На это скажи: так говорит Господь Бог: хотя Я и удалил их к народам и хотя рассеял их по землям, но Я буду для них некоторым святилищем в тех землях, куда пошли они.
\vs Eze 11:17 Затем скажи: так говорит Господь Бог: Я соберу вас из народов, и возвращу вас из земель, в которые вы рассеяны; и дам вам землю Израилеву.
\vs Eze 11:18 И придут туда, и извергнут из нее все гнусности ее и все мерзости ее.
\vs Eze 11:19 И дам им сердце единое, и дух новый вложу в них, и возьму из плоти их сердце каменное, и дам им сердце плотяное,
\vs Eze 11:20 чтобы они ходили по заповедям Моим, и соблюдали уставы Мои, и выполняли их; и будут Моим народом, а Я буду их Богом.
\vs Eze 11:21 А чье сердце увлечется вслед гнусностей их и мерзостей их, поведение тех обращу на их голову, говорит Господь Бог.
\vs Eze 11:22 Тогда Херувимы подняли крылья свои, и колеса подле них; и слава Бога Израилева вверху над ними.
\vs Eze 11:23 И поднялась слава Господа из среды города и остановилась над горою, которая на восток от города.
\rsbpar\vs Eze 11:24 И дух поднял меня и перенес меня в Халдею, к переселенцам, в видении, Духом Божиим. И отошло от меня видение, которое я видел.
\vs Eze 11:25 И я пересказал переселенцам все слова Господа, которые Он открыл мне.
\vs Eze 12:1 И было ко мне слово Господне:
\vs Eze 12:2 сын человеческий! ты живешь среди дома мятежного; у них есть глаза, чтобы видеть, а не видят; у них есть уши, чтобы слышать, а не слышат; потому что они~--- мятежный дом.
\vs Eze 12:3 Ты же, сын человеческий, изготовь себе нужное для переселения, и среди дня переселяйся перед глазами их, и переселяйся с места твоего в другое место перед глазами их; может быть, они уразумеют, хотя они~--- дом мятежный;
\vs Eze 12:4 и вещи твои вынеси, как вещи нужные при переселении, днем, перед глазами их, и сам выйди вечером перед глазами их, как выходят для переселения.
\vs Eze 12:5 Перед глазами их проломай себе отверстие в стене, и вынеси через него.
\vs Eze 12:6 Перед глазами их возьми ношу на плечо, впотьмах вынеси ее, лице твое закрой, чтобы не видеть земли; ибо Я поставил тебя знамением дому Израилеву.
\vs Eze 12:7 И сделал я, как повелено было мне; вещи мои, как вещи нужные при переселении, вынес днем, а вечером проломал себе рукою отверстие в стене, впотьмах вынес ношу и поднял на плечо перед глазами их.
\vs Eze 12:8 И было ко мне слово Господне поутру:
\vs Eze 12:9 сын человеческий! не говорил ли тебе дом Израилев, дом мятежный: <<что ты делаешь?>>
\vs Eze 12:10 Скажи им: так говорит Господь Бог: это~--- предвещание для начальствующего в Иерусалиме и для всего дома Израилева, который находится там.
\vs Eze 12:11 Скажи: я знамение для вас; что делаю я, то будет с ними,~--- в переселение, в плен пойдут они.
\vs Eze 12:12 И начальствующий, который среди них, впотьмах поднимет \bibemph{ношу} на плечо и выйдет. Стену проломают, чтобы отправить \bibemph{его} через нее; он закроет лице свое, так что не увидит глазами земли сей.
\vs Eze 12:13 И раскину на него сеть Мою, и будет пойман в тенета Мои, и отведу его в Вавилон, в землю Халдейскую, но он не увидит ее, и там умрет.
\vs Eze 12:14 А всех, которые вокруг него, споборников его и все войско его развею по всем ветрам, и обнажу вслед их меч.
\vs Eze 12:15 И узнают, что Я Господь, когда рассею их по народам и развею их по землям.
\vs Eze 12:16 Но небольшое число их Я сохраню от меча, голода и язвы, чтобы они рассказали у народов, к которым пойдут, о всех своих мерзостях; и узнают, что Я Господь.
\vs Eze 12:17 И было ко мне слово Господне:
\vs Eze 12:18 сын человеческий! хлеб твой ешь с трепетом, и воду твою пей с дрожанием и печалью.
\vs Eze 12:19 И скажи народу земли: так говорит Господь Бог о жителях Иерусалима, о земле Израилевой: они хлеб свой будут есть с печалью и воду свою будут пить в унынии, потому что земля его будет лишена всего изобилия своего за неправды всех живущих на ней.
\vs Eze 12:20 И будут разорены населенные города, и земля сделается пустою, и узнаете, что Я Господь.
\vs Eze 12:21 И было ко мне слово Господне:
\vs Eze 12:22 сын человеческий! что за поговорка у вас, в земле Израилевой: <<много дней пройдет, и всякое пророческое видение исчезнет>>?
\vs Eze 12:23 Посему скажи им: так говорит Господь Бог: уничтожу эту поговорку, и не будут уже употреблять такой поговорки у Израиля; но скажи им: близки дни и исполнение всякого видения пророческого.
\vs Eze 12:24 Ибо уже не останется втуне никакое видение пророческое, и ни одно предвещание не будет ложным в доме Израилевом.
\vs Eze 12:25 Ибо Я Господь, Я говорю; и слово, которое Я говорю, исполнится, и не будет отложено; в ваши дни, мятежный дом, Я изрек слово, и исполню его, говорит Господь Бог.
\vs Eze 12:26 И было ко мне слово Господне:
\vs Eze 12:27 сын человеческий! вот, дом Израилев говорит: <<пророческое видение, которое видел он, \bibemph{сбудется} после многих дней, и он пророчествует об отдаленных временах>>.
\vs Eze 12:28 Посему скажи им: так говорит Господь Бог: ни одно из слов Моих уже не будет отсрочено, но слово, которое Я скажу, сбудется, говорит Господь Бог.
\vs Eze 13:1 И было ко мне слово Господне:
\vs Eze 13:2 сын человеческий! изреки пророчество на пророков Израилевых пророчествующих, и скажи пророкам от собственного сердца: слушайте слово Господне!
\vs Eze 13:3 Так говорит Господь Бог: горе безумным пророкам, которые водятся своим духом и ничего не видели!
\vs Eze 13:4 Пророки твои, Израиль, как лисицы в развалинах.
\vs Eze 13:5 В проломы вы не вх\acc{о}дите и не ограждаете стеною дома Израилева, чтобы твердо стоять в сражении в день Господа.
\vs Eze 13:6 Они видят пустое и предвещают ложь, говоря: <<Господь сказал>>; а Господь не посылал их; и обнадеживают, что слово сбудется.
\vs Eze 13:7 Не пустое ли видение видели вы? и не лживое ли предвещание изрекаете, говоря: <<Господь сказал>>, а Я не говорил?
\vs Eze 13:8 Посему так говорит Господь Бог: так как вы говорите пустое и видите в видениях ложь, за то вот Я~--- на вас, говорит Господь Бог.
\vs Eze 13:9 И будет рука Моя против этих пророков, видящих пустое и предвещающих ложь; в совете народа Моего они не будут, и в список дома Израилева не впишутся, и в землю Израилеву не войдут; и узнаете, что Я Господь Бог.
\vs Eze 13:10 За то, что они вводят народ Мой в заблуждение, говоря: <<мир>>, тогда как нет мира; и когда он строит стену, они обмазывают ее грязью,
\vs Eze 13:11 скажи обмазывающим стену грязью, что она упадет. Пойдет проливной дождь, и вы, каменные градины, падете, и бурный ветер разорвет ее.
\vs Eze 13:12 И вот, падет стена; тогда не скажут ли вам: <<где та обмазка, которою вы обмазывали?>>
\vs Eze 13:13 Посему так говорит Господь Бог: Я пущу бурный ветер во гневе Моем, и пойдет проливной дождь в ярости Моей, и камни града в негодовании Моем, для истребления.
\vs Eze 13:14 И разрушу стену, которую вы обмазывали грязью, и повергну ее на землю, и откроется основание ее, и падет, и вы вместе с нею погибнете; и узнаете, что Я Господь.
\vs Eze 13:15 И истощу ярость Мою на стене и на обмазывающих ее грязью, и скажу вам: нет стены, и нет обмазывавших ее,
\vs Eze 13:16 пророков Израилевых, которые пророчествовали Иерусалиму и возвещали ему видения мира, тогда как нет мира, говорит Господь Бог.
\vs Eze 13:17 Ты же, сын человеческий, обрати лице твое к дщерям народа твоего, пророчествующим от собственного своего сердца, и изреки на них пророчество,
\vs Eze 13:18 и скажи: так говорит Господь Бог: горе сшивающим чародейные мешочки под мышки и делающим покрывала для головы всякого роста, чтобы уловлять души! Неужели, уловляя души народа Моего, вы спасете ваши души?
\vs Eze 13:19 И бесславите Меня пред народом Моим за горсти ячменя и за куски хлеба, умерщвляя души, которые не должны умереть, и оставляя жизнь душам, которые не должны жить, обманывая народ, который слушает ложь.
\vs Eze 13:20 Посему так говорит Господь Бог: вот, Я~--- на ваши чародейные мешочки, которыми вы там уловляете души, чтобы они прилетали, и вырву их из-под мышц ваших, и пущу на свободу души, которые вы уловляете, чтобы прилетали к вам.
\vs Eze 13:21 И раздеру покрывала ваши, и избавлю народ Мой от рук ваших, и не будут уже в ваших руках добычею, и узнаете, что Я Господь.
\vs Eze 13:22 За то, что вы ложью опечаливаете сердце праведника, которое Я не хотел опечаливать, и поддерживаете руки беззаконника, чтобы он не обратился от порочного пути своего и не сохранил жизни своей,~---
\vs Eze 13:23 за это уже не будете иметь пустых видений и впредь не будете предугадывать; и Я избавлю народ Мой от рук ваших, и узнаете, что Я Господь.
\vs Eze 14:1 И пришли ко мне несколько человек из старейшин Израилевых и сели перед лицем моим.
\vs Eze 14:2 И было ко мне слово Господне:
\vs Eze 14:3 сын человеческий! Сии люди допустили идолов своих в сердце свое и поставили соблазн нечестия своего перед лицем своим: могу ли Я отвечать им?
\vs Eze 14:4 Посему говори с ними и скажи им: так говорит Господь Бог: если кто из дома Израилева допустит идолов своих в сердце свое и поставит соблазн нечестия своего перед лицем своим, и придет к пророку,~--- то Я, Господь, могу ли, при множестве идолов его, дать ему ответ?
\vs Eze 14:5 Пусть дом Израилев поймет в сердце своем, что все они через своих идолов сделались чужими для Меня.
\vs Eze 14:6 Посему скажи дому Израилеву: так говорит Господь Бог: обратитесь и отвратитесь от идолов ваших, и от всех мерзостей ваших отвратите лице ваше.
\vs Eze 14:7 Ибо если кто из дома Израилева и из пришельцев, которые живут у Израиля, отложится от Меня и допустит идолов своих в сердце свое, и поставит соблазн нечестия своего перед лицем своим, и придет к пророку вопросить Меня через него,~--- то Я, Господь, дам ли ему ответ от Себя?
\vs Eze 14:8 Я обращу лице Мое против того человека и сокрушу его в знамение и притчу, и истреблю его из народа Моего, и узнаете, что Я Господь.
\vs Eze 14:9 А если пророк допустит обольстить себя и скажет слово так, как бы Я, Господь, научил этого пророка, то Я простру на него руку Мою и истреблю его из народа Моего, Израиля.
\vs Eze 14:10 И понесут вину беззакония своего: какова вина вопрошающего, такова будет вина и пророка,
\vs Eze 14:11 чтобы впредь дом Израилев не уклонялся от Меня и чтобы более не оскверняли себя всякими беззакониями своими, но чтобы были Моим народом, и Я был их Богом, говорит Господь Бог.
\vs Eze 14:12 И было ко мне слово Господне:
\vs Eze 14:13 сын человеческий! если бы какая земля согрешила предо Мною, вероломно отступив от Меня, и Я простер на нее руку Мою, и истребил в ней хлебную опору, и послал на нее голод, и стал губить на ней людей и скот;
\vs Eze 14:14 и если бы нашлись в ней сии три мужа: Ной, Даниил и Иов,~--- то они праведностью своею спасли бы только свои души, говорит Господь Бог.
\vs Eze 14:15 Или, если бы Я послал на эту землю лютых зверей, которые осиротили бы ее, и она по причине зверей сделалась пустою и непроходимою:
\vs Eze 14:16 то сии три мужа среди нее,~--- живу Я, говорит Господь Бог,~--- не спасли бы ни сыновей, ни дочерей, а они, только они спаслись бы, земля же сделалась бы пустынею.
\vs Eze 14:17 Или, если бы Я навел на ту землю меч и сказал: <<меч, пройди по земле!>>, и стал истреблять на ней людей и скот,
\vs Eze 14:18 то сии три мужа среди нее,~--- живу Я, говорит Господь Бог,~--- не спасли бы ни сыновей, ни дочерей, а они только спаслись бы.
\vs Eze 14:19 Или, если бы Я послал на ту землю моровую язву и излил на нее ярость Мою в кровопролитии, чтобы истребить на ней людей и скот:
\vs Eze 14:20 то Ной, Даниил и Иов среди нее,~--- живу Я, говорит Господь Бог,~--- не спасли бы ни сыновей, ни дочерей; праведностью своею они спасли бы только свои души.
\vs Eze 14:21 Ибо так говорит Господь Бог: если и четыре тяжкие казни Мои: меч, и голод, и лютых зверей, и моровую язву пошлю на Иерусалим, чтобы истребить в нем людей и скот,
\vs Eze 14:22 и тогда останется в нем остаток, сыновья и дочери, которые будут выведены оттуда; вот, они выйдут к вам, и вы увидите поведение их и дела их, и утешитесь о том бедствии, которое Я навел на Иерусалим, о всем, что Я навел на него.
\vs Eze 14:23 Они утешат вас, когда вы увидите поведение их и дела их; и узнаете, что Я не напрасно сделал все то, что сделал в нем, говорит Господь Бог.
\vs Eze 15:1 И было ко мне слово Господне:
\vs Eze 15:2 сын человеческий! какое преимущество имеет дерево виноградной лозы перед всяким другим деревом и ветви виноградной лозы~--- между деревами в лесу?
\vs Eze 15:3 Берут ли от него кусок на какое-либо изделие? Берут ли от него хотя на гвоздь, чтобы вешать на нем какую-либо вещь?
\vs Eze 15:4 Вот, оно отдается огню на съедение; оба конца его огонь поел, и обгорела середина его: годится ли оно на какое-нибудь изделие?
\vs Eze 15:5 И тогда, как оно было цело, не годилось ни на какое изделие; тем паче, когда огонь поел его, и оно обгорело, годится ли оно на какое-нибудь изделие?
\vs Eze 15:6 Посему так говорит Господь Бог: как дерево виноградной лозы между деревами лесными Я отдал огню на съедение, так отдам ему и жителей Иерусалима.
\vs Eze 15:7 И обращу лице Мое против них; из одного огня выйдут, и другой огонь пожрет их,~--- и узнаете, что Я Господь, когда обращу против них лице Мое.
\vs Eze 15:8 И сделаю эту землю пустынею за то, что они вероломно поступали, говорит Господь Бог.
\vs Eze 16:1 И было ко мне слово Господне:
\vs Eze 16:2 сын человеческий! выскажи Иерусалиму мерзости его
\vs Eze 16:3 и скажи: так говорит Господь Бог \bibemph{дщери} Иерусалима: твой корень и твоя родина в земле Ханаанской; отец твой Аморрей, и мать твоя Хеттеянка;
\vs Eze 16:4 при рождении твоем, в день, когда ты родилась, пупа твоего не отрезали, и водою ты не была омыта для очищения, и солью не была осолена, и пеленами не повита.
\vs Eze 16:5 Ничей глаз не сжалился над тобою, чтобы из милости к тебе сделать тебе что-нибудь из этого; но ты выброшена была на поле, по презрению к жизни твоей, в день рождения твоего.
\vs Eze 16:6 И проходил Я мимо тебя, и увидел тебя, брошенную на попрание в кровях твоих, и сказал тебе: <<в кровях твоих живи!>> Так, Я сказал тебе: <<в кровях твоих живи!>>
\vs Eze 16:7 Умножил тебя как полевые растения; ты выросла и стала большая, и достигла превосходной красоты: поднялись груди, и волоса у тебя выросли; но ты была нага и непокрыта.
\vs Eze 16:8 И проходил Я мимо тебя, и увидел тебя, и вот, это было время твое, время любви; и простер Я воскрилия \bibemph{риз} Моих на тебя, и покрыл наготу твою; и поклялся тебе и вступил в союз с тобою, говорит Господь Бог,~--- и ты стала Моею.
\vs Eze 16:9 Омыл Я тебя водою и смыл с тебя кровь твою и помазал тебя елеем.
\vs Eze 16:10 И надел на тебя узорчатое платье, и обул тебя в сафьянные сандалии, и опоясал тебя виссоном, и покрыл тебя шелковым покрывалом.
\vs Eze 16:11 И нарядил тебя в наряды, и положил на руки твои запястья и на шею твою ожерелье.
\vs Eze 16:12 И дал тебе кольцо на твой нос и серьги к ушам твоим и на голову твою прекрасный венец.
\vs Eze 16:13 Так украшалась ты золотом и серебром, и одежда твоя \bibemph{была} виссон и шелк и узорчатые ткани; питалась ты хлебом из лучшей пшеничной муки, медом и елеем, и была чрезвычайно красива, и достигла царственного величия.
\vs Eze 16:14 И пронеслась по народам слава твоя ради красоты твоей, потому что она была вполне совершенна при том великолепном наряде, который Я возложил на тебя, говорит Господь Бог.
\vs Eze 16:15 Но ты понадеялась на красоту твою, и, пользуясь славою твоею, стала блудить и расточала блудодейство твое на всякого мимоходящего, отдаваясь ему.
\vs Eze 16:16 И взяла из одежд твоих, и сделала себе разноцветные высоты, и блудодействовала на них, как никогда не случится и не будет.
\vs Eze 16:17 И взяла нарядные твои вещи из Моего золота и из Моего серебра, которые Я дал тебе, и сделала себе мужские изображения, и блудодействовала с ними.
\vs Eze 16:18 И взяла узорчатые платья твои, и одела их ими, и ставила перед ними елей Мой и фимиам Мой,
\vs Eze 16:19 и хлеб Мой, который Я давал тебе, пшеничную муку, и елей, и мед, которыми Я питал тебя, ты поставляла перед ними в приятное благовоние; и это было, говорит Господь Бог.
\vs Eze 16:20 И взяла сыновей твоих и дочерей твоих, которых ты родила Мне, и приносила в жертву на снедение им. Мало ли тебе было блудодействовать?
\vs Eze 16:21 Но ты и сыновей Моих заколала и отдавала им, проводя их \bibemph{через огонь}.
\vs Eze 16:22 И при всех твоих мерзостях и блудодеяниях твоих ты не вспомнила о днях юности твоей, когда ты была нага и непокрыта и брошена в крови твоей на попрание.
\vs Eze 16:23 И после всех злодеяний твоих,~--- горе, горе тебе! говорит Господь Бог,~---
\vs Eze 16:24 ты построила себе блудилища и наделала себе возвышений на всякой площади;
\vs Eze 16:25 при начале всякой дороги устроила себе возвышения, позорила красоту твою и раскидывала ноги твои для всякого мимоходящего, и умножила блудодеяния твои.
\vs Eze 16:26 Блудила с сыновьями Египта, соседями твоими, людьми великорослыми, и умножала блудодеяния твои, прогневляя Меня.
\vs Eze 16:27 И вот, Я простер на тебя руку Мою, и уменьшил назначенное тебе, и отдал тебя на произвол ненавидящим тебя дочерям Филистимским, которые устыдились срамного поведения твоего.
\vs Eze 16:28 И блудила ты с сынами Ассура и не насытилась; блудила с ними, но тем не удовольствовалась;
\vs Eze 16:29 и умножила блудодеяния твои в земле Ханаанской до Халдеи, но и тем не удовольствовалась.
\vs Eze 16:30 Как истомлено должно быть сердце твое, говорит Господь Бог, когда ты все это делала, как необузданная блудница!
\vs Eze 16:31 Когда ты строила себе блудилища при начале всякой дороги и делала себе возвышения на всякой площади, ты была не как блудница, потому что отвергала подарки,
\vs Eze 16:32 но как прелюбодейная жена, принимающая вместо своего мужа чужих.
\vs Eze 16:33 Всем блудницам дают подарки, а ты сама давала подарки всем любовникам твоим и подкупала их, чтобы они со всех сторон приходили к тебе блудить с тобою.
\vs Eze 16:34 У тебя в блудодеяниях твоих было противное тому, что бывает с женщинами: не за тобою гонялись, но ты давала подарки, а тебе не давали подарков; и потому ты поступала в противность другим.
\vs Eze 16:35 Посему выслушай, блудница, слово Господне!
\vs Eze 16:36 Так говорит Господь Бог: за то, что ты так сыпала деньги твои, и в блудодеяниях твоих раскрываема была нагота твоя перед любовниками твоими и перед всеми мерзкими идолами твоими, и за кровь сыновей твоих, которых ты отдавала им,~---
\vs Eze 16:37 за то вот, Я соберу всех любовников твоих, которыми ты услаждалась и которых ты любила, со всеми теми, которых ненавидела, и соберу их отовсюду против тебя, и раскрою перед ними наготу твою, и увидят весь срам твой.
\vs Eze 16:38 Я буду судить тебя судом прелюбодейц и проливающих кровь,~--- и предам тебя кровавой ярости и ревности;
\vs Eze 16:39 предам тебя в руки их и они разорят блудилища твои, и раскидают возвышения твои, и сорвут с тебя одежды твои, и возьмут наряды твои, и оставят тебя нагою и непокрытою.
\vs Eze 16:40 И созовут на тебя собрание, и побьют тебя камнями, и разрубят тебя мечами своими.
\vs Eze 16:41 Сожгут домы твои огнем и совершат над тобою суд перед глазами многих жен; и положу конец блуду твоему, и не будешь уже давать подарков.
\vs Eze 16:42 И утолю над тобою гнев Мой, и отступит от тебя негодование Мое, и успокоюсь, и уже не буду гневаться.
\vs Eze 16:43 За то, что ты не вспомнила о днях юности твоей и всем этим раздражала Меня, вот, и Я поведение твое обращу на \bibemph{твою} голову, говорит Господь Бог, чтобы ты не предавалась более разврату после всех твоих мерзостей.
\vs Eze 16:44 Вот, всякий, кто говорит притчами, может сказать о тебе: <<какова мать, такова и дочь>>.
\vs Eze 16:45 Ты дочь в мать твою, которая бросила мужа своего и детей своих,~--- и ты сестра в сестер твоих, которые бросили мужей своих и детей своих. Мать ваша Хеттеянка, и отец ваш Аморрей.
\vs Eze 16:46 Б\acc{о}льшая же сестра твоя~--- Самария, с дочерями своими живущая влево от тебя; а меньшая сестра твоя, живущая от тебя вправо, есть Содома с дочерями ее.
\vs Eze 16:47 Но ты и не их путями ходила и не по их мерзостям поступала; этого было мало: ты поступала развратнее их на всех путях твоих.
\vs Eze 16:48 Живу Я, говорит Господь Бог; Содома, сестра твоя, не делала того сама и ее дочери, что делала ты и дочери твои.
\vs Eze 16:49 Вот в чем было беззаконие Содомы, сестры твоей и дочерей ее: в гордости, пресыщении и праздности, и она руки бедного и нищего не поддерживала.
\vs Eze 16:50 И возгордились они, и делали мерзости пред лицем Моим, и, увидев это, Я отверг их.
\vs Eze 16:51 И Самария половины грехов твоих не нагрешила; ты превзошла их мерзостями твоими, и через твои мерзости, какие делала ты, сестры твои оказались правее тебя.
\vs Eze 16:52 Неси же посрамление твое и ты, которая осуждала сестер твоих; по грехам твоим, какими ты опозорила себя более их, они правее тебя. Красней же от стыда и ты, и неси посрамление твое, так оправдав сестер твоих.
\vs Eze 16:53 Но Я возвращу плен их, плен Содомы и дочерей ее, плен Самарии и дочерей ее, и между ними плен плененных твоих,
\vs Eze 16:54 дабы ты несла посрамление твое и стыдилась всего того, что делала, служа для них утешением.
\vs Eze 16:55 И сестры твои, Содома и дочери ее, возвратятся в прежнее состояние свое; и Самария и дочери ее возвратятся в прежнее состояние свое, и ты и дочери твои возвратитесь в прежнее состояние ваше.
\vs Eze 16:56 О сестре твоей Содоме и помина не было в устах твоих во дни гордыни твоей,
\vs Eze 16:57 доколе еще не открыто было нечестие твое, как во время посрамления от дочерей Сирии и всех окружавших ее, от дочерей Филистимы, смотревших на тебя с презрением со всех сторон.
\vs Eze 16:58 За разврат твой и за мерзости твои терпишь ты, говорит Господь.
\vs Eze 16:59 Ибо так говорит Господь Бог: Я поступлю с тобою, как поступила ты, презрев клятву нарушением союза.
\vs Eze 16:60 Но Я вспомню союз Мой с тобою во дни юности твоей, и восстановлю с тобою вечный союз.
\vs Eze 16:61 И ты вспомнишь о путях твоих, и будет стыдно тебе, когда станешь принимать к себе сестер твоих, б\acc{о}льших тебя, как и меньших тебя, и когда Я буду давать тебе их в дочерей, но не от твоего союза.
\vs Eze 16:62 Я восстановлю союз Мой с тобою, и узнаешь, что Я Господь,
\vs Eze 16:63 для того, чтобы ты помнила и стыдилась, и чтобы вперед нельзя было тебе и рта открыть от стыда, когда Я прощу тебе все, что ты делала, говорит Господь Бог.
\vs Eze 17:1 И было ко мне слово Господне:
\vs Eze 17:2 сын человеческий! предложи загадку и скажи притчу к дому Израилеву.
\vs Eze 17:3 Скажи: так говорит Господь Бог: большой орел с большими крыльями, с длинными перьями, пушистый, пестрый, прилетел на Ливан и снял с кедра верхушку,
\vs Eze 17:4 сорвал верхний из молодых побегов его и принес его в землю Ханаанскую, в городе торговцев положил его;
\vs Eze 17:5 и взял от семени этой земли, и посадил на земле семени, поместил у больших вод, как сажают иву.
\vs Eze 17:6 И оно выросло, и сделалось виноградною лозою, широкою, низкою ростом, которой ветви клонились к ней, и корни ее были под нею же, и стало виноградною лозою, и дало отрасли, и пустило ветви.
\vs Eze 17:7 И еще был орел с большими крыльями и пушистый; и вот, эта виноградная лоза потянулась к нему своими корнями и простерла к нему ветви свои, чтобы он поливал ее из борозд рассадника своего.
\vs Eze 17:8 Она была посажена на хорошем поле, у больших вод, так что могла пускать ветви и приносить плод, сделаться лозою великолепною.
\vs Eze 17:9 Скажи: так говорит Господь Бог: будет ли ей успех? Не вырвут ли корней ее, и не оборвут ли плодов ее, так что она засохнет? все молодые ветви, отросшие от нее, засохнут. И не с большою силою и не со многими людьми сорвут ее с корней ее.
\vs Eze 17:10 И вот, хотя она посажена, но будет ли успех? Не иссохнет ли она, как скоро коснется ее восточный ветер? иссохнет на грядах, где выросла.
\vs Eze 17:11 И было ко мне слово Господне:
\vs Eze 17:12 скажи мятежному дому: разве не знаете, что это значит?~--- Скажи: вот, пришел царь Вавилонский в Иерусалим, и взял царя его и князей его, и привел их к себе в Вавилон.
\vs Eze 17:13 И взял \bibemph{другого} из царского рода, и заключил с ним союз, и обязал его клятвою, и взял сильных земли той с собою,
\vs Eze 17:14 чтобы царство было покорное, чтобы не могло подняться, чтобы сохраняем был союз и стоял твердо.
\vs Eze 17:15 Но тот отложился от него, послав послов своих в Египет, чтобы дали ему коней и много людей. Будет ли ему успех? Уцелеет ли тот, кто это делает? Он нарушил союз и уцелеет ли?
\vs Eze 17:16 Живу Я, говорит Господь Бог: в местопребывании царя, который поставил его царем, и которому данную клятву он презрел, и нарушил союз свой с ним, он умрет у него в Вавилоне.
\vs Eze 17:17 С великою силою и с многочисленным народом фараон ничего не сделает для него в этой войне, когда будет насыпан вал и построены будут осадные башни на погибель многих душ.
\vs Eze 17:18 Он презрел клятву, чтобы нарушить союз, и вот, дал руку свою и сделал все это; он не уцелеет.
\vs Eze 17:19 Посему так говорит Господь Бог: живу Я! клятву Мою, которую он презрел, и союз Мой, который он нарушил, Я обращу на его голову.
\vs Eze 17:20 И закину на него сеть Мою, и пойман будет в тенета Мои; и приведу его в Вавилон, и там буду судиться с ним за вероломство его против Меня.
\vs Eze 17:21 А все беглецы его из всех полков его падут от меча, а оставшиеся развеяны будут по всем ветрам; и узнаете, что Я, Господь, сказал это.
\vs Eze 17:22 Так говорит Господь Бог: и возьму Я с вершины высокого кедра, и посажу; с верхних побегов его оторву нежную отрасль и посажу на высокой и величественной горе.
\vs Eze 17:23 На высокой горе Израилевой посажу его, и пустит ветви, и принесет плод, и сделается величественным кедром, и будут обитать под ним всякие птицы, всякие пернатые будут обитать в тени ветвей его.
\vs Eze 17:24 И узнают все дерева полевые, что Я, Господь, высокое дерево понижаю, низкое дерево повышаю, зеленеющее дерево иссушаю, а сухое дерево делаю цветущим: Я, Господь, сказал, и сделаю.
\vs Eze 18:1 И было ко мне слово Господне:
\vs Eze 18:2 зачем вы употребляете в земле Израилевой эту пословицу, говоря: <<отцы ели кислый виноград, а у детей на зубах оскомина>>?
\vs Eze 18:3 Живу Я! говорит Господь Бог,~--- не будут вперед говорить пословицу эту в Израиле.
\vs Eze 18:4 Ибо вот, все души~--- Мои: как душа отца, так и душа сына~--- Мои: душа согрешающая, та умрет.
\vs Eze 18:5 Если кто праведен и творит суд и правду,
\vs Eze 18:6 на горах жертвенного не ест и к идолам дома Израилева не обращает глаз своих, жены ближнего своего не оскверняет и к своей жене во время очищения нечистот ее не приближается,
\vs Eze 18:7 никого не притесняет, должнику возвращает залог его, хищения не производит, хлеб свой дает голодному и нагого покрывает одеждою,
\vs Eze 18:8 в рост не отдает и лихвы не берет, от неправды удерживает руку свою, суд человеку с человеком производит правильный,
\vs Eze 18:9 поступает по заповедям Моим и соблюдает постановления Мои искренно: то он праведник, он непременно будет жив, говорит Господь Бог.
\vs Eze 18:10 Но если у него родился сын разбойник, проливающий кровь, и делает что-нибудь из всего того,
\vs Eze 18:11 чего он сам не делал совсем, и на горах ест жертвенное, и жену ближнего своего оскверняет,
\vs Eze 18:12 бедного и нищего притесняет, насильно отнимает, залога не возвращает, и к идолам обращает глаза свои, делает мерзость,
\vs Eze 18:13 в рост дает, и берет лихву; то будет ли он жив? \bibemph{Нет}, он не будет жив. Кто делает все такие мерзости, тот непременно умрет, кровь его будет на нем.
\vs Eze 18:14 Но если у кого родился сын, который, видя все грехи отца своего, какие он делает, видит и не делает подобного им:
\vs Eze 18:15 на горах жертвенного не ест, к идолам дома Израилева не обращает глаз своих, жены ближнего своего не оскверняет,
\vs Eze 18:16 и человека не притесняет, залога не берет, и насильно не отнимает, хлеб свой дает голодному, и нагого покрывает одеждою,
\vs Eze 18:17 от \bibemph{обиды} бедному удерживает руку свою, роста и лихвы не берет, исполняет Мои повеления и поступает по заповедям Моим,~--- то сей не умрет за беззаконие отца своего; он будет жив.
\vs Eze 18:18 А отец его, так как он жестоко притеснял, грабил брата и недоброе делал среди народа своего, вот, он умрет за свое беззаконие.
\vs Eze 18:19 Вы говорите: <<почему же сын не несет вины отца своего?>> Потому что сын поступает законно и праведно, все уставы Мои соблюдает и исполняет их; он будет жив.
\vs Eze 18:20 Душа согрешающая, она умрет; сын не понесет вины отца, и отец не понесет вины сына, правда праведного при нем и остается, и беззаконие беззаконного при нем и остается.
\vs Eze 18:21 И беззаконник, если обратится от всех грехов своих, какие делал, и будет соблюдать все уставы Мои и поступать законно и праведно, жив будет, не умрет.
\vs Eze 18:22 Все преступления его, какие делал он, не припомнятся ему: в правде своей, которую будет делать, он жив будет.
\vs Eze 18:23 Разве Я хочу смерти беззаконника? говорит Господь Бог. Не того ли, чтобы он обратился от путей своих и был жив?
\vs Eze 18:24 И праведник, если отступит от правды своей и будет поступать неправедно, будет делать все те мерзости, какие делает беззаконник, будет ли он жив? все добрые дела его, какие он делал, не припомнятся; за беззаконие свое, какое делает, и за грехи свои, в каких грешен, он умрет.
\vs Eze 18:25 Но вы говорите: <<неправ путь Господа!>> Послушайте, дом Израилев! Мой ли путь неправ? не ваши ли пути неправы?
\vs Eze 18:26 Если праведник отступает от правды своей и делает беззаконие и за то умирает, то он умирает за беззаконие свое, которое сделал.
\vs Eze 18:27 И беззаконник, если обращается от беззакония своего, какое делал, и творит суд и правду,~--- к жизни возвратит душу свою.
\vs Eze 18:28 Ибо он увидел и обратился от всех преступлений своих, какие делал; он будет жив, не умрет.
\vs Eze 18:29 А дом Израилев говорит: <<неправ путь Господа!>> Мои ли пути неправы, дом Израилев? не ваши ли пути неправы?
\vs Eze 18:30 Посему Я буду судить вас, дом Израилев, каждого по путям его, говорит Господь Бог; покайтесь и обратитесь от всех преступлений ваших, чтобы нечестие не было вам преткновением.
\vs Eze 18:31 Отвергните от себя все грехи ваши, которыми согрешали вы, и сотворите себе новое сердце и новый дух; и зачем вам умирать, дом Израилев?
\vs Eze 18:32 Ибо Я не хочу смерти умирающего, говорит Господь Бог; но обратитесь, и живите!
\vs Eze 19:1 А ты подними плач о князьях Израиля
\vs Eze 19:2 и скажи: что за львица мать твоя? расположилась среди львов, между молодыми львами растила львенков своих.
\vs Eze 19:3 И вскормила одного из львенков своих; он сделался молодым львом и научился ловить добычу, ел людей.
\vs Eze 19:4 И услышали о нем народы; он пойман был в яму их, и в цепях отвели его в землю Египетскую.
\vs Eze 19:5 И когда, пождав, увидела она, что надежда ее пропала, тогда взяла другого из львенков своих и сделала его молодым львом.
\vs Eze 19:6 И, сделавшись молодым львом, он стал ходить между львами и научился ловить добычу, ел людей
\vs Eze 19:7 и осквернял вдов их и города их опустошал; и опустела земля и все селения ее от рыкания его.
\vs Eze 19:8 Тогда восстали на него народы из окрестных областей и раскинули на него сеть свою; он пойман был в яму их.
\vs Eze 19:9 И посадили его в клетку на цепи и отвели его к царю Вавилонскому; отвели его в крепость, чтобы не слышен уже был голос его на горах Израилевых.
\vs Eze 19:10 Твоя мать была, как виноградная лоза, посаженная у воды; плодовита и ветвиста была она от обилия воды.
\vs Eze 19:11 И были у нее ветви крепкие для скипетров властителей, и высоко поднялся ствол ее между густыми ветвями; и выдавалась она высотою своею со множеством ветвей своих.
\vs Eze 19:12 Но во гневе вырвана, брошена на землю, и восточный ветер иссушил плод ее; отторжены и иссохли крепкие ветви ее, огонь пожрал их.
\vs Eze 19:13 А теперь она пересажена в пустыню, в землю сухую и жаждущую.
\vs Eze 19:14 И вышел огонь из ствола ветвей ее, пожрал плоды ее и не осталось на ней ветвей крепких для скипетра властителя. Это плачевная песнь, и останется для плача.
\vs Eze 20:1 В седьмом году, в пятом \bibemph{месяце}, в десятый день месяца, пришли мужи из старейшин Израилевых вопросить Господа и сели перед лицем моим.
\vs Eze 20:2 И было ко мне слово Господне:
\vs Eze 20:3 сын человеческий! говори со старейшинами Израилевыми и скажи им: так говорит Господь Бог: вы пришли вопросить Меня? Живу Я, не дам вам ответа, говорит Господь Бог.
\vs Eze 20:4 Хочешь ли судиться с ними, хочешь ли судиться, сын человеческий? выскажи им мерзости отцов их
\vs Eze 20:5 и скажи им: так говорит Господь Бог: в тот день, когда Я избрал Израиля и, подняв руку Мою, \bibemph{поклялся} племени дома Иаковлева, и открыл Себя им в земле Египетской, и, подняв руку, сказал им: <<Я Господь Бог ваш!>>~---
\vs Eze 20:6 в тот день, подняв руку Мою, Я поклялся им вывести их из земли Египетской в землю, которую Я усмотрел для них, текущую молоком и медом, красу всех земель,
\vs Eze 20:7 и сказал им: отвергните каждый мерзости от очей ваших и не оскверняйте себя идолами Египетскими: Я Господь Бог ваш.
\vs Eze 20:8 Но они возмутились против Меня и не хотели слушать Меня; никто не отверг мерзостей от очей своих и не оставил идолов Египетских. И Я сказал: изолью на них гнев Мой, истощу на них ярость Мою среди земли Египетской.
\vs Eze 20:9 Но Я поступил ради имени Моего, чтобы оно не хулилось перед народами, среди которых находились они и перед глазами которых Я открыл Себя им, чтобы вывести их из земли Египетской.
\vs Eze 20:10 И Я вывел их из земли Египетской и привел их в пустыню,
\vs Eze 20:11 и дал им заповеди Мои, и объявил им Мои постановления, исполняя которые человек жив был бы через них;
\vs Eze 20:12 дал им также субботы Мои, чтобы они были знамением между Мною и ими, чтобы знали, что Я Господь, освящающий их.
\vs Eze 20:13 Но дом Израилев возмутился против Меня в пустыне: по заповедям Моим не поступали и отвергли постановления Мои, исполняя которые человек жив был бы через них, и субботы Мои нарушали, и Я сказал: изолью на них ярость Мою в пустыне, чтобы истребить их.
\vs Eze 20:14 Но Я поступил ради имени Моего, чтобы оно не хулилось перед народами, в глазах которых Я вывел их.
\vs Eze 20:15 Даже Я, подняв руку Мою против них в пустыне, \bibemph{поклялся}, что не введу их в землю, которую Я назначил,~--- текущую молоком и медом, красу всех земель,~---
\vs Eze 20:16 за то, что они отвергли постановления Мои, и не поступали по заповедям Моим, и нарушали субботы Мои; ибо сердце их стремилось к идолам их.
\vs Eze 20:17 Но око Мое пожалело погубить их; и Я не истребил их в пустыне.
\vs Eze 20:18 И говорил Я сыновьям их в пустыне: не ходите по правилам отцов ваших, и не соблюдайте установлений их, и не оскверняйте себя идолами их.
\vs Eze 20:19 Я Господь Бог ваш: по Моим заповедям поступайте, и Мои уставы соблюдайте, и исполняйте их.
\vs Eze 20:20 И святите субботы Мои, чтобы они были знамением между Мною и вами, дабы вы знали, что Я Господь Бог ваш.
\vs Eze 20:21 Но и сыновья возмутились против Меня: по заповедям Моим не поступали и уставов Моих не соблюдали, не исполняли того, что исполняя, человек был бы жив, нарушали субботы Мои,~--- и Я сказал: изолью на них гнев Мой, истощу над ними ярость Мою в пустыне;
\vs Eze 20:22 но Я отклонил руку Мою и поступил ради имени Моего, чтобы оно не хулилось перед народами, перед глазами которых Я вывел их.
\vs Eze 20:23 Также, подняв руку Мою в пустыне, Я \bibemph{поклялся} рассеять их по народам и развеять их по землям
\vs Eze 20:24 за то, что они постановлений Моих не исполняли и заповеди Мои отвергли, и нарушали субботы мои, и глаза их обращались к идолам отцов их.
\vs Eze 20:25 И попустил им учреждения недобрые и постановления, от которых они не могли быть живы,
\vs Eze 20:26 и попустил им оскверниться жертвоприношениями их, когда они стали проводить через огонь всякий первый плод утробы, чтобы разорить их, дабы знали, что Я Господь.
\vs Eze 20:27 Посему говори дому Израилеву, сын человеческий, и скажи им: так говорит Господь Бог: вот чем еще хулили Меня отцы ваши, вероломно поступая против Меня:
\vs Eze 20:28 Я привел их в землю, которую клятвенно обещал дать им, подняв руку Мою,~--- а они, высмотрев себе всякий высокий холм и всякое ветвистое дерево, стали заколать там жертвы свои, и ставили там оскорбительные для Меня приношения свои и благовонные курения свои, и возливали там возлияния свои.
\vs Eze 20:29 И Я говорил им: что это за высота, куда ходите вы? поэтому именем Бама называется она и до сего дня.
\vs Eze 20:30 Посему скажи дому Израилеву: так говорит Господь Бог: не оскверняете ли вы себя по примеру отцов ваших и не блудодействуете ли вслед мерзостей их?
\vs Eze 20:31 Принося дары ваши и проводя сыновей ваших через огонь, вы оскверняете себя всеми идолами вашими до сего дня, и хотите вопросить Меня, дом Израилев? живу Я, говорит Господь Бог, не дам вам ответа.
\vs Eze 20:32 И что приходит вам на ум, совсем не сбудется. Вы говорите: <<будем, как язычники, как племена иноземные, служить дереву и камню>>.
\vs Eze 20:33 Живу Я, говорит Господь Бог: рукою крепкою и мышцею простертою и излиянием ярости буду господствовать над вами.
\vs Eze 20:34 И выведу вас из народов и из стран, по которым вы рассеяны, и соберу вас рукою крепкою и мышцею простертою и излиянием ярости.
\vs Eze 20:35 И приведу вас в пустыню народов, и там буду судиться с вами лицом к лицу.
\vs Eze 20:36 Как Я судился с отцами вашими в пустыне земли Египетской, так буду судиться с вами, говорит Господь Бог.
\vs Eze 20:37 И проведу вас под жезлом и введу вас в узы завета.
\vs Eze 20:38 И выделю из вас мятежников и непокорных Мне. Из земли пребывания их выведу их, но в землю Израилеву они не войдут, и узнаете, что Я Господь.
\vs Eze 20:39 А вы, дом Израилев,~--- так говорит Господь Бог,~--- идите каждый к своим идолам и служите им, если Меня не слушаете, но не оскверняйте более святаго имени Моего дарами вашими и идолами вашими,
\vs Eze 20:40 потому что на Моей святой горе, на горе высокой Израилевой,~--- говорит Господь Бог,~--- там будет служить Мне весь дом Израилев,~--- весь, сколько ни есть его на земле; там Я с благоволением приму их, и там потребую приношений ваших и начатков ваших со всеми святынями вашими.
\vs Eze 20:41 Приму вас, как благовонное курение, когда выведу вас из народов и соберу вас из стран, по которым вы рассеяны, и буду святиться в вас перед глазами народов.
\vs Eze 20:42 И узнаете, что Я Господь, когда введу вас в землю Израилеву,~--- в землю, которую Я \bibemph{клялся} дать отцам вашим, подняв руку Мою.
\vs Eze 20:43 И вспомните там о путях ваших и обо всех делах ваших, какими вы оскверняли себя, и возгнушаетесь самими собою за все злодеяния ваши, какие вы делали.
\vs Eze 20:44 И узнаете, что Я Господь, когда буду поступать с вами ради имени Моего, не по злым вашим путям и вашим делам развратным, дом Израилев,~--- говорит Господь Бог.
\rsbpar\vs Eze 20:45 И было ко мне слово Господне:
\vs Eze 20:46 сын человеческий! обрати лице твое на путь к полудню, и произнеси слово на полдень, и изреки пророчество на лес южного поля.
\vs Eze 20:47 И скажи южному лесу: слушай слово Господа; так говорит Господь Бог: вот, Я зажгу в тебе огонь, и он пожрет в тебе всякое дерево зеленеющее и всякое дерево сухое; не погаснет пылающий пламень, и все будет опалено им от юга до севера.
\vs Eze 20:48 И увидит всякая плоть, что Я, Господь, зажег его, и он не погаснет.
\vs Eze 20:49 И сказал я: о, Господи Боже! они говорят обо мне: <<не говорит ли он притчи?>>
\vs Eze 21:1 И было ко мне слово Господне:
\vs Eze 21:2 сын человеческий! обрати лице твое к Иерусалиму и произнеси слово на святилища, и изреки пророчество на землю Израилеву,
\vs Eze 21:3 и скажи земле Израилевой: так говорит Господь Бог: вот, Я~--- на тебя, и извлеку меч Мой из ножен его и истреблю у тебя праведного и нечестивого.
\vs Eze 21:4 А для того, чтобы истребить у тебя праведного и нечестивого, меч Мой из ножен своих пойдет на всякую плоть от юга до севера.
\vs Eze 21:5 И узнает всякая плоть, что Я, Господь, извлек меч Мой из ножен его, и он уже не возвратится.
\vs Eze 21:6 Ты же, сын человеческий, стенай, сокрушая бедра твои, и в горести стенай перед глазами их.
\vs Eze 21:7 И когда скажут тебе: <<отчего ты стенаешь?>>, скажи: <<от слуха, что идет>>,~--- и растает всякое сердце, и все руки опустятся, и всякий дух изнеможет, и все колени задрожат, как вода. Вот, это придет и сбудется, говорит Господь Бог.
\rsbpar\vs Eze 21:8 И было ко мне слово Господне:
\vs Eze 21:9 сын человеческий! изреки пророчество и скажи: так говорит Господь Бог: скажи: меч, меч наострен и вычищен;
\vs Eze 21:10 наострен для того, чтобы больше заколать; вычищен, чтобы сверкал, как молния. Радоваться ли нам, что жезл сына Моего презирает всякое дерево?
\vs Eze 21:11 Я дал его вычистить, чтобы взять в руку; уже наострен этот меч и вычищен, чтобы отдать его в руку убийцы.
\vs Eze 21:12 Стенай и рыдай, сын человеческий, ибо он~--- на народ Мой, на всех князей Израиля; они отданы будут под меч с народом Моим; посему ударяй себя по бедрам.
\vs Eze 21:13 Ибо он уже испытан. И что, если он презирает и жезл? сей не устоит, говорит Господь Бог.
\vs Eze 21:14 Ты же, сын человеческий, пророчествуй и ударяй рукою об руку; и удвоится меч и утроится, меч на поражаемых, меч на поражение великого, проникающий во внутренность жилищ их.
\vs Eze 21:15 Чтобы растаяли сердца и чтобы павших было более, Я у всех ворот их поставлю грозный меч, увы! сверкающий, как молния, наостренный для заклания.
\vs Eze 21:16 Соберись и иди направо или иди налево, куда бы ни обратилось лице твое.
\vs Eze 21:17 И Я буду рукоплескать и утолю гнев Мой; Я, Господь, сказал.
\vs Eze 21:18 И было ко мне слово Господне:
\vs Eze 21:19 и ты, сын человеческий, представь себе две дороги, по которым должно идти мечу царя Вавилонского,~--- обе они должны выходить из одной земли; и начертай руку, начертай при начале дорог в города.
\vs Eze 21:20 Представь дорогу, по которой меч шел бы в Равву сынов Аммоновых и в Иудею, в укрепленный Иерусалим;
\vs Eze 21:21 потому что царь Вавилонский остановился на распутье, при начале двух дорог, для гаданья: трясет стрелы, вопрошает терафимов, рассматривает печень.
\vs Eze 21:22 В правой руке у него гаданье: <<в Иерусалим>>, где должно поставить тараны, открыть для побоища уста, возвысить голос для военного крика, подвести тараны к воротам, насыпать вал, построить осадные башни.
\vs Eze 21:23 Это гаданье показалось в глазах их лживым; но так как они клялись клятвою, то он, вспомнив о таком их вероломстве, положил взять его.
\vs Eze 21:24 Посему так говорит Господь Бог: так как вы сами приводите на память беззаконие ваше, делая явными преступления ваши, выставляя на вид грехи ваши во всех делах ваших, и сами приводите это на память, то вы будете взяты руками.
\vs Eze 21:25 И ты, недостойный, преступный вождь Израиля, которого день наступил ныне, когда нечестию его положен будет конец!
\vs Eze 21:26 так говорит Господь Бог: сними с себя диадему и сложи венец; этого уже не будет; униженное возвысится и высокое унизится.
\vs Eze 21:27 Низложу, низложу, низложу и его не будет, доколе не придет Тот, Кому \bibemph{принадлежит} он, и Я дам Ему.
\vs Eze 21:28 И ты, сын человеческий, изреки пророчество и скажи: так говорит Господь Бог о сынах Аммона и о поношении их; и скажи: меч, меч обнажен для заклания, вычищен для истребления, чтобы сверкал, как молния,
\vs Eze 21:29 чтобы, тогда как представляют тебе пустые видения и ложно гадают тебе, и тебя приложил к обезглавленным нечестивцам, которых день наступил, когда нечестию их положен будет конец.
\vs Eze 21:30 Возвратить ли его в ножны его?~--- на месте, где ты сотворен, на земле происхождения твоего буду судить тебя:
\vs Eze 21:31 и изолью на тебя негодование Мое, дохну на тебя огнем ярости Моей и отдам тебя в руки людей свирепых, опытных в убийстве.
\vs Eze 21:32 Ты будешь пищею огню, кровь твоя останется на земле; не будут и вспоминать о тебе; ибо Я, Господь, сказал это.
\vs Eze 22:1 И было ко мне слово Господне:
\vs Eze 22:2 и ты, сын человеческий, хочешь ли судить, судить город кровей? выскажи ему все мерзости его.
\vs Eze 22:3 И скажи: так говорит Господь Бог: о, город, проливающий кровь среди себя, чтобы наступило время твое, и делающий у себя идолов, чтобы осквернять себя!
\vs Eze 22:4 Кровью, которую ты пролил, ты сделал себя виновным, и идолами, каких ты наделал, ты осквернил себя, и приблизил дни твои и достиг годины твоей. За это отдам тебя на посмеяние народам, на поругание всем землям.
\vs Eze 22:5 Близкие и далекие от тебя будут ругаться над тобою, осквернившим имя твое, прославившимся буйством.
\vs Eze 22:6 Вот, начальствующие у Израиля, каждый по мере сил своих, были у тебя, чтобы проливать кровь.
\vs Eze 22:7 У тебя отца и мать злословят, пришельцу делают обиду среди тебя, сироту и вдову притесняют у тебя.
\vs Eze 22:8 Святынь Моих ты не уважаешь и субботы Мои нарушаешь.
\vs Eze 22:9 Клеветники находятся в тебе, чтобы проливать кровь, и на горах едят у тебя \bibemph{идоложертвенное}, среди тебя производят гнусность.
\vs Eze 22:10 Наготу отца открывают у тебя, жену во время очищения нечистот ее насилуют у тебя.
\vs Eze 22:11 Иной делает мерзость с женою ближнего своего, иной оскверняет сноху свою, иной насилует сестру свою, дочь отца своего.
\vs Eze 22:12 Взятки берут у тебя, чтобы проливать кровь; ты берешь рост и лихву и насилием вымогаешь корысть у ближнего твоего, а Меня забыл, говорит Господь Бог.
\vs Eze 22:13 И вот, Я всплеснул руками Моими о корыстолюбии твоем, какое обнаруживается у тебя, и о кровопролитии, которое совершается среди тебя.
\vs Eze 22:14 Устоит ли сердце твое, будут ли тверды руки твои в те дни, в которые буду действовать против тебя? Я, Господь, сказал и сделаю.
\vs Eze 22:15 И рассею тебя по народам, и развею тебя по землям, и положу конец мерзостям твоим среди тебя.
\vs Eze 22:16 И сделаешь сам себя презренным перед глазами народов, и узнаешь, что Я Господь.
\vs Eze 22:17 И было ко мне слово Господне:
\vs Eze 22:18 сын человеческий! дом Израилев сделался у Меня изгарью; все они~--- олово, медь и железо и свинец в горниле, сделались, как изгарь серебра.
\vs Eze 22:19 Посему так говорит Господь Бог: так как все вы сделались изгарью, за то вот, Я соберу вас в Иерусалим.
\vs Eze 22:20 Как в горнило кладут вместе серебро, и медь, и железо, и свинец, и олово, чтобы раздуть на них огонь и расплавить; так Я во гневе Моем и в ярости Моей соберу, и положу, и расплавлю вас.
\vs Eze 22:21 Соберу вас и дохну на вас огнем негодования Моего, и расплавитесь среди него.
\vs Eze 22:22 Как серебро расплавляется в горниле, так расплавитесь и вы среди него, и узнаете, что Я, Господь, излил ярость Мою на вас.
\vs Eze 22:23 И было ко мне слово Господне:
\vs Eze 22:24 сын человеческий! скажи ему: ты~--- земля неочищенная, не орошаемая дождем в день гнева!
\vs Eze 22:25 Заговор пророков ее среди нее~--- как лев рыкающий, терзающий добычу; съедают души, обирают имущество и драгоценности, и умножают число вдов.
\vs Eze 22:26 Священники ее нарушают закон Мой и оскверняют святыни Мои, не отделяют святаго от несвятаго и не указывают различия между чистым и нечистым, и от суббот Моих они закрыли глаза свои, и Я уничижен у них.
\vs Eze 22:27 Князья у нее как волки, похищающие добычу; проливают кровь, губят души, чтобы приобрести корысть.
\vs Eze 22:28 А пророки ее всё замазывают грязью, видят пустое и предсказывают им ложное, говоря: <<так говорит Господь Бог>>, тогда как не говорил Господь.
\vs Eze 22:29 А в народе угнетают друг друга, грабят и притесняют бедного и нищего, и пришельца угнетают несправедливо.
\vs Eze 22:30 Искал Я у них человека, который поставил бы стену и стал бы предо Мною в проломе за сию землю, чтобы Я не погубил ее, но не нашел.
\vs Eze 22:31 Итак изолью на них негодование Мое, огнем ярости Моей истреблю их, поведение их обращу им на голову, говорит Господь Бог.
\vs Eze 23:1 И было ко мне слово Господне:
\vs Eze 23:2 сын человеческий! были две женщины, дочери одной матери,
\vs Eze 23:3 и блудили они в Египте, блудили в своей молодости; там измяты груди их, и там растлили девственные сосцы их.
\vs Eze 23:4 Имена им: большой~--- Огола, а сестре ее~--- Оголива. И были они Моими, и рождали сыновей и дочерей; и именовались~--- Огола Самариею, а Оголива Иерусалимом.
\vs Eze 23:5 И стала Огола блудить от Меня и пристрастилась к своим любовникам, к Ассириянам, к соседям своим,
\vs Eze 23:6 к одевавшимся в ткани яхонтового цвета, к областеначальникам и градоправителям, ко всем красивым юношам, всадникам, ездящим на конях;
\vs Eze 23:7 и расточала блудодеяния свои со всеми отборными из сынов Ассура, и оскверняла себя всеми идолами тех, к кому ни пристращалась;
\vs Eze 23:8 не переставала блудить и с Египтянами, потому что они с нею спали в молодости ее и растлевали девственные сосцы ее, и изливали на нее похоть свою.
\vs Eze 23:9 За то Я и отдал ее в руки любовников ее, в руки сынов Ассура, к которым она пристрастилась.
\vs Eze 23:10 Они открыли наготу ее, взяли сыновей ее и дочерей ее, а ее убили мечом. И она сделалась позором между женщинами, когда совершили над нею казнь.
\vs Eze 23:11 Сестра ее, Оголива, видела это, и еще развращеннее была в любви своей, и блужение ее превзошло блужение сестры ее.
\vs Eze 23:12 Она пристрастилась к сынам Ассуровым, к областеначальникам и градоправителям, соседям ее, пышно одетым, к всадникам, ездящим на конях, ко всем отборным юношам.
\vs Eze 23:13 И Я видел, что она осквернила себя, \bibemph{и что} у обеих их одна дорога.
\vs Eze 23:14 Но эта еще умножила блудодеяния свои, потому что, увидев вырезанных на стене мужчин, красками нарисованные изображения Халдеев,
\vs Eze 23:15 опоясанных по чреслам своим поясом, с роскошными на голове их повязками, имеющих вид военачальников, похожих на сынов Вавилона, которых родина земля Халдейская,
\vs Eze 23:16 она влюбилась в них по одному взгляду очей своих и послала к ним в Халдею послов.
\vs Eze 23:17 И пришли к ней сыны Вавилона на любовное ложе, и осквернили ее блудодейством своим, и она осквернила себя ими; и отвратилась от них душа ее.
\vs Eze 23:18 Когда же она явно предалась блудодеяниям своим и открыла наготу свою, тогда и от нее отвратилась душа Моя, как отвратилась душа Моя от сестры ее.
\vs Eze 23:19 И она умножала блудодеяния свои, вспоминая дни молодости своей, когда блудила в земле Египетской;
\vs Eze 23:20 и пристрастилась к любовникам своим, у которых плоть~--- плоть ослиная, и похоть, как у жеребцов.
\vs Eze 23:21 Так ты вспомнила распутство молодости твоей, когда Египтяне жали сосцы твои из-за девственных грудей твоих.
\vs Eze 23:22 Посему, Оголива, так говорит Господь Бог: вот, Я возбужу против тебя любовников твоих, от которых отвратилась душа твоя, и приведу их против тебя со всех сторон:
\vs Eze 23:23 сынов Вавилона и всех Халдеев, из Пехода, из Шоа и Коа, и с ними всех сынов Ассура, красивых юношей, областеначальников и градоправителей, сановных и именитых, всех искусных наездников.
\vs Eze 23:24 И придут на тебя с оружием, с конями и колесницами и с множеством народа, и обступят тебя кругом в латах, со щитами и в шлемах, и отдам им тебя на суд, и будут судить тебя своим судом.
\vs Eze 23:25 И обращу ревность Мою против тебя, и поступят с тобою яростно: отрежут у тебя нос и уши, а остальное твое от меча падет; возьмут сыновей твоих и дочерей твоих, а остальное твое огнем будет пожрано;
\vs Eze 23:26 и снимут с тебя одежды твои, возьмут наряды твои.
\vs Eze 23:27 И положу конец распутству твоему и блужению твоему, принесенному из земли Египетской, и не будешь обращать к ним глаз твоих, и о Египте уже не вспомнишь.
\vs Eze 23:28 Ибо так говорит Господь Бог: вот, Я предаю тебя в руки тех, которых ты возненавидела, в руки тех, от которых отвратилась душа твоя.
\vs Eze 23:29 И поступят с тобою жестоко, и возьмут у тебя все, нажитое трудами, и оставят тебя нагою и непокрытою, и открыта будет срамная нагота твоя, и распутство твое, и блудодейство твое.
\vs Eze 23:30 Это будет сделано с тобою за блудодейство твое с народами, которых идолами ты осквернила себя.
\vs Eze 23:31 Ты ходила дорогою сестры твоей; за то и дам в руку тебе чашу ее.
\vs Eze 23:32 Так говорит Господь Бог: ты будешь пить чашу сестры твоей, глубокую и широкую, и подвергнешься посмеянию и позору, по огромной вместительности ее.
\vs Eze 23:33 Опьянения и горести будешь исполнена: чаша ужаса и опустошения~--- чаша сестры твоей, Самарии!
\vs Eze 23:34 И выпьешь ее, и осушишь, и черепки ее оближешь, и груди твои истерзаешь: ибо Я сказал это, говорит Господь Бог.
\vs Eze 23:35 Посему так говорит Господь Бог: так как ты забыла Меня и отвратилась от Меня, то и терпи за беззаконие твое и за блудодейство твое.
\vs Eze 23:36 И сказал мне Господь: сын человеческий! хочешь ли судить Оголу и Оголиву? выскажи им мерзости их;
\vs Eze 23:37 ибо они прелюбодействовали, и кровь на руках их, и с идолами своими прелюбодействовали, и сыновей своих, которых родили Мне, через огонь проводили в пищу им.
\vs Eze 23:38 Еще вот что они делали Мне: оскверняли святилище Мое в тот же день, и нарушали субботы Мои;
\vs Eze 23:39 потому что, когда они заколали детей своих для идолов своих, в тот же день приходили в святилище Мое, чтобы осквернять его: вот как поступали они в доме Моем!
\vs Eze 23:40 Кроме сего посылали за людьми, приходившими издалека; к ним отправляли послов, и вот, они приходили, и ты для них умывалась, сурьмила глаза твои и украшалась нарядами,
\vs Eze 23:41 и садились на великолепное ложе, перед которым приготовляем был стол и на котором предлагала ты благовонные курения Мои и елей Мой.
\vs Eze 23:42 И раздавался голос народа, ликовавшего у нее, и к людям из толпы народной вводимы были пьяницы из пустыни; и они возлагали на руки их запястья и на головы их красивые венки.
\vs Eze 23:43 Тогда сказал Я об одряхлевшей в прелюбодействе: теперь кончатся блудодеяния ее вместе с нею.
\vs Eze 23:44 Но приходили к ней, как приходят к жене блуднице, так приходили к Оголе и Оголиве, к распутным женам.
\vs Eze 23:45 Но мужи праведные будут судить их; они будут судить их судом прелюбодейц и судом проливающих кровь, потому что они прелюбодейки, и у них кровь на руках.
\vs Eze 23:46 Ибо так сказал Господь Бог: созвать на них собрание и предать их озлоблению и грабежу.
\vs Eze 23:47 И собрание побьет их камнями, и изрубит их мечами своими, и убьет сыновей их и дочерей их, и домы их сожжет огнем.
\vs Eze 23:48 Так положу конец распутству на сей земле, и все женщины примут урок, и не будут делать срамных дел подобно вам;
\vs Eze 23:49 и возложат на вас ваше распутство, и понесете наказание за грехи с идолами вашими, и узнаете, что Я Господь Бог.
\vs Eze 24:1 И было ко мне слово Господне в девятом году, в десятом месяце, в десятый день месяца:
\vs Eze 24:2 сын человеческий! запиши себе имя этого дня, этого самого дня: в этот самый день царь Вавилонский подступит к Иерусалиму.
\vs Eze 24:3 И произнеси на мятежный дом притчу, и скажи им: так говорит Господь Бог: поставь котел, поставь и налей в него воды;
\vs Eze 24:4 сложи в него куски мяса, все лучшие куски, бедра и плеча, и наполни отборными костями;
\vs Eze 24:5 отборных овец возьми, и \bibemph{разожги} под ним кости, и кипяти до того, чтобы и кости разварились в нем.
\vs Eze 24:6 Посему так говорит Господь Бог: горе городу кровей! горе котлу, в котором есть накипь и с которого накипь его не сходит! кусок за куском его выбрасывайте из него, не выбирая по жребию.
\vs Eze 24:7 Ибо кровь его среди него; он оставил ее на голой скале; не на землю проливал ее, где она могла бы покрыться пылью.
\vs Eze 24:8 Чтобы возбудить гнев для совершения мщения, Я оставил кровь его на голой скале, чтобы она не скрылась.
\vs Eze 24:9 Посему так говорит Господь Бог: горе городу кровей! и Я разложу большой костер.
\vs Eze 24:10 Прибавь дров, разведи огонь, вывари мясо; пусть все сгустится, и кости перегорят.
\vs Eze 24:11 И когда котел будет пуст, поставь его на уголья, чтобы он разгорелся, и чтобы медь его раскалилась, и расплавилась в нем нечистота его, и вся накипь его исчезла.
\vs Eze 24:12 Труд будет тяжелый; но большая накипь его не сойдет с него; и в огне \bibemph{останется} на нем накипь его.
\vs Eze 24:13 В нечистоте твоей такая мерзость, что, сколько Я ни чищу тебя, ты все нечист; от нечистоты твоей ты и впредь не очистишься, доколе ярости Моей Я не утолю над тобою.
\vs Eze 24:14 Я Господь, Я говорю: это придет и Я сделаю; не отменю и не пощажу, и не помилую. По путям твоим и по делам твоим будут судить тебя, говорит Господь Бог.
\rsbpar\vs Eze 24:15 И было ко мне слово Господне:
\vs Eze 24:16 сын человеческий! вот, Я возьму у тебя язвою утеху очей твоих; но ты не сетуй и не плачь, и слезы да не выступают у тебя;
\vs Eze 24:17 вздыхай в безмолвии, плача по умершим не совершай; но обвязывай себя повязкою и обувай ноги твои в обувь твою, и бород\acc{ы} не закрывай, и хлеба от чужих не ешь.
\vs Eze 24:18 И после того, как говорил я поутру слово к народу, вечером умерла жена моя, и на другой день я сделал так, как повелено было мне.
\vs Eze 24:19 И сказал мне народ: не скажешь ли нам, какое для нас значение в том, что ты делаешь?
\vs Eze 24:20 И сказал я им: ко мне было слово Господне:
\vs Eze 24:21 скажи дому Израилеву: так говорит Господь Бог: вот, Я отдам на поругание святилище Мое, опору силы вашей, утеху очей ваших и отраду души вашей, а сыновья ваши и дочери ваши, которых вы оставили, падут от меча.
\vs Eze 24:22 И вы будете делать то же, что делал я; бород\acc{ы} не будете закрывать, и хлеба от чужих не будете есть;
\vs Eze 24:23 и повязки ваши будут на головах ваших, и обувь ваша на ногах ваших; не будете сетовать и плакать, но будете истаявать от грехов ваших и воздыхать друг перед другом.
\vs Eze 24:24 И будет для вас Иезекииль знамением: все, что он делал, и вы будете делать; и когда это сбудется, узнаете, что Я Господь Бог.
\vs Eze 24:25 А что до тебя, сын человеческий, то в тот день, когда Я возьму у них украшение славы их, утеху очей их и отраду души их, сыновей их и дочерей их,~---
\vs Eze 24:26 в тот день придет к тебе спасшийся \bibemph{оттуда}, чтобы подать весть в уши твои.
\vs Eze 24:27 В тот день при этом спасшемся откроются уста твои, и ты будешь говорить, и не останешься уже безмолвным, и будешь знамением для них, и узнают, что Я Господь.
\vs Eze 25:1 И было ко мне слово Господне:
\vs Eze 25:2 сын человеческий! обрати лице твое к сынам Аммоновым и изреки на них пророчество,
\vs Eze 25:3 и скажи сынам Аммоновым: слушайте слово Господа Бога: так говорит Господь Бог: за то, что ты о святилище Моем говоришь: <<а! а!>>, потому что оно поругано,~--- и о земле Израилевой, потому что она опустошена, и о доме Иудином, потому что они пошли в плен,~---
\vs Eze 25:4 за то вот, Я отдам тебя в наследие сынам востока, и построят у тебя овчарни свои, и поставят у тебя шатры свои, и будут есть плоды твои и пить молоко твое.
\vs Eze 25:5 Я сделаю Равву стойлом для верблюдов, и сынов Аммоновых~--- пастухами овец, и узнаете, что Я Господь.
\vs Eze 25:6 Ибо так говорит Господь Бог: за то, что ты рукоплескал и топал ногою, и со всем презрением к земле Израилевой душевно радовался,~---
\vs Eze 25:7 за то вот, Я простру руку Мою на тебя и отдам тебя на расхищение народам, и истреблю тебя из числа народов, и изглажу тебя из числа земель; сокрушу тебя, и узнаешь, что Я Господь.
\vs Eze 25:8 Так говорит Господь Бог: за то, что Моав и Сеир говорят: <<вот и дом Иудин, как все народы!>>,
\vs Eze 25:9 за то вот, Я, \bibemph{начиная} от городов, от всех пограничных городов его, красы земли, от Беф-Иешимофа, Ваалмеона и Кириафаима, открою бок Моава
\vs Eze 25:10 для сынов востока и отдам его в наследие \bibemph{им}, вместе с сынами Аммоновыми, чтобы сыны Аммона не упоминались более среди народов.
\vs Eze 25:11 И над Моавом произведу суд, и узнают, что Я Господь.
\vs Eze 25:12 Так говорит Господь Бог: за то, что Едом жестоко мстил дому Иудину и тяжко согрешил, совершая над ним мщение,
\vs Eze 25:13 за то, так говорит Господь Бог: простру руку Мою на Едома и истреблю у него людей и скот, и сделаю его пустынею; от Фемана до Дедана все падут от меча.
\vs Eze 25:14 И совершу мщение Мое над Едомом рукою народа Моего, Израиля; и они будут действовать в Идумее по Моему гневу и Моему негодованию, и узнают мщение Мое, говорит Господь Бог.
\vs Eze 25:15 Так говорит Господь Бог: за то, что Филистимляне поступили мстительно и мстили с презрением в душе, на погибель, по вечной неприязни,
\vs Eze 25:16 за то, так говорит Господь Бог: вот, Я простру руку Мою на Филистимлян, и истреблю Критян, и уничтожу остаток их на берегу моря;
\vs Eze 25:17 и совершу над ними великое мщение наказаниями яростными; и узнают, что Я Господь, когда совершу над ними Мое мщение.
\vs Eze 26:1 В одиннадцатом году, в первый день первого месяца, было ко мне слово Господне:
\vs Eze 26:2 сын человеческий! за то, что Тир говорит о Иерусалиме: <<а! а! он сокрушен~--- врата народов; он обращается ко мне; наполнюсь; он опустошен>>,~---
\vs Eze 26:3 за то, так говорит Господь Бог: вот, Я~--- на тебя, Тир, и подниму на тебя многие народы, как море поднимает волны свои.
\vs Eze 26:4 И разобьют стены Тира и разрушат башни его; и вымету из него прах его и сделаю его голою скалою.
\vs Eze 26:5 Местом для расстилания сетей будет он среди моря; ибо Я сказал это, говорит Господь Бог: и будет он на расхищение народам.
\vs Eze 26:6 А дочери его, которые на земле, убиты будут мечом, и узнают, что Я Господь.
\vs Eze 26:7 Ибо так говорит Господь Бог: вот, Я приведу против Тира от севера Навуходоносора, царя Вавилонского, царя царей, с конями и с колесницами, и со всадниками, и с войском, и с многочисленным народом.
\vs Eze 26:8 Дочерей твоих на земле он побьет мечом и устроит против тебя осадные башни, и насыплет против тебя вал, и поставит против тебя щиты;
\vs Eze 26:9 и к стенам твоим придвинет стенобитные машины и башни твои разрушит секирами своими.
\vs Eze 26:10 От множества коней его покроет тебя пыль, от шума всадников и колес и колесниц потрясутся стены твои, когда он будет входить в ворота твои, как входят в разбитый город.
\vs Eze 26:11 Копытами коней своих он истопчет все улицы твои, народ твой побьет мечом и памятники могущества твоего повергнет на землю.
\vs Eze 26:12 И разграбят богатство твое, и расхитят товары твои, и разрушат стены твои, и разобьют красивые домы твои, и камни твои и дерева твои, и землю твою бросят в воду.
\vs Eze 26:13 И прекращу шум песней твоих, и звук цитр твоих уже не будет слышен.
\vs Eze 26:14 И сделаю тебя голою скалою, будешь местом для расстилания сетей; не будешь вновь построен: ибо Я, Господь, сказал это, говорит Господь Бог.
\vs Eze 26:15 Так говорит Господь Бог Тиру: от шума падения твоего, от стона раненых, когда будет производимо среди тебя избиение, не содрогнутся ли острова?
\vs Eze 26:16 И сойдут все князья моря с престолов своих, и сложат с себя мантии свои, и снимут с себя узорчатые одежды свои, облекутся в трепет, сядут на землю, и ежеминутно будут содрогаться и изумляться о тебе.
\vs Eze 26:17 И поднимут плач о тебе и скажут тебе: как погиб ты, населенный мореходцами, город знаменитый, который был силен на море, сам и жители его, наводившие страх на всех обитателей его!
\vs Eze 26:18 Ныне, в день падения твоего, содрогнулись острова; острова на море приведены в смятение погибелью твоею.
\vs Eze 26:19 Ибо так говорит Господь Бог: когда Я сделаю тебя городом опустелым, подобным городам необитаемым, когда подниму на тебя пучину, и покроют тебя большие воды;
\vs Eze 26:20 тогда низведу тебя с отходящими в могилу к народу давно бывшему, и помещу тебя в преисподних земли, в пустынях вечных, с отшедшими в могилу, чтобы ты не был более населен; и явлю Я славу на земле живых.
\vs Eze 26:21 Ужасом сделаю тебя, и не будет тебя, и будут искать тебя, но уже не найдут тебя во веки, говорит Господь Бог.
\vs Eze 27:1 И было ко мне слово Господне:
\vs Eze 27:2 и ты, сын человеческий, подними плач о Тире
\vs Eze 27:3 и скажи Тиру, поселившемуся на выступах в море, торгующему с народами на многих островах: так говорит Господь Бог: Тир! ты говоришь: <<я совершенство красоты!>>
\vs Eze 27:4 Пределы твои в сердце морей; строители твои усовершили красоту твою:
\vs Eze 27:5 из Сенирских кипарисов устроили все помосты твои; брали с Ливана кедр, чтобы сделать на тебе мачты;
\vs Eze 27:6 из дубов Васанских делали весла твои; скамьи твои делали из букового дерева, с оправою из слоновой кости с островов Киттимских;
\vs Eze 27:7 узорчатые полотна из Египта употреблялись на паруса твои и служили флагом; голубого и пурпурового цвета ткани с островов Елисы были покрывалом твоим.
\vs Eze 27:8 Жители Сидона и Арвада были у тебя гребцами; свои знатоки были у тебя, Тир; они были у тебя кормчими.
\vs Eze 27:9 Старшие из Гевала и знатоки его были у тебя, чтобы заделывать пробоины твои. Всякие морские корабли и корабельщики их находились у тебя для производства торговли твоей.
\vs Eze 27:10 Перс и Лидиянин и Ливиец находились в войске твоем и были у тебя ратниками, вешали на тебе щит и шлем; они придавали тебе величие.
\vs Eze 27:11 Сыны Арвада с собственным твоим войском стояли кругом на стенах твоих, и Гамадимы были на башнях твоих; кругом по стенам твоим они вешали колчаны свои; они довершали красу твою.
\vs Eze 27:12 Фарсис, торговец твой, по множеству всякого богатства, платил за товары твои серебром, железом, свинцом и оловом.
\vs Eze 27:13 Иаван, Фувал и Мешех торговали с тобою, выменивая товары твои на души человеческие и медную посуду.
\vs Eze 27:14 Из дома Фогарма за товары твои доставляли тебе лошадей и строевых коней и лошаков.
\vs Eze 27:15 Сыны Дедана торговали с тобою; многие острова производили с тобою мену, в уплату тебе доставляли слоновую кость и черное дерево.
\vs Eze 27:16 По причине большого торгового производства твоего торговали с тобою Арамеяне; за товары твои они платили карбункулами, тканями пурпуровыми, узорчатыми, и виссонами, и кораллами, и рубинами.
\vs Eze 27:17 Иудея и земля Израилева торговали с тобою; за товар твой платили пшеницею Миннифскою и сластями, и медом, и деревянным маслом, и бальзамом.
\vs Eze 27:18 Дамаск, по причине большого торгового производства твоего, по изобилию всякого богатства, торговал с тобою вином Хелбонским и белою шерстью.
\vs Eze 27:19 Дан и Иаван из Узала платили тебе за товары твои выделанным железом; кассия и благовонная трость шли на обмен тебе.
\vs Eze 27:20 Дедан торговал с тобою драгоценными попонами для верховой езды.
\vs Eze 27:21 Аравия и все князья Кидарские производили мену с тобою: ягнят и баранов и козлов променивали тебе.
\vs Eze 27:22 Купцы из Савы и Раемы торговали с тобою всякими лучшими благовониями и всякими дорогими камнями, и золотом платили за товары твои.
\vs Eze 27:23 Харан и Хане и Еден, купцы Савейские, Ассур и Хилмад торговали с тобою.
\vs Eze 27:24 Они торговали с тобою драгоценными одеждами, шелковыми и узорчатыми материями, которые они привозили на твои рынки в дорогих ящиках, сделанных из кедра и хорошо упакованных.
\vs Eze 27:25 Фарсисские корабли были твоими караванами в твоей торговле, и ты сделался богатым и весьма славным среди морей.
\vs Eze 27:26 Гребцы твои завели тебя в большие воды; восточный ветер разбил тебя среди морей.
\vs Eze 27:27 Богатство твое и товары твои, все склады твои, корабельщики твои и кормчие твои, заделывавшие пробоины твои и распоряжавшиеся торговлею твоею, и все ратники твои, какие у тебя были, и все множество народа в тебе, в день падения твоего упадет в сердце морей.
\vs Eze 27:28 От вопля кормчих твоих содрогнутся окрестности.
\vs Eze 27:29 И с кораблей своих сойдут все гребцы, корабельщики, все кормчие моря, и станут на землю;
\vs Eze 27:30 и зарыдают о тебе громким голосом, и горько застенают, посыпав пеплом головы свои и валяясь во прахе;
\vs Eze 27:31 и остригут по тебе волосы догола, и опояшутся вретищами, и заплачут о тебе от душевной скорби горьким плачем;
\vs Eze 27:32 и в сетовании своем поднимут плачевную песнь о тебе, и так зарыдают о тебе: <<кто как Тир, так разрушенный посреди моря!
\vs Eze 27:33 Когда приходили с морей товары твои, ты насыщал многие народы; множеством богатства твоего и торговлею твоею обогащал царей земли.
\vs Eze 27:34 А когда ты разбит морями в пучине вод, товары твои и все толпившееся в тебе упало.
\vs Eze 27:35 Все обитатели островов ужаснулись о тебе, и цари их содрогнулись, изменились в лицах.
\vs Eze 27:36 Торговцы других народов свистнули о тебе; ты сделался ужасом,~--- и не будет тебя во веки>>.
\vs Eze 28:1 И было ко мне слово Господне:
\vs Eze 28:2 сын человеческий! скажи начальствующему в Тире: так говорит Господь Бог: за то, что вознеслось сердце твое и ты говоришь: <<я бог, восседаю на седалище божием, в сердце морей>>, и будучи человеком, а не Богом, ставишь ум твой наравне с умом Божиим,~---
\vs Eze 28:3 вот, ты премудрее Даниила, нет тайны, сокрытой от тебя;
\vs Eze 28:4 твоею мудростью и твоим разумом ты приобрел себе богатство и в сокровищницы твои собрал золота и серебра;
\vs Eze 28:5 большою мудростью твоею, посредством торговли твоей, ты умножил богатство твое, и ум твой возгордился богатством твоим,~---
\vs Eze 28:6 за то так говорит Господь Бог: так как ты ум твой ставишь наравне с умом Божиим,
\vs Eze 28:7 вот, Я приведу на тебя иноземцев, лютейших из народов, и они обнажат мечи свои против красы твоей мудрости и помрачат блеск твой;
\vs Eze 28:8 низведут тебя в могилу, и умрешь в сердце морей смертью убитых.
\vs Eze 28:9 Скажешь ли тогда перед твоим убийцею: <<я бог>>, тогда как в руке поражающего тебя ты будешь человек, а не бог?
\vs Eze 28:10 Ты умрешь от руки иноземцев смертью необрезанных; ибо Я сказал это, говорит Господь Бог.
\vs Eze 28:11 И было ко мне слово Господне:
\vs Eze 28:12 сын человеческий! плачь о царе Тирском и скажи ему: так говорит Господь Бог: ты печать совершенства, полнота мудрости и венец красоты.
\vs Eze 28:13 Ты находился в Едеме, в саду Божием; твои одежды были украшены всякими драгоценными камнями; рубин, топаз и алмаз, хризолит, оникс, яспис, сапфир, карбункул и изумруд и золото, все, искусно усаженное у тебя в гнездышках и нанизанное на тебе, приготовлено было в день сотворения твоего.
\vs Eze 28:14 Ты был помазанным херувимом, чтобы осенять, и Я поставил тебя на то; ты был на святой горе Божией, ходил среди огнистых камней.
\vs Eze 28:15 Ты совершен был в путях твоих со дня сотворения твоего, доколе не нашлось в тебе беззакония.
\vs Eze 28:16 От обширности торговли твоей внутреннее твое исполнилось неправды, и ты согрешил; и Я низвергнул тебя, как нечистого, с горы Божией, изгнал тебя, херувим осеняющий, из среды огнистых камней.
\vs Eze 28:17 От красоты твоей возгордилось сердце твое, от тщеславия твоего ты погубил мудрость твою; за то Я повергну тебя на землю, перед царями отдам тебя на позор.
\vs Eze 28:18 Множеством беззаконий твоих в неправедной торговле твоей ты осквернил святилища твои; и Я извлеку из среды тебя огонь, который и пожрет тебя: и Я превращу тебя в пепел на земле перед глазами всех, видящих тебя.
\vs Eze 28:19 Все, знавшие тебя среди народов, изумятся о тебе; ты сделаешься ужасом, и не будет тебя во веки.
\rsbpar\vs Eze 28:20 И было ко мне слово Господне:
\vs Eze 28:21 сын человеческий! обрати лице твое к Сидону и изреки на него пророчество,
\vs Eze 28:22 и скажи: вот, Я~--- на тебя, Сидон, и прославлюсь среди тебя, и узнают, что Я Господь, когда произведу суд над ним и явлю в нем святость Мою;
\vs Eze 28:23 и пошлю на него моровую язву и кровопролитие на улицы его, и падут среди него убитые мечом, пожирающим его отовсюду; и узнают, что Я Господь.
\vs Eze 28:24 И не будет он впредь для дома Израилева колючим терном и причиняющим боль волчцом, более всех соседей зложелательствующим ему, и узнают, что Я Господь Бог.
\vs Eze 28:25 Так говорит Господь Бог: когда Я соберу дом Израилев из народов, между которыми они рассеяны, и явлю в них святость Мою перед глазами племен, и они будут жить на земле своей, которую Я дал рабу Моему Иакову:
\vs Eze 28:26 тогда они будут жить на ней безопасно, и построят домы, и насадят виноградники, и будут жить в безопасности, потому что Я произведу суд над всеми зложелателями их вокруг них, и узнают, что Я Господь Бог их.
\vs Eze 29:1 В десятом году, в десятом \bibemph{месяце}, в двенадцатый \bibemph{день} месяца, было ко мне слово Господне:
\vs Eze 29:2 сын человеческий! обрати лице твое к фараону, царю Египетскому, и изреки пророчество на него и на весь Египет.
\vs Eze 29:3 Говори и скажи: так говорит Господь Бог: вот, Я~--- на тебя, фараон, царь Египетский, большой крокодил, который, лежа среди рек своих, говоришь: <<моя река, и я создал ее для себя>>.
\vs Eze 29:4 Но Я вложу крюк в челюсти твои и к чешуе твоей прилеплю рыб из рек твоих, и вытащу тебя из рек твоих со всею рыбою рек твоих, прилипшею к чешуе твоей;
\vs Eze 29:5 и брошу тебя в пустыне, тебя и всю рыбу из рек твоих, ты упадешь на открытое поле, не уберут и не подберут тебя; отдам тебя на съедение зверям земным и птицам небесным.
\vs Eze 29:6 И узнают все обитатели Египта, что Я Господь; потому что они дому Израилеву были подпорою тростниковою.
\vs Eze 29:7 Когда они ухватились за тебя рукою, ты расщепился и все плечо исколол им; и когда они оперлись о тебя, ты сломился и изранил все чресла им.
\vs Eze 29:8 Посему так говорит Господь Бог: вот, Я наведу на тебя меч, и истреблю у тебя людей и скот.
\vs Eze 29:9 И сделается земля Египетская пустынею и степью; и узнают, что Я Господь. Так как он говорит: <<моя река, и я создал ее>>;
\vs Eze 29:10 то вот, Я~--- на реки твои, и сделаю землю Египетскую пустынею из пустынь от Мигдола до Сиены, до самого предела Ефиопии.
\vs Eze 29:11 Не будет проходить по ней нога человеческая, и нога скотов не будет проходить по ней, и не будут обитать на ней сорок лет.
\vs Eze 29:12 И сделаю землю Египетскую пустынею среди земель опустошенных; и города ее среди опустелых городов будут пустыми сорок лет, и рассею Египтян по народам, и развею их по землям.
\vs Eze 29:13 Ибо так говорит Господь Бог: по окончании сорока лет Я соберу Египтян из народов, между которыми они будут рассеяны;
\vs Eze 29:14 и возвращу плен Египта, и обратно приведу их в землю Пафрос, в землю происхождения их, и там они будут царством слабым.
\vs Eze 29:15 Оно будет слабее \bibemph{других} царств, и не будет более возноситься над народами; Я умалю их, чтобы они не господствовали над народами.
\vs Eze 29:16 И не будут впредь дому Израилеву опорою, припоминающею беззаконие их, когда они обращались к нему; и узнают, что Я Господь Бог.
\rsbpar\vs Eze 29:17 В двадцать седьмом году, в первом \bibemph{месяце}, в первый \bibemph{день} месяца, было ко мне слово Господне:
\vs Eze 29:18 сын человеческий! Навуходоносор, царь Вавилонский, утомил свое войско большими работами при Тире; все головы оплешивели и все плечи стерты; а ни ему, ни войску его нет вознаграждения от Тира за работы, которые он употребил против него.
\vs Eze 29:19 Посему так говорит Господь Бог: вот, Я Навуходоносору, царю Вавилонскому, даю землю Египетскую, чтобы он обобрал богатство ее и произвел грабеж в ней, и ограбил награбленное ею, и это будет вознаграждением войску его.
\vs Eze 29:20 В награду за дело, которое он произвел в нем, Я отдаю ему землю Египетскую, потому что они делали это для Меня, сказал Господь Бог.
\vs Eze 29:21 В тот день возвращу рог дому Израилеву, и тебе открою уста среди них, и узнают, что Я Господь.
\vs Eze 30:1 И было ко мне слово Господне:
\vs Eze 30:2 сын человеческий! изреки пророчество и скажи: так говорит Господь Бог: рыдайте! о, злосчастный день!
\vs Eze 30:3 Ибо близок день, так! близок день Господа, день мрачный; година народов наступает.
\vs Eze 30:4 И пойдет меч на Египет, и ужас распространится в Ефиопии, когда в Египте будут падать пораженные, когда возьмут богатство его, и основания его будут разрушены;
\vs Eze 30:5 Ефиопия и Ливия, и Лидия, и весь смешанный народ, и Хуб, и сыны земли завета вместе с ними падут от меча.
\vs Eze 30:6 Так говорит Господь: падут подпоры Египта, и упадет гордыня могущества его; от Мигдола до Сиены будут падать в нем от меча, сказал Господь Бог.
\vs Eze 30:7 И опустеет он среди опустошенных земель, и города его будут среди опустошенных городов.
\vs Eze 30:8 И узнают, что Я Господь, когда пошлю огонь на Египет, и все подпоры его будут сокрушены.
\vs Eze 30:9 В тот день пойдут от Меня вестники на кораблях, чтобы устрашить беспечных Ефиоплян, и распространится у них ужас, как в день Египта; ибо вот, он идет.
\vs Eze 30:10 Так говорит Господь Бог: положу конец многолюдству Египта рукою Навуходоносора, царя Вавилонского.
\vs Eze 30:11 Он и с ним народ его, лютейший из народов, приведены будут на погибель сей земли, и обнажат мечи свои на Египет, и наполнят землю пораженными.
\vs Eze 30:12 И реки сделаю сушею и предам землю в руки злым, и рукою иноземцев опустошу землю и все, наполняющее ее. Я, Господь, сказал это.
\vs Eze 30:13 Так говорит Господь Бог: истреблю идолов и уничтожу лжебогов в Мемфисе, и из земли Египетской не будет уже властителя, и наведу страх на землю Египетскую.
\vs Eze 30:14 И опустошу Пафрос и пошлю огонь на Цоан, и произведу суд над Но.
\vs Eze 30:15 И изолью ярость Мою на Син, крепость Египта, и истреблю многолюдие в Но.
\vs Eze 30:16 И пошлю огонь на Египет; вострепещет Син, и Но рушится, и на Мемфис нападут враги среди дня.
\vs Eze 30:17 Молодые люди Она и Бубаста падут от меча, а прочие пойдут в плен.
\vs Eze 30:18 И в Тафнисе померкнет день, когда Я сокрушу там ярмо Египта, и прекратится в нем гордое могущество его. Облако закроет его, и дочери его пойдут в плен.
\vs Eze 30:19 Так произведу Я суд над Египтом, и узнают, что Я Господь.
\rsbpar\vs Eze 30:20 В одиннадцатом году, в первом месяце, в седьмой день \bibemph{месяца}, было ко мне слово Господне:
\vs Eze 30:21 сын человеческий! Я уже сокрушил мышцу фараону, царю Египетскому; и вот, она еще не обвязана для излечения ее и не обвита врачебными перевязками, от которых она получила бы силу держать меч.
\vs Eze 30:22 Посему так говорит Господь Бог: вот, Я~--- на фараона, царя Египетского, и сокрушу мышцы его, здоровую и переломленную, так что меч выпадет из руки его.
\vs Eze 30:23 И рассею Египтян по народам, и развею их по землям.
\vs Eze 30:24 А мышцы царя Вавилонского сделаю крепкими и дам ему меч Мой в руку, мышцы же фараона сокрушу, и он изъязвленный будет сильно стонать перед ним.
\vs Eze 30:25 Укреплю мышцы царя Вавилонского, а мышцы у фараона опустятся; и узнают, что Я Господь, когда меч Мой дам в руку царю Вавилонскому, и он прострет его на землю Египетскую.
\vs Eze 30:26 И рассею Египтян по народам, и развею их по землям, и узнают, что Я Господь.
\vs Eze 31:1 В одиннадцатом году, в третьем \bibemph{месяце}, в первый день месяца, было ко мне слово Господне:
\vs Eze 31:2 сын человеческий! скажи фараону, царю Египетскому, и народу его: кому ты равняешь себя в величии твоем?
\vs Eze 31:3 Вот, Ассур был кедр на Ливане, с красивыми ветвями и тенистою листвою, и высокий ростом; вершина его находилась среди толстых сучьев.
\vs Eze 31:4 Воды растили его, бездна поднимала его, реки ее окружали питомник его, и она протоки свои посылала ко всем деревам полевым.
\vs Eze 31:5 Оттого высота его перевысила все дерева полевые, и сучьев на нем было много, и ветви его умножались, и сучья его становились длинными от множества вод, когда он разрастался.
\vs Eze 31:6 На сучьях его вили гнезда всякие птицы небесные, под ветвями его выводили детей всякие звери полевые, и под тенью его жили всякие многочисленные народы.
\vs Eze 31:7 Он красовался высотою роста своего, длиною ветвей своих, ибо корень его был у великих вод.
\vs Eze 31:8 Кедры в саду Божием не затемняли его; кипарисы не равнялись сучьям его, и каштаны не были величиною с ветви его, ни одно дерево в саду Божием не равнялось с ним красотою своею.
\vs Eze 31:9 Я украсил его множеством ветвей его, так что все дерева Едемские в саду Божием завидовали ему.
\vs Eze 31:10 Посему так сказал Господь Бог: за то, что ты высок стал ростом и вершину твою выставил среди толстых сучьев, и сердце его возгордилось величием его,~---
\vs Eze 31:11 за то Я отдал его в руки властителю народов; он поступил с ним, как надобно; за беззаконие его Я отверг его.
\vs Eze 31:12 И срубили его чужеземцы, лютейшие из народов, и повергли его на горы; и на все долины упали ветви его; и сучья его сокрушились на всех лощинах земли, и из-под тени его ушли все народы земли, и оставили его.
\vs Eze 31:13 На обломках его поместились всякие птицы небесные, и в сучьях были всякие полевые звери.
\vs Eze 31:14 Это для того, чтобы никакие дерева при водах не величались высоким ростом своим и не поднимали вершины своей из среды толстых сучьев, и чтобы не прилеплялись к ним из-за высоты их дерева, пьющие воду; ибо все они будут преданы смерти, в преисподнюю страну вместе с сынами человеческими, отшедшими в могилу.
\vs Eze 31:15 Так говорит Господь Бог: в тот день, когда он сошел в могилу, Я сделал сетование о нем, затворил ради него бездну и остановил реки ее, и задержал большие воды и омрачил по нем Ливан, и все дерева полевые были в унынии по нем.
\vs Eze 31:16 Шумом падения его Я привел в трепет народы, когда низвел его в преисподнюю, к отшедшим в могилу, и обрадовались в преисподней стране все дерева Едема, отличные и наилучшие Ливанские, все, пьющие воду;
\vs Eze 31:17 ибо и они с ним отошли в преисподнюю, к пораженным мечом, и союзники его, жившие под тенью его, среди народов.
\vs Eze 31:18 Итак которому из дерев Едемских равнялся ты в славе и величии? Но теперь наравне с деревами Едемскими ты будешь низведен в преисподнюю, будешь лежать среди необрезанных, с пораженными мечом. Это фараон и все множество народа его, говорит Господь Бог.
\vs Eze 32:1 В двенадцатом году, в двенадцатом месяце, в первый \bibemph{день} месяца, было ко мне слово Господне:
\vs Eze 32:2 сын человеческий! подними плач о фараоне, царе Египетском, и скажи ему: ты, как молодой лев между народами и как чудовище в морях, кидаешься в реках твоих, и мутишь ногами твоими воды, и попираешь потоки их.
\vs Eze 32:3 Так говорит Господь Бог: Я закину на тебя сеть Мою в собрании многих народов, и они вытащат тебя Моею мрежею.
\vs Eze 32:4 И выкину тебя на землю, на открытом поле брошу тебя, и будут садиться на тебя всякие небесные птицы, и насыщаться тобою звери всей земли.
\vs Eze 32:5 И раскидаю мясо твое по горам, и долины наполню твоими трупами.
\vs Eze 32:6 И землю плавания твоего напою кровью твоею до самых гор; и рытвины будут наполнены тобою.
\vs Eze 32:7 И когда ты угаснешь, закрою небеса и звезды их помрачу, солнце закрою облаком, и луна не будет светить светом своим.
\vs Eze 32:8 Все светила, светящиеся на небе, помрачу над тобою и на землю твою наведу тьму, говорит Господь Бог.
\vs Eze 32:9 Приведу в смущение сердце многих народов, когда разглашу о падении твоем между народами, по землям, которых ты не знал.
\vs Eze 32:10 И приведу тобою в ужас многие народы, и цари их содрогнутся о тебе в страхе, когда мечом Моим потрясу перед лицем их, и поминутно будут трепетать каждый за душу свою в день падения твоего.
\vs Eze 32:11 Ибо так говорит Господь Бог: меч царя Вавилонского придет на тебя.
\vs Eze 32:12 От мечей сильных падет народ твой; все они~--- лютейшие из народов, и уничтожат гордость Египта, и погибнет все множество его.
\vs Eze 32:13 И истреблю весь скот его при великих водах, и вперед не будет мутить их нога человеческая, и копыта скота не будут мутить их.
\vs Eze 32:14 Тогда дам покой водам их, и сделаю, что реки их потекут, как масло, говорит Господь Бог.
\vs Eze 32:15 Когда сделаю землю Египетскую пустынею, и когда лишится земля всего, наполняющего ее; когда поражу всех живущих на ней, тогда узнают, что Я Господь.
\vs Eze 32:16 Вот плачевная песнь, которую будут петь; дочери народов будут петь ее; о Египте и обо всем множестве его будут петь ее, говорит Господь Бог.
\rsbpar\vs Eze 32:17 В двенадцатом году, в пятнадцатый \bibemph{день того же} месяца, было ко мне слово Господне:
\vs Eze 32:18 сын человеческий! оплачь народ Египетский, и низринь его, его и дочерей знаменитых народов в преисподнюю, с отходящими в могилу.
\vs Eze 32:19 Кого ты превосходишь? сойди, и лежи с необрезанными.
\vs Eze 32:20 Те падут среди убитых мечом, и он отдан мечу; влеките его и все множество его.
\vs Eze 32:21 Среди преисподней будут говорить о нем и о союзниках его первые из героев; они пали и лежат там между необрезанными, сраженные мечом.
\vs Eze 32:22 Там Ассур и все полчище его, вокруг него гробы их, все пораженные, павшие от меча.
\vs Eze 32:23 Гробы его поставлены в самой глубине преисподней, и полчище его вокруг гробницы его, все пораженные, павшие от меча, те, которые распространяли ужас на земле живых.
\vs Eze 32:24 Там Елам со всем множеством своим вокруг гробницы его, все они пораженные, павшие от меча, которые необрезанными сошли в преисподнюю, которые распространили собою ужас на земле живых и несут позор свой с отшедшими в могилу.
\vs Eze 32:25 Среди пораженных дали ложе ему со всем множеством его; вокруг него гробы их, все необрезанные, пораженные мечом; и как они распространяли ужас на земле живых, то и несут на себе позор наравне с отшедшими в могилу и положены среди пораженных.
\vs Eze 32:26 Там Мешех и Фувал со всем множеством своим; вокруг него гробы их, все необрезанные, пораженные мечом, потому что они распространяли ужас на земле живых.
\vs Eze 32:27 Не должны ли \bibemph{и} они лежать с павшими героями необрезанными, которые с воинским оружием своим сошли в преисподнюю и мечи свои положили себе под головы, и осталось беззаконие их на костях их, потому что они, как сильные, были ужасом на земле живых.
\vs Eze 32:28 И ты будешь сокрушен среди необрезанных и лежать с пораженными мечом.
\vs Eze 32:29 Там Едом и цари его и все князья его, которые при всей своей храбрости положены среди пораженных мечом; они лежат с необрезанными и сошедшими в могилу.
\vs Eze 32:30 Там властелины севера, все они и все Сидоняне, которые сошли туда с пораженными, быв посрамлены в могуществе своем, наводившем ужас, и лежат они с необрезанными, пораженными мечом, и несут позор свой с отшедшими в могилу.
\vs Eze 32:31 Увидит их фараон и утешится о всем множестве своем, пораженном мечом, фараон и все войско его, говорит Господь Бог.
\vs Eze 32:32 Ибо Я распространю страх Мой на земле живых, и положен будет фараон и все множество его среди необрезанных с пораженными мечом, говорит Господь Бог.
\vs Eze 33:1 И было ко мне слово Господне:
\vs Eze 33:2 сын человеческий! изреки слово к сынам народа твоего и скажи им: если Я на какую-либо землю наведу меч, и народ той земли возьмет из среды себя человека и поставит его у себя стражем;
\vs Eze 33:3 и он, увидев меч, идущий на землю, затрубит в трубу и предостережет народ;
\vs Eze 33:4 и если кто будет слушать голос трубы, но не остережет себя,~--- то, когда меч придет и захватит его, кровь его будет на его голове.
\vs Eze 33:5 Голос трубы он слышал, но не остерег себя, кровь его на нем будет; а кто остерегся, тот спас жизнь свою.
\vs Eze 33:6 Если же страж видел идущий меч и не затрубил в трубу, и народ не был предостережен,~--- то, когда придет меч и отнимет у кого из них жизнь, сей схвачен будет за грех свой, но кровь его взыщу от руки стража.
\vs Eze 33:7 И тебя, сын человеческий, Я поставил стражем дому Израилеву, и ты будешь слышать из уст Моих слово и вразумлять их от Меня.
\vs Eze 33:8 Когда Я скажу беззаконнику: <<беззаконник! ты смертью умрешь>>, а ты не будешь ничего говорить, чтобы предостеречь беззаконника от пути его,~--- то беззаконник тот умрет за грех свой, но кровь его взыщу от руки твоей.
\vs Eze 33:9 Если же ты остерегал беззаконника от пути его, чтобы он обратился от него, но он от пути своего не обратился,~--- то он умирает за грех свой, а ты спас душу твою.
\vs Eze 33:10 И ты, сын человеческий, скажи дому Израилеву: вы говорите так: <<преступления наши и грехи наши на нас, и мы истаеваем в них: как же можем мы жить?>>
\vs Eze 33:11 Скажи им: живу Я, говорит Господь Бог: не хочу смерти грешника, но чтобы грешник обратился от пути своего и жив был. Обратитесь, обратитесь от злых путей ваших; для чего умирать вам, дом Израилев?
\vs Eze 33:12 И ты, сын человеческий, скажи сынам народа твоего: праведность праведника не спасет в день преступления его, и беззаконник за беззаконие свое не падет в день обращения от беззакония своего, равно как и праведник в день согрешения своего не может остаться в живых за свою праведность.
\vs Eze 33:13 Когда Я скажу праведнику, что он будет жив, а он понадеется на свою праведность и сделает неправду,~--- то все праведные дела его не помянутся, и он умрет от неправды своей, какую сделал.
\vs Eze 33:14 А когда скажу беззаконнику: <<ты смертью умрешь>>, и он обратится от грехов своих и будет творить суд и правду,
\vs Eze 33:15 \bibemph{если} этот беззаконник возвратит залог, за похищенное заплатит, будет ходить по законам жизни, не делая ничего худого,~--- то он будет жив, не умрет.
\vs Eze 33:16 Ни один из грехов его, какие он сделал, не помянется ему; он стал творить суд и правду, он будет жив.
\vs Eze 33:17 А сыны народа твоего говорят: <<неправ путь Господа>>, тогда как их путь неправ.
\vs Eze 33:18 Когда праведник отступил от праведности своей и начал делать беззаконие,~--- то он умрет за то.
\vs Eze 33:19 И когда беззаконник обратился от беззакония своего и стал творить суд и правду, он будет за то жив.
\vs Eze 33:20 А вы говорите: <<неправ путь Господа!>> Я буду судить вас, дом Израилев, каждого по путям его.
\rsbpar\vs Eze 33:21 В двенадцатом году нашего переселения, в десятом \bibemph{месяце}, в пятый \bibemph{день} месяца, пришел ко мне один из спасшихся из Иерусалима и сказал: <<разрушен город!>>
\vs Eze 33:22 Но еще до прихода сего спасшегося вечером была на мне рука Господа, и Он открыл мне уста, прежде нежели тот пришел ко мне поутру. И открылись уста мои, и я уже не был безмолвен.
\vs Eze 33:23 И было ко мне слово Господне:
\vs Eze 33:24 сын человеческий! живущие на опустелых местах в земле Израилевой говорят: <<Авраам был один, и получил во владение землю сию, а нас много; \bibemph{итак} нам дана земля сия во владение>>.
\vs Eze 33:25 Посему скажи им: так говорит Господь Бог: вы едите с кровью и поднимаете глаза ваши к идолам вашим, и проливаете кровь; и хотите владеть землею?
\vs Eze 33:26 Вы опираетесь на меч ваш, делаете мерзости, оскверняете один жену другого, и хотите владеть землею?
\vs Eze 33:27 Вот что скажи им: так говорит Господь Бог: живу Я! те, которые на местах разоренных, падут от меча; а кто в поле, того отдам зверям на съедение; а которые в укреплениях и пещерах, те умрут от моровой язвы.
\vs Eze 33:28 И сделаю землю пустынею из пустынь, и гордое могущество ее престанет, и горы Израилевы опустеют, так что не будет проходящих.
\vs Eze 33:29 И узнают, что Я Господь, когда сделаю землю пустынею из пустынь за все мерзости их, какие они делали.
\vs Eze 33:30 А о тебе, сын человеческий, сыны народа твоего разговаривают у стен и в дверях домов и говорят один другому, брат брату: <<пойдите и послушайте, какое слово вышло от Господа>>.
\vs Eze 33:31 И они приходят к тебе, как на народное сходбище, и садится перед лицем твоим народ Мой, и слушают слова твои, но не исполняют их; ибо они в устах своих делают из этого забаву, сердце их увлекается за корыстью их.
\vs Eze 33:32 И вот, ты для них~--- как забавный певец с приятным голосом и хорошо играющий; они слушают слова твои, но не исполняют их.
\vs Eze 33:33 Но когда сбудется,~--- вот, уже и сбывается,~--- тогда узнают, что среди них был пророк.
\vs Eze 34:1 И было ко мне слово Господне:
\vs Eze 34:2 сын человеческий! изреки пророчество на пастырей Израилевых, изреки пророчество и скажи им, пастырям: так говорит Господь Бог: горе пастырям Израилевым, которые пасли себя самих! не стадо ли должны пасти пастыри?
\vs Eze 34:3 Вы ели тук и в\acc{о}лною одевались, откормленных овец заколали, \bibemph{а} стада не пасли.
\vs Eze 34:4 Слабых не укрепляли, и больной овцы не врачевали, и пораненной не перевязывали, и угнанной не возвращали, и потерянной не искали, а правили ими с насилием и жестокостью.
\vs Eze 34:5 И рассеялись они без пастыря и, рассеявшись, сделались пищею всякому зверю полевому.
\vs Eze 34:6 Блуждают овцы Мои по всем горам и по всякому высокому холму, и по всему лицу земли рассеялись овцы Мои, и никто не разведывает о них, и никто не ищет их.
\vs Eze 34:7 Посему, пастыри, выслушайте слово Господне.
\vs Eze 34:8 Живу Я! говорит Господь Бог; за то, что овцы Мои оставлены были на расхищение и без пастыря сделались овцы Мои пищею всякого зверя полевого, и пастыри Мои не искали овец Моих, и пасли пастыри самих себя, а овец Моих не пасли,~---
\vs Eze 34:9 за то, пастыри, выслушайте слово Господне.
\vs Eze 34:10 Так говорит Господь Бог: вот, Я~--- на пастырей, и взыщу овец Моих от руки их, и не дам им более пасти овец, и не будут более пастыри пасти самих себя, и исторгну овец Моих из челюстей их, и не будут они пищею их.
\vs Eze 34:11 Ибо так говорит Господь Бог: вот, Я Сам отыщу овец Моих и осмотрю их.
\vs Eze 34:12 Как пастух поверяет стадо свое в тот день, когда находится среди стада своего рассеянного, так Я пересмотрю овец Моих и высвобожу их из всех мест, в которые они были рассеяны в день облачный и мрачный.
\vs Eze 34:13 И выведу их из народов, и соберу их из стран, и приведу их в землю их, и буду пасти их на горах Израилевых, при потоках и на всех обитаемых местах земли сей.
\vs Eze 34:14 Буду пасти их на хорошей пажити, и загон их будет на высоких горах Израилевых; там они будут отдыхать в хорошем загоне и будут пастись на тучной пажити, на горах Израилевых.
\vs Eze 34:15 Я буду пасти овец Моих и Я буду покоить их, говорит Господь Бог.
\vs Eze 34:16 Потерявшуюся отыщу и угнанную возвращу, и пораненную перевяжу, и больную укреплю, а разжиревшую и буйную истреблю; буду пасти их по правде.
\vs Eze 34:17 Вас же, овцы Мои,~--- так говорит Господь Бог,~--- вот, Я буду судить между овцою и овцою, между бараном и козлом.
\vs Eze 34:18 Разве мало вам того, что пасетесь на хорошей пажити, а между тем остальное на пажити вашей топчете ногами вашими, пьете чистую воду, а оставшуюся мутите ногами вашими,
\vs Eze 34:19 так что овцы Мои должны питаться тем, что потоптано ногами вашими, и пить то, что возмущено ногами вашими?
\vs Eze 34:20 Посему так говорит им Господь Бог: вот, Я Сам буду судить между овцою тучною и овцою тощею.
\vs Eze 34:21 Так как вы толкаете боком и плечом, и рогами своими бодаете всех слабых, доколе не вытолкаете их вон,~---
\vs Eze 34:22 то Я спасу овец Моих, и они не будут уже расхищаемы, и рассужу между овцою и овцою.
\vs Eze 34:23 И поставлю над ними одного пастыря, который будет пасти их, раба Моего Давида; он будет пасти их и он будет у них пастырем.
\vs Eze 34:24 И Я, Господь, буду их Богом, и раб Мой Давид будет князем среди них. Я, Господь, сказал это.
\vs Eze 34:25 И заключу с ними завет мира и удалю с земли лютых зверей, так что безопасно будут жить в степи и спать в лесах.
\vs Eze 34:26 Дарую им и окрестностям холма Моего благословение, и дождь буду ниспосылать в свое время; это будут дожди благословения.
\vs Eze 34:27 И полевое дерево будет давать плод свой, и земля будет давать произведения свои; и будут они безопасны на земле своей, и узнают, что Я Господь, когда сокрушу связи ярма их и освобожу их из руки поработителей их.
\vs Eze 34:28 Они не будут уже добычею для народов, и полевые звери не будут пожирать их; они будут жить безопасно, и никто не будет устрашать \bibemph{их}.
\vs Eze 34:29 И произведу у них насаждение славное, и не будут уже погибать от голода на земле и терпеть посрамления от народов.
\vs Eze 34:30 И узнают, что Я, Господь Бог их, с ними, и они, дом Израилев, Мой народ, говорит Господь Бог,
\vs Eze 34:31 и что вы~--- овцы Мои, овцы паствы Моей; вы~--- человеки, \bibemph{а} Я Бог ваш, говорит Господь Бог.
\vs Eze 35:1 И было ко мне слово Господне:
\vs Eze 35:2 сын человеческий! обрати лице твое к горе Сеир и изреки на нее пророчество
\vs Eze 35:3 и скажи ей: так говорит Господь Бог: вот, Я~--- на тебя, гора Сеир! и простру на тебя руку Мою и сделаю тебя пустою и необитаемою.
\vs Eze 35:4 Города твои превращу в развалины, и ты сама опустеешь и узнаешь, что Я Господь.
\vs Eze 35:5 Так как у тебя вечная вражда, и ты предавала сынов Израилевых в руки мечу во время несчастья их, во время окончательной гибели:
\vs Eze 35:6 за это~--- живу Я! говорит Господь Бог~--- сделаю тебя кровью, и кровь будет преследовать тебя; так как ты не ненавидела крови, то кровь и будет преследовать тебя.
\vs Eze 35:7 И сделаю гору Сеир пустою и безлюдною степью и истреблю на ней приходящего и возвращающегося.
\vs Eze 35:8 И наполню высоты ее убитыми ее; на холмах твоих и в долинах твоих, и во всех рытвинах твоих будут падать сраженные мечом.
\vs Eze 35:9 Сделаю тебя пустынею вечною, и в городах твоих не будут жить, и узнаете, что Я Господь.
\vs Eze 35:10 Так как ты говорила: <<эти два народа и эти две земли будут мои, и мы завладеем ими, хотя и Господь был там>>:
\vs Eze 35:11 за то,~--- живу Я! говорит Господь Бог,~--- поступлю с тобою по мере ненависти твоей и зависти твоей, какую ты выказала из ненависти твоей к ним, и явлю Себя им, когда буду судить тебя.
\vs Eze 35:12 И узнаешь, что Я, Господь, слышал все глумления твои, какие ты произносила на горы Израилевы, говоря: <<опустели! нам отданы на съедение!>>
\vs Eze 35:13 Вы величались предо Мною языком вашим и умножали речи ваши против Меня; Я слышал это.
\vs Eze 35:14 Так говорит Господь Бог: когда вся земля будет радоваться, Я сделаю тебя пустынею.
\vs Eze 35:15 Как ты радовалась тому, что удел дома Израилева опустел, так сделаю Я и с тобою: опустошена будешь, гора Сеир, и вся Идумея вместе, и узнают, что Я Господь.
\vs Eze 36:1 И ты, сын человеческий, изреки пророчество на горы Израилевы и скажи: горы Израилевы! слушайте слово Господне.
\vs Eze 36:2 Так говорит Господь Бог: так как враг говорит о вас: <<а! а! и вечные высоты достались нам в удел>>,
\vs Eze 36:3 то изреки пророчество и скажи: так говорит Господь Бог: за то, именно за то, что опустошают вас и поглощают вас со всех сторон, чтобы вы сделались достоянием прочих народов и подверглись злоречию и пересудам людей,~---
\vs Eze 36:4 за это, горы Израилевы, выслушайте слово Господа Бога: так говорит Господь Бог горам и холмам, лощинам и долинам, и опустелым развалинам, и оставленным городам, которые сделались добычею и посмеянием прочим окрестным народам;
\vs Eze 36:5 за это так говорит Господь Бог: в огне ревности Моей Я изрек слово на прочие народы и на всю Идумею, которые назначили землю Мою во владение себе, с сердечною радостью и с презрением в душе обрекая ее в добычу себе.
\vs Eze 36:6 Посему изреки пророчество о земле Израилевой и скажи горам и холмам, лощинам и долинам: так говорит Господь Бог: вот, Я изрек сие в ревности Моей и в ярости Моей, потому что вы несете на себе посмеяние от народов.
\vs Eze 36:7 Посему так говорит Господь Бог: Я поднял руку Мою с клятвою, что народы, которые вокруг вас, сами понесут срам свой.
\vs Eze 36:8 А вы, горы Израилевы, распустите ветви ваши и будете приносить плоды ваши народу Моему Израилю; ибо они скоро придут.
\vs Eze 36:9 Ибо вот, Я к вам обращусь, и вы будете возделываемы и засеваемы.
\vs Eze 36:10 И поселю на вас множество людей, весь дом Израилев, весь, и заселены будут города и застроены развалины.
\vs Eze 36:11 И умножу на вас людей и скот, и они будут плодиться и размножаться, и заселю вас, как было в прежние времена ваши, и буду благотворить вам больше, нежели в прежние времена ваши, и узнаете, что Я Господь.
\vs Eze 36:12 И приведу на вас людей, народ Мой, Израиля, и они будут владеть тобою, \bibemph{земля}! и ты будешь наследием их и не будешь более делать их бездетными.
\vs Eze 36:13 Так говорит Господь Бог: за то, что говорят о вас: <<ты~--- \bibemph{земля}, поедающая людей и делающая народ твой бездетным>>:
\vs Eze 36:14 за то уже не будешь поедать людей и народа твоего не будешь вперед делать бездетным, говорит Господь Бог.
\vs Eze 36:15 И не будешь более слышать посмеяния от народов, и поругания от племен не понесешь уже на себе, и народа твоего вперед не будешь делать бездетным, говорит Господь Бог.
\rsbpar\vs Eze 36:16 И было ко мне слово Господне:
\vs Eze 36:17 сын человеческий! когда дом Израилев жил на земле своей, он осквернял ее поведением своим и делами своими; путь их пред лицем Моим был как нечистота женщины во время очищения ее.
\vs Eze 36:18 И Я излил на них гнев Мой за кровь, которую они проливали на этой земле, и за то, что они оскверняли ее идолами своими.
\vs Eze 36:19 И Я рассеял их по народам, и они развеяны по землям; Я судил их по путям их и по делам их.
\vs Eze 36:20 И пришли они к народам, куда пошли, и обесславили святое имя Мое, потому что о них говорят: <<они~--- народ Господа, и вышли из земли Его>>.
\vs Eze 36:21 И пожалел Я святое имя Мое, которое обесславил дом Израилев у народов, куда пришел.
\vs Eze 36:22 Посему скажи дому Израилеву: так говорит Господь Бог: не для вас Я сделаю это, дом Израилев, а ради святаго имени Моего, которое вы обесславили у народов, куда пришли.
\vs Eze 36:23 И освящу великое имя Мое, бесславимое у народов, среди которых вы обесславили его, и узнают народы, что Я Господь, говорит Господь Бог, когда явлю на вас святость Мою перед глазами их.
\vs Eze 36:24 И возьму вас из народов, и соберу вас из всех стран, и приведу вас в землю вашу.
\vs Eze 36:25 И окроплю вас чистою водою, и вы очиститесь от всех скверн ваших, и от всех идолов ваших очищу вас.
\vs Eze 36:26 И дам вам сердце новое, и дух новый дам вам; и возьму из плоти вашей сердце каменное, и дам вам сердце плотяное.
\vs Eze 36:27 Вложу внутрь вас дух Мой и сделаю то, что вы будете ходить в заповедях Моих и уставы Мои будете соблюдать и выполнять.
\vs Eze 36:28 И будете жить на земле, которую Я дал отцам вашим, и будете Моим народом, и Я буду вашим Богом.
\vs Eze 36:29 И освобожу вас от всех нечистот ваших, и призову хлеб, и умножу его, и не дам вам терпеть голода.
\vs Eze 36:30 И умножу плоды на деревах и произведения полей, чтобы вперед не терпеть вам поношения от народов из-за голода.
\vs Eze 36:31 Тогда вспомните о злых путях ваших и недобрых делах ваших и почувствуете отвращение к самим себе за беззакония ваши и за мерзости ваши.
\vs Eze 36:32 Не ради вас Я сделаю это, говорит Господь Бог, да будет вам известно. Краснейте и стыдитесь путей ваших, дом Израилев.
\vs Eze 36:33 Так говорит Господь Бог: в тот день, когда очищу вас от всех беззаконий ваших и населю города, и обстроены будут развалины,
\vs Eze 36:34 и опустошенная земля будет возделываема, быв пустынею в глазах всякого мимоходящего,
\vs Eze 36:35 тогда скажут: <<эта опустелая земля сделалась, как сад Едемский; и эти развалившиеся и опустелые и разоренные города укреплены и населены>>.
\vs Eze 36:36 И узнают народы, которые останутся вокруг вас, что Я, Господь, вновь созидаю разрушенное, засаждаю опустелое. Я, Господь, сказал~--- и сделал.
\vs Eze 36:37 Так говорит Господь Бог: вот, еще и в том явлю милость Мою дому Израилеву, умножу их людьми как стадо.
\vs Eze 36:38 Как много бывает жертвенных овец в Иерусалиме во время праздников его, так полны будут людьми опустелые города, и узнают, что Я Господь.
\vs Eze 37:1 Была на мне рука Господа, и Господь вывел меня духом и поставил меня среди поля, и оно было полно костей,
\vs Eze 37:2 и обвел меня кругом около них, и вот весьма много их на поверхности поля, и вот они весьма сухи.
\vs Eze 37:3 И сказал мне: сын человеческий! оживут ли кости сии? Я сказал: Господи Боже! Ты знаешь это.
\vs Eze 37:4 И сказал мне: изреки пророчество на кости сии и скажи им: <<кости сухие! слушайте слово Господне!>>
\vs Eze 37:5 Так говорит Господь Бог костям сим: вот, Я введу дух в вас, и оживете.
\vs Eze 37:6 И обложу вас жилами, и выращу на вас плоть, и покрою вас кожею, и введу в вас дух, и оживете, и узнаете, что Я Господь.
\vs Eze 37:7 Я изрек пророчество, как повелено было мне; и когда я пророчествовал, произошел шум, и вот движение, и стали сближаться кости, кость с костью своею.
\vs Eze 37:8 И видел я: и вот, жилы были на них, и плоть выросла, и кожа покрыла их сверху, а духа не было в них.
\vs Eze 37:9 Тогда сказал Он мне: изреки пророчество духу, изреки пророчество, сын человеческий, и скажи духу: так говорит Господь Бог: от четырех ветров приди, дух, и дохни на этих убитых, и они оживут.
\vs Eze 37:10 И я изрек пророчество, как Он повелел мне, и вошел в них дух, и они ожили, и стали на ноги свои~--- весьма, весьма великое полчище.
\vs Eze 37:11 И сказал Он мне: сын человеческий! кости сии~--- весь дом Израилев. Вот, они говорят: <<иссохли кости наши, и погибла надежда наша, мы оторваны от корня>>.
\vs Eze 37:12 Посему изреки пророчество и скажи им: так говорит Господь Бог: вот, Я открою гробы ваши и выведу вас, народ Мой, из гробов ваших и введу вас в землю Израилеву.
\vs Eze 37:13 И узнаете, что Я Господь, когда открою гробы ваши и выведу вас, народ Мой, из гробов ваших,
\vs Eze 37:14 и вложу в вас дух Мой, и оживете, и помещу вас на земле вашей, и узнаете, что Я, Господь, сказал это~--- и сделал, говорит Господь.
\rsbpar\vs Eze 37:15 И было ко мне слово Господне:
\vs Eze 37:16 ты же, сын человеческий, возьми себе один жезл и напиши на нем: <<Иуде и сынам Израилевым, союзным с ним>>; и еще возьми жезл и напиши на нем: <<Иосифу>>; это жезл Ефрема и всего дома Израилева, союзного с ним.
\vs Eze 37:17 И сложи их у себя один с другим в один жезл, чтобы они в руке твоей были одно.
\vs Eze 37:18 И когда спросят у тебя сыны народа твоего: <<не объяснишь ли нам, что это у тебя?>>,
\vs Eze 37:19 тогда скажи им: так говорит Господь Бог: вот, Я возьму жезл Иосифов, который в руке Ефрема и союзных с ним колен Израилевых, и приложу их к нему, к жезлу Иуды, и сделаю их одним жезлом, и будут одно в руке Моей.
\vs Eze 37:20 Когда же оба жезла, на которых ты напишешь, будут в руке твоей перед глазами их,
\vs Eze 37:21 то скажи им: так говорит Господь Бог: вот, Я возьму сынов Израилевых из среды народов, между которыми они находятся, и соберу их отовсюду и приведу их в землю их.
\vs Eze 37:22 На этой земле, на горах Израиля Я сделаю их одним народом, и один Царь будет царем у всех их, и не будут более двумя народами, и уже не будут вперед разделяться на два царства.
\vs Eze 37:23 И не будут уже осквернять себя идолами своими и мерзостями своими и всякими пороками своими, и освобожу их из всех мест жительства их, где они грешили, и очищу их, и будут Моим народом, и Я буду их Богом.
\vs Eze 37:24 А раб Мой Давид будет Царем над ними и Пастырем всех их, и они будут ходить в заповедях Моих, и уставы Мои будут соблюдать и выполнять их.
\vs Eze 37:25 И будут жить на земле, которую Я дал рабу Моему Иакову, на которой жили отцы их; там будут жить они и дети их, и дети детей их во веки; и раб Мой Давид будет князем у них вечно.
\vs Eze 37:26 И заключу с ними завет мира, завет вечный будет с ними. И устрою их, и размножу их, и поставлю среди них святилище Мое на веки.
\vs Eze 37:27 И будет у них жилище Мое, и буду их Богом, а они будут Моим народом.
\vs Eze 37:28 И узнают народы, что Я Господь, освящающий Израиля, когда святилище Мое будет среди них во веки.
\vs Eze 38:1 И было ко мне слово Господне:
\vs Eze 38:2 сын человеческий! обрати лице твое к Гогу в земле Магог, князю Роша, Мешеха и Фувала, и изреки на него пророчество
\vs Eze 38:3 и скажи: так говорит Господь Бог: вот, Я~--- на тебя, Гог, князь Роша, Мешеха и Фувала!
\vs Eze 38:4 И поверну тебя, и вложу удила в челюсти твои, и выведу тебя и все войско твое, коней и всадников, всех в полном вооружении, большое полчище, в бронях и со щитами, всех вооруженных мечами,
\vs Eze 38:5 Персов, Ефиоплян и Ливийцев с ними, всех со щитами и в шлемах,
\vs Eze 38:6 Гомера со всеми отрядами его, дом Фогарма, от пределов севера, со всеми отрядами его, многие народы с тобою.
\vs Eze 38:7 Готовься и снаряжайся, ты и все полчища твои, собравшиеся к тебе, и будь им вождем.
\vs Eze 38:8 После многих дней ты понадобишься; в последние годы ты придешь в землю, избавленную от меча, собранную из многих народов, на горы Израилевы, которые были в постоянном запустении, но теперь жители ее будут возвращены из народов, и все они будут жить безопасно.
\vs Eze 38:9 И поднимешься, как буря, пойдешь, как туча, чтобы покрыть землю, ты и все полчища твои и многие народы с тобою.
\vs Eze 38:10 Так говорит Господь Бог: в тот день придут тебе на сердце мысли, и ты задумаешь злое предприятие
\vs Eze 38:11 и скажешь: <<поднимусь я на землю неогражденную, пойду на беззаботных, живущих беспечно,~--- все они живут без стен, и нет у них ни запоров, ни дверей,~---
\vs Eze 38:12 чтобы произвести грабеж и набрать добычи, наложить руку на вновь заселенные развалины и на народ, собранный из народов, занимающийся хозяйством и торговлею, живущий на вершине земли>>.
\vs Eze 38:13 Сава и Дедан и купцы Фарсисские со всеми молодыми львами их скажут тебе: <<ты пришел, чтобы произвести грабеж, собрал полчище твое, чтобы набрать добычи, взять серебро и золото, отнять скот и имущество, захватить большую добычу?>>
\vs Eze 38:14 Посему изреки пророчество, сын человеческий, и скажи Гогу: так говорит Господь Бог: не так ли? в тот день, когда народ Мой Израиль будет жить безопасно, ты узнаешь это;
\vs Eze 38:15 и пойдешь с места твоего, от пределов севера, ты и многие народы с тобою, все сидящие на конях, сборище великое и войско многочисленное.
\vs Eze 38:16 И поднимешься на народ Мой, на Израиля, как туча, чтобы покрыть землю: это будет в последние дни, и Я приведу тебя на землю Мою, чтобы народы узнали Меня, когда Я над тобою, Гог, явлю святость Мою пред глазами их.
\vs Eze 38:17 Так говорит Господь Бог: не ты ли тот самый, о котором Я говорил в древние дни чрез рабов Моих, пророков Израилевых, которые пророчествовали в те времена, что Я приведу тебя на них?
\vs Eze 38:18 И будет в тот день, когда Гог придет на землю Израилеву, говорит Господь Бог, гнев Мой воспылает в ярости Моей.
\vs Eze 38:19 И в ревности Моей, в огне негодования Моего Я сказал: истинно в тот день произойдет великое потрясение на земле Израилевой.
\vs Eze 38:20 И вострепещут от лица Моего рыбы морские и птицы небесные, и звери полевые и все пресмыкающееся, ползающее по земле, и все люди, которые на лице земли, и обрушатся горы, и упадут утесы, и все стены падут на землю.
\vs Eze 38:21 И по всем горам Моим призову меч против него, говорит Господь Бог; меч каждого человека будет против брата его.
\vs Eze 38:22 И буду судиться с ним моровою язвою и кровопролитием, и пролью на него и на полки его и на многие народы, которые с ним, всепотопляющий дождь и каменный град, огонь и серу;
\vs Eze 38:23 и покажу Мое величие и святость Мою, и явлю Себя пред глазами многих народов, и узнают, что Я Господь.
\vs Eze 39:1 Ты же, сын человеческий, изреки пророчество на Гога и скажи: так говорит Господь Бог: вот, Я~--- на тебя, Гог, князь Роша, Мешеха и Фувала!
\vs Eze 39:2 И поверну тебя, и поведу тебя, и выведу тебя от краев севера, и приведу тебя на горы Израилевы.
\vs Eze 39:3 И выбью лук твой из левой руки твоей, и выброшу стрелы твои из правой руки твоей.
\vs Eze 39:4 Падешь ты на горах Израилевых, ты и все полки твои, и народы, которые с тобою; отдам тебя на съедение всякого рода хищным птицам и зверям полевым.
\vs Eze 39:5 На открытом поле падешь; ибо Я сказал это, говорит Господь Бог.
\vs Eze 39:6 И пошлю огонь на землю Магог и на жителей островов, живущих беспечно, и узнают, что Я Господь.
\vs Eze 39:7 И явлю святое имя Мое среди народа Моего, Израиля, и не дам вперед бесславить святаго имени Моего, и узнают народы, что Я Господь, Святый в Израиле.
\vs Eze 39:8 Вот, это придет и сбудется, говорит Господь Бог,~--- это тот день, о котором Я сказал.
\vs Eze 39:9 Тогда жители городов Израилевых выйдут, и разведут огонь, и будут сожигать оружие, щиты и латы, луки и стрелы, и булавы и копья; семь лет буду жечь их.
\vs Eze 39:10 И не будут носить дров с поля, ни рубить из лесов, но будут жечь только оружие; и ограбят грабителей своих, и оберут обирателей своих, говорит Господь Бог.
\vs Eze 39:11 И будет в тот день: дам Гогу место для могилы в Израиле, долину прохожих на восток от моря, и она будет задерживать прохожих; и похоронят там Гога и все полчище его, и будут называть ее долиною полчища Гогова.
\vs Eze 39:12 И дом Израилев семь месяцев будет хоронить их, чтобы очистить землю.
\vs Eze 39:13 И весь народ земли будет хоронить \bibemph{их}, и знаменит будет у них день, в который Я прославлю Себя, говорит Господь Бог.
\vs Eze 39:14 И назначат людей, которые постоянно обходили бы землю и с помощью прохожих погребали бы оставшихся на поверхности земли, для очищения ее; по прошествии семи месяцев они начнут делать поиски;
\vs Eze 39:15 и когда кто из обходящих землю увидит кость человеческую, то поставит возле нее знак, доколе погребатели не похоронят ее в долине полчища Гогова.
\vs Eze 39:16 И будет имя городу: Гамона [полчище]. И так очистят они землю.
\vs Eze 39:17 Ты же, сын человеческий, так говорит Господь Бог, скажи всякого рода птицам и всем зверям полевым: собирайтесь и идите, со всех сторон сходитесь к жертве Моей, которую Я заколю для вас, к великой жертве на горах Израилевых; и будете есть мясо и пить кровь.
\vs Eze 39:18 Мясо мужей сильных будете есть, и будете пить кровь князей земли, баранов, ягнят, козлов и тельцов, всех откормленных на Васане;
\vs Eze 39:19 и будете есть жир до сытости и пить кровь до опьянения от жертвы Моей, которую Я заколю для вас.
\vs Eze 39:20 И насытитесь за столом Моим конями и всадниками, мужами сильными и всякими людьми военными, говорит Господь Бог.
\vs Eze 39:21 И явлю славу Мою между народами, и все народы увидят суд Мой, который Я произведу, и руку Мою, которую Я наложу на них.
\vs Eze 39:22 И будет знать дом Израилев, что Я Господь Бог их, от сего дня и далее.
\vs Eze 39:23 И узнают народы, что дом Израилев был переселен за неправду свою; за то, что они поступали вероломно предо Мною, Я сокрыл от них лице Мое и отдал их в руки врагов их, и все они пали от меча.
\vs Eze 39:24 За нечистоты их и за их беззаконие Я сделал это с ними, и сокрыл от них лице Мое.
\vs Eze 39:25 Посему так говорит Господь Бог: ныне возвращу плен Иакова, и помилую весь дом Израиля, и возревную по святом имени Моем.
\vs Eze 39:26 И почувствуют они бесчестие свое и все беззакония свои, какие делали предо Мною, когда будут жить на земле своей безопасно, и никто не будет устрашать их,
\vs Eze 39:27 когда Я возвращу их из народов, и соберу их из земель врагов их, и явлю в них святость Мою пред глазами многих народов.
\vs Eze 39:28 И узнают, что Я Господь Бог их, когда, рассеяв их между народами, опять соберу их в землю их и не оставлю уже там ни одного из них;
\vs Eze 39:29 и не буду уже скрывать от них лица Моего, потому что Я изолью дух Мой на дом Израилев, говорит Господь Бог.
\vs Eze 40:1 В двадцать пятом году по переселении нашем, в начале года, в десятый \bibemph{день} месяца, в четырнадцатом году по разрушении города, в тот самый день была на мне рука Господа, и Он повел меня туда.
\vs Eze 40:2 В видениях Божиих привел Он меня в землю Израилеву и поставил меня на весьма высокой горе, и на ней, с южной стороны, были как бы городские здания;
\vs Eze 40:3 и привел меня туда. И вот муж, которого вид как бы вид блестящей меди, и льняная вервь в руке его и трость измерения, и стоял он у ворот.
\vs Eze 40:4 И сказал мне этот муж: <<сын человеческий! смотри глазами твоими и слушай ушами твоими, и прилагай сердце твое ко всему, что я буду показывать тебе, ибо ты для того и приведен сюда, чтоб я показал тебе \bibemph{это}; все, что увидишь, возвести дому Израилеву>>.
\vs Eze 40:5 И вот, вне храма стена со всех сторон \bibemph{его}, и в руке того мужа трость измерения в шесть локтей, \bibemph{считая каждый локоть} в локоть с ладонью; и намерил он в этом здании одну трость толщины и одну трость вышины.
\vs Eze 40:6 Потом пошел к воротам, обращенным лицом к востоку, и взошел по ступеням их, и нашел меры в одном пороге ворот одну трость ширины и в другом пороге одну трость ширины.
\vs Eze 40:7 И в каждой боковой комнате одна трость длины и одна трость ширины, а между комнатами пять локтей, и в пороге ворот у притвора ворот внутри одна же трость.
\vs Eze 40:8 И смерил он в притворе ворот внутри одну трость,
\vs Eze 40:9 а в притворе у ворот намерил восемь локтей и два локтя в столбах. Этот притвор у ворот со стороны храма.
\vs Eze 40:10 Боковых комнат у восточных ворот три~--- с одной стороны и три~--- с другой; одна мера во всех трех и одна мера в столбах с той и другой стороны.
\vs Eze 40:11 Ширины в отверстии ворот он намерил десять локтей, а длины ворот тринадцать локтей.
\vs Eze 40:12 А перед комнатами выступ в один локоть, и в один же локоть с другой стороны выступ; эти комнаты с одной стороны \bibemph{имели} шесть локтей и шесть же локтей с другой стороны.
\vs Eze 40:13 Потом намерил он в воротах от крыши одной комнаты до крыши другой двадцать пять локтей ширины; дверь была против двери.
\vs Eze 40:14 А в столбах он насчитал шестьдесят локтей, в каждом столбе около двора и у ворот,
\vs Eze 40:15 и от передней стороны входа в ворота до передней стороны внутренних ворот пятьдесят локтей.
\vs Eze 40:16 Решетчатые окна были и в боковых комнатах и в столбах их, внутрь ворот кругом, также и в притворах окна были кругом на внутреннюю сторону, и на столбах~--- пальмы.
\vs Eze 40:17 И привел он меня на внешний двор, и вот там комнаты, и каменный помост кругом двора был сделан; тридцать комнат на том помосте.
\vs Eze 40:18 И помост этот был по бокам ворот, соответственно длине ворот; этот помост был ниже.
\vs Eze 40:19 И намерил он в ширину от нижних ворот до внешнего края внутреннего двора сто локтей, к востоку и к северу.
\vs Eze 40:20 Он измерил также длину и ширину ворот внешнего двора, обращенных лицом к северу,
\vs Eze 40:21 и боковые комнаты при них, три с одной стороны и три с другой; и столбы их, и выступы их были такой же меры, как у прежних ворот: длина их пятьдесят локтей, а ширина двадцать пять локтей.
\vs Eze 40:22 И окна их, и выступы их, и пальмы их~--- той же меры, как у ворот, обращенных лицом к востоку; и входят к ним семью ступенями, и перед ними выступы.
\vs Eze 40:23 И во внутренний двор есть ворота против ворот северных и восточных; и намерил он от ворот до ворот сто локтей.
\vs Eze 40:24 И повел меня на юг, и вот там ворота южные; и намерил он в столбах и выступах такую же меру.
\vs Eze 40:25 И окна в них и в преддвериях их такие же, как те окна: длины пятьдесят локтей, а ширины двадцать пять локтей.
\vs Eze 40:26 Подъем к ним~--- в семь ступеней, и преддверия перед ними; и пальмовые украшения~--- одно с той стороны и одно с другой на столбах их.
\vs Eze 40:27 И во внутренний двор были южные ворота; и намерил он от ворот до ворот южных сто локтей.
\vs Eze 40:28 И привел он меня через южные ворота во внутренний двор; и намерил в южных воротах ту же меру.
\vs Eze 40:29 И боковые комнаты их, и столбы их, и притворы их~--- той же меры, и окна в них в притворах их были кругом; всего в длину пятьдесят локтей, а в ширину двадцать пять локтей.
\vs Eze 40:30 Притворы были кругом длиною в двадцать пять локтей, а шириною в пять локтей.
\vs Eze 40:31 И притворы были у них на внешний двор, и пальмы были на столбах их; подъем к ним~--- в восемь ступеней.
\vs Eze 40:32 И повел меня восточными воротами на внутренний двор; и намерил в этих воротах ту же меру.
\vs Eze 40:33 И боковые комнаты их, и столбы их, и притворы их были той же меры; и окна в них и притворах их были кругом; длина пятьдесят локтей, а ширина двадцать пять локтей.
\vs Eze 40:34 Притворы у них были на внешний двор, и пальмы на столбах их с той и другой стороны; подъем к ним~--- в восемь ступеней.
\vs Eze 40:35 Потом привел меня к северным воротам, и намерил в них ту же меру.
\vs Eze 40:36 Боковые комнаты при них, столбы их и притворы их, и окна в них были кругом; всего в длину пятьдесят локтей, и в ширину двадцать пять локтей.
\vs Eze 40:37 Притворы у них были на внешний двор, и пальмы на столбах их с той и с другой стороны; подъем к ним~--- в восемь ступеней.
\vs Eze 40:38 Была также комната, со входом в нее, у столбов ворот: там омывают жертвы всесожжения.
\vs Eze 40:39 А в притворе у ворот два стола с одной стороны и два с другой стороны, чтобы заколать на них жертвы всесожжения и жертвы за грех и жертвы за преступление.
\vs Eze 40:40 И у наружного бока при входе в отверстие северных ворот были два стола, и у другого бока, подле притвора у ворот, два стола.
\vs Eze 40:41 Четыре стола с одной стороны и четыре стола с другой стороны, по бокам ворот: \bibemph{всего} восемь столов, на которых заколают \bibemph{жертвы}.
\vs Eze 40:42 И четыре стола для приготовления всесожжения были из тесаных камней, длиною в полтора локтя, и шириною в полтора локтя, а вышиною в один локоть; на них кладут орудия для заклания жертвы всесожжения и \bibemph{других} жертв.
\vs Eze 40:43 И крюки в одну ладонь приделаны были к стенам здания кругом, а на столах клали жертвенное мясо.
\vs Eze 40:44 Снаружи внутренних ворот были комнаты для певцов; на внутреннем дворе, сбоку северных ворот, одна обращена лицом к югу, а другая, сбоку южных ворот, обращена лицом к северу.
\vs Eze 40:45 И сказал он мне: <<эта комната, которая лицом к югу, для священников, бодрствующих на страже храма;
\vs Eze 40:46 а комната, которая лицом к северу, для священников, бодрствующих на страже жертвенника: это сыны Садока, которые одни из сынов Левия приближаются к Господу, чтобы служить Ему>>.
\vs Eze 40:47 И намерил он во дворе сто локтей длины и сто локтей ширины: \bibemph{он} был четыреугольный; а перед храмом стоял жертвенник.
\vs Eze 40:48 И привел он меня к притвору храма, и намерил в столбах притвора пять локтей с одной стороны и пять локтей с другой; а в воротах три локтя ширины с одной стороны и три локтя с другой.
\vs Eze 40:49 Длина притвора~--- в двадцать локтей, а ширина~--- в одиннадцать локтей, и всходят в него по десяти ступеням; и были подпоры у столбов, одна с одной стороны, а другая с другой.
\vs Eze 41:1 Потом ввел меня в храм и намерил в столбах шесть локтей ширины с одной стороны и шесть локтей ширины с другой стороны, в ширину скинии.
\vs Eze 41:2 В дверях десять локтей ширины, и по бокам дверей пять локтей с одной стороны и пять локтей с другой стороны; и намерил длины в храме сорок локтей, а ширины двадцать локтей.
\vs Eze 41:3 И пошел внутрь, и намерил в столбах у дверей два локтя и в дверях шесть локтей, а ширина двери~--- в семь локтей.
\vs Eze 41:4 И отмерил в нем двадцать локтей в длину и двадцать локтей в ширину храма, и сказал мне: <<это~--- Святое Святых>>.
\vs Eze 41:5 И намерил в стене храма шесть локтей, а ширины в боковых комнатах, кругом храма, по четыре локтя.
\vs Eze 41:6 Боковых комнат было тридцать три, комната подле комнаты; они вдаются в стену, которая у храма для комнат кругом, так что они в связи с нею, но стен\acc{ы} самого храма не касаются.
\vs Eze 41:7 И он более и более расширялся кругом вверх боковыми комнатами, потому что окружность храма восходила выше и выше вокруг храма, и потому храм имел б\acc{о}льшую ширину вверху, и из нижнего этажа восходили в верхний через средний.
\vs Eze 41:8 И я видел верх дома во всю окружность; боковые комнаты в основании имели там меры цельную трость, шесть полных локтей.
\vs Eze 41:9 Ширина стены боковых комнат, выходящих наружу, пять локтей, и открытое пространство есть подле боковых комнат храма.
\vs Eze 41:10 И между комнатами расстояние двадцать локтей кругом всего храма.
\vs Eze 41:11 Двери боковых комнат \bibemph{ведут} на открытое пространство, одни двери~--- на северную сторону, а другие двери~--- на южную сторону; а ширина этого открытого пространства~--- пять локтей кругом.
\vs Eze 41:12 Здание перед площадью на западной стороне~--- шириною в семьдесят локтей; стена же этого здания~--- в пять локтей ширины кругом, а длина ее~--- девяносто локтей.
\vs Eze 41:13 И намерил он в храме сто локтей длины, и в площади и в пристройке, и в стенах его также сто локтей длины.
\vs Eze 41:14 И ширина храма по лицевой стороне и площади к востоку сто же локтей.
\vs Eze 41:15 И в длине здания перед площадью на задней стороне ее с боковыми комнатами его по ту и другую сторону он намерил сто локтей, со внутренностью храма и притворами двора.
\vs Eze 41:16 Дверные брусья и решетчатые окна, и боковые комнаты кругом, во всех трех \bibemph{ярусах}, против порогов обшиты деревом и от пола по окна; окна были закрыты.
\vs Eze 41:17 От верха дверей как внутри храма, так и снаружи, и по всей стене кругом, внутри и снаружи, были резные изображения,
\vs Eze 41:18 сделаны были херувимы и пальмы: пальма между двумя херувимами, и у каждого херувима два лица.
\vs Eze 41:19 С одной стороны к пальме обращено лицо человеческое, а с другой стороны к пальме~--- лице львиное; так сделано во всем храме кругом.
\vs Eze 41:20 От пола до верха дверей сделаны были херувимы и пальмы, также и по стене храма.
\vs Eze 41:21 В храме были четырехугольные дверные косяки, и святилище имело такой же вид, как я видел.
\vs Eze 41:22 Жертвенник был деревянный в три локтя вышины и в два локтя длины; и углы его, и подножие его, и стенки его~--- из дерева. И сказал он мне: <<это трапеза, которая пред Господом>>.
\vs Eze 41:23 В храме и во святилище по две двери,
\vs Eze 41:24 и двери сии о двух досках, обе доски подвижные, две у одной двери и две доски у другой;
\vs Eze 41:25 и сделаны на них, на дверях храма, херувимы и пальмы такие же, какие сделаны по стенам; а перед притвором снаружи был деревянный помост.
\vs Eze 41:26 И решетчатые окна с пальмами, по ту и другую сторону, были по бокам притвора и в боковых комнатах храма и на деревянной обшивке.
\vs Eze 42:1 И вывел меня ко внешнему двору северною дорогою, и привел меня к комнатам, которые против площади и против здания на севере,
\vs Eze 42:2 к тому месту, которое у северных дверей имеет в длину сто локтей, а в ширину пятьдесят локтей.
\vs Eze 42:3 Напротив двадцати \bibemph{локтей} внутреннего двора и напротив помоста, который на внешнем дворе, были галерея против галереи в три яруса.
\vs Eze 42:4 А перед комнатами ход в десять локтей ширины, а внутрь в один локоть; двери их лицом к северу.
\vs Eze 42:5 Верхние комнаты \acc{у}же, потому что галереи отнимают у них несколько против нижних и средних \bibemph{комнат} этого здания.
\vs Eze 42:6 Они в три яруса, и таких столбов, какие на дворах, нет у них; потому они и сделаны \acc{у}же против нижних и средних комнат, начиная от пола.
\vs Eze 42:7 А наружная стена напротив этих комнат от внешнего двора, составляющая лицевую сторону комнат, имеет длины пятьдесят локтей;
\vs Eze 42:8 потому что \bibemph{и} комнаты на внешнем дворе занимают длины только пятьдесят локтей, и вот перед храмом сто локтей.
\vs Eze 42:9 А снизу ход к этим комнатам с восточной стороны, когда подходят к ним со внешнего двора.
\vs Eze 42:10 В ширину стены двора к востоку перед площадью и перед зданием были комнаты.
\vs Eze 42:11 И ход перед ними такой же, как и у тех комнат, которые обращены к северу, такая же длина, как и у тех, и такая же ширина, и все выходы их, и устройство их, и двери их такие же, как и у тех.
\vs Eze 42:12 Такие же двери, как и у комнат, которые на юг, и для входа в них дверь у самой дороги, которая шла прямо вдоль стены на восток.
\vs Eze 42:13 И сказал он мне: <<комнаты на север \bibemph{и} комнаты на юг, которые перед площадью, суть комнаты священные, в которых священники, приближающиеся к Господу, съедают священнейшие жертвы; там же они кладут священнейшие жертвы, и хлебное приношение, и жертву за грех, и жертву за преступление, ибо это место святое.
\vs Eze 42:14 Когда войдут \bibemph{туда} священники, то они не должны выходить из этого святаго места на внешний двор, доколе не оставят там одежд своих, в которых служили, ибо они священны; они должны надеть на себя другие одежды и тогда выходить к народу>>.
\vs Eze 42:15 Когда кончил он измерения внутреннего храма, то вывел меня воротами, обращенными лицом к востоку, и стал измерять его кругом.
\vs Eze 42:16 Он измерил восточную сторону тростью измерения и \bibemph{намерил} тростью измерения всего пятьсот тростей;
\vs Eze 42:17 в северной стороне той же тростью измерения намерил всего пятьсот тростей;
\vs Eze 42:18 в южной стороне намерил тростью измерения также пятьсот тростей.
\vs Eze 42:19 Поворотив к западной стороне, намерил тростью измерения пятьсот тростей.
\vs Eze 42:20 Со всех четырех сторон он измерил его; кругом него была стена длиною в пятьсот \bibemph{тростей} и в пятьсот \bibemph{тростей} шириною, чтобы отделить святое место от несвятого.
\vs Eze 43:1 И привел меня к воротам, к тем воротам, которые обращены лицом к востоку.
\vs Eze 43:2 И вот, слава Бога Израилева шла от востока, и глас Его~--- как шум вод многих, и земля осветилась от славы Его.
\vs Eze 43:3 Это видение было такое же, какое я видел прежде, точно такое, какое я видел, когда приходил возвестить гибель городу, и видения, подобные видениям, какие видел я у реки Ховара. И я пал на лице мое.
\vs Eze 43:4 И слава Господа вошла в храм путем ворот, обращенных лицом к востоку.
\rsbpar\vs Eze 43:5 И поднял меня дух, и ввел меня во внутренний двор, и вот, слава Господа наполнила весь храм.
\vs Eze 43:6 И я слышал кого-то, говорящего мне из храма, а тот муж стоял подле меня,
\vs Eze 43:7 и сказал мне: сын человеческий! это место престола Моего и место стопам ног Моих, где Я буду жить среди сынов Израилевых во веки; и дом Израилев не будет более осквернять святаго имени Моего, ни они, ни цари их, блужением своим и трупами царей своих на высотах их.
\vs Eze 43:8 Они ставили порог свой у порога Моего и вереи дверей своих подле Моих верей, так что одна стена \bibemph{была} между Мною и ими, и оскверняли святое имя Мое мерзостями своими, какие делали, и за то Я погубил их во гневе Моем.
\vs Eze 43:9 А теперь они удалят от Меня блужение свое и трупы царей своих, и Я буду жить среди них во веки.
\vs Eze 43:10 Ты, сын человеческий, возвести дому Израилеву о храме сем, чтобы они устыдились беззаконий своих и чтобы сняли с него меру.
\vs Eze 43:11 И если они устыдятся всего того, что делали, то покажи им вид храма и расположение его, и выходы его, и входы его, и все очертания его, и все уставы его, и все образы его, и все законы его, и напиши при глазах их, чтобы они сохраняли все очертания его и все уставы его и поступали по ним.
\vs Eze 43:12 Вот закон храма: на вершине горы все пространство его вокруг~--- Святое Святых; вот закон храма!
\vs Eze 43:13 И вот размеры жертвенника локтями, \bibemph{считая} локоть в локоть с ладонью: основание в локоть, ширина в локоть же, и пояс по всем краям его в одну пядень; и вот задняя сторона жертвенника.
\vs Eze 43:14 От основания, что в земле, до нижнего выступа два локтя, а шириною он в один локоть; от малого выступа до большого выступа четыре локтя, а ширина его~--- в один локоть.
\vs Eze 43:15 Самый жертвенник вышиною в четыре локтя; и из жертвенника \bibemph{поднимаются} вверх четыре рога.
\vs Eze 43:16 Жертвенник имеет двенадцать \bibemph{локтей} длины \bibemph{и} двенадцать ширины; он четырехугольный на все свои четыре стороны.
\vs Eze 43:17 А в площадке четырнадцать \bibemph{локтей} длины и четырнадцать ширины на все четыре стороны ее, и вокруг нее пояс в пол-локтя, а основание ее в локоть вокруг, ступени же к нему~--- с востока.
\vs Eze 43:18 И сказал он мне: сын человеческий! так говорит Господь Бог: вот уставы жертвенника к тому дню, когда он будет сделан для приношения на нем всесожжений и для кропления на него кровью.
\vs Eze 43:19 Священникам от колена Левиина, которые из племени Садока, приближающимся ко Мне, чтобы служить Мне, говорит Господь Бог, дай тельца из стада волов, в жертву за грех.
\vs Eze 43:20 И возьми крови его, и покропи на четыре рога его, и на четыре угла площадки, и на пояс кругом, и так очисти его и освяти его.
\vs Eze 43:21 И возьми тельца, \bibemph{в жертву} за грех, и сожги его на назначенном месте дома вне святилища.
\vs Eze 43:22 А на другой день в жертву за грех принеси из козьего стада козла без порока, и пусть очистят жертвенник так же, как очищали тельцом.
\vs Eze 43:23 Когда же кончишь очищение, приведи из стада волов тельца без порока и из стада овец овна без порока;
\vs Eze 43:24 и принеси их пред лице Господа; и священники бросят на них соли, и вознесут их во всесожжение Господу.
\vs Eze 43:25 Семь дней приноси в жертву за грех по козлу в день; также пусть приносят в жертву по тельцу из стада волов и по овну из стада овец без порока.
\vs Eze 43:26 Семь дней они должны очищать жертвенник и освящать его и наполнять руки свои.
\vs Eze 43:27 По окончании же сих дней, в восьмой день и далее, священники будут возносить на жертвеннике ваши всесожжения и благодарственные жертвы; и Я буду милостив к вам, говорит Господь Бог.
\vs Eze 44:1 И привел он меня обратно ко внешним воротам святилища, обращенным лицом на восток, и они были затворены.
\vs Eze 44:2 И сказал мне Господь: ворота сии будут затворены, не отворятся, и никакой человек не войдет ими, ибо Господь, Бог Израилев, вошел ими, и они будут затворены.
\vs Eze 44:3 Что до князя, он, \bibemph{как} князь, сядет в них, чтобы есть хлеб пред Господом; войдет путем притвора этих ворот, и тем же путем выйдет.
\vs Eze 44:4 Потом привел меня путем ворот северных перед лице храма, и я видел, и вот, слава Господа наполняла дом Господа, и пал я на лице мое.
\vs Eze 44:5 И сказал мне Господь: сын человеческий! прилагай сердце твое \bibemph{ко всему}, и смотри глазами твоими, и слушай ушами твоими все, что Я говорю тебе о всех постановлениях дома Господа и всех законах его; и прилагай сердце твое ко входу в храм и ко всем выходам из святилища.
\vs Eze 44:6 И скажи мятежному дому Израилеву: так говорит Господь Бог: довольно вам, дом Израилев, делать все мерзости ваши,
\vs Eze 44:7 вводить сынов чужой, необрезанных сердцем и необрезанных плотью, чтобы они были в Моем святилище и оскверняли храм Мой, подносить хлеб Мой, тук и кровь, и разрушать завет Мой всякими мерзостями вашими.
\vs Eze 44:8 Вы не исполняли стражи у святынь Моих, а ставили вместо себя их для стражи в Моем святилище.
\vs Eze 44:9 Так говорит Господь Бог: никакой сын чужой, необрезанный сердцем и необрезанный плотью, не должен входить во святилище Мое, даже и тот сын чужой, который \bibemph{живет} среди сынов Израиля.
\vs Eze 44:10 Равно и левиты, которые удалились от Меня во время отступничества Израилева, которые, оставив Меня, блуждали вслед идолов своих, понесут наказание за вину свою.
\vs Eze 44:11 Они будут служить во святилище Моем, как сторожа у ворот храма и прислужники у храма; они будут заколать для народа всесожжение и другие жертвы, и будут стоять пред ними для служения им.
\vs Eze 44:12 За то, что они служили им пред идолами их и были для дома Израилева соблазном к нечестию, Я поднял на них руку Мою, говорит Господь Бог, и они понесут наказание за вину свою;
\vs Eze 44:13 они не будут приближаться ко Мне, чтобы священнодействовать предо Мною и приступать ко всем святыням Моим, к Святому Святых, но будут нести на себе бесславие свое и мерзости свои, какие делали.
\vs Eze 44:14 Сделаю их стражами храма для всех служб его и для всего, что производится в нем.
\vs Eze 44:15 А священники из колена Левиина, сыны Садока, которые, во время отступления сынов Израилевых от Меня, постоянно стояли на страже святилища Моего, те будут приближаться ко Мне, чтобы служить Мне, и будут предстоять пред лицем Моим, чтобы приносить Мне тук и кровь, говорит Господь Бог.
\vs Eze 44:16 Они будут входить во святилище Мое и приближаться к трапезе Моей, чтобы служить Мне и соблюдать стражу Мою.
\vs Eze 44:17 Когда придут к воротам внутреннего двора, тогда оденутся в одежды льняные, а шерстяное не должно быть на них во время служения их в воротах внутреннего двора и внутри храма.
\vs Eze 44:18 Увясла на головах их должны быть также льняные; и исподняя одежда на чреслах их должна быть также льняная; в поту они не должны опоясываться.
\vs Eze 44:19 А когда надобно будет выйти на внешний двор, на внешний двор к народу, тогда они должны будут снять одежды свои, в которых они служили, и оставить их в священных комнатах, и одеться в другие одежды, чтобы священными одеждами своими не прикасаться к народу.
\vs Eze 44:20 И головы своей они не должны брить, и не должны отпускать волос, а пусть непременно стригут головы свои.
\vs Eze 44:21 И вина не должен пить ни один священник, когда идет во внутренний двор.
\vs Eze 44:22 Ни вдовы, ни разведенной с мужем они не должны брать себе в жены, а только могут брать себе девиц из племени дома Израилева и вдову, оставшуюся вдовою от священника.
\vs Eze 44:23 Они должны учить народ Мой отличать священное от несвященного и объяснять им, чт\acc{о} нечисто и чт\acc{о} чисто.
\vs Eze 44:24 При спорных делах они должны присутствовать в суде, и по уставам Моим судить их, и наблюдать законы Мои и постановления Мои о всех праздниках Моих, и свято хранить субботы Мои.
\vs Eze 44:25 К мертвому человеку никто из них не должен подходить, чтобы не сделаться нечистым; только ради отца и матери, ради сына и дочери, брата и сестры, которая не была замужем, можно им сделать себя нечистыми.
\vs Eze 44:26 По очищении же такого, еще семь дней надлежит отсчитать ему.
\vs Eze 44:27 И в тот день, когда ему надобно будет приступать ко святыне во внутреннем дворе, чтобы служить при святыне, он должен принести жертву за грех, говорит Господь Бог.
\vs Eze 44:28 А что до удела их, то Я их удел. И владения не давайте им в Израиле: Я их владение.
\vs Eze 44:29 Они будут есть от хлебного приношения, от жертвы за грех и жертвы за преступление; и все заклятое у Израиля им же принадлежит.
\vs Eze 44:30 И начатки из всех плодов ваших и всякого рода приношения, из чего ни состояли бы приношения ваши, принадлежат священникам; и начатки молотого вами отдавайте священнику, чтобы над домом твоим почивало благословение.
\vs Eze 44:31 Никакой мертвечины и ничего, растерзанного зверем, ни из птиц, ни из скота, не должны есть священники.
\vs Eze 45:1 Когда будете по жребию делить землю на уделы, тогда отделите священный участок Господу в двадцать пять тысяч \bibemph{тростей} длины и десять тысяч ширины; да будет свято это место во всем объеме своем, кругом.
\vs Eze 45:2 От него к святилищу отойдет четырехугольник по пятисот \bibemph{тростей} кругом, и кругом него площадь в пятьдесят локтей.
\vs Eze 45:3 Из этой меры отмерь двадцать пять тысяч \bibemph{тростей} в длину и десять тысяч в ширину, где будет находиться святилище, Святое Святых.
\vs Eze 45:4 Эта священная часть земли принадлежать будет священникам, служителям святилища, приступающим к служению Господу: это будет для них местом для домов и святынею для святилища.
\vs Eze 45:5 Двадцать пять тысяч \bibemph{тростей} длины и десять тысяч ширины будут принадлежать левитам, служителям храма, как их владение для обитания их.
\vs Eze 45:6 И во владение городу дайте пять тысяч ширины и двадцать пять тысяч длины, против священного места, отделенного Господу; это принадлежать должно всему дому Израилеву.
\vs Eze 45:7 И князю \bibemph{дайте} долю по ту и другую сторону, как подле священного места, отделенного \bibemph{Господу}, так и подле городского владения, к западу с западной стороны и к востоку с восточной стороны, длиною наравне с одним из оных уделов от западного предела до восточного.
\vs Eze 45:8 Это его земля, его владение в Израиле, чтобы князья Мои вперед не теснили народа Моего и чтобы предоставили землю дому Израилеву по коленам его.
\vs Eze 45:9 Так говорит Господь Бог: довольно вам, князья Израилевы! отложите обиды и угнетения и творите суд и правду, перестаньте вытеснять народ Мой из владения его, говорит Господь Бог.
\vs Eze 45:10 Да будут у вас правильные весы и правильная ефа и правильный бат.
\vs Eze 45:11 Ефа и бат должны быть одинаковой меры, так чтобы бат вмещал в себе десятую часть хомера и ефа десятую часть хомера; мера их должна определяться по хомеру.
\vs Eze 45:12 В сикле двадцать гер; а двадцать сиклей, двадцать пять сиклей и пятнадцать сиклей составлять будут у вас мину.
\vs Eze 45:13 Вот дань, какую вы должны давать \bibemph{князю}: шестую часть ефы от хомера пшеницы и шестую часть ефы от хомера ячменя;
\vs Eze 45:14 постановление об елее: от кора елея десятую часть бата; десять батов \bibemph{составят} хомер, потому что в хомере десять батов;
\vs Eze 45:15 одну овцу от стада в двести овец с тучной пажити Израиля: все это для хлебного приношения и всесожжения, и благодарственной жертвы, в очищение их, говорит Господь Бог.
\vs Eze 45:16 Весь народ земли обязывается делать сие приношение князю в Израиле.
\vs Eze 45:17 А на обязанности князя будут лежать всесожжение и хлебное приношение, и возлияние в праздники и в новомесячия, и в субботы, во все торжества дома Израилева; он должен будет приносить жертву за грех и хлебное приношение, и всесожжение, и жертву благодарственную для очищения дома Израилева.
\vs Eze 45:18 Так говорит Господь Бог: в первом \bibemph{месяце}, в первый \bibemph{день} месяца, возьми из стада волов тельца без порока, и очисти святилище.
\vs Eze 45:19 Священник пусть возьмет крови от этой жертвы за грех и покропит ею на вереи храма и на четыре угла площадки у жертвенника и на вереи ворот внутреннего двора.
\vs Eze 45:20 То же сделай и в седьмой \bibemph{день} месяца за согрешающих умышленно и по простоте, и так очищайте храм.
\vs Eze 45:21 В первом \bibemph{месяце}, в четырнадцатый день месяца, должна быть у вас Пасха, праздник семидневный, когда должно есть опресноки.
\vs Eze 45:22 В этот день князь за себя и за весь народ земли принесет тельца в жертву за грех.
\vs Eze 45:23 И в эти семь дней праздника он должен приносить во всесожжение Господу каждый день по семи тельцов и по семи овнов без порока, и в жертву за грех каждый день по козлу из козьего стада.
\vs Eze 45:24 Хлебного приношения он должен приносить по ефе на тельца и по ефе на овна и по гину елея на ефу.
\vs Eze 45:25 В седьмом \bibemph{месяце}, в пятнадцатый день месяца, в праздник, в течение семи дней он должен приносить то же: такую же жертву за грех, такое же всесожжение, и столько же хлебного приношения и столько же елея.
\vs Eze 46:1 Так говорит Господь Бог: ворота внутреннего двора, обращенные лицом к востоку, должны быть заперты в продолжение шести рабочих дней, а в субботний день они должны быть отворены и в день новомесячия должны быть отворены.
\vs Eze 46:2 Князь пойдет через внешний притвор ворот и станет у вереи этих ворот; и священники совершат его всесожжение и его благодарственную жертву; и он у порога ворот поклонится \bibemph{Господу}, и выйдет, а ворота остаются незапертыми до вечера.
\vs Eze 46:3 И народ земли будет поклоняться пред Господом, при входе в ворота, в субботы и новомесячия.
\vs Eze 46:4 Всесожжение, которое князь принесет Господу в субботний день, должно быть из шести агнцев без порока и из овна без порока;
\vs Eze 46:5 хлебного приношения ефа на овна, а на агнцев хлебного приношения, сколько рука его подаст, а елея гин на ефу.
\vs Eze 46:6 В день новомесячия будут приносимы им из стада волов телец без порока, также шесть агнцев и овен без порока.
\vs Eze 46:7 Хлебного приношения он принесет ефу на тельца и ефу на овна, а на агнцев, сколько рука его подаст, и елея гин на ефу.
\vs Eze 46:8 И когда приходить будет князь, то должен входить через притвор ворот и тем же путем выходить.
\vs Eze 46:9 А когда народ земли будет приходить пред лице Господа в праздники, то вошедший северными воротами для поклонения должен выходить воротами южными, а вошедший южными воротами должен выходить воротами северными; он не должен выходить теми же воротами, которыми вошел, а должен выходить противоположными.
\vs Eze 46:10 И князь должен находиться среди них; когда они входят, входит и он; и когда они выходят, выходит и он.
\vs Eze 46:11 И в праздники и в торжественные дни хлебного приношения \bibemph{от него} должно быть по ефе на тельца и по ефе на овна, а на агнцев, сколько подаст рука его, и елея по гину на ефу.
\vs Eze 46:12 А если князь, по усердию своему, захочет принести всесожжение или благодарственную жертву Господу, то должны отворить ему ворота, обращенные к востоку, и он совершит свое всесожжение и свою благодарственную жертву так же, как совершил в субботний день, и после сего он выйдет, и по выходе его ворота запрутся.
\vs Eze 46:13 Каждый день приноси Господу во всесожжение однолетнего агнца без порока; каждое утро приноси его.
\vs Eze 46:14 А хлебного приношения прилагай к нему каждое утро шестую часть ефы и елея третью часть гина, чтобы растворить муку; таково вечное постановление о хлебном приношении Господу, навсегда.
\vs Eze 46:15 Пусть приносят во всесожжение агнца и хлебное приношение и елей каждое утро постоянно.
\vs Eze 46:16 Так говорит Господь Бог: если князь дает кому из сыновей своих подарок, то это должно пойти в наследство и его сыновьям; это владение их должно быть наследственным.
\vs Eze 46:17 Если же он даст из наследия своего кому-либо из рабов своих подарок, то это будет принадлежать ему только до года освобождения, и тогда возвратится к князю. Только к сыновьям его должно переходить наследие его.
\vs Eze 46:18 Но князь не может брать из наследственного участка народа, вытесняя их из владения их; из своего только владения он может уделять детям своим, чтобы никто из народа Моего не был изгоняем из своего владения.
\vs Eze 46:19 И привел он меня тем ходом, который сбоку ворот, к священным комнатам для священников, обращенным к северу, и вот там одно место на краю к западу.
\vs Eze 46:20 И сказал мне: <<это~--- место, где священники должны варить жертву за преступление и жертву за грех, где должны печь хлебное приношение, не вынося его на внешний двор, для освящения народа>>.
\vs Eze 46:21 И вывел меня на внешний двор, и провел меня по четырем углам двора, и вот, в каждом углу двора еще двор.
\vs Eze 46:22 Во всех четырех углах двора были покрытые дворы в сорок \bibemph{локтей} длины и тридцать ширины, одной меры во всех четырех углах.
\vs Eze 46:23 И кругом всех их четырех~--- стены, а у стен сделаны очаги кругом.
\vs Eze 46:24 И сказал мне: <<вот поварни, в которых служители храма варят жертвы народные>>.
\vs Eze 47:1 Потом привел он меня обратно к дверям храма, и вот, из-под порога храма течет вода на восток, ибо храм стоял лицом на восток, и вода текла из-под правого бока храма, по южную сторону жертвенника.
\vs Eze 47:2 И вывел меня северными воротами, и внешним путем обвел меня к внешним воротам, путем, обращенным к востоку; и вот, вода течет по правую сторону.
\vs Eze 47:3 Когда тот муж пошел на восток, то в руке держал шнур, и отмерил тысячу локтей, и повел меня по воде; воды было по лодыжку.
\vs Eze 47:4 И \bibemph{еще} отмерил тысячу, и повел меня по воде; воды было по колено. И еще отмерил тысячу, и повел меня; воды было по поясницу.
\vs Eze 47:5 И еще отмерил тысячу, и уже тут был такой поток, через который я не мог идти, потому что вода была так высока, что надлежало плыть, а переходить нельзя было этот поток.
\vs Eze 47:6 И сказал мне: <<видел, сын человеческий?>> и повел меня обратно к берегу этого потока.
\vs Eze 47:7 И когда я пришел назад, и вот, на берегах потока много было дерев по ту и другую сторону.
\vs Eze 47:8 И сказал мне: эта вода течет в восточную сторону земли, сойдет на равнину и войдет в море; и воды его сделаются здоровыми.
\vs Eze 47:9 И всякое живущее существо, пресмыкающееся там, где войдут две струи, будет живо; и рыбы будет весьма много, потому что войдет туда эта вода, и в\acc{о}ды \bibemph{в море} сделаются здоровыми, и, куда войдет этот поток, все будет живо там.
\vs Eze 47:10 И будут стоять подле него рыболовы от Ен-Гадди до Эглаима, будут закидывать сети. Рыба будет в своем виде и, как в большом море, рыбы будет весьма много.
\vs Eze 47:11 Болота его и лужи его, которые не сделаются здоровыми, будут оставлены для соли.
\vs Eze 47:12 У потока по берегам его, с той и другой стороны, будут расти всякие дерева, доставляющие пищу: листья их не будут увядать, и плоды на них не будут истощаться; каждый месяц будут созревать новые, потому что вода для них течет из святилища; плоды их будут употребляемы в пищу, а листья на врачевание.
\rsbpar\vs Eze 47:13 Так говорит Господь Бог: вот распределение, по которому вы должны разделить землю в наследие двенадцати коленам Израилевым: Иосифу два удела.
\vs Eze 47:14 И наследуйте ее, как один, так и другой; так как Я, подняв руку Мою, клялся отдать ее отцам вашим, то и будет земля сия наследием вашим.
\vs Eze 47:15 И вот предел земли: на северном конце, начиная от великого моря, через Хетлон, по дороге в Цедад,
\vs Eze 47:16 Емаф, Берот, Сивраим, находящийся между Дамасскою и Емафскою областями Гацар-Тихон, который на границе Аврана.
\vs Eze 47:17 И будет граница от моря до Гацар-Енон, граница с Дамаском, и далее на севере область Емаф; и вот северный край.
\vs Eze 47:18 Черту восточного края ведите между Авраном и Дамаском, между Галаадом и землею Израильскою, по Иордану, от северного края до восточного моря; это восточный край.
\vs Eze 47:19 А южный край с полуденной стороны от Тамары до вод пререкания при Кадисе, и по течению потока до великого моря; это полуденный край на юге.
\vs Eze 47:20 Западный же предел~--- великое море, от южной границы до места против Емафа; это западный край.
\vs Eze 47:21 И разделите себе землю сию на уделы по коленам Израилевым.
\vs Eze 47:22 И разделите ее по жребию в наследие себе и иноземцам, живущим у вас, которые родили у вас детей; и они среди сынов Израилевых должны считаться наравне с природными жителями, и они с вами войдут в долю среди колен Израилевых.
\vs Eze 47:23 В котором колене живет иноземец, в том и дайте ему наследие его, говорит Господь Бог.
\vs Eze 48:1 Вот имена колен. На северном краю по дороге от Хетлона, ведущей в Емаф, Гацар-Енон, от северной границы Дамаска по пути к Емафу: все это от востока до моря один удел Дану.
\vs Eze 48:2 Подле границы Дана, от восточного края до западного, это один удел Асиру.
\vs Eze 48:3 Подле границы Асира, от восточного края до западного, это один удел Неффалиму.
\vs Eze 48:4 Подле границы Неффалима, от восточного края до западного, это один удел Манассии.
\vs Eze 48:5 Подле границы Манассии, от восточного края до западного, это один удел Ефрему.
\vs Eze 48:6 Подле границы Ефрема, от восточного края до западного, это один удел Рувиму.
\vs Eze 48:7 Подле границы Рувима, от восточного края до западного, это один удел Иуде.
\vs Eze 48:8 А подле границы Иуды, от восточного края до западного, священный участок, шириною в двадцать пять тысяч \bibemph{тростей}, а длиною наравне с другими уделами, от восточного края до западного; среди него будет святилище.
\vs Eze 48:9 Участок, который вы посвятите Господу, длиною будет в двадцать пять тысяч, а шириною в десять тысяч \bibemph{тростей}.
\vs Eze 48:10 И этот священный участок должен принадлежать священникам, к северу двадцать пять тысяч и к морю в ширину десять тысяч, и к востоку в ширину десять тысяч, а к югу в длину двадцать пять тысяч \bibemph{тростей}, и среди него будет святилище Господне.
\vs Eze 48:11 Это посвятите священникам из сынов Садока, которые стояли на страже Моей, которые во время отступничества сынов Израилевых не отступили от Меня, как отступили \bibemph{другие} левиты.
\vs Eze 48:12 Им будет принадлежать эта часть земли из священного участка, святыня из святынь, у предела левитов.
\vs Eze 48:13 И левиты получат также у священнического предела двадцать пять тысяч в длину и десять тысяч \bibemph{тростей} в ширину; вся длина двадцать пять тысяч, а ширина десять тысяч \bibemph{тростей}.
\vs Eze 48:14 И из этой части они не могут ни продать, ни променять; и начатки земли не могут переходить к другим, потому что это святыня Господня.
\vs Eze 48:15 А остальные пять тысяч в ширину с двадцатью пятью тысячами \bibemph{в длину} назначаются для города в общее употребление, на заселение и на предместья; город будет в средине.
\vs Eze 48:16 И вот размеры его: северная сторона четыре тысячи пятьсот и южная сторона четыре тысячи пятьсот, восточная сторона четыре тысячи пятьсот и западная сторона четыре тысячи пятьсот \bibemph{тростей}.
\vs Eze 48:17 А предместья города к северу двести пятьдесят, и к востоку двести пятьдесят, и к югу двести пятьдесят, и к западу двести пятьдесят \bibemph{тростей}.
\vs Eze 48:18 А что остается из длины против священного участка, десять тысяч к востоку и десять тысяч к западу, против священного участка, произведения с этой земли должны быть для продовольствия работающих в городе.
\vs Eze 48:19 Работать же в городе могут работники из всех колен Израилевых.
\vs Eze 48:20 Весь отделенный участок в двадцать пять тысяч длины и в двадцать пять тысяч ширины, четырехугольный, выделите в священный удел, со включением владений города;
\vs Eze 48:21 а остальное князю. Как со стороны священного участка, так и со стороны владений города, против двадцати пяти тысяч \bibemph{тростей} до восточной границы участка, и на запад против двадцати пяти тысяч у западной границы соразмерно с сими уделами, удел князю, так что священный участок и святилище будет в средине его.
\vs Eze 48:22 И то, что от владений левитских \bibemph{и} от владений города остается в промежутке, принадлежит также князю; промежуток между границею Иуды и между границею Вениамина будет принадлежать князю.
\vs Eze 48:23 Остальное же от колен, от восточного края до западного~--- один удел Вениамину.
\vs Eze 48:24 Подле границы Вениамина, от восточного края до западного~--- один удел Симеону.
\vs Eze 48:25 Подле границы Симеона, от восточного края до западного~--- один удел Иссахару.
\vs Eze 48:26 Подле границы Иссахара, от восточного края до западного~--- один удел Завулону.
\vs Eze 48:27 Подле границы Завулона, от восточного края до западного~--- один удел Гаду.
\vs Eze 48:28 А подле границы Гада на южной стороне идет южный предел от Тамары к водам пререкания при Кадисе, вдоль потока до великого моря.
\vs Eze 48:29 Вот земля, которую вы по жребию разделите коленам Израилевым, и вот участки их, говорит Господь Бог.
\vs Eze 48:30 И вот выходы города: с северной стороны меры четыре тысячи пятьсот;
\vs Eze 48:31 и ворота города называются именами колен Израилевых; к северу трое ворот: ворота Рувимовы одни, ворота Иудины одни, ворота Левиины одни.
\vs Eze 48:32 И с восточной стороны \bibemph{меры} четыре тысячи пятьсот, и трое ворот: ворота Иосифовы одни, ворота Вениаминовы одни, ворота Дановы одни;
\vs Eze 48:33 и с южной стороны меры четыре тысячи пятьсот, и трое ворот: ворота Симеоновы одни, ворота Иссахаровы одни, ворота Завулоновы одни.
\vs Eze 48:34 С морской стороны \bibemph{меры} четыре тысячи пятьсот, ворот здесь трое же: ворота Гадовы одни, ворота Асировы одни, ворота Неффалимовы одни.
\vs Eze 48:35 Всего кругом восемнадцать тысяч. А имя городу с того дня будет: <<Господь там>>.

\bibbookdescr{Dan}{
  inline={\LARGE Книга\\\Huge Пророка Даниила},
  toc={Даниил},
  bookmark={Даниил},
  header={Даниил},
  %headerleft={},
  %headerright={},
  abbr={Дан}
}
\vs Dan 1:1 В третий год царствования Иоакима, царя Иудейского, пришел Навуходоносор, царь Вавилонский, к Иерусалиму и осадил его.
\vs Dan 1:2 И предал Господь в руку его Иоакима, царя Иудейского, и часть сосудов дома Божия, и он отправил их в землю Сеннаар, в дом бога своего, и внес эти сосуды в сокровищницу бога своего.
\vs Dan 1:3 И сказал царь Асфеназу, начальнику евнухов своих, чтобы он из сынов Израилевых, из рода царского и княжеского, привел
\vs Dan 1:4 отроков, у которых нет никакого телесного недостатка, красивых видом, и понятливых для всякой науки, и разумеющих науки, и смышленых и годных служить в чертогах царских, и чтобы научил их книгам и языку Халдейскому.
\vs Dan 1:5 И назначил им царь ежедневную пищу с царского стола и вино, которое сам пил, и велел воспитывать их три года, по истечении которых они должны были предстать пред царя.
\vs Dan 1:6 Между ними были из сынов Иудиных Даниил, Анания, Мисаил и Азария.
\vs Dan 1:7 И переименовал их начальник евнухов~--- Даниила Валтасаром, Ананию Седрахом, Мисаила Мисахом и Азарию Авденаго.
\vs Dan 1:8 Даниил положил в сердце своем не оскверняться яствами со стола царского и вином, какое пьет царь, и потому просил начальника евнухов о том, чтобы не оскверняться ему.
\vs Dan 1:9 Бог даровал Даниилу милость и благорасположение начальника евнухов;
\vs Dan 1:10 и начальник евнухов сказал Даниилу: боюсь я господина моего, царя, который сам назначил вам пищу и питье; если он увидит лица ваши худощавее, нежели у отроков, сверстников ваших, то вы сделаете голову мою виновною перед царем.
\vs Dan 1:11 Тогда сказал Даниил Амелсару, которого начальник евнухов приставил к Даниилу, Анании, Мисаилу и Азарии:
\vs Dan 1:12 сделай опыт над рабами твоими в течение десяти дней; пусть дают нам в пищу овощи и воду для питья;
\vs Dan 1:13 и потом пусть явятся перед тобою лица наши и лица тех отроков, которые питаются царскою пищею, и затем поступай с рабами твоими, как увидишь.
\vs Dan 1:14 Он послушался их в этом и испытывал их десять дней.
\vs Dan 1:15 По истечении же десяти дней лица их оказались красивее, и телом они были полнее всех тех отроков, которые питались царскими яствами.
\vs Dan 1:16 Тогда Амелсар брал их кушанье и вино для питья и давал им овощи.
\vs Dan 1:17 И даровал Бог четырем сим отрокам знание и разумение всякой книги и мудрости, а Даниилу еще даровал разуметь и всякие видения и сны.
\vs Dan 1:18 По окончании тех дней, когда царь приказал представить их, начальник евнухов представил их Навуходоносору.
\vs Dan 1:19 И царь говорил с ними, и из всех \bibemph{отроков} не нашлось подобных Даниилу, Анании, Мисаилу и Азарии, и стали они служить пред царем.
\vs Dan 1:20 И во всяком деле мудрого уразумения, о чем ни спрашивал их царь, он находил их в десять раз выше всех тайноведцев и волхвов, какие были во всем царстве его.
\vs Dan 1:21 И был там Даниил до первого года царя Кира.
\vs Dan 2:1 Во второй год царствования Навуходоносора снились Навуходоносору сны, и возмутился дух его, и сон удалился от него.
\vs Dan 2:2 И велел царь созвать тайноведцев, и гадателей, и чародеев, и Халдеев, чтобы они рассказали царю сновидения его. Они пришли, и стали перед царем.
\vs Dan 2:3 И сказал им царь: сон снился мне, и тревожится дух мой; желаю знать этот сон.
\vs Dan 2:4 И сказали Халдеи царю по-арамейски: царь! вовеки живи! скажи сон рабам твоим, и мы объясним значение его.
\vs Dan 2:5 Отвечал царь и сказал Халдеям: слово отступило от меня; если вы не скажете мне сновидения и значения его, то в куски будете изрублены, и домы ваши обратятся в развалины.
\vs Dan 2:6 Если же расскажете сон и значение его, то получите от меня дары, награду и великую почесть; итак скажите мне сон и значение его.
\vs Dan 2:7 Они вторично отвечали и сказали: да скажет царь рабам своим сновидение, и мы объясним его значение.
\vs Dan 2:8 Отвечал царь и сказал: верно знаю, что вы хотите выиграть время, потому что видите, что слово отступило от меня.
\vs Dan 2:9 Так как вы не объявляете мне сновидения, то у вас один умысел: вы собираетесь сказать мне ложь и обман, пока минет время; итак расскажите мне сон, и тогда я узнаю, что вы можете объяснить мне и значение его.
\vs Dan 2:10 Халдеи отвечали царю и сказали: нет на земле человека, который мог бы открыть это дело царю, и потому ни один царь, великий и могущественный, не требовал подобного ни от какого тайноведца, гадателя и Халдея.
\vs Dan 2:11 Дело, которого царь требует, так трудно, что никто другой не может открыть его царю, кроме богов, которых обитание не с плотью.
\vs Dan 2:12 Рассвирепел царь и сильно разгневался на это, и приказал истребить всех мудрецов Вавилонских.
\rsbpar\vs Dan 2:13 Когда вышло это повеление, чтобы убивать мудрецов, искали Даниила и товарищей его, чтобы умертвить их.
\vs Dan 2:14 Тогда Даниил обратился с советом и мудростью к Ариоху, начальнику царских телохранителей, который вышел убивать мудрецов Вавилонских;
\vs Dan 2:15 и спросил Ариоха, сильного при царе: <<почему такое грозное повеление от царя?>> Тогда Ариох рассказал все дело Даниилу.
\vs Dan 2:16 И Даниил вошел, и упросил царя дать ему время, и он представит царю толкование \bibemph{сна}.
\vs Dan 2:17 Даниил пришел в дом свой, и рассказал дело Анании, Мисаилу и Азарии, товарищам своим,
\vs Dan 2:18 чтобы они просили милости у Бога небесного об этой тайне, дабы Даниил и товарищи его не погибли с прочими мудрецами Вавилонскими.
\vs Dan 2:19 И тогда открыта была тайна Даниилу в ночном видении, и Даниил благословил Бога небесного.
\vs Dan 2:20 И сказал Даниил: да будет благословенно имя Господа от века и до века! ибо у Него мудрость и сила;
\vs Dan 2:21 Он изменяет времена и лета, низлагает царей и поставляет царей; дает мудрость мудрым и разумение разумным;
\vs Dan 2:22 Он открывает глубокое и сокровенное, знает, что во мраке, и свет обитает с Ним.
\vs Dan 2:23 Славлю и величаю Тебя, Боже отцов моих, что Ты даровал мне мудрость и силу и открыл мне то, о чем мы молили Тебя; ибо Ты открыл нам дело царя.
\vs Dan 2:24 После сего Даниил вошел к Ариоху, которому царь повелел умертвить мудрецов Вавилонских, пришел и сказал ему: не убивай мудрецов Вавилонских; введи меня к царю, и я открою значение \bibemph{сна}.
\vs Dan 2:25 Тогда Ариох немедленно привел Даниила к царю и сказал ему: я нашел из пленных сынов Иудеи человека, который может открыть царю значение \bibemph{сна}.
\vs Dan 2:26 Царь сказал Даниилу, который назван был Валтасаром: можешь ли ты сказать мне сон, который я видел, и значение его?
\rsbpar\vs Dan 2:27 Даниил отвечал царю и сказал: тайны, о которой царь спрашивает, не могут открыть царю ни мудрецы, ни обаятели, ни тайноведцы, ни гадатели.
\vs Dan 2:28 Но есть на небесах Бог, открывающий тайны; и Он открыл царю Навуходоносору, что будет в последние дни. Сон твой и видения главы твоей на ложе твоем были такие:
\vs Dan 2:29 ты, царь, на ложе твоем думал о том, что будет после сего? и Открывающий тайны показал тебе то, что будет.
\vs Dan 2:30 А мне тайна сия открыта не потому, чтобы я был мудрее всех живущих, но для того, чтобы открыто было царю разумение и чтобы ты узнал помышления сердца твоего.
\vs Dan 2:31 Тебе, царь, было такое видение: вот, какой-то большой истукан; огромный был этот истукан, в чрезвычайном блеске стоял он пред тобою, и страшен был вид его.
\vs Dan 2:32 У этого истукана голова была из чистого золота, грудь его и руки его~--- из серебра, чрево его и бедра его медные,
\vs Dan 2:33 голени его железные, ноги его частью железные, частью глиняные.
\vs Dan 2:34 Ты видел его, доколе камень не оторвался от горы без содействия рук, ударил в истукана, в железные и глиняные ноги его, и разбил их.
\vs Dan 2:35 Тогда все вместе раздробилось: железо, глина, медь, серебро и золото сделались как прах на летних гумнах, и ветер унес их, и следа не осталось от них; а камень, разбивший истукана, сделался великою горою и наполнил всю землю.
\vs Dan 2:36 Вот сон! Скажем пред царем и значение его.
\vs Dan 2:37 Ты, царь, царь царей, которому Бог небесный даровал царство, власть, силу и славу,
\vs Dan 2:38 и всех сынов человеческих, где бы они ни жили, зверей земных и птиц небесных Он отдал в твои руки и поставил тебя владыкою над всеми ими. Ты~--- это золотая голова!
\vs Dan 2:39 После тебя восстанет другое царство, ниже твоего, и еще третье царство, медное, которое будет владычествовать над всею землею.
\vs Dan 2:40 А четвертое царство будет крепко, как железо; ибо как железо разбивает и раздробляет все, так и оно, подобно всесокрушающему железу, будет раздроблять и сокрушать.
\vs Dan 2:41 А что ты видел ноги и пальцы на ногах частью из глины горшечной, а частью из железа, то будет царство разделенное, и в нем останется несколько крепости железа, так как ты видел железо, смешанное с горшечною глиною.
\vs Dan 2:42 И как персты ног были частью из железа, а частью из глины, так и царство будет частью крепкое, частью хрупкое.
\vs Dan 2:43 А что ты видел железо, смешанное с глиною горшечною, это значит, что они смешаются через семя человеческое, но не сольются одно с другим, как железо не смешивается с глиною.
\vs Dan 2:44 И во дни тех царств Бог небесный воздвигнет царство, которое вовеки не разрушится, и царство это не будет передано другому народу; оно сокрушит и разрушит все царства, а само будет стоять вечно,
\vs Dan 2:45 так как ты видел, что камень отторгнут был от горы не руками и раздробил железо, медь, глину, серебро и золото. Великий Бог дал знать царю, что будет после сего. И верен этот сон, и точно истолкование его!
\vs Dan 2:46 Тогда царь Навуходоносор пал на лице свое и поклонился Даниилу, и велел принести ему дары и благовонные курения.
\vs Dan 2:47 И сказал царь Даниилу: истинно Бог ваш есть Бог богов и Владыка царей, открывающий тайны, когда ты мог открыть эту тайну!
\vs Dan 2:48 Тогда возвысил царь Даниила и дал ему много больших подарков, и поставил его над всею областью Вавилонскою и главным начальником над всеми мудрецами Вавилонскими.
\vs Dan 2:49 Но Даниил просил царя, и он поставил Седраха, Мисаха и Авденаго над делами страны Вавилонской, а Даниил остался при дворе царя.
\vs Dan 3:1 Царь Навуходоносор сделал золотой истукан, вышиною в шестьдесят локтей, шириною в шесть локтей, поставил его на поле Деире, в области Вавилонской.
\vs Dan 3:2 И послал царь Навуходоносор собрать сатрапов, наместников, воевод, верховных судей, казнохранителей, законоведцев, блюстителей суда и всех областных правителей, чтобы они пришли на торжественное открытие истукана, которого поставил царь Навуходоносор.
\vs Dan 3:3 И собрались сатрапы, наместники, военачальники, верховные судьи, казнохранители, законоведцы, блюстители суда и все областные правители на открытие истукана, которого Навуходоносор царь поставил, и стали перед истуканом, которого воздвиг Навуходоносор.
\vs Dan 3:4 Тогда глашатай громко воскликнул: объявляется вам, народы, племена и языки:
\vs Dan 3:5 в то время, как услышите звук трубы, свирели, цитры, цевницы, гуслей и симфонии и всяких музыкальных орудий, падите и поклонитесь золотому истукану, которого поставил царь Навуходоносор.
\vs Dan 3:6 А кто не падет и не поклонится, тотчас брошен будет в печь, раскаленную огнем.
\vs Dan 3:7 Посему, когда все народы услышали звук трубы, свирели, цитры, цевницы, гуслей и всякого рода музыкальных орудий, то пали все народы, племена и языки, и поклонились золотому истукану, которого поставил Навуходоносор царь.
\rsbpar\vs Dan 3:8 В это самое время приступили некоторые из Халдеев и донесли на Иудеев.
\vs Dan 3:9 Они сказали царю Навуходоносору: царь, вовеки живи!
\vs Dan 3:10 Ты, царь, дал повеление, чтобы каждый человек, который услышит звук трубы, свирели, цитры, цевницы, гуслей и симфонии и всякого рода музыкальных орудий, пал и поклонился золотому истукану;
\vs Dan 3:11 а кто не падет и не поклонится, тот должен быть брошен в печь, раскаленную огнем.
\vs Dan 3:12 Есть мужи Иудейские, которых ты поставил над делами страны Вавилонской: Седрах, Мисах и Авденаго; эти мужи не повинуются повелению твоему, царь, богам твоим не служат и золотому истукану, которого ты поставил, не поклоняются.
\vs Dan 3:13 Тогда Навуходоносор во гневе и ярости повелел привести Седраха, Мисаха и Авденаго; и приведены были эти мужи к царю.
\vs Dan 3:14 Навуходоносор сказал им: с умыслом ли вы, Седрах, Мисах и Авденаго, богам моим не служите, и золотому истукану, которого я поставил, не поклоняетесь?
\vs Dan 3:15 Отныне, если вы готовы, как скоро услышите звук трубы, свирели, цитры, цевницы, гуслей, симфонии и всякого рода музыкальных орудий, падите и поклонитесь истукану, которого я сделал; если же не поклонитесь, то в тот же час брошены будете в печь, раскаленную огнем, и тогда какой Бог избавит вас от руки моей?
\vs Dan 3:16 И отвечали Седрах, Мисах и Авденаго, и сказали царю Навуходоносору: нет нужды нам отвечать тебе на это.
\vs Dan 3:17 Бог наш, Которому мы служим, силен спасти нас от печи, раскаленной огнем, и от руки твоей, царь, избавит.
\vs Dan 3:18 Если же и не будет того, то да будет известно тебе, царь, что мы богам твоим служить не будем и золотому истукану, которого ты поставил, не поклонимся.
\vs Dan 3:19 Тогда Навуходоносор исполнился ярости, и вид лица его изменился на Седраха, Мисаха и Авденаго, и он повелел разжечь печь в семь раз сильнее, нежели как обыкновенно разжигали ее,
\vs Dan 3:20 и самым сильным мужам из войска своего приказал связать Седраха, Мисаха и Авденаго и бросить их в печь, раскаленную огнем.
\vs Dan 3:21 Тогда мужи сии связаны были в исподнем и верхнем платье своем, в головных повязках и в прочих одеждах своих, и брошены в печь, раскаленную огнем.
\vs Dan 3:22 И как повеление царя было строго, и печь раскалена была чрезвычайно, то пламя огня убило тех людей, которые бросали Седраха, Мисаха и Авденаго.
\vs Dan 3:23 А сии три мужа, Седрах, Мисах и Авденаго, упали в раскаленную огнем печь связанные.
\vs Dan 3:24 \fns{Стихи с 24-го по 90-й переведены с греческого, потому что в еврейском тексте их нет.}[И ходили посреди пламени, воспевая Бога и благословляя Господа.
\vs Dan 3:25 И став Азария молился и, открыв уста свои среди огня, возгласил:
\rsbpar\vs Dan 3:26 <<Благословен Ты, Господи Боже отцов наших, хвально и прославлено имя Твое вовеки.
\vs Dan 3:27 Ибо праведен Ты во всем, что соделал с нами, и все дела Твои истинны и пути Твои правы, и все суды Твои истинны.
\vs Dan 3:28 Ты совершил истинные суды во всем, что навел на нас и на святый град отцов наших Иерусалим, потому что по истине и по суду навел Ты все это на нас за грехи наши.
\vs Dan 3:29 Ибо согрешили мы, и поступили беззаконно, отступив от Тебя, и во всем согрешили.
\vs Dan 3:30 Заповедей Твоих не слушали и не соблюдали их, и не поступали, как Ты повелел нам, чтобы благо нам было.
\vs Dan 3:31 И все, что Ты навел на нас, и все, что Ты соделал с нами, соделал по истинному суду.
\vs Dan 3:32 И предал нас в руки врагов беззаконных, ненавистнейших отступников, и царю неправосудному и злейшему на всей земле.
\vs Dan 3:33 И ныне мы не можем открыть уст наших; мы сделались стыдом и поношением для рабов Твоих и чтущих Тебя.
\vs Dan 3:34 Но не предай нас навсегда ради имени Твоего, и не разруши завета Твоего.
\vs Dan 3:35 Не отними от нас милости Твоей ради Авраама, возлюбленного Тобою, ради Исаака, раба Твоего, и Израиля, святаго Твоего,
\vs Dan 3:36 которым Ты говорил, что умножишь семя их, как звезды небесные и как песок на берегу моря.
\vs Dan 3:37 Мы умалены, Господи, паче всех народов, и унижены ныне на всей земле за грехи наши,
\vs Dan 3:38 и нет у нас в настоящее время ни князя, ни пророка, ни вождя, ни всесожжения, ни жертвы, ни приношения, ни фимиама, ни места, чтобы нам принести жертву Тебе и обрести милость Твою.
\vs Dan 3:39 Но с сокрушенным сердцем и смиренным духом да будем приняты.
\vs Dan 3:40 Как при всесожжении овнов и тельцов и как при тысячах тучных агнцев, так да будет жертва наша пред Тобою ныне благоугодною Тебе; ибо нет стыда уповающим на Тебя.
\vs Dan 3:41 И ныне мы следуем за Тобою всем сердцем и боимся Тебя и ищем лица Твоего.
\vs Dan 3:42 Не посрами нас, но сотвори с нами по снисхождению Твоему и по множеству милости Твоей
\vs Dan 3:43 и избави нас силою чудес Твоих, и дай славу имени Твоему, Господи,
\vs Dan 3:44 и да постыдятся все, делающие рабам Твоим зло, и да постыдятся со всем могуществом, и сила их да сокрушится,
\vs Dan 3:45 и да познают, что Ты Господь Бог един и славен по всей вселенной>>.
\rsbpar\vs Dan 3:46 А между тем слуги царя, ввергшие их, не переставали разжигать печь нефтью, смолою, паклею и хворостом,
\vs Dan 3:47 и поднимался пламень над печью на сорок девять локтей
\vs Dan 3:48 и вырывался, и сожигал тех из Халдеев, которых достигал около печи.
\vs Dan 3:49 Но Ангел Господень сошел в печь вместе с Азариею и бывшими с ним
\vs Dan 3:50 и выбросил пламень огня из печи, и сделал, что в средине печи был как бы шумящий влажный ветер, и огонь нисколько не прикоснулся к ним, и не повредил им, и не смутил их.
\vs Dan 3:51 Тогда сии трое, как бы одними устами, воспели в печи, и благословили и прославили Бога:
\rsbpar\vs Dan 3:52 <<Благословен Ты, Господи Боже отцов наших, и хвальный и превозносимый во веки, и благословенно имя славы Твоей, святое и прехвальное и превозносимое во веки.
\vs Dan 3:53 Благословен Ты в храме святой славы Твоей, и прехвальный и преславный во веки.
\vs Dan 3:54 Благословен Ты, видящий бездны, восседающий на Херувимах, и прехвальный и превозносимый во веки.
\vs Dan 3:55 Благословен Ты на престоле славы царства Твоего, и прехвальный и превозносимый во веки.
\vs Dan 3:56 Благословен Ты на тверди небесной, и прехвальный и превозносимый во веки.
\vs Dan 3:57 Благословите, все дела Господни, Господа, пойте и превозносите Его во веки.
\vs Dan 3:58 Благословите, Ангелы Господни, Господа, пойте и превозносите Его во веки.
\vs Dan 3:59 Благословите, небеса, Господа, пойте и превозносите Его во веки.
\vs Dan 3:60 Благословите Господа, все воды, которые превыше небес, пойте и превозносите Его во веки.
\vs Dan 3:61 Благословите, все силы Господни, Господа, пойте и превозносите Его во веки.
\vs Dan 3:62 Благословите, солнце и луна, Господа, пойте и превозносите Его во веки.
\vs Dan 3:63 Благословите, звезды небесные, Господа, пойте и превозносите Его во веки.
\vs Dan 3:64 Благословите, всякий дождь и роса, Господа, пойте и превозносите Его во веки.
\vs Dan 3:65 Благословите, все ветры, Господа, пойте и превозносите Его во веки.
\vs Dan 3:66 Благословите, огонь и жар, Господа, пойте и превозносите Его во веки.
\vs Dan 3:67 Благословите, холод и зной, Господа, пойте и превозносите Его во веки.
\vs Dan 3:68 Благословите, росы и инеи, Господа, пойте и превозносите Его во веки.
\vs Dan 3:69 Благословите, ночи и дни, Господа, пойте и превозносите Его во веки.
\vs Dan 3:70 Благословите, свет и тьма, Господа, пойте и превозносите Его во веки.
\vs Dan 3:71 Благословите, лед и мороз, Господа, пойте и превозносите Его во веки.
\vs Dan 3:72 Благословите, иней и снег, Господа, пойте и превозносите Его во веки.
\vs Dan 3:73 Благословите, молнии и облака, Господа, пойте и превозносите Его во веки.
\vs Dan 3:74 Да благословит земля Господа, да поет и превозносит Его во веки.
\vs Dan 3:75 Благословите, горы и холмы, Господа, пойте и превозносите Его во веки.
\vs Dan 3:76 Благословите Господа, все произрастания на земле, пойте и превозносите Его во веки.
\vs Dan 3:77 Благословите, источники, Господа, пойте и превозносите Его во веки.
\vs Dan 3:78 Благословите, моря и реки, Господа, пойте и превозносите Его во веки.
\vs Dan 3:79 Благословите Господа, киты и все, движущееся в водах, пойте и превозносите Его во веки.
\vs Dan 3:80 Благословите, все птицы небесные, Господа, пойте и превозносите Его во веки.
\vs Dan 3:81 Благословите Господа, звери и весь скот, пойте и превозносите Его во веки.
\vs Dan 3:82 Благословите, сыны человеческие, Господа, пойте и превозносите Его во веки.
\vs Dan 3:83 Благослови, Израиль, Господа, пой и превозноси Его во веки.
\vs Dan 3:84 Благословите, священники Господни, Господа, пойте и превозносите Его во веки.
\vs Dan 3:85 Благословите, рабы Господни, Господа, пойте и превозносите Его во веки.
\vs Dan 3:86 Благословите, духи и души праведных, Господа, пойте и превозносите Его во веки.
\vs Dan 3:87 Благословите, праведные и смиренные сердцем, Господа, пойте и превозносите Его во веки.
\vs Dan 3:88 Благословите, Анания, Азария и Мисаил, Господа, пойте и превозносите Его во веки; ибо Он извлек нас из ада и спас нас от руки смерти, и избавил нас из среды печи горящего пламени, и из среды огня избавил нас.
\vs Dan 3:89 Славьте Господа, ибо Он благ, ибо вовек милость Его.
\vs Dan 3:90 Благословите, все чтущие Господа, Бога богов, пойте и славьте, ибо вовек милость Его>>.]
\rsbpar\vs Dan 3:91 Навуходоносор царь, [услышав, что они поют,] изумился, и поспешно встал, и сказал вельможам своим: не троих ли мужей бросили мы в огонь связанными? Они в ответ сказали царю: истинно так, царь!
\vs Dan 3:92 На это он сказал: вот, я вижу четырех мужей несвязанных, ходящих среди огня, и нет им вреда; и вид четвертого подобен сыну Божию.
\vs Dan 3:93 Тогда подошел Навуходоносор к устью печи, раскаленной огнем, и сказал: Седрах, Мисах и Авденаго, рабы Бога Всевышнего! выйдите и подойдите! Тогда Седрах, Мисах и Авденаго вышли из среды огня.
\vs Dan 3:94 И, собравшись, сатрапы, наместники, военачальники и советники царя усмотрели, что над телами мужей сих огонь не имел силы, и волосы на голове не опалены, и одежды их не изменились, и даже запаха огня не было от них.
\vs Dan 3:95 Тогда Навуходоносор сказал: благословен Бог Седраха, Мисаха и Авденаго, Который послал Ангела Своего и избавил рабов Своих, которые надеялись на Него и не послушались царского повеления, и предали тела свои [огню], чтобы не служить и не поклоняться иному богу, кроме Бога своего!
\vs Dan 3:96 И от меня дается повеление, чтобы из всякого народа, племени и языка кто произнесет хулу на Бога Седраха, Мисаха и Авденаго, был изрублен в куски, и дом его обращен в развалины, ибо нет иного бога, который мог бы так спасать.
\vs Dan 3:97 Тогда царь возвысил Седраха, Мисаха и Авденаго в стране Вавилонской [и возвеличил их и удостоил их начальства над прочими Иудеями в его царстве].
\rsbpar\vs Dan 3:98 Навуходоносор царь всем народам, племенам и языкам, живущим по всей земле: мир вам да умножится!
\vs Dan 3:99 Знамения и чудеса, какие совершил надо мною Всевышний Бог, угодно мне возвестить вам.
\vs Dan 3:100 Как велики знамения Его и как могущественны чудеса Его! Царство Его~--- царство вечное, и владычество Его~--- в роды и роды.
\vs Dan 4:1 Я, Навуходоносор, спокоен был в доме моем и благоденствовал в чертогах моих.
\vs Dan 4:2 Но я видел сон, который устрашил меня, и размышления на ложе моем и видения головы моей смутили меня.
\vs Dan 4:3 И дано было мною повеление привести ко мне всех мудрецов Вавилонских, чтобы они сказали мне значение сна.
\vs Dan 4:4 Тогда пришли тайноведцы, обаятели, Халдеи и гадатели; я рассказал им сон, но они не могли мне объяснить значения его.
\vs Dan 4:5 Наконец вошел ко мне Даниил, которому имя было Валтасар, по имени бога моего, и в котором дух святаго Бога; ему рассказал я сон.
\vs Dan 4:6 Валтасар, глава мудрецов! я знаю, что в тебе дух святаго Бога, и никакая тайна не затрудняет тебя; объясни мне видения сна моего, который я видел, и значение его.
\vs Dan 4:7 Видения же головы моей на ложе моем были такие: я видел, вот, среди земли дерево весьма высокое.
\vs Dan 4:8 Большое было это дерево и крепкое, и высота его достигала до неба, и оно видимо было до краев всей земли.
\vs Dan 4:9 Листья его прекрасные, и плодов на нем множество, и пища на нем для всех; под ним находили тень полевые звери, и в ветвях его гнездились птицы небесные, и от него питалась всякая плоть.
\vs Dan 4:10 И видел я в видениях головы моей на ложе моем, и вот, нисшел с небес Бодрствующий и Святый.
\vs Dan 4:11 Воскликнув громко, Он сказал: <<срубите это дерево, обрубите ветви его, стрясите листья с него и разбросайте плоды его; пусть удалятся звери из-под него и птицы с ветвей его;
\vs Dan 4:12 но главный корень его оставьте в земле, и пусть он в узах железных и медных среди полевой травы орошается небесною росою, и с животными пусть будет часть его в траве земной.
\vs Dan 4:13 Сердце человеческое отнимется от него и дастся ему сердце звериное, и пройдут над ним семь времен.
\vs Dan 4:14 Повелением Бодрствующих это определено, и по приговору Святых назначено, дабы знали живущие, что Всевышний владычествует над царством человеческим, и дает его, кому хочет, и поставляет над ним уничиженного между людьми>>.
\vs Dan 4:15 Такой сон видел я, царь Навуходоносор; а ты, Валтасар, скажи значение его, так как никто из мудрецов в моем царстве не мог объяснить его значения, а ты можешь, потому что дух святаго Бога в тебе.
\rsbpar\vs Dan 4:16 Тогда Даниил, которому имя Валтасар, около часа пробыл в изумлении, и мысли его смущали его. Царь начал говорить и сказал: Валтасар! да не смущает тебя этот сон и значение его. Валтасар отвечал и сказал: господин мой! твоим бы ненавистникам этот сон, и врагам твоим значение его!
\vs Dan 4:17 Дерево, которое ты видел, которое было большое и крепкое, высотою своею достигало до небес и видимо было по всей земле,
\vs Dan 4:18 на котором листья были прекрасные и множество плодов и пропитание для всех, под которым обитали звери полевые и в ветвях которого гнездились птицы небесные,
\vs Dan 4:19 это ты, царь, возвеличившийся и укрепившийся, и величие твое возросло и достигло до небес, и власть твоя~--- до краев земли.
\vs Dan 4:20 А что царь видел Бодрствующего и Святаго, сходящего с небес, Который сказал: <<срубите дерево и истребите его, только главный корень его оставьте в земле, и пусть он в узах железных и медных, среди полевой травы, орошается росою небесною, и с полевыми зверями пусть будет часть его, доколе не пройдут над ним семь времен>>,~---
\vs Dan 4:21 то вот значение этого, царь, и вот определение Всевышнего, которое постигнет господина моего, царя:
\vs Dan 4:22 тебя отлучат от людей, и обитание твое будет с полевыми зверями; травою будут кормить тебя, как вола, росою небесною ты будешь орошаем, и семь времен пройдут над тобою, доколе познаешь, что Всевышний владычествует над царством человеческим и дает его, кому хочет.
\vs Dan 4:23 А что повелено было оставить главный корень дерева, это значит, что царство твое останется при тебе, когда ты познаешь власть небесную.
\vs Dan 4:24 Посему, царь, да будет благоугоден тебе совет мой: искупи грехи твои правдою и беззакония твои милосердием к бедным; вот чем может продлиться мир твой.
\vs Dan 4:25 Все это сбылось над царем Навуходоносором.
\rsbpar\vs Dan 4:26 По прошествии двенадцати месяцев, расхаживая по царским чертогам в Вавилоне,
\vs Dan 4:27 царь сказал: это ли не величественный Вавилон, который построил я в дом царства силою моего могущества и в славу моего величия!
\vs Dan 4:28 Еще речь сия была в устах царя, как был с неба голос: <<тебе говорят, царь Навуходоносор: царство отошло от тебя!
\vs Dan 4:29 И отлучат тебя от людей, и будет обитание твое с полевыми зверями; травою будут кормить тебя, как вола, и семь времен пройдут над тобою, доколе познаешь, что Всевышний владычествует над царством человеческим и дает его, кому хочет!>>
\vs Dan 4:30 Тотчас и исполнилось это слово над Навуходоносором, и отлучен он был от людей, ел траву, как вол, и орошалось тело его росою небесною, так что волосы у него выросли как у льва, и ногти у него~--- как у птицы.
\vs Dan 4:31 По окончании же дней тех, я, Навуходоносор, возвел глаза мои к небу, и разум мой возвратился ко мне; и благословил я Всевышнего, восхвалил и прославил Присносущего, Которого владычество~--- владычество вечное, и Которого царство~--- в роды и роды.
\vs Dan 4:32 И все, живущие на земле, ничего не значат; по воле Своей Он действует как в небесном воинстве, так и у живущих на земле; и нет никого, кто мог бы противиться руке Его и сказать Ему: <<что Ты сделал?>>
\vs Dan 4:33 В то время возвратился ко мне разум мой, и к славе царства моего возвратились ко мне сановитость и прежний вид мой; тогда взыскали меня советники мои и вельможи мои, и я восстановлен на царство мое, и величие мое еще более возвысилось.
\vs Dan 4:34 Ныне я, Навуходоносор, славлю, превозношу и величаю Царя Небесного, Которого все дела истинны и пути праведны, и Который силен смирить ходящих гордо.
\vs Dan 5:1 Валтасар царь сделал большое пиршество для тысячи вельмож своих и перед глазами тысячи пил вино.
\vs Dan 5:2 Вкусив вина, Валтасар приказал принести золотые и серебряные сосуды, которые Навуходоносор, отец его, вынес из храма Иерусалимского, чтобы пить из них царю, вельможам его, женам его и наложницам его.
\vs Dan 5:3 Тогда принесли золотые сосуды, которые взяты были из святилища дома Божия в Иерусалиме; и пили из них царь и вельможи его, жены его и наложницы его.
\vs Dan 5:4 Пили вино, и славили богов золотых и серебряных, медных, железных, деревянных и каменных.
\vs Dan 5:5 В тот самый час вышли персты руки человеческой и писали против лампады на извести стены чертога царского, и царь видел кисть руки, которая писала.
\vs Dan 5:6 Тогда царь изменился в лице своем; мысли его смутили его, связи чресл его ослабели, и колени его стали биться одно о другое.
\vs Dan 5:7 Сильно закричал царь, чтобы привели обаятелей, Халдеев и гадателей. Царь начал говорить, и сказал мудрецам Вавилонским: кто прочитает это написанное и объяснит мне значение его, тот будет облечен в багряницу, и золотая цепь будет на шее у него, и третьим властелином будет в царстве.
\vs Dan 5:8 И вошли все мудрецы царя, но не могли прочитать написанного и объяснить царю значения его.
\vs Dan 5:9 Царь Валтасар чрезвычайно встревожился, и вид лица его изменился на нем, и вельможи его смутились.
\vs Dan 5:10 Царица же, по поводу слов царя и вельмож его, вошла в палату пиршества; начала говорить царица и сказала: царь, вовеки живи! да не смущают тебя мысли твои, и да не изменяется вид лица твоего!
\vs Dan 5:11 Есть в царстве твоем муж, в котором дух святаго Бога; во дни отца твоего найдены были в нем свет, разум и мудрость, подобная мудрости богов, и царь Навуходоносор, отец твой, поставил его главою тайноведцев, обаятелей, Халдеев и гадателей,~--- сам отец твой, царь,
\vs Dan 5:12 потому что в нем, в Данииле, которого царь переименовал Валтасаром, оказались высокий дух, ведение и разум, способный изъяснять сны, толковать загадочное и разрешать узлы. Итак пусть призовут Даниила и он объяснит значение.
\vs Dan 5:13 Тогда введен был Даниил пред царя, и царь начал речь и сказал Даниилу: ты ли Даниил, один из пленных сынов Иудейских, которых отец мой, царь, привел из Иудеи?
\vs Dan 5:14 Я слышал о тебе, что дух Божий в тебе и свет, и разум, и высокая мудрость найдена в тебе.
\vs Dan 5:15 Вот, приведены были ко мне мудрецы и обаятели, чтобы прочитать это написанное и объяснить мне значение его; но они не могли объяснить мне этого.
\vs Dan 5:16 А о тебе я слышал, что ты можешь объяснять значение и разрешать узлы; итак, если можешь прочитать это написанное и объяснить мне значение его, то облечен будешь в багряницу, и золотая цепь будет на шее твоей, и третьим властелином будешь в царстве.
\vs Dan 5:17 Тогда отвечал Даниил, и сказал царю: дары твои пусть останутся у тебя, и почести отдай другому; а написанное я прочитаю царю и значение объясню ему.
\rsbpar\vs Dan 5:18 Царь! Всевышний Бог даровал отцу твоему Навуходоносору царство, величие, честь и славу.
\vs Dan 5:19 Пред величием, которое Он дал ему, все народы, племена и языки трепетали и страшились его: кого хотел, он убивал, и кого хотел, оставлял в живых; кого хотел, возвышал, и кого хотел, унижал.
\vs Dan 5:20 Но когда сердце его надмилось и дух его ожесточился до дерзости, он был свержен с царского престола своего и лишен славы своей,
\vs Dan 5:21 и отлучен был от сынов человеческих, и сердце его уподобилось звериному, и жил он с дикими ослами; кормили его травою, как вола, и тело его орошаемо было небесною росою, доколе он познал, что над царством человеческим владычествует Всевышний Бог и поставляет над ним, кого хочет.
\vs Dan 5:22 И ты, сын его Валтасар, не смирил сердца твоего, хотя знал все это,
\vs Dan 5:23 но вознесся против Господа небес, и сосуды дома Его принесли к тебе, и ты и вельможи твои, жены твои и наложницы твои пили из них вино, и ты славил богов серебряных и золотых, медных, железных, деревянных и каменных, которые ни видят, ни слышат, ни разумеют; а Бога, в руке Которого дыхание твое и у Которого все пути твои, ты не прославил.
\vs Dan 5:24 За это и послана от Него кисть руки, и начертано это писание.
\vs Dan 5:25 И вот что начертано: мене, мене, текел, упарсин.
\vs Dan 5:26 Вот и значение слов: мене~--- исчислил Бог царство твое и положил конец ему;
\vs Dan 5:27 Текел~--- ты взвешен на весах и найден очень легким;
\vs Dan 5:28 Перес~--- разделено царство твое и дано Мидянам и Персам.
\rsbpar\vs Dan 5:29 Тогда по повелению Валтасара облекли Даниила в багряницу и возложили золотую цепь на шею его, и провозгласили его третьим властелином в царстве.
\vs Dan 5:30 В ту же самую ночь Валтасар, царь Халдейский, был убит,
\vs Dan 5:31 и Дарий Мидянин принял царство, будучи шестидесяти двух лет.
\vs Dan 6:1 Угодно было Дарию поставить над царством сто двадцать сатрапов, чтобы они были во всем царстве,
\vs Dan 6:2 а над ними трех князей,~--- из которых один был Даниил,~--- чтобы сатрапы давали им отчет и чтобы царю не было никакого обременения.
\vs Dan 6:3 Даниил превосходил прочих князей и сатрапов, потому что в нем был высокий дух, и царь помышлял уже поставить его над всем царством.
\vs Dan 6:4 Тогда князья и сатрапы начали искать предлога к обвинению Даниила по управлению царством; но никакого предлога и погрешностей не могли найти, потому что он был верен, и никакой погрешности или вины не оказывалось в нем.
\vs Dan 6:5 И эти люди сказали: не найти нам предлога против Даниила, если мы не найдем его против него в законе Бога его.
\vs Dan 6:6 Тогда эти князья и сатрапы приступили к царю и так сказали ему: царь Дарий! вовеки живи!
\vs Dan 6:7 Все князья царства, наместники, сатрапы, советники и военачальники согласились между собою, чтобы сделано было царское постановление и издано повеление, чтобы, кто в течение тридцати дней будет просить какого-либо бога или человека, кроме тебя, царь, того бросить в львиный ров.
\vs Dan 6:8 Итак утверди, царь, это определение и подпиши указ, чтобы он был неизменен, как закон Мидийский и Персидский, и чтобы он не был нарушен.
\vs Dan 6:9 Царь Дарий подписал указ и это повеление.
\vs Dan 6:10 Даниил же, узнав, что подписан такой указ, пошел в дом свой; окна же в горнице его были открыты против Иерусалима, и он три раза в день преклонял колени, и молился своему Богу, и славословил Его, как это делал он и прежде того.
\vs Dan 6:11 Тогда эти люди подсмотрели и нашли Даниила молящегося и просящего милости пред Богом своим,
\vs Dan 6:12 потом пришли и сказали царю о царском повелении: не ты ли подписал указ, чтобы всякого человека, который в течение тридцати дней будет просить какого-либо бога или человека, кроме тебя, царь, бросать в львиный ров? Царь отвечал и сказал: это слово твердо, как закон Мидян и Персов, не допускающий изменения.
\vs Dan 6:13 Тогда отвечали они и сказали царю, что Даниил, который из пленных сынов Иудеи, не обращает внимания ни на тебя, царь, ни на указ, тобою подписанный, но три раза в день молится своими молитвами.
\vs Dan 6:14 Царь, услышав это, сильно опечалился и положил в сердце своем спасти Даниила, и даже до захождения солнца усиленно старался избавить его.
\vs Dan 6:15 Но те люди приступили к царю и сказали ему: знай, царь, что по закону Мидян и Персов никакое определение или постановление, утвержденное царем, не может быть изменено.
\vs Dan 6:16 Тогда царь повелел, и привели Даниила, и бросили в ров львиный; при этом царь сказал Даниилу: Бог твой, Которому ты неизменно служишь, Он спасет тебя!
\vs Dan 6:17 И принесен был камень и положен на отверстие рва, и царь запечатал его перстнем своим, и перстнем вельмож своих, чтобы ничто не переменилось в распоряжении о Данииле.
\vs Dan 6:18 Затем царь пошел в свой дворец, лег спать без ужина, и даже не велел вносить к нему пищи, и сон бежал от него.
\vs Dan 6:19 Поутру же царь встал на рассвете и поспешно пошел ко рву львиному,
\vs Dan 6:20 и, подойдя ко рву, жалобным голосом кликнул Даниила, и сказал царь Даниилу: Даниил, раб Бога живаго! Бог твой, Которому ты неизменно служишь, мог ли спасти тебя от львов?
\vs Dan 6:21 Тогда Даниил сказал царю: царь! вовеки живи!
\vs Dan 6:22 Бог мой послал Ангела Своего и заградил пасть львам, и они не повредили мне, потому что я оказался пред Ним чист, да и перед тобою, царь, я не сделал преступления.
\vs Dan 6:23 Тогда царь чрезвычайно возрадовался о нем и повелел поднять Даниила изо рва; и поднят был Даниил изо рва, и никакого повреждения не оказалось на нем, потому что он веровал в Бога своего.
\vs Dan 6:24 И приказал царь, и приведены были те люди, которые обвиняли Даниила, и брошены в львиный ров, как они сами, так и дети их и жены их; и они не достигли до дна рва, как львы овладели ими и сокрушили все кости их.
\rsbpar\vs Dan 6:25 После того царь Дарий написал всем народам, племенам и языкам, живущим по всей земле: <<Мир вам да умножится!
\vs Dan 6:26 Мною дается повеление, чтобы во всякой области царства моего трепетали и благоговели пред Богом Данииловым, потому что Он есть Бог живый и присносущий, и царство Его несокрушимо, и владычество Его бесконечно.
\vs Dan 6:27 Он избавляет и спасает, и совершает чудеса и знамения на небе и на земле; Он избавил Даниила от силы львов>>.
\vs Dan 6:28 И Даниил благоуспевал и в царствование Дария, и в царствование Кира Персидского.
\vs Dan 7:1 В первый год Валтасара, царя Вавилонского, Даниил видел сон и пророческие видения головы своей на ложе своем. Тогда он записал этот сон, изложив сущность дела.
\vs Dan 7:2 Начав речь, Даниил сказал: видел я в ночном видении моем, и вот, четыре ветра небесных боролись на великом море,
\vs Dan 7:3 и четыре больших зверя вышли из моря, непохожие один на другого.
\vs Dan 7:4 Первый~--- как лев, но у него крылья орлиные; я смотрел, доколе не вырваны были у него крылья, и он поднят был от земли, и стал на ноги, как человек, и сердце человеческое дано ему.
\vs Dan 7:5 И вот еще зверь, второй, похожий на медведя, стоял с одной стороны, и три клыка во рту у него, между зубами его; ему сказано так: <<встань, ешь мяса много!>>
\vs Dan 7:6 Затем видел я, вот еще зверь, как барс; на спине у него четыре птичьих крыла, и четыре головы были у зверя сего, и власть дана была ему.
\vs Dan 7:7 После сего видел я в ночных видениях, и вот зверь четвертый, страшный и ужасный и весьма сильный; у него большие железные зубы; он пожирает и сокрушает, остатки же попирает ногами; он отличен был от всех прежних зверей, и десять рогов было у него.
\vs Dan 7:8 Я смотрел на эти рога, и вот, вышел между ними еще небольшой рог, и три из прежних рогов с корнем исторгнуты были перед ним, и вот, в этом роге были глаза, как глаза человеческие, и уста, говорящие высокомерно.
\vs Dan 7:9 Видел я, наконец, что поставлены были престолы, и воссел Ветхий днями; одеяние на Нем было бело, как снег, и волосы главы Его~--- как чистая в\acc{о}лна; престол Его~--- как пламя огня, колеса Его~--- пылающий огонь.
\vs Dan 7:10 Огненная река выходила и проходила пред Ним; тысячи тысяч служили Ему и тьмы тем предстояли пред Ним; судьи сели, и раскрылись книги.
\vs Dan 7:11 Видел я тогда, что за изречение высокомерных слов, какие говорил рог, зверь был убит в глазах моих, и тело его сокрушено и предано на сожжение огню.
\vs Dan 7:12 И у прочих зверей отнята власть их, и продолжение жизни дано им только на время и на срок.
\vs Dan 7:13 Видел я в ночных видениях, вот, с облаками небесными шел как бы Сын человеческий, дошел до Ветхого днями и подведен был к Нему.
\vs Dan 7:14 И Ему дана власть, слава и царство, чтобы все народы, племена и языки служили Ему; владычество Его~--- владычество вечное, которое не прейдет, и царство Его не разрушится.
\vs Dan 7:15 Вострепетал дух мой во мне, Данииле, в теле моем, и видения головы моей смутили меня.
\vs Dan 7:16 Я подошел к одному из предстоящих и спросил у него об истинном значении всего этого, и он стал говорить со мною, и объяснил мне смысл сказанного:
\vs Dan 7:17 <<эти большие звери, которых четыре, \bibemph{означают}, что четыре царя восстанут от земли.
\vs Dan 7:18 Потом примут царство святые Всевышнего и будут владеть царством вовек и во веки веков>>.
\vs Dan 7:19 Тогда пожелал я точного объяснения о четвертом звере, который был отличен от всех и очень страшен, с зубами железными и когтями медными, пожирал и сокрушал, а остатки попирал ногами,
\vs Dan 7:20 и о десяти рогах, которые были на голове у него, и о другом, вновь вышедшем, перед которым выпали три, о том самом роге, у которого были глаза и уста, говорящие высокомерно, и который по виду стал больше прочих.
\vs Dan 7:21 Я видел, как этот рог вел брань со святыми и превозмогал их,
\vs Dan 7:22 доколе не пришел Ветхий днями, и суд дан был святым Всевышнего, и наступило время, чтобы царством овладели святые.
\vs Dan 7:23 Об этом он сказал: зверь четвертый~--- четвертое царство будет на земле, отличное от всех царств, которое будет пожирать всю землю, попирать и сокрушать ее.
\vs Dan 7:24 А десять рогов значат, что из этого царства восстанут десять царей, и после них восстанет иной, отличный от прежних, и уничижит трех царей,
\vs Dan 7:25 и против Всевышнего будет произносить слова и угнетать святых Всевышнего; даже возмечтает отменить у них \bibemph{праздничные} времена и закон, и они преданы будут в руку его до времени и времен и полувремени.
\vs Dan 7:26 Затем воссядут судьи и отнимут у него власть губить и истреблять до конца.
\vs Dan 7:27 Царство же и власть и величие царственное во всей поднебесной дано будет народу святых Всевышнего, Которого царство~--- царство вечное, и все властители будут служить и повиноваться Ему.
\vs Dan 7:28 Здесь конец слова. Меня, Даниила, сильно смущали размышления мои, и лице мое изменилось на мне; но слово я сохранил в сердце моем.
\vs Dan 8:1 В третий год царствования Валтасара царя явилось мне, Даниилу, видение после того, которое явилось мне прежде.
\vs Dan 8:2 И видел я в видении, и когда видел, я был в Сузах, престольном городе в области Еламской, и видел я в видении,~--- как бы я был у реки Улая.
\vs Dan 8:3 Поднял я глаза мои и увидел: вот, один овен стоит у реки; у него два рога, и рога высокие, но один выше другого, и высший поднялся после.
\vs Dan 8:4 Видел я, как этот овен бодал к западу и к северу и к югу, и никакой зверь не мог устоять против него, и никто не мог спасти от него; он делал, что хотел, и величался.
\vs Dan 8:5 Я внимательно смотрел на это, и вот, с запада шел козел по лицу всей земли, не касаясь земли; у этого козла был видный рог между его глазами.
\vs Dan 8:6 Он пошел на того овна, имеющего рога, которого я видел стоящим у реки, и бросился на него в сильной ярости своей.
\vs Dan 8:7 И я видел, как он, приблизившись к овну, рассвирепел на него и поразил овна, и сломил у него оба рога; и недостало силы у овна устоять против него, и он поверг его на землю и растоптал его, и не было никого, кто мог бы спасти овна от него.
\vs Dan 8:8 Тогда козел чрезвычайно возвеличился; но когда он усилился, то сломился большой рог, и на место его вышли четыре, обращенные на четыре ветра небесных.
\vs Dan 8:9 От одного из них вышел небольшой рог, который чрезвычайно разросся к югу и к востоку и к прекрасной стране,
\vs Dan 8:10 и вознесся до воинства небесного, и низринул на землю часть сего воинства и звезд, и попрал их,
\vs Dan 8:11 и даже вознесся на Вождя воинства сего, и отнята была у Него ежедневная жертва, и поругано было место святыни Его.
\vs Dan 8:12 И воинство предано вместе с ежедневною жертвою за нечестие, и он, повергая истину на землю, действовал и успевал.
\vs Dan 8:13 И услышал я одного святого говорящего, и сказал этот святой кому-то, вопрошавшему: <<на сколько времени простирается это видение о ежедневной жертве и об опустошительном нечестии, когда святыня и воинство будут попираемы?>>
\vs Dan 8:14 И сказал мне: <<на две тысячи триста вечеров и утр; и тогда святилище очистится>>.
\vs Dan 8:15 И было: когда я, Даниил, увидел это видение и искал значения его, вот, стал предо мною как облик мужа.
\vs Dan 8:16 И услышал я от средины Улая голос человеческий, который воззвал и сказал: <<Гавриил! объясни ему это видение!>>
\vs Dan 8:17 И он подошел к тому месту, где я стоял, и когда он пришел, я ужаснулся и пал на лице мое; и сказал он мне: <<знай, сын человеческий, что видение относится к концу времени!>>
\vs Dan 8:18 И когда он говорил со мною, я без чувств лежал лицем моим на земле; но он прикоснулся ко мне и поставил меня на место мое,
\vs Dan 8:19 и сказал: <<вот, я открываю тебе, чт\acc{о} будет в последние дни гнева; ибо это относится к концу определенного времени.
\vs Dan 8:20 Овен, которого ты видел с двумя рогами, это цари Мидийский и Персидский.
\vs Dan 8:21 А козел косматый~--- царь Греции, а большой рог, который между глазами его, это первый ее царь;
\vs Dan 8:22 он сломился, и вместо него вышли другие четыре: это~--- четыре царства восстанут из этого народа, но не с его силою.
\vs Dan 8:23 Под конец же царства их, когда отступники исполнят меру беззаконий своих, восстанет царь наглый и искусный в коварстве;
\vs Dan 8:24 и укрепится сила его, хотя и не его силою, и он будет производить удивительные опустошения и успевать и действовать и губить сильных и народ святых,
\vs Dan 8:25 и при уме его и коварство будет иметь успех в руке его, и сердцем своим он превознесется, и среди мира погубит многих, и против Владыки владык восстанет, но будет сокрушен~--- не рукою.
\vs Dan 8:26 Видение же о вечере и утре, о котором сказано, истинно; но ты сокрой это видение, ибо оно относится к отдаленным временам>>.
\rsbpar\vs Dan 8:27 И я, Даниил, изнемог, и болел несколько дней; потом встал и начал заниматься царскими делами; я изумлен был видением сим и не понимал его.
\vs Dan 9:1 В первый год Дария, сына Ассуирова, из рода Мидийского, который поставлен был царем над царством Халдейским,
\vs Dan 9:2 в первый год царствования его я, Даниил, сообразил по книгам число лет, о котором было слово Господне к Иеремии пророку, что семьдесят лет исполнятся над опустошением Иерусалима.
\vs Dan 9:3 И обратил я лице мое к Господу Богу с молитвою и молением, в посте и вретище и пепле.
\vs Dan 9:4 И молился я Господу Богу моему, и исповедовался и сказал: <<Молю Тебя, Господи Боже великий и дивный, хранящий завет и милость любящим Тебя и соблюдающим повеления Твои!
\vs Dan 9:5 Согрешили мы, поступали беззаконно, действовали нечестиво, упорствовали и отступили от заповедей Твоих и от постановлений Твоих;
\vs Dan 9:6 и не слушали рабов Твоих, пророков, которые Твоим именем говорили царям нашим, и вельможам нашим, и отцам нашим, и всему народу страны.
\vs Dan 9:7 У Тебя, Господи, правда, а у нас на лицах стыд, как день сей, у каждого Иудея, у жителей Иерусалима и у всего Израиля, у ближних и дальних, во всех странах, куда Ты изгнал их за отступление их, с каким они отступили от Тебя.
\vs Dan 9:8 Господи! у нас на лицах стыд, у царей наших, у князей наших и у отцов наших, потому что мы согрешили пред Тобою.
\vs Dan 9:9 А у Господа Бога нашего милосердие и прощение, ибо мы возмутились против Него
\vs Dan 9:10 и не слушали гласа Господа Бога нашего, чтобы поступать по законам Его, которые Он дал нам через рабов Своих, пророков.
\vs Dan 9:11 И весь Израиль преступил закон Твой и отвратился, чтобы не слушать гласа Твоего; и за то излились на нас проклятие и клятва, которые написаны в законе Моисея, раба Божия: ибо мы согрешили пред Ним.
\vs Dan 9:12 И Он исполнил слова Свои, которые изрек на нас и на судей наших, судивших нас, наведя на нас великое бедствие, какого не бывало под небесами и какое совершилось над Иерусалимом.
\vs Dan 9:13 Как написано в законе Моисея, так все это бедствие постигло нас; но мы не умоляли Господа Бога нашего, чтобы нам обратиться от беззаконий наших и уразуметь истину Твою.
\vs Dan 9:14 Наблюдал Господь это бедствие и навел его на нас: ибо праведен Господь Бог наш во всех делах Своих, которые совершает, но мы не слушали гласа Его.
\vs Dan 9:15 И ныне, Господи Боже наш, изведший народ Твой из земли Египетской рукою сильною и явивший славу Твою, как день сей! согрешили мы, поступали нечестиво.
\vs Dan 9:16 Господи! по всей правде Твоей да отвратится гнев Твой и негодование Твое от града Твоего, Иерусалима, от святой горы Твоей; ибо за грехи наши и беззакония отцов наших Иерусалим и народ Твой в поругании у всех, окружающих нас.
\vs Dan 9:17 И ныне услыши, Боже наш, молитву раба Твоего и моление его и воззри светлым лицем Твоим на опустошенное святилище Твое, ради Тебя, Господи.
\vs Dan 9:18 Приклони, Боже мой, ухо Твое и услыши, открой очи Твои и воззри на опустошения наши и на город, на котором наречено имя Твое; ибо мы повергаем моления наши пред Тобою, уповая не на праведность нашу, но на Твое великое милосердие.
\vs Dan 9:19 Господи! услыши; Господи! прости; Господи! внемли и соверши, не умедли ради Тебя Самого, Боже мой, ибо Твое имя наречено на городе Твоем и на народе Твоем>>.
\vs Dan 9:20 И когда я еще говорил и молился, и исповедовал грехи мои и грехи народа моего, Израиля, и повергал мольбу мою пред Господом Богом моим о святой горе Бога моего;
\vs Dan 9:21 когда я еще продолжал молитву, муж Гавриил, которого я видел прежде в видении, быстро прилетев, коснулся меня около времени вечерней жертвы
\vs Dan 9:22 и вразумлял меня, говорил со мною и сказал: <<Даниил! теперь я исшел, чтобы научить тебя разумению.
\vs Dan 9:23 В начале моления твоего вышло слово, и я пришел возвестить \bibemph{его тебе}, ибо ты муж желаний; итак вникни в слово и уразумей видение.
\vs Dan 9:24 Семьдесят седмин определены для народа твоего и святаго города твоего, чтобы покрыто было преступление, запечатаны были грехи и заглажены беззакония, и чтобы приведена была правда вечная, и запечатаны были видение и пророк, и помазан был Святый святых.
\vs Dan 9:25 Итак знай и разумей: с того времени, как выйдет повеление о восстановлении Иерусалима, до Христа Владыки семь седмин и шестьдесят две седмины; и возвратится \bibemph{народ} и обстроятся улицы и стены, но в трудные времена.
\vs Dan 9:26 И по истечении шестидесяти двух седмин предан будет смерти Христос, и не будет; а город и святилище разрушены будут народом вождя, который придет, и конец его будет как от наводнения, и до конца войны будут опустошения.
\vs Dan 9:27 И утвердит завет для многих одна седмина, а в половине седмины прекратится жертва и приношение, и на крыле \bibemph{святилища} будет мерзость запустения, и окончательная предопределенная гибель постигнет опустошителя>>.
\vs Dan 10:1 В третий год Кира, царя Персидского, было откровение Даниилу, который назывался именем Валтасара; и истинно было это откровение и великой силы. Он понял это откровение и уразумел это видение.
\vs Dan 10:2 В эти дни я, Даниил, был в сетовании три седмицы дней.
\vs Dan 10:3 Вкусного хлеба я не ел; мясо и вино не входило в уста мои, и мастями я не умащал себя до исполнения трех седмиц дней.
\vs Dan 10:4 А в двадцать четвертый день первого месяца был я на берегу большой реки Тигра,
\vs Dan 10:5 и поднял глаза мои, и увидел: вот один муж, облеченный в льняную одежду, и чресла его опоясаны золотом из Уфаза.
\vs Dan 10:6 Тело его~--- как топаз, лице его~--- как вид молнии; очи его~--- как горящие светильники, руки его и ноги его по виду~--- как блестящая медь, и глас речей его~--- как голос множества людей.
\vs Dan 10:7 И только один я, Даниил, видел это видение, а бывшие со мною люди не видели этого видения; но сильный страх напал на них и они убежали, чтобы скрыться.
\vs Dan 10:8 И остался я один и смотрел на это великое видение, но во мне не осталось крепости, и вид лица моего чрезвычайно изменился, не стало во мне бодрости.
\vs Dan 10:9 И услышал я глас слов его; и как только услышал глас слов его, в оцепенении пал я на лице мое и лежал лицем к земле.
\vs Dan 10:10 Но вот, коснулась меня рука и поставила меня на колени мои и на длани рук моих.
\vs Dan 10:11 И сказал он мне: <<Даниил, муж желаний! вникни в слова, которые я скажу тебе, и стань прямо на ноги твои; ибо к тебе я послан ныне>>. Когда он сказал мне эти слова, я встал с трепетом.
\vs Dan 10:12 Но он сказал мне: <<не бойся, Даниил; с первого дня, как ты расположил сердце твое, чтобы достигнуть разумения и смирить тебя пред Богом твоим, слова твои услышаны, и я пришел бы по словам твоим.
\vs Dan 10:13 Но князь царства Персидского стоял против меня двадцать один день; но вот, Михаил, один из первых князей, пришел помочь мне, и я остался там при царях Персидских.
\vs Dan 10:14 А теперь я пришел возвестить тебе, что будет с народом твоим в последние времена, так как видение относится к отдаленным дням>>.
\vs Dan 10:15 Когда он говорил мне такие слова, я припал лицем моим к земле и онемел.
\vs Dan 10:16 Но вот, некто, по виду похожий на сынов человеческих, коснулся уст моих, и я открыл уста мои, стал говорить и сказал стоящему передо мною: <<господин мой! от этого видения внутренности мои повернулись во мне, и не стало во мне силы.
\vs Dan 10:17 И как может говорить раб такого господина моего с таким господином моим? ибо во мне нет силы, и дыхание замерло во мне>>.
\vs Dan 10:18 Тогда снова прикоснулся ко мне тот человеческий облик и укрепил меня
\vs Dan 10:19 и сказал: <<не бойся, муж желаний! мир тебе; мужайся, мужайся!>> И когда он говорил со мною, я укрепился и сказал: <<говори, господин мой; ибо ты укрепил меня>>.
\vs Dan 10:20 И он сказал: <<знаешь ли, для чего я пришел к тебе? Теперь я возвращусь, чтобы бороться с князем Персидским; а когда я выйду, то вот, придет князь Греции.
\vs Dan 10:21 Впрочем я возвещу тебе, что начертано в истинном писании; и нет никого, кто поддерживал бы меня в том, кроме Михаила, князя вашего.
\vs Dan 11:1 Итак я с первого года Дария Мидянина стал ему подпорою и подкреплением.
\vs Dan 11:2 Теперь возвещу тебе истину: вот, еще три царя восстанут в Персии; потом четвертый превзойдет всех великим богатством, и когда усилится богатством своим, то поднимет всех против царства Греческого.
\vs Dan 11:3 И восстанет царь могущественный, который будет владычествовать с великою властью, и будет действовать по своей воле.
\vs Dan 11:4 Но когда он восстанет, царство его разрушится и разделится по четырем ветрам небесным, и не к его потомкам перейдет, и не с тою властью, с какою он владычествовал; ибо раздробится царство его и достанется другим, кроме этих.
\vs Dan 11:5 И усилится южный царь и один из князей его пересилит его и будет владычествовать, и велико будет владычество его.
\vs Dan 11:6 Но через несколько лет они сблизятся, и дочь южного царя придет к царю северному, чтобы установить правильные отношения между ними; но она не удержит силы в руках своих, не устоит и род ее, но преданы будут как она, так и сопровождавшие ее, и рожденный ею, и помогавшие ей в те времена.
\vs Dan 11:7 Но восстанет отрасль от корня ее, придет к войску и войдет в укрепления царя северного, и будет действовать в них, и усилится.
\vs Dan 11:8 Даже и богов их, истуканы их с драгоценными сосудами их, серебряными и золотыми, увезет в плен в Египет и на несколько лет будет стоять выше царя северного.
\vs Dan 11:9 Хотя этот и сделает нашествие на царство южного царя, но возвратится в свою землю.
\vs Dan 11:10 Потом вооружатся сыновья его и соберут многочисленное войско, и один из них быстро пойдет, наводнит и пройдет, и потом, возвращаясь, будет сражаться с ним до укреплений его.
\vs Dan 11:11 И раздражится южный царь, и выступит, сразится с ним, с царем северным, и выставит большое войско, и предано будет войско в руки его.
\vs Dan 11:12 И ободрится войско, и сердце \bibemph{царя} вознесется; он низложит многие тысячи, но от этого не будет сильнее.
\vs Dan 11:13 Ибо царь северный возвратится и выставит войско больше прежнего, и через несколько лет быстро придет с огромным войском и большим богатством.
\vs Dan 11:14 В те времена многие восстанут против южного царя, и мятежные из сынов твоего народа поднимутся, чтобы исполнилось видение, и падут.
\vs Dan 11:15 И придет царь северный, устроит вал и овладеет укрепленным городом, и не устоят мышцы юга, ни отборное войско его; недостанет силы противостоять.
\vs Dan 11:16 И кто выйдет к нему, будет действовать по воле его, и никто не устоит перед ним; и на славной земле поставит стан свой, и она пострадает от руки его.
\vs Dan 11:17 И вознамерится войти со всеми силами царства своего, и праведные с ним, и совершит это; и дочь жен отдаст ему, на погибель ее, но этот замысел не состоится, и ему не будет пользы из того.
\vs Dan 11:18 Потом обратит лице свое к островам и овладеет многими; но некий вождь прекратит нанесенный им позор и даже свой позор обратит на него.
\vs Dan 11:19 Затем он обратит лице свое на крепости своей земли; но споткнется, падет и не станет его.
\vs Dan 11:20 На место его восстанет некий, который пошлет сборщика податей, пройти по царству славы; но и он после немногих дней погибнет, и не от возмущения и не в сражении.
\vs Dan 11:21 И восстанет на место его презренный, и не воздадут ему царских почестей, но он придет без шума и лестью овладеет царством.
\vs Dan 11:22 И всепотопляющие полчища будут потоплены и сокрушены им, даже и сам вождь завета.
\vs Dan 11:23 Ибо после того, как он вступит в союз с ним, он будет действовать обманом, и взойдет, и одержит верх с малым народом.
\vs Dan 11:24 Он войдет в мирные и плодоносные страны, и совершит то, чего не делали отцы его и отцы отцов его; добычу, награбленное имущество и богатство будет расточать своим и на крепости будет иметь замыслы свои, но только до времени.
\vs Dan 11:25 Потом возбудит силы свои и дух свой с многочисленным войском против царя южного, и южный царь выступит на войну с великим и еще более сильным войском, но не устоит, потому что будет против него коварство.
\vs Dan 11:26 Даже участники трапезы его погубят его, и войско его разольется, и падет много убитых.
\vs Dan 11:27 У обоих царей сих на сердце будет коварство, и за одним столом будут говорить ложь, но успеха не будет, потому что конец еще отложен до времени.
\vs Dan 11:28 И отправится он в землю свою с великим богатством и враждебным намерением против святаго завета, и он исполнит его, и возвратится в свою землю.
\vs Dan 11:29 В назначенное время опять пойдет он на юг; но последний \bibemph{поход} не такой будет, как прежний,
\vs Dan 11:30 ибо в одно время с ним придут корабли Киттимские; и он упадет духом, и возвратится, и озлобится на святый завет, и исполнит свое намерение, и опять войдет в соглашение с отступниками от святаго завета.
\vs Dan 11:31 И поставлена будет им часть войска, которая осквернит святилище могущества, и прекратит ежедневную жертву, и поставит мерзость запустения.
\vs Dan 11:32 Поступающих нечестиво против завета он привлечет к себе лестью; но люди, чтущие своего Бога, усилятся и будут действовать.
\vs Dan 11:33 И разумные из народа вразумят многих, хотя будут несколько времени страдать от меча и огня, от плена и грабежа;
\vs Dan 11:34 и во время страдания своего будут иметь некоторую помощь, и многие присоединятся к ним, но притворно.
\vs Dan 11:35 Пострадают некоторые и из разумных для испытания их, очищения и для убеления к последнему времени; ибо есть еще время до срока.
\vs Dan 11:36 И будет поступать царь тот по своему произволу, и вознесется и возвеличится выше всякого божества, и о Боге богов станет говорить хульное и будет иметь успех, доколе не совершится гнев: ибо, что предопределено, то исполнится.
\vs Dan 11:37 И о богах отцов своих он не помыслит, и ни желания жен, ни даже божества никакого не уважит; ибо возвеличит себя выше всех.
\vs Dan 11:38 Но богу крепостей на месте его будет он воздавать честь, и этого бога, которого не знали отцы его, он будет чествовать золотом и серебром, и дорогими камнями, и разными драгоценностями,
\vs Dan 11:39 и устроит твердую крепость с чужим богом: которые призн\acc{а}ют его, тем увеличит почести и даст власть над многими, и землю раздаст в награду.
\vs Dan 11:40 Под конец же времени сразится с ним царь южный, и царь северный устремится как буря на него с колесницами, всадниками и многочисленными кораблями, и нападет на области, наводнит их, и пройдет через них.
\vs Dan 11:41 И войдет он в прекраснейшую из земель, и многие области пострадают и спасутся от руки его только Едом, Моав и большая часть сынов Аммоновых.
\vs Dan 11:42 И прострет руку свою на разные страны; не спасется и земля Египетская.
\vs Dan 11:43 И завладеет он сокровищами золота и серебра и разными драгоценностями Египта; Ливийцы и Ефиопляне последуют за ним.
\vs Dan 11:44 Но слухи с востока и севера встревожат его, и выйдет он в величайшей ярости, чтобы истреблять и губить многих,
\vs Dan 11:45 и раскинет он царские шатры свои между морем и горою преславного святилища; но придет к своему концу, и никто не поможет ему.
\vs Dan 12:1 И восстанет в то время Михаил, князь великий, стоящий за сынов народа твоего; и наступит время тяжкое, какого не бывало с тех пор, как существуют люди, до сего времени; но спасутся в это время из народа твоего все, которые найдены будут записанными в книге.
\vs Dan 12:2 И многие из спящих в прахе земли пробудятся, одни для жизни вечной, другие на вечное поругание и посрамление.
\vs Dan 12:3 И разумные будут сиять, как светила на тверди, и обратившие многих к правде~--- как звезды, вовеки, навсегда.
\vs Dan 12:4 А ты, Даниил, сокрой слова сии и запечатай книгу сию до последнего времени; многие прочитают ее, и умножится ведение>>.
\vs Dan 12:5 Тогда я, Даниил, посмотрел, и вот, стоят двое других, один на этом берегу реки, другой на том берегу реки.
\vs Dan 12:6 И \bibemph{один} сказал мужу в льняной одежде, который стоял над водами реки: <<когда будет конец этих чудных происшествий?>>
\vs Dan 12:7 И слышал я, как муж в льняной одежде, находившийся над водами реки, подняв правую и левую руку к небу, клялся Живущим вовеки, что к концу времени и времен и полувремени, и по совершенном низложении силы народа святого, все это совершится.
\vs Dan 12:8 Я слышал это, но не понял, и потому сказал: <<господин мой! что же после этого будет?>>
\vs Dan 12:9 И отвечал он: <<иди, Даниил; ибо сокрыты и запечатаны слова сии до последнего времени.
\vs Dan 12:10 Многие очистятся, убелятся и переплавлены будут \bibemph{в искушении}; нечестивые же будут поступать нечестиво, и не уразумеет сего никто из нечестивых, а мудрые уразумеют.
\vs Dan 12:11 Со времени прекращения ежедневной жертвы и поставления мерзости запустения пройдет тысяча двести девяносто дней.
\vs Dan 12:12 Блажен, кто ожидает и достигнет тысячи трехсот тридцати пяти дней.
\vs Dan 12:13 А ты иди к твоему концу и упокоишься, и восстанешь для получения твоего жребия в конце дней>>.
\vs Dan 13:1 \fns{13-я и 14-я главы переведены с греческого, потому что в еврейском тексте их нет.}В Вавилоне жил муж, по имени Иоаким.
\vs Dan 13:2 И взял он жену, по имени Сусанну, дочь Хелкия, очень красивую и богобоязненную.
\vs Dan 13:3 Родители ее были праведные и научили дочь свою закону Моисееву.
\vs Dan 13:4 Иоаким был очень богат, и был у него сад близ дома его; и сходились к нему Иудеи, потому что он был почетнейший из всех.
\vs Dan 13:5 И были поставлены два старца из народа судьями в том году, о которых Господь сказал, что беззаконие вышло из Вавилона от старейшин-судей, которые казались управляющими народом.
\vs Dan 13:6 Они постоянно бывали в доме Иоакима, и к ним приходили все, имевшие спорные дела.
\vs Dan 13:7 Когда народ уходил около полудня, Сусанна входила в сад своего мужа для прогулки.
\vs Dan 13:8 И видели ее оба старейшины всякий день приходящую и прогуливающуюся, и в них родилась похоть к ней,
\vs Dan 13:9 и извратили ум свой, и уклонили глаза свои, чтобы не смотреть на небо и не вспоминать о праведных судах.
\vs Dan 13:10 Оба они были уязвлены похотью к ней, но не открывали друг другу боли своей,
\vs Dan 13:11 потому что стыдились объявить о вожделении своем, что хотели совокупиться с нею.
\vs Dan 13:12 И они прилежно сторожили каждый день, чтобы видеть ее, и говорили друг другу:
\vs Dan 13:13 <<пойдем домой, потому что час обеда>>,~--- и, выйдя, расходились друг от друга,
\vs Dan 13:14 и, возвратившись, приходили на то же самое место, и когда допытывались друг у друга о причине того, признались в похоти своей, и тогда вместе назначили время, когда могли бы найти ее одну.
\vs Dan 13:15 И было, когда они выжидали удобного дня, Сусанна вошла, как вчера и третьего дня, с двумя только служанками и захотела мыться в саду, потому что было жарко.
\vs Dan 13:16 И не было там никого, кроме двух старейшин, которые спрятались и сторожили ее.
\vs Dan 13:17 И сказала она служанкам: принесите мне масла и мыла, и заприте двери сада, чтобы мне помыться.
\vs Dan 13:18 Они так и сделали, как она сказала: заперли двери сада и вышли боковыми дверями, чтобы принести, что приказано было им, и не видали старейшин, потому что они спрятались.
\vs Dan 13:19 И вот, когда служанки вышли, встали оба старейшины, и прибежали к ней, и сказали:
\vs Dan 13:20 Вот, двери сада заперты и никто нас не видит, и мы имеем похотение к тебе, поэтому согласись с нами и побудь с нами.
\vs Dan 13:21 Если же не так, то мы будем свидетельствовать против тебя, что с тобою был юноша, и ты поэтому отослала от себя служанок твоих.
\vs Dan 13:22 Тогда застонала Сусанна и сказала: тесно мне отовсюду; ибо, если я сделаю это, смерть мне, а если не сделаю, то не избегну от рук ваших.
\vs Dan 13:23 Лучше для меня не сделать этого и впасть в руки ваши, нежели согрешить пред Господом.
\vs Dan 13:24 И закричала Сусанна громким голосом; закричали также и оба старейшины против нее,
\vs Dan 13:25 и один побежал и отворил двери сада.
\vs Dan 13:26 Когда же находившиеся в доме услышали крик в саду, вскочили боковыми дверями, чтобы видеть, что случилось с нею.
\vs Dan 13:27 И когда старейшины сказали слова свои, слуги ее чрезвычайно были пристыжены, потому что никогда ничего такого о Сусанне говорено не было.
\vs Dan 13:28 И было на другой день, когда собрался народ к Иоакиму, мужу ее, пришли и оба старейшины, полные беззаконного умысла против Сусанны, чтобы предать ее смерти.
\vs Dan 13:29 И сказали они перед народом: пошлите за Сусанною, дочерью Хелкия, женою Иоакима. И послали.
\vs Dan 13:30 И пришла она, и родители ее, и дети ее, и все родственники ее.
\vs Dan 13:31 Сусанна была очень нежна и красива лицем,
\vs Dan 13:32 и эти беззаконники приказали открыть \bibemph{лице} ее, так как оно было закрыто, чтобы насытиться красотою ее.
\vs Dan 13:33 Родственники же и все, которые смотрели на нее, плакали.
\vs Dan 13:34 А оба старейшины, встав посреди народа, положили руки на голову ее.
\vs Dan 13:35 Она же в слезах смотрела на небо, ибо сердце ее уповало на Господа.
\vs Dan 13:36 И сказали старейшины: когда мы ходили по саду одни, вошла эта с двумя служанками и затворила двери сада, и отослала служанок;
\vs Dan 13:37 и пришел к ней юноша, который скрывался там, и лег с нею.
\vs Dan 13:38 Мы находясь в углу сада и видя такое беззаконие, побежали на них,
\vs Dan 13:39 и увидели их совокупляющимися, и того не могли удержать, потому что он был сильнее нас и, отворив двери, выскочил.
\vs Dan 13:40 Но эту мы схватили и допрашивали: кто был этот юноша? но она не хотела объявить нам. Об этом мы свидетельствуем.
\vs Dan 13:41 И поверило им собрание, как старейшинам народа и судьям, и осудили ее на смерть.
\rsbpar\vs Dan 13:42 Возопила Сусанна громким голосом и сказала: Боже вечный, ведающий сокровенное и знающий все прежде бытия его!
\vs Dan 13:43 Ты знаешь, что они ложно свидетельствовали против меня, и вот, я умираю, не сделав ничего, что эти люди злостно выдумали на меня.
\vs Dan 13:44 И услышал Господь голос ее.
\vs Dan 13:45 И когда она ведена была на смерть, возбудил Бог святой дух молодого юноши, по имени Даниила,
\vs Dan 13:46 и он закричал громким голосом: чист я от крови ее!
\vs Dan 13:47 Тогда обратился к нему весь народ и сказал: что это за слово, которое ты сказал?
\vs Dan 13:48 Тогда он, став посреди них, сказал: так ли вы неразумны, сыны Израиля, что, не исследовав и не узнав истины, осудили дочь Израиля?
\vs Dan 13:49 Возвратитесь в суд, ибо эти ложно против нее засвидетельствовали.
\vs Dan 13:50 И тотчас весь народ возвратился, и сказали ему старейшины: садись посреди нас и объяви нам, потому что Бог дал тебе старейшинство.
\vs Dan 13:51 И сказал им Даниил: отделите их друг от друга подальше, и я допрошу их.
\vs Dan 13:52 Когда же они отделены были один от другого, призвал одного из них и сказал ему: состарившийся в злых днях! ныне обнаружились грехи твои, которые ты делал прежде,
\vs Dan 13:53 производя суды неправедные, осуждая невинных и оправдывая виновных, тогда как Господь говорит: <<невинного и правого не умерщвляй>>.
\vs Dan 13:54 Итак, если ты сию видел, скажи, под каким деревом видел ты их разговаривающими друг с другом? Он сказал: под мастиковым.
\vs Dan 13:55 Даниил сказал: точно, солгал ты на твою голову; ибо вот, Ангел Божий, приняв решение от Бога, рассечет тебя пополам.
\vs Dan 13:56 Удалив его, он приказал привести другого и сказал ему: племя Ханаана, а не Иуды! красота прельстила тебя, и похоть развратила сердце твое.
\vs Dan 13:57 Так поступали вы с дочерями Израиля, и они из страха имели общение с вами; но дочь Иуды не потерпела беззакония вашего.
\vs Dan 13:58 Итак скажи мне: под каким деревом ты застал их разговаривающими между собою? Он сказал: под зеленым дубом.
\vs Dan 13:59 Даниил сказал ему: точно, солгал ты на твою голову; ибо Ангел Божий с мечом ждет, чтобы рассечь тебя пополам, чтобы истребить вас.
\vs Dan 13:60 Тогда все собрание закричало громким голосом, и благословили Бога, спасающего надеющихся на Него,
\vs Dan 13:61 и восстали на обоих старейшин, потому что Даниил их устами обличил их, что они ложно свидетельствовали;
\vs Dan 13:62 и поступили с ними так, как они злоумыслили против ближнего, по закону Моисееву, и умертвили их; и спасена была в тот день кровь невинная.
\vs Dan 13:63 Хелкия же и жена его прославили Бога за дочь свою Сусанну с Иоакимом, мужем ее, и со всеми родственниками, потому что не найдено было в ней постыдного дела.
\vs Dan 13:64 И Даниил стал велик перед народом с того дня и потом.
\vs Dan 14:1 Царь Астиаг приложился к отцам своим, и Кир, Персиянин, принял царство его.
\vs Dan 14:2 И Даниил жил вместе с царем и был славнее всех друзей его.
\vs Dan 14:3 Был у Вавилонян идол, по имени Вил, и издерживали на него каждый день двадцать больших мер пшеничной муки, сорок овец и вина шесть мер.
\vs Dan 14:4 Царь чтил его и ходил каждый день поклоняться ему; Даниил же поклонялся Богу своему. И сказал ему царь: почему ты не поклоняешься Вилу?
\vs Dan 14:5 Он отвечал: потому что я не поклоняюсь идолам, сделанным руками, но \bibemph{поклоняюсь} живому Богу, сотворившему небо и землю и владычествующему над всякою плотью.
\vs Dan 14:6 Царь сказал: не думаешь ли ты, что Вил неживой бог? не видишь ли, сколько он ест и пьет каждый день?
\vs Dan 14:7 Даниил, улыбнувшись, сказал: не обманывайся, царь; ибо он внутри глина, а снаружи медь, и никогда ни ел, ни пил.
\vs Dan 14:8 Тогда царь, разгневавшись, призвал жрецов своих и сказал им: если вы не скажете мне, кто съедает все это, то умрете.
\vs Dan 14:9 Если же вы докажете мне, что съедает это Вил, то умрет Даниил, потому что произнес хулу на Вила. И сказал Даниил царю: да будет по слову твоему.
\vs Dan 14:10 Жрецов Вила было семьдесят, кроме жен и детей.
\vs Dan 14:11 И пришел царь с Даниилом в храм Вила, и сказали жрецы Вила: вот, мы выйдем вон, а ты, царь, поставь пищу и, налив вина, запри двери и запечатай перстнем твоим.
\vs Dan 14:12 И если завтра ты придешь и не найдешь, что все съедено Вилом, мы умрем, или Даниил, который солгал на нас.
\vs Dan 14:13 Они не обращали на это внимания, потому что под столом сделали потаенный вход, и им всегда входили, и съедали это.
\vs Dan 14:14 Когда они вышли, царь поставил пищу перед Вилом, а Даниил приказал слугам своим, и они принесли пепел, и посыпали весь храм в присутствии одного царя, и, выйдя, заперли двери, и запечатали царским перстнем, и отошли.
\vs Dan 14:15 Жрецы же, по обычаю своему, пришли ночью с женами и детьми своими, и все съели и выпили.
\vs Dan 14:16 На другой день царь встал рано и Даниил с ним,
\vs Dan 14:17 и сказал: целы ли печати, Даниил? Он сказал: целы, царь.
\vs Dan 14:18 И как скоро отворены были двери, царь, взглянув на стол, воскликнул громким голосом: велик ты, Вил, и нет никакого обмана в тебе!
\vs Dan 14:19 Даниил, улыбнувшись, удержал царя, чтобы он не входил внутрь, и сказал: посмотри на пол и заметь, чьи это следы.
\vs Dan 14:20 Царь сказал: вижу следы мужчин, женщин и детей.
\vs Dan 14:21 И, разгневавшись, царь приказал схватить жрецов, жен их и детей и они показали потаенные двери, которыми они входили и съедали, что было на столе.
\vs Dan 14:22 Тогда царь повелел умертвить их и отдал Вила Даниилу, и он разрушил его и храм его.
\rsbpar\vs Dan 14:23 Был на том месте большой дракон, и Вавилоняне чтили его.
\vs Dan 14:24 И сказал царь Даниилу: не скажешь ли и об этом, что он медь? вот, он живой, и ест и пьет; ты не можешь сказать, что этот бог неживой; итак поклонись ему.
\vs Dan 14:25 Даниил сказал: Господу Богу моему поклоняюсь, потому что Он Бог живой.
\vs Dan 14:26 Но ты, царь, дай мне позволение, и я умерщвлю дракона без меча и жезла. Царь сказал: даю тебе.
\vs Dan 14:27 Тогда Даниил взял смолы, жира и вол\acc{о}с, сварил это вместе и, сделав из этого ком, бросил его в пасть дракону, и дракон расселся. И сказал \bibemph{Даниил}: вот ваши святыни!
\vs Dan 14:28 Когда же Вавилоняне услышали о том, сильно вознегодовали и восстали против царя, и сказали: царь сделался Иудеем, Вила разрушил и убил дракона, и предал смерти жрецов,
\vs Dan 14:29 и, придя к царю, сказали: предай нам Даниила, иначе мы умертвим тебя и дом твой.
\vs Dan 14:30 И когда царь увидел, что они сильно настаивают, принужден был предать им Даниила,
\vs Dan 14:31 они же бросили его в ров львиный, и он пробыл там шесть дней.
\vs Dan 14:32 Во рве было семь львов, и давалось им каждый день по два тела и по две овцы; в это время им не давали их, чтобы они съели Даниила.
\rsbpar\vs Dan 14:33 Был в Иудее пророк Аввакум, который, сварив похлебку и накрошив хлеба в блюдо, шел на поле, чтобы отнести это жнецам.
\vs Dan 14:34 Но Ангел Господень сказал Аввакуму: отнеси этот обед, который у тебя, в Вавилон к Даниилу, в ров львиный.
\vs Dan 14:35 Аввакум сказал: господин! Вавилона я \bibemph{никогда} не видал и рва не знаю.
\vs Dan 14:36 Тогда Ангел Господень взял его за темя и, подняв его за волосы головы его, поставил его в Вавилоне над рвом силою духа своего.
\vs Dan 14:37 И воззвал Аввакум и сказал: Даниил! Даниил! возьми обед, который Бог послал тебе.
\vs Dan 14:38 Даниил сказал: вспомнил Ты обо мне, Боже, и не оставил любящих Тебя.
\vs Dan 14:39 И встал Даниил и ел; Ангел же Божий мгновенно поставил Аввакума на его место.
\rsbpar\vs Dan 14:40 В седьмой день пришел царь, чтобы поскорбеть о Данииле и, подойдя ко рву, взглянул в него, и вот, Даниил сидел.
\vs Dan 14:41 И воскликнул царь громким голосом, и сказал: велик Ты, Господь Бог Даниилов, и нет иного кроме Тебя!
\vs Dan 14:42 И приказал вынуть \bibemph{Даниила}, а виновников его погубления бросить в ров,~--- и они тотчас были съедены в присутствии его.
\newbookpage
\bibbookdescr{Hos}{
  inline={\LARGE Книга\\\Huge Пророка Осии},
  toc={Осия},
  bookmark={Осия},
  header={Осия},
  %headerleft={},
  %headerright={},
  abbr={Ос}
}
\vs Hos 1:1 Слово Господне, которое было к Осии, сыну Беериину, во дни Озии, Иоафама, Ахаза, Езекии, царей Иудейских, и во дни Иеровоама, сына Иоасова, царя Израильского.
\rsbpar\vs Hos 1:2 Начало слова Господня к Осии. И сказал Господь Осии: иди, возьми себе жену блудницу и детей блуда; ибо сильно блудодействует земля сия, отступив от Господа.
\vs Hos 1:3 И пошел он и взял Гомерь, дочь Дивлаима; и она зачала и родила ему сына.
\vs Hos 1:4 И Господь сказал ему: нареки ему имя Изреель, потому что еще немного пройдет, и Я взыщу кровь Изрееля с дома Ииуева, и положу конец царству дома Израилева,
\vs Hos 1:5 и будет в тот день, Я сокрушу лук Израилев в долине Изреель.
\vs Hos 1:6 И зачала еще, и родила дочь, и Он сказал ему: нареки ей имя Лорухама\fns{Непомилованная.}; ибо Я уже не буду более миловать дома Израилева, чтобы прощать им.
\vs Hos 1:7 А дом Иудин помилую и спасу их в Господе Боге их, спасу их ни луком, ни мечом, ни войною, ни конями и всадниками.
\vs Hos 1:8 И, откормив грудью Непомилованную, она зачала, и родила сына.
\vs Hos 1:9 И сказал Он: нареки ему имя Лоамми\fns{Не Мой народ.}, потому что вы не Мой народ, и Я не буду вашим [Богом].
\vs Hos 1:10 Но будет число сынов Израилевых как песок морской, которого нельзя ни измерить, ни исчислить; и там, где говорили им: <<вы не Мой народ>>, будут говорить им: <<вы сыны Бога живаго>>.
\vs Hos 1:11 И соберутся сыны Иудины и сыны Израилевы вместе, и поставят себе одну главу, и выйдут из земли \bibemph{переселения}; ибо велик день Изрееля!
\vs Hos 2:1 Говорите братьям вашим: <<Мой народ>>, и сестрам вашим: <<Помилованная>>.
\vs Hos 2:2 Судитесь с вашею матерью, судитесь; ибо она не жена Моя, и Я не муж ее; пусть она удалит блуд от лица своего и прелюбодеяние от грудей своих,
\vs Hos 2:3 дабы Я не разоблачил ее донага и не выставил ее, как в день рождения ее, не сделал ее пустынею, не обратил ее в землю сухую и не уморил ее жаждою.
\vs Hos 2:4 И детей ее не помилую, потому что они дети блуда.
\vs Hos 2:5 Ибо блудодействовала мать их и осрамила себя зачавшая их; ибо говорила: <<пойду за любовниками моими, которые дают мне хлеб и воду, шерсть и лен, елей и напитки>>.
\vs Hos 2:6 За то вот, Я загорожу путь ее тернами и обнесу ее оградою, и она не найдет стезей своих,
\vs Hos 2:7 и погонится за любовниками своими, но не догонит их, и будет искать их, но не найдет, и скажет: <<пойду я, и возвращусь к первому мужу моему; ибо тогда лучше было мне, нежели теперь>>.
\vs Hos 2:8 А не знала она, что Я, Я давал ей хлеб и вино и елей и умножил у нее серебро и золото, из которого сделали \bibemph{истукана} Ваала.
\vs Hos 2:9 За то Я возьму назад хлеб Мой в его время и вино Мое в его пору и отниму шерсть и лен Мой, чем покрывается нагота ее.
\vs Hos 2:10 И ныне открою срамоту ее пред глазами любовников ее, и никто не исторгнет ее из руки Моей.
\vs Hos 2:11 И прекращу у нее всякое веселье, праздники ее и новомесячия ее, и субботы ее, и все торжества ее.
\vs Hos 2:12 И опустошу виноградные лозы ее и смоковницы ее, о которых она говорит: <<это у меня подарки, которые надарили мне любовники мои>>; и Я превращу их в лес, и полевые звери поедят их.
\vs Hos 2:13 И накажу ее за дни служения Ваалам, когда она кадила им и, украсив себя серьгами и ожерельями, ходила за любовниками своими, а Меня забывала, говорит Господь.
\vs Hos 2:14 Посему вот, и Я увлеку ее, приведу ее в пустыню, и буду говорить к сердцу ее.
\vs Hos 2:15 И дам ей оттуда виноградники ее и долину Ахор, в преддверие надежды; и она будет петь там, как во дни юности своей и как в день выхода своего из земли Египетской.
\vs Hos 2:16 И будет в тот день, говорит Господь, ты будешь звать Меня: <<муж мой>>, и не будешь более звать Меня: <<Ваали\fns{Господин мой.}>>.
\vs Hos 2:17 И удалю имена Ваалов от уст ее, и не будут более вспоминаемы имена их.
\vs Hos 2:18 И заключу в то время для них союз с полевыми зверями и с птицами небесными, и с пресмыкающимися по земле; и лук, и меч, и войну истреблю от земли той, и дам им жить в безопасности.
\vs Hos 2:19 И обручу тебя Мне навек, и обручу тебя Мне в правде и суде, в благости и милосердии.
\vs Hos 2:20 И обручу тебя Мне в верности, и ты познаешь Господа.
\vs Hos 2:21 И будет в тот день, Я услышу, говорит Господь, услышу небо, и оно услышит землю,
\vs Hos 2:22 и земля услышит хлеб и вино и елей; а сии услышат Изреель.
\vs Hos 2:23 И посею ее для Себя на земле, и помилую Непомилованную, и скажу не Моему народу: <<ты Мой народ>>, а он скажет: <<Ты мой Бог!>>
\vs Hos 3:1 И сказал мне Господь: иди еще, и полюби женщину, любимую мужем, но прелюбодействующую, подобно тому, как любит Господь сынов Израилевых, а они обращаются к другим богам и любят виноградные лепешки их.
\vs Hos 3:2 И приобрел я ее себе за пятнадцать сребреников и за хомер ячменя и полхомера ячменя
\vs Hos 3:3 и сказал ей: много дней оставайся у меня; не блуди, и не будь с другим; так же и я буду для тебя.
\vs Hos 3:4 Ибо долгое время сыны Израилевы будут оставаться без царя и без князя и без жертвы, без жертвенника, без ефода и терафима.
\vs Hos 3:5 После того обратятся сыны Израилевы и взыщут Господа Бога своего и Давида, царя своего, и будут благоговеть пред Господом и благостью Его в последние дни.
\vs Hos 4:1 Слушайте слово Господне, сыны Израилевы; ибо суд у Господа с жителями сей земли, потому что нет ни истины, ни милосердия, ни Богопознания на земле.
\vs Hos 4:2 Клятва и обман, убийство и воровство, и прелюбодейство крайне распространились, и кровопролитие следует за кровопролитием.
\vs Hos 4:3 За то восплачет земля сия, и изнемогут все, живущие на ней, со зверями полевыми и птицами небесными, даже и рыбы морские погибнут.
\vs Hos 4:4 Но никто не спорь, никто не обличай другого; и твой народ~--- как спорящие со священником.
\vs Hos 4:5 И ты падешь днем, и пророк падет с тобою ночью, и истреблю матерь твою.
\vs Hos 4:6 Истреблен будет народ Мой за недостаток ведения: так как ты отверг ведение, то и Я отвергну тебя от священнодействия предо Мною; и как ты забыл закон Бога твоего, то и Я забуду детей твоих.
\vs Hos 4:7 Чем больше они умножаются, тем больше грешат против Меня; славу их обращу в бесславие.
\vs Hos 4:8 Грехами народа Моего кормятся они, и к беззаконию его стремится душа их.
\vs Hos 4:9 И что будет с народом, то и со священником; и накажу его по путям его, и воздам ему по делам его.
\vs Hos 4:10 Будут есть, и не насытятся; будут блудить, и не размножатся; ибо оставили служение Господу.
\vs Hos 4:11 Блуд, вино и напитки завладели сердцем их.
\vs Hos 4:12 Народ Мой вопрошает свое дерево и жезл его дает ему ответ; ибо дух блуда ввел их в заблуждение, и, блудодействуя, они отступили от Бога своего.
\vs Hos 4:13 На вершинах гор они приносят жертвы и на холмах совершают каждение под дубом и тополем и теревинфом, потому что хороша от них тень; поэтому любодействуют дочери ваши и прелюбодействуют невестки ваши.
\vs Hos 4:14 Я оставлю наказывать дочерей ваших, когда они блудодействуют, и невесток ваших, когда они прелюбодействуют, потому что вы сами на стороне блудниц и с любодейцами приносите жертвы, а невежественный народ гибнет.
\vs Hos 4:15 Если ты, Израиль, блудодействуешь, то пусть не грешил бы Иуда; и не ходите в Галгал, и не восходите в Беф-Авен, и не клянитесь: <<жив Господь!>>
\vs Hos 4:16 Ибо как упрямая телица, упорен стал Израиль; посему будет ли теперь Господь пасти их, как агнцев на пространном пастбище?
\vs Hos 4:17 Привязался к идолам Ефрем; оставь его!
\vs Hos 4:18 Отвратительно пьянство их, совершенно предались блудодеянию; князья их любят постыдное.
\vs Hos 4:19 Охватит их ветер своими крыльями, и устыдятся они жертв своих.
\vs Hos 5:1 Слушайте это, священники, и внимайте, дом Израилев, и приклоните ухо, дом царя; ибо вам будет суд, потому что вы были западнею в Массифе и сетью, раскинутою на Фаворе.
\vs Hos 5:2 Глубоко погрязли они в распутстве; но Я накажу всех их.
\vs Hos 5:3 Ефрема Я знаю, и Израиль не сокрыт от Меня; ибо ты блудодействуешь, Ефрем, и Израиль осквернился.
\vs Hos 5:4 Дела их не допускают их обратиться к Богу своему, ибо дух блуда внутри них, и Господа они не познали.
\vs Hos 5:5 И гордость Израиля унижена в глазах их; и Израиль и Ефрем падут от нечестия своего; падет и Иуда с ними.
\vs Hos 5:6 С овцами своими и волами своими пойдут искать Господа и не найдут Его: Он удалился от них.
\vs Hos 5:7 Господу они изменили, потому что родили чужих детей; ныне новый месяц поест их с их имуществом.
\vs Hos 5:8 Вострубите рогом в Гиве, трубою в Раме; возглашайте в Беф-Авене: <<за тобою, Вениамин!>>
\vs Hos 5:9 Ефрем сделается пустынею в день наказания; между коленами Израилевыми Я возвестил это.
\vs Hos 5:10 Вожди Иудины стали подобны передвигающим межи: изолью на них гнев Мой, как воду.
\vs Hos 5:11 Угнетен Ефрем, поражен судом; ибо захотел ходить вслед суетных.
\vs Hos 5:12 И буду как моль для Ефрема и как червь для дома Иудина.
\vs Hos 5:13 И увидел Ефрем болезнь свою, и Иуда~--- свою рану, и пошел Ефрем к Ассуру, и послал к царю Иареву; но он не может исцелить вас, и не излечит вас от раны.
\vs Hos 5:14 Ибо Я как лев для Ефрема и как скимен для дома Иудина; Я, Я растерзаю, и уйду; унесу, и никто не спасет.
\vs Hos 5:15 Пойду, возвращусь в Мое место, доколе они не признают себя виновными и не взыщут лица Моего.
\vs Hos 6:1 В скорби своей они с раннего утра будут искать Меня и говорить: <<пойдем и возвратимся к Господу! ибо Он уязвил~--- и Он исцелит нас, поразил~--- и перевяжет наши раны;
\vs Hos 6:2 оживит нас через два дня, в третий день восставит нас, и мы будем жить пред лицем Его.
\vs Hos 6:3 Итак познаем, будем стремиться познать Господа; как утренняя заря~--- явление Его, и Он придет к нам, как дождь, как поздний дождь оросит землю>>.
\vs Hos 6:4 Что сделаю тебе, Ефрем? что сделаю тебе, Иуда? благочестие ваше, как утренний туман и как роса, скоро исчезающая.
\vs Hos 6:5 Посему Я поражал через пророков и бил их словами уст Моих, и суд Мой, как восходящий свет.
\vs Hos 6:6 Ибо Я милости хочу, а не жертвы, и Боговедения более, нежели всесожжений.
\vs Hos 6:7 Они же, подобно Адаму, нарушили завет и там изменили Мне.
\vs Hos 6:8 Галаад~--- город нечестивцев, запятнанный кровью.
\vs Hos 6:9 Как разбойники подстерегают человека, так сборище священников убивают на пути в Сихем и совершают мерзости.
\vs Hos 6:10 В доме Израиля Я вижу ужасное; там блудодеяние у Ефрема, осквернился Израиль.
\vs Hos 6:11 И тебе, Иуда, назначена жатва, когда Я возвращу плен народа Моего.
\vs Hos 7:1 Когда Я врачевал Израиля, открылась неправда Ефрема и злодейство Самарии: ибо они поступают лживо; и входит вор, и разбойник грабит по улицам.
\vs Hos 7:2 Не помышляют они в сердце своем, что Я помню все злодеяния их; теперь окружают их дела их; они пред лицем Моим.
\vs Hos 7:3 Злодейством своим они увеселяют царя и обманами своими~--- князей.
\vs Hos 7:4 Все они пылают прелюбодейством, как печь, растопленная пекарем, который перестает поджигать ее, когда замесит тесто и оно вскиснет.
\vs Hos 7:5 <<День нашего царя!>> \bibemph{говорят} князья, разгоряченные до болезни вином, а он протягивает руку свою к кощунам.
\vs Hos 7:6 Ибо они коварством своим делают сердце свое подобным печи: пекарь их спит всю ночь, а утром она горит, как пылающий огонь.
\vs Hos 7:7 Все они распалены, как печь, и пожирают судей своих; все цари их падают, и никто из них не взывает ко Мне.
\vs Hos 7:8 Ефрем смешался с народами, Ефрем стал, как неповороченный хлеб.
\vs Hos 7:9 Чужие пожирали силу его и он не замечал; седина покрыла его, а он не знает.
\vs Hos 7:10 И гордость Израиля унижена в глазах их и при всем том они не обратились к Господу Богу своему и не взыскали Его.
\vs Hos 7:11 И стал Ефрем, как глупый голубь, без сердца: зовут Египтян, идут в Ассирию.
\vs Hos 7:12 Когда они пойдут, Я закину на них сеть Мою; как птиц небесных низвергну их; накажу их, как слышало собрание их.
\vs Hos 7:13 Горе им, что они удалились от Меня; гибель им, что они отпали от Меня! Я спасал их, а они ложь говорили на Меня.
\vs Hos 7:14 И не взывали ко Мне сердцем своим, когда вопили на ложах своих; собираются из-за хлеба и вина, а от Меня удаляются.
\vs Hos 7:15 Я вразумлял \bibemph{их} и укреплял мышцы их, а они умышляли злое против Меня.
\vs Hos 7:16 Они обращались, но не к Всевышнему, стали~--- как неверный лук; падут от меча князья их за дерзость языка своего; это будет посмеянием над ними в земле Египетской.
\vs Hos 8:1 Трубу к устам твоим! Как орел \bibemph{налетит} на дом Господень за то, что они нарушили завет Мой и преступили закон Мой!
\vs Hos 8:2 Ко Мне будут взывать: <<Боже мой! мы познали Тебя, мы~--- Израиль>>.
\vs Hos 8:3 Отверг Израиль доброе; враг будет преследовать его.
\vs Hos 8:4 Поставляли царей сами, без Меня; ставили князей, но без Моего ведома; из серебра своего и золота своего сделали для себя идолов: оттуда гибель.
\vs Hos 8:5 Оставил тебя телец твой, Самария! воспылал гнев Мой на них; доколе не могут они очиститься?
\vs Hos 8:6 Ибо и он~--- дело Израиля: художник сделал его, и потому он не бог; в куски обратится телец Самарийский!
\vs Hos 8:7 Так как они сеяли ветер, то и пожнут бурю: хлеба на корню не будет у него; зерно не даст муки; а если и даст, то чужие проглотят ее.
\vs Hos 8:8 Поглощен Израиль; теперь они будут среди народов, как негодный сосуд.
\vs Hos 8:9 Они пошли к Ассуру, как дикий осел, одиноко бродящий; Ефрем приобретал подарками расположение к себе.
\vs Hos 8:10 Хотя они и посылали дары к народам, но скоро Я соберу их, и они начнут страдать от бремени царя князей;
\vs Hos 8:11 ибо много жертвенников настроил Ефрем для греха,~--- ко греху послужили ему эти жертвенники.
\vs Hos 8:12 Написал Я ему важные законы Мои, но они сочтены им как бы чужие.
\vs Hos 8:13 В жертвоприношениях Мне они приносят мясо и едят его; Господу неугодны они; ныне Он вспомнит нечестие их и накажет их за грехи их: они возвратятся в Египет.
\vs Hos 8:14 Забыл Израиль Создателя своего и устроил капища, и Иуда настроил много укрепленных городов; но Я пошлю огонь на города его, и пожрет чертоги его.
\vs Hos 9:1 Не радуйся, Израиль, до восторга, как \bibemph{другие} народы, ибо ты блудодействуешь, удалившись от Бога твоего: любишь блудодейные дары на всех гумнах.
\vs Hos 9:2 Гумно и точило не будут питать их, и \bibemph{надежда} на виноградный сок обманет их.
\vs Hos 9:3 Не будут они жить на земле Господней: Ефрем возвратится в Египет, и в Ассирии будут есть нечистое.
\vs Hos 9:4 Не будут возливать Господу вина, и неугодны Ему будут жертвы их; они будут для них, как хлеб похоронный: все, которые будут есть его, осквернятся, ибо хлеб их~--- для души их, а в дом Господень он не войдет.
\vs Hos 9:5 Что будете делать в день торжества и в день праздника Господня?
\vs Hos 9:6 Ибо вот, они уйдут по причине опустошения; Египет соберет их, Мемфис похоронит их; драгоценностями их из серебра завладеет крапива, колючий терн будет в шатрах их.
\vs Hos 9:7 Пришли дни посещения, пришли дни воздаяния; да узнает Израиль, что глуп прорицатель, безумен выдающий себя за вдохновенного, по причине множества беззаконий твоих и великой враждебности.
\vs Hos 9:8 Ефрем~--- страж подле Бога моего; пророк~--- сеть птицелова на всех путях его; соблазн в доме Бога его.
\vs Hos 9:9 Глубоко упали они, развратились, как во дни Гивы; Он вспомнит нечестие их, накажет их за грехи их.
\vs Hos 9:10 Как виноград в пустыне, Я нашел Израиля; как первую ягоду на смоковнице, в первое время ее, увидел Я отцов ваших,~--- но они пошли к Ваал-Фегору и предались постыдному, и сами стали мерзкими, как те, которых возлюбили.
\vs Hos 9:11 У Ефремлян, как птица улетит слава [чадородия]: ни рождения, ни беременности, ни зачатия [не будет].
\vs Hos 9:12 А хотя бы они и воспитали детей своих, отниму их; ибо горе им, когда удалюсь от них!
\vs Hos 9:13 Ефрем, как Я видел его до Тира, насажден на прекрасной местности; однако Ефрем выведет детей своих к убийце.
\vs Hos 9:14 Дай им, Господи: что Ты дашь им? дай им утробу нерождающую и сухие сосцы.
\vs Hos 9:15 Все зло их в Галгале: там Я возненавидел их за злые дела их; изгоню их из дома Моего, не буду больше любить их; все князья их~--- отступники.
\vs Hos 9:16 Поражен Ефрем; иссох корень их,~--- не будут приносить они плода, а если и будут рождать, Я умерщвлю вожделенный плод утробы их.
\vs Hos 9:17 Отвергнет их Бог мой, потому что они не послушались Его, и будут скитальцами между народами.
\vs Hos 10:1 Израиль~--- ветвистый виноград, умножает для себя плод: чем более у него плодов, тем более умножает жертвенники; чем лучше земля у него, тем более украшают они кумиры.
\vs Hos 10:2 Разделилось сердце их, за то они и будут наказаны: Он разрушит жертвенники их, сокрушит кумиры их.
\vs Hos 10:3 Теперь они говорят: <<нет у нас царя, ибо мы не убоялись Господа; а царь,~--- чт\acc{о} он нам сделает?>>
\vs Hos 10:4 Говорят слова \bibemph{пустые}, клянутся ложно, заключают союзы; за то явится суд над ними, как ядовитая трава на бороздах поля.
\vs Hos 10:5 За тельца Беф-Авена вострепещут жители Самарии; восплачет о нем народ его, и жрецы его, радовавшиеся о нем, будут плакать о славе его, потому что она отойдет от него.
\vs Hos 10:6 И сам он отнесен будет в Ассирию, в дар царю Иареву; постыжен будет Ефрем, и посрамится Израиль от замысла своего.
\vs Hos 10:7 Исчезнет в Самарии царь ее, как пена на поверхности воды.
\vs Hos 10:8 И истреблены будут высоты Авена, грех Израиля; терние и волчцы вырастут на жертвенниках их, и скажут они горам: <<покройте нас>>, и холмам: <<падите на нас>>.
\vs Hos 10:9 Больше, нежели во дни Гивы, грешил ты, Израиль; там они устояли; война в Гаваоне против сынов нечестия не постигла их.
\vs Hos 10:10 По желанию Моему накажу их, и соберутся против них народы, и они будут связаны за двойное преступление их.
\vs Hos 10:11 Ефрем~--- обученная телица, привычная к молотьбе, и Я Сам возложу ярмо на тучную шею его; на Ефреме будут верхом ездить, Иуда будет пахать, Иаков будет боронить.
\vs Hos 10:12 Сейте себе в правду, и пожнете милость; распахивайте у себя новину, ибо время взыскать Господа, чтобы Он, когда придет, дождем пролил на вас правду.
\vs Hos 10:13 Вы возделывали нечестие, пожинаете беззаконие, едите плод лжи, потому что ты надеялся на путь твой, на множество ратников твоих.
\vs Hos 10:14 И произойдет смятение в народе твоем, и все твердыни твои будут разрушены, как Салман разрушил Бет-Арбел в день брани: мать была убита с детьми.
\vs Hos 10:15 Вот что причинит вам Вефиль за крайнее нечестие ваше.
\vs Hos 11:1 На заре погибнет царь Израилев! Когда Израиль был юн, Я любил его и из Египта вызвал сына Моего.
\vs Hos 11:2 Звали их, а они уходили прочь от лица их: приносили жертву Ваалам и кадили истуканам.
\vs Hos 11:3 Я Сам приучал Ефрема ходить, носил его на руках Своих, а они не сознавали, что Я врачевал их.
\vs Hos 11:4 Узами человеческими влек Я их, узами любви, и был для них как бы поднимающий ярмо с челюстей их, и ласково подкладывал пищу им.
\vs Hos 11:5 Не возвратится он в Египет, но Ассур~--- он будет царем его, потому что они не захотели обратиться \bibemph{ко Мне}.
\vs Hos 11:6 И падет меч на города его, и истребит затворы его, и пожрет их за умыслы их.
\vs Hos 11:7 Народ Мой закоснел в отпадении от Меня, и хотя призывают его к горнему, он не возвышается единодушно.
\vs Hos 11:8 Как поступлю с тобою, Ефрем? как предам тебя, Израиль? Поступлю ли с тобою, как с Адамою, сделаю ли тебе, что Севоиму? Повернулось во Мне сердце Мое, возгорелась вся жалость Моя!
\vs Hos 11:9 Не сделаю по ярости гнева Моего, не истреблю Ефрема, ибо Я Бог, а не человек; среди тебя Святый; Я не войду в город.
\vs Hos 11:10 Вслед Господа пойдут они; как лев, Он даст глас Свой, даст глас Свой, и встрепенутся к Нему сыны с запада,
\vs Hos 11:11 встрепенутся из Египта, как птицы, и из земли Ассирийской, как голуби, и вселю их в домы их, говорит Господь.
\vs Hos 11:12 Окружил Меня Ефрем ложью и дом Израилев лукавством; Иуда держался еще Бога и верен был со святыми.
\vs Hos 12:1 Ефрем пасет ветер и гоняется за восточным ветром, каждый день умножает ложь и разорение; заключают они союз с Ассуром, и в Египет отвозится елей.
\vs Hos 12:2 Но и с Иудою у Господа суд и Он посетит Иакова по путям его, воздаст ему по делам его.
\vs Hos 12:3 Еще во чреве матери запинал он брата своего, а возмужав боролся с Богом.
\vs Hos 12:4 Он боролся с Ангелом~--- и превозмог; плакал и умолял Его; в Вефиле Он нашел нас и там говорил с нами.
\vs Hos 12:5 А Господь есть Бог Саваоф; Сущий [Иегова]~--- имя Его.
\vs Hos 12:6 Обратись и ты к Богу твоему; наблюдай милость и суд и уповай на Бога твоего всегда.
\vs Hos 12:7 Хананеянин с неверными весами в руке любит обижать;
\vs Hos 12:8 и Ефрем говорит: <<однако я разбогател; накопил себе имущества, хотя во всех моих трудах не найдут ничего незаконного, что было бы грехом>>.
\vs Hos 12:9 А Я, Господь Бог твой от самой земли Египетской, опять поселю тебя в кущах, как во дни праздника.
\vs Hos 12:10 Я говорил к пророкам, и умножал видения, и чрез пророков употреблял притчи.
\vs Hos 12:11 Если Галаад сделался Авеном, то они стали суетны, в Галгалах заколали в жертву тельцов, и жертвенники их стояли как груды камней на межах поля.
\vs Hos 12:12 Убежал Иаков на поля Сирийские, и служил Израиль за жену, и за жену стерег \bibemph{овец}.
\vs Hos 12:13 Чрез пророка вывел Господь Израиля из Египта, и чрез пророка Он охранял его.
\vs Hos 12:14 Сильно раздражил Ефрем \bibemph{Господа} и за то кровь его оставит на нем, и поношение его обратит Господь на него.
\vs Hos 13:1 Когда Ефрем говорил, все трепетали. Он был высок в Израиле; но сделался виновным через Ваала, и погиб.
\vs Hos 13:2 И ныне прибавили они ко греху: сделали для себя литых истуканов из серебра своего, по понятию своему,~--- полная работа художников,~--- и говорят они приносящим жертву людям: <<целуйте тельцов!>>
\vs Hos 13:3 За то они будут как утренний туман, как роса, скоро исчезающая, как мякина, свеваемая с гумна, и как дым из трубы.
\vs Hos 13:4 Но Я~--- Господь Бог твой от земли Египетской,~--- и ты не должен знать другого бога, кроме Меня, и нет спасителя, кроме Меня.
\vs Hos 13:5 Я признал тебя в пустыне, в земле жаждущей.
\vs Hos 13:6 Имея пажити, они были сыты; а когда насыщались, то превозносилось сердце их, и потому они забывали Меня.
\vs Hos 13:7 И Я буду для них как лев, как скимен буду подстерегать при дороге.
\vs Hos 13:8 Буду нападать на них, как лишенная детей медведица, и раздирать вместилище сердца их, и поедать их там, как львица; полевые звери будут терзать их.
\vs Hos 13:9 Погубил ты себя, Израиль, ибо только во Мне опора твоя.
\vs Hos 13:10 Где царь твой теперь? Пусть он спасет тебя во всех городах твоих! Где судьи твои, о которых говорил ты: <<дай нам царя и начальников>>?
\vs Hos 13:11 И Я дал тебе царя во гневе Моем, и отнял в негодовании Моем.
\vs Hos 13:12 Связано в узел беззаконие Ефрема, сбережен его грех.
\vs Hos 13:13 М\acc{у}ки родильницы постигнут его; он~--- сын неразумный, иначе не стоял бы долго в положении рождающихся детей.
\vs Hos 13:14 От власти ада Я искуплю их, от смерти избавлю их. Смерть! где твое жало? ад! где твоя победа? Раскаяния в том не будет у Меня.
\vs Hos 13:15 Хотя \bibemph{Ефрем} плодовит между братьями, но придет восточный ветер, поднимется ветер Господень из пустыни, и иссохнет родник его, и иссякнет источник его; он опустошит сокровищницу всех драгоценных сосудов.
\vs Hos 14:1 Опустошена будет Самария, потому что восстала против Бога своего; от меча падут они; младенцы их будут разбиты, и беременные их будут рассечены.
\vs Hos 14:2 Обратись, Израиль, к Господу Богу твоему; ибо ты упал от нечестия твоего.
\vs Hos 14:3 Возьмите с собою \bibemph{молитвенные} слова и обратитесь к Господу; говорите Ему: <<отними всякое беззаконие и прими во благо, и мы принесем жертву уст наших.
\vs Hos 14:4 Ассур не будет уже спасать нас; не станем садиться на коня и не будем более говорить изделию рук наших: боги наши; потому что у Тебя милосердие для сирот>>.
\vs Hos 14:5 Уврачую отпадение их, возлюблю их по благоволению; ибо гнев Мой отвратился от них.
\vs Hos 14:6 Я буду росою для Израиля; он расцветет, как лилия, и пустит корни свои, как Ливан.
\vs Hos 14:7 Расширятся ветви его, и будет красота его, как маслины, и благоухание от него, как от Ливана.
\vs Hos 14:8 Возвратятся сидевшие под тенью его, будут изобиловать хлебом, и расцветут, как виноградная лоза, славны будут, как вино Ливанское.
\vs Hos 14:9 <<Что мне еще за дело до идолов?>>~--- скажет Ефрем.~--- Я услышу его и призрю на него; Я буду как зеленеющий кипарис; от Меня будут тебе плоды.
\vs Hos 14:10 Кто мудр, чтобы разуметь это? кто разумен, чтобы познать это? Ибо правы пути Господни, и праведники ходят по ним, а беззаконные падут на них.

\bibbookdescr{Joe}{
  inline={\LARGE Книга\\\Huge Пророка Иоиля},
  toc={Иоиль},
  bookmark={Иоиль},
  header={Иоиль},
  %headerleft={},
  %headerright={},
  abbr={Иоил}
}
\vs Joe 1:1 Слово Господне, которое было к Иоилю, сыну Вафуила.
\vs Joe 1:2 Слушайте это, старцы, и внимайте, все жители земли сей: бывало ли такое во дни ваши, или во дни отцов ваших?
\vs Joe 1:3 Передайте об этом детям вашим; а дети ваши пусть скажут своим детям, а их дети следующему роду:
\vs Joe 1:4 оставшееся от гусеницы ела саранча, оставшееся от саранчи ели черви, а оставшееся от червей доели жуки.
\vs Joe 1:5 Пробудитесь, пьяницы, и плачьте и рыдайте, все пьющие вино, о виноградном соке, ибо он отнят от уст ваших!
\vs Joe 1:6 Ибо пришел на землю Мою народ сильный и бесчисленный; зубы у него~--- зубы львиные, и челюсти у него~--- как у львицы.
\vs Joe 1:7 Опустошил он виноградную лозу Мою, и смоковницу Мою обломал, ободрал ее догола, и бросил; сделались белыми ветви ее.
\vs Joe 1:8 Рыдай, как молодая жена, препоясавшись \bibemph{вретищем}, о муже юности своей!
\vs Joe 1:9 Прекратилось хлебное приношение и возлияние в доме Господнем; плачут священники, служители Господни.
\vs Joe 1:10 Опустошено поле, сетует земля; ибо истреблен хлеб, высох виноградный сок, завяла маслина.
\vs Joe 1:11 Краснейте от стыда, земледельцы, рыдайте, виноградари, о пшенице и ячмене, потому что погибла жатва в поле,
\vs Joe 1:12 засохла виноградная лоза и смоковница завяла; гранатовое дерево, пальма и яблоня, все дерева в поле посохли; потому и веселье у сынов человеческих исчезло.
\vs Joe 1:13 Препояшьтесь \bibemph{вретищем} и плачьте, священники! рыдайте, служители алтаря! войдите, ночуйте во вретищах, служители Бога моего! ибо не стало в доме Бога вашего хлебного приношения и возлияния.
\vs Joe 1:14 Назначьте пост, объявите торжественное собрание, созовите старцев и всех жителей страны сей в дом Господа Бога вашего, и взывайте к Господу.
\vs Joe 1:15 О, какой день! ибо день Господень близок; как опустошение от Всемогущего придет он.
\vs Joe 1:16 Не пред нашими ли глазами отнимается пища, от дома Бога нашего~--- веселье и радость?
\vs Joe 1:17 Истлели зерна под глыбами своими, опустели житницы, разрушены кладовые, ибо не стало хлеба.
\vs Joe 1:18 Как стонет скот! уныло ходят стада волов, ибо нет для них пажити; томятся и стада овец.
\vs Joe 1:19 К Тебе, Господи, взываю; ибо огонь пожрал злачные пастбища пустыни, и пламя попалило все дерева в поле.
\vs Joe 1:20 Даже и животные на поле взывают к Тебе, потому что иссохли потоки вод, и огонь истребил пастбища пустыни.
\vs Joe 2:1 Трубите трубою на Сионе и бейте тревогу на святой горе Моей; да трепещут все жители земли, ибо наступает день Господень, ибо он близок~---
\vs Joe 2:2 день тьмы и мрака, день облачный и туманный: как утренняя заря распространяется по горам народ многочисленный и сильный, какого не бывало от века и после того не будет в роды родов.
\vs Joe 2:3 Перед ним пожирает огонь, а за ним палит пламя; перед ним земля как сад Едемский, а позади него будет опустошенная степь, и никому не будет спасения от него.
\vs Joe 2:4 Вид его как вид коней, и скачут они как всадники;
\vs Joe 2:5 скачут по вершинам гор как бы со стуком колесниц, как бы с треском огненного пламени, пожирающего солому, как сильный народ, выстроенный к битве.
\vs Joe 2:6 При виде его затрепещут народы, у всех лица побледнеют.
\vs Joe 2:7 Как борцы бегут они и как храбрые воины влезают на стену, и каждый идет своею дорогою, и не сбивается с путей своих.
\vs Joe 2:8 Не давят друг друга, каждый идет своею стезею, и падают на копья, но остаются невредимы.
\vs Joe 2:9 Бегают по городу, поднимаются на стены, влезают на дома, входят в окна, как вор.
\vs Joe 2:10 Перед ними потрясется земля, поколеблется небо; солнце и луна помрачатся, и звезды потеряют свой свет.
\vs Joe 2:11 И Господь даст глас Свой пред воинством Своим, ибо весьма многочисленно полчище Его и могуществен исполнитель слова Его; ибо велик день Господень и весьма страшен, и кто выдержит его?
\vs Joe 2:12 Но и ныне еще говорит Господь: обратитесь ко Мне всем сердцем своим в посте, плаче и рыдании.
\vs Joe 2:13 Раздирайте сердца ваши, а не одежды ваши, и обратитесь к Господу Богу вашему; ибо Он благ и милосерд, долготерпелив и многомилостив и сожалеет о бедствии.
\vs Joe 2:14 Кто знает, не сжалится ли Он, и не оставит ли благословения, хлебного приношения и возлияния Господу Богу вашему?
\vs Joe 2:15 Вострубите трубою на Сионе, назначьте пост и объявите торжественное собрание.
\vs Joe 2:16 Соберите народ, созовите собрание, пригласите старцев, соберите отроков и грудных младенцев; пусть выйдет жених из чертога своего и невеста из своей горницы.
\vs Joe 2:17 Между притвором и жертвенником да плачут священники, служители Господни, и говорят: <<пощади, Господи, народ Твой, не предай наследия Твоего на поругание, чтобы не издевались над ним народы; для чего будут говорить между народами: где Бог их?>>
\vs Joe 2:18 И тогда возревнует Господь о земле Своей, и пощадит народ Свой.
\vs Joe 2:19 И ответит Господь, и скажет народу Своему: вот, Я пошлю вам хлеб и вино и елей, и будете насыщаться ими, и более не отдам вас на поругание народам.
\vs Joe 2:20 И пришедшего от севера удалю от вас, и изгоню в землю безводную и пустую, переднее полчище его~--- в море восточное, а заднее~--- в море западное, и пойдет от него зловоние, и поднимется от него смрад, так как он много наделал \bibemph{зла}.
\vs Joe 2:21 Не бойся, земля: радуйся и веселись, ибо Господь велик, чтобы совершить это.
\vs Joe 2:22 Не бойтесь, животные, ибо пастбища пустыни произрастят траву, дерево принесет плод свой, смоковница и виноградная лоза окажут свою силу.
\vs Joe 2:23 И вы, чада Сиона, радуйтесь и веселитесь о Господе Боге вашем; ибо Он даст вам дождь в меру и будет ниспосылать вам дождь, дождь ранний и поздний, как прежде.
\vs Joe 2:24 И наполнятся гумна хлебом, и переполнятся подточилия виноградным соком и елеем.
\vs Joe 2:25 И воздам вам за те годы, которые пожирали саранча, черви, жуки и гусеница, великое войско Мое, которое послал Я на вас.
\vs Joe 2:26 И до сытости будете есть и насыщаться и славить имя Господа Бога вашего, Который дивное соделал с вами, и не посрамится народ Мой во веки.
\vs Joe 2:27 И узнаете, что Я посреди Израиля, и Я~--- Господь Бог ваш, и нет другого, и Мой народ не посрамится во веки.
\rsbpar\vs Joe 2:28 И будет после того, излию от Духа Моего на всякую плоть, и будут пророчествовать сыны ваши и дочери ваши; старцам вашим будут сниться сны, и юноши ваши будут видеть видения.
\vs Joe 2:29 И также на рабов и на рабынь в те дни излию от Духа Моего.
\vs Joe 2:30 И покажу знамения на небе и на земле: кровь и огонь и столпы дыма.
\vs Joe 2:31 Солнце превратится во тьму и луна~--- в кровь, прежде нежели наступит день Господень, великий и страшный.
\vs Joe 2:32 И будет: всякий, кто призовет имя Господне, спасется; ибо на горе Сионе и в Иерусалиме будет спасение, как сказал Господь, и у остальных, которых призовет Господь.
\vs Joe 3:1 Ибо вот, в те дни и в то самое время, когда Я возвращу плен Иуды и Иерусалима,
\vs Joe 3:2 Я соберу все народы, и приведу их в долину Иосафата, и там произведу над ними суд за народ Мой и за наследие Мое, Израиля, который они рассеяли между народами, и землю Мою разделили.
\vs Joe 3:3 И о народе Моем они бросали жребий, и отдавали отрока за блудницу, и продавали отроковицу за вино, и пили.
\vs Joe 3:4 И что вы Мне, Тир и Сидон и все округи Филистимские? Хотите ли воздать Мне возмездие? хотите ли воздать Мне? Легко и скоро Я обращу возмездие ваше на головы ваши,
\vs Joe 3:5 потому что вы взяли серебро Мое и золото Мое, и наилучшие драгоценности Мои внесли в капища ваши,
\vs Joe 3:6 и сынов Иуды и сынов Иерусалима продавали сынам Еллинов, чтобы удалить их от пределов их.
\vs Joe 3:7 Вот, Я подниму их из того места, куда вы продали их, и обращу мзду вашу на голову вашу.
\vs Joe 3:8 И предам сыновей ваших и дочерей ваших в руки сынов Иуды, и они продадут их Савеям, народу отдаленному; так Господь сказал.
\rsbpar\vs Joe 3:9 Провозгласите об этом между народами, приготовьтесь к войне, возбудите храбрых; пусть выступят, поднимутся все ратоборцы.
\vs Joe 3:10 Перекуйте орала ваши на мечи и серпы ваши на копья; слабый пусть говорит: <<я силен>>.
\vs Joe 3:11 Спешите и сходитесь, все народы окрестные, и соберитесь; туда, Господи, веди Твоих героев.
\vs Joe 3:12 Пусть воспрянут народы и низойдут в долину Иосафата; ибо там Я воссяду, чтобы судить все народы отовсюду.
\vs Joe 3:13 Пустите в дело серпы, ибо жатва созрела; идите, спуститесь, ибо точило полно и подточилия переливаются, потому что злоба их велика.
\vs Joe 3:14 Толпы, толпы в долине суда! ибо близок день Господень к долине суда!
\vs Joe 3:15 Солнце и луна померкнут и звезды потеряют блеск свой.
\vs Joe 3:16 И возгремит Господь с Сиона, и даст глас Свой из Иерусалима; содрогнутся небо и земля; но Господь будет защитою для народа Своего и обороною для сынов Израилевых.
\vs Joe 3:17 Тогда узнаете, что Я Господь Бог ваш, обитающий на Сионе, на святой горе Моей; и будет Иерусалим святынею, и не будут уже иноплеменники проходить через него.
\vs Joe 3:18 И будет в тот день: горы будут капать вином и холмы потекут молоком, и все русла Иудейские наполнятся водою, а из дома Господня выйдет источник, и будет напоять долину Ситтим.
\vs Joe 3:19 Египет сделается пустынею и Едом будет пустою степью~--- за то, что они притесняли сынов Иудиных и проливали невинную кровь в земле их.
\vs Joe 3:20 А Иуда будет жить вечно и Иерусалим~--- в роды родов.
\vs Joe 3:21 Я смою кровь их, которую не смыл еще, и Господь будет обитать на Сионе.
\newbookpage
\bibbookdescr{Amo}{
  inline={\LARGE Книга\\\Huge Пророка Амоса},
  toc={Амос},
  bookmark={Амос},
  header={Амос},
  %headerleft={},
  %headerright={},
  abbr={Ам}
}
\vs Amo 1:1 Слова Амоса, одного из пастухов Фекойских, которые он \bibemph{слышал} в видении об Израиле во дни Озии, царя Иудейского, и во дни Иеровоама, сына Иоасова, царя Израильского, за два года перед землетрясением.
\rsbpar\vs Amo 1:2 И сказал он: Господь возгремит с Сиона и даст глас Свой из Иерусалима, и восплачут хижины пастухов, и иссохнет вершина Кармила.
\vs Amo 1:3 Так говорит Господь: за три преступления Дамаска и за четыре не пощажу его, потому что они молотили Галаад железными молотилами.
\vs Amo 1:4 И пошлю огонь на дом Азаила, и пожрет он чертоги Венадада.
\vs Amo 1:5 И сокрушу затворы Дамаска, и истреблю жителей долины Авен и держащего скипетр~--- из дома Еденова, и пойдет народ Арамейский в плен в Кир, говорит Господь.
\vs Amo 1:6 Так говорит Господь: за три преступления Газы и за четыре не пощажу ее, потому что они вывели всех в плен, чтобы предать их Едому.
\vs Amo 1:7 И пошлю огонь в стены Газы,~--- и пожрет чертоги ее.
\vs Amo 1:8 И истреблю жителей Азота и держащего скипетр в Аскалоне, и обращу руку Мою на Екрон, и погибнет остаток Филистимлян, говорит Господь Бог.
\vs Amo 1:9 Так говорит Господь: за три преступления Тира и за четыре не пощажу его, потому что они передали всех пленных Едому и не вспомнили братского союза.
\vs Amo 1:10 Пошлю огонь в стены Тира, и пожрет чертоги его.
\vs Amo 1:11 Так говорит Господь: за три преступления Едома и за четыре не пощажу его, потому что он преследовал брата своего мечом, подавил чувства родства, свирепствовал постоянно во гневе своем и всегда сохранял ярость свою.
\vs Amo 1:12 И пошлю огонь на Феман, и пожрет чертоги Восора.
\vs Amo 1:13 Так говорит Господь: за три преступления сынов Аммоновых и за четыре не пощажу их, потому что они рассекали беременных в Галааде, чтобы расширить пределы свои.
\vs Amo 1:14 И запалю огонь в стенах Раввы, и пожрет чертоги ее, среди крика в день брани, с вихрем в день бури.
\vs Amo 1:15 И пойдет царь их в плен, он и князья его вместе с ним, говорит Господь.
\vs Amo 2:1 Так говорит Господь: за три преступления Моава и за четыре не пощажу его, потому что он пережег кости царя Едомского в известь.
\vs Amo 2:2 И пошлю огонь на Моава, и пожрет чертоги Кериофа, и погибнет Моав среди разгрома с шумом, при звуке трубы.
\vs Amo 2:3 Истреблю судью из среды его и умерщвлю всех князей его вместе с ним, говорит Господь.
\vs Amo 2:4 Так говорит Господь: за три преступления Иуды и за четыре не пощажу его, потому что отвергли закон Господень и постановлений Его не сохранили, и идолы их, вслед которых ходили отцы их, совратили их с пути.
\vs Amo 2:5 И пошлю огонь на Иуду, и пожрет чертоги Иерусалима.
\vs Amo 2:6 Так говорит Господь: за три преступления Израиля и за четыре не пощажу его, потому что продают правого за серебро и бедного~--- за пару сандалий.
\vs Amo 2:7 Жаждут, чтобы прах земной был на голове бедных, и путь кротких извращают; даже отец и сын ходят к одной женщине, чтобы бесславить святое имя Мое.
\vs Amo 2:8 На одеждах, взятых в залог, возлежат при всяком жертвеннике, и вино, \bibemph{взыскиваемое} с обвиненных, пьют в доме богов своих.
\vs Amo 2:9 А Я истребил перед лицем их Аморрея, которого высота была как высота кедра и который был крепок как дуб; Я истребил плод его вверху и корни его внизу.
\vs Amo 2:10 Вас же Я вывел из земли Египетской и водил вас в пустыне сорок лет, чтобы вам наследовать землю Аморрейскую.
\vs Amo 2:11 Из сыновей ваших Я избирал в пророки и из юношей ваших~--- в назореи; не так ли это, сыны Израиля? говорит Господь.
\vs Amo 2:12 А вы назореев поили вином и пророкам приказывали, говоря: <<не пророчествуйте>>.
\vs Amo 2:13 Вот, Я придавлю вас, как давит колесница, нагруженная снопами,~---
\vs Amo 2:14 и у проворного не станет силы бежать, и крепкий не удержит крепости своей, и храбрый не спасет своей жизни,
\vs Amo 2:15 ни стреляющий из лука не устоит, ни скороход не убежит, ни сидящий на коне не спасет своей жизни.
\vs Amo 2:16 И самый отважный из храбрых убежит нагой в тот день, говорит Господь.
\vs Amo 3:1 Слушайте слово сие, которое Господь изрек на вас, сыны Израилевы, на все племя, которое вывел Я из земли Египетской, говоря:
\vs Amo 3:2 только вас признал Я из всех племен земли, потому и взыщу с вас за все беззакония ваши.
\vs Amo 3:3 Пойдут ли двое вместе, не сговорившись между собою?
\vs Amo 3:4 Ревет ли лев в лесу, когда нет перед ним добычи? подает ли свой голос львенок из логовища своего, когда он ничего не поймал?
\vs Amo 3:5 Попадет ли птица в петлю на земле, когда силка нет для нее? Поднимется ли с земли петля, когда ничего не попало в нее?
\vs Amo 3:6 Трубит ли в городе труба,~--- и народ не испугался бы? Бывает ли в городе бедствие, которое не Господь попустил бы?
\vs Amo 3:7 Ибо Господь Бог ничего не делает, не открыв Своей тайны рабам Своим, пророкам.
\vs Amo 3:8 Лев начал рыкать,~--- кто не содрогнется? Господь Бог сказал,~--- кто не будет пророчествовать?
\vs Amo 3:9 Провозгласите на кровлях в Азоте и на кровлях в земле Египетской и скажите: соберитесь на горы Самарии и посмотрите на великое бесчинство в ней и на притеснения среди нее.
\vs Amo 3:10 Они не умеют поступать справедливо, говорит Господь: насилием и грабежом собирают сокровища в чертоги свои.
\vs Amo 3:11 Посему так говорит Господь Бог: вот неприятель, и притом вокруг всей земли! он низложит могущество твое, и ограблены будут чертоги твои.
\vs Amo 3:12 Так говорит Господь: как \bibemph{иногда} пастух исторгает из пасти львиной две голени или часть уха, так спасены будут сыны Израилевы, сидящие в Самарии в углу постели и в Дамаске на ложе.
\vs Amo 3:13 Слушайте и засвидетельствуйте дому Иакова, говорит Господь Бог, Бог Саваоф.
\vs Amo 3:14 Ибо в тот день, когда Я взыщу с Израиля за преступления его, взыщу и за жертвенники в Вефиле, и отсечены будут роги алтаря, и падут на землю.
\vs Amo 3:15 И поражу дом зимний вместе с домом летним, и исчезнут домы с украшениями из слоновой кости, и не станет многих домов, говорит Господь.
\vs Amo 4:1 Слушайте слово сие, телицы Васанские, которые на горе Самарийской, вы, притесняющие бедных, угнетающие нищих, говорящие господам своим: <<подавай, и мы будем пить!>>
\vs Amo 4:2 Клялся Господь Бог святостью Своею, что вот, придут на вас дни, когда повлекут вас крюками и остальных ваших удами.
\vs Amo 4:3 И сквозь проломы стен выйдете, каждая, как случится, и бросите все убранство чертогов, говорит Господь.
\vs Amo 4:4 Идите в Вефиль~--- и грешите, в Галгал~--- и умножайте преступления; приносите жертвы ваши каждое утро, десятины ваши хотя через каждые три дня.
\vs Amo 4:5 Приносите в жертву благодарения квасное, провозглашайте о добровольных приношениях ваших и разглашайте о них, ибо это вы любите, сыны Израилевы, говорит Господь Бог.
\vs Amo 4:6 За то и дал Я вам голые зубы во всех городах ваших и недостаток хлеба во всех селениях ваших; но вы не обратились ко Мне, говорит Господь.
\vs Amo 4:7 И удерживал от вас дождь за три месяца до жатвы; проливал дождь на один город, а на другой город не проливал дождя; один участок напояем был дождем, а другой, не окропленный дождем, засыхал.
\vs Amo 4:8 И сходились два-три города в один город, чтобы напиться воды, и не могли досыта напиться; но и тогда вы не обратились ко Мне, говорит Господь.
\vs Amo 4:9 Я поражал вас ржою и блеклостью хлеба; множество садов ваших и виноградников ваших, и смоковниц ваших, и маслин ваших пожирала гусеница,~--- и при всем том вы не обратились ко Мне, говорит Господь.
\vs Amo 4:10 Посылал Я на вас моровую язву, подобную Египетской, убивал мечом юношей ваших, отводя коней в плен, так что смрад от станов ваших поднимался в ноздри ваши; и при всем том вы не обратились ко Мне, говорит Господь.
\vs Amo 4:11 Производил Я среди вас разрушения, как разрушил Бог Содом и Гоморру, и вы были выхвачены, как головня из огня,~--- и при всем том вы не обратились ко Мне, говорит Господь.
\vs Amo 4:12 Посему так поступлю Я с тобою, Израиль; и как Я так поступлю с тобою, то приготовься к сретению Бога твоего, Израиль,
\vs Amo 4:13 ибо вот Он, Который образует горы, и творит ветер, и объявляет человеку намерения его, утренний свет обращает в мрак, и шествует превыше земли; Господь Бог Саваоф~--- имя Ему.
\vs Amo 5:1 Слушайте это слово, в котором я подниму плач о вас, дом Израилев.
\vs Amo 5:2 Упала, не встает более дева Израилева! повержена на земле своей, и некому поднять ее.
\vs Amo 5:3 Ибо так говорит Господь Бог: город, выступавший тысячею, останется только с сотнею, и выступавший сотнею, останется с десятком у дома Израилева.
\vs Amo 5:4 Ибо так говорит Господь дому Израилеву: взыщите Меня, и будете живы.
\vs Amo 5:5 Не ищите Вефиля и не ходите в Галгал, и в Вирсавию не странствуйте, ибо Галгал весь пойдет в плен и Вефиль обратится в ничто.
\vs Amo 5:6 Взыщите Господа, и будете живы, чтобы Он не устремился на дом Иосифов как огонь, который пожрет его, и некому будет погасить его в Вефиле.
\vs Amo 5:7 О, вы, которые суд превращаете в отраву и правду повергаете на землю!
\vs Amo 5:8 Кто сотворил семизвездие и Орион, и претворяет смертную тень в ясное утро, а день делает темным как ночь, призывает воды морские и разливает их по лицу земли?~--- Господь имя Ему!
\vs Amo 5:9 Он укрепляет опустошителя против сильного, и опустошитель входит в крепость.
\vs Amo 5:10 А они ненавидят обличающего в воротах и гнушаются тем, кто говорит правду.
\vs Amo 5:11 Итак за то, что вы попираете бедного и берете от него подарки хлебом, вы построите домы из тесаных камней, но жить не будете в них; разведете прекрасные виноградники, а вино из них не будете пить.
\vs Amo 5:12 Ибо Я знаю, как многочисленны преступления ваши и как тяжки грехи ваши: вы враги правого, берете взятки и извращаете в суде дела бедных.
\vs Amo 5:13 Поэтому разумный безмолвствует в это время, ибо злое это время.
\vs Amo 5:14 Ищите добра, а не зла, чтобы вам остаться в живых,~--- и тогда Господь Бог Саваоф будет с вами, как вы говорите.
\vs Amo 5:15 Возненавидьте зло и возлюбите добро, и восстановите у ворот правосудие; может быть, Господь Бог Саваоф помилует остаток Иосифов.
\vs Amo 5:16 Посему так говорит Господь Бог Саваоф, Вседержитель: на всех улицах будет плач, и на всех дорогах будут восклицать: <<увы, увы!>>, и призовут земледельца сетовать и искусных в плачевных песнях~--- плакать,
\vs Amo 5:17 и во всех виноградниках будет плач, ибо Я пройду среди тебя, говорит Господь.
\vs Amo 5:18 Горе желающим дня Господня! для чего вам этот день Господень? он тьма, а не свет,
\vs Amo 5:19 то же, как если бы кто убежал от льва, и попался бы ему навстречу медведь, или если бы пришел домой и оперся рукою о стену, и змея ужалила бы его.
\vs Amo 5:20 Разве день Господень не мрак, а свет? он тьма, и нет в нем сияния.
\vs Amo 5:21 Ненавижу, отвергаю праздники ваши и не обоняю жертв во время торжественных собраний ваших.
\vs Amo 5:22 Если вознесете Мне всесожжение и хлебное приношение, Я не приму их и не призрю на благодарственную жертву из тучных тельцов ваших.
\vs Amo 5:23 Удали от Меня шум песней твоих, ибо звуков гуслей твоих Я не буду слушать.
\vs Amo 5:24 Пусть, как вода, течет суд, и правда~--- как сильный поток!
\vs Amo 5:25 Приносили ли вы Мне жертвы и хлебные дары в пустыне в течение сорока лет, дом Израилев?
\vs Amo 5:26 Вы носили скинию Молохову и звезду бога вашего Ремфана, изображения, которые вы сделали для себя.
\vs Amo 5:27 За то Я переселю вас за Дамаск, говорит Господь; Бог Саваоф~--- имя Ему!
\vs Amo 6:1 Горе беспечным на Сионе и надеющимся на гору Самарийскую именитым первенствующего народа, к которым приходит дом Израиля!
\vs Amo 6:2 Пройдите в Калне и посмотрите, оттуда перейдите в Емаф великий и спуститесь в Геф Филистимский: не лучше ли они сих царств? не обширнее ли пределы их пределов ваших?
\vs Amo 6:3 Вы, которые день бедствия считаете далеким и приближаете торжество насилия,~---
\vs Amo 6:4 вы, которые лежите на ложах из слоновой кости и нежитесь на постелях ваших, едите лучших овнов из стада и тельцов с тучного пастбища,
\vs Amo 6:5 поете под звуки гуслей, думая, что владеете музыкальным орудием, как Давид,
\vs Amo 6:6 пьете из чаш вино, мажетесь наилучшими мастями, и не болезнуете о бедствии Иосифа!
\vs Amo 6:7 За то ныне пойдут они в плен во главе пленных, и кончится ликование изнеженных.
\vs Amo 6:8 Клянется Господь Бог Самим Собою, и так говорит Господь Бог Саваоф: гнушаюсь высокомерием Иакова и ненавижу чертоги его, и предам город и все, что наполняет его.
\vs Amo 6:9 И будет: если в каком доме останется десять человек, то умрут и они,
\vs Amo 6:10 и возьмет их родственник их или сожигатель, чтобы вынести кости их из дома, и скажет находящемуся при доме: есть ли еще у тебя кто? Тот ответит: нет никого. И скажет сей: молчи! ибо нельзя упоминать имени Господня.
\vs Amo 6:11 Ибо вот, Господь даст повеление и поразит большие дома расселинами, а малые дома~--- трещинами.
\vs Amo 6:12 Бегают ли кони по скале? можно ли распахивать ее волами? Вы между тем суд превращаете в яд и плод правды в горечь;
\vs Amo 6:13 вы, которые восхищаетесь ничтожными вещами и говорите: <<не своею ли силою мы приобрели себе могущество?>>
\vs Amo 6:14 Вот Я, говорит Господь Бог Саваоф, воздвигну народ против вас, дом Израилев, и будут теснить вас от входа в Емаф до потока в пустыне.
\vs Amo 7:1 Такое видение открыл мне Господь Бог: вот, Он создал саранчу в начале произрастания поздней травы, и это была трава после царского покоса.
\vs Amo 7:2 И было, когда она окончила есть траву на земле, я сказал: Господи Боже! пощади; как устоит Иаков? он очень мал.
\vs Amo 7:3 И пожалел Господь о том; <<не будет сего>>, сказал Господь.
\rsbpar\vs Amo 7:4 Такое видение открыл мне Господь Бог: вот, Господь Бог произвел для суда огонь,~--- и он пожрал великую пучину, пожрал и часть земли.
\vs Amo 7:5 И сказал я: Господи Боже! останови; как устоит Иаков? он очень мал.
\vs Amo 7:6 И пожалел Господь о том; <<и этого не будет>>, сказал Господь Бог.
\rsbpar\vs Amo 7:7 Такое видение открыл Он мне: вот, Господь стоял на отвесной стене, и в руке у Него свинцовый отвес.
\vs Amo 7:8 И сказал мне Господь: что ты видишь, Амос? Я ответил: отвес. И Господь сказал: вот, положу отвес среди народа Моего, Израиля; не буду более прощать ему.
\vs Amo 7:9 И опустошены будут \bibemph{жертвенные} высоты Исааковы, и разрушены будут святилища Израилевы, и восстану с мечом против дома Иеровоамова.
\rsbpar\vs Amo 7:10 И послал Амасия, священник Вефильский, к Иеровоаму, царю Израильскому, сказать: Амос производит возмущение против тебя среди дома Израилева; земля не может терпеть всех слов его.
\vs Amo 7:11 Ибо так говорит Амос: <<от меча умрет Иеровоам, а Израиль непременно отведен будет пленным из земли своей>>.
\vs Amo 7:12 И сказал Амасия Амосу: провидец! пойди и удались в землю Иудину; там ешь хлеб, и там пророчествуй,
\vs Amo 7:13 а в Вефиле больше не пророчествуй, ибо он святыня царя и дом царский.
\vs Amo 7:14 И отвечал Амос и сказал Амасии: я не пророк и не сын пророка; я был пастух и собирал сикоморы.
\vs Amo 7:15 Но Господь взял меня от овец и сказал мне Господь: <<иди, пророчествуй к народу Моему, Израилю>>.
\vs Amo 7:16 Теперь выслушай слово Господне. Ты говоришь: <<не пророчествуй на Израиля и не произноси слов на дом Исааков>>.
\vs Amo 7:17 За это, вот что говорит Господь: жена твоя будет обесчещена в городе, сыновья и дочери твои падут от меча, земля твоя будет разделена межевою вервью, а ты умрешь в земле нечистой, и Израиль непременно выведен будет из земли своей.
\vs Amo 8:1 Такое видение открыл мне Господь Бог: вот корзина со спелыми плодами.
\vs Amo 8:2 И сказал Он: что ты видишь, Амос? Я ответил: корзину со спелыми плодами. Тогда Господь сказал мне: приспел конец народу Моему, Израилю: не буду более прощать ему.
\vs Amo 8:3 Песни чертога в тот день обратятся в рыдание, говорит Господь Бог; много будет трупов, на всяком месте будут бросать их молча.
\rsbpar\vs Amo 8:4 Выслушайте это, алчущие поглотить бедных и погубить нищих,~---
\vs Amo 8:5 вы, которые говорите: <<когда-то пройдет новолуние, чтобы нам продавать хлеб, и суббота, чтобы открыть житницы, уменьшить меру, увеличить цену сикля и обманывать неверными весами,
\vs Amo 8:6 чтобы покупать неимущих за серебро и бедных за пару обуви, а высевки из хлеба продавать>>.
\vs Amo 8:7 Клялся Господь славою Иакова: поистине во веки не забуду ни одного из дел их!
\vs Amo 8:8 Не поколеблется ли от этого земля, и не восплачет ли каждый, живущий на ней? Взволнуется вся она, как река, и будет подниматься и опускаться, как река Египетская.
\vs Amo 8:9 И будет в тот день, говорит Господь Бог: произведу закат солнца в полдень и омрачу землю среди светлого дня.
\vs Amo 8:10 И обращу праздники ваши в сетование и все песни ваши в плач, и возложу на все чресла вретище и плешь на всякую голову; и произведу \bibemph{в стране} плач, как о единственном сыне, и конец ее будет~--- как горький день.
\vs Amo 8:11 Вот наступают дни, говорит Господь Бог, когда Я пошлю на землю голод,~--- не голод хлеба, не жажду воды, но жажду слышания слов Господних.
\vs Amo 8:12 И будут ходить от моря до моря и скитаться от севера к востоку, ища слова Господня, и не найдут его.
\vs Amo 8:13 В тот день истаявать будут от жажды красивые девы и юноши,
\vs Amo 8:14 которые клянутся грехом Самарийским и говорят: <<жив бог твой, Дан! и жив путь в Вирсавию!>>~--- Они падут и уже не встанут.
\vs Amo 9:1 Видел я Господа стоящим над жертвенником, и Он сказал: ударь в притолоку над воротами, чтобы потряслись косяки, и обрушь их на головы всех их, остальных же из них Я поражу мечом: не убежит у них никто бегущий и не спасется из них никто, желающий спастись.
\vs Amo 9:2 Хотя бы они зарылись в преисподнюю, и оттуда рука Моя возьмет их; хотя бы взошли на небо, и оттуда свергну их.
\vs Amo 9:3 И хотя бы они скрылись на вершине Кармила, и там отыщу и возьму их; хотя бы сокрылись от очей Моих на дне моря, и там повелю морскому змею уязвить их.
\vs Amo 9:4 И если пойдут в плен впереди врагов своих, то повелю мечу и там убить их. Обращу на них очи Мои на беду им, а не во благо.
\vs Amo 9:5 Ибо Господь Бог Саваоф коснется земли,~--- и она растает, и восплачут все живущие на ней; и поднимется вся она как река, и опустится как река Египетская.
\vs Amo 9:6 Он устроил горние чертоги Свои на небесах и свод Свой утвердил на земле, призывает воды морские, и изливает их по лицу земли; Господь имя Ему.
\vs Amo 9:7 Не таковы ли, как сыны Ефиоплян, и вы для Меня, сыны Израилевы? говорит Господь. Не Я ли вывел Израиля из земли Египетской и Филистимлян~--- из Кафтора, и Арамлян~--- из Кира?
\vs Amo 9:8 Вот, очи Господа Бога~--- на грешное царство, и Я истреблю его с лица земли; но дом Иакова не совсем истреблю, говорит Господь.
\vs Amo 9:9 Ибо вот, Я повелю и рассыплю дом Израилев по всем народам, как рассыпают зерна в решете, и ни одно не падает на землю.
\vs Amo 9:10 От меча умрут все грешники из народа Моего, которые говорят: <<не постигнет нас и не придет к нам это бедствие!>>
\vs Amo 9:11 В тот день Я восстановлю скинию Давидову падшую, заделаю трещины в ней и разрушенное восстановлю, и устрою ее, как в дни древние,
\vs Amo 9:12 чтобы они овладели остатком Едома и всеми народами, между которыми возвестится имя Мое, говорит Господь, творящий все сие.
\vs Amo 9:13 Вот, наступят дни, говорит Господь, когда пахарь застанет еще жнеца, а топчущий виноград~--- сеятеля; и горы источать будут виноградный сок, и все холмы потекут.
\vs Amo 9:14 И возвращу из плена народ Мой, Израиля, и застроят опустевшие города и поселятся в них, насадят виноградники и будут пить вино из них, разведут сады и станут есть плоды из них.
\vs Amo 9:15 И водворю их на земле их, и они не будут более исторгаемы из земли своей, которую Я дал им, говорит Господь Бог твой.

\bibbookdescr{Oba}{
  inline={\LARGE Книга\\\Huge Пророка Авдия},
  toc={Авдий},
  bookmark={Авдий},
  header={Авдий},
  %headerleft={},
  %headerright={},
  abbr={Авд}
}
\vs Oba 1:1 Видение Авдия. Так говорит Господь Бог об Едоме: весть услышали мы от Господа, и посол послан \bibemph{объявить} народам: <<вставайте, и выступим против него войною!>>
\vs Oba 1:2 Вот, Я сделал тебя малым между народами, и ты в большом презрении.
\vs Oba 1:3 Гордость сердца твоего обольстила тебя; ты живешь в расселинах скал, на возвышенном месте, и говоришь в сердце твоем: <<кто низринет меня на землю?>>
\vs Oba 1:4 Но хотя бы ты, как орел, поднялся высоко и среди звезд устроил гнездо твое, то и оттуда Я низрину тебя, говорит Господь.
\vs Oba 1:5 Не воры ли приходили к тебе? не ночные ли грабители, что ты так разорен? Но они украли бы столько, сколько надобно им. Если бы проникли к тебе обиратели винограда, то и они разве не оставили бы несколько ягод?
\vs Oba 1:6 Как обобрано все у Исава и обысканы тайники его!
\vs Oba 1:7 До границы выпроводят тебя все союзники твои, обманут тебя, одолеют тебя живущие с тобою в мире, ядущие хлеб твой нанесут тебе удар. Нет в нем смысла!
\vs Oba 1:8 Не в тот ли день это будет, говорит Господь, когда Я истреблю мудрых в Едоме и благоразумных на горе Исава?
\vs Oba 1:9 Поражены будут страхом храбрецы твои, Феман, дабы все на горе Исава истреблены были убийством.
\vs Oba 1:10 За притеснение брата твоего, Иакова, покроет тебя стыд и ты истреблен будешь навсегда.
\vs Oba 1:11 В тот день, когда ты стоял напротив, в тот день, когда чужие уводили войско его в плен и иноплеменники вошли в ворота его и бросали жребий о Иерусалиме, ты был как один из них.
\vs Oba 1:12 Не следовало бы тебе злорадно смотреть на день брата твоего, на день отчуждения его; не следовало бы радоваться о сынах Иуды в день гибели их и расширять рот в день бедствия.
\vs Oba 1:13 Не следовало бы тебе входить в ворота народа Моего в день несчастья его и даже смотреть на злополучие его в день погибели его, ни касаться имущества его в день бедствия его,
\vs Oba 1:14 ни стоять на перекрестках для убивания бежавших его, ни выдавать уцелевших из него в день бедствия.
\vs Oba 1:15 Ибо близок день Господень на все народы: как ты поступал, так поступлено будет и с тобою; воздаяние твое обратится на голову твою.
\vs Oba 1:16 Ибо, как вы пили на святой горе Моей, так все народы всегда будут пить, будут пить, проглотят и будут, как бы их не было.
\vs Oba 1:17 А на горе Сионе будет спасение, и будет она святынею; и дом Иакова получит во владение наследие свое.
\vs Oba 1:18 И дом Иакова будет огнем, и дом Иосифа~--- пламенем, а дом Исавов~--- соломою: зажгут его, и истребят его, и никого не останется из дома Исава: ибо Господь сказал это.
\vs Oba 1:19 И завладеют те, которые к югу, горою Исава, а которые в долине,~--- Филистимлянами; и завладеют полем Ефрема и полем Самарии, а Вениамин завладеет Галаадом.
\vs Oba 1:20 И переселенные из войска сынов Израилевых завладеют землею Ханаанскою до Сарепты, а переселенные из Иерусалима, находящиеся в Сефараде, получат во владение города южные.
\vs Oba 1:21 И придут спасители на гору Сион, чтобы судить гору Исава, и будет царство Господа.
\newbookpage
\bibbookdescr{Jon}{
  inline={\LARGE Книга\\\Huge Пророка Ионы},
  toc={Иона},
  bookmark={Иона},
  header={Иона},
  %headerleft={},
  %headerright={},
  abbr={Иона}
}
\vs Jon 1:1 И было слово Господне к Ионе, сыну Амафиину:
\vs Jon 1:2 встань, иди в Ниневию, город великий, и проповедуй в нем, ибо злодеяния его дошли до Меня.
\vs Jon 1:3 И встал Иона, чтобы бежать в Фарсис от лица Господня, и пришел в Иоппию, и нашел корабль, отправлявшийся в Фарсис, отдал плату за провоз и вошел в него, чтобы плыть с ними в Фарсис от лица Господа.
\vs Jon 1:4 Но Господь воздвиг на море крепкий ветер, и сделалась на море великая буря, и корабль готов был разбиться.
\vs Jon 1:5 И устрашились корабельщики, и взывали каждый к своему богу, и стали бросать в море кладь с корабля, чтобы облегчить его от нее; Иона же спустился во внутренность корабля, лег и крепко заснул.
\vs Jon 1:6 И пришел к нему начальник корабля и сказал ему: что ты спишь? встань, воззови к Богу твоему; может быть, Бог вспомнит о нас, и мы не погибнем.
\vs Jon 1:7 И сказали друг другу: пойдем, бросим жребии, чтобы узнать, за кого постигает нас эта беда. И бросили жребии, и пал жребий на Иону.
\vs Jon 1:8 Тогда сказали ему: скажи нам, за кого постигла нас эта беда? какое твое занятие, и откуда идешь ты? где твоя страна, и из какого ты народа?
\vs Jon 1:9 И он сказал им: я Еврей, чту Господа Бога небес, сотворившего море и сушу.
\vs Jon 1:10 И устрашились люди страхом великим и сказали ему: для чего ты это сделал? Ибо узнали эти люди, что он бежит от лица Господня, как он сам объявил им.
\vs Jon 1:11 И сказали ему: что сделать нам с тобою, чтобы море утихло для нас? Ибо море не переставало волноваться.
\vs Jon 1:12 Тогда он сказал им: возьмите меня и бросьте меня в море, и море утихнет для вас, ибо я знаю, что ради меня постигла вас эта великая буря.
\vs Jon 1:13 Но эти люди начали усиленно грести, чтобы пристать к земле, но не могли, потому что море все продолжало бушевать против них.
\vs Jon 1:14 Тогда воззвали они к Господу и сказали: молим Тебя, Господи, да не погибнем за душу человека сего, и да не вменишь нам кровь невинную; ибо Ты, Господи, соделал, что угодно Тебе!
\vs Jon 1:15 И взяли Иону и бросили его в море, и утихло море от ярости своей.
\vs Jon 1:16 И устрашились эти люди Господа великим страхом, и принесли Господу жертву, и дали обеты.
\vs Jon 2:1 И повелел Господь большому киту поглотить Иону; и был Иона во чреве этого кита три дня и три ночи.
\vs Jon 2:2 И помолился Иона Господу Богу своему из чрева кита
\vs Jon 2:3 и сказал: к Господу воззвал я в скорби моей, и Он услышал меня; из чрева преисподней я возопил, и Ты услышал голос мой.
\vs Jon 2:4 Ты вверг меня в глубину, в сердце моря, и потоки окружили меня, все воды Твои и волны Твои проходили надо мною.
\vs Jon 2:5 И я сказал: отринут я от очей Твоих, однако я опять увижу святый храм Твой.
\vs Jon 2:6 Объяли меня воды до души моей, бездна заключила меня; морскою травою обвита была голова моя.
\vs Jon 2:7 До основания гор я нисшел, земля своими запорами навек заградила меня; но Ты, Господи Боже мой, изведешь душу мою из ада.
\vs Jon 2:8 Когда изнемогла во мне душа моя, я вспомнил о Господе, и молитва моя дошла до Тебя, до храма святаго Твоего.
\vs Jon 2:9 Чтущие суетных и ложных \bibemph{богов} оставили Милосердаго своего,
\vs Jon 2:10 а я гласом хвалы принесу Тебе жертву; что обещал, исполню: у Господа спасение!
\vs Jon 2:11 И сказал Господь киту, и он изверг Иону на сушу.
\vs Jon 3:1 И было слово Господне к Ионе вторично:
\vs Jon 3:2 встань, иди в Ниневию, город великий, и проповедуй в ней, чт\acc{о} Я повелел тебе.
\vs Jon 3:3 И встал Иона и пошел в Ниневию, по слову Господню; Ниневия же была город великий у Бога, на три дня ходьбы.
\vs Jon 3:4 И начал Иона ходить по городу, сколько можно пройти в один день, и проповедовал, говоря: еще сорок дней и Ниневия будет разрушена!
\vs Jon 3:5 И поверили Ниневитяне Богу, и объявили пост, и оделись во вретища, от большого из них до малого.
\vs Jon 3:6 Это слово дошло до царя Ниневии, и он встал с престола своего, и снял с себя царское облачение свое, и оделся во вретище, и сел на пепле,
\vs Jon 3:7 и повелел провозгласить и сказать в Ниневии от имени царя и вельмож его: <<чтобы ни люди, ни скот, ни волы, ни овцы ничего не ели, не ходили на пастбище и воды не пили,
\vs Jon 3:8 и чтобы покрыты были вретищем люди и скот и крепко вопияли к Богу, и чтобы каждый обратился от злого пути своего и от насилия рук своих.
\vs Jon 3:9 Кто знает, может быть, еще Бог умилосердится и отвратит от нас пылающий гнев Свой, и мы не погибнем>>.
\vs Jon 3:10 И увидел Бог дела их, что они обратились от злого пути своего, и пожалел Бог о бедствии, о котором сказал, что наведет на них, и не навел.
\vs Jon 4:1 Иона сильно огорчился этим и был раздражен.
\vs Jon 4:2 И молился он Господу и сказал: о, Господи! не это ли говорил я, когда еще был в стране моей? Потому я и побежал в Фарсис, ибо знал, что Ты Бог благий и милосердый, долготерпеливый и многомилостивый и сожалеешь о бедствии.
\vs Jon 4:3 И ныне, Господи, возьми душу мою от меня, ибо лучше мне умереть, нежели жить.
\vs Jon 4:4 И сказал Господь: неужели это огорчило тебя так сильно?
\vs Jon 4:5 И вышел Иона из города, и сел с восточной стороны у города, и сделал себе там кущу, и сел под нею в тени, чтобы увидеть, что будет с городом.
\vs Jon 4:6 И произрастил Господь Бог растение, и оно поднялось над Ионою, чтобы над головою его была тень и чтобы избавить его от огорчения его; Иона весьма обрадовался этому растению.
\vs Jon 4:7 И устроил Бог так, что на другой день при появлении зари червь подточил растение, и оно засохло.
\vs Jon 4:8 Когда же взошло солнце, навел Бог знойный восточный ветер, и солнце стало палить голову Ионы, так что он изнемог и просил себе смерти, и сказал: лучше мне умереть, нежели жить.
\vs Jon 4:9 И сказал Бог Ионе: неужели так сильно огорчился ты за растение? Он сказал: очень огорчился, даже до смерти.
\vs Jon 4:10 Тогда сказал Господь: ты сожалеешь о растении, над которым ты не трудился и которого не растил, которое в одну ночь выросло и в одну же ночь и пропало:
\vs Jon 4:11 Мне ли не пожалеть Ниневии, города великого, в котором более ста двадцати тысяч человек, не умеющих отличить правой руки от левой, и множество скота?

\bibbookdescr{Mic}{
  inline={\LARGE Книга\\\Huge Пророка Михея},
  toc={Михей},
  bookmark={Михей},
  header={Михей},
  %headerleft={},
  %headerright={},
  abbr={Мих}
}
\vs Mic 1:1 Слово Господне, которое было к Михею Морасфитину во дни Иоафама, Ахаза и Езекии, царей Иудейских, и которое открыто ему о Самарии и Иерусалиме.
\rsbpar\vs Mic 1:2 Слушайте, все народы, внимай, земля и все, что наполняет ее! Да будет Господь Бог свидетелем против вас, Господь из святаго храма Своего!
\vs Mic 1:3 Ибо вот, Господь исходит от места Своего, низойдет и наступит на высоты земли,~---
\vs Mic 1:4 и горы растают под Ним, долины распадутся, как воск от огня, как воды, льющиеся с крутизны.
\vs Mic 1:5 Все это~--- за нечестие Иакова, за грех дома Израилева. От кого нечестие Иакова? не от Самарии ли? Кто \bibemph{устроил} высоты в Иудее? не Иерусалим ли?
\vs Mic 1:6 За то сделаю Самарию грудою развалин в поле, местом для разведения винограда; низрину в долину камни ее и обнажу основания ее.
\vs Mic 1:7 Все истуканы ее будут разбиты и все любодейные дары ее сожжены будут огнем, и всех идолов ее предам разрушению, ибо из любодейных даров она устраивала их, на любодейные дары они и будут обращены.
\vs Mic 1:8 Об этом буду я плакать и рыдать, буду ходить, как ограбленный и обнаженный, выть, как шакалы, и плакать, как страусы,
\vs Mic 1:9 потому что болезненно поражение ее, дошло до Иуды, достигло даже до ворот народа моего, до Иерусалима.
\vs Mic 1:10 Не объявляйте об этом в Гефе, не плачьте там громко; но в селении Офра покрой себя пеплом.
\vs Mic 1:11 Переселяйтесь, жительницы Шафира, срамно обнаженные; не убежит и живущая в Цаане; плач в селении Ецель не даст вам остановиться в нем.
\vs Mic 1:12 Горюет о своем добре жительница Марофы, ибо сошло бедствие от Господа к воротам Иерусалима.
\vs Mic 1:13 Запрягай в колесницу быстрых, жительница Лахиса; ты~--- начало греха дщери Сионовой, ибо у тебя появились преступления Израиля.
\vs Mic 1:14 Посему ты посылать будешь дары в Морешеф-Геф; но селения Ахзива будут обманом для царей Израилевых.
\vs Mic 1:15 Еще наследника приведу к тебе, жительница Мореша; он пройдет до Одоллама, славы Израиля.
\vs Mic 1:16 Сними с себя волосы, остригись, скорбя о нежно любимых сынах твоих; расширь из-за них лысину, как у линяющего орла, ибо они переселены будут от тебя.
\vs Mic 2:1 Горе замышляющим беззаконие и на ложах своих придумывающим злодеяния, которые совершают утром на рассвете, потому что есть в руке их сила!
\vs Mic 2:2 Пожелают полей и берут их силою, домов,~--- и отнимают их; обирают человека и его дом, мужа и его наследие.
\vs Mic 2:3 Посему так говорит Господь: вот, Я помышляю навести на этот род такое бедствие, которого вы не свергнете с шеи вашей, и не будете ходить выпрямившись; ибо это время злое.
\vs Mic 2:4 В тот день произнесут о вас притчу и будут плакать горьким плачем и говорить: <<мы совершенно разорены! удел народа моего отдан другим; как возвратится ко мне! поля наши уже разделены иноплеменникам>>.
\vs Mic 2:5 Посему не будет у тебя никого, кто бросил бы жребий для измерения в собрании пред Господом.
\vs Mic 2:6 Не пророчествуйте, пророки; не пророчествуйте им, чтобы не постигло вас бесчестие.
\vs Mic 2:7 О, называющийся домом Иакова! разве умалился Дух Господень? таковы ли действия Его? не благотворны ли слова Мои для того, кто поступает справедливо?
\vs Mic 2:8 Народ же, который был прежде Моим, восстал как враг, и вы отнимаете как верхнюю, так и нижнюю одежду у проходящих мирно, отвращающихся войны.
\vs Mic 2:9 Жен народа Моего вы изгоняете из приятных домов их; у детей их вы навсегда отнимаете украшение Мое.
\vs Mic 2:10 Встаньте и уходите, ибо \bibemph{страна} сия не есть место покоя; за нечистоту она будет разорена и притом жестоким разорением.
\vs Mic 2:11 Если бы какой-либо ветреник выдумал ложь и сказал: <<я буду проповедовать тебе о вине и сикере>>, то он и был бы угодным проповедником для этого народа.
\vs Mic 2:12 Непременно соберу всего тебя, Иаков, непременно соединю остатки Израиля, совокуплю их воедино, как овец в Восоре, как стадо в овечьем загоне; зашумят они от многолюдства.
\vs Mic 2:13 Перед ними пойдет стенорушитель; они сокрушат преграды, войдут сквозь ворота и выйдут ими; и царь их пойдет перед ними, а во главе их Господь.
\vs Mic 3:1 И сказал я: слушайте, главы Иакова и князья дома Израилева: не вам ли должно знать правду?
\vs Mic 3:2 А вы ненавидите доброе и любите злое; сдираете с них кожу их и плоть с костей их,
\vs Mic 3:3 едите плоть народа Моего и сдираете с них кожу их, а кости их ломаете и дробите как бы в горшок, и плоть~--- как бы в котел.
\vs Mic 3:4 И будут они взывать к Господу, но Он не услышит их и сокроет лице Свое от них на то время, как они злодействуют.
\vs Mic 3:5 Так говорит Господь на пророков, вводящих в заблуждение народ Мой, которые грызут зубами своими~--- и проповедуют мир, а кто ничего не кладет им в рот, против того объявляют войну.
\vs Mic 3:6 Посему ночь будет вам вместо видения, и тьма~--- вместо предвещаний; зайдет солнце над пророками и потемнеет день над ними.
\vs Mic 3:7 И устыдятся прозорливцы, и посрамлены будут гадатели, и закроют уста свои все они, потому что не будет ответа от Бога.
\vs Mic 3:8 А я исполнен силы Духа Господня, правоты и твердости, чтобы высказать Иакову преступление его и Израилю грех его.
\vs Mic 3:9 Слушайте же это, главы дома Иаковлева и князья дома Израилева, гнушающиеся правосудием и искривляющие все прямое,
\vs Mic 3:10 созидающие Сион кровью и Иерусалим~--- неправдою!
\vs Mic 3:11 Главы его судят за подарки и священники его учат за плату, и пророки его предвещают за деньги, а между тем опираются на Господа, говоря: <<не среди ли нас Господь? не постигнет нас беда!>>
\vs Mic 3:12 Посему за вас Сион распахан будет как поле, и Иерусалим сделается грудою развалин, и гора дома сего будет лесистым холмом.
\vs Mic 4:1 И будет в последние дни: гора дома Господня поставлена будет во главу гор и возвысится над холмами, и потекут к ней народы.
\vs Mic 4:2 И пойдут многие народы и скажут: придите, и взойдем на гору Господню и в дом Бога Иаковлева, и Он научит нас путям Своим, и будем ходить по стезям Его, ибо от Сиона выйдет закон и слово Господне~--- из Иерусалима.
\vs Mic 4:3 И будет Он судить многие народы, и обличит многие племена в отдаленных странах; и перекуют они мечи свои на орала и копья свои~--- на серпы; не поднимет народ на народ меча, и не будут более учиться воевать.
\vs Mic 4:4 Но каждый будет сидеть под своею виноградною лозою и под своею смоковницею, и никто не будет устрашать их, ибо уста Господа Саваофа изрекли это.
\vs Mic 4:5 Ибо все народы ходят, каждый во имя своего бога; а мы будем ходить во имя Господа Бога нашего во веки веков.
\vs Mic 4:6 В тот день, говорит Господь, соберу хромлющее и совокуплю разогнанное и тех, на кого Я навел бедствие.
\vs Mic 4:7 И сделаю хромлющее остатком и далеко рассеянное сильным народом, и Господь будет царствовать над ними на горе Сионе отныне и до века.
\vs Mic 4:8 А ты, башня стада, холм дщери Сиона! к тебе придет и возвратится прежнее владычество, царство~--- к дщерям Иерусалима.
\vs Mic 4:9 Для чего же ты ныне так громко вопиешь? Разве нет у тебя царя? Или не стало у тебя советника, что тебя схватили муки, как рождающую?
\vs Mic 4:10 Страдай и мучься болями, дщерь Сиона, как рождающая, ибо ныне ты выйдешь из города и будешь жить в поле, и дойдешь до Вавилона: там будешь спасена, там искупит тебя Господь от руки врагов твоих.
\vs Mic 4:11 А теперь собрались против тебя многие народы и говорят: <<да будет она осквернена, и да наглядится око наше на Сион!>>
\vs Mic 4:12 Но они не знают мыслей Господних и не разумеют совета Его, что Он собрал их как снопы на гумно.
\vs Mic 4:13 Встань и молоти, дщерь Сиона, ибо Я сделаю рог твой железным и копыта твои сделаю медными, и сокрушишь многие народы, и посвятишь Господу стяжания их и богатства их Владыке всей земли.
\vs Mic 5:1 Теперь ополчись, дщерь полчищ; обложили нас осадою, тростью будут бить по ланите судью Израилева.
\rsbpar\vs Mic 5:2 И ты, Вифлеем-Ефрафа, мал ли ты между тысячами Иудиными? из тебя произойдет Мне Тот, Который должен быть Владыкою в Израиле и Которого происхождение из начала, от дней вечных.
\vs Mic 5:3 Посему Он оставит их до времени, доколе не родит имеющая родить; тогда возвратятся к сынам Израиля и оставшиеся братья их.
\vs Mic 5:4 И станет Он, и будет пасти в силе Господней, в величии имени Господа Бога Своего, и они будут жить безопасно, ибо тогда Он будет великим до краев земли.
\vs Mic 5:5 И будет Он мир. Когда Ассур придет в нашу землю и вступит в наши чертоги, мы выставим против него семь пастырей и восемь князей.
\vs Mic 5:6 И будут они пасти землю Ассура мечом и землю Немврода в самых воротах ее, и Он-то избавит от Ассура, когда тот придет в землю нашу и когда вступит в пределы наши.
\vs Mic 5:7 И будет остаток Иакова среди многих народов как роса от Господа, как ливень на траве, и он не будет зависеть от человека и полагаться на сынов Адамовых.
\vs Mic 5:8 И будет остаток Иакова между народами, среди многих племен, как лев среди зверей лесных, как скимен среди стада овец, который, когда выступит, то попирает и терзает, и никто не спасет от него.
\vs Mic 5:9 Поднимется рука твоя над врагами твоими, и все неприятели твои будут истреблены.
\vs Mic 5:10 И будет в тот день, говорит Господь: истреблю коней твоих из среды твоей и уничтожу колесницы твои,
\vs Mic 5:11 истреблю города в земле твоей и разрушу все укрепления твои,
\vs Mic 5:12 исторгну чародеяния из руки твоей, и гадающих по облакам не будет у тебя;
\vs Mic 5:13 истреблю истуканов твоих и кумиров из среды твоей, и не будешь более поклоняться изделиям рук твоих.
\vs Mic 5:14 Искореню из среды твоей священные рощи твои и разорю города твои.
\vs Mic 5:15 И совершу в гневе и негодовании мщение над народами, которые будут непослушны.
\vs Mic 6:1 Слушайте, что говорит Господь: встань, судись перед горами, и холмы да слышат голос твой!
\vs Mic 6:2 Слушайте, горы, суд Господень, и вы, твердые основы земли: ибо у Господа суд с народом Своим, и с Израилем Он состязуется.
\vs Mic 6:3 Народ Мой! что сделал Я тебе и чем отягощал тебя? отвечай Мне.
\vs Mic 6:4 Я вывел тебя из земли Египетской и искупил тебя из дома рабства, и послал перед тобою Моисея, Аарона и Мариам.
\vs Mic 6:5 Народ Мой! вспомни, что замышлял Валак, царь Моавитский, и что отвечал ему Валаам, сын Веоров, и что \bibemph{происходило} от Ситтима до Галгал, чтобы познать тебе праведные действия Господни.
\rsbpar\vs Mic 6:6 <<С чем предстать мне пред Господом, преклониться пред Богом небесным? Предстать ли пред Ним со всесожжениями, с тельцами однолетними?
\vs Mic 6:7 Но можно ли угодить Господу тысячами овнов или неисчетными потоками елея? Разве дам Ему первенца моего за преступление мое и плод чрева моего~--- за грех души моей?>>
\vs Mic 6:8 О, человек! сказано тебе, чт\acc{о}~--- добро и чего требует от тебя Господь: действовать справедливо, любить дела милосердия и смиренномудренно ходить пред Богом твоим.
\vs Mic 6:9 Глас Господа взывает к городу, и мудрость благоговеет пред именем Твоим: слушайте жезл и Того, Кто поставил его.
\vs Mic 6:10 Не находятся ли и теперь в доме нечестивого сокровища нечестия и уменьшенная мера, отвратительная?
\vs Mic 6:11 Могу ли я быть чистым с весами неверными и с обманчивыми гирями в суме?
\vs Mic 6:12 Так как богачи его исполнены неправды, и жители его говорят ложь, и язык их есть обман в устах их,
\vs Mic 6:13 то и Я неисцельно поражу тебя опустошением за грехи твои.
\vs Mic 6:14 Ты будешь есть, и не будешь сыт; пустота будет внутри тебя; будешь хранить, но не убережешь, а что сбережешь, то предам мечу.
\vs Mic 6:15 Будешь сеять, а жать не будешь; будешь давить оливки, и не будешь умащаться елеем; выжмешь виноградный сок, а вина пить не будешь.
\vs Mic 6:16 Сохранились у вас обычаи Амврия и все дела дома Ахавова, и вы поступаете по советам их; и предам Я тебя опустошению и жителей твоих посмеянию, и вы понесете поругание народа Моего.
\vs Mic 7:1 Горе мне! ибо со мною теперь~--- как по собрании летних плодов, как по уборке винограда: ни одной ягоды для еды, ни спелого плода, которого желает душа моя.
\vs Mic 7:2 Не стало милосердых на земле, нет правдивых между людьми; все строят ковы, чтобы проливать кровь; каждый ставит брату своему сеть.
\vs Mic 7:3 Руки их обращены к тому, чтобы уметь делать зло; начальник требует подарков, и судья судит за взятки, а вельможи высказывают злые хотения души своей и извращают дело.
\vs Mic 7:4 Лучший из них~--- как терн, и справедливый~--- хуже колючей изгороди, день провозвестников Твоих, посещение Твое наступает; ныне постигнет их смятение.
\vs Mic 7:5 Не верьте другу, не полагайтесь на приятеля; от лежащей на лоне твоем стереги двери уст твоих.
\vs Mic 7:6 Ибо сын позорит отца, дочь восстает против матери, невестка~--- против свекрови своей; враги человеку~--- домашние его.
\vs Mic 7:7 А я буду взирать на Господа, уповать на Бога спасения моего: Бог мой услышит меня.
\vs Mic 7:8 Не радуйся ради меня, неприятельница моя! хотя я упал, но встану; хотя я во мраке, но Господь свет для меня.
\vs Mic 7:9 Гнев Господень я буду нести, потому что согрешил пред Ним, доколе Он не решит дела моего и не совершит суда надо мною; тогда Он выведет меня на свет, и я увижу правду Его.
\vs Mic 7:10 И увидит это неприятельница моя и стыд покроет ее, говорившую мне: <<где Господь Бог твой?>> Насмотрятся на нее глаза мои, как она будет попираема подобно грязи на улицах.
\vs Mic 7:11 В день сооружения стен твоих, в этот день отдалится определение.
\vs Mic 7:12 В тот день придут к тебе из Ассирии и городов Египетских, и от Египта до реки \bibemph{Евфрата}, и от моря до моря, и от горы до горы.
\vs Mic 7:13 А земля та будет пустынею за \bibemph{вину} жителей ее, за плоды деяний их.
\vs Mic 7:14 Паси народ Твой жезлом Твоим, овец наследия Твоего, обитающих уединенно в лесу среди Кармила; да пасутся они на Васане и Галааде, как во дни древние!
\vs Mic 7:15 Как во дни исхода твоего из земли Египетской, явлю ему дивные дела.
\vs Mic 7:16 Увидят это народы и устыдятся при всем могуществе своем; положат руку на уста, уши их сделаются глухими;
\vs Mic 7:17 будут лизать прах как змея, как черви земные выползут они из укреплений своих; устрашатся Господа Бога нашего и убоятся Тебя.
\vs Mic 7:18 Кто Бог, как Ты, прощающий беззаконие и не вменяющий преступления остатку наследия Твоего? не вечно гневается Он, потому что любит миловать.
\vs Mic 7:19 Он опять умилосердится над нами, изгладит беззакония наши. Ты ввергнешь в пучину морскую все грехи наши.
\vs Mic 7:20 Ты явишь верность Иакову, милость Аврааму, которую с клятвою обещал отцам нашим от дней первых.

\bibbookdescr{Nah}{
  inline={\LARGE Книга\\\Huge Пророка Наума},
  toc={Наум},
  bookmark={Наум},
  header={Наум},
  %headerleft={},
  %headerright={},
  abbr={Наум}
}
\vs Nah 1:1 Пророчество о Ниневии; книга видений Наума Елкосеянина.
\rsbpar\vs Nah 1:2 Господь есть Бог ревнитель и мститель; мститель Господь и страшен в гневе: мстит Господь врагам Своим и не пощадит противников Своих.
\vs Nah 1:3 Господь долготерпелив и велик могуществом, и не оставляет без наказания; в вихре и в буре шествие Господа, облако~--- пыль от ног Его.
\vs Nah 1:4 Запретит Он морю, и оно высыхает, и все реки иссякают; вянет Васан и Кармил, и блекнет цвет на Ливане.
\vs Nah 1:5 Горы трясутся пред Ним, и холмы тают, и земля колеблется пред лицем Его, и вселенная и все живущие в ней.
\vs Nah 1:6 Пред негодованием Его кто устоит? И кто стерпит пламя гнева Его? Гнев Его разливается как огонь; скалы распадаются пред Ним.
\vs Nah 1:7 Благ Господь, убежище в день скорби, и знает надеющихся на Него.
\vs Nah 1:8 Но всепотопляющим наводнением разрушит до основания \bibemph{Ниневию}, и врагов Его постигнет мрак.
\vs Nah 1:9 Что умышляете вы против Господа? Он совершит истребление, и бедствие уже не повторится,
\vs Nah 1:10 ибо сплетшиеся между собою как терновник и упившиеся как пьяницы, они пожраны будут совершенно, как сухая солома.
\vs Nah 1:11 Из тебя произошел умысливший злое против Господа, составивший совет нечестивый.
\rsbpar\vs Nah 1:12 Так говорит Господь: хотя они безопасны и многочисленны, но они будут посечены и исчезнут; а тебя, хотя Я отягощал, более не буду отягощать.
\vs Nah 1:13 И ныне Я сокрушу ярмо его, лежащее на тебе, и узы твои разорву.
\vs Nah 1:14 А о тебе, \bibemph{Ассур}, Господь определил: не будет более семени с твоим именем; из дома бога твоего истреблю истуканов и кумиров; приготовлю тебе в нем могилу, потому что ты будешь в презрении.
\vs Nah 1:15 Вот, на горах~--- стопы благовестника, возвещающего мир: празднуй, Иудея, праздники твои, исполняй обеты твои, ибо не будет более проходить по тебе нечестивый: он совсем уничтожен.
\vs Nah 2:1 Поднимается на тебя разрушитель: охраняй твердыни, стереги дорогу, укрепи чресла, собирайся с силами.
\vs Nah 2:2 Ибо восстановит Господь величие Иакова, как величие Израиля, потому что опустошили их опустошители и виноградные ветви их истребили.
\vs Nah 2:3 Щит героев его красен; воины его в одеждах багряных; огнем сверкают колесницы в день приготовления к бою, и лес копьев волнуется.
\vs Nah 2:4 По улицам несутся колесницы, гремят на площадях; блеск от них, как от огня; сверкают, как молния.
\vs Nah 2:5 Он вызывает храбрых своих, но они спотыкаются на ходу своем; поспешают на стены города, но осада уже устроена.
\vs Nah 2:6 Речные ворота отворяются, и дворец разрушается.
\vs Nah 2:7 Решено: она будет обнажена и отведена в плен, и рабыни ее будут стонать как голуби, ударяя себя в грудь.
\vs Nah 2:8 Ниневия со времени существования своего была как пруд, полный водою, а они бегут. <<Стойте, стойте!>> Но никто не оглядывается.
\vs Nah 2:9 Расхищайте серебро, расхищайте золото! нет конца запасам всякой драгоценной утвари.
\vs Nah 2:10 Разграблена, опустошена и разорена она,~--- и тает сердце, колени трясутся; у всех в чреслах сильная боль, и лица у всех потемнели.
\vs Nah 2:11 Где теперь логовище львов и то пастбище для львенков, по которому ходил лев, львица и львенок, и никто не пугал их,~---
\vs Nah 2:12 лев, похищающий для насыщения щенков своих, и задушающий для львиц своих, и наполняющий добычею пещеры свои и логовища свои похищенным?
\vs Nah 2:13 Вот, Я~--- на тебя! говорит Господь Саваоф. И сожгу в дыму колесницы твои, и меч пожрет львенков твоих, и истреблю с земли добычу твою, и не будет более слышим голос послов твоих.
\vs Nah 3:1 Горе городу кровей! весь он полон обмана и убийства; не прекращается в нем грабительство.
\vs Nah 3:2 Слышны хлопанье бича и стук крутящихся колес, ржание коня и грохот скачущей колесницы.
\vs Nah 3:3 Несется конница, сверкает меч и блестят копья; убитых множество и груды трупов: нет конца трупам, спотыкаются о трупы их.
\vs Nah 3:4 Это~--- за многие блудодеяния развратницы приятной наружности, искусной в чародеянии, которая блудодеяниями своими продает народы и чарованиями своими~--- племена.
\vs Nah 3:5 Вот, Я~--- на тебя! говорит Господь Саваоф. И подниму на лице твое края одежды твоей и покажу народам наготу твою и царствам срамоту твою.
\vs Nah 3:6 И забросаю тебя мерзостями, сделаю тебя презренною и выставлю тебя на позор.
\vs Nah 3:7 И будет то, что всякий, увидев тебя, побежит от тебя и скажет: <<разорена Ниневия! Кто пожалеет о ней? где найду я утешителей для тебя?>>
\vs Nah 3:8 Разве ты лучше Но-Аммона, находящегося между реками, окруженного водою, которого вал было море, и море служило стеною его?
\vs Nah 3:9 Ефиопия и Египет с бесчисленным множеством других служили ему подкреплением; Копты и Ливийцы приходили на помощь тебе.
\vs Nah 3:10 Но и он переселен, пошел в плен; даже и младенцы его разбиты на перекрестках всех улиц, а о знатных его бросали жребий, и все вельможи его окованы цепями.
\vs Nah 3:11 Так и ты~--- опьянеешь и скроешься; так и ты будешь искать защиты от неприятеля.
\vs Nah 3:12 Все укрепления твои подобны смоковнице со спелыми плодами: если тряхнуть их, то они упадут прямо в рот желающего есть.
\vs Nah 3:13 Вот, и народ твой, как женщины у тебя: врагам твоим настежь отворятся ворота земли твоей, огонь пожрет запоры твои.
\vs Nah 3:14 Начерпай воды на время осады; укрепляй крепости твои; пойди в грязь, топчи глину, исправь печь для обжигания кирпичей.
\vs Nah 3:15 Там пожрет тебя огонь, посечет тебя меч, поест тебя как гусеница, хотя бы ты умножился как гусеница, умножился как саранча.
\vs Nah 3:16 Купцов у тебя стало более, нежели звезд на небе; но эта саранча рассеется и улетит.
\vs Nah 3:17 Князья твои~--- как саранча, и военачальники твои~--- как рои мошек, которые во время холода гнездятся в щелях \bibemph{стен}, и когда взойдет солнце, то разлетаются, и не узнаешь места, где они были.
\vs Nah 3:18 Спят пастыри твои, царь Ассирийский, покоятся вельможи твои; народ твой рассеялся по горам, и некому собрать его.
\vs Nah 3:19 Нет врачевства для раны твоей, болезненна язва твоя. Все, услышавшие весть о тебе, будут рукоплескать о тебе, ибо на кого не простиралась беспрестанно злоба твоя?
\newbookpage
\bibbookdescr{Hab}{
  inline={\LARGE Книга\\\Huge Пророка Аввакума},
  toc={Аввакум},
  bookmark={Аввакум},
  header={Аввакум},
  %headerleft={},
  %headerright={},
  abbr={Авв}
}
\vs Hab 1:1 Пророческое видение, которое видел пророк Аввакум.
\rsbpar\vs Hab 1:2 Доколе, Господи, я буду взывать, и Ты не слышишь, буду вопиять к Тебе о насилии, и Ты не спасаешь?
\vs Hab 1:3 Для чего даешь мне видеть злодейство и смотреть на бедствия? Грабительство и насилие предо мною, и восстает вражда и поднимается раздор.
\vs Hab 1:4 От этого закон потерял силу, и суда правильного нет: так как нечестивый одолевает праведного, то и суд происходит превратный.
\vs Hab 1:5 Посмотрите между народами и внимательно вглядитесь, и вы сильно изумитесь; ибо Я сделаю во дни ваши такое дело, которому вы не поверили бы, если бы вам рассказывали.
\vs Hab 1:6 Ибо вот, Я подниму Халдеев, народ жестокий и необузданный, который ходит по широтам земли, чтобы завладеть не принадлежащими ему селениями.
\vs Hab 1:7 Страшен и грозен он; от него самого происходит суд его и власть его.
\vs Hab 1:8 Быстрее барсов кони его и прытче вечерних волков; скачет в разные стороны конница его; издалека приходят всадники его, прилетают как орел, бросающийся на добычу.
\vs Hab 1:9 Весь он идет для грабежа; устремив лице свое вперед, он забирает пленников, как песок.
\vs Hab 1:10 И над царями он издевается, и князья служат ему посмешищем; над всякою крепостью он смеется: насыплет осадный вал и берет ее.
\vs Hab 1:11 Тогда надмевается дух его, и он ходит и буйствует; сила его~--- бог его.
\vs Hab 1:12 Но не Ты ли издревле Господь Бог мой, Святый мой? мы не умрем! Ты, Господи, только для суда попустил его. Скала моя! для наказания Ты назначил его.
\vs Hab 1:13 Чистым очам Твоим не свойственно глядеть на злодеяния, и смотреть на притеснение Ты не можешь; для чего же Ты смотришь на злодеев и безмолвствуешь, когда нечестивец поглощает того, кто праведнее его,
\vs Hab 1:14 и оставляешь людей как рыбу в море, как пресмыкающихся, у которых нет властителя?
\vs Hab 1:15 Всех их таскает удою, захватывает в сеть свою и забирает их в неводы свои, и оттого радуется и торжествует.
\vs Hab 1:16 За то приносит жертвы сети своей и кадит неводу своему, потому что от них тучна часть его и роскошна пища его.
\vs Hab 1:17 Неужели для этого он должен опорожнять свою сеть и непрестанно избивать народы без пощады?
\vs Hab 2:1 На стражу мою стал я и, стоя на башне, наблюдал, чтобы узнать, что скажет Он во мне, и что мне отвечать по жалобе моей?
\vs Hab 2:2 И отвечал мне Господь и сказал: запиши видение и начертай ясно на скрижалях, чтобы читающий легко мог прочитать,
\vs Hab 2:3 ибо видение относится еще к определенному времени и говорит о конце и не обманет; и хотя бы и замедлило, жди его, ибо непременно сбудется, не отменится.
\vs Hab 2:4 Вот, душа надменная не успокоится, а праведный своею верою жив будет.
\vs Hab 2:5 Надменный человек, как бродящее вино, не успокаивается, так что расширяет душу свою как ад, и как смерть он ненасытен, и собирает к себе все народы, и захватывает себе все племена.
\vs Hab 2:6 Но не все ли они будут произносить о нем притчу и насмешливую песнь: <<горе тому, кто без меры обогащает себя не своим,~--- на долго ли?~--- и обременяет себя залогами!>>
\vs Hab 2:7 Не восстанут ли внезапно те, которые будут терзать тебя, и не поднимутся ли против тебя грабители, и ты достанешься им на расхищение?
\vs Hab 2:8 Так как ты ограбил многие народы, то и тебя ограбят все остальные народы за пролитие крови человеческой, за разорение страны, города и всех живущих в нем.
\rsbpar\vs Hab 2:9 Горе тому, кто жаждет неправедных приобретений для дома своего, чтобы устроить гнездо свое на высоте и тем обезопасить себя от руки несчастья!
\vs Hab 2:10 Бесславие измыслил ты для твоего дома, истребляя многие народы, и согрешил против души твоей.
\vs Hab 2:11 Камни из стен возопиют и перекладины из дерева будут отвечать им:
\vs Hab 2:12 <<горе строящему город на крови и созидающему крепости неправдою!>>
\vs Hab 2:13 Вот, не от Господа ли Саваофа это, что народы трудятся для огня и племена мучат себя напрасно?
\vs Hab 2:14 Ибо земля наполнится познанием славы Господа, как воды наполняют море.
\vs Hab 2:15 Горе тебе, который подаешь ближнему твоему питье с примесью злобы твоей и делаешь его пьяным, чтобы видеть срамоту его!
\vs Hab 2:16 Ты пресытился стыдом вместо славы; пей же и ты и показывай срамоту,~--- обратится и к тебе чаша десницы Господней и посрамление на славу твою.
\vs Hab 2:17 Ибо злодейство твое на Ливане обрушится на тебя за истребление устрашенных животных, за пролитие крови человеческой, за опустошение страны, города и всех живущих в нем.
\vs Hab 2:18 Что за польза от истукана, сделанного художником, этого литого лжеучителя, хотя ваятель, делая немые кумиры, полагается на свое произведение?
\vs Hab 2:19 Горе тому, кто говорит дереву: <<встань!>> и бессловесному камню: <<пробудись!>> Научит ли он чему-нибудь? Вот, он обложен золотом и серебром, но дыхания в нем нет.
\vs Hab 2:20 А Господь~--- во святом храме Своем: да молчит вся земля пред лицем Его!
\vs Hab 3:1 Молитва Аввакума пророка, для пения.
\vs Hab 3:2 Господи! услышал я слух Твой и убоялся. Господи! соверши дело Твое среди лет, среди лет яви его; во гневе вспомни о милости.
\vs Hab 3:3 Бог от Фемана грядет и Святый~--- от горы Фаран. Покрыло небеса величие Его, и славою Его наполнилась земля.
\vs Hab 3:4 Блеск ее~--- как солнечный свет; от руки Его лучи, и здесь тайник Его силы!
\vs Hab 3:5 Пред лицем Его идет язва, а по стопам Его~--- жгучий ветер.
\vs Hab 3:6 Он стал и поколебал землю; воззрел, и в трепет привел народы; вековые горы распались, первобытные холмы опали; пути Его вечные.
\vs Hab 3:7 Грустными видел я шатры Ефиопские; сотряслись палатки земли Мадиамской.
\vs Hab 3:8 Разве на реки воспылал, Господи, гнев Твой? разве на реки~--- негодование Твое, или на море~--- ярость Твоя, что Ты восшел на коней Твоих, на колесницы Твои спасительные?
\vs Hab 3:9 Ты обнажил лук Твой по клятвенному обетованию, данному коленам. Ты потоками рассек землю.
\vs Hab 3:10 Увидев Тебя, вострепетали горы, ринулись воды; бездна дала голос свой, высоко подняла руки свои;
\vs Hab 3:11 солнце и луна остановились на месте своем пред светом летающих стрел Твоих, пред сиянием сверкающих копьев Твоих.
\vs Hab 3:12 Во гневе шествуешь Ты по земле и в негодовании попираешь народы.
\vs Hab 3:13 Ты выступаешь для спасения народа Твоего, для спасения помазанного Твоего. Ты сокрушаешь главу нечестивого дома, обнажая его от основания до верха.
\vs Hab 3:14 Ты пронзаешь копьями его главу вождей его, когда они как вихрь ринулись разбить меня, в радости, как бы думая поглотить бедного скрытно.
\vs Hab 3:15 Ты с конями Твоими проложил путь по морю, через пучину великих вод.
\vs Hab 3:16 Я услышал, и вострепетала внутренность моя; при вести о сем задрожали губы мои, боль проникла в кости мои, и колеблется место подо мною; а я должен быть спокоен в день бедствия, когда придет на народ мой грабитель его.
\vs Hab 3:17 Хотя бы не расцвела смоковница и не было плода на виноградных лозах, и маслина изменила, и нива не дала пищи, хотя бы не стало овец в загоне и рогатого скота в стойлах,~---
\vs Hab 3:18 но и тогда я буду радоваться о Господе и веселиться о Боге спасения моего.
\vs Hab 3:19 Господь Бог~--- сила моя: Он сделает ноги мои как у оленя и на высоты мои возведет меня! (Начальнику хора.)

\bibbookdescr{Zep}{
  inline={\LARGE Книга\\\Huge Пророка Софонии},
  toc={Софония},
  bookmark={Софония},
  header={Софония},
  %headerleft={},
  %headerright={},
  abbr={Соф}
}
\vs Zep 1:1 Слово Господне, которое было к Софонии, сыну Хусия, сыну Годолии, сыну Амории, сыну Езекии, во дни Иосии, сына Амонова, царя Иудейского.
\rsbpar\vs Zep 1:2 Все истреблю с лица земли, говорит Господь:
\vs Zep 1:3 истреблю людей и скот, истреблю птиц небесных и рыб морских, и соблазны вместе с нечестивыми; истреблю людей с лица земли, говорит Господь.
\vs Zep 1:4 И простру руку Мою на Иудею и на всех жителей Иерусалима: истреблю с места сего остатки Ваала, имя жрецов со священниками,
\vs Zep 1:5 и тех, которые на кровлях поклоняются воинству небесному, и тех поклоняющихся, которые клянутся Господом и клянутся царем своим,
\vs Zep 1:6 и тех, которые отступили от Господа, не искали Господа и не вопрошали о Нем.
\vs Zep 1:7 Умолкни пред лицем Господа Бога! ибо близок день Господень: уже приготовил Господь жертвенное заклание, назначил, кого позвать.
\vs Zep 1:8 И будет в день жертвы Господней: Я посещу князей и сыновей царя и всех, одевающихся в одежду иноплеменников;
\vs Zep 1:9 посещу в тот день всех, которые перепрыгивают через порог, которые дом Господа своего наполняют насилием и обманом.
\vs Zep 1:10 И будет в тот день, говорит Господь, вопль у ворот рыбных и рыдание у других ворот и великое разрушение на холмах.
\vs Zep 1:11 Рыдайте, жители нижней части города, ибо исчезнет весь торговый народ и истреблены будут обремененные серебром.
\vs Zep 1:12 И будет в то время: Я со светильником осмотрю Иерусалим и накажу тех, которые сидят на дрожжах своих и говорят в сердце своем: <<не делает Господь ни добра, ни зла>>.
\vs Zep 1:13 И обратятся богатства их в добычу и домы их~--- в запустение; они построят домы, а жить в них не будут, насадят виноградники, а вина из них не будут пить.
\rsbpar\vs Zep 1:14 Близок великий день Господа, близок, и очень поспешает: уже слышен голос дня Господня; горько возопиет тогда и самый храбрый!
\vs Zep 1:15 День гнева~--- день сей, день скорби и тесноты, день опустошения и разорения, день тьмы и мрака, день облака и мглы,
\vs Zep 1:16 день трубы и бранного крика против укрепленных городов и высоких башен.
\vs Zep 1:17 И Я стесню людей, и они будут ходить, как слепые, потому что они согрешили против Господа, и разметана будет кровь их, как прах, и плоть их~--- как помет.
\vs Zep 1:18 Ни серебро их, ни золото их не может спасти их в день гнева Господа, и огнем ревности Его пожрана будет вся эта земля, ибо истребление, и притом внезапное, совершит Он над всеми жителями земли.
\vs Zep 2:1 Исследуйте себя внимательно, исследуйте, народ необузданный,
\vs Zep 2:2 доколе не пришло определение~--- день пролетит как мякина~--- доколе не пришел на вас пламенный гнев Господень, доколе не наступил для вас день ярости Господней.
\vs Zep 2:3 Взыщите Господа, все смиренные земли, исполняющие законы Его; взыщите правду, взыщите смиренномудрие; может быть, вы укроетесь в день гнева Господня.
\vs Zep 2:4 Ибо Газа будет покинута и Аскалон опустеет, Азот будет выгнан среди дня и Екрон искоренится.
\rsbpar\vs Zep 2:5 Горе жителям приморской страны, народу Критскому! Слово Господне на вас, Хананеи, земля Филистимская! Я истреблю тебя, и не будет у тебя жителей,~---
\vs Zep 2:6 и будет приморская страна пастушьим овчарником и загоном для скота.
\vs Zep 2:7 И достанется этот край остаткам дома Иудина, и будут пасти там, и в домах Аскалона будут вечером отдыхать, ибо Господь Бог их посетит их и возвратит плен их.
\vs Zep 2:8 Слышал Я поношение Моава и ругательства сынов Аммоновых, как они издевались над Моим народом и величались на пределах его.
\vs Zep 2:9 Посему, живу Я! говорит Господь Саваоф, Бог Израилев: Моав будет, как Содом, и сыны Аммона будут, как Гоморра, достоянием крапивы, соляною рытвиною, пустынею навеки; остаток народа Моего возьмет их в добычу, и уцелевшие из людей Моих получат их в наследие.
\vs Zep 2:10 Это им за высокомерие их, за то, что они издевались и величались над народом Господа Саваофа.
\vs Zep 2:11 Страшен будет для них Господь, ибо истребит всех богов земли, и Ему будут поклоняться, каждый со своего места, все острова народов.
\vs Zep 2:12 И вы, Ефиопляне, избиты будете мечом Моим.
\vs Zep 2:13 И прострет Он руку Свою на север, и уничтожит Ассура, и обратит Ниневию в развалины, в место сухое, как пустыня,
\vs Zep 2:14 и покоиться будут среди нее стада и всякого рода животные; пеликан и еж будут ночевать в резных украшениях ее; голос их будет раздаваться в окнах, разрушение обнаружится на дверных столбах, ибо не станет на них кедровой обшивки.
\vs Zep 2:15 Вот чем будет город торжествующий, живущий беспечно, говорящий в сердце своем: <<я, и нет иного кроме меня>>. Как он стал развалиною, логовищем для зверей! Всякий, проходя мимо него, посвищет и махнет рукою.
\vs Zep 3:1 Горе городу нечистому и оскверненному, притеснителю!
\vs Zep 3:2 Не слушает голоса, не принимает наставления, на Господа не уповает, к Богу своему не приближается.
\vs Zep 3:3 Князья его посреди него~--- рыкающие львы, судьи его~--- вечерние волки, не оставляющие до утра ни одной кости.
\vs Zep 3:4 Пророки его~--- люди легкомысленные, вероломные; священники его оскверняют святыню, попирают закон.
\vs Zep 3:5 Господь праведен посреди него, не делает неправды, каждое утро являет суд Свой неизменно; но беззаконник не знает стыда.
\vs Zep 3:6 Я истребил народы, разрушены твердыни их; пустыми сделал улицы их, так что никто уже не ходит по ним; разорены города их: нет ни одного человека, нет жителей.
\vs Zep 3:7 Я говорил: <<бойся только Меня, принимай наставление!>> и не будет истреблено жилище его, и не постигнет его зло, какое Я постановил о нем; а они прилежно старались портить все свои действия.
\rsbpar\vs Zep 3:8 Итак ждите Меня, говорит Господь, до того дня, когда Я восстану для опустошения, ибо Мною определено собрать народы, созвать царства, чтобы излить на них негодование Мое, всю ярость гнева Моего; ибо огнем ревности Моей пожрана будет вся земля.
\vs Zep 3:9 Тогда опять Я дам народам уста чистые, чтобы все призывали имя Господа и служили Ему единодушно.
\vs Zep 3:10 Из заречных стран Ефиопии поклонники Мои, дети рассеянных Моих, принесут Мне дары.
\vs Zep 3:11 В тот день ты не будешь срамить себя всякими поступками твоими, какими ты грешил против Меня, ибо тогда Я удалю из среды твоей тщеславящихся твоею знатностью, и не будешь более превозноситься на святой горе Моей.
\vs Zep 3:12 Но оставлю среди тебя народ смиренный и простой, и они будут уповать на имя Господне.
\vs Zep 3:13 Остатки Израиля не будут делать неправды, не станут говорить лжи, и не найдется в устах их языка коварного, ибо сами будут пастись и покоиться, и никто не потревожит их.
\vs Zep 3:14 Ликуй, дщерь Сиона! торжествуй, Израиль! веселись и радуйся от всего сердца, дщерь Иерусалима!
\vs Zep 3:15 Отменил Господь приговор над тобою, прогнал врага твоего! Господь, царь Израилев, посреди тебя: уже более не увидишь зла.
\vs Zep 3:16 В тот день скажут Иерусалиму: <<не бойся>>, и Сиону: <<да не ослабевают руки твои!>>
\vs Zep 3:17 Господь Бог твой среди тебя, Он силен спасти тебя; возвеселится о тебе радостью, будет милостив по любви Своей, будет торжествовать о тебе с ликованием.
\vs Zep 3:18 Сетующих о торжественных празднествах Я соберу: твои они, на них тяготеет поношение.
\vs Zep 3:19 Вот, Я стесню всех притеснителей твоих в то время и спасу хромлющее, и соберу рассеянное, и приведу их в почет и именитость на всей этой земле поношения их.
\vs Zep 3:20 В то время приведу вас и тогда же соберу вас, ибо сделаю вас именитыми и почетными между всеми народами земли, когда возвращу плен ваш перед глазами вашими, говорит Господь.

\bibbookdescr{Hag}{
  inline={\LARGE Книга\\\Huge Пророка Аггея},
  toc={Аггей},
  bookmark={Аггей},
  header={Аггей},
  %headerleft={},
  %headerright={},
  abbr={Агг}
}
\vs Hag 1:1 Во второй год царя Дария, в шестой месяц, в первый день месяца, было слово Господне через Аггея пророка к Зоровавелю, сыну Салафиилеву, правителю Иудеи, и к Иисусу, сыну Иоседекову, великому иерею:
\vs Hag 1:2 так сказал Господь Саваоф: народ сей говорит: <<не пришло еще время, не время строить дом Господень>>.
\rsbpar\vs Hag 1:3 И было слово Господне через Аггея пророка:
\vs Hag 1:4 а вам самим время жить в домах ваших украшенных, тогда как дом сей в запустении?
\vs Hag 1:5 Посему ныне так говорит Господь Саваоф: обратите сердце ваше на пути ваши.
\vs Hag 1:6 Вы сеете много, а собираете мало; едите, но не в сытость; пьете, но не напиваетесь; одеваетесь, а не согреваетесь; зарабатывающий плату зарабатывает для дырявого кошелька.
\vs Hag 1:7 Так говорит Господь Саваоф: обратите сердце ваше на пути ваши.
\vs Hag 1:8 Взойдите на гору и носите дерева, и стройте храм; и Я буду благоволить к нему, и прославлюсь, говорит Господь.
\vs Hag 1:9 Ожидаете многого, а выходит мало; и что принесете домой, то Я развею.~--- За что? говорит Господь Саваоф: за Мой дом, который в запустении, тогда как вы бежите, каждый к своему дому.
\vs Hag 1:10 Посему-то небо заключилось и не дает вам росы, и земля не дает своих произведений.
\vs Hag 1:11 И Я призвал засуху на землю, на горы, на хлеб, на виноградный сок, на елей и на все, что производит земля, и на человека, и на скот, и на всякий ручной труд.
\vs Hag 1:12 И послушались Зоровавель, сын Салафиилев, и Иисус, сын Иоседеков, и весь прочий народ гласа Господа Бога своего и слов Аггея пророка, как посланного Господом Богом их, и народ убоялся Господа.
\rsbpar\vs Hag 1:13 Тогда Аггей, вестник Господень, посланный от Господа, сказал к народу: Я с вами! говорит Господь.
\vs Hag 1:14 И возбудил Господь дух Зоровавеля, сына Салафиилева, правителя Иудеи, и дух Иисуса, сына Иоседекова, великого иерея, и дух всего остатка народа, и они пришли, и стали производить работы в доме Господа Саваофа, Бога своего,
\vs Hag 1:15 в двадцать четвертый день шестого месяца, во второй год царя Дария.
\vs Hag 2:1 В седьмой месяц, в двадцать первый день месяца, было слово Господне через Аггея пророка:
\vs Hag 2:2 скажи теперь Зоровавелю, сыну Салафиилеву, правителю Иудеи, и Иисусу, сыну Иоседекову, великому иерею, и остатку народа:
\vs Hag 2:3 кто остался между вами, который видел этот дом в прежней его славе, и каким видите вы его теперь? Не есть ли он в глазах ваших как бы ничто?
\vs Hag 2:4 Но ободрись ныне, Зоровавель, говорит Господь, ободрись, Иисус, сын Иоседеков, великий иерей! ободрись, весь народ земли, говорит Господь, и производите работы, ибо Я с вами, говорит Господь Саваоф.
\vs Hag 2:5 Завет Мой, который Я заключил с вами при исшествии вашем из Египта, и дух Мой пребывает среди вас: не бойтесь!
\vs Hag 2:6 Ибо так говорит Господь Саваоф: еще раз, и это будет скоро, Я потрясу небо и землю, море и сушу,
\vs Hag 2:7 и потрясу все народы, и придет Желаемый всеми народами, и наполню дом сей славою, говорит Господь Саваоф.
\vs Hag 2:8 Мое серебро и Мое золото, говорит Господь Саваоф.
\vs Hag 2:9 Слава сего последнего храма будет больше, нежели прежнего, говорит Господь Саваоф; и на месте сем Я дам мир, говорит Господь Саваоф.
\rsbpar\vs Hag 2:10 В двадцать четвертый день девятого месяца, во второй год Дария, было слово Господне через Аггея пророка:
\vs Hag 2:11 так говорит Господь Саваоф: спроси священников о законе и скажи:
\vs Hag 2:12 если бы кто нес освященное мясо в пол\acc{е} одежды своей и полою своею коснулся хлеба, или чего-либо вареного, или вина, или елея, или какой-нибудь пищи: сделается ли это священным? И отвечали священники и сказали: нет.
\vs Hag 2:13 Потом сказал Аггей: а если прикоснется ко всему этому кто-либо, осквернившийся от прикосновения к мертвецу: сделается ли это нечистым? И отвечали священники и сказали: будет нечистым.
\vs Hag 2:14 Тогда отвечал Аггей и сказал: таков этот народ, таково это племя предо Мною, говорит Господь, и таковы все дела рук их! И что они приносят там, все нечисто.
\vs Hag 2:15 Теперь обратите сердце ваше на время от сего дня и назад, когда еще не был положен камень на камень в храме Господнем.
\vs Hag 2:16 Приходили бывало к копне, могущей приносить двадцать мер, и оказывалось только десять; приходили к подточилию, чтобы начерпать пятьдесят мер из подточилия, а оказывалось только двадцать.
\vs Hag 2:17 Поражал Я вас ржавчиною и блеклостью хлеба и градом все труды рук ваших; но вы не обращались ко Мне, говорит Господь.
\vs Hag 2:18 Обратите же сердце ваше на время от сего дня и назад, от двадцать четвертого дня девятого месяца, от того дня, когда основан был храм Господень; обратите сердце ваше:
\vs Hag 2:19 есть ли еще в житницах семена? Доселе ни виноградная лоза, ни смоковница, ни гранатовое дерево, ни маслина не давали плода; а от сего дня Я благословлю их.
\rsbpar\vs Hag 2:20 И было слово Господне к Аггею вторично в двадцать четвертый день месяца, и сказано:
\vs Hag 2:21 скажи Зоровавелю, правителю Иудеи: потрясу Я небо и землю;
\vs Hag 2:22 и ниспровергну престолы царств, и истреблю силу царств языческих, опрокину колесницы и сидящих на них, и низринуты будут кони и всадники их, один мечом другого.
\vs Hag 2:23 В тот день, говорит Господь Саваоф, Я возьму тебя, Зоровавель, сын Салафиилев, раб Мой, говорит Господь, и буду держать тебя как печать, ибо Я избрал тебя, говорит Господь Саваоф.

\bibbookdescr{Zec}{
  inline={\LARGE Книга\\\Huge Пророка Захарии},
  toc={Захария},
  bookmark={Захария},
  header={Захария},
  %headerleft={},
  %headerright={},
  abbr={Зах}
}
\vs Zec 1:1 В восьмом месяце, во второй год Дария, было слово Господне к Захарии, сыну Варахиину, сыну Аддову, пророку:
\vs Zec 1:2 прогневался Господь на отцов ваших великим гневом,
\vs Zec 1:3 и ты скажи им: так говорит Господь Саваоф: обратитесь ко Мне, говорит Господь Саваоф, и Я обращусь к вам, говорит Господь Саваоф.
\vs Zec 1:4 Не будьте такими, как отцы ваши, к которым взывали прежде бывшие пророки, говоря: <<так говорит Господь Саваоф: обратитесь от злых путей ваших и от злых дел ваших>>; но они не слушались и не внимали Мне, говорит Господь.
\vs Zec 1:5 Отцы ваши~--- где они? да и пророки, будут ли они вечно жить?
\vs Zec 1:6 Но слова Мои и определения Мои, которые заповедал Я рабам Моим, пророкам, разве не постигли отцов ваших? и они обращались и говорили: <<как определил Господь Саваоф поступить с нами по нашим путям и по нашим делам, так и поступил с нами>>.
\rsbpar\vs Zec 1:7 В двадцать четвертый день одиннадцатого месяца,~--- это месяц Шеват,~--- во второй год Дария, было слово Господне к Захарии, сыну Варахиину, сыну Аддову, пророку:
\vs Zec 1:8 видел я ночью: вот, муж на рыжем коне стоит между миртами, которые в углублении, а позади него кони рыжие, пегие и белые,~---
\vs Zec 1:9 и сказал я: кто они, господин мой? И сказал мне Ангел, говоривший со мною: я покажу тебе, кто они.
\vs Zec 1:10 И отвечал муж, который стоял между миртами, и сказал: это те, которых Господь послал обойти землю.
\vs Zec 1:11 И они отвечали Ангелу Господню, стоявшему между миртами, и сказали: обошли мы землю, и вот, вся земля населена и спокойна.
\vs Zec 1:12 И отвечал Ангел Господень и сказал: Господи Вседержителю! Доколе Ты не умилосердишься над Иерусалимом и над городами Иуды, на которые Ты гневаешься вот уже семьдесят лет?
\vs Zec 1:13 Тогда в ответ Ангелу, говорившему со мною, изрек Господь слова благие, слова утешительные.
\vs Zec 1:14 И сказал мне Ангел, говоривший со мною: провозгласи и скажи: так говорит Господь Саваоф: возревновал Я о Иерусалиме и о Сионе ревностью великою;
\vs Zec 1:15 и великим негодованием негодую на народы, живущие в покое; ибо, когда Я мало прогневался, они усилили зло.
\vs Zec 1:16 Посему так говорит Господь: Я обращаюсь к Иерусалиму с милосердием; в нем соорудится дом Мой, говорит Господь Саваоф, и землемерная вервь протянется по Иерусалиму.
\vs Zec 1:17 Еще провозгласи и скажи: так говорит Господь Саваоф: снова переполнятся города Мои добром, и утешит Господь Сион, и снова изберет Иерусалим.
\vs Zec 1:18 И поднял я глаза мои и увидел: вот четыре рога.
\vs Zec 1:19 И сказал я Ангелу, говорившему со мною: что это? И он ответил мне: это роги, которые разбросали Иуду, Израиля и Иерусалим.
\vs Zec 1:20 Потом показал мне Господь четырех рабочих.
\vs Zec 1:21 И сказал я: что они идут делать? Он сказал мне так: эти роги разбросали Иуду, так что никто не может поднять головы своей; а сии пришли устрашить их, сбить роги народов, поднявших рог свой против земли Иуды, чтобы рассеять ее.
\vs Zec 2:1 И снова я поднял глаза мои и увидел: вот муж, у которого в руке землемерная вервь.
\vs Zec 2:2 Я спросил: куда ты идешь? и он сказал мне: измерять Иерусалим, чтобы видеть, какая широта его и какая длина его.
\vs Zec 2:3 И вот Ангел, говоривший со мною, выходит, а другой Ангел идет навстречу ему,
\vs Zec 2:4 и сказал он этому: иди скорее, скажи этому юноше: Иерусалим заселит окрестности по причине множества людей и скота в нем.
\vs Zec 2:5 И Я буду для него, говорит Господь, огненною стеною вокруг него и прославлюсь посреди него.
\vs Zec 2:6 Эй, эй! бегите из северной страны, говорит Господь: ибо по четырем ветрам небесным Я рассеял вас, говорит Господь.
\vs Zec 2:7 Спасайся, Сион, обитающий у дочери Вавилона.
\vs Zec 2:8 Ибо так говорит Господь Саваоф: для славы Он послал Меня к народам, грабившим вас, ибо касающийся вас касается зеницы ока Его.
\vs Zec 2:9 И вот, Я подниму руку Мою на них, и они сделаются добычею рабов своих, и тогда узнаете, что Господь Саваоф послал Меня.
\vs Zec 2:10 Ликуй и веселись, дщерь Сиона! Ибо вот, Я приду и поселюсь посреди тебя, говорит Господь.
\vs Zec 2:11 И прибегнут к Господу многие народы в тот день, и будут Моим народом; и Я поселюсь посреди тебя, и узнаешь, что Господь Саваоф послал Меня к тебе.
\vs Zec 2:12 Тогда Господь возьмет во владение Иуду, Свой удел на святой земле, и снова изберет Иерусалим.
\vs Zec 2:13 Да молчит всякая плоть пред лицем Господа! Ибо Он поднимается от святаго жилища Своего.
\vs Zec 3:1 И показал он мне Иисуса, великого иерея, стоящего перед Ангелом Господним, и сатану, стоящего по правую руку его, чтобы противодействовать ему.
\vs Zec 3:2 И сказал Господь сатане: Господь да запретит тебе, сатана, да запретит тебе Господь, избравший Иерусалим! не головня ли он, исторгнутая из огня?
\vs Zec 3:3 Иисус же одет был в запятнанные одежды и стоял перед Ангелом,
\vs Zec 3:4 который отвечал и сказал стоявшим перед ним так: снимите с него запятнанные одежды. А ему самому сказал: смотри, Я снял с тебя вину твою и облекаю тебя в одежды торжественные.
\vs Zec 3:5 И сказал: возложите на голову его чистый кидар. И возложили чистый кидар на голову его и облекли его в одежду; Ангел же Господень стоял.
\vs Zec 3:6 И засвидетельствовал Ангел Господень и сказал Иисусу:
\vs Zec 3:7 так говорит Господь Саваоф: если ты будешь ходить по Моим путям и если будешь на страже Моей, то будешь судить дом Мой и наблюдать за дворами Моими. Я дам тебе ходить между сими, стоящими здесь.
\vs Zec 3:8 Выслушай же, Иисус, иерей великий, ты и собратия твои, сидящие перед тобою, мужи знаменательные: вот, Я привожу раба Моего, ОТРАСЛЬ.
\vs Zec 3:9 Ибо вот тот камень, который Я полагаю перед Иисусом; на этом одном камне семь очей; вот, Я вырежу на нем начертания его, говорит Господь Саваоф, и изглажу грех земли сей в один день.
\vs Zec 3:10 В тот день, говорит Господь Саваоф, будете друг друга приглашать под виноград и под смоковницу.
\vs Zec 4:1 И возвратился тот Ангел, который говорил со мною, и пробудил меня, как пробуждают человека от сна его.
\vs Zec 4:2 И сказал он мне: что ты видишь? И отвечал я: вижу, вот светильник весь из золота, и чашечка для елея наверху его, и семь лампад на нем, и по семи трубочек у лампад, которые наверху его;
\vs Zec 4:3 и две маслины на нем, одна с правой стороны чашечки, другая с левой стороны ее.
\vs Zec 4:4 И отвечал я и сказал Ангелу, говорившему со мною: что это, господин мой?
\vs Zec 4:5 И Ангел, говоривший со мною, отвечал и сказал мне: ты не знаешь, что это? И сказал я: не знаю, господин мой.
\vs Zec 4:6 Тогда отвечал он и сказал мне так: это слово Господа к Зоровавелю, выражающее: не воинством и не силою, но Духом Моим, говорит Господь Саваоф.
\vs Zec 4:7 Кто ты, великая гора, перед Зоровавелем? ты~--- равнина, и вынесет он краеугольный камень при шумных восклицаниях: <<благодать, благодать на нем!>>
\vs Zec 4:8 И было ко мне слово Господне:
\vs Zec 4:9 руки Зоровавеля положили основание дому сему; его руки и окончат его, и узнаешь, что Господь Саваоф послал Меня к вам.
\vs Zec 4:10 Ибо кто может считать день сей маловажным, когда радостно смотрят на строительный отвес в руках Зоровавеля те семь,~--- это очи Господа, которые объемлют взором всю землю?
\vs Zec 4:11 Тогда отвечал я и сказал ему: что значат те две маслины с правой стороны светильника и с левой стороны его?
\vs Zec 4:12 Вторично стал я говорить и сказал ему: что значат две масличные ветви, которые через две золотые трубочки изливают из себя золото?
\vs Zec 4:13 И сказал он мне: ты не знаешь, что это? Я отвечал: не знаю, господин мой.
\vs Zec 4:14 И сказал он: это два помазанные елеем, предстоящие Господу всей земли.
\vs Zec 5:1 И опять поднял я глаза мои и увидел: вот летит свиток.
\vs Zec 5:2 И сказал он мне: что видишь ты? Я отвечал: вижу летящий свиток; длина его двадцать локтей, а ширина его десять локтей.
\vs Zec 5:3 Он сказал мне: это проклятие, исходящее на лице всей земли; ибо всякий, кто крадет, будет истреблен, как написано на одной стороне, и всякий, клянущийся ложно, истреблен будет, как написано на другой стороне.
\vs Zec 5:4 Я навел его, говорит Господь Саваоф, и оно войдет в дом татя и в дом клянущегося Моим именем ложно, и пребудет в доме его, и истребит его, и дерева его, и камни его.
\vs Zec 5:5 И вышел Ангел, говоривший со мною, и сказал мне: подними еще глаза твои и посмотри, что это выходит?
\vs Zec 5:6 Когда же я сказал: что это? Он отвечал: это выходит ефа, и сказал: это образ их по всей земле.
\vs Zec 5:7 И вот, кусок свинца поднялся, и там сидела одна женщина посреди ефы.
\vs Zec 5:8 И сказал он: эта \bibemph{женщина}~--- само нечестие, и бросил ее в средину ефы, а на отверстие ее бросил свинцовый кусок.
\vs Zec 5:9 И поднял я глаза мои и увидел: вот, появились две женщины, и ветер был в крыльях их, и крылья у них как крылья аиста; и подняли они ефу и понесли ее между землею и небом.
\vs Zec 5:10 И сказал я Ангелу, говорившему со мною: куда несут они эту ефу?
\vs Zec 5:11 Тогда сказал он мне: чтобы устроить для нее дом в земле Сеннаар, и когда будет все приготовлено, то она поставится там на своей основе.
\vs Zec 6:1 И опять поднял я глаза мои и вижу: вот, четыре колесницы выходят из ущелья между двумя горами; и горы те \bibemph{были} горы медные.
\vs Zec 6:2 В первой колеснице кони рыжие, а во второй колеснице кони вороные;
\vs Zec 6:3 в третьей колеснице кони белые, а в четвертой колеснице кони пегие, сильные.
\vs Zec 6:4 И, начав речь, я сказал Ангелу, говорившему со мною: что это, господин мой?
\vs Zec 6:5 И отвечал Ангел и сказал мне: это выходят четыре духа небесных, которые предстоят пред Господом всей земли.
\vs Zec 6:6 Вороные кони там выходят к стране северной и белые идут за ними, а пегие идут к стране полуденной.
\vs Zec 6:7 И сильные вышли и стремились идти, чтобы пройти землю; и он сказал: идите, пройдите землю,~--- и они прошли землю.
\vs Zec 6:8 Тогда позвал он меня и сказал мне так: смотри, вышедшие в землю северную успокоили дух Мой на земле северной.
\vs Zec 6:9 И было слово Господне ко мне:
\vs Zec 6:10 возьми у пришедших из плена, у Хелдая, у Товии и у Иедая, и пойди в тот самый день, пойди в дом Иосии, сына Софониева, куда они пришли из Вавилона,
\vs Zec 6:11 возьми \bibemph{у них} серебро и золото и сделай венцы, и возложи на голову Иисуса, сына Иоседекова, иерея великого,
\vs Zec 6:12 и скажи ему: так говорит Господь Саваоф: вот Муж,~--- имя Ему ОТРАСЛЬ, Он произрастет из Своего корня и создаст храм Господень.
\vs Zec 6:13 Он создаст храм Господень и примет славу, и воссядет, и будет владычествовать на престоле Своем; будет и священником на престоле Своем, и совет мира будет между тем и другим.
\vs Zec 6:14 А венцы те будут Хелему и Товии, Иедаю и Хену, сыну Софониеву, на память в храме Господнем.
\vs Zec 6:15 И издали придут, и примут участие в построении храма Господня, и вы узнаете, что Господь Саваоф послал меня к вам, и это будет, если вы усердно будете слушаться гласа Господа Бога вашего.
\vs Zec 7:1 В четвертый год царя Дария было слово Господне к Захарии, в четвертый день девятого месяца, Хаслева,
\vs Zec 7:2 когда Вефиль послал Сарецера и Регем-Мелеха и спутников его помолиться пред лицем Господа
\vs Zec 7:3 и спросить у священников, которые в доме Господа Саваофа, и у пророков, говоря: <<плакать ли мне в пятый месяц и поститься, как я делал это уже много лет?>>
\vs Zec 7:4 И было ко мне слово Господа Саваофа:
\vs Zec 7:5 скажи всему народу земли сей и священникам так: когда вы постились и плакали в пятом и седьмом месяце, притом уже семьдесят лет, для Меня ли вы постились? для Меня ли?
\vs Zec 7:6 И когда вы едите и когда пьете, не для себя ли вы едите, не для себя ли вы пьете?
\vs Zec 7:7 Не те же ли слова провозглашал Господь через прежних пророков, когда еще Иерусалим был населен и покоен, и города вокруг него, южная страна и низменность, были населены?
\vs Zec 7:8 И было слово Господне к Захарии:
\vs Zec 7:9 так говорил тогда Господь Саваоф: производ\acc{и}те суд справедливый и оказывайте милость и сострадание каждый брату своему;
\vs Zec 7:10 вдов\acc{ы} и сирот\acc{ы}, пришельца и бедного не притесняйте и зла друг против друга не мыслите в сердце вашем.
\vs Zec 7:11 Но они не хотели внимать, отворотились от Меня, и уши свои отяготили, чтобы не слышать.
\vs Zec 7:12 И сердце свое окаменили, чтобы не слышать закона и слов, которые посылал Господь Саваоф Духом Своим через прежних пророков; за то и постиг их великий гнев Господа Саваофа.
\vs Zec 7:13 И было: как Он взывал, а они не слушали, так и они взывали, а Я не слушал, говорит Господь Саваоф.
\vs Zec 7:14 И Я развеял их по всем народам, которых они не знали, и земля сия опустела после них, так что никто не ходил по ней ни взад, ни вперед, и они сделали вожделенную страну пустынею.
\vs Zec 8:1 И было слово Господа Саваофа:
\vs Zec 8:2 так говорит Господь Саваоф: возревновал Я о Сионе ревностью великою, и с великим гневом возревновал Я о нем.
\vs Zec 8:3 Так говорит Господь: обращусь Я к Сиону и буду жить в Иерусалиме, и будет называться Иерусалим городом истины, и гора Господа Саваофа~--- горою святыни.
\vs Zec 8:4 Так говорит Господь Саваоф: опять старцы и старицы будут сидеть на улицах в Иерусалиме, каждый с посохом в руке, от множества дней.
\vs Zec 8:5 И улицы города сего наполнятся отроками и отроковицами, играющими на улицах его.
\vs Zec 8:6 Так говорит Господь Саваоф: если это в глазах оставшегося народа покажется дивным во дни сии, то неужели оно дивно и в Моих очах? говорит Господь Саваоф.
\vs Zec 8:7 Так говорит Господь Саваоф: вот, Я спасу народ Мой из страны востока и из страны захождения солнца;
\vs Zec 8:8 и приведу их, и будут они жить в Иерусалиме, и будут Моим народом, и Я буду их Богом, в истине и правде.
\rsbpar\vs Zec 8:9 Так говорит Господь Саваоф: укрепите руки ваши вы, слышащие ныне слова сии из уст пророков, бывших при основании дома Господа Саваофа, для создания храма.
\vs Zec 8:10 Ибо прежде дней тех не было возмездия для человека, ни возмездия за труд животных; ни уходящему, ни приходящему не было покоя от врага; и попускал Я всякого человека враждовать против другого.
\vs Zec 8:11 А ныне для остатка этого народа Я не такой, как в прежние дни, говорит Господь Саваоф.
\vs Zec 8:12 Ибо посев будет в мире; виноградная лоза даст плод свой, и земля даст произведения свои, и небеса будут давать росу свою, и все это Я отдам во владение оставшемуся народу сему.
\vs Zec 8:13 И будет: как вы, дом Иудин и дом Израилев, были проклятием у народов, так Я спасу вас, и вы будете благословением; не бойтесь; да укрепятся руки ваши!
\vs Zec 8:14 Ибо так говорит Господь Саваоф; как Я определил наказать вас, когда отцы ваши прогневали Меня, говорит Господь Саваоф, и не отменил,
\vs Zec 8:15 так опять Я определил в эти дни соделать доброе Иерусалиму и дому Иудину; не бойтесь!
\vs Zec 8:16 Вот дела, которые вы должны делать: говорите истину друг другу; по истине и миролюбно судите у ворот ваших.
\vs Zec 8:17 Никто из вас да не мыслит в сердце своем зла против ближнего своего, и ложной клятвы не любите, ибо все это Я ненавижу, говорит Господь.
\rsbpar\vs Zec 8:18 И было ко мне слово Господа Саваофа:
\vs Zec 8:19 так говорит Господь Саваоф: пост четвертого месяца и пост пятого, и пост седьмого, и пост десятого соделается для дома Иудина радостью и веселым торжеством; только любите истину и мир.
\vs Zec 8:20 Так говорит Господь Саваоф: еще будут приходить народы и жители многих городов;
\vs Zec 8:21 и пойдут жители одного города к жителям другого и скажут: пойдем молиться лицу Господа и взыщем Господа Саваофа; \bibemph{и каждый скажет}: пойду и я.
\vs Zec 8:22 И будут приходить многие племена и сильные народы, чтобы взыскать Господа Саваофа в Иерусалиме и помолиться лицу Господа.
\vs Zec 8:23 Так говорит Господь Саваоф: будет в те дни, возьмутся десять человек из всех разноязычных народов, возьмутся за полу Иудея и будут говорить: мы пойдем с тобою, ибо мы слышали, что с вами Бог.
\vs Zec 9:1 Пророческое слово Господа на землю Хадрах, и на Дамаске оно остановится,~--- ибо око Господа на всех людей, как и на все колена Израилевы,~---
\vs Zec 9:2 и на Емаф, смежный с ним, на Тир и Сидон, ибо он очень умудрился.
\vs Zec 9:3 И устроил себе Тир крепость, накопил серебра, как пыли, и золота, как уличной грязи.
\vs Zec 9:4 Вот, Господь сделает его бедным и поразит силу его в море, и сам он будет истреблен огнем.
\vs Zec 9:5 Увидит это Аскалон и ужаснется, и Газа, и вострепещет сильно, и Екрон; ибо посрамится надежда его: не станет царя в Газе, и Аскалон будет необитаем.
\vs Zec 9:6 Чужое племя будет жить в Азоте, и Я уничтожу высокомерие Филистимлян.
\vs Zec 9:7 Исторгну кровь из уст его и мерзости его из зубов его, и он достанется Богу нашему, и будет как тысяченачальник в Иуде, и Екрон будет, как Иевусей.
\vs Zec 9:8 И Я расположу стан у дома Моего против войска, против проходящих вперед и назад, и не будет более проходить притеснитель, ибо ныне Моими очами Я буду взирать на это.
\rsbpar\vs Zec 9:9 Ликуй от радости, дщерь Сиона, торжествуй, дщерь Иерусалима: се Царь твой грядет к тебе, праведный и спасающий, кроткий, сидящий на ослице и на молодом осле, сыне подъяремной.
\vs Zec 9:10 Тогда истреблю колесницы у Ефрема и коней в Иерусалиме, и сокрушен будет бранный лук; и Он возвестит мир народам, и владычество Его будет от моря до моря и от реки до концов земли.
\vs Zec 9:11 А что до тебя, ради крови завета твоего Я освобожу узников твоих изо рва, в котором нет воды.
\vs Zec 9:12 Возвращайтесь на твердыню вы, пленники надеющиеся! Что теперь возвещаю, воздам тебе вдвойне.
\vs Zec 9:13 Ибо как лук Я натяну Себе Иуду и наполню лук Ефремом, и воздвигну сынов твоих, Сион, против сынов твоих, Иония, и сделаю тебя мечом ратоборца.
\vs Zec 9:14 И явится над ними Господь, и как молния вылетит стрела Его, и возгремит Господь Бог трубою, и шествовать будет в бурях полуденных.
\vs Zec 9:15 Господь Саваоф будет защищать их, и они будут истреблять и попирать пращные камни, и будут пить и шуметь как бы от вина, и наполнятся как жертвенные чаши, как углы жертвенника.
\vs Zec 9:16 И спасет их Господь Бог их в тот день, как овец, народ Свой; ибо, подобно камням в венце, они воссияют на земле Его.
\vs Zec 9:17 О, как велика благость его и какая красота его! Хлеб одушевит язык у юношей и вино~--- у отроковиц!
\vs Zec 10:1 Прос\acc{и}те у Господа дождя во время благопотребное; Господь блеснет молниею и даст вам обильный дождь, каждому злак на поле.
\vs Zec 10:2 Ибо терафимы говорят пустое, и вещуны видят ложное и рассказывают сны лживые; они утешают пустотою; поэтому они бродят как овцы, бедствуют, потому что нет пастыря.
\vs Zec 10:3 На пастырей воспылал гнев Мой, и козлов Я накажу; ибо посетит Господь Саваоф стадо Свое, дом Иудин, и поставит их, как славного коня Своего на брани.
\vs Zec 10:4 Из него будет краеугольный камень, из него~--- гвоздь, из него~--- лук для брани, из него произойдут все народоправители.
\vs Zec 10:5 И они будут, как герои, попирающие \bibemph{врагов} на войне, как уличную грязь, и сражаться, потому что Господь с ними, и посрамят всадников на конях.
\vs Zec 10:6 И укреплю дом Иудин, и спасу дом Иосифов, и возвращу их, потому что Я умилосердился над ними, и они будут, как бы Я не оставлял их: ибо Я Господь Бог их, и услышу их.
\vs Zec 10:7 Как герой будет Ефрем; возвеселится сердце их, как от вина; и увидят это сыны их и возрадуются; в восторге будет сердце их о Господе.
\vs Zec 10:8 Я дам им знак и соберу их, потому что Я искупил их; они будут так же многочисленны, как прежде;
\vs Zec 10:9 и расселю их между народами, и в отдаленных странах они будут воспоминать обо Мне и будут жить с детьми своими, и возвратятся;
\vs Zec 10:10 и возвращу их из земли Египетской, и из Ассирии соберу их, и приведу их в землю Галаадскую и на Ливан, и недостанет \bibemph{места} для них.
\vs Zec 10:11 И пройдет бедствие по морю, и поразит волны морские, и иссякнут все глубины реки, и смирится гордость Ассура, и скипетр отнимется у Египта.
\vs Zec 10:12 Укреплю их в Господе, и они будут ходить во имя Его, говорит Господь.
\vs Zec 11:1 Отворяй, Ливан, ворота твои, и да пожрет огонь кедры твои.
\vs Zec 11:2 Рыдай, кипарис, ибо упал кедр, ибо и величавые опустошены; рыдайте, дубы Васанские, ибо повалился непроходимый лес.
\vs Zec 11:3 Слышен голос рыдания пастухов, потому что опустошено приволье их; слышно рыкание молодых львов, потому что опустошена краса Иордана.
\vs Zec 11:4 Так говорит Господь Бог мой: паси овец, обреченных на заклание,
\vs Zec 11:5 которых купившие убивают ненаказанно, а продавшие говорят: <<благословен Господь; я разбогател!>> и пастухи их не жалеют о них.
\vs Zec 11:6 Ибо Я не буду более миловать жителей земли сей, говорит Господь; и вот, Я предам людей, каждого в руки ближнего его и в руки царя его, и они будут поражать землю, и Я не избавлю от рук их.
\vs Zec 11:7 И буду пасти овец, обреченных на заклание, овец поистине бедных. И возьму Себе два жезла, и назову один~--- благоволением, другой~--- узами, и ими буду пасти овец.
\vs Zec 11:8 И истреблю трех из пастырей в один месяц; и отвратится душа Моя от них, как и их душа отвращается от Меня.
\vs Zec 11:9 Тогда скажу: не буду пасти вас: умирающая~--- пусть умирает, и гибнущая~--- пусть гибнет, а остающиеся пусть едят плоть одна другой.
\vs Zec 11:10 И возьму жезл Мой~--- благоволения и переломлю его, чтобы уничтожить завет, который заключил Я со всеми народами.
\vs Zec 11:11 И он уничтожен будет в тот день, и тогда узнают бедные из овец, ожидающие Меня, что это слово Господа.
\vs Zec 11:12 И скажу им: если угодно вам, то дайте Мне плату Мою; если же нет,~--- не давайте; и они отвесят в уплату Мне тридцать сребреников.
\vs Zec 11:13 И сказал мне Господь: брось их в церковное хранилище,~--- высокая цена, в какую они оценили Меня! И взял Я тридцать сребреников и бросил их в дом Господень для горшечника.
\vs Zec 11:14 И переломил Я другой жезл Мой~--- <<узы>>, чтобы расторгнуть братство между Иудою и Израилем.
\vs Zec 11:15 И Господь сказал мне: еще возьми себе снаряд одного из глупых пастухов.
\vs Zec 11:16 Ибо вот, Я поставлю на этой земле пастуха, который о погибающих не позаботится, потерявшихся не будет искать и больных не будет лечить, здоровых не будет кормить, а мясо тучных будет есть и копыта их оторвет.
\vs Zec 11:17 Горе негодному пастуху, оставляющему стадо! меч на руку его и на правый глаз его! рука его совершенно иссохнет, и правый глаз его совершенно потускнет.
\vs Zec 12:1 Пророческое слово Господа об Израиле. Господь, распростерший небо, основавший землю и образовавший дух человека внутри него, говорит:
\vs Zec 12:2 вот, Я сделаю Иерусалим чашею исступления для всех окрестных народов, и также для Иуды во время осады Иерусалима.
\vs Zec 12:3 И будет в тот день, сделаю Иерусалим тяжелым камнем для всех племен; все, которые будут поднимать его, надорвут себя, а соберутся против него все народы земли.
\vs Zec 12:4 В тот день, говорит Господь, Я поражу всякого коня бешенством и всадника его безумием, а на дом Иудин отверзу очи Мои; всякого же коня у народов поражу слепотою.
\vs Zec 12:5 И скажут князья Иудины в сердцах своих: сила моя~--- жители Иерусалима в Господе Саваофе, Боге их.
\vs Zec 12:6 В тот день Я сделаю князей Иудиных, как жаровню с огнем между дровами и как горящий светильник среди снопов, и они истребят все окрестные народы, справа и слева, и снова населен будет Иерусалим на своем месте, в Иерусалиме.
\vs Zec 12:7 И спасет Господь сначала шатры Иуды, чтобы величие дома Давидова и величие жителей Иерусалима не возносилось над Иудою.
\vs Zec 12:8 В тот день защищать будет Господь жителей Иерусалима, и самый слабый между ними в тот день будет как Давид, а дом Давида будет как Бог, как Ангел Господень перед ними.
\vs Zec 12:9 И будет в тот день, Я истреблю все народы, нападающие на Иерусалим.
\vs Zec 12:10 А на дом Давида и на жителей Иерусалима изолью дух благодати и умиления, и они воззрят на Него, Которого пронзили, и будут рыдать о Нем, как рыдают об единородном сыне, и скорбеть, как скорбят о первенце.
\vs Zec 12:11 В тот день поднимется большой плач в Иерусалиме, как плач Гададриммона в долине Мегиддонской.
\vs Zec 12:12 И будет рыдать земля, каждое племя особо: племя дома Давидова особо, и жены их особо; племя дома Нафанова особо, и жены их особо;
\vs Zec 12:13 племя дома Левиина особо, и жены их особо; племя Симеоново особо, и жены их особо.
\vs Zec 12:14 Все остальные племена~--- каждое племя особо, и жены их особо.
\vs Zec 13:1 В тот день откроется источник дому Давидову и жителям Иерусалима для омытия греха и нечистоты.
\vs Zec 13:2 И будет в тот день, говорит Господь Саваоф, Я истреблю имена идолов с этой земли, и они не будут более упоминаемы, равно как лжепророков и нечистого духа удалю с земли.
\vs Zec 13:3 Тогда, если кто будет прорицать, то отец его и мать его, родившие его, скажут ему: тебе не должно жить, потому что ты ложь говоришь во имя Господа; и поразят его отец его и мать его, родившие его, когда он будет прорицать.
\vs Zec 13:4 И будет в тот день, устыдятся такие прорицатели, каждый видения своего, когда будут прорицать, и не будут надевать на себя власяницы, чтобы обманывать.
\vs Zec 13:5 И каждый скажет: я не пророк, я земледелец, потому что некто сделал меня рабом от детства моего.
\vs Zec 13:6 Ему скажут: отчего же на руках у тебя рубцы? И он ответит: оттого, что меня били в доме любящих меня.
\vs Zec 13:7 О, меч! поднимись на пастыря Моего и на ближнего Моего, говорит Господь Саваоф: порази пастыря, и рассеются овцы! И Я обращу руку Мою на малых.
\vs Zec 13:8 И будет на всей земле, говорит Господь, две части на ней будут истреблены, вымрут, а третья останется на ней.
\vs Zec 13:9 И введу эту третью часть в огонь, и расплавлю их, как плавят серебро, и очищу их, как очищают золото: они будут призывать имя Мое, и Я услышу их и скажу: <<это Мой народ>>, и они скажут: <<Господь~--- Бог мой!>>
\vs Zec 14:1 Вот наступает день Господень, и разделят награбленное у тебя среди тебя.
\vs Zec 14:2 И соберу все народы на войну против Иерусалима, и взят будет город, и разграблены будут домы, и обесчещены будут жены, и половина города пойдет в плен; но остальной народ не будет истреблен из города.
\vs Zec 14:3 Тогда выступит Господь и ополчится против этих народов, как ополчился в день брани.
\vs Zec 14:4 И станут ноги Его в тот день на горе Елеонской, которая перед лицем Иерусалима к востоку; и раздвоится гора Елеонская от востока к западу весьма большою долиною, и половина горы отойдет к северу, а половина ее~--- к югу.
\vs Zec 14:5 И вы побежите в долину гор Моих, ибо долина гор будет простираться до Асила; и вы побежите, как бежали от землетрясения во дни Озии, царя Иудейского; и придет Господь Бог мой и все святые с Ним.
\vs Zec 14:6 И будет в тот день: не станет света, светила удалятся.
\vs Zec 14:7 День этот будет единственный, ведомый только Господу: ни день, ни ночь; лишь в вечернее время явится свет.
\vs Zec 14:8 И будет в тот день, живые воды потекут из Иерусалима, половина их к морю восточному и половина их к морю западному: летом и зимой так будет.
\vs Zec 14:9 И Господь будет Царем над всею землею; в тот день будет Господь един, и имя Его едино.
\vs Zec 14:10 Вся эта земля будет, как равнина, от Гаваона до Реммона, на юг от Иерусалима, который высоко будет стоять на своем месте и населится от ворот Вениаминовых до места первых ворот, до угловых ворот, и от башни Анамеила до царских точил.
\vs Zec 14:11 И будут жить в нем, и проклятия не будет более, но будет стоять Иерусалим безопасно.
\vs Zec 14:12 И вот какое будет поражение, которым поразит Господь все народы, которые воевали против Иерусалима: у каждого исчахнет тело его, когда он еще стоит на своих ногах, и глаза у него истают в яминах своих, и язык его иссохнет во рту у него.
\vs Zec 14:13 И будет в тот день: произойдет между ними великое смятение от Господа, так что один схватит руку другого, и поднимется рука его на руку ближнего его.
\vs Zec 14:14 Но и сам Иуда будет воевать против Иерусалима, и собрано будет богатство всех окрестных народов: золото, серебро и одежды в великом множестве.
\vs Zec 14:15 Будет такое же поражение и коней, и лошаков, и верблюдов, и ослов, и всякого скота, какой будет в станах у них.
\vs Zec 14:16 Затем все остальные из всех народов, приходивших против Иерусалима, будут приходить из года в год для поклонения Царю, Господу Саваофу, и для празднования праздника кущей.
\vs Zec 14:17 И будет: если какое из племен земных не пойдет в Иерусалим для поклонения Царю, Господу Саваофу, то не будет дождя у них.
\vs Zec 14:18 И если племя Египетское не поднимется в путь и не придет [сюда], то и у него не будет \bibemph{дождя} и постигнет его поражение, каким поразит Господь народы, не приходящие праздновать праздника кущей.
\vs Zec 14:19 Вот что будет за грех Египта и за грех всех народов, которые не придут праздновать праздника кущей!
\vs Zec 14:20 В то время даже на конских уборах будет \bibemph{начертано}: <<Святыня Господу>>, и котлы в доме Господнем будут, как жертвенные чаши перед алтарем.
\vs Zec 14:21 И все котлы в Иерусалиме и Иудее будут святынею Господа Саваофа, и будут приходить все приносящие жертву и брать их и варить в них, и не будет более ни одного Хананея в доме Господа Саваофа в тот день.

\bibbookdescr{Mal}{
  inline={\LARGE Книга\\\Huge Пророка Малахии},
  toc={Малахия},
  bookmark={Малахия},
  header={Малахия},
  %headerleft={},
  %headerright={},
  abbr={Мал}
}
\vs Mal 1:1 Пророческое слово Господа к Израилю через Малахию.
\rsbpar\vs Mal 1:2 Я возлюбил вас, говорит Господь. А вы говорите: <<в чем явил Ты любовь к нам?>>~--- Не брат ли Исав Иакову? говорит Господь; и однако же Я возлюбил Иакова,
\vs Mal 1:3 а Исава возненавидел и предал горы его опустошению, и владения его~--- шакалам пустыни.
\vs Mal 1:4 Если Едом скажет: <<мы разорены, но мы восстановим разрушенное>>, то Господь Саваоф говорит: они построят, а Я разрушу, и прозовут их областью нечестивою, народом, на который Господь прогневался навсегда.
\vs Mal 1:5 И увидят это глаза ваши, и вы скажете: <<возвеличился Господь над пределами Израиля!>>
\vs Mal 1:6 Сын чтит отца и раб~--- господина своего; если Я Отец, то где почтение ко Мне? и если Я Господь, то где благоговение предо Мною? говорит Господь Саваоф вам, священники, бесславящие имя Мое. Вы говорите: <<чем мы бесславим имя Твое?>>
\vs Mal 1:7 Вы приносите на жертвенник Мой нечистый хлеб, а говорите: <<чем мы бесславим Тебя?>>~--- Тем, что говорите: <<трапеза Господня не стоит уважения>>.
\vs Mal 1:8 И когда приносите в жертву слепое, не худо ли это? или когда приносите хромое и больное, не худо ли это? Поднеси это твоему князю; будет ли он доволен тобою и благосклонно ли примет тебя? говорит Господь Саваоф.
\vs Mal 1:9 Итак молитесь Богу, чтобы помиловал нас; а когда такое исходит из рук ваших, то может ли Он милостиво принимать вас? говорит Господь Саваоф.
\vs Mal 1:10 Лучше кто-нибудь из вас запер бы двери, чтобы напрасно не держали огня на жертвеннике Моем. Нет Моего благоволения к вам, говорит Господь Саваоф, и приношение из рук ваших неблагоугодно Мне.
\vs Mal 1:11 Ибо от востока солнца до запада велико будет имя Мое между народами, и на всяком месте будут приносить фимиам имени Моему, чистую жертву; велико будет имя Мое между народами, говорит Господь Саваоф.
\vs Mal 1:12 А вы хулите его тем, что говорите: <<трапеза Господня не стоит уважения, и доход от нее~--- пища ничтожная>>.
\vs Mal 1:13 Притом говорите: <<вот сколько труда!>> и пренебрегаете ею, говорит Господь Саваоф, и приносите украденное, хромое и больное, и такого же свойства приносите хлебный дар: могу ли с благоволением принимать это из рук ваших? говорит Господь.
\vs Mal 1:14 Проклят лживый, у которого в стаде есть неиспорченный самец, и он дал обет, а приносит в жертву Господу поврежденное: ибо Я Царь великий, и имя Мое страшно у народов.
\vs Mal 2:1 Итак для вас, священники, эта заповедь:
\vs Mal 2:2 если вы не послушаетесь и если не примете к сердцу, чтобы воздавать славу имени Моему, говорит Господь Саваоф, то Я пошлю на вас проклятие и прокляну ваши благословения, и уже проклинаю, потому что вы не хотите приложить к тому сердца.
\vs Mal 2:3 Вот, Я отниму у вас плечо, и помет раскидаю на лица ваши, помет праздничных жертв ваших, и выбросят вас вместе с ним.
\vs Mal 2:4 И вы узнаете, что Я дал эту заповедь для сохранения завета Моего с Левием, говорит Господь Саваоф.
\vs Mal 2:5 Завет Мой с ним был \bibemph{завет} жизни и мира, и Я дал его ему для страха, и он боялся Меня и благоговел пред именем Моим.
\vs Mal 2:6 Закон истины был в устах его, и неправды не обреталось на языке его; в мире и правде он ходил со Мною и многих отвратил от греха.
\vs Mal 2:7 Ибо уста священника должны хранить ведение, и закона ищут от уст его, потому что он вестник Господа Саваофа.
\vs Mal 2:8 Но вы уклонились от пути сего, для многих послужили соблазном в законе, разрушили завет Левия, говорит Господь Саваоф.
\vs Mal 2:9 За то и Я сделаю вас презренными и униженными перед всем народом, так как вы не соблюдаете путей Моих, лицеприятствуете в делах закона.
\vs Mal 2:10 Не один ли у всех нас Отец? Не один ли Бог сотворил нас? Почему же мы вероломно поступаем друг против друга, нарушая тем завет отцов наших?
\vs Mal 2:11 Вероломно поступает Иуда, и мерзость совершается в Израиле и в Иерусалиме; ибо унизил Иуда святыню Господню, которую любил, и женился на дочери чужого бога.
\vs Mal 2:12 У того, кто делает это, истребит Господь из шатров Иаковлевых бдящего на страже и отвечающего, и приносящего жертву Господу Саваофу.
\vs Mal 2:13 И вот еще что вы делаете: вы заставляете обливать слезами жертвенник Господа с рыданием и воплем, так что Он уже не призирает более на приношение и не принимает умилостивительной жертвы из рук ваших.
\vs Mal 2:14 Вы скажете: <<за что?>> За то, что Господь был свидетелем между тобою и женою юности твоей, против которой ты поступил вероломно, между тем как она подруга твоя и законная жена твоя.
\vs Mal 2:15 Но не сделал ли того же один, и в нем пребывал превосходный дух? что же сделал этот один? он желал получить от Бога потомство. Итак берегите дух ваш, и никто не поступай вероломно против жены юности своей.
\vs Mal 2:16 Если ты ненавидишь ее, отпусти, говорит Господь Бог Израилев; обида покроет одежду его, говорит Господь Саваоф; посему наблюдайте за духом вашим и не поступайте вероломно.
\vs Mal 2:17 Вы прогневляете Господа словами вашими и говорите: <<чем прогневляем мы Его?>> Тем, что говорите: <<всякий, делающий зло, хорош пред очами Господа, и к таким Он благоволит>>, или: <<где Бог правосудия?>>
\vs Mal 3:1 Вот, Я посылаю Ангела Моего, и он приготовит путь предо Мною, и внезапно придет в храм Свой Господь, Которого вы ищете, и Ангел завета, Которого вы желаете; вот, Он идет, говорит Господь Саваоф.
\vs Mal 3:2 И кто выдержит день пришествия Его, и кто устоит, когда Он явится? Ибо Он~--- как огонь расплавляющий и как щелок очищающий,
\vs Mal 3:3 и сядет переплавлять и очищать серебро, и очистит сынов Левия и переплавит их, как золото и как серебро, чтобы приносили жертву Господу в правде.
\vs Mal 3:4 Тогда благоприятна будет Господу жертва Иуды и Иерусалима, как во дни древние и как в лета прежние.
\vs Mal 3:5 И приду к вам для суда и буду скорым обличителем чародеев и прелюбодеев и тех, которые клянутся ложно и удерживают плату у наемника, притесняют вдову и сироту, и отталкивают пришельца, и Меня не боятся, говорит Господь Саваоф.
\vs Mal 3:6 Ибо Я~--- Господь, Я не изменяюсь; посему вы, сыны Иакова, не уничтожились.
\vs Mal 3:7 Со дней отцов ваших вы отступили от уставов Моих и не соблюдаете их; обратитесь ко Мне, и Я обращусь к вам, говорит Господь Саваоф. Вы скажете: <<как нам обратиться?>>
\vs Mal 3:8 Можно ли человеку обкрадывать Бога? А вы обкрадываете Меня. Скажете: <<чем обкрадываем мы Тебя?>> Десятиною и приношениями.
\vs Mal 3:9 Проклятием вы прокляты, потому что вы~--- весь народ~--- обкрадываете Меня.
\vs Mal 3:10 Принесите все десятины в дом хранилища, чтобы в доме Моем была пища, и хотя в этом испытайте Меня, говорит Господь Саваоф: не открою ли Я для вас отверстий небесных и не изолью ли на вас благословения до избытка?
\vs Mal 3:11 Я для вас запрещу пожирающим истреблять у вас плоды земные, и виноградная лоза на поле у вас не лишится плодов своих, говорит Господь Саваоф.
\vs Mal 3:12 И блаженными называть будут вас все народы, потому что вы будете землею вожделенною, говорит Господь Саваоф.
\vs Mal 3:13 Дерзостны предо Мною слова ваши, говорит Господь. Вы скажете: <<что мы говорим против Тебя?>>
\vs Mal 3:14 Вы говорите: <<тщетно служение Богу, и что пользы, что мы соблюдали постановления Его и ходили в печальной одежде пред лицем Господа Саваофа?
\vs Mal 3:15 И ныне мы считаем надменных счастливыми: лучше устраивают себя делающие беззакония, и хотя искушают Бога, но остаются целы>>.
\vs Mal 3:16 Но боящиеся Бога говорят друг другу: <<внимает Господь и слышит это, и пред лицем Его пишется памятная книга о боящихся Господа и чтущих имя Его>>.
\vs Mal 3:17 И они будут Моими, говорит Господь Саваоф, собственностью Моею в тот день, который Я соделаю, и буду миловать их, как милует человек сына своего, служащего ему.
\vs Mal 3:18 И тогда снова увидите различие между праведником и нечестивым, между служащим Богу и не служащим Ему.
\vs Mal 4:1 Ибо вот, придет день, пылающий как печь; тогда все надменные и поступающие нечестиво будут как солома, и попалит их грядущий день, говорит Господь Саваоф, так что не оставит у них ни корня, ни ветвей.
\vs Mal 4:2 А для вас, благоговеющие пред именем Моим, взойдет Солнце правды и исцеление в лучах Его, и вы выйдете и взыграете, как тельцы упитанные;
\vs Mal 4:3 и будете попирать нечестивых, ибо они будут прахом под стопами ног ваших в тот день, который Я соделаю, говорит Господь Саваоф.
\vs Mal 4:4 Помните закон Моисея, раба Моего, который Я заповедал ему на Хориве для всего Израиля, равно как и правила и уставы.
\vs Mal 4:5 Вот, Я пошлю к вам Илию пророка пред наступлением дня Господня, великого и страшного.
\vs Mal 4:6 И он обратит сердца отцов к детям и сердца детей к отцам их, чтобы Я, придя, не поразил земли проклятием.

\bibbookdescr{1Ma}{
  inline={\LARGE Первая книга\\\Huge Маккавейская\fns{Книги Маккавейские переведены с греческого, потому что в еврейском тексте их нет.}},
  toc={1-я Маккавейская*},
  bookmark={1-я Маккавейская},
  header={1-я Маккавейская},
  %headerleft={},
  %headerright={},
  abbr={1~Мак}
}
\vs 1Ma 1:1 После того как Александр, сын Филиппа, Македонянин, который вышел из земли Киттим, поразил Дария, царя Персидского и Мидийского, и воцарился вместо него прежде над Елладою,~---
\vs 1Ma 1:2 он произвел много войн и овладел многими укрепленными местами, и убивал царей земли.
\vs 1Ma 1:3 И прошел до пределов земли и взял добычу от множества народов; и умолкла земля пред ним, и он возвысился, и вознеслось сердце его.
\vs 1Ma 1:4 Он собрал весьма сильное войско и господствовал над областями и народами и властителями, и они сделались его данниками.
\vs 1Ma 1:5 После того он слег в постель и, почувствовав, что умирает,
\vs 1Ma 1:6 призвал знатных из слуг своих, которые были воспитаны с ним от юности, и разделил им свое царство еще при жизни своей.
\rsbpar\vs 1Ma 1:7 Александр царствовал двенадцать лет и умер.
\vs 1Ma 1:8 И владычествовали слуги его каждый в своем месте.
\vs 1Ma 1:9 И по смерти его все они возложили на себя венцы, а после них и сыновья их в течение многих лет; и умножили зло на земле.
\vs 1Ma 1:10 И вышел от них корень греха~--- Антиох Епифан, сын царя Антиоха, который был заложником в Риме, и воцарился в сто тридцать седьмом году царства Еллинского.
\rsbpar\vs 1Ma 1:11 В те дни вышли из Израиля сыны беззаконные и убеждали многих, говоря: пойдем и заключим союз с народами, окружающими нас, ибо с тех пор, как мы отделились от них, постигли нас многие бедствия.
\vs 1Ma 1:12 И добрым показалось это слово в глазах их.
\vs 1Ma 1:13 Некоторые из народа изъявили желание и отправились к царю; и он дал им право исполнять установления языческие.
\vs 1Ma 1:14 Они построили в Иерусалиме училище по обычаю языческому
\vs 1Ma 1:15 и установили у себя необрезание, и отступили от святаго завета, и соединились с язычниками, и продались, чтобы делать зло.
\rsbpar\vs 1Ma 1:16 Когда Антиох увидел, что царство укрепилось, предпринял воцариться над Египтом, чтобы царствовать над двумя царствами,
\vs 1Ma 1:17 и вошел он в Египет с сильным ополчением, с колесницами, и слонами, и всадниками, и множеством кораблей;
\vs 1Ma 1:18 и вступил в сражение с Птоломеем, царем Египетским; и убоялся Птоломей от лица его и обратился в бегство, и много пало раненых.
\vs 1Ma 1:19 И овладели они укрепленными городами в земле Египетской, и взял он добычу из земли Египетской.
\rsbpar\vs 1Ma 1:20 После поражения Египта Антиох возвратился в сто сорок третьем году и пошел против Израиля, и вступил в Иерусалим с сильным ополчением;
\vs 1Ma 1:21 вошел во святилище с надменностью и взял золотой жертвенник, светильник и все сосуды его,
\vs 1Ma 1:22 и трапезу предложения, и возлияльники, и чаши, и кадильницы золотые, и завесу, и венцы, и золотое украшение, бывшее снаружи храма, и всё обобрал.
\vs 1Ma 1:23 Взял и серебро, и золото, и драгоценные сосуды, и взял скрытые сокровища, какие отыскал.
\vs 1Ma 1:24 И, взяв всё, отправился в землю свою и совершил убийства, и говорил с великою надменностью.
\vs 1Ma 1:25 Посему был великий плач в Израиле, во всех местах его.
\vs 1Ma 1:26 Стенали начальники и старейшины, изнемогали девы и юноши, и изменилась красота женская.
\vs 1Ma 1:27 Всякий жених предавался плачу, и сидящая в брачном чертоге была в скорби.
\vs 1Ma 1:28 Вострепетала земля за обитающих на ней, и весь дом Иакова облекся стыдом.
\rsbpar\vs 1Ma 1:29 По прошествии двух лет послал царь начальника податей в города Иуды, и он пришел в Иерусалим с большою толпою;
\vs 1Ma 1:30 коварно говорил им слова мира, и они поверили ему; но он внезапно напал на город и поразил его великим поражением, и погубил множество народа Израильского;
\vs 1Ma 1:31 взял добычи из города и сожег его огнем, и разрушил домы его и стены его кругом;
\vs 1Ma 1:32 и увели в плен жен и детей, и овладели скотом.
\vs 1Ma 1:33 Оградили город Давидов большою и крепкою стеною и крепкими башнями, и сделался он для них крепостью.
\vs 1Ma 1:34 И поместили там народ нечестивый, людей беззаконных, и они укрепились в ней;
\vs 1Ma 1:35 запаслись оружием и продовольствием и, собрав добычи Иерусалимские, сложили там, и сделались большою сетью.
\vs 1Ma 1:36 И было это постоянною засадою для святилища и злым диаволом для Израиля.
\vs 1Ma 1:37 Они проливали невинную кровь вокруг святилища и оскверняли святилище.
\vs 1Ma 1:38 Жители же Иерусалима разбежались ради них, и он сделался жилищем чужих и стал чужим для своего рода, и дети его оставили его.
\vs 1Ma 1:39 Святилище его запустело, как пустыня, праздники его обратились в плач, субботы его~--- в поношение, честь его~--- в уничижение.
\vs 1Ma 1:40 По мере славы его увеличилось бесчестие его, и высота его обратилась в печаль.
\rsbpar\vs 1Ma 1:41 Царь Антиох написал всему царству своему, чтобы все были одним народом
\vs 1Ma 1:42 и чтобы каждый оставил свой закон. И согласились все народы по слову царя.
\vs 1Ma 1:43 И многие из Израиля приняли идолослужение его и принесли жертвы идолам, и осквернили субботу.
\vs 1Ma 1:44 Царь послал через вестников грамоты в Иерусалим и в города Иудейские, чтобы они следовали узаконениям, чужим для сей земли,
\vs 1Ma 1:45 и чтобы не допускались всесожжения и жертвоприношения, и возлияние в святилище, чтобы ругались над субботами и праздниками
\vs 1Ma 1:46 и оскверняли святилище и святых,
\vs 1Ma 1:47 чтобы строили жертвенники, храмы и капища идольские, и приносили в жертву свиные мяса и скотов нечистых,
\vs 1Ma 1:48 и оставляли сыновей своих необрезанными, и оскверняли души их всякою нечистотою и мерзостью,
\vs 1Ma 1:49 для того, чтобы забыли закон и изменили все постановления.
\vs 1Ma 1:50 А если кто не сделает по слову царя, да будет предан смерти.
\vs 1Ma 1:51 Согласно этому писал он всему царству своему и поставил надзирателей над всем народом, и повелел городам Иудейским приносить жертвы во всяком городе.
\vs 1Ma 1:52 И собрались к ним многие из народа, все, которые оставили закон,~--- и совершили зло в земле;
\vs 1Ma 1:53 и заставили Израиля укрываться во всяком убежище его.
\rsbpar\vs 1Ma 1:54 В пятнадцатый день Хаслева, сто сорок пятого года, устроили на жертвеннике мерзость запустения, и в городах Иудейских вокруг построили жертвенники,
\vs 1Ma 1:55 и перед дверями домов и на улицах совершали курения,
\vs 1Ma 1:56 и книги закона, какие находили, разрывали и сожигали огнем;
\vs 1Ma 1:57 у кого находили книгу завета и кто держался закона, того, по повелению царя, предавали смерти.
\vs 1Ma 1:58 С таким насилием поступали они с Израильтянами, приходившими каждый месяц в города.
\vs 1Ma 1:59 И в двадцать пятый день месяца, принося жертвы на жертвеннике, который был над алтарем,
\vs 1Ma 1:60 они, по данному повелению, убивали жен, обрезавших детей своих,
\vs 1Ma 1:61 а младенцев вешали за шеи их, домы их расхищали и совершавших над ними обрезание убивали.
\vs 1Ma 1:62 Но многие в Израиле остались твердыми и укрепились, чтобы не есть нечистого,
\vs 1Ma 1:63 и предпочли умереть, чтобы не оскверниться пищею и не поругать святаго завета,~--- и умирали.
\vs 1Ma 1:64 И был весьма великий гнев над Израилем.
\vs 1Ma 2:1 В те дни восстал Маттафия, сын Иоанна, сына Симеонова, священник из сынов Иоарива из Иерусалима; жил он в Модине.
\vs 1Ma 2:2 У него было пять сыновей: Иоанн, прозываемый Гаддис,
\vs 1Ma 2:3 Симон, называемый Фасси,
\vs 1Ma 2:4 Иуда, прозываемый Маккавей,
\vs 1Ma 2:5 Елеазар, прозываемый Аваран, Ионафан, прозываемый Апфус.
\vs 1Ma 2:6 Видя богохульства, происходившие в Иудее и Иерусалиме,
\vs 1Ma 2:7 он сказал: горе мне! для чего родился я видеть разорение народа моего и разорение святаго города и оставаться здесь, когда он предан в руки врагов и святилище~--- в руки чужих?
\vs 1Ma 2:8 Храм его сделался, как муж бесславный,
\vs 1Ma 2:9 драгоценные сосуды его унесены в плен, младенцы его избиты на улицах, юноши его пали от меча врага.
\vs 1Ma 2:10 Какой народ не занимал царства его и не овладевал добычами его?
\vs 1Ma 2:11 Все украшение его отнято; из свободного он сделался рабом.
\vs 1Ma 2:12 И вот святыни наши, и благолепие наше, и слава наша опустели, и язычники осквернили их.
\vs 1Ma 2:13 Для чего нам еще жить?
\vs 1Ma 2:14 И разодрал Маттафия и сыновья его одежды свои, и облеклись во вретища, и горько плакали.
\vs 1Ma 2:15 И пришли от царя в город Модин принуждавшие к отступничеству, чтобы приносить жертвы.
\vs 1Ma 2:16 И многие из Израиля пристали к ним; а Маттафия и сыновья его устояли.
\vs 1Ma 2:17 И отвечали пришедшие от царя и сказали Маттафии: ты вождь, ты славен и велик в этом городе и имеешь опору в сыновьях и братьях.
\vs 1Ma 2:18 Итак, приступи теперь первый и исполни повеление царя, как сделали это все народы и мужи Иудейские и оставшиеся в Иерусалиме, и будешь ты и дом твой в числе друзей царских, и ты и сыновья твои будете почтены и серебром, и золотом, и многими дарами.
\vs 1Ma 2:19 И отвечал Маттафия и сказал громким голосом: если и все народы в области царства царя послушают его и отступят каждый от богослужения отцов своих, и согласятся на повеления его,
\vs 1Ma 2:20 то я и сыновья мои и братья мои будем поступать по завету отцов наших.
\vs 1Ma 2:21 Помилуй нас Бог, чтобы оставить закон и постановления!
\vs 1Ma 2:22 Не послушаем мы слов царя, чтобы отступить нам от нашего богослужения вправо или влево.
\vs 1Ma 2:23 Когда перестал он говорить эти слова, подошел муж Иудеянин пред глазами всех, чтобы принести по повелению царя идольскую жертву на жертвеннике, который был в Модине.
\vs 1Ma 2:24 Увидев это, Маттафия возревновал, и затрепетала внутренность его, и воспламенилась ярость его по законе, и он, подбежав, убил его при жертвеннике.
\vs 1Ma 2:25 И в то же время убил мужа царского, принуждавшего приносить жертву, и разрушил жертвенник.
\vs 1Ma 2:26 И возревновал он по законе, как это сделал Финеес с Замврием, сыном Салома.
\vs 1Ma 2:27 И воскликнул Маттафия в городе громким голосом: всякий, кто ревнует по законе и стоит в завете, да идет вслед за мною!
\vs 1Ma 2:28 И убежал сам и сыновья его в горы, оставив всё, что имели в городе.
\vs 1Ma 2:29 Тогда многие, преданные правде и закону, ушли в пустыню и оставались там,
\vs 1Ma 2:30 сами и сыновья их, и жены их, и скоты их, потому что умножились беды над ними.
\vs 1Ma 2:31 И возвещено было мужам царским и войску, находившемуся в Иерусалиме, городе Давидовом, что некоторые мужи, нарушив царское повеление, ушли в сокровенные места в пустыне.
\vs 1Ma 2:32 И погнались за ними многие и, настигнув их, ополчились, и выстроились к сражению против них в день субботний,
\vs 1Ma 2:33 и сказали им: теперь еще можно; выходите и сделайте по слову царя, и останетесь живы.
\vs 1Ma 2:34 Но они отвечали: не выйдем и не сделаем по слову царя, не оскверним дня субботнего.
\vs 1Ma 2:35 Тогда поспешили начать сражение против них.
\vs 1Ma 2:36 Но они не отвечали им, ни даже камня не бросили на них, ни заградили тайных убежищ своих,
\vs 1Ma 2:37 и сказали: мы все умрем в невинности нашей; небо и земля свидетели за нас, что вы несправедливо губите нас.
\vs 1Ma 2:38 Нападали на них по субботам, и умерло их, и жен их, и детей их со скотом их, до тысячи душ.
\vs 1Ma 2:39 Когда узнал о том Маттафия и друзья его, горько плакали о них;
\vs 1Ma 2:40 и говорили друг другу: если все мы будем поступать так, как поступали эти братья наши, и не будем сражаться с язычниками за жизнь нашу и постановления наши, то они скоро истребят нас с земли.
\vs 1Ma 2:41 И решили они в тот день и сказали: кто бы ни пошел на войну против нас в день субботний, будем сражаться против него, дабы нам не умереть всем, как умерли братья наши в тайных убежищах.
\vs 1Ma 2:42 Тогда собрались к ним множество Иудеев, крепкие силою из Израиля, все верные закону.
\vs 1Ma 2:43 И все, бежавшие от бедствия, присоединились к ним и сделались подкреплением для них.
\vs 1Ma 2:44 Так составили они войско и поражали в гневе своем нечестивых и в ярости своей мужей беззаконных; остальные же бежали для спасения к язычникам.
\vs 1Ma 2:45 И обходил вокруг Маттафия и друзья его, и разрушали жертвенники,
\vs 1Ma 2:46 и небоязненно обрезывали необрезанных детей, сколько находили в пределах Израильских,
\vs 1Ma 2:47 и преследовали сынов гордыни, и дело успешно шло в руках их.
\vs 1Ma 2:48 Так защищали они закон от руки язычников и от руки царей и не дали восторжествовать грешнику.
\rsbpar\vs 1Ma 2:49 Приблизились дни смерти Маттафии, и он сказал сыновьям своим: ныне усилилась гордость и испытание, ныне время переворота и гнев ярости.
\vs 1Ma 2:50 Итак, дети, возревнуйте о законе и отдайте жизнь вашу за завет отцов наших.
\vs 1Ma 2:51 Вспомните о делах отцов наших, которые они совершили во времена свои, и вы приобретете великую славу и вечное имя.
\vs 1Ma 2:52 Авраам не в искушении ли найден был верным? и это вменилось ему в праведность.
\vs 1Ma 2:53 Иосиф в стесненном положении своем сохранил заповедь и сделался господином Египта.
\vs 1Ma 2:54 Финеес, отец наш, за то, что возревновал ревностью, получил завет вечного священства.
\vs 1Ma 2:55 Иисус за исполнение слова сделался судьею над Израилем.
\vs 1Ma 2:56 Халев за свидетельство перед собранием получил в наследие землю.
\vs 1Ma 2:57 Давид за свое милосердие наследовал престол царства навеки.
\vs 1Ma 2:58 Илия за великую ревность по законе взят даже на небо.
\vs 1Ma 2:59 Анания, Азария, Мисаил верою спаслись от пламени.
\vs 1Ma 2:60 Даниил за свою невинность избавлен от челюстей львов.
\vs 1Ma 2:61 Итак, припоминайте от рода до рода, что все, надеющиеся на Него, не изнемогут.
\vs 1Ma 2:62 Не убойтесь слов мужа грешного, ибо слава его обратится в навоз и в червей.
\vs 1Ma 2:63 Сегодня он превозносится, а завтра не найдут его, ибо он обратился в прах свой, и замысел его погиб.
\vs 1Ma 2:64 Но вы, дети мои, крепитесь и мужественно стойте в законе, ибо чрез него вы прославитесь.
\vs 1Ma 2:65 Вот~--- Симон, брат ваш: знаю, что он~--- муж совета, слушайтесь его во все дни; он будет вам вместо отца.
\vs 1Ma 2:66 А Иуда Маккавей, крепкий силою от юности своей, да будет у вас начальником войска, и будет вести войну с народами.
\vs 1Ma 2:67 Итак, соберите к себе всех исполнителей закона и отмщайте за обиды народа вашего;
\vs 1Ma 2:68 воздайте воздаяние язычникам и будьте внимательны к повелениям закона.
\vs 1Ma 2:69 И благословил их и приложился к отцам своим.
\vs 1Ma 2:70 Умер же он на сто сорок шестом году; и сыновья его похоронили его в гробе отцов своих в Модине, и весь Израиль оплакивал его горьким плачем.
\vs 1Ma 3:1 И восстал вместо него Иуда, называемый Маккавей, сын его.
\vs 1Ma 3:2 И помогали ему все братья его и все, которые были привержены к отцу его, и вели войну Израиля с радостью.
\vs 1Ma 3:3 Он распространил славу народа своего; он облекался бронею, как исполин, опоясывался воинскими доспехами своими и вел войну, защищая ополчение мечом;
\vs 1Ma 3:4 он уподоблялся льву в делах своих и был как скимен, рыкающий на добычу;
\vs 1Ma 3:5 он преследовал беззаконных, отыскивая их, и возмущающих народ его сожигал.
\vs 1Ma 3:6 И смирились беззаконные из страха пред ним, и все делатели беззакония смутились пред ним, и благоуспешно было спасение рукою его.
\vs 1Ma 3:7 Он огорчил многих царей и возвеселил Иакова делами своими, и память его до века в благословении;
\vs 1Ma 3:8 прошел по городам Иудеи и истребил в ней нечестивых, и отвратил гнев от Израиля,
\vs 1Ma 3:9 и сделался именитым до последних пределов земли, и собрал погибавших.
\rsbpar\vs 1Ma 3:10 Тогда Аполлоний собрал язычников и из Самарии многочисленное войско, чтобы воевать против Израиля.
\vs 1Ma 3:11 Иуда узнал о том и вышел к нему навстречу, и поразил, и убил его; и много пало пораженных, а остальные убежали.
\vs 1Ma 3:12 И взял Иуда добычу их, и взял меч Аполлония, и сражался им во все дни.
\rsbpar\vs 1Ma 3:13 И услышал Сирон, военачальник Сирии, что Иуда собрал вокруг себя людей и сонм верных, выступающих с ним на войну,
\vs 1Ma 3:14 и сказал: сделаю себе имя и прославлюсь в царстве, и сражусь с Иудою и с теми, которые вместе с ним и которые презирают слово царское.
\vs 1Ma 3:15 И решился он идти, и пошло с ним сильное полчище нечестивых помогать ему и сделать отмщение на сынах Израиля.
\vs 1Ma 3:16 Когда они приблизились к возвышенности Вефорона, Иуда вышел к ним навстречу с очень немногими,
\vs 1Ma 3:17 которые, когда увидели идущее навстречу им войско, сказали Иуде: как можем мы в таком малом числе сражаться против такого сильного множества? И мы же совсем ослабели, еще не евши ныне.
\vs 1Ma 3:18 Но Иуда сказал им: легко и многим попасть в руки немногих, и у Бога небесного нет различия, многими ли спасти, или немногими;
\vs 1Ma 3:19 ибо не от множества войска бывает победа на войне, но с неба приходит сила.
\vs 1Ma 3:20 Они идут против нас во множестве надменности и нечестия, чтобы истребить нас и жен наших и детей наших, чтобы ограбить нас;
\vs 1Ma 3:21 а мы сражаемся за души наши и законы наши.
\vs 1Ma 3:22 Он Сам сокрушит их пред лицем нашим; вы же не страшитесь их.
\vs 1Ma 3:23 Перестав говорить, он внезапно бросился на них, и поражен был Сирон и войско его перед ним.
\vs 1Ma 3:24 И они преследовали его по спуску Вефорона до самой равнины; и пало из них до восьмисот мужей, прочие же убежали в землю Филистимскую.
\vs 1Ma 3:25 И начал страх перед Иудою и братьями его и боязнь нападать на всех окрестных язычников.
\vs 1Ma 3:26 Дошло и до царя имя его, и все народы рассказывали о битвах Иуды.
\rsbpar\vs 1Ma 3:27 Когда же услышал эти речи царь Антиох, то воспылал гневом и, послав, собрал все силы царства своего, весьма сильное ополчение;
\vs 1Ma 3:28 и открыл казнохранилище свое, и выдал войскам своим годовое жалованье, и приказал им быть готовыми на всякую надобность.
\vs 1Ma 3:29 Но увидел, что истощилось серебро в казнохранилищах, а подати страны скудны по причине волнения и разорения, которое он произвел в земле той, уничтожая законы, существовавшие от дней древних.
\vs 1Ma 3:30 И начал он опасаться, что у него недостанет, разве только на раз или два, на издержки и подарки, которые прежде раздавал щедрою рукою и превзошел в том прежних царей.
\vs 1Ma 3:31 Сильно озабоченный в душе своей, он решился идти в Персию и взять подати со стран и собрать побольше серебра.
\vs 1Ma 3:32 А дела царские от реки Евфрата до пределов Египта предоставил Лисию, человеку знаменитому, происходившему от рода царского,
\vs 1Ma 3:33 также и воспитание сына своего, Антиоха, до его возвращения;
\vs 1Ma 3:34 и передал ему половину войск и слонов, дав ему приказания о всем, чего хотел, и о жителях Иудеи и Иерусалима,
\vs 1Ma 3:35 чтобы он послал против них войско сокрушить и уничтожить могущество Израиля и остаток Иерусалима, и истребить память их от места того,
\vs 1Ma 3:36 и поселить во всех пределах их сынов иноплеменных, и разделить по жребию землю их.
\vs 1Ma 3:37 Царь же взял остальную половину войска и отправился из Антиохии, престольного города своего, в сто сорок седьмом году и, перейдя реку Евфрат, прошел верхние страны.
\vs 1Ma 3:38 Лисий избрал Птоломея, сына Дорименова, и Никанора и Горгия, мужей сильных из друзей царя,
\vs 1Ma 3:39 и послал с ними сорок тысяч мужей и семь тысяч всадников, чтобы идти в землю Иудейскую и разорить ее по слову царя.
\vs 1Ma 3:40 Они отправились со всем войском своим и, придя, расположились на равнине близ Еммаума.
\vs 1Ma 3:41 Купцы этой страны услышали имя их и, взяв весьма много серебра и золота и слуг, пришли в стан покупать сынов Израиля в рабы; к ним присоединилось и войско Сирии и земл\acc{и} иноплеменных.
\rsbpar\vs 1Ma 3:42 Увидел Иуда и братья его, что умножились бедствия и войска расположились станом в пределах их; узнали и о повелении царя, которое он приказал исполнить над народом к погублению и истреблению его.
\vs 1Ma 3:43 И говорили каждый ближнему своему: восставим низверженный народ наш и сразимся за народ наш и за святыню.
\vs 1Ma 3:44 И собрался сонм, чтобы быть готовыми к войне и помолиться, и испросить милости и сожаления.
\vs 1Ma 3:45 Иерусалим был необитаем, как пустыня; не было ни входящего в него, ни выходящего из него из природных жителей его; святилище было попрано, и сыновья инородных были в крепости его; он стал жилищем язычников; и отнято веселье у Иакова, и не слышно стало свирели и цитры.
\vs 1Ma 3:46 Итак, они собрались и пошли в Массифу, напротив Иерусалима, ибо место молитвы у Израильтян было прежде в Массифе.
\vs 1Ma 3:47 И постились в этот день, и возложили на себя вретища и пепел на головы свои, и разодрали одежды свои,
\vs 1Ma 3:48 раскрыли книгу закона из тех, которые язычники отыскивали, чтобы сделать на них изображения своих идолов,
\vs 1Ma 3:49 и принесли священнические облачения и первородных и десятины; и созвали назореев, исполнивших дни свои,
\vs 1Ma 3:50 и громко возопили к небу: что нам делать с ними и куда отвести их?
\vs 1Ma 3:51 Святилище Твое попрано и осквернено, и священники Твои в скорби и уничижении.
\vs 1Ma 3:52 И вот, собрались против нас язычники, чтобы истребить нас. Ты знаешь, что умышляют они против нас.
\vs 1Ma 3:53 Как можем мы устоять пред лицем их, если Ты не поможешь нам?
\vs 1Ma 3:54 И вострубили трубами и воскликнули громким голосом.
\rsbpar\vs 1Ma 3:55 После сего Иуда поставил вождей для народа~--- тысяченачальников, стоначальников, пятидесятиначальников и десятиначальников.
\vs 1Ma 3:56 И сказали тем, которые строили дома, обручились с женами, насадили виноградники, и людям боязливым, чтобы каждый из них, по закону, возвратился в свой дом.
\vs 1Ma 3:57 Тогда двинулось ополчение и расположилось станом на юге от Еммаума.
\vs 1Ma 3:58 И сказал Иуда: опояшьтесь и будьте мужественны и готовы к утру сразиться с этими язычниками, которые собрались против нас, чтобы погубить нас и святыню нашу.
\vs 1Ma 3:59 Ибо лучше нам умереть в сражении, нежели видеть бедствия нашего народа и святыни.
\vs 1Ma 3:60 А какая будет воля на небе, так да сотворит!
\vs 1Ma 4:1 И взял Горгий пять тысяч мужей и тысячу отборных всадников, и двинулось ополчение ночью,
\vs 1Ma 4:2 чтобы напасть на ополчение Иудеев и поразить их внезапно, а жившие в крепости служили ему проводниками.
\vs 1Ma 4:3 И услышал Иуда и выступил сам и храбрые мужи, чтобы поразить войско царя в Еммауме,
\vs 1Ma 4:4 доколе силы неприятельские были еще в отдаленности от стана.
\vs 1Ma 4:5 И пришел Горгий в стан Иуды ночью, и никого не нашел, и искал их по горам, ибо говорил: они бегут от нас.
\vs 1Ma 4:6 Но с рассветом дня Иуда явился на равнине с тремя тысячами мужей, но они не имели ни щитов, ни мечей, как того желали.
\vs 1Ma 4:7 Когда увидели они крепкое и вооруженное ополчение язычников и окружающую его конницу, обученных для войны,
\vs 1Ma 4:8 Иуда сказал бывшим с ним мужам: не бойтесь множества их и не страшитесь нападения их.
\vs 1Ma 4:9 Вспомните, как спасены были отцы наши в Чермном море, когда фараон преследовал их с войском.
\vs 1Ma 4:10 И ныне возопием на небо; может быть, Он умилосердится над нами, воспомянув завет с отцами нашими, и сокрушит ныне это ополчение перед лицем нашим;
\vs 1Ma 4:11 и все язычники познают, что есть Избавляющий и Спасающий Израиля.
\vs 1Ma 4:12 Иноплеменники, подняв глаза свои, увидели, что идут против них,
\vs 1Ma 4:13 и вышли из стана на сражение, а бывшие с Иудою затрубили,
\vs 1Ma 4:14 и сошлись, и разбиты были язычники, и побежали на равнину,
\vs 1Ma 4:15 а все остальные пали от меча; и преследовали их до Газера и до равнин Идумеи, Азота и Иамнии, и пали из них до трех тысяч мужей.
\vs 1Ma 4:16 И возвратился Иуда и войско его от преследования их
\vs 1Ma 4:17 и сказал народу: не бросайтесь на добычу, ибо война еще предстоит нам;
\vs 1Ma 4:18 Горгий и войско его на горе близ нас; станьте теперь против врагов наших и сражайтесь с ними, а после смело возьмете добычу.
\vs 1Ma 4:19 Когда еще говорил это Иуда, показалась некоторая толпа, выступавшая с горы.
\vs 1Ma 4:20 И увидел он, что их обратили в бегство и жгут лагерь; ибо поднимающийся дым показывал, что произошло.
\vs 1Ma 4:21 Когда они увидели это, очень испугались; увидев же и войско Иуды на равнине, готовое к сражению,
\vs 1Ma 4:22 все побежали в землю иноплеменников.
\vs 1Ma 4:23 Тогда Иуда обратился на добычу стана, и захватили много золота и серебра, гиацинтовых и багряных одежд и великое богатство.
\vs 1Ma 4:24 И, возвращаясь, воспевали и благословляли Господа небесного, потому что Он благ и что вовек милость Его.
\vs 1Ma 4:25 И было в тот день великое спасение Израилю.
\vs 1Ma 4:26 Уцелевшие же из иноплеменников пришли к Лисию и возвестили о всем случившемся.
\vs 1Ma 4:27 Он, услышав, уныл и опечалился, что не то случилось с Израилем, чего он хотел, и не то вышло, что повелел ему царь.
\vs 1Ma 4:28 И на следующий год Лисий собрал шестьдесят тысяч избранных мужей и пять тысяч всадников, чтобы победить их.
\vs 1Ma 4:29 И пришли они в Идумею и расположились станом в Вефсурах; а Иуда встретил их с десятью тысячами мужей.
\vs 1Ma 4:30 Увидев сильное ополчение, он молился и говорил: благословен Ты, Спаситель Израиля, сокрушивший нападение сильного рукою раба Твоего Давида и предавший полк иноплеменников в руки Ионафана, сына Саулова, и оруженосца его.
\vs 1Ma 4:31 Предай войско сие в руки народа Твоего~--- Израиля, и да будут они постыжены в силе и коннице их;
\vs 1Ma 4:32 наведи на них страх и сокруши дерзость силы их; да будут они потрясены поражением своим;
\vs 1Ma 4:33 низложи их мечом любящих Тебя, и да прославят Тебя в песнях все знающие имя Твое.
\vs 1Ma 4:34 И сразились они, и пало из войска Лисия до пяти тысяч мужей, пали перед ними.
\vs 1Ma 4:35 Лисий, увидев бегство войска своего и храбрость воинов Иуды и что они готовы или жить, или умереть отважно, отправился в Антиохию, набрал чужеземцев и, увеличив бывшее войско, думал снова идти в Иудею.
\rsbpar\vs 1Ma 4:36 Иуда же и братья его сказали: вот, враги наши сокрушены, взойдем очистить и обновить святилище.
\vs 1Ma 4:37 И собралось все ополчение, и взошли на гору Сион.
\vs 1Ma 4:38 И увидели, что святилище опустошено, жертвенник осквернен, ворота сожжены, и в притворах, как в лесу или на какой-либо горе, поросл\acc{и} растения, и хранилища разрушены,
\vs 1Ma 4:39 и разодрали они одежды свои, плакали горьким плачем и сыпали пепел на свои головы,
\vs 1Ma 4:40 и падали лицом на землю и трубили вестовыми трубами, и вопили к небу.
\vs 1Ma 4:41 Тогда отрядил Иуда мужей воевать против находившихся в крепости, доколе он очистит святилище.
\vs 1Ma 4:42 И избрал священников беспорочных, ревнителей закона.
\vs 1Ma 4:43 Они очистили святилище и оскверненные камни вынесли в нечистое место.
\vs 1Ma 4:44 Потом они рассуждали об оскверненном жертвеннике всесожжения, как поступить с ним.
\vs 1Ma 4:45 И пришла им добрая мысль разрушить его, чтобы он когда-нибудь не послужил им в поношение, так как язычники осквернили его; и разрушили они жертвенник,
\vs 1Ma 4:46 и камни сложили на горе храма в приличном месте, пока придет пророк и даст ответ о них.
\vs 1Ma 4:47 Взяли камни целые, по закону, и построили новый жертвенник по-прежнему;
\vs 1Ma 4:48 потом устроили святыни и внутренние части храма и освятили притворы;
\vs 1Ma 4:49 устроили новую священную утварь и внесли в храм свещник и алтарь всесожжений и фимиамов и трапезу;
\vs 1Ma 4:50 и воскурили на алтаре фимиам и зажгли светильники на свещнике, и осветили храм;
\vs 1Ma 4:51 и положили на трапезу хлебы, и развесили завесы, и окончили все дела, которые предприняли.
\rsbpar\vs 1Ma 4:52 В двадцать пятый день девятого месяца~--- это месяц Хаслев~--- сто сорок восьмого года встали весьма рано
\vs 1Ma 4:53 и принесли жертву по закону на новоустроенном жертвеннике всесожжений.
\vs 1Ma 4:54 В то время, в тот самый день, в который язычники осквернили жертвенник, обновлен он с песнями, с цитрами, гуслями и кимвалами.
\vs 1Ma 4:55 И весь народ падал на лицо свое, и молились и воссылали благодарение на небо Благопоспешившему им.
\vs 1Ma 4:56 Так совершали обновление жертвенника восемь дней с весельем, принося всесожжения и вознося жертву спасения и хвалы.
\vs 1Ma 4:57 И украсили переднюю сторону храма золотыми венцами и щитами и возобновили ворота и хранилища, и сделали для них двери.
\vs 1Ma 4:58 И была весьма великая радость в народе, и отвращено было поношение язычников.
\vs 1Ma 4:59 И установил Иуда и братья его и все собрание Израиля, чтобы дни обновления жертвенника празднуемы были с веселием и радостью в свое время, каждый год восемь дней, от двадцатого дня месяца Хаслева.
\vs 1Ma 4:60 В то же время обстроили гору Сион вокруг высокими стенами и крепкими башнями, чтобы язычники, придя когда-нибудь, не попрали их, как сделали это прежде.
\vs 1Ma 4:61 И расположил там Иуда войско стеречь гору, и укрепили для охранения ее Вефсуру, чтобы народ имел крепость против Идумеи.
\vs 1Ma 5:1 Когда окрестные народы услышали, что построен жертвенник и возобновлено святилище, как прежде, сильно вознегодовали;
\vs 1Ma 5:2 и решились истребить род Иакова, живший среди них, и начали убивать и истреблять людей в этом народе.
\vs 1Ma 5:3 Тогда Иуда ополчился против сынов Исава в Идумее, в Акравиме, так как они держали в осаде Израиля, и поразил их великим поражением, и смирил их, и взял добычи их.
\vs 1Ma 5:4 Вспомнил он и о злобе сынов Веана, которые были для народа сетью и претыканием, строя ему засады на дорогах.
\vs 1Ma 5:5 Хотя они заперлись от него в башнях, но он ополчился против них, предал их заклятию и сожег огнем башни их со всеми, бывшими в них.
\vs 1Ma 5:6 Потом он перешел к сынам Аммона и встретил сильное войско и многочисленный народ и Тимофея, предводителя их.
\vs 1Ma 5:7 Он имел с ними много сражений, и они были разбиты пред лицем его; он поразил их;
\vs 1Ma 5:8 взял Иазер и селения его и возвратился в Иудею.
\vs 1Ma 5:9 Тогда собрались язычники, жившие в Галааде, против Израильтян, находившихся в пределах их, чтобы истребить их; но они бежали в крепость Дафему.
\vs 1Ma 5:10 И послали письма к Иуде и братьям его и сказали: собрались против нас окружающие нас язычники, чтобы истребить нас,
\vs 1Ma 5:11 и готовятся идти и сделать нападение на крепость, в которую мы убежали, и Тимофей предводительствует войском их.
\vs 1Ma 5:12 Итак, приди и избавь нас от руки их, ибо множество из нас погибло;
\vs 1Ma 5:13 и все братья наши, бывшие в пределах Това, преданы смерти, а жен их и детей их и имущество взяли в плен, и погубили там около тысячи мужей.
\vs 1Ma 5:14 Еще читались эти письма, как вот, пришли другие вестники из Галилеи в разодранных одеждах с таким извещением:
\vs 1Ma 5:15 собрались против нас из Птолемаиды и из Тира и Сидона, и из всей Галилеи языческой, чтобы погубить нас.
\vs 1Ma 5:16 Когда услышал эти слова Иуда и народ, то собралось великое собрание для совещания, что сделать для сих братьев, находящихся в бедствии и угрожаемых войною от тех язычников?
\vs 1Ma 5:17 Тогда Иуда сказал Симону, брату своему: выбери себе мужей и иди и защити братьев твоих, находящихся в Галилее; а я и Ионафан, брат мой, пойдем в Галаад.
\vs 1Ma 5:18 И оставил он Иосифа, сына Захарии, и Азарию начальниками над народом с остатком войска в Иудее на охранение.
\vs 1Ma 5:19 И дал им повеление, сказав: управляйте народом сим, но не начинайте войны против язычников до нашего возвращения.
\vs 1Ma 5:20 Симону отделены для похода в Галилею три тысячи мужей, Иуде же~--- в Галаад восемь тысяч мужей.
\vs 1Ma 5:21 И отправился Симон в Галилею и произвел много сражений с язычниками, и разбиты им язычники.
\vs 1Ma 5:22 Он преследовал их до ворот Птолемаиды, и пало из язычников до трех тысяч мужей, и он взял добычи их.
\vs 1Ma 5:23 Также взял он с собою находившихся в Галилее и Арваттах \bibemph{Иудеев} с женами и детьми и со всем имением их и привел в Иудею с великою радостью.
\vs 1Ma 5:24 А Иуда Маккавей и Ионафан, брат его, перешли Иордан и совершили трехдневный путь в пустыне.
\vs 1Ma 5:25 Их встретили Навуфеи и приняли мирно, и рассказали им все, случившееся с братьями их в Галааде,
\vs 1Ma 5:26 и что многие из них заперты в Васаре и Восоре, в Алемах, Хасфоре, Македе и Карнаине~--- все сии города укреплены и велики~---
\vs 1Ma 5:27 и в прочих городах Галаада находятся в осаде, и что завтра назначено напасть на эти укрепления и взять их и погубить всех их в один день.
\vs 1Ma 5:28 Посему Иуда со своим войском вдруг направил путь свой в пустыню к Восору и взял этот город, и избил весь мужеский пол острием меча, и взял все добычи их, и сожег его огнем;
\vs 1Ma 5:29 а оттуда отправился ночью и шел до укрепления.
\vs 1Ma 5:30 Когда наступало утро, и подняли глаза, и вот, народ многочисленный, которому числа не было, поднимают лестницы и машины, чтобы взять укрепление, и осаждают бывших в нем.
\vs 1Ma 5:31 Увидел Иуда, что началась битва и вопль города восходил на небо трубами и громким криком,
\vs 1Ma 5:32 и сказал воинам: сражайтесь теперь за братьев ваших.
\vs 1Ma 5:33 Он обошел врагов с тыла с тремя отрядами, и затрубили трубами и воскликнули с молитвою;
\vs 1Ma 5:34 и узнало войско Тимофея, что это~--- Маккавей, и побежали от лица его, и он поразил их великим поражением, и пало из них в этот день до восьми тысяч мужей.
\vs 1Ma 5:35 Тогда поворотил он в Масфу и осадил и взял ее, избил весь мужеский пол в ней, взял добычи ее и сожег ее огнем;
\vs 1Ma 5:36 отправившись оттуда, он взял Хасфон, Макед, Восор и прочие города Галаадские.
\rsbpar\vs 1Ma 5:37 После этих событий Тимофей собрал другое войско и расположился станом перед Рафоном по ту сторону потока.
\vs 1Ma 5:38 И послал Иуда осмотреть войско, и объявили ему и сказали: собрались к ним все окружающие нас язычники~--- сила весьма многочисленная,
\vs 1Ma 5:39 и они наняли в помощь себе Аравитян и расположились станом за потоком, будучи готовы идти против тебя войною. И пошел Иуда навстречу им.
\vs 1Ma 5:40 Тогда Тимофей сказал своим военачальникам, когда Иуда и войско его приближались к потоку воды: если он перейдет к нам прежде, то мы не в силах будем устоять против него, ибо он превозможет нас.
\vs 1Ma 5:41 Если же он убоится и расположится станом по ту сторону потока, то мы перейдем к нему и превозможем его.
\vs 1Ma 5:42 Как только подошел Иуда к потоку воды, то поставил при потоке народных писцов и приказал им, сказав: не оставляйте ни одного человека в стане, но пусть все идут на сражение.
\vs 1Ma 5:43 И переправился к ним первый и весь народ за ним. И сокрушены были пред лицем его все язычники, и бросили оружие свое, и убежали в капище, которое было в Карнаине.
\vs 1Ma 5:44 Тогда взяли они этот город и сожгли огнем капище со всеми находившимися в нем; и побежден был Карнаин и не мог более противостоять Иуде.
\vs 1Ma 5:45 И собрал Иуда всех Израильтян, находившихся в Галааде, от малого до большого, и жен их, и детей их, и имение, очень большое ополчение, чтобы идти в землю Иудейскую.
\vs 1Ma 5:46 И дошли они до Ефрона. Это был большой город, весьма укрепленный, на пути; невозможно было уклониться от него ни вправо, ни влево; надобно было пройти посреди него,
\vs 1Ma 5:47 а жители заперлись в нем и ворота завалили камнями.
\vs 1Ma 5:48 Иуда послал к ним с мирным предложением: мы пройдем по земле вашей, чтобы идти нам в землю нашу, и никто не обидит вас, только ногами нашими пройдем. Но они не захотели отворить ему.
\vs 1Ma 5:49 Тогда Иуда приказал объявить в ополчении, чтобы каждый ополчился на своем месте;
\vs 1Ma 5:50 и ополчились воины и осаждали город весь тот день и всю ночь, и сдался город в руки его.
\vs 1Ma 5:51 И побил он весь мужеский пол острием меча и до основания разрушил город, и взял добычи его, и прошел через город по убитым.
\vs 1Ma 5:52 И переправились через Иордан на великую равнину против Вефсана.
\vs 1Ma 5:53 И собирал Иуда отставших и ободрял народ в продолжение всего пути, доколе не пришли в землю Иудейскую.
\vs 1Ma 5:54 И взошли на гору Сион с весельем и радостью и принесли всесожжения, потому что никто не пал из них до самого возвращения в мире.
\rsbpar\vs 1Ma 5:55 В те дни, когда Иуда и Ионафан находились в Галааде, а Симон, брат его,~--- в Галилее перед Птолемаидою,
\vs 1Ma 5:56 услышали Иосиф, сын Захарии, и Азарий, военачальники, о славных воинских подвигах, совершенных ими,
\vs 1Ma 5:57 и сказали: сделаем и мы себе имя; пойдем воевать с язычниками, окружающими нас.
\vs 1Ma 5:58 Так объявили они бывшему при них войску и пошли на Иамнию.
\vs 1Ma 5:59 И вышел Горгий из города и воины его навстречу им на сражение.
\vs 1Ma 5:60 И, обратившись в бегство, Иосиф и Азария были преследуемы до пределов Иудеи; и пали в этот день из народа Израильского до двух тысяч мужей.
\vs 1Ma 5:61 И было великое замешательство в народе Израильском, потому что не послушались Иуды и братьев его, мечтая показать храбрость,
\vs 1Ma 5:62 тогда как они не были от семени тех мужей, руке которых предоставлено спасение Израиля.
\vs 1Ma 5:63 Но муж Иуда и братья его весьма прославились перед всем Израилем и перед всеми народами, где только слышно было имя их,~---
\vs 1Ma 5:64 и собирались к ним приветствующие.
\rsbpar\vs 1Ma 5:65 После того вышел Иуда и братья его и воевали против сынов Исава в земле, лежащей к югу, и поразил Хеврон и селения его, и разрушил укрепление его, и сожег башни его вокруг него,
\vs 1Ma 5:66 и поднялся, чтобы идти в землю иноплеменников, и прошел Самарию.
\vs 1Ma 5:67 В то время пали в сражении священники, желавшие прославиться храбростью и безрассудно вышедшие на войну.
\vs 1Ma 5:68 И обратился Иуда в Азот, землю иноплеменников, разрушил жертвенники их, сожег огнем резные изображения богов их, взял добычи городов и возвратился в землю Иудейскую.
\vs 1Ma 6:1 Между тем царь Антиох, проходя верхние области, услышал, что есть в Персии город Елимаис, славящийся богатством, серебром и золотом,
\vs 1Ma 6:2 и в нем~--- храм, весьма богатый, и есть там золотые покровы, брони и оружия, которые оставил там Александр, сын Филиппа, царь Македонский,~--- первый, воцарившийся над Еллинами.
\vs 1Ma 6:3 И он пришел и старался взять этот город и ограбить его, но не мог, потому что намерение его стало известно жителям города.
\vs 1Ma 6:4 Они поднялись против него войною, и он обратился в бегство и ушел оттуда с великою скорбью, чтобы отправиться в Вавилон.
\vs 1Ma 6:5 Тогда пришел некто к нему в Персию с известием, что ополчения, ходившие в землю Иуды, обращены в бегство,
\vs 1Ma 6:6 что Лисий ходил с сильным войском впереди всех, но был поражен \bibemph{Иудеями}, и они усилились и оружием, и войском, и многими добычами, которые взяли от пораженных ими войск,
\vs 1Ma 6:7 и что они разрушили мерзость, которую он воздвиг над жертвенником в Иерусалиме, а святилище по-прежнему обнесли высокими стенами, также и Вефсуру, город его.
\vs 1Ma 6:8 Когда царь услышал слова сии, сильно испугался и встревожился, упал на постель и впал в изнеможение от печали, что не сбылось так, как он желал.
\vs 1Ma 6:9 И много дней пробыл он там, ибо возобновлялась в нем сильная печаль; он думал, что умирает.
\vs 1Ma 6:10 И созвал он всех друзей своих и сказал им: удалился сон от глаз моих, и я изнемог сердцем от печали.
\vs 1Ma 6:11 И сказал я в сердце моем: до какой скорби дошел я и до какого великого смущения, в котором нахожусь теперь! А был я полезен и любим во владычестве моем.
\vs 1Ma 6:12 Теперь же я воспоминаю о тех злодеяниях, которые я совершил в Иерусалиме, и как взял все находившиеся в нем золотые и серебряные сосуды и посылал истреблять обитающих в Иудее напрасно.
\vs 1Ma 6:13 Теперь я позна\acc{ю}, что за это постигли меня эти беды,~--- и вот, я погибаю от великой печали в чужой земле.
\vs 1Ma 6:14 И призвал он Филиппа, одного из друзей своих, и поставил его правителем над всем царством своим;
\vs 1Ma 6:15 и дал ему венец и царскую одежду свою и перстень, чтобы он руководил Антиоха, сына его, и воспитывал его для царствования.
\vs 1Ma 6:16 И умер царь Антиох в сто сорок девятом году.
\rsbpar\vs 1Ma 6:17 Когда Лисий узнал, что царь умер, то поставил вместо него на царство сына его, Антиоха, которого воспитывал в юности его, и назвал его именем Евпатора.
\vs 1Ma 6:18 Между тем находившиеся в крепости теснили Израиля вокруг святилища и всегда старались делать ему зло, а язычникам служить опорою;
\vs 1Ma 6:19 тогда Иуда решил выгнать их и созвал весь народ, чтобы осадить их.
\vs 1Ma 6:20 Все собрались и осадили их в сто пятидесятом году, и устроил он против них стрелометательные орудия и машины.
\vs 1Ma 6:21 Но некоторые из осажденных вышли, и к ним пристали некоторые из нечестивых Израильтян;
\vs 1Ma 6:22 и пошли они к царю и сказали: доколе ты не сделаешь суда и не отмстишь за братьев наших?
\vs 1Ma 6:23 Мы согласились служить отцу твоему и ходить по заповедям его и следовать повелениям его;
\vs 1Ma 6:24 а сыны народа нашего осадили крепость и за то чуждаются нас, и кого из нас находят, умерщвляют, и имущества наши расхищают,
\vs 1Ma 6:25 и не на нас только простерли они руку, но и на все пределы наши.
\vs 1Ma 6:26 И вот, теперь осадили они крепость в Иерусалиме, чтобы овладеть ею, а святилище и Вефсуру укрепили.
\vs 1Ma 6:27 Если ты не поспешишь предупредить их, то они сделают больше этого, и тогда ты не в силах будешь удержать их.
\vs 1Ma 6:28 Услышав это, царь разгневался и собрал всех друзей своих и начальников войска своего и начальников конницы;
\vs 1Ma 6:29 пришли к нему и из других царств и с морских островов войска наемные,
\vs 1Ma 6:30 так что число войск его было: сто тысяч пеших, двадцать тысяч всадников и тридцать два слона, приученных к войне.
\vs 1Ma 6:31 И прошли они через Идумею и расположились станом против Вефсуры, и сражались много дней и устроили машины; но те сделали вылазку и сожгли их огнем и сразились мужественно.
\vs 1Ma 6:32 После сего Иуда отступил от крепости и расположился станом в Вефсахаре против стана царского.
\vs 1Ma 6:33 Царь же, встав рано утром, поспешно отправился с войском своим по дороге к Вефсахаре, и приготовились войска к сражению и затрубили трубами.
\vs 1Ma 6:34 Слонам показывали кровь винограда и тутовых ягод, чтобы возбудить их к битве,
\vs 1Ma 6:35 и разделили этих животных на отряды и приставили к каждому слону по тысяче мужей в железных кольчугах и с медными шлемами на головах, сверх того по пятисот отборных всадников назначено было к каждому слону.
\vs 1Ma 6:36 Они становились заблаговременно там, где был слон, и куда он шел, шли и они вместе, не отставая от него.
\vs 1Ma 6:37 Притом на них были крепкие деревянные башни, покрывавшие каждого слона, укрепленные на них помочами, и в каждой из них по тридцати по два сильных мужей, которые сражались на них, и при слоне Индиец его.
\vs 1Ma 6:38 Остальных же всадников расставили здесь и там~--- на двух сторонах ополчения, чтобы подавать знаки и подкреплять в тесных местах.
\vs 1Ma 6:39 Когда солнце блеснуло на золотых и медных щитах, то заблистали от них горы и светились, как огненные светильники.
\vs 1Ma 6:40 Одна часть царского войска протянута была по высоким горам, а другие~--- по низменным местам; и шли они твердо и стройно.
\vs 1Ma 6:41 И смутились все, слышавшие шум множества их и шествие такого полчища и стук оружий, ибо войско было весьма великое и сильное.
\vs 1Ma 6:42 И вступил Иуда и войско его в сражение~--- и пали из ополчения царского шестьсот мужей.
\vs 1Ma 6:43 Тогда Елеазар, сын Саварана, увидел, что один из слонов покрыт бронею царскою и превосходил всех, и казалось, что на нем был царь,~---
\vs 1Ma 6:44 и он предал себя, чтобы спасти народ свой и приобрести себе вечное имя;
\vs 1Ma 6:45 и смело побежал к нему в средину отряда, поражая направо и налево, и расступались от него и в ту, и в другую сторону;
\vs 1Ma 6:46 и подбежал он под того слона, лег под него и убил его, и пал на него слон на землю, и он умер там.
\vs 1Ma 6:47 Но, увидев силу царского ополчения и стремительность войск, Иудеи уклонились от них.
\vs 1Ma 6:48 Царские же войска пошли против них на Иерусалим: царь направил войска на Иудею и на гору Сион.
\vs 1Ma 6:49 И заключил он мир с бывшими в Вефсуре, которые вышли из города, ибо не было у них продовольствия, чтобы держаться в нем в осаде, потому что был субботний год на земле.
\vs 1Ma 6:50 И овладел царь Вефсурою и оставил в ней стражу, чтобы стеречь ее.
\vs 1Ma 6:51 Потом много дней осаждал святилище и поставил там стрелометательные орудия и машины, и огнеметательные, и камнеметательные, и копьеметательные, чтобы бросать стрелы и камни.
\vs 1Ma 6:52 Но и Иудеи устроили машины против их машин и сражались много дней;
\vs 1Ma 6:53 съестных же припасов недостало в хранилищах, потому что был седьмой год, и искавшие в Иудее безопасности от язычников издержали остатки запасов;
\vs 1Ma 6:54 и осталось при святилище немного мужей, ибо одолел их голод, и разошлись каждый в свое место.
\rsbpar\vs 1Ma 6:55 Услышал Лисий, что Филипп, которому царь Антиох еще при жизни поручил воспитывать сына своего, Антиоха, для царствования,
\vs 1Ma 6:56 возвратился из Персии и Мидии и с ним ходившие с царем войска, и что он домогается принять на себя дела царства.
\vs 1Ma 6:57 Почему поспешно пошел и сказал царю, начальникам войска и вельможам: мы каждый день терпим недостаток и продовольствия у нас мало, а место, осаждаемое нами, крепко, между тем лежит на нас попечение о царстве.
\vs 1Ma 6:58 Итак, подадим правую руку этим людям и заключим с ними мир и со всем народом их,
\vs 1Ma 6:59 и предоставим им поступать по законам их, как прежде; ибо за свои законы, которые мы отменили, они раздражились и сделали всё это.
\vs 1Ma 6:60 И угодно было это слово царю и начальникам,~--- и послал он к ним, чтобы заключить мир, что они и приняли;
\vs 1Ma 6:61 и клялся им царь и начальники. После сего они вышли из крепости.
\vs 1Ma 6:62 И взошел царь на гору Сион и, осмотрев укрепленные места, пренебрег клятвою, которою клялся, и велел разорить стены кругом.
\vs 1Ma 6:63 Потом поспешно отправился, и, возвратившись в Антиохию, он нашел, что Филипп владеет городом, вступил с ним в сражение и силою взял город.
\vs 1Ma 7:1 В сто пятьдесят первом году вышел из Рима Димитрий, сын Селевка, и с немногими людьми вошел в один приморский город и там воцарился.
\vs 1Ma 7:2 Когда же он входил в царственный дом отцов своих, войско схватило Антиоха и Лисия, чтобы привести их к нему.
\vs 1Ma 7:3 Это стало известно ему, и он сказал: не показывайте мне лиц их.
\vs 1Ma 7:4 Тогда воины убили их, и воссел Димитрий на престоле царства своего.
\vs 1Ma 7:5 И пришли к нему все мужи беззаконные и нечестивые из Израильтян, и Алким предводительствовал ими, домогаясь священства;
\vs 1Ma 7:6 и обвиняли они перед царем народ, говоря: погубил Иуда и братья его друзей твоих, и нас выгнали из земли нашей.
\vs 1Ma 7:7 Итак, пошли теперь мужа, кому ты доверяешь; пусть он пойдет и увидит все разорение, которое они причинили нам и стране царя, и пусть накажет их и всех, помогающих им.
\vs 1Ma 7:8 Царь избрал Вакхида из друзей царских, который управлял по ту сторону реки, был велик в царстве и верен царю,
\vs 1Ma 7:9 и послал его и нечестивого Алкима, предоставив ему священство, и повелел ему сделать отмщение сынам Израиля.
\vs 1Ma 7:10 Они отправились и пришли в землю Иудейскую с большим войском; и он послал к Иуде и братьям его послов с мирным, но коварным предложением.
\vs 1Ma 7:11 Но они не вняли словам их, ибо видели, что они пришли с большим войском.
\vs 1Ma 7:12 К Алкиму же и Вакхиду сошлось собрание книжников искать справедливости.
\vs 1Ma 7:13 Первые из сынов Израилевых были Асидеи; они искали у них мира,
\vs 1Ma 7:14 ибо говорили: священник от племени Аарона пришел вместе с войском и не обидит нас.
\vs 1Ma 7:15 И он говорил с ними мирно и клялся им, и сказал: мы не сделаем зла вам и друзьям вашим.
\vs 1Ma 7:16 И они поверили ему, а он, захватив из них шестьдесят мужей, умертвил их в один день, как сказано в Писании:
\vs 1Ma 7:17 <<тела святых Твоих и кровь их пролили вокруг Иерусалима, и некому было похоронить их>>.
\vs 1Ma 7:18 И напал от них страх и ужас на весь народ, и говорили: нет в них истины и правды, ибо они нарушили постановление и клятву, которою клялись.
\vs 1Ma 7:19 Тогда Вакхид отступил от Иерусалима и расположился станом при Визефе, и, послав, поймал многих из бежавших от него мужей и некоторых из народа, заколол и бросил их в глубокий колодезь.
\rsbpar\vs 1Ma 7:20 Потом, поручив страну Алкиму и оставив с ним войско на помощь ему, Вакхид отправился к царю.
\vs 1Ma 7:21 Алким же домогался первосвященства.
\vs 1Ma 7:22 И собрались к нему все возмущавшие народ свой, и овладели землею Иудейскою, и произвели великое поражение в Израиле.
\vs 1Ma 7:23 И увидел Иуда все зло, какое причинил Алким со своими сообщниками сынам Израилевым,~--- больше, нежели язычники;
\vs 1Ma 7:24 и, обойдя все пределы Иудеи, сделал отмщение отступникам,~--- и они перестали входить в эту страну.
\vs 1Ma 7:25 Когда же Алким увидел, что Иуда и находящиеся с ним усилились, и понял, что не может противостоять им, возвратился к царю и жестоко обвинял их.
\vs 1Ma 7:26 Тогда царь послал Никанора, одного из славных вождей своих, ненавистника и враждебного Израилю, и приказал ему истребить этот народ.
\vs 1Ma 7:27 Никанор, придя в Иерусалим с большим войском, послал к Иуде и братьям его коварно со словами мирными:
\vs 1Ma 7:28 да не будет войны между мною и вами; я войду с немногими людьми, чтобы видеть лица ваши в мире.
\vs 1Ma 7:29 И пришел он к Иуде, и приветствовали они друг друга мирно; а между тем воины были приготовлены схватить Иуду.
\vs 1Ma 7:30 Иуде сделалось известным, что он пришел к нему с коварством, поэтому он убоялся его и не хотел более видеть лица его.
\vs 1Ma 7:31 Когда Никанор узнал, что умысел его открылся, вышел против Иуды на сражение близ Хафарсаламы.
\vs 1Ma 7:32 И пало из бывших при Никаноре около пяти тысяч мужей, а прочие убежали в город Давидов.
\vs 1Ma 7:33 После того Никанор взошел на гору Сион; и вышли из святилища некоторые из священников и старейшин народа, чтобы мирно приветствовать его и показать ему всесожжение, приносимое за царя.
\vs 1Ma 7:34 Но он осмеял их, надругался над ними и осквернил их, и говорил высокомерно,
\vs 1Ma 7:35 и, поклявшись, с гневом сказал: если не предан будет ныне Иуда и войско его в мои руки, то, когда возвращусь благополучно, сожгу дом сей. И ушел с великим гневом.
\vs 1Ma 7:36 А священники вошли и стали пред лицем жертвенника и храма, заплакали и сказали:
\vs 1Ma 7:37 Ты, Господи, избрал дом сей, чтобы на нем нарицалось имя Твое и чтобы он был домом молитвы и моления для народа Твоего.
\vs 1Ma 7:38 Сделай отмщение человеку сему и войску его, и пусть падут они от меча; вспомни злохуления их и не дай им оставаться долее.
\vs 1Ma 7:39 И вышел Никанор из Иерусалима и расположился станом при Вефороне, и пристало к нему здесь войско Сирийское.
\vs 1Ma 7:40 А Иуда с тремя тысячами мужей расположился станом при Адасе; и помолился Иуда, и сказал:
\vs 1Ma 7:41 Господи! когда посланные царя Ассирийского произносили злохуления, то пришел Ангел Твой и поразил из них сто восемьдесят пять тысяч.
\vs 1Ma 7:42 Так сокруши ныне пред нами сие полчище, да познают прочие, что они произносили хулу на святыни Твои, и суди их по злобе их.
\vs 1Ma 7:43 И вступили войска в сражение в тринадцатый день месяца Адара, и разбито было войско Никанора, и он первый пал в сражении.
\vs 1Ma 7:44 Когда же воины его увидели, что Никанор пал, то, побросав оружие свое, обратились в бегство.
\vs 1Ma 7:45 И преследовали их Израильтяне целый день, от Адаса до самой Газиры, и трубили вслед их вестовыми трубами.
\vs 1Ma 7:46 И выходили из всех окрестных селений Иудейских и окружали их,~--- и они, оборачиваясь к преследовавшим их, все пали от меча, и ни одного не осталось из них.
\vs 1Ma 7:47 И взяли \bibemph{Иудеи} добычи их и награбленное ими, и отрубили голову Никанора и правую руку его, которую он простирал надменно, и принесли и повесили перед Иерусалимом.
\vs 1Ma 7:48 Народ весьма радовался и провел тот день, как день великого веселья;
\vs 1Ma 7:49 и установили ежегодно праздновать этот день тринадцатого числа Адара.
\vs 1Ma 7:50 И успокоилась земля Иудейская на некоторое время.
\vs 1Ma 8:1 Иуда услышал о славе Римлян, что они могущественны и сильны и благосклонно принимают всех, обращающихся к ним, и кто ни приходил к ним, со всеми заключали они дружбу.
\vs 1Ma 8:2 А что они могущественны и сильны,~--- рассказывали ему о войнах их, о мужественных подвигах, которые они показали над Галатами, как они покорили их и сделали данниками;
\vs 1Ma 8:3 также о том, что сделали они в стране Испанской, чтобы овладеть находящимися там серебряными и золотыми рудниками,
\vs 1Ma 8:4 и своим благоразумием и твердостью овладели всем краем, хотя тот край весьма далеко отстоял от них, равно о царях, которые приходили против них от конца земли, и они сокрушили их и поразили великим поражением, а прочие платят им ежегодно дань;
\vs 1Ma 8:5 они также сокрушили на войне и покорили себе Филиппа и Персея, царя Китийского, и других, восставших против них,
\vs 1Ma 8:6 и Антиоха, великого царя Азии, который вышел против них на войну со ста двадцатью слонами, и с конницею, и колесницами, и весьма многочисленным войском и был разбит ими;
\vs 1Ma 8:7 они взяли его живого и заставили платить им великую дань,~--- как его, так и следующих после него царей,~--- дать заложников и допустить раздел,
\vs 1Ma 8:8 а страну Индийскую и Мидию, и Лидию, и другие из лучших областей его, взяв от него, отдали царю Евмению;
\vs 1Ma 8:9 и о том, как Еллины вознамерились прийти и истребить их,
\vs 1Ma 8:10 но это намерение сделалось им известным, и они послали против них одного военачальника и воевали против них,~--- и много из них пало пораженных, и взяли в плен жен их и детей их и разграбили их, и овладели их землею, и разорили крепости их, и поработили их до сего дня;
\vs 1Ma 8:11 и другие царства и острова, которые когда-либо восставали против них, они разорили и поработили.
\vs 1Ma 8:12 А с друзьями своими и с доверявшимися им они сохраняли дружбу; и овладели царствами ближними и дальними, и все, слышавшие имя их, боялись их.
\vs 1Ma 8:13 Если захотят кому помочь и кого воцарить, те царствуют, и кого хотят, сменяют, и они весьма возвысились;
\vs 1Ma 8:14 но при всем том никто из них не возлагал на себя венца и не облекался в порфиру, чтобы величаться ею.
\vs 1Ma 8:15 Они составили у себя совет, и постоянно каждый день триста двадцать человек совещаются обо всем, что относится до народа и благоустроения его;
\vs 1Ma 8:16 и каждый год одному человеку вверяют они начальство над собою и господство над всею землею их, и все слушают одного, и не бывает ни зависти, ни ревности между ними.
\rsbpar\vs 1Ma 8:17 Тогда избрал Иуда Евполема, сына Иоаннова, сына Аккосова, и Иасона, сына Елеазарова, и послал их в Рим, чтобы заключить с ними дружбу и союз
\vs 1Ma 8:18 и чтобы они сняли с них иго, ибо они видят, что Еллинское царство хочет поработить Израиля.
\vs 1Ma 8:19 Итак, они отправились в Рим, хотя путь был очень долгий, и вошли в собрание совета и, приступив, сказали:
\vs 1Ma 8:20 Иуда Маккавей и братья его и весь народ Иудейский послали нас к вам, чтобы заключить с вами союз и мир и чтобы вы вписали нас в число соратников и друзей ваших.
\vs 1Ma 8:21 И угодно было это слово перед ними.
\vs 1Ma 8:22 И вот список того послания, которое написали они в ответ на медных досках и послали в Иерусалим, чтобы оно служило для них там памятником мира и союза:
\vs 1Ma 8:23 <<благо да будет Римлянам и народу Иудейскому на море и на суше навеки, и меч и враг да будут далеко от них!
\vs 1Ma 8:24 Если же настанет война прежде у Римлян или у всех союзников их во всем владении их,
\vs 1Ma 8:25 то народ Иудейский должен оказать им всем сердцем помощь в войне, как потребует того время;
\vs 1Ma 8:26 и воюющим они не будут ни давать, ни доставлять ни хлеба, ни оружия, ни денег, ни кораблей, ибо так угодно Римлянам; они должны исполнять обязанность свою, ничего не получая.
\vs 1Ma 8:27 Точно так же, если прежде случится война у народа Иудейского, Римляне от души будут помогать им в войне, как потребует того время,
\vs 1Ma 8:28 и помогающим в войне не будут давать ни хлеба, ни оружия, ни денег, ни кораблей: так угодно Риму; они должны исполнять свои обязанности~--- и без обмана>>.
\vs 1Ma 8:29 На таких условиях заключили Римляне союз с народом Иудейским.
\vs 1Ma 8:30 Если же после сих условий те и другие вздумают что-нибудь прибавить или убавить, пусть сделают это по их общему произволению, и то, что они прибавят или убавят, будет иметь силу.
\vs 1Ma 8:31 А о том зле, какое делает \bibemph{Иудеям} царь Димитрий, мы написали ему так: <<для чего ты наложил тяжкое твое иго на друзей наших и союзников~--- Иудеев?
\vs 1Ma 8:32 Если они еще обратятся к нам с жалобою на тебя, то мы окажем им справедливость и будем воевать против тебя на море и на суше>>.
\vs 1Ma 9:1 Когда Димитрий услышал, что Никанор и воины его пали в сражении, послал Вакхида и Алкима во второй раз в землю Иудейскую и правое крыло с ними.
\vs 1Ma 9:2 И отправились они по дороге в Галгалы и расположились станом при Месалофе, что в Арвилах, и, овладев им, погубили множество людей.
\vs 1Ma 9:3 В первом месяце сто пятьдесят второго года расположились они станом у Иерусалима,
\vs 1Ma 9:4 но снялись и пошли к Верее с двадцатью тысячами мужей и двумя тысячами конницы.
\vs 1Ma 9:5 А Иуда расположился станом при Елеасе, и три тысячи избранных мужей с ним.
\vs 1Ma 9:6 Но, увидев множество войска, как оно многочисленно, они весьма устрашились, и многие из стана его разбежались, и осталось из них не более восьмисот мужей.
\vs 1Ma 9:7 Когда увидел Иуда, что разбежалось ополчение его, а война тревожила его, он смутился сердцем, потому что не имел времени собрать их.
\vs 1Ma 9:8 Он опечалился и сказал оставшимся: встанем и пойдем на противников наших; может быть, мы в силах будем сражаться с ними.
\vs 1Ma 9:9 Но они отклоняли его и говорили: мы не в силах, но будем теперь спасать жизнь нашу, и потом возвратимся с братьями нашими, и тогда будем сражаться против них, а теперь нас мало.
\vs 1Ma 9:10 Но Иуда сказал: нет, да не будет этого со мною, чтобы бежать от них; а если пришел час наш, то умрем мужественно за братьев наших и не оставим нарекания на славу нашу.
\vs 1Ma 9:11 И двинулось войско из стана и стало против них; и разделилась конница на две части, а впереди войска шли пращники и стрельцы и все сильные передовые воины.
\vs 1Ma 9:12 Вакхид же находился на правом крыле, и приближались отряды с обеих сторон и трубили трубами.
\vs 1Ma 9:13 Затрубили трубами и бывшие с Иудою, и поколебалась земля от шума войск, и было упорное сражение от утра до вечера.
\vs 1Ma 9:14 Когда увидел Иуда, что Вакхид и крепчайшая часть его войска находится на правой стороне, то собрались к нему все храбрые сердцем,~---
\vs 1Ma 9:15 и разбито ими правое крыло, и они преследовали их до горы Азота.
\vs 1Ma 9:16 Когда находившиеся на левом крыле увидели, что правое крыло разбито, то обратились вслед за Иудою и бывшими с ним, с тыла.
\vs 1Ma 9:17 И сражение было жестокое, и много пало пораженных с той и другой стороны,
\vs 1Ma 9:18 пал и Иуда, а прочие обратились в бегство.
\vs 1Ma 9:19 И взяли Ионафан и Симон Иуду, брата своего, и похоронили его во гробе отцов его в Модине.
\vs 1Ma 9:20 И оплакивали его и рыдали о нем сильно все Израильтяне, и печалились много дней и говорили:
\vs 1Ma 9:21 как пал сильный, спасавший Израиля?
\vs 1Ma 9:22 Прочие же дела Иуды, и сражения, и мужественные подвиги, которые совершил он, и величие его не описаны, ибо их было весьма много.
\rsbpar\vs 1Ma 9:23 По смерти же Иуды во всех пределах Израильских явились люди беззаконные, и поднялись все делатели неправды.
\vs 1Ma 9:24 В те самые дни был очень сильный голод, и страна пристала к ним.
\vs 1Ma 9:25 И выбрал Вакхид нечестивых мужей и поставил их начальниками страны.
\vs 1Ma 9:26 Они разведывали и разыскивали друзей Иуды и приводили их к Вакхиду, а он мстил им и издевался над ними.
\vs 1Ma 9:27 И была великая скорбь в Израиле, какой не бывало с того дня, как не видно стало у них пророка.
\vs 1Ma 9:28 Тогда собрались все друзья Иуды и сказали Ионафану:
\vs 1Ma 9:29 с того времени, как скончался брат твой Иуда, нет подобного ему мужа, чтобы выйти против врагов и Вакхида и против ненавистников нашего народа.
\vs 1Ma 9:30 Итак, теперь мы тебя избрали~--- быть нам вместо него начальником и вождем, чтобы вести войну нашу.
\vs 1Ma 9:31 И принял Ионафан в то время предводительство и стал на место Иуды, брата своего.
\vs 1Ma 9:32 И узнал о том Вакхид и искал убить его.
\vs 1Ma 9:33 Об этом узнали Ионафан и Симон, брат его, и все бывшие с ним и убежали в пустыню Фекое и расположились станом при водах озера Асфар.
\vs 1Ma 9:34 Вакхид, узнав о том в день субботний, переправился сам и все войско его за Иордан.
\vs 1Ma 9:35 А Ионафан отправил брата своего~--- предводителя народа~--- и просил друзей своих, Наватеев, чтобы сложить у них большой запас свой.
\vs 1Ma 9:36 Но вышли из Мидавы сыны Иамври и схватили Иоанна и все, что он имел, и ушли.
\vs 1Ma 9:37 После сих происшествий сказали Ионафану и Симону, брату его, что сыны Иамври торжественно совершают знатный брак и провожают из Надавафа с великою пышностью невесту, дочь одного из знатных вельмож Хананейских.
\vs 1Ma 9:38 Тогда вспомнили они об Иоанне, брате своем, и вышли, и скрылись под кровом горы.
\vs 1Ma 9:39 Подняв глаза свои, они увидели: вот восклицания и большое приданое; навстречу вышел жених и друзья его и братья его с тимпанами и музыкою и со многими оружиями.
\vs 1Ma 9:40 Тогда бывшие с Ионафаном поднялись на них из засады и побили их, и много пало пораженных, а остальные убежали на гору; и взяли они всю добычу их.
\vs 1Ma 9:41 И обратилось брачное торжество в печаль, и звук музыки их~--- в плач.
\vs 1Ma 9:42 Так отмстили они за кровь брата своего и возвратились к болотистому месту у Иордана.
\vs 1Ma 9:43 И услышал об этом Вакхид~--- и в день субботний пришел к берегам Иордана с большим войском.
\vs 1Ma 9:44 Тогда сказал Ионафан бывшим с ним: встанем теперь и сразимся за жизнь нашу, ибо ныне~--- не то, что вчера и третьего дня.
\vs 1Ma 9:45 Вот, неприятель и спереди нас и сзади нас, вода Иордана с той и с другой стороны, и болото и лес, и нет места, куда уклониться.
\vs 1Ma 9:46 Итак, теперь воззовите на небо, чтобы избавиться вам от руки врагов ваших.
\vs 1Ma 9:47 И началось сражение. И простер Ионафан руку свою, чтобы поразить Вакхида, но тот уклонился от него назад.
\vs 1Ma 9:48 И бросился Ионафан и бывшие с ним в Иордан и переплыли на другой берег, а те не перешли за ними Иордана.
\vs 1Ma 9:49 И пало у Вакхида в тот день до тысячи мужей.
\vs 1Ma 9:50 И возвратился он в Иерусалим и построил в Иудее крепкие города: крепость в Иерихоне, и Еммаум и Вефорон, и Вефиль и Фамнафу в Фарафоне, и Тефон с высокими стенами, воротами и запорами,
\vs 1Ma 9:51 и поставил в них стражу, чтобы враждебно действовать против Израиля.
\vs 1Ma 9:52 Укрепил также город в Вефсуре и Газару и крепость и оставил в них войско со съестными запасами,
\vs 1Ma 9:53 и взял в заложники сыновей вождей страны и поместил их в Иерусалимской крепости под стражею.
\rsbpar\vs 1Ma 9:54 В сто пятьдесят третьем году, во втором месяце, Алким велел разорить стену внутреннего двора храма и разрушить дело пророков, и уже начал разрушение.
\vs 1Ma 9:55 Но в то самое время Алким поражен был ударом, и остановились предприятия его; уста его сомкнулись, он онемел и не мог более вымолвить ни одного слова и завещать о доме своем.
\vs 1Ma 9:56 И умер Алким в то же время в тяжких мучениях.
\vs 1Ma 9:57 Когда Вакхид узнал, что Алким умер, возвратился к царю; и земля Иудейская два года оставалась в покое.
\vs 1Ma 9:58 Тогда все беззаконники совещались и говорили: вот, Ионафан и находящиеся с ним живут безопасно в покое; приведем теперь Вакхида, и он схватит всех их в одну ночь.
\vs 1Ma 9:59 Пошли и предложили ему такой совет.
\vs 1Ma 9:60 Он решился идти с большим войском и послал тайно письма всем союзникам своим, которые находились в Иудее, чтобы они схватили Ионафана и находящихся с ним, но они не могли, потому что замысел их сделался известен им.
\vs 1Ma 9:61 И поймали они из мужей страны виновников этого злодейства до пятидесяти человек и убили их.
\vs 1Ma 9:62 После сего удалились Ионафан и Симон и бывшие с ними в Вефваси, что в пустыне, и возобновили разрушенное там и укрепили город.
\vs 1Ma 9:63 Узнав об этом, Вакхид собрал все войско свое, известив и тех, которые находились в Иудее,
\vs 1Ma 9:64 пришел и осадил Вефваси, и сражался против него много дней и устроил машины.
\vs 1Ma 9:65 Ионафан же оставил в городе Симона, брата своего, а сам вышел в страну, и вышел с небольшим числом,
\vs 1Ma 9:66 и поразил Одоааррина и братьев его и сыновей Фасирона в шатрах их и начал поражать и наступать с силою.
\vs 1Ma 9:67 Тогда и Симон и бывшие с ним выступили из города и сожгли машины,
\vs 1Ma 9:68 и сражались против Вакхида, и он был разбит ими; этим они сильно опечалили его, потому что замысел его и поход остался тщетным.
\vs 1Ma 9:69 Сильно разгневался он на мужей беззаконных, которые присоветовали ему идти в эту страну, и многих из них умертвил, и решился возвратиться в землю свою.
\vs 1Ma 9:70 Узнав об этом, Ионафан послал к нему старейшин, чтобы заключить с ним мир и чтобы он отдал пленных.
\vs 1Ma 9:71 Он принял это и сделал по словам его, и поклялся не причинять ему никакого зла во все дни жизни своей,
\vs 1Ma 9:72 и отдал ему пленных, которых прежде взял в плен в земле Иудейской, и возвратился в землю свою и не приходил более в пределы их.
\vs 1Ma 9:73 И унялся меч в Израиле, и поселился Ионафан в Махмасе; и начал Ионафан судить народ и истребил нечестивых из среды Израиля.
\vs 1Ma 10:1 В сто шестидесятом году выступил Александр, сын Антиоха Епифана, и овладел Птолемаидою: и приняли его, и он воцарился там.
\vs 1Ma 10:2 Когда услышал о том царь Димитрий, собрал весьма многочисленное войско и вышел против него на войну.
\vs 1Ma 10:3 И послал Димитрий письма Ионафану с мирным предложением, как бы желая возвеличить его,
\vs 1Ma 10:4 ибо говорил: предупредим заключить с ним мир, прежде нежели он заключит с Александром против нас:
\vs 1Ma 10:5 тогда он припомнит все зло, которое мы сделали против него и братьев его и народа его.
\vs 1Ma 10:6 И он дал ему власть набирать войско и приготовлять оружия, чтобы быть союзником его, и велел отдать ему заложников, которые находились в крепости.
\vs 1Ma 10:7 Ионафан пришел в Иерусалим и прочитал письма вслух всего народа и бывших в крепости;
\vs 1Ma 10:8 и убоялись все великим страхом, услышав, что царь дал ему власть набирать войско;
\vs 1Ma 10:9 а бывшие в крепости выдали Ионафану заложников, и он возвратил их родителям их.
\vs 1Ma 10:10 И жил Ионафан в Иерусалиме; и начал строить и возобновлять город,
\vs 1Ma 10:11 и сказал производившим работы, чтобы они строили стены и вокруг горы Сиона для твердости из четырехугольных камней,~--- и делали так.
\vs 1Ma 10:12 Тогда иноплеменные, бывшие в крепостях, построенных Вакхидом, бежали:
\vs 1Ma 10:13 каждый оставил свое место и ушел в свою землю.
\vs 1Ma 10:14 Только в Вефсуре остались некоторые из тех, которые оставили закон и заповеди, ибо это место служило для них убежищем.
\vs 1Ma 10:15 И услышал царь Александр о тех обещаниях, какие Димитрий послал Ионафану, и рассказали ему о войнах и храбрых подвигах, которые совершил Ионафан и братья его, и о трудностях, понесенных ими.
\vs 1Ma 10:16 Тогда он сказал: найдем ли мы еще такого мужа, как этот? Сделаем же его нашим другом и союзником.
\vs 1Ma 10:17 И написал и послал ему письмо в таких словах:
\vs 1Ma 10:18 <<Царь Александр брату Ионафану~--- радоваться.
\vs 1Ma 10:19 Услышали мы о тебе, что ты~--- муж, крепкий силою и достойный быть нашим другом.
\vs 1Ma 10:20 Итак, мы поставляем тебя ныне первосвященником народа твоего; и ты будешь именоваться другом царя (он послал ему порфиру и золотой венец) и будешь держать нашу сторону и хранить дружбу с нами>>.
\vs 1Ma 10:21 И облекся Ионафан в священную одежду в седьмом месяце сто шестидесятого года, в праздник кущей, и собрал войско и заготовил множество оружий.
\vs 1Ma 10:22 И услышал об этом Димитрий и огорчился, и сказал:
\vs 1Ma 10:23 что это мы сделали, что Александр предупредил нас заключить дружбу с Иудеями в подкрепление себе?
\vs 1Ma 10:24 Напишу и я им слова приветствия, восхваления и обещаний, чтобы были они в помощь мне.
\vs 1Ma 10:25 И послал им письмо в таких словах: <<Царь Димитрий народу Иудейскому~--- радоваться.
\vs 1Ma 10:26 Слышали мы и радовались, что вы сохраняете договоры наши, пребываете в дружбе с нами и не склоняетесь к врагам нашим.
\vs 1Ma 10:27 Продолжайте и ныне сохранять верность к нам, и мы воздадим вам добром за то, что вы делаете для нас:
\vs 1Ma 10:28 сделаем вам многие уступки и дадим вам дары.
\vs 1Ma 10:29 Ныне же разрешаю вас и освобождаю всех Иудеев от податей и пошлины с соли и с венцов;
\vs 1Ma 10:30 и за третью часть семян и половинную часть древесных плодов, принадлежащую мне, отныне и впредь я отменяю брать с земли Иудейской и с трех областей, присоединенных к ней от Самарии и Галилеи, от нынешнего дня и на вечные времена.
\vs 1Ma 10:31 И Иерусалим да будет священным и свободным и пределы его, десятины и доходы его.
\vs 1Ma 10:32 Предоставляю и власть над крепостью Иерусалимскою и даю право первосвященнику поставить в ней людей, каких он сам изберет, для охранения ее;
\vs 1Ma 10:33 и всякого человека из Иудеев, взятого в плен из земли Иудейской, во всем царстве моем отпускаю на свободу даром: пусть все будут свободны от повинностей за себя и за скот свой.
\vs 1Ma 10:34 Все праздники и субботы и новомесячия, и дни установленные~--- три дня пред праздником и три дня после праздника,~--- все эти дни пусть будут днями льготы и свободы всем Иудеям, находящимся в моем царстве.
\vs 1Ma 10:35 Никто не будет иметь права притеснять и отягощать кого-нибудь из них ни по какому делу.
\vs 1Ma 10:36 И пусть из Иудеев записываются в царские войска до тридцати тысяч человек,~--- и им будет даваться жалованье наравне со всеми войсками царскими.
\vs 1Ma 10:37 И из них да будут поставляемы начальствующими над большими крепостями царскими, из них же да будут поставляемы и над делами царства, требующими верности, и их приставники и начальники да будут из них же, и пусть они живут по своим законам, как повелел царь в земле Иудейской.
\vs 1Ma 10:38 И три области, присоединенные к Иудее от страны Самарийской, пусть останутся присоединенными к Иудее, чтобы считаться и быть им за одну и не подлежать другой власти, кроме власти первосвященника.
\vs 1Ma 10:39 Птолемаиду с округом ее я отдаю в дар святилищу в Иерусалиме на издержки, потребные для святилища;
\vs 1Ma 10:40 я же даю ежегодно пятнадцать тысяч сиклей серебра из царских сборов с подлежащих мест.
\vs 1Ma 10:41 И все остальное, чего не отдали заведующие сборами, как в прежние годы, отныне будут отдавать на работы храма.
\vs 1Ma 10:42 Сверх того пять тысяч сиклей серебра, которые брали от доходов святилища из ежегодного сбора, и те уступаются, как принадлежащие служащим священникам.
\vs 1Ma 10:43 И все, которые убегут в храм Иерусалимский и во все пределы его по причине повинностей царских и всех других, пусть будут свободны со всем, что принадлежит им в царстве моем.
\vs 1Ma 10:44 И на строение и возобновление святилища издержки будут выдаваемы из сборов царских.
\vs 1Ma 10:45 И на построение стен Иерусалима и укрепление их вокруг издержки будут выдаваемы из доходов царских, а также на построение стен в Иудее>>.
\vs 1Ma 10:46 Ионафан и народ, выслушав эти слова, не поверили им и не приняли их, ибо вспомнили о тех великих бедствиях, которые нанес Димитрий Израильтянам, жестоко притеснив их,
\vs 1Ma 10:47 и предпочли союз с Александром, ибо он первый сделал им мирные предложения,~--- и помогали ему в войнах во все дни.
\rsbpar\vs 1Ma 10:48 Царь Александр собрал большое войско и ополчился против Димитрия.
\vs 1Ma 10:49 И вступили два царя в сражение, и войско Димитрия обратилось в бегство; Александр преследовал его, и превозмог,
\vs 1Ma 10:50 и весьма настойчиво продолжал сражение до самого захождения солнца,~--- и пал Димитрий в этот день.
\rsbpar\vs 1Ma 10:51 После того Александр отправил послов к Птоломею, царю Египетскому, с такими словами:
\vs 1Ma 10:52 <<Я возвратился в землю царства моего и воссел на престоле отцов моих, принял верховную власть, сокрушил Димитрия и стал обладателем страны нашей.
\vs 1Ma 10:53 Я вступил с ним в сражение, и он разбит нами и войско его, и воссели мы на престоле царства его.
\vs 1Ma 10:54 Итак, заключим теперь дружбу между нами, и ты дай мне дочь твою в жену, и буду я тебе зятем и дам тебе и ей дары, достойные тебя>>.
\vs 1Ma 10:55 И отвечал царь Птоломей так: <<Счастлив день, в который ты возвратился в землю отцов твоих и воссел на престоле царства их.
\vs 1Ma 10:56 Ныне я исполню для тебя то, о чем ты писал, только ты выйди ко мне в Птолемаиду, чтобы нам видеть друг друга, и я породнюсь с тобою, как ты сказал>>.
\vs 1Ma 10:57 И отправился Птоломей из Египта сам и Клеопатра, дочь его, и прибыли в Птолемаиду в сто шестьдесят втором году.
\vs 1Ma 10:58 Царь Александр встретил его, и он выдал за него Клеопатру, дочь свою, и устроил брак ее в Птолемаиде, как прилично царям, с великою пышностью.
\vs 1Ma 10:59 Писал также царь Александр Ионафану, чтобы он вышел к нему навстречу.
\vs 1Ma 10:60 И отправился Ионафан в Птолемаиду с пышностью, и представлялся обоим царям и одарил их и приближенных их серебром и золотом и многими дарами, и приобрел благоволение их.
\vs 1Ma 10:61 И собрались против него мужи зловредные из среды Израиля, мужи беззаконные, чтобы оклеветать его; но царь не внял им.
\vs 1Ma 10:62 И повелел царь снять с Ионафана одежды его и облечь его в порфиру,~--- и сделали так.
\vs 1Ma 10:63 И посадил его царь с собою и сказал своим правителям: выйдите с ним на средину города и провозгласите, чтобы никто не смел клеветать на него ни в каком деле и никто не тревожил его никаким делом.
\vs 1Ma 10:64 Когда клеветавшие увидели славу его, как он был провозглашаем и как облечен в порфиру, все разбежались.
\vs 1Ma 10:65 Так прославил его царь и вписал его в число первых друзей, и назначил его военачальником и областным правителем.
\vs 1Ma 10:66 И возвратился Ионафан в Иерусалим с миром и веселием.
\rsbpar\vs 1Ma 10:67 Но в сто шестьдесят пятом году пришел из Крита Димитрий, сын Димитрия, в землю отцов своих.
\vs 1Ma 10:68 Услышав о том, царь Александр весьма огорчился и возвратился в Антиохию.
\vs 1Ma 10:69 И поставил Димитрий военачальником Аполлония, правителя Келе-Сирии,~--- и он собрал большое войско и расположился станом при Иамнии и послал к первосвященнику Ионафану сказать:
\vs 1Ma 10:70 ты только один превозносишься над нами, я же подвергся осмеянию и посрамлению через тебя. Зачем ты противостоишь нам в горах?
\vs 1Ma 10:71 Если ты надеешься на твои военные силы, то сойди к нам на равнину, и там мы померяемся, ибо со мною войско городов.
\vs 1Ma 10:72 Спроси и узнай, кто я и прочие помогающие нам, и скажут тебе: невозможно вам устоять пред лицем нашим, ибо дважды обращены были в бегство отцы твои в земле своей.
\vs 1Ma 10:73 И ныне ты не можешь устоять против такой конницы и такого войска на равнине, где нет ни камней, ни ущелий, ни места для убежища.
\vs 1Ma 10:74 Когда Ионафан выслушал эти слова Аполлония, то подвигся духом и, избрав десять тысяч мужей, вышел из Иерусалима, и брат его Симон сошелся с ним на помощь ему.
\vs 1Ma 10:75 И расположился станом при Иоппии; но не впустили его в город, ибо в Иоппии была стража Аполлония, и они начали воевать против нее.
\vs 1Ma 10:76 Тогда устрашенные жители отворили ему город, и Ионафан овладел Иоппиею.
\vs 1Ma 10:77 Услышав о сем, Аполлоний взял три тысячи конницы и большое войско и пошел в Азот, как бы делая переход, а между тем прошел на равнину, ибо имел множество конницы и надеялся на нее.
\vs 1Ma 10:78 Ионафан же преследовал его до Азота, и вступили войска в сражение.
\vs 1Ma 10:79 Между тем Аполлоний оставил тысячу всадников в скрытном месте позади них;
\vs 1Ma 10:80 но Ионафан узнал, что есть засада сзади него. И обступили войско его и бросали в народ стрелы с утра до вечера,
\vs 1Ma 10:81 народ же стоял, как приказал Ионафан; наконец всадники утомились.
\vs 1Ma 10:82 Тогда Симон подвел войско свое и напал на отряд, ибо всадники изнемогли,~--- и были разбиты им и обратились в бегство.
\vs 1Ma 10:83 И рассеялись всадники по равнине и убежали в Азот, и вошли в Бетдагон, капище их, чтобы спастись.
\vs 1Ma 10:84 Но Ионафан сжег Азот и окрестные города и взял добычу их, и капище Дагона с убежавшими в него сжег огнем.
\vs 1Ma 10:85 И было павших от меча с сожженными до восьми тысяч мужей.
\vs 1Ma 10:86 Отправившись оттуда, Ионафан расположился станом против Аскалона; но жители города вышли к нему навстречу с великою почестью.
\vs 1Ma 10:87 И возвратился Ионафан со всеми бывшими при нем в Иерусалим, имея при себе много добычи.
\vs 1Ma 10:88 Когда царь Александр услышал о сих событиях, то вновь почтил Ионафана
\vs 1Ma 10:89 и послал ему золотую пряжку, какая по обычаю давалась царским родственникам, и подарил ему Аккарон и всю область его в наследственное владение.
\vs 1Ma 11:1 Между тем царь Египетский, собрав многочисленное войско, как песок на берегу морском, и множество кораблей, домогался овладеть царством Александра хитростью и присоединить его к своему царству.
\vs 1Ma 11:2 Он пришел в Сирию с мирными речами, и жители отворяли ему города и выходили навстречу, ибо дано было от царя Александра повеление встречать его, потому что он был тесть его.
\vs 1Ma 11:3 Когда же Птоломей входил в города, то оставлял войско для стражи в каждом городе.
\vs 1Ma 11:4 Когда приблизился он к Азоту, то показали ему сожженное капище Дагона, и Азот и окрестные города разрушенные, и тела пораженные и сожженные во время сражения, ибо сложили их в груды по пути его,
\vs 1Ma 11:5 и рассказали царю о всем, что сделал Ионафан, жалуясь на него; но царь промолчал.
\vs 1Ma 11:6 Тогда вышел Ионафан навстречу царю в Иоппию с почетом, и приветствовали друг друга и ночевали там.
\vs 1Ma 11:7 И шел Ионафан с царем до реки, называемой Елевфера, и потом возвратился в Иерусалим.
\vs 1Ma 11:8 Царь же Птоломей овладел городами на морском берегу до Селевкии приморской и составлял злые замыслы против Александра.
\vs 1Ma 11:9 И послал послов к царю Димитрию, говоря: приди сюда, заключим между собою союз, и я дам тебе дочь мою, которую имеет Александр, и ты будешь царствовать в царстве отца твоего.
\vs 1Ma 11:10 Я раскаиваюсь, что отдал ему дочь мою, ибо он старался убить меня.
\vs 1Ma 11:11 Так клеветал он на него, потому что сам домогался царства его.
\vs 1Ma 11:12 И, отняв у него дочь свою, отдал ее Димитрию, и стал чужим для Александра, и обнаружилась вражда их.
\vs 1Ma 11:13 И вошел Птоломей в Антиохию и возложил на свою голову два венца~--- Азии и Египта.
\vs 1Ma 11:14 Царь Александр находился в то время в Киликии, потому что жители тех мест отпали от него.
\vs 1Ma 11:15 Услышав об этом, Александр пошел против него воевать; тогда Птоломей вывел войско и встретил его с крепкою силою, и обратил его в бегство.
\vs 1Ma 11:16 И убежал Александр в Аравию, чтобы укрыться там; царь же Птоломей возвысился.
\vs 1Ma 11:17 Завдиил, Аравитянин, снял голову с Александра и послал ее Птоломею.
\vs 1Ma 11:18 Царь же Птоломей на третий день умер, а оставшиеся в крепостях истреблены были жителями крепостей.
\vs 1Ma 11:19 И воцарился Димитрий в сто шестьдесят седьмом году.
\rsbpar\vs 1Ma 11:20 В те дни собрал Ионафан Иудеев, чтобы завоевать крепость Иерусалимскую, и устроил перед нею множество машин.
\vs 1Ma 11:21 Но некоторые ненавистники народа своего, отступники от закона, пошли к царю и донесли, что Ионафан облагает крепость.
\vs 1Ma 11:22 Когда он услышал об этом, разгневался и, поспешно собравшись, отправился в Птолемаиду, и написал Ионафану, чтобы он не облагал крепости, а как можно скорее шел к нему навстречу в Птолемаиду, чтобы переговорить с ним.
\vs 1Ma 11:23 Но Ионафан, выслушав это, приказал продолжать осаду и, избрав из старейшин Израильских и священников, решился подвергнуться опасности.
\vs 1Ma 11:24 Взяв серебра и золота, одежды и много других даров, он пошел к царю в Птолемаиду и приобрел благоволение его.
\vs 1Ma 11:25 И хотя некоторые отступники из того же народа клеветали на него,
\vs 1Ma 11:26 но царь поступил с ним так же, как поступали с ним предшественники его, и возвысил его пред всеми друзьями своими,
\vs 1Ma 11:27 и утвердил за ним первосвященство и другие почетные отличия, какие он имел прежде, и сделал его одним из первых друзей своих.
\vs 1Ma 11:28 И просил Ионафан царя освободить от податей Иудею и три области и Самарию и обещал ему триста талантов.
\vs 1Ma 11:29 Царь согласился и написал Ионафану обо всем этом письмо такого содержания:
\vs 1Ma 11:30 <<Царь Димитрий брату Ионафану и народу Иудейскому~--- радоваться.
\vs 1Ma 11:31 Список письма, которое мы писали о вас Ласфену, родственнику нашему, посылаем и к вам, чтобы вы знали.
\vs 1Ma 11:32 Царь Димитрий Ласфену-отцу~--- радоваться.
\vs 1Ma 11:33 Народу Иудейскому, друзьям нашим, верно исполняющим свои обязанности перед нами, мы рассудили оказать благодеяние за их доброе расположение к нам.
\vs 1Ma 11:34 Итак, мы утверждаем за ними как пределы Иудеи, так и три области: Аферему, Лидду и Рамафем, которые присоединены к Иудее от Самарии, и все, принадлежащее всем жрецам их в Иерусалиме, за те царские оброки, которые прежде ежегодно получал от них царь с произрастаний земли и с плодов древесных,
\vs 1Ma 11:35 и все прочее, принадлежащее нам отныне из десятин и даней, следующих нам, соленые озера и венечный сбор, нам принадлежащий, все вполне уступаем им.
\vs 1Ma 11:36 И ничего не будет отменено из сего отныне и навсегда.
\vs 1Ma 11:37 Итак, позаботьтесь сделать список с сего, и пусть будет отдан он Ионафану и положен на святой горе в известном месте>>.
\rsbpar\vs 1Ma 11:38 И увидел царь Димитрий, что преклонилась земля пред ним и ничто не противилось ему, и отпустил все войска свои, каждого в свое место, кроме войск чужеземных, которые он нанял с островов чужих народов, за что все войска отцов его ненавидели его.
\vs 1Ma 11:39 Трифон, один из прежних приверженцев Александра, видя, что все войска ропщут на Димитрия, отправился к Емалкую Аравитянину, который воспитывал Антиоха, малолетнего сына Александрова;
\vs 1Ma 11:40 и настаивал, чтобы он выдал его ему, дабы сделать его царем вместо него; и рассказал ему обо всем, что сделал Димитрий, и о неприязни, которую имеют к нему войска его, и пробыл там много дней.
\rsbpar\vs 1Ma 11:41 И послал Ионафан к царю Димитрию, чтобы он вывел оставленных им в Иерусалимской крепости и укреплениях, ибо они нападали на Израиля.
\vs 1Ma 11:42 Димитрий послал сказать Ионафану: не только это сделаю для тебя и для народа твоего, но и почту тебя и народ твой великою честью, как скоро буду иметь благоприятное время.
\vs 1Ma 11:43 Теперь же ты справедливо поступишь, если пришлешь мне людей на помощь в войне, ибо отложились от меня все войска мои.
\vs 1Ma 11:44 И послал к нему Ионафан в Антиохию три тысячи храбрых мужей, и пришли они к царю, и обрадовался царь прибытию их.
\vs 1Ma 11:45 Граждане же, собравшись на средину города до ста двадцати тысяч человек, хотели убить царя.
\vs 1Ma 11:46 Но царь убежал во дворец, а граждане заняли все улицы города и начали осаждать его.
\vs 1Ma 11:47 Тогда царь призвал на помощь Иудеев, и все они тотчас собрались к нему, и вдруг рассыпались по городу, и умертвили в тот день в городе до ста тысяч,
\vs 1Ma 11:48 и зажгли город, и взяли в тот день много добычи, и спасли царя.
\vs 1Ma 11:49 И увидели граждане, что Иудеи овладели городом, как хотели, и упали духом, и начали взывать к царю, умоляя и говоря:
\vs 1Ma 11:50 прости нас, и пусть Иудеи перестанут нападать на нас и на город.
\vs 1Ma 11:51 И сложили оружие и заключили мир. И прославились Иудеи перед царем и перед всеми в царстве его и возвратились в Иерусалим с большою добычею.
\vs 1Ma 11:52 И воссел царь Димитрий на престоле царства своего, и успокоилась земля пред ним.
\vs 1Ma 11:53 Но он солгал во всем, что обещал, и изменил Ионафану и не воздал за сделанное ему добро и сильно оскорбил его.
\vs 1Ma 11:54 После того возвратился Трифон и с ним Антиох, еще очень юный; он воцарился и возложил на себя венец.
\vs 1Ma 11:55 И собрались к нему все войска, которые распустил Димитрий, и начали воевать с ним, и он обратился в бегство, и был поражен.
\vs 1Ma 11:56 И взял Трифон слонов и овладел Антиохиею.
\vs 1Ma 11:57 И писал юный Антиох Ионафану, говоря: предоставляю тебе первосвященство и поставляю тебя над четырьмя областями, и ты будешь в числе друзей царских.
\vs 1Ma 11:58 И послал ему золотые сосуды и домашнюю утварь и дал ему право пить из золотых сосудов и носить порфиру и золотую пряжку,
\vs 1Ma 11:59 а Симона, брата его, поставил военачальником от области Тирской до пределов Египта.
\vs 1Ma 11:60 И выступил Ионафан в поход, и проходил по ту сторону реки \bibemph{(Иордана)} и по городам, и собрались к нему на помощь все Сирийские войска; и пришел он к Аскалону, и встретили его жители города с честью.
\vs 1Ma 11:61 Оттуда пошел он в Газу; но жители Газы заперлись; и осадил он город, и сжег огнем предместья его, и опустошил их.
\vs 1Ma 11:62 И упросили жители Газы Ионафана, и он примирился с ними, только взял в заложники сыновей начальников их и отослал их в Иерусалим, и прошел страну до Дамаска.
\rsbpar\vs 1Ma 11:63 И услышал Ионафан, что пришли в Кадис, в Галилее, военачальники Димитрия с многочисленным войском, чтобы удалить его от страны.
\vs 1Ma 11:64 Но он пошел навстречу им, брата же своего, Симона, оставил в стране.
\vs 1Ma 11:65 И расположил Симон стан свой при Вефсуре, и осаждал его многие дни, и запер его.
\vs 1Ma 11:66 И просили его о мире, и он согласился, но выгнал их оттуда, и овладел городом, и поставил в нем стражу.
\vs 1Ma 11:67 А Ионафан и войско его расположились станом при водах Геннисаретских и утром стали на равнине Насор.
\vs 1Ma 11:68 И вот, войско иноплеменников встретилось с ним на равнине, оставив против него засаду в горах, само же шло навстречу ему с противной стороны.
\vs 1Ma 11:69 И вышли бывшие в засаде из своих мест, и начали сражаться: тогда все бывшие с Ионафаном обратились в бегство,
\vs 1Ma 11:70 и ни одного из них не осталось, кроме Маттафии, сына Авессаломова, и Иуды, сына Халфиева, начальников воинских отрядов.
\vs 1Ma 11:71 И разодрал Ионафан одежды свои, и посыпал землю на голову свою, и молился.
\vs 1Ma 11:72 Потом возвратился сражаться с ними и поразил их, и они бежали.
\vs 1Ma 11:73 Увидев это, убежавшие от него возвратились к нему, и с ним преследовали их до Кадиса, до самого стана их, и там остановились.
\vs 1Ma 11:74 В тот день пало от иноплеменников до трех тысяч мужей; и возвратился Ионафан в Иерусалим.
\vs 1Ma 12:1 Ионафан, видя, что время благоприятствует ему, избрал мужей и послал в Рим установить и возобновить дружбу с Римлянами,
\vs 1Ma 12:2 и к Спартанцам и в другие места послал письма о том же.
\vs 1Ma 12:3 И пришли они в Рим, и вошли в совет, и сказали: <<Ионафан-первосвященник и народ Иудейский прислали нас, чтобы возобновить дружбу с вами и союз по-прежнему>>.
\vs 1Ma 12:4 И там дали им письма к местным начальникам, чтобы проводили их в землю Иудейскую с миром.
\vs 1Ma 12:5 Вот список письма, которое писал Ионафан Спартанцам:
\vs 1Ma 12:6 <<Первосвященник Ионафан и народные старейшины и священники и остальной народ Иудейский братьям Спартанцам~--- радоваться.
\vs 1Ma 12:7 Еще прежде от Дария [Арея], царствовавшего у вас, присланы были к первосвященнику Онии письма, что вы~--- братья наши, как показывает список.
\vs 1Ma 12:8 И принял Ония посланного мужа с честью, и получил письма, в которых ясно говорилось о союзе и дружбе.
\vs 1Ma 12:9 Мы же, хотя и не имеем надобности в них, имея утешением священные книги, которые в руках наших,
\vs 1Ma 12:10 но предприняли послать к вам для возобновления братства и дружбы, чтобы не отчуждаться от вас, ибо много прошло времени после того, как вы присылали к нам.
\vs 1Ma 12:11 Мы неопустительно во всякое время, как в праздники, так и в прочие установленные дни, воспоминаем о вас при жертвоприношениях наших и молитвах, как должно и прилично воспоминать братьев.
\vs 1Ma 12:12 Мы радуемся о вашей славе;
\vs 1Ma 12:13 нас же обстоят многие беды и частые войны; ибо воевали против нас окрестные цари.
\vs 1Ma 12:14 Но мы не хотели беспокоить вас и прочих союзников и друзей наших в этих войнах,
\vs 1Ma 12:15 ибо мы имеем помощь небесную, помогающую нам; мы избавились от врагов наших, и враги наши усмирены.
\vs 1Ma 12:16 Теперь мы избрали Нуминия, сына Антиохова, и Антипатра, сына Иасонова, и послали их к Римлянам возобновить дружбу с ними и прежний союз.
\vs 1Ma 12:17 Поручили им идти и к вам, приветствовать вас и вручить вам письма от нас о возобновлении и с вами нашего братства.
\vs 1Ma 12:18 И вы хорошо сделаете, ответив нам на них>>.
\rsbpar\vs 1Ma 12:19 Вот и список писем, которые прислал Дарий [Арей]:
\vs 1Ma 12:20 <<Царь Спартанский Онии первосвященнику~--- радоваться.
\vs 1Ma 12:21 Найдено в писании о Спартанцах и Иудеях, что они~--- братья и от рода Авраамова.
\vs 1Ma 12:22 Теперь, когда мы узнали об этом, вы хорошо сделаете, написав нам о благосостоянии вашем.
\vs 1Ma 12:23 Мы же уведомляем вас: скот ваш и имущество ваше~--- наши, а что у нас есть, то ваше. И мы повелели объявить вам о том>>.
\rsbpar\vs 1Ma 12:24 И услышал Ионафан, что возвратились военачальники Димитрия с б\acc{о}льшим войском, нежели прежде, чтобы воевать против него,
\vs 1Ma 12:25 и вышел из Иерусалима, и встретил их в стране Амафитской, и не дал им времени войти в страну его.
\vs 1Ma 12:26 И послал соглядатаев в стан их, которые, возвратившись, объявили ему, что они готовятся напасть на них в эту ночь.
\vs 1Ma 12:27 Посему, когда зашло солнце, Ионафан приказал своим бодрствовать, быть в вооружении и готовиться к сражению всю ночь, и поставил вокруг стана передовых сторожей.
\vs 1Ma 12:28 И услышали неприятели, что Ионафан со своими приготовился к сражению, и устрашились, и затрепетали сердцем своим, и, зажегши огни в стане своем, ушли.
\vs 1Ma 12:29 Ионафан же и бывшие с ним не знали о том до утра, ибо видели горящие огни.
\vs 1Ma 12:30 И погнался Ионафан за ними, но не настиг их, потому что они перешли реку Елевферу.
\vs 1Ma 12:31 Тогда Ионафан обратился на Арабов, называемых Заведеями, поразил их и взял добычу их.
\vs 1Ma 12:32 Потом, возвратившись, пришел в Дамаск и прошел по всей той стране.
\vs 1Ma 12:33 И Симон вышел, и прошел до Аскалона и ближайших крепостей, и обратился в Иоппию, и овладел ею
\vs 1Ma 12:34 ибо он услышал, что \bibemph{Иоппияне} хотят сдать крепость войскам Димитрия,~--- и поставил там стражу, чтобы охранять ее.
\vs 1Ma 12:35 И возвратился Ионафан, и созвал старейшин народа, и советовался с ними, чтобы построить крепости в Иудее,
\vs 1Ma 12:36 возвысить стены Иерусалима и воздвигнуть высокую стену между крепостью и городом, дабы отделить ее от города, так чтобы она была особо и не было бы в ней ни купли, ни продажи.
\vs 1Ma 12:37 Когда собрались устроить город и дошли до стены у потока с восточной стороны, то построили так называемую Хафенафу.
\vs 1Ma 12:38 А Симон построил Адиду в Сефиле и укрепил ворота и запоры.
\rsbpar\vs 1Ma 12:39 Между тем Трифон домогался сделаться царем Азии и возложить на себя венец и поднять руку на царя Антиоха,
\vs 1Ma 12:40 но опасался, как бы не воспрепятствовал ему Ионафан и не начал против него войны; поэтому искал случая, чтобы взять Ионафана и убить, и, поднявшись, пошел в Вефсан.
\vs 1Ma 12:41 И вышел Ионафан навстречу ему с сорока тысячами избранных мужей, готовых к битве, и пришел в Вефсан.
\vs 1Ma 12:42 Когда Трифон увидел, что Ионафан идет с многочисленным войском, то побоялся поднять на него руки.
\vs 1Ma 12:43 И принял его с честью, и представил его всем друзьям своим, дал ему подарки, приказал войскам своим повиноваться ему, как себе самому.
\vs 1Ma 12:44 Потом сказал Ионафану: для чего ты утруждаешь весь этот народ, когда не предстоит нам войны?
\vs 1Ma 12:45 Итак, отпусти их теперь в домы их, а для себя избери немногих мужей, которые были бы с тобою, и пойдем со мною в Птолемаиду, и я передам ее тебе и другие крепости и остальные войска и всех, заведующих сборами, и потом возвращусь; ибо для этого я и нахожусь здесь.
\vs 1Ma 12:46 И поверил ему Ионафан, и сделал так, как он сказал, и отпустил войска, и они отправились в землю Иудейскую;
\vs 1Ma 12:47 с собою же оставил три тысячи мужей, из которых две тысячи оставил в Галилее, тысяча же отправилась с ним.
\vs 1Ma 12:48 Но как скоро вошел Ионафан в Птолемаиду, Птолемаидяне заперли ворота, и схватили его, и всех вошедших с ним убили мечом.
\vs 1Ma 12:49 Тогда Трифон послал войско и конницу в Галилею и на великую равнину, чтобы истребить всех бывших с Ионафаном.
\vs 1Ma 12:50 Но они, услышав, что Ионафан схвачен и погиб и бывшие с ним, ободрили друг друга и вышли густым строем, готовые сразиться.
\vs 1Ma 12:51 И увидели преследующие, что дело идет о жизни, и возвратились назад.
\vs 1Ma 12:52 А они все благополучно пришли в землю Иудейскую и оплакивали Ионафана и бывших с ним, и были в большом страхе, и весь Израиль плакал горьким плачем.
\vs 1Ma 12:53 Тогда все окрестные народы искали истребить их, ибо говорили: теперь нет у них начальника и поборника; итак, будем теперь воевать против них и истребим из среды людей память их.
\vs 1Ma 13:1 Услышал Симон, что Трифон собрал большое войско, чтобы идти в землю Иудейскую и разорить ее.
\vs 1Ma 13:2 И, видя, что народ в страхе и трепете, взошел в Иерусалим и собрал народ.
\vs 1Ma 13:3 И, ободряя их, говорил им: сами вы знаете, сколько я и братья мои и дом отца моего сделали ради этих законов и святыни, знаете войны и угнетения, какие мы испытали.
\vs 1Ma 13:4 Потому и погибли все братья мои за Израиля, и остался я один.
\vs 1Ma 13:5 И ныне да не будет того, чтобы я стал щадить жизнь мою во все время угнетения, ибо я не лучше братьев моих.
\vs 1Ma 13:6 Но буду мстить за народ мой и за святилище, и за жен и за детей наших, ибо соединились все народы, чтобы истребить нас по неприязни.
\vs 1Ma 13:7 И воспламенился дух народа, как только услышал он такие слова;
\vs 1Ma 13:8 и отвечали громким голосом, и сказали: ты~--- наш вождь на место Иуды и Ионафана, брата твоего.
\vs 1Ma 13:9 Веди нашу войну, и, что ты ни скажешь нам, мы всё сделаем.
\vs 1Ma 13:10 Тогда собрал он всех мужей ратных, и поспешил окончить стены Иерусалима, и со всех сторон укрепил его.
\vs 1Ma 13:11 Потом послал Ионафана, сына Авессаломова, и с ним достаточное число войска в Иоппию, и он выгнал бывших в ней и остался там.
\rsbpar\vs 1Ma 13:12 Между тем Трифон поднялся из Птолемаиды с многочисленным войском, чтобы войти в землю Иудейскую; с ним был и Ионафан под стражею.
\vs 1Ma 13:13 Симон же расположил стан при Адиде напротив равнины.
\vs 1Ma 13:14 Когда Трифон узнал, что Симон заступил место Ионафана, брата своего, и намеревается вступить в сражение с ним, то послал к нему послов сказать:
\vs 1Ma 13:15 за серебро, которым брат твой Ионафан задолжал царской казне по надобностям, какие он имел, мы удержали его.
\vs 1Ma 13:16 Итак, пришли теперь сто талантов серебра и в заложники двух сыновей его, чтобы он, быв отпущен, не отложился от нас,~--- и мы отпустим его.
\vs 1Ma 13:17 Симон понимал, что они говорят с ним коварно, но послал серебро и детей, чтобы не навлечь большой ненависти от народа,
\vs 1Ma 13:18 который сказал бы: оттого, что я не послал ему серебра и детей, \bibemph{Ионафан} погиб.
\vs 1Ma 13:19 Итак, послал детей и сто талантов; но Трифон обманул и не отпустил Ионафана.
\vs 1Ma 13:20 После сего Трифон пошел, чтобы войти в страну и разорить ее, и пошел окольным путем на Адару. Но Симон и войско его следовали за ним повсюду, куда он ни шел.
\vs 1Ma 13:21 Бывшие же в крепости послали к Трифону послов, чтобы побудить его прийти к ним чрез пустыню и прислать им съестных припасов.
\vs 1Ma 13:22 И приготовил Трифон всю свою конницу, чтобы идти в ту же ночь, но был очень большой снег, и он не пошел по причине снега, а, поднявшись, отправился в Галаад.
\vs 1Ma 13:23 Когда же приблизился к Васкаме, умертвил Ионафана, и он погребен там.
\vs 1Ma 13:24 И возвратился Трифон и ушел в землю свою.
\vs 1Ma 13:25 Тогда Симон послал и взял кости Ионафана, брата своего, и похоронил их в Модине, городе отцов своих.
\vs 1Ma 13:26 И оплакивал его весь Израиль горьким плачем, и сокрушались о нем многие дни.
\vs 1Ma 13:27 И воздвиг Симон здание над гробом отца своего и братьев своих и вывел его высоко, для благовидности, из тесаного камня с передней и задней стороны,
\vs 1Ma 13:28 и поставил на нем семь пирамид, одну против другой, отцу и матери и четырем братьям;
\vs 1Ma 13:29 сделал на них искусные украшения, поставив вокруг высокие столбы, а на столбах полное вооружение~--- на вечную память, и подле оружий~--- изваянные корабли, так что они были видимы всеми, плавающими по морю.
\vs 1Ma 13:30 Этот надгробный памятник, который сделал он в Модине, стоит до сего дня.
\rsbpar\vs 1Ma 13:31 Трифон же с коварством отправился в путь с юным царем Антиохом и убил его,
\vs 1Ma 13:32 и воцарился вместо него, и возложил на себя венец Азии, и произвел великое поражение на земле.
\vs 1Ma 13:33 А Симон строил крепости в Иудее, укрепляя их высокими башнями и большими стенами, воротами и запорами, и складывал в крепостях съестные запасы.
\vs 1Ma 13:34 Потом избрал Симон мужей и послал к царю Димитрию просить, чтобы он сделал облегчение стране, ибо все деяния Трифона были грабительские.
\vs 1Ma 13:35 И послал ему царь Димитрий ответ на эти слова и написал такое письмо:
\vs 1Ma 13:36 <<Царь Димитрий Симону, первосвященнику и другу царей, и старейшинам и народу Иудейскому~--- радоваться.
\vs 1Ma 13:37 Золотой венец и пальмовую ветвь, посланную вами, мы получили и готовы заключить с вами полный мир и написать заведующим сборами, чтобы отпустить вам дани.
\vs 1Ma 13:38 И всё, что мы постановили о вас, да будет неизменно, и крепости, которые вы построили, пусть принадлежат вам.
\vs 1Ma 13:39 Прощаем вам также неумышленные проступки ваши до сего дня и венечный сбор, который платить вы обязаны, и если другое что взимаемо было в Иерусалиме, более не будет взиматься.
\vs 1Ma 13:40 И если найдутся из вас способные быть вписанными в число состоящих при нас, пусть записываются, и да будет между нами мир>>.
\rsbpar\vs 1Ma 13:41 В сто семидесятом году снято иго язычников с Израиля;
\vs 1Ma 13:42 и народ Израильский в переписке и договорах начал писать: <<Первого года при Симоне, великом первосвященнике, вожде и правителе Иудеев>>.
\vs 1Ma 13:43 В это время Симон сделал нападение на Газу, окружил ее войском, устроил осадные машины и придвинул их к городу, разбил одну башню и овладел ею.
\vs 1Ma 13:44 А бывшие на машине вскочили в город, и произошло в городе великое смятение.
\vs 1Ma 13:45 И взошли граждане с женами и детьми на стену, разодрав одежды свои, и громко взывали, умоляя Симона дать им помилование,
\vs 1Ma 13:46 и говорили: поступи с нами не по злым делам нашим, но по милости твоей.
\vs 1Ma 13:47 И умилосердился над ними Симон, и не сражался с ними, а только выгнал их из города, и очистил домы, в которых находились идолы, и так вошел в город с славословиями и благословениями.
\vs 1Ma 13:48 И выбросил из него все нечистое, и поселил там мужей, соблюдающих закон, и укрепил его, и устроил в нем для себя жилище.
\vs 1Ma 13:49 Бывшим же в Иерусалимской крепости не позволяли ни выходить, ни вступать в страну, ни покупать, ни продавать, и они терпели сильный голод, и многие из них погибли от голода.
\vs 1Ma 13:50 Тогда воззвали они к Симону о мире, и он дал им его, но выгнал их оттуда и очистил крепость от осквернения,
\vs 1Ma 13:51 и взошел в нее в двадцать третий день второго месяца сто семьдесят первого года с славословиями, пальмовыми ветвями, с гуслями, кимвалами и цитрами, с псалмами и песнями, ибо сокрушен великий враг Израиля.
\vs 1Ma 13:52 И установил каждогодно проводить этот день с весельем, и укрепил гору храма, находящуюся близ крепости, и поселился там сам и бывшие с ним.
\vs 1Ma 13:53 И увидел Симон, что сын его Иоанн возмужал, и поставил его начальником над всеми войсками, и поселился в Газаре.
\vs 1Ma 14:1 В сто семьдесят втором году царь Димитрий собрал войска свои и отправился в Мидию, чтобы получить помощь себе для войны против Трифона.
\vs 1Ma 14:2 Но Арсак, царь Персидский и Мидийский, услышав, что Димитрий пришел в пределы его, послал одного из военачальников своих взять его живого.
\vs 1Ma 14:3 Тот отправился и разбил войско Димитрия, взял его и привел к Арсаку, который заключил его в темницу.
\rsbpar\vs 1Ma 14:4 И покоилась земля Иудейская во все дни Симона; он старался о благе народа своего, и нравилась им власть и слава его во все дни.
\vs 1Ma 14:5 И ко всей своей славе, он взял еще Иоппию для пристани и открыл вход островам морским,
\vs 1Ma 14:6 и распространил пределы народа своего, и овладел тою страною.
\vs 1Ma 14:7 Он набрал множество пленных и господствовал над Газарою и Вефсурою и над крепостью, очистил ее от осквернения, и не было противящегося ему.
\vs 1Ma 14:8 \bibemph{Иудеи} спокойно возделывали землю свою, и земля давала произведения свои и дерева в полях~--- плод свой.
\vs 1Ma 14:9 Старцы, сидя на улицах, все совещались о пользах общественных, и юноши облекались в пышные и воинские одежды.
\vs 1Ma 14:10 Городам доставлял он съестные припасы и делал их местами укрепленными, так что славное имя его произносилось до конца земли.
\vs 1Ma 14:11 Он восстановил мир в стране, и радовался Израиль великою радостью.
\vs 1Ma 14:12 И сидел каждый под виноградом своим и под смоковницею своею, и никто не страшил их.
\vs 1Ma 14:13 И не осталось никого на земле, кто воевал бы против них, и цари смирились в те дни.
\vs 1Ma 14:14 Он подкреплял всех бедных в народе своем, требовал исполнения закона и истреблял всякого беззаконника и злодея,
\vs 1Ma 14:15 украсил святилище и умножил священную утварь.
\rsbpar\vs 1Ma 14:16 Когда дошел слух до Рима и до Спарты, что Ионафан умер, они весьма опечалились.
\vs 1Ma 14:17 Когда же услышали, что Симон, брат его, сделался вместо него первосвященником и господствует над страною и находящимися в ней городами,
\vs 1Ma 14:18 то написали к нему на медных досках, чтобы возобновить с ним дружбу и союз, заключенный ими с братьями его Иудою и Ионафаном.
\vs 1Ma 14:19 Они были прочитаны в Иерусалиме пред собранием.
\vs 1Ma 14:20 Вот список с писем, присланных Спартанцами: <<Спартанские начальники и город Симону первосвященнику, старейшинам и священникам и всему народу Иудейскому, братьям нашим~--- радоваться.
\vs 1Ma 14:21 Послы, присланные к народу нашему, рассказали нам о вашей славе и чести, и мы возрадовались прибытию их
\vs 1Ma 14:22 и записали сказанное ими в народном совете так: Нуминий, сын Антиоха, и Антипатр, сын Иасона, послы Иудейские, пришли к нам возобновить с нами дружбу.
\vs 1Ma 14:23 И угодно было народу принять этих мужей с честью и внести запись слов их в открытые народные книги, на память народу Спартанскому. А список с этого мы написали для первосвященника Симона>>.
\rsbpar\vs 1Ma 14:24 После того Симон послал Нуминия в Рим с большим золотым щитом, весом в тысячу мин, чтобы заключить с ними союз.
\vs 1Ma 14:25 Когда услышал об этом народ, то сказал: какую благодарность воздадим мы Симону и сыновьям его?
\vs 1Ma 14:26 Ибо он твердо стоял и братья его и дом отца его, и отразили врагов Израиля, и доставили ему свободу.
\vs 1Ma 14:27 И написали о том на медных досках и выставили их на столбах на горе Сион. Вот список написанного: <<В восемнадцатый день Елула сто семьдесят второго года~--- это был третий год при первосвященнике Симоне~---
\vs 1Ma 14:28 в Сарамели, в великом собрании священников и народа и князей народных и старейшин страны, объявлено нам:
\vs 1Ma 14:29 так как много раз бывали войны в этой стране, то Симон, сын Маттафии, сын сынов Иарива, и братья его, подвергая себя опасности, противостали врагам народа своего, чтобы сохранить святилище его и закон, и великою славою прославили народ свой.
\vs 1Ma 14:30 Ионафан собрал народ свой и сделался первосвященником его, но он приложился к народу своему.
\rsbpar\vs 1Ma 14:31 Когда же враги их вознамерились войти в страну их, чтобы разорить страну их и простереть руки на святилище их,
\vs 1Ma 14:32 тогда восстал Симон и воевал за народ свой и издержал много собственных денег, снабжая храбрых мужей народа своего оружием и давая им жалованье.
\vs 1Ma 14:33 Он укрепил города Иудеи и Вефсуру на границах Иудеи, где прежде находились оружия неприятелей, и поставил там стражу из Иудеев.
\vs 1Ma 14:34 Также укрепил Иоппию при море и Газару на пределах Азота, в которой прежде обитали враги, и поселил там Иудеев, снабдив эти \bibemph{места} всем, что нужно было к восстановлению их.
\vs 1Ma 14:35 И видел народ деяния Симона и славу, какую старался он доставить народу своему, и поставил его своим начальником и первосвященником за то, что все это сделал он, и за справедливость и верность, которую он хранил к племени своему, всячески стараясь возвысить народ свой.
\vs 1Ma 14:36 Во дни его руками его успешно изгнаны из страны язычники и занимавшие город Давидов в Иерусалиме, которые, устроив себе крепость, выходили из нее и оскверняли все вокруг святилища и много вредили святыне.
\vs 1Ma 14:37 Он поселил в ней Иудеев и укрепил ее для безопасности страны и города и возвысил стены Иерусалима.
\vs 1Ma 14:38 Посему и царь Димитрий утвердил за ним первосвященство,
\vs 1Ma 14:39 и причислил его к друзьям своим, и почтил его великою славою.
\vs 1Ma 14:40 Ибо он услышал, что Римляне назвали Иудеев друзьями и союзниками и братьями и с честью приняли послов Симона,
\vs 1Ma 14:41 что Иудеи и священники согласились, чтобы Симон был у них начальником и первосвященником навек, доколе восстанет Пророк верный,
\vs 1Ma 14:42 чтобы он был у них военачальником и имел попечение о святых и поставлял их над работами их, и над областью, и над оружиями, и над крепостями,
\vs 1Ma 14:43 чтобы имел попечение о святилище и все слушались его, чтобы все договоры в стране писались на его имя и чтобы он одевался в порфиру и носил золотые украшения.
\vs 1Ma 14:44 И никому из народа и священников да не будет позволено отменить что-либо из сего или противоречить словам его, или без него созывать собрание в стране и одеваться в порфиру и носить золотую пряжку.
\vs 1Ma 14:45 А кто сделает что-нибудь против сего или отменит что из сего, будет повинен>>.
\vs 1Ma 14:46 И согласился весь народ подчиниться Симону и поступать по словам сим.
\vs 1Ma 14:47 Симон принял и согласился быть первосвященником и военачальником и правителем Иудеев и священников и начальствовать над всеми.
\vs 1Ma 14:48 И решили начертать запись сию на медных досках и поставить их в ограде храма на видном месте,
\vs 1Ma 14:49 а списки с них положить в сокровищнице, чтобы имел их Симон и сыновья его.
\vs 1Ma 15:1 И прислал Антиох, сын царя Димитрия, письма с островов морских к Симону, великому священнику и правителю народа Иудейского, и всему народу.
\vs 1Ma 15:2 Они были такого содержания: <<Царь Антиох Симону, первосвященнику и правителю народа, и народу Иудейскому~--- радоваться.
\vs 1Ma 15:3 Так как люди зловредные овладели царством отцов наших, то я хочу возвратить царство, чтобы восстановить его, как оно было прежде. Я набрал множество войска и приготовил военные корабли;
\vs 1Ma 15:4 и хочу пройти по области, чтобы наказать тех, которые опустошили область нашу и разорили многие города в царстве.
\vs 1Ma 15:5 Оставляю теперь за тобою все дани, какие уступали тебе цари, бывшие прежде меня, и другие дары, какие они уступали тебе;
\vs 1Ma 15:6 дозволяю тебе чеканить свою монету в стране твоей.
\vs 1Ma 15:7 Иерусалим и святилище пусть будут свободны; и все оружия, которые ты заготовил, и крепости, построенные тобою, которыми ты владеешь, пусть остаются у тебя.
\vs 1Ma 15:8 И всякий долг царский и будущие царские долги отныне и навсегда пусть будут отпущены тебе.
\vs 1Ma 15:9 Когда же мы овладеем царством нашим, тогда почтим тебя и народ твой и храм великою честью, чтобы слава ваша стала известна по всей земле>>.
\rsbpar\vs 1Ma 15:10 В сто семьдесят четвертом году вступил Антиох в землю отцов своих, и собрались к нему все войска, так что оставшихся с Трифоном было немного.
\vs 1Ma 15:11 И преследовал его царь Антиох, и он убежал в Дору, которая при море;
\vs 1Ma 15:12 ибо он увидел, что обрушились на него беды и оставили его войска.
\vs 1Ma 15:13 И пришел Антиох к Доре и с ним сто двадцать тысяч воинов и восемь тысяч конницы
\vs 1Ma 15:14 и окружил город, а корабли подошли с моря, и теснил он город с суши и моря, и не давал никому ни выйти, ни войти.
\rsbpar\vs 1Ma 15:15 Тогда пришел из Рима Нуминий и сопровождавшие его с письмами к царям и странам, в которых было написано следующее:
\vs 1Ma 15:16 <<Левкий, консул Римский, царю Птоломею~--- радоваться.
\vs 1Ma 15:17 Пришли к нам Иудейские послы, друзья наши и союзники, посланные от первосвященника Симона и народа Иудейского, возобновить давнюю дружбу и союз,
\vs 1Ma 15:18 и принесли золотой щит в тысячу мин.
\vs 1Ma 15:19 Итак, мы заблагорассудили написать царям и странам, чтобы они не причиняли им зла, и не воевали против них и городов их и страны их, и не помогали воюющим против них.
\vs 1Ma 15:20 Мы рассудили принять от них щит.
\vs 1Ma 15:21 Итак, если какие зловредные люди убежали к вам из страны их, выдайте их первосвященнику Симону, чтобы он наказал их по закону их>>.
\vs 1Ma 15:22 То же самое написал он царю Димитрию и Атталу, Ариарафе и Арсаку,
\vs 1Ma 15:23 и во все области, и Сампсаме и Спартанцам, и в Делос и в Минд, и в Сикион, и в Карию, и в Самос, и в Памфилию, и в Ликию, и в Галикарнасс, и в Родос, и в Фасилиду, и в Кос, и в Сиду, и в Арад, и в Гортину, и в Книду, и в Кипр, и в Киринию.
\vs 1Ma 15:24 Список с этих писем написали Симону первосвященнику.
\rsbpar\vs 1Ma 15:25 Царь же Антиох обложил Дору вторично, нападая на нее со всех сторон и устраивая машины, и запер Трифона так, что невозможно было ему ни войти, ни выйти.
\vs 1Ma 15:26 И послал к нему Симон две тысячи избранных мужей в помощь ему, и серебро и золото, и довольно запасов;
\vs 1Ma 15:27 но он не захотел принять это и отверг все, в чем прежде условился с ним, и отчуждился от него.
\vs 1Ma 15:28 И послал к нему Афиновия, одного из друзей своих, чтобы переговорить с ним и сказать: <<Вы владеете Иоппиею и Газарою и крепостью Иерусалимскою~--- городами царства моего;
\vs 1Ma 15:29 вы опустошили пределы их и произвели великое поражение на земле, и овладели многими местами в царстве моем.
\vs 1Ma 15:30 Итак, отдайте теперь города, которые вы взяли, и дани с тех мест, которыми вы владеете вне пределов Иудейских.
\vs 1Ma 15:31 Если же не так, то дайте за них пятьсот талантов серебра, и за опустошение, которое произвели, и за дани с городов другие пятьсот талантов; а если не дадите, то мы придем и будем сражаться с вами>>.
\vs 1Ma 15:32 И пришел Афиновий, друг царя, в Иерусалим, и когда увидел славу Симона и сокровищницу с золотою и серебряною утварью и окружающее великолепие, то изумился и объявил ему слова царя.
\vs 1Ma 15:33 Симон сказал ему в ответ: мы ни чужой земли не брали, ни господствовали над чужим, но \bibemph{владеем} наследием отцов наших, которое враги наши в одно время неправедно присвоили себе.
\vs 1Ma 15:34 Мы же, улучив время, опять возвратили себе наследие отцов наших.
\vs 1Ma 15:35 Что касается до Иоппии и Газары, которых ты требуешь, то они сами причинили много зла народу в стране нашей; за них мы дадим сто талантов. На это Афиновий ничего не отвечал;
\vs 1Ma 15:36 но, с досадою возвратившись к царю, рассказал ему эти слова и о славе Симона, и о всем, что видел, и царь сильно разгневался.
\vs 1Ma 15:37 Трифон же, сев на корабль, убежал в Орфосиаду.
\vs 1Ma 15:38 Тогда царь, сделав военачальником приморской страны Кендевея, вручил ему пешие и конные войска
\vs 1Ma 15:39 и приказал ему идти войною против Иудеи, приказал ему также построить Кедрон и укрепить ворота, и как воевать с народом; сам же царь погнался за Трифоном.
\vs 1Ma 15:40 И пришел Кендевей в Иамнию, и начал вызывать на бой народ и вторгаться в Иудею и брать народ в плен и убивать;
\vs 1Ma 15:41 и построил Кедрон, и расположил там конницу и войско, чтобы они, выходя оттуда, обходили пути Иудеи, как приказал ему царь.
\vs 1Ma 16:1 И возвратился Иоанн из Газары и рассказал Симону, отцу своему, о том, что делал Кендевей.
\vs 1Ma 16:2 Тогда Симон призвал двух старших сыновей своих, Иуду и Иоанна, и сказал им: я и братья мои и дом отца моего воевали против врагов Израиля от юности до сего дня и много раз успешно спасали руками нашими Израиля.
\vs 1Ma 16:3 Но вот, я состарился, а вы по милости \bibemph{Божией} находитесь в летах зрелых: заступите место мое и брата моего, идите и сражайтесь за народ наш, и да будет с вами помощь небесная.
\vs 1Ma 16:4 И избрал из страны двадцать тысяч воинов и всадников, и пошли они против Кендевея, и ночевали в Модине.
\vs 1Ma 16:5 Встав же утром, вышли на равнину, и вот многочисленное войско навстречу им, пешие и конные, и между ними был поток.
\vs 1Ma 16:6 И двинулся против них сам и народ его, и, видя, что народ боится переходить поток, он перешел первый, и увидели это воины, и перешли за ним.
\vs 1Ma 16:7 И разделил он народ, поставив конных среди пеших; конница же неприятелей была весьма многочисленна.
\vs 1Ma 16:8 И затрубили священными трубами; и Кендевей обратился в бегство и войско его, и пало у них много раненых, остальные же бежали в крепость.
\vs 1Ma 16:9 Тогда был ранен Иуда, брат Иоанна; но Иоанн преследовал их, доколе не пришел в Кедрон, который он построил.
\vs 1Ma 16:10 И убежали они в башни, находящиеся в области Азота, но он сжег его огнем, и погибло из них до двух тысяч мужей; и возвратился он с миром в землю Иудейскую.
\rsbpar\vs 1Ma 16:11 Птоломей же, сын Авува, поставлен был военачальником на равнине Иерихонской и имел много серебра и золота;
\vs 1Ma 16:12 ибо он был зять первосвященника.
\vs 1Ma 16:13 И надмилось сердце его, и захотел он овладеть страною, и делал коварные замыслы против Симона и сыновей его, чтобы погубить их.
\vs 1Ma 16:14 Между тем Симон, посещая города страны и заботясь о потребностях их, пришел в Иерихон, сам и Маттафия и Иуда, сыновья его, в сто семьдесят седьмом году в одиннадцатом месяце~--- это месяц Сават.
\vs 1Ma 16:15 И с коварством принял их радушно сын Авувов в небольшую крепость, называемую Док, им устроенную, и сделал для них большой пир, и спрятал там людей.
\vs 1Ma 16:16 И когда опьянел Симон и сыновья его, тогда встал Птоломей и бывшие при нем, взяли оружия свои и вошли к Симону во время пира и убили его и двух сыновей его и некоторых из служителей его.
\vs 1Ma 16:17 Так совершил он великое вероломство и воздал за добро злом.
\vs 1Ma 16:18 Птоломей написал об этом и послал к царю, чтобы прислал ему войско на помощь, и он предаст ему страну их и города.
\vs 1Ma 16:19 И некоторых послал в Газару убить Иоанна, а тысяченачальникам послал письма, чтобы они пришли к нему, и он даст им серебра и золота и подарки;
\vs 1Ma 16:20 а других послал овладеть Иерусалимом и горою храма.
\vs 1Ma 16:21 Но некто, прибежав к Иоанну в Газару, известил его, что отец его и братья умерщвлены и что \bibemph{Птоломей} послал убить и его.
\vs 1Ma 16:22 Услышав об этом, \bibemph{Иоанн} весьма смутился и, схватив мужей, пришедших погубить его, убил их, ибо узнал, что они искали погубить его.
\rsbpar\vs 1Ma 16:23 Прочие же дела Иоанна и в\acc{о}йны его и мужественные подвиги его, славно совершенные, и сооружение стен, им воздвигнутых, и другие деяния его,
\vs 1Ma 16:24 вот, они описаны в книге дней первосвященства его, с того времени, как сделался он первосвященником после отца своего.
\newbookpage
\bibbookdescr{2Ma}{
  inline={\LARGE Вторая книга\\\Huge Маккавейская\fns{Книги Маккавейские переведены с греческого, потому что в еврейском тексте их нет.}},
  toc={2-я Маккавейская*},
  bookmark={2-я Маккавейская},
  header={2-я Маккавейская},
  %headerleft={},
  %headerright={},
  abbr={2~Мак}
}
\vs 2Ma 1:1 Братьям Иудеям в Египте~--- радоваться; братья Иудеи в Иерусалиме и во всей стране Иудейской желают счастливого мира.
\vs 2Ma 1:2 Да благодетельствует вам Бог и да помянет завет Свой с верными рабами Своими: Авраамом, Исааком и Иаковом!
\vs 2Ma 1:3 Да даст всем вам сердце, чтобы чтить Его и исполнять волю Его всем сердцем и усердною душею!
\vs 2Ma 1:4 Да откроет сердце ваше для закона Его и повелений и дарует мир!
\vs 2Ma 1:5 Да услышит моления ваши и да будет милостив к вам, и да не оставит вас во время бедствия!
\vs 2Ma 1:6 Так ныне здесь мы молимся о вас.
\rsbpar\vs 2Ma 1:7 В царствование Димитрия, в сто шестьдесят девятом году, мы, Иудеи, писали к вам в скорби и страданиях, постигших нас в те годы, как отложился Иасон и соумышленники его от святой земли и царства.
\vs 2Ma 1:8 Они сожгли ворота и пролили невинную кровь. Тогда мы молились Господу и были услышаны, и приносили жертву и семидал, и возжигали светильники, и предлагали хлебы.
\vs 2Ma 1:9 И ныне совершайте праздник кущей в месяце Хаслеве.
\rsbpar\vs 2Ma 1:10 В сто восемьдесят восьмом году живущие в Иерусалиме и в Иудее, и старейшины и Иуда~--- Аристовулу, учителю царя Птоломея, происходящему из рода помазанных священников, и пребывающим в Египте Иудеям~--- радоваться и здравствовать.
\vs 2Ma 1:11 Избавленные Богом от великих опасностей, мы торжественно благодарим Его, как бы сражавшиеся против царя,
\vs 2Ma 1:12 так как Он изгнал ополчившихся на святый град.
\vs 2Ma 1:13 Ибо когда царь пошел в Персию и с ним войско, которое казалось непобедимым, они поражены были в храме Нанеи через обман, употребленный жрецами Нанеи.
\vs 2Ma 1:14 Именно, когда Антиох, как бы намереваясь сочетаться с нею, пришел на то место, а бывшие с ним друзья пришли взять деньги как приданое,
\vs 2Ma 1:15 и жрецы Нанеи предложили их, и Антиох с немногими вошел во внутренность храма,~--- тогда они заключили храм, как только вошел Антиох,
\vs 2Ma 1:16 и, отворив потаенное отверстие в своде, стали бросать камни, и поразили предводителя и бывших с ним, и, рассекши на части и отрубив головы, выбросили их к находившимся снаружи.
\vs 2Ma 1:17 Во всем благословен Бог наш, предавший нечестивцев.
\vs 2Ma 1:18 Итак, намереваясь в двадцать пятый день Хаслева праздновать очищение храма, мы почли нужным известить вас, чтобы и вы совершили праздник кущей и огня, подобно тому как Неемия, построив храм и жертвенник, принес жертву.
\vs 2Ma 1:19 Ибо, когда отцы наши отведены были в Персию, тогда благочестивые священники, взяв огня с жертвенника тайно, скрыли его во глубине колодезя, имевшего безводное дно, и в нем безопасно сохранили его, так как никому не известно было это место.
\rsbpar\vs 2Ma 1:20 По прошествии же многих лет, когда угодно было Богу, Неемия, присланный от Персидского царя, послал за сим огнем потомков тех священников, которые скрыли его. Когда же объявили нам, что не нашли огня, а только густую воду,
\vs 2Ma 1:21 тогда он приказал им, почерпнув, принести ее; и когда потом приготовлены были жертвы, Неемия приказал священникам окропить этою водою дрова и положенное на них.
\vs 2Ma 1:22 Когда же это было сделано и наступило время, когда просияло солнце, прежде закрытое облаками, тогда воспламенился большой огонь, так что все удивились.
\vs 2Ma 1:23 Священники же, доколе горела жертва, совершали молитву, священники и все; Ионафан начинал, а прочие припевали, как и Неемия.
\vs 2Ma 1:24 Молитва же была такая: <<Господи, Господи Боже, Создателю всех, страшный и сильный, и праведный и милостивый, единый Царь и благодетель,
\vs 2Ma 1:25 единый податель всего, единый праведный и всемогущий и вечный, избавляющий Израиля от всякого зла, избравший отцов и освятивший их!
\vs 2Ma 1:26 Прими жертву сию за весь народ Твой~--- Израиля, и сохрани сей удел Твой, и освяти его;
\vs 2Ma 1:27 собери рассеяние наше, освободи порабощенных язычниками, призри на уничиженных и презренных, и да познают язычники, что Ты Бог наш;
\vs 2Ma 1:28 покарай угнетающих и обижающих нас с надмением,
\vs 2Ma 1:29 насади народ Твой на святом месте Твоем, как сказал Моисей>>.
\vs 2Ma 1:30 Священники воспевали при сем торжественные песни.
\vs 2Ma 1:31 Когда же жертва была сожжена, Неемия приказал оставшеюся водою полить большие камни.
\vs 2Ma 1:32 Как только это было исполнено, вспыхнуло пламя, но от света, воссиявшего от жертвенника, оно исчезло.
\vs 2Ma 1:33 Когда это событие сделалось известным и донесено было царю Персов, что в том месте, где переселенные священники скрыли огонь, оказалась вода, которою Неемия и бывшие с ним освятили жертвы;
\vs 2Ma 1:34 царь, по исследовании дела, оградил это место, как священное.
\vs 2Ma 1:35 И тем, к кому царь благоволил, он раздавал много даров, которые сам получал.
\vs 2Ma 1:36 Бывшие с Неемиею прозвали это место Нефтар, что значит: <<очищение>>; многими же называется оно Нефтай.
\vs 2Ma 2:1 В записях пророка Иеремии находится, что он приказал переселяемым взять от огня, как показано
\vs 2Ma 2:2 и как заповедал пророк, дав переселяемым закон, чтобы они не забывали повелений Господних и не заблуждались мыслями своими, смотря на золотые и серебряные кумиры и на украшение их.
\vs 2Ma 2:3 Говоря и другое, подобное сему, он увещевал их не удалять закона из сердца своего.
\vs 2Ma 2:4 Было также в писании, что сей пророк, по бывшему ему Божественному откровению, повелел скинии и ковчегу следовать за ним, когда он восходил на гору, с которой Моисей, взойдя, видел наследие Божие.
\vs 2Ma 2:5 Придя туда, Иеремия нашел жилище в пещере и внес туда скинию и ковчег и жертвенник кадильный, и заградил вход.
\vs 2Ma 2:6 Когда потом пришли некоторые из сопутствовавших, чтобы заметить вход, то не могли найти его.
\vs 2Ma 2:7 Когда же Иеремия узнал о сем, то, упрекая их, сказал, что это место останется неизвестным, доколе Бог, умилосердившись, не соберет сонма народа.
\vs 2Ma 2:8 И тогда Господь покажет его, и явится слава Господня и облако, как явилось при Моисее, как и Соломон просил, чтобы особенно святилось место.
\vs 2Ma 2:9 Было сказано и то, как он, исполненный премудрости, принес жертву обновления и совершения храма.
\vs 2Ma 2:10 Как Моисей молился Господу, и сошел огонь с неба, и потребил жертву, так и Соломон молился, и сошедший огонь истребил жертвы всесожжения.
\vs 2Ma 2:11 И сказал Моисей: так как жертва о грехе не употреблена в пищу, то потреблена огнем.
\vs 2Ma 2:12 Точно так и Соломон торжествовал восемь дней.
\rsbpar\vs 2Ma 2:13 Повествуется также в записях и памятных книгах Неемии, как он, составляя библиотеку, собрал сказания о царях и пророках и о Давиде и письма царей о священных приношениях.
\vs 2Ma 2:14 Подобным образом и Иуда затерянное, по случаю бывшей у нас войны, всё собрал, и оно есть у нас.
\vs 2Ma 2:15 Итак, если вы имеете в этом надобность, пришлите людей, которые вам доставят.
\vs 2Ma 2:16 Намереваясь праздновать очищение, мы писали вам об этом; хорошо сделаете и вы, если будете праздновать эти дни.
\vs 2Ma 2:17 Бог же, сохранивший весь народ Свой и возвративший всем наследие и царство и священство и святилище,
\vs 2Ma 2:18 как обещал в законе,~--- надеемся на Бога,~--- Он скоро помилует нас и соберет от поднебесной в место святое.
\vs 2Ma 2:19 Ибо Он избавил нас от великих бед и очистил место.
\vs 2Ma 2:20 О делах же Иуды Маккавея и братьев его и об очищении великого храма и обновлении жертвенника,
\vs 2Ma 2:21 также о войнах против Антиоха Епифана и против сына его Евпатора,
\vs 2Ma 2:22 и о бывших с неба явлениях тем, которые подвизались за Иудеев столь ревностно, что, быв весьма малочисленны, очищали всю страну и преследовали многочисленные толпы неприятелей,
\vs 2Ma 2:23 и воссоздали славный во всей вселенной храм, и освободили город, и восстановили клонившиеся к разрушению законы, когда Господь с великим снисхождением умилосердился над ними;
\vs 2Ma 2:24 о всем этом изложенное Иасоном Киринейским в пяти книгах мы попытаемся кратко начертать в одной книге.
\vs 2Ma 2:25 Ибо, имея в виду множество чисел и трудность, происходящую от обилия содержания, для желающих заняться историческими повествованиями,
\vs 2Ma 2:26 мы озаботились доставить душевное назидание желающим читать, облегчение старающимся удержать в памяти и всем, кому случится читать, пользу;
\vs 2Ma 2:27 хотя для нас, принявших на себя труд сокращения, это нелегкое дело, требующее напряжения и бдительности,
\vs 2Ma 2:28 как нелегко бывает тому, кто готовит пиршество и желает пользы другим. Но, имея в виду благодарность многих, мы охотно принимаем на себя этот труд,
\vs 2Ma 2:29 предоставляя точное изложение подробностей историку и стараясь последовать примерам сокращенного изложения.
\vs 2Ma 2:30 Ибо как строителю нового дома предлежит заботиться обо всем строении, а тому, кто должен заняться резьбою и живописью, надлежит изыскивать только потребное к украшению, так мы думаем и о себе.
\vs 2Ma 2:31 Углубляться и говорить обо всем и исследовать каждую частность свойственно начальному писателю истории.
\vs 2Ma 2:32 Тому же, кто делает сокращение, должно быть предоставлено преследовать только краткость речи и избегать подробных изысканий.
\vs 2Ma 2:33 Итак, в связи с сказанным, начнем теперь повествование: ибо неразумно увеличивать предисловие к истории, а самую историю сокращать.
\vs 2Ma 3:1 Когда в святом граде жили еще в полном мире и тщательно соблюдались законы, по благочестию и отвращению от зла первосвященника Онии,
\vs 2Ma 3:2 бывало, и сами цари чтили это место, и прославляли святилище отличными дарами,
\vs 2Ma 3:3 так что и Селевк, царь Азии, давал из своих доходов на все издержки, потребные для жертвенного служения.
\vs 2Ma 3:4 Но некто Симон из колена Вениаминова, поставленный попечителем храма, вошел в спор с первосвященником о нарушении законов в городе.
\vs 2Ma 3:5 И как он не мог превозмочь Онии, то пошел к Аполлонию, сыну Фрасея, который в то время был военачальником Келе-Сирии и Финикии,
\vs 2Ma 3:6 и объявил ему, что Иерусалимская сокровищница наполнена несметными богатствами, равно как несчетное множество денег скоплено, и нет в них нужды для приношения жертв, но все это может быть обращено во власть царя.
\vs 2Ma 3:7 Аполлоний же, увидевшись с царем, объявил ему об означенных богатствах, а он, назначив Илиодора, поставленного над государственными делами, послал его и дал приказ вывезти упомянутые сокровища.
\vs 2Ma 3:8 Илиодор тотчас отправился в путь, под предлогом обозрения городов Келе-Сирии и Финикии, а на самом деле для того, чтобы исполнить волю царя.
\vs 2Ma 3:9 Прибыв же в Иерусалим и быв дружелюбно принят первосвященником города, он сообщил ему о сделанном указании и объявил, за чем пришел, притом спрашивал: действительно ли все это так?
\vs 2Ma 3:10 Хотя первосвященник показал, что это есть вверенное на сохранение имущество вдов и сирот
\vs 2Ma 3:11 и частью Гиркана, сына Товии, мужа весьма знаменитого, а не так, как клеветал нечестивый Симон, и что всего четыреста талантов серебра и двести золота;
\vs 2Ma 3:12 обижать же положившихся на святость места, на уважение и неприкосновенность храма, чтимого во всей вселенной, никак не следует.
\vs 2Ma 3:13 Но Илиодор, имея царский приказ, решительно говорил, что это должно быть взято в царское казнохранилище.
\rsbpar\vs 2Ma 3:14 Назначив день, он вошел, чтобы сделать осмотр этого, и произошло немалое волнение во всем городе.
\vs 2Ma 3:15 Священники в священных одеждах, повергшись пред жертвенником, взывали на небо, чтобы Тот, Который дал закон о вверяемом святилищу имуществе, в целости сохранил его вверившим.
\vs 2Ma 3:16 Кто смотрел на лице первосвященника, испытывал душевное потрясение; ибо взгляд его и изменившийся цвет лица обличал в нем душевное смущение.
\vs 2Ma 3:17 Его объял ужас и дрожание тела, из чего явна была смотревшим скорбь его сердца.
\vs 2Ma 3:18 Иные семьями выбегали из домов на всенародное моление, ибо предстояло священному месту испытать поругание;
\vs 2Ma 3:19 женщины, опоясав грудь вретищами, толпами ходили по улицам; уединенные девы иные бежали к воротам, другие~--- на стены, а иные смотрели из окон,
\vs 2Ma 3:20 все же, простирая к небу руки, молились.
\vs 2Ma 3:21 Трогательно было, как народ толпами бросался ниц, а сильно смущенный первосвященник стоял в ожидании.
\vs 2Ma 3:22 Они умоляли Вседержителя Бога вверенное сохранить в целости вверившим.
\vs 2Ma 3:23 А Илиодор исполнял предположенное.
\vs 2Ma 3:24 Когда же он с вооруженными людьми вошел уже в сокровищницу, Господь отцов и Владыка всякой власти явил великое знамение: все, дерзнувшие войти с ним, быв поражены страхом силы Божией, пришли в изнеможение и ужас,
\vs 2Ma 3:25 ибо явился им конь со страшным всадником, покрытый прекрасным покровом: быстро несясь, он поразил Илиодора передними копытами, а сидевший на нем, казалось, имел золотое всеоружие.
\vs 2Ma 3:26 Явились ему и еще другие два юноши, цветущие силою, прекрасные видом, благолепно одетые, которые, став с той и другой стороны, непрерывно бичевали его, налагая ему многие раны.
\vs 2Ma 3:27 Когда он внезапно упал на землю и объят был великою тьмою, тогда подняли его и положили на носилки.
\vs 2Ma 3:28 Того, который с большою свитою и телохранителями только что вошел в означенную сокровищницу, вынесли как беспомощного, ясно познав всемогущество Божие.
\vs 2Ma 3:29 Божественною силою он повергнут был безгласным и лишенным всякой надежды и спасения.
\vs 2Ma 3:30 Они же благословляли Господа, прославившего Свое жилище; и храм, который незадолго пред тем наполнен был страхом и смущением, явлением Господа Вседержителя наполнился радостью и веселием.
\vs 2Ma 3:31 Вскоре некоторые из близких Илиодора, \bibemph{придя}, умоляли Онию призвать Всевышнего и даровать жизнь лежавшему уже при последнем издыхании.
\vs 2Ma 3:32 Первосвященник, опасаясь, чтобы царь не подумал, что сделано Иудеями какое-нибудь злоумышление против Илиодора, принес жертву о его спасении.
\vs 2Ma 3:33 Когда же первосвященник приносил умилостивительную жертву, те же юноши опять явились Илиодору, украшенные теми же одеждами, и, представ, сказали ему: воздай великую благодарность первосвященнику Онии, ибо для него Господь даровал тебе жизнь;
\vs 2Ma 3:34 ты же, наказанный от Него, возвещай всем великую силу Бога. И, сказав сие, они стали невидимы.
\vs 2Ma 3:35 Илиодор же, принеся жертву Господу, и обещав многие обеты Сохранившему ему жизнь, и возблагодарив Онию, возвратился с воинами к царю
\vs 2Ma 3:36 и пред всеми свидетельствовал о делах великого Бога, которые он видел своими глазами.
\vs 2Ma 3:37 Когда же царь спросил Илиодора, кто был бы способен, чтобы еще раз послать в Иерусалим, он отвечал:
\vs 2Ma 3:38 если ты имеешь какого-нибудь врага и противника твоему правлению, то пошли его туда, и встретишь его наказанным, если только останется он в живых, ибо на месте сем истинно пребывает сила Божия:
\vs 2Ma 3:39 Он Сам, обитающий на небе, есть страж и заступник того места и приходящих с злым намерением поражает и умерщвляет.
\vs 2Ma 3:40 Вот что произошло с Илиодором, и так спасена сокровищница храма.
\vs 2Ma 4:1 А выше упоминаемый Симон, сделавшись предателем сокровищ и отечества, клеветал на Онию, будто он сам поощрял Илиодора и был виновником зол.
\vs 2Ma 4:2 Благодетеля города, попечителя о соплеменниках и ревнителя законов дерзал он называть противником правительства.
\vs 2Ma 4:3 Когда же вражда дошла до того, что чрез одного из доверенных людей Симона стали совершаться убийства,
\vs 2Ma 4:4 тогда Ония, видя, что борьба опасна, что Аполлоний, как военачальник Келе-Сирии и Финикии, неистовствует, увеличивая злобу Симона,
\vs 2Ma 4:5 отправился к царю, не как обвинитель сограждан, но имея в виду пользу каждого и всего народа,
\vs 2Ma 4:6 ибо он видел, что без царской попечительности невозможно мирно устроить дела и Симон не оставит своего безумия.
\vs 2Ma 4:7 Но когда умер Селевк и получил царство Антиох, по прозванию Епифан, тогда домогался священноначалия Иасон, брат Онии,
\vs 2Ma 4:8 обещав царю при свидании триста шестьдесят талантов серебра и с некоторых доходов восемьдесят талантов.
\vs 2Ma 4:9 Сверх того обещал и еще подписать сто пятьдесят талантов, если предоставлено ему будет властью его устроить училище для телесного упражнения юношей и писать Иерусалимлян Антиохиянами.
\vs 2Ma 4:10 Когда царь дал согласие и он получил власть, тотчас начал склонять одноплеменников своих к Еллинским нравам.
\vs 2Ma 4:11 Он отверг человеколюбиво предоставленные Иудеям царские льготы по ходатайству Иоанна, отца Евполемова, который предпринимал посольство к Римлянам о дружбе и союзе; нарушая законные учреждения, он вводил противные закону обычаи.
\vs 2Ma 4:12 Намеренно под самою крепостью построил он училище для телесного упражнения юношей и, привлекши лучших из юношей, подводил их под срамную покрышку.
\vs 2Ma 4:13 Так явилась склонность к Еллинизму и сближение с иноплеменничеством вследствие непомерного нечестия Иасона, этого безбожника, а не первосвященника,
\vs 2Ma 4:14 так что священники перестали быть ревностными к служению жертвеннику и, презирая храм и нерадя о жертвах, спешили принимать участие в противных закону играх палестры по призыву бросаемого диска.
\vs 2Ma 4:15 Ни во что ставили они отечественный почет; только Еллинские почести признавали наилучшими.
\vs 2Ma 4:16 За это постигло их тяжкое посещение, и те самые, которым они соревновали в образе жизни и хотели во всем уподобиться, стали их врагами и мучителями;
\vs 2Ma 4:17 ибо нечестиво поступать против Божественных законов невозможно ненаказанно, как показывает наступающее за тем время.
\vs 2Ma 4:18 Когда праздновались в Тире пятилетние игры и царь присутствовал там,
\vs 2Ma 4:19 тогда нечестивый Иасон послал туда зрителями Антиохиян из Иерусалима, чтобы доставить триста драхм серебра на жертву Геркулесу; но сами принесшие просили не употреблять их на жертву, считая это неприличным, а назначить на другие расходы:
\vs 2Ma 4:20 итак, им посланы эти деньги в жертву Геркулесу от имени посылавшего, а принесшими они обращены на устройство гребных судов.
\vs 2Ma 4:21 Когда затем Аполлоний, сын Менесфея, послан был в Египет по случаю восшествия на престол царя Птоломея Филометора, Антиох заподозрил его враждебным себе и начал стараться обезопасить себя против него; посему, отправившись в Иоппию, он пришел в Иерусалим.
\vs 2Ma 4:22 Великолепно принятый Иасоном и городом, он вошел при светильниках и восклицаниях и оттуда отправился с войском в Финикию.
\vs 2Ma 4:23 По прошествии трех лет Иасон послал Менелая, брата вышеозначенного Симона, чтобы он доставил царю деньги и сделал представление о некоторых нужных делах.
\vs 2Ma 4:24 Он же, представившись царю и польстив его власти, восхитил себе священноначалие, надбавив триста талантов серебра против Иасона.
\vs 2Ma 4:25 Получив от царя приказания, он возвратился, не принеся с собою ничего достойного первосвященства, а только гнев жестокого тирана и ярость дикого зверя.
\vs 2Ma 4:26 Так Иасон, обманувший своего брата, сам был обманут другим и, как изгнанник, удалился в страну Аммонитскую.
\vs 2Ma 4:27 Менелай же получил власть, но нисколько не заботился об обещанных царю деньгах, хотя Сострат, начальник городской крепости, и делал требования,
\vs 2Ma 4:28 ибо на нем лежал сбор даней; по этой причине оба они были вызваны царем.
\vs 2Ma 4:29 Менелай оставил преемником первосвященства брата своего, Лисимаха, а Сострат~--- Кратита, начальника Кипрян.
\rsbpar\vs 2Ma 4:30 В то время, как это происходило, взбунтовались Тарсяне и Маллоты за то, что они отданы были в дар Антиохиде, наложнице царской.
\vs 2Ma 4:31 Посему царь поспешно отправился, чтобы привести дела в порядок, оставив вместо себя Андроника, одного из почетных сановников.
\vs 2Ma 4:32 Тогда Менелай, думая воспользоваться благоприятным случаем, похитил из храма некоторые золотые сосуды и подарил Андронику, а другие продал в Тире и окрестных городах.
\vs 2Ma 4:33 Верно дознав о том, Ония изобличил его и удалился в безопасное место~--- Дафну, лежащую при Антиохии.
\vs 2Ma 4:34 Посему Менелай, улучив наедине Андроника, просил его убить Онию; и он, придя к Онии и коварно уверив его, дав руку с клятвою, хотя и был в подозрении, убедил его выйти из убежища и тотчас убил, не устыдившись правды.
\vs 2Ma 4:35 Этим раздражены были не только Иудеи, но и многие из других народов, и негодовали на беззаконное убийство этого мужа.
\vs 2Ma 4:36 Когда же царь возвратился из стран Киликии, то бывшие в городе Иудеи с вознегодовавшими Еллинами донесли ему, что Ония убит безвинно.
\vs 2Ma 4:37 Антиох, душевно огорченный и тронутый сожалением, оплакивал добродетель и великое благочиние умершего
\vs 2Ma 4:38 и в гневе на Андроника, тотчас совлекши с него порфиру и изодрав одежды, приказал водить его по всему городу и на том самом месте, где он злодейски погубил Онию, казнить убийцу, чем Господь воздал ему заслуженное наказание.
\rsbpar\vs 2Ma 4:39 Когда же в городе были произведены многие святотатства Лисимахом, с соизволения Менелая, и разнесся о том слух, то народ восстал на Лисимаха, ибо похищено было множество золотых сосудов.
\vs 2Ma 4:40 Когда восстал народ, исполненный гнева, то Лисимах вооружил до трех тысяч человек и начал беззаконное насилие под предводительством одного тирана, старого летами и не менее застаревшего в безумии.
\vs 2Ma 4:41 Увидев такое насилие Лисимаха, одни схватили камни, другие~--- толстые колья, а иные, хватая с земли пыль, бросали все вместе на людей Лисимаха
\vs 2Ma 4:42 и таким образом многих из них ранили, других поразили и всех обратили в бегство, а самого святотатца умертвили близ сокровищницы.
\vs 2Ma 4:43 Об этом состоялся суд над Менелаем.
\vs 2Ma 4:44 Когда царь прибыл в Тир, то посланные от собрания старейшин три мужа представили ему жалобу.
\vs 2Ma 4:45 Менелай, уже взятый, обещал Птоломею, сыну Дорименову, большие деньги, если он упросит за него царя.
\vs 2Ma 4:46 И Птоломей, отозвав царя в притвор под предлогом отдохновения, извратил дело.
\vs 2Ma 4:47 Менелая, виновника всего зла, освободил от обвинений, а несчастных, которые, если бы и пред Скифами говорили, были бы отпущены неосужденными, осудил на смерть.
\vs 2Ma 4:48 Так скоро понесли неправедную казнь говорившие в защиту города, народа и священных сосудов.
\vs 2Ma 4:49 Тиряне, негодуя на то, щедро доставили потребное для погребения их.
\vs 2Ma 4:50 А Менелай, при любостяжании начальствующих, удержал за собою власть и, возрастая в злобе, сделался жестоким врагом граждан.
\vs 2Ma 5:1 Около этого времени Антиох предпринял другой поход в Египет.
\vs 2Ma 5:2 Случилось, что над всем городом почти в продолжение сорока дней являлись в воздухе носившиеся всадники в золотых одеждах и наподобие воинов вооруженные копьями,
\vs 2Ma 5:3 и стройные отряды конницы, и нападения и отступления с обеих сторон, обращение щитов, множество копьев и взмахи мечей, бросание стрел и блеск золотых доспехов и всякого рода вооружения.
\vs 2Ma 5:4 Почему все молились, чтобы это явление было ко благу.
\rsbpar\vs 2Ma 5:5 Когда потом разнесся ложный слух, будто Антиох умер, Иасон, собрав не менее тысячи мужей, сделал внезапное нападение на город; когда они взошли на стену и наконец город был взят, Менелай убежал в крепость.
\vs 2Ma 5:6 А Иасон нещадно производил кровопролитие между своими согражданами, не размышляя о том, что успех против одноплеменников есть величайшее несчастье, и воображая получить трофеи как бы над врагами, а не одноплеменными.
\vs 2Ma 5:7 Впрочем, он не достиг начальства, а концом его злоумышлений было то, что он с позором, как беглец, опять ушел в страну Аммонитскую.
\vs 2Ma 5:8 Концом его злобной жизни было то, что, обвиненный пред Аретою, владетелем Аравийским, он бегал из города в город, всеми преследуемый и ненавидимый, как отступник от законов, и, презираемый, как враг отечества и сограждан, был изгнан в Египет.
\vs 2Ma 5:9 Тот, который столь многих изгнал из отечества, сам погиб на чужой стороне, придя к Лакедемонянам и надеясь, по сродству происхождения, найти у них прибежище.
\vs 2Ma 5:10 Оставивший многих без погребения, он сам остался неоплаканным, и не удостоен ни погребения, ни отеческого гроба.
\rsbpar\vs 2Ma 5:11 Когда все происшедшее дошло до слуха царя, он подумал, что Иудея отлагается от него, поднялся из Египта, рассвирепев в душе, и взял город вооруженною рукою.
\vs 2Ma 5:12 Он приказал воинам нещадно бить всех, кто попадется, и умерщвлять, кто станет скрываться в домы.
\vs 2Ma 5:13 Так совершилось избиение юных и старых, умерщвление мужей, жен и детей, заклание дев и младенцев.
\vs 2Ma 5:14 В продолжение трех дней погибло восемьдесят тысяч: сорок тысяч пало от руки убийц, и не меньше убитых было продано.
\vs 2Ma 5:15 Но, не удовольствовавшись этим, он дерзнул войти в святейший на всей земле храм, имея проводником Менелая, этого предателя законов и отечества.
\vs 2Ma 5:16 Скверными руками принимая священные сосуды и иные вещи, пожертвованные от других царей на возвеличение и славу и честь святаго места, восхищая нечестивыми руками, раздавал.
\vs 2Ma 5:17 И превознесся Антиох в своих мыслях, не разумея, что Господь на краткое время прогневался за грехи обитающих в городе, почему и осталось без призрения это место.
\vs 2Ma 5:18 Если бы они не были объяты многими грехами, тогда, подобно Илиодору, посланному царем Селевком осмотреть сокровищницу, и он, лишь только бы вторгся, тотчас был бы наказан и оставил бы свою дерзость.
\vs 2Ma 5:19 Но Господь избрал не для места народ, а для народа это место.
\vs 2Ma 5:20 Посему и самое место, сделавшись причастным бывшим народным несчастьям, приобщилось потом благодеяний Господа и, быв оставлено Всемогущим во гневе, опять, с умилостивлением верховного Владыки, восстало во всей славе.
\rsbpar\vs 2Ma 5:21 Итак, Антиох, похитив из храма тысячу восемьсот талантов, поспешно удалился в Антиохию, в превозношении сердца находя возможным сделать землю судоходною и море сухопутным.
\vs 2Ma 5:22 Между тем он оставил приставников, чтобы угнетать народ, в Иерусалиме~--- Филиппа, родом Фригийца, нравом же человека еще более жестокого, нежели каков был поставивший его,
\vs 2Ma 5:23 а в Гаризине~--- Андроника и сверх того Менелая, который превзошел прочих злобою к жителям и имел враждебное расположение к гражданам Иудейским.
\vs 2Ma 5:24 Он послал виновника нечестия, Аполлония, с двадцатью двумя тысячами войска, повелев всех взрослых избить, а женщин и детей продавать.
\vs 2Ma 5:25 Он же, придя в Иерусалим и притворно храня мир, медлил до святаго дня субботы и, застигнув Иудеев во время покоя, велел своим людям вооружиться.
\vs 2Ma 5:26 Всех, вышедших на это зрелище, он умертвил и, вторгшись с войском в город, избил множество народа.
\vs 2Ma 5:27 А Иуда Маккавей, десятый в роде своем, удалился в пустыню и жил со своими приверженцами в горах по подобию зверей, питаясь травами, чтобы не сделаться причастным осквернения.
\vs 2Ma 6:1 Спустя немного времени царь послал одного старца, Афинянина, принуждать Иудеев отступить от законов отеческих и не жить по законам Божиим,
\vs 2Ma 6:2 а также осквернить храм Иерусалимский и наименовать его храмом Юпитера Олимпийского, а храм в Гаризине, так как обитатели того места пришельцы,~--- храмом Юпитера Странноприимного.
\vs 2Ma 6:3 Тяжело и невыносимо было для народа наступившее бедствие.
\vs 2Ma 6:4 Храм наполнился любодейством и бесчинием от язычников, которые, обращаясь с блудницами, смешивались с женщинами в самых священных притворах и вносили внутрь вещи недозволенные.
\vs 2Ma 6:5 И жертвенник наполнился непотребными, запрещенными законом вещами.
\vs 2Ma 6:6 Нельзя было ни хранить субботы, ни соблюдать отеческих праздников, ни даже называться Иудеем.
\vs 2Ma 6:7 С тяжким принуждением водили их каждый месяц в день рождения царя на идольские жертвы, а на празднике Диониса принуждали Иудеев в плющевых венках идти в торжественном ходе в честь Диониса.
\vs 2Ma 6:8 Такое повеление вышло и соседним Еллинским городам, по наущению Птоломея, чтобы они так же действовали против Иудеев и заставляли их приносить идольские жертвы,
\vs 2Ma 6:9 а не соглашавшихся переходить к Еллинским обычаям убивали. Тогда-то можно было видеть настоящее бедствие.
\vs 2Ma 6:10 Две женщины обвинены были в том, что обрезали своих детей; и за это, привесив к сосцам их младенцев и пред народом проведя по городу, низвергли их со стены.
\vs 2Ma 6:11 Другие бежали в ближние пещеры, чтобы втайне праздновать седьмой день, но, быв указаны Филиппу, были сожжены, ибо неправедным считали защищаться по уважению к святости дня.
\rsbpar\vs 2Ma 6:12 Тех, кому случится читать эту книгу, прошу не страшиться напастей и уразуметь, что эти страдания служат не к погублению, а к вразумлению рода нашего.
\vs 2Ma 6:13 Ибо то самое, что нечестивцам не дается много времени, но скоро подвергаются они карам, есть знамение великого благодеяния.
\vs 2Ma 6:14 Ибо не так, как к другим народам, продолжает Господь долготерпение, чтобы карать их, когда они достигнут полноты грехов, не так судил Он о нас,
\vs 2Ma 6:15 чтобы покарать нас после, когда уже достигнем до конца грехов.
\vs 2Ma 6:16 Он никогда не удаляет от нас Своей милости и, наказывая несчастьями, не оставляет Своего народа.
\vs 2Ma 6:17 Впрочем, пусть будет это сказано на память нам: после этих немногих слов возвратимся к повествованию.
\rsbpar\vs 2Ma 6:18 Был некто Елеазар, из первых книжников, муж, уже достигший старости, но весьма красивой наружности; его принуждали, раскрывая ему рот, есть свиное мясо.
\vs 2Ma 6:19 Предпочитая славную смерть опозоренной жизни, он добровольно пошел на мучение и плевал,
\vs 2Ma 6:20 как надлежало решившимся устоять против того, чего из любви к жизни не дозволено вкушать.
\vs 2Ma 6:21 Тогда приставленные к беззаконному жертвоприношению, знавшие этого мужа с давнего времени, отозвав его, наедине убеждали его принести им самим приготовленные мяса, которые мог бы он употреблять, и притвориться, будто ест назначенные от царя жертвенные мяса,
\vs 2Ma 6:22 дабы через это избавиться от смерти и по давней с ними дружбе воспользоваться их человеколюбием.
\vs 2Ma 6:23 Но он, утвердившись в доброй мысли, достойной его возраста и почтенной старости и достигнутой им славной седины и благочестивого издетства воспитания, а более всего~--- святаго и Богом данного законоположения, соответственно сему отвечал и сказал: немедленно предать смерти;
\vs 2Ma 6:24 ибо недостойно нашего возраста лицемерить, дабы многие из юных, узнав, что девяностолетний Елеазар перешел в язычество,
\vs 2Ma 6:25 и сами вследствие моего лицемерия, ради краткой и ничтожной жизни, не впали через меня в заблуждение, и через то я положил бы бесчестие и пятно на мою старость.
\vs 2Ma 6:26 Если в настоящее время я и избавлюсь мучения от людей, но не избегну десницы Всемогущего ни в сей жизни, ни по смерти.
\vs 2Ma 6:27 Посему, мужественно расставаясь теперь с жизнью, сам я явлюсь достойным старости,
\vs 2Ma 6:28 а юным оставлю добрый пример~--- охотно и доблестно принимать смерть за досточтимые и святые законы. Сказав это, он тотчас пошел на мучение.
\vs 2Ma 6:29 Тогда и те, которые вели его, незадолго пред сим оказанное ему доброжелательство изменили в ненависть по причине вышесказанных слов, ибо они почли их за безумие.
\vs 2Ma 6:30 Готовясь уже умереть под ударами, он, восстенав, произнес: Господу, имеющему совершенное ведение, известно, что я, имея возможность избавиться от смерти, принимаю бичуемым телом жестокие страдания, а душею охотно терплю их по страху пред Ним.
\vs 2Ma 6:31 И так скончался он, оставив в смерти своей не только юношам, но и весьма многим из народа образец доблести и памятник добродетели.
\vs 2Ma 7:1 Случилось также, что были схвачены семь братьев с матерью и принуждаемы царем есть недозволенное свиное мясо, быв терзаемы бичами и жилами.
\vs 2Ma 7:2 Один из них, приняв на себя ответ, сказал: о чем ты хочешь спрашивать или что узнать от нас? Мы готовы лучше умереть, нежели преступить отеческие законы.
\vs 2Ma 7:3 Тогда царь, озлобившись, приказал разжечь сковороды и котлы.
\vs 2Ma 7:4 Когда они были разожжены, тотчас приказал принявшему на себя ответ отрезать язык и, содрав кожу с него, отсечь члены тела в виду прочих братьев и матери.
\vs 2Ma 7:5 Лишенного всех членов, но еще дышащего велел отнести к костру и жечь на сковороде; когда же от сковороды распространилось сильное испарение, они вместе с матерью увещевали друг друга мужественно претерпеть смерть, говоря:
\vs 2Ma 7:6 Господь Бог видит и поистине умилосердится над нами, как Моисей возвестил в своей песни пред лицем народа: <<и над рабами Своими умилосердится>>.
\vs 2Ma 7:7 Когда умер первый, вывели на поругание второго и, содрав с головы кожу с волосами, спрашивали, будет ли он есть, прежде нежели будут мучить по частям его тело?
\vs 2Ma 7:8 Он же, отвечая на отечественном языке, сказал: нет. Поэтому и он принял мучение таким же образом, как первый.
\vs 2Ma 7:9 Быв же при последнем издыхании, сказал: ты, мучитель, лишаешь нас настоящей жизни, но Царь мира воскресит нас, умерших за Его законы, для жизни вечной.
\vs 2Ma 7:10 После того третий подвергнут был поруганию и на требование дать язык тотчас выставил его, неустрашимо протянув и руки,
\vs 2Ma 7:11 и мужественно сказал: от неба я получил их и за законы Его не жалею их, и от Него надеюсь опять получить их.
\vs 2Ma 7:12 Сам царь и бывшие с ним изумлены были таким мужеством отрока, как он ни во что вменял страдания.
\vs 2Ma 7:13 Когда скончался и этот, таким же образом терзали и мучили четвертого.
\vs 2Ma 7:14 Будучи близок к смерти, он так говорил: умирающему от людей вожделенно возлагать надежду на Бога, что Он опять оживит; для тебя же не будет воскресения в жизнь.
\vs 2Ma 7:15 Затем привели и начали мучить пятого.
\vs 2Ma 7:16 Он, смотря на царя, сказал: имея власть над людьми, ты, сам подверженный тлению, делаешь, что хочешь; но не думай, чтобы род наш оставлен был Богом.
\vs 2Ma 7:17 Подожди, и ты увидишь великую силу Его, как Он накажет тебя и семя твое.
\vs 2Ma 7:18 После этого привели шестого, который, готовясь на смерть, сказал: не заблуждайся напрасно, ибо мы терпим это за себя, согрешив пред Богом нашим, оттого и произошло достойное удивления.
\vs 2Ma 7:19 Но не думай остаться безнаказанным ты, дерзнувший противоборствовать Богу.
\vs 2Ma 7:20 Наиболее же достойна удивления и славной памяти мать, которая, видя, как семь ее сыновей умерщвлены в течение одного дня, благодушно переносила это в надежде на Господа.
\vs 2Ma 7:21 Исполненная доблестных чувств и укрепляя женское рассуждение мужеским духом, она поощряла каждого из них на отечественном языке и говорила им:
\vs 2Ma 7:22 я не знаю, как вы явились во чреве моем; не я дала вам дыхание и жизнь; не мною образовался состав каждого.
\vs 2Ma 7:23 Итак, Творец мира, Который образовал природу человека и устроил происхождение всех, опять даст вам дыхание и жизнь с милостью, так как вы теперь не щадите самих себя за Его законы.
\vs 2Ma 7:24 Антиох же, думая, что его презирают, и принимая эту речь за поругание себе, убеждал самого младшего, который еще оставался, не только словами, но и клятвенными уверениями, что и обогатит и осчастливит его, если он отступит от отеческих законов, что будет иметь его другом и вверит ему почетные должности.
\vs 2Ma 7:25 Но как юноша нисколько не внимал, то царь, призвав мать, убеждал ее посоветовать сыну сберечь себя.
\vs 2Ma 7:26 После многих его убеждений она согласилась уговаривать сына.
\vs 2Ma 7:27 Наклонившись же к нему и посмеиваясь жестокому мучителю, она так говорила на отечественном языке: сын! сжалься надо мною, которая девять месяцев носила тебя во чреве, три года питала тебя молоком, вскормила и вырастила и воспитала тебя.
\vs 2Ma 7:28 Умоляю тебя, дитя мое, посмотри на небо и землю и, видя все, что на них, познай, что все сотворил Бог из ничего и что так произошел и род человеческий.
\vs 2Ma 7:29 Не страшись этого убийцы, но будь достойным братьев твоих и прими смерть, чтобы я по милости \bibemph{Божией} опять приобрела тебя с братьями твоими.
\rsbpar\vs 2Ma 7:30 Когда она еще продолжала говорить, юноша сказал: чего вы ожидаете? Я не слушаю повеления царя, а повинуюсь повелению закона, данного отцам нашим чрез Моисея.
\vs 2Ma 7:31 Ты же, изобретатель всех зол для Евреев, не избегнешь рук Божиих.
\vs 2Ma 7:32 Мы страдаем за свои грехи.
\vs 2Ma 7:33 Если для вразумления и наказания нашего живый Господь и прогневался на нас на малое время, то Он опять умилостивится над рабами Своими;
\vs 2Ma 7:34 ты же, нечестивый и преступнейший из всех людей, не возносись напрасно, надмеваясь ложными надеждами, что ты воздвигнешь руку на рабов Его,
\vs 2Ma 7:35 ибо ты не ушел еще от суда всемогущего и всевидящего Бога.
\vs 2Ma 7:36 Братья наши, претерпев ныне краткое мучение, по завету Божию получили жизнь вечную, а ты по суду Божию понесешь праведное наказание за превозношение.
\vs 2Ma 7:37 Я же, как и братья мои, предаю и душу и тело за отеческие законы, призывая Бога, чтобы Он скоро умилосердился над народом, и чтобы ты с муками и карами исповедал, что Он един есть Бог,
\vs 2Ma 7:38 и чтобы на мне и на братьях моих окончился гнев Всемогущего, праведно постигший весь род наш.
\vs 2Ma 7:39 Тогда разгневанный царь поступил с ним еще жесточе, нежели с прочими, негодуя на посмеяние.
\vs 2Ma 7:40 Так и этот кончил жизнь чистым, всецело положившись на Господа.
\vs 2Ma 7:41 После сыновей скончалась и мать.
\rsbpar\vs 2Ma 7:42 О жертвах идольских и о необыкновенных муках сказанного довольно.
\vs 2Ma 8:1 Между тем Иуда Маккавей и бывшие с ним, тайно входя в селения, созывали сродников и, принимая оставшихся в Иудействе, собрали до шести тысяч мужей.
\vs 2Ma 8:2 Они взывали к Господу, чтобы Он призрел на народ, всеми попираемый, и пожалел храм, оскверненный людьми нечестивыми;
\vs 2Ma 8:3 чтобы помиловал разоренный город, близкий к тому, чтобы сравняться с землею, и услышал вопиющую к Нему кровь;
\vs 2Ma 8:4 чтобы вспомнил о беззаконном погублении невинных младенцев и о бывших хулениях имени Его, и вознегодовал на злых.
\vs 2Ma 8:5 Окружив себя множеством, Маккавей сделался непобедим для язычников, когда гнев Господа преложился на милость.
\vs 2Ma 8:6 Внезапно нападая на города и селения, он сожигал их и, занимая удобные места, немало победил врагов, обращая их в бегство;
\vs 2Ma 8:7 преимущественно он избирал себе в помощь для таких предприятий ночи, и слух о его мужестве разносился повсюду.
\rsbpar\vs 2Ma 8:8 Филипп, видя, что этот муж мало-помалу приходит в силу, а чаще бывает счастлив в делах, писал к Птоломею, военачальнику Келе-Сирии и Финикии, чтобы он помог делам царя.
\vs 2Ma 8:9 Он же, немедленно избрав Никанора, сына Патроклова, одного из первых своих друзей, послал его, подчинив ему не менее двадцати тысяч человек из разных народов, истребить весь род Иудеев; присоединил к нему и Горгия военачальника, опытного в делах военных.
\vs 2Ma 8:10 Никанор постановил: дань в две тысячи талантов, которую царь должен был Римлянам, пополнить от пленения Иудеев.
\vs 2Ma 8:11 Почему тотчас послал в приморские города, приглашая их покупать в рабы Иудеев и обещая доставлять по девяносто пленников за один талант; но не ожидал он того мщения, которое готово было прийти на него от Всемогущего.
\vs 2Ma 8:12 Иуде же дано было знать о приходе Никанора, и, когда он передал бывшим с ним о прибытии войска,
\vs 2Ma 8:13 тогда боязливые и не веровавшие в воздаяние Божие разбежались, оставив места свои.
\vs 2Ma 8:14 Другие же продавали все оставшееся у них и умоляли Господа избавить их, проданных нечестивым Никанором прежде сражения,
\vs 2Ma 8:15 если не для них, то ради заветов с отцами их и наречения на них святаго и славного имени Его.
\rsbpar\vs 2Ma 8:16 Тогда Маккавей собрал бывших с ним, числом шесть тысяч мужей, и увещевал их не страшиться врагов и не бояться множества язычников, неправедно идущих на них, но мужественно сражаться,
\vs 2Ma 8:17 имея пред глазами неправедно нанесенное ими оскорбление святому месту и разорение поруганного города и нарушение праотеческих учреждений.
\vs 2Ma 8:18 Ибо, говорил он, они надеются на оружие и отважность, а мы надеемся на всемогущего Бога, Который одним мановением может ниспровергнуть и идущих на нас, и весь мир.
\vs 2Ma 8:19 Он рассказал им и о том заступлении, какое получали их предки, и как при Сеннахириме погублены сто восемьдесят пять тысяч мужей,
\vs 2Ma 8:20 и о бывшем в Вавилоне сражении против Галатов, как они пришли на брань в числе только восьми тысяч с четырьмя тысячами Македонян, и когда Македоняне смешались, то эти восемь тысяч погубили сто двадцать тысяч бывшею им с неба помощью и получили великую добычу.
\vs 2Ma 8:21 Такими рассказами сделав их неустрашимыми и готовыми умереть за законы и отечество, он разделил войско на четыре отряда,
\vs 2Ma 8:22 назначив вождями каждого отряда братьев своих: Симона, Иосифа и Ионафана~--- и подчинив каждому по тысяче пятисот человек.
\vs 2Ma 8:23 Потом приказал Елеазару читать священную книгу, и, обнадежив Божиею помощью, сам принял предводительство над передовым отрядом и вступил в сражение с Никанором.
\vs 2Ma 8:24 Так как помощником их был Всемогущий, то они побили врагов более девяти тысяч, и еще б\acc{о}льшую часть Никанорова войска оставили ранеными и изувеченными, и всех принудили бежать.
\vs 2Ma 8:25 Взяли и деньги у пришедших покупать их; преследовали их на значительное расстояние и возвратились, будучи остановлены временем.
\vs 2Ma 8:26 Ибо это был день пред субботою; по этой причине они и не продолжали гнаться за ними.
\vs 2Ma 8:27 Собрав же за ними оружие и сняв доспехи с врагов, они праздновали субботу, усердно благодаря и прославляя Господа, спасшего их в тот день и начавшего являть им Свое милосердие.
\vs 2Ma 8:28 После субботы, уделив из добычи увечным, вдовам и сиротам, остальное разделили между собою и детьми своими.
\vs 2Ma 8:29 Окончив это, они учредили общественную молитву и умоляли милосердого Господа совершенно примириться с рабами Своими.
\vs 2Ma 8:30 И тогда, как Тимофей и Вакхид напали на них совокупно, они избили более двадцати тысяч и легко овладели высокими крепостями; они разделили весьма много добычи по равным частям между собою и увечными и сиротами и вдовами, еще же и старейшинами.
\vs 2Ma 8:31 Собрав после них оружие, тщательно сложили всё в удобных местах, остальную же добычу принесли в Иерусалим.
\vs 2Ma 8:32 Убили и вождя войск Тимофеевых, человека нечестивейшего, который причинил много бед Иудеям.
\vs 2Ma 8:33 Потом, торжествуя победу в отечестве, они сожгли Каллисфена и некоторых других, которые сожгли священные ворота и убежали в один дом, так что эти за свое нечестие понесли достойное возмездие.
\vs 2Ma 8:34 А преступнейший Никанор, который привел тысячу купцов для покупки Иудеев,
\vs 2Ma 8:35 при помощи Божией посрамлен был теми, которых считал за ничто, и, сбросив пышную одежду, под видом беглого раба чрез внутренние земли пришел один в Антиохию, крайне огорченный поражением войска.
\vs 2Ma 8:36 Тот, который взялся доставить Римлянам дань от пленных в Иерусалиме, объявил, что Иудеи имеют защитником Бога и таким образом остаются невредимы, потому что повинуются установленным от Бога законам.
\vs 2Ma 9:1 Около того же времени Антиох с бесславием возвращался из пределов Персии.
\vs 2Ma 9:2 Ибо он вошел в так называемый Персеполь и покушался ограбить храм и овладеть городом. Поэтому сбежался народ, и обратились к помощи оружия, и Антиох, обращенный жителями в бегство, должен был со стыдом возвратиться назад.
\vs 2Ma 9:3 Когда находился он близ Екбатаны, донесли ему о том, что случилось с Никанором и с Тимофеем.
\vs 2Ma 9:4 Воспылав гневом, он думал выместить на Иудеях зло обративших его в бегство; поэтому приказал правящему колесницею непрестанно погонять и ускорять путешествие, тогда как небесный суд уже следовал за ним. Ибо он сказал с высокомерием: кладбищем для Иудеев сделаю Иерусалим, когда приду туда.
\vs 2Ma 9:5 Но всевидящий Господь, Бог Израилев, поразил его неисцельным и невидимым ударом: как только кончил он эти слова, схватила его нестерпимая болезнь живота и жестокие внутренние муки,
\vs 2Ma 9:6 и совершенно праведно; ибо он многими и необычайными муками терзал утробы других.
\vs 2Ma 9:7 Но он нисколько не оставлял своей гордости и еще более исполнился высокомерия, дыша огнем ярости на Иудеев и приказывая ускорять путешествие. Тогда случилось, что он упал с колесницы, которая неслась быстро, и тяжким падением повредил все члены тела.
\vs 2Ma 9:8 И тот, который только что мнил по гордости, более нежели человеческой, повелевать волнам моря и думал на весах взвесить высоты гор, повержен был на землю и несен был на носилках, показуя всем явную силу Божию,
\vs 2Ma 9:9 так что из тела нечестивца во множестве выползали черви и еще у живого выпадали части тела от болезней и страданий; смрад же зловония от него невыносим был в целом войске.
\vs 2Ma 9:10 И того, который незадолго перед тем мечтал касаться звезд небесных, никто не мог носить по причине невыносимого зловония.
\vs 2Ma 9:11 Теперь-то, будучи сокрушен, начал он оставлять свое великое высокомерие и приходить в познание, когда по наказанию Божию страдания его усиливались с каждою минутою.
\vs 2Ma 9:12 Сам не в силах сносить своего зловония, он так говорил: праведно покоряться Богу, и смертному не должно думать высокомерно быть равным Богу.
\vs 2Ma 9:13 Нечестивец молил Господа, уже не миловавшего его, и говорил:
\vs 2Ma 9:14 <<Святый город, который спешил я сравнять с землею и сделать кладбищем, объявляю свободным;
\vs 2Ma 9:15 Иудеев, которых положил не удостоивать погребения, а выбрасывать вместе с детьми их хищным птицам и зверям, сделаю всех равными Афинянам;
\vs 2Ma 9:16 святый храм, который прежде ограбил, украшу отличнейшими дарами, священные сосуды возвращу все, и еще в большем количестве, и необходимые для жертв издержки буду производить из моих доходов;
\vs 2Ma 9:17 сверх того, сам сделаюсь Иудеем и, проходя по всякому обитаемому месту, буду возвещать силу Божию>>.
\vs 2Ma 9:18 Но когда боли нисколько не умалялись, ибо пришел уже на него праведный суд Божий, он, отчаиваясь в себе, написал к Иудеям письмо, имевшее значение мольбы, следующего содержания:
\vs 2Ma 9:19 <<Царь и военачальник Антиох добрым Иудеям-гражданам~--- много радоваться и здравствовать и благоденствовать.
\vs 2Ma 9:20 Если вы здравствуете с детьми вашими и дела ваши идут по вашему желанию, то я воздаю Богу величайшую благодарность, возлагая надежду на небо.
\vs 2Ma 9:21 Я же лежу в болезни и с любовью воспоминаю о вашей почтительности и благорасположении ко мне. Возвращаясь из пределов Персии и подвергшись тяжкой болезни, я за нужное почел позаботиться об общей безопасности всех.
\vs 2Ma 9:22 Хотя я не отчаиваюсь в себе и имею полную надежду освободиться от болезни,
\vs 2Ma 9:23 но, зная, что и отец мой, когда воевал в верхних странах, объявил преемника,
\vs 2Ma 9:24 дабы, если последует что-нибудь неожиданное или объявлена будет какая невзгода, жители страны знали, кому предоставлено правление, и не приходили в смущение;
\vs 2Ma 9:25 сверх того, замечая, что окрестные владетели и соседние с нашим государством наблюдают время и выжидают, какой будет исход, я назначил царем сына моего Антиоха, которого я уже часто во время походов в верхние сатрапии весьма многим из вас препоручал и представлял; и к нему я написал особо.
\vs 2Ma 9:26 Итак, убеждаю вас и прошу, чтобы вы, помня мои благодеяния вообще и в частности, сохранили ваше теперешнее благорасположение ко мне и к сыну моему.
\vs 2Ma 9:27 Ибо я уверен, что он, следуя моему желанию, будет обращаться с вами милостиво и человеколюбиво>>.
\rsbpar\vs 2Ma 9:28 Так этот человекоубийца и богохульник, претерпев тяжкие страдания, какие причинял другим, кончил жизнь на чужой стороне в горах самою жалкою смертью.
\vs 2Ma 9:29 Тело его привез Филипп, совоспитанник его, который, боясь сына Антиохова, удалился к Птоломею Филопатору в Египет.
\vs 2Ma 10:1 Маккавей же и бывшие с ним, под водительством Господа, опять заняли храм и город,
\vs 2Ma 10:2 а построенные иноплеменниками на площади жертвенники и капища разрушили.
\vs 2Ma 10:3 Очистив храм, они соорудили другой жертвенник; разжегши камни и взяв из них огонь, принесли жертву после двухгодичного промежутка, сделали кадильницу и свещники и предложение хлебов.
\vs 2Ma 10:4 Устроив все это, они молили Господа, падая ниц, чтобы им не подвергаться более таким бедствиям; если же когда и согрешат, то да накажет Он их милостиво, не предавая богохульным и жестоким язычникам.
\vs 2Ma 10:5 В тот самый день, в какой осквернен был храм иноплеменниками, совершилось и очищение храма, в двадцать пятый день того же месяца Хаслева.
\vs 2Ma 10:6 И провели они в весельи восемь дней по подобию праздника кущей, воспоминая, как незадолго пред тем временем они проводили праздник кущей, подобно зверям, в горах и пещерах.
\vs 2Ma 10:7 Поэтому они с жезлами, обвитыми плющом, и с цветущими ветвями и пальмами возносили хвалебные песни Тому, Который благопоспешил очистить место Свое.
\vs 2Ma 10:8 И общим решением и приговором определили~--- всему Иудейскому народу праздновать эти дни каждогодно.
\vs 2Ma 10:9 Такова была кончина Антиоха, прозванного Епифаном.
\rsbpar\vs 2Ma 10:10 Теперь изложим, что происходило при Антиохе Евпаторе, сыне того нечестивца, ограничиваясь бедствиями войн.
\vs 2Ma 10:11 Приняв царство, он вручил управление некоему Лисию, главному военачальнику Келе-Сирии и Финикии.
\vs 2Ma 10:12 Ибо Птоломей, по прозванию Макрон, почел за лучшее соблюдать справедливость к Иудеям, после бывших к ним несправедливостей, и старался дела с ними оканчивать мирно.
\vs 2Ma 10:13 Поэтому он был оклеветан любимцами пред Евпатором, и, повсюду слыша название предателя за то, что он оставил вверенный ему от Филометора Кипр и перешел к Антиоху Епифану, он, не имея почетной власти, от печали отравил себя и так окончил жизнь свою.
\rsbpar\vs 2Ma 10:14 Горгий же, сделавшись в тех местах военачальником, содержал наемные войска и непрерывно поддерживал войну против Иудеев.
\vs 2Ma 10:15 Вместе с ним и Идумеи, владевшие удобными укреплениями, тревожили Иудеев и, принимая к себе изгнанных из Иерусалима, предпринимали войны.
\vs 2Ma 10:16 Бывшие же с Маккавеем, совершая моление и прося Бога быть помощником им в войне, устремились на укрепления Идумеев.
\vs 2Ma 10:17 И, сделав на них сильное нападение, они овладели этими местами, отмстили всем сражавшимся на стенах, умерщвляли всех попадавшихся навстречу и побили не менее двадцати тысяч.
\vs 2Ma 10:18 Не менее девяти тысяч бежали в две весьма крепкие башни, снабженные всем против осады.
\vs 2Ma 10:19 Оставив Симона и Иосифа и еще Закхея с довольным числом людей для осаждения их, Маккавей сам отправился в такие места, где он более нужен был.
\vs 2Ma 10:20 А бывшие с Симоном, будучи сребролюбивы, дали некоторым из находившихся в башнях подкупить себя деньгами; получив семьдесят тысяч драхм, дозволили некоторым убежать.
\vs 2Ma 10:21 Когда донесено было Маккавею о происшедшем, он, собрав народных вождей, укорял их, что они за серебро продали братьев, отпустив врагов их.
\vs 2Ma 10:22 Этих людей, сделавшихся предателями, он предал смерти и тотчас овладел двумя башнями.
\vs 2Ma 10:23 Имея постоянно успех в оружии, которое было в руках его, он истребил в этих двух укреплениях более двадцати тысяч человек.
\rsbpar\vs 2Ma 10:24 Тимофей же, прежде побежденный Иудеями, собрал весьма многочисленное войско из чужеземцев, собрал немало и бывших в Азии всадников и явился в Иудею в намерении завоевать ее.
\vs 2Ma 10:25 При его приближении бывшие с Маккавеем обратились к молитве Богу, посыпав землею головы и опоясав чресла вретищами.
\vs 2Ma 10:26 Припадая к подножию жертвенника, они умоляли Его, чтобы Он был милостив к ним, был врагом врагам их и противником противникам, как говорит закон.
\vs 2Ma 10:27 Совершив молитву, они взяли оружие и далеко отошли от города; приблизившись же ко врагам, остановились.
\vs 2Ma 10:28 С наступлением восхода солнечного те и другие вступили в бой~--- одни, при доблести своей, имея залогом успеха и победы прибежище к Господу, другие~--- поставляя предводителем брани ярость.
\vs 2Ma 10:29 Когда произошло упорное сражение, то противникам явились с неба пять величественных мужей на конях с золотыми уздами, и двое из них предводительствовали Иудеями:
\vs 2Ma 10:30 они взяли Маккавея в средину к себе и, покрывая своим вооружением, сохраняли его невредимым, на противников же бросали стрелы и молнии, так что они, смешавшись от ослепления и исполненные страха, сами себя поражали.
\vs 2Ma 10:31 Побито было двадцать тысяч пятьсот пеших и шестьсот конных.
\vs 2Ma 10:32 Сам Тимофей убежал в крепость, называемую Газара, весьма твердую и состоявшую под начальством Херея.
\vs 2Ma 10:33 Бывшие с Маккавеем весело осаждали эту крепость в продолжение четырех дней.
\vs 2Ma 10:34 А находившиеся в крепости, уверенные в недоступности этого места, чрезмерно злословили и произносили хульные речи.
\vs 2Ma 10:35 На рассвете пятого дня двадцать юношей из бывших с Маккавеем, воспламенившись гневом от такого злословия, храбро устремились на стену и с зверскою яростью поражали каждого, кто попадался.
\vs 2Ma 10:36 Другие также бросились во время смятения на находившихся внутри, зажигали башни и, разжегши костры, сожигали хульников живыми; иные разбивали ворота и, впустив в них остальное войско, овладели городом;
\vs 2Ma 10:37 Тимофея же, скрывшегося во рву, убили, равно как и брата его Херея, и Аполлофана.
\vs 2Ma 10:38 Совершив это, они с песнями и славословиями возблагодарили Господа, Который так много облагодетельствовал Израиля и даровал им победу.
\vs 2Ma 11:1 Спустя очень немного времени Лисий, опекун и родственник царя, наместник царский, с большим огорчением перенося то, что случилось,
\vs 2Ma 11:2 собрал до восьмидесяти тысяч пехоты и всю конницу \bibemph{и} отправился против Иудеев с намерением город их сделать местом жительства Еллинов,
\vs 2Ma 11:3 храм обложить налогом, подобно прочим языческим капищам, а священноначалие сделать ежегодно продажным.
\vs 2Ma 11:4 Нисколько не подумал он о силе Божией, понадеявшись на десятки тысяч пехоты, на тысячи конницы и на восемьдесят слонов.
\vs 2Ma 11:5 Вступив в Иудею и приблизившись к Вефсуре, месту укрепленному, отстоящему от Иерусалима стадий на пять, он обложил его.
\vs 2Ma 11:6 Когда Маккавей и бывшие с ним узнали, что он осаждает твердыни, то с плачем и слезами вместе с народом умоляли Господа, чтобы Он послал доброго Ангела ко спасению Израиля.
\vs 2Ma 11:7 Маккавей же, сам первый взяв оружие, убеждал других вместе с ним, подвергая себя опасностям, помочь братьям; и они тотчас охотно выступили с ним в поход.
\vs 2Ma 11:8 Когда они были близ Иерусалима, тотчас явился предводителем их всадник в белой одежде, потрясавший золотым оружием.
\vs 2Ma 11:9 Все они вместе возблагодарили милосердого Бога и укрепились духом, готовые сокрушить не только людей, но и лютых зверей и даже железные стены.
\vs 2Ma 11:10 Так пришли они, под покровом небесного споборника, по милости к ним Господа.
\vs 2Ma 11:11 Как львы бросились они на неприятелей и поразили из них одиннадцать тысяч пеших и тысячу шестьсот конных, а всех прочих обратили в бегство.
\vs 2Ma 11:12 Многие из них, быв ранены, спасались раздетыми, и сам Лисий спасся постыдным бегством.
\vs 2Ma 11:13 Будучи же небессмыслен и обсуждая сам с собою случившееся с ним поражение, он понял, что Евреи непобедимы, потому что всемогущий Бог споборствует им; посему, послав к ним,
\vs 2Ma 11:14 уверял, что он соглашается на все законные требования и убедит царя быть другом им.
\vs 2Ma 11:15 Маккавей, заботясь о пользе, согласился на все, что предъявлял Лисий; ибо царь одобрил все, что предложил Маккавей Лисию на письме относительно Иудеев.
\vs 2Ma 11:16 Письмо же, писанное Лисием к Иудеям, было следующего содержания: <<Лисий народу Иудейскому~--- радоваться.
\vs 2Ma 11:17 Иоанн и Авессалом, вами посланные, передав подписанный ответ, ходатайствовали о том, что было означено в нем.
\vs 2Ma 11:18 Итак, о чем следовало донести царю, я объяснил, и, что можно было принять, на то он согласился.
\vs 2Ma 11:19 Посему, если вы будете сохранять доброе расположение к правлению, то и на будущее время я постараюсь содействовать вам ко благу.
\vs 2Ma 11:20 О частностях же я поручил как вашим, так и моим посланным переговорить с вами.
\vs 2Ma 11:21 Будьте здоровы! Сто сорок восьмого года, месяца Диоскоринфия, двадцать четвертого дня>>.
\vs 2Ma 11:22 Письмо же царя было такого содержания: <<Царь Антиох брату Лисию~--- радоваться.
\vs 2Ma 11:23 С того времени, как отец мой отошел к богам, наше желание то, чтобы подданные царства оставались безмятежными в отправлении дел своих.
\vs 2Ma 11:24 Когда же мы услышали, что Иудеи не соглашаются на предпринятое отцом моим нововведение Еллинских обычаев, а предпочитают собственные установления и потому просят, чтобы позволено им было соблюдать свои законы,
\vs 2Ma 11:25 то, желая, чтобы и этот народ не был беспокоим, определяем, чтобы храм их был восстановлен и чтобы жили они по обычаю своих предков.
\vs 2Ma 11:26 Итак, ты хорошо сделаешь, если пошлешь к ним и заключишь мир с ними, чтобы они, зная наши намерения, были благодушны и весело продолжали заниматься делами своими>>.
\vs 2Ma 11:27 К народу же письмо царя было такое: <<Царь Антиох старейшинам Иудейским и прочим Иудеям~--- радоваться.
\vs 2Ma 11:28 Если вы здравствуете, то этого мы и желаем: мы также здравствуем.
\vs 2Ma 11:29 Менелай объявил нам, что вы желаете сходить к вашим, которые у нас.
\vs 2Ma 11:30 Итак, тем, которые будут приходить до тридцатого дня месяца Ксанфика, готова правая рука в уверение их безопасности:
\vs 2Ma 11:31 Иудеи могут употреблять свою пищу и хранить свои законы, как и прежде, и никто из них никаким образом не будет обеспокоен за бывшие опущения.
\vs 2Ma 11:32 Я послал к вам Менелая, чтобы он успокоил вас.
\vs 2Ma 11:33 Будьте здоровы! Сто сорок восьмого года, пятнадцатого дня Ксанфика>>.
\vs 2Ma 11:34 Прислали к ним письмо и Римляне следующего содержания: <<Квинт Меммий и Тит Манлий, старейшины Римские, Иудейскому народу~--- радоваться.
\vs 2Ma 11:35 Что уступил вам Лисий, родственник царя, то и мы подтверждаем.
\vs 2Ma 11:36 А что признал он нужным доложить царю, о том, рассудив немедленно, пошлите кого-нибудь, чтобы мы могли сделать, что для вас нужно, ибо мы отправляемся в Антиохию.
\vs 2Ma 11:37 Посему поспешите и пошлите кого-нибудь, чтобы и мы могли знать, какого вы мнения.
\vs 2Ma 11:38 Будьте здоровы! Сто сорок восьмого года, пятнадцатого дня Ксанфика>>.
\vs 2Ma 12:1 По окончании этих договоров Лисий отправился к царю, а Иудеи занялись земледелием.
\vs 2Ma 12:2 Но из местных военачальников Тимофей и Аполлоний, сын Генея, равно как Иероним и Димофон, и сверх того Никанор, начальник Кипра, не давали им жить в покое и безопасности.
\vs 2Ma 12:3 Иоппийцы же совершили такое безбожное дело: они пригласили живущих с ними Иудеев с их женами и детьми взойти на приготовленные ими лодки, как бы не имея против них никакого зла.
\vs 2Ma 12:4 Когда же они согласились, ибо желали сохранить мир и не имели никакого подозрения, тогда, по общему приговору города, Иоппийцы, отплыв, потопили их, не менее двухсот человек.
\vs 2Ma 12:5 Когда Иуда узнал о такой жестокости, совершенной над одноплеменниками, объявил о том бывшим с ним
\vs 2Ma 12:6 и, призвав праведного Судию Бога, пошел против скверных убийц братьев его, зажег ночью пристань и сжег лодки, а сбежавшихся туда умертвил.
\vs 2Ma 12:7 А так как это место было заперто, то он отошел, в намерении опять прийти и истребить все общество Иоппийцев.
\vs 2Ma 12:8 Узнав же, что и жители Иамнии хотят таким же образом поступить с обитающими там Иудеями,
\vs 2Ma 12:9 он напал ночью и на Иамнитян и зажег пристань с кораблями, так что пламя видно было в Иерусалиме за двести сорок стадий.
\rsbpar\vs 2Ma 12:10 Когда же они отошли оттуда на девять стадий, направляясь против Тимофея, то напали на них Арабы, не менее пяти тысяч и пятисот всадников.
\vs 2Ma 12:11 Сражение было жестокое, и когда бывшие с Иудою при помощи Божией одержали победу, то потерпевшие поражение номады Арабы просили Иуду о мире, обещая доставлять им скот и в другом быть полезными им.
\vs 2Ma 12:12 Иуда же, понимая, что они действительно во многом могут быть полезны, согласился заключить с ними мир; заключив же мир, они удалились в свои палатки.
\vs 2Ma 12:13 Еще напал он на один город с крепким мостом, окруженный стенами и населенный разными народами, по имени Каспин.
\vs 2Ma 12:14 Жители, надеясь на крепость стен и запас продовольствия, поступили очень дерзко, злословя бывших с Иудою, богохульствуя и произнося неподобающие речи.
\vs 2Ma 12:15 Но бывшие с Иудою, призвав на помощь великого Владыку мира, Который без стенобитных машин и орудий разрушил Иерихон во времена Иисуса, зверски бросились на стену.
\vs 2Ma 12:16 При помощи Божией они взяли город и произвели бесчисленные убийства, так что близлежащее озеро, имевшее две стадии в ширину, казалось наполненным кровью.
\rsbpar\vs 2Ma 12:17 Отойдя оттуда на семьсот пятьдесят стадий, они пришли в Харак к Иудеям, называемым Тувиинами;
\vs 2Ma 12:18 но не застали там Тимофея, который, ничего не сделав, удалился из этой страны, оставив, впрочем, в одном месте очень крепкую стражу.
\vs 2Ma 12:19 Посему Досифей и Сосипатр, из бывших с Маккавеем вождей, отправились и побили оставленных Тимофеем в крепости людей, более десяти тысяч.
\vs 2Ma 12:20 Тогда Маккавей, разделив свое войско на отряды, поставил их над этими отрядами и устремился на Тимофея, который имел при себе сто двадцать тысяч пеших и тысячу пятьсот конных.
\vs 2Ma 12:21 Когда узнал Тимофей о приближении Иуды, то отослал жен и детей и прочий обоз в так называемый Карнион, ибо эта крепость была неудобна для осады и недоступна по тесноте всей местности.
\vs 2Ma 12:22 Когда же показался первый отряд Иуды, страх напал на врагов, и ужас объял их от явления Всевидящего: они обратились в бегство, стремясь один туда, другой сюда, так что большею частью поражаемы были своими, пронзаемы острием своих мечей.
\vs 2Ma 12:23 Иуда настойчиво продолжал преследовать, убивал беззаконных и истребил до тридцати тысяч человек.
\vs 2Ma 12:24 Сам Тимофей попался в руки бывших с Досифеем и Сосипатром и с большим ухищрением умолял отпустить его живым, ибо у него находились многих \bibemph{Иудеев} родители, а некоторых братья и они не будут пощажены, если он умрет.
\vs 2Ma 12:25 Когда он многими словами уверил в своем обещании, что возвратит их невредимыми, они отпустили его, ради спасения братьев.
\rsbpar\vs 2Ma 12:26 Потом \bibemph{Иуда} пошел против Карниона и Атаргатиона и избил двадцать пять тысяч человек.
\vs 2Ma 12:27 После победы над ними и поражения Иуда отправился против укрепленного города Ефрона, в котором имел пребывание Лисий и множество разноплеменных: сильные юноши, стоявшие пред стенами, сражались упорно; там же находились большие запасы орудий и стрел.
\vs 2Ma 12:28 Но они, призвав на помощь Всесильного, сокрушающего Своим могуществом силы врагов, овладели этим городом и избили бывших в нем до двадцати пяти тысяч.
\vs 2Ma 12:29 Поднявшись оттуда, они устремились на город Скифов, отстоящий от Иерусалима на шестьсот стадий.
\vs 2Ma 12:30 Но как обитавшие там Иудеи свидетельствовали о благорасположении, какое имеют к ним Скифские жители, и о кротком обхождении с ними во времена бедствий,
\vs 2Ma 12:31 то, поблагодарив их и попросив и на будущее время быть благосклонными к роду их, они отправились в Иерусалим, потому что приближался праздник седмиц.
\rsbpar\vs 2Ma 12:32 После праздника, называемого Пятидесятницею, пошли они против Горгия, военачальника Идумеи.
\vs 2Ma 12:33 Выступил же Иуда с тремя тысячами пеших и четырьмя стами конных.
\vs 2Ma 12:34 Когда они вступили в сражение, случилось пасть немногим из Иудеев.
\vs 2Ma 12:35 Досифей же, один из бывших под начальством Вакинора, всадник, муж сильный, поймал Горгия и, схватив его за плащ, влек его сильно, чтобы взять проклятого в плен живым; но один из всадников Фракийских наскакал на него и отсек ему плечо, и Горгий убежал в Марису.
\vs 2Ma 12:36 Когда же бывшие с Ездрином, долго сражаясь, изнемогли, Иуда призвал на помощь Господа, да будет Он началовождем в сражении.
\vs 2Ma 12:37 Начав на отечественном языке песнопение громким голосом, он воскликнул и, неожиданно устремившись на бывших с Горгием, обратил их в бегство.
\rsbpar\vs 2Ma 12:38 Потом Иуда, взяв с собою войско, отправился в город Одоллам, и так как наступал седьмой день, то они очистились по обычаю и праздновали субботу.
\vs 2Ma 12:39 На другой день бывшие с Иудою пошли, как требовал долг, перенести тела павших и положить их вместе со сродниками в отеческих гробницах.
\vs 2Ma 12:40 И нашли они у каждого из умерших под хитонами посвященные Иамнийским идолам вещи, что закон запрещал Иудеям; и сделалось всем явно, по какой причине они пали.
\vs 2Ma 12:41 Итак, все прославили праведного Судию Господа, открывающего сокровенное,
\vs 2Ma 12:42 и обратились к молитве, прося, да будет совершенно изглажен содеянный грех; а доблестный Иуда увещевал народ хранить себя от грехов, видя своими глазами, что случилось по вине падших.
\vs 2Ma 12:43 Сделав же сбор по числу мужей до двух тысяч драхм серебра, он послал в Иерусалим, чтобы принести жертву за грех, и поступил весьма хорошо и благочестно, помышляя о воскресении;
\vs 2Ma 12:44 ибо, если бы он не надеялся, что павшие в сражении воскреснут, то излишне и напрасно было бы молиться о мертвых.
\vs 2Ma 12:45 Но он помышлял, что скончавшимся в благочестии уготована превосходная награда,~--- какая святая и благочестивая мысль! Посему принес за умерших умилостивительную жертву, да разрешатся от греха.
\vs 2Ma 13:1 В сто сорок девятом году дошел слух до бывших с Иудою, что Антиох Евпатор идет на Иудею со множеством войска
\vs 2Ma 13:2 и с ним Лисий, опекун и государственный правитель, и у каждого Еллинское войско, сто десять тысяч пеших, пять тысяч триста конных, двадцать два слона и триста колесниц с косами.
\vs 2Ma 13:3 Присоединился к ним и Менелай, с большим притворством побуждая Антиоха, не ради спасения отечества, но в надежде получить начальство.
\vs 2Ma 13:4 Но Царь царей воздвиг гнев Антиоха на преступника, и когда Лисий объяснил, что \bibemph{Менелай} был виновником всех зол, то он приказал отвести его в Берию и по тамошнему обычаю умертвить.
\vs 2Ma 13:5 В том месте находится башня в пятьдесят локтей, наполненная пеплом; в ней было орудие, обращавшееся вокруг и спускавшееся в пепел.
\vs 2Ma 13:6 Там всегда низвергают на погибель виновного в святотатстве или превзошедшего меру других зол.
\vs 2Ma 13:7 Такою-то смертью пришлось умереть нечестивому Менелаю и не иметь погребения в земле,~--- и весьма справедливо.
\vs 2Ma 13:8 Ибо когда он совершил много грехов против алтаря \bibemph{Господня}, которого огонь и пепел был свят, то и получил смерть в пепле.
\rsbpar\vs 2Ma 13:9 Между тем царь, ожесточившийся в своих замыслах, продолжал шествие, намереваясь причинить Иудеям бедствия горшие тех, какие были при отце его.
\vs 2Ma 13:10 Когда узнал об этом Иуда, то велел народу день и ночь призывать Господа, чтобы Он и ныне, как и прежде, явил им Свою помощь при опасности лишиться закона и отечества и святаго храма
\vs 2Ma 13:11 и чтобы народ, только что немного успокоившийся, не отдал в порабощение злохульным язычникам.
\vs 2Ma 13:12 Все единодушно исполнили это и в продолжение трех дней с плачем и постом и коленопреклонением непрестанно молились милосердому Господу; тогда Иуда, ободрив их, приказал им быть в готовности.
\vs 2Ma 13:13 Оставшись же наедине со старейшинами, держал совет, намереваясь прежде, нежели царское войско войдет в Иудею и овладеет городом, выйти и решить дело с помощью Господа.
\vs 2Ma 13:14 Предоставив попечение о себе Создателю мира, он убеждал бывших с ним сражаться мужественно до смерти за законы, за храм, город, отечество и права гражданские и расположил войско около Модина.
\vs 2Ma 13:15 Дав бывшим с ним условный знак <<Божия победа>>, он с избранными сильными юношами ночью устремился на царский шатер, убил в войске до четырех тысяч человек и, кроме того, самого большого слона с помещавшимся на нем народом.
\vs 2Ma 13:16 Наконец, исполнив войско страха и смятения, они благополучно отошли.
\vs 2Ma 13:17 Произошло это уже на рассвете дня, при покровительстве Господа.
\vs 2Ma 13:18 Царь же, опытом дознав отважность Иудеев, пытался овладеть местами посредством хитрости.
\vs 2Ma 13:19 И приступил он к Вефсуре, твердой крепости Иудейской, но был обращен в бегство и потерпел поражение и потерю;
\vs 2Ma 13:20 Иуда же присылал бывшим в крепости все нужное.
\vs 2Ma 13:21 Некто Родок из войска Иудейского объявил врагам об этой тайне, но был отыскан, схвачен и заключен.
\vs 2Ma 13:22 Во второй раз царь вступил в переговоры с жителями Вефсуры; дал им и от них получил мир, удалился и обратился против бывших с Иудою, но был побежден.
\vs 2Ma 13:23 Узнав же, что Филипп, оставленный в Антиохии правителем, отложился, он пришел в смущение: стал уговаривать Иудеев, смирился и клялся исполнить все справедливые требования, затем примирился с ними и принес жертву, почтил храм и оказал милости городу,
\vs 2Ma 13:24 принял Маккавея и поставил его военачальником от Птолемаиды до самого Геррин.
\vs 2Ma 13:25 Потом пошел он в Птолемаиду: Птолемаидяне недовольны были договором, негодовали на условия и хотели отменить их.
\vs 2Ma 13:26 Вошел на судилище Лисий, защищался по возможности, уговорил их, успокоил, сделал благосклонными и отправился в Антиохию. Так окончилось нашествие и возвращение царя.
\vs 2Ma 14:1 Спустя три года дошел слух до Иуды и бывших с ним, что Димитрий, сын Селевка, приплыл в пристань Трипольскую с сильным сухопутным и морским войском
\vs 2Ma 14:2 и, овладев страною, умертвил Антиоха и опекуна его Лисия.
\rsbpar\vs 2Ma 14:3 Алким же некто, бывший прежде первосвященником, но добровольно осквернившийся в смутные времена, размыслив, что никаким образом нет ему спасения и нет доступа до священного жертвенника,
\vs 2Ma 14:4 в сто пятьдесят первом году пришел к царю Димитрию и принес ему золотой венец и пальму и сверх того масличные ветви, считавшиеся принадлежностями храма,~--- и в этот день Алким ничего не предпринял.
\vs 2Ma 14:5 Улучив же время, благоприятное его безумному замыслу, когда он позван был Димитрием в собрание совета и спрошен, в каком расположении и настроении находятся Иудеи, он сказал на это:
\vs 2Ma 14:6 так называемые из Иудеев Асидеи, вождем которых Иуда Маккавей, поддерживают войну и воздвигают мятежи, не давая царству достигнуть благосостояния.
\vs 2Ma 14:7 Посему я, лишенный чести предков моих, то есть священноначалия, пришел теперь сюда,
\vs 2Ma 14:8 во-первых, искренно радея о том, что принадлежит царю, во-вторых, имея в виду своих сограждан; ибо от безрассудства названных людей немало бедствует весь род наш.
\vs 2Ma 14:9 Ты же, царь, узнав обо всем этом, попекись о стране и об угнетенном роде нашем, по доступному для всех человеколюбию твоему:
\vs 2Ma 14:10 доколе остается Иуда, не может быть спокойствия.
\vs 2Ma 14:11 Когда это было сказано им, прочие советники, имевшие неприязнь к Иуде, еще более возбудили Димитрия.
\vs 2Ma 14:12 Он тотчас призвал Никанора, заведовавшего слонами, и, назначив его военачальником в Иудею, послал его,
\vs 2Ma 14:13 дав приказание, чтобы Иуду умертвить, сообщников его рассеять, Алкима же поставить первосвященником великого храма.
\vs 2Ma 14:14 Тогда язычники, бежавшие из Иудеи от Иуды, толпами сходились к Никанору в надежде, что несчастья и беды Иудеев сделаются их благоденствием.
\vs 2Ma 14:15 \bibemph{Иудеи} же, услышав о походе Никанора и присоединении к нему язычников, посыпали головы землею и молились Тому, Который до века установил народ Свой и всегда видимо защищал удел Свой.
\vs 2Ma 14:16 По повелению вождя своего они поспешно поднялись оттуда и сошлись с ними при селении Дессау.
\vs 2Ma 14:17 Симон, брат Иуды, вступил в бой с Никанором, но вскоре, при внезапном наступлении противников, потерпел небольшое поражение.
\vs 2Ma 14:18 Впрочем, Никанор, слышав, какую храбрость имели находившиеся с Иудою и какую отважность в битвах за отечество, побоялся решить дело кровопролитием;
\vs 2Ma 14:19 посему послал Посидония, Феодота и Маттафию~--- заключить с Иудеями мир.
\vs 2Ma 14:20 После долгого рассуждения о сем и когда вождь сообщил о том народу, состоялось единодушное мнение, и они согласились на переговоры
\vs 2Ma 14:21 и назначили день, в который бы сойтись им вместе наедине, и когда он наступил, поставили для каждого особые седалища.
\vs 2Ma 14:22 Иуда же поставил в удобных местах вооруженных людей в готовности, дабы от врагов внезапно не последовало какого-нибудь злодейства,~--- и имели они мирное совещание.
\vs 2Ma 14:23 Никанор пробыл в Иерусалиме несколько времени, и не сделал ничего неуместного, и отпустил собранный народ.
\vs 2Ma 14:24 Он постоянно имел Иуду с собою и душевно расположился к этому мужу;
\vs 2Ma 14:25 убедил его жениться, чтобы рождать детей. Иуда женился, успокоился и наслаждался жизнью.
\vs 2Ma 14:26 Алким же, видя взаимное их друг ко другу расположение и состоявшийся между ними союз, собрался с духом, пришел к Димитрию и сказал, что Никанор имеет враждебные для царства намерения, ибо назначил Иуду, злоумышленника против царства, своим преемником.
\vs 2Ma 14:27 Царь, разгневанный и раздраженный этими клеветами злодея, писал к Никанору, выражая, что ему тяжело переносить такой договор, и приказывал тотчас же прислать Маккавея в Антиохию в оковах.
\vs 2Ma 14:28 Когда узнал об этом Никанор, то смутился и огорчен был тем, что должен был отвергнуть установленный союз с человеком, который не сделал ничего несправедливого.
\vs 2Ma 14:29 Но как нельзя было противиться царю, то он выжидал благоприятного случая исполнить это хитростью.
\rsbpar\vs 2Ma 14:30 Маккавей же, заметив, что Никанор начал обходиться с ним суровее и в обычных встречах стал грубее, и заключив, что не от доброго происходит эта суровость, и собрав немалое число из находившихся при нем, скрылся от Никанора.
\vs 2Ma 14:31 Когда последний узнал, что Иуда искусно предварил его хитростью, то пришел в великий и святый храм, когда священники приносили установленные жертвы, и приказывал, чтобы они выдали того мужа.
\vs 2Ma 14:32 Когда же они с клятвою говорили, что не знают, где находится тот, кого он ищет,
\vs 2Ma 14:33 то он, простерши правую руку на храм, поклялся, сказав: если вы не выдадите мне Иуду связанным, то я этот храм Божий сравняю с землею, раскопаю жертвенник и воздвигну здесь славный храм Дионису.
\vs 2Ma 14:34 Сказав это, он удалился. Священники же, простирая руки к небу, умоляли всегдашнего Защитника народа нашего и говорили:
\vs 2Ma 14:35 Ты, Господи, не имея ни в чем нужды, благоволил храму сему быть местом Твоего обитания между нами.
\vs 2Ma 14:36 И ныне, Святый Господь всякой святыни, сохрани навеки неоскверненным сей недавно очищенный дом и загради уста неправедные.
\rsbpar\vs 2Ma 14:37 Никанору же указали на некоего Разиса из Иерусалимских старейшин как на друга граждан, имевшего весьма добрую славу и за свое доброжелательство прозванного отцом Иудеев.
\vs 2Ma 14:38 Он в предшествовавшие смутные времена стоял на стороне Иудейства и со всем усердием отдавал за Иудейство и тело и душу.
\vs 2Ma 14:39 Никанор, желая показать, какую он имеет ненависть против Иудеев, послал более пятисот воинов, чтобы схватить его,
\vs 2Ma 14:40 ибо думал, что, взяв его, причинит им несчастье.
\vs 2Ma 14:41 Когда же толпа хотела овладеть башнею и врывалась в ворота двора и уже приказано было принести огня, чтобы зажечь ворота, тогда он, в неизбежной опасности быть захваченным, пронзил себя мечом,
\vs 2Ma 14:42 желая лучше доблестно умереть, нежели попасться в руки беззаконников и недостойно обесчестить свое благородство.
\vs 2Ma 14:43 Но как удар оказался от поспешности неверен, а толпы уже вторгались в двери, то он, отважно вбежав на стену, мужественно бросился с нее на толпу народа.
\vs 2Ma 14:44 Когда же стоявшие поспешно расступились и осталось пустое пространство, то он упал в средину на чрево.
\vs 2Ma 14:45 Дыша еще и сгорая негодованием, несмотря на лившуюся ручьем кровь и тяжелые раны, встал и, пробежав сквозь толпу народа, остановился на одной крутой скале.
\vs 2Ma 14:46 Совершенно уже истекая кровью, он вырвал у себя внутренности и, взяв их обеими руками, бросил в толпу и, моля Господа жизни и духа опять дать ему жизнь и дыхание, кончил таким образом жизнь.
\vs 2Ma 15:1 Когда узнал Никанор, что бывшие с Иудою находятся в стране Самарийской, то думал совершенно безнаказанно напасть на них в день покоя.
\vs 2Ma 15:2 Когда же поневоле сопровождавшие его Иудеи говорили: <<Не губи их так жестоко и бесчеловечно, воздай честь дню, освященному Всевидящим>>;
\vs 2Ma 15:3 тогда этот нечестивец спросил: <<Неужели есть Владыка на небе, повелевший праздновать день субботний?>>
\vs 2Ma 15:4 И когда они отвечали: <<Есть живый Господь, Владыка небесный, повелевший чтить седьмой день>>,
\vs 2Ma 15:5 то он сказал: <<А я~--- господин на земле, повелевающий взять оружие и исполнять царскую службу>>. Впрочем, он не успел совершить своего умысла.
\vs 2Ma 15:6 Превозносясь с великою гордостью, Никанор думал одержать всеобщую победу над бывшими с Иудою.
\vs 2Ma 15:7 Маккавей же не переставал надеяться с полною уверенностью, что получит заступление от Господа.
\vs 2Ma 15:8 Он убеждал бывших с ним не страшиться нашествия язычников, но, воспоминая прежде бывшие опыты небесной помощи, и ныне ожидать себе победы и помощи от Вседержителя.
\vs 2Ma 15:9 Утешая их обетованиями закона и пророков, припоминая им подвиги, совершенные ими самими, он одушевил их мужеством.
\vs 2Ma 15:10 Возбуждая дух их, он убеждал их, указывая притом на вероломство язычников и нарушение ими клятв.
\vs 2Ma 15:11 Вооружил же он каждого не столько крепкими щитами и копьями, сколько убедительными добрыми речами, и притом всех обрадовал рассказом о достойном вероятия сновидении.
\rsbpar\vs 2Ma 15:12 Видение же его было такое: он видел Онию, бывшего первосвященника, мужа честного и доброго, почтенного видом, кроткого нравом, приятного в речах, издетства ревностно усвоившего все, что касалось добродетели,~--- видел, что он, простирая руки, молится за весь народ Иудейский.
\vs 2Ma 15:13 Потом явился другой муж, украшенный сединами и славою, окруженный дивным и необычайным величием.
\vs 2Ma 15:14 И сказал Ония: это братолюбец, который много молится о народе и святом городе, Иеремия, пророк Божий.
\vs 2Ma 15:15 Тогда Иеремия, простерши правую руку, дал Иуде золотой меч и, подавая его, сказал:
\vs 2Ma 15:16 возьми этот святый меч, дар от Бога, которым ты сокрушишь врагов.
\rsbpar\vs 2Ma 15:17 Утешенные столь добрыми речами Иуды, которые могли возбуждать к мужеству и укреплять сердца юных, Иудеи решились не располагаться станом, а отважно напасть и, с полным мужеством вступив в бой, решить дело, ибо город и святыня и храм находились в опасности.
\vs 2Ma 15:18 Борьба за жен и детей, братьев и родных казалась им делом менее важным; величайшее и преимущественное опасение было за святый храм.
\vs 2Ma 15:19 Для тех, которые остались в городе, также немало было беспокойства, ибо они тревожились о сражении, имеющем быть в поле.
\rsbpar\vs 2Ma 15:20 Итак, когда все ожидали, что наступает решение дела, когда враги уже соединились и войско было поставлено в строй, слоны размещены в надлежащих местах и конница расположена по сторонам,~---
\vs 2Ma 15:21 Маккавей, видя наступление многочисленного войска, пестроту приготовленного оружия и свирепость зверей, простер руки к небу и призывал Господа, творящего чудеса и всевидящего, зная, что не оружием одерживается победа, но Сам Он, как Ему угодно, дарует победу достойным.
\vs 2Ma 15:22 В молитве своей он так говорил: Ты, Господи, при Езекии, царе Иудейском, послал Ангела,~--- и он поразил из полка Сеннахиримова сто восемьдесят пять тысяч.
\vs 2Ma 15:23 И ныне, Господи небес, пошли доброго Ангела пред нами на страх и трепет врагам.
\vs 2Ma 15:24 Силою мышцы Твоей да будут поражены пришедшие с хулением на святый народ Твой. Сим он кончил.
\vs 2Ma 15:25 Бывшие с Никанором шли со звуком труб и криками,
\vs 2Ma 15:26 а находившиеся с Иудою с призыванием и молитвами вступили в сражение с неприятелями.
\vs 2Ma 15:27 Руками сражаясь, а сердцами молясь Богу, они избили не менее тридцати пяти тысяч, весьма обрадованные видимою помощью Божиею.
\vs 2Ma 15:28 Окончив дело и радостно возвращаясь, они узнали, что Никанор пал в своем всеоружии.
\vs 2Ma 15:29 Когда крик и шум утихли, они восхвалили Господа на отечественном языке.
\vs 2Ma 15:30 Тогда Иуда, первоподвижник за сограждан и телом и душею и лучшие лета свои сохранивший для одноплеменников, дал приказание, чтобы отсекли голову Никанора и руку с плечом и несли в Иерусалим.
\vs 2Ma 15:31 Придя туда, он созвал одноплеменников и поставил пред жертвенником священников, призвал и тех, которые находились в крепости,
\vs 2Ma 15:32 и, показав голову скверного Никанора и руку злохульника, которую он простирал на святый дом Вседержителя и превозносился,
\vs 2Ma 15:33 приказал вырезать язык у нечестивого Никанора и, раздробив его, разбросать птицам, руку же безумца повесить против храма.
\vs 2Ma 15:34 Тогда все, обращаясь к небу, прославляли явившего помощь Господа и говорили: благословен Сохранивший неоскверненным место Свое!
\vs 2Ma 15:35 Голову же Никанора повесил он на крепости в видимое для всех и ясное знамение помощи Господней.
\vs 2Ma 15:36 И все общим приговором определили: никогда не оставлять без торжества день сей, чтить же празднеством тринадцатый день двенадцатого месяца, называемого на Сирском языке Адаром, за день до дня Мардохеева.
\rsbpar\vs 2Ma 15:37 Так окончилось дело с Никанором; и как с того времени город остался во власти Евреев, то я и кончу здесь мое слово.
\vs 2Ma 15:38 Если я изложил его хорошо и удовлетворительно, то я сего и желал; если же слабо и посредственно, то я сделал то, что было по силам моим.
\vs 2Ma 15:39 Неприятно пить особо вино и тотчас же особо воду, между тем вино, смешанное с водою, сладко и доставляет удовольствие; так и состав сочинения приятно занимает слух читателя при соразмерности. Здесь да будет конец.

\bibbookdescr{3Ma}{
  inline={\LARGE Третья книга\\\Huge Маккавейская\fns{Книги Маккавейские переведены с греческого, потому что в еврейском тексте их нет.}},
  toc={3-я Маккавейская*},
  bookmark={3-я Маккавейская},
  header={3-я Маккавейская},
  %headerleft={},
  %headerright={},
  abbr={3~Мак}
}
\vs 3Ma 1:1 Филопатор, узнав от прибывших к нему, что Антиохом отняты бывшие в его владении местности, отдал приказ всем войскам своим, пешим и конным, и, взяв с собою сестру свою Арсиною, отправился в страну Рафию, где расположены были станом войска Антиоха.
\vs 3Ma 1:2 Тогда некто Феодот решился исполнить свой замысел, взял с собою лучших из вверенных ему Птоломеем вооруженных людей и ночью проник в палатку Птоломея, чтобы наедине убить его и тем предотвратить войну.
\vs 3Ma 1:3 Но его обманул Досифей, сын Дримила, родом Иудей, впоследствии изменивший закону и отступивший от отеческой веры: он поместил в палатке одного незначительного человека, которому и пришлось принять назначенную Птоломею смерть.
\vs 3Ma 1:4 Когда же произошло упорное сражение и дело Антиоха превозмогало, то Арсиноя, распустив волосы, с плачем и слезами ходила по войскам, усильно убеждая, чтобы храбрее сражались за себя, за детей и жен, и обещая, если победят, дать каждому по две мины золота.
\vs 3Ma 1:5 И так случилось, что противники поражены были в рукопашном бое, и многие взяты в плен.
\vs 3Ma 1:6 Достигнув своей цели, Филопатор рассудил пройти по ближним городам, чтобы ободрить их.
\vs 3Ma 1:7 Исполнив это и снабдив капища дарами, он одушевил мужеством подвластных ему.
\rsbpar\vs 3Ma 1:8 Когда потом Иудеи отправили к нему от совета и старейшин послов поздравить его, поднести дары и изъявить радость о случившемся, то он пожелал как можно скорее прийти к ним.
\vs 3Ma 1:9 Прибыв же в Иерусалим, он принес жертву великому Богу, воздал благодарение и прочее исполнил, приличествующее священному месту;
\vs 3Ma 1:10 и когда вошел туда, то изумлен был величием и благолепием и, удивляясь благоустройству храма, пожелал войти во святилище.
\vs 3Ma 1:11 Ему сказали, что не следует этого делать, ибо никому и из своего народа непозволительно входить туда, и даже священникам, но только одному начальствующему над всеми первосвященнику, и притом однажды в год; но он никак не хотел слушать.
\vs 3Ma 1:12 Прочитали ему закон, но и тогда не оставил он своего намерения, говоря, что он должен войти: пусть они будут лишены этой чести, но не я. И спрашивал, почему, когда он входил в храм, никто из присутствовавших не возбранил ему?
\vs 3Ma 1:13 И когда некто неосмотрительно сказал, что это худо было сделано, он отвечал: но когда это уже сделано, по какой бы то ни было причине, то не должно ли ему во всяком случае войти, хотят ли они того, или не хотят.
\vs 3Ma 1:14 Тогда священники в священных одеждах пали ниц и молились великому Богу, чтобы Он помог им в настоящей крайности и удержал стремление насильственно вторгающегося; храм наполнился воплем и слезами, а остававшиеся в городе сбежались в смущении, полагая, что случилось нечто необычайное.
\vs 3Ma 1:15 И заключенные в своих покоях девы выбегали с матерями и, посыпая пеплом и прахом головы, оглашали улицы рыданиями и стонами.
\vs 3Ma 1:16 Другие же во всем наряде, оставив приготовленный для встречи брачный чертог и подобающий стыд, беспорядочно бегали по городу.
\vs 3Ma 1:17 А матери и кормилицы, оставляя и здесь и там новорожденных детей, иные в домах, другие~--- на улицах, неудержимо сбегались во всесвятейший храм.
\vs 3Ma 1:18 Так разнообразна была молитва собравшихся по случаю святотатственного покушения.
\vs 3Ma 1:19 Вместе с тем некоторые из граждан возымели смелость не допускать домогавшегося вторгнуться и исполнить свое намерение. Они воззвали, что нужно взяться за оружие и мужественно умереть за закон отеческий, и произвели в храме великое смятение:
\vs 3Ma 1:20 с трудом быв удержаны старейшинами и священниками, они остались в том же молитвенном положении.
\vs 3Ma 1:21 Народ, как и прежде, продолжал молиться. Даже бывшие с царем старейшины многократно пытались отвлечь надменный его ум от предпринятого намерения.
\vs 3Ma 1:22 Но, исполненный дерзости и все пренебрегший, он уже делал шаг вперед, чтобы совершенно исполнить сказанное прежде.
\vs 3Ma 1:23 Видя это, и бывшие с ним начали призывать вместе с нашими Вседержителя, чтобы Он помог в настоящей нужде и не попустил такого беззаконного и надменного поступка.
\vs 3Ma 1:24 От совокупного, напряженного и тяжкого народного вопля происходил невыразимый гул.
\vs 3Ma 1:25 Казалось, что не только люди, но и самые стены и все основания вопияли, как бы умирая уже за осквернение священного места.
\vs 3Ma 2:1 А первосвященник Симон, преклонив колени пред святилищем и благоговейно распростерши руки, творил молитву:
\vs 3Ma 2:2 <<Господи, Господи, Царь небес и Владыка всякого создания, Святый во святых, Единовластвующий, Вседержитель! Призри на нас, угнетаемых от безбожника и нечестивца, надменного дерзостью и силою.
\vs 3Ma 2:3 Ибо Ты, все создавший и всем управляющий, праведный Владыка: Ты судишь тех, которые делают что-либо с дерзостью и превозношением.
\vs 3Ma 2:4 Ты некогда погубил делавших беззаконие, между которыми были исполины, надеявшиеся на силу и дерзость, и навел на них безмерную воду.
\vs 3Ma 2:5 Ты сожег огнем и серою Содомлян, поступавших надменно, явно делавших зло, и поставил их в пример потомкам.
\vs 3Ma 2:6 Ты дерзкого фараона, поработившего Твой святый народ, Израиля, посетил различными и многими казнями, явил Твою власть и показал Твою великую силу.
\vs 3Ma 2:7 И когда он погнался за ним, Ты потопил его с колесницами и множеством народа во глубине моря, а тех, которые надеялись на Тебя, Владыку всякого создания, Ты провел невредимо, и они, увидев дела руки Твоей, восхвалили Тебя, Вседержителя.
\vs 3Ma 2:8 Ты, Царь, создавший беспредельную и неизмеримую землю, избрал этот город, и освятил это место во славу Тебе, ни в чем не имеющему нужды, и прославил его Твоим величественным явлением, обращая его к славе Твоего великого и досточтимого имени.
\vs 3Ma 2:9 По любви к дому Израилеву Ты обещал, что, если постигнет нас несчастье и обымет угнетенье и мы, придя на место сие, помолимся, Ты услышишь молитву нашу.
\vs 3Ma 2:10 И Ты верен и истинен, и много раз, когда отцы наши подвергались бедствиям, Ты помогал им в их скорби и избавлял их от великих опасностей.
\vs 3Ma 2:11 Вот и мы, Святый Царь, за многие и великие грехи наши бедствуем, преданы врагам нашим и изнемогли от скорбей.
\vs 3Ma 2:12 В таком упадке нашем этот дерзкий нечестивец покушается оскорбить это святое место, посвященное на земле славному имени Твоему.
\vs 3Ma 2:13 Ибо, хотя жилище Твое, небо небес, недостижимо для людей, но Ты, благоволив явить славу Твою народу Твоему, Израилю, освятил место сие.
\vs 3Ma 2:14 Не отмщай нам за нечистоту их и не накажи нас за осквернение, чтобы не тщеславились беззаконники в мыслях своих и не торжествовали в превозношении языка своего, говоря: мы попрали дом святыни, как попираются домы скверны.
\vs 3Ma 2:15 Оставь грехи наши, отпусти неправды наши и яви милость Твою в час сей; скоро да предварят нас щедроты Твои; дай хвалу устам упадших духом и сокрушенных сердцем; даруй нам мир>>.
\vs 3Ma 2:16 Тогда всевидящий Бог и над всеми Святый во святых, услышав молитву смирения, поразил надмевавшегося насилием и дерзостью, сотрясая его туда и сюда, как тростник ветром, так что он, лежа недвижим на помосте и будучи расслаблен членами, не мог подать даже голоса, постигнутый праведным судом.
\vs 3Ma 2:17 Тогда его друзья и телохранители, видя внезапную и тяжкую казнь, постигшую его, и опасаясь, чтобы он не лишился жизни, поспешно вынесли его, будучи сами поражены чрезвычайным страхом.
\vs 3Ma 2:18 Через несколько времени, придя в себя после испытанного наказания, он нисколько не пришел в раскаяние и удалился с жестокими угрозами.
\rsbpar\vs 3Ma 2:19 Возвратившись в Египет и умножая дела своей злобы, он с упомянутыми участниками в пиршествах и друзьями, забывшими всякую справедливость, не только пресыщался бесчисленными студодействами, но дошел до такой дерзости, что произносил там проклятие \bibemph{на Иудеев}, и многие из друзей его, смотря на пример царя, и сами следовали его желаниям.
\vs 3Ma 2:20 Наконец он решился публично предать позору народ \bibemph{Иудейский}, и поставил на башне своего дворца столб, сделав на нем надпись: <<Кто не приносит жертв, тому не входить в свои священные места; Иудеев же всех внести в перепись простого народа и зачислить в рабское состояние, а кто будет противиться, тех брать силою и лишать жизни;
\vs 3Ma 2:21 внесенных же в перепись отмечать, выжигая им на теле знак Диониса~--- лист плюща, после чего отпускать их в назначенное им состояние с ограниченными правами>>.
\vs 3Ma 2:22 Но чтобы не сделаться ненавистным для всех, он прибавил в надписи, что, если кто из них пожелает жить по обрядам языческим, тем давать равные права с Александрийскими гражданами.
\vs 3Ma 2:23 Посему некоторые, ради права гражданского презрев отечественное благочестие, поспешно передались, как будто могли они от будущего общения с царем приобщиться великой славы.
\vs 3Ma 2:24 Но б\acc{о}льшая часть укрепились мужеством духа и не отпали от благочестия; они отдавали деньги за жизнь свою, и небоязненно пытались избавиться от записи, имея добрую надежду получить помощь, и от отпавших отвращались, почитая их врагами \bibemph{своего} народа и избегая всякого общения с ними и дружественного обхождения.
\vs 3Ma 3:1 Узнав о том, нечестивец пришел в такое неистовство, что не только озлобился против Иудеев, живших в Александрии, но обнаружил жестокую вражду и против обитавших в целой стране, приказав немедленно собрать всех вместе и предать позорнейшей смерти.
\vs 3Ma 3:2 Когда готовилось это дело, распространен был людьми, одномысленными злодейству, злой слух против народа Иудейского по поводу к такому распоряжению, будто они уклоняются от исполнения законных обязанностей.
\vs 3Ma 3:3 Между тем Иудеи хранили доброе расположение и неизменную верность к царям; но они почитали Бога, жили по Его закону и потому в некоторых случаях допускали отступления и отмены: по этой причине они и казались некоторым враждебными; у всех же других людей добрым исполнением всего справедливого они приобретали благоволение.
\vs 3Ma 3:4 Несмотря на то, известный добрый образ жизни этого народа иноплеменники считали ни во что. Они замечали только различие в богопочтении и пище и говорили, что эти люди не допускают общения трапезы ни с царем, ни с вельможами, что они завистники и великие противники государства, и таким образом разглашали о них намеренные хулы.
\vs 3Ma 3:5 Жившие в городе Еллины, не испытавшие от них никакой обиды, видя неожиданное волнение против этих людей и внезапное их стечение, хотя не могли помочь им,~--- ибо царское было распоряжение,~--- однако утешали их, негодовали и надеялись, что дело переменится:
\vs 3Ma 3:6 ибо нельзя было пренебрегать таким множеством народа, ни в чем не повинного.
\vs 3Ma 3:7 Впрочем, некоторые соседи и друзья и производившие с ними торговлю, тайно принимая некоторых из них, обещали помогать им и делать все возможное к их защите.
\vs 3Ma 3:8 А он, надмеваясь временным благополучием и не помышляя о власти величайшего Бога, думал неизменно остаться в том же умысле и написал против них такое письмо:
\vs 3Ma 3:9 <<Царь Птоломей Филопатор обитателям Египта и местным военачальникам и воинам~--- радоваться и здравствовать. Я же сам здоров, и дела наши благоуспешны.
\vs 3Ma 3:10 После похода, предпринятого нами в Азию, который, как вы сами знаете, неожиданною помощью богов и нашею силою, согласно нашему намерению, достиг счастливого окончания, мы думали благоустроить народы, обитающие в Келе-Сирии и Финикии, не силою оружия, но снисхождением и великим человеколюбием, охотно благодетельствуя им.
\vs 3Ma 3:11 Давая по городам богатые вклады в храмы, мы пришли и в Иерусалим, положив почтить святилище этих негодных людей, никогда не оставляющих своего безумия.
\vs 3Ma 3:12 Они же, приняв наше прибытие на словах охотно, а на деле коварно, когда мы желали войти в храм и почтить его подобающими и наилучшими дарами, напыщенные своею древнею гордостью, возбранили нам вход, не потерпев от нас насилия по человеколюбию, какое мы имеем ко всем людям.
\vs 3Ma 3:13 Явно обнаружив свою враждебность против нас, они одни только из всех народов упорно противятся царям и своим благодетелям и не хотят исполнять ничего справедливого.
\vs 3Ma 3:14 Мы же, снисходя их безумию, и тогда, как возвращались с победою, и в самом Египте, принимая человеколюбиво все народы, поступали, как надлежало.
\vs 3Ma 3:15 Между прочим, объявляя всем о нашем непамятозлобии к их одноплеменникам, мы решились ввести перемены: так как они служили нам на войне и занимались весьма многими делами, издавна по простоте предоставленными им, то мы хотели даже удостоить их прав Александрийского гражданства и сделать участниками исконного жречества.
\vs 3Ma 3:16 Они же, приняв это в противность себе и, по сродному им злонравию, отвергая доброе и склоняясь всегда к худому, не только презрели неоценимое право гражданства, но и гласно и негласно гнушаются тех немногих из них, которые искренно расположены к нам, постоянно надеясь, что мы вследствие беспорядочного образа жизни их скоро отменим наши установления.
\vs 3Ma 3:17 Посему мы, достаточно убедившись опытами, что они при всяком случае питают неприязненные против нас замыслы, и предвидя, что когда-нибудь, при возникшем неожиданно против нас возмущении, мы будем иметь за собою в лице этих нечестивцев предателей и жестоких врагов,
\vs 3Ma 3:18 повелеваем, как скоро будет получено это письмо, тотчас упомянутых нами людей с их женами и детьми, с насилиями и истязаниями заключив в железные оковы, отовсюду выслать к нам на смертную казнь, беспощадную и позорную, достойную таких злоумышленников.
\vs 3Ma 3:19 Если они в один раз будут наказаны, то мы надеемся, что на будущее время наши государственные дела придут в совершенное благоустройство и наилучший порядок.
\vs 3Ma 3:20 Если же кто укроет кого из Иудеев, от старика до ребенка, не исключая грудных младенцев, должен быть истреблен со всем его домом жесточайшим образом.
\vs 3Ma 3:21 А кто откроет кого-либо, тот получит имение виновного и еще две тысячи драхм из царской казны, получит свободу и будет почтен.
\vs 3Ma 3:22 Всякое место, где будет пойман укрывающийся Иудей, должно быть опустошено и выжжено, так чтобы никому из смертных ни на что не было годно на вечные времена>>. Таков был смысл письма.
\vs 3Ma 4:1 Везде, куда приходило это повеление, у язычников учреждались народные пиршества с радостными кликами, как будто закореневшая издавна в душе вражда теперь обнаружилась дерзновенно.
\vs 3Ma 4:2 А у Иудеев началась неутешная скорбь, горький плач и рыдание; ибо жгли сердце достигавшие со всех сторон стоны оплакивающих неожиданную, внезапно определенную им погибель.
\vs 3Ma 4:3 Какая область или город, или какое обитаемое место, или какие дороги не наполнились их плачем и воплями?
\vs 3Ma 4:4 Жестоко и без всякой жалости они были вместе высылаемы властями каждого города, так что при виде этой необыкновенной кары и некоторые из врагов, смотря на общее страдание и помышляя о неведомой превратности жизни, оплакивали злополучнейшее их изгнание.
\vs 3Ma 4:5 Гнали толпу престарелых, покрытых сединами, сгорбленных от старческой слабости в ногах, и по требованию насильственного изгнания бесстыдно принуждали их к скорейшему шествию.
\vs 3Ma 4:6 Отроковицы, только что сочетавшиеся супружеским союзом и вошедшие в брачный чертог, вместо ликования начали плач, посыпали пеплом благоухавшие от мастей волосы, были ведены непокрытыми и вместо брачных песней поднимали общий вопль, будучи мучимы истязаниями иноплеменных. В оковах они открыто влекомы были с насилием, до ввержения в корабль.
\vs 3Ma 4:7 А их супруги, вместо венков перевязанные по шеям веревками, в цветущем юношеском возрасте, вместо пиршества и наслаждения молодости, проводили остальные дни брака в плаче, ибо под ногами у себя видели открытый ад.
\vs 3Ma 4:8 Везены они были по подобию зверей под игом железных оков; одни прикованы были за шеи к корабельным скамьям, другие крепкими узами привязаны были за ноги. Кроме того, накрытые плотным помостом, они отлучены были от света, так что, со всех сторон окруженные тьмою, во все время плавания содержались подобно злоумышленникам.
\vs 3Ma 4:9 Когда же они привезены были на место, называемое Схедия, и плавание было окончено, как назначено было царем, тогда он приказал поставить их перед городом на конском ристалище, которое имело обширную окружность и весьма удобно было для примерного поругания в виду всех, шедших в город и обратно отправлявшихся внутрь страны, так чтобы они ни с войском не имели сообщения, ни вообще не были удостоены никакого крова.
\vs 3Ma 4:10 Когда это было исполнено и \bibemph{царь} услышал, что одноплеменники их часто выходят тайно из города оплакивать позорное бедствие братьев, то весьма разгневался и приказал и с этими поступить точно так же, как и с теми, чтобы они никак не меньшее получили наказание.
\vs 3Ma 4:11 Он велел переписать весь народ по именам, не для тяжкого рабского служения, незадолго пред сим возвещенного, а для того, чтобы, измучив их объявленными казнями, вконец погубить в один день.
\vs 3Ma 4:12 И хотя эта перепись производилась с крайнею поспешностью и ревностным старанием от восхода до захождения солнца, но совершенно окончить ее не могли в продолжение сорока дней.
\vs 3Ma 4:13 Царь же, чрезмерно и непрестанно предаваясь удовольствию, пред всеми идолами учреждал пиршества, и умом, далеко уклонившимся от истины, и нечистыми устами славословил тех, которые глухи и не могут говорить или подать помощи, а на величайшего Бога произносил неподобающее.
\rsbpar\vs 3Ma 4:14 После сказанного промежутка времени писцы донесли царю, что они не в состоянии сделать переписи Иудеев, по причине бесчисленного их множества; притом еще большее число их находится в областях; одни остаются в домах, другие рассеяны по разным местам, так что сделать этого невозможно даже всем властям в Египте.
\vs 3Ma 4:15 Когда же царь еще строже угрожал им, предполагая, что они подкуплены дарами и коварно избегали наказания, тогда пришлось осязательно убедить его в том. Они доказали, что недостает у них ни хартий, ни необходимых для того письменных тростей.
\vs 3Ma 4:16 Это было действие непобедимого небесного Промысла, помогавшего Иудеям.
\vs 3Ma 5:1 Тогда царь, исполненный сильного гнева и неизменный в своей ненависти, призвал Ермона, заведовавшего слонами, и приказал на следующий день всех слонов, числом пятьсот, накормить ладаном в возможно больших приемах и вдоволь напоить цельным вином, и когда они рассвирепеют от данного им в изобилии питья, вывести их на Иудеев, обреченных встретить смерть.
\vs 3Ma 5:2 Дав такое приказание, он отправился на пиршество, пригласив особенно тех из своих друзей и воинов, которые враждовали против Иудеев; а Ермон, начальствующий над слонами, в точности исполнил его повеление.
\vs 3Ma 5:3 Назначенные при этом служители пошли вечером вязать руки несчастным и другие принимали против них предосторожности, думая, что через ночь весь народ подвергнется конечной гибели.
\vs 3Ma 5:4 Иудеи же, казавшиеся язычникам лишенными всякой защиты, ибо отовсюду стеснены они были тяжкими узами, призывали всемогущего Господа, властвующего над всякою властью, своего милосердого Бога и Отца, призывали все непрестающим воплем со слезами, умоляя отвратить от них нечестивый умысел и спасти их от приготовленной им смерти Своим славным явлением.
\rsbpar\vs 3Ma 5:5 Прилежное моление их взошло на небо. Ермон, напоив неукротимых слонов, после обильной дачи им вина и ладана, утром явился во дворец донести о сем царю.
\vs 3Ma 5:6 Но Бог послал царю крепкий сон, этот добрый дар, от века ниспосылаемый Им и в нощи и во дни всем, кому Он хочет.
\vs 3Ma 5:7 Божиим устроением погруженный в приятный и глубокий сон, он забыл о своем беззаконном предприятии и совершенно обманулся в своем непременном решении.
\vs 3Ma 5:8 Иудеи же, избавившись предназначенного часа, восхваляли святаго Бога своего и снова умоляли Благопримирительного показать гордым язычникам силу всемогущей десницы Своей.
\vs 3Ma 5:9 Когда прошла уже половина десятого часа, служитель, которому поручены были приглашения, видя, что приглашенные уже собрались, вошел к царю будить его. С трудом разбудив его, он объявил, что время пиршества проходит, и дал отчет в своем поручении. Поверив его и отправившись пить, царь приказал пришедшим на пир возлечь прямо против себя.
\vs 3Ma 5:10 Когда это было исполнено, он поощрял собравшихся на пиршество проводить настоящую часть пиршества в полном веселье.
\vs 3Ma 5:11 Во время продолжительной беседы царь, призвав Ермона, строго и грозно спрашивал, по какой причине Иудеи допущены пережить настоящий день?
\vs 3Ma 5:12 Тот объявил, что еще ночью исполнил порученное ему, и друзья царя подтвердили это. Тогда \bibemph{царь}, в жестокости лютый более, нежели Фаларис, сказал, что они должны быть благодарны сегодняшнему сну:
\vs 3Ma 5:13 <<А ты непременно на завтрашний день так же приготовь слонов на истребление беззаконных Иудеев>>.
\vs 3Ma 5:14 Когда царь сказал это, все присутствовавшие с удовольствием и радостью изъявили ему свое одобрение, и разошлись каждый в свой дом. Время ночи употреблено было не столько на сон, сколько на изобретение всяких поруганий над мнимыми преступниками.
\vs 3Ma 5:15 Рано утром, лишь только запел петух, Ермон вывел зверей и стал раздражать их на обширном дворе. В городе толпы народа собрались на плачевное зрелище, с нетерпением ожидая рассвета.
\vs 3Ma 5:16 Иудеи непрестанно, томясь духом, творили молитву со многими слезами и плачевными песнями и, простирая руки к небу, умоляли величайшего Бога опять послать им скорую помощь.
\vs 3Ma 5:17 Не распространились еще лучи солнца, и царь еще принимал своих друзей, как предстал пред ним Ермон и приглашал на выход, донося, что все готово, чего желал царь.
\vs 3Ma 5:18 Выслушав это и изумившись предложению необычного выхода, он совершенно обо всем забыл и спрашивал: что это за дело, которое он с такою поспешностью исполнил? Было же это действием властвующего над всем Бога, Который навел на ум его забвение обо всем, что он сам прежде придумал.
\vs 3Ma 5:19 Ермон и все друзья объясняли, говоря: царь! звери и войска приготовлены по твоему настоятельному повелению.
\vs 3Ma 5:20 Он же исполнился сильного гнева на такие речи,~--- ибо промыслом Божиим разрушено было все его умышление,~--- и, сверкая глазами, сказал с угрозою:
\vs 3Ma 5:21 если бы у тебя были родители или дети, то они послужили бы изобильною пищею для диких зверей вместо невинных Иудеев, которые мне и предкам моим сохраняли неизменную и совершенную верность. Если бы не привязанность моя к тебе по воспитанию и не заслуги твои, то ты вместо них был бы лишен жизни.
\vs 3Ma 5:22 Так встретил Ермон неожиданную и страшную угрозу и изменился во взоре и лице, а каждый из друзей вышел с неудовольствием, и всех собравшихся отпустили каждого на свое дело.
\vs 3Ma 5:23 Когда Иудеи услышали о такой благосклонности царя, то восхвалили Бога и Царя царей за помощь, полученную от Него.
\rsbpar\vs 3Ma 5:24 После таких решений царь опять учредил пиршество и приглашал предаться веселью. Призвав же Ермона, грозно сказал: сколько раз я должен приказывать тебе, негодный, об одном и том же? Вооружи опять слонов на утро для погубления Иудеев.
\vs 3Ma 5:25 Тогда возлежавшие вместе с ним родственники, удивляясь непостоянным его мыслям, сказали: долго ли, царь, ты будешь искушать нас как несмысленных, в третий раз повелевая истребить их, и опять, когда дойдет до дела, отменяешь и уничтожаешь свои повеления?
\vs 3Ma 5:26 От этого и город от ожидания находится в тревоге, наполняется толпами народа и часто подвергается опасности разграбления.
\vs 3Ma 5:27 После этого царь, совершенно, как Фаларис, исполнившись безрассудства и почитая за ничто происходившие в нем душевные перемены в пользу Иудеев,
\vs 3Ma 5:28 подтвердил нечестивейшею клятвою и определил немедленно послать их в ад, изувеченных ногами и ступнями зверей, затем предпринять поход на Иудею, вскоре опустошить ее огнем и мечом, и недоступный нам, говорил он, храм их сжечь огнем и сделать его навсегда пустым для всех, желающих приносить там жертвы.
\vs 3Ma 5:29 Тогда друзья и родственники, весьма обрадованные, разошлись с доверием и расположили в городе в удобнейших местах войска для стражи.
\vs 3Ma 5:30 А начальствующий над слонами, приведя зверей, можно сказать, в бешеное состояние благоуханным питьем вина, приправленного ладаном, вооружил их страшными орудиями, и рано утром, когда уже бесчисленные толпы стремились из города на конское ристалище, пришел он во дворец и напомнил царю о том, что предлежало исполнить.
\vs 3Ma 5:31 Царь же, полный сильного гнева, с нечестивым замыслом, вышел целым походом со зверями, желая по жестокости сердца видеть собственными глазами плачевную и бедственную гибель упомянутых людей.
\vs 3Ma 5:32 Когда Иудеи увидели пыль, поднимавшуюся от слонов, выходивших из ворот, и следовавшего с ними вооруженного войска и также от множества народа, и услышали сильно раздавшиеся клики, то подумали, что настала последняя минута их жизни и конец их несчастнейшего ожидания.
\vs 3Ma 5:33 Подняв плач и вопль, они целовали друг друга, обнимались с родными, бросаясь на шеи~--- отцы сыновьям, а матери дочерям,
\vs 3Ma 5:34 иные же держали при грудях новорожденных младенцев, сосавших последнее молоко.
\vs 3Ma 5:35 Зная, однако же, прежде бывшие им заступления с неба, они единодушно пали ниц, отняв от грудей младенцев,
\vs 3Ma 5:36 и громко взывали к Властвующему над всякою властью, умоляя Его помиловать их и явить помощь им, стоящим уже при вратах ада.
\vs 3Ma 6:1 Между тем некто Елеазар, уважаемый муж, из священников страны, уже достигший старческого возраста и украшенный в жизни своей всякою добродетелью, пригласил стоявших вокруг него старцев призывать святаго Бога и молился так:
\vs 3Ma 6:2 <<Царь всесильный, высочайший, Бог Вседержитель, милостиво управляющий всем созданием! призри, Отец, на семя Авраама, на детей освященного Иакова, на народ святаго удела Твоего, странствующий в земле чужой и неправедно погубляемый.
\vs 3Ma 6:3 Ты фараона, прежнего властителя Египта, имевшего множество колесниц, превознесшегося беззаконною дерзостью и высокомерными речами, погубил с гордым его войском, потопив в море, а роду Израильскому явил свет милости.
\vs 3Ma 6:4 Ты жестокого царя Ассирийского Сеннахирима, тщеславившегося бесчисленными войсками, покорившего мечом всю землю и восставшего на святый город Твой, в гордости и дерзости произносившего хулы, низложил, явно показав многим народам Твою силу.
\vs 3Ma 6:5 Ты трех отроков в Вавилоне, добровольно предавших жизнь свою огню, чтобы не служить суетным идолам, сохранил невредимыми до волоса, оросив разжженную печь, а пламень обратил на всех врагов.
\vs 3Ma 6:6 Ты Даниила, клеветами зависти вверженного в ров на растерзание львам, вывел на свет невредимым; Ты, Отец, и Иону, когда он безнадежно томился во чреве кита, обитающего во глубине моря, невредимым показал всем его присным.
\vs 3Ma 6:7 И ныне, Отмститель обид, многомилостивый, покровитель всех, явись вскоре сущим от рода Израилева, обидимым от гнусных беззаконных язычников.
\vs 3Ma 6:8 Если же жизнь наша в преселении наполнилась нечестием, то, избавив нас от руки врагов, погуби нас, Господи, какою Тебе благоугодно, смертью,
\vs 3Ma 6:9 да не славословят суеверы суетных идолов за погибель возлюбленных Твоих, говоря: не избавил их Бог их.
\vs 3Ma 6:10 Ты же, Вечный, имеющий всю силу и всякую власть, призри ныне:
\vs 3Ma 6:11 помилуй нас, по несмысленному насилию беззаконных лишаемых жизни, подобно злоумышленникам.
\vs 3Ma 6:12 Да устрашатся теперь язычники непобедимого могущества Твоего, Преславный, обладающий силою спасти род Иакова.
\vs 3Ma 6:13 Умоляет Тебя все множество младенцев и родители их со слезами: да будет явно всем язычникам, что с нами Ты, Господи, и не отвратил лица Твоего от нас;
\vs 3Ma 6:14 соверши так, как сказал Ты, Господи, что и в земле врагов их Ты не презришь их>>.
\rsbpar\vs 3Ma 6:15 Только что Елеазар окончил молитву, как царь со зверями и со всем страшным войском пришел на ристалище.
\vs 3Ma 6:16 Когда увидели его Иудеи, подняли громкий вопль к небу, так что и близлежащие долины огласились эхом, и возбудили неудержимое сострадание во всем войске.
\vs 3Ma 6:17 Тогда великославный Вседержитель и истинный Бог, явив святое лице Свое, отверз небесные врата, из которых сошли два славных и страшных Ангела, видимые всем, кроме Иудеев.
\vs 3Ma 6:18 Они стали против войска, и исполнили врагов смятением и страхом, и связали неподвижными узами; также и тело царя объял трепет, и раздраженную дерзость его постигло забвение.
\vs 3Ma 6:19 Тогда слоны обратились на сопровождавшие их вооруженные войска, попирали их и погубляли.
\vs 3Ma 6:20 Гнев царя превратился в жалость и слезы о том, что пред тем он ухищрялся исполнить.
\vs 3Ma 6:21 Ибо, когда услышал он крик \bibemph{Иудеев} и увидел их всех преклонившимися на погибель, то, заплакав, с гневом угрожал друзьям своим и говорил:
\vs 3Ma 6:22 вы злоупотребляете властью, и превзошли жестокостью тиранов, и меня самого, вашего благодетеля, покушаетесь лишить власти и жизни, замышляя тайно неполезное для царства.
\vs 3Ma 6:23 Тех, которые так верно охраняли укрепления нашей страны, кто безумно собрал сюда, удалив каждого из дома?
\vs 3Ma 6:24 Тех, которые издревле превосходили все народы преданностью нам во всем и часто терпели самые тяжкие угнетения от людей, кто подверг столь незаслуженному позору?
\vs 3Ma 6:25 Разрешите, разрешите неправедные узы, отпустите их с миром в свои домы, испросив прощение в том, что прежде сделано; освободите сынов небесного Вседержителя, живаго Бога, Который от времен наших предков доныне подавал непрерывное благоденствие и славу нашему царству.
\vs 3Ma 6:26 Вот что сказал царь. В ту же минуту разрешенные Иудеи, избавившись от смерти, прославляли своего святаго Спасителя Бога.
\rsbpar\vs 3Ma 6:27 После того царь, возвратившись в город и призвав заведующего расходами, приказал в продолжение семи дней давать Иудеям вино и прочее потребное для пиршества, положив, чтобы они на том же месте, на котором ожидали себе погибели, в полном веселье праздновали свое спасение.
\vs 3Ma 6:28 Тогда они, бывшие перед тем в поругании и находившиеся близ ада или, лучше, нисходившие в ад, вместо горькой и плачевной смерти учредили пиршество спасения и, полные радости, разделили для возлежания место, приготовленное им на погибель и могилу.
\vs 3Ma 6:29 Оставив жалостнейшую песнь плача, они начали песнь отцов, восхваляя Спасителя Израилева и Чудотворца Бога, и, отвергнув все сетование и рыдание, составили хоры в знамение мирного веселья.
\vs 3Ma 6:30 Равно и царь, составив по сему случаю многолюдное пиршество, выражал свою признательность к небу за славное, торжественно дарованное им спасение.
\vs 3Ma 6:31 Те же, которые обрекали их на погибель и на пищу хищным птицам и с радостью делали им перепись, теперь, объятые стыдом, восстенали, и дышавшая огнем дерзость угасла с позором.
\vs 3Ma 6:32 А Иудеи, как сказали мы, составив упомянутый хор, отправляли празднество с радостными славословиями и псалмопениями.
\vs 3Ma 6:33 Они сделали даже общественное постановление, чтобы во всяком населении их в роды и роды радостно праздновать означенные дни, не для питья и пресыщения, но в память бывшего им от Бога спасения.
\vs 3Ma 6:34 Потом они предстали царю и просили отпустить их в домы.
\vs 3Ma 6:35 Перепись их производилась с двадцать пятого дня месяца Пахона до четвертого дня месяца Епифа, в продолжение сорока дней; погубление их назначалось от пятого дня месяца Епифа до седьмого, в течение трех дней, в которые славным образом явил Свою милость Владыка всех и спас их невредимо и всецело.
\vs 3Ma 6:36 Праздновали они, довольствуемые всем от царя, до четырнадцатого дня, в который они и представили прошение об отпуске их.
\vs 3Ma 6:37 Царь, соизволив им, великодушно написал в их пользу, за своею подписью, следующее послание к городским начальникам:
\vs 3Ma 7:1 <<Царь Птоломей Филопатор начальникам Египетским и всем поставленным в должностях~--- радоваться и здравствовать. Здравствуем и мы и дети наши, ибо великий Бог благопоспешествует нам в делах по нашему желанию.
\vs 3Ma 7:2 Некоторые из друзей наших по злоумышлению своему часто представляли нам и убеждали нас собрать всех Иудеев, находящихся в царстве, и замучить необычайными казнями, как изменников,
\vs 3Ma 7:3 присовокупляя, что, доколе не будет этого сделано, дела нашего царства никогда не будут благоустроены по ненависти, которую питают они ко всем народам.
\vs 3Ma 7:4 Они-то привели их в оковах, с насилием, как невольников, или, лучше, как наветников, и без всякого рассмотрения и исследования покушались погубить их, изобретая жестокости, лютейшие даже Скифских обычаев.
\vs 3Ma 7:5 Мы строго воспретили это и по благоволению, которое питаем ко всем людям, тотчас даровали им жизнь; а когда узнали, что небесный Бог есть верный покров Иудеев и всегда защищает их, как отец сынов, еще же приняв во внимание известное их доброжелательство к нам и к предкам нашим, мы справедливо освободили их от всякого обвинения в чем бы то ни было
\vs 3Ma 7:6 и приказали всем и каждому возвратиться в свои домы, так чтобы нигде никто ни в чем не оскорблял их и не укорял в том, что произошло без их вины.
\vs 3Ma 7:7 Знайте, что если мы предпримем против них что-либо злое или вообще оскорбим их, то будем иметь против себя не человека, но властвующего над всякою властью всевышнего Бога отмстителем за дела наши во всем и всегда неизбежно. Будьте здравы>>.
\rsbpar\vs 3Ma 7:8 Получив это послание, Иудеи не спешили тотчас отправиться, но просили царя, чтобы те из рода Иудейского, которые самовольно оставили святаго Бога и закон Божий, получили через них должное наказание,
\vs 3Ma 7:9 присовокупляя, что преступившие ради чрева постановления Божественные никогда не будут иметь добрых расположений и к правлению царя.
\vs 3Ma 7:10 Царь нашел, что они говорят правду, одобрил их и дал им полномочие на всё, чтобы они преступивших закон Божий истребили во всяком месте царства его беспрепятственно, без особого позволения или надзора царя.
\vs 3Ma 7:11 Тогда, возблагодарив его, как надлежало, священники и все народное множество воспели <<аллилуия>> и радостно отправились.
\vs 3Ma 7:12 Всякого соплеменника из осквернившихся, которого встречали на пути, они наказывали и убивали в пример другим.
\vs 3Ma 7:13 В этот день они умертвили более трехсот мужей и торжествовали с весельем, умерщвляя нечистых.
\vs 3Ma 7:14 Сами же, пребыв с Богом до смерти и получив полную радость спасения, поднялись из города, увенчанные всякими благоуханными цветами, с весельем и восклицаниями, хвалами и благозвучными песнями, благодаря Бога отцов, вечного Спасителя Израиля.
\rsbpar\vs 3Ma 7:15 Придя в Птолемаиду, называемую по свойству места Родофором \bibemph{(розоносною)}, в которой по общему их уговору ожидали их корабли семь дней,
\vs 3Ma 7:16 они учредили там пиршество спасения, ибо царь щедро снабдил их всем, что потребно было каждому до прибытия в свой дом.
\vs 3Ma 7:17 Так как они достигли сюда в мире, с приличными благодарениями, то и здесь также установили весело праздновать эти дни во время пребывания своего.
\vs 3Ma 7:18 Освятив эти дни и утвердив свой обет поставлением столба на месте пиршества, они отправились далее сушею и морем и рекою, каждый в свое жилище, невредимые, свободные, в полной радости, охраняемые царским повелением. Тогда-то приобрели они б\acc{о}льшую, нежели прежде, силу и славу и сделались страшными для врагов, ни от кого нисколько не притесняемые в своем владении,
\vs 3Ma 7:19 и все получили свое по описи, так что, кто имел что-либо у себя, с величайшим страхом отдавали им, ибо величайшие благодеяния явил им величайший Бог на спасение их.
\vs 3Ma 7:20 Благословен Спаситель Израиля на вечные времена! Аминь.

\bibbookdescr{3Ez}{
  inline={\LARGE Третья книга\\\Huge Ездры\fns{Книги этой нет ни на еврейском, ни на греческом языках. Как славянский, так и русский переводы сделаны с Вульгаты. В последней она разделена на две книги: первую составляют главы 3--14 по славянскому переводу, а вторая заключает в себе главы 1, 2, 15 и 16. В русском переводе удержан порядок глав славянского перевода.}},
  toc={3-я Ездры*},
  bookmark={3-я Ездры},
  header={3-я Ездры},
  %headerleft={},
  %headerright={},
  abbr={3~Езд}
}
\vs 3Ez 1:1 Вторая книга Ездры пророка, сына Сераии, сына Азарии, сына Хелкии, сына Шаллума, сына Садока, сына Ахитува,
\vs 3Ez 1:2 сына Ахии, сына Финееса, сына Илия, сына Амарии, сына Асиела, сына Мерайофа, сына Арна, сына Уззия, сына Ворифа, сына Авишуя, сына Финееса, сына Елеазара,
\vs 3Ez 1:3 сына Аарона от колена Левиина, который был пленником в стране Мидийской, в царствование Артаксеркса, царя Персидского.
\rsbpar\vs 3Ez 1:4 Было слово Господне ко мне:
\vs 3Ez 1:5 иди и возвести народу Моему злые дела их и сыновьям их~--- беззакония, которые они совершили против Меня, чтобы они возвестили сынам сынов своих;
\vs 3Ez 1:6 ибо грехи родителей их возросли в них; забыв Меня, они приносили жертвы богам чужим.
\vs 3Ez 1:7 Не Я ли вывел их из земли Египетской, из дома рабства? а они прогневали Меня и советы Мои презрели.
\vs 3Ez 1:8 Ты остриги волосы головы твоей, и брось на них все злое, ибо они не слушались закона моего~--- народ необузданный!
\vs 3Ez 1:9 Доколе Я буду терпеть их, которым сделал столько благодеяний?
\vs 3Ez 1:10 Ради них Я многих царей низложил; поразил фараона с рабами его и со всем войском его;
\vs 3Ez 1:11 всех язычников от лица их погубил, и на востоке народ двух областей, Тира и Сидона, рассеял и всех врагов их истребил.
\vs 3Ez 1:12 Ты же так скажи им: так говорит Господь:
\vs 3Ez 1:13 именно Я провел вас через море и по дну его проложил вам огражденную улицу, дал вам вождя Моисея и Аарона священника,
\vs 3Ez 1:14 дал вам свет в столпе огненном, и многие чудеса сотворил среди вас; а вы Меня забыли, говорит Господь.
\rsbpar\vs 3Ez 1:15 Так говорит Господь Вседержитель: перепелы были вам в знамение. Я дал вам станы для защиты, но вы и там роптали
\vs 3Ez 1:16 и не радовались во имя Мое о погибели врагов ваших, но даже доныне еще ропщете.
\vs 3Ez 1:17 Где те благодеяния, которые Я сделал вам? Не в пустыне ли, когда вы, взалкав, вопияли ко Мне,
\vs 3Ez 1:18 говоря: <<зачем Ты привел нас в эту пустыню? уморить нас? лучше нам было служить Египтянам, нежели умереть в этой пустыне>>?
\vs 3Ez 1:19 Я сжалился на стенания ваши, и дал вам манну в пищу: вы ели хлеб ангельский.
\vs 3Ez 1:20 Когда вы жаждали, не рассек ли Я камень, и потекли воды до сытости? от зноя покрывал вас листьями древесными.
\vs 3Ez 1:21 Разделил вам земли тучные; Хананеев, Ферезеев и Филистимлян изгнал от лица вашего. Что еще сделаю вам? говорит Господь.
\vs 3Ez 1:22 Так говорит Господь Вседержитель: когда вы были в пустыне, на реке Мерры, и жаждущие хулили имя Мое,
\vs 3Ez 1:23 не огонь послал Я на вас за богохульства, но вложил дерево в воду и реку сделал сладкою.
\vs 3Ez 1:24 Что сделаю тебе, Иаков? Не хотел ты повиноваться, Иуда. Переселюсь к другим народам и дам им имя Мое, чтобы соблюдали законы Мои.
\vs 3Ez 1:25 Так как вы Меня оставили, то и Я оставлю вас; просящих у Меня милости не помилую.
\vs 3Ez 1:26 Когда будете призывать Меня, Я не услышу вас, ибо вы осквернили руки ваши кровью, и ноги ваши быстры на совершение человекоубийства.
\vs 3Ez 1:27 Вы как бы не Меня оставили, а вас самих, говорит Господь.
\vs 3Ez 1:28 Так говорит Господь Вседержитель: не Я ли умолял вас, как отец сыновей и как мать дочерей и как кормилица питомцев своих,
\vs 3Ez 1:29 чтобы вы были Мне народом и Я вам Богом, чтобы вы были Мне сынами и Я вам Отцом?
\vs 3Ez 1:30 Я собрал вас, как курица птенцов своих под крылья свои. Что ныне сделаю вам? Отвергну вас от лица Моего.
\vs 3Ez 1:31 Когда принесете Мне приношение, отвращу лице Мое от вас; ибо ваши дни праздничные и новомесячия и обрезания Я отринул.
\vs 3Ez 1:32 Я послал к вам рабов Моих, пророков; вы, схватив их, умертвили и растерзали тела их. Кровь их Я взыщу, говорит Господь.
\rsbpar\vs 3Ez 1:33 Так говорит Господь Вседержитель: дом ваш пуст. Развею вас, как ветер мякину,
\vs 3Ez 1:34 и сыновья не будут иметь потомства, потому что заповедь Мою презрели и делали то, что зло предо Мною.
\vs 3Ez 1:35 Предам домы ваши людям грядущим, которые, не слышав Меня, уверуют, которые, хотя Я не показывал им знамений, исполнят то, что Я заповедал,
\vs 3Ez 1:36 не видев пророков, воспомянут о своих беззакониях.
\vs 3Ez 1:37 Завещеваю благодать людям грядущим, дети которых, не видев Меня очами плотскими, но духом веруя тому, что Я сказал, торжествуют с весельем.
\vs 3Ez 1:38 Итак теперь смотри, брат, какая слава,~--- смотри на людей, грядущих с востока,
\vs 3Ez 1:39 которым Я дам в вожди Авраама, Исаака и Иакова, и Осию, и Амоса, и Михея, и Иоиля, и Авдия, и Иону,
\vs 3Ez 1:40 и Наума, и Аввакума, Софонию, Аггея, Захарию и Малахию, который наречен и Ангелом Господним.
\vs 3Ez 2:1 Так говорит Господь: Я вывел народ сей из работы, дал им повеление через рабов Моих, пророков, которых они не захотели слушать, но отвергли Мои советы.
\vs 3Ez 2:2 Мать, которая родила их, говорит им: <<идите, дети; ибо я вдова и оставлена.
\vs 3Ez 2:3 Я воспитала вас с радостью, и отпустила с плачем и горестью, потому что вы согрешили пред Господом Богом вашим, и сделали злое пред Ним.
\vs 3Ez 2:4 Ныне же что сделаю для вас? Я вдова и оставлена: идите, дети, и просите у Господа милости>>.
\vs 3Ez 2:5 Тебя, Отче, призываю во свидетеля на мать сыновей, которые не захотели хранить завета моего.
\vs 3Ez 2:6 Предай их посрамлению и мать их~--- на расхищение, чтобы не было рода их.
\vs 3Ez 2:7 Пусть рассеются имена их по народам и изгладятся от земли, ибо они презрели завет мой.
\vs 3Ez 2:8 Горе тебе, Ассур, скрывающий у себя нечестивых! Род лукавый! вспомни, что Я сделал Содому и Гоморре.
\vs 3Ez 2:9 Земля их лежит в смоляных глыбах и холмах пепельных. Так поступлю Я с теми, которые Меня не слушались, говорит Господь Вседержитель.
\vs 3Ez 2:10 Так говорит Господь к Ездре: возвести народу Моему, что Я дам им царство Иерусалимское, которое обещал Израилю,
\vs 3Ez 2:11 и прииму славу от них и дам им обители вечные, которые приготовил для них.
\vs 3Ez 2:12 Древо жизни будет для них мастью благовонною; не будут изнуряемы трудом и не изнемогут.
\vs 3Ez 2:13 Идите и получ\acc{и}те; прос\acc{и}те себе дней малых, дабы они не замедлили. Уже готово для вас царство: бодрствуйте.
\vs 3Ez 2:14 Свидетельствуй, небо и земля, ибо Я стер злое и сотворил доброе. Живу Я! говорит Господь.
\vs 3Ez 2:15 Мать! обними сыновей твоих, воспитывай их с радостью; как голубица укрепляй ноги их, ибо Я избрал тебя, говорит Господь.
\vs 3Ez 2:16 И воскрешу мертвых от мест их и из гробов выведу их, потому что Я познал имя Мое в Израиле.
\vs 3Ez 2:17 Не бойся, мать сынов, ибо Я избрал тебя, говорит Господь.
\vs 3Ez 2:18 Я пошлю тебе в помощь рабов Моих Исаию и Иеремию, по совету которых Я освятил и приготовил тебе двенадцать дерев, обремененных различными плодами,
\vs 3Ez 2:19 и столько же источников, текущих молоком и медом, и семь гор величайших, произращающих розу и лилию, через которые исполню радостью сынов твоих.
\vs 3Ez 2:20 Оправдай вдову, дай суд бедному, помоги нищему, защити сироту, одень нагого,
\vs 3Ez 2:21 о расслабленном и немощном попекись, над хромым не смейся, безрукого защити, и слепого приведи к видению света Моего,
\vs 3Ez 2:22 старца и юношу в стенах твоих сохрани,
\vs 3Ez 2:23 мертвых, где найдешь, запечатлев, предай гробу, и Я дам тебе первое место в Моем воскресении.
\vs 3Ez 2:24 Отдыхай и покойся, народ Мой, ибо придет покой твой.
\vs 3Ez 2:25 Корми сынов твоих, добрая кормилица, укрепляй ноги их.
\vs 3Ez 2:26 Из рабов, которых Я дал тебе, никто да не погибнет, ибо Я взыщу их от тебя.
\vs 3Ez 2:27 Не ослабевай. Когда придет день печали и тесноты, другие будут плакать и сокрушаться, а ты будешь весела и изобильна.
\vs 3Ez 2:28 Язычники будут завидовать тебе, но ничего против тебя сделать не могут, говорит Господь.
\vs 3Ez 2:29 Руки Мои покроют тебя, чтобы сыны твои не видели геенны.
\vs 3Ez 2:30 Утешайся, мать, с сынами твоими, ибо Я спасу тебя.
\vs 3Ez 2:31 Помни о сынах твоих почивающих. Я выведу их от краев земли и окажу им милость, ибо Я милостив, говорит Господь Вседержитель.
\vs 3Ez 2:32 Обними детей твоих, доколе Я приду и сделаю им милость; ибо источники Мои обильны и благодать Моя не оскудеет.
\rsbpar\vs 3Ez 2:33 Я, Ездра, получил на горе Орив повеление от Господа идти к Израилю. Когда я пришел к ним, они отвергли меня и презрели заповедь Господню.
\vs 3Ez 2:34 Посему вам говорю, язычники, которые можете слышать и понимать: ожидайте Пастыря вашего, Он даст вам покой вечный, ибо близко Тот, Который придет в скончание века.
\vs 3Ez 2:35 Будьте готовы к воздаянию царствия, ибо свет немерцающий воссияет вам на вечное время.
\vs 3Ez 2:36 Избегайте тени века сего; приимите сладость славы вашей. Я открыто свидетельствую о Спасителе моем.
\vs 3Ez 2:37 Вверенный дар приимите, и наслаждайтесь, благодаря Того, Кто призвал вас в небесное царство.
\vs 3Ez 2:38 Встаньте и стойте, и смотрите, какое число знаменованных на вечери Господней,
\vs 3Ez 2:39 которые, переселившись от тени века сего, получили от Господа светлые одежды.
\vs 3Ez 2:40 Приими число твое, Сион, и заключи твоих, одетых в белые одеяния, которые исполнили закон Господень.
\vs 3Ez 2:41 Число желанных сынов твоих полно. Проси державу Господа, чтобы освятился народ твой, призванный от начала.
\rsbpar\vs 3Ez 2:42 Я, Ездра, видел на горе Сионской сонм великий, которого не мог исчислить, и все они песнями прославляли Господа.
\vs 3Ez 2:43 Посреди них был юноша величественный, превосходящий всех их, и возлагал венцы на главу каждого из них и тем более возвышался; я поражен был удивлением.
\vs 3Ez 2:44 Тогда я спросил Ангела: кто сии, господин мой?
\vs 3Ez 2:45 Он в ответ мне сказал: это те, которые сложили смертную одежду и облеклись в бессмертную и исповедали имя Божие; они теперь увенчиваются и принимают победные пальмы.
\vs 3Ez 2:46 Я спросил: а кто сей юноша, который возлагает на них венцы и вручает им пальмы?
\vs 3Ez 2:47 Он отвечал мне: Сам Сын Божий, Которого они прославляли в веке сем. И я начал славить их, мужественно стоявших за имя Господне.
\vs 3Ez 2:48 Тогда Ангел сказал мне: иди и возвести народу моему, какие видел ты дивные дела Господа Бога.
\vs 3Ez 3:1 В тридцатом году по разорении города был я в Вавилоне, и смущался, лежа на постели моей, и помышления всходили на сердце мое,
\vs 3Ez 3:2 ибо я видел опустошение Сиона и богатство живущих в Вавилоне.
\vs 3Ez 3:3 И возмутился дух мой, и я начал со страхом говорить ко Всевышнему,
\vs 3Ez 3:4 и сказал: Владыко Господи! Ты сказал от начала, когда един основал землю, и повелел персти,
\vs 3Ez 3:5 и дал Адаму тело смертное, которое было также создание рук Твоих, и вдохнул в него дух жизни, и он сделался живым пред Тобою,
\vs 3Ez 3:6 и ввел его в рай, который насадила десница Твоя, прежде нежели земля произрастила плоды;
\vs 3Ez 3:7 Ты повелел ему хранить заповедь Твою, но он нарушил ее, и Ты осудил его на смерть, и род его и происшедшие от него поколения и племена, народы и отрасли их, которым нет числа.
\vs 3Ez 3:8 Каждый народ стал ходить по своему хотению, делал пред Тобою дела неразумные и презирал заповеди Твои.
\vs 3Ez 3:9 По времени, Ты навел потоп на обитателей земли и истребил их,
\vs 3Ez 3:10 и исполнилось на каждом из них,~--- как на Адаме смерть, так на сих потоп.
\vs 3Ez 3:11 Одного из них Ты оставил~--- Ноя с семейством его, и от него произошли все праведные.
\vs 3Ez 3:12 Когда начали размножаться обитающие на земле, и умножились сыны и народы и поколения многие, и опять начали предаваться нечестию, более нежели прежние,
\vs 3Ez 3:13 когда начали делать пред Тобою беззаконие: Ты избрал Себе из них мужа, которому имя Авраам,
\vs 3Ez 3:14 и возлюбил его и открыл ему одному волю Твою,
\vs 3Ez 3:15 и положил ему завет вечный, и сказал ему, что никогда не оставишь семени его. И дал ему Исаака, и Исааку дал Иакова и Исава;
\vs 3Ez 3:16 Ты избрал Себе Иакова, Исава же отринул. И умножился Иаков чрезвычайно.
\vs 3Ez 3:17 Когда Ты вывел из Египта семя его и привел к горе Синайской,
\vs 3Ez 3:18 тогда преклонил небеса, уставил землю, поколебал вселенную, привел в трепет бездны и весь мир в смятение.
\vs 3Ez 3:19 И прошла слава Твоя в четырех \bibemph{явлениях}: в огне, землетрясении, бурном ветре и морозе, чтобы дать закон семени Иакова и радение роду Израиля,
\vs 3Ez 3:20 но не отнял у них сердца лукавого, чтобы закон Твой принес в них плод.
\vs 3Ez 3:21 С сердцем лукавым первый Адам преступил заповедь, и побежден был; так и все, от него происшедшие.
\vs 3Ez 3:22 Осталась немощь и закон в сердце народа с корнем зла, и отступило доброе, и осталось злое.
\vs 3Ez 3:23 Прошли времена и окончились лета,~--- и Ты воздвиг Себе раба, именем Давида;
\vs 3Ez 3:24 повелел ему построить город имени Твоему и в нем приносить Тебе фимиам и жертвы.
\vs 3Ez 3:25 Много лет это исполнялось, и потом согрешили населяющие город,
\vs 3Ez 3:26 во всем поступая так, как поступил Адам и все его потомки; ибо и у них было сердце лукавое.
\vs 3Ez 3:27 И Ты предал город Твой в руки врагов Твоих.
\vs 3Ez 3:28 Неужели лучше живут обитатели Вавилона и за это владеют Сионом?
\vs 3Ez 3:29 Когда я пришел сюда, видел нечестия, которым нет числа, и в этом тридцатом году пленения видит душа моя многих грешников,~--- и изныло сердце мое,
\vs 3Ez 3:30 ибо я видел, как Ты поддерживаешь сих грешников и щадишь нечестивцев, а народ Твой погубил, врагов же Твоих сохранил и не явил о том никакого знамения.
\vs 3Ez 3:31 Не понимаю, как этот путь мог измениться. Неужели Вавилон поступает лучше, нежели Сион?
\vs 3Ez 3:32 Или иной народ познал Тебя, кроме Израиля? или какие племена веровали заветам Твоим, как Иаков?
\vs 3Ez 3:33 Ни воздаяние им не равномерно, ни труд их не принес плода, ибо я прошел среди народов, и видел, что они живут в изобилии, хотя и не вспоминают о заповедях Твоих.
\vs 3Ez 3:34 Итак взвесь на весах и наши беззакония и дела живущих на земле, и нигде не найдется имя Твое, как только у Израиля.
\vs 3Ez 3:35 Когда не грешили пред Тобою живущие на земле? или какой народ так сохранил заповеди Твои?
\vs 3Ez 3:36 Между сими хотя по именам найдешь хранящих заповеди Твои, а у других народов не найдешь.
\vs 3Ez 4:1 Тогда отвечал мне посланный ко мне Ангел, которому имя Уриил,
\vs 3Ez 4:2 и сказал: сердце твое слишком далеко зашло в этом веке, что ты помышляешь постигнуть путь Всевышнего.
\vs 3Ez 4:3 Я отвечал: так, господин мой. Он же сказал мне: три пути послан я показать тебе и три подобия предложить тебе.
\vs 3Ez 4:4 Если ты одно из них объяснишь мне, то и я покажу тебе путь, который желаешь ты видеть, и научу тебя, откуда произошло сердце лукавое.
\vs 3Ez 4:5 Тогда я сказал: говори, господин мой. Он же сказал мне: иди и взвесь тяжесть огня, или измерь мне дуновение ветра, или возврати мне день, который уже прошел.
\vs 3Ez 4:6 Какой человек, отвечал я, может сделать то, чего ты требуешь от меня?
\vs 3Ez 4:7 А он сказал мне: если бы я спросил тебя, сколько обиталищ в сердце морском, или сколько источников в самом основании бездны, или сколько жил над твердью, или какие пределы у рая,
\vs 3Ez 4:8 ты, может быть, сказал бы мне: <<в бездну я не сходил, и в ад также, и на небо никогда не восходил>>.
\vs 3Ez 4:9 Теперь же я спросил тебя только об огне, ветре и дне, который ты пережил, и о том, без чего ты быть не можешь, и на это ты не отвечал мне.
\vs 3Ez 4:10 И сказал мне: ты и того, что твое и с тобою от юности, не можешь познать;
\vs 3Ez 4:11 как же сосуд твой мог бы вместить в себе путь Всевышнего и в этом уже заметно растленном веке понять растление, которое очевидно в глазах моих?
\vs 3Ez 4:12 На это сказал я: лучше было бы нам вовсе не быть, нежели жить в нечестиях и страдать, не зная, почему.
\vs 3Ez 4:13 Он же в ответ сказал мне: вот, я отправился в полевой лес, и застал дерева держащими совет.
\vs 3Ez 4:14 Они говорили: <<придите, и пойдем и объявим войну морю, чтобы оно отступило перед нами, и мы там возрастим для себя другие леса>>.
\vs 3Ez 4:15 Подобным образом и волны морские имели совещание: <<придите>>, говорили они, <<поднимемся и завоюем леса полевые, чтобы и там приобрести для себя другое место>>.
\vs 3Ez 4:16 Но замысел леса оказался тщетным, ибо пришел огонь и сжег его.
\vs 3Ez 4:17 Подобным образом кончился и замысел волн морских, ибо стал песок, и воспрепятствовал им.
\vs 3Ez 4:18 Если бы ты был судьею их, кого бы ты стал оправдывать или кого обвинять?
\vs 3Ez 4:19 Подлинно, отвечал я, замыслы их были суетны, ибо земля дана лесу, дано место и морю, чтобы носить свои волны.
\vs 3Ez 4:20 Он же в ответ сказал мне: справедливо рассудил ты; почему же ты не судил таким же образом себя самого?
\vs 3Ez 4:21 Ибо как земля дана лесу, а море волнам его, так обитающие на земле могут разуметь только то, что на земле; а обитающие на небесах могут разуметь, что на высоте небес.
\vs 3Ez 4:22 И отвечал я, и сказал: молю Тебя, Господи, да дастся мне смысл разумения.
\vs 3Ez 4:23 Не хотел я вопрошать Тебя о высшем, а о том, что ежедневно бывает у нас: почему Израиль предан на поругание язычникам? почему народ, который Ты возлюбил, отдан нечестивым племенам, и закон отцов наших доведен до ничтожества, и писанных постановлений нигде нет?
\vs 3Ez 4:24 Переходим из века сего, как саранча, жизнь наша проходит в страхе и ужасе, и мы сделались недостойными милосердия.
\vs 3Ez 4:25 Но что сделает Он с именем Своим, которое наречено на нас? вот о чем я вопрошал.
\vs 3Ez 4:26 Он же отвечал мне: чем больше будешь испытывать, тем больше будешь удивляться; потому что быстро спешит век сей к своему исходу,
\vs 3Ez 4:27 и не может вместить того, что обещано праведным в будущие времена, потому что век сей исполнен неправдою и немощами.
\vs 3Ez 4:28 А о том, о чем ты спрашивал меня, скажу тебе: посеяно зло, а еще не пришло время искоренения его.
\vs 3Ez 4:29 Посему, доколе посеянное не исторгнется, и место, на котором насеяно зло, не упразднится,~--- не придет место, на котором всеяно добро.
\vs 3Ez 4:30 Ибо зерно злого семени посеяно в сердце Адама изначала, и сколько нечестия народило оно доселе и будет рождать до тех пор, пока не настанет молотьба!
\vs 3Ez 4:31 Рассуди с собою, сколько зерно злого семени народило плодов нечестия!
\vs 3Ez 4:32 Когда будут пожаты бесчисленные колосья его, какое огромное понадобится для сего гумно!
\vs 3Ez 4:33 Как же и когда это будет? спросил я его; почему наши лета малы и несчастны?
\vs 3Ez 4:34 Не спеши подниматься, отвечал он, выше Всевышнего; ибо напрасно спешишь быть выше Его: слишком далеко заходишь.
\vs 3Ez 4:35 Не о том же ли вопрошали души праведных в затворах своих, говоря: <<доколе таким образом будем мы надеяться? И когда плод нашего возмездия?>>
\vs 3Ez 4:36 На это отвечал мне Иеремиил Архангел: <<когда исполнится число семян в вас, ибо Всевышний на весах взвесил век сей,
\vs 3Ez 4:37 и мерою измерил времена, и числом исчислил часы, и не подвинет и не ускорит до тех пор, доколе не исполнится определенная мера>>.
\vs 3Ez 4:38 Я же в ответ на это сказал ему: о, Владыко Господи! а мы все преисполнены нечестием.
\vs 3Ez 4:39 И, может быть, из-за нас не наполняются житницы праведных, и ради грехов живущих на земле.
\vs 3Ez 4:40 На это он отвечал мне: пойди, спроси беременную женщину, могут ли, по исполнении девятимесячного срока, ложесна ее удержать в себе плод?
\vs 3Ez 4:41 Я сказал: не могут. Тогда он сказал мне: подобны ложеснам и обиталища душ в преисподней.
\vs 3Ez 4:42 Как рождающая спешит родить, чтобы освободиться от болезней рождения, так и эти спешат отдать вверенное им.
\vs 3Ez 4:43 Сначала будет показано тебе то, что ты желаешь видеть.
\vs 3Ez 4:44 Если я обрел благодать пред очами твоими, отвечал я, и если это возможно и я способен к тому,
\vs 3Ez 4:45 покажи мне: имеющее прийти более ли того, что прошло, или сбывшееся более того, что будет?
\vs 3Ez 4:46 Что прошло, я это знаю, а что придет, не ведаю.
\vs 3Ez 4:47 Он сказал мне: стань на правую сторону, и я объясню тебе значение подобием.
\vs 3Ez 4:48 И я стал, и увидел: вот горящая печь проходит передо мною; и когда пламя прошло, я увидел: остался дым.
\vs 3Ez 4:49 После сего прошло предо мною облако, наполненное водою, и пролился из него сильный дождь; но как скоро стремительность дождя остановилась, остались капли.
\vs 3Ez 4:50 Тогда он сказал мне: размышляй себе: как дождь более капель, а огонь больше дыма, так мера прошедшего превысила, а остались капли и дым.
\vs 3Ez 4:51 Тогда я умолял его и сказал: думаешь ли ты, что я доживу до этих дней? и что будет в эти дни?
\vs 3Ez 4:52 На это отвечал он, и сказал: о знамениях, о которых ты спрашиваешь меня, я отчасти могу сказать тебе, а о жизни твоей я не послан говорить с тобою, да и не знаю.
\vs 3Ez 5:1 О знамениях: вот, настанут дни, в которые многие из живущих на земле, обладающие в\acc{е}дением, будут вос\-х\acc{и}\-ще\-ны, и путь истины сокроется, и вселенная оскудеет верою,
\vs 3Ez 5:2 и умножится неправда, которую теперь ты видишь и о которой издавна слышал.
\vs 3Ez 5:3 И будет, что страна, которую ты теперь видишь господствующею, подвергнется опустошению.
\vs 3Ez 5:4 А если Всевышний даст тебе дожить, то увидишь, что после третьей трубы внезапно воссияет среди ночи солнце и луна трижды в день;
\vs 3Ez 5:5 и с дерева будет капать кровь, камень даст голос свой, и народы поколеблются.
\vs 3Ez 5:6 Тогда будет царствовать тот, которого живущие на земле не ожидают, и птицы перелетят на другие места.
\vs 3Ez 5:7 Море Содомское извергнет рыб, будет издавать ночью голос, неведомый для многих; однако же все услышат голос его.
\vs 3Ez 5:8 Будет смятение во многих местах, часто будет посылаем с неба огонь; дикие звери переменят места свои, и нечистые женщины будут рождать чудовищ.
\vs 3Ez 5:9 Сладкие воды сделаются солеными, и все друзья ополчатся друг против друга; тогда сокроется ум, и разум удалится в свое хранилище.
\vs 3Ez 5:10 Многие будут искать его, но не найдут, и умножится на земле неправда и невоздержание.
\vs 3Ez 5:11 Одна область будет спрашивать другую соседнюю: <<не проходила ли по тебе правда, делающая праведным?>> И та скажет: <<нет>>.
\vs 3Ez 5:12 Люди в то время будут надеяться, и не достигнут желаемого, будут трудиться, и не управятся пути их.
\vs 3Ez 5:13 Об этих знамениях мне дозволено сказать тебе, и если снова помолишься и поплачешь, как теперь, и попостишься семь дней, то услышишь еще больше того.
\vs 3Ez 5:14 И я пришел в себя, и тело мое сильно дрожало, и душа моя изнемогла, как будто исчезала.
\vs 3Ez 5:15 Но пришедший ко мне Ангел поддержал меня и укрепил меня, и поставил на ноги.
\vs 3Ez 5:16 И было, во вторую ночь пришел ко мне Салафиил, вождь народа, и спросил меня: где ты был, и отчего лице твое так печально?
\vs 3Ez 5:17 Разве не знаешь, что тебе вверен Израиль в стране преселения его?
\vs 3Ez 5:18 Итак встань и вкуси хлеба, и не оставляй нас, как пастырь своего стада, в руках лукавых волков.
\vs 3Ez 5:19 Тогда сказал я ему: отойди от меня, и не приближайся ко мне. И он, услышав это, удалился от меня.
\vs 3Ez 5:20 А я семь дней постился, стеная и плача, как повелел мне Ангел Уриил.
\vs 3Ez 5:21 И после семи дней помышления сердца моего опять были для меня крайне тягостны;
\vs 3Ez 5:22 но душа моя прияла дух разумения, и я снова начал говорить пред Всевышним
\vs 3Ez 5:23 и сказал: о, Владыко Господи! Ты из всех лесов на земле и из всех дерев на ней избрал только одну виноградную лозу;
\vs 3Ez 5:24 Ты из всего круга земного избрал Себе одну пещеру, и из всех цветов во вселенной Ты избрал Себе одну лилию;
\vs 3Ez 5:25 Ты из всех пучин морских наполнил для Себя один источник, а из всех построенных городов освятил для Себя один Сион.
\vs 3Ez 5:26 Из всех сотворенных птиц Ты наименовал Себе одну голубицу, и из всех сотворенных скотов Ты избрал Себе одну овцу;
\vs 3Ez 5:27 из всех многочисленных народов Ты приобрел Себе один народ, и возлюбил его, дал ему закон совершенный.
\vs 3Ez 5:28 Но ныне, Господи, отчего же Ты предал одного многим, и на одном корне Ты насадил другие отрасли и рассеял Твой единственный народ между многими народами?
\vs 3Ez 5:29 И попрали его противники обетованиям Твоим и заветам Твоим не веровавшие.
\vs 3Ez 5:30 И если уже Ты сильно возненавидел народ Твой, то пусть бы он Твоими руками наказывался.
\rsbpar\vs 3Ez 5:31 Когда я произносил слова сии, послан был ко мне Ангел, который приходил ко мне прежде ночью,
\vs 3Ez 5:32 и сказал мне: послушай меня, и я научу тебя; внимай мне, и я скажу тебе еще более.
\vs 3Ez 5:33 Говори, сказал я, господин мой. И он сказал мне: ты слишком далеко зашел пытливостью ума твоего об Израиле; неужели ты больше любишь его, нежели Тот, Который сотворил его?
\vs 3Ez 5:34 Нет, господин мой, отвечал я, но говорил от великой скорби. Внутренность моя мучает меня всякий час, когда я стараюсь постигнуть путь Всевышнего и исследовать хотя часть суда Его.
\vs 3Ez 5:35 Он отвечал: не можешь. Почему же, господин мой? спросил я. Лучше бы я не родился, и утроба матерняя сделалась для меня гробом, нежели видеть угнетение Иакова и изнурение рода Израильского.
\vs 3Ez 5:36 И он сказал мне: исчисли мне, что еще не пришло, и собери мне рассеянные капли, и оживи иссохшие цветы;
\vs 3Ez 5:37 открой заключенные хранилища и выведи мне заключенные в них ветры, и покажи мне образ голоса: и тогда я покажу тебе то, что ты усиливаешься видеть.
\vs 3Ez 5:38 Владыко Господи! отвечал я, кто может знать это, разве только тот, кто не живет с человеками?
\vs 3Ez 5:39 А я безумен, и как могу говорить о том, о чем Ты спросил меня?
\vs 3Ez 5:40 Тогда Он сказал мне: как ты не можешь сделать ничего из сказанного, так не можешь познать судеб Моих, ни предела любви, которую обещал Я народу.
\vs 3Ez 5:41 Но вот, Господи, Ты близок к тем, которые к концу близятся, и что будут делать те, которые прежде меня были, или мы, или которые после нас будут?
\vs 3Ez 5:42 Он сказал мне: венцу уподоблю я суд Мой; как нет запоздания последних, так и ускорения первых.
\vs 3Ez 5:43 Отвечал я и сказал: не мог ли бы Ты соединить воедино как тех, которые сотворены были прежде, так и тех, которые существуют и которые будут, дабы скорее объявить им суд Твой?
\vs 3Ez 5:44 Он отвечал мне: не может ускорить творение Творца своего, ни век сей не может вместить в себе всех вместе, которые должны быть сотворены.
\vs 3Ez 5:45 И сказал я: как же Ты сказал рабу Твоему, что Ты дал жизнь созданному творению вкупе, и однако творение выдержало это; посему могли бы понести и ныне существующие вкупе.
\vs 3Ez 5:46 Он сказал мне: спроси женщину, и скажи ей: <<если ты рождаешь десять, то почему рождаешь по временам?>>, и проси ее, чтобы она родила десять вдруг.
\vs 3Ez 5:47 Я же сказал Ему: невозможно это, но должно быть по времени.
\vs 3Ez 5:48 Тогда Он сказал мне: и Я дал недрам земли способность посеянное на ней возращать по временам.
\vs 3Ez 5:49 Как младенец не может производить того, что свойственно старцам, так Я устроил созданный Мною век.
\vs 3Ez 5:50 Тогда я вопросил Его и сказал: когда Ты открыл мне путь, то позволь мне сказать Тебе: мать наша, о которой Ты говорил Мне, молода ли еще, или приближается к старости?
\vs 3Ez 5:51 Спроси об этом рождающую, и она скажет тебе.
\vs 3Ez 5:52 Скажи ей: <<почему рождаемые тобою ныне не подобны тем, которые рождены были прежде, но меньше их ростом?>>
\vs 3Ez 5:53 И она скажет тебе: <<одни рождены мною в крепости молодой силы, а другие рождены под старость, когда ложесна начали терять свою силу>>.
\vs 3Ez 5:54 Рассуди же ты: вы теперь меньше станом, нежели те, которые были прежде вас;
\vs 3Ez 5:55 и те, которые после вас родятся, будут еще меньше вас, так как творения, уже состаривающиеся, и крепость юноши уже миновала.
\vs 3Ez 5:56 И сказал я: если я приобрел благоволение пред очами Твоими, покажи рабу Твоему, через кого Ты посещаешь творение Твое?
\vs 3Ez 6:1 И сказал Он мне: от начала творения круга земного и прежде нежели установлены были пределы века, и прежде нежели подули ветры;
\vs 3Ez 6:2 прежде нежели услышаны были гласы громов, прежде нежели возблистали молнии, прежде нежели утвердились основания рая;
\vs 3Ez 6:3 прежде нежели показались прекрасные цветы, прежде нежели утвердились силы подвижные, и прежде нежели собрались бесчисленные воинства Ангелов;
\vs 3Ez 6:4 прежде нежели поднялись высоты воздушные, прежде нежели определились меры твердей, прежде нежели возгорелись огни на Сионе;
\vs 3Ez 6:5 прежде нежели исследованы были лета, и отделены те, которые грешат ныне, и запечатлены те, которые хранили веру, как сокровище:
\vs 3Ez 6:6 тогда Я помыслил, и сотворено было все Мною одним, а не чрез кого-либо иного; от Меня также последует и конец, а не от кого-либо иного.
\vs 3Ez 6:7 Тогда я отвечал: какое разделение времен, и когда будет конец первого и начало последнего?
\vs 3Ez 6:8 От Авраама даже до Исаака, когда родились от него Иаков и Исав, рука Иакова держала от начала пяту Исава.
\vs 3Ez 6:9 Конец сего века~--- Исав, а начало следующего~--- Иаков.
\vs 3Ez 6:10 Рука человека~--- начало его, а конец~--- пята его. О другом, Ездра, не спрашивай Меня.
\vs 3Ez 6:11 Я же в ответ сказал Ему: о, Владыко Господи! если я обрел благодать пред очами Твоими,
\vs 3Ez 6:12 молю Тебя, покажи рабу Твоему конец знамений Твоих, которых часть показал Ты мне в прошедшую ночь.
\vs 3Ez 6:13 Он отвечал мне и сказал: встань на ноги твои, и слушай голос, исполненный шума,
\vs 3Ez 6:14 и будет как бы землетрясение, но место, на котором ты стоишь, не поколеблется.
\vs 3Ez 6:15 Посему, когда будет говорить, ты не ужасайся; ибо о конце будет слово, и основания земли разумеются.
\vs 3Ez 6:16 А как речь идет о них самих, то земля вострепещет и поколеблется, ибо знает, что конец их должен измениться.
\rsbpar\vs 3Ez 6:17 И было, когда я услышал голос, встал на ноги мои, и слышал, и вот голос говорящий, и шум его, как шум вод многих,
\vs 3Ez 6:18 и сказал: вот, наступают дни, когда Я начну приближаться, чтобы посетить живущих на земле,
\vs 3Ez 6:19 когда начну Я взыскивать с тех, которые неправдою своею произвели неправедно великий вред, и когда исполнится мера уничижения Сиона.
\vs 3Ez 6:20 А когда назнаменается век, который начнет проходить, то вот знамения, которые Я покажу: книги раскроются пред лицем тверди, и все вместе увидят;
\vs 3Ez 6:21 и однолетние младенцы заговорят своими голосами, и беременные женщины будут рождать недозрелых младенцев через три и четыре месяца, и они останутся живыми и укрепятся;
\vs 3Ez 6:22 засеянные поля внезапно явятся как незасеянные, и полные житницы окажутся пустыми;
\vs 3Ez 6:23 затем вострубит труба с шумом, и когда услышат ее, все внезапно ужаснутся.
\vs 3Ez 6:24 И будет в то время, вооружатся друзья против друзей, как враги, и устрашится земля с живущими на ней, и жилы источников остановятся и три часа не будут течь.
\vs 3Ez 6:25 Всякий, кто после всего этого, о чем Я предсказал тебе, останется в живых, сам спасется, и увидит спасение Мое и конец вашего века.
\vs 3Ez 6:26 И увидят люди избранные, которые не испытали смерти от рождения своего, и изменится сердце живущих и обратится в чувство иное.
\vs 3Ez 6:27 Ибо зло истребится, и исчезнет лукавство;
\vs 3Ez 6:28 процветет вера, побеждено будет растление, явится истина, которая столько времени оставалась без плода.
\vs 3Ez 6:29 Когда Он говорил, я взглянул на того, пред которым стоял.
\vs 3Ez 6:30 И он сказал мне: я пришел показать тебе время грядущей ночи.
\vs 3Ez 6:31 Итак, если ты опять помолишься и опять семь дней попостишься, то я покажу тебе больше в день, в который я услышал тебя.
\vs 3Ez 6:32 Голос твой услышан у Всевышнего; увидел Крепкий правильное действие, увидел и чистоту, которую хранил ты от юности твоей.
\vs 3Ez 6:33 Посему Он послал меня показать тебе все это и сказать: уповай и не бойся;
\vs 3Ez 6:34 не спеши с первыми временами помышлять суетное, дабы не судить тебе с такою же поспешностью о временах последних.
\rsbpar\vs 3Ez 6:35 После сего я снова со слезами молился, и также постился семь дней, чтобы исполнить три седмицы, заповеданные мне.
\vs 3Ez 6:36 В восьмую же ночь сердце мое пришло снова в возбуждение, и я начал говорить пред Всевышним,
\vs 3Ez 6:37 ибо дух мой воспламенялся сильно, и душа моя томилась.
\vs 3Ez 6:38 И сказал я: Господи! Ты от начала творения говорил; в первый день сказал: <<да будет небо и земля>>, и слово Твое было совершившимся делом.
\vs 3Ez 6:39 Тогда носился Дух, и тьма облегала вокруг и молчание: звука человеческого голоса еще не было.
\vs 3Ez 6:40 Тогда повелел Ты из сокровищниц Твоих выйти обильному свету, чтобы явилось дело Твое.
\vs 3Ez 6:41 Во второй день сотворил Ты дух тверди и повелел ему отделить и произвести разделение между водами, чтобы некоторая часть их поднялась вверх, а прочая осталась внизу.
\vs 3Ez 6:42 В третий день Ты повелел водам собраться на седьмой части земли, а шесть частей осушил, чтобы они служили пред Тобою к обсеменению и обработанию.
\vs 3Ez 6:43 Слово Твое исходило, и тотчас являлось дело;
\vs 3Ez 6:44 вдруг явилось безмерное множество плодов и многоразличные приятности для вкуса, цветы в виде своем неизменные, с запахом, несказанно благоуханным: все это совершено было в третий день.
\vs 3Ez 6:45 В четвертый день Ты повелел быть сиянию солнца, свету луны, расположению звезд
\vs 3Ez 6:46 и повелел, чтобы они служили имеющему быть созданным человеку.
\vs 3Ez 6:47 В пятый день Ты сказал седьмой части, в которой была собрана вода, чтобы она произвела животных, летающих и рыб, что и сделалось.
\vs 3Ez 6:48 Вода немая и бездушная, по мановению Божию, произвела животных, чтобы все роды возвещали дивные дела Твои.
\vs 3Ez 6:49 Тогда Ты сохранил двух животных: одно называлось бегемотом, а другое левиафаном.
\vs 3Ez 6:50 И Ты отделил их друг от друга, потому что седьмая часть, где была собрана вода, не могла принять их вместе.
\vs 3Ez 6:51 Бегемоту Ты дал одну часть из земли, осушенной в третий день, да обитает в ней, в которой тысячи гор.
\vs 3Ez 6:52 Левиафану дал седьмую часть водяную, и сохранил его, чтобы он был пищею тем, кому Ты хочешь, и когда хочешь.
\vs 3Ez 6:53 В шестый же день повелел Ты земле произвести пред Тобою скотов, зверей и пресмыкающихся;
\vs 3Ez 6:54 а после них Ты сотворил Адама, которого поставил властелином над всеми Твоими тварями и от которого происходим все мы и народ, который Ты избрал.
\vs 3Ez 6:55 Все это сказал я пред Тобою, Господи, потому что для нас создал Ты век сей.
\vs 3Ez 6:56 О прочих же народах, происшедших от Адама, Ты сказал, что они ничто, но подобны слюне, и все множество их Ты уподобил каплям, каплющим из сосуда.
\vs 3Ez 6:57 И ныне, Господи, вот, эти народы, за ничто Тобою признанные, начали владычествовать над нами и пожирать нас.
\vs 3Ez 6:58 Мы же, народ Твой, который Ты назвал Твоим первенцем, единородным, возлюбленным Твоим, преданы в руки их.
\vs 3Ez 6:59 Если для нас создан век сей, то почему не получаем мы наследия с веком? И доколе это?
\vs 3Ez 7:1 Когда я окончил говорить эти слова, послан был ко мне Ангел, который посылаем был ко мне в прежние ночи,
\vs 3Ez 7:2 и сказал мне: встань, Ездра, и слушай слов\acc{а}, которые я пришел говорить тебе.
\vs 3Ez 7:3 Я сказал: говори, господин мой. И он сказал мне: море расположено в пространном месте, чтобы быть глубоким и безмерным;
\vs 3Ez 7:4 но вход в него находится в тесном месте, так что подобен рекам.
\vs 3Ez 7:5 Кто пожелал бы войти в море и видеть его, или господствовать над ним, тот, если не пройдет тесноты, как может дойти до широты?
\vs 3Ez 7:6 Или иное подобие: город построен и расположен на равнине, и наполнен всеми благами;
\vs 3Ez 7:7 но вход в него тесен и расположен на крутизне так, что по правую сторону огонь, а по левую глубокая вода.
\vs 3Ez 7:8 Между ними, то есть между огнем и водою, лежит лишь одна стезя, на которой может поместиться не более, как только ступень человека.
\vs 3Ez 7:9 Если город этот будет дан в наследство человеку, то как он получит свое наследство, если никогда не перейдет лежащей на пути опасности?
\vs 3Ez 7:10 Я сказал: так, Господи. И Он сказал мне: такова и доля Израиля.
\vs 3Ez 7:11 Для них Я сотворил век; но когда Адам нарушил Мои постановления, определено быть тому, что сделано.
\vs 3Ez 7:12 И сделались входы века сего тесными, болезненными, утомительными, также узкими, лукавыми, исполненными бедствий и требующими великого труда.
\vs 3Ez 7:13 А входы будущего века пространны, безопасны, и приносят плод бессмертия.
\vs 3Ez 7:14 Итак, если входящие, которые живут, не войдут в это тесное и бедственное, они не могут получить, что уготовано.
\vs 3Ez 7:15 Зачем же смущаешься, когда ты тленен, и что мятешься, когда смертен?
\vs 3Ez 7:16 Зачем не принял ты в сердце твоем того, что будущее, а принял то, что в настоящем?
\vs 3Ez 7:17 Я отвечал и сказал: Владыко Господи! вот, Ты определил законом Твоим, что праведники наследуют это, а грешники погибнут.
\vs 3Ez 7:18 Праведники потерпят тесноту, надеясь пространного, а нечестиво жившие, хотя потерпели тесноту, не увидят пространного.
\vs 3Ez 7:19 И Он сказал мне: нет судии выше Бога, нет разумеющего более Всевышнего.
\vs 3Ez 7:20 Погибают многие в этой жизни, потому что нерадят о предложенном им законе Божием.
\vs 3Ez 7:21 Ибо строго повелел Бог приходящим, когда они пришли, что делая, они будут живы, и что соблюдая, не будут наказаны.
\vs 3Ez 7:22 А они не послушались, и воспротивились Ему, утвердили в себе помышление суетное.
\vs 3Ez 7:23 Увлеклись греховными обольщениями, сказали о Всевышнем, что \bibemph{Его} нет, не познали путей Его,
\vs 3Ez 7:24 презрели закон Его, отвергли обетования Его, не имели веры к обрядовым установлениям Его, не совершали дел Его.
\vs 3Ez 7:25 И потому, Ездра, пустым пустое, а полным полное.
\vs 3Ez 7:26 Вот, придет время, когда придут знамения, которые Я предсказал тебе, и явится невеста, и являясь покажется,~--- скрываемая ныне землею.
\vs 3Ez 7:27 И всякий, кто избавится от прежде исчисленных зол, сам увидит чудеса Мои.
\vs 3Ez 7:28 Ибо откроется Сын Мой Иисус с теми, которые с Ним, и оставшиеся будут наслаждаться четыреста лет.
\vs 3Ez 7:29 А после этих лет умрет Сын Мой Христос и все люди, имеющие дыхание.
\vs 3Ez 7:30 И обратится век в древнее молчание на семь дней, подобно тому, как было прежде, так что не останется никого.
\vs 3Ez 7:31 После же семи дней восстанет век усыпленный, и умрет поврежденный.
\vs 3Ez 7:32 И отдаст земля тех, которые в ней спят, и прах тех, которые молчаливо в нем обитают, а хранилища отдадут вверенные им души.
\vs 3Ez 7:33 Тогда явится Всевышний на престоле суда, и пройдут беды, и окончится долготерпение.
\vs 3Ez 7:34 Суд будет один, истина утвердится, вера укрепится.
\vs 3Ez 7:35 Затем последует дело, откроется воздаяние, восстанет правда, перестанет господствовать неправда.
\vs 3Ez 7:(36) \fns{70 стихов, находящихся между 35 и 36 стихами 7-й главы, имеются в русском переводе в <<Толковой Библии>> А.П.Лопухина (Петербург, 1913) и в т.н. <<Брюссельской>> Библии (Брюссель, 1973). В Синодальной Библии их нет.}И откроется озеро мучения, а против него место покоя; видна будет печь геенны, а против нее рай сладости.
\vs 3Ez 7:(37) И скажет тогда Всевышний пробудившимся народам: <<посмотрите и поймите, Кого вы отвергли, Кому вы не служили и Чьи заповеди вы презрели.
\vs 3Ez 7:(38) Взгляните прямо пред собою и напротив: там сладость и покой, а тут огонь и мучения>>. Вот что скажешь Ты им в день суда.
\vs 3Ez 7:(39) Этот день таков, что не имеет ни солнца, ни луны, ни звезд,
\vs 3Ez 7:(40) ни облака, ни грома, ни молнии, ни ветра, ни дождя, ни тумана, ни мрака, ни вечера, ни утра,
\vs 3Ez 7:(41) ни лета, ни весны, ни жары, ни зимы, ни мороза, ни холода, ни града, ни дождя, ни росы,
\vs 3Ez 7:(42) ни полдня, ни ночи, ни предрассветных сумерек, ни блеска, ни ясности, ни света, кроме одного лишь сияния светлости Всевышнего, вследствие чего все могут видеть то, что пред ними.
\vs 3Ez 7:(43) Его длительность будет такая же, как седьмины лет.
\vs 3Ez 7:(44) Таков суд Мой и его порядок. Одному тебе Я открыл это.
\vs 3Ez 7:(45) И я отвечал: <<я говорил уже, и теперь скажу: блаженны живущие и исполняющие заповеданное Тобою.
\vs 3Ez 7:(46) Но я молил о следующем: найдется ли кто из живущих, чтобы не грешил, или найдется ли кто из родившихся, чтобы не нарушал Твоего завета?
\vs 3Ez 7:(47) И теперь я вижу, что будущий век принесет сладость немногим, а мучения многим.
\vs 3Ez 7:(48) Ибо внутри нас выросло сердце злое, которое удалило нас от Него и привело нас к тлению и путям смерти, показало нам тропинки погибели и удалило нас от жизни, притом не малое количество, но почти всех, кто был сотворен>>.
\vs 3Ez 7:(49) И Он отвечал мне и сказал: выслушай Меня, и Я наставлю тебя и вразумлю тебя относительно имеющего быть.
\vs 3Ez 7:(50) В виду этого Бог и сотворил не один век, а два.
\vs 3Ez 7:(51) Что же касается твоих слов, что праведных не много, но мало, тогда как нечестивых множество, то выслушай на это вот что:
\vs 3Ez 7:(52) <<если у тебя будет весьма немного драгоценных камней, то ты станешь складывать их у себя по числу их; свинца же и глины изобилие>>.
\vs 3Ez 7:(53) И я сказал: <<как же это возможно?>>
\vs 3Ez 7:(54) И Он сказал мне: <<не только это, но спроси землю, и та скажет тебе, подойди к ней с лестью, и та поведает тебе.
\vs 3Ez 7:(55) Ты скажешь ей: ты производишь золото, серебро и медь, а также железо, свинец и глину.
\vs 3Ez 7:(56) Серебра же больше, чем золота, меди больше, чем серебра, железа больше, чем меди, свинца больше, чем железа, и глины больше, чем свинца.
\vs 3Ez 7:(57) Посуди теперь сам, что драгоценно и влечет к себе, то ли, чего много, или то, что является редкостью>>.
\vs 3Ez 7:(58) И я сказал: <<Владыка Господи! Что встречается в избытке, то хуже, а что попадается реже, то драгоценнее>>.
\vs 3Ez 7:(59) И Он отвечал мне и сказал: <<взвесь про себя то, что ты подумал: кто владеет тем, что с трудом добывается, бывает рад больше того, кто обладает тем, что встречается в избытке.
\vs 3Ez 7:(60) Так обстоит дело и с обещанною Мною тварью. Я рад буду немногим спасшимся, потому что они утвердили ныне владычество Моей славы и на них наречено ныне же Мое имя.
\vs 3Ez 7:(61) Меня не будет огорчать множество погибших: ведь это те самые, которые теперь уже уподоблены пару и приравнены к огню и дыму. Вот они вспыхнули, запылали и погасли>>.
\vs 3Ez 7:(62) И я отвечал и сказал: <<о, земля! что же ты породила, если разум произошел из праха, как и остальная тварь?
\vs 3Ez 7:(63) Лучше было бы не появляться самому праху, чтобы из него не возник разум.
\vs 3Ez 7:(64) А теперь, разум возрастает вместе с нами, и из-за этого мы мучимся, так как сознательно идем к гибели.
\vs 3Ez 7:(65) Пусть рыдает род человеческий, и радуются полевые звери; пусть рыдают все, кто родился, и веселятся четвероногие и скоты.
\vs 3Ez 7:(66) Ибо им гораздо лучше, чем нам, так как они не ждут суда; им неведомы ни мучения, ни блаженство, обещанные им после смерти.
\vs 3Ez 7:(67) Что нам пользы в том, что мы будем снова жить, но будем жестоко мучиться?
\vs 3Ez 7:(68) Ведь все, кто родился, пропитаны беззакониями, полны грехов и отягчены преступлениями.
\vs 3Ez 7:(69) И быть может, лучше было бы нам, если бы нам не нужно было идти на суд>>.
\vs 3Ez 7:(70) И Он отвечал мне и сказал: <<раньше, чем Всевышний сотворил век с Адамом и всеми, происшедшими от него, Он приготовил суд и то, что относится к суду.
\vs 3Ez 7:(71) Теперь же уразумей на основании своих собственных слов; ведь ты сказал, что разум возрастает с нами.
\vs 3Ez 7:(72) Поэтому те, кто живет на земле, терпят здесь мучения, потому что, имея разум, они совершали беззакония и, получая заповеди, не исполняли их, и, будучи последователями закона, отвергали закон, полученный ими.
\vs 3Ez 7:(73) Что же имеют они сказать на суде или какой ответ дадут они в ближайшее время?
\vs 3Ez 7:(74) В самом деле, сколько времени Всевышний проявлял долготерпение к тем, кто населяет век, и не ради их самих, а ради исполнения предусмотренного Им срока>>.
\vs 3Ez 7:(75) И я отвечал и сказал: <<если я нашел благодать пред Тобою, Господи, то покажи рабу Твоему еще следующее. Будем ли мы после смерти, то есть когда каждый из нас отдаст душу свою, пребывать в покое, пока не наступят те времена, когда Ты начнешь обновлять тварь, или же тотчас будем терпеть мучения?>>
\vs 3Ez 7:(76) И Он отвечал мне и сказал: <<покажу тебе и это. Но ты не смешивай себя с теми, кто презирал, и не причисляй себя к тем, которые терпят мучения,
\vs 3Ez 7:(77) ибо у тебя есть сокровище дел, сохраняемое у Всевышнего; но оно не будет пока дано тебе до наступления последнего времени.
\vs 3Ez 7:(78) Теперь будет речь о смерти, когда выйдет от Всевышнего приговор относительно срока, чтобы умереть человеку, и когда дух выйдет из тела, чтобы снова вернуться к Тому, Кто дал его, для поклонения прежде всего славе Всевышнего.
\vs 3Ez 7:(79) И если это будут души тех, кто презирал и не сохранял путей Всевышнего, пренебрегал Его законом и ненавидел боящихся Бога,
\vs 3Ez 7:(80) то таковые души не войдут в обители, но немедленно начнут в мучениях, в постоянной скорби и печали блуждать по семи путям.
\vs 3Ez 7:(81) Первый путь это то, что они презрели закон Всевышнего.
\vs 3Ez 7:(82) Второй путь: они уже не могут принести доброе раскаяние, чтобы жить.
\vs 3Ez 7:(83) Третий путь: они увидят награду, сохраняемую для тех, кто верен заветам Всевышнего.
\vs 3Ez 7:(84) Четвертый путь: они увидят мучения, сохраняемые для них на самое последнее время.
\vs 3Ez 7:(85) Пятый путь: они видят жилища других, охраняемые в глубочайшем молчании ангелами.
\vs 3Ez 7:(86) Шестой путь: они видят, что немедленно же отсюда они перейдут на мучения.
\vs 3Ez 7:(87) Седьмой путь, превосходящий все названные выше пути, состоит в том, что они тают от смятения, их снедает стыд, они изнемогают от страха, при виде славы Всевышнего, пред которой они грешили при жизни и пред которой им предстоит суд в последние времена.
\vs 3Ez 7:(88) Что же касается тех, кто сохранял пути Всевышнего, то удел их по разлучению с тленным сосудом будет следующий:
\vs 3Ez 7:(89) во время пребывания в нем они с трудностями служили Всевышнему и каждый час подвергались опасностям, лишь бы всецело сохранить закон Законодателя.
\vs 3Ez 7:(90) Поэтому приговор о них будет такой:
\vs 3Ez 7:(91) прежде всего они увидят с великою радостью славу Того, Кто принимает их к Себе; покой же они будут вкушать семи видов.
\vs 3Ez 7:(92) Первый вид это то, что они с великим трудом вели борьбу, с целью преодолеть помышление злое, созданное вместе с ними, чтобы оно не могло отвлекать их от жизни к смерти.
\vs 3Ez 7:(93) Второй вид: они созерцают смятение, в каком блуждают души нечестивых, и наказание, предстоящее им.
\vs 3Ez 7:(94) Третий вид: они созерцают данное им их Создателем свидетельство, что они при жизни сохранили закон, вверенный им.
\vs 3Ez 7:(95) Четвертый вид: они сознают свой покой, которым они наслаждаются ныне, собравшись в своих хранилищах и оберегаемые в глубоком молчании ангелами, и прославление, ожидающее их в последние времена.
\vs 3Ez 7:(96) Пятый вид: они ликуют по поводу того, что покинули ныне тленное и получат будущее наследие; они видят кроме того ту тесноту, полную тягостей, от которой они освободились, и начинают чувствовать простор, блаженные и бессмертные.
\vs 3Ez 7:(97) Шестой вид: им показано будет, как лицо их засияет подобно солнцу и они уподобятся по блеску звездам, став тотчас же нетленными.
\vs 3Ez 7:(98) Седьмой вид, превосходящий все ранее названные: они будут ликовать с уверенностью, надеяться без посрамления и радоваться без страха, так как они спешат увидеть лицо Того, Кому они служили при жизни, и от Кого они должны получить награду, состоящую в прославлении.
\vs 3Ez 7:(99) Таков удел душ праведников, возвещаемый им тотчас же. Ранее были названы пути тех мучений, которые терпят немедленно же грешники>>.
\vs 3Ez 7:(100) И я отвечал и сказал: <<значит, душам по разлучении их с телом будет дано время, чтобы видеть то, о чем Ты мне сказал>>.
\vs 3Ez 7:(101) И Он сказал мне: <<семь дней будет длиться их свобода, чтобы они за семь дней увидели то, о чем была выше речь, а после этого они соберутся в свои жилища>>.
\vs 3Ez 7:(102) И я отвечал и сказал: <<если я нашел милость пред очами Твоими, то покажи мне, рабу Твоему, кроме того, могут ли в день суда праведники достигнуть оправдания нечестивых или молить за них Всевышнего,
\vs 3Ez 7:(103) отцы за сыновей, сыновья за родителей, братья за братьев, родственники за своих близких, или друзья за дорогих для них лиц>>.
\vs 3Ez 7:(104) Он отвечал мне и сказал: <<так как ты нашел милость пред очами Моими, то Я покажу тебе и это. День суда решительный и являет всем печать истины. Подобно тому, как ныне отец не посылает сына или сын отца, или господин раба, или друг самого дорогого для него человека с тем, чтобы тот думал за него, или спал, или ел, или лечился,
\vs 3Ez 7:(105) так никогда никто не будет за кого-либо ходатайствовать, но каждый принесет тогда свои правды или неправды>>.
\vs 3Ez 7:36 Я сказал: Авраам первый молился о Содомлянах; Моисей~--- за отцов, согрешивших в пустыне;
\vs 3Ez 7:37 Иисус после него~--- за Израиля во дни Ахана;
\vs 3Ez 7:38 Самуил и Давид~--- за погубляемых, Соломон~--- за тех, которые пришли на освящение;
\vs 3Ez 7:39 Илия~--- за тех, которые приняли дождь, и за мертвеца, чтобы он ожил;
\vs 3Ez 7:40 Езекия~--- за народ во дни Сеннахирима, и многие~--- за многих.
\vs 3Ez 7:41 Итак, если тогда, когда усилилось растление и умножилась неправда, праведные молились за неправедных, то почему же не быть тому и ныне?
\vs 3Ez 7:42 Он отвечал мне и сказал: настоящий век не есть конец; славы в нем часто не бывает, потому молились за немощных.
\vs 3Ez 7:43 День же суда будет концом времени сего и началом времени будущего бессмертия, когда пройдет тление,
\vs 3Ez 7:44 прекратится невоздержание, пресечется неверие, а возрастет правда, воссияет истина.
\vs 3Ez 7:45 Тогда никто не возможет спасти погибшего, ни погубить победившего.
\vs 3Ez 7:46 Я отвечал и сказал: вот мое слово первое и последнее: лучше было не давать земли Адаму, или, когда уже дана, удержать его, чтобы не согрешил.
\vs 3Ez 7:47 Что пользы людям~--- в настоящем веке жить в печали, а по смерти ожидать наказания?
\vs 3Ez 7:48 О, что сделал ты, Адам? Когда ты согрешил, то совершилось падение не тебя только одного, но и нас, которые от тебя происходим.
\vs 3Ez 7:49 Что пользы нам, если нам обещано бессмертное время, а мы делали смертные дела?
\vs 3Ez 7:50 Нам предсказана вечная надежда, а мы, непотребные, сделались суетными.
\vs 3Ez 7:51 Нам уготованы жилища здоровья и покоя, а мы жили худо;
\vs 3Ez 7:52 уготована слава Всевышнего, чтобы покрыть тех, которые жили кротко, а мы ходили по путям злым.
\vs 3Ez 7:53 Показан будет рай, плод которого пребывает нетленным и в котором покой и врачевство;
\vs 3Ez 7:54 но мы не войдем \bibemph{в него}, потому что обращались в местах неплодных.
\vs 3Ez 7:55 Светлее звезд воссияют лица тех, которые имели воздержание, а наши лица~--- чернее тьмы.
\vs 3Ez 7:56 Мы не помышляли в жизни, когда делали беззаконие, что по смерти будем страдать.
\vs 3Ez 7:57 Он отвечал и сказал: это~--- помышление о борьбе, которую должен вести на земле родившийся человек,
\vs 3Ez 7:58 чтобы, если будет побежден, потерпеть то, о чем ты сказал, а если победит, получить то, о чем Я говорю.
\vs 3Ez 7:59 Это та жизнь, о которой сказал Моисей, когда жил, к народу, говоря: <<избери себе жизнь, чтобы жить>>.
\vs 3Ez 7:60 Но они не поверили ему, ни пророкам после него, ни Мне, говорившему к ним,
\vs 3Ez 7:61 что не будет скорби о погибели их, как будет радость о тех, которым уготовано спасение.
\vs 3Ez 7:62 Я отвечал и сказал: знаю, Господи, что Всевышний называется милосердым, потому что помилует тех, которые еще не пришли в мир,
\vs 3Ez 7:63 и милует тех, которые провождают жизнь в законе Его.
\vs 3Ez 7:64 Он долготерпелив, ибо оказывает долготерпение к согрешившим, как к Своему творению.
\vs 3Ez 7:65 Он щедр, ибо готов давать по надобности,
\vs 3Ez 7:66 и многомилостив, ибо умножает милости Свои к живущим ныне и к жившим и к тем, которые будут жить.
\vs 3Ez 7:67 Ибо, если бы не умножал Он Своих милостей, то не мог бы век продолжать жить с теми, которые обитают в нем.
\vs 3Ez 7:68 Он подает дары; ибо если бы не даровал по благости Своей, да облегчатся совершившие нечестие от своих беззаконий, то не могла бы оставаться в живых десятитысячная часть людей.
\vs 3Ez 7:69 Он судия, и если бы не прощал тех, которые сотворены словом Его, и не истребил множества преступлений,
\vs 3Ez 7:70 может быть, из бесчисленного множества остались бы только весьма немногие.
\vs 3Ez 8:1 Он отвечал мне и сказал: этот век Всевышний сотворил для многих, а будущий для немногих.
\vs 3Ez 8:2 Скажу тебе, Ездра, подобие. Как если спросишь землю, она скажет тебе, что дает очень много вещества, из которого делаются глиняные вещи, а не много праха, из которого бывает золото, так и дела настоящего века.
\vs 3Ez 8:3 Многие сотворены, но немногие спасутся.
\vs 3Ez 8:4 Я отвечал и сказал: душа! пожри смысл и поглоти мудрость.
\vs 3Ez 8:5 Ибо ты обещала слушать, и пожелала пророчествовать, а тебе дано время только, чтобы жить.
\vs 3Ez 8:6 О, Господи! неужели Ты не позволишь рабу Твоему, чтобы мы молились пред Тобою о даровании сердцу нашему семени и разуму возделания, чтобы произошел плод, которым мог бы жить всякий растленный, кто будет носить имя человека?
\vs 3Ez 8:7 Ты един, и мы единое творение рук Твоих, как сказал Ты.
\vs 3Ez 8:8 И как же ныне во чреве матернем образуется тело, и Ты даешь члены, как сохраняется Твое творение в огне и воде, и как девять месяцев терпит в себе Твое же создание Твою тварь, которая в нем сотворена?
\vs 3Ez 8:9 И хранящее и хранимое, и то и другое сохраняются, и чрево матери в свое время отдает то сохраненное, что в нем произросло.
\vs 3Ez 8:10 Ты повелел из самих членов, то есть из сосцов, давать молоко, плод сосцов,
\vs 3Ez 8:11 да питается созданное до некоторого времени, а после передашь его Твоему милосердию.
\vs 3Ez 8:12 Ты воспитал его Твоею правдою, научил его Твоему закону, наставил его Твоим разумом,
\vs 3Ez 8:13 и умертвишь его, как Твое творение, и опять оживишь, как Твое дело.
\vs 3Ez 8:14 Если Ты погубишь созданного с таким попечением, то повелению Твоему легко устроить, чтобы и сохранялось то, что было создано.
\vs 3Ez 8:15 И ныне, Господи, я скажу: о всяком человеке Ты больше знаешь; но \bibemph{скажу} о народе Твоем, о котором болезную,
\vs 3Ez 8:16 о наследии Твоем, о котором проливаю слезы, об Израиле, о котором скорблю, об Иакове, о котором сокрушаюсь.
\vs 3Ez 8:17 Начну молиться пред Тобою за себя и за них, ибо вижу грехопадения нас, обитающих на земле.
\vs 3Ez 8:18 Но я слышал, что скоро придет Судия.
\vs 3Ez 8:19 Посему услышь мой голос, вонми словам моим, и я буду говорить пред Тобою. [Начало слов Ездры, прежде нежели он был взят.]
\vs 3Ez 8:20 Я сказал: Господи, живущий вечно, Которого очи обращены на выспреннее и небесное,
\vs 3Ez 8:21 Которого престол неоценим и слава непостижима, Которому с трепетом предстоят воинства Ангелов, служащих в ветре и огне, Которого слово истинно и глаголы непреложны,
\vs 3Ez 8:22 повеление сильно и правление страшно, Которого взор иссушает бездны, гнев расплавляет горы и истина пребывает во веки!
\vs 3Ez 8:23 Услышь молитву раба Твоего, и вонми молению создания Твоего.
\vs 3Ez 8:24 Доколе живу, буду говорить, и доколе разумею, буду отвечать. Не взирай на грехи народа Твоего, но на тех, которые Тебе в истине служат;
\vs 3Ez 8:25 не обращай внимания на нечестивые дела язычников, но на тех, которые заветы Твои сохранили среди бедствий;
\vs 3Ez 8:26 не помышляй о тех, которые пред Тобою лживо поступали, но помяни тех, которые, по воле Твоей, познали страх;
\vs 3Ez 8:27 не погубляй тех, которые жили по-скотски, но воззри на тех, которые ясно учили закону Твоему;
\vs 3Ez 8:28 не прогневайся на тех, которые признаны худшими зверей;
\vs 3Ez 8:29 но возлюби тех, которые всегда надеются на правду Твою и славу.
\vs 3Ez 8:30 Ибо мы и отцы наши такими болезнями страдаем;
\vs 3Ez 8:31 а Ты, ради нас~--- грешных, назовешься милосердым.
\vs 3Ez 8:32 Если Ты пожелаешь помиловать нас, то назовешься милосердым, потому что мы не имеем дел правды.
\vs 3Ez 8:33 Праведники же, у которых много дел приобретено, по собственным делам получат воздаяние.
\vs 3Ez 8:34 Что есть человек, чтобы Ты гневался на него, и род растленный, чтобы Ты столько огорчался им?
\vs 3Ez 8:35 Поистине, нет никого из рожденных, кто не поступил бы нечестиво, и из исповедающих \bibemph{Тебя} нет никого, кто не согрешил бы.
\vs 3Ez 8:36 В том-то и возвестится правда Твоя и благость Твоя, Господи, когда помилуешь тех, которые не имеют существа добрых дел.
\vs 3Ez 8:37 Он отвечал мне и сказал: справедливо ты сказал нечто, и по словам твоим так и будет.
\vs 3Ez 8:38 Ибо истинно не помышляю Я о делах тех созданий, которые согрешили, прежде смерти, прежде суда, прежде погибели;
\vs 3Ez 8:39 но услаждаюсь подвигами праведных, и воспоминаю, как они странствовали, как спасались и старались заслужить награду.
\vs 3Ez 8:40 Как сказал Я, так и есть.
\vs 3Ez 8:41 Как земледелец сеет на земле многие семена и садит многие растения, но не все посеянное сохранится со временем, и не все посаженное укоренится, так и те, которые посеяны в веке \bibemph{сем}, не все спасутся.
\vs 3Ez 8:42 Я отвечал и сказал: если я обрел благодать, то буду говорить.
\vs 3Ez 8:43 Как семя земледельца, если не взойдет, или не примет вовремя дождя Твоего, или повредится от множества дождя, погибает:
\vs 3Ez 8:44 так и человек, созданный руками Твоими,~--- и Ты называешься его первообразом, потому что Ты подобен ему, для которого создал все и которого Ты уподобил семени земледельца.
\vs 3Ez 8:45 Не гневайся на нас, но пощади народ Твой и помилуй наследие Твое,~--- а Ты милосерд к созданию Твоему.
\vs 3Ez 8:46 Он отвечал мне и сказал: настоящее настоящим и будущее будущим.
\vs 3Ez 8:47 Многого недостает тебе, чтобы ты мог возлюбить создание Мое более Меня, хотя Я часто приближался к тебе самому, а к неправедным никогда.
\vs 3Ez 8:48 Но и в том дивен ты пред Всевышним,
\vs 3Ez 8:49 что смирил себя, как прилично тебе, и не судил о себе так, чтобы много славиться между праведными.
\vs 3Ez 8:50 Многие и горестные бедствия постигнут тех, которые населяют век, в последнее время, потому что они ходили в великой гордыне.
\vs 3Ez 8:51 А ты заботься о себе, и подобным тебе ищи славы;
\vs 3Ez 8:52 ибо вам открыт рай, насаждено древо жизни, предназначено будущее время, готово изобилие, построен город, приготовлен покой, совершенная благость и совершенная премудрость.
\vs 3Ez 8:53 Корень зла запечатан от вас, немощь и тля сокрыты от вас, и растление бежит в ад в забвение.
\vs 3Ez 8:54 Прошли болезни, и в конце показалось сокровище бессмертия.
\vs 3Ez 8:55 Не старайся более испытывать о множестве погибающих.
\vs 3Ez 8:56 Ибо они, получив свободу, презрели Всевышнего, пренебрегли закон Его и оставили пути Его,
\vs 3Ez 8:57 а еще и праведных Его попрали,
\vs 3Ez 8:58 и говорили в сердце своем: <<нет Бога>>, хотя и знали, что они смертны.
\vs 3Ez 8:59 Как вас ожидает то, о чем сказано прежде, так и их~--- жажда и мучение, которые приготовлены. Бог не хотел погубить человека,
\vs 3Ez 8:60 но сами сотворенные обесславили имя Того, Кто сотворил их, и были неблагодарными к Тому, Кто предуготовил им жизнь.
\vs 3Ez 8:61 Посему суд Мой ныне приближается,~---
\vs 3Ez 8:62 о чем Я не всем открыл, а только тебе и немногим, тебе подобным. Я отвечал и сказал:
\vs 3Ez 8:63 вот ныне, Господи, Ты показал мне множество знамений, которые Ты начнешь творить при кончине, но не показал, в какое время.
\vs 3Ez 9:1 Он отвечал мне и сказал: измеряя измеряй время в себе самом, и когда увидишь, что прошла некоторая часть знамений, прежде указанных,
\vs 3Ez 9:2 тогда уразумеешь, что это и есть то время, в которое начнет Всевышний посещать век, Им созданный.
\vs 3Ez 9:3 Когда обнаружится в веке колебание мест, смятение народов,
\vs 3Ez 9:4 тогда уразумеешь, что об этом говорил Всевышний от дней, бывших прежде тебя, от начала.
\vs 3Ez 9:5 Как все, сотворенное в веке, имеет начало, равно и конец, и окончание бывает явно:
\vs 3Ez 9:6 так и времена Всевышнего имеют начала, открывающиеся чудесами и силами, и окончания, являемые действиями и знамениями.
\vs 3Ez 9:7 Всякий, кто спасется и возможет делами своими и верою, которою веруете, избежать от преждесказанных бед,
\vs 3Ez 9:8 останется, и увидит спасение Мое на земле Моей и в пределах Моих, которые Я освятил Себе от века.
\vs 3Ez 9:9 Тогда пожалеют отступившие ныне от путей Моих, и отвергшие их с презрением пребудут в муках.
\vs 3Ez 9:10 Те, которые не познали Меня, получая при жизни благодеяния,
\vs 3Ez 9:11 и возгнушались законом Моим, не уразумели его, но презрели, когда еще имели свободу и когда еще отверсто было им место для покаяния,
\vs 3Ez 9:12 те познают Меня по смерти в мучении.
\vs 3Ez 9:13 Ты не любопытствуй более, как нечестивые будут мучиться, но исследуй, как спасутся праведные, которым принадлежит век и ради которых век, и когда.
\vs 3Ez 9:14 Я отвечал и сказал:
\vs 3Ez 9:15 я прежде говорил, и теперь говорю, и после буду говорить, что больше тех, которые погибнут, нежели тех, которые спасутся, как волна больше капли.
\vs 3Ez 9:16 Он отвечал мне и сказал:
\vs 3Ez 9:17 какова нива, таковы и семена; каковы цветы, таковы и краски; каков делатель, таково и дело; каков земледелец, таково и возделывание; ибо то было время века.
\vs 3Ez 9:18 Когда Я уготовлял век, прежде нежели он был, для обитания тех, которые живут ныне в нем, никто Мне не противоречил.
\vs 3Ez 9:19 А ныне, когда век сей был создан, нравы сотворенных повредились при неоскудевающей жатве, при неисследимом законе.
\vs 3Ez 9:20 И рассмотрел Я век, и вот, оказалась опасность от замыслов, которые появились в нем.
\vs 3Ez 9:21 Я увидел и пощадил его, и сохранил для Себя одну ягоду из виноградной кисти и одно насаждение из множества.
\vs 3Ez 9:22 Пусть погибнет множество, которое напрасно родилось, и сохранится ягода Моя и насаждение Мое, которое Я вырастил с большим трудом.
\vs 3Ez 9:23 А ты, когда по прошествии семи дней иных, не постясь однако в них,
\vs 3Ez 9:24 выйдешь на цветущее поле, где нет построенного дома, и станешь питаться только от полевых цветов и не вкушать мяса, ни пить вина, а только цветы,
\vs 3Ez 9:25 молись ко Всевышнему непрестанно, и Я приду и буду говорить с тобою.
\vs 3Ez 9:26 И пошел я, как Он сказал мне, на поле, которое называется Ардаф, и сел там в цветах и вкушал от полевых трав, и была мне пища от них в насыщение.
\vs 3Ez 9:27 После семи дней лежал я на траве, и сердце мое опять смущалось, как прежде.
\vs 3Ez 9:28 И отверзлись уста мои, и я начал говорить пред Всевышним и сказал:
\vs 3Ez 9:29 о, Господи! являя Себя нам, Ты явился отцам нашим в пустыне непроходимой и бесплодной, когда они вышли из Египта,
\vs 3Ez 9:30 и сказал: <<слушай Меня, Израиль, и внимай словам Моим, семя Иакова.
\vs 3Ez 9:31 Вот, Я сею в вас закон Мой, и принесет в вас плод, и вы будете славиться в нем вечно>>.
\vs 3Ez 9:32 Но отцы наши, приняв закон, не исполнили его и постановлений Твоих не сохранили, и хотя плод закона Твоего не погиб и не мог погибнуть, потому что был Твой,
\vs 3Ez 9:33 но принявшие \bibemph{закон} погибли, не сохранив того, что в нем было посеяно.
\vs 3Ez 9:34 Обыкновенно бывает, что если земля приняла семя, или море корабль, или какой-либо сосуд пищу или питье, и если будет повреждено то, в чем посеяно, или то, в чем помещено,
\vs 3Ez 9:35 в таком случае погибает вместе и самое посеянное, или помещенное, или принятое, и принятого уже не остается пред нами. Но с нами не так.
\vs 3Ez 9:36 Мы, принявшие закон, согрешая, погибли, равно и сердце наше, которое приняло его;
\vs 3Ez 9:37 но закон не погиб, и остается в своей силе.
\vs 3Ez 9:38 Когда я говорил это в сердце моем, я воззрел глазами моими, и увидел на правой стороне женщину; и вот, она плакала и рыдала с великим воплем, и сильно болела душею; одежда ее была разодрана, а на голове ее пепел.
\vs 3Ez 9:39 Тогда оставил я размышления, которыми был занят, и, обратившись к ней, сказал ей:
\vs 3Ez 9:40 о чем плачешь ты, и о чем так скорбишь душею?
\vs 3Ez 9:41 Она сказала: оставь меня, господин мой, да плачу о себе и усугублю скорбь, ибо я весьма огорчена душею и весьма унижена.
\vs 3Ez 9:42 Я спросил ее: что потерпела ты? скажи мне. И она отвечала мне:
\vs 3Ez 9:43 я была неплодна, раба твоя, и не рождала, имея мужа, тридцать лет.
\vs 3Ez 9:44 Каждый час, каждый день в эти тридцать лет я молила Всевышнего непрестанно,
\vs 3Ez 9:45 и услышал меня Бог, рабу твою, после тридцати лет, увидел смирение мое, внял скорби моей и дал мне сына, и я сильно обрадовалась ему, и муж мой, и все сограждане мои, и мы много прославляли Всевышнего.
\vs 3Ez 9:46 Я вскормила его с великим трудом,
\vs 3Ez 9:47 и когда он возрос и пошел взять себе жену, я устроила день пиршества.
\vs 3Ez 10:1 Но когда сын мой вошел в брачный чертог свой, он упал, и умер.
\vs 3Ez 10:2 И опрокинули все мы светильники, и все сограждане мои поднялись утешать меня, и я почила до ночи другого дня.
\vs 3Ez 10:3 Когда же все перестали утешать меня, чтобы оставить меня в покое, я, встав ночью, побежала и пришла, как видишь, на это поле.
\vs 3Ez 10:4 И думаю уже не возвращаться в город, но оставаться здесь, ни есть, ни пить, но непрестанно плакать и поститься, доколе не умру.
\vs 3Ez 10:5 Оставив размышления, которыми занимался, я с гневом отвечал ей и сказал:
\vs 3Ez 10:6 о, безумнейшая из всех жен! не видишь ли скорби нашей и приключившегося нам,~---
\vs 3Ez 10:7 что Сион, мать наша, печалится безмерно, крайне унижена, и плачет горько?
\vs 3Ez 10:8 И теперь, когда все мы скорбим и печалимся, потому что все опечалены, будешь ли ты печалиться об одном сыне твоем?
\vs 3Ez 10:9 Спроси землю, и она скажет тебе, что ей-то должно оплакивать падение столь многих рождающихся на ней;
\vs 3Ez 10:10 ибо все рожденные из нее от начала и другие, которые имеют произойти, едва не все погибают, и толикое множество их предаются истреблению.
\vs 3Ez 10:11 Итак кто должен более печалиться, как не та, которая потеряла толикое множество, а не ты, скорбящая об одном?
\vs 3Ez 10:12 Если ты скажешь мне: <<плач мой не подобен плачу земли, ибо я лишилась плода чрева моего, который я носила с печалью и родила с болезнью;
\vs 3Ez 10:13 а земля~--- по свойству земли; на ней настоящее множество как отходит, так и приходит>>:
\vs 3Ez 10:14 и я скажу тебе, что как ты с трудом родила, так и земля дает плод свой человеку, который от начала возделывает ее.
\vs 3Ez 10:15 Посему воздержись теперь от скорби твоей и мужественно переноси случившуюся тебе потерю.
\vs 3Ez 10:16 Ибо если ты признаешь праведным определение Божие, то в свое время получишь сына, и между женами будешь прославлена.
\vs 3Ez 10:17 Итак возвратись в город к мужу твоему.
\vs 3Ez 10:18 Но она сказала: не сделаю так, не возвращусь в город, но здесь умру.
\vs 3Ez 10:19 Продолжая говорить с нею, я сказал:
\vs 3Ez 10:20 не делай этого, но послушай совета моего. Ибо сколько бед Сиону? Утешься ради скорби Иерусалима.
\vs 3Ez 10:21 Ибо ты видишь, что святилище наше опустошено, алтарь наш ниспровергнут, храм наш разрушен,
\vs 3Ez 10:22 псалтирь наш уничижен, песни умолкли, радость наша исчезла, свет светильника нашего угас, ковчег завета нашего расхищен, Святое наше осквернено, и имя, которое наречено на нас, едва не поругано, дети наши потерпели позор, священники наши избиты, левиты наши отведены в плен, девицы наши осквернены, жены наши потерпели насилие, праведники наши увлечены, отроки наши погибли, юноши наши в рабстве, крепкие наши изнемогли;
\vs 3Ez 10:23 и что всего тяжелее, знамя Сиона лишено славы своей, потому что предано в руки ненавидящих нас.
\vs 3Ez 10:24 Посему оставь великую печаль твою, и отложи множество скорбей, чтобы помиловал тебя Крепкий, и Всевышний даровал тебе успокоение и облегчение трудов.
\vs 3Ez 10:25 При сих словах моих к ней, внезапно просияло лице и взор ее, и вот, вид сделался блистающим, так что я, устрашенный ею, помышлял, что бы это было.
\vs 3Ez 10:26 И вот, она внезапно испустила столь громкий и столь страшный звук голоса, что от сего звука жены поколебалась земля.
\vs 3Ez 10:27 И я видел, и вот, жена более не являлась мне, но созидался город, и место его обозначалось на обширных основаниях, и я устрашенный громко воскликнул и сказал:
\vs 3Ez 10:28 где Ангел Уриил, который вначале приходил ко мне? ибо он привел меня в такое исступление ума, в котором цель моего стремления исчезла, и молитва моя обратилась в поношение.
\vs 3Ez 10:29 Когда я говорил это, он пришел ко мне;
\vs 3Ez 10:30 и увидел меня, и вот, я лежал, как мертвый и в бессознательном состоянии; он взял меня за правую руку, укрепил меня и, поставив на ноги, сказал мне:
\vs 3Ez 10:31 что с тобою? отчего смущены разум твой и чувства сердца твоего? отчего смущаешься?
\vs 3Ez 10:32 Оттого, отвечал я ему, что ты оставил меня, и я, поступая по словам твоим, вышел на поле, и вот увидел и еще вижу то, о чем не могу рассказать.
\vs 3Ez 10:33 А он сказал мне: стой мужественно, и я объясню тебе.
\vs 3Ez 10:34 Говори мне, господин мой, сказал я, только не оставляй меня, чтобы я не умер напрасно;
\vs 3Ez 10:35 ибо я видел, чего не знал, и слышал, чего не знаю.
\vs 3Ez 10:36 Чувство ли мое обманывает меня, или душа моя грезит во сне?
\vs 3Ez 10:37 Посему прошу тебя объяснить мне, рабу твоему, это исступление ума моего. Отвечая мне, сказал он:
\vs 3Ez 10:38 внимай мне, и я научу тебя, и изъясню тебе то, что устрашило тебя: ибо Всевышний откроет тебе многие тайны.
\vs 3Ez 10:39 Он видит правый путь твой, что ты непрестанно скорбишь о народе твоем и сильно печалишься о Сионе.
\vs 3Ez 10:40 Таково значение видения, которое пред сим явилось тебе:
\vs 3Ez 10:41 жена, которую ты видел плачущею и старался утешать,
\vs 3Ez 10:42 которая потом сделалась невидима, но явился тебе город созидаемый,
\vs 3Ez 10:43 и которая тебе рассказала о смерти сына своего, вот что значит:
\vs 3Ez 10:44 жена, которую ты видел, это Сион. А что сказала тебе та, которую ты видел, как город только что созидаемый,
\vs 3Ez 10:45 что она тридцать лет была неплодна, этим указывается на то, что в продолжение тридцати лет в Сионе еще не была приносима жертва.
\vs 3Ez 10:46 По истечении тридцати лет неплодная родила сына: это было тогда, когда Соломон создал город и принес жертвы.
\vs 3Ez 10:47 А что она сказала тебе, что с трудом воспитала его, это было обитание в Иерусалиме.
\vs 3Ez 10:48 А что сын ее, как она сказала тебе, входя в чертог свой, упал и умер, это было падение Иерусалима.
\vs 3Ez 10:49 И вот, ты видел подобие ее, и как она скорбела о сыне, старался утешать ее в случившемся: то надлежало открыть тебе о сем.
\vs 3Ez 10:50 Ныне же Всевышний, видя, что ты скорбишь душею и всем сердцем болезнуешь о нем, показал тебе светлость славы его и красоту его.
\vs 3Ez 10:51 Для сего-то я повелел тебе жить в поле, где нет дома.
\vs 3Ez 10:52 Я знал, что Всевышний покажет тебе это;
\vs 3Ez 10:53 для того и повелел, чтобы ты пришел на поле, где не положено основания здания.
\vs 3Ez 10:54 Ибо не могло дело человеческого созидания существовать там, где начинал показываться город Всевышнего.
\vs 3Ez 10:55 Итак не бойся, и да не страшится сердце твое, но войди и посмотри на светлость и великолепие созидания, сколько могут видеть глаза твои.
\vs 3Ez 10:56 После того услышишь, сколько могут слышать уши твои.
\vs 3Ez 10:57 Ты блаженнее многих и призван к Всевышнему, как немногие.
\vs 3Ez 10:58 На завтрашнюю ночь оставайся здесь,
\vs 3Ez 10:59 и Всевышний покажет тебе видение величайших дел, которые Он сотворит для обитателей земли в последние дни.
\vs 3Ez 10:60 И спал я в ту ночь и в следующую, как он повелел мне.
\vs 3Ez 11:1 И видел я сон, и вот, поднялся с моря орел, у которого было двенадцать крыльев пернатых и три головы.
\vs 3Ez 11:2 И видел я: вот, он распростирал крылья свои над всею землею, и все ветры небесные дули на него и собирались облака.
\vs 3Ez 11:3 И видел я, что из перьев его выходили другие малые перья, и из тех выходили еще меньшие и короткие.
\vs 3Ez 11:4 Головы его покоились, и средняя голова была больше других голов, но также покоилась с ними.
\vs 3Ez 11:5 И видел я: вот орел летал на крыльях своих и царствовал над землею и над всеми обитателями ее.
\vs 3Ez 11:6 И видел я, что все поднебесное было покорно ему, и никто не сопротивлялся ему, ни одна из тварей, существующих на земле.
\vs 3Ez 11:7 И вот, орел стал на когти свои и испустил голос к перьям своим и сказал:
\vs 3Ez 11:8 не бодрствуйте все вместе; спите каждое на своем месте, и бодрствуйте поочередно,
\vs 3Ez 11:9 а головы пусть сохраняются на последнее время.
\vs 3Ez 11:10 Видел я, что голос его исходил не из голов его, но из средины тела его.
\vs 3Ez 11:11 Я сосчитал малые перья его; их было восемь.
\vs 3Ez 11:12 И вот, с правой стороны поднялось одно перо и воцарилось над всею землею.
\vs 3Ez 11:13 И когда воцарилось, пришел конец его, и не видно стало места его; потом поднялось другое перо и царствовало; это владычествовало долгое время.
\vs 3Ez 11:14 Когда оно царствовало и приблизился конец его, чтобы оно так же исчезло, как и первое,
\vs 3Ez 11:15 и вот, слышен был голос, говорящий ему:
\vs 3Ez 11:16 слушай ты, которое столько времени обладало землею! вот что я возвещаю тебе, прежде нежели начнешь исчезать:
\vs 3Ez 11:17 никто после тебя не будет владычествовать столько времени, как ты, и даже половины того.
\vs 3Ez 11:18 И поднялось третье перо, и владычествовало, как и прежние, но исчезло и оно.
\vs 3Ez 11:19 Так было и со всеми другими: они владычествовали и потом исчезали навсегда.
\vs 3Ez 11:20 Я видел, что по времени с правой стороны поднимались следующие перья, чтобы и им иметь начальство, и некоторые из них начальствовали, но тотчас исчезали;
\vs 3Ez 11:21 иные же из них поднимались, но не получали начальства.
\vs 3Ez 11:22 После сего не являлись более двенадцать перьев, ни два малых пера;
\vs 3Ez 11:23 и не осталось в теле орла ничего, кроме двух голов покоящихся и шести малых перьев.
\vs 3Ez 11:24 Я видел, и вот, из шести малых перьев отделились два и остались под головою, которая была с правой стороны, а четыре оставались на своем месте.
\vs 3Ez 11:25 Потом подкрыльные перья покушались подняться и начальствовать;
\vs 3Ez 11:26 и вот, одно поднялось, но тотчас исчезло;
\vs 3Ez 11:27 а следующие исчезали еще скорее, нежели прежние.
\vs 3Ez 11:28 И видел я: вот, два остававшиеся пера покушались также царствовать.
\vs 3Ez 11:29 Когда они покушались, одна из покоящихся голов, которая была средняя, пробудилась, и она была более других двух голов.
\vs 3Ez 11:30 И видел я, что две другие головы соединились с нею.
\vs 3Ez 11:31 И эта голова, обратившись с теми, которые были соединены с нею, пожрала два подкрыльных пера, которые покушались царствовать.
\vs 3Ez 11:32 Эта голова устрашила всю землю и владычествовала над обитателями земли с великим угнетением, и удерживала власть на земном шаре более всех крыльев, которые были.
\vs 3Ez 11:33 После того я видел, что и средняя голова внезапно исчезла, как и крылья;
\vs 3Ez 11:34 оставались две головы, которые подобным образом царствовали на земле и над ее обитателями.
\vs 3Ez 11:35 И вот, голова с правой стороны пожрала ту, которая была с левой.
\vs 3Ez 11:36 И слышал я голос, говорящий мне: смотри перед собою, и размышляй о том, что видишь.
\vs 3Ez 11:37 И видел я: вот, как бы лев, выбежавший из леса и рыкающий, испустил человеческий голос к орлу и сказал:
\vs 3Ez 11:38 слушай, что я буду говорить тебе и что скажет тебе Всевышний:
\vs 3Ez 11:39 не ты ли оставшийся из числа четырех животных, которых Я поставил царствовать в веке Моем, чтобы через них пришел конец времен тех?
\vs 3Ez 11:40 И четвертое из них пришло, победило всех прежде бывших животных и держало век в большом трепете и всю вселенную в лютом угнетении, и с тягостнейшим утеснением подвластных, и столь долгое время обитало на земле с коварством.
\vs 3Ez 11:41 Ты судил землю не по правде;
\vs 3Ez 11:42 ты утеснял кротких, обижал миролюбивых, любил лжецов, разорял жилища тех, которые приносили пользу, и разрушал стены тех, которые не делали тебе вреда.
\vs 3Ez 11:43 И взошла ко Всевышнему обида твоя, и гордыня твоя~--- к Крепкому.
\vs 3Ez 11:44 И воззрел Всевышний на времена гордыни, и вот, они кончились, и исполнилась мера злодейств ее.
\vs 3Ez 11:45 Поэтому исчезни ты, орел, с страшными крыльями твоими, с гнусными перьями твоими, со злыми головами твоими, с жестокими когтями твоими и со всем негодным телом твоим,
\vs 3Ez 11:46 чтобы отдохнула вся земля и освободилась от твоего насилия, и надеялась на суд и милосердие своего Создателя.
\vs 3Ez 12:1 Когда лев говорил к орлу эти слова, я увидел,
\vs 3Ez 12:2 что не являлась более голова, которая оставалась вместе с четырьмя крыльями, которые перешли к ней и поднимались, чтобы царствовать, но которых царство было слабо и исполнено возмущений.
\vs 3Ez 12:3 И я видел, и вот они исчезли, и все тело орла сгорало, и ужаснулась земля, и я от тревоги, исступления ума и от великого страха пробудился и сказал духу моему:
\vs 3Ez 12:4 вот, ты причинил мне это тем, что испытываешь пути Всевышнего.
\vs 3Ez 12:5 Вот, я еще трепещу сердцем и весьма изнемог духом моим, и нет во мне нисколько силы от великого страха, которым я поражен в эту ночь.
\vs 3Ez 12:6 Итак ныне я помолюсь Всевышнему, чтобы Он укрепил меня до конца.
\vs 3Ez 12:7 И сказал я: Владыко Господи! если я обрел благодать пред очами Твоими, если Ты нашел меня праведным пред многими, и если молитва моя подлинно взошла пред лице Твое,
\vs 3Ez 12:8 укрепи меня и покажи мне, рабу Твоему, значение сего страшного видения, чтобы вполне успокоить душу мою:
\vs 3Ez 12:9 ибо Ты судил меня достойным, чтобы показать мне последние времена. И Он сказал мне:
\vs 3Ez 12:10 Таково значение видения сего:
\vs 3Ez 12:11 орел, которого ты видел восходящим от моря, есть царство, показанное в видении Даниилу, брату твоему;
\vs 3Ez 12:12 но ему не было изъяснено то, что ныне Я изъясню тебе.
\vs 3Ez 12:13 Вот, приходят дни, когда восстанет на земле царство более страшное, нежели все царства, бывшие прежде него.
\vs 3Ez 12:14 В нем будут царствовать, один после другого, двенадцать царей.
\vs 3Ez 12:15 Второй из них начнет царствовать, и удержит власть более продолжительное время, нежели прочие двенадцать.
\vs 3Ez 12:16 Таково значение двенадцати крыльев, виденных тобою.
\vs 3Ez 12:17 А что ты слышал говоривший голос, исходящий не от голов орла, но из средины тела его,
\vs 3Ez 12:18 это означает, что после времени того царства произойдут немалые распри, и царство подвергнется опасности падения; но оно не падет тогда и восстановится в первоначальное состояние свое.
\vs 3Ez 12:19 А что ты видел восемь малых подкрыльных перьев, соединенных с крыльями, это означает,
\vs 3Ez 12:20 что восстанут в царстве восемь царей, которых времена будут легки и годы скоротечны, и два из них погибнут.
\vs 3Ez 12:21 Когда будет приближаться среднее время, четыре сохранятся до того времени, когда будет близок конец его; а два сохранятся до конца.
\vs 3Ez 12:22 А что ты видел три головы покоящиеся, это означает,
\vs 3Ez 12:23 что в последние дни царства Всевышний воздвигнет три царства и покорит им многие другие, и они будут владычествовать над землею и обитателями ее
\vs 3Ez 12:24 с б\acc{о}льшим утеснением, нежели все прежде бывшие; поэтому они и названы головами орла,
\vs 3Ez 12:25 ибо они-то довершат беззакония его и положат конец ему.
\vs 3Ez 12:26 А что ты видел, что большая голова не являлась более, это означает, что один из царей умрет на постели своей, впрочем с мучением,
\vs 3Ez 12:27 а двух остальных пожрет меч;
\vs 3Ez 12:28 меч одного пожрет того, который с ним, но и он в последствие времени умрет от меча.
\vs 3Ez 12:29 А что ты видел, два подкрыльных пера перешли на голову, находящуюся с правой стороны,
\vs 3Ez 12:30 это те, которых Всевышний сохранил к концу царства, то есть царство скудное и исполненное беспокойств.
\vs 3Ez 12:31 Лев, которого ты видел поднявшимся из леса и рыкающим, говорящим к орлу и обличающим его в неправдах его всеми словами его, которые ты слышал,
\vs 3Ez 12:32 это~--- Помазанник, сохраненный Всевышним к концу против них и нечестий их, Который обличит их и представит пред ними притеснения их.
\vs 3Ez 12:33 Он поставит их на суд живых и, обличив их, накажет их.
\vs 3Ez 12:34 Он по милосердию избавит остаток народа Моего, тех, которые сохранились в пределах Моих, и обрадует их, доколе не придет конец, день суда, о котором Я сказал тебе вначале.
\vs 3Ez 12:35 Таков сон, виденный тобою, и таково значение его.
\vs 3Ez 12:36 Ты один был достоин знать эту тайну Всевышнего.
\vs 3Ez 12:37 Все это, виденное тобою, напиши в книге и положи в сокровенном месте;
\vs 3Ez 12:38 и научи этому мудрых из народа твоего, которых сердц\acc{а} призн\acc{а}ешь способными принять и хранить сии тайны.
\vs 3Ez 12:39 А ты пребудь здесь еще семь дней, чтобы тебе показано было, что Всевышнему угодно будет показать тебе. И отошел от меня.
\rsbpar\vs 3Ez 12:40 Когда по истечении семи дней весь народ услышал, что я не возвратился в город, собрались все от малого до большого и, придя ко мне, говорили мне:
\vs 3Ez 12:41 чем согрешили мы против тебя? И чем обидели тебя, что ты, оставив нас, сидишь на этом месте?
\vs 3Ez 12:42 Ты один из всего народа остался нам, как гроздь от винограда, как светильник в темном месте и как пристань и корабль, спасенный от бури.
\vs 3Ez 12:43 Неужели мало бедствий, приключившихся нам?
\vs 3Ez 12:44 Если ты оставишь нас, то лучше было бы для нас сгореть, когда горел Сион.
\vs 3Ez 12:45 Ибо мы не лучше тех, которые умерли там. И плакали они с громким воплем. Отвечая им, я сказал:
\vs 3Ez 12:46 надейся, Израиль, и не скорби, дом Иакова;
\vs 3Ez 12:47 ибо помнит о вас Всевышний, и Крепкий не забыл вас в напасти.
\vs 3Ez 12:48 И я не оставил вас и не ушел от вас, но пришел на это место, чтобы помолиться о разоренном Сионе и просить милосердия уничиженной святыне вашей.
\vs 3Ez 12:49 Теперь идите каждый в дом свой, и я приду к вам после сих дней.
\vs 3Ez 12:50 И пошел народ, как я сказал ему, в город,
\vs 3Ez 12:51 а я оставался в поле в продолжение семи дней, как повелено мне, и питался в те дни только цветами полевыми, и трава была мне пищею.
\vs 3Ez 13:1 И было после семи дней, я видел ночью сон:
\vs 3Ez 13:2 вот, поднялся ветер с моря, чтобы возмутить все волны его.
\vs 3Ez 13:3 Я смотрел, и вот, вышел крепкий муж с воинством небесным, и куда он ни обращал лице свое, чтобы взглянуть, все трепетало, что виднелось под ним;
\vs 3Ez 13:4 и куда ни выходил голос из уст его, загорались все, которые слышали голос его, подобно тому, как тает воск, когда почувствует огонь.
\vs 3Ez 13:5 И после этого видел я: вот, собралось множество людей, которым не было числа, от четырех ветров небесных, чтобы преодолеть этого мужа, который поднялся с моря.
\vs 3Ez 13:6 Видел я, и вот, он изваял себе большую гору и взлетел на нее.
\vs 3Ez 13:7 Я старался увидеть ту страну или место, откуда изваяна была эта гора, но не мог.
\vs 3Ez 13:8 После сего видел я, что все, которые собрались победить его, очень испугались и однако же осмелились воевать.
\vs 3Ez 13:9 Он же, когда увидел устремление идущего множества, не поднял руки своей, ни копья не держал и никакого оружия воинского;
\vs 3Ez 13:10 но только, как я видел, он испускал из уст своих как бы дуновение огня и из губ своих~--- как бы дыхание пламени и с языка своего пускал искры и бури, и все это смешалось вместе: и дуновение огня и дыхание пламени и сильная буря.
\vs 3Ez 13:11 И стремительно напал он на это множество, которое приготовилось сразиться, и сжег всех, так что ничего не видно было из бесчисленного множества, кроме праха, и только был запах от дыма; увидел я это, и устрашился.
\vs 3Ez 13:12 После сего я видел того мужа сходящим с горы и призывающим к себе другое множество, мирное.
\vs 3Ez 13:13 И многие приступали к нему, иные с лицами веселыми, а иные с печальными, иные были связаны, иных приносили,~--- и я изнемог от великого страха, пробудился и сказал:
\vs 3Ez 13:14 Ты от начала показал рабу Твоему чудеса сии и судил меня достойным, чтобы принять молитву мою;
\vs 3Ez 13:15 покажи же мне и значение сна сего,
\vs 3Ez 13:16 потому что, как я понимаю разумом моим, горе тем, которые оставлены будут до тех дней, а еще более горе тем, которые не оставлены.
\vs 3Ez 13:17 Ибо те, которые не оставлены, были печальны.
\vs 3Ez 13:18 Теперь я понимаю, что то, что отложено на последние дни, встретит их, но и тех, которые оставлены.
\vs 3Ez 13:19 Поэтому они пришли в большие опасности и большие затруднения, как показывают эти сны.
\vs 3Ez 13:20 Но легче находящемуся в опасности потерпеть это, нежели перейти подобно облаку из мира сего и не видеть того, что будет в последние времена. Он отвечал мне и сказал:
\vs 3Ez 13:21 И значение видения Я скажу тебе, и о чем ты говорил, открою тебе.
\vs 3Ez 13:22 Так как ты говорил о тех, которые оставлены, то вот объяснение:
\vs 3Ez 13:23 кто выдержит опасность в то время, тот сохранил себя, а которые впадут в опасность, это те, которые имеют дела и веру во Всемогущего.
\vs 3Ez 13:24 Итак знай, что те, которые оставлены, блаженнее умерших.
\vs 3Ez 13:25 Вот объяснение видения: так как ты видел мужа, восходящего из средины моря,
\vs 3Ez 13:26 это тот, которого Всевышний хранит многие времена, который самим собою избавит творение свое и управит тех, которые оставлены.
\vs 3Ez 13:27 А что ты видел исходивший из уст его как бы ветер, огонь и бурю,
\vs 3Ez 13:28 и что он не держал ни копья и никакого воинского оружия, но устремление его поразило множество, которое пришло, чтобы победить его, то вот объяснение:
\vs 3Ez 13:29 вот, наступают дни, когда Всевышний начнет избавлять тех, которые на земле,
\vs 3Ez 13:30 и приведет в изумление живущих на земле.
\vs 3Ez 13:31 И будут предпринимать войны одни против других, город против города, одно место против другого, народ против народа, царство против царства.
\vs 3Ez 13:32 Когда это будет и явятся знамения, которые Я показал тебе прежде, тогда откроется Сын Мой, Которого ты видел, как мужа восходящего.
\vs 3Ez 13:33 И когда все народы услышат глас Его, каждый оставит войну в своей собственной стране, которую они имеют между собою.
\vs 3Ez 13:34 И соберется в одно собрание множество бесчисленное, как бы желая идти и победить Его.
\vs 3Ez 13:35 Он же станет на верху горы Сиона.
\vs 3Ez 13:36 И Сион придет и покажется всем приготовленный и устроенный, как ты видел гору, изваянную без рук.
\vs 3Ez 13:37 Сын же Мой обличит нечестия, изобретенные этими народами, которые своими злыми помышлениями приблизили бурю и мучения, которыми они начнут мучиться,
\vs 3Ez 13:38 и которые подобны огню; и Он истребит их без труда законом, который подобен огню.
\vs 3Ez 13:39 А что ты видел, что Он собирал к себе другое, мирное общество:
\vs 3Ez 13:40 это десять колен, которые отведены были пленными из земли своей во дни царя Осии, которого отвел в плен Салманассар, царь Ассирийский, и перевел их за реку, и переведены были в землю иную.
\vs 3Ez 13:41 Они же положили в совете своем, чтобы оставить множество язычников и отправиться в дальнюю страну, где никогда не обитал род человеческий,
\vs 3Ez 13:42 чтобы там соблюдать законы свои, которых они не соблюдали в стране своей.
\vs 3Ez 13:43 Тесными входами подошли они к реке Евфрату;
\vs 3Ez 13:44 ибо Всевышний сотворил тогда для них чудеса и остановил жилы реки, доколе они проходили;
\vs 3Ez 13:45 ибо через эту страну шли они долго, полтора года; эта страна называется Арсареф.
\vs 3Ez 13:46 Там жили они до последнего времени. И ныне, когда они начнут приходить,
\vs 3Ez 13:47 Всевышний снова остановит жилы реки, чтобы они могли пройти; поэтому ты видел множество мирное.
\vs 3Ez 13:48 Но которые оставлены от народа твоего, это те, которые находятся внутри пределов Моих.
\vs 3Ez 13:49 Ибо, когда начнет Он истреблять множество собравшихся вместе народов, Он защитит народ Свой, который останется.
\vs 3Ez 13:50 И тогда покажет им множество чудес.
\vs 3Ez 13:51 Я сказал: Владыко Господи! Объясни мне это, для чего видел я мужа, восходящего из средины моря?
\vs 3Ez 13:52 И Он сказал мне: как не можешь ты исследовать и познать того, что во глубине моря, так никто не может на земле видеть Сына Моего, ни тех, которые с Ним, разве только во время дня Его.
\vs 3Ez 13:53 Вот истолкование сна, который ты видел и которым ты один здесь просвещен.
\vs 3Ez 13:54 Ты оставил дела твои и упражнялся в законе Моем, и взыскал его,
\vs 3Ez 13:55 ибо жизнь твою ты устроил в мудрости и рассудительность назвал твоею матерью.
\vs 3Ez 13:56 Поэтому Я показал тебе воздаяния у Всевышнего; после трех дней Я покажу тебе другое и открою тебе важное и чудное.
\vs 3Ez 13:57 Тогда я пошел и вышел в поле, много славя и благодаря Всевышнего за чудеса, которые Он совершал по временам,
\vs 3Ez 13:58 и что Он управляет настоящим и тем, что произойдет во времена,~--- и там я сидел три дня.
\vs 3Ez 14:1 И было после трех дней, я сидел под дубом, и вот, голос вышел из куста против меня и сказал: Ездра, Ездра!
\vs 3Ez 14:2 Я сказал: вот я, Господи. И встал на ноги мои.
\vs 3Ez 14:3 Тогда сказал Он мне: в кусте Я открылся и говорил Моисею, когда народ Мой был рабом в Египте;
\vs 3Ez 14:4 и послал его и вывел народ Мой из Египта, и привел его к горе Синаю и держал его у Себя много дней,
\vs 3Ez 14:5 и открыл ему много чудес и показал тайны времен и конец, и заповедал ему, сказав:
\vs 3Ez 14:6 <<Эти слова объяви, а прочие скрой>>.
\vs 3Ez 14:7 И ныне тебе говорю:
\vs 3Ez 14:8 знамения, которые Я показал тебе, и сны, которые ты видел, и толкования, которые слышал, положи в сердце твоем;
\vs 3Ez 14:9 потому что ты взят будешь от людей и будешь обращаться с Сыном Моим и с подобными тебе, доколе не окончатся времена.
\vs 3Ez 14:10 Ибо век потерял свою юность, и времена приближаются к старости,
\vs 3Ez 14:11 так как век разделен на двенадцать частей, и девять частей его и половина десятой части уже прошли,
\vs 3Ez 14:12 и остается то, что после половины десятой части.
\vs 3Ez 14:13 Итак ныне устрой дом твой и вразуми народ твой, утешь уничиженных и отрекись тления,
\vs 3Ez 14:14 и отпусти от себя смертные помышления, отбрось тягости людские, сними с себя немощи естества и отложи в сторону тягостные для тебя помыслы, и готовься переселиться от времен сих.
\vs 3Ez 14:15 Ибо после больше будет бедствий, нежели сколько ты видел ныне.
\vs 3Ez 14:16 Сколько будет слабеть век от старости, столько будет умножаться зло для живущих.
\vs 3Ez 14:17 Еще дальше удалится истина, и приблизится ложь; уже поспешает прийти видение, которое ты видел.
\vs 3Ez 14:18 Тогда отвечал я и сказал: вот, я~--- пред Тобою, Господи;
\vs 3Ez 14:19 я пойду, как Ты повелел мне, и вразумлю нынешний народ; но кто научит тех, которые потом родятся?
\vs 3Ez 14:20 Ибо век во тьме лежит, и живущие в нем~--- без света;
\vs 3Ez 14:21 потому что закон Твой сожжен, и оттого никто не знает, что соделано Тобою или что должно им делать.
\vs 3Ez 14:22 Но если я приобрел милость у Тебя, ниспошли на меня Духа Святаго, чтобы я написал все, что было соделано в мире от начала, что было написано в законе Твоем, дабы люди могли найти стезю и дабы те, которые захотят жить в последние времена, могли жить.
\vs 3Ez 14:23 И Он в ответ сказал мне: иди, собери народ и скажи ему, чтобы он не искал тебя в продолжение сорока дней.
\vs 3Ez 14:24 Ты же приготовь себе побольше дощечек и возьми с собою Сария, Даврия, Салемия, Ехана и Асиеля, этих пять, способных писать скоро.
\vs 3Ez 14:25 И приди сюда, и Я возжгу в сердце твоем светильник разума, который не угаснет, доколе не окончится то, что ты начнешь писать.
\vs 3Ez 14:26 И когда ты совершишь это, то иное объяви, а иное тайно передай мудрым. Завтра в этот час ты начнешь писать.
\vs 3Ez 14:27 Тогда я пошел, как Он повелел мне, и собрал весь народ и сказал:
\vs 3Ez 14:28 слушай, Израиль, слова сии:
\vs 3Ez 14:29 отцы наши были странниками в Египте, и освобождены были оттуда,
\vs 3Ez 14:30 и приняли закон жизни, которого не сохранили, который и вы после них нарушили.
\vs 3Ez 14:31 И дана была вам земля в наследие и земля Сион; но отцы ваши и вы делали беззаконие и не держались тех путей, которые Всевышний заповедал вам.
\vs 3Ez 14:32 И Он, как праведный судия, отнял у вас ныне, что даровал вам.
\vs 3Ez 14:33 И ныне вы здесь и братья ваши между вами.
\vs 3Ez 14:34 Если вы будете управлять чувством вашим и образуете сердце ваше, то сохраните жизнь и по смерти пол\acc{у}чите милость.
\vs 3Ez 14:35 Ибо по смерти настанет суд, когда мы оживем; и тогда имена праведных будут объявлены и показаны дела нечестивых.
\vs 3Ez 14:36 Никто не приходи ко мне ныне и не ищи меня до сорока дней.
\vs 3Ez 14:37 И взял я пять мужей, как Он заповедал мне, и пошли мы в поле и остались там.
\vs 3Ez 14:38 И вот, на другой день голос воззвал ко мне: Ездра! открой уста твои и выпей то, чем Я напою тебя.
\vs 3Ez 14:39 Я открыл уста мои, и вот полная чаша подана была мне, которая была наполнена как бы водою, но цвет того был подобен огню.
\vs 3Ez 14:40 И взял я и пил; и когда я пил, сердце мое дышало разумом и в груди моей возрастала мудрость, ибо дух мой подкреплялся памятью;
\vs 3Ez 14:41 уста мои были открыты и больше не закрывались.
\vs 3Ez 14:42 Всевышний даровал разум пяти мужам, и они ночью писали по порядку, что было говорено им и чего они не знали.
\vs 3Ez 14:43 Ночью они ели хлеб; а я говорил днем и не молчал ночью.
\vs 3Ez 14:44 Написаны же были в сорок дней девяносто четыре книги.
\vs 3Ez 14:45 И когда исполнилось сорок дней,
\vs 3Ez 14:46 Всевышний сказал: первые, которые ты написал, положи открыто, чтобы могли читать и достойные и недостойные,
\vs 3Ez 14:47 но последние семьдесят сбереги, чтобы передать их мудрым из народа;
\vs 3Ez 14:48 потому что в них проводник разума, источник мудрости и река знания. Так я и сделал.
\vs 3Ez 15:1 Говори вслух народа Моего слова пророчества, которые вложу Я в уста твои, говорит Господь;
\vs 3Ez 15:2 и сделай, чтобы они написаны были на хартии, потому что они верны и истинны.
\vs 3Ez 15:3 Не бойся, что будут замышлять против тебя, и да не смущает тебя неверие тех, которые будут говорить против тебя,
\vs 3Ez 15:4 ибо всякий неверующий в неверии своем умрет.
\vs 3Ez 15:5 Вот, Я наведу, говорит Господь, на круг земной бедствия: меч и голод, и смерть и пагубу
\vs 3Ez 15:6 за то, что нечестие \bibemph{людей} осквернило всю землю, и пагубные дела их переполнились.
\vs 3Ez 15:7 Посему говорит Господь:
\vs 3Ez 15:8 Я уже не буду молчать о беззакониях, которые совершают они нечестиво, и не буду терпеть в них того, что они делают преступно: вот, кровь неповинная и праведная вопиет ко Мне, и души праведных вопиют непрестанно.
\vs 3Ez 15:9 Отмщу им, говорит Господь, и возьму от них к Себе всякую кровь неповинную.
\vs 3Ez 15:10 Вот, народ Мой ведется как стадо на заклание; не потерплю более, чтобы он жил в Египте,
\vs 3Ez 15:11 но выведу его рукою сильною и мышцею высокою, и поражу Египет казнью, как прежде, и погублю всю землю его.
\vs 3Ez 15:12 Восплачет Египет и основания его, пораженные казнью и мщением, которое наведет на него Бог.
\vs 3Ez 15:13 Восплачут земледельцы, возделывающие землю, потому что оскудеют у них семена от ржавчины и от града и от страшной звезды.
\vs 3Ez 15:14 Горе веку и тем, которые живут в нем,
\vs 3Ez 15:15 ибо приблизился меч и истребление их, и восстанет народ на народ для войны, и мечи в руках их.
\vs 3Ez 15:16 Люди сделаются непостоянными и, одни других одолевая, вознерадят о царе своем, и начальники~--- о ходе дел своих в пределах своей власти.
\vs 3Ez 15:17 Пожелает человек идти в город, и не возможет,
\vs 3Ez 15:18 ибо, по причине их гордости, города возмутятся, домы будут разорены, на людей нападет страх.
\vs 3Ez 15:19 Не сжалится человек над ближним своим, предавая домы их на разорение оружием, расхищая имущество их по причине голода и многих бед.
\vs 3Ez 15:20 Вот, Я созываю, говорит Бог, всех царей земли, от востока и юга, от севера и Ливана, чтобы благоговели предо Мною и обратились к себе самим, и чтобы воздать им, что они делали тем.
\vs 3Ez 15:21 Как поступают они даже доселе с избранными Моими, так поступлю \bibemph{с ними} и воздам в недро их, говорит Господь Бог.
\vs 3Ez 15:22 Не пощадит десница Моя грешников, и меч не перестанет поражать проливающих на землю неповинную кровь.
\vs 3Ez 15:23 Исшел огонь из гнева Его и истребил основания земли и грешников, как зажженную солому.
\vs 3Ez 15:24 Горе грешникам и не соблюдающим заповедей Моих! говорит Господь.
\vs 3Ez 15:25 Не пощажу их. Удалитесь, сыновья отступников, не оскверняйте святыни Моей.
\vs 3Ez 15:26 Господь знает всех, которые грешат против Него; потому предал их на смерть и на убиение.
\vs 3Ez 15:27 На круг земной пришли уже бедствия, и вы пребудете в них. Бог не избавит вас, потому что вы согрешили против Него.
\vs 3Ez 15:28 Вот, видение грозное, и лице его от востока.
\vs 3Ez 15:29 Выступят порождения драконов Аравийских на многих колесницах и с быстротою ветра понесутся по земле, так что наведут страх и трепет на всех, которые услышат о них.
\vs 3Ez 15:30 Выйдут, как вепри из леса, Кармоняне, неистовствующие в ярости, и придут в великой силе, вступят в борьбу с ними и опустошат часть земли Ассирийской.
\vs 3Ez 15:31 Потом драконы, помнящие происхождение свое, одержат верх и, обладая великою силою, обратятся преследовать тех.
\vs 3Ez 15:32 Те смутятся, умолкнут перед силою их и обратят ноги свои в бегство.
\vs 3Ez 15:33 Но находящийся в засаде со стороны Ассириян окружит их и умертвит одного из них; в войске их произойдет страх и трепет и ропот на царей их.
\vs 3Ez 15:34 Вот, облака от востока и от севера до юга, и вид их весьма грозен, исполнен свирепости и бури.
\vs 3Ez 15:35 Они столкнутся между собою, и свергнут много звезд на землю и звезду их; и будет кровь от меча до чрева,
\vs 3Ez 15:36 и помет человеческий~--- до седла верблюда; страх и трепет великий будет на земле.
\vs 3Ez 15:37 Ужаснутся \bibemph{все}, которые увидят эту свирепость, и вострепещут.
\vs 3Ez 15:38 После того много раз будут подниматься бури от юга и севера и частью от запада,
\vs 3Ez 15:39 и ветры сильные поднимутся от востока и откроют его и облако, которое Я подвигнул во гневе; а звезда, назначенная для устрашения при восточном и западном ветре, повредится.
\vs 3Ez 15:40 И поднимутся облака, великие и сильные, полные свирепости, и звезда, чтобы устрашить всю землю и жителей ее; и прольют на всякое место, высокое и возвышенное, страшную звезду,
\vs 3Ez 15:41 огонь и град, мечи летающие и многие воды, чтобы наполнить все поля и все источники множеством вод.
\vs 3Ez 15:42 И затопят город, и стены, и горы, и холмы, и дерева в лесах, и траву в лугах, и хлебные растения их;
\vs 3Ez 15:43 и пройдут безостановочно до Вавилона и сокрушат его;
\vs 3Ez 15:44 соберутся к нему и окружат его; прольют звезду и ярость на него. И поднимется пыль и дым до самого неба, и все кругом будут оплакивать его,
\vs 3Ez 15:45 а те, которые останутся подвластными ему, будут служить тем, которые навели страх.
\vs 3Ez 15:46 И ты, Асия, соучастница в надежде Вавилона и в славе его:
\vs 3Ez 15:47 горе тебе, бедная, за то, что уподоблялась ему и украшала дочерей твоих в блудодеянии, чтобы они нравились и славились у любовников твоих, которые желали всегда блудодействовать с тобою.
\vs 3Ez 15:48 Ты подражала ненавистному во всех делах и предприятиях его.
\vs 3Ez 15:49 За то, говорит Бог, пошлю на тебя бедствия: вдовство, нищету, и голод, и меч, и язву, чтобы опустошить домы твои насилием и смертью.
\vs 3Ez 15:50 И слава могущества твоего засохнет, как цвет, когда настанет зной, посланный на тебя.
\vs 3Ez 15:51 Ты изнеможешь, как нищая, избитая и израненная женщинами, чтобы люди знатные и любовники не могли принимать тебя.
\vs 3Ez 15:52 Стал ли бы Я так ненавидеть тебя, говорит Господь,
\vs 3Ez 15:53 если бы ты не убивала избранных Моих во всякое время, поднимая руки на поражение их и глумясь над смертью их, когда ты была в опьянении?
\vs 3Ez 15:54 Украшай твое лице.
\vs 3Ez 15:55 Мзда блудодеяния твоего в недре твоем; за то и получишь ты воздаяние.
\vs 3Ez 15:56 Как поступала ты с избранными Моими, говорит Господь, так с тобою поступит Бог, и подвергнет тебя бедствиям.
\vs 3Ez 15:57 Дети твои погибнут от голода, ты падешь от меча, города твои будут разрушены, и все твои падут в поле от меча.
\vs 3Ez 15:58 А которые на горах, те погибнут от голода, и будут есть плоть свою по недостатку хлеба и пить кровь по недостатку воды.
\vs 3Ez 15:59 В несчастии пойдешь по морям,~--- и там встретишь беды.
\vs 3Ez 15:60 Во время переходов твоих они бросятся на опустошенный город, и истребят часть земли твоей, и часть славы твоей уничтожат.
\vs 3Ez 15:61 Разоренная, ты послужишь для них соломою, а они для тебя будут огнем;
\vs 3Ez 15:62 и истребят тебя, и города твои, землю твою, горы твои, все леса твои и дерева плодоносные сожгут огнем.
\vs 3Ez 15:63 Сыновей твоих уведут в плен, имущество твое захватят в добычу, и славу твою истребят.
\vs 3Ez 16:1 Горе тебе, Вавилон и Асия, горе тебе, Египет и Сирия!
\vs 3Ez 16:2 Препояшьтесь вретищем и власяницами, оплакивайте сыновей ваших, и болезнуйте, потому что приблизилась ваша погибель.
\vs 3Ez 16:3 Послан на вас меч,~--- и кто отклонит его?
\vs 3Ez 16:4 Послан на вас огонь,~--- и кто угасит его?
\vs 3Ez 16:5 Посланы на вас бедствия,~--- и кто отвратит их?
\vs 3Ez 16:6 Прогонит ли кто голодного льва в лесу, или угасит ли мгновенно огонь в соломе, когда он начнет разгораться?
\vs 3Ez 16:7 Отразит ли кто стрелу, пущенную стрелком сильным?
\vs 3Ez 16:8 Господь сильный посылает бедствия,~--- и кто отвратит их?
\vs 3Ez 16:9 Исшел огонь от гнева Его,~--- и кто угасит его?
\vs 3Ez 16:10 Он блеснет молнией,~--- и кто не убоится? Возгремит,~--- и кто не ужаснется?
\vs 3Ez 16:11 Господь воззрит грозно,~--- и кто не сокрушится до основания от лица Его?
\vs 3Ez 16:12 Содрогнулась земля и основания ее; море волнуется со дна, и волны его возмущаются и рыбы его от лица Господа и от величия силы Его.
\vs 3Ez 16:13 Ибо сильна Его десница, напрягающая лук, остры Его стрелы, пускаемые Им, не ослабеют, когда будут посылаемы до концов земли.
\vs 3Ez 16:14 Вот, посылаются бедствия, и не возвратятся, доколе не придут на землю.
\vs 3Ez 16:15 Возгорается огонь, и не угаснет, доколе не попалит основания земли.
\vs 3Ez 16:16 Как стрела, пущенная сильным стрелком, не возвращается, так не возвратятся бедствия, которые будут посланы на землю.
\vs 3Ez 16:17 Горе мне, горе мне! Кто избавит меня в те дни?
\vs 3Ez 16:18 Начнутся болезни,~--- и многие восстенают; начнется голод,~--- и многие будут гибнуть; начнутся войны,~--- и начальствующими овладеет страх; начнутся бедствия,~--- и все вострепещут.
\vs 3Ez 16:19 Что мне делать тогда, когда придут бедствия?
\vs 3Ez 16:20 Вот, голод и язва, и скорбь и теснота посланы как бичи для исправления:
\vs 3Ez 16:21 но при всем этом \bibemph{люди} не обратятся от беззаконий своих и о бичах не всегда будут помнить.
\vs 3Ez 16:22 Вот, на земле будет дешевизна во всем, и подумают, что настал мир; но тогда-то и постигнут землю бедствия~--- меч, голод и великое смятение.
\vs 3Ez 16:23 От голода погибнут очень многие жители земли, а прочие, которые перенесут голод, падут от меча.
\vs 3Ez 16:24 И трупы, как навоз, будут выбрасываемы, и некому будет оплакивать их, ибо земля опустеет, и города ее будут разрушены.
\vs 3Ez 16:25 Не останется никого, кто возделывал бы землю и сеял на ней.
\vs 3Ez 16:26 Дерева дадут плоды, и кто будет собирать их?
\vs 3Ez 16:27 Виноград созреет, и кто будет топтать его? Ибо повсюду будет великое запустение.
\vs 3Ez 16:28 Трудно будет человеку увидеть человека, или услышать голос его,
\vs 3Ez 16:29 ибо из жителей города останется не более десяти, и из поселян~--- человека два, которые скроются в густых рощах и расселинах скал.
\vs 3Ez 16:30 Как в масличном саду остаются иногда на деревах три или четыре маслины,
\vs 3Ez 16:31 или в винограднике обобранном не досмотрят несколько гроздей те, которые внимательно обирают виноград:
\vs 3Ez 16:32 так в те дни останутся трое или четверо при обыске домов их с мечом.
\vs 3Ez 16:33 Земля останется в запустении, поля ее заглохнут, дороги ее и все тропинки ее зарастут терном, потому что некому будет ходить по ним.
\vs 3Ez 16:34 Плакать будут девицы, не имея женихов; плакать будут жены, не имея мужей; плакать будут дочери их, не имея помощи.
\vs 3Ez 16:35 Женихов их убьют на войне, и мужья их погибнут от голода.
\vs 3Ez 16:36 Слушайте это, и вразумляйтесь, рабы Господни!
\vs 3Ez 16:37 Это~--- слово Господа: внимайте ему, и не верьте богам, о которых говорит Господь.
\vs 3Ez 16:38 Вот, приближаются бедствия, и не замедлят.
\vs 3Ez 16:39 Как у беременной женщины, когда в девятый месяц настанет ей пора родить сына, часа за два или за три до рождения, боли охватывают чрево ее и, при выходе младенца из чрева, не замедлят ни на одну минуту:
\vs 3Ez 16:40 так не замедлят прийти на землю бедствия, и люди того времени восстенают; боли охватят их.
\vs 3Ez 16:41 Слушай слово, народ мой: готовьтесь на брань, и среди бедствий будьте как пришельцы земли.
\vs 3Ez 16:42 Продающий пусть будет, как собирающийся в бегство, и покупающий~--- как готовящийся на погибель;
\vs 3Ez 16:43 торгующий~--- как не ожидающий никакой прибыли, и строящий дом~--- как не надеющийся жить в нем.
\vs 3Ez 16:44 Сеятель пусть думает, что не пожнет, и виноградарь,~--- что не соберет винограда;
\vs 3Ez 16:45 вступающие в брак,~--- что не будут рождать детей, и не вступающие,~--- как вдовцы.
\vs 3Ez 16:46 Посему все трудящиеся без пользы трудятся,
\vs 3Ez 16:47 ибо плодами трудов их воспользуются чужеземцы, и имущество их расхитят, домы их разрушат и сыновей их поработят, потому что в плену и в голоде они рождают детей своих.
\vs 3Ez 16:48 Кто занимается хищничеством, тех, чем дольше украшают они города и домы свои, владения и лица свои,
\vs 3Ez 16:49 тем более возненавижу за грехи их, говорит Господь.
\vs 3Ez 16:50 Как блудница ненавидит женщину честную и весьма благонравную,
\vs 3Ez 16:51 так правда возненавидит неправду, украшающую себя, и обвинит ее в лице, когда придет Тот, Кто будет защищать преследующего всякий грех на земле.
\vs 3Ez 16:52 Потому не подражайте неправде и делам ее,
\vs 3Ez 16:53 ибо еще немного, и неправда будет удалена с земли, а правда воцарится над вами.
\vs 3Ez 16:54 Пусть не говорит грешник, что он не согрешил, потому что горящие угли возгорятся на голове того, кто говорит: я не согрешил пред Господом Богом и славою Его.
\vs 3Ez 16:55 Господь знает все дела людей и начинания их, и помышления их и сердца их.
\vs 3Ez 16:56 Он сказал: <<да будет земля>>,~--- и земля явилась; <<да будет небо>>,~--- и было.
\vs 3Ez 16:57 Словом Его сотворены звезды, и Он знает число звезд.
\vs 3Ez 16:58 Он созерцает бездны и сокровенное в них, измерил море и что в нем.
\vs 3Ez 16:59 Словом Своим Он заключил море среди вод и землю повесил на водах.
\vs 3Ez 16:60 Он простер небо, как шатер, на водах основал его.
\vs 3Ez 16:61 Он поместил в пустыне источники вод и озера на вершинах гор, для низведения рек с высоких скал, чтобы напоять землю.
\vs 3Ez 16:62 Он сотворил человека и положил сердце его в средине тела, и вложил в него дух, жизнь и разум
\vs 3Ez 16:63 и дыхание Бога всемогущего, Который сотворил все и созерцает все сокровенное в сокровенных земли.
\vs 3Ez 16:64 Он знает намерение ваше и что помышляете вы в сердцах ваших, когда грешите и хотите скрыть грехи ваши.
\vs 3Ez 16:65 Потому Господь совершенно ясно видит все дела ваши, и обличит всех вас;
\vs 3Ez 16:66 и вы будете посрамлены, когда грехи ваши откроются перед людьми, и беззакония предстанут обвинителями в тот день.
\vs 3Ez 16:67 Что вы сделаете и как скроете грехи ваши пред Богом и Ангелами Его?
\vs 3Ez 16:68 Вот, Бог~--- Судия; бойтесь Его; оставьте грехи ваши и навсегда перестаньте делать беззакония, и Бог изведет вас и избавит от всякой скорби.
\vs 3Ez 16:69 Ибо вот, возгорается на вас ярость многочисленного полчища, и схватят некоторых из вас и умертвят для принесения в жертву идолам.
\vs 3Ez 16:70 Кто будет единомыслен с ними, тех подвергнут они посмеянию, поношению и попранию.
\vs 3Ez 16:71 Ибо по всем местам и в соседних городах многие восстанут против боящихся Господа.
\vs 3Ez 16:72 Будут, как исступленные, без пощады расхищать и опустошать все у боящихся Господа.
\vs 3Ez 16:73 Опустошат и расхитят имущество их, и из домов их изгонят их.
\vs 3Ez 16:74 Тогда настанет испытание избранным Моим, как золото испытывается огнем.
\vs 3Ez 16:75 Слушайте, возлюбленные Мои, говорит Господь: вот перед вами дни скорби, и от них Я избавлю вас.
\vs 3Ez 16:76 Не бойтесь и не сомневайтесь, ибо вождь ваш~--- Бог.
\vs 3Ez 16:77 Если будете исполнять заповеди и повеления Мои, говорит Господь Бог, то грехи ваши не будут бременем, подавляющим вас, и беззакония ваши не превозмогут вас.
\vs 3Ez 16:78 Горе тем, которые связаны грехами своими и покрыты беззакониями своими! Это~--- поле, которое заросло кустарником и через которое путь покрыт терном, так что человек проходить не может: оно оставляется, и обрекается огню на истребление.

\bibpart{Книги Нового Завета}{Новый Завет}{NT}
\bibbookdescr{Mat}{
  inline={От Матфея\\\LARGE святое благовествование},
  toc={От Матфея},
  bookmark={От Матфея},
  header={От Матфея},
  %headerleft={},
  %headerright={},
  abbr={Мф}
}
\vs Mat 1:1 Родословие Иисуса Христа, Сына Давидова, Сына Авраамова.
\rsbpar\vs Mat 1:2 Авраам родил Исаака; Исаак родил Иакова; Иаков родил Иуду и братьев его;
\vs Mat 1:3 Иуда родил Фареса и Зару от Фамари; Фарес родил Есрома; Есром родил Арама;
\vs Mat 1:4 Арам родил Аминадава; Аминадав родил Наассона; Наассон родил Салмона;
\vs Mat 1:5 Салмон родил Вооза от Рахавы; Вооз родил Овида от Руфи; Овид родил Иессея;
\vs Mat 1:6 Иессей родил Давида царя; Давид царь родил Соломона от бывшей за Уриею;
\vs Mat 1:7 Соломон родил Ровоама; Ровоам родил Авию; Авия родил Асу;
\vs Mat 1:8 Аса родил Иосафата; Иосафат родил Иорама; Иорам родил Озию;
\vs Mat 1:9 Озия родил Иоафама; Иоафам родил Ахаза; Ахаз родил Езекию;
\vs Mat 1:10 Езекия родил Манассию; Манассия родил Амона; Амон родил Иосию;
\vs Mat 1:11 Иосия родил Иоакима; Иоаким родил Иехонию и братьев его, перед переселением в Вавилон.
\vs Mat 1:12 По переселении же в Вавилон, Иехония родил Салафииля; Салафииль родил Зоровавеля;
\vs Mat 1:13 Зоровавель родил Авиуда; Авиуд родил Елиакима; Елиаким родил Азора;
\vs Mat 1:14 Азор родил Садока; Садок родил Ахима; Ахим родил Елиуда;
\vs Mat 1:15 Елиуд родил Елеазара; Елеазар родил Матфана; Матфан родил Иакова;
\vs Mat 1:16 Иаков родил Иосифа, мужа Марии, от Которой родился Иисус, называемый Христос.
\vs Mat 1:17 Итак всех родов от Авраама до Давида четырнадцать родов; и от Давида до переселения в Вавилон четырнадцать родов; и от переселения в Вавилон до Христа четырнадцать родов.
\rsbpar\vs Mat 1:18 Рождество Иисуса Христа было так: по обручении Матери Его Марии с Иосифом, прежде нежели сочетались они, оказалось, что Она имеет во чреве от Духа Святаго.
\vs Mat 1:19 Иосиф же муж Ее, будучи праведен и не желая огласить Ее, хотел тайно отпустить Ее.
\vs Mat 1:20 Но когда он помыслил это,~--- се, Ангел Господень явился ему во сне и сказал: Иосиф, сын Давидов! не бойся принять Марию, жену твою, ибо родившееся в Ней есть от Духа Святаго;
\vs Mat 1:21 родит же Сына, и наречешь Ему имя Иисус, ибо Он спасет людей Своих от грехов их.
\vs Mat 1:22 А все сие произошло, да сбудется реченное Господом через пророка, который говорит:
\vs Mat 1:23 се, Дева во чреве приимет и родит Сына, и нарекут имя Ему Еммануил, что значит: с нами Бог.
\vs Mat 1:24 Встав от сна, Иосиф поступил, как повелел ему Ангел Господень, и принял жену свою,
\vs Mat 1:25 и не знал Ее, как наконец Она родила Сына Своего первенца, и он нарек Ему имя: Иисус.
\vs Mat 2:1 Когда же Иисус родился в Вифлееме Иудейском во дни царя Ирода, пришли в Иерусалим волхвы\fns{Мудрецы.} с востока и говорят:
\vs Mat 2:2 где родившийся Царь Иудейский? ибо мы видели звезду Его на востоке и пришли поклониться Ему.
\vs Mat 2:3 Услышав это, Ирод царь встревожился, и весь Иерусалим с ним.
\vs Mat 2:4 И, собрав всех первосвященников и книжников народных, спрашивал у них: где должно родиться Христу?
\vs Mat 2:5 Они же сказали ему: в Вифлееме Иудейском, ибо так написано через пророка:
\vs Mat 2:6 и ты, Вифлеем, земля Иудина, ничем не меньше воеводств Иудиных, ибо из тебя произойдет Вождь, Который упасет народ Мой, Израиля.
\vs Mat 2:7 Тогда Ирод, тайно призвав волхвов, выведал от них время появления звезды
\vs Mat 2:8 и, послав их в Вифлеем, сказал: пойдите, тщательно разведайте о Младенце и, когда найдете, известите меня, чтобы и мне пойти поклониться Ему.
\vs Mat 2:9 Они, выслушав царя, пошли. И се, звезда, которую видели они на востоке, шла перед ними, \bibemph{как} наконец пришла и остановилась над \bibemph{местом}, где был Младенец.
\vs Mat 2:10 Увидев же звезду, они возрадовались радостью весьма великою,
\vs Mat 2:11 и, войдя в дом, увидели Младенца с Мариею, Матерью Его, и, пав, поклонились Ему; и, открыв сокровища свои, принесли Ему дары: золото, ладан и смирну.
\vs Mat 2:12 И, получив во сне откровение не возвращаться к Ироду, иным путем отошли в страну свою.
\rsbpar\vs Mat 2:13 Когда же они отошли,~--- се, Ангел Господень является во сне Иосифу и говорит: встань, возьми Младенца и Матерь Его и беги в Египет, и будь там, доколе не скажу тебе, ибо Ирод хочет искать Младенца, чтобы погубить Его.
\vs Mat 2:14 Он встал, взял Младенца и Матерь Его ночью и пошел в Египет,
\vs Mat 2:15 и там был до смерти Ирода, да сбудется реченное Господом через пророка, который говорит: из Египта воззвал Я Сына Моего.
\rsbpar\vs Mat 2:16 Тогда Ирод, увидев себя осмеянным волхвами, весьма разгневался, и послал избить всех младенцев в Вифлееме и во всех пределах его, от двух лет и ниже, по времени, которое выведал от волхвов.
\vs Mat 2:17 Тогда сбылось реченное через пророка Иеремию, который говорит:
\vs Mat 2:18 глас в Раме слышен, плач и рыдание и вопль великий; Рахиль плачет о детях своих и не хочет утешиться, ибо их нет.
\rsbpar\vs Mat 2:19 По смерти же Ирода,~--- се, Ангел Господень во сне является Иосифу в Египте
\vs Mat 2:20 и говорит: встань, возьми Младенца и Матерь Его и иди в землю Израилеву, ибо умерли искавшие души Младенца.
\vs Mat 2:21 Он встал, взял Младенца и Матерь Его и пришел в землю Израилеву.
\vs Mat 2:22 Услышав же, что Архелай царствует в Иудее вместо Ирода, отца своего, убоялся туда идти; но, получив во сне откровение, пошел в пределы Галилейские
\vs Mat 2:23 и, придя, поселился в городе, называемом Назарет, да сбудется реченное через пророков, что Он Назореем наречется.
\vs Mat 3:1 В те дни приходит Иоанн Креститель и проповедует в пустыне Иудейской
\vs Mat 3:2 и говорит: покайтесь, ибо приблизилось Царство Небесное.
\vs Mat 3:3 Ибо он тот, о котором сказал пророк Исаия: глас вопиющего в пустыне: приготовьте путь Господу, прямыми сделайте стези Ему.
\vs Mat 3:4 Сам же Иоанн имел одежду из верблюжьего волоса и пояс кожаный на чреслах своих, а пищею его были акриды и дикий мед.
\vs Mat 3:5 Тогда Иерусалим и вся Иудея и вся окрестность Иорданская выходили к нему
\vs Mat 3:6 и крестились от него в Иордане, исповедуя грехи свои.
\vs Mat 3:7 Увидев же Иоанн многих фарисеев и саддукеев, идущих к нему креститься, сказал им: порождения ехиднины! кто внушил вам бежать от будущего гнева?
\vs Mat 3:8 сотворите же достойный плод покаяния
\vs Mat 3:9 и не думайте говорить в себе: <<отец у нас Авраам>>, ибо говорю вам, что Бог может из камней сих воздвигнуть детей Аврааму.
\vs Mat 3:10 Уже и секира при корне дерев лежит: всякое дерево, не приносящее доброго плода, срубают и бросают в огонь.
\vs Mat 3:11 Я крещу вас в воде в покаяние, но Идущий за мною сильнее меня; я не достоин понести обувь Его; Он будет крестить вас Духом Святым и огнем;
\vs Mat 3:12 лопата\fns{Которою веют хлеб.} Его в руке Его, и Он очистит гумно Свое и соберет пшеницу Свою в житницу, а солому сожжет огнем неугасимым.
\rsbpar\vs Mat 3:13 Тогда приходит Иисус из Галилеи на Иордан к Иоанну креститься от него.
\vs Mat 3:14 Иоанн же удерживал Его и говорил: мне надобно креститься от Тебя, и Ты ли приходишь ко мне?
\vs Mat 3:15 Но Иисус сказал ему в ответ: оставь теперь, ибо так надлежит нам исполнить всякую правду. Тогда \bibemph{Иоанн} допускает Его.
\vs Mat 3:16 И, крестившись, Иисус тотчас вышел из воды,~--- и се, отверзлись Ему небеса, и увидел \bibemph{Иоанн} Духа Божия, Который сходил, как голубь, и ниспускался на Него.
\vs Mat 3:17 И се, глас с небес глаголющий: Сей есть Сын Мой возлюбленный, в Котором Мое благоволение.
\vs Mat 4:1 Тогда Иисус возведен был Духом в пустыню, для искушения от диавола,
\vs Mat 4:2 и, постившись сорок дней и сорок ночей, напоследок взалкал.
\vs Mat 4:3 И приступил к Нему искуситель и сказал: если Ты Сын Божий, скажи, чтобы камни сии сделались хлебами.
\vs Mat 4:4 Он же сказал ему в ответ: написано: не хлебом одним будет жить человек, но всяким словом, исходящим из уст Божиих.
\vs Mat 4:5 Потом берет Его диавол в святой город и поставляет Его на крыле храма,
\vs Mat 4:6 и говорит Ему: если Ты Сын Божий, бросься вниз, ибо написано: Ангелам Своим заповедает о Тебе, и на руках понесут Тебя, да не преткнешься о камень ногою Твоею.
\vs Mat 4:7 Иисус сказал ему: написано также: не искушай Господа Бога твоего.
\vs Mat 4:8 Опять берет Его диавол на весьма высокую гору и показывает Ему все царства мира и славу их,
\vs Mat 4:9 и говорит Ему: всё это дам Тебе, если, пав, поклонишься мне.
\vs Mat 4:10 Тогда Иисус говорит ему: отойди от Меня, сатана, ибо написано: Господу Богу твоему поклоняйся и Ему одному служи.
\vs Mat 4:11 Тогда оставляет Его диавол, и се, Ангелы приступили и служили Ему.
\rsbpar\vs Mat 4:12 Услышав же Иисус, что Иоанн отдан \bibemph{под стражу}, удалился в Галилею
\vs Mat 4:13 и, оставив Назарет, пришел и поселился в Капернауме приморском, в пределах Завулоновых и Неффалимовых,
\vs Mat 4:14 да сбудется реченное через пророка Исаию, который говорит:
\vs Mat 4:15 земля Завулонова и земля Неффалимова, на пути приморском, за Иорданом, Галилея языческая,
\vs Mat 4:16 народ, сидящий во тьме, увидел свет великий, и сидящим в стране и тени смертной воссиял свет.
\vs Mat 4:17 С того времени Иисус начал проповедовать и говорить: покайтесь, ибо приблизилось Царство Небесное.
\rsbpar\vs Mat 4:18 Проходя же близ моря Галилейского, Он увидел двух братьев: Симона, называемого Петром, и Андрея, брата его, закидывающих сети в море, ибо они были рыболовы,
\vs Mat 4:19 и говорит им: идите за Мною, и Я сделаю вас ловцами человеков.
\vs Mat 4:20 И они тотчас, оставив сети, последовали за Ним.
\vs Mat 4:21 Оттуда, идя далее, увидел Он других двух братьев, Иакова Зеведеева и Иоанна, брата его, в лодке с Зеведеем, отцом их, починивающих сети свои, и призвал их.
\vs Mat 4:22 И они тотчас, оставив лодку и отца своего, последовали за Ним.
\rsbpar\vs Mat 4:23 И ходил Иисус по всей Галилее, уча в синагогах их и проповедуя Евангелие Царствия, и исцеляя всякую болезнь и всякую немощь в людях.
\vs Mat 4:24 И прошел о Нем слух по всей Сирии; и приводили к Нему всех немощных, одержимых различными болезнями и припадками, и бесноватых, и лунатиков, и расслабленных, и Он исцелял их.
\vs Mat 4:25 И следовало за Ним множество народа из Галилеи и Десятиградия, и Иерусалима, и Иудеи, и из-за Иордана.
\vs Mat 5:1 Увидев народ, Он взошел на гору; и, когда сел, приступили к Нему ученики Его.
\vs Mat 5:2 И Он, отверзши уста Свои, учил их, говоря:
\rsbpar\vs Mat 5:3 Блаженны нищие духом, ибо их есть Царство Небесное.
\rsbpar\vs Mat 5:4 Блаженны плачущие, ибо они утешатся.
\rsbpar\vs Mat 5:5 Блаженны кроткие, ибо они наследуют землю.
\rsbpar\vs Mat 5:6 Блаженны алчущие и жаждущие правды, ибо они насытятся.
\rsbpar\vs Mat 5:7 Блаженны милостивые, ибо они помилованы будут.
\rsbpar\vs Mat 5:8 Блаженны чистые сердцем, ибо они Бога узрят.
\rsbpar\vs Mat 5:9 Блаженны миротворцы, ибо они будут наречены сынами Божиими.
\rsbpar\vs Mat 5:10 Блаженны изгнанные за правду, ибо их есть Царство Небесное.
\rsbpar\vs Mat 5:11 Блаженны вы, когда будут поносить вас и гнать и всячески неправедно злословить за Меня.
\vs Mat 5:12 Радуйтесь и веселитесь, ибо велика ваша награда на небесах: так гнали \bibemph{и} пророков, бывших прежде вас.
\rsbpar\vs Mat 5:13 Вы~--- соль земли. Если же соль потеряет силу, то чем сделаешь ее соленою? Она уже ни к чему негодна, как разве выбросить ее вон на попрание людям.
\rsbpar\vs Mat 5:14 Вы~--- свет мира. Не может укрыться город, стоящий на верху горы.
\vs Mat 5:15 И, зажегши свечу, не ставят ее под сосудом, но на подсвечнике, и светит всем в доме.
\vs Mat 5:16 Так да светит свет ваш пред людьми, чтобы они видели ваши добрые дела и прославляли Отца вашего Небесного.
\rsbpar\vs Mat 5:17 Не думайте, что Я пришел нарушить закон или пророков: не нарушить пришел Я, но исполнить.
\vs Mat 5:18 Ибо истинно говорю вам: доколе не прейдет небо и земля, ни одна иота или ни одна черта не прейдет из закона, пока не исполнится все.
\vs Mat 5:19 Итак, кто нарушит одну из заповедей сих малейших и научит так людей, тот малейшим наречется в Царстве Небесном; а кто сотворит и научит, тот великим наречется в Царстве Небесном.
\vs Mat 5:20 Ибо, говорю вам, если праведность ваша не превзойдет праведности книжников и фарисеев, то вы не войдете в Царство Небесное.
\rsbpar\vs Mat 5:21 Вы слышали, что сказано древним: не убивай, кто же убьет, подлежит суду.
\vs Mat 5:22 А Я говорю вам, что всякий, гневающийся на брата своего напрасно, подлежит суду; кто же скажет брату своему: <<рак\acc{а}>>\fns{Пустой человек.}, подлежит синедриону\fns{Верховное судилище.}; а кто скажет: <<безумный>>, подлежит геенне огненной.
\vs Mat 5:23 Итак, если ты принесешь дар твой к жертвеннику и там вспомнишь, что брат твой имеет что-нибудь против тебя,
\vs Mat 5:24 оставь там дар твой пред жертвенником, и пойди прежде примирись с братом твоим, и тогда приди и принеси дар твой.
\vs Mat 5:25 Мирись с соперником твоим скорее, пока ты еще на пути с ним, чтобы соперник не отдал тебя судье, а судья не отдал бы тебя слуге, и не ввергли бы тебя в темницу;
\vs Mat 5:26 истинно говорю тебе: ты не выйдешь оттуда, пока не отдашь до последнего кодранта.
\rsbpar\vs Mat 5:27 Вы слышали, что сказано древним: не прелюбодействуй.
\vs Mat 5:28 А Я говорю вам, что всякий, кто смотрит на женщину с вожделением, уже прелюбодействовал с нею в сердце своем.
\vs Mat 5:29 Если же правый глаз твой соблазняет тебя, вырви его и брось от себя, ибо лучше для тебя, чтобы погиб один из членов твоих, а не все тело твое было ввержено в геенну.
\vs Mat 5:30 И если правая твоя рука соблазняет тебя, отсеки ее и брось от себя, ибо лучше для тебя, чтобы погиб один из членов твоих, а не все тело твое было ввержено в геенну.
\rsbpar\vs Mat 5:31 Сказано также, что если кто разведется с женою своею, пусть даст ей разводную.
\vs Mat 5:32 А Я говорю вам: кто разводится с женою своею, кроме вины любодеяния, тот подает ей повод прелюбодействовать; и кто женится на разведенной, тот прелюбодействует.
\rsbpar\vs Mat 5:33 Еще слышали вы, что сказано древним: не преступай клятвы, но исполняй пред Господом клятвы твои.
\vs Mat 5:34 А Я говорю вам: не клянись вовсе: ни небом, потому что оно престол Божий;
\vs Mat 5:35 ни землею, потому что она подножие ног Его; ни Иерусалимом, потому что он город великого Царя;
\vs Mat 5:36 ни головою твоею не клянись, потому что не можешь ни одного волоса сделать белым или черным.
\vs Mat 5:37 Но да будет слово ваше: да, да; нет, нет; а что сверх этого, то от лукавого.
\rsbpar\vs Mat 5:38 Вы слышали, что сказано: око за око и зуб за зуб.
\vs Mat 5:39 А Я говорю вам: не противься злому. Но кто ударит тебя в правую щеку твою, обрати к нему и другую;
\vs Mat 5:40 и кто захочет судиться с тобою и взять у тебя рубашку, отдай ему и верхнюю одежду;
\vs Mat 5:41 и кто принудит тебя идти с ним одно поприще, иди с ним два.
\vs Mat 5:42 Просящему у тебя дай, и от хотящего занять у тебя не отвращайся.
\rsbpar\vs Mat 5:43 Вы слышали, что сказано: люби ближнего твоего и ненавидь врага твоего.
\vs Mat 5:44 А Я говорю вам: люб\acc{и}те врагов ваших, благословляйте проклинающих вас, благотворите ненавидящим вас и молитесь за обижающих вас и гонящих вас,
\vs Mat 5:45 да будете сынами Отца вашего Небесного, ибо Он повелевает солнцу Своему восходить над злыми и добрыми и посылает дождь на праведных и неправедных.
\vs Mat 5:46 Ибо если вы будете любить любящих вас, какая вам награда? Не то же ли делают и мытари\fns{Сборщики податей.}?
\vs Mat 5:47 И если вы приветствуете только братьев ваших, что особенного делаете? Не так же ли поступают и язычники?
\rsbpar\vs Mat 5:48 Итак будьте совершенны, как совершен Отец ваш Небесный.
\vs Mat 6:1 Смотрите, не творите милостыни вашей пред людьми с тем, чтобы они видели вас: иначе не будет вам награды от Отца вашего Небесного.
\vs Mat 6:2 Итак, когда творишь милостыню, не труби перед собою, как делают лицемеры в синагогах и на улицах, чтобы прославляли их люди. Истинно говорю вам: они уже получают награду свою.
\vs Mat 6:3 У тебя же, когда творишь милостыню, пусть левая рука твоя не знает, что делает правая,
\vs Mat 6:4 чтобы милостыня твоя была втайне; и Отец твой, видящий тайное, воздаст тебе явно.
\rsbpar\vs Mat 6:5 И, когда молишься, не будь, как лицемеры, которые любят в синагогах и на углах улиц, останавливаясь, молиться, чтобы показаться перед людьми. Истинно говорю вам, что они уже получают награду свою.
\vs Mat 6:6 Ты же, когда молишься, войди в комнату твою и, затворив дверь твою, помолись Отцу твоему, Который втайне; и Отец твой, видящий тайное, воздаст тебе явно.
\vs Mat 6:7 А молясь, не говорите лишнего, как язычники, ибо они думают, что в многословии своем будут услышаны;
\vs Mat 6:8 не уподобляйтесь им, ибо знает Отец ваш, в чем вы имеете нужду, прежде вашего прошения у Него.
\vs Mat 6:9 Молитесь же так:\rsbpar Отче наш, сущий на небесах! да святится имя Твое;
\vs Mat 6:10 да приидет Царствие Твое; да будет воля Твоя и на земле, как на небе;
\vs Mat 6:11 хлеб наш насущный дай нам на сей день;
\vs Mat 6:12 и прости нам долги наши, как и мы прощаем должникам нашим;
\vs Mat 6:13 и не введи нас в искушение, но избавь нас от лукавого. Ибо Твое есть Царство и сила и слава во веки. Аминь.
\rsbpar\vs Mat 6:14 Ибо если вы будете прощать людям согрешения их, то простит и вам Отец ваш Небесный,
\vs Mat 6:15 а если не будете прощать людям согрешения их, то и Отец ваш не простит вам согрешений ваших.
\rsbpar\vs Mat 6:16 Также, когда поститесь, не будьте унылы, как лицемеры, ибо они принимают на себя мрачные лица, чтобы показаться людям постящимися. Истинно говорю вам, что они уже получают награду свою.
\vs Mat 6:17 А ты, когда постишься, помажь голову твою и умой лице твое,
\vs Mat 6:18 чтобы явиться постящимся не пред людьми, но пред Отцом твоим, Который втайне; и Отец твой, видящий тайное, воздаст тебе явно.
\rsbpar\vs Mat 6:19 Не собирайте себе сокровищ на земле, где моль и ржа истребляют и где воры подкапывают и крадут,
\vs Mat 6:20 но собирайте себе сокровища на небе, где ни моль, ни ржа не истребляют и где воры не подкапывают и не крадут,
\vs Mat 6:21 ибо где сокровище ваше, там будет и сердце ваше.
\rsbpar\vs Mat 6:22 Светильник для тела есть око. Итак, если око твое будет чисто, то всё тело твое будет светло;
\vs Mat 6:23 если же око твое будет худо, то всё тело твое будет темно. Итак, если свет, который в тебе, тьма, то какова же тьма?
\rsbpar\vs Mat 6:24 Никто не может служить двум господам: ибо или одного будет ненавидеть, а другого любить; или одному станет усердствовать, а о другом нерадеть. Не можете служить Богу и маммоне\fns{Богатству.}.
\vs Mat 6:25 Посему говорю вам: не заботьтесь для души вашей, что вам есть и что пить, ни для тела вашего, во что одеться. Душа не больше ли пищи, и тело одежды?
\vs Mat 6:26 Взгляните на птиц небесных: они ни сеют, ни жнут, ни собирают в житницы; и Отец ваш Небесный питает их. Вы не гораздо ли лучше их?
\vs Mat 6:27 Да и кто из вас, заботясь, может прибавить себе росту \bibemph{хотя} на один локоть?
\vs Mat 6:28 И об одежде что заботитесь? Посмотрите на полевые лилии, как они растут: ни трудятся, ни прядут;
\vs Mat 6:29 но говорю вам, что и Соломон во всей славе своей не одевался так, к\acc{а}к всякая из них;
\vs Mat 6:30 если же траву полевую, которая сегодня есть, а завтра будет брошена в печь, Бог так одевает, кольми паче вас, маловеры!
\vs Mat 6:31 Итак не заботьтесь и не говорите: что нам есть? или что пить? или во что одеться?
\vs Mat 6:32 потому что всего этого ищут язычники, и потому что Отец ваш Небесный знает, что вы имеете нужду во всем этом.
\vs Mat 6:33 Ищите же прежде Царства Божия и правды Его, и это все приложится вам.
\vs Mat 6:34 Итак не заботьтесь о завтрашнем дне, ибо завтрашний \bibemph{сам} будет заботиться о своем: довольно для \bibemph{каждого} дня своей заботы.
\vs Mat 7:1 Не суд\acc{и}те, да не судимы будете,
\vs Mat 7:2 ибо каким судом с\acc{у}дите, \bibemph{таким} будете судимы; и какою мерою мерите, \bibemph{такою} и вам будут мерить.
\vs Mat 7:3 И что ты смотришь на сучок в глазе брата твоего, а бревна в твоем глазе не чувствуешь?
\vs Mat 7:4 Или как скажешь брату твоему: <<дай, я выну сучок из глаза твоего>>, а вот, в твоем глазе бревно?
\vs Mat 7:5 Лицемер! вынь прежде бревно из твоего глаза и тогда увидишь, \bibemph{как} вынуть сучок из глаза брата твоего.
\rsbpar\vs Mat 7:6 Не давайте святыни псам и не бросайте жемчуга вашего перед свиньями, чтобы они не попрали его ногами своими и, обратившись, не растерзали вас.
\rsbpar\vs Mat 7:7 Прос\acc{и}те, и дано будет вам; ищите, и найдете; стучите, и отворят вам;
\vs Mat 7:8 ибо всякий просящий получает, и ищущий находит, и стучащему отворят.
\vs Mat 7:9 Есть ли между вами такой человек, который, когда сын его попросит у него хлеба, подал бы ему камень?
\vs Mat 7:10 и когда попросит рыбы, подал бы ему змею?
\vs Mat 7:11 Итак если вы, будучи злы, умеете даяния благие давать детям вашим, тем более Отец ваш Небесный даст блага просящим у Него.
\rsbpar\vs Mat 7:12 Итак во всем, как хотите, чтобы с вами поступали люди, т\acc{а}к поступайте и вы с ними, ибо в этом закон и пророки.
\rsbpar\vs Mat 7:13 Входите тесными вратами, потому что широк\acc{и} врата и пространен путь, ведущие в погибель, и многие идут ими;
\vs Mat 7:14 потому что тесн\acc{ы} врата и узок путь, ведущие в жизнь, и немногие находят их.
\rsbpar\vs Mat 7:15 Берегитесь лжепророков, которые приходят к вам в овечьей одежде, а внутри суть волки хищные.
\vs Mat 7:16 По плодам их узн\acc{а}ете их. Собирают ли с терновника виноград, или с репейника смоквы?
\vs Mat 7:17 Т\acc{а}к всякое дерево доброе приносит и плоды добрые, а худое дерево приносит и плоды худые.
\vs Mat 7:18 Не может дерево доброе приносить плоды худые, ни дерево худое приносить плоды добрые.
\vs Mat 7:19 Всякое дерево, не приносящее плода доброго, срубают и бросают в огонь.
\vs Mat 7:20 Итак по плодам их узн\acc{а}ете их.
\vs Mat 7:21 Не всякий, говорящий Мне: <<Господи! Господи!>>, войдет в Царство Небесное, но исполняющий волю Отца Моего Небесного.
\vs Mat 7:22 Многие скажут Мне в тот день: Господи! Господи! не от Твоего ли имени мы пророчествовали? и не Твоим ли именем бесов изгоняли? и не Твоим ли именем многие чудеса творили?
\vs Mat 7:23 И тогда объявлю им: Я никогда не знал вас; отойдите от Меня, делающие беззаконие.
\rsbpar\vs Mat 7:24 Итак всякого, кто слушает слова Мои сии и исполняет их, уподоблю мужу благоразумному, который построил дом свой на камне;
\vs Mat 7:25 и пошел дождь, и разлились реки, и подули ветры, и устремились на дом тот, и он не упал, потому что основан был на камне.
\vs Mat 7:26 А всякий, кто слушает сии слова Мои и не исполняет их, уподобится человеку безрассудному, который построил дом свой на песке;
\vs Mat 7:27 и пошел дождь, и разлились реки, и подули ветры, и налегли на дом тот; и он упал, и было падение его великое.
\rsbpar\vs Mat 7:28 И когда Иисус окончил слова сии, народ дивился учению Его,
\vs Mat 7:29 ибо Он учил их, как власть имеющий, а не как книжники и фарисеи.
\vs Mat 8:1 Когда же сошел Он с горы, за Ним последовало множество народа.
\vs Mat 8:2 И вот подошел прокаженный и, кланяясь Ему, сказал: Господи! если хочешь, можешь меня очистить.
\vs Mat 8:3 Иисус, простерши руку, коснулся его и сказал: хочу, очистись. И он тотчас очистился от проказы.
\vs Mat 8:4 И говорит ему Иисус: смотри, никому не сказывай, но пойди, покажи себя священнику и принеси дар, какой повелел Моисей, во свидетельство им.
\rsbpar\vs Mat 8:5 Когда же вошел Иисус в Капернаум, к Нему подошел сотник и просил Его:
\vs Mat 8:6 Господи! слуга мой лежит дома в расслаблении и жестоко страдает.
\vs Mat 8:7 Иисус говорит ему: Я приду и исцелю его.
\vs Mat 8:8 Сотник же, отвечая, сказал: Господи! я недостоин, чтобы Ты вошел под кров мой, но скажи только слово, и выздоровеет слуга мой;
\vs Mat 8:9 ибо я и подвластный человек, но, имея у себя в подчинении воинов, говорю одному: пойди, и идет; и другому: приди, и приходит; и слуге моему: сделай то, и делает.
\vs Mat 8:10 Услышав сие, Иисус удивился и сказал идущим за Ним: истинно говорю вам, и в Израиле не нашел Я такой веры.
\vs Mat 8:11 Говорю же вам, что многие придут с востока и запада и возлягут с Авраамом, Исааком и Иаковом в Царстве Небесном;
\vs Mat 8:12 а сыны царства извержены будут во тьму внешнюю: там будет плач и скрежет зубов.
\vs Mat 8:13 И сказал Иисус сотнику: иди, и, как ты веровал, да будет тебе. И выздоровел слуга его в тот час.
\rsbpar\vs Mat 8:14 Придя в дом Петров, Иисус увидел тещу его, лежащую в горячке,
\vs Mat 8:15 и коснулся руки ее, и горячка оставила ее; и она встала и служила им.
\rsbpar\vs Mat 8:16 Когда же настал вечер, к Нему привели многих бесноватых, и Он изгнал духов словом и исцелил всех больных,
\vs Mat 8:17 да сбудется реченное через пророка Исаию, который говорит: Он взял на Себя наши немощи и понес болезни.
\rsbpar\vs Mat 8:18 Увидев же Иисус вокруг Себя множество народа, велел \bibemph{ученикам} отплыть на другую сторону.
\vs Mat 8:19 Тогда один книжник, подойдя, сказал Ему: Учитель! я пойду за Тобою, куда бы Ты ни пошел.
\vs Mat 8:20 И говорит ему Иисус: лисицы имеют норы и птицы небесные~--- гнезда, а Сын Человеческий не имеет, где приклонить голову.
\vs Mat 8:21 Другой же из учеников Его сказал Ему: Господи! позволь мне прежде пойти и похоронить отца моего.
\vs Mat 8:22 Но Иисус сказал ему: иди за Мною, и предоставь мертвым погребать своих мертвецов.
\rsbpar\vs Mat 8:23 И когда вошел Он в лодку, за Ним последовали ученики Его.
\vs Mat 8:24 И вот, сделалось великое волнение на море, так что лодка покрывалась волнами; а Он спал.
\vs Mat 8:25 Тогда ученики Его, подойдя к Нему, разбудили Его и сказали: Господи! спаси нас, погибаем.
\vs Mat 8:26 И говорит им: что вы \bibemph{так} боязливы, маловерные? Потом, встав, запретил ветрам и морю, и сделалась великая тишина.
\vs Mat 8:27 Люди же, удивляясь, говорили: кто это, что и ветры и море повинуются Ему?
\rsbpar\vs Mat 8:28 И когда Он прибыл на другой берег в страну Гергесинскую, Его встретили два бесноватые, вышедшие из гробов\fns{Из пещер, где погребали.}, весьма свирепые, так что никто не смел проходить тем путем.
\vs Mat 8:29 И вот, они закричали: что Тебе до нас, Иисус, Сын Божий? пришел Ты сюда прежде времени мучить нас.
\vs Mat 8:30 Вдали же от них паслось большое стадо свиней.
\vs Mat 8:31 И бесы просили Его: если выгонишь нас, то пошли нас в стадо свиней.
\vs Mat 8:32 И Он сказал им: идите. И они, выйдя, пошли в стадо свиное. И вот, всё стадо свиней бросилось с крутизны в море и погибло в воде.
\vs Mat 8:33 Пастухи же побежали и, придя в город, рассказали обо всем, и о том, что было с бесноватыми.
\vs Mat 8:34 И вот, весь город вышел навстречу Иисусу; и, увидев Его, просили, чтобы Он отошел от пределов их.
\vs Mat 9:1 Тогда Он, войдя в лодку, переправился \bibemph{обратно} и прибыл в Свой город.
\vs Mat 9:2 И вот, принесли к Нему расслабленного, положенного на постели. И, видя Иисус веру их, сказал расслабленному: дерзай, чадо! прощаются тебе грехи твои.
\vs Mat 9:3 При сем некоторые из книжников сказали сами в себе: Он богохульствует.
\vs Mat 9:4 Иисус же, видя помышления их, сказал: для чего вы мыслите худое в сердцах ваших?
\vs Mat 9:5 ибо что легче сказать: прощаются тебе грехи, или сказать: встань и ходи?
\vs Mat 9:6 Но чтобы вы знали, что Сын Человеческий имеет власть на земле прощать грехи,~--- тогда говорит расслабленному: встань, возьми постель твою, и иди в дом твой.
\vs Mat 9:7 И он встал, \bibemph{взял постель свою} и пошел в дом свой.
\vs Mat 9:8 Народ же, видев это, удивился и прославил Бога, давшего такую власть человекам.
\rsbpar\vs Mat 9:9 Проходя оттуда, Иисус увидел человека, сидящего у сбора пошлин, по имени Матфея, и говорит ему: следуй за Мною. И он встал и последовал за Ним.
\vs Mat 9:10 И когда Иисус возлежал в доме, многие мытари и грешники пришли и возлегли с Ним и учениками Его.
\vs Mat 9:11 Увидев то, фарисеи сказали ученикам Его: для чего Учитель ваш ест и пьет с мытарями и грешниками?
\vs Mat 9:12 Иисус же, услышав это, сказал им: не здоровые имеют нужду во враче, но больные,
\vs Mat 9:13 пойдите, научитесь, чт\acc{о} значит: милости хочу, а не жертвы? Ибо Я пришел призвать не праведников, но грешников к покаянию.
\rsbpar\vs Mat 9:14 Тогда приходят к Нему ученики Иоанновы и говорят: почему мы и фарисеи постимся много, а Твои ученики не постятся?
\vs Mat 9:15 И сказал им Иисус: могут ли печалиться сыны чертога брачного, пока с ними жених? Но придут дни, когда отнимется у них жених, и тогда будут поститься.
\vs Mat 9:16 И никто к ветхой одежде не приставляет заплаты из небеленой ткани, ибо вновь пришитое отдерет от старого, и дыра будет еще хуже.
\vs Mat 9:17 Не вливают также вина молодого в мехи ветхие; а иначе прорываются мехи, и вино вытекает, и мехи пропадают, но вино молодое вливают в новые мехи, и сберегается то и другое.
\rsbpar\vs Mat 9:18 Когда Он говорил им сие, подошел к Нему некоторый начальник и, кланяясь Ему, говорил: дочь моя теперь умирает; но приди, возложи на нее руку Твою, и она будет жива.
\vs Mat 9:19 И встав, Иисус пошел за ним, и ученики Его.
\rsbpar\vs Mat 9:20 И вот, женщина, двенадцать лет страдавшая кровотечением, подойдя сзади, прикоснулась к краю одежды Его,
\vs Mat 9:21 ибо она говорила сама в себе: если только прикоснусь к одежде Его, выздоровею.
\vs Mat 9:22 Иисус же, обратившись и увидев ее, сказал: дерзай, дщерь! вера твоя спасла тебя. Женщина с того часа стала здорова.
\rsbpar\vs Mat 9:23 И когда пришел Иисус в дом начальника и увидел свирельщиков и народ в смятении,
\vs Mat 9:24 сказал им: выйдите вон, ибо не умерла девица, но спит. И смеялись над Ним.
\vs Mat 9:25 Когда же народ был выслан, Он, войдя, взял ее за руку, и девица встала.
\vs Mat 9:26 И разнесся слух о сем по всей земле той.
\rsbpar\vs Mat 9:27 Когда Иисус шел оттуда, за Ним следовали двое слепых и кричали: помилуй нас, Иисус, сын Давидов!
\vs Mat 9:28 Когда же Он пришел в дом, слепые приступили к Нему. И говорит им Иисус: веруете ли, что Я могу это сделать? Они говорят Ему: ей, Господи!
\vs Mat 9:29 Тогда Он коснулся глаз их и сказал: по вере вашей да будет вам.
\vs Mat 9:30 И открылись глаза их; и Иисус строго сказал им: смотрите, чтобы никто не узнал.
\vs Mat 9:31 А они, выйдя, разгласили о Нем по всей земле той.
\rsbpar\vs Mat 9:32 Когда же те выходили, то привели к Нему человека немого бесноватого.
\vs Mat 9:33 И когда бес был изгнан, немой стал говорить. И народ, удивляясь, говорил: никогда не бывало такого явления в Израиле.
\vs Mat 9:34 А фарисеи говорили: Он изгоняет бесов силою князя бесовского.
\rsbpar\vs Mat 9:35 И ходил Иисус по всем городам и селениям, уча в синагогах их, проповедуя Евангелие Царствия и исцеляя всякую болезнь и всякую немощь в людях.
\vs Mat 9:36 Видя толпы народа, Он сжалился над ними, что они были изнурены и рассеяны, как овцы, не имеющие пастыря.
\vs Mat 9:37 Тогда говорит ученикам Своим: жатвы много, а делателей мало;
\vs Mat 9:38 итак молите Господина жатвы, чтобы выслал делателей на жатву Свою.
\vs Mat 10:1 И призвав двенадцать учеников Своих, Он дал им власть над нечистыми духами, чтобы изгонять их и врачевать всякую болезнь и всякую немощь.
\vs Mat 10:2 Двенадцати же Апостолов имена суть сии: первый Симон, называемый Петром, и Андрей, брат его, Иаков Зеведеев и Иоанн, брат его,
\vs Mat 10:3 Филипп и Варфоломей, Фома и Матфей мытарь, Иаков Алфеев и Леввей, прозванный Фаддеем,
\vs Mat 10:4 Симон Кананит и Иуда Искариот, который и предал Его.
\vs Mat 10:5 Сих двенадцать послал Иисус, и заповедал им, говоря: на путь к язычникам не ходите, и в город Самарянский не входите;
\vs Mat 10:6 а идите наипаче к погибшим овцам дома Израилева;
\vs Mat 10:7 ходя же, проповедуйте, что приблизилось Царство Небесное;
\vs Mat 10:8 больных исцеляйте, прокаженных очищайте, мертвых воскрешайте, бесов изгоняйте; даром получили, даром давайте.
\vs Mat 10:9 Не берите с собою ни золота, ни серебра, ни меди в поясы свои,
\vs Mat 10:10 ни сумы на дорогу, ни двух одежд, ни обуви, ни посоха, ибо трудящийся достоин пропитания.
\vs Mat 10:11 В какой бы город или селение ни вошли вы, наведывайтесь, кто в нем достоин, и там оставайтесь, пока не выйдете;
\vs Mat 10:12 а входя в дом, приветствуйте его, говоря: мир дому сему;
\vs Mat 10:13 и если дом будет достоин, то мир ваш придет на него; если же не будет достоин, то мир ваш к вам возвратится.
\vs Mat 10:14 А если кто не примет вас и не послушает слов ваших, то, выходя из дома или из города того, отрясите прах от ног ваших;
\vs Mat 10:15 истинно говорю вам: отраднее будет земле Содомской и Гоморрской в день суда, нежели городу тому.
\vs Mat 10:16 Вот, Я посылаю вас, как овец среди волков: итак будьте мудры, как змии, и просты, как голуби.
\vs Mat 10:17 Остерегайтесь же людей: ибо они будут отдавать вас в судилища и в синагогах своих будут бить вас,
\vs Mat 10:18 и поведут вас к правителям и царям за Меня, для свидетельства перед ними и язычниками.
\vs Mat 10:19 Когда же будут предавать вас, не заботьтесь, как или что сказать; ибо в тот час дано будет вам, что сказать,
\vs Mat 10:20 ибо не вы будете говорить, но Дух Отца вашего будет говорить в вас.
\vs Mat 10:21 Предаст же брат брата на смерть, и отец~--- сына; и восстанут дети на родителей, и умертвят их;
\vs Mat 10:22 и будете ненавидимы всеми за имя Мое; претерпевший же до конца спасется.
\vs Mat 10:23 Когда же будут гнать вас в одном городе, бегите в другой. Ибо истинно говорю вам: не успеете обойти городов Израилевых, как приидет Сын Человеческий.
\vs Mat 10:24 Ученик не выше учителя, и слуга не выше господина своего:
\vs Mat 10:25 довольно для ученика, чтобы он был, как учитель его, и для слуги, чтобы он был, как господин его. Если хозяина дома назвали веельзевулом, не тем ли более домашних его?
\vs Mat 10:26 Итак не бойтесь их, ибо нет ничего сокровенного, что не открылось бы, и тайного, что не было бы узнано.
\vs Mat 10:27 Что говорю вам в темноте, говорите при свете; и что на ухо слышите, проповедуйте на кровлях.
\vs Mat 10:28 И не бойтесь убивающих тело, душ\acc{и} же не могущих убить; а бойтесь более Того, Кто может и душу и тело погубить в геенне.
\vs Mat 10:29 Не две ли малые птицы продаются за ассарий\fns{Мелкая монета.}? И ни одна из них не упадет на землю без \bibemph{воли} Отца вашего;
\vs Mat 10:30 у вас же и волосы на голове все сочтены;
\vs Mat 10:31 не бойтесь же: вы лучше многих малых птиц.
\vs Mat 10:32 Итак всякого, кто исповедает Меня пред людьми, того исповедаю и Я пред Отцем Моим Небесным;
\vs Mat 10:33 а кто отречется от Меня пред людьми, отрекусь от того и Я пред Отцем Моим Небесным.
\vs Mat 10:34 Не думайте, что Я пришел принести мир на землю; не мир пришел Я принести, но меч,
\vs Mat 10:35 ибо Я пришел разделить человека с отцом его, и дочь с матерью ее, и невестку со свекровью ее.
\vs Mat 10:36 И враги человеку~--- домашние его.
\vs Mat 10:37 Кто любит отца или мать более, нежели Меня, не достоин Меня; и кто любит сына или дочь более, нежели Меня, не достоин Меня;
\vs Mat 10:38 и кто не берет креста своего и следует за Мною, тот не достоин Меня.
\vs Mat 10:39 Сберегший душу свою потеряет ее; а потерявший душу свою ради Меня сбережет ее.
\vs Mat 10:40 Кто принимает вас, принимает Меня, а кто принимает Меня, принимает Пославшего Меня;
\vs Mat 10:41 кто принимает пророка, во имя пророка, получит награду пророка; и кто принимает праведника, во имя праведника, получит награду праведника.
\vs Mat 10:42 И кто напоит одного из малых сих только чашею холодной воды, во имя ученика, истинно говорю вам, не потеряет награды своей.
\vs Mat 11:1 И когда окончил Иисус наставления двенадцати ученикам Своим, перешел оттуда учить и проповедовать в городах их.
\vs Mat 11:2 Иоанн же, услышав в темнице о делах Христовых, послал двоих из учеников своих
\vs Mat 11:3 сказать Ему: Ты ли Тот, Который должен прийти, или ожидать нам другого?
\vs Mat 11:4 И сказал им Иисус в ответ: пойдите, скажите Иоанну, что слышите и видите:
\vs Mat 11:5 слепые прозревают и хромые ходят, прокаженные очищаются и глухие слышат, мертвые воскресают и нищие благовествуют;
\vs Mat 11:6 и блажен, кто не соблазнится о Мне.
\rsbpar\vs Mat 11:7 Когда же они пошли, Иисус начал говорить народу об Иоанне: чт\acc{о} смотреть ходили вы в пустыню? трость ли, ветром колеблемую?
\vs Mat 11:8 Чт\acc{о} же смотреть ходили вы? человека ли, одетого в мягкие одежды? Носящие мягкие одежды находятся в чертогах царских.
\vs Mat 11:9 Чт\acc{о} же смотреть ходили вы? пророка? Да, говорю вам, и больше пророка.
\vs Mat 11:10 Ибо он тот, о котором написано: се, Я посылаю Ангела Моего пред лицем Твоим, который приготовит путь Твой пред Тобою.
\vs Mat 11:11 Истинно говорю вам: из рожденных женами не восставал больший Иоанна Крестителя; но меньший в Царстве Небесном больше его.
\vs Mat 11:12 От дней же Иоанна Крестителя доныне Царство Небесное силою берется, и употребляющие усилие восхищают его,
\vs Mat 11:13 ибо все пророки и закон прорекли до Иоанна.
\vs Mat 11:14 И если хотите принять, он есть Илия, которому должно прийти.
\vs Mat 11:15 Кто имеет уши слышать, да слышит!
\vs Mat 11:16 Но кому уподоблю род сей? Он подобен детям, которые сидят на улице и, обращаясь к своим товарищам,
\vs Mat 11:17 говорят: мы играли вам на свирели, и вы не плясали; мы пели вам печальные песни, и вы не рыдали.
\vs Mat 11:18 Ибо пришел Иоанн, ни ест, ни пьет; и говорят: в нем бес.
\vs Mat 11:19 Пришел Сын Человеческий, ест и пьет; и говорят: вот человек, который любит есть и пить вино, друг мытарям и грешникам. И оправдана премудрость чадами ее.
\rsbpar\vs Mat 11:20 Тогда начал Он укорять города, в которых наиболее явлено было сил Его, за то, что они не покаялись:
\vs Mat 11:21 горе тебе, Хоразин! горе тебе, Вифсаида! ибо если бы в Тире и Сидоне явлены были силы, явленные в вас, то давно бы они во вретище и пепле покаялись,
\vs Mat 11:22 но говорю вам: Тиру и Сидону отраднее будет в день суда, нежели вам.
\vs Mat 11:23 И ты, Капернаум, до неба вознесшийся, до ада низвергнешься, ибо если бы в Содоме явлены были силы, явленные в тебе, то он оставался бы до сего дня;
\vs Mat 11:24 но говорю вам, что земле Содомской отраднее будет в день суда, нежели тебе.
\vs Mat 11:25 В то время, продолжая речь, Иисус сказал: славлю Тебя, Отче, Господи неба и земли, что Ты утаил сие от мудрых и разумных и открыл то младенцам;
\vs Mat 11:26 ей, Отче! ибо таково было Твое благоволение.
\rsbpar\vs Mat 11:27 Все предано Мне Отцем Моим, и никто не знает Сына, кроме Отца; и Отца не знает никто, кроме Сына, и кому Сын хочет открыть.
\vs Mat 11:28 Придите ко Мне все труждающиеся и обремененные, и Я успокою вас;
\vs Mat 11:29 возьмите иго Мое на себя и научитесь от Меня, ибо Я кроток и смирен сердцем, и найдете покой душам вашим;
\vs Mat 11:30 ибо иго Мое благо, и бремя Мое легко.
\vs Mat 12:1 В то время проходил Иисус в субботу засеянными полями; ученики же Его взалкали и начали срывать колосья и есть.
\vs Mat 12:2 Фарисеи, увидев это, сказали Ему: вот, ученики Твои делают, чего не должно делать в субботу.
\vs Mat 12:3 Он же сказал им: разве вы не читали, что сделал Давид, когда взалкал сам и бывшие с ним?
\vs Mat 12:4 как он вошел в дом Божий и ел хлебы предложения, которых не должно было есть ни ему, ни бывшим с ним, а только одним священникам?
\vs Mat 12:5 Или не читали ли вы в законе, что в субботы священники в храме нарушают субботу, однако невиновны?
\vs Mat 12:6 Но говорю вам, что здесь Тот, Кто больше храма;
\vs Mat 12:7 если бы вы знали, чт\acc{о} значит: милости хочу, а не жертвы, то не осудили бы невиновных,
\vs Mat 12:8 ибо Сын Человеческий есть господин и субботы.
\rsbpar\vs Mat 12:9 И, отойдя оттуда, вошел Он в синагогу их.
\vs Mat 12:10 И вот, там был человек, имеющий сухую руку. И спросили Иисуса, чтобы обвинить Его: можно ли исцелять в субботы?
\vs Mat 12:11 Он же сказал им: кто из вас, имея одну овцу, если она в субботу упадет в яму, не возьмет ее и не вытащит?
\vs Mat 12:12 Сколько же лучше человек овцы! Итак можно в субботы делать добро.
\vs Mat 12:13 Тогда говорит человеку тому: протяни руку твою. И он протянул, и стала она здорова, как другая.
\rsbpar\vs Mat 12:14 Фарисеи же, выйдя, имели совещание против Него, как бы погубить Его. Но Иисус, узнав, удалился оттуда.
\vs Mat 12:15 И последовало за Ним множество народа, и Он исцелил их всех
\vs Mat 12:16 и запретил им объявлять о Нем,
\vs Mat 12:17 да сбудется реченное через пророка Исаию, который говорит:
\vs Mat 12:18 Се, Отрок Мой, Которого Я избрал, Возлюбленный Мой, Которому благоволит душа Моя. Положу дух Мой на Него, и возвестит народам суд;
\vs Mat 12:19 не воспрекословит, не возопиет, и никто не услышит на улицах голоса Его;
\vs Mat 12:20 трости надломленной не переломит, и льна курящегося не угасит, доколе не доставит суду победы;
\vs Mat 12:21 и на имя Его будут уповать народы.
\rsbpar\vs Mat 12:22 Тогда привели к Нему бесноватого слепого и немого; и исцелил его, так что слепой и немой стал и говорить и видеть.
\vs Mat 12:23 И дивился весь народ и говорил: не это ли Христос, сын Давидов?
\vs Mat 12:24 Фарисеи же, услышав \bibemph{сие}, сказали: Он изгоняет бесов не иначе, как \bibemph{силою} веельзевула, князя бесовского.
\vs Mat 12:25 Но Иисус, зная помышления их, сказал им: всякое царство, разделившееся само в себе, опустеет; и всякий город или дом, разделившийся сам в себе, не устоит.
\vs Mat 12:26 И если сатана сатану изгоняет, то он разделился сам с собою: как же устоит царство его?
\vs Mat 12:27 И если Я \bibemph{силою} веельзевула изгоняю бесов, то сыновья ваши чьею \bibemph{силою} изгоняют? Посему они будут вам судьями.
\vs Mat 12:28 Если же Я Духом Божиим изгоняю бесов, то конечно достигло до вас Царствие Божие.
\vs Mat 12:29 Или, как может кто войти в дом сильного и расхитить вещи его, если прежде не свяжет сильного? и тогда расхитит дом его.
\vs Mat 12:30 Кто не со Мною, тот против Меня; и кто не собирает со Мною, тот расточает.
\vs Mat 12:31 Посему говорю вам: всякий грех и хула простятся человекам, а хула на Духа не простится человекам;
\vs Mat 12:32 если кто скажет слово на Сына Человеческого, простится ему; если же кто скажет на Духа Святаго, не простится ему ни в сем веке, ни в будущем.
\vs Mat 12:33 Или признайте дерево хорошим и плод его хорошим; или признайте дерево худым и плод его худым, ибо дерево познается по плоду.
\vs Mat 12:34 Порождения ехиднины! как вы можете говорить доброе, будучи злы? Ибо от избытка сердца говорят уста.
\vs Mat 12:35 Добрый человек из доброго сокровища выносит доброе, а злой человек из злого сокровища выносит злое.
\vs Mat 12:36 Говорю же вам, что за всякое праздное слово, какое скажут люди, дадут они ответ в день суда:
\vs Mat 12:37 ибо от слов своих оправдаешься, и от слов своих осудишься.
\rsbpar\vs Mat 12:38 Тогда некоторые из книжников и фарисеев сказали: Учитель! хотелось бы нам видеть от Тебя знамение.
\vs Mat 12:39 Но Он сказал им в ответ: род лукавый и прелюбодейный ищет знамения; и знамение не дастся ему, кроме знамения Ионы пророка;
\vs Mat 12:40 ибо как Иона был во чреве кита три дня и три ночи, так и Сын Человеческий будет в сердце земли три дня и три ночи.
\vs Mat 12:41 Ниневитяне восстанут на суд с родом сим и осудят его, ибо они покаялись от проповеди Иониной; и вот, здесь больше Ионы.
\vs Mat 12:42 Царица южная восстанет на суд с родом сим и осудит его, ибо она приходила от пределов земли послушать мудрости Соломоновой; и вот, здесь больше Соломона.
\vs Mat 12:43 Когда нечистый дух выйдет из человека, то ходит по безводным местам, ища покоя, и не находит;
\vs Mat 12:44 тогда говорит: возвращусь в дом мой, откуда я вышел. И, придя, находит \bibemph{его} незанятым, выметенным и убранным;
\vs Mat 12:45 тогда идет и берет с собою семь других духов, злейших себя, и, войдя, живут там; и бывает для человека того последнее хуже первого. Так будет и с этим злым родом.
\rsbpar\vs Mat 12:46 Когда же Он еще говорил к народу, Матерь и братья Его стояли вне \bibemph{дома}, желая говорить с Ним.
\vs Mat 12:47 И некто сказал Ему: вот Матерь Твоя и братья Твои стоят вне, желая говорить с Тобою.
\vs Mat 12:48 Он же сказал в ответ говорившему: кто Матерь Моя? и кто братья Мои?
\vs Mat 12:49 И, указав рукою Своею на учеников Своих, сказал: вот матерь Моя и братья Мои;
\vs Mat 12:50 ибо, кто будет исполнять волю Отца Моего Небесного, тот Мне брат, и сестра, и матерь.
\vs Mat 13:1 Выйдя же в день тот из дома, Иисус сел у моря.
\vs Mat 13:2 И собралось к Нему множество народа, так что Он вошел в лодку и сел; а весь народ стоял на берегу.
\vs Mat 13:3 И поучал их много притчами, говоря: вот, вышел сеятель сеять;
\vs Mat 13:4 и когда он сеял, иное упало при дороге, и налетели птицы и поклевали то;
\vs Mat 13:5 иное упало на места каменистые, где немного было земли, и скоро взошло, потому что земля была неглубока.
\vs Mat 13:6 Когда же взошло солнце, увяло, и, как не имело корня, засохло;
\vs Mat 13:7 иное упало в терние, и выросло терние и заглушило его;
\vs Mat 13:8 иное упало на добрую землю и принесло плод: одно во сто крат, а другое в шестьдесят, иное же в тридцать.
\vs Mat 13:9 Кто имеет уши слышать, да слышит!
\vs Mat 13:10 И, приступив, ученики сказали Ему: для чего притчами говоришь им?
\vs Mat 13:11 Он сказал им в ответ: для того, что вам дано знать тайны Царствия Небесного, а им не дано,
\vs Mat 13:12 ибо кто имеет, тому дано будет и приумножится, а кто не имеет, у того отнимется и то, что имеет;
\vs Mat 13:13 потому говорю им притчами, что они видя не видят, и слыша не слышат, и не разумеют;
\vs Mat 13:14 и сбывается над ними пророчество Исаии, которое говорит: слухом услышите~--- и не уразумеете, и глазами смотреть будете~--- и не увидите,
\vs Mat 13:15 ибо огрубело сердце людей сих и ушами с трудом слышат, и глаза свои сомкнули, да не увидят глазами и не услышат ушами, и не уразумеют сердцем, и да не обратятся, чтобы Я исцелил их.
\vs Mat 13:16 Ваши же блаженны очи, что видят, и уши ваши, что слышат,
\vs Mat 13:17 ибо истинно говорю вам, что многие пророки и праведники желали видеть, чт\acc{о} вы видите, и не видели, и слышать, чт\acc{о} вы слышите, и не слышали.
\vs Mat 13:18 Вы же выслушайте \bibemph{значение} притчи о сеятеле:
\vs Mat 13:19 ко всякому, слушающему слово о Царствии и не разумеющему, приходит лукавый и похищает посеянное в сердце его~--- вот кого означает посеянное при дороге.
\vs Mat 13:20 А посеянное на каменистых местах означает того, кто слышит слово и тотчас с радостью принимает его;
\vs Mat 13:21 но не имеет в себе корня и непостоянен: когда настанет скорбь или гонение за слово, тотчас соблазняется.
\vs Mat 13:22 А посеянное в тернии означает того, кто слышит слово, но забота века сего и обольщение богатства заглушает слово, и оно бывает бесплодно.
\vs Mat 13:23 Посеянное же на доброй земле означает слышащего слово и разумеющего, который и бывает плодоносен, так что иной приносит плод во сто крат, иной в шестьдесят, а иной в тридцать.
\rsbpar\vs Mat 13:24 Другую притчу предложил Он им, говоря: Царство Небесное подобно человеку, посеявшему доброе семя на поле своем;
\vs Mat 13:25 когда же люди спали, пришел враг его и посеял между пшеницею плевелы и ушел;
\vs Mat 13:26 когда взошла зелень и показался плод, тогда явились и плевелы.
\vs Mat 13:27 Придя же, рабы домовладыки сказали ему: господин! не доброе ли семя сеял ты на поле твоем? откуда же на нем плевелы?
\vs Mat 13:28 Он же сказал им: враг человек сделал это. А рабы сказали ему: хочешь ли, мы пойдем, выберем их?
\vs Mat 13:29 Но он сказал: нет~--- чтобы, выбирая плевелы, вы не выдергали вместе с ними пшеницы,
\vs Mat 13:30 оставьте расти вместе т\acc{о} и другое до жатвы; и во время жатвы я скажу жнецам: соберите прежде плевелы и свяжите их в снопы, чтобы сжечь их, а пшеницу уберите в житницу мою.
\rsbpar\vs Mat 13:31 Иную притчу предложил Он им, говоря: Царство Небесное подобно зерну горчичному, которое человек взял и посеял на поле своем,
\vs Mat 13:32 которое, хотя меньше всех семян, но, когда вырастет, бывает больше всех злаков и становится деревом, так что прилетают птицы небесные и укрываются в ветвях его.
\rsbpar\vs Mat 13:33 Иную притчу сказал Он им: Царство Небесное подобно закваске, которую женщина, взяв, положила в три меры муки, доколе не вскисло всё.
\rsbpar\vs Mat 13:34 Всё сие Иисус говорил народу притчами, и без притчи не говорил им,
\vs Mat 13:35 да сбудется реченное через пророка, который говорит: отверзу в притчах уста Мои; изреку сокровенное от создания мира.
\rsbpar\vs Mat 13:36 Тогда Иисус, отпустив народ, вошел в дом. И, приступив к Нему, ученики Его сказали: изъясни нам притчу о плевелах на поле.
\vs Mat 13:37 Он же сказал им в ответ: сеющий доброе семя есть Сын Человеческий;
\vs Mat 13:38 поле есть мир; доброе семя, это сыны Царствия, а плевелы~--- сыны лукавого;
\vs Mat 13:39 враг, посеявший их, есть диавол; жатва есть кончина века, а жнецы суть Ангелы.
\vs Mat 13:40 Посему как собирают плевелы и огнем сжигают, так будет при кончине века сего:
\vs Mat 13:41 пошлет Сын Человеческий Ангелов Своих, и соберут из Царства Его все соблазны и делающих беззаконие,
\vs Mat 13:42 и ввергнут их в печь огненную; там будет плач и скрежет зубов;
\vs Mat 13:43 тогда праведники воссияют, как солнце, в Царстве Отца их. Кто имеет уши слышать, да слышит!
\rsbpar\vs Mat 13:44 Еще подобно Царство Небесное сокровищу, скрытому на поле, которое, найдя, человек утаил, и от радости о нем идет и продает всё, что имеет, и покупает поле то.
\rsbpar\vs Mat 13:45 Еще подобно Царство Небесное купцу, ищущему хороших жемчужин,
\vs Mat 13:46 который, найдя одну драгоценную жемчужину, пошел и продал всё, что имел, и купил ее.
\rsbpar\vs Mat 13:47 Еще подобно Царство Небесное неводу, закинутому в море и захватившему рыб всякого рода,
\vs Mat 13:48 который, когда наполнился, вытащили на берег и, сев, хорошее собрали в сосуды, а худое выбросили вон.
\vs Mat 13:49 Так будет при кончине века: изыдут Ангелы, и отделят злых из среды праведных,
\vs Mat 13:50 и ввергнут их в печь огненную: там будет плач и скрежет зубов.
\vs Mat 13:51 И спросил их Иисус: поняли ли вы всё это? Они говорят Ему: т\acc{а}к, Господи!
\vs Mat 13:52 Он же сказал им: поэтому всякий книжник, наученный Царству Небесному, подобен хозяину, который выносит из сокровищницы своей новое и старое.
\rsbpar\vs Mat 13:53 И, когда окончил Иисус притчи сии, пошел оттуда.
\vs Mat 13:54 И, придя в отечество Свое, учил их в синагоге их, так что они изумлялись и говорили: откуда у Него такая премудрость и силы?
\vs Mat 13:55 не плотников ли Он сын? не Его ли Мать называется Мария, и братья Его Иаков и Иосий, и Симон, и Иуда?
\vs Mat 13:56 и сестры Его не все ли между нами? откуда же у Него всё это?
\vs Mat 13:57 И соблазнялись о Нем. Иисус же сказал им: не бывает пророк без чести, разве только в отечестве своем и в доме своем.
\vs Mat 13:58 И не совершил там многих чудес по неверию их.
\vs Mat 14:1 В то время Ирод четвертовластник услышал молву об Иисусе
\vs Mat 14:2 и сказал служащим при нем: это Иоанн Креститель; он воскрес из мертвых, и потому чудеса делаются им.
\vs Mat 14:3 Ибо Ирод, взяв Иоанна, связал его и посадил в темницу за Иродиаду, жену Филиппа, брата своего,
\vs Mat 14:4 потому что Иоанн говорил ему: не должно тебе иметь ее.
\vs Mat 14:5 И хотел убить его, но боялся народа, потому что его почитали за пророка.
\vs Mat 14:6 Во время же \bibemph{празднования} дня рождения Ирода дочь Иродиады плясала перед собранием и угодила Ироду,
\vs Mat 14:7 посему он с клятвою обещал ей дать, чего она ни попросит.
\vs Mat 14:8 Она же, по наущению матери своей, сказала: дай мне здесь на блюде голову Иоанна Крестителя.
\vs Mat 14:9 И опечалился царь, но, ради клятвы и возлежащих с ним, повелел дать ей,
\vs Mat 14:10 и послал отсечь Иоанну голову в темнице.
\vs Mat 14:11 И принесли голову его на блюде и дали девице, а она отнесла матери своей.
\vs Mat 14:12 Ученики же его, придя, взяли тело его и погребли его; и пошли, возвестили Иисусу.
\rsbpar\vs Mat 14:13 И, услышав, Иисус удалился оттуда на лодке в пустынное место один; а народ, услышав о том, пошел за Ним из городов пешком.
\vs Mat 14:14 И, выйдя, Иисус увидел множество людей и сжалился над ними, и исцелил больных их.
\vs Mat 14:15 Когда же настал вечер, приступили к Нему ученики Его и сказали: место здесь пустынное и время уже позднее; отпусти народ, чтобы они пошли в селения и купили себе пищи.
\vs Mat 14:16 Но Иисус сказал им: не нужно им идти, вы дайте им есть.
\vs Mat 14:17 Они же говорят Ему: у нас здесь только пять хлебов и две рыбы.
\vs Mat 14:18 Он сказал: принесите их Мне сюда.
\vs Mat 14:19 И велел народу возлечь на траву и, взяв пять хлебов и две рыбы, воззрел на небо, благословил и, преломив, дал хлебы ученикам, а ученики народу.
\vs Mat 14:20 И ели все и насытились; и набрали оставшихся кусков двенадцать коробов полных;
\vs Mat 14:21 а евших было около пяти тысяч человек, кроме женщин и детей.
\rsbpar\vs Mat 14:22 И тотчас понудил Иисус учеников Своих войти в лодку и отправиться прежде Его на другую сторону, пока Он отпустит народ.
\vs Mat 14:23 И, отпустив народ, Он взошел на гору помолиться наедине; и вечером оставался там один.
\vs Mat 14:24 А лодка была уже на средине моря, и ее било волнами, потому что ветер был противный.
\vs Mat 14:25 В четвертую же стражу ночи пошел к ним Иисус, идя по морю.
\vs Mat 14:26 И ученики, увидев Его идущего по морю, встревожились и говорили: это призрак; и от страха вскричали.
\vs Mat 14:27 Но Иисус тотчас заговорил с ними и сказал: ободритесь; это Я, не бойтесь.
\vs Mat 14:28 Петр сказал Ему в ответ: Господи! если это Ты, повели мне прийти к Тебе по воде.
\vs Mat 14:29 Он же сказал: иди. И, выйдя из лодки, Петр пошел по воде, чтобы подойти к Иисусу,
\vs Mat 14:30 но, видя сильный ветер, испугался и, начав утопать, закричал: Господи! спаси меня.
\vs Mat 14:31 Иисус тотчас простер руку, поддержал его и говорит ему: маловерный! зачем ты усомнился?
\vs Mat 14:32 И, когда вошли они в лодку, ветер утих.
\vs Mat 14:33 Бывшие же в лодке подошли, поклонились Ему и сказали: истинно Ты Сын Божий.
\rsbpar\vs Mat 14:34 И, переправившись, прибыли в землю Геннисаретскую.
\vs Mat 14:35 Жители того места, узнав Его, послали во всю окрестность ту и принесли к Нему всех больных,
\vs Mat 14:36 и просили Его, чтобы только прикоснуться к краю одежды Его; и которые прикасались, исцелялись.
\vs Mat 15:1 Тогда приходят к Иисусу Иерусалимские книжники и фарисеи и говорят:
\vs Mat 15:2 зачем ученики Твои преступают предание старцев? ибо не умывают рук своих, когда едят хлеб.
\vs Mat 15:3 Он же сказал им в ответ: зачем и вы преступаете заповедь Божию ради предания вашего?
\vs Mat 15:4 Ибо Бог заповедал: почитай отца и мать; и: злословящий отца или мать смертью да умрет.
\vs Mat 15:5 А вы говорите: если кто скажет отцу или матери: дар \bibemph{Богу} то, чем бы ты от меня пользовался,
\vs Mat 15:6 тот может и не почтить отца своего или мать свою; таким образом вы устранили заповедь Божию преданием вашим.
\vs Mat 15:7 Лицемеры! хорошо пророчествовал о вас Исаия, говоря:
\vs Mat 15:8 приближаются ко Мне люди сии устами своими, и чтут Меня языком, сердце же их далеко отстоит от Меня;
\vs Mat 15:9 но тщетно чтут Меня, уча учениям, заповедям человеческим.
\vs Mat 15:10 И, призвав народ, сказал им: слушайте и разумейте!
\vs Mat 15:11 не т\acc{о}, чт\acc{о} входит в уста, оскверняет человека, но т\acc{о}, чт\acc{о} выходит из уст, оскверняет человека.
\vs Mat 15:12 Тогда ученики Его, приступив, сказали Ему: знаешь ли, что фарисеи, услышав слово сие, соблазнились?
\vs Mat 15:13 Он же сказал в ответ: всякое растение, которое не Отец Мой Небесный насадил, искоренится;
\vs Mat 15:14 оставьте их: они~--- слепые вожди слепых; а если слепой ведет слепого, то оба упадут в яму.
\vs Mat 15:15 Петр же, отвечая, сказал Ему: изъясни нам притчу сию.
\vs Mat 15:16 Иисус сказал: неужели и вы еще не разумеете?
\vs Mat 15:17 еще ли не понимаете, что всё, входящее в уста, проходит в чрево и извергается вон?
\vs Mat 15:18 а исходящее из уст~--- из сердца исходит~--- сие оскверняет человека,
\vs Mat 15:19 ибо из сердца исходят злые помыслы, убийства, прелюбодеяния, любодеяния, кражи, лжесвидетельства, хуления~---
\vs Mat 15:20 это оскверняет человека; а есть неумытыми руками~--- не оскверняет человека.
\rsbpar\vs Mat 15:21 И, выйдя оттуда, Иисус удалился в страны Тирские и Сидонские.
\vs Mat 15:22 И вот, женщина Хананеянка, выйдя из тех мест, кричала Ему: помилуй меня, Господи, сын Давидов, дочь моя жестоко беснуется.
\vs Mat 15:23 Но Он не отвечал ей ни слова. И ученики Его, приступив, просили Его: отпусти ее, потому что кричит за нами.
\vs Mat 15:24 Он же сказал в ответ: Я послан только к погибшим овцам дома Израилева.
\vs Mat 15:25 А она, подойдя, кланялась Ему и говорила: Господи! помоги мне.
\vs Mat 15:26 Он же сказал в ответ: нехорошо взять хлеб у детей и бросить псам.
\vs Mat 15:27 Она сказала: так, Господи! но и псы едят крохи, которые падают со стола господ их.
\vs Mat 15:28 Тогда Иисус сказал ей в ответ: о, женщина! велик\acc{а} вера твоя; да будет тебе по желанию твоему. И исцелилась дочь ее в тот час.
\rsbpar\vs Mat 15:29 Перейдя оттуда, пришел Иисус к морю Галилейскому и, взойдя на гору, сел там.
\vs Mat 15:30 И приступило к Нему множество народа, имея с собою хромых, слепых, немых, увечных и иных многих, и повергли их к ногам Иисусовым; и Он исцелил их;
\vs Mat 15:31 так что народ дивился, видя немых говорящими, увечных здоровыми, хромых ходящими и слепых видящими; и прославлял Бога Израилева.
\rsbpar\vs Mat 15:32 Иисус же, призвав учеников Своих, сказал им: жаль Мне народа, что уже три дня находятся при Мне, и нечего им есть; отпустить же их неевшими не хочу, чтобы не ослабели в дороге.
\vs Mat 15:33 И говорят Ему ученики Его: откуда нам взять в пустыне столько хлебов, чтобы накормить столько народа?
\vs Mat 15:34 Говорит им Иисус: сколько у вас хлебов? Они же сказали: семь, и немного рыбок.
\vs Mat 15:35 Тогда велел народу возлечь на землю.
\vs Mat 15:36 И, взяв семь хлебов и рыбы, воздал благодарение, преломил и дал ученикам Своим, а ученики народу.
\vs Mat 15:37 И ели все и насытились; и набрали оставшихся кусков семь корзин полных,
\vs Mat 15:38 а евших было четыре тысячи человек, кроме женщин и детей.
\rsbpar\vs Mat 15:39 И, отпустив народ, Он вошел в лодку и прибыл в пределы Магдалинские.
\vs Mat 16:1 И приступили фарисеи и саддукеи и, искушая Его, просили показать им знамение с неба.
\vs Mat 16:2 Он же сказал им в ответ: вечером вы говорите: будет вёдро, потому что небо красно;
\vs Mat 16:3 и поутру: сегодня ненастье, потому что небо багрово. Лицемеры! различать лице неба вы умеете, а знамений времен не можете.
\vs Mat 16:4 Род лукавый и прелюбодейный знамения ищет, и знамение не дастся ему, кроме знамения Ионы пророка. И, оставив их, отошел.
\rsbpar\vs Mat 16:5 Переправившись на другую сторону, ученики Его забыли взять хлебов.
\vs Mat 16:6 Иисус сказал им: смотрите, берегитесь закваски фарисейской и саддукейской.
\vs Mat 16:7 Они же помышляли в себе и говорили: \bibemph{это значит}, что хлебов мы не взяли.
\vs Mat 16:8 Уразумев то, Иисус сказал им: что помышляете в себе, маловерные, что хлебов не взяли?
\vs Mat 16:9 Еще ли не понимаете и не помните о пяти хлебах на пять тысяч \bibemph{человек}, и сколько коробов вы набрали?
\vs Mat 16:10 ни о семи хлебах на четыре тысячи, и сколько корзин вы набрали?
\vs Mat 16:11 как не разумеете, что не о хлебе сказал Я вам: берегитесь закваски фарисейской и саддукейской?
\vs Mat 16:12 Тогда они поняли, что Он говорил им беречься не закваски хлебной, но учения фарисейского и саддукейского.
\rsbpar\vs Mat 16:13 Придя же в страны Кесарии Филипповой, Иисус спрашивал учеников Своих: за кого люди почитают Меня, Сына Человеческого?
\vs Mat 16:14 Они сказали: одни за Иоанна Крестителя, другие за Илию, а иные за Иеремию, или за одного из пророков.
\vs Mat 16:15 Он говорит им: а вы за кого почитаете Меня?
\vs Mat 16:16 Симон же Петр, отвечая, сказал: Ты~--- Христос, Сын Бога живаго.
\vs Mat 16:17 Тогда Иисус сказал ему в ответ: блажен ты, Симон, сын Ионин, потому что не плоть и кровь открыли тебе это, но Отец Мой, Сущий на небесах;
\vs Mat 16:18 и Я говорю тебе: ты~--- Петр\fns{Камень.}, и на сем камне Я создам Церковь Мою, и врата ада не одолеют ее;
\vs Mat 16:19 и дам тебе ключи Царства Небесного: и чт\acc{о} свяжешь на земле, т\acc{о} будет связано на небесах, и чт\acc{о} разрешишь на земле, т\acc{о} будет разрешено на небесах.
\vs Mat 16:20 Тогда \bibemph{Иисус} запретил ученикам Своим, чтобы никому не сказывали, что Он есть Иисус Христос.
\rsbpar\vs Mat 16:21 С того времени Иисус начал открывать ученикам Своим, что Ему должно идти в Иерусалим и много пострадать от старейшин и первосвященников и книжников, и быть убиту, и в третий день воскреснуть.
\vs Mat 16:22 И, отозвав Его, Петр начал прекословить Ему: будь милостив к Себе, Господи! да не будет этого с Тобою!
\vs Mat 16:23 Он же, обратившись, сказал Петру: отойди от Меня, сатана! ты Мне соблазн! потому что думаешь не о том, чт\acc{о} Божие, но чт\acc{о} человеческое.
\vs Mat 16:24 Тогда Иисус сказал ученикам Своим: если кто хочет идти за Мною, отвергнись себя, и возьми крест свой, и следуй за Мною,
\vs Mat 16:25 ибо кто хочет душу\fns{Жизнь.} свою сберечь, тот потеряет ее, а кто потеряет душу свою ради Меня, тот обретет ее;
\vs Mat 16:26 какая польза человеку, если он приобретет весь мир, а душе своей повредит? или какой выкуп даст человек за душу свою?
\vs Mat 16:27 ибо приидет Сын Человеческий во славе Отца Своего с Ангелами Своими и тогда воздаст каждому по делам его.
\vs Mat 16:28 Истинно говорю вам: есть некоторые из стоящих здесь, которые не вкусят смерти, как уже увидят Сына Человеческого, грядущего в Царствии Своем.
\vs Mat 17:1 По прошествии дней шести, взял Иисус Петра, Иакова и Иоанна, брата его, и возвел их на гору высокую одних,
\vs Mat 17:2 и преобразился пред ними: и просияло лице Его, как солнце, одежды же Его сделались белыми, как свет.
\vs Mat 17:3 И вот, явились им Моисей и Илия, с Ним беседующие.
\vs Mat 17:4 При сем Петр сказал Иисусу: Господи! хорошо нам здесь быть; если хочешь, сделаем здесь три кущи: Тебе одну, и Моисею одну, и одну Илии.
\vs Mat 17:5 Когда он еще говорил, се, облако светлое осенило их; и се, глас из облака глаголющий: Сей есть Сын Мой Возлюбленный, в Котором Мое благоволение; Его слушайте.
\vs Mat 17:6 И, услышав, ученики пали на лица свои и очень испугались.
\vs Mat 17:7 Но Иисус, приступив, коснулся их и сказал: встаньте и не бойтесь.
\vs Mat 17:8 Возведя же очи свои, они никого не увидели, кроме одного Иисуса.
\vs Mat 17:9 И когда сходили они с горы, Иисус запретил им, говоря: никому не сказывайте о сем видении, доколе Сын Человеческий не воскреснет из мертвых.
\vs Mat 17:10 И спросили Его ученики Его: как же книжники говорят, что Илии надлежит прийти прежде?
\vs Mat 17:11 Иисус сказал им в ответ: правда, Илия \bibemph{должен} прийти прежде и устроить всё;
\vs Mat 17:12 но говорю вам, что Илия уже пришел, и не узнали его, а поступили с ним, как хотели; т\acc{а}к и Сын Человеческий пострадает от них.
\vs Mat 17:13 Тогда ученики поняли, что Он говорил им об Иоанне Крестителе.
\rsbpar\vs Mat 17:14 Когда они пришли к народу, то подошел к Нему человек и, преклоняя пред Ним колени,
\vs Mat 17:15 сказал: Господи! помилуй сына моего; он в новолуния \bibemph{беснуется} и тяжко страдает, ибо часто бросается в огонь и часто в воду,
\vs Mat 17:16 я приводил его к ученикам Твоим, и они не могли исцелить его.
\vs Mat 17:17 Иисус же, отвечая, сказал: о, род неверный и развращенный! доколе буду с вами? доколе буду терпеть вас? приведите его ко Мне сюда.
\vs Mat 17:18 И запретил ему Иисус, и бес вышел из него; и отрок исцелился в тот час.
\vs Mat 17:19 Тогда ученики, приступив к Иисусу наедине, сказали: почему мы не могли изгнать его?
\vs Mat 17:20 Иисус же сказал им: по неверию вашему; ибо истинно говорю вам: если вы будете иметь веру с горчичное зерно и скажете горе сей: <<перейди отсюда туда>>, и она перейдет; и ничего не будет невозможного для вас;
\vs Mat 17:21 сей же род изгоняется только молитвою и постом.
\rsbpar\vs Mat 17:22 Во время пребывания их в Галилее, Иисус сказал им: Сын Человеческий предан будет в руки человеческие,
\vs Mat 17:23 и убьют Его, и в третий день воскреснет. И они весьма опечалились.
\rsbpar\vs Mat 17:24 Когда же пришли они в Капернаум, то подошли к Петру собиратели дидрахм\fns{Две драхмы~--- определенная дань на храм.} и сказали: Учитель ваш не даст ли дидрахмы?
\vs Mat 17:25 Он говорит: да. И когда вошел он в дом, то Иисус, предупредив его, сказал: как тебе кажется, Симон? цари земные с кого берут пошлины или подати? с сынов ли своих, или с посторонних?
\vs Mat 17:26 Петр говорит Ему: с посторонних. Иисус сказал ему: итак сыны свободны;
\vs Mat 17:27 но, чтобы нам не соблазнить их, пойди на море, брось уду, и первую рыбу, которая попадется, возьми, и, открыв у ней рот, найдешь статир\fns{Четыре драхмы.}; возьми его и отдай им за Меня и за себя.
\vs Mat 18:1 В то время ученики приступили к Иисусу и сказали: кто больше в Царстве Небесном?
\vs Mat 18:2 Иисус, призвав дитя, поставил его посреди них
\vs Mat 18:3 и сказал: истинно говорю вам, если не обратитесь и не будете как дети, не войдете в Царство Небесное;
\vs Mat 18:4 итак, кто умалится, как это дитя, тот и больше в Царстве Небесном;
\vs Mat 18:5 и кто примет одно такое дитя во имя Мое, тот Меня принимает;
\vs Mat 18:6 а кто соблазнит одного из малых сих, верующих в Меня, тому лучше было бы, если бы повесили ему мельничный жернов на шею и потопили его во глубине морской.
\vs Mat 18:7 Горе миру от соблазнов, ибо надобно прийти соблазнам; но горе тому человеку, через которого соблазн приходит.
\vs Mat 18:8 Если же рука твоя или нога твоя соблазняет тебя, отсеки их и брось от себя: лучше тебе войти в жизнь без руки или без ноги, нежели с двумя руками и с двумя ногами быть ввержену в огонь вечный;
\vs Mat 18:9 и если глаз твой соблазняет тебя, вырви его и брось от себя: лучше тебе с одним глазом войти в жизнь, нежели с двумя глазами быть ввержену в геенну огненную.
\rsbpar\vs Mat 18:10 Смотрите, не презирайте ни одного из малых сих; ибо говорю вам, что Ангелы их на небесах всегда видят лице Отца Моего Небесного.
\vs Mat 18:11 Ибо Сын Человеческий пришел взыскать и спасти погибшее.
\vs Mat 18:12 Как вам кажется? Если бы у кого было сто овец, и одна из них заблудилась, то не оставит ли он девяносто девять в горах и не пойдет ли искать заблудившуюся?
\vs Mat 18:13 и если случится найти ее, то, истинно говорю вам, он радуется о ней более, нежели о девяноста девяти незаблудившихся.
\vs Mat 18:14 Т\acc{а}к, нет воли Отца вашего Небесного, чтобы погиб один из малых сих.
\rsbpar\vs Mat 18:15 Если же согрешит против тебя брат твой, пойди и обличи его между тобою и им одним; если послушает тебя, то приобрел ты брата твоего;
\vs Mat 18:16 если же не послушает, возьми с собою еще одного или двух, дабы устами двух или трех свидетелей подтвердилось всякое слово;
\vs Mat 18:17 если же не послушает их, скажи церкви; а если и церкви не послушает, то да будет он тебе, как язычник и мыт\acc{а}рь.
\vs Mat 18:18 Истинно говорю вам: чт\acc{о} вы свяжете на земле, т\acc{о} будет связано на небе; и чт\acc{о} разрешите на земле, т\acc{о} будет разрешено на небе.
\vs Mat 18:19 Истинно также говорю вам, что если двое из вас согласятся на земле просить о всяком деле, то, чего бы ни попросили, будет им от Отца Моего Небесного,
\vs Mat 18:20 ибо, где двое или трое собраны во имя Мое, там Я посреди них.
\rsbpar\vs Mat 18:21 Тогда Петр приступил к Нему и сказал: Господи! сколько раз прощать брату моему, согрешающему против меня? до семи ли раз?
\vs Mat 18:22 Иисус говорит ему: не говорю тебе: до семи, но до седмижды семидесяти раз.
\rsbpar\vs Mat 18:23 Посему Царство Небесное подобно царю, который захотел сосчитаться с рабами своими;
\vs Mat 18:24 когда начал он считаться, приведен был к нему некто, который должен был ему десять тысяч талантов\fns{Вес серебра.};
\vs Mat 18:25 а как он не имел, чем заплатить, то государь его приказал продать его, и жену его, и детей, и всё, что он имел, и заплатить;
\vs Mat 18:26 тогда раб тот пал, и, кланяясь ему, говорил: государь! потерпи на мне, и всё тебе заплачу.
\vs Mat 18:27 Государь, умилосердившись над рабом тем, отпустил его и долг простил ему.
\vs Mat 18:28 Раб же тот, выйдя, нашел одного из товарищей своих, который должен был ему сто динариев, и, схватив его, душил, говоря: отдай мне, чт\acc{о} должен.
\vs Mat 18:29 Тогда товарищ его пал к ногам его, умолял его и говорил: потерпи на мне, и всё отдам тебе.
\vs Mat 18:30 Но тот не захотел, а пошел и посадил его в темницу, пока не отдаст долга.
\vs Mat 18:31 Товарищи его, видев происшедшее, очень огорчились и, придя, рассказали государю своему всё бывшее.
\vs Mat 18:32 Тогда государь его призывает его и говорит: злой раб! весь долг тот я простил тебе, потому что ты упросил меня;
\vs Mat 18:33 не надлежало ли и тебе помиловать товарища твоего, к\acc{а}к и я помиловал тебя?
\vs Mat 18:34 И, разгневавшись, государь его отдал его истязателям, пока не отдаст ему всего долга.
\vs Mat 18:35 Т\acc{а}к и Отец Мой Небесный поступит с вами, если не простит каждый из вас от сердца своего брату своему согрешений его.
\vs Mat 19:1 Когда Иисус окончил слова сии, то вышел из Галилеи и пришел в пределы Иудейские, Заиорданскою стороною.
\vs Mat 19:2 За Ним последовало много людей, и Он исцелил их там.
\rsbpar\vs Mat 19:3 И приступили к Нему фарисеи и, искушая Его, говорили Ему: по всякой ли причине позволительно человеку разводиться с женою своею?
\vs Mat 19:4 Он сказал им в ответ: не читали ли вы, что Сотворивший вначале мужчину и женщину сотворил их?
\vs Mat 19:5 И сказал: посему оставит человек отца и мать и прилепится к жене своей, и будут два одною плотью,
\vs Mat 19:6 так что они уже не двое, но одна плоть. Итак, что Бог сочетал, того человек да не разлучает.
\vs Mat 19:7 Они говорят Ему: как же Моисей заповедал давать разводное письмо и разводиться с нею?
\vs Mat 19:8 Он говорит им: Моисей по жестокосердию вашему позволил вам разводиться с женами вашими, а сначала не было так;
\vs Mat 19:9 но Я говорю вам: кто разведется с женою своею не за прелюбодеяние и женится на другой, \bibemph{тот} прелюбодействует; и женившийся на разведенной прелюбодействует.
\vs Mat 19:10 Говорят Ему ученики Его: если такова обязанность человека к жене, то лучше не жениться.
\vs Mat 19:11 Он же сказал им: не все вмещают слово сие, но кому дано,
\vs Mat 19:12 ибо есть скопцы, которые из чрева матернего родились так; и есть скопцы, которые оскоплены от людей; и есть скопцы, которые сделали сами себя скопцами для Царства Небесного. Кто может вместить, да вместит.
\rsbpar\vs Mat 19:13 Тогда приведены были к Нему дети, чтобы Он возложил на них руки и помолился; ученики же возбраняли им.
\vs Mat 19:14 Но Иисус сказал: пустите детей и не препятствуйте им приходить ко Мне, ибо таковых есть Царство Небесное.
\vs Mat 19:15 И, возложив на них руки, пошел оттуда.
\rsbpar\vs Mat 19:16 И вот, некто, подойдя, сказал Ему: Учитель благий! что сделать мне доброго, чтобы иметь жизнь вечную?
\vs Mat 19:17 Он же сказал ему: что ты называешь Меня благим? Никто не благ, как только один Бог. Если же хочешь войти в жизнь \bibemph{вечную}, соблюди заповеди.
\vs Mat 19:18 Говорит Ему: какие? Иисус же сказал: не убивай; не прелюбодействуй; не кради; не лжесвидетельствуй;
\vs Mat 19:19 почитай отца и мать; и: люби ближнего твоего, как самого себя.
\vs Mat 19:20 Юноша говорит Ему: всё это сохранил я от юности моей; чего еще недостает мне?
\vs Mat 19:21 Иисус сказал ему: если хочешь быть совершенным, пойди, продай имение твое и раздай нищим; и будешь иметь сокровище на небесах; и приходи и следуй за Мною.
\vs Mat 19:22 Услышав слово сие, юноша отошел с печалью, потому что у него было большое имение.
\vs Mat 19:23 Иисус же сказал ученикам Своим: истинно говорю вам, что трудно богатому войти в Царство Небесное;
\vs Mat 19:24 и еще говорю вам: удобнее верблюду пройти сквозь игольные уши, нежели богатому войти в Царство Божие.
\vs Mat 19:25 Услышав это, ученики Его весьма изумились и сказали: так кто же может спастись?
\vs Mat 19:26 А Иисус, воззрев, сказал им: человекам это невозможно, Богу же всё возможно.
\rsbpar\vs Mat 19:27 Тогда Петр, отвечая, сказал Ему: вот, мы оставили всё и последовали за Тобою; что же будет нам?
\vs Mat 19:28 Иисус же сказал им: истинно говорю вам, что вы, последовавшие за Мною,~--- в пакибытии, когда сядет Сын Человеческий на престоле славы Своей, сядете и вы на двенадцати престолах судить двенадцать колен Израилевых.
\vs Mat 19:29 И всякий, кто оставит д\acc{о}мы, или братьев, или сестер, или отца, или мать, или жену, или детей, или з\acc{е}мли, ради имени Моего, получит во сто крат и наследует жизнь вечную.
\vs Mat 19:30 Многие же будут первые последними, и последние первыми.
\vs Mat 20:1 Ибо Царство Небесное подобно хозяину дома, который вышел рано поутру нанять работников в виноградник свой
\vs Mat 20:2 и, договорившись с работниками по динарию на день, послал их в виноградник свой;
\vs Mat 20:3 выйдя около третьего часа, он увидел других, стоящих на торжище праздно,
\vs Mat 20:4 и им сказал: идите и вы в виноградник мой, и чт\acc{о} следовать будет, дам вам. Они пошли.
\vs Mat 20:5 Опять выйдя около шестого и девятого часа, сделал т\acc{о} же.
\vs Mat 20:6 Наконец, выйдя около одиннадцатого часа, он нашел других, стоящих праздно, и говорит им: чт\acc{о} вы стоите здесь целый день праздно?
\vs Mat 20:7 Они говорят ему: никто нас не нанял. Он говорит им: идите и вы в виноградник мой, и чт\acc{о} следовать будет, пол\acc{у}чите.
\vs Mat 20:8 Когда же наступил вечер, говорит господин виноградника управителю своему: позови работников и отдай им плату, начав с последних до первых.
\vs Mat 20:9 И пришедшие около одиннадцатого часа получили по динарию.
\vs Mat 20:10 Пришедшие же первыми думали, что они получат больше, но получили и они по динарию;
\vs Mat 20:11 и, получив, стали роптать на хозяина дома
\vs Mat 20:12 и говорили: эти последние работали один час, и ты сравнял их с нами, перенесшими тягость дня и зной.
\vs Mat 20:13 Он же в ответ сказал одному из них: друг! я не обижаю тебя; не за динарий ли ты договорился со мною?
\vs Mat 20:14 возьми свое и пойди; я же хочу дать этому последнему \bibemph{т\acc{о} же}, чт\acc{о} и тебе;
\vs Mat 20:15 разве я не властен в своем делать, чт\acc{о} хочу? или глаз твой завистлив оттого, что я добр?
\vs Mat 20:16 Так будут последние первыми, и первые последними, ибо много званых, а мало избранных.
\rsbpar\vs Mat 20:17 И, восходя в Иерусалим, Иисус дорогою отозвал двенадцать учеников одних, и сказал им:
\vs Mat 20:18 вот, мы восходим в Иерусалим, и Сын Человеческий предан будет первосвященникам и книжникам, и осудят Его на смерть;
\vs Mat 20:19 и предадут Его язычникам на поругание и биение и распятие; и в третий день воскреснет.
\rsbpar\vs Mat 20:20 Тогда приступила к Нему мать сыновей Зеведеевых с сыновьями своими, кланяясь и чего-то прося у Него.
\vs Mat 20:21 Он сказал ей: чего ты хочешь? Она говорит Ему: скажи, чтобы сии два сына мои сели у Тебя один по правую сторону, а другой по левую в Царстве Твоем.
\vs Mat 20:22 Иисус сказал в ответ: не знаете, чего просите. Можете ли пить чашу, которую Я буду пить, или креститься крещением, которым Я крещусь? Они говорят Ему: можем.
\vs Mat 20:23 И говорит им: чашу Мою будете пить, и крещением, которым Я крещусь, будете креститься, но дать сесть у Меня по правую сторону и по левую~--- не от Меня \bibemph{зависит}, но кому уготовано Отцем Моим.
\vs Mat 20:24 Услышав \bibemph{сие, прочие} десять \bibemph{учеников} вознегодовали на двух братьев.
\vs Mat 20:25 Иисус же, подозвав их, сказал: вы знаете, что князья народов господствуют над ними, и вельможи властвуют ими;
\vs Mat 20:26 но между вами да не будет так: а кто хочет между вами быть б\acc{о}льшим, да будет вам слугою;
\vs Mat 20:27 и кто хочет между вами быть первым, да будет вам рабом;
\vs Mat 20:28 так как Сын Человеческий не \bibemph{для того} пришел, чтобы Ему служили, но чтобы послужить и отдать душу Свою для искупления многих.
\rsbpar\vs Mat 20:29 И когда выходили они из Иерихона, за Ним следовало множество народа.
\vs Mat 20:30 И вот, двое слепых, сидевшие у дороги, услышав, что Иисус идет мимо, начали кричать: помилуй нас, Господи, Сын Давидов!
\vs Mat 20:31 Народ же заставлял их молчать; но они еще громче стали кричать: помилуй нас, Господи, Сын Давидов!
\vs Mat 20:32 Иисус, остановившись, подозвал их и сказал: чего вы хотите от Меня?
\vs Mat 20:33 Они говорят Ему: Господи! чтобы открылись глаза наши.
\vs Mat 20:34 Иисус же, умилосердившись, прикоснулся к глазам их; и тотчас прозрели глаза их, и они пошли за Ним.
\vs Mat 21:1 И когда приблизились к Иерусалиму и пришли в Виффагию к горе Елеонской, тогда Иисус послал двух учеников,
\vs Mat 21:2 сказав им: пойдите в селение, которое прямо перед вами; и тотчас найдете ослицу привязанную и молодого осла с нею; отвязав, приведите ко Мне;
\vs Mat 21:3 и если кто скажет вам что-нибудь, отвечайте, что они надобны Господу; и тотчас пошлет их.
\vs Mat 21:4 Всё же сие было, да сбудется реченное через пророка, который говорит:
\vs Mat 21:5 Скажите дщери Сионовой: се, Царь твой грядет к тебе кроткий, сидя на ослице и молодом осле, сыне подъяремной.
\vs Mat 21:6 Ученики пошли и поступили так, как повелел им Иисус:
\vs Mat 21:7 привели ослицу и молодого осла и положили на них одежды свои, и Он сел поверх их.
\vs Mat 21:8 Множество же народа постилали свои одежды по дороге, а другие резали ветви с дерев и постилали по дороге;
\vs Mat 21:9 народ же, предшествовавший и сопровождавший, восклицал: осанна\fns{Спасение.} Сыну Давидову! благословен Грядущий во имя Господне! осанна в вышних!
\vs Mat 21:10 И когда вошел Он в Иерусалим, весь город пришел в движение и говорил: кто Сей?
\vs Mat 21:11 Народ же говорил: Сей есть Иисус, Пророк из Назарета Галилейского.
\rsbpar\vs Mat 21:12 И вошел Иисус в храм Божий и выгнал всех продающих и покупающих в храме, и опрокинул столы меновщиков и скамьи продающих голубей,
\vs Mat 21:13 и говорил им: написано,~--- дом Мой домом молитвы наречется; а вы сделали его вертепом разбойников.
\vs Mat 21:14 И приступили к Нему в храме слепые и хромые, и Он исцелил их.
\vs Mat 21:15 Видев же первосвященники и книжники чудеса, которые Он сотворил, и детей, восклицающих в храме и говорящих: осанна Сыну Давидову!~--- вознегодовали
\vs Mat 21:16 и сказали Ему: слышишь ли, что они говорят? Иисус же говорит им: да! разве вы никогда не читали: из уст младенцев и грудных детей Ты устроил хвалу?
\vs Mat 21:17 И, оставив их, вышел вон из города в Вифанию и провел там ночь.
\rsbpar\vs Mat 21:18 Поутру же, возвращаясь в город, взалкал;
\vs Mat 21:19 и увидев при дороге одну смоковницу, подошел к ней и, ничего не найдя на ней, кроме одних листьев, говорит ей: да не будет же впредь от тебя плода вовек. И смоковница тотчас засохла.
\vs Mat 21:20 Увидев это, ученики удивились и говорили: как это тотчас засохла смоковница?
\vs Mat 21:21 Иисус же сказал им в ответ: истинно говорю вам, если будете иметь веру и не усомнитесь, не только сделаете т\acc{о}, чт\acc{о} \bibemph{сделано} со смоковницею, но если и горе сей скажете: поднимись и ввергнись в море,~--- будет;
\vs Mat 21:22 и всё, чего ни попросите в молитве с верою, пол\acc{у}чите.
\rsbpar\vs Mat 21:23 И когда пришел Он в храм и учил, приступили к Нему первосвященники и старейшины народа и сказали: какою властью Ты это делаешь? и кто Тебе дал такую власть?
\vs Mat 21:24 Иисус сказал им в ответ: спрошу и Я вас об одном; если о том скажете Мне, то и Я вам скажу, какою властью это делаю;
\vs Mat 21:25 крещение Иоанново откуда было: с небес, или от человеков? Они же рассуждали между собою: если скажем: с небес, то Он скажет нам: почему же вы не поверили ему?
\vs Mat 21:26 а если сказать: от человеков,~--- боимся народа, ибо все почитают Иоанна за пророка.
\vs Mat 21:27 И сказали в ответ Иисусу: не знаем. Сказал им и Он: и Я вам не скажу, какою властью это делаю.
\rsbpar\vs Mat 21:28 А к\acc{а}к вам кажется? У одного человека было два сына; и он, подойдя к первому, сказал: сын! пойди сегодня работай в винограднике моем.
\vs Mat 21:29 Но он сказал в ответ: не хочу; а после, раскаявшись, пошел.
\vs Mat 21:30 И подойдя к другому, он сказал т\acc{о} же. Этот сказал в ответ: иду, государь, и не пошел.
\vs Mat 21:31 Который из двух исполнил волю отца? Говорят Ему: первый. Иисус говорит им: истинно говорю вам, что мытари и блудницы вперед вас идут в Царство Божие,
\vs Mat 21:32 ибо пришел к вам Иоанн путем праведности, и вы не поверили ему, а мытари и блудницы поверили ему; вы же, и видев это, не раскаялись после, чтобы поверить ему.
\rsbpar\vs Mat 21:33 Выслушайте другую притчу: был некоторый хозяин дома, который насадил виноградник, обнес его оградою, выкопал в нем точило, построил башню и, отдав его виноградарям, отлучился.
\vs Mat 21:34 Когда же приблизилось время плодов, он послал своих слуг к виноградарям взять свои плоды;
\vs Mat 21:35 виноградари, схватив слуг его, иного прибили, иного убили, а иного побили камнями.
\vs Mat 21:36 Опять послал он других слуг, больше прежнего; и с ними поступили так же.
\vs Mat 21:37 Наконец, послал он к ним своего сына, говоря: постыдятся сына моего.
\vs Mat 21:38 Но виноградари, увидев сына, сказали друг другу: это наследник; пойдем, убьем его и завладеем наследством его.
\vs Mat 21:39 И, схватив его, вывели вон из виноградника и убили.
\vs Mat 21:40 Итак, когда придет хозяин виноградника, что сделает он с этими виноградарями?
\vs Mat 21:41 Говорят Ему: злодеев сих предаст злой смерти, а виноградник отдаст другим виноградарям, которые будут отдавать ему плоды во времена свои.
\vs Mat 21:42 Иисус говорит им: неужели вы никогда не читали в Писании: камень, который отвергли строители, тот самый сделался главою угла? Это от Господа, и есть дивно в очах наших?
\vs Mat 21:43 Потому сказываю вам, что отнимется от вас Царство Божие и дано будет народу, приносящему плоды его;
\vs Mat 21:44 и тот, кто упадет на этот камень, разобьется, а на кого он упадет, того раздавит.
\vs Mat 21:45 И слышав притчи Его, первосвященники и фарисеи поняли, что Он о них говорит,
\vs Mat 21:46 и старались схватить Его, но побоялись народа, потому что Его почитали за Пророка.
\vs Mat 22:1 Иисус, продолжая говорить им притчами, сказал:
\vs Mat 22:2 Царство Небесное подобно человеку царю, который сделал брачный пир для сына своего
\vs Mat 22:3 и послал рабов своих звать званых на брачный пир; и не хотели прийти.
\vs Mat 22:4 Опять послал других рабов, сказав: скажите званым: вот, я приготовил обед мой, тельцы мои и что откормлено, заколото, и всё готово; приходите на брачный пир.
\vs Mat 22:5 Но они, пренебрегши то, пошли, кто на поле свое, а кто на торговлю свою;
\vs Mat 22:6 прочие же, схватив рабов его, оскорбили и убили \bibemph{их}.
\vs Mat 22:7 Услышав о сем, царь разгневался, и, послав войск\acc{а} свои, истребил убийц оных и сжег город их.
\vs Mat 22:8 Тогда говорит он рабам своим: брачный пир готов, а званые не были достойны;
\vs Mat 22:9 итак пойдите на распутия и всех, кого найдете, зовите на брачный пир.
\vs Mat 22:10 И рабы те, выйдя на дороги, собрали всех, кого только нашли, и злых и добрых; и брачный пир наполнился возлежащими.
\vs Mat 22:11 Царь, войдя посмотреть возлежащих, увидел там человека, одетого не в брачную одежду,
\vs Mat 22:12 и говорит ему: друг! как ты вошел сюда не в брачной одежде? Он же молчал.
\vs Mat 22:13 Тогда сказал царь слугам: связав ему руки и ноги, возьмите его и бросьте во тьму внешнюю; там будет плач и скрежет зубов;
\vs Mat 22:14 ибо много званых, а мало избранных.
\rsbpar\vs Mat 22:15 Тогда фарисеи пошли и совещались, как бы уловить Его в словах.
\vs Mat 22:16 И посылают к Нему учеников своих с иродианами, говоря: Учитель! мы знаем, что Ты справедлив, и истинно пути Божию учишь, и не заботишься об угождении кому-либо, ибо не смотришь ни на какое лице;
\vs Mat 22:17 итак скажи нам: как Тебе кажется? позволительно ли давать подать кесарю, или нет?
\vs Mat 22:18 Но Иисус, видя лукавство их, сказал: что искушаете Меня, лицемеры?
\vs Mat 22:19 покажите Мне монету, которою платится подать. Они принесли Ему динарий.
\vs Mat 22:20 И говорит им: чье это изображение и надпись?
\vs Mat 22:21 Говорят Ему: кесаревы. Тогда говорит им: итак отдавайте кесарево кесарю, а Божие Богу.
\vs Mat 22:22 Услышав это, они удивились и, оставив Его, ушли.
\rsbpar\vs Mat 22:23 В тот день приступили к Нему саддукеи, которые говорят, что нет воскресения, и спросили Его:
\vs Mat 22:24 Учитель! Моисей сказал: если кто умрет, не имея детей, то брат его пусть возьмет за себя жену его и восстановит семя брату своему;
\vs Mat 22:25 было у нас семь братьев; первый, женившись, умер и, не имея детей, оставил жену свою брату своему;
\vs Mat 22:26 подобно и второй, и третий, даже до седьмого;
\vs Mat 22:27 после же всех умерла и жена;
\vs Mat 22:28 итак, в воскресении, которого из семи будет она женою? ибо все имели ее.
\vs Mat 22:29 Иисус сказал им в ответ: заблуждаетесь, не зная Писаний, ни силы Божией,
\vs Mat 22:30 ибо в воскресении ни женятся, ни выходят замуж, но пребывают, как Ангелы Божии на небесах.
\vs Mat 22:31 А о воскресении мертвых не читали ли вы реченного вам Богом:
\vs Mat 22:32 Я Бог Авраама, и Бог Исаака, и Бог Иакова? Бог не есть Бог мертвых, но живых.
\vs Mat 22:33 И, слыша, народ дивился учению Его.
\rsbpar\vs Mat 22:34 А фарисеи, услышав, что Он привел саддукеев в молчание, собрались вместе.
\vs Mat 22:35 И один из них, законник, искушая Его, спросил, говоря:
\vs Mat 22:36 Учитель! какая наибольшая заповедь в законе?
\vs Mat 22:37 Иисус сказал ему: возлюби Господа Бога твоего всем сердцем твоим и всею душею твоею и всем разумением твоим:
\vs Mat 22:38 сия есть первая и наибольшая заповедь;
\vs Mat 22:39 вторая же подобная ей: возлюби ближнего твоего, как самого себя;
\vs Mat 22:40 на сих двух заповедях утверждается весь закон и пророки.
\rsbpar\vs Mat 22:41 Когда же собрались фарисеи, Иисус спросил их:
\vs Mat 22:42 чт\acc{о} вы думаете о Христе? чей Он сын? Говорят Ему: Давидов.
\vs Mat 22:43 Говорит им: к\acc{а}к же Давид, по вдохновению, называет Его Господом, когда говорит:
\vs Mat 22:44 сказал Господь Господу моему: седи одесную Меня, доколе положу врагов Твоих в подножие ног Твоих?
\vs Mat 22:45 Итак, если Давид называет Его Господом, как же Он сын ему?
\vs Mat 22:46 И никто не мог отвечать Ему ни слова; и с того дня никто уже не смел спрашивать Его.
\vs Mat 23:1 Тогда Иисус начал говорить народу и ученикам Своим
\vs Mat 23:2 и сказал: на Моисеевом седалище сели книжники и фарисеи;
\vs Mat 23:3 итак всё, что они велят вам соблюдать, соблюдайте и делайте; по делам же их не поступайте, ибо они говорят, и не делают:
\vs Mat 23:4 связывают бремена тяжелые и неудобоносимые и возлагают на плечи людям, а сами не хотят и перстом двинуть их;
\vs Mat 23:5 все же дела свои делают с тем, чтобы видели их люди: расширяют хранилища\fns{Повязки на лбу и на руках со словами из закона.} свои и увеличивают воскрилия одежд своих;
\vs Mat 23:6 также любят предвозлежания на пиршествах и председания в синагогах
\vs Mat 23:7 и приветствия в народных собраниях, и чтобы люди звали их: учитель! учитель!
\vs Mat 23:8 А вы не называйтесь учителями, ибо один у вас Учитель~--- Христос, все же вы~--- братья;
\vs Mat 23:9 и отцом себе не называйте никого на земле, ибо один у вас Отец, Который на небесах;
\vs Mat 23:10 и не называйтесь наставниками, ибо один у вас Наставник~--- Христос.
\vs Mat 23:11 Больший из вас да будет вам слуга:
\vs Mat 23:12 ибо, кто возвышает себя, тот унижен будет, а кто унижает себя, тот возвысится.
\rsbpar\vs Mat 23:13 Горе вам, книжники и фарисеи, лицемеры, что затворяете Царство Небесное человекам, ибо сами не вх\acc{о}дите и хотящих войти не допускаете.
\vs Mat 23:14 Горе вам, книжники и фарисеи, лицемеры, что поедаете домы вдов и лицемерно долго м\acc{о}литесь: за т\acc{о} примете тем б\acc{о}льшее осуждение.
\vs Mat 23:15 Горе вам, книжники и фарисеи, лицемеры, что обходите море и сушу, дабы обратить хотя одного; и когда это случится, делаете его сыном геенны, вдвое худшим вас.
\vs Mat 23:16 Горе вам, вожди слепые, которые говорите: если кто поклянется храмом, то ничего, а если кто поклянется золотом храма, то повинен.
\vs Mat 23:17 Безумные и слепые! что больше: золото, или храм, освящающий золото?
\vs Mat 23:18 Также: если кто поклянется жертвенником, то ничего, если же кто поклянется даром, который на нем, то повинен.
\vs Mat 23:19 Безумные и слепые! что больше: дар, или жертвенник, освящающий дар?
\vs Mat 23:20 Итак клянущийся жертвенником клянется им и всем, что на нем;
\vs Mat 23:21 и клянущийся храмом клянется им и Живущим в нем;
\vs Mat 23:22 и клянущийся небом клянется Престолом Божиим и Сидящим на нем.
\vs Mat 23:23 Горе вам, книжники и фарисеи, лицемеры, что даете десятину с мяты, аниса и тмина, и оставили важнейшее в законе: суд, милость и веру; сие надлежало делать, и того не оставлять.
\vs Mat 23:24 Вожди слепые, оцеживающие комара, а верблюда поглощающие!
\vs Mat 23:25 Горе вам, книжники и фарисеи, лицемеры, что очищаете внешность чаши и блюда, между тем как внутри они полны хищения и неправды.
\vs Mat 23:26 Фарисей слепой! очисти прежде внутренность чаши и блюда, чтобы чиста была и внешность их.
\vs Mat 23:27 Горе вам, книжники и фарисеи, лицемеры, что уподобляетесь окрашенным гробам, которые снаружи кажутся красивыми, а внутри полны костей мертвых и всякой нечистоты;
\vs Mat 23:28 так и вы по наружности кажетесь людям праведными, а внутри исполнены лицемерия и беззакония.
\vs Mat 23:29 Горе вам, книжники и фарисеи, лицемеры, что строите гробницы пророкам и украшаете памятники праведников,
\vs Mat 23:30 и говорите: если бы мы были во дни отцов наших, то не были бы сообщниками их в \bibemph{пролитии} крови пророков;
\vs Mat 23:31 таким образом вы сами против себя свидетельствуете, что вы сыновья тех, которые избили пророков;
\vs Mat 23:32 дополняйте же меру отцов ваших.
\vs Mat 23:33 Змии, порождения ехиднины! как убежите вы от осуждения в геенну?
\vs Mat 23:34 Посему, вот, Я посылаю к вам пророков, и мудрых, и книжников; и вы иных убьете и распнете, а иных будете бить в синагогах ваших и гнать из города в город;
\vs Mat 23:35 да придет на вас вся кровь праведная, пролитая на земле, от крови Авеля праведного до крови Захарии, сына Варахиина, которого вы убили между храмом и жертвенником.
\vs Mat 23:36 Истинно говорю вам, что всё сие придет на род сей.
\vs Mat 23:37 Иерусалим, Иерусалим, избивающий пророков и камнями побивающий посланных к тебе! сколько раз хотел Я собрать детей твоих, как птица собирает птенцов своих под крылья, и вы не захотели!
\vs Mat 23:38 Се, оставляется вам дом ваш пуст.
\vs Mat 23:39 Ибо сказываю вам: не увидите Меня отныне, доколе не воскликнете: благословен Грядый во имя Господне!
\vs Mat 24:1 И выйдя, Иисус шел от храма; и приступили ученики Его, чтобы показать Ему здания храма.
\vs Mat 24:2 Иисус же сказал им: видите ли всё это? Истинно говорю вам: не останется здесь камня на камне; всё будет разрушено.
\rsbpar\vs Mat 24:3 Когда же сидел Он на горе Елеонской, то приступили к Нему ученики наедине и спросили: скажи нам, когда это будет? и какой признак Твоего пришествия и кончины века?
\vs Mat 24:4 Иисус сказал им в ответ: берегитесь, чтобы кто не прельстил вас,
\vs Mat 24:5 ибо многие придут под именем Моим, и будут говорить: <<я Христос>>, и многих прельстят.
\vs Mat 24:6 Также услышите о войнах и о военных слухах. Смотрите, не ужасайтесь, ибо надлежит всему тому быть, но это еще не конец:
\vs Mat 24:7 ибо восстанет народ на народ, и царство на царство; и будут глады, моры и землетрясения по местам;
\vs Mat 24:8 всё же это~--- начало болезней.
\vs Mat 24:9 Тогда будут предавать вас на мучения и убивать вас; и вы будете ненавидимы всеми народами за имя Мое;
\vs Mat 24:10 и тогда соблазнятся многие, и друг друга будут предавать, и возненавидят друг друга;
\vs Mat 24:11 и многие лжепророки восстанут, и прельстят многих;
\vs Mat 24:12 и, по причине умножения беззакония, во многих охладеет любовь;
\vs Mat 24:13 претерпевший же до конца спасется.
\vs Mat 24:14 И проповедано будет сие Евангелие Царствия по всей вселенной, во свидетельство всем народам; и тогда придет конец.
\vs Mat 24:15 Итак, когда увидите мерзость запустения, реченную через пророка Даниила, стоящую на святом месте,~--- читающий да разумеет,~---
\vs Mat 24:16 тогда находящиеся в Иудее да бегут в горы;
\vs Mat 24:17 и кто на кровле, тот да не сходит взять что-нибудь из дома своего;
\vs Mat 24:18 и кто на поле, тот да не обращается назад взять одежды свои.
\vs Mat 24:19 Горе же беременным и питающим сосцами в те дни!
\vs Mat 24:20 Мол\acc{и}тесь, чтобы не случилось бегство ваше зимою или в субботу,
\vs Mat 24:21 ибо тогда будет великая скорбь, какой не было от начала мира доныне, и не будет.
\vs Mat 24:22 И если бы не сократились те дни, то не спаслась бы никакая плоть; но ради избранных сократятся те дни.
\vs Mat 24:23 Тогда, если кто скажет вам: вот, здесь Христос, или там,~--- не верьте.
\vs Mat 24:24 Ибо восстанут лжехристы и лжепророки, и дадут великие знамения и чудеса, чтобы прельстить, если возможно, и избранных.
\vs Mat 24:25 Вот, Я наперед сказал вам.
\vs Mat 24:26 Итак, если скажут вам: <<вот, \bibemph{Он} в пустыне>>,~--- не выход\acc{и}те; <<вот, \bibemph{Он} в потаенных комнатах>>,~--- не верьте;
\vs Mat 24:27 ибо, как молния исходит от востока и видна бывает даже до запада, так будет пришествие Сына Человеческого;
\vs Mat 24:28 ибо, где будет труп, там соберутся орлы.
\vs Mat 24:29 И вдруг, после скорби дней тех, солнце померкнет, и луна не даст света своего, и звезды спадут с неба, и силы небесные поколеблются;
\vs Mat 24:30 тогда явится знамение Сына Человеческого на небе; и тогда восплачутся все племена земные и увидят Сына Человеческого, грядущего на облаках небесных с силою и славою великою;
\vs Mat 24:31 и пошлет Ангелов Своих с трубою громогласною, и соберут избранных Его от четырех ветров, от края небес до края их.
\vs Mat 24:32 От смоковницы возьмите подобие: когда ветви ее становятся уже мягки и пускают листья, то знаете, что близко лето;
\vs Mat 24:33 так, когда вы увидите всё сие, знайте, что близко, при дверях.
\vs Mat 24:34 Истинно говорю вам: не прейдет род сей, как всё сие будет;
\vs Mat 24:35 небо и земля прейдут, но слова Мои не прейдут.
\vs Mat 24:36 О дне же том и часе никто не знает, ни Ангелы небесные, а только Отец Мой один;
\vs Mat 24:37 но, к\acc{а}к было во дни Ноя, так будет и в пришествие Сына Человеческого:
\vs Mat 24:38 ибо, к\acc{а}к во дни перед потопом ели, пили, женились и выходили замуж, до того дня, как вошел Ной в ковчег,
\vs Mat 24:39 и не думали, пока не пришел потоп и не истребил всех,~--- так будет и пришествие Сына Человеческого;
\vs Mat 24:40 тогда будут двое на поле: один берется, а другой оставляется;
\vs Mat 24:41 две мелющие в жерновах: одна берется, а другая оставляется.
\vs Mat 24:42 Итак бодрствуйте, потому что не знаете, в который час Господь ваш приидет.
\vs Mat 24:43 Но это вы знаете, что, если бы ведал хозяин дома, в какую стражу придет вор, то бодрствовал бы и не дал бы подкопать дома своего.
\vs Mat 24:44 Потому и вы будьте готовы, ибо в который час не думаете, приидет Сын Человеческий.
\vs Mat 24:45 Кт\acc{о} же верный и благоразумный раб, которого господин его поставил над слугами своими, чтобы давать им пищу во время?
\vs Mat 24:46 Блажен тот раб, которого господин его, придя, найдет поступающим так;
\vs Mat 24:47 истинно говорю вам, что над всем имением своим поставит его.
\vs Mat 24:48 Если же раб тот, будучи зол, скажет в сердце своем: не скоро придет господин мой,
\vs Mat 24:49 и начнет бить товарищей своих и есть и пить с пьяницами,~---
\vs Mat 24:50 то придет господин раба того в день, в который он не ожидает, и в час, в который не думает,
\vs Mat 24:51 и рассечет его, и подвергнет его одной участи с лицемерами; там будет плач и скрежет зубов.
\vs Mat 25:1 Тогда подобно будет Царство Небесное десяти девам, которые, взяв светильники свои, вышли навстречу жениху.
\vs Mat 25:2 Из них пять было мудрых и пять неразумных.
\vs Mat 25:3 Неразумные, взяв светильники свои, не взяли с собою масла.
\vs Mat 25:4 Мудрые же, вместе со светильниками своими, взяли масла в сосудах своих.
\vs Mat 25:5 И как жених замедлил, то задремали все и уснули.
\vs Mat 25:6 Но в полночь раздался крик: вот, жених идет, выходите навстречу ему.
\vs Mat 25:7 Тогда встали все девы те и поправили светильники свои.
\vs Mat 25:8 Неразумные же сказали мудрым: дайте нам вашего масла, потому что светильники наши гаснут.
\vs Mat 25:9 А мудрые отвечали: чтобы не случилось недостатка и у нас и у вас, пойдите лучше к продающим и купите себе.
\vs Mat 25:10 Когда же пошли они покупать, пришел жених, и готовые вошли с ним на брачный пир, и двери затворились;
\vs Mat 25:11 после приходят и прочие девы, и говорят: Господи! Господи! отвори нам.
\vs Mat 25:12 Он же сказал им в ответ: истинно говорю вам: не знаю вас.
\vs Mat 25:13 Итак, бодрствуйте, потому что не знаете ни дня, ни часа, в который приидет Сын Человеческий.
\rsbpar\vs Mat 25:14 Ибо \bibemph{Он поступит}, как человек, который, отправляясь в чужую страну, призвал рабов своих и поручил им имение свое:
\vs Mat 25:15 и одному дал он пять талантов, другому два, иному один, каждому по его силе; и тотчас отправился.
\vs Mat 25:16 Получивший пять талантов пошел, употребил их в дело и приобрел другие пять талантов;
\vs Mat 25:17 точно так же и получивший два таланта приобрел другие два;
\vs Mat 25:18 получивший же один талант пошел и закопал \bibemph{его} в землю и скрыл серебро господина своего.
\vs Mat 25:19 По долгом времени, приходит господин рабов тех и требует у них отчета.
\vs Mat 25:20 И, подойдя, получивший пять талантов принес другие пять талантов и говорит: господин! пять талантов ты дал мне; вот, другие пять талантов я приобрел на них.
\vs Mat 25:21 Господин его сказал ему: хорошо, добрый и верный раб! в малом ты был верен, над многим тебя поставлю; войди в радость господина твоего.
\vs Mat 25:22 Подошел также и получивший два таланта и сказал: господин! два таланта ты дал мне; вот, другие два таланта я приобрел на них.
\vs Mat 25:23 Господин его сказал ему: хорошо, добрый и верный раб! в малом ты был верен, над многим тебя поставлю; войди в радость господина твоего.
\vs Mat 25:24 Подошел и получивший один талант и сказал: господин! я знал тебя, что ты человек жестокий, жнешь, где не сеял, и собираешь, где не рассып\acc{а}л,
\vs Mat 25:25 и, убоявшись, пошел и скрыл талант твой в земле; вот тебе твое.
\vs Mat 25:26 Господин же его сказал ему в ответ: лукавый раб и ленивый! ты знал, что я жну, где не сеял, и собираю, где не рассып\acc{а}л;
\vs Mat 25:27 посему надлежало тебе отдать серебро мое торгующим, и я, придя, получил бы мое с прибылью;
\vs Mat 25:28 итак, возьмите у него талант и дайте имеющему десять талантов,
\vs Mat 25:29 ибо всякому имеющему дастся и приумножится, а у неимеющего отнимется и т\acc{о}, чт\acc{о} имеет;
\vs Mat 25:30 а негодного раба выбросьте во тьму внешнюю: там будет плач и скрежет зубов. Сказав сие, возгласил: кто имеет уши слышать, да слышит!
\rsbpar\vs Mat 25:31 Когда же приидет Сын Человеческий во славе Своей и все святые Ангелы с Ним, тогда сядет на престоле славы Своей,
\vs Mat 25:32 и соберутся пред Ним все народы; и отделит одних от других, как пастырь отделяет овец от козлов;
\vs Mat 25:33 и поставит овец по правую Свою сторону, а козлов~--- по левую.
\vs Mat 25:34 Тогда скажет Царь тем, которые по правую сторону Его: приидите, благословенные Отца Моего, наследуйте Царство, уготованное вам от создания мира:
\vs Mat 25:35 ибо алкал Я, и вы дали Мне есть; жаждал, и вы напоили Меня; был странником, и вы приняли Меня;
\vs Mat 25:36 был наг, и вы одели Меня; был болен, и вы посетили Меня; в темнице был, и вы пришли ко Мне.
\vs Mat 25:37 Тогда праведники скажут Ему в ответ: Господи! когда мы видели Тебя алчущим, и накормили? или жаждущим, и напоили?
\vs Mat 25:38 когда мы видели Тебя странником, и приняли? или нагим, и одели?
\vs Mat 25:39 когда мы видели Тебя больным, или в темнице, и пришли к Тебе?
\vs Mat 25:40 И Царь скажет им в ответ: истинно говорю вам: так как вы сделали это одному из сих братьев Моих меньших, то сделали Мне.
\vs Mat 25:41 Тогда скажет и тем, которые по левую сторону: идите от Меня, проклятые, в огонь вечный, уготованный диаволу и ангелам его:
\vs Mat 25:42 ибо алкал Я, и вы не дали Мне есть; жаждал, и вы не напоили Меня;
\vs Mat 25:43 был странником, и не приняли Меня; был наг, и не одели Меня; болен и в темнице, и не посетили Меня.
\vs Mat 25:44 Тогда и они скажут Ему в ответ: Господи! когда мы видели Тебя алчущим, или жаждущим, или странником, или нагим, или больным, или в темнице, и не послужили Тебе?
\vs Mat 25:45 Тогда скажет им в ответ: истинно говорю вам: так как вы не сделали этого одному из сих меньших, то не сделали Мне.
\vs Mat 25:46 И пойдут сии в м\acc{у}ку вечную, а праведники в жизнь вечную.
\vs Mat 26:1 Когда Иисус окончил все слова сии, то сказал ученикам Своим:
\vs Mat 26:2 вы знаете, что через два дня будет Пасха, и Сын Человеческий предан будет на распятие.
\rsbpar\vs Mat 26:3 Тогда собрались первосвященники и книжники и старейшины народа во двор первосвященника, по имени Каиафы,
\vs Mat 26:4 и положили в совете взять Иисуса хитростью и убить;
\vs Mat 26:5 но говорили: только не в праздник, чтобы не сделалось возмущения в народе.
\rsbpar\vs Mat 26:6 Когда же Иисус был в Вифании, в доме Симона прокаженного,
\vs Mat 26:7 приступила к Нему женщина с алавастровым сосудом мира драгоценного и возливала Ему возлежащему на голову.
\vs Mat 26:8 Увидев это, ученики Его вознегодовали и говорили: к чему такая трата?
\vs Mat 26:9 Ибо можно было бы продать это миро за большую цену и дать нищим.
\vs Mat 26:10 Но Иисус, уразумев сие, сказал им: что смущаете женщину? она доброе дело сделала для Меня:
\vs Mat 26:11 ибо нищих всегда имеете с собою, а Меня не всегда имеете;
\vs Mat 26:12 возлив миро сие на тело Мое, она приготовила Меня к погребению;
\vs Mat 26:13 истинно говорю вам: где ни будет проповедано Евангелие сие в целом мире, сказано будет в память ее и о том, что она сделала.
\vs Mat 26:14 Тогда один из двенадцати, называемый Иуда Искариот, пошел к первосвященникам
\vs Mat 26:15 и сказал: что вы дадите мне, и я вам предам Его? Они предложили ему тридцать сребреников;
\vs Mat 26:16 и с того времени он искал удобного случая предать Его.
\vs Mat 26:17 В первый же день опресночный приступили ученики к Иисусу и сказали Ему: где велишь нам приготовить Тебе пасху?
\vs Mat 26:18 Он сказал: пойдите в город к такому-то и скажите ему: Учитель говорит: время Мое близко; у тебя совершу пасху с учениками Моими.
\vs Mat 26:19 Ученики сделали, как повелел им Иисус, и приготовили пасху.
\rsbpar\vs Mat 26:20 Когда же настал вечер, Он возлег с двенадцатью учениками;
\vs Mat 26:21 и когда они ели, сказал: истинно говорю вам, что один из вас предаст Меня.
\vs Mat 26:22 Они весьма опечалились, и начали говорить Ему, каждый из них: не я ли, Господи?
\vs Mat 26:23 Он же сказал в ответ: опустивший со Мною руку в блюдо, этот предаст Меня;
\vs Mat 26:24 впрочем Сын Человеческий идет, как писано о Нем, но горе тому человеку, которым Сын Человеческий предается: лучше было бы этому человеку не родиться.
\vs Mat 26:25 При сем и Иуда, предающий Его, сказал: не я ли, Равв\acc{и}? \bibemph{Иисус} говорит ему: ты сказал.
\rsbpar\vs Mat 26:26 И когда они ели, Иисус взял хлеб и, благословив, преломил и, раздавая ученикам, сказал: приимите, ядите: сие есть Тело Мое.
\vs Mat 26:27 И, взяв чашу и благодарив, подал им и сказал: пейте из нее все,
\vs Mat 26:28 ибо сие есть Кровь Моя Нового Завета, за многих изливаемая во оставление грехов.
\vs Mat 26:29 Сказываю же вам, что отныне не буду пить от плода сего виноградного до того дня, когда буду пить с вами новое \bibemph{вино} в Царстве Отца Моего.
\rsbpar\vs Mat 26:30 И, воспев, пошли на гору Елеонскую.
\vs Mat 26:31 Тогда говорит им Иисус: все вы соблазнитесь о Мне в эту ночь, ибо написано: поражу пастыря, и рассеются овцы стада;
\vs Mat 26:32 по воскресении же Моем предварю вас в Галилее.
\vs Mat 26:33 Петр сказал Ему в ответ: если и все соблазнятся о Тебе, я никогда не соблазнюсь.
\vs Mat 26:34 Иисус сказал ему: истинно говорю тебе, что в эту ночь, прежде нежели пропоет петух, трижды отречешься от Меня.
\vs Mat 26:35 Говорит Ему Петр: хотя бы надлежало мне и умереть с Тобою, не отрекусь от Тебя. Подобное говорили и все ученики.
\rsbpar\vs Mat 26:36 Потом приходит с ними Иисус на место, называемое Гефсимания, и говорит ученикам: посидите тут, пока Я пойду, помолюсь там.
\vs Mat 26:37 И, взяв с Собою Петра и обоих сыновей Зеведеевых, начал скорбеть и тосковать.
\vs Mat 26:38 Тогда говорит им Иисус: душа Моя скорбит смертельно; побудьте здесь и бодрствуйте со Мною.
\vs Mat 26:39 И, отойдя немного, пал на лице Свое, молился и говорил: Отче Мой! если возможно, да минует Меня чаша сия; впрочем не как Я хочу, но как Ты.
\vs Mat 26:40 И приходит к ученикам и находит их спящими, и говорит Петру: т\acc{а}к ли не могли вы один час бодрствовать со Мною?
\vs Mat 26:41 бодрствуйте и молитесь, чтобы не впасть в искушение: дух бодр, плоть же немощна.
\vs Mat 26:42 Еще, отойдя в другой раз, молился, говоря: Отче Мой! если не может чаша сия миновать Меня, чтобы Мне не пить ее, да будет воля Твоя.
\vs Mat 26:43 И, придя, находит их опять спящими, ибо у них глаза отяжелели.
\vs Mat 26:44 И, оставив их, отошел опять и помолился в третий раз, сказав то же слово.
\vs Mat 26:45 Тогда приходит к ученикам Своим и говорит им: вы всё еще спите и почиваете? вот, приблизился час, и Сын Человеческий предается в руки грешников;
\vs Mat 26:46 встаньте, пойдем: вот, приблизился предающий Меня.
\rsbpar\vs Mat 26:47 И, когда еще говорил Он, вот Иуда, один из двенадцати, пришел, и с ним множество народа с мечами и кольями, от первосвященников и старейшин народных.
\vs Mat 26:48 Предающий же Его дал им знак, сказав: Кого я поцелую, Тот и есть, возьмите Его.
\vs Mat 26:49 И, тотчас подойдя к Иисусу, сказал: радуйся, Равв\acc{и}! И поцеловал Его.
\vs Mat 26:50 Иисус же сказал ему: друг, для чего ты пришел? Тогда подошли и возложили руки на Иисуса, и взяли Его.
\vs Mat 26:51 И вот, один из бывших с Иисусом, простерши руку, извлек меч свой и, ударив раба первосвященникова, отсек ему ухо.
\vs Mat 26:52 Тогда говорит ему Иисус: возврати меч твой в его место, ибо все, взявшие меч, мечом погибнут;
\vs Mat 26:53 или думаешь, что Я не могу теперь умолить Отца Моего, и Он представит Мне более, нежели двенадцать легионов Ангелов?
\vs Mat 26:54 как же сбудутся Писания, что т\acc{а}к должно быть?
\vs Mat 26:55 В тот час сказал Иисус народу: как будто на разбойника вышли вы с мечами и кольями взять Меня; каждый день с вами сидел Я, уча в храме, и вы не брали Меня.
\vs Mat 26:56 Сие же всё было, да сбудутся писания пророков. Тогда все ученики, оставив Его, бежали.
\rsbpar\vs Mat 26:57 А взявшие Иисуса отвели Его к Каиафе первосвященнику, куда собрались книжники и старейшины.
\vs Mat 26:58 Петр же следовал за Ним издали, до двора первосвященникова; и, войдя внутрь, сел со служителями, чтобы видеть конец.
\vs Mat 26:59 Первосвященники и старейшины и весь синедрион\fns{Верховное судилище.} искали лжесвидетельства против Иисуса, чтобы предать Его смерти,
\vs Mat 26:60 и не находили; и, хотя много лжесвидетелей приходило, не нашли. Но наконец пришли два лжесвидетеля
\vs Mat 26:61 и сказали: Он говорил: могу разрушить храм Божий и в три дня создать его.
\vs Mat 26:62 И, встав, первосвященник сказал Ему: \bibemph{что же} ничего не отвечаешь? чт\acc{о} они против Тебя свидетельствуют?
\vs Mat 26:63 Иисус молчал. И первосвященник сказал Ему: заклинаю Тебя Богом живым, скажи нам, Ты ли Христос, Сын Божий?
\vs Mat 26:64 Иисус говорит ему: ты сказал; даже сказываю вам: отныне \acc{у}зрите Сына Человеческого, сидящего одесную силы и грядущего на облаках небесных.
\vs Mat 26:65 Тогда первосвященник разодрал одежды свои и сказал: Он богохульствует! на чт\acc{о} еще нам свидетелей? вот, теперь вы слышали богохульство Его!
\vs Mat 26:66 как вам кажется? Они же сказали в ответ: повинен смерти.
\vs Mat 26:67 Тогда плевали Ему в лице и заушали Его; другие же ударяли Его по ланитам
\vs Mat 26:68 и говорили: прореки нам, Христос, кто ударил Тебя?
\rsbpar\vs Mat 26:69 Петр же сидел вне на дворе. И подошла к нему одна служанка и сказала: и ты был с Иисусом Галилеянином.
\vs Mat 26:70 Но он отрекся перед всеми, сказав: не знаю, что ты говоришь.
\vs Mat 26:71 Когда же он выходил за ворота, увидела его другая, и говорит бывшим там: и этот был с Иисусом Назореем.
\vs Mat 26:72 И он опять отрекся с клятвою, что не знает Сего Человека.
\vs Mat 26:73 Немного спустя подошли стоявшие там и сказали Петру: точно и ты из них, ибо и речь твоя обличает тебя.
\vs Mat 26:74 Тогда он начал клясться и божиться, что не знает Сего Человека. И вдруг запел петух.
\vs Mat 26:75 И вспомнил Петр слово, сказанное ему Иисусом: прежде нежели пропоет петух, трижды отречешься от Меня. И выйдя вон, плакал горько.
\vs Mat 27:1 Когда же настало утро, все первосвященники и старейшины народа имели совещание об Иисусе, чтобы предать Его смерти;
\vs Mat 27:2 и, связав Его, отвели и предали Его Понтию Пилату, правителю.
\rsbpar\vs Mat 27:3 Тогда Иуда, предавший Его, увидев, что Он осужден, и, раскаявшись, возвратил тридцать сребреников первосвященникам и старейшинам,
\vs Mat 27:4 говоря: согрешил я, предав кровь невинную. Они же сказали ему: чт\acc{о} нам до того? смотри сам.
\vs Mat 27:5 И, бросив сребреники в храме, он вышел, пошел и удавился.
\vs Mat 27:6 Первосвященники, взяв сребреники, сказали: непозволительно положить их в сокровищницу церковную, потому что это цена крови.
\vs Mat 27:7 Сделав же совещание, купили на них землю горшечника, для погребения странников;
\vs Mat 27:8 посему и называется земля та <<землею крови>> до сего дня.
\vs Mat 27:9 Тогда сбылось реченное через пророка Иеремию, который говорит: и взяли тридцать сребреников, цену Оцененного, Которого оценили сыны Израиля,
\vs Mat 27:10 и дали их за землю горшечника, как сказал мне Господь.
\rsbpar\vs Mat 27:11 Иисус же стал пред правителем. И спросил Его правитель: Ты Царь Иудейский? Иисус сказал ему: ты говоришь.
\vs Mat 27:12 И когда обвиняли Его первосвященники и старейшины, Он ничего не отвечал.
\vs Mat 27:13 Тогда говорит Ему Пилат: не слышишь, сколько свидетельствуют против Тебя?
\vs Mat 27:14 И не отвечал ему ни на одно слово, так что правитель весьма дивился.
\vs Mat 27:15 На праздник же \bibemph{Пасхи} правитель имел обычай отпускать народу одного узника, которого хотели.
\vs Mat 27:16 Был тогда у них известный узник, называемый Варавва;
\vs Mat 27:17 итак, когда собрались они, сказал им Пилат: кого хотите, чтобы я отпустил вам: Варавву, или Иисуса, называемого Христом?
\vs Mat 27:18 ибо знал, что предали Его из зависти.
\vs Mat 27:19 Между тем, как сидел он на судейском месте, жена его послала ему сказать: не делай ничего Праведнику Тому, потому что я ныне во сне много пострадала за Него.
\vs Mat 27:20 Но первосвященники и старейшины возбудили народ просить Варавву, а Иисуса погубить.
\vs Mat 27:21 Тогда правитель спросил их: кого из двух хотите, чтобы я отпустил вам? Они сказали: Варавву.
\vs Mat 27:22 Пилат говорит им: чт\acc{о} же я сделаю Иисусу, называемому Христом? Говорят ему все: да будет распят.
\vs Mat 27:23 Правитель сказал: какое же зло сделал Он? Но они еще сильнее кричали: да будет распят.
\vs Mat 27:24 Пилат, видя, что ничто не помогает, но смятение увеличивается, взял воды и умыл руки перед народом, и сказал: невиновен я в крови Праведника Сего; смотрите вы.
\vs Mat 27:25 И, отвечая, весь народ сказал: кровь Его на нас и на детях наших.
\vs Mat 27:26 Тогда отпустил им Варавву, а Иисуса, бив, предал на распятие.
\rsbpar\vs Mat 27:27 Тогда воины правителя, взяв Иисуса в преторию\fns{Судилище преторское.}, собрали на Него весь полк
\vs Mat 27:28 и, раздев Его, надели на Него багряницу;
\vs Mat 27:29 и, сплетши венец из терна, возложили Ему на голову и дали Ему в правую руку трость; и, становясь пред Ним на колени, насмехались над Ним, говоря: радуйся, Царь Иудейский!
\vs Mat 27:30 и плевали на Него и, взяв трость, били Его по голове.
\rsbpar\vs Mat 27:31 И когда насмеялись над Ним, сняли с Него багряницу, и одели Его в одежды Его, и повели Его на распятие.
\vs Mat 27:32 Выходя, они встретили одного Киринеянина, по имени Симона; сего заставили нести крест Его.
\vs Mat 27:33 И, придя на место, называемое Голгофа, что значит: Лобное место,
\vs Mat 27:34 дали Ему пить уксуса, смешанного с желчью; и, отведав, не хотел пить.
\vs Mat 27:35 Распявшие же Его делили одежды Его, бросая жребий;
\vs Mat 27:36 и, сидя, стерегли Его там;
\vs Mat 27:37 и поставили над головою Его надпись, означающую вину Его: Сей есть Иисус, Царь Иудейский.
\vs Mat 27:38 Тогда распяты с Ним два разбойника: один по правую сторону, а другой по левую.
\vs Mat 27:39 Проходящие же злословили Его, кивая головами своими
\vs Mat 27:40 и говоря: Разрушающий храм и в три дня Созидающий! спаси Себя Самого; если Ты Сын Божий, сойди с креста.
\vs Mat 27:41 Подобно и первосвященники с книжниками и старейшинами и фарисеями, насмехаясь, говорили:
\vs Mat 27:42 других спасал, а Себя Самого не может спасти; если Он Царь Израилев, пусть теперь сойдет с креста, и уверуем в Него;
\vs Mat 27:43 уповал на Бога; пусть теперь избавит Его, если Он угоден Ему. Ибо Он сказал: Я Божий Сын.
\vs Mat 27:44 Также и разбойники, распятые с Ним, поносили Его.
\rsbpar\vs Mat 27:45 От шестого же часа тьма была по всей земле до часа девятого;
\vs Mat 27:46 а около девятого часа возопил Иисус громким голосом: Ил\acc{и}, Ил\acc{и}! лам\acc{а} савахфан\acc{и}? то есть: Боже Мой, Боже Мой! для чего Ты Меня оставил?
\vs Mat 27:47 Некоторые из стоявших там, слыша это, говорили: Илию зовет Он.
\vs Mat 27:48 И тотчас побежал один из них, взял губку, наполнил уксусом и, налож\acc{и}в на трость, давал Ему пить;
\vs Mat 27:49 а другие говорили: постой, посмотрим, придет ли Илия спасти Его.
\vs Mat 27:50 Иисус же, опять возопив громким голосом, испустил дух.
\vs Mat 27:51 И вот, завеса в храме раздралась надвое, сверху донизу; и земля потряслась; и камни расселись;
\vs Mat 27:52 и гробы отверзлись; и многие тела усопших святых воскресли
\vs Mat 27:53 и, выйдя из гробов по воскресении Его, вошли во святый град и явились многим.
\vs Mat 27:54 Сотник же и те, которые с ним стерегли Иисуса, видя землетрясение и все бывшее, устрашились весьма и говорили: воистину Он был Сын Божий.
\vs Mat 27:55 Там были также и смотрели издали многие женщины, которые следовали за Иисусом из Галилеи, служа Ему;
\vs Mat 27:56 между ними были Мария Магдалина и Мария, мать Иакова и Иосии, и мать сыновей Зеведеевых.
\rsbpar\vs Mat 27:57 Когда же настал вечер, пришел богатый человек из Аримафеи, именем Иосиф, который также учился у Иисуса;
\vs Mat 27:58 он, придя к Пилату, просил тела Иисусова. Тогда Пилат приказал отдать тело;
\vs Mat 27:59 и, взяв тело, Иосиф обвил его чистою плащаницею\fns{Полотном.}
\vs Mat 27:60 и положил его в новом своем гробе, который высек он в скале; и, привалив большой камень к двери гроба, удалился.
\vs Mat 27:61 Была же там Мария Магдалина и другая Мария, которые сидели против гроба.
\rsbpar\vs Mat 27:62 На другой день, который следует за пятницею, собрались первосвященники и фарисеи к Пилату
\vs Mat 27:63 и говорили: господин! Мы вспомнили, что обманщик тот, еще будучи в живых, сказал: после трех дней воскресну;
\vs Mat 27:64 итак прикажи охранять гроб до третьего дня, чтобы ученики Его, придя ночью, не украли Его и не сказали народу: воскрес из мертвых; и будет последний обман хуже первого.
\vs Mat 27:65 Пилат сказал им: имеете стражу; пойдите, охраняйте, как знаете.
\vs Mat 27:66 Они пошли и поставили у гроба стражу, и приложили к камню печать.
\vs Mat 28:1 По прошествии же субботы, на рассвете первого дня недели, пришла Мария Магдалина и другая Мария посмотреть гроб.
\rsbpar\vs Mat 28:2 И вот, сделалось великое землетрясение, ибо Ангел Господень, сошедший с небес, приступив, отвалил камень от двери гроба и сидел на нем;
\vs Mat 28:3 вид его был, как молния, и одежда его бела, как снег;
\vs Mat 28:4 устрашившись его, стерегущие пришли в трепет и стали, как мертвые;
\rsbpar\vs Mat 28:5 Ангел же, обратив речь к женщинам, сказал: не бойтесь, ибо знаю, что вы ищете Иисуса распятого;
\vs Mat 28:6 Его нет здесь~--- Он воскрес, как сказал. Подойдите, посмотрите место, где лежал Господь,
\vs Mat 28:7 и пойдите скорее, скажите ученикам Его, что Он воскрес из мертвых и предваряет вас в Галилее; там Его увидите. Вот, я сказал вам.
\vs Mat 28:8 И, выйдя поспешно из гроба, они со страхом и радостью великою побежали возвестить ученикам Его.
\rsbpar\vs Mat 28:9 Когда же шли они возвестить ученикам Его, и се Иисус встретил их и сказал: радуйтесь! И они, приступив, ухватились за ноги Его и поклонились Ему.
\vs Mat 28:10 Тогда говорит им Иисус: не бойтесь; пойдите, возвестите братьям Моим, чтобы шли в Галилею, и там они увидят Меня.
\vs Mat 28:11 Когда же они шли, то некоторые из стражи, войдя в город, объявили первосвященникам о всем бывшем.
\vs Mat 28:12 И сии, собравшись со старейшинами и сделав совещание, довольно денег дали воинам,
\vs Mat 28:13 и сказали: скажите, что ученики Его, придя ночью, украли Его, когда мы спали;
\vs Mat 28:14 и, если слух об этом дойдет до правителя, мы убедим его, и вас от неприятности избавим.
\vs Mat 28:15 Они, взяв деньги, поступили, как научены были; и пронеслось слово сие между Иудеями до сего дня.
\rsbpar\vs Mat 28:16 Одиннадцать же учеников пошли в Галилею, на гору, куда повелел им Иисус,
\vs Mat 28:17 и, увидев Его, поклонились Ему, а иные усомнились.
\vs Mat 28:18 И приблизившись Иисус сказал им: дана Мне всякая власть на небе и на земле.
\vs Mat 28:19 Итак идите, научите все народы, крестя их во имя Отца и Сына и Святаго Духа,
\vs Mat 28:20 уча их соблюдать всё, что Я повелел вам; и се, Я с вами во все дни до скончания века. Аминь.

\bibbookdescr{Mar}{
  inline={От Марка\\\LARGE святое благовествование},
  toc={От Марка},
  bookmark={От Марка},
  header={От Марка},
  %headerleft={},
  %headerright={},
  abbr={Мк}
}
\vs Mar 1:1 Начало Евангелия Иисуса Христа, Сына Божия,
\vs Mar 1:2 как написано у пророков: вот, Я посылаю Ангела Моего пред лицем Твоим, который приготовит путь Твой пред Тобою.
\vs Mar 1:3 Глас вопиющего в пустыне: приготовьте путь Господу, прямыми сделайте стези Ему.
\rsbpar\vs Mar 1:4 Явился Иоанн, крестя в пустыне и проповедуя крещение покаяния для прощения грехов.
\vs Mar 1:5 И выходили к нему вся страна Иудейская и Иерусалимляне, и крестились от него все в реке Иордане, исповедуя грехи свои.
\vs Mar 1:6 Иоанн же носил одежду из верблюжьего волоса и пояс кожаный на чреслах своих, и ел акриды и дикий мед.
\vs Mar 1:7 И проповедовал, говоря: идет за мною Сильнейший меня, у Которого я недостоин, наклонившись, развязать ремень обуви Его;
\vs Mar 1:8 я крестил вас водою, а Он будет крестить вас Духом Святым.
\rsbpar\vs Mar 1:9 И было в те дни, пришел Иисус из Назарета Галилейского и крестился от Иоанна в Иордане.
\vs Mar 1:10 И когда выходил из воды, тотчас увидел \bibemph{Иоанн} разверзающиеся небеса и Духа, как голубя, сходящего на Него.
\vs Mar 1:11 И глас был с небес: Ты Сын Мой возлюбленный, в Котором Мое благоволение.
\rsbpar\vs Mar 1:12 Немедленно после того Дух ведет Его в пустыню.
\vs Mar 1:13 И был Он там в пустыне сорок дней, искушаемый сатаною, и был со зверями; и Ангелы служили Ему.
\rsbpar\vs Mar 1:14 После же того, как предан был Иоанн, пришел Иисус в Галилею, проповедуя Евангелие Царствия Божия
\vs Mar 1:15 и говоря, что исполнилось время и приблизилось Царствие Божие: покайтесь и веруйте в Евангелие.
\rsbpar\vs Mar 1:16 Проходя же близ моря Галилейского, увидел Симона и Андрея, брата его, закидывающих сети в море, ибо они были рыболовы.
\vs Mar 1:17 И сказал им Иисус: идите за Мною, и Я сделаю, что вы будете ловцами человеков.
\vs Mar 1:18 И они тотчас, оставив свои сети, последовали за Ним.
\vs Mar 1:19 И, пройдя оттуда немного, Он увидел Иакова Зеведеева и Иоанна, брата его, также в лодке починивающих сети;
\vs Mar 1:20 и тотчас призвал их. И они, оставив отца своего Зеведея в лодке с работниками, последовали за Ним.
\rsbpar\vs Mar 1:21 И приходят в Капернаум; и вскоре в субботу вошел Он в синагогу и учил.
\vs Mar 1:22 И дивились Его учению, ибо Он учил их, как власть имеющий, а не как книжники.
\vs Mar 1:23 В синагоге их был человек, \bibemph{одержимый} духом нечистым, и вскричал:
\vs Mar 1:24 оставь! что Тебе до нас, Иисус Назарянин? Ты пришел погубить нас! знаю Тебя, кто Ты, Святый Божий.
\vs Mar 1:25 Но Иисус запретил ему, говоря: замолчи и выйди из него.
\vs Mar 1:26 Тогда дух нечистый, сотрясши его и вскричав громким голосом, вышел из него.
\vs Mar 1:27 И все ужаснулись, так что друг друга спрашивали: что это? что это за новое учение, что Он и духам нечистым повелевает со властью, и они повинуются Ему?
\vs Mar 1:28 И скоро разошлась о Нем молва по всей окрестности в Галилее.
\rsbpar\vs Mar 1:29 Выйдя вскоре из синагоги, пришли в дом Симона и Андрея, с Иаковом и Иоанном.
\vs Mar 1:30 Теща же Симонова лежала в горячке; и тотчас говорят Ему о ней.
\vs Mar 1:31 Подойдя, Он поднял ее, взяв ее за руку; и горячка тотчас оставила ее, и она стала служить им.
\vs Mar 1:32 При наступлении же вечера, когда заходило солнце, приносили к Нему всех больных и бесноватых.
\vs Mar 1:33 И весь город собрался к дверям.
\vs Mar 1:34 И Он исцелил многих, страдавших различными болезнями; изгнал многих бесов, и не позволял бесам говорить, что они знают, что Он Христос.
\rsbpar\vs Mar 1:35 А утром, встав весьма рано, вышел и удалился в пустынное место, и там молился.
\vs Mar 1:36 Симон и бывшие с ним пошли за Ним
\vs Mar 1:37 и, найдя Его, говорят Ему: все ищут Тебя.
\vs Mar 1:38 Он говорит им: пойдем в ближние селения и города, чтобы Мне и там проповедовать, ибо Я для того пришел.
\vs Mar 1:39 И Он проповедовал в синагогах их по всей Галилее и изгонял бесов.
\rsbpar\vs Mar 1:40 Приходит к Нему прокаженный и, умоляя Его и падая пред Ним на колени, говорит Ему: если хочешь, можешь меня очистить.
\vs Mar 1:41 Иисус, умилосердившись над ним, простер руку, коснулся его и сказал ему: хочу, очистись.
\vs Mar 1:42 После сего слова проказа тотчас сошла с него, и он стал чист.
\vs Mar 1:43 И, посмотрев на него строго, тотчас отослал его
\vs Mar 1:44 и сказал ему: смотри, никому ничего не говори, но пойди, покажись священнику и принеси за очищение твое, что повелел Моисей, во свидетельство им.
\vs Mar 1:45 А он, выйдя, начал провозглашать и рассказывать о происшедшем, так что \bibemph{Иисус} не мог уже явно войти в город, но находился вне, в местах пустынных. И приходили к Нему отовсюду.
\vs Mar 2:1 Через \bibemph{несколько} дней опять пришел Он в Капернаум; и слышно стало, что Он в доме.
\vs Mar 2:2 Тотчас собрались многие, так что уже и у дверей не было места; и Он говорил им слово.
\vs Mar 2:3 И пришли к Нему с расслабленным, которого несли четверо;
\vs Mar 2:4 и, не имея возможности приблизиться к Нему за многолюдством, раскрыли кровлю \bibemph{дома}, где Он находился, и, прокопав ее, спустили постель, на которой лежал расслабленный.
\vs Mar 2:5 Иисус, видя веру их, говорит расслабленному: чадо! прощаются тебе грехи твои.
\vs Mar 2:6 Тут сидели некоторые из книжников и помышляли в сердцах своих:
\vs Mar 2:7 что Он так богохульствует? кто может прощать грехи, кроме одного Бога?
\vs Mar 2:8 Иисус, тотчас узнав духом Своим, что они так помышляют в себе, сказал им: для чего так помышляете в сердцах ваших?
\vs Mar 2:9 Что легче? сказать ли расслабленному: прощаются тебе грехи? или сказать: встань, возьми свою постель и ходи?
\vs Mar 2:10 Но чтобы вы знали, что Сын Человеческий имеет власть на земле прощать грехи,~--- говорит расслабленному:
\vs Mar 2:11 тебе говорю: встань, возьми постель твою и иди в дом твой.
\vs Mar 2:12 Он тотчас встал и, взяв постель, вышел перед всеми, так что все изумлялись и прославляли Бога, говоря: никогда ничего такого мы не видали.
\rsbpar\vs Mar 2:13 И вышел \bibemph{Иисус} опять к морю; и весь народ пошел к Нему, и Он учил их.
\vs Mar 2:14 Проходя, увидел Он Левия Алфеева, сидящего у сбора пошлин, и говорит ему: следуй за Мною. И \bibemph{он}, встав, последовал за Ним.
\vs Mar 2:15 И когда Иисус возлежал в доме его, возлежали с Ним и ученики Его и многие мытари и грешники: ибо много их было, и они следовали за Ним.
\vs Mar 2:16 Книжники и фарисеи, увидев, что Он ест с мытарями и грешниками, говорили ученикам Его: как это Он ест и пьет с мытарями и грешниками?
\vs Mar 2:17 Услышав \bibemph{сие}, Иисус говорит им: не здоровые имеют нужду во враче, но больные; Я пришел призвать не праведников, но грешников к покаянию.
\rsbpar\vs Mar 2:18 Ученики Иоанновы и фарисейские постились. Приходят к Нему и говорят: почему ученики Иоанновы и фарисейские постятся, а Твои ученики не постятся?
\vs Mar 2:19 И сказал им Иисус: могут ли поститься сыны чертога брачного, когда с ними жених? Доколе с ними жених, не могут поститься,
\vs Mar 2:20 но придут дни, когда отнимется у них жених, и тогда будут поститься в те дни.
\vs Mar 2:21 Никто к ветхой одежде не приставляет заплаты из небеленой ткани: иначе вновь пришитое отдерет от старого, и дыра будет еще хуже.
\vs Mar 2:22 Никто не вливает вина молодого в мехи ветхие: иначе молодое вино прорвет мехи, и вино вытечет, и мехи пропадут; но вино молодое надобно вливать в мехи новые.
\rsbpar\vs Mar 2:23 И случилось Ему в субботу проходить засеянными \bibemph{полями}, и ученики Его дорогою начали срывать колосья.
\vs Mar 2:24 И фарисеи сказали Ему: смотри, чт\acc{о} они делают в субботу, чего не должно \bibemph{делать}?
\vs Mar 2:25 Он сказал им: неужели вы не читали никогда, чт\acc{о} сделал Давид, когда имел нужду и взалкал сам и бывшие с ним?
\vs Mar 2:26 как вошел он в дом Божий при первосвященнике Авиафаре и ел хлебы предложения, которых не должно было есть никому, кроме священников, и дал и бывшим с ним?
\vs Mar 2:27 И сказал им: суббота для человека, а не человек для субботы;
\vs Mar 2:28 посему Сын Человеческий есть господин и субботы.
\vs Mar 3:1 И пришел опять в синагогу; там был человек, имевший иссохшую руку.
\vs Mar 3:2 И наблюдали за Ним, не исцелит ли его в субботу, чтобы обвинить Его.
\vs Mar 3:3 Он же говорит человеку, имевшему иссохшую руку: стань на средину.
\vs Mar 3:4 А им говорит: должно ли в субботу добро делать, или зло делать? душу спасти, или погубить? Но они молчали.
\vs Mar 3:5 И, воззрев на них с гневом, скорбя об ожесточении сердец их, говорит тому человеку: протяни руку твою. Он протянул, и стала рука его здорова, как другая.
\rsbpar\vs Mar 3:6 Фарисеи, выйдя, немедленно составили с иродианами совещание против Него, как бы погубить Его.
\vs Mar 3:7 Но Иисус с учениками Своими удалился к морю; и за Ним последовало множество народа из Галилеи, Иудеи,
\vs Mar 3:8 Иерусалима, Идумеи и из-за Иордана. И \bibemph{живущие} в окрестностях Тира и Сидона, услышав, что Он делал, шли к Нему в великом множестве.
\vs Mar 3:9 И сказал ученикам Своим, чтобы готова была для Него лодка по причине многолюдства, дабы не теснили Его.
\vs Mar 3:10 Ибо многих Он исцелил, так что имевшие язвы бросались к Нему, чтобы коснуться Его.
\vs Mar 3:11 И духи нечистые, когда видели Его, падали пред Ним и кричали: Ты Сын Божий.
\vs Mar 3:12 Но Он строго запрещал им, чтобы не делали Его известным.
\rsbpar\vs Mar 3:13 Потом взошел на гору и позвал к Себе, кого Сам хотел; и пришли к Нему.
\vs Mar 3:14 И поставил \bibemph{из них} двенадцать, чтобы с Ним были и чтобы посылать их на проповедь,
\vs Mar 3:15 и чтобы они имели власть исцелять от болезней и изгонять бесов;
\vs Mar 3:16 \bibemph{поставил} Симона, нарекши ему имя Петр,
\vs Mar 3:17 Иакова Зеведеева и Иоанна, брата Иакова, нарекши им имена Воанергес, то есть <<сыны громовы>>,
\vs Mar 3:18 Андрея, Филиппа, Варфоломея, Матфея, Фому, Иакова Алфеева, Фаддея, Симона Кананита
\vs Mar 3:19 и Иуду Искариотского, который и предал Его.
\rsbpar\vs Mar 3:20 Приходят в дом; и опять сходится народ, так что им невозможно было и хлеба есть.
\vs Mar 3:21 И, услышав, ближние Его пошли взять Его, ибо говорили, что Он вышел из себя.
\vs Mar 3:22 А книжники, пришедшие из Иерусалима, говорили, что Он имеет \bibemph{в Себе} веельзевула и что изгоняет бесов силою бесовского князя.
\vs Mar 3:23 И, призвав их, говорил им притчами: как может сатана изгонять сатану?
\vs Mar 3:24 Если царство разделится само в себе, не может устоять царство т\acc{о};
\vs Mar 3:25 и если дом разделится сам в себе, не может устоять дом тот;
\vs Mar 3:26 и если сатана восстал на самого себя и разделился, не может устоять, но пришел конец его.
\vs Mar 3:27 Никто, войдя в дом сильного, не может расхитить вещей его, если прежде не свяжет сильного, и тогда расхитит дом его.
\vs Mar 3:28 Истинно говорю вам: будут прощены сынам человеческим все грехи и хуления, какими бы ни хулили;
\vs Mar 3:29 но кто будет хулить Духа Святаго, тому не будет прощения вовек, но подлежит он вечному осуждению.
\vs Mar 3:30 \bibemph{Сие сказал Он}, потому что говорили: в Нем нечистый дух.
\rsbpar\vs Mar 3:31 И пришли Матерь и братья Его и, стоя вне \bibemph{дома}, послали к Нему звать Его.
\vs Mar 3:32 Около Него сидел народ. И сказали Ему: вот, Матерь Твоя и братья Твои и сестры Твои, вне \bibemph{дома}, спрашивают Тебя.
\vs Mar 3:33 И отвечал им: кто матерь Моя и братья Мои?
\vs Mar 3:34 И обозрев сидящих вокруг Себя, говорит: вот матерь Моя и братья Мои;
\vs Mar 3:35 ибо кто будет исполнять волю Божию, тот Мне брат, и сестра, и матерь.
\vs Mar 4:1 И опять начал учить при море; и собралось к Нему множество народа, так что Он вошел в лодку и сидел на море, а весь народ был на земле, у моря.
\vs Mar 4:2 И учил их притчами много, и в учении Своем говорил им:
\vs Mar 4:3 слушайте: вот, вышел сеятель сеять;
\vs Mar 4:4 и, когда сеял, случилось, что иное упало при дороге, и налетели птицы и поклевали т\acc{о}.
\vs Mar 4:5 Иное упало на каменистое \bibemph{место}, где немного было земли, и скоро взошло, потому что земля была неглубока;
\vs Mar 4:6 когда же взошло солнце, увяло и, как не имело корня, засохло.
\vs Mar 4:7 Иное упало в терние, и терние выросло, и заглушило \bibemph{семя}, и оно не дало плода.
\vs Mar 4:8 И иное упало на добрую землю и дало плод, который взошел и вырос, и принесло иное тридцать, иное шестьдесят, и иное сто.
\vs Mar 4:9 И сказал им: кто имеет уши слышать, да слышит!
\vs Mar 4:10 Когда же остался без народа, окружающие Его, вместе с двенадцатью, спросили Его о притче.
\vs Mar 4:11 И сказал им: вам дано знать тайны Царствия Божия, а тем внешним все бывает в притчах;
\vs Mar 4:12 так что они своими глазами смотрят, и не видят; своими ушами слышат, и не разумеют, да не обратятся, и прощены будут им грехи.
\vs Mar 4:13 И говорит им: не понимаете этой притчи? Как же вам уразуметь все притчи?
\vs Mar 4:14 Сеятель слово сеет.
\vs Mar 4:15 \bibemph{Посеянное} при дороге означает тех, в которых сеется слово, но \bibemph{к которым}, когда услышат, тотчас приходит сатана и похищает слово, посеянное в сердцах их.
\vs Mar 4:16 Подобным образом и посеянное на каменистом \bibemph{месте} означает тех, которые, когда услышат слово, тотчас с радостью принимают его,
\vs Mar 4:17 но не имеют в себе корня и непостоянны; потом, когда настанет скорбь или гонение за слово, тотчас соблазняются.
\vs Mar 4:18 Посеянное в тернии означает слышащих слово,
\vs Mar 4:19 но в которых заботы века сего, обольщение богатством и другие пожелания, входя в них, заглушают слово, и оно бывает без плода.
\vs Mar 4:20 А посеянное на доброй земле означает тех, которые слушают слово и принимают, и приносят плод, один в тридцать, другой в шестьдесят, иной во сто крат.
\rsbpar\vs Mar 4:21 И сказал им: для того ли приносится свеча, чтобы поставить ее под сосуд или под кровать? не для того ли, чтобы поставить ее на подсвечнике?
\vs Mar 4:22 Нет ничего тайного, что не сделалось бы явным, и ничего не бывает потаенного, что не вышло бы наружу.
\vs Mar 4:23 Если кто имеет уши слышать, да слышит!
\vs Mar 4:24 И сказал им: замечайте, что слышите: какою мерою мерите, такою отмерено будет вам и прибавлено будет вам, слушающим.
\vs Mar 4:25 Ибо кто имеет, тому дано будет, а кто не имеет, у того отнимется и то, что имеет.
\rsbpar\vs Mar 4:26 И сказал: Царствие Божие подобно тому, как если человек бросит семя в землю,
\vs Mar 4:27 и спит, и встает ночью и днем; и к\acc{а}к семя всходит и растет, не знает он,
\vs Mar 4:28 ибо земля сама собою производит сперва зелень, потом колос, потом полное зерно в колосе.
\vs Mar 4:29 Когда же созреет плод, немедленно посылает серп, потому что настала жатва.
\rsbpar\vs Mar 4:30 И сказал: чему уподобим Царствие Божие? или какою притчею изобразим его?
\vs Mar 4:31 Оно~--- как зерно горчичное, которое, когда сеется в землю, есть меньше всех семян на земле;
\vs Mar 4:32 а когда посеяно, всходит и становится больше всех злаков, и пускает большие ветви, так что под тенью его могут укрываться птицы небесные.
\vs Mar 4:33 И таковыми многими притчами проповедовал им слово, сколько они могли слышать.
\vs Mar 4:34 Без притчи же не говорил им, а ученикам наедине изъяснял все.
\rsbpar\vs Mar 4:35 Вечером того дня сказал им: переправимся на ту сторону.
\vs Mar 4:36 И они, отпустив народ, взяли Его с собою, как Он был в лодке; с Ним были и другие лодки.
\vs Mar 4:37 И поднялась великая буря; волны били в лодку, так что она уже наполнялась \bibemph{водою}.
\vs Mar 4:38 А Он спал на корме на возглавии. Его будят и говорят Ему: Учитель! неужели Тебе нужды нет, что мы погибаем?
\vs Mar 4:39 И, встав, Он запретил ветру и сказал морю: умолкни, перестань. И ветер утих, и сделалась великая тишина.
\vs Mar 4:40 И сказал им: что вы так боязливы? как у вас нет веры?
\vs Mar 4:41 И убоялись страхом великим и говорили между собою: кто же Сей, что и ветер и море повинуются Ему?
\vs Mar 5:1 И пришли на другой берег моря, в страну Гадаринскую.
\vs Mar 5:2 И когда вышел Он из лодки, тотчас встретил Его вышедший из гробов человек, \bibemph{одержимый} нечистым духом,
\vs Mar 5:3 он имел жилище в гробах, и никто не мог его связать даже цепями,
\vs Mar 5:4 потому что многократно был он скован оковами и цепями, но разрывал цепи и разбивал оковы, и никто не в силах был укротить его;
\vs Mar 5:5 всегда, ночью и днем, в горах и гробах, кричал он и бился о камни;
\vs Mar 5:6 увидев же Иисуса издалека, прибежал и поклонился Ему,
\vs Mar 5:7 и, вскричав громким голосом, сказал: что Тебе до меня, Иисус, Сын Бога Всевышнего? заклинаю Тебя Богом, не мучь меня!
\vs Mar 5:8 Ибо \bibemph{Иисус} сказал ему: выйди, дух нечистый, из сего человека.
\vs Mar 5:9 И спросил его: как тебе имя? И он сказал в ответ: легион имя мне, потому что нас много.
\vs Mar 5:10 И много просили Его, чтобы не высылал их вон из страны той.
\vs Mar 5:11 Паслось же там при горе большое стадо свиней.
\vs Mar 5:12 И просили Его все бесы, говоря: пошли нас в свиней, чтобы нам войти в них.
\vs Mar 5:13 Иисус тотчас позволил им. И нечистые духи, выйдя, вошли в свиней; и устремилось стадо с крутизны в море, а их было около двух тысяч; и потонули в море.
\vs Mar 5:14 Пасущие же свиней побежали и рассказали в городе и в деревнях. И \bibemph{жители} вышли посмотреть, что случилось.
\vs Mar 5:15 Приходят к Иисусу и видят, что бесновавшийся, в котором был легион, сидит и одет, и в здравом уме; и устрашились.
\vs Mar 5:16 Видевшие рассказали им о том, как это произошло с бесноватым, и о свиньях.
\vs Mar 5:17 И начали просить Его, чтобы отошел от пределов их.
\vs Mar 5:18 И когда Он вошел в лодку, бесновавшийся просил Его, чтобы быть с Ним.
\vs Mar 5:19 Но Иисус не дозволил ему, а сказал: иди домой к своим и расскажи им, что сотворил с тобою Господь и \bibemph{как} помиловал тебя.
\vs Mar 5:20 И пошел и начал проповедовать в Десятиградии, что сотворил с ним Иисус; и все дивились.
\rsbpar\vs Mar 5:21 Когда Иисус опять переправился в лодке на другой берег, собралось к Нему множество народа. Он был у моря.
\vs Mar 5:22 И вот, приходит один из начальников синагоги, по имени Иаир, и, увидев Его, падает к ногам Его
\vs Mar 5:23 и усильно просит Его, говоря: дочь моя при смерти; приди и возложи на нее руки, чтобы она выздоровела и осталась жива.
\vs Mar 5:24 \bibemph{Иисус} пошел с ним. За Ним следовало множество народа, и теснили Его.
\rsbpar\vs Mar 5:25 Одна женщина, которая страдала кровотечением двенадцать лет,
\vs Mar 5:26 много потерпела от многих врачей, истощила всё, что было у ней, и не получила никакой пользы, но пришла еще в худшее состояние,~---
\vs Mar 5:27 услышав об Иисусе, подошла сзади в народе и прикоснулась к одежде Его,
\vs Mar 5:28 ибо говорила: если хотя к одежде Его прикоснусь, то выздоровею.
\vs Mar 5:29 И тотчас иссяк у ней источник крови, и она ощутила в теле, что исцелена от болезни.
\vs Mar 5:30 В то же время Иисус, почувствовав Сам в Себе, что вышла из Него сила, обратился в народе и сказал: кто прикоснулся к Моей одежде?
\vs Mar 5:31 Ученики сказали Ему: Ты видишь, что народ теснит Тебя, и говоришь: кто прикоснулся ко Мне?
\vs Mar 5:32 Но Он смотрел вокруг, чтобы видеть ту, которая сделала это.
\vs Mar 5:33 Женщина в страхе и трепете, зная, что с нею произошло, подошла, пала пред Ним и сказала Ему всю истину.
\vs Mar 5:34 Он же сказал ей: дщерь! вера твоя спасла тебя; иди в мире и будь здорова от болезни твоей.
\rsbpar\vs Mar 5:35 Когда Он еще говорил сие, приходят от начальника синагоги и говорят: дочь твоя умерла; что еще утруждаешь Учителя?
\vs Mar 5:36 Но Иисус, услышав сии слова, тотчас говорит начальнику синагоги: не бойся, только веруй.
\vs Mar 5:37 И не позволил никому следовать за Собою, кроме Петра, Иакова и Иоанна, брата Иакова.
\vs Mar 5:38 Приходит в дом начальника синагоги и видит смятение и плачущих и вопиющих громко.
\vs Mar 5:39 И, войдя, говорит им: что смущаетесь и плачете? девица не умерла, но спит.
\vs Mar 5:40 И смеялись над Ним. Но Он, выслав всех, берет с Собою отца и мать девицы и бывших с Ним и входит туда, где девица лежала.
\vs Mar 5:41 И, взяв девицу за руку, говорит ей: <<талиф\acc{а} кум\acc{и}>>, что значит: девица, тебе говорю, встань.
\vs Mar 5:42 И девица тотчас встала и начала ходить, ибо была лет двенадцати. \bibemph{Видевшие} пришли в великое изумление.
\vs Mar 5:43 И Он строго приказал им, чтобы никто об этом не знал, и сказал, чтобы дали ей есть.
\vs Mar 6:1 Оттуда вышел Он и пришел в Свое отечество; за Ним следовали ученики Его.
\vs Mar 6:2 Когда наступила суббота, Он начал учить в синагоге; и многие слышавшие с изумлением говорили: откуда у Него это? что за премудрость дана Ему, и как такие чудеса совершаются руками Его?
\vs Mar 6:3 Не плотник ли Он, сын Марии, брат Иакова, Иосии, Иуды и Симона? Не здесь ли, между нами, Его сестры? И соблазнялись о Нем.
\vs Mar 6:4 Иисус же сказал им: не бывает пророк без чести, разве только в отечестве своем и у сродников и в доме своем.
\vs Mar 6:5 И не мог совершить там никакого чуда, только на немногих больных возложив руки, исцелил \bibemph{их}.
\vs Mar 6:6 И дивился неверию их; потом ходил по окрестным селениям и учил.
\rsbpar\vs Mar 6:7 И, призвав двенадцать, начал посылать их по два, и дал им власть над нечистыми духами.
\vs Mar 6:8 И заповедал им ничего не брать в дорогу, кроме одного посоха: ни сумы, ни хлеба, ни меди в поясе,
\vs Mar 6:9 но обуваться в простую обувь и не носить двух одежд.
\vs Mar 6:10 И сказал им: если где войдете в дом, оставайтесь в нем, доколе не выйдете из того места.
\vs Mar 6:11 И если кто не примет вас и не будет слушать вас, то, выходя оттуда, отрясите прах от ног ваших, во свидетельство на них. Истинно говорю вам: отраднее будет Содому и Гоморре в день суда, нежели тому городу.
\vs Mar 6:12 Они пошли и проповедовали покаяние;
\vs Mar 6:13 изгоняли многих бесов и многих больных мазали маслом и исцеляли.
\rsbpar\vs Mar 6:14 Царь Ирод, услышав \bibemph{об Иисусе} (ибо имя Его стало гласно), говорил: это Иоанн Креститель воскрес из мертвых, и потому чудеса делаются им.
\vs Mar 6:15 Другие говорили: это Илия, а иные говорили: это пророк, или как один из пророков.
\vs Mar 6:16 Ирод же, услышав, сказал: это Иоанн, которого я обезглавил; он воскрес из мертвых.
\vs Mar 6:17 Ибо сей Ирод, послав, взял Иоанна и заключил его в темницу за Иродиаду, жену Филиппа, брата своего, потому что женился на ней.
\vs Mar 6:18 Ибо Иоанн говорил Ироду: не должно тебе иметь жену брата твоего.
\vs Mar 6:19 Иродиада же, злобясь на него, желала убить его; но не могла.
\vs Mar 6:20 Ибо Ирод боялся Иоанна, зная, что он муж праведный и святой, и берёг его; многое делал, слушаясь его, и с удовольствием слушал его.
\vs Mar 6:21 Настал удобный день, когда Ирод, по случаю \bibemph{дня} рождения своего, делал пир вельможам своим, тысяченачальникам и старейшинам Галилейским,~---
\vs Mar 6:22 дочь Иродиады вошла, плясала и угодила Ироду и возлежавшим с ним; царь сказал девице: проси у меня, чего хочешь, и дам тебе;
\vs Mar 6:23 и клялся ей: чего ни попросишь у меня, дам тебе, даже до половины моего царства.
\vs Mar 6:24 Она вышла и спросила у матери своей: чего просить? Та отвечала: головы Иоанна Крестителя.
\vs Mar 6:25 И она тотчас пошла с поспешностью к царю и просила, говоря: хочу, чтобы ты дал мне теперь же на блюде голову Иоанна Крестителя.
\vs Mar 6:26 Царь опечалился, но ради клятвы и возлежавших с ним не захотел отказать ей.
\vs Mar 6:27 И тотчас, послав оруженосца, царь повелел принести голову его.
\vs Mar 6:28 Он пошел, отсек ему голову в темнице, и принес голову его на блюде, и отдал ее девице, а девица отдала ее матери своей.
\vs Mar 6:29 Ученики его, услышав, пришли и взяли тело его, и положили его во гробе.
\rsbpar\vs Mar 6:30 И собрались Апостолы к Иисусу и рассказали Ему всё, и что сделали, и чему научили.
\vs Mar 6:31 Он сказал им: пойдите вы одни в пустынное место и отдохните немного,~--- ибо много было приходящих и отходящих, так что и есть им было некогда.
\vs Mar 6:32 И отправились в пустынное место в лодке одни.
\vs Mar 6:33 Народ увидел, \bibemph{как} они отправлялись, и многие узнали их; и бежали туда пешие из всех городов, и предупредили их, и собрались к Нему.
\vs Mar 6:34 Иисус, выйдя, увидел множество народа и сжалился над ними, потому что они были, как овцы, не имеющие пастыря; и начал учить их много.
\vs Mar 6:35 И как времени прошло много, ученики Его, приступив к Нему, говорят: место \bibemph{здесь} пустынное, а времени уже много,~---
\vs Mar 6:36 отпусти их, чтобы они пошли в окрестные деревни и селения и купили себе хлеба, ибо им нечего есть.
\vs Mar 6:37 Он сказал им в ответ: вы дайте им есть. И сказали Ему: разве нам пойти купить хлеба динариев на двести и дать им есть?
\vs Mar 6:38 Но Он спросил их: сколько у вас хлебов? пойдите, посмотрите. Они, узнав, сказали: пять хлебов и две рыбы.
\vs Mar 6:39 Тогда повелел им рассадить всех отделениями на зеленой траве.
\vs Mar 6:40 И сели рядами, по сто и по пятидесяти.
\vs Mar 6:41 Он взял пять хлебов и две рыбы, воззрев на небо, благословил и преломил хлебы и дал ученикам Своим, чтобы они раздали им; и две рыбы разделил на всех.
\vs Mar 6:42 И ели все, и насытились.
\vs Mar 6:43 И набрали кусков хлеба и \bibemph{остатков} от рыб двенадцать полных коробов.
\vs Mar 6:44 Было же евших хлебы около пяти тысяч мужей.
\rsbpar\vs Mar 6:45 И тотчас понудил учеников Своих войти в лодку и отправиться вперед на другую сторону к Вифсаиде, пока Он отпустит народ.
\vs Mar 6:46 И, отпустив их, пошел на гору помолиться.
\vs Mar 6:47 Вечером лодка была посреди моря, а Он один на земле.
\vs Mar 6:48 И увидел их бедствующих в плавании, потому что ветер им был противный; около же четвертой стражи ночи подошел к ним, идя по морю, и хотел миновать их.
\vs Mar 6:49 Они, увидев Его идущего по морю, подумали, что это призрак, и вскричали.
\vs Mar 6:50 Ибо все видели Его и испугались. И тотчас заговорил с ними и сказал им: ободритесь; это Я, не бойтесь.
\vs Mar 6:51 И вошел к ним в лодку, и ветер утих. И они чрезвычайно изумлялись в себе и дивились,
\vs Mar 6:52 ибо не вразумились \bibemph{чудом} над хлебами, потому что сердце их было окаменено.
\vs Mar 6:53 И, переправившись, прибыли в землю Геннисаретскую и пристали \bibemph{к берегу}.
\vs Mar 6:54 Когда вышли они из лодки, тотчас \bibemph{жители}, узнав Его,
\vs Mar 6:55 обежали всю окрестность ту и начали на постелях приносить больных туда, где Он, как слышно было, находился.
\vs Mar 6:56 И куда ни приходил Он, в селения ли, в города ли, в деревни ли, клали больных на открытых местах и просили Его, чтобы им прикоснуться хотя к краю одежды Его; и которые прикасались к Нему, исцелялись.
\vs Mar 7:1 Собрались к Нему фарисеи и некоторые из книжников, пришедшие из Иерусалима,
\vs Mar 7:2 и, увидев некоторых из учеников Его, евших хлеб нечистыми, то есть неумытыми, руками, укоряли.
\vs Mar 7:3 Ибо фарисеи и все Иудеи, держась предания старцев, не едят, не умыв тщательно рук;
\vs Mar 7:4 и, \bibemph{придя} с торга, не едят не омывшись. Есть и многое другое, чего они приняли держаться: наблюдать омовение чаш, кружек, котлов и скамей.
\vs Mar 7:5 Потом спрашивают Его фарисеи и книжники: зачем ученики Твои не поступают по преданию старцев, но неумытыми руками едят хлеб?
\vs Mar 7:6 Он сказал им в ответ: хорошо пророчествовал о вас, лицемерах, Исаия, как написано: люди сии чтут Меня устами, сердце же их далеко отстоит от Меня,
\vs Mar 7:7 но тщетно чтут Меня, уча учениям, заповедям человеческим.
\vs Mar 7:8 Ибо вы, оставив заповедь Божию, держитесь предания человеческого, омовения кружек и чаш, и делаете многое другое, сему подобное.
\vs Mar 7:9 И сказал им: хорошо ли, \bibemph{что} вы отменяете заповедь Божию, чтобы соблюсти свое предание?
\vs Mar 7:10 Ибо Моисей сказал: почитай отца своего и мать свою; и: злословящий отца или мать смертью да умрет.
\vs Mar 7:11 А вы говорите: кто скажет отцу или матери: корван, то есть дар \bibemph{Богу} т\acc{о}, чем бы ты от меня пользовался,
\vs Mar 7:12 тому вы уже попускаете ничего не делать для отца своего или матери своей,
\vs Mar 7:13 устраняя слово Божие преданием вашим, которое вы установили; и делаете многое сему подобное.
\vs Mar 7:14 И, призвав весь народ, говорил им: слушайте Меня все и разумейте:
\vs Mar 7:15 ничто, входящее в человека извне, не может осквернить его; но что исходит из него, то оскверняет человека.
\vs Mar 7:16 Если кто имеет уши слышать, да слышит!
\vs Mar 7:17 И когда Он от народа вошел в дом, ученики Его спросили Его о притче.
\vs Mar 7:18 Он сказал им: неужели и вы так непонятливы? Неужели не разумеете, что ничто, извне входящее в человека, не может осквернить его?
\vs Mar 7:19 Потому что не в сердце его входит, а в чрево, и выходит вон, \bibemph{чем} очищается всякая пища.
\vs Mar 7:20 Далее сказал: исходящее из человека оскверняет человека.
\vs Mar 7:21 Ибо извнутрь, из сердца человеческого, исходят злые помыслы, прелюбодеяния, любодеяния, убийства,
\vs Mar 7:22 кражи, лихоимство, злоба, коварство, непотребство, завистливое око, богохульство, гордость, безумство,~---
\vs Mar 7:23 всё это зло извнутрь исходит и оскверняет человека.
\rsbpar\vs Mar 7:24 И, отправившись оттуда, пришел в пределы Тирские и Сидонские; и, войдя в дом, не хотел, чтобы кто узнал; но не мог утаиться.
\vs Mar 7:25 Ибо услышала о Нем женщина, у которой дочь одержима была нечистым духом, и, придя, припала к ногам Его;
\vs Mar 7:26 а женщина та была язычница, родом сирофиникиянка; и просила Его, чтобы изгнал беса из ее дочери.
\vs Mar 7:27 Но Иисус сказал ей: дай прежде насытиться детям, ибо нехорошо взять хлеб у детей и бросить псам.
\vs Mar 7:28 Она же сказала Ему в ответ: так, Господи; но и псы под столом едят крохи у детей.
\vs Mar 7:29 И сказал ей: за это слово, пойди; бес вышел из твоей дочери.
\vs Mar 7:30 И, придя в свой дом, она нашла, что бес вышел и дочь лежит на постели.
\rsbpar\vs Mar 7:31 Выйдя из пределов Тирских и Сидонских, \bibemph{Иисус} опять пошел к морю Галилейскому через пределы Десятиградия.
\vs Mar 7:32 Привели к Нему глухого косноязычного и просили Его возложить на него руку.
\vs Mar 7:33 \bibemph{Иисус}, отведя его в сторону от народа, вложил персты Свои в уши ему и, плюнув, коснулся языка его;
\vs Mar 7:34 и, воззрев на небо, вздохнул и сказал ему: <<еффаф\acc{а}>>, то есть: отверзись.
\vs Mar 7:35 И тотчас отверзся у него слух и разрешились узы его языка, и стал говорить чисто.
\vs Mar 7:36 И повелел им не сказывать никому. Но сколько Он ни запрещал им, они еще более разглашали.
\vs Mar 7:37 И чрезвычайно дивились, и говорили: всё хорошо делает,~--- и глухих делает слышащими, и немых~--- говорящими.
\vs Mar 8:1 В те дни, когда собралось весьма много народа и нечего было им есть, Иисус, призвав учеников Своих, сказал им:
\vs Mar 8:2 жаль Мне народа, что уже три дня находятся при Мне, и нечего им есть.
\vs Mar 8:3 Если неевшими отпущу их в домы их, ослабеют в дороге, ибо некоторые из них пришли издалека.
\vs Mar 8:4 Ученики Его отвечали Ему: откуда мог бы кто \bibemph{взять} здесь в пустыне хлебов, чтобы накормить их?
\vs Mar 8:5 И спросил их: сколько у вас хлебов? Они сказали: семь.
\vs Mar 8:6 Тогда велел народу возлечь на землю; и, взяв семь хлебов и воздав благодарение, преломил и дал ученикам Своим, чтобы они раздали; и они раздали народу.
\vs Mar 8:7 Было у них и немного рыбок: благословив, Он велел раздать и их.
\vs Mar 8:8 И ели, и насытились; и набрали оставшихся кусков семь корзин.
\vs Mar 8:9 Евших же было около четырех тысяч. И отпустил их.
\rsbpar\vs Mar 8:10 И тотчас войдя в лодку с учениками Своими, прибыл в пределы Далмануфские.
\vs Mar 8:11 Вышли фарисеи, начали с Ним спорить и требовали от Него знамения с неба, искушая Его.
\vs Mar 8:12 И Он, глубоко вздохнув, сказал: для чего род сей требует знамения? Истинно говорю вам, не дастся роду сему знамение.
\vs Mar 8:13 И, оставив их, опять вошел в лодку и отправился на ту сторону.
\vs Mar 8:14 При сем ученики Его забыли взять хлебов и кроме одного хлеба не имели с собою в лодке.
\vs Mar 8:15 А Он заповедал им, говоря: смотрите, берегитесь закваски фарисейской и закваски Иродовой.
\vs Mar 8:16 И, рассуждая между собою, говорили: \bibemph{это значит}, что хлебов нет у нас.
\vs Mar 8:17 Иисус, уразумев, говорит им: что рассуждаете о том, что нет у вас хлебов? Еще ли не понимаете и не разумеете? Еще ли окаменено у вас сердце?
\vs Mar 8:18 Имея очи, не видите? имея уши, не слышите? и не помните?
\vs Mar 8:19 Когда Я пять хлебов преломил для пяти тысяч \bibemph{человек}, сколько полных коробов набрали вы кусков? Говорят Ему: двенадцать.
\vs Mar 8:20 А когда семь для четырех тысяч, сколько корзин набрали вы оставшихся кусков? Сказали: семь.
\vs Mar 8:21 И сказал им: как же не разумеете?
\rsbpar\vs Mar 8:22 Приходит в Вифсаиду; и приводят к Нему слепого и просят, чтобы прикоснулся к нему.
\vs Mar 8:23 Он, взяв слепого за руку, вывел его вон из селения и, плюнув ему на глаза, возложил на него руки и спросил его: видит ли что?
\vs Mar 8:24 Он, взглянув, сказал: вижу проходящих людей, как деревья.
\vs Mar 8:25 Потом опять возложил руки на глаза ему и велел ему взглянуть. И он исцелел и стал видеть все ясно.
\vs Mar 8:26 И послал его домой, сказав: не заходи в селение и не рассказывай никому в селении.
\rsbpar\vs Mar 8:27 И пошел Иисус с учениками Своими в селения Кесарии Филипповой. Дорогою Он спрашивал учеников Своих: за кого почитают Меня люди?
\vs Mar 8:28 Они отвечали: за Иоанна Крестителя; другие же~--- за Илию; а иные~--- за одного из пророков.
\vs Mar 8:29 Он говорит им: а вы за кого почитаете Меня? Петр сказал Ему в ответ: Ты Христос.
\vs Mar 8:30 И запретил им, чтобы никому не говорили о Нем.
\vs Mar 8:31 И начал учить их, что Сыну Человеческому много должно пострадать, быть отвержену старейшинами, первосвященниками и книжниками, и быть убиту, и в третий день воскреснуть.
\vs Mar 8:32 И говорил о сем открыто. Но Петр, отозвав Его, начал прекословить Ему.
\vs Mar 8:33 Он же, обратившись и взглянув на учеников Своих, воспретил Петру, сказав: отойди от Меня, сатана, потому что ты думаешь не о том, что Божие, но что человеческое.
\vs Mar 8:34 И, подозвав народ с учениками Своими, сказал им: кто хочет идти за Мною, отвергнись себя, и возьми крест свой, и следуй за Мною.
\vs Mar 8:35 Ибо кто хочет душу свою сберечь, тот потеряет ее, а кто потеряет душу свою ради Меня и Евангелия, тот сбережет ее.
\vs Mar 8:36 Ибо какая польза человеку, если он приобретет весь мир, а душе своей повредит?
\vs Mar 8:37 Или какой выкуп даст человек за душу свою?
\vs Mar 8:38 Ибо кто постыдится Меня и Моих слов в роде сем прелюбодейном и грешном, того постыдится и Сын Человеческий, когда приидет в славе Отца Своего со святыми Ангелами.
\vs Mar 9:1 И сказал им: истинно говорю вам: есть некоторые из стоящих здесь, которые не вкусят смерти, как уже увидят Царствие Божие, пришедшее в силе.
\vs Mar 9:2 И, по прошествии дней шести, взял Иисус Петра, Иакова и Иоанна, и возвел на гору высокую особо их одних, и преобразился перед ними.
\vs Mar 9:3 Одежды Его сделались блистающими, весьма белыми, как снег, как на земле белильщик не может выбелить.
\vs Mar 9:4 И явился им Илия с Моисеем; и беседовали с Иисусом.
\vs Mar 9:5 При сем Петр сказал Иисусу: Равв\acc{и}! хорошо нам здесь быть; сделаем три кущи: Тебе одну, Моисею одну, и одну Илии.
\vs Mar 9:6 Ибо не знал, что сказать; потому что они были в страхе.
\vs Mar 9:7 И явилось облако, осеняющее их, и из облака исшел глас, глаголющий: Сей есть Сын Мой возлюбленный; Его слушайте.
\vs Mar 9:8 И, внезапно посмотрев вокруг, никого более с собою не видели, кроме одного Иисуса.
\vs Mar 9:9 Когда же сходили они с горы, Он не велел никому рассказывать о том, что видели, доколе Сын Человеческий не воскреснет из мертвых.
\vs Mar 9:10 И они удержали это слово, спрашивая друг друга, что значит: воскреснуть из мертвых.
\vs Mar 9:11 И спросили Его: как же книжники говорят, что Илии надлежит прийти прежде?
\vs Mar 9:12 Он сказал им в ответ: правда, Илия должен прийти прежде и устроить всё; и Сыну Человеческому, как написано о Нем, \bibemph{надлежит} много пострадать и быть уничижену.
\vs Mar 9:13 Но говорю вам, что и Илия пришел, и поступили с ним, как хотели, как написано о нем.
\rsbpar\vs Mar 9:14 Придя к ученикам, увидел много народа около них и книжников, спорящих с ними.
\vs Mar 9:15 Тотчас, увидев Его, весь народ изумился, и, подбегая, приветствовали Его.
\vs Mar 9:16 Он спросил книжников: о чем спорите с ними?
\vs Mar 9:17 Один из народа сказал в ответ: Учитель! я привел к Тебе сына моего, одержимого духом немым:
\vs Mar 9:18 где ни схватывает его, повергает его на землю, и он испускает пену, и скрежещет зубами своими, и цепенеет. Говорил я ученикам Твоим, чтобы изгнали его, и они не могли.
\vs Mar 9:19 Отвечая ему, Иисус сказал: о, род неверный! доколе буду с вами? доколе буду терпеть вас? Приведите его ко Мне.
\vs Mar 9:20 И привели его к Нему. Как скоро \bibemph{бесноватый} увидел Его, дух сотряс его; он упал на землю и валялся, испуская пену.
\vs Mar 9:21 И спросил \bibemph{Иисус} отца его: как давно это сделалось с ним? Он сказал: с детства;
\vs Mar 9:22 и многократно \bibemph{дух} бросал его и в огонь и в воду, чтобы погубить его; но, если что можешь, сжалься над нами и помоги нам.
\vs Mar 9:23 Иисус сказал ему: если сколько-нибудь можешь веровать, всё возможно верующему.
\vs Mar 9:24 И тотчас отец отрока воскликнул со слезами: верую, Господи! помоги моему неверию.
\vs Mar 9:25 Иисус, видя, что сбегается народ, запретил духу нечистому, сказав ему: дух немой и глухой! Я повелеваю тебе, выйди из него и впредь не входи в него.
\vs Mar 9:26 И, вскрикнув и сильно сотрясши его, вышел; и он сделался, как мертвый, так что многие говорили, что он умер.
\vs Mar 9:27 Но Иисус, взяв его за руку, поднял его; и он встал.
\vs Mar 9:28 И как вошел \bibemph{Иисус} в дом, ученики Его спрашивали Его наедине: почему мы не могли изгнать его?
\vs Mar 9:29 И сказал им: сей род не может выйти иначе, как от молитвы и поста.
\rsbpar\vs Mar 9:30 Выйдя оттуда, проходили через Галилею; и Он не хотел, чтобы кто узнал.
\vs Mar 9:31 Ибо учил Своих учеников и говорил им, что Сын Человеческий предан будет в руки человеческие и убьют Его, и, по убиении, в третий день воскреснет.
\vs Mar 9:32 Но они не разумели сих слов, а спросить Его боялись.
\rsbpar\vs Mar 9:33 Пришел в Капернаум; и когда был в доме, спросил их: о чем дорогою вы рассуждали между собою?
\vs Mar 9:34 Они молчали; потому что дорогою рассуждали между собою, кто больше.
\vs Mar 9:35 И, сев, призвал двенадцать и сказал им: кто хочет быть первым, будь из всех последним и всем слугою.
\vs Mar 9:36 И, взяв дитя, поставил его посреди них и, обняв его, сказал им:
\vs Mar 9:37 кто примет одно из таких детей во имя Мое, тот принимает Меня; а кто Меня примет, тот не Меня принимает, но Пославшего Меня.
\vs Mar 9:38 При сем Иоанн сказал: Учитель! мы видели человека, который именем Твоим изгоняет бесов, а не ходит за нами; и запретили ему, потому что не ходит за нами.
\vs Mar 9:39 Иисус сказал: не запрещайте ему, ибо никто, сотворивший чудо именем Моим, не может вскоре злословить Меня.
\vs Mar 9:40 Ибо кто не против вас, тот за вас.
\vs Mar 9:41 И кто напоит вас чашею воды во имя Мое, потому что вы Христовы, истинно говорю вам, не потеряет награды своей.
\vs Mar 9:42 А кто соблазнит одного из малых сих, верующих в Меня, тому лучше было бы, если бы повесили ему жерновный камень на шею и бросили его в море.
\vs Mar 9:43 И если соблазняет тебя рука твоя, отсеки ее: лучше тебе увечному войти в жизнь, нежели с двумя руками идти в геенну, в огонь неугасимый,
\vs Mar 9:44 где червь их не умирает и огонь не угасает.
\vs Mar 9:45 И если нога твоя соблазняет тебя, отсеки ее: лучше тебе войти в жизнь хромому, нежели с двумя ногами быть ввержену в геенну, в огонь неугасимый,
\vs Mar 9:46 где червь их не умирает и огонь не угасает.
\vs Mar 9:47 И если глаз твой соблазняет тебя, вырви его: лучше тебе с одним глазом войти в Царствие Божие, нежели с двумя глазами быть ввержену в геенну огненную,
\vs Mar 9:48 где червь их не умирает и огонь не угасает.
\vs Mar 9:49 Ибо всякий огнем осолится, и всякая жертва солью осолится.
\vs Mar 9:50 Соль~--- добрая \bibemph{вещь}; но ежели соль не солона будет, чем вы ее поправите? Имейте в себе соль, и мир имейте между собою.
\vs Mar 10:1 Отправившись оттуда, приходит в пределы Иудейские за Иорданскою стороною. Опять собирается к Нему народ, и, по обычаю Своему, Он опять учил их.
\vs Mar 10:2 Подошли фарисеи и спросили, искушая Его: позволительно ли разводиться мужу с женою?
\vs Mar 10:3 Он сказал им в ответ: что заповедал вам Моисей?
\vs Mar 10:4 Они сказали: Моисей позволил писать разводное письмо и разводиться.
\vs Mar 10:5 Иисус сказал им в ответ: по жестокосердию вашему он написал вам сию заповедь.
\vs Mar 10:6 В начале же создания, Бог мужчину и женщину сотворил их.
\vs Mar 10:7 Посему оставит человек отца своего и мать
\vs Mar 10:8 и прилепится к жене своей, и будут два одною плотью; так что они уже не двое, но одна плоть.
\vs Mar 10:9 Итак, что Бог сочетал, того человек да не разлучает.
\vs Mar 10:10 В доме ученики Его опять спросили Его о том же.
\vs Mar 10:11 Он сказал им: кто разведется с женою своею и женится на другой, тот прелюбодействует от нее;
\vs Mar 10:12 и если жена разведется с мужем своим и выйдет за другого, прелюбодействует.
\rsbpar\vs Mar 10:13 Приносили к Нему детей, чтобы Он прикоснулся к ним; ученики же не допускали приносящих.
\vs Mar 10:14 Увидев \bibemph{то}, Иисус вознегодовал и сказал им: пустите детей приходить ко Мне и не препятствуйте им, ибо таковых есть Царствие Божие.
\vs Mar 10:15 Истинно говорю вам: кто не примет Царствия Божия, как дитя, тот не войдет в него.
\vs Mar 10:16 И, обняв их, возложил руки на них и благословил их.
\rsbpar\vs Mar 10:17 Когда выходил Он в путь, подбежал некто, пал пред Ним на колени и спросил Его: Учитель благий! что мне делать, чтобы наследовать жизнь вечную?
\vs Mar 10:18 Иисус сказал ему: что ты называешь Меня благим? Никто не благ, как только один Бог.
\vs Mar 10:19 Знаешь заповеди: не прелюбодействуй, не убивай, не кради, не лжесвидетельствуй, не обижай, почитай отца твоего и мать.
\vs Mar 10:20 Он же сказал Ему в ответ: Учитель! всё это сохранил я от юности моей.
\vs Mar 10:21 Иисус, взглянув на него, полюбил его и сказал ему: одного тебе недостает: пойди, всё, что имеешь, продай и раздай нищим, и будешь иметь сокровище на небесах; и приходи, последуй за Мною, взяв крест.
\vs Mar 10:22 Он же, смутившись от сего слова, отошел с печалью, потому что у него было большое имение.
\vs Mar 10:23 И, посмотрев вокруг, Иисус говорит ученикам Своим: как трудно имеющим богатство войти в Царствие Божие!
\vs Mar 10:24 Ученики ужаснулись от слов Его. Но Иисус опять говорит им в ответ: дети! как трудно надеющимся на богатство войти в Царствие Божие!
\vs Mar 10:25 Удобнее верблюду пройти сквозь игольные уши, нежели богатому войти в Царствие Божие.
\vs Mar 10:26 Они же чрезвычайно изумлялись и говорили между собою: кто же может спастись?
\vs Mar 10:27 Иисус, воззрев на них, говорит: человекам это невозможно, но не Богу, ибо всё возможно Богу.
\rsbpar\vs Mar 10:28 И начал Петр говорить Ему: вот, мы оставили всё и последовали за Тобою.
\vs Mar 10:29 Иисус сказал в ответ: истинно говорю вам: нет никого, кто оставил бы дом, или братьев, или сестер, или отца, или мать, или жену, или детей, или з\acc{е}мли, ради Меня и Евангелия,
\vs Mar 10:30 и не получил бы ныне, во время сие, среди гонений, во сто крат более домов, и братьев, и сестер, и отцов, и матерей, и детей, и земель, а в веке грядущем жизни вечной.
\vs Mar 10:31 Многие же будут первые последними, и последние первыми.
\rsbpar\vs Mar 10:32 Когда были они на пути, восходя в Иерусалим, Иисус шел впереди их, а они ужасались и, следуя за Ним, были в страхе. Подозвав двенадцать, Он опять начал им говорить о том, чт\acc{о} будет с Ним:
\vs Mar 10:33 вот, мы восходим в Иерусалим, и Сын Человеческий предан будет первосвященникам и книжникам, и осудят Его на смерть, и предадут Его язычникам,
\vs Mar 10:34 и поругаются над Ним, и будут бить Его, и оплюют Его, и убьют Его; и в третий день воскреснет.
\vs Mar 10:35 \bibemph{Тогда} подошли к Нему сыновья Зеведеевы Иаков и Иоанн и сказали: Учитель! мы желаем, чтобы Ты сделал нам, о чем попросим.
\vs Mar 10:36 Он сказал им: что хотите, чтобы Я сделал вам?
\vs Mar 10:37 Они сказали Ему: дай нам сесть у Тебя, одному по правую сторону, а другому по левую в славе Твоей.
\vs Mar 10:38 Но Иисус сказал им: не знаете, чего просите. Можете ли пить чашу, которую Я пью, и креститься крещением, которым Я крещусь?
\vs Mar 10:39 Они отвечали: можем. Иисус же сказал им: чашу, которую Я пью, будете пить, и крещением, которым Я крещусь, будете креститься;
\vs Mar 10:40 а дать сесть у Меня по правую сторону и по левую~--- не от Меня \bibemph{зависит}, но кому уготовано.
\vs Mar 10:41 И, услышав, десять начали негодовать на Иакова и Иоанна.
\vs Mar 10:42 Иисус же, подозвав их, сказал им: вы знаете, что почитающиеся князьями народов господствуют над ними, и вельможи их властвуют ими.
\vs Mar 10:43 Но между вами да не будет так: а кто хочет быть б\acc{о}льшим между вами, да будем вам слугою;
\vs Mar 10:44 и кто хочет быть первым между вами, да будет всем рабом.
\vs Mar 10:45 Ибо и Сын Человеческий не для того пришел, чтобы Ему служили, но чтобы послужить и отдать душу Свою для искупления многих.
\rsbpar\vs Mar 10:46 Приходят в Иерихон. И когда выходил Он из Иерихона с учениками Своими и множеством народа, Вартимей, сын Тимеев, слепой сидел у дороги, прося \bibemph{милостыни}.
\vs Mar 10:47 Услышав, что это Иисус Назорей, он начал кричать и говорить: Иисус, Сын Давидов! помилуй меня.
\vs Mar 10:48 Многие заставляли его молчать; но он еще более стал кричать: Сын Давидов! помилуй меня.
\vs Mar 10:49 Иисус остановился и велел его позвать. Зовут слепого и говорят ему: не бойся, вставай, зовет тебя.
\vs Mar 10:50 Он сбросил с себя верхнюю одежду, встал и пришел к Иисусу.
\vs Mar 10:51 Отвечая ему, Иисус спросил: чего ты хочешь от Меня? Слепой сказал Ему: Учитель! чтобы мне прозреть.
\vs Mar 10:52 Иисус сказал ему: иди, вера твоя спасла тебя. И он тотчас прозрел и пошел за Иисусом по дороге.
\vs Mar 11:1 Когда приблизились к Иерусалиму, к Виффагии и Вифании, к горе Елеонской, \bibemph{Иисус} посылает двух из учеников Своих
\vs Mar 11:2 и говорит им: пойдите в селение, которое прямо перед вами; входя в него, тотчас найдете привязанного молодого осла, на которого никто из людей не садился; отвязав его, приведите.
\vs Mar 11:3 И если кто скажет вам: что вы это делаете?~--- отвечайте, что он надобен Господу; и тотчас пошлет его сюда.
\vs Mar 11:4 Они пошли, и нашли молодого осла, привязанного у ворот на улице, и отвязали его.
\vs Mar 11:5 И некоторые из стоявших там говорили им: что делаете? \bibemph{зачем} отвязываете осленка?
\vs Mar 11:6 Они отвечали им, к\acc{а}к повелел Иисус; и те отпустили их.
\vs Mar 11:7 И привели осленка к Иисусу, и возложили на него одежды свои; \bibemph{Иисус} сел на него.
\vs Mar 11:8 Многие же постилали одежды свои по дороге; а другие резали ветви с дерев и постилали по дороге.
\vs Mar 11:9 И предшествовавшие и сопровождавшие восклицали: осанна! благословен Грядущий во имя Господне!
\vs Mar 11:10 благословенно грядущее во имя Господа царство отца нашего Давида! осанна в вышних!
\rsbpar\vs Mar 11:11 И вошел Иисус в Иерусалим и в храм; и, осмотрев всё, как время уже было позднее, вышел в Вифанию с двенадцатью.
\rsbpar\vs Mar 11:12 На другой день, когда они вышли из Вифании, Он взалкал;
\vs Mar 11:13 и, увидев издалека смоковницу, покрытую листьями, пошел, не найдет ли чего на ней; но, придя к ней, ничего не нашел, кроме листьев, ибо еще не время было \bibemph{собирания} смокв.
\vs Mar 11:14 И сказал ей Иисус: отныне да не вкушает никто от тебя плода вовек! И слышали т\acc{о} ученики Его.
\vs Mar 11:15 Пришли в Иерусалим. Иисус, войдя в храм, начал выгонять продающих и покупающих в храме; и столы меновщиков и скамьи продающих голубей опрокинул;
\vs Mar 11:16 и не позволял, чтобы кто пронес через храм какую-либо вещь.
\vs Mar 11:17 И учил их, говоря: не написано ли: дом Мой домом молитвы наречется для всех народов? а вы сделали его вертепом разбойников.
\vs Mar 11:18 Услышали \bibemph{это} книжники и первосвященники, и искали, как бы погубить Его, ибо боялись Его, потому что весь народ удивлялся учению Его.
\vs Mar 11:19 Когда же стало поздно, Он вышел вон из города.
\rsbpar\vs Mar 11:20 Поутру, проходя мимо, увидели, что смоковница засохла до корня.
\vs Mar 11:21 И, вспомнив, Петр говорит Ему: Равв\acc{и}! посмотри, смоковница, которую Ты проклял, засохла.
\vs Mar 11:22 Иисус, отвечая, говорит им:
\vs Mar 11:23 имейте веру Божию, ибо истинно говорю вам, если кто скажет горе сей: поднимись и ввергнись в море, и не усомнится в сердце своем, но поверит, что сбудется по словам его,~--- будет ему, что ни скажет.
\vs Mar 11:24 Потому говорю вам: всё, чего ни будете просить в молитве, верьте, что получите,~--- и будет вам.
\vs Mar 11:25 И когда стоите на молитве, прощайте, если чт\acc{о} имеете на кого, дабы и Отец ваш Небесный простил вам согрешения ваши.
\vs Mar 11:26 Если же не прощаете, то и Отец ваш Небесный не простит вам согрешений ваших.
\rsbpar\vs Mar 11:27 Пришли опять в Иерусалим. И когда Он ходил в храме, подошли к Нему первосвященники и книжники, и старейшины
\vs Mar 11:28 и говорили Ему: какою властью Ты это делаешь? и кто Тебе дал власть делать это?
\vs Mar 11:29 Иисус сказал им в ответ: спрошу и Я вас об одном, отвечайте Мне; \bibemph{тогда} и Я скажу вам, какою властью это делаю.
\vs Mar 11:30 Крещение Иоанново с небес было, или от человеков? отвечайте Мне.
\vs Mar 11:31 Они рассуждали между собою: если скажем: с небес,~--- то Он скажет: почему же вы не поверили ему?
\vs Mar 11:32 а сказать: от человеков~--- боялись народа, потому что все полагали, что Иоанн точно был пророк.
\vs Mar 11:33 И сказали в ответ Иисусу: не знаем. Тогда Иисус сказал им в ответ: и Я не скажу вам, какою властью это делаю.
\vs Mar 12:1 И начал говорить им притчами: некоторый человек насадил виноградник и обнес оградою, и выкопал точило, и построил башню, и, отдав его виноградарям, отлучился.
\vs Mar 12:2 И послал в свое время к виноградарям слугу~--- принять от виноградарей плодов из виноградника.
\vs Mar 12:3 Они же, схватив его, били, и отослали ни с чем.
\vs Mar 12:4 Опять послал к ним другого слугу; и тому камнями разбили голову и отпустили его с бесчестьем.
\vs Mar 12:5 И опять иного послал: и того убили; и многих других то били, то убивали.
\vs Mar 12:6 Имея же еще одного сына, любезного ему, напоследок послал и его к ним, говоря: постыдятся сына моего.
\vs Mar 12:7 Но виноградари сказали друг другу: это наследник; пойдем, убьем его, и наследство будет наше.
\vs Mar 12:8 И, схватив его, убили и выбросили вон из виноградника.
\vs Mar 12:9 Что же сделает хозяин виноградника?~--- Придет и предаст смерти виноградарей, и отдаст виноградник другим.
\vs Mar 12:10 Неужели вы не читали сего в Писании: камень, который отвергли строители, тот самый сделался главою угла;
\vs Mar 12:11 это от Господа, и есть дивно в очах наших.
\vs Mar 12:12 И старались схватить Его, но побоялись народа, ибо поняли, что о них сказал притчу; и, оставив Его, отошли.
\rsbpar\vs Mar 12:13 И посылают к Нему некоторых из фарисеев и иродиан, чтобы уловить Его в слове.
\vs Mar 12:14 Они же, придя, говорят Ему: Учитель! мы знаем, что Ты справедлив и не заботишься об угождении кому-либо, ибо не смотришь ни на какое лице, но истинно пути Божию учишь. Позволительно ли давать п\acc{о}дать кесарю или нет? давать ли нам или не давать?
\vs Mar 12:15 Но Он, зная их лицемерие, сказал им: что искушаете Меня? принесите Мне динарий, чтобы Мне видеть его.
\vs Mar 12:16 Они принесли. Тогда говорит им: чье это изображение и надпись? Они сказали Ему: кесаревы.
\vs Mar 12:17 Иисус сказал им в ответ: отдавайте кесарево кесарю, а Божие Богу. И дивились Ему.
\rsbpar\vs Mar 12:18 Потом пришли к Нему саддукеи, которые говорят, что нет воскресения, и спросили Его, говоря:
\vs Mar 12:19 Учитель! Моисей написал нам: если у кого умрет брат и оставит жену, а детей не оставит, то брат его пусть возьмет жену его и восстановит семя брату своему.
\vs Mar 12:20 Было семь братьев: первый взял жену и, умирая, не оставил детей.
\vs Mar 12:21 Взял ее второй и умер, и он не оставил детей; также и третий.
\vs Mar 12:22 Брали ее \bibemph{за себя} семеро и не оставили детей. После всех умерла и жена.
\vs Mar 12:23 Итак, в воскресении, когда воскреснут, которого из них будет она женою? Ибо семеро имели ее женою.
\vs Mar 12:24 Иисус сказал им в ответ: этим ли приводитесь вы в заблуждение, не зная Писаний, ни силы Божией?
\vs Mar 12:25 Ибо, когда из мертвых воскреснут, \bibemph{тогда} не будут ни жениться, ни замуж выходить, но будут, как Ангелы на небесах.
\vs Mar 12:26 А о мертвых, что они воскреснут, разве не читали вы в книге Моисея, как Бог при купине сказал ему: Я Бог Авраама, и Бог Исаака, и Бог Иакова?
\vs Mar 12:27 \bibemph{Бог} не есть Бог мертвых, но Бог живых. Итак, вы весьма заблуждаетесь.
\rsbpar\vs Mar 12:28 Один из книжников, слыша их прения и видя, что \bibemph{Иисус} хорошо им отвечал, подошел и спросил Его: какая первая из всех заповедей?
\vs Mar 12:29 Иисус отвечал ему: первая из всех заповедей: слушай, Израиль! Господь Бог наш есть Господь единый;
\vs Mar 12:30 и возлюби Господа Бога твоего всем сердцем твоим, и всею душею твоею, и всем разумением твоим, и всею крепостию твоею,~--- вот первая заповедь!
\vs Mar 12:31 Вторая подобная ей: возлюби ближнего твоего, как самого себя. Иной большей сих заповеди нет.
\vs Mar 12:32 Книжник сказал Ему: хорошо, Учитель! истину сказал Ты, что один есть Бог и нет иного, кроме Его;
\vs Mar 12:33 и любить Его всем сердцем и всем умом, и всею душею, и всею крепостью, и любить ближнего, как самого себя, есть больше всех всесожжений и жертв.
\vs Mar 12:34 Иисус, видя, что он разумно отвечал, сказал ему: недалеко ты от Царствия Божия. После того никто уже не смел спрашивать Его.
\rsbpar\vs Mar 12:35 Продолжая учить в храме, Иисус говорил: как говорят книжники, что Христос есть Сын Давидов?
\vs Mar 12:36 Ибо сам Давид сказал Духом Святым: сказал Господь Господу моему: седи одесную Меня, доколе положу врагов Твоих в подножие ног Твоих.
\vs Mar 12:37 Итак, сам Давид называет Его Господом: как же Он Сын ему? И множество народа слушало Его с услаждением.
\vs Mar 12:38 И говорил им в учении Своем: остерегайтесь книжников, любящих ходить в длинных одеждах и \bibemph{принимать} приветствия в народных собраниях,
\vs Mar 12:39 сидеть впереди в синагогах и возлежать на первом \bibemph{месте} на пиршествах,~---
\vs Mar 12:40 сии, поядающие домы вдов и напоказ долго молящиеся, примут тягчайшее осуждение.
\rsbpar\vs Mar 12:41 И сел Иисус против сокровищницы и смотрел, как народ кладет деньги в сокровищницу. Многие богатые клали много.
\vs Mar 12:42 Придя же, одна бедная вдова положила две лепты, что составляет кодрант.
\vs Mar 12:43 Подозвав учеников Своих, \bibemph{Иисус} сказал им: истинно говорю вам, что эта бедная вдова положила больше всех, клавших в сокровищницу,
\vs Mar 12:44 ибо все клали от избытка своего, а она от скудости своей положила всё, что имела, всё пропитание свое.
\vs Mar 13:1 И когда выходил Он из храма, говорит Ему один из учеников Его: Учитель! посмотри, какие камни и какие здания!
\vs Mar 13:2 Иисус сказал ему в ответ: видишь сии великие здания? всё это будет разрушено, так что не останется здесь камня на камне.
\vs Mar 13:3 И когда Он сидел на горе Елеонской против храма, спрашивали Его наедине Петр, и Иаков, и Иоанн, и Андрей:
\vs Mar 13:4 скажи нам, когда это будет, и какой признак, когда всё сие должно совершиться?
\vs Mar 13:5 Отвечая им, Иисус начал говорить: берегитесь, чтобы кто не прельстил вас,
\vs Mar 13:6 ибо многие придут под именем Моим и будут говорить, что это Я; и многих прельстят.
\vs Mar 13:7 Когда же услышите о войнах и о военных слухах, не ужасайтесь: ибо надлежит \bibemph{сему} быть,~--- но \bibemph{это} еще не конец.
\vs Mar 13:8 Ибо восстанет народ на народ и царство на царство; и будут землетрясения по местам, и будут глады и смятения. Это~--- начало болезней.
\vs Mar 13:9 Но вы смотр\acc{и}те за собою, ибо вас будут предавать в судилища и бить в синагогах, и перед правителями и царями поставят вас за Меня, для свидетельства перед ними.
\vs Mar 13:10 И во всех народах прежде должно быть проповедано Евангелие.
\vs Mar 13:11 Когда же поведут предавать вас, не заботьтесь наперед, чт\acc{о} вам говорить, и не обдумывайте; но чт\acc{о} дано будет вам в тот час, т\acc{о} и говорите, ибо не вы будете говорить, но Дух Святый.
\vs Mar 13:12 Предаст же брат брата на смерть, и отец~--- детей; и восстанут дети на родителей и умертвят их.
\vs Mar 13:13 И будете ненавидимы всеми за имя Мое; претерпевший же до конца спасется.
\vs Mar 13:14 Когда же увидите мерзость запустения, реченную пророком Даниилом, стоящую, где не должно,~--- читающий да разумеет,~--- тогда находящиеся в Иудее да бегут в горы;
\vs Mar 13:15 а кто на кровле, тот не сходи в дом и не входи взять что-нибудь из дома своего;
\vs Mar 13:16 и кто на поле, не обращайся назад взять одежду свою.
\vs Mar 13:17 Горе беременным и питающим сосцами в те дни.
\vs Mar 13:18 Мол\acc{и}тесь, чтобы не случилось бегство ваше зимою.
\vs Mar 13:19 Ибо в те дни будет такая скорбь, какой не было от начала творения, которое сотворил Бог, даже доныне, и не будет.
\vs Mar 13:20 И если бы Господь не сократил тех дней, то не спаслась бы никакая плоть; но ради избранных, которых Он избрал, сократил те дни.
\vs Mar 13:21 Тогда, если кто вам скажет: вот, здесь Христос, или: вот, там,~--- не верьте.
\vs Mar 13:22 Ибо восстанут лжехристы и лжепророки и дадут знамения и чудеса, чтобы прельстить, если возможно, и избранных.
\vs Mar 13:23 Вы же берегитесь. Вот, Я наперед сказал вам всё.
\vs Mar 13:24 Но в те дни, после скорби той, солнце померкнет, и луна не даст света своего,
\vs Mar 13:25 и звезды спадут с неба, и силы небесные поколеблются.
\vs Mar 13:26 Тогда увидят Сына Человеческого, грядущего на облаках с силою многою и славою.
\vs Mar 13:27 И тогда Он пошлет Ангелов Своих и соберет избранных Своих от четырех ветров, от края земли до края неба.
\vs Mar 13:28 От смоковницы возьмите подобие: когда ветви ее становятся уже мягки и пускают листья, то знаете, что близко лето.
\vs Mar 13:29 Так и когда вы увидите т\acc{о} сбывающимся, знайте, что близко, при дверях.
\vs Mar 13:30 Истинно говорю вам: не прейдет род сей, как всё это будет.
\vs Mar 13:31 Небо и земля прейдут, но слова Мои не прейдут.
\vs Mar 13:32 О дне же том, или часе, никто не знает, ни Ангелы небесные, ни Сын, но только Отец.
\vs Mar 13:33 Смотрите, бодрствуйте, молитесь, ибо не знаете, когда наступит это время.
\vs Mar 13:34 Подобно как бы кто, отходя в путь и оставляя дом свой, дал слугам своим власть и каждому свое дело, и приказал привратнику бодрствовать.
\vs Mar 13:35 Итак бодрствуйте, ибо не знаете, когда придет хозяин дома: вечером, или в полночь, или в пение петухов, или поутру;
\vs Mar 13:36 чтобы, придя внезапно, не нашел вас спящими.
\vs Mar 13:37 А чт\acc{о} вам говорю, говорю всем: бодрствуйте.
\vs Mar 14:1 Через два дня \bibemph{надлежало} быть \bibemph{празднику} Пасхи и опресноков. И искали первосвященники и книжники, как бы взять Его хитростью и убить;
\vs Mar 14:2 но говорили: \bibemph{только} не в праздник, чтобы не произошло возмущения в народе.
\rsbpar\vs Mar 14:3 И когда был Он в Вифании, в доме Симона прокаженного, и возлежал,~--- пришла женщина с алавастровым сосудом мира из нарда чистого, драгоценного и, разбив сосуд, возлила Ему на голову.
\vs Mar 14:4 Некоторые же вознегодовали и говорили между собою: к чему сия трата мира?
\vs Mar 14:5 Ибо можно было бы продать его более нежели за триста динариев и раздать нищим. И роптали на нее.
\vs Mar 14:6 Но Иисус сказал: оставьте ее; чт\acc{о} ее смущаете? Она доброе дело сделала для Меня.
\vs Mar 14:7 Ибо нищих всегда имеете с собою и, когда захотите, можете им благотворить; а Меня не всегда имеете.
\vs Mar 14:8 Она сделала, чт\acc{о} могла: предварила помазать тело Мое к погребению.
\vs Mar 14:9 Истинно говорю вам: где ни будет проповедано Евангелие сие в целом мире, сказано будет, в память ее, и о том, чт\acc{о} она сделала.
\rsbpar\vs Mar 14:10 И пошел Иуда Искариот, один из двенадцати, к первосвященникам, чтобы предать Его им.
\vs Mar 14:11 Они же, услышав, обрадовались, и обещали дать ему сребреники. И он искал, как бы в удобное время предать Его.
\rsbpar\vs Mar 14:12 В первый день опресноков, когда заколали пасхального \bibemph{агнца}, говорят Ему ученики Его: где хочешь есть пасху? мы пойдем и приготовим.
\vs Mar 14:13 И посылает двух из учеников Своих и говорит им: пойдите в город; и встретится вам человек, несущий кувшин воды; последуйте за ним
\vs Mar 14:14 и куда он войдет, скажите хозяину дома того: Учитель говорит: где комната, в которой бы Мне есть пасху с учениками Моими?
\vs Mar 14:15 И он покажет вам горницу большую, устланную, готовую: там приготовьте нам.
\vs Mar 14:16 И пошли ученики Его, и пришли в город, и нашли, как сказал им; и приготовили пасху.
\vs Mar 14:17 Когда настал вечер, Он приходит с двенадцатью.
\vs Mar 14:18 И, когда они возлежали и ели, Иисус сказал: истинно говорю вам, один из вас, ядущий со Мною, предаст Меня.
\vs Mar 14:19 Они опечалились и стали говорить Ему, один за другим: не я ли? и другой: не я ли?
\vs Mar 14:20 Он же сказал им в ответ: один из двенадцати, обмакивающий со Мною в блюдо.
\vs Mar 14:21 Впрочем Сын Человеческий идет, как писано о Нем; но горе тому человеку, которым Сын Человеческий предается: лучше было бы тому человеку не родиться.
\rsbpar\vs Mar 14:22 И когда они ели, Иисус, взяв хлеб, благословил, преломил, дал им и сказал: приимите, ядите; сие есть Тело Мое.
\vs Mar 14:23 И, взяв чашу, благодарив, подал им: и пили из нее все.
\vs Mar 14:24 И сказал им: сие есть Кровь Моя Нового Завета, за многих изливаемая.
\vs Mar 14:25 Истинно говорю вам: Я уже не буду пить от плода виноградного до того дня, когда буду пить новое вино в Царствии Божием.
\rsbpar\vs Mar 14:26 И, воспев, пошли на гору Елеонскую.
\vs Mar 14:27 И говорит им Иисус: все вы соблазнитесь о Мне в эту ночь; ибо написано: поражу пастыря, и рассеются овцы.
\vs Mar 14:28 По воскресении же Моем, Я предварю вас в Галилее.
\vs Mar 14:29 Петр сказал Ему: если и все соблазнятся, но не я.
\vs Mar 14:30 И говорит ему Иисус: истинно говорю тебе, что ты ныне, в эту ночь, прежде нежели дважды пропоет петух, трижды отречешься от Меня.
\vs Mar 14:31 Но он еще с б\acc{о}льшим усилием говорил: хотя бы мне надлежало и умереть с Тобою, не отрекусь от Тебя. Т\acc{о} же и все говорили.
\rsbpar\vs Mar 14:32 Пришли в селение, называемое Гефсимания; и Он сказал ученикам Своим: посидите здесь, пока Я помолюсь.
\vs Mar 14:33 И взял с Собою Петра, Иакова и Иоанна; и начал ужасаться и тосковать.
\vs Mar 14:34 И сказал им: душа Моя скорбит смертельно; побудьте здесь и бодрствуйте.
\vs Mar 14:35 И, отойдя немного, пал на землю и молился, чтобы, если возможно, миновал Его час сей;
\vs Mar 14:36 и говорил: Авва Отче! всё возможно Тебе; пронеси чашу сию мимо Меня; но не чего Я хочу, а чего Ты.
\vs Mar 14:37 Возвращается и находит их спящими, и говорит Петру: Симон! ты спишь? не мог ты бодрствовать один час?
\vs Mar 14:38 Бодрствуйте и молитесь, чтобы не впасть в искушение: дух бодр, плоть же немощна.
\vs Mar 14:39 И, опять отойдя, молился, сказав то же слово.
\vs Mar 14:40 И, возвратившись, опять нашел их спящими, ибо глаза у них отяжелели, и они не знали, чт\acc{о} Ему отвечать.
\vs Mar 14:41 И приходит в третий раз и говорит им: вы всё еще спите и почиваете? Кончено, пришел час: вот, предается Сын Человеческий в руки грешников.
\vs Mar 14:42 Встаньте, пойдем; вот, приблизился предающий Меня.
\rsbpar\vs Mar 14:43 И тотчас, как Он еще говорил, приходит Иуда, один из двенадцати, и с ним множество народа с мечами и кольями, от первосвященников и книжников и старейшин.
\vs Mar 14:44 Предающий же Его дал им знак, сказав: Кого я поцелую, Тот и есть, возьмите Его и ведите осторожно.
\vs Mar 14:45 И, придя, тотчас подошел к Нему и говорит: Равв\acc{и}! Равв\acc{и}! и поцеловал Его.
\vs Mar 14:46 А они возложили на Него руки свои и взяли Его.
\vs Mar 14:47 Один же из стоявших тут извлек меч, ударил раба первосвященникова и отсек ему ухо.
\vs Mar 14:48 Тогда Иисус сказал им: как будто на разбойника вышли вы с мечами и кольями, чтобы взять Меня.
\vs Mar 14:49 Каждый день бывал Я с вами в храме и учил, и вы не брали Меня. Но да сбудутся Писания.
\vs Mar 14:50 Тогда, оставив Его, все бежали.
\vs Mar 14:51 Один юноша, завернувшись по нагому телу в покрывало, следовал за Ним; и воины схватили его.
\vs Mar 14:52 Но он, оставив покрывало, нагой убежал от них.
\rsbpar\vs Mar 14:53 И привели Иисуса к первосвященнику; и собрались к нему все первосвященники и старейшины и книжники.
\vs Mar 14:54 Петр издали следовал за Ним, даже внутрь двора первосвященникова; и сидел со служителями, и грелся у огня.
\vs Mar 14:55 Первосвященники же и весь синедрион искали свидетельства на Иисуса, чтобы предать Его смерти; и не находили.
\vs Mar 14:56 Ибо многие лжесвидетельствовали на Него, но свидетельства сии не были достаточны.
\vs Mar 14:57 И некоторые, встав, лжесвидетельствовали против Него и говорили:
\vs Mar 14:58 мы слышали, как Он говорил: Я разрушу храм сей рукотворенный, и через три дня воздвигну другой, нерукотворенный.
\vs Mar 14:59 Но и такое свидетельство их не было достаточно.
\vs Mar 14:60 Тогда первосвященник стал посреди и спросил Иисуса: чт\acc{о} Ты ничего не отвечаешь? чт\acc{о} они против Тебя свидетельствуют?
\vs Mar 14:61 Но Он молчал и не отвечал ничего. Опять первосвященник спросил Его и сказал Ему: Ты ли Христос, Сын Благословенного?
\vs Mar 14:62 Иисус сказал: Я; и вы \acc{у}зрите Сына Человеческого, сидящего одесную силы и грядущего на облаках небесных.
\vs Mar 14:63 Тогда первосвященник, разодрав одежды свои, сказал: на что еще нам свидетелей?
\vs Mar 14:64 Вы слышали богохульство; как вам кажется? Они же все признали Его повинным смерти.
\vs Mar 14:65 И некоторые начали плевать на Него и, закрывая Ему лице, ударять Его и говорить Ему: прореки. И слуги били Его по ланитам.
\rsbpar\vs Mar 14:66 Когда Петр был на дворе внизу, пришла одна из служанок первосвященника
\vs Mar 14:67 и, увидев Петра греющегося и всмотревшись в него, сказала: и ты был с Иисусом Назарянином.
\vs Mar 14:68 Но он отрекся, сказав: не знаю и не понимаю, что ты говоришь. И вышел вон на передний двор; и запел петух.
\vs Mar 14:69 Служанка, увидев его опять, начала говорить стоявшим тут: этот из них.
\vs Mar 14:70 Он опять отрекся. Спустя немного, стоявшие тут опять стали говорить Петру: точно ты из них; ибо ты Галилеянин, и наречие твое сходно.
\vs Mar 14:71 Он же начал клясться и божиться: не знаю Человека Сего, о Котором говорите.
\vs Mar 14:72 Тогда петух запел во второй раз. И вспомнил Петр слово, сказанное ему Иисусом: прежде нежели петух пропоет дважды, трижды отречешься от Меня; и начал плакать.
\vs Mar 15:1 Немедленно поутру первосвященники со старейшинами и книжниками и весь синедрион составили совещание и, связав Иисуса, отвели и предали Пилату.
\vs Mar 15:2 Пилат спросил Его: Ты Царь Иудейский? Он же сказал ему в ответ: ты говоришь.
\vs Mar 15:3 И первосвященники обвиняли Его во многом.
\vs Mar 15:4 Пилат же опять спросил Его: Ты ничего не отвечаешь? видишь, как много против Тебя обвинений.
\vs Mar 15:5 Но Иисус и на это ничего не отвечал, так что Пилат дивился.
\vs Mar 15:6 На всякий же праздник отпускал он им одного узника, о котором просили.
\vs Mar 15:7 Тогда был в узах \bibemph{некто}, по имени Варавва, со своими сообщниками, которые во время мятежа сделали убийство.
\vs Mar 15:8 И народ начал кричать и просить \bibemph{Пилата} о том, чт\acc{о} он всегда делал для них.
\vs Mar 15:9 Он сказал им в ответ: хотите ли, отпущу вам Царя Иудейского?
\vs Mar 15:10 Ибо знал, что первосвященники предали Его из зависти.
\vs Mar 15:11 Но первосвященники возбудили народ \bibemph{просить}, чтобы отпустил им лучше Варавву.
\vs Mar 15:12 Пилат, отвечая, опять сказал им: что же хотите, чтобы я сделал с Тем, Которого вы называете Царем Иудейским?
\vs Mar 15:13 Они опять закричали: распни Его.
\vs Mar 15:14 Пилат сказал им: какое же зло сделал Он? Но они еще сильнее закричали: распни Его.
\vs Mar 15:15 Тогда Пилат, желая сделать угодное народу, отпустил им Варавву, а Иисуса, бив, предал на распятие.
\rsbpar\vs Mar 15:16 А воины отвели Его внутрь двора, то есть в преторию, и собрали весь полк,
\vs Mar 15:17 и одели Его в багряницу, и, сплетши терновый венец, возложили на Него;
\vs Mar 15:18 и начали приветствовать Его: радуйся, Царь Иудейский!
\vs Mar 15:19 И били Его по голове тростью, и плевали на Него, и, становясь на колени, кланялись Ему.
\rsbpar\vs Mar 15:20 Когда же насмеялись над Ним, сняли с Него багряницу, одели Его в собственные одежды Его и повели Его, чтобы распять Его.
\vs Mar 15:21 И заставили проходящего некоего Киринеянина Симона, отца Александрова и Руфова, идущего с поля, нести крест Его.
\vs Mar 15:22 И привели Его на место Голгофу, чт\acc{о} значит: Лобное место.
\vs Mar 15:23 И давали Ему пить вино со смирною; но Он не принял.
\vs Mar 15:24 Распявшие Его делили одежды Его, бросая жребий, кому чт\acc{о} взять.
\vs Mar 15:25 Был час третий, и распяли Его.
\vs Mar 15:26 И была надпись вины Его: Царь Иудейский.
\vs Mar 15:27 С Ним распяли двух разбойников, одного по правую, а другого по левую \bibemph{сторону} Его.
\vs Mar 15:28 И сбылось слово Писания: и к злодеям причтен.
\vs Mar 15:29 Проходящие злословили Его, кивая головами своими и говоря: э! разрушающий храм, и в три дня созидающий!
\vs Mar 15:30 спаси Себя Самого и сойди со креста.
\vs Mar 15:31 Подобно и первосвященники с книжниками, насмехаясь, говорили друг другу: других спасал, а Себя не может спасти.
\vs Mar 15:32 Христос, Царь Израилев, пусть сойдет теперь с креста, чтобы мы видели, и уверуем. И распятые с Ним поносили Его.
\rsbpar\vs Mar 15:33 В шестом же часу настала тьма по всей земле и \bibemph{продолжалась} до часа девятого.
\vs Mar 15:34 В девятом часу возопил Иисус громким голосом: Эло\acc{и}! Эло\acc{и}! ламм\acc{а} савахфан\acc{и}?~--- что значит: Боже Мой! Боже Мой! для чего Ты Меня оставил?
\vs Mar 15:35 Некоторые из стоявших тут, услышав, говорили: вот, Илию зовет.
\vs Mar 15:36 А один побежал, наполнил губку уксусом и, наложив на трость, давал Ему пить, говоря: постойте, посмотрим, придет ли Илия снять Его.
\vs Mar 15:37 Иисус же, возгласив громко, испустил дух.
\vs Mar 15:38 И завеса в храме раздралась надвое, сверху донизу.
\vs Mar 15:39 Сотник, стоявший напротив Его, увидев, что Он, т\acc{а}к возгласив, испустил дух, сказал: истинно Человек Сей был Сын Божий.
\vs Mar 15:40 Были \bibemph{тут} и женщины, которые смотрели издали: между ними была и Мария Магдалина, и Мария, мать Иакова меньшего и Иосии, и Саломия,
\vs Mar 15:41 которые и тогда, как Он был в Галилее, следовали за Ним и служили Ему, и другие многие, вместе с Ним пришедшие в Иерусалим.
\rsbpar\vs Mar 15:42 И как уже настал вечер,~--- потому что была пятница, то есть \bibemph{день} перед субботою,~---
\vs Mar 15:43 пришел Иосиф из Аримафеи, знаменитый член совета, который и сам ожидал Царствия Божия, осмелился войти к Пилату, и просил тела Иисусова.
\vs Mar 15:44 Пилат удивился, что Он уже умер, и, призвав сотника, спросил его, давно ли умер?
\vs Mar 15:45 И, узнав от сотника, отдал тело Иосифу.
\vs Mar 15:46 Он, купив плащаницу и сняв Его, обвил плащаницею, и положил Его во гробе, который был высечен в скале, и привалил камень к двери гроба.
\vs Mar 15:47 Мария же Магдалина и Мария Иосиева смотрели, где Его полагали.
\vs Mar 16:1 По прошествии субботы Мария Магдалина и Мария Иаковлева и Саломия купили ароматы, чтобы идти помазать Его.
\vs Mar 16:2 И весьма рано, в первый \bibemph{день} недели, приходят ко гробу, при восходе солнца,
\vs Mar 16:3 и говорят между собою: кто отвалит нам камень от двери гроба?
\vs Mar 16:4 И, взглянув, видят, что камень отвален; а он был весьма велик.
\vs Mar 16:5 И, войдя во гроб, увидели юношу, сидящего на правой стороне, облеченного в белую одежду; и ужаснулись.
\vs Mar 16:6 Он же говорит им: не ужасайтесь. Иисуса ищете Назарянина, распятого; Он воскрес, Его нет здесь. Вот место, где Он был положен.
\vs Mar 16:7 Но идите, скажите ученикам Его и Петру, что Он предваряет вас в Галилее; там Его увидите, как Он сказал вам.
\vs Mar 16:8 И, выйдя, побежали от гроба; их объял трепет и ужас, и никому ничего не сказали, потому что боялись.
\rsbpar\vs Mar 16:9 Воскреснув рано в первый \bibemph{день} недели, \bibemph{Иисус} явился сперва Марии Магдалине, из которой изгнал семь бесов.
\vs Mar 16:10 Она пошла и возвестила бывшим с Ним, плачущим и рыдающим;
\vs Mar 16:11 но они, услышав, что Он жив и она видела Его,~--- не поверили.
\rsbpar\vs Mar 16:12 После сего явился в ином образе двум из них на дороге, когда они шли в селение.
\vs Mar 16:13 И те, возвратившись, возвестили прочим; но и им не поверили.
\rsbpar\vs Mar 16:14 Наконец, явился самим одиннадцати, возлежавшим \bibemph{на вечери}, и упрекал их за неверие и жестокосердие, что видевшим Его воскресшего не поверили.
\vs Mar 16:15 И сказал им: идите по всему миру и проповедуйте Евангелие всей твари.
\vs Mar 16:16 Кто будет веровать и креститься, спасен будет; а кто не будет веровать, осужден будет.
\vs Mar 16:17 Уверовавших же будут сопровождать сии знамения: именем Моим будут изгонять бесов; будут говорить новыми языками;
\vs Mar 16:18 будут брать змей; и если чт\acc{о} смертоносное выпьют, не повредит им; возложат руки на больных, и они будут здоровы.
\rsbpar\vs Mar 16:19 И так Господь, после беседования с ними, вознесся на небо и воссел одесную Бога.
\vs Mar 16:20 А они пошли и проповедовали везде, при Господнем содействии и подкреплении слова последующими знамениями. Аминь.

\bibbookdescr{Luk}{
  inline={От Луки\\\LARGE святое благовествование},
  toc={От Луки},
  bookmark={От Луки},
  header={От Луки},
  %headerleft={},
  %headerright={},
  abbr={Лк}
}
\vs Luk 1:1 Как уже многие начали составлять повествования о совершенно известных между нами событиях,
\vs Luk 1:2 как передали нам т\acc{о} бывшие с самого начала очевидцами и служителями Слова,
\vs Luk 1:3 то рассудилось и мне, по тщательном исследовании всего сначала, по порядку описать тебе, достопочтенный Феофил,
\vs Luk 1:4 чтобы ты узнал твердое основание того учения, в котором был наставлен.
\rsbpar\vs Luk 1:5 Во дни Ирода, царя Иудейского, был священник из Авиевой чреды, именем Захария, и жена его из рода Ааронова, имя ей Елисавета.
\vs Luk 1:6 Оба они были праведны пред Богом, поступая по всем заповедям и уставам Господним беспорочно.
\vs Luk 1:7 У них не было детей, ибо Елисавета была неплодна, и оба были уже в летах преклонных.
\vs Luk 1:8 Однажды, когда он в порядке своей чреды служил пред Богом,
\vs Luk 1:9 по жребию, как обыкновенно было у священников, досталось ему войти в храм Господень для каждения,
\vs Luk 1:10 а всё множество народа молилось вне во время каждения,~---
\vs Luk 1:11 тогда явился ему Ангел Господень, стоя по правую сторону жертвенника кадильного.
\vs Luk 1:12 Захария, увидев его, смутился, и страх напал на него.
\vs Luk 1:13 Ангел же сказал ему: не бойся, Захария, ибо услышана молитва твоя, и жена твоя Елисавета родит тебе сына, и наречешь ему имя: Иоанн;
\vs Luk 1:14 и будет тебе радость и веселие, и многие о рождении его возрадуются,
\vs Luk 1:15 ибо он будет велик пред Господом; не будет пить вина и сикера, и Духа Святаго исполнится еще от чрева матери своей;
\vs Luk 1:16 и многих из сынов Израилевых обратит к Господу Богу их;
\vs Luk 1:17 и предъидет пред Ним в духе и силе Илии, чтобы возвратить сердца отцов детям, и непокоривым образ мыслей праведников, дабы представить Господу народ приготовленный.
\vs Luk 1:18 И сказал Захария Ангелу: по чему я узн\acc{а}ю это? ибо я стар, и жена моя в летах преклонных.
\vs Luk 1:19 Ангел сказал ему в ответ: я Гавриил, предстоящий пред Богом, и послан говорить с тобою и благовестить тебе сие;
\vs Luk 1:20 и вот, ты будешь молчать и не будешь иметь возможности говорить до того дня, как это сбудется, за т\acc{о}, что ты не поверил словам моим, которые сбудутся в свое время.
\vs Luk 1:21 Между тем народ ожидал Захарию и дивился, что он медлит в храме.
\vs Luk 1:22 Он же, выйдя, не мог говорить к ним; и они поняли, что он видел видение в храме; и он объяснялся с ними знаками, и оставался нем.
\vs Luk 1:23 А когда окончились дни службы его, возвратился в дом свой.
\vs Luk 1:24 После сих дней зачала Елисавета, жена его, и таилась пять месяцев и говорила:
\vs Luk 1:25 так сотворил мне Господь во дни сии, в которые призрел на меня, чтобы снять с меня поношение между людьми.
\rsbpar\vs Luk 1:26 В шестой же месяц послан был Ангел Гавриил от Бога в город Галилейский, называемый Назарет,
\vs Luk 1:27 к Деве, обрученной мужу, именем Иосифу, из дома Давидова; имя же Деве: Мария.
\vs Luk 1:28 Ангел, войдя к Ней, сказал: радуйся, Благодатная! Господь с Тобою; благословенна Ты между женами.
\vs Luk 1:29 Она же, увидев его, смутилась от слов его и размышляла, чт\acc{о} бы это было за приветствие.
\vs Luk 1:30 И сказал Ей Ангел: не бойся, Мария, ибо Ты обрела благодать у Бога;
\vs Luk 1:31 и вот, зачнешь во чреве, и родишь Сына, и наречешь Ему имя: Иисус.
\vs Luk 1:32 Он будет велик и наречется Сыном Всевышнего, и даст Ему Господь Бог престол Давида, отца Его;
\vs Luk 1:33 и будет царствовать над домом Иакова во веки, и Царству Его не будет конца.
\vs Luk 1:34 Мария же сказала Ангелу: к\acc{а}к будет это, когда Я мужа не знаю?
\vs Luk 1:35 Ангел сказал Ей в ответ: Дух Святый найдет на Тебя, и сила Всевышнего осенит Тебя; посему и рождаемое Святое наречется Сыном Божиим.
\vs Luk 1:36 Вот и Елисавета, родственница Твоя, называемая неплодною, и она зачала сына в старости своей, и ей уже шестой месяц,
\vs Luk 1:37 ибо у Бога не останется бессильным никакое слово.
\vs Luk 1:38 Тогда Мария сказала: се, Раба Господня; да будет Мне по слову твоему. И отошел от Нее Ангел.
\rsbpar\vs Luk 1:39 Встав же Мария во дни сии, с поспешностью пошла в нагорную страну, в город Иудин,
\vs Luk 1:40 и вошла в дом Захарии, и приветствовала Елисавету.
\vs Luk 1:41 Когда Елисавета услышала приветствие Марии, взыграл младенец во чреве ее; и Елисавета исполнилась Святаго Духа,
\vs Luk 1:42 и воскликнула громким голосом, и сказала: благословенна Ты между женами, и благословен плод чрева Твоего!
\vs Luk 1:43 И откуда это мне, что пришла Матерь Господа моего ко мне?
\vs Luk 1:44 Ибо когда голос приветствия Твоего дошел до слуха моего, взыграл младенец радостно во чреве моем.
\vs Luk 1:45 И блаженна Уверовавшая, потому что совершится сказанное Ей от Господа.
\vs Luk 1:46 И сказала Мария: величит душа Моя Господа,
\vs Luk 1:47 и возрадовался дух Мой о Боге, Спасителе Моем,
\vs Luk 1:48 что призрел Он на смирение Рабы Своей, ибо отныне будут ублажать Меня все роды;
\vs Luk 1:49 что сотворил Мне величие Сильный, и свято имя Его;
\vs Luk 1:50 и милость Его в роды родов к боящимся Его;
\vs Luk 1:51 явил силу мышцы Своей; рассеял надменных помышлениями с\acc{е}рдца их;
\vs Luk 1:52 низложил сильных с престолов, и вознес смиренных;
\vs Luk 1:53 алчущих исполнил благ, и богатящихся отпустил ни с чем;
\vs Luk 1:54 воспринял Израиля, отрока Своего, воспомянув милость,
\vs Luk 1:55 к\acc{а}к говорил отцам нашим, к Аврааму и семени его до века.
\vs Luk 1:56 Пребыла же Мария с нею около трех месяцев, и возвратилась в дом свой.
\rsbpar\vs Luk 1:57 Елисавете же настало время родить, и она родила сына.
\vs Luk 1:58 И услышали соседи и родственники ее, что возвеличил Господь милость Свою над нею, и радовались с нею.
\vs Luk 1:59 В восьмой день пришли обрезать младенца и хотели назвать его, по имени отца его, Захариею.
\vs Luk 1:60 На это мать его сказала: нет, а назвать его Иоанном.
\vs Luk 1:61 И сказали ей: никого нет в родстве твоем, кто назывался бы сим именем.
\vs Luk 1:62 И спрашивали знаками у отца его, к\acc{а}к бы он хотел назвать его.
\vs Luk 1:63 Он потребовал дощечку и написал: Иоанн имя ему. И все удивились.
\vs Luk 1:64 И тотчас разрешились уста его и язык его, и он стал говорить, благословляя Бога.
\vs Luk 1:65 И был страх на всех живущих вокруг них; и рассказывали обо всем этом по всей нагорной стране Иудейской.
\vs Luk 1:66 Все слышавшие положили это на сердце своем и говорили: чт\acc{о} будет младенец сей? И рука Господня была с ним.
\vs Luk 1:67 И Захария, отец его, исполнился Святаго Духа и пророчествовал, говоря:
\vs Luk 1:68 благословен Господь Бог Израилев, что посетил народ Свой и сотворил избавление ему,
\vs Luk 1:69 и воздвиг рог спасения нам в дому Давида, отрока Своего,
\vs Luk 1:70 к\acc{а}к возвестил устами бывших от века святых пророков Своих,
\vs Luk 1:71 что спасет нас от врагов наших и от руки всех ненавидящих нас;
\vs Luk 1:72 сотворит милость с отцами нашими и помянет святой завет Свой,
\vs Luk 1:73 клятву, которою клялся Он Аврааму, отцу нашему, дать нам,
\vs Luk 1:74 небоязненно, по избавлении от руки врагов наших,
\vs Luk 1:75 служить Ему в святости и правде пред Ним, во все дни жизни нашей.
\vs Luk 1:76 И ты, младенец, наречешься пророком Всевышнего, ибо предъидешь пред лицем Господа приготовить пути Ему,
\vs Luk 1:77 дать уразуметь народу Его спасение в прощении грехов их,
\vs Luk 1:78 по благоутробному милосердию Бога нашего, которым посетил нас Восток свыше,
\vs Luk 1:79 просветить сидящих во тьме и тени смертной, направить ноги наши на путь мира.
\rsbpar\vs Luk 1:80 Младенец же возрастал и укреплялся духом, и был в пустынях до дня явления своего Израилю.
\vs Luk 2:1 В те дни вышло от кесаря Августа повеление сделать перепись по всей земле.
\vs Luk 2:2 Эта перепись была первая в правление Квириния Сириею.
\vs Luk 2:3 И пошли все записываться, каждый в свой город.
\vs Luk 2:4 Пошел также и Иосиф из Галилеи, из города Назарета, в Иудею, в город Давидов, называемый Вифлеем, потому что он был из дома и рода Давидова,
\vs Luk 2:5 записаться с Мариею, обрученною ему женою, которая была беременна.
\vs Luk 2:6 Когда же они были там, наступило время родить Ей;
\vs Luk 2:7 и родила Сына своего Первенца, и спеленала Его, и положила Его в ясли, потому что не было им места в гостинице.
\rsbpar\vs Luk 2:8 В той стране были на поле пастухи, которые содержали ночную стражу у стада своего.
\vs Luk 2:9 Вдруг предстал им Ангел Господень, и слава Господня осияла их; и убоялись страхом великим.
\vs Luk 2:10 И сказал им Ангел: не бойтесь; я возвещаю вам великую радость, которая будет всем людям:
\vs Luk 2:11 ибо ныне родился вам в городе Давидовом Спаситель, Который есть Христос Господь;
\vs Luk 2:12 и вот вам знак: вы найдете Младенца в пеленах, лежащего в яслях.
\vs Luk 2:13 И внезапно явилось с Ангелом многочисленное воинство небесное, славящее Бога и взывающее:
\vs Luk 2:14 слава в вышних Богу, и на земле мир, в человеках благоволение!
\vs Luk 2:15 Когда Ангелы отошли от них на небо, пастухи сказали друг другу: пойдем в Вифлеем и посмотрим, чт\acc{о} там случилось, о чем возвестил нам Господь.
\vs Luk 2:16 И, поспешив, пришли и нашли Марию и Иосифа, и Младенца, лежащего в яслях.
\vs Luk 2:17 Увидев же, рассказали о том, чт\acc{о} было возвещено им о Младенце Сем.
\vs Luk 2:18 И все слышавшие дивились тому, чт\acc{о} рассказывали им пастухи.
\vs Luk 2:19 А Мария сохраняла все слова сии, слагая в сердце Своем.
\vs Luk 2:20 И возвратились пастухи, славя и хваля Бога за всё т\acc{о}, что слышали и видели, к\acc{а}к им сказано было.
\rsbpar\vs Luk 2:21 По прошествии восьми дней, когда надлежало обрезать \bibemph{Младенца}, дали Ему имя Иисус, нареченное Ангелом прежде зачатия Его во чреве.
\rsbpar\vs Luk 2:22 А когда исполнились дни очищения их по закону Моисееву, принесли Его в Иерусалим, чтобы представить пред Господа,
\vs Luk 2:23 как предписано в законе Господнем, чтобы всякий младенец мужеского пола, разверзающий ложесна, был посвящен Господу,
\vs Luk 2:24 и чтобы принести в жертву, по реченному в законе Господнем, две горлицы или двух птенцов голубиных.
\vs Luk 2:25 Тогда был в Иерусалиме человек, именем Симеон. Он был муж праведный и благочестивый, чающий утешения Израилева; и Дух Святый был на нем.
\vs Luk 2:26 Ему было предсказано Духом Святым, что он не увидит смерти, доколе не увидит Христа Господня.
\vs Luk 2:27 И пришел он по вдохновению в храм. И, когда родители принесли Младенца Иисуса, чтобы совершить над Ним законный обряд,
\vs Luk 2:28 он взял Его на руки, благословил Бога и сказал:
\rsbpar\vs Luk 2:29 Ныне отпускаешь раба Твоего, Владыко, по слову Твоему, с миром,
\vs Luk 2:30 ибо видели очи мои спасение Твое,
\vs Luk 2:31 которое Ты уготовал пред лицем всех народов,
\vs Luk 2:32 свет к просвещению язычников и славу народа Твоего Израиля.
\rsbpar\vs Luk 2:33 Иосиф же и Матерь Его дивились сказанному о Нем.
\vs Luk 2:34 И благословил их Симеон и сказал Марии, Матери Его: се, лежит Сей на падение и на восстание многих в Израиле и в предмет пререканий,~---
\vs Luk 2:35 и Тебе Самой оружие пройдет душу,~--- да откроются помышления многих сердец.
\vs Luk 2:36 Тут была также Анна пророчица, дочь Фануилова, от колена Асирова, достигшая глубокой старости, прожив с мужем от девства своего семь лет,
\vs Luk 2:37 вдова лет восьмидесяти четырех, которая не отходила от храма, постом и молитвою служа Богу день и ночь.
\vs Luk 2:38 И она в то время, подойдя, славила Господа и говорила о Нем всем, ожидавшим избавления в Иерусалиме.
\rsbpar\vs Luk 2:39 И когда они совершили всё по закону Господню, возвратились в Галилею, в город свой Назарет.
\vs Luk 2:40 Младенец же возрастал и укреплялся духом, исполняясь премудрости, и благодать Божия была на Нем.
\vs Luk 2:41 Каждый год родители Его ходили в Иерусалим на праздник Пасхи.
\vs Luk 2:42 И когда Он был двенадцати лет, пришли они также по обычаю в Иерусалим на праздник.
\vs Luk 2:43 Когда же, по окончании дней \bibemph{праздника}, возвращались, остался Отрок Иисус в Иерусалиме; и не заметили того Иосиф и Матерь Его,
\vs Luk 2:44 но думали, что Он идет с другими. Пройдя же дневной путь, стали искать Его между родственниками и знакомыми
\vs Luk 2:45 и, не найдя Его, возвратились в Иерусалим, ища Его.
\vs Luk 2:46 Через три дня нашли Его в храме, сидящего посреди учителей, слушающего их и спрашивающего их;
\vs Luk 2:47 все слушавшие Его дивились разуму и ответам Его.
\vs Luk 2:48 И, увидев Его, удивились; и Матерь Его сказала Ему: Чадо! чт\acc{о} Ты сделал с нами? Вот, отец Твой и Я с великою скорбью искали Тебя.
\vs Luk 2:49 Он сказал им: зачем было вам искать Меня? или вы не знали, что Мне должно быть в том, чт\acc{о} принадлежит Отцу Моему?
\vs Luk 2:50 Но они не поняли сказанных Им слов.
\vs Luk 2:51 И Он пошел с ними и пришел в Назарет; и был в повиновении у них. И Матерь Его сохраняла все слова сии в сердце Своем.
\vs Luk 2:52 Иисус же преуспевал в премудрости и возрасте и в любви у Бога и человеков.
\vs Luk 3:1 В пятнадцатый же год правления Тиверия кесаря, когда Понтий Пилат начальствовал в Иудее, Ирод был четвертовластником в Галилее, Филипп, брат его, четвертовластником в Итурее и Трахонитской области, а Лисаний четвертовластником в Авилинее,
\vs Luk 3:2 при первосвященниках Анне и Каиафе, был глагол Божий к Иоанну, сыну Захарии, в пустыне.
\vs Luk 3:3 И он проходил по всей окрестной стране Иорданской, проповедуя крещение покаяния для прощения грехов,
\vs Luk 3:4 как написано в книге слов пророка Исаии, который говорит: глас вопиющего в пустыне: приготовьте путь Господу, прямыми сделайте стези Ему;
\vs Luk 3:5 всякий дол да наполнится, и всякая гора и холм да понизятся, кривизны выпрямятся и неровные пути сделаются гладкими;
\vs Luk 3:6 и узрит всякая плоть спасение Божие.
\vs Luk 3:7 \bibemph{Иоанн} приходившему креститься от него народу говорил: порождения ехиднины! кто внушил вам бежать от будущего гнева?
\vs Luk 3:8 Сотворите же достойные плоды покаяния и не думайте говорить в себе: отец у нас Авраам, ибо говорю вам, что Бог может из камней сих воздвигнуть детей Аврааму.
\vs Luk 3:9 Уже и секира при корне дерев лежит: всякое дерево, не приносящее доброго плода, срубают и бросают в огонь.
\vs Luk 3:10 И спрашивал его народ: что же нам делать?
\vs Luk 3:11 Он сказал им в ответ: у кого две одежды, тот дай неимущему, и у кого есть пища, делай то же.
\vs Luk 3:12 Пришли и мытари креститься, и сказали ему: учитель! что нам делать?
\vs Luk 3:13 Он отвечал им: ничего не требуйте более определенного вам.
\vs Luk 3:14 Спрашивали его также и воины: а нам что делать? И сказал им: никого не обижайте, не клевещите, и довольствуйтесь своим жалованьем.
\vs Luk 3:15 Когда же народ был в ожидании, и все помышляли в сердцах своих об Иоанне, не Христос ли он,~---
\vs Luk 3:16 Иоанн всем отвечал: я крещу вас водою, но идёт Сильнейший меня, у Которого я недостоин развязать ремень обуви; Он будет крестить вас Духом Святым и огнем.
\vs Luk 3:17 Лопата Его в руке Его, и Он очистит гумно Свое и соберет пшеницу в житницу Свою, а солому сожжет огнем неугасимым.
\vs Luk 3:18 Многое и другое благовествовал он народу, поучая его.
\rsbpar\vs Luk 3:19 Ирод же четвертовластник, обличаемый от него за Иродиаду, жену брата своего, и за всё, что сделал Ирод худого,
\vs Luk 3:20 прибавил ко всему прочему и т\acc{о}, что заключил Иоанна в темницу.
\rsbpar\vs Luk 3:21 Когда же крестился весь народ, и Иисус, крестившись, молился: отверзлось небо,
\vs Luk 3:22 и Дух Святый нисшел на Него в телесном виде, как голубь, и был глас с небес, глаголющий: Ты Сын Мой Возлюбленный; в Тебе Мое благоволение!
\rsbpar\vs Luk 3:23 Иисус, начиная \bibemph{Своё служение}, был лет тридцати, и был, как думали, Сын Иосифов, Илиев,
\vs Luk 3:24 Матфатов, Левиин, Мелхиев, Ианнаев, Иосифов,
\vs Luk 3:25 Маттафиев, Амосов, Наумов, Еслимов, Наггеев,
\vs Luk 3:26 Маафов, Маттафиев, Семеиев, Иосифов, Иудин,
\vs Luk 3:27 Иоаннанов, Рисаев, Зоровавелев, Салафиилев, Нириев,
\vs Luk 3:28 Мелхиев, Аддиев, Косамов, Елмодамов, Иров,
\vs Luk 3:29 Иосиев, Елиезеров, Иоримов, Матфатов, Левиин,
\vs Luk 3:30 Симеонов, Иудин, Иосифов, Ионанов, Елиакимов,
\vs Luk 3:31 Мелеаев, Маинанов, Маттафаев, Нафанов, Давидов,
\vs Luk 3:32 Иессеев, Овидов, Воозов, Салмонов, Наассонов,
\vs Luk 3:33 Аминадавов, Арамов, Есромов, Фаресов, Иудин,
\vs Luk 3:34 Иаковлев, Исааков, Авраамов, Фаррин, Нахоров,
\vs Luk 3:35 Серухов, Рагавов, Фалеков, Еверов, Салин,
\vs Luk 3:36 Каинанов, Арфаксадов, Симов, Ноев, Ламехов,
\vs Luk 3:37 Мафусалов, Енохов, Иаредов, Малелеилов, Каинанов,
\vs Luk 3:38 Еносов, Сифов, Адамов, Божий.
\vs Luk 4:1 Иисус, исполненный Духа Святаго, возвратился от Иордана и поведен был Духом в пустыню.
\vs Luk 4:2 Там сорок дней Он был искушаем от диавола и ничего не ел в эти дни, а по прошествии их напоследок взалкал.
\vs Luk 4:3 И сказал Ему диавол: если Ты Сын Божий, то вели этому камню сделаться хлебом.
\vs Luk 4:4 Иисус сказал ему в ответ: написано, что не хлебом одним будет жить человек, но всяким словом Божиим.
\vs Luk 4:5 И, возведя Его на высокую гору, диавол показал Ему все царства вселенной во мгновение времени,
\vs Luk 4:6 и сказал Ему диавол: Тебе дам власть над всеми сими \bibemph{царствами} и славу их, ибо она предана мне, и я, кому хочу, даю ее;
\vs Luk 4:7 итак, если Ты поклонишься мне, то всё будет Твое.
\vs Luk 4:8 Иисус сказал ему в ответ: отойди от Меня, сатана; написано: Господу Богу твоему поклоняйся, и Ему одному служи.
\vs Luk 4:9 И повел Его в Иерусалим, и поставил Его на крыле храма, и сказал Ему: если Ты Сын Божий, бросься отсюда вниз,
\vs Luk 4:10 ибо написано: Ангелам Своим заповедает о Тебе сохранить Тебя;
\vs Luk 4:11 и на руках понесут Тебя, да не преткнешься о камень ногою Твоею.
\vs Luk 4:12 Иисус сказал ему в ответ: сказано: не искушай Господа Бога твоего.
\vs Luk 4:13 И, окончив всё искушение, диавол отошел от Него до времени.
\rsbpar\vs Luk 4:14 И возвратился Иисус в силе Духа в Галилею; и разнеслась молва о Нем по всей окрестной стране.
\vs Luk 4:15 Он учил в синагогах их, и от всех был прославляем.
\rsbpar\vs Luk 4:16 И пришел в Назарет, где был воспитан, и вошел, по обыкновению Своему, в день субботний в синагогу, и встал читать.
\vs Luk 4:17 Ему подали книгу пророка Исаии; и Он, раскрыв книгу, нашел место, где было написано:
\vs Luk 4:18 Дух Господень на Мне; ибо Он помазал Меня благовествовать нищим, и послал Меня исцелять сокрушенных сердцем, проповедовать пленным освобождение, слепым прозрение, отпустить измученных на свободу,
\vs Luk 4:19 проповедовать лето Господне благоприятное.
\vs Luk 4:20 И, закрыв книгу и отдав служителю, сел; и глаза всех в синагоге были устремлены на Него.
\vs Luk 4:21 И Он начал говорить им: ныне исполнилось писание сие, слышанное вами.
\vs Luk 4:22 И все засвидетельствовали Ему это, и дивились словам благодати, исходившим из уст Его, и говорили: не Иосифов ли это сын?
\vs Luk 4:23 Он сказал им: конечно, вы скажете Мне присловие: врач! исцели Самого Себя; сделай и здесь, в Твоем отечестве, т\acc{о}, чт\acc{о}, мы слышали, было в Капернауме.
\vs Luk 4:24 И сказал: истинно говорю вам: никакой пророк не принимается в своем отечестве.
\vs Luk 4:25 Поистине говорю вам: много вдов было в Израиле во дни Илии, когда заключено было небо три года и шесть месяцев, так что сделался большой голод по всей земле,
\vs Luk 4:26 и ни к одной из них не был послан Илия, а только ко вдове в Сарепту Сидонскую;
\vs Luk 4:27 много также было прокаженных в Израиле при пророке Елисее, и ни один из них не очистился, кроме Неемана Сириянина.
\vs Luk 4:28 Услышав это, все в синагоге исполнились ярости
\vs Luk 4:29 и, встав, выгнали Его вон из города и повели на вершину горы, на которой город их был построен, чтобы свергнуть Его;
\vs Luk 4:30 но Он, пройдя посреди них, удалился.
\rsbpar\vs Luk 4:31 И пришел в Капернаум, город Галилейский, и учил их в дни субботние.
\vs Luk 4:32 И дивились учению Его, ибо слово Его было со властью.
\vs Luk 4:33 Был в синагоге человек, имевший нечистого духа бесовского, и он закричал громким голосом:
\vs Luk 4:34 оставь; чт\acc{о} Тебе до нас, Иисус Назарянин? Ты пришел погубить нас; знаю Тебя, кто Ты, Святый Божий.
\vs Luk 4:35 Иисус запретил ему, сказав: замолчи и выйди из него. И бес, повергнув его посреди \bibemph{синагоги}, вышел из него, нимало не повредив ему.
\vs Luk 4:36 И напал на всех ужас, и рассуждали между собою: что это значит, что Он со властью и силою повелевает нечистым духам, и они выходят?
\vs Luk 4:37 И разнесся слух о Нем по всем окрестным местам.
\rsbpar\vs Luk 4:38 Выйдя из синагоги, Он вошел в дом Симона; тёща же Симонова была одержима сильною горячкою; и просили Его о ней.
\vs Luk 4:39 Подойдя к ней, Он запретил горячке; и оставила ее. Она тотчас встала и служила им.
\vs Luk 4:40 При захождении же солнца все, имевшие больных различными болезнями, приводили их к Нему и Он, возлагая на каждого из них руки, исцелял их.
\vs Luk 4:41 Выходили также и бесы из многих с криком и говорили: Ты Христос, Сын Божий. А Он запрещал им сказывать, что они знают, что Он Христос.
\rsbpar\vs Luk 4:42 Когда же настал день, Он, выйдя \bibemph{из дома}, пошел в пустынное место, и народ искал Его и, придя к Нему, удерживал Его, чтобы не уходил от них.
\vs Luk 4:43 Но Он сказал им: и другим городам благовествовать Я должен Царствие Божие, ибо на то Я послан.
\vs Luk 4:44 И проповедовал в синагогах галилейских.
\vs Luk 5:1 Однажды, когда народ теснился к Нему, чтобы слышать слово Божие, а Он стоял у озера Геннисаретского,
\vs Luk 5:2 увидел Он две лодки, стоящие на озере; а рыболовы, выйдя из них, вымывали сети.
\vs Luk 5:3 Войдя в одну лодку, которая была Симонова, Он просил его отплыть несколько от берега и, сев, учил народ из лодки.
\vs Luk 5:4 Когда же перестал учить, сказал Симону: отплыви на глубину и закиньте сети свои для лова.
\vs Luk 5:5 Симон сказал Ему в ответ: Наставник! мы трудились всю ночь и ничего не поймали, но по слову Твоему закину сеть.
\vs Luk 5:6 Сделав это, они поймали великое множество рыбы, и даже сеть у них прорывалась.
\vs Luk 5:7 И дали знак товарищам, находившимся на другой лодке, чтобы пришли помочь им; и пришли, и наполнили обе лодки, так что они начинали тонуть.
\vs Luk 5:8 Увидев это, Симон Петр припал к коленям Иисуса и сказал: выйди от меня, Господи! потому что я человек грешный.
\vs Luk 5:9 Ибо ужас объял его и всех, бывших с ним, от этого лова рыб, ими пойманных;
\vs Luk 5:10 также и Иакова и Иоанна, сыновей Зеведеевых, бывших товарищами Симону. И сказал Симону Иисус: не бойся; отныне будешь ловить человеков.
\vs Luk 5:11 И, вытащив обе лодки на берег, оставили всё и последовали за Ним.
\rsbpar\vs Luk 5:12 Когда Иисус был в одном городе, пришел человек весь в проказе и, увидев Иисуса, пал ниц, умоляя Его и говоря: Господи! если хочешь, можешь меня очистить.
\vs Luk 5:13 Он простер руку, прикоснулся к нему и сказал: хочу, очистись. И тотчас проказа сошла с него.
\vs Luk 5:14 И Он повелел ему никому не сказывать, а пойти показаться священнику и принести \bibemph{жертву} за очищение свое, к\acc{а}к повелел Моисей, во свидетельство им.
\vs Luk 5:15 Но тем более распространялась молва о Нём, и великое множество народа стекалось к Нему слушать и врачеваться у Него от болезней своих.
\vs Luk 5:16 Но Он уходил в пустынные места и молился.
\rsbpar\vs Luk 5:17 В один день, когда Он учил, и сидели тут фарисеи и законоучители, пришедшие из всех мест Галилеи и Иудеи и из Иерусалима, и сила Господня являлась в исцелении \bibemph{больных},~---
\vs Luk 5:18 вот, принесли некоторые на постели человека, который был расслаблен, и старались внести его \bibemph{в дом} и положить перед Иисусом;
\vs Luk 5:19 и, не найдя, где пронести его за многолюдством, влезли на верх дома и сквозь кровлю спустили его с постелью на средину пред Иисуса.
\vs Luk 5:20 И Он, видя веру их, сказал человеку тому: прощаются тебе грехи твои.
\vs Luk 5:21 Книжники и фарисеи начали рассуждать, говоря: кто это, который богохульствует? кто может прощать грехи, кроме одного Бога?
\vs Luk 5:22 Иисус, уразумев помышления их, сказал им в ответ: чт\acc{о} вы помышляете в сердцах ваших?
\vs Luk 5:23 Чт\acc{о} легче сказать: прощаются тебе грехи твои, или сказать: встань и ходи?
\vs Luk 5:24 Но чтобы вы знали, что Сын Человеческий имеет власть на земле прощать грехи,~--- сказал Он расслабленному: тебе говорю: встань, возьми постель твою и иди в дом твой.
\vs Luk 5:25 И он тотчас встал перед ними, взял, на чём лежал, и пошел в дом свой, славя Бога.
\vs Luk 5:26 И ужас объял всех, и славили Бога и, быв исполнены страха, говорили: ч\acc{у}дные дела видели мы ныне.
\rsbpar\vs Luk 5:27 После сего \bibemph{Иисус} вышел и увидел мытаря, именем Левия, сидящего у сбора пошлин, и говорит ему: следуй за Мною.
\vs Luk 5:28 И он, оставив всё, встал и последовал за Ним.
\vs Luk 5:29 И сделал для Него Левий в доме своем большое угощение; и там было множество мытарей и других, которые возлежали с ними.
\vs Luk 5:30 Книжники же и фарисеи роптали и говорили ученикам Его: зачем вы едите и пьете с мытарями и грешниками?
\vs Luk 5:31 Иисус же сказал им в ответ: не здоровые имеют нужду во враче, но больные;
\vs Luk 5:32 Я пришел призвать не праведников, а грешников к покаянию.
\vs Luk 5:33 Они же сказали Ему: почему ученики Иоанновы постятся часто и молитвы творят, также и фарисейские, а Твои едят и пьют?
\vs Luk 5:34 Он сказал им: можете ли заставить сынов чертога брачного поститься, когда с ними жених?
\vs Luk 5:35 Но придут дни, когда отнимется у них жених, и тогда будут поститься в те дни.
\vs Luk 5:36 При сем сказал им притчу: никто не приставляет заплаты к ветхой одежде, отодрав от новой одежды; а иначе и новую раздерет, и к старой не подойдет заплата от новой.
\vs Luk 5:37 И никто не вливает молодого вина в мехи ветхие; а иначе молодое вино прорвет мехи, и само вытечет, и мехи пропадут;
\vs Luk 5:38 но молодое вино должно вливать в мехи новые; тогда сбережется и т\acc{о} и другое.
\vs Luk 5:39 И никто, пив старое \bibemph{вино}, не захочет тотчас молодого, ибо говорит: старое лучше.
\vs Luk 6:1 В субботу, первую по втором дне Пасхи, случилось Ему проходить засеянными полями, и ученики Его срывали колосья и ели, растирая руками.
\vs Luk 6:2 Некоторые же из фарисеев сказали им: зачем вы делаете то, чего не должно делать в субботы?
\vs Luk 6:3 Иисус сказал им в ответ: разве вы не читали, что сделал Давид, когда взалкал сам и бывшие с ним?
\vs Luk 6:4 К\acc{а}к он вошел в дом Божий, взял хлебы предложения, которых не должно было есть никому, кроме одних священников, и ел, и дал бывшим с ним?
\vs Luk 6:5 И сказал им: Сын Человеческий есть господин и субботы.
\rsbpar\vs Luk 6:6 Случилось же и в другую субботу войти Ему в синагогу и учить. Там был человек, у которого правая рука была сухая.
\vs Luk 6:7 Книжники же и фарисеи наблюдали за Ним, не исцелит ли в субботу, чтобы найти обвинение против Него.
\vs Luk 6:8 Но Он, зная помышления их, сказал человеку, имеющему сухую руку: встань и выступи на средину. И он встал и выступил.
\vs Luk 6:9 Тогда сказал им Иисус: спрошу Я вас: чт\acc{о} должно делать в субботу? добро, или зло? спасти душу, или погубить? Они молчали.
\vs Luk 6:10 И, посмотрев на всех их, сказал тому человеку: протяни руку твою. Он так и сделал; и стала рука его здорова, как другая.
\vs Luk 6:11 Они же пришли в бешенство и говорили между собою, чт\acc{о} бы им сделать с Иисусом.
\rsbpar\vs Luk 6:12 В те дни взошел Он на гору помолиться и пробыл всю ночь в молитве к Богу.
\vs Luk 6:13 Когда же настал день, призвал учеников Своих и избрал из них двенадцать, которых и наименовал Апостолами:
\vs Luk 6:14 Симона, которого и назвал Петром, и Андрея, брата его, Иакова и Иоанна, Филиппа и Варфоломея,
\vs Luk 6:15 Матфея и Фому, Иакова Алфеева и Симона, прозываемого Зилотом,
\vs Luk 6:16 Иуду Иаковлева и Иуду Искариота, который потом сделался предателем.
\rsbpar\vs Luk 6:17 И, сойдя с ними, стал Он на ровном месте, и множество учеников Его, и много народа из всей Иудеи и Иерусалима и приморских мест Тирских и Сидонских,
\vs Luk 6:18 которые пришли послушать Его и исцелиться от болезней своих, также и страждущие от нечистых духов; и исцелялись.
\vs Luk 6:19 И весь народ искал прикасаться к Нему, потому что от Него исходила сила и исцеляла всех.
\vs Luk 6:20 И Он, возведя очи Свои на учеников Своих, говорил:\rsbpar Блаженны нищие духом, ибо ваше есть Царствие Божие.
\rsbpar\vs Luk 6:21 Блаженны алчущие ныне, ибо насытитесь.\rsbpar Блаженны плачущие ныне, ибо воссмеетесь.
\rsbpar\vs Luk 6:22 Блаженны вы, когда возненавидят вас люди и когда отлучат вас, и будут поносить, и пронесут имя ваше, как бесчестное, за Сына Человеческого.
\vs Luk 6:23 Возрадуйтесь в тот день и возвеселитесь, ибо велика вам награда на небесах. Так поступали с пророками отцы их.
\rsbpar\vs Luk 6:24 Напротив, горе вам, богатые! ибо вы уже получили свое утешение.
\vs Luk 6:25 Горе вам, пресыщенные ныне! ибо взалчете. Горе вам, смеющиеся ныне! ибо восплачете и возрыдаете.
\vs Luk 6:26 Горе вам, когда все люди будут говорить о вас хорошо! ибо так поступали с лжепророками отцы их.
\rsbpar\vs Luk 6:27 Но вам, слушающим, говорю: люб\acc{и}те врагов ваших, благотворите ненавидящим вас,
\vs Luk 6:28 благословляйте проклинающих вас и молитесь за обижающих вас.
\vs Luk 6:29 Ударившему тебя по щеке подставь и другую, и отнимающему у тебя верхнюю одежду не препятствуй взять и рубашку.
\vs Luk 6:30 Всякому, просящему у тебя, давай, и от взявшего твое не требуй назад.
\vs Luk 6:31 И к\acc{а}к хотите, чтобы с вами поступали люди, т\acc{а}к и вы поступайте с ними.
\vs Luk 6:32 И если любите любящих вас, какая вам за то благодарность? ибо и грешники любящих их любят.
\vs Luk 6:33 И если делаете добро тем, которые вам делают добро, какая вам за то благодарность? ибо и грешники т\acc{о} же делают.
\vs Luk 6:34 И если взаймы даёте тем, от которых надеетесь получить обратно, какая вам за то благодарность? ибо и грешники дают взаймы грешникам, чтобы получить обратно столько же.
\vs Luk 6:35 Но вы люб\acc{и}те врагов ваших, и благотворите, и взаймы давайте, не ожидая ничего; и будет вам награда великая, и будете сынами Всевышнего; ибо Он благ и к неблагодарным и злым.
\vs Luk 6:36 Итак, будьте милосерды, как и Отец ваш милосерд.
\rsbpar\vs Luk 6:37 Не суд\acc{и}те, и не будете судимы; не осуждайте, и не будете осуждены; прощайте, и прощены будете;
\vs Luk 6:38 давайте, и дастся вам: мерою доброю, утрясенною, нагнетенною и переполненною отсыплют вам в лоно ваше; ибо, какою мерою мерите, такою же отмерится и вам.
\vs Luk 6:39 Сказал также им притчу: может ли слепой водить слепого? не оба ли упадут в яму?
\vs Luk 6:40 Ученик не бывает выше своего учителя; но, и усовершенствовавшись, будет всякий, как учитель его.
\vs Luk 6:41 Что ты смотришь на сучок в глазе брата твоего, а бревна в твоем глазе не чувствуешь?
\vs Luk 6:42 Или, как можешь сказать брату твоему: брат! дай, я выну сучок из глаза твоего, когда сам не видишь бревна в твоем глазе? Лицемер! вынь прежде бревно из твоего глаза, и тогда увидишь, как вынуть сучок из глаза брата твоего.
\vs Luk 6:43 Нет доброго дерева, которое приносило бы худой плод; и нет худого дерева, которое приносило бы плод добрый,
\vs Luk 6:44 ибо всякое дерево познаётся по плоду своему, потому что не собирают смокв с терновника и не снимают винограда с кустарника.
\vs Luk 6:45 Добрый человек из доброго сокровища сердца своего выносит доброе, а злой человек из злого сокровища сердца своего выносит злое, ибо от избытка сердца говорят уста его.
\rsbpar\vs Luk 6:46 Чт\acc{о} вы зовете Меня: Господи! Господи!~--- и не делаете того, чт\acc{о} Я говорю?
\vs Luk 6:47 Всякий, приходящий ко Мне и слушающий слова Мои и исполняющий их, скажу вам, кому подобен.
\vs Luk 6:48 Он подобен человеку, строящему дом, который копал, углубился и положил основание на камне; почему, когда случилось наводнение и вода напёрла на этот дом, то не могла поколебать его, потому что он основан был на камне.
\vs Luk 6:49 А слушающий и неисполняющий подобен человеку, построившему дом на земле без основания, который, когда напёрла на него вода, тотчас обрушился; и разрушение дома сего было великое.
\vs Luk 7:1 Когда Он окончил все слова Свои к слушавшему народу, то вошел в Капернаум.
\vs Luk 7:2 У одного сотника слуга, которым он дорожил, был болен при смерти.
\vs Luk 7:3 Услышав об Иисусе, он послал к Нему Иудейских старейшин просить Его, чтобы пришел исцелить слугу его.
\vs Luk 7:4 И они, придя к Иисусу, просили Его убедительно, говоря: он достоин, чтобы Ты сделал для него это,
\vs Luk 7:5 ибо он любит народ наш и построил нам синагогу.
\vs Luk 7:6 Иисус пошел с ними. И когда Он недалеко уже был от дома, сотник прислал к Нему друзей сказать Ему: не трудись, Господи! ибо я недостоин, чтобы Ты вошел под кров мой;
\vs Luk 7:7 потому и себя самого не почел я достойным прийти к Тебе; но скажи слово, и выздоровеет слуга мой.
\vs Luk 7:8 Ибо я и подвластный человек, но, имея у себя в подчинении воинов, говорю одному: пойди, и идет; и другому: приди, и приходит; и слуге моему: сделай т\acc{о}, и делает.
\vs Luk 7:9 Услышав сие, Иисус удивился ему и, обратившись, сказал идущему за Ним народу: сказываю вам, что и в Израиле не нашел Я такой веры.
\vs Luk 7:10 Посланные, возвратившись в дом, нашли больного слугу выздоровевшим.
\rsbpar\vs Luk 7:11 После сего Иисус пошел в город, называемый Наин; и с Ним шли многие из учеников Его и множество народа.
\vs Luk 7:12 Когда же Он приблизился к городским воротам, тут выносили умершего, единственного сына у матери, а она была вдова; и много народа шло с нею из города.
\vs Luk 7:13 Увидев ее, Господь сжалился над нею и сказал ей: не плачь.
\vs Luk 7:14 И, подойдя, прикоснулся к одру; несшие остановились, и Он сказал: юноша! тебе говорю, встань!
\vs Luk 7:15 Мертвый, поднявшись, сел и стал говорить; и отдал его \bibemph{Иисус} матери его.
\vs Luk 7:16 И всех объял страх, и славили Бога, говоря: великий пророк восстал между нами, и Бог посетил народ Свой.
\vs Luk 7:17 Такое мнение о Нём распространилось по всей Иудее и по всей окрестности.
\rsbpar\vs Luk 7:18 И возвестили Иоанну ученики его о всём том.
\vs Luk 7:19 Иоанн, призвав двоих из учеников своих, послал к Иисусу спросить: Ты ли Тот, Который должен прийти, или ожидать нам другого?
\vs Luk 7:20 Они, придя к \bibemph{Иисусу}, сказали: Иоанн Креститель послал нас к Тебе спросить: Ты ли Тот, Которому должно прийти, или другого ожидать нам?
\vs Luk 7:21 А в это время Он многих исцелил от болезней и недугов и от злых духов, и многим слепым даровал зрение.
\vs Luk 7:22 И сказал им Иисус в ответ: пойдите, скажите Иоанну, чт\acc{о} вы видели и слышали: слепые прозревают, хромые ходят, прокаженные очищаются, глухие слышат, мертвые воскресают, нищие благовествуют;
\vs Luk 7:23 и блажен, кто не соблазнится о Мне!
\rsbpar\vs Luk 7:24 По отшествии же посланных Иоанном, начал говорить к народу об Иоанне: чт\acc{о} смотреть ходили вы в пустыню? трость ли, ветром колеблемую?
\vs Luk 7:25 Чт\acc{о} же смотреть ходили вы? человека ли, одетого в мягкие одежды? Но одевающиеся пышно и роскошно живущие находятся при дворах царских.
\vs Luk 7:26 Чт\acc{о} же смотреть ходили вы? пророка ли? Да, говорю вам, и больше пророка.
\vs Luk 7:27 Сей есть, о котором написано: вот, Я посылаю Ангела Моего пред лицем Твоим, который приготовит путь Твой пред Тобою.
\vs Luk 7:28 Ибо говорю вам: из рожденных женами нет ни одного пророка больше Иоанна Крестителя; но меньший в Царствии Божием больше его.
\vs Luk 7:29 И весь народ, слушавший \bibemph{Его}, и мытари воздали славу Богу, крестившись крещением Иоанновым;
\vs Luk 7:30 а фарисеи и законники отвергли волю Божию о себе, не крестившись от него.
\vs Luk 7:31 Тогда Господь сказал: с кем сравню людей рода сего? и кому они подобны?
\vs Luk 7:32 Они подобны детям, которые сидят на улице, кличут друг друга и говорят: мы играли вам на свирели, и вы не плясали; мы пели вам плачевные песни, и вы не плакали.
\vs Luk 7:33 Ибо пришел Иоанн Креститель: ни хлеба не ест, ни вина не пьет; и говорите: в нем бес.
\vs Luk 7:34 Пришел Сын Человеческий: ест и пьет; и говорите: вот человек, который любит есть и пить вино, друг мытарям и грешникам.
\vs Luk 7:35 И оправдана премудрость всеми чадами ее.
\rsbpar\vs Luk 7:36 Некто из фарисеев просил Его вкусить с ним пищи; и Он, войдя в дом фарисея, возлег.
\vs Luk 7:37 И вот, женщина того города, которая была грешница, узнав, что Он возлежит в доме фарисея, принесла алавастровый сосуд с миром
\vs Luk 7:38 и, став позади у ног Его и плача, начала обливать ноги Его слезами и отирать волосами головы своей, и целовала ноги Его, и мазала миром.
\vs Luk 7:39 Видя это, фарисей, пригласивший Его, сказал сам в себе: если бы Он был пророк, то знал бы, кто и какая женщина прикасается к Нему, ибо она грешница.
\vs Luk 7:40 Обратившись к нему, Иисус сказал: Симон! Я имею нечто сказать тебе. Он говорит: скажи, Учитель.
\vs Luk 7:41 Иисус сказал: у одного заимодавца было два должника: один должен был пятьсот динариев, а другой пятьдесят,
\vs Luk 7:42 но как они не имели чем заплатить, он простил обоим. Скажи же, который из них более возлюбит его?
\vs Luk 7:43 Симон отвечал: думаю, тот, которому более простил. Он сказал ему: правильно ты рассудил.
\vs Luk 7:44 И, обратившись к женщине, сказал Симону: видишь ли ты эту женщину? Я пришел в дом твой, и ты воды Мне на ноги не дал, а она слезами облила Мне ноги и волосами головы своей отёрла;
\vs Luk 7:45 ты целования Мне не дал, а она, с тех пор как Я пришел, не перестает целовать у Меня ноги;
\vs Luk 7:46 ты головы Мне маслом не помазал, а она миром помазала Мне ноги.
\vs Luk 7:47 А потому сказываю тебе: прощаются грехи её многие за то, что она возлюбила много, а кому мало прощается, тот мало любит.
\vs Luk 7:48 Ей же сказал: прощаются тебе грехи.
\vs Luk 7:49 И возлежавшие с Ним начали говорить про себя: кто это, что и грехи прощает?
\vs Luk 7:50 Он же сказал женщине: вера твоя спасла тебя, иди с миром.
\vs Luk 8:1 После сего Он проходил по городам и селениям, проповедуя и благовествуя Царствие Божие, и с Ним двенадцать,
\vs Luk 8:2 и некоторые женщины, которых Он исцелил от злых духов и болезней: Мария, называемая Магдалиною, из которой вышли семь бесов,
\vs Luk 8:3 и Иоанна, жена Хузы, домоправителя Иродова, и Сусанна, и многие другие, которые служили Ему имением своим.
\rsbpar\vs Luk 8:4 Когда же собралось множество народа, и из всех городов жители сходились к Нему, Он начал говорить притчею:
\vs Luk 8:5 вышел сеятель сеять семя свое, и когда он сеял, иное упало при дороге и было потоптано, и птицы небесные поклевали его;
\vs Luk 8:6 а иное упало на камень и, взойдя, засохло, потому что не имело влаги;
\vs Luk 8:7 а иное упало между тернием, и выросло терние и заглушило его;
\vs Luk 8:8 а иное упало на добрую землю и, взойдя, принесло плод сторичный. Сказав сие, возгласил: кто имеет уши слышать, да слышит!
\vs Luk 8:9 Ученики же Его спросили у Него: что бы значила притча сия?
\vs Luk 8:10 Он сказал: вам дано знать тайны Царствия Божия, а прочим в притчах, так что они видя не видят и слыша не разумеют.
\vs Luk 8:11 Вот что значит притча сия: семя есть слово Божие;
\vs Luk 8:12 а упавшее при пути, это суть слушающие, к которым пот\acc{о}м приходит диавол и уносит слово из сердца их, чтобы они не уверовали и не спаслись;
\vs Luk 8:13 а упавшее на камень, это те, которые, когда услышат слово, с радостью принимают, но которые не имеют корня, и временем веруют, а во время искушения отпадают;
\vs Luk 8:14 а упавшее в терние, это те, которые слушают слово, но, отходя, заботами, богатством и наслаждениями житейскими подавляются и не приносят плода;
\vs Luk 8:15 а упавшее на добрую землю, это те, которые, услышав слово, хранят его в добром и чистом сердце и приносят плод в терпении. Сказав это, Он возгласил: кто имеет уши слышать, да слышит!
\vs Luk 8:16 Никто, зажегши свечу, не покрывает ее сосудом, или не ставит под кровать, а ставит на подсвечник, чтобы входящие видели свет.
\vs Luk 8:17 Ибо нет ничего тайного, чт\acc{о} не сделалось бы явным, ни сокровенного, чт\acc{о} не сделалось бы известным и не обнаружилось бы.
\vs Luk 8:18 Итак, наблюдайте, как вы слушаете: ибо, кто имеет, тому дано будет, а кто не имеет, у того отнимется и т\acc{о}, чт\acc{о} он думает иметь.
\rsbpar\vs Luk 8:19 И пришли к Нему Матерь и братья Его, и не могли подойти к Нему по причине народа.
\vs Luk 8:20 И дали знать Ему: Матерь и братья Твои стоят вне, желая видеть Тебя.
\vs Luk 8:21 Он сказал им в ответ: матерь Моя и братья Мои суть слушающие слово Божие и исполняющие его.
\rsbpar\vs Luk 8:22 В один день Он вошел с учениками Своими в лодку и сказал им: переправимся на ту сторону озера. И отправились.
\vs Luk 8:23 Во время плавания их Он заснул. На озере поднялся бурный ветер, и заливало их \bibemph{волнами}, и они были в опасности.
\vs Luk 8:24 И, подойдя, разбудили Его и сказали: Наставник! Наставник! погибаем. Но Он, встав, запретил ветру и волнению воды; и перестали, и сделалась тишина.
\vs Luk 8:25 Тогда Он сказал им: где вера ваша? Они же в страхе и удивлении говорили друг другу: кто же это, что и ветрам повелевает и воде, и повинуются Ему?
\rsbpar\vs Luk 8:26 И приплыли в страну Гадаринскую, лежащую против Галилеи.
\vs Luk 8:27 Когда же вышел Он на берег, встретил Его один человек из города, одержимый бесами с давнего времени, и в одежду не одевавшийся, и живший не в доме, а в гробах.
\vs Luk 8:28 Он, увидев Иисуса, вскричал, пал пред Ним и громким голосом сказал: чт\acc{о} Тебе до меня, Иисус, Сын Бога Всевышнего? умоляю Тебя, не мучь меня.
\vs Luk 8:29 Ибо \bibemph{Иисус} повелел нечистому духу выйти из сего человека, потому что он долгое время мучил его, так что его связывали цепями и узами, сберегая его; но он разрывал узы и был гоним бесом в пустыни.
\vs Luk 8:30 Иисус спросил его: как тебе имя? Он сказал: легион,~--- потому что много бесов вошло в него.
\vs Luk 8:31 И они просили Иисуса, чтобы не повелел им идти в бездну.
\vs Luk 8:32 Тут же на горе паслось большое стадо свиней; и \bibemph{бесы} просили Его, чтобы позволил им войти в них. Он позволил им.
\vs Luk 8:33 Бесы, выйдя из человека, вошли в свиней, и бросилось стадо с крутизны в озеро и потонуло.
\vs Luk 8:34 Пастухи, видя происшедшее, побежали и рассказали в городе и в селениях.
\vs Luk 8:35 И вышли видеть происшедшее; и, придя к Иисусу, нашли человека, из которого вышли бесы, сидящего у ног Иисуса, одетого и в здравом уме; и ужаснулись.
\vs Luk 8:36 Видевшие же рассказали им, как исцелился бесновавшийся.
\vs Luk 8:37 И просил Его весь народ Гадаринской окрестности удалиться от них, потому что они объяты были великим страхом. Он вошел в лодку и возвратился.
\vs Luk 8:38 Человек же, из которого вышли бесы, просил Его, чтобы быть с Ним. Но Иисус отпустил его, сказав:
\vs Luk 8:39 возвратись в дом твой и расскажи, чт\acc{о} сотворил тебе Бог. Он пошел и проповедовал по всему городу, что сотворил ему Иисус.
\rsbpar\vs Luk 8:40 Когда же возвратился Иисус, народ принял Его, потому что все ожидали Его.
\vs Luk 8:41 И вот, пришел человек, именем Иаир, который был начальником синагоги; и, пав к ногам Иисуса, просил Его войти к нему в дом,
\vs Luk 8:42 потому что у него была одна дочь, лет двенадцати, и та была при смерти. Когда же Он шел, народ теснил Его.
\vs Luk 8:43 И женщина, страдавшая кровотечением двенадцать лет, которая, издержав на врачей всё имение, ни одним не могла быть вылечена,
\vs Luk 8:44 подойдя сзади, коснулась края одежды Его; и тотчас течение крови у ней остановилось.
\vs Luk 8:45 И сказал Иисус: кто прикоснулся ко Мне? Когда же все отрицались, Петр сказал и бывшие с Ним: Наставник! народ окружает Тебя и теснит,~--- и Ты говоришь: кто прикоснулся ко Мне?
\vs Luk 8:46 Но Иисус сказал: прикоснулся ко Мне некто, ибо Я чувствовал силу, исшедшую из Меня.
\vs Luk 8:47 Женщина, видя, что она не утаилась, с трепетом подошла и, пав пред Ним, объявила Ему перед всем народом, по какой причине прикоснулась к Нему и как тотчас исцелилась.
\vs Luk 8:48 Он сказал ей: дерзай, дщерь! вера твоя спасла тебя; иди с миром.
\vs Luk 8:49 Когда Он еще говорил это, приходит некто из дома начальника синагоги и говорит ему: дочь твоя умерла; не утруждай Учителя.
\vs Luk 8:50 Но Иисус, услышав это, сказал ему: не бойся, только веруй, и спасена будет.
\vs Luk 8:51 Придя же в дом, не позволил войти никому, кроме Петра, Иоанна и Иакова, и отца девицы, и матери.
\vs Luk 8:52 Все плакали и рыдали о ней. Но Он сказал: не плачьте; она не умерла, но спит.
\vs Luk 8:53 И смеялись над Ним, зная, что она умерла.
\vs Luk 8:54 Он же, выслав всех вон и взяв ее за руку, возгласил: девица! встань.
\vs Luk 8:55 И возвратился дух ее; она тотчас встала, и Он велел дать ей есть.
\vs Luk 8:56 И удивились родители ее. Он же повелел им не сказывать никому о происшедшем.
\vs Luk 9:1 Созвав же двенадцать, дал силу и власть над всеми бесами и врачевать от болезней,
\vs Luk 9:2 и послал их проповедовать Царствие Божие и исцелять больных.
\vs Luk 9:3 И сказал им: ничего не берите на дорогу: ни посоха, ни сум\acc{ы}, ни хлеба, ни серебра, и не имейте по две одежды;
\vs Luk 9:4 и в какой дом войдете, там оставайтесь и оттуда отправляйтесь \bibemph{в путь}.
\vs Luk 9:5 А если где не примут вас, то, выходя из того города, отрясите и прах от ног ваших во свидетельство на них.
\vs Luk 9:6 Они пошли и проходили по селениям, благовествуя и исцеляя повсюду.
\rsbpar\vs Luk 9:7 Услышал Ирод четвертовластник о всём, что делал \bibemph{Иисус}, и недоумевал: ибо одни говорили, что это Иоанн восстал из мертвых;
\vs Luk 9:8 другие, что Илия явился, а иные, что один из древних пророков воскрес.
\vs Luk 9:9 И сказал Ирод: Иоанна я обезглавил; кто же Этот, о Котором я слышу такое? И искал увидеть Его.
\rsbpar\vs Luk 9:10 Апостолы, возвратившись, рассказали Ему, чт\acc{о} они сделали; и Он, взяв их с Собою, удалился особо в пустое место, близ города, называемого Вифсаидою.
\rsbpar\vs Luk 9:11 Но народ, узнав, пошел за Ним; и Он, приняв их, беседовал с ними о Царствии Божием и требовавших исцеления исцелял.
\vs Luk 9:12 День же начал склоняться к вечеру. И, приступив к Нему, двенадцать говорили Ему: отпусти народ, чтобы они пошли в окрестные селения и деревни ночевать и достали пищи; потому что мы здесь в пустом месте.
\vs Luk 9:13 Но Он сказал им: вы дайте им есть. Они сказали: у нас нет более пяти хлебов и двух рыб; разве нам пойти купить пищи для всех сих людей?
\vs Luk 9:14 Ибо их было около пяти тысяч человек. Но Он сказал ученикам Своим: рассадите их рядами по пятидесяти.
\vs Luk 9:15 И сделали так, и рассадили всех.
\vs Luk 9:16 Он же, взяв пять хлебов и две рыбы и воззрев на небо, благословил их, преломил и дал ученикам, чтобы раздать народу.
\vs Luk 9:17 И ели, и насытились все; и оставшихся у них кусков набрано двенадцать коробов.
\rsbpar\vs Luk 9:18 В одно время, когда Он молился в уединенном месте, и ученики были с Ним, Он спросил их: за кого почитает Меня народ?
\vs Luk 9:19 Они сказали в ответ: за Иоанна Крестителя, а иные за Илию; другие же \bibemph{говорят}, что один из древних пророков воскрес.
\vs Luk 9:20 Он же спросил их: а вы за кого почитаете Меня? Отвечал Петр: за Христа Божия.
\vs Luk 9:21 Но Он строго приказал им никому не говорить о сем,
\vs Luk 9:22 сказав, что Сыну Человеческому должно много пострадать, и быть отвержену старейшинами, первосвященниками и книжниками, и быть убиту, и в третий день воскреснуть.
\rsbpar\vs Luk 9:23 Ко всем же сказал: если кто хочет идти за Мною, отвергнись себя, и возьми крест свой, и следуй за Мною.
\vs Luk 9:24 Ибо кто хочет душу свою сберечь, тот потеряет ее; а кто потеряет душу свою ради Меня, тот сбережет ее.
\vs Luk 9:25 Ибо что пользы человеку приобрести весь мир, а себя самого погубить или повредить себе?
\vs Luk 9:26 Ибо кто постыдится Меня и Моих слов, того Сын Человеческий постыдится, когда приидет во славе Своей и Отца и святых Ангелов.
\vs Luk 9:27 Говорю же вам истинно: есть некоторые из стоящих здесь, которые не вкусят смерти, как уже увидят Царствие Божие.
\rsbpar\vs Luk 9:28 После сих слов, дней через восемь, взяв Петра, Иоанна и Иакова, взошел Он на гору помолиться.
\vs Luk 9:29 И когда молился, вид лица Его изменился, и одежда Его сделалась белою, блистающею.
\vs Luk 9:30 И вот, два мужа беседовали с Ним, которые были Моисей и Илия;
\vs Luk 9:31 явившись во славе, они говорили об исходе Его, который Ему надлежало совершить в Иерусалиме.
\vs Luk 9:32 Петр же и бывшие с ним отягчены были сном; но, пробудившись, увидели славу Его и двух мужей, стоявших с Ним.
\vs Luk 9:33 И когда они отходили от Него, сказал Петр Иисусу: Наставник! хорошо нам здесь быть; сделаем три кущи: одну Тебе, одну Моисею и одну Илии,~--- не зная, чт\acc{о} говорил.
\vs Luk 9:34 Когда же он говорил это, явилось облако и осенило их; и устрашились, когда вошли в облако.
\vs Luk 9:35 И был из облака глас, глаголющий: Сей есть Сын Мой Возлюбленный, Его слушайте.
\vs Luk 9:36 Когда был глас сей, остался Иисус один. И они умолчали, и никому не говорили в те дни о том, что видели.
\rsbpar\vs Luk 9:37 В следующий же день, когда они сошли с горы, встретило Его много народа.
\vs Luk 9:38 Вдруг некто из народа воскликнул: Учитель! умоляю Тебя взглянуть на сына моего, он один у меня:
\vs Luk 9:39 его схватывает дух, и он внезапно вскрикивает, и терзает его, так что он испускает пену; и насилу отступает от него, измучив его.
\vs Luk 9:40 Я просил учеников Твоих изгнать его, и они не могли.
\vs Luk 9:41 Иисус же, отвечая, сказал: о, род неверный и развращенный! доколе буду с вами и буду терпеть вас? приведи сюда сына твоего.
\vs Luk 9:42 Когда же тот еще шел, бес поверг его и стал бить; но Иисус запретил нечистому духу, и исцелил отрока, и отдал его отцу его.
\vs Luk 9:43 И все удивлялись величию Божию.\rsbpar Когда же все дивились всему, что творил Иисус, Он сказал ученикам Своим:
\vs Luk 9:44 вложите вы себе в уши слова сии: Сын Человеческий будет предан в руки человеческие.
\vs Luk 9:45 Но они не поняли сл\acc{о}ва сего, и оно было закрыто от них, так что они не постигли его, а спросить Его о сем слове боялись.
\vs Luk 9:46 Пришла же им мысль: кто бы из них был больше?
\vs Luk 9:47 Иисус же, видя помышление сердца их, взяв дитя, поставил его пред Собою
\vs Luk 9:48 и сказал им: кто примет сие дитя во имя Мое, тот Меня принимает; а кто примет Меня, тот принимает Пославшего Меня; ибо кто из вас меньше всех, тот будет велик.
\vs Luk 9:49 При сем Иоанн сказал: Наставник! мы видели человека, именем Твоим изгоняющего бесов, и запретили ему, потому что он не ходит с нами.
\vs Luk 9:50 Иисус сказал ему: не запрещайте, ибо кто не против вас, тот за вас.
\rsbpar\vs Luk 9:51 Когда же приближались дни взятия Его \bibemph{от мира}, Он восхотел идти в Иерусалим;
\vs Luk 9:52 и послал вестников пред лицем Своим; и они пошли и вошли в селение Самарянское; чтобы приготовить для Него;
\vs Luk 9:53 но \bibemph{там} не приняли Его, потому что Он имел вид путешествующего в Иерусалим.
\vs Luk 9:54 Видя т\acc{о}, ученики Его, Иаков и Иоанн, сказали: Господи! хочешь ли, мы скажем, чтобы огонь сошел с неба и истребил их, как и Илия сделал?
\vs Luk 9:55 Но Он, обратившись к ним, запретил им и сказал: не знаете, какого вы духа;
\vs Luk 9:56 ибо Сын Человеческий пришел не губить души человеческие, а спасать. И пошли в другое селение.
\rsbpar\vs Luk 9:57 Случилось, что когда они были в пути, некто сказал Ему: Господи! я пойду за Тобою, куда бы Ты ни пошел.
\vs Luk 9:58 Иисус сказал ему: лисицы имеют норы, и птицы небесные~--- гнезда; а Сын Человеческий не имеет, где приклонить голову.
\vs Luk 9:59 А другому сказал: следуй за Мною. Тот сказал: Господи! позволь мне прежде пойти и похоронить отца моего.
\vs Luk 9:60 Но Иисус сказал ему: предоставь мертвым погребать своих мертвецов, а ты иди, благовествуй Царствие Божие.
\vs Luk 9:61 Еще другой сказал: я пойду за Тобою, Господи! но прежде позволь мне проститься с домашними моими.
\vs Luk 9:62 Но Иисус сказал ему: никто, возложивший руку свою на плуг и озирающийся назад, не благонадежен для Царствия Божия.
\vs Luk 10:1 После сего избрал Господь и других семьдесят \bibemph{учеников}, и послал их по два пред лицем Своим во всякий город и место, куда Сам хотел идти,
\vs Luk 10:2 и сказал им: жатвы много, а делателей мало; итак, молите Господина жатвы, чтобы выслал делателей на жатву Свою.
\vs Luk 10:3 Идите! Я посылаю вас, как агнцев среди волков.
\vs Luk 10:4 Не берите ни мешка, ни сум\acc{ы}, ни обуви, и никого на дороге не приветствуйте.
\vs Luk 10:5 В какой дом войдете, сперва говорите: мир дому сему;
\vs Luk 10:6 и если будет там сын мира, то почиет на нём мир ваш, а если нет, то к вам возвратится.
\vs Luk 10:7 В доме же том оставайтесь, ешьте и пейте, что у них есть, ибо трудящийся достоин награды за труды свои; не переходите из дома в дом.
\vs Luk 10:8 И если придёте в какой город и примут вас, ешьте, что вам предложат,
\vs Luk 10:9 и исцеляйте находящихся в нём больных, и говорите им: приблизилось к вам Царствие Божие.
\vs Luk 10:10 Если же придете в какой город и не примут вас, то, выйдя на улицу, скажите:
\vs Luk 10:11 и прах, прилипший к нам от вашего города, отрясаем вам; однако же знайте, что приблизилось к вам Царствие Божие.
\vs Luk 10:12 Сказываю вам, что Содому в день оный будет отраднее, нежели городу тому.
\vs Luk 10:13 Горе тебе, Хоразин! горе тебе, Вифсаида! ибо если бы в Тире и Сидоне явлены были силы, явленные в вас, то давно бы они, сидя во вретище и пепле, покаялись;
\vs Luk 10:14 но и Тиру и Сидону отраднее будет на суде, нежели вам.
\vs Luk 10:15 И ты, Капернаум, до неба вознесшийся, до ада низвергнешься.
\vs Luk 10:16 Слушающий вас Меня слушает, и отвергающийся вас Меня отвергается; а отвергающийся Меня отвергается Пославшего Меня.
\rsbpar\vs Luk 10:17 Семьдесят \bibemph{учеников} возвратились с радостью и говорили: Господи! и бесы повинуются нам о имени Твоем.
\vs Luk 10:18 Он же сказал им: Я видел сатану, спадшего с неба, как молнию;
\vs Luk 10:19 се, даю вам власть наступать на змей и скорпионов и на всю силу вражью, и ничто не повредит вам;
\vs Luk 10:20 однако ж тому не радуйтесь, что духи вам повинуются, но радуйтесь тому, что имена ваши написаны на небесах.
\vs Luk 10:21 В тот час возрадовался духом Иисус и сказал: славлю Тебя, Отче, Господи неба и земли, что Ты утаил сие от мудрых и разумных и открыл младенцам. Ей, Отче! Ибо таково было Твое благоволение.
\vs Luk 10:22 И, обратившись к ученикам, сказал: всё предано Мне Отцем Моим; и кто есть Сын, не знает никто, кроме Отца, и кто есть Отец, \bibemph{не знает никто}, кроме Сына, и кому Сын хочет открыть.
\vs Luk 10:23 И, обратившись к ученикам, сказал им особо: блаженны очи, видящие то, что вы видите!
\vs Luk 10:24 ибо сказываю вам, что многие пророки и цари желали видеть, чт\acc{о} вы видите, и не видели, и слышать, чт\acc{о} вы слышите, и не слышали.
\rsbpar\vs Luk 10:25 И вот, один законник встал и, искушая Его, сказал: Учитель! чт\acc{о} мне делать, чтобы наследовать жизнь вечную?
\vs Luk 10:26 Он же сказал ему: в законе чт\acc{о} написано? к\acc{а}к читаешь?
\vs Luk 10:27 Он сказал в ответ: возлюби Господа Бога твоего всем сердцем твоим, и всею душею твоею, и всею крепостию твоею, и всем разумением твоим, и ближнего твоего, как самого себя.
\vs Luk 10:28 \bibemph{Иисус} сказал ему: правильно ты отвечал; так поступай, и будешь жить.
\vs Luk 10:29 Но он, желая оправдать себя, сказал Иисусу: а кто мой ближний?
\vs Luk 10:30 На это сказал Иисус: некоторый человек шел из Иерусалима в Иерихон и попался разбойникам, которые сняли с него одежду, изранили его и ушли, оставив его едва живым.
\vs Luk 10:31 По случаю один священник шел тою дорогою и, увидев его, прошел мимо.
\vs Luk 10:32 Также и левит, быв на том месте, подошел, посмотрел и прошел мимо.
\vs Luk 10:33 Самарянин же некто, проезжая, нашел на него и, увидев его, сжалился
\vs Luk 10:34 и, подойдя, перевязал ему раны, возливая масло и вино; и, посадив его на своего осла, привез его в гостиницу и позаботился о нем;
\vs Luk 10:35 а на другой день, отъезжая, вынул два динария, дал содержателю гостиницы и сказал ему: позаботься о нем; и если издержишь что более, я, когда возвращусь, отдам тебе.
\vs Luk 10:36 Кто из этих троих, думаешь ты, был ближний попавшемуся разбойникам?
\vs Luk 10:37 Он сказал: оказавший ему милость. Тогда Иисус сказал ему: иди, и ты поступай так же.
\rsbpar\vs Luk 10:38 В продолжение пути их пришел Он в одно селение; здесь женщина, именем Марфа, приняла Его в дом свой;
\vs Luk 10:39 у неё была сестра, именем Мария, которая села у ног Иисуса и слушала слово Его.
\vs Luk 10:40 Марфа же заботилась о большом угощении и, подойдя, сказала: Господи! или Тебе нужды нет, что сестра моя одну меня оставила служить? скажи ей, чтобы помогла мне.
\vs Luk 10:41 Иисус же сказал ей в ответ: Марфа! Марфа! ты заботишься и суетишься о многом,
\vs Luk 10:42 а одно только нужно; Мария же избрала благую часть, которая не отнимется у неё.
\vs Luk 11:1 Случилось, что когда Он в одном месте молился, и перестал, один из учеников Его сказал Ему: Господи! научи нас молиться, как и Иоанн научил учеников своих.
\vs Luk 11:2 Он сказал им: когда м\acc{о}литесь, говорите:\rsbpar Отче наш, сущий на небесах! да святится имя Твое; да приидет Царствие Твое; да будет воля Твоя и на земле, как на небе;
\vs Luk 11:3 хлеб наш насущный подавай нам на каждый день;
\vs Luk 11:4 и прости нам грехи наши, ибо и мы прощаем всякому должнику нашему; и не введи нас в искушение, но избавь нас от лукавого.
\rsbpar\vs Luk 11:5 И сказал им: \bibemph{положим, что} кто-нибудь из вас, имея друга, придёт к нему в полночь и скажет ему: друг! дай мне взаймы три хлеба,
\vs Luk 11:6 ибо друг мой с дороги зашел ко мне, и мне нечего предложить ему;
\vs Luk 11:7 а тот изнутри скажет ему в ответ: не беспокой меня, двери уже заперты, и дети мои со мною на постели; не могу встать и дать тебе.
\vs Luk 11:8 Если, говорю вам, он не встанет и не даст ему по дружбе с ним, то по неотступности его, встав, даст ему, сколько просит.
\vs Luk 11:9 И Я скажу вам: прос\acc{и}те, и дано будет вам; ищите, и найдете; стучите, и отворят вам,
\vs Luk 11:10 ибо всякий просящий получает, и ищущий находит, и стучащему отворят.
\vs Luk 11:11 Какой из вас отец, \bibemph{когда} сын попросит у него хлеба, подаст ему камень? или, \bibemph{когда попросит} рыбы, подаст ему змею вместо рыбы?
\vs Luk 11:12 Или, если попросит яйца, подаст ему скорпиона?
\vs Luk 11:13 Итак, если вы, будучи злы, умеете даяния благие давать детям вашим, тем более Отец Небесный даст Духа Святаго просящим у Него.
\rsbpar\vs Luk 11:14 Однажды изгнал Он беса, который был нем; и когда бес вышел, немой стал говорить; и народ удивился.
\vs Luk 11:15 Некоторые же из них говорили: Он изгоняет бесов силою веельзевула, князя бесовского.
\vs Luk 11:16 А другие, искушая, требовали от Него знамения с неба.
\vs Luk 11:17 Но Он, зная помышления их, сказал им: всякое царство, разделившееся само в себе, опустеет, и дом, \bibemph{разделившийся} сам в себе, падет;
\vs Luk 11:18 если же и сатана разделится сам в себе, то к\acc{а}к устоит царство его? а вы говорите, что Я силою веельзевула изгоняю бесов;
\vs Luk 11:19 и если Я силою веельзевула изгоняю бесов, то сыновья ваши чьею силою изгоняют их? Посему они будут вам судьями.
\vs Luk 11:20 Если же Я перстом Божиим изгоняю бесов, то, конечно, достигло до вас Царствие Божие.
\vs Luk 11:21 Когда сильный с оружием охраняет свой дом, тогда в безопасности его имение;
\vs Luk 11:22 когда же сильнейший его нападет на него и победит его, тогда возьмет всё оружие его, на которое он надеялся, и разделит похищенное у него.
\vs Luk 11:23 Кто не со Мною, тот против Меня; и кто не собирает со Мною, тот расточает.
\vs Luk 11:24 Когда нечистый дух выйдет из человека, то ходит по безводным местам, ища покоя, и, не находя, говорит: возвращусь в дом мой, откуда вышел;
\vs Luk 11:25 и, придя, находит его выметенным и убранным;
\vs Luk 11:26 тогда идет и берет с собою семь других духов, злейших себя, и, войдя, живут там,~--- и бывает для человека того последнее хуже первого.
\vs Luk 11:27 Когда же Он говорил это, одна женщина, возвысив голос из народа, сказала Ему: блаженно чрево, носившее Тебя, и сосцы, Тебя питавшие!
\vs Luk 11:28 А Он сказал: блаженны слышащие слово Божие и соблюдающие его.
\rsbpar\vs Luk 11:29 Когда же народ стал сходиться во множестве, Он начал говорить: род сей лукав, он ищет знамения, и знамение не дастся ему, кроме знамения Ионы пророка;
\vs Luk 11:30 ибо к\acc{а}к Иона был знамением для Ниневитян, т\acc{а}к будет и Сын Человеческий для рода сего.
\vs Luk 11:31 Царица южная восстанет на суд с людьми рода сего и осудит их, ибо она приходила от пределов земли послушать мудрости Соломоновой; и вот, здесь больше Соломона.
\vs Luk 11:32 Ниневитяне восстанут на суд с родом сим и осудят его, ибо они покаялись от проповеди Иониной, и вот, здесь больше Ионы.
\rsbpar\vs Luk 11:33 Никто, зажегши свечу, не ставит ее в сокровенном месте, ни под сосудом, но на подсвечнике, чтобы входящие видели свет.
\vs Luk 11:34 Светильник тела есть око; итак, если око твое будет чисто, то и все тело твое будет светло; а если оно будет худо, то и тело твое будет темно.
\vs Luk 11:35 Итак, смотри: свет, который в тебе, не есть ли тьма?
\vs Luk 11:36 Если же тело твое всё светло и не имеет ни одной темной части, то будет светло всё т\acc{а}к, как бы светильник освещал тебя сиянием.
\rsbpar\vs Luk 11:37 Когда Он говорил это, один фарисей просил Его к себе обедать. Он пришел и возлег.
\vs Luk 11:38 Фарисей же удивился, увидев, что Он не умыл \bibemph{рук} перед обедом.
\vs Luk 11:39 Но Господь сказал ему: ныне вы, фарисеи, внешность чаши и блюда очищаете, а внутренность ваша исполнена хищения и лукавства.
\vs Luk 11:40 Неразумные! не Тот же ли, Кто сотворил внешнее, сотворил и внутреннее?
\vs Luk 11:41 Подавайте лучше милостыню из того, чт\acc{о} у вас есть, тогда всё будет у вас чисто.
\vs Luk 11:42 Но горе вам, фарисеям, что даете десятину с мяты, руты и всяких овощей, и нерадите о суде и любви Божией: сие надлежало делать, и того не оставлять.
\vs Luk 11:43 Горе вам, фарисеям, что любите председания в синагогах и приветствия в народных собраниях.
\vs Luk 11:44 Горе вам, книжники и фарисеи, лицемеры, что вы~--- как гробы скрытые, над которыми люди ходят и не знают того.
\vs Luk 11:45 На это некто из законников сказал Ему: Учитель! говоря это, Ты и нас обижаешь.
\vs Luk 11:46 Но Он сказал: и вам, законникам, горе, что налагаете на людей бремена неудобоносимые, а сами и одним перстом своим не дотрагиваетесь до них.
\vs Luk 11:47 Горе вам, что строите гробницы пророкам, которых избили отцы ваши:
\vs Luk 11:48 сим вы свидетельствуете о делах отцов ваших и соглашаетесь с ними, ибо они избили пророков, а вы строите им гробницы.
\vs Luk 11:49 Потому и премудрость Божия сказала: пошлю к ним пророков и Апостолов, и из них одних убьют, а других изгонят,
\vs Luk 11:50 да взыщется от рода сего кровь всех пророков, пролитая от создания мира,
\vs Luk 11:51 от крови Авеля до крови Захарии, убитого между жертвенником и храмом. Ей, говорю вам, взыщется от рода сего.
\vs Luk 11:52 Горе вам, законникам, что вы взяли ключ разумения: сами не вошли, и входящим воспрепятствовали.
\vs Luk 11:53 Когда Он говорил им это, книжники и фарисеи начали сильно приступать к Нему, вынуждая у Него ответы на многое,
\vs Luk 11:54 подыскиваясь под Него и стараясь уловить что-нибудь из уст Его, чтобы обвинить Его.
\vs Luk 12:1 Между тем, когда собрались тысячи народа, так что теснили друг друга, Он начал говорить сперва ученикам Своим: берегитесь закваски фарисейской, которая есть лицемерие.
\vs Luk 12:2 Нет ничего сокровенного, что не открылось бы, и тайного, чего не узнали бы.
\vs Luk 12:3 Посему, чт\acc{о} вы сказали в темноте, т\acc{о} услышится во свете; и чт\acc{о} говорили на ухо внутри дома, т\acc{о} будет провозглашено на кровлях.
\vs Luk 12:4 Говорю же вам, друзьям Моим: не бойтесь убивающих тело и потом не могущих ничего более сделать;
\vs Luk 12:5 но скажу вам, кого бояться: бойтесь Того, Кто, по убиении, может ввергнуть в геенну: ей, говорю вам, Того бойтесь.
\vs Luk 12:6 Не пять ли малых птиц продаются за два ассария? и ни одна из них не забыта у Бога.
\vs Luk 12:7 А у вас и волосы на голове все сочтены. Итак не бойтесь: вы дороже многих малых птиц.
\vs Luk 12:8 Сказываю же вам: всякого, кто исповедает Меня пред человеками, и Сын Человеческий исповедает пред Ангелами Божиими;
\vs Luk 12:9 а кто отвергнется Меня пред человеками, тот отвержен будет пред Ангелами Божиими.
\vs Luk 12:10 И всякому, кто скажет слово на Сына Человеческого, прощено будет; а кто скажет хулу на Святаго Духа, тому не простится.
\vs Luk 12:11 Когда же приведут вас в синагоги, к начальствам и властям, не заботьтесь, к\acc{а}к или чт\acc{о} отвечать, или чт\acc{о} говорить,
\vs Luk 12:12 ибо Святый Дух научит вас в тот час, чт\acc{о} должно говорить.
\rsbpar\vs Luk 12:13 Некто из народа сказал Ему: Учитель! скажи брату моему, чтобы он разделил со мною наследство.
\vs Luk 12:14 Он же сказал человеку тому: кто поставил Меня судить или делить вас?
\vs Luk 12:15 При этом сказал им: смотрите, берегитесь любостяжания, ибо жизнь человека не зависит от изобилия его имения.
\vs Luk 12:16 И сказал им притчу: у одного богатого человека был хороший урожай в поле;
\vs Luk 12:17 и он рассуждал сам с собою: что мне делать? некуда мне собрать плодов моих?
\vs Luk 12:18 И сказал: вот что сделаю: сломаю житницы мои и построю б\acc{о}льшие, и соберу туда весь хлеб мой и всё добро мое,
\vs Luk 12:19 и скажу душе моей: душа! много добра лежит у тебя на многие годы: покойся, ешь, пей, веселись.
\vs Luk 12:20 Но Бог сказал ему: безумный! в сию ночь душу твою возьмут у тебя; кому же достанется то, что ты заготовил?
\vs Luk 12:21 Так \bibemph{бывает с тем}, кто собирает сокровища для себя, а не в Бога богатеет.
\vs Luk 12:22 И сказал ученикам Своим: посему говорю вам,~--- не заботьтесь для души вашей, что вам есть, ни для тела, во что одеться:
\vs Luk 12:23 душа больше пищи, и тело~--- одежды.
\vs Luk 12:24 Посмотрите на воронов: они не сеют, не жнут; нет у них ни хранилищ, ни житниц, и Бог питает их; сколько же вы лучше птиц?
\vs Luk 12:25 Да и кто из вас, заботясь, может прибавить себе роста хотя на один локоть?
\vs Luk 12:26 Итак, если и малейшего сделать не можете, чт\acc{о} заботитесь о прочем?
\vs Luk 12:27 Посмотрите на лилии, как они растут: не трудятся, не прядут; но говорю вам, что и Соломон во всей славе своей не одевался так, как всякая из них.
\vs Luk 12:28 Если же траву на поле, которая сегодня есть, а завтра будет брошена в печь, Бог так одевает, то кольми паче вас, маловеры!
\vs Luk 12:29 Итак, не ищите, чт\acc{о} вам есть, или чт\acc{о} пить, и не беспокойтесь,
\vs Luk 12:30 потому что всего этого ищут люди мира сего; ваш же Отец знает, что вы имеете нужду в том;
\vs Luk 12:31 наипаче ищите Царствия Божия, и это всё приложится вам.
\vs Luk 12:32 Не бойся, малое стадо! ибо Отец ваш благоволил дать вам Царство.
\vs Luk 12:33 Продавайте имения ваши и давайте милостыню. Приготовляйте себе влагалища не ветшающие, сокровище неоскудевающее на небесах, куда вор не приближается и где моль не съедает,
\vs Luk 12:34 ибо где сокровище ваше, там и сердце ваше будет.
\rsbpar\vs Luk 12:35 Да будут чресла ваши препоясаны и светильники горящи.
\vs Luk 12:36 И вы будьте подобны людям, ожидающим возвращения господина своего с брака, дабы, когда придёт и постучит, тотчас отворить ему.
\vs Luk 12:37 Блаженны рабы те, которых господин, придя, найдёт бодрствующими; истинно говорю вам, он препояшется и посадит их, и, подходя, станет служить им.
\vs Luk 12:38 И если придет во вторую стражу, и в третью стражу придет, и найдет их так, то блаженны рабы те.
\vs Luk 12:39 Вы знаете, что если бы ведал хозяин дома, в который час придет вор, то бодрствовал бы и не допустил бы подкопать дом свой.
\vs Luk 12:40 Будьте же и вы готовы, ибо, в который час не думаете, приидет Сын Человеческий.
\vs Luk 12:41 Тогда сказал Ему Петр: Господи! к нам ли притчу сию говоришь, или и ко всем?
\vs Luk 12:42 Господь же сказал: кт\acc{о} верный и благоразумный домоправитель, которого господин поставил над слугами своими раздавать им в своё время меру хлеба?
\vs Luk 12:43 Блажен раб тот, которого господин его, придя, найдет поступающим так.
\vs Luk 12:44 Истинно говорю вам, что над всем имением своим поставит его.
\vs Luk 12:45 Если же раб тот скажет в сердце своем: не скоро придет господин мой, и начнет бить слуг и служанок, есть и пить и напиваться,~---
\vs Luk 12:46 то придет господин раба того в день, в который он не ожидает, и в час, в который не думает, и рассечет его, и подвергнет его одной участи с неверными.
\vs Luk 12:47 Раб же тот, который знал волю господина своего, и не был готов, и не делал по воле его, бит будет много;
\vs Luk 12:48 а который не знал, и сделал достойное наказания, бит будет меньше. И от всякого, кому дано много, много и потребуется, и кому много вверено, с того больше взыщут.
\vs Luk 12:49 Огонь пришел Я низвести на землю, и как желал бы, чтобы он уже возгорелся!
\vs Luk 12:50 Крещением должен Я креститься; и как Я томлюсь, пока сие совершится!
\vs Luk 12:51 Думаете ли вы, что Я пришел дать мир земле? Нет, говорю вам, но разделение;
\vs Luk 12:52 ибо отныне пятеро в одном доме станут разделяться, трое против двух, и двое против трех:
\vs Luk 12:53 отец будет против сына, и сын против отца; мать против дочери, и дочь против матери; свекровь против невестки своей, и невестка против свекрови своей.
\vs Luk 12:54 Сказал же и народу: когда вы видите облако, поднимающееся с запада, тотчас говорите: дождь будет, и бывает так;
\vs Luk 12:55 и когда дует южный ветер, говорите: зной будет, и бывает.
\vs Luk 12:56 Лицемеры! лице земли и неба распознавать умеете, как же времени сего не узнаете?
\vs Luk 12:57 Зачем же вы и по самим себе не судите, чему быть должно?
\vs Luk 12:58 Когда ты идешь с соперником своим к начальству, то на дороге постарайся освободиться от него, чтобы он не привел тебя к судье, а судья не отдал тебя истязателю, а истязатель не вверг тебя в темницу.
\vs Luk 12:59 Сказываю тебе: не выйдешь оттуда, пока не отдашь и последней полушки.
\vs Luk 13:1 В это время пришли некоторые и рассказали Ему о Галилеянах, которых кровь Пилат смешал с жертвами их.
\vs Luk 13:2 Иисус сказал им на это: думаете ли вы, что эти Галилеяне были грешнее всех Галилеян, что так пострадали?
\vs Luk 13:3 Нет, говорю вам, но, если не покаетесь, все т\acc{а}к же погибнете.
\vs Luk 13:4 Или думаете ли, что те восемнадцать человек, на которых упала башня Силоамская и побила их, виновнее были всех, живущих в Иерусалиме?
\vs Luk 13:5 Нет, говорю вам, но, если не покаетесь, все т\acc{а}к же погибнете.
\vs Luk 13:6 И сказал сию притчу: некто имел в винограднике своем посаженную смоковницу, и пришел искать плода на ней, и не нашел;
\vs Luk 13:7 и сказал виноградарю: вот, я третий год прихожу искать плода на этой смоковнице и не нахожу; сруби ее: на что она и землю занимает?
\vs Luk 13:8 Но он сказал ему в ответ: господин! оставь ее и на этот год, пока я окопаю ее и обложу навозом,~---
\vs Luk 13:9 не принесет ли плода; если же нет, то в следующий \bibemph{год} срубишь ее.
\rsbpar\vs Luk 13:10 В одной из синагог учил Он в субботу.
\vs Luk 13:11 Там была женщина, восемнадцать лет имевшая духа немощи: она была скорчена и не могла выпрямиться.
\vs Luk 13:12 Иисус, увидев ее, подозвал и сказал ей: женщина! ты освобождаешься от недуга твоего.
\vs Luk 13:13 И возложил на нее руки, и она тотчас выпрямилась и стала славить Бога.
\vs Luk 13:14 При этом начальник синагоги, негодуя, что Иисус исцелил в субботу, сказал народу: есть шесть дней, в которые должно делать; в те и приход\acc{и}те исцеляться, а не в день субботний.
\vs Luk 13:15 Господь сказал ему в ответ: лицемер! не отвязывает ли каждый из вас вола своего или осла от яслей в субботу и не ведет ли поить?
\vs Luk 13:16 сию же дочь Авраамову, которую связал сатана вот уже восемнадцать лет, не надлежало ли освободить от уз сих в день субботний?
\vs Luk 13:17 И когда говорил Он это, все противившиеся Ему стыдились; и весь народ радовался о всех славных делах Его.
\rsbpar\vs Luk 13:18 Он же сказал: чему подобно Царствие Божие? и чему уподоблю его?
\vs Luk 13:19 Оно подобно зерну горчичному, которое, взяв, человек посадил в саду своем; и выросло, и стало большим деревом, и птицы небесные укрывались в ветвях его.
\vs Luk 13:20 Ещё сказал: чему уподоблю Царствие Божие?
\vs Luk 13:21 Оно подобно закваске, которую женщина, взяв, положила в три меры муки, доколе не вскисло всё.
\rsbpar\vs Luk 13:22 И проходил по городам и селениям, уча и направляя путь к Иерусалиму.
\rsbpar\vs Luk 13:23 Некто сказал Ему: Господи! неужели мало спасающихся? Он же сказал им:
\vs Luk 13:24 подвизайтесь войти сквозь тесные врата, ибо, сказываю вам, многие поищут войти, и не возмогут.
\vs Luk 13:25 Когда хозяин дома встанет и затворит двери, тогда вы, стоя вне, станете стучать в двери и говорить: Господи! Господи! отвори нам; но Он скажет вам в ответ: не знаю вас, откуда вы.
\vs Luk 13:26 Тогда станете говорить: мы ели и пили пред Тобою, и на улицах наших учил Ты.
\vs Luk 13:27 Но Он скажет: говорю вам: не знаю вас, откуда вы; отойдите от Меня все делатели неправды.
\vs Luk 13:28 Там будет плач и скрежет зубов, когда увидите Авраама, Исаака и Иакова и всех пророков в Царствии Божием, а себя изгоняемыми вон.
\vs Luk 13:29 И придут от востока и запада, и севера и юга, и возлягут в Царствии Божием.
\vs Luk 13:30 И вот, есть последние, которые будут первыми, и есть первые, которые будут последними.
\rsbpar\vs Luk 13:31 В тот день пришли некоторые из фарисеев и говорили Ему: выйди и удались отсюда, ибо Ирод хочет убить Тебя.
\vs Luk 13:32 И сказал им: пойдите, скажите этой лисице: се, изгоняю бесов и совершаю исцеления сегодня и завтра, и в третий \bibemph{день} кончу;
\vs Luk 13:33 а впрочем, Мне должно ходить сегодня, завтра и в последующий день, потому что не бывает, чтобы пророк погиб вне Иерусалима.
\vs Luk 13:34 Иерусалим! Иерусалим! избивающий пророков и камнями побивающий посланных к тебе! сколько раз хотел Я собрать чад твоих, как птица птенцов своих под крылья, и вы не захотели!
\vs Luk 13:35 Се, оставляется вам дом ваш пуст. Сказываю же вам, что вы не увидите Меня, пока не придет время, когда скажете: благословен Грядый во имя Господне!
\vs Luk 14:1 Случилось Ему в субботу прийти в дом одного из начальников фарисейских вкусить хлеба, и они наблюдали за Ним.
\vs Luk 14:2 И вот, предстал пред Него человек, страждущий водяною болезнью.
\vs Luk 14:3 По сему случаю Иисус спросил законников и фарисеев: позволительно ли врачевать в субботу?
\vs Luk 14:4 Они молчали. И, прикоснувшись, исцелил его и отпустил.
\vs Luk 14:5 При сем сказал им: если у кого из вас осёл или вол упадет в колодезь, не тотчас ли вытащит его и в субботу?
\vs Luk 14:6 И не могли отвечать Ему на это.
\rsbpar\vs Luk 14:7 Замечая же, как званые выбирали первые места, сказал им притчу:
\vs Luk 14:8 когда ты будешь позван кем на брак, не садись на первое место, чтобы не случился кто из званых им почетнее тебя,
\vs Luk 14:9 и звавший тебя и его, подойдя, не сказал бы тебе: уступи ему место; и тогда со стыдом должен будешь занять последнее место.
\vs Luk 14:10 Но когда зван будешь, придя, садись на последнее место, чтобы звавший тебя, подойдя, сказал: друг! пересядь выше; тогда будет тебе честь пред сидящими с тобою,
\vs Luk 14:11 ибо всякий возвышающий сам себя унижен будет, а унижающий себя возвысится.
\vs Luk 14:12 Сказал же и позвавшему Его: когда делаешь обед или ужин, не зови друзей твоих, ни братьев твоих, ни родственников твоих, ни соседей богатых, чтобы и они тебя когда не позвали, и не получил ты воздаяния.
\vs Luk 14:13 Но, когда делаешь пир, зови нищих, увечных, хромых, слепых,
\vs Luk 14:14 и блажен будешь, что они не могут воздать тебе, ибо воздастся тебе в воскресение праведных.
\vs Luk 14:15 Услышав это, некто из возлежащих с Ним сказал Ему: блажен, кто вкусит хлеба в Царствии Божием!
\vs Luk 14:16 Он же сказал ему: один человек сделал большой ужин и звал многих,
\vs Luk 14:17 и когда наступило время ужина, послал раба своего сказать званым: идите, ибо уже всё готово.
\vs Luk 14:18 И начали все, как бы сговорившись, извиняться. Первый сказал ему: я купил землю и мне нужно пойти посмотреть ее; прошу тебя, извини меня.
\vs Luk 14:19 Другой сказал: я купил пять пар волов и иду испытать их; прошу тебя, извини меня.
\vs Luk 14:20 Третий сказал: я женился и потому не могу прийти.
\vs Luk 14:21 И, возвратившись, раб тот донес о сем господину своему. Тогда, разгневавшись, хозяин дома сказал рабу своему: пойди скорее по улицам и переулкам города и приведи сюда нищих, увечных, хромых и слепых.
\vs Luk 14:22 И сказал раб: господин! исполнено, как приказал ты, и еще есть место.
\vs Luk 14:23 Господин сказал рабу: пойди по дорогам и изгородям и убеди прийти, чтобы наполнился дом мой.
\vs Luk 14:24 Ибо сказываю вам, что никто из тех званых не вкусит моего ужина, ибо много званых, но мало избранных.
\rsbpar\vs Luk 14:25 С Ним шло множество народа; и Он, обратившись, сказал им:
\vs Luk 14:26 если кто приходит ко Мне и не возненавидит отца своего и матери, и жены и детей, и братьев и сестер, а притом и самой жизни своей, тот не может быть Моим учеником;
\vs Luk 14:27 и кто не несет креста своего и идёт за Мною, не может быть Моим учеником.
\vs Luk 14:28 Ибо кто из вас, желая построить башню, не сядет прежде и не вычислит издержек, имеет ли он, что нужно для совершения ее,
\vs Luk 14:29 дабы, когда положит основание и не возможет совершить, все видящие не стали смеяться над ним,
\vs Luk 14:30 говоря: этот человек начал строить и не мог окончить?
\vs Luk 14:31 Или какой царь, идя на войну против другого царя, не сядет и не посоветуется прежде, силен ли он с десятью тысячами противостать идущему на него с двадцатью тысячами?
\vs Luk 14:32 Иначе, пока тот еще далеко, он пошлет к нему посольство просить о мире.
\vs Luk 14:33 Так всякий из вас, кто не отрешится от всего, что имеет, не может быть Моим учеником.
\vs Luk 14:34 Соль~--- добрая вещь; но если соль потеряет силу, чем исправить ее?
\vs Luk 14:35 ни в землю, ни в навоз не годится; вон выбрасывают ее. Кто имеет уши слышать, да слышит!
\vs Luk 15:1 Приближались к Нему все мытари и грешники слушать Его.
\vs Luk 15:2 Фарисеи же и книжники роптали, говоря: Он принимает грешников и ест с ними.
\vs Luk 15:3 Но Он сказал им следующую притчу:
\vs Luk 15:4 кто из вас, имея сто овец и потеряв одну из них, не оставит девяноста девяти в пустыне и не пойдет за пропавшею, пока не найдет ее?
\vs Luk 15:5 А найдя, возьмет ее на плечи свои с радостью
\vs Luk 15:6 и, придя домой, созовет друзей и соседей и скажет им: порадуйтесь со мною: я нашел мою пропавшую овцу.
\vs Luk 15:7 Сказываю вам, что так на небесах более радости будет об одном грешнике кающемся, нежели о девяноста девяти праведниках, не имеющих нужды в покаянии.
\vs Luk 15:8 Или какая женщина, имея десять драхм, если потеряет одну драхму, не зажжет свеч\acc{и} и не станет мести комнату и искать тщательно, пока не найдет,
\vs Luk 15:9 а найдя, созовет подруг и соседок и скажет: порадуйтесь со мною: я нашла потерянную драхму.
\vs Luk 15:10 Так, говорю вам, бывает радость у Ангелов Божиих и об одном грешнике кающемся.
\rsbpar\vs Luk 15:11 Еще сказал: у некоторого человека было два сына;
\vs Luk 15:12 и сказал младший из них отцу: отче! дай мне следующую \bibemph{мне} часть имения. И \bibemph{отец} разделил им имение.
\vs Luk 15:13 По прошествии немногих дней младший сын, собрав всё, пошел в дальнюю сторону и там расточил имение свое, живя распутно.
\vs Luk 15:14 Когда же он прожил всё, настал великий голод в той стране, и он начал нуждаться;
\vs Luk 15:15 и пошел, пристал к одному из жителей страны той, а тот послал его на поля свои пасти свиней;
\vs Luk 15:16 и он рад был наполнить чрево свое рожк\acc{а}ми, которые ели свиньи, но никто не давал ему.
\vs Luk 15:17 Придя же в себя, сказал: сколько наемников у отца моего избыточествуют хлебом, а я умираю от голода;
\vs Luk 15:18 встану, пойду к отцу моему и скажу ему: отче! я согрешил против неба и пред тобою
\vs Luk 15:19 и уже недостоин называться сыном твоим; прими меня в число наемников твоих.
\vs Luk 15:20 Встал и пошел к отцу своему. И когда он был еще далеко, увидел его отец его и сжалился; и, побежав, пал ему на шею и целовал его.
\vs Luk 15:21 Сын же сказал ему: отче! я согрешил против неба и пред тобою и уже недостоин называться сыном твоим.
\vs Luk 15:22 А отец сказал рабам своим: принесите лучшую одежду и оденьте его, и дайте перстень на руку его и обувь на ноги;
\vs Luk 15:23 и приведите откормленного теленка, и заколите; станем есть и веселиться!
\vs Luk 15:24 ибо этот сын мой был мертв и ожил, пропадал и нашелся. И начали веселиться.
\vs Luk 15:25 Старший же сын его был на поле; и возвращаясь, когда приблизился к дому, услышал пение и ликование;
\vs Luk 15:26 и, призвав одного из слуг, спросил: что это такое?
\vs Luk 15:27 Он сказал ему: брат твой пришел, и отец твой заколол откормленного теленка, потому что принял его здоровым.
\vs Luk 15:28 Он осердился и не хотел войти. Отец же его, выйдя, звал его.
\vs Luk 15:29 Но он сказал в ответ отцу: вот, я столько лет служу тебе и никогда не преступал приказания твоего, но ты никогда не дал мне и козлёнка, чтобы мне повеселиться с друзьями моими;
\vs Luk 15:30 а когда этот сын твой, расточивший имение своё с блудницами, пришел, ты заколол для него откормленного теленка.
\vs Luk 15:31 Он же сказал ему: сын мой! ты всегда со мною, и всё мое твое,
\vs Luk 15:32 а о том надобно было радоваться и веселиться, что брат твой сей был мертв и ожил, пропадал и нашелся.
\vs Luk 16:1 Сказал же и к ученикам Своим: один человек был богат и имел управителя, на которого донесено было ему, что расточает имение его;
\vs Luk 16:2 и, призвав его, сказал ему: что это я слышу о тебе? дай отчет в управлении твоем, ибо ты не можешь более управлять.
\vs Luk 16:3 Тогда управитель сказал сам в себе: что мне делать? господин мой отнимает у меня управление домом; копать не могу, просить стыжусь;
\vs Luk 16:4 знаю, что сделать, чтобы приняли меня в домы свои, когда отставлен буду от управления домом.
\vs Luk 16:5 И, призвав должников господина своего, каждого порознь, сказал первому: сколько ты должен господину моему?
\vs Luk 16:6 Он сказал: сто мер масла. И сказал ему: возьми твою расписку и садись скорее, напиши: пятьдесят.
\vs Luk 16:7 Потом другому сказал: а ты сколько должен? Он отвечал: сто мер пшеницы. И сказал ему: возьми твою расписку и напиши: восемьдесят.
\vs Luk 16:8 И похвалил господин управителя неверного, что догадливо поступил; ибо сыны века сего догадливее сынов света в своем роде.
\vs Luk 16:9 И Я говорю вам: приобретайте себе друзей богатством неправедным, чтобы они, когда обнищаете, приняли вас в вечные обители.
\vs Luk 16:10 Верный в малом и во многом верен, а неверный в малом неверен и во многом.
\vs Luk 16:11 Итак, если вы в неправедном богатстве не были верны, кто поверит вам истинное?
\vs Luk 16:12 И если в чужом не были верны, кто даст вам ваше?
\vs Luk 16:13 Никакой слуга не может служить двум господам, ибо или одного будет ненавидеть, а другого любить, или одному станет усердствовать, а о другом нерадеть. Не можете служить Богу и маммоне.
\rsbpar\vs Luk 16:14 Слышали всё это и фарисеи, которые были сребролюбивы, и они смеялись над Ним.
\vs Luk 16:15 Он сказал им: вы выказываете себя праведниками пред людьми, но Бог знает сердц\acc{а} ваши, ибо что высоко у людей, т\acc{о} мерзость пред Богом.
\vs Luk 16:16 Закон и пророки до Иоанна; с сего времени Царствие Божие благовествуется, и всякий усилием входит в него.
\vs Luk 16:17 Но скорее небо и земля прейдут, нежели одна черта из закона пропадет.
\vs Luk 16:18 Всякий, разводящийся с женою своею и женящийся на другой, прелюбодействует, и всякий, женящийся на разведенной с мужем, прелюбодействует.
\rsbpar\vs Luk 16:19 Некоторый человек был богат, одевался в порфиру и виссон и каждый день пиршествовал блистательно.
\vs Luk 16:20 Был также некоторый нищий, именем Лазарь, который лежал у ворот его в струпьях
\vs Luk 16:21 и желал напитаться крошками, падающими со стола богача, и псы, приходя, лизали струпья его.
\vs Luk 16:22 Умер нищий и отнесен был Ангелами на лоно Авраамово. Умер и богач, и похоронили его.
\vs Luk 16:23 И в аде, будучи в муках, он поднял глаза свои, увидел вдали Авраама и Лазаря на лоне его
\vs Luk 16:24 и, возопив, сказал: отче Аврааме! умилосердись надо мною и пошли Лазаря, чтобы омочил конец перста своего в воде и прохладил язык мой, ибо я мучаюсь в пламени сем.
\vs Luk 16:25 Но Авраам сказал: чадо! вспомни, что ты получил уже доброе твое в жизни твоей, а Лазарь~--- злое; ныне же он здесь утешается, а ты страдаешь;
\vs Luk 16:26 и сверх всего того между нами и вами утверждена великая пропасть, так что хотящие перейти отсюда к вам не могут, также и оттуда к нам не переходят.
\vs Luk 16:27 Тогда сказал он: так прошу тебя, отче, пошли его в дом отца моего,
\vs Luk 16:28 ибо у меня пять братьев; пусть он засвидетельствует им, чтобы и они не пришли в это место мучения.
\vs Luk 16:29 Авраам сказал ему: у них есть Моисей и пророки; пусть слушают их.
\vs Luk 16:30 Он же сказал: нет, отче Аврааме, но если кто из мертвых придет к ним, покаются.
\vs Luk 16:31 Тогда \bibemph{Авраам} сказал ему: если Моисея и пророков не слушают, то если бы кто и из мертвых воскрес, не поверят.
\vs Luk 17:1 Сказал также \bibemph{Иисус} ученикам: невозможно не прийти соблазнам, но горе тому, через кого они приходят;
\vs Luk 17:2 лучше было бы ему, если бы мельничный жернов повесили ему на шею и бросили его в море, нежели чтобы он соблазнил одного из малых сих.
\vs Luk 17:3 Наблюдайте за собою. Если же согрешит против тебя брат твой, выговори ему; и если покается, прости ему;
\vs Luk 17:4 и если семь раз в день согрешит против тебя и семь раз в день обратится, и скажет: каюсь,~--- прости ему.
\rsbpar\vs Luk 17:5 И сказали Апостолы Господу: умножь в нас веру.
\vs Luk 17:6 Господь сказал: если бы вы имели веру с зерно горчичное и сказали смоковнице сей: исторгнись и пересадись в море, то она послушалась бы вас.
\vs Luk 17:7 Кто из вас, имея раба п\acc{а}шущего или пасущего, по возвращении его с поля, скажет ему: пойди скорее, садись за стол?
\vs Luk 17:8 Напротив, не скажет ли ему: приготовь мне поужинать и, подпоясавшись, служи мне, пока буду есть и пить, и потом ешь и пей сам?
\vs Luk 17:9 Станет ли он благодарить раба сего за то, что он исполнил приказание? Не думаю.
\vs Luk 17:10 Так и вы, когда исполните всё повеленное вам, говорите: мы рабы ничего не стоящие, потому что сделали, чт\acc{о} должны были сделать.
\rsbpar\vs Luk 17:11 Идя в Иерусалим, Он проходил между Самариею и Галилеею.
\vs Luk 17:12 И когда входил Он в одно селение, встретили Его десять человек прокаженных, которые остановились вдали
\vs Luk 17:13 и громким голосом говорили: Иисус Наставник! помилуй нас.
\vs Luk 17:14 Увидев \bibemph{их}, Он сказал им: пойдите, покажитесь священникам. И когда они шли, очистились.
\vs Luk 17:15 Один же из них, видя, что исцелен, возвратился, громким голосом прославляя Бога,
\vs Luk 17:16 и пал ниц к ногам Его, благодаря Его; и это был Самарянин.
\vs Luk 17:17 Тогда Иисус сказал: не десять ли очистились? где же девять?
\vs Luk 17:18 как они не возвратились воздать славу Богу, кроме сего иноплеменника?
\vs Luk 17:19 И сказал ему: встань, иди; вера твоя спасла тебя.
\rsbpar\vs Luk 17:20 Быв же спрошен фарисеями, когда придет Царствие Божие, отвечал им: не придет Царствие Божие приметным образом,
\vs Luk 17:21 и не скажут: вот, оно здесь, или: вот, там. Ибо вот, Царствие Божие внутрь вас есть.
\vs Luk 17:22 Сказал также ученикам: придут дни, когда пожелаете видеть хотя один из дней Сына Человеческого, и не увидите;
\vs Luk 17:23 и скажут вам: вот, здесь, или: вот, там,~--- не ходите и не гоняйтесь,
\vs Luk 17:24 ибо, как молния, сверкнувшая от одного края неба, блистает до другого края неба, так будет Сын Человеческий в день Свой.
\vs Luk 17:25 Но прежде надлежит Ему много пострадать и быть отвержену родом сим.
\vs Luk 17:26 И как было во дни Ноя, так будет и во дни Сына Человеческого:
\vs Luk 17:27 ели, пили, женились, выходили замуж, до того дня, как вошел Ной в ковчег, и пришел потоп и погубил всех.
\vs Luk 17:28 Т\acc{а}к же, к\acc{а}к было и во дни Лота: ели, пили, покупали, продавали, садили, строили;
\vs Luk 17:29 но в день, в который Лот вышел из Содома, пролился с неба дождь огненный и серный и истребил всех;
\vs Luk 17:30 так будет и в тот день, когда Сын Человеческий явится.
\vs Luk 17:31 В тот день, кто будет на кровле, а вещи его в доме, тот не сходи взять их; и кто будет на поле, также не обращайся назад.
\vs Luk 17:32 Вспоминайте жену Лотову.
\vs Luk 17:33 Кто станет сберегать душу свою, тот погубит ее; а кто погубит ее, тот оживит ее.
\vs Luk 17:34 Сказываю вам: в ту ночь будут двое на одной постели: один возьмется, а другой оставится;
\vs Luk 17:35 две будут молоть вместе: одна возьмется, а другая оставится;
\vs Luk 17:36 двое будут на поле: один возьмется, а другой оставится.
\vs Luk 17:37 На это сказали Ему: где, Господи? Он же сказал им: где труп, там соберутся и орлы.
\vs Luk 18:1 Сказал также им притчу о том, что должно всегда молиться и не унывать,
\vs Luk 18:2 говоря: в одном городе был судья, который Бога не боялся и людей не стыдился.
\vs Luk 18:3 В том же городе была одна вдова, и она, приходя к нему, говорила: защити меня от соперника моего.
\vs Luk 18:4 Но он долгое время не хотел. А после сказал сам в себе: хотя я и Бога не боюсь и людей не стыжусь,
\vs Luk 18:5 но, как эта вдова не дает мне покоя, защищу ее, чтобы она не приходила больше докучать мне.
\vs Luk 18:6 И сказал Господь: слышите, что говорит судья неправедный?
\vs Luk 18:7 Бог ли не защитит избранных Своих, вопиющих к Нему день и ночь, хотя и медлит защищать их?
\vs Luk 18:8 сказываю вам, что подаст им защиту вскоре. Но Сын Человеческий, придя, найдет ли веру на земле?
\rsbpar\vs Luk 18:9 Сказал также к некоторым, которые уверены были о себе, что они праведны, и уничижали других, следующую притчу:
\vs Luk 18:10 два человека вошли в храм помолиться: один фарисей, а другой мытарь.
\vs Luk 18:11 Фарисей, став, молился сам в себе так: Боже! благодарю Тебя, что я не таков, как прочие люди, грабители, обидчики, прелюбодеи, или как этот мытарь:
\vs Luk 18:12 пощусь два раза в неделю, даю десятую часть из всего, чт\acc{о} приобретаю.
\vs Luk 18:13 Мытарь же, стоя вдали, не смел даже поднять глаз на небо; но, ударяя себя в грудь, говорил: Боже! будь милостив ко мне грешнику!
\vs Luk 18:14 Сказываю вам, что сей пошел оправданным в дом свой более, нежели тот: ибо всякий, возвышающий сам себя, унижен будет, а унижающий себя возвысится.
\rsbpar\vs Luk 18:15 Приносили к Нему и младенцев, чтобы Он прикоснулся к ним; ученики же, видя то, возбраняли им.
\vs Luk 18:16 Но Иисус, подозвав их, сказал: пустите детей приходить ко Мне и не возбраняйте им, ибо таковых есть Царствие Божие.
\vs Luk 18:17 Истинно говорю вам: кто не примет Царствия Божия, как дитя, тот не войдет в него.
\rsbpar\vs Luk 18:18 И спросил Его некто из начальствующих: Учитель благий! что мне делать, чтобы наследовать жизнь вечную?
\vs Luk 18:19 Иисус сказал ему: что ты называешь Меня благим? никто не благ, как только один Бог;
\vs Luk 18:20 знаешь заповеди: не прелюбодействуй, не убивай, не кради, не лжесвидетельствуй, почитай отца твоего и матерь твою.
\vs Luk 18:21 Он же сказал: все это сохранил я от юности моей.
\vs Luk 18:22 Услышав это, Иисус сказал ему: еще одного недостает тебе: все, что имеешь, продай и раздай нищим, и будешь иметь сокровище на небесах, и приходи, следуй за Мною.
\vs Luk 18:23 Он же, услышав сие, опечалился, потому что был очень богат.
\vs Luk 18:24 Иисус, видя, что он опечалился, сказал: как трудно имеющим богатство войти в Царствие Божие!
\vs Luk 18:25 ибо удобнее верблюду пройти сквозь игольные уши, нежели богатому войти в Царствие Божие.
\vs Luk 18:26 Слышавшие сие сказали: кто же может спастись?
\vs Luk 18:27 Но Он сказал: невозможное человекам возможно Богу.
\rsbpar\vs Luk 18:28 Петр же сказал: вот, мы оставили все и последовали за Тобою.
\vs Luk 18:29 Он сказал им: истинно говорю вам: нет никого, кто оставил бы дом, или родителей, или братьев, или сестер, или жену, или детей для Царствия Божия,
\vs Luk 18:30 и не получил бы гораздо более в сие время, и в век будущий жизни вечной.
\rsbpar\vs Luk 18:31 Отозвав же двенадцать учеников Своих, сказал им: вот, мы восходим в Иерусалим, и совершится все, написанное через пророков о Сыне Человеческом,
\vs Luk 18:32 ибо предадут Его язычникам, и поругаются над Ним, и оскорбят Его, и оплюют Его,
\vs Luk 18:33 и будут бить, и убьют Его: и в третий день воскреснет.
\vs Luk 18:34 Но они ничего из этого не поняли; слова сии были для них сокровенны, и они не разумели сказанного.
\rsbpar\vs Luk 18:35 Когда же подходил Он к Иерихону, один слепой сидел у дороги, прося милостыни,
\vs Luk 18:36 и, услышав, что мимо него проходит народ, спросил: что это такое?
\vs Luk 18:37 Ему сказали, что Иисус Назорей идет.
\vs Luk 18:38 Тогда он закричал: Иисус, Сын Давидов! помилуй меня.
\vs Luk 18:39 Шедшие впереди заставляли его молчать; но он еще громче кричал: Сын Давидов! помилуй меня.
\vs Luk 18:40 Иисус, остановившись, велел привести его к Себе: и, когда тот подошел к Нему, спросил его:
\vs Luk 18:41 чего ты хочешь от Меня? Он сказал: Господи! чтобы мне прозреть.
\vs Luk 18:42 Иисус сказал ему: прозри! вера твоя спасла тебя.
\vs Luk 18:43 И он тотчас прозрел и пошел за Ним, славя Бога; и весь народ, видя это, воздал хвалу Богу.
\vs Luk 19:1 Потом \bibemph{Иисус} вошел в Иерихон и проходил через него.
\vs Luk 19:2 И вот, некто, именем Закхей, начальник мытарей и человек богатый,
\vs Luk 19:3 искал видеть Иисуса, кто Он, но не мог за народом, потому что мал был ростом,
\vs Luk 19:4 и, забежав вперед, взлез на смоковницу, чтобы увидеть Его, потому что Ему надлежало проходить мимо нее.
\vs Luk 19:5 Иисус, когда пришел на это место, взглянув, увидел его и сказал ему: Закхей! сойди скорее, ибо сегодня надобно Мне быть у тебя в доме.
\vs Luk 19:6 И он поспешно сошел и принял Его с радостью.
\vs Luk 19:7 И все, видя то, начали роптать, и говорили, что Он зашел к грешному человеку;
\vs Luk 19:8 Закхей же, став, сказал Господу: Господи! половину имения моего я отдам нищим, и, если кого чем обидел, воздам вчетверо.
\vs Luk 19:9 Иисус сказал ему: ныне пришло спасение дому сему, потому что и он сын Авраама,
\vs Luk 19:10 ибо Сын Человеческий пришел взыскать и спасти погибшее.
\rsbpar\vs Luk 19:11 Когда же они слушали это, присовокупил притчу: ибо Он был близ Иерусалима, и они думали, что скоро должно открыться Царствие Божие.
\vs Luk 19:12 Итак сказал: некоторый человек высокого рода отправлялся в дальнюю страну, чтобы получить себе царство и возвратиться;
\vs Luk 19:13 призвав же десять рабов своих, дал им десять мин\fns{Фунтов серебра.} и сказал им: употребляйте их в оборот, пока я возвращусь.
\vs Luk 19:14 Но граждане ненавидели его и отправили вслед за ним посольство, сказав: не хотим, чтобы он царствовал над нами.
\vs Luk 19:15 И когда возвратился, получив царство, велел призвать к себе рабов тех, которым дал серебро, чтобы узнать, кто что приобрел.
\vs Luk 19:16 Пришел первый и сказал: господин! мина твоя принесла десять мин.
\vs Luk 19:17 И сказал ему: хорошо, добрый раб! за то, что ты в малом был верен, возьми в управление десять городов.
\vs Luk 19:18 Пришел второй и сказал: господин! мина твоя принесла пять мин.
\vs Luk 19:19 Сказал и этому: и ты будь над пятью городами.
\vs Luk 19:20 Пришел третий и сказал: господин! вот твоя мина, которую я хранил, завернув в платок,
\vs Luk 19:21 ибо я боялся тебя, потому что ты человек жестокий: берешь, чего не клал, и жнешь, чего не сеял.
\vs Luk 19:22 \bibemph{Господин} сказал ему: твоими устами буду судить тебя, лукавый раб! ты знал, что я человек жестокий, беру, чего не клал, и жну, чего не сеял;
\vs Luk 19:23 для чего же ты не отдал серебра моего в оборот, чтобы я, придя, получил его с прибылью?
\vs Luk 19:24 И сказал предстоящим: возьмите у него мину и дайте имеющему десять мин.
\vs Luk 19:25 И сказали ему: господин! у него есть десять мин.
\vs Luk 19:26 Сказываю вам, что всякому имеющему дано будет, а у неимеющего отнимется и то, что имеет;
\vs Luk 19:27 врагов же моих тех, которые не хотели, чтобы я царствовал над ними, приведите сюда и избейте предо мною.
\vs Luk 19:28 Сказав это, Он пошел далее, восходя в Иерусалим.
\rsbpar\vs Luk 19:29 И когда приблизился к Виффагии и Вифании, к горе, называемой Елеонскою, послал двух учеников Своих,
\vs Luk 19:30 сказав: пойдите в противолежащее селение; войдя в него, найдете молодого осла привязанного, на которого никто из людей никогда не садился; отвязав его, приведите;
\vs Luk 19:31 и если кто спросит вас: зачем отвязываете? скажите ему так: он надобен Господу.
\vs Luk 19:32 Посланные пошли и нашли, как Он сказал им.
\vs Luk 19:33 Когда же они отвязывали молодого осла, хозяева его сказали им: зачем отвязываете осленка?
\vs Luk 19:34 Они отвечали: он надобен Господу.
\vs Luk 19:35 И привели его к Иисусу, и, накинув одежды свои на осленка, посадили на него Иисуса.
\vs Luk 19:36 И, когда Он ехал, постилали одежды свои по дороге.
\vs Luk 19:37 А когда Он приблизился к спуску с горы Елеонской, все множество учеников начало в радости велегласно славить Бога за все чудеса, какие видели они,
\vs Luk 19:38 говоря: благословен Царь, грядущий во имя Господне! мир на небесах и слава в вышних!
\vs Luk 19:39 И некоторые фарисеи из среды народа сказали Ему: Учитель! запрети ученикам Твоим.
\vs Luk 19:40 Но Он сказал им в ответ: сказываю вам, что если они умолкнут, то камни возопиют.
\vs Luk 19:41 И когда приблизился к городу, то, смотря на него, заплакал о нем
\vs Luk 19:42 и сказал: о, если бы и ты хотя в сей твой день узнал, что служит к миру твоему! Но это сокрыто ныне от глаз твоих,
\vs Luk 19:43 ибо придут на тебя дни, когда враги твои обложат тебя окопами и окружат тебя, и стеснят тебя отовсюду,
\vs Luk 19:44 и разорят тебя, и побьют детей твоих в тебе, и не оставят в тебе камня на камне за т\acc{о}, что ты не узнал времени посещения твоего.
\vs Luk 19:45 И, войдя в храм, начал выгонять продающих в нем и покупающих,
\vs Luk 19:46 говоря им: написано: дом Мой есть дом молитвы, а вы сделали его вертепом разбойников.
\vs Luk 19:47 И учил каждый день в храме. Первосвященники же и книжники и старейшины народа искали погубить Его,
\vs Luk 19:48 и не находили, что бы сделать с Ним; потому что весь народ неотступно слушал Его.
\vs Luk 20:1 В один из тех дней, когда Он учил народ в храме и благовествовал, приступили первосвященники и книжники со старейшинами,
\vs Luk 20:2 и сказали Ему: скажи нам, какою властью Ты это делаешь, или кто дал Тебе власть сию?
\vs Luk 20:3 Он сказал им в ответ: спрошу и Я вас об одном, и скажите Мне:
\vs Luk 20:4 крещение Иоанново с небес было, или от человеков?
\vs Luk 20:5 Они же, рассуждая между собою, говорили: если скажем: с небес, то скажет: почему же вы не поверили ему?
\vs Luk 20:6 а если скажем: от человеков, то весь народ побьет нас камнями, ибо он уверен, что Иоанн есть пророк.
\vs Luk 20:7 И отвечали: не знаем откуда.
\vs Luk 20:8 Иисус сказал им: и Я не скажу вам, какою властью это делаю.
\rsbpar\vs Luk 20:9 И начал Он говорить к народу притчу сию: один человек насадил виноградник и отдал его виноградарям, и отлучился на долгое время;
\vs Luk 20:10 и в свое время послал к виноградарям раба, чтобы они дали ему плодов из виноградника; но виноградари, прибив его, отослали ни с чем.
\vs Luk 20:11 Еще послал другого раба; но они и этого, прибив и обругав, отослали ни с чем.
\vs Luk 20:12 И еще послал третьего; но они и того, изранив, выгнали.
\vs Luk 20:13 Тогда сказал господин виноградника: что мне делать? Пошлю сына моего возлюбленного; может быть, увидев его, постыдятся.
\vs Luk 20:14 Но виноградари, увидев его, рассуждали между собою, говоря: это наследник; пойдем, убьем его, и наследство его будет наше.
\vs Luk 20:15 И, выведя его вон из виноградника, убили. Что же сделает с ними господин виноградника?
\vs Luk 20:16 Придет и погубит виноградарей тех, и отдаст виноградник другим. Слышавшие же это сказали: да не будет!
\vs Luk 20:17 Но Он, взглянув на них, сказал: что значит сие написанное: камень, который отвергли строители, тот самый сделался главою угла?
\vs Luk 20:18 Всякий, кто упадет на тот камень, разобьется, а на кого он упадет, того раздавит.
\vs Luk 20:19 И искали в это время первосвященники и книжники, чтобы наложить на Него руки, но побоялись народа, ибо поняли, что о них сказал Он эту притчу.
\vs Luk 20:20 И, наблюдая за Ним, подослали лукавых людей, которые, притворившись благочестивыми, уловили бы Его в каком-либо слове, чтобы предать Его начальству и власти правителя.
\vs Luk 20:21 И они спросили Его: Учитель! мы знаем, что Ты правдиво говоришь и учишь и не смотришь на лице, но истинно пути Божию учишь;
\vs Luk 20:22 позволительно ли нам давать подать кесарю, или нет?
\vs Luk 20:23 Он же, уразумев лукавство их, сказал им: что вы Меня искушаете?
\vs Luk 20:24 Покажите Мне динарий: чье на нем изображение и надпись? Они отвечали: кесаревы.
\vs Luk 20:25 Он сказал им: итак, отдавайте кесарево кесарю, а Божие Богу.
\vs Luk 20:26 И не могли уловить Его в слове перед народом, и, удивившись ответу Его, замолчали.
\rsbpar\vs Luk 20:27 Тогда пришли некоторые из саддукеев, отвергающих воскресение, и спросили Его:
\vs Luk 20:28 Учитель! Моисей написал нам, что если у кого умрет брат, имевший жену, и умрет бездетным, то брат его должен взять его жену и восставить семя брату своему.
\vs Luk 20:29 Было семь братьев, первый, взяв жену, умер бездетным;
\vs Luk 20:30 взял ту жену второй, и тот умер бездетным;
\vs Luk 20:31 взял ее третий; также и все семеро, и умерли, не оставив детей;
\vs Luk 20:32 после всех умерла и жена;
\vs Luk 20:33 итак, в воскресение которого из них будет она женою, ибо семеро имели ее женою?
\vs Luk 20:34 Иисус сказал им в ответ: чада века сего женятся и выходят замуж;
\vs Luk 20:35 а сподобившиеся достигнуть того века и воскресения из мертвых ни женятся, ни замуж не выходят,
\vs Luk 20:36 и умереть уже не могут, ибо они равны Ангелам и суть сыны Божии, будучи сынами воскресения.
\vs Luk 20:37 А что мертвые воскреснут, и Моисей показал при купине, когда назвал Господа Богом Авраама и Богом Исаака и Богом Иакова.
\vs Luk 20:38 Бог же не есть \bibemph{Бог} мертвых, но живых, ибо у Него все живы.
\vs Luk 20:39 На это некоторые из книжников сказали: Учитель! Ты хорошо сказал.
\vs Luk 20:40 И уже не смели спрашивать Его ни о чем. Он же сказал им:
\vs Luk 20:41 к\acc{а}к говорят, что Христос есть Сын Давидов,
\vs Luk 20:42 а сам Давид говорит в книге псалмов: сказал Господь Господу моему: седи одесную Меня,
\vs Luk 20:43 доколе положу врагов Твоих в подножие ног Твоих?
\vs Luk 20:44 Итак, Давид Господом называет Его; как же Он Сын ему?
\vs Luk 20:45 И когда слушал весь народ, Он сказал ученикам Своим:
\vs Luk 20:46 остерегайтесь книжников, которые любят ходить в длинных одеждах и любят приветствия в народных собраниях, председания в синагогах и предвозлежания на пиршествах,
\vs Luk 20:47 которые поедают д\acc{о}мы вдов и лицемерно долго молятся; они примут тем большее осуждение.
\vs Luk 21:1 Взглянув же, Он увидел богатых, клавших дары свои в сокровищницу;
\vs Luk 21:2 увидел также и бедную вдову, положившую туда две лепты,
\vs Luk 21:3 и сказал: истинно говорю вам, что эта бедная вдова больше всех положила;
\vs Luk 21:4 ибо все те от избытка своего положили в дар Богу, а она от скудости своей положила все пропитание свое, какое имела.
\rsbpar\vs Luk 21:5 И когда некоторые говорили о храме, что он украшен дорогими камнями и вкладами, Он сказал:
\vs Luk 21:6 придут дни, в которые из того, что вы здесь видите, не останется камня на камне; все будет разрушено.
\vs Luk 21:7 И спросили Его: Учитель! когда же это будет? и какой признак, когда это должно произойти?
\vs Luk 21:8 Он сказал: берегитесь, чтобы вас не ввели в заблуждение, ибо многие придут под именем Моим, говоря, что это Я; и это время близко: не ходите вслед их.
\vs Luk 21:9 Когда же услышите о войнах и смятениях, не ужасайтесь, ибо этому надлежит быть прежде; но не тотчас конец.
\vs Luk 21:10 Тогда сказал им: восстанет народ на народ, и царство на царство;
\vs Luk 21:11 будут большие землетрясения по местам, и глады, и моры, и ужасные явления, и великие знамения с неба.
\vs Luk 21:12 Прежде же всего того возложат на вас руки и будут гнать \bibemph{вас}, предавая в синагоги и в темницы, и поведут пред царей и правителей за имя Мое;
\vs Luk 21:13 будет же это вам для свидетельства.
\vs Luk 21:14 Итак положите себе на сердце не обдумывать заранее, что отвечать,
\vs Luk 21:15 ибо Я дам вам уста и премудрость, которой не возмогут противоречить ни противостоять все, противящиеся вам.
\vs Luk 21:16 Преданы также будете и родителями, и братьями, и родственниками, и друзьями, и некоторых из вас умертвят;
\vs Luk 21:17 и будете ненавидимы всеми за имя Мое,
\vs Luk 21:18 но и волос с головы вашей не пропадет,~---
\vs Luk 21:19 терпением вашим спасайте души ваши.
\vs Luk 21:20 Когда же увидите Иерусалим, окруженный войсками, тогда знайте, что приблизилось запустение его:
\vs Luk 21:21 тогда находящиеся в Иудее да бегут в горы; и кто в городе, выходи из него; и кто в окрестностях, не входи в него,
\vs Luk 21:22 потому что это дни отмщения, да исполнится все написанное.
\vs Luk 21:23 Горе же беременным и питающим сосцами в те дни; ибо великое будет бедствие на земле и гнев на народ сей:
\vs Luk 21:24 и падут от острия меча, и отведутся в плен во все народы; и Иерусалим будет попираем язычниками, доколе не окончатся времена язычников.
\vs Luk 21:25 И будут знамения в солнце и луне и звездах, а на земле уныние народов и недоумение; и море восшумит и возмутится;
\vs Luk 21:26 люди будут издыхать от страха и ожидания \bibemph{бедствий}, грядущих на вселенную, ибо силы небесные поколеблются,
\vs Luk 21:27 и тогда увидят Сына Человеческого, грядущего на облаке с силою и славою великою.
\vs Luk 21:28 Когда же начнет это сбываться, тогда восклонитесь и поднимите головы ваши, потому что приближается избавление ваше.
\vs Luk 21:29 И сказал им притчу: посмотрите на смоковницу и на все деревья:
\vs Luk 21:30 когда они уже распускаются, то, видя это, знаете сами, что уже близко лето.
\vs Luk 21:31 Так, и когда вы увидите то сбывающимся, знайте, что близко Царствие Божие.
\vs Luk 21:32 Истинно говорю вам: не прейдет род сей, как все это будет;
\vs Luk 21:33 небо и земля прейдут, но слова Мои не прейдут.
\vs Luk 21:34 Смотрите же за собою, чтобы сердца ваши не отягчались объядением и пьянством и заботами житейскими, и чтобы день тот не постиг вас внезапно,
\vs Luk 21:35 ибо он, как сеть, найдет на всех живущих по всему лицу земному;
\vs Luk 21:36 итак бодрствуйте на всякое время и молитесь, да сподобитесь избежать всех сих будущих \bibemph{бедствий} и предстать пред Сына Человеческого.
\rsbpar\vs Luk 21:37 Днем Он учил в храме, а ночи, выходя, проводил на горе, называемой Елеонскою.
\vs Luk 21:38 И весь народ с утра приходил к Нему в храм слушать Его.
\vs Luk 22:1 Приближался праздник опресноков, называемый Пасхою,
\vs Luk 22:2 и искали первосвященники и книжники, как бы погубить Его, потому что боялись народа.
\vs Luk 22:3 Вошел же сатана в Иуду, прозванного Искариотом, одного из числа двенадцати,
\vs Luk 22:4 и он пошел, и говорил с первосвященниками и начальниками, как Его предать им.
\vs Luk 22:5 Они обрадовались и согласились дать ему денег;
\vs Luk 22:6 и он обещал, и искал удобного времени, чтобы предать Его им не при народе.
\rsbpar\vs Luk 22:7 Настал же день опресноков, в который надлежало заколать пасхального \bibemph{агнца},
\vs Luk 22:8 и послал \bibemph{Иисус} Петра и Иоанна, сказав: пойдите, приготовьте нам есть пасху.
\vs Luk 22:9 Они же сказали Ему: где велишь нам приготовить?
\vs Luk 22:10 Он сказал им: вот, при входе вашем в город, встретится с вами человек, несущий кувшин воды; последуйте за ним в дом, в который войдет он,
\vs Luk 22:11 и скажите хозяину дома: Учитель говорит тебе: где комната, в которой бы Мне есть пасху с учениками Моими?
\vs Luk 22:12 И он покажет вам горницу большую устланную; там приготовьте.
\vs Luk 22:13 Они пошли, и нашли, как сказал им, и приготовили пасху.
\rsbpar\vs Luk 22:14 И когда настал час, Он возлег, и двенадцать Апостолов с Ним,
\vs Luk 22:15 и сказал им: очень желал Я есть с вами сию пасху прежде Моего страдания,
\vs Luk 22:16 ибо сказываю вам, что уже не буду есть ее, пока она не совершится в Царствии Божием.
\vs Luk 22:17 И, взяв чашу и благодарив, сказал: приимите ее и разделите между собою,
\vs Luk 22:18 ибо сказываю вам, что не буду пить от плода виноградного, доколе не придет Царствие Божие.
\vs Luk 22:19 И, взяв хлеб и благодарив, преломил и подал им, говоря: сие есть тело Мое, которое за вас предается; сие творите в Мое воспоминание.
\vs Luk 22:20 Также и чашу после вечери, говоря: сия чаша \bibemph{есть} Новый Завет в Моей крови, которая за вас проливается.
\vs Luk 22:21 И вот, рука предающего Меня со Мною за столом;
\vs Luk 22:22 впрочем, Сын Человеческий идет по предназначению, но горе тому человеку, которым Он предается.
\vs Luk 22:23 И они начали спрашивать друг друга, кто бы из них был, который это сделает.
\vs Luk 22:24 Был же и спор между ними, кто из них должен почитаться б\acc{о}льшим.
\vs Luk 22:25 Он же сказал им: цари господствуют над народами, и владеющие ими благодетелями называются,
\vs Luk 22:26 а вы не так: но кто из вас больше, будь как меньший, и начальствующий~--- как служащий.
\vs Luk 22:27 Ибо кто больше: возлежащий, или служащий? не возлежащий ли? А Я посреди вас, как служащий.
\vs Luk 22:28 Но вы пребыли со Мною в напастях Моих,
\vs Luk 22:29 и Я завещаваю вам, как завещал Мне Отец Мой, Царство,
\vs Luk 22:30 да ядите и пиете за трапезою Моею в Царстве Моем, и сядете на престолах судить двенадцать колен Израилевых.
\vs Luk 22:31 И сказал Господь: Симон! Симон! се, сатана просил, чтобы сеять вас как пшеницу,
\vs Luk 22:32 но Я молился о тебе, чтобы не оскудела вера твоя; и ты некогда, обратившись, утверди братьев твоих.
\vs Luk 22:33 Он отвечал Ему: Господи! с Тобою я готов и в темницу и на смерть идти.
\vs Luk 22:34 Но Он сказал: говорю тебе, Петр, не пропоет петух сегодня, как ты трижды отречешься, что не знаешь Меня.
\vs Luk 22:35 И сказал им: когда Я посылал вас без мешка и без сум\acc{ы} и без обуви, имели ли вы в чем недостаток? Они отвечали: ни в чем.
\vs Luk 22:36 Тогда Он сказал им: но теперь, кто имеет мешок, тот возьми его, также и сум\acc{у}; а у кого нет, продай одежду свою и купи меч;
\vs Luk 22:37 ибо сказываю вам, что должно исполниться на Мне и сему написанному: и к злодеям причтен. Ибо то, что о Мне, приходит к концу.
\vs Luk 22:38 Они сказали: Господи! вот, здесь два меча. Он сказал им: довольно.
\rsbpar\vs Luk 22:39 И, выйдя, пошел по обыкновению на гору Елеонскую, за Ним последовали и ученики Его.
\vs Luk 22:40 Придя же на место, сказал им: молитесь, чтобы не впасть в искушение.
\vs Luk 22:41 И Сам отошел от них на вержение камня, и, преклонив колени, молился,
\vs Luk 22:42 говоря: Отче! о, если бы Ты благоволил пронести чашу сию мимо Меня! впрочем не Моя воля, но Твоя да будет.
\vs Luk 22:43 Явился же Ему Ангел с небес и укреплял Его.
\vs Luk 22:44 И, находясь в борении, прилежнее молился, и был пот Его, как капли крови, падающие на землю.
\vs Luk 22:45 Встав от молитвы, Он пришел к ученикам, и нашел их спящими от печали
\vs Luk 22:46 и сказал им: что вы спите? встаньте и молитесь, чтобы не впасть в искушение.
\rsbpar\vs Luk 22:47 Когда Он еще говорил это, появился народ, а впереди его шел один из двенадцати, называемый Иуда, и он подошел к Иисусу, чтобы поцеловать Его. Ибо он такой им дал знак: Кого я поцелую, Тот и есть.
\vs Luk 22:48 Иисус же сказал ему: Иуда! целованием ли предаешь Сына Человеческого?
\vs Luk 22:49 Бывшие же с Ним, видя, к чему идет дело, сказали Ему: Господи! не ударить ли нам мечом?
\vs Luk 22:50 И один из них ударил раба первосвященникова, и отсек ему правое ухо.
\vs Luk 22:51 Тогда Иисус сказал: оставьте, довольно. И, коснувшись уха его, исцелил его.
\vs Luk 22:52 Первосвященникам же и начальникам храма и старейшинам, собравшимся против Него, сказал Иисус: как будто на разбойника вышли вы с мечами и кольями, чтобы взять Меня?
\vs Luk 22:53 Каждый день бывал Я с вами в храме, и вы не поднимали на Меня рук, но теперь ваше время и власть тьмы.
\rsbpar\vs Luk 22:54 Взяв Его, повели и привели в дом первосвященника. Петр же следовал издали.
\vs Luk 22:55 Когда они развели огонь среди двора и сели вместе, сел и Петр между ними.
\vs Luk 22:56 Одна служанка, увидев его сидящего у огня и всмотревшись в него, сказала: и этот был с Ним.
\vs Luk 22:57 Но он отрекся от Него, сказав женщине: я не знаю Его.
\vs Luk 22:58 Вскоре потом другой, увидев его, сказал: и ты из них. Но Петр сказал этому человеку: нет!
\vs Luk 22:59 Прошло с час времени, еще некто настоятельно говорил: точно и этот был с Ним, ибо он Галилеянин.
\vs Luk 22:60 Но Петр сказал тому человеку: не знаю, что ты говоришь. И тотчас, когда еще говорил он, запел петух.
\vs Luk 22:61 Тогда Господь, обратившись, взглянул на Петра, и Петр вспомнил слово Господа, как Он сказал ему: прежде нежели пропоет петух, отречешься от Меня трижды.
\vs Luk 22:62 И, выйдя вон, горько заплакал.
\rsbpar\vs Luk 22:63 Люди, державшие Иисуса, ругались над Ним и били Его;
\vs Luk 22:64 и, закрыв Его, ударяли Его по лицу и спрашивали Его: прореки, кто ударил Тебя?
\vs Luk 22:65 И много иных хулений произносили против Него.
\rsbpar\vs Luk 22:66 И как настал день, собрались старейшины народа, первосвященники и книжники, и ввели Его в свой синедрион
\vs Luk 22:67 и сказали: Ты ли Христос? скажи нам. Он сказал им: если скажу вам, вы не поверите;
\vs Luk 22:68 если же и спрошу вас, не будете отвечать Мне и не отпустите \bibemph{Меня};
\vs Luk 22:69 отныне Сын Человеческий воссядет одесную силы Божией.
\vs Luk 22:70 И сказали все: итак, Ты Сын Божий? Он отвечал им: вы говорите, что Я.
\vs Luk 22:71 Они же сказали: какое еще нужно нам свидетельство? ибо мы сами слышали из уст Его.
\vs Luk 23:1 И поднялось все множество их, и повели Его к Пилату,
\vs Luk 23:2 и начали обвинять Его, говоря: мы нашли, что Он развращает народ наш и запрещает давать подать кесарю, называя Себя Христом Царем.
\vs Luk 23:3 Пилат спросил Его: Ты Царь Иудейский? Он сказал ему в ответ: ты говоришь.
\vs Luk 23:4 Пилат сказал первосвященникам и народу: я не нахожу никакой вины в этом человеке.
\vs Luk 23:5 Но они настаивали, говоря, что Он возмущает народ, уча по всей Иудее, начиная от Галилеи до сего места.
\vs Luk 23:6 Пилат, услышав о Галилее, спросил: разве Он Галилеянин?
\vs Luk 23:7 И, узнав, что Он из области Иродовой, послал Его к Ироду, который в эти дни был также в Иерусалиме.
\vs Luk 23:8 Ирод, увидев Иисуса, очень обрадовался, ибо давно желал видеть Его, потому что много слышал о Нем, и надеялся увидеть от Него какое-нибудь чудо,
\vs Luk 23:9 и предлагал Ему многие вопросы, но Он ничего не отвечал ему.
\vs Luk 23:10 Первосвященники же и книжники стояли и усильно обвиняли Его.
\vs Luk 23:11 Но Ирод со своими воинами, уничижив Его и насмеявшись над Ним, одел Его в светлую одежду и отослал обратно к Пилату.
\vs Luk 23:12 И сделались в тот день Пилат и Ирод друзьями между собою, ибо прежде были во вражде друг с другом.
\vs Luk 23:13 Пилат же, созвав первосвященников и начальников и народ,
\vs Luk 23:14 сказал им: вы привели ко мне человека сего, как развращающего народ; и вот, я при вас исследовал и не нашел человека сего виновным ни в чем том, в чем вы обвиняете Его;
\vs Luk 23:15 и Ирод также, ибо я посылал Его к нему; и ничего не найдено в Нем достойного смерти;
\vs Luk 23:16 итак, наказав Его, отпущу.
\vs Luk 23:17 А ему и нужно было для праздника отпустить им одного \bibemph{узника}.
\vs Luk 23:18 Но весь народ стал кричать: смерть Ему! а отпусти нам Варавву.
\vs Luk 23:19 Варавва был посажен в темницу за произведенное в городе возмущение и убийство.
\vs Luk 23:20 Пилат снова возвысил голос, желая отпустить Иисуса.
\vs Luk 23:21 Но они кричали: распни, распни Его!
\vs Luk 23:22 Он в третий раз сказал им: какое же зло сделал Он? я ничего достойного смерти не нашел в Нем; итак, наказав Его, отпущу.
\vs Luk 23:23 Но они продолжали с великим криком требовать, чтобы Он был распят; и превозмог крик их и первосвященников.
\vs Luk 23:24 И Пилат решил быть по прошению их,
\vs Luk 23:25 и отпустил им посаженного за возмущение и убийство в темницу, которого они просили; а Иисуса предал в их волю.
\rsbpar\vs Luk 23:26 И когда повели Его, то, захватив некоего Симона Киринеянина, шедшего с поля, возложили на него крест, чтобы нес за Иисусом.
\vs Luk 23:27 И шло за Ним великое множество народа и женщин, которые плакали и рыдали о Нем.
\vs Luk 23:28 Иисус же, обратившись к ним, сказал: дщери Иерусалимские! не плачьте обо Мне, но плачьте о себе и о детях ваших,
\vs Luk 23:29 ибо приходят дни, в которые скажут: блаженны неплодные, и утробы неродившие, и сосцы непитавшие!
\vs Luk 23:30 тогда начнут говорить горам: падите на нас! и холмам: покройте нас!
\vs Luk 23:31 Ибо если с зеленеющим деревом это делают, то с сухим что будет?
\rsbpar\vs Luk 23:32 Вели с Ним на смерть и двух злодеев.
\vs Luk 23:33 И когда пришли на место, называемое Лобное, там распяли Его и злодеев, одного по правую, а другого по левую сторону.
\vs Luk 23:34 Иисус же говорил: Отче! прости им, ибо не знают, что делают. И делили одежды Его, бросая жребий.
\vs Luk 23:35 И стоял народ и смотрел. Насмехались же вместе с ними и начальники, говоря: других спасал; пусть спасет Себя Самого, если Он Христос, избранный Божий.
\vs Luk 23:36 Также и воины ругались над Ним, подходя и поднося Ему уксус
\vs Luk 23:37 и говоря: если Ты Царь Иудейский, спаси Себя Самого.
\vs Luk 23:38 И была над Ним надпись, написанная словами греческими, римскими и еврейскими: Сей есть Царь Иудейский.
\vs Luk 23:39 Один из повешенных злодеев злословил Его и говорил: если Ты Христос, спаси Себя и нас.
\vs Luk 23:40 Другой же, напротив, унимал его и говорил: или ты не боишься Бога, когда и сам осужден на то же?
\vs Luk 23:41 и мы \bibemph{осуждены} справедливо, потому что достойное по делам нашим приняли, а Он ничего худого не сделал.
\vs Luk 23:42 И сказал Иисусу: помяни меня, Господи, когда приидешь в Царствие Твое!
\vs Luk 23:43 И сказал ему Иисус: истинно говорю тебе, ныне же будешь со Мною в раю.
\rsbpar\vs Luk 23:44 Было же около шестого часа дня, и сделалась тьма по всей земле до часа девятого:
\vs Luk 23:45 и померкло солнце, и завеса в храме раздралась по средине.
\vs Luk 23:46 Иисус, возгласив громким голосом, сказал: Отче! в руки Твои предаю дух Мой. И, сие сказав, испустил дух.
\vs Luk 23:47 Сотник же, видев происходившее, прославил Бога и сказал: истинно человек этот был праведник.
\vs Luk 23:48 И весь народ, сшедшийся на сие зрелище, видя происходившее, возвращался, бия себя в грудь.
\vs Luk 23:49 Все же, знавшие Его, и женщины, следовавшие за Ним из Галилеи, стояли вдали и смотрели на это.
\rsbpar\vs Luk 23:50 Тогда некто, именем Иосиф, член совета, человек добрый и правдивый,
\vs Luk 23:51 не участвовавший в совете и в деле их; из Аримафеи, города Иудейского, ожидавший также Царствия Божия,
\vs Luk 23:52 пришел к Пилату и просил тела Иисусова;
\vs Luk 23:53 и, сняв его, обвил плащаницею и положил его в гробе, высеченном \bibemph{в скале}, где еще никто не был положен.
\vs Luk 23:54 День тот был пятница, и наступала суббота.
\vs Luk 23:55 Последовали также и женщины, пришедшие с Иисусом из Галилеи, и смотрели гроб, и как полагалось тело Его;
\vs Luk 23:56 возвратившись же, приготовили благовония и масти; и в субботу остались в покое по заповеди.
\vs Luk 24:1 В первый же день недели, очень рано, неся приготовленные ароматы, пришли они ко гробу, и вместе с ними некоторые другие;
\vs Luk 24:2 но нашли камень отваленным от гроба.
\vs Luk 24:3 И, войдя, не нашли тела Господа Иисуса.
\vs Luk 24:4 Когда же недоумевали они о сем, вдруг предстали перед ними два мужа в одеждах блистающих.
\vs Luk 24:5 И когда они были в страхе и наклонили лица \bibemph{свои} к земле, сказали им: что вы ищете живого между мертвыми?
\vs Luk 24:6 Его нет здесь: Он воскрес; вспомните, как Он говорил вам, когда был еще в Галилее,
\vs Luk 24:7 сказывая, что Сыну Человеческому надлежит быть предану в руки человеков грешников, и быть распяту, и в третий день воскреснуть.
\vs Luk 24:8 И вспомнили они слова Его;
\vs Luk 24:9 и, возвратившись от гроба, возвестили всё это одиннадцати и всем прочим.
\vs Luk 24:10 То были Магдалина Мария, и Иоанна, и Мария, \bibemph{мать} Иакова, и другие с ними, которые сказали о сем Апостолам.
\vs Luk 24:11 И показались им слова их пустыми, и не поверили им.
\vs Luk 24:12 Но Петр, встав, побежал ко гробу и, наклонившись, увидел только пелены лежащие, и пошел назад, дивясь сам в себе происшедшему.
\rsbpar\vs Luk 24:13 В тот же день двое из них шли в селение, отстоящее стадий на шестьдесят от Иерусалима, называемое Эммаус;
\vs Luk 24:14 и разговаривали между собою о всех сих событиях.
\vs Luk 24:15 И когда они разговаривали и рассуждали между собою, и Сам Иисус, приблизившись, пошел с ними.
\vs Luk 24:16 Но глаза их были удержаны, так что они не узнали Его.
\vs Luk 24:17 Он же сказал им: о чем это вы, идя, рассуждаете между собою, и отчего вы печальны?
\vs Luk 24:18 Один из них, именем Клеопа, сказал Ему в ответ: неужели Ты один из пришедших в Иерусалим не знаешь о происшедшем в нем в эти дни?
\vs Luk 24:19 И сказал им: о чем? Они сказали Ему: что было с Иисусом Назарянином, Который был пророк, сильный в деле и слове пред Богом и всем народом;
\vs Luk 24:20 как предали Его первосвященники и начальники наши для осуждения на смерть и распяли Его.
\vs Luk 24:21 А мы надеялись было, что Он есть Тот, Который должен избавить Израиля; но со всем тем, уже третий день ныне, как это произошло.
\vs Luk 24:22 Но и некоторые женщины из наших изумили нас: они были рано у гроба
\vs Luk 24:23 и не нашли тела Его и, придя, сказывали, что они видели и явление Ангелов, которые говорят, что Он жив.
\vs Luk 24:24 И пошли некоторые из наших ко гробу и нашли так, как и женщины говорили, но Его не видели.
\vs Luk 24:25 Тогда Он сказал им: о, несмысленные и медлительные сердцем, чтобы веровать всему, что предсказывали пророки!
\vs Luk 24:26 Не так ли надлежало пострадать Христу и войти в славу Свою?
\vs Luk 24:27 И, начав от Моисея, из всех пророков изъяснял им сказанное о Нем во всем Писании.
\vs Luk 24:28 И приблизились они к тому селению, в которое шли; и Он показывал им вид, что хочет идти далее.
\vs Luk 24:29 Но они удерживали Его, говоря: останься с нами, потому что день уже склонился к вечеру. И Он вошел и остался с ними.
\vs Luk 24:30 И когда Он возлежал с ними, то, взяв хлеб, благословил, преломил и подал им.
\vs Luk 24:31 Тогда открылись у них глаза, и они узнали Его. Но Он стал невидим для них.
\vs Luk 24:32 И они сказали друг другу: не горело ли в нас сердце наше, когда Он говорил нам на дороге и когда изъяснял нам Писание?
\vs Luk 24:33 И, встав в тот же час, возвратились в Иерусалим и нашли вместе одиннадцать \bibemph{Апостолов} и бывших с ними,
\vs Luk 24:34 которые говорили, что Господь истинно воскрес и явился Симону.
\vs Luk 24:35 И они рассказывали о происшедшем на пути, и как Он был узнан ими в преломлении хлеба.
\rsbpar\vs Luk 24:36 Когда они говорили о сем, Сам Иисус стал посреди них и сказал им: мир вам.
\vs Luk 24:37 Они, смутившись и испугавшись, подумали, что видят духа.
\vs Luk 24:38 Но Он сказал им: что смущаетесь, и для чего такие мысли входят в сердца ваши?
\vs Luk 24:39 Посмотрите на руки Мои и на ноги Мои; это Я Сам; осяжите Меня и рассмотр\acc{и}те; ибо дух плоти и костей не имеет, как видите у Меня.
\vs Luk 24:40 И, сказав это, показал им руки и ноги.
\vs Luk 24:41 Когда же они от радости еще не верили и дивились, Он сказал им: есть ли у вас здесь какая пища?
\vs Luk 24:42 Они подали Ему часть печеной рыбы и сотового меда.
\vs Luk 24:43 И, взяв, ел пред ними.
\vs Luk 24:44 И сказал им: вот то, о чем Я вам говорил, еще быв с вами, что надлежит исполниться всему, написанному о Мне в законе Моисеевом и в пророках и псалмах.
\vs Luk 24:45 Тогда отверз им ум к уразумению Писаний.
\vs Luk 24:46 И сказал им: так написано, и так надлежало пострадать Христу, и воскреснуть из мертвых в третий день,
\vs Luk 24:47 и проповедану быть во имя Его покаянию и прощению грехов во всех народах, начиная с Иерусалима.
\vs Luk 24:48 Вы же свидетели сему.
\vs Luk 24:49 И Я пошлю обетование Отца Моего на вас; вы же оставайтесь в городе Иерусалиме, доколе не облечетесь силою свыше.
\rsbpar\vs Luk 24:50 И вывел их вон \bibemph{из города} до Вифании и, подняв руки Свои, благословил их.
\vs Luk 24:51 И, когда благословлял их, стал отдаляться от них и возноситься на небо.
\vs Luk 24:52 Они поклонились Ему и возвратились в Иерусалим с великою радостью.
\vs Luk 24:53 И пребывали всегда в храме, прославляя и благословляя Бога. Аминь.

\bibbookdescr{Joh}{
  inline={От Иоанна\\\LARGE святое благовествование},
  toc={От Иоанна},
  bookmark={От Иоанна},
  header={От Иоанна},
  %headerleft={},
  %headerright={},
  abbr={Ин}
}
\vs Joh 1:1 В начале было Слово, и Слово было у Бога, и Слово было Бог.
\vs Joh 1:2 Оно было в начале у Бога.
\vs Joh 1:3 Все чрез Него н\acc{а}чало быть, и без Него ничто не н\acc{а}чало быть, что н\acc{а}чало быть.
\vs Joh 1:4 В Нем была жизнь, и жизнь была свет человеков.
\vs Joh 1:5 И свет во тьме светит, и тьма не объяла его.
\rsbpar\vs Joh 1:6 Был человек, посланный от Бога; имя ему Иоанн.
\vs Joh 1:7 Он пришел для свидетельства, чтобы свидетельствовать о Свете, дабы все уверовали чрез него.
\vs Joh 1:8 Он не был свет, но \bibemph{был послан}, чтобы свидетельствовать о Свете.
\rsbpar\vs Joh 1:9 Был Свет истинный, Который просвещает всякого человека, приходящего в мир.
\vs Joh 1:10 В мире был, и мир чрез Него н\acc{а}чал быть, и мир Его не познал.
\vs Joh 1:11 Пришел к своим, и свои Его не приняли.
\vs Joh 1:12 А тем, которые приняли Его, верующим во имя Его, дал власть быть чадами Божиими,
\vs Joh 1:13 которые ни от крови, ни от хотения плоти, ни от хотения мужа, но от Бога родились.
\rsbpar\vs Joh 1:14 И Слово стало плотию, и обитало с нами, полное благодати и истины; и мы видели славу Его, славу, как Единородного от Отца.
\vs Joh 1:15 Иоанн свидетельствует о Нем и, восклицая, говорит: Сей был Тот, о Котором я сказал, что Идущий за мною стал впереди меня, потому что был прежде меня.
\vs Joh 1:16 И от полноты Его все мы приняли и благодать на благодать,
\vs Joh 1:17 ибо закон дан чрез Моисея; благодать же и истина произошли чрез Иисуса Христа.
\vs Joh 1:18 Бога не видел никто никогда; Единородный Сын, сущий в недре Отчем, Он явил.
\rsbpar\vs Joh 1:19 И вот свидетельство Иоанна, когда Иудеи прислали из Иерусалима священников и левитов спросить его: кто ты?
\vs Joh 1:20 Он объявил, и не отрекся, и объявил, что я не Христос.
\vs Joh 1:21 И спросили его: что же? ты Илия? Он сказал: нет. Пророк? Он отвечал: нет.
\vs Joh 1:22 Сказали ему: кто же ты? чтобы нам дать ответ пославшим нас: что ты скажешь о себе самом?
\vs Joh 1:23 Он сказал: я глас вопиющего в пустыне: исправьте путь Господу, как сказал пророк Исаия.
\vs Joh 1:24 А посланные были из фарисеев;
\vs Joh 1:25 И они спросили его: что же ты крестишь, если ты ни Христос, ни Илия, ни пророк?
\vs Joh 1:26 Иоанн сказал им в ответ: я крещу в воде; но стоит среди вас \bibemph{Некто}, Которого вы не знаете.
\vs Joh 1:27 Он-то Идущий за мною, но Который стал впереди меня. Я недостоин развязать ремень у обуви Его.
\vs Joh 1:28 Это происходило в Вифаваре при Иордане, где крестил Иоанн.
\rsbpar\vs Joh 1:29 На другой день видит Иоанн идущего к нему Иисуса и говорит: вот Агнец Божий, Который берет \bibemph{на Себя} грех мира.
\vs Joh 1:30 Сей есть, о Котором я сказал: за мною идет Муж, Который стал впереди меня, потому что Он был прежде меня.
\vs Joh 1:31 Я не знал Его; но для того пришел крестить в воде, чтобы Он явлен был Израилю.
\vs Joh 1:32 И свидетельствовал Иоанн, говоря: я видел Духа, сходящего с неба, как голубя, и пребывающего на Нем.
\vs Joh 1:33 Я не знал Его; но Пославший меня крестить в воде сказал мне: на Кого увидишь Духа сходящего и пребывающего на Нем, Тот есть крестящий Духом Святым.
\vs Joh 1:34 И я видел и засвидетельствовал, что Сей есть Сын Божий.
\rsbpar\vs Joh 1:35 На другой день опять стоял Иоанн и двое из учеников его.
\vs Joh 1:36 И, увидев идущего Иисуса, сказал: вот Агнец Божий.
\vs Joh 1:37 Услышав от него сии слова, оба ученика пошли за Иисусом.
\vs Joh 1:38 Иисус же, обратившись и увидев их идущих, говорит им: что вам надобно? Они сказали Ему: Равв\acc{и},~--- что значит: учитель,~--- где живешь?
\vs Joh 1:39 Говорит им: пойдите и увидите. Они пошли и увидели, где Он живет; и пробыли у Него день тот. Было около десятого часа.
\vs Joh 1:40 Один из двух, слышавших от Иоанна \bibemph{об Иисусе} и последовавших за Ним, был Андрей, брат Симона Петра.
\vs Joh 1:41 Он первый находит брата своего Симона и говорит ему: мы нашли Мессию, что значит: Христос;
\vs Joh 1:42 и привел его к Иисусу. Иисус же, взглянув на него, сказал: ты~--- Симон, сын Ионин; ты наречешься Кифа, что значит: камень (Петр).
\rsbpar\vs Joh 1:43 На другой день \bibemph{Иисус} восхотел идти в Галилею, и находит Филиппа и говорит ему: иди за Мною.
\vs Joh 1:44 Филипп же был из Вифсаиды, из \bibemph{одного} города с Андреем и Петром.
\vs Joh 1:45 Филипп находит Нафанаила и говорит ему: мы нашли Того, о Котором писали Моисей в законе и пророки, Иисуса, сына Иосифова, из Назарета.
\vs Joh 1:46 Но Нафанаил сказал ему: из Назарета может ли быть что доброе? Филипп говорит ему: пойди и посмотри.
\vs Joh 1:47 Иисус, увидев идущего к Нему Нафанаила, говорит о нем: вот подлинно Израильтянин, в котором нет лукавства.
\vs Joh 1:48 Нафанаил говорит Ему: почему Ты знаешь меня? Иисус сказал ему в ответ: прежде нежели позвал тебя Филипп, когда ты был под смоковницею, Я видел тебя.
\vs Joh 1:49 Нафанаил отвечал Ему: Равв\acc{и}! Ты Сын Божий, Ты Царь Израилев.
\vs Joh 1:50 Иисус сказал ему в ответ: ты веришь, потому что Я тебе сказал: Я видел тебя под смоковницею; увидишь больше сего.
\vs Joh 1:51 И говорит ему: истинно, истинно говорю вам: отныне будете видеть небо отверстым и Ангелов Божиих восходящих и нисходящих к Сыну Человеческому.
\vs Joh 2:1 На третий день был брак в Кане Галилейской, и Матерь Иисуса была там.
\vs Joh 2:2 Был также зван Иисус и ученики Его на брак.
\vs Joh 2:3 И как недоставало вина, то Матерь Иисуса говорит Ему: вина нет у них.
\vs Joh 2:4 Иисус говорит Ей: что Мне и Тебе, Ж\acc{е}но? еще не пришел час Мой.
\vs Joh 2:5 Матерь Его сказала служителям: что скажет Он вам, то сделайте.
\vs Joh 2:6 Было же тут шесть каменных водоносов, стоявших \bibemph{по обычаю} очищения Иудейского, вмещавших по две или по три меры.
\vs Joh 2:7 Иисус говорит им: наполните сосуды водою. И наполнили их до верха.
\vs Joh 2:8 И говорит им: теперь почерпните и несите к распорядителю пира. И понесли.
\vs Joh 2:9 Когда же распорядитель отведал воды, сделавшейся вином,~--- а он не знал, откуда \bibemph{это вино}, знали только служители, почерпавшие воду,~--- тогда распорядитель зовет жениха
\vs Joh 2:10 и говорит ему: всякий человек подает сперва хорошее вино, а когда напьются, тогда худшее; а ты хорошее вино сберег доселе.
\vs Joh 2:11 Так положил Иисус начало чудесам в Кане Галилейской и явил славу Свою; и уверовали в Него ученики Его.
\vs Joh 2:12 После сего пришел Он в Капернаум, Сам и Матерь Его, и братья Его, и ученики Его; и там пробыли немного дней.
\vs Joh 2:13 Приближалась Пасха Иудейская, и Иисус пришел в Иерусалим
\vs Joh 2:14 и нашел, что в храме продавали волов, овец и голубей, и сидели меновщики денег.
\vs Joh 2:15 И, сделав бич из веревок, выгнал из храма всех, \bibemph{также} и овец и волов; и деньги у меновщиков рассыпал, а столы их опрокинул.
\vs Joh 2:16 И сказал продающим голубей: возьмите это отсюда и д\acc{о}ма Отца Моего не делайте домом торговли.
\vs Joh 2:17 При сем ученики Его вспомнили, что написано: ревность по доме Твоем снедает Меня.
\vs Joh 2:18 На это Иудеи сказали: каким знамением докажешь Ты нам, что \bibemph{имеешь власть} так поступать?
\vs Joh 2:19 Иисус сказал им в ответ: разрушьте храм сей, и Я в три дня воздвигну его.
\vs Joh 2:20 На это сказали Иудеи: сей храм строился сорок шесть лет, и Ты в три дня воздвигнешь его?
\vs Joh 2:21 А Он говорил о храме тела Своего.
\vs Joh 2:22 Когда же воскрес Он из мертвых, то ученики Его вспомнили, что Он говорил это, и поверили Писанию и слову, которое сказал Иисус.
\vs Joh 2:23 И когда Он был в Иерусалиме на празднике Пасхи, то многие, видя чудеса, которые Он творил, уверовали во имя Его.
\vs Joh 2:24 Но Сам Иисус не вверял Себя им, потому что знал всех
\vs Joh 2:25 и не имел нужды, чтобы кто засвидетельствовал о человеке, ибо Сам знал, что в человеке.
\vs Joh 3:1 Между фарисеями был некто, именем Никодим, \bibemph{один} из начальников Иудейских.
\vs Joh 3:2 Он пришел к Иисусу ночью и сказал Ему: Равв\acc{и}! мы знаем, что Ты учитель, пришедший от Бога; ибо таких чудес, какие Ты творишь, никто не может творить, если не будет с ним Бог.
\vs Joh 3:3 Иисус сказал ему в ответ: истинно, истинно говорю тебе, если кто не родится свыше, не может увидеть Царствия Божия.
\vs Joh 3:4 Никодим говорит Ему: как может человек родиться, будучи стар? неужели может он в другой раз войти в утробу матери своей и родиться?
\vs Joh 3:5 Иисус отвечал: истинно, истинно говорю тебе, если кто не родится от воды и Духа, не может войти в Царствие Божие.
\vs Joh 3:6 Рожденное от плоти есть плоть, а рожденное от Духа есть дух.
\vs Joh 3:7 Не удивляйся тому, что Я сказал тебе: должно вам родиться свыше.
\vs Joh 3:8 Дух дышит, где хочет, и голос его слышишь, а не знаешь, откуда приходит и куда уходит: так бывает со всяким, рожденным от Духа.
\vs Joh 3:9 Никодим сказал Ему в ответ: как это может быть?
\vs Joh 3:10 Иисус отвечал и сказал ему: ты~--- учитель Израилев, и этого ли не знаешь?
\vs Joh 3:11 Истинно, истинно говорю тебе: Мы говорим о том, что знаем, и свидетельствуем о том, что видели, а вы свидетельства Нашего не принимаете.
\vs Joh 3:12 Если Я сказал вам о земном, и вы не верите,~--- как поверите, если буду говорить вам о небесном?
\vs Joh 3:13 Никто не восходил на небо, как только сшедший с небес Сын Человеческий, сущий на небесах.
\vs Joh 3:14 И как Моисей вознес змию в пустыне, так должно вознесену быть Сыну Человеческому,
\vs Joh 3:15 дабы всякий, верующий в Него, не погиб, но имел жизнь вечную.
\vs Joh 3:16 Ибо так возлюбил Бог мир, что отдал Сына Своего Единородного, дабы всякий верующий в Него, не погиб, но имел жизнь вечную.
\vs Joh 3:17 Ибо не послал Бог Сына Своего в мир, чтобы судить мир, но чтобы мир спасен был чрез Него.
\vs Joh 3:18 Верующий в Него не судится, а неверующий уже осужден, потому что не уверовал во имя Единородного Сына Божия.
\vs Joh 3:19 Суд же состоит в том, что свет пришел в мир; но люди более возлюбили тьму, нежели свет, потому что дела их были злы;
\vs Joh 3:20 ибо всякий, делающий злое, ненавидит свет и не идет к свету, чтобы не обличились дела его, потому что они злы,
\vs Joh 3:21 а поступающий по правде идет к свету, дабы явны были дела его, потому что они в Боге соделаны.
\rsbpar\vs Joh 3:22 После сего пришел Иисус с учениками Своими в землю Иудейскую и там жил с ними и крестил.
\vs Joh 3:23 А Иоанн также крестил в Еноне, близ Салима, потому что там было много воды; и приходили \bibemph{туда} и крестились,
\vs Joh 3:24 ибо Иоанн еще не был заключен в темницу.
\vs Joh 3:25 Тогда у Иоанновых учеников произошел спор с Иудеями об очищении.
\vs Joh 3:26 И пришли к Иоанну и сказали ему: равв\acc{и}! Тот, Который был с тобою при Иордане и о Котором ты свидетельствовал, вот Он крестит, и все идут к Нему.
\vs Joh 3:27 Иоанн сказал в ответ: не может человек ничего принимать \bibemph{на себя}, если не будет дано ему с неба.
\vs Joh 3:28 Вы сами мне свидетели в том, что я сказал: не я Христос, но я послан пред Ним.
\vs Joh 3:29 Имеющий невесту есть жених, а друг жениха, стоящий и внимающий ему, радостью радуется, слыша голос жениха. Сия-то радость моя исполнилась.
\vs Joh 3:30 Ему должно расти, а мне умаляться.
\vs Joh 3:31 Приходящий свыше и есть выше всех; а сущий от земли земной и есть и говорит, как сущий от земли; Приходящий с небес есть выше всех,
\vs Joh 3:32 и что Он видел и слышал, о том и свидетельствует; и никто не принимает свидетельства Его.
\vs Joh 3:33 Принявший Его свидетельство сим запечатлел, что Бог истинен,
\vs Joh 3:34 ибо Тот, Которого послал Бог, говорит слова Божии; ибо не мерою дает Бог Духа.
\vs Joh 3:35 Отец любит Сына и все дал в руку Его.
\vs Joh 3:36 Верующий в Сына имеет жизнь вечную, а не верующий в Сына не увидит жизни, но гнев Божий пребывает на нем.
\vs Joh 4:1 Когда же узнал Иисус о \bibemph{дошедшем до} фарисеев слухе, что Он более приобретает учеников и крестит, нежели Иоанн,~---
\vs Joh 4:2 хотя Сам Иисус не крестил, а ученики Его,~---
\vs Joh 4:3 то оставил Иудею и пошел опять в Галилею.
\rsbpar\vs Joh 4:4 Надлежало же Ему проходить через Самарию.
\vs Joh 4:5 Итак приходит Он в город Самарийский, называемый Сихарь, близ участка земли, данного Иаковом сыну своему Иосифу.
\vs Joh 4:6 Там был колодезь Иаковлев. Иисус, утрудившись от пути, сел у колодезя. Было около шестого часа.
\vs Joh 4:7 Приходит женщина из Самарии почерпнуть воды. Иисус говорит ей: дай Мне пить.
\vs Joh 4:8 Ибо ученики Его отлучились в город купить пищи.
\vs Joh 4:9 Женщина Самарянская говорит Ему: как ты, будучи Иудей, просишь пить у меня, Самарянки? ибо Иудеи с Самарянами не сообщаются.
\vs Joh 4:10 Иисус сказал ей в ответ: если бы ты знала дар Божий и Кто говорит тебе: дай Мне пить, то ты сама просила бы у Него, и Он дал бы тебе воду живую.
\vs Joh 4:11 Женщина говорит Ему: господин! тебе и почерпнуть нечем, а колодезь глубок; откуда же у тебя вода живая?
\vs Joh 4:12 Неужели ты больше отца нашего Иакова, который дал нам этот колодезь и сам из него пил, и дети его, и скот его?
\vs Joh 4:13 Иисус сказал ей в ответ: всякий, пьющий воду сию, возжаждет опять,
\vs Joh 4:14 а кто будет пить воду, которую Я дам ему, тот не будет жаждать вовек; но вода, которую Я дам ему, сделается в нем источником воды, текущей в жизнь вечную.
\vs Joh 4:15 Женщина говорит Ему: господин! дай мне этой воды, чтобы мне не иметь жажды и не приходить сюда черпать.
\vs Joh 4:16 Иисус говорит ей: пойди, позови мужа твоего и приди сюда.
\vs Joh 4:17 Женщина сказала в ответ: у меня нет мужа. Иисус говорит ей: правду ты сказала, что у тебя нет мужа,
\vs Joh 4:18 ибо у тебя было пять мужей, и тот, которого ныне имеешь, не муж тебе; это справедливо ты сказала.
\vs Joh 4:19 Женщина говорит Ему: Господи! вижу, что Ты пророк.
\vs Joh 4:20 Отцы наши поклонялись на этой горе, а вы говорите, что место, где должно поклоняться, находится в Иерусалиме.
\vs Joh 4:21 Иисус говорит ей: поверь Мне, что наступает время, когда и не на горе сей, и не в Иерусалиме будете поклоняться Отцу.
\vs Joh 4:22 Вы не знаете, чему кланяетесь, а мы знаем, чему кланяемся, ибо спасение от Иудеев.
\vs Joh 4:23 Но настанет время и настало уже, когда истинные поклонники будут поклоняться Отцу в духе и истине, ибо таких поклонников Отец ищет Себе.
\vs Joh 4:24 Бог есть дух, и поклоняющиеся Ему должны поклоняться в духе и истине.
\vs Joh 4:25 Женщина говорит Ему: знаю, что придет Мессия, то есть Христос; когда Он придет, то возвестит нам все.
\vs Joh 4:26 Иисус говорит ей: это Я, Который говорю с тобою.
\vs Joh 4:27 В это время пришли ученики Его, и удивились, что Он разговаривал с женщиною; однако ж ни один не сказал: чего Ты требуешь? или: о чем говоришь с нею?
\vs Joh 4:28 Тогда женщина оставила водонос свой и пошла в город, и говорит людям:
\vs Joh 4:29 пойдите, посмотрите Человека, Который сказал мне все, что я сделала: не Он ли Христос?
\vs Joh 4:30 Они вышли из города и пошли к Нему.
\vs Joh 4:31 Между тем ученики просили Его, говоря: Равв\acc{и}! ешь.
\vs Joh 4:32 Но Он сказал им: у Меня есть пища, которой вы не знаете.
\vs Joh 4:33 Посему ученики говорили между собою: разве кто принес Ему есть?
\vs Joh 4:34 Иисус говорит им: Моя пища есть творить волю Пославшего Меня и совершить дело Его.
\vs Joh 4:35 Не говорите ли вы, что еще четыре месяца, и наступит жатва? А Я говорю вам: возведите очи ваши и посмотрите на нивы, как они побелели и поспели к жатве.
\vs Joh 4:36 Жнущий получает награду и собирает плод в жизнь вечную, так что и сеющий и жнущий вместе радоваться будут,
\vs Joh 4:37 ибо в этом случае справедливо изречение: один сеет, а другой жнет.
\vs Joh 4:38 Я послал вас жать то, над чем вы не трудились: другие трудились, а вы вошли в труд их.
\vs Joh 4:39 И многие Самаряне из города того уверовали в Него по слову женщины, свидетельствовавшей, что Он сказал ей все, что она сделала.
\vs Joh 4:40 И потому, когда пришли к Нему Самаряне, то просили Его побыть у них; и Он пробыл там два дня.
\vs Joh 4:41 И еще большее число уверовали по Его слову.
\vs Joh 4:42 А женщине той говорили: уже не по твоим речам веруем, ибо сами слышали и узнали, что Он истинно Спаситель мира, Христос.
\rsbpar\vs Joh 4:43 По прошествии же двух дней Он вышел оттуда и пошел в Галилею,
\vs Joh 4:44 ибо Сам Иисус свидетельствовал, что пророк не имеет чести в своем отечестве.
\vs Joh 4:45 Когда пришел Он в Галилею, то Галилеяне приняли Его, видев все, что Он сделал в Иерусалиме в праздник,~--- ибо и они ходили на праздник.
\vs Joh 4:46 Итак Иисус опять пришел в Кану Галилейскую, где претворил воду в вино. В Капернауме был некоторый царедворец, у которого сын был болен.
\vs Joh 4:47 Он, услышав, что Иисус пришел из Иудеи в Галилею, пришел к Нему и просил Его прийти и исцелить сына его, который был при смерти.
\vs Joh 4:48 Иисус сказал ему: вы не уверуете, если не увидите знамений и чудес.
\vs Joh 4:49 Царедворец говорит Ему: Господи! приди, пока не умер сын мой.
\vs Joh 4:50 Иисус говорит ему: пойди, сын твой здоров. Он поверил слову, которое сказал ему Иисус, и пошел.
\vs Joh 4:51 На дороге встретили его слуги его и сказали: сын твой здоров.
\vs Joh 4:52 Он спросил у них: в котором часу стало ему легче? Ему сказали: вчера в седьмом часу горячка оставила его.
\vs Joh 4:53 Из этого отец узнал, что это был тот час, в который Иисус сказал ему: сын твой здоров, и уверовал сам и весь дом его.
\vs Joh 4:54 Это второе чудо сотворил Иисус, возвратившись из Иудеи в Галилею.
\vs Joh 5:1 После сего был праздник Иудейский, и пришел Иисус в Иерусалим.
\vs Joh 5:2 Есть же в Иерусалиме у Овечьих \bibemph{ворот} купальня, называемая по-еврейски Вифезда\fns{Дом милосердия.}, при которой было пять крытых ходов.
\vs Joh 5:3 В них лежало великое множество больных, слепых, хромых, иссохших, ожидающих движения воды,
\vs Joh 5:4 ибо Ангел Господень по временам сходил в купальню и возмущал воду, и кто первый входил \bibemph{в нее} по возмущении воды, тот выздоравливал, какою бы ни был одержим болезнью.
\vs Joh 5:5 Тут был человек, находившийся в болезни тридцать восемь лет.
\vs Joh 5:6 Иисус, увидев его лежащего и узнав, что он лежит уже долгое время, говорит ему: хочешь ли быть здоров?
\vs Joh 5:7 Больной отвечал Ему: так, Господи; но не имею человека, который опустил бы меня в купальню, когда возмутится вода; когда же я прихожу, другой уже сходит прежде меня.
\vs Joh 5:8 Иисус говорит ему: встань, возьми постель твою и ходи.
\vs Joh 5:9 И он тотчас выздоровел, и взял постель свою и пошел. Было же это в день субботний.
\vs Joh 5:10 Посему Иудеи говорили исцеленному: сегодня суббота; не должно тебе брать постели.
\vs Joh 5:11 Он отвечал им: Кто меня исцелил, Тот мне сказал: возьми постель твою и ходи.
\vs Joh 5:12 Его спросили: кто Тот Человек, Который сказал тебе: возьми постель твою и ходи?
\vs Joh 5:13 Исцеленный же не знал, кто Он, ибо Иисус скрылся в народе, бывшем на том месте.
\vs Joh 5:14 Потом Иисус встретил его в храме и сказал ему: вот, ты выздоровел; не греши больше, чтобы не случилось с тобою чего хуже.
\vs Joh 5:15 Человек сей пошел и объявил Иудеям, что исцеливший его есть Иисус.
\vs Joh 5:16 И стали Иудеи гнать Иисуса и искали убить Его за то, что Он делал такие \bibemph{дела} в субботу.
\rsbpar\vs Joh 5:17 Иисус же говорил им: Отец Мой доныне делает, и Я делаю.
\vs Joh 5:18 И еще более искали убить Его Иудеи за то, что Он не только нарушал субботу, но и Отцем Своим называл Бога, делая Себя равным Богу.
\vs Joh 5:19 На это Иисус сказал: истинно, истинно говорю вам: Сын ничего не может творить Сам от Себя, если не увидит Отца творящего: ибо, что творит Он, то и Сын творит также.
\vs Joh 5:20 Ибо Отец любит Сына и показывает Ему все, что творит Сам; и покажет Ему дела больше сих, так что вы удивитесь.
\vs Joh 5:21 Ибо, как Отец воскрешает мертвых и оживляет, так и Сын оживляет, кого хочет.
\vs Joh 5:22 Ибо Отец и не судит никого, но весь суд отдал Сыну,
\vs Joh 5:23 дабы все чтили Сына, как чтут Отца. Кто не чтит Сына, тот не чтит и Отца, пославшего Его.
\vs Joh 5:24 Истинно, истинно говорю вам: слушающий слово Мое и верующий в Пославшего Меня имеет жизнь вечную, и на суд не приходит, но перешел от смерти в жизнь.
\vs Joh 5:25 Истинно, истинно говорю вам: наступает время, и настало уже, когда мертвые услышат глас Сына Божия и, услышав, оживут.
\vs Joh 5:26 Ибо, как Отец имеет жизнь в Самом Себе, так и Сыну дал иметь жизнь в Самом Себе.
\vs Joh 5:27 И дал Ему власть производить и суд, потому что Он есть Сын Человеческий.
\vs Joh 5:28 Не дивитесь сему; ибо наступает время, в которое все, находящиеся в гробах, услышат глас Сына Божия;
\vs Joh 5:29 и изыдут творившие добро в воскресение жизни, а делавшие зло~--- в воскресение осуждения.
\vs Joh 5:30 Я ничего не могу творить Сам от Себя. Как слышу, так и сужу, и суд Мой праведен; ибо не ищу Моей воли, но воли пославшего Меня Отца.
\vs Joh 5:31 Если Я свидетельствую Сам о Себе, то свидетельство Мое не есть истинно.
\vs Joh 5:32 Есть другой, свидетельствующий о Мне; и Я знаю, что истинно то свидетельство, которым он свидетельствует о Мне.
\vs Joh 5:33 Вы посылали к Иоанну, и он засвидетельствовал об истине.
\vs Joh 5:34 Впрочем Я не от человека принимаю свидетельство, но говорю это для того, чтобы вы спаслись.
\vs Joh 5:35 Он был светильник, горящий и светящий; а вы хотели малое время порадоваться при свете его.
\vs Joh 5:36 Я же имею свидетельство больше Иоаннова: ибо дела, которые Отец дал Мне совершить, самые дела сии, Мною творимые, свидетельствуют о Мне, что Отец послал Меня.
\vs Joh 5:37 И пославший Меня Отец Сам засвидетельствовал о Мне. А вы ни гласа Его никогда не слышали, ни лица Его не видели;
\vs Joh 5:38 и не имеете слова Его пребывающего в вас, потому что вы не веруете Тому, Которого Он послал.
\vs Joh 5:39 Исследуйте Писания, ибо вы думаете чрез них иметь жизнь вечную; а они свидетельствуют о Мне.
\vs Joh 5:40 Но вы не хотите прийти ко Мне, чтобы иметь жизнь.
\vs Joh 5:41 Не принимаю славы от человеков,
\vs Joh 5:42 но знаю вас: вы не имеете в себе любви к Богу.
\vs Joh 5:43 Я пришел во имя Отца Моего, и не принимаете Меня; а если иной придет во имя свое, его примете.
\vs Joh 5:44 Как вы можете веровать, когда друг от друга принимаете славу, а славы, которая от Единого Бога, не ищете?
\vs Joh 5:45 Не думайте, что Я буду обвинять вас пред Отцем: есть на вас обвинитель Моисей, на которого вы уповаете.
\vs Joh 5:46 Ибо если бы вы верили Моисею, то поверили бы и Мне, потому что он писал о Мне.
\vs Joh 5:47 Если же его писаниям не верите, как поверите Моим словам?
\vs Joh 6:1 После сего пошел Иисус на ту сторону моря Галилейского, \bibemph{в окрестности} Тивериады.
\vs Joh 6:2 За Ним последовало множество народа, потому что видели чудеса, которые Он творил над больными.
\vs Joh 6:3 Иисус взошел на гору и там сидел с учениками Своими.
\vs Joh 6:4 Приближалась же Пасха, праздник Иудейский.
\vs Joh 6:5 Иисус, возведя очи и увидев, что множество народа идет к Нему, говорит Филиппу: где нам купить хлебов, чтобы их накормить?
\vs Joh 6:6 Говорил же это, испытывая его; ибо Сам знал, что хотел сделать.
\vs Joh 6:7 Филипп отвечал Ему: им на двести динариев не довольно будет хлеба, чтобы каждому из них досталось хотя понемногу.
\vs Joh 6:8 Один из учеников Его, Андрей, брат Симона Петра, говорит Ему:
\vs Joh 6:9 здесь есть у одного мальчика пять хлебов ячменных и две рыбки; но что это для такого множества?
\vs Joh 6:10 Иисус сказал: велите им возлечь. Было же на том месте много травы. Итак возлегло людей числом около пяти тысяч.
\vs Joh 6:11 Иисус, взяв хлебы и воздав благодарение, раздал ученикам, а ученики возлежавшим, также и рыбы, сколько кто хотел.
\vs Joh 6:12 И когда насытились, то сказал ученикам Своим: соберите оставшиеся куски, чтобы ничего не пропало.
\vs Joh 6:13 И собрали, и наполнили двенадцать коробов кусками от пяти ячменных хлебов, оставшимися у тех, которые ели.
\vs Joh 6:14 Тогда люди, видевшие чудо, сотворенное Иисусом, сказали: это истинно Тот Пророк, Которому должно прийти в мир.
\vs Joh 6:15 Иисус же, узнав, что хотят прийти, нечаянно взять Его и сделать царем, опять удалился на гору один.
\rsbpar\vs Joh 6:16 Когда же настал вечер, то ученики Его сошли к морю
\vs Joh 6:17 и, войдя в лодку, отправились на ту сторону моря, в Капернаум. Становилось темно, а Иисус не приходил к ним.
\vs Joh 6:18 Дул сильный ветер, и море волновалось.
\vs Joh 6:19 Проплыв около двадцати пяти или тридцати стадий, они увидели Иисуса, идущего по морю и приближающегося к лодке, и испугались.
\vs Joh 6:20 Но Он сказал им: это Я; не бойтесь.
\vs Joh 6:21 Они хотели принять Его в лодку; и тотчас лодка пристала к берегу, куда плыли.
\rsbpar\vs Joh 6:22 На другой день народ, стоявший по ту сторону моря, видел, что там, кроме одной лодки, в которую вошли ученики Его, иной не было, и что Иисус не входил в лодку с учениками Своими, а отплыли одни ученики Его.
\vs Joh 6:23 Между тем пришли из Тивериады другие лодки близко к тому месту, где ели хлеб по благословении Господнем.
\vs Joh 6:24 Итак, когда народ увидел, что тут нет Иисуса, ни учеников Его, то вошли в лодки и приплыли в Капернаум, ища Иисуса.
\vs Joh 6:25 И, найдя Его на той стороне моря, сказали Ему: Равв\acc{и}! когда Ты сюда пришел?
\vs Joh 6:26 Иисус сказал им в ответ: истинно, истинно говорю вам: вы ищете Меня не потому, что видели чудеса, но потому, что ели хлеб и насытились.
\vs Joh 6:27 Старайтесь не о пище тленной, но о пище, пребывающей в жизнь вечную, которую даст вам Сын Человеческий, ибо на Нем положил печать \bibemph{Свою} Отец, Бог.
\vs Joh 6:28 Итак сказали Ему: чт\acc{о} нам делать, чтобы творить дела Божии?
\vs Joh 6:29 Иисус сказал им в ответ: вот дело Божие, чтобы вы веровали в Того, Кого Он послал.
\vs Joh 6:30 На это сказали Ему: какое же Ты дашь знамение, чтобы мы увидели и поверили Тебе? чт\acc{о} Ты делаешь?
\vs Joh 6:31 Отцы наши ели манну в пустыне, как написано: хлеб с неба дал им есть.
\vs Joh 6:32 Иисус же сказал им: истинно, истинно говорю вам: не Моисей дал вам хлеб с неба, а Отец Мой дает вам истинный хлеб с небес.
\vs Joh 6:33 Ибо хлеб Божий есть Тот, Который сходит с небес и дает жизнь миру.
\vs Joh 6:34 На это сказали Ему: Господи! подавай нам всегда такой хлеб.
\vs Joh 6:35 Иисус же сказал им: Я есмь хлеб жизни; приходящий ко Мне не будет алкать, и верующий в Меня не будет жаждать никогда.
\vs Joh 6:36 Но Я сказал вам, что вы и видели Меня, и не веруете.
\vs Joh 6:37 Все, что дает Мне Отец, ко Мне придет; и приходящего ко Мне не изгоню вон,
\vs Joh 6:38 ибо Я сошел с небес не для того, чтобы творить волю Мою, но волю пославшего Меня Отца.
\vs Joh 6:39 Воля же пославшего Меня Отца есть та, чтобы из того, что Он Мне дал, ничего не погубить, но все то воскресить в последний день.
\vs Joh 6:40 Воля Пославшего Меня есть та, чтобы всякий, видящий Сына и верующий в Него, имел жизнь вечную; и Я воскрешу его в последний день.
\vs Joh 6:41 Возроптали на Него Иудеи за то, что Он сказал: Я есмь хлеб, сшедший с небес.
\vs Joh 6:42 И говорили: не Иисус ли это, сын Иосифов, Которого отца и Мать мы знаем? Как же говорит Он: Я сшел с небес?
\vs Joh 6:43 Иисус сказал им в ответ: не ропщите между собою.
\vs Joh 6:44 Никто не может прийти ко Мне, если не привлечет его Отец, пославший Меня; и Я воскрешу его в последний день.
\vs Joh 6:45 У пророков написано: и будут все научены Богом. Всякий, слышавший от Отца и научившийся, приходит ко Мне.
\vs Joh 6:46 Это не то, чтобы кто видел Отца, кроме Того, Кто есть от Бога; Он видел Отца.
\vs Joh 6:47 Истинно, истинно говорю вам: верующий в Меня имеет жизнь вечную.
\vs Joh 6:48 Я есмь хлеб жизни.
\vs Joh 6:49 Отцы ваши ели манну в пустыне и умерли;
\vs Joh 6:50 хлеб же, сходящий с небес, таков, что ядущий его не умрет.
\vs Joh 6:51 Я хлеб живый, сшедший с небес; ядущий хлеб сей будет жить вовек; хлеб же, который Я дам, есть Плоть Моя, которую Я отдам за жизнь мира.
\vs Joh 6:52 Тогда Иудеи стали спорить между собою, говоря: как Он может дать нам есть Плоть Свою?
\vs Joh 6:53 Иисус же сказал им: истинно, истинно говорю вам: если не будете есть Плоти Сына Человеческого и пить Крови Его, то не будете иметь в себе жизни.
\vs Joh 6:54 Ядущий Мою Плоть и пиющий Мою Кровь имеет жизнь вечную, и Я воскрешу его в последний день.
\vs Joh 6:55 Ибо Плоть Моя истинно есть пища, и Кровь Моя истинно есть питие.
\vs Joh 6:56 Ядущий Мою Плоть и пиющий Мою Кровь пребывает во Мне, и Я в нем.
\vs Joh 6:57 Как послал Меня живый Отец, и Я живу Отцем, \bibemph{так} и ядущий Меня жить будет Мною.
\vs Joh 6:58 Сей-то есть хлеб, сшедший с небес. Не так, как отцы ваши ели манну и умерли: ядущий хлеб сей жить будет вовек.
\vs Joh 6:59 Сие говорил Он в синагоге, уча в Капернауме.
\vs Joh 6:60 Многие из учеников Его, слыша то, говорили: какие странные слова! кто может это слушать?
\vs Joh 6:61 Но Иисус, зная Сам в Себе, что ученики Его ропщут на то, сказал им: это ли соблазняет вас?
\vs Joh 6:62 Что ж, если увидите Сына Человеческого восходящего \bibemph{туда}, где был прежде?
\vs Joh 6:63 Дух животворит; плоть не пользует нимало. Слова, которые говорю Я вам, суть дух и жизнь.
\vs Joh 6:64 Но есть из вас некоторые неверующие. Ибо Иисус от начала знал, кто суть неверующие и кто предаст Его.
\vs Joh 6:65 И сказал: для того-то и говорил Я вам, что никто не может прийти ко Мне, если то не дано будет ему от Отца Моего.
\vs Joh 6:66 С этого времени многие из учеников Его отошли от Него и уже не ходили с Ним.
\vs Joh 6:67 Тогда Иисус сказал двенадцати: не хотите ли и вы отойти?
\vs Joh 6:68 Симон Петр отвечал Ему: Господи! к кому нам идти? Ты имеешь глаголы вечной жизни:
\vs Joh 6:69 и мы уверовали и познали, что Ты Христос, Сын Бога живаго.
\vs Joh 6:70 Иисус отвечал им: не двенадцать ли вас избрал Я? но один из вас диавол.
\vs Joh 6:71 Это говорил Он об Иуде Симонове Искариоте, ибо сей хотел предать Его, будучи один из двенадцати.
\vs Joh 7:1 После сего Иисус ходил по Галилее, ибо по Иудее не хотел ходить, потому что Иудеи искали убить Его.
\rsbpar\vs Joh 7:2 Приближался праздник Иудейский~--- поставление кущей.
\vs Joh 7:3 Тогда братья Его сказали Ему: выйди отсюда и пойди в Иудею, чтобы и ученики Твои видели дела, которые Ты делаешь.
\vs Joh 7:4 Ибо никто не делает чего-либо втайне, и ищет сам быть известным. Если Ты творишь такие дела, то яви Себя миру.
\vs Joh 7:5 Ибо и братья Его не веровали в Него.
\vs Joh 7:6 На это Иисус сказал им: Мое время еще не настало, а для вас всегда время.
\vs Joh 7:7 Вас мир не может ненавидеть, а Меня ненавидит, потому что Я свидетельствую о нем, что дела его злы.
\vs Joh 7:8 Вы пойдите на праздник сей; а Я еще не пойду на сей праздник, потому что Мое время еще не исполнилось.
\vs Joh 7:9 Сие сказав им, остался в Галилее.
\rsbpar\vs Joh 7:10 Но когда пришли братья Его, тогда и Он пришел на праздник не явно, а как бы тайно.
\vs Joh 7:11 Иудеи же искали Его на празднике и говорили: где Он?
\vs Joh 7:12 И много толков было о Нем в народе: одни говорили, что Он добр; а другие говорили: нет, но обольщает народ.
\vs Joh 7:13 Впрочем никто не говорил о Нем явно, боясь Иудеев.
\vs Joh 7:14 Но в половине уже праздника вошел Иисус в храм и учил.
\vs Joh 7:15 И дивились Иудеи, говоря: как Он знает Писания, не учившись?
\vs Joh 7:16 Иисус, отвечая им, сказал: Мое учение~--- не Мое, но Пославшего Меня;
\vs Joh 7:17 кто хочет творить волю Его, тот узнает о сем учении, от Бога ли оно, или Я Сам от Себя говорю.
\vs Joh 7:18 Говорящий сам от себя ищет славы себе; а Кто ищет славы Пославшему Его, Тот истинен, и нет неправды в Нем.
\vs Joh 7:19 Не дал ли вам Моисей закона? и никто из вас не поступает по закону. За что ищете убить Меня?
\vs Joh 7:20 Народ сказал в ответ: не бес ли в Тебе? кто ищет убить Тебя?
\vs Joh 7:21 Иисус, продолжая речь, сказал им: одно дело сделал Я, и все вы дивитесь.
\vs Joh 7:22 Моисей дал вам обрезание (хотя оно не от Моисея, но от отцов), и в субботу вы обрезываете человека.
\vs Joh 7:23 Если в субботу принимает человек обрезание, чтобы не был нарушен закон Моисеев,~--- на Меня ли негодуете за то, что Я всего человека исцелил в субботу?
\vs Joh 7:24 Не суд\acc{и}те по наружности, но суд\acc{и}те судом праведным.
\vs Joh 7:25 Тут некоторые из Иерусалимлян говорили: не Тот ли это, Которого ищут убить?
\vs Joh 7:26 Вот, Он говорит явно, и ничего не говорят Ему: не удостоверились ли начальники, что Он подлинно Христос?
\vs Joh 7:27 Но мы знаем Его, откуда Он; Христос же когда придет, никто не будет знать, откуда Он.
\vs Joh 7:28 Тогда Иисус возгласил в храме, уча и говоря: и знаете Меня, и знаете, откуда Я; и Я пришел не Сам от Себя, но истинен Пославший Меня, Которого вы не знаете.
\vs Joh 7:29 Я знаю Его, потому что Я от Него, и Он послал Меня.
\vs Joh 7:30 И искали схватить Его, но никто не наложил на Него рук\acc{и}, потому что еще не пришел час Его.
\vs Joh 7:31 Многие же из народа уверовали в Него и говорили: когда придет Христос, неужели сотворит больше знамений, нежели сколько Сей сотворил?
\vs Joh 7:32 Услышали фарисеи такие толки о Нем в народе, и послали фарисеи и первосвященники служителей~--- схватить Его.
\vs Joh 7:33 Иисус же сказал им: еще недолго быть Мне с вами, и пойду к Пославшему Меня;
\vs Joh 7:34 будете искать Меня, и не найдете; и где буду Я, \bibemph{туда} вы не можете прийти.
\vs Joh 7:35 При сем Иудеи говорили между собою: куда Он хочет идти, так что мы не найдем Его? Не хочет ли Он идти в Еллинское рассеяние и учить Еллинов?
\vs Joh 7:36 Что значат сии слова, которые Он сказал: будете искать Меня, и не найдете; и где буду Я, \bibemph{туда} вы не можете прийти?
\rsbpar\vs Joh 7:37 В последний же великий день праздника стоял Иисус и возгласил, говоря: кто жаждет, иди ко Мне и пей.
\vs Joh 7:38 Кто верует в Меня, у того, как сказано в Писании, из чрева потекут реки воды живой.
\vs Joh 7:39 Сие сказал Он о Духе, Которого имели принять верующие в Него: ибо еще не было на них Духа Святаго, потому что Иисус еще не был прославлен.
\vs Joh 7:40 Многие из народа, услышав сии слова, говорили: Он точно пророк.
\vs Joh 7:41 Другие говорили: это Христос. А иные говорили: разве из Галилеи Христос придет?
\vs Joh 7:42 Не сказано ли в Писании, что Христос придет от семени Давидова и из Вифлеема, из того места, откуда был Давид?
\vs Joh 7:43 Итак произошла о Нем распря в народе.
\vs Joh 7:44 Некоторые из них хотели схватить Его; но никто не наложил на Него рук.
\vs Joh 7:45 Итак служители возвратились к первосвященникам и фарисеям, и сии сказали им: для чего вы не привели Его?
\vs Joh 7:46 Служители отвечали: никогда человек не говорил так, как Этот Человек.
\vs Joh 7:47 Фарисеи сказали им: неужели и вы прельстились?
\vs Joh 7:48 Уверовал ли в Него кто из начальников, или из фарисеев?
\vs Joh 7:49 Но этот народ невежда в законе, проклят он.
\vs Joh 7:50 Никодим, приходивший к Нему ночью, будучи один из них, говорит им:
\vs Joh 7:51 судит ли закон наш человека, если прежде не выслушают его и не узнают, что он делает?
\vs Joh 7:52 На это сказали ему: и ты не из Галилеи ли? рассмотри и увидишь, что из Галилеи не приходит пророк.
\vs Joh 7:53 И разошлись все по домам.
\vs Joh 8:1 Иисус же пошел на гору Елеонскую.
\vs Joh 8:2 А утром опять пришел в храм, и весь народ шел к Нему. Он сел и учил их.
\vs Joh 8:3 Тут книжники и фарисеи привели к Нему женщину, взятую в прелюбодеянии, и, поставив ее посреди,
\vs Joh 8:4 сказали Ему: Учитель! эта женщина взята в прелюбодеянии;
\vs Joh 8:5 а Моисей в законе заповедал нам побивать таких камнями: Ты что скажешь?
\vs Joh 8:6 Говорили же это, искушая Его, чтобы найти что-нибудь к обвинению Его. Но Иисус, наклонившись низко, писал перстом на земле, не обращая на них внимания.
\vs Joh 8:7 Когда же продолжали спрашивать Его, Он, восклонившись, сказал им: кто из вас без греха, первый брось на нее камень.
\vs Joh 8:8 И опять, наклонившись низко, писал на земле.
\vs Joh 8:9 Они же, услышав \bibemph{то} и будучи обличаемы совестью, стали уходить один за другим, начиная от старших до последних; и остался один Иисус и женщина, стоящая посреди.
\vs Joh 8:10 Иисус, восклонившись и не видя никого, кроме женщины, сказал ей: женщина! где твои обвинители? никто не осудил тебя?
\vs Joh 8:11 Она отвечала: никто, Господи. Иисус сказал ей: и Я не осуждаю тебя; иди и впредь не греши.
\rsbpar\vs Joh 8:12 Опять говорил Иисус \bibemph{к народу} и сказал им: Я свет миру; кто последует за Мною, тот не будет ходить во тьме, но будет иметь свет жизни.
\vs Joh 8:13 Тогда фарисеи сказали Ему: Ты Сам о Себе свидетельствуешь, свидетельство Твое не истинно.
\vs Joh 8:14 Иисус сказал им в ответ: если Я и Сам о Себе свидетельствую, свидетельство Мое истинно; потому что Я знаю, откуда пришел и куда иду; а вы не знаете, откуда Я и куда иду.
\vs Joh 8:15 Вы с\acc{у}дите по плоти; Я не сужу никого.
\vs Joh 8:16 А если и сужу Я, то суд Мой истинен, потому что Я не один, но Я и Отец, пославший Меня.
\vs Joh 8:17 А и в законе вашем написано, что двух человек свидетельство истинно.
\vs Joh 8:18 Я Сам свидетельствую о Себе, и свидетельствует о Мне Отец, пославший Меня.
\vs Joh 8:19 Тогда сказали Ему: где Твой Отец? Иисус отвечал: вы не знаете ни Меня, ни Отца Моего; если бы вы знали Меня, то знали бы и Отца Моего.
\vs Joh 8:20 Сии слова говорил Иисус у сокровищницы, когда учил в храме; и никто не взял Его, потому что еще не пришел час Его.
\vs Joh 8:21 Опять сказал им Иисус: Я отхожу, и будете искать Меня, и умрете во грехе вашем. Куда Я иду, \bibemph{туда} вы не можете прийти.
\vs Joh 8:22 Тут Иудеи говорили: неужели Он убьет Сам Себя, что говорит: <<куда Я иду, вы не можете прийти>>?
\vs Joh 8:23 Он сказал им: вы от нижних, Я от вышних; вы от мира сего, Я не от сего мира.
\vs Joh 8:24 Потому Я и сказал вам, что вы умрете во грехах ваших; ибо если не уверуете, что это Я, то умрете во грехах ваших.
\vs Joh 8:25 Тогда сказали Ему: кто же Ты? Иисус сказал им: от начала Сущий, как и говорю вам.
\vs Joh 8:26 Много имею говорить и судить о вас; но Пославший Меня есть истинен, и что Я слышал от Него, то и говорю миру.
\vs Joh 8:27 Не поняли, что Он говорил им об Отце.
\vs Joh 8:28 Итак Иисус сказал им: когда вознесете Сына Человеческого, тогда узнаете, что это Я и что ничего не делаю от Себя, но как научил Меня Отец Мой, так и говорю.
\vs Joh 8:29 Пославший Меня есть со Мною; Отец не оставил Меня одного, ибо Я всегда делаю то, что Ему угодно.
\vs Joh 8:30 Когда Он говорил это, многие уверовали в Него.
\vs Joh 8:31 Тогда сказал Иисус к уверовавшим в Него Иудеям: если пребудете в слове Моем, то вы истинно Мои ученики,
\vs Joh 8:32 и позн\acc{а}ете истину, и истина сделает вас свободными.
\vs Joh 8:33 Ему отвечали: мы семя Авраамово и не были рабами никому никогда; как же Ты говоришь: сделаетесь свободными?
\vs Joh 8:34 Иисус отвечал им: истинно, истинно говорю вам: всякий, делающий грех, есть раб греха.
\vs Joh 8:35 Но раб не пребывает в доме вечно; сын пребывает вечно.
\vs Joh 8:36 Итак, если Сын освободит вас, то истинно свободны будете.
\vs Joh 8:37 Знаю, что вы семя Авраамово; однако ищете убить Меня, потому что слово Мое не вмещается в вас.
\vs Joh 8:38 Я говорю то, что видел у Отца Моего; а вы делаете то, что видели у отца вашего.
\vs Joh 8:39 Сказали Ему в ответ: отец наш есть Авраам. Иисус сказал им: если бы вы были дети Авраама, то дела Авраамовы делали бы.
\vs Joh 8:40 А теперь ищете убить Меня, Человека, сказавшего вам истину, которую слышал от Бога: Авраам этого не делал.
\vs Joh 8:41 Вы делаете дела отца вашего. На это сказали Ему: мы не от любодеяния рождены; одного Отца имеем, Бога.
\vs Joh 8:42 Иисус сказал им: если бы Бог был Отец ваш, то вы любили бы Меня, потому что Я от Бога исшел и пришел; ибо Я не Сам от Себя пришел, но Он послал Меня.
\vs Joh 8:43 Почему вы не понимаете речи Моей? Потому что не можете слышать сл\acc{о}ва Моего.
\vs Joh 8:44 Ваш отец диавол; и вы хотите исполнять похоти отца вашего. Он был человекоубийца от начала и не устоял в истине, ибо нет в нем истины. Когда говорит он ложь, говорит свое, ибо он лжец и отец лжи.
\vs Joh 8:45 А как Я истину говорю, то не верите Мне.
\vs Joh 8:46 Кто из вас обличит Меня в неправде? Если же Я говорю истину, почему вы не верите Мне?
\vs Joh 8:47 Кто от Бога, тот слушает слова Божии. Вы потому не слушаете, что вы не от Бога.
\vs Joh 8:48 На это Иудеи отвечали и сказали Ему: не правду ли мы говорим, что Ты Самарянин и что бес в Тебе?
\vs Joh 8:49 Иисус отвечал: во Мне беса нет; но Я чту Отца Моего, а вы бесчестите Меня.
\vs Joh 8:50 Впрочем Я не ищу Моей славы: есть Ищущий и Судящий.
\vs Joh 8:51 Истинно, истинно говорю вам: кто соблюдет слово Мое, тот не увидит смерти вовек.
\vs Joh 8:52 Иудеи сказали Ему: теперь узнали мы, что бес в Тебе. Авраам умер и пророки, а Ты говоришь: кто соблюдет слово Мое, тот не вкусит смерти вовек.
\vs Joh 8:53 Неужели Ты больше отца нашего Авраама, который умер? и пророки умерли: чем Ты Себя делаешь?
\vs Joh 8:54 Иисус отвечал: если Я Сам Себя славлю, то слава Моя ничто. Меня прославляет Отец Мой, о Котором вы говорите, что Он Бог ваш.
\vs Joh 8:55 И вы не познали Его, а Я знаю Его; и если скажу, что не знаю Его, то буду подобный вам лжец. Но Я знаю Его и соблюдаю слово Его.
\vs Joh 8:56 Авраам, отец ваш, рад был увидеть день Мой; и увидел и возрадовался.
\vs Joh 8:57 На это сказали Ему Иудеи: Тебе нет еще пятидесяти лет,~--- и Ты видел Авраама?
\vs Joh 8:58 Иисус сказал им: истинно, истинно говорю вам: прежде нежели был Авраам, Я есмь.
\vs Joh 8:59 Тогда взяли каменья, чтобы бросить на Него; но Иисус скрылся и вышел из храма, пройдя посреди них, и пошел далее.
\vs Joh 9:1 И, проходя, увидел человека, слепого от рождения.
\vs Joh 9:2 Ученики Его спросили у Него: Равв\acc{и}! кто согрешил, он или родители его, что родился слепым?
\vs Joh 9:3 Иисус отвечал: не согрешил ни он, ни родители его, но \bibemph{это для того}, чтобы на нем явились дела Божии.
\vs Joh 9:4 Мне должно делать дела Пославшего Меня, доколе есть день; приходит ночь, когда никто не может делать.
\vs Joh 9:5 Доколе Я в мире, Я свет миру.
\vs Joh 9:6 Сказав это, Он плюнул на землю, сделал брение из плюновения и помазал брением глаза слепому,
\vs Joh 9:7 и сказал ему: пойди, умойся в купальне Силоам, что значит: посланный. Он пошел и умылся, и пришел зрячим.
\vs Joh 9:8 Тут соседи и видевшие прежде, что он был слеп, говорили: не тот ли это, который сидел и просил милостыни?
\vs Joh 9:9 Иные говорили: это он, а иные: похож на него. Он же говорил: это я.
\vs Joh 9:10 Тогда спрашивали у него: как открылись у тебя глаза?
\vs Joh 9:11 Он сказал в ответ: Человек, называемый Иисус, сделал брение, помазал глаза мои и сказал мне: пойди на купальню Силоам и умойся. Я пошел, умылся и прозрел.
\vs Joh 9:12 Тогда сказали ему: где Он? Он отвечал: не знаю.
\vs Joh 9:13 Повели сего бывшего слепца к фарисеям.
\vs Joh 9:14 А была суббота, когда Иисус сделал брение и отверз ему очи.
\vs Joh 9:15 Спросили его также и фарисеи, как он прозрел. Он сказал им: брение положил Он на мои глаза, и я умылся, и вижу.
\vs Joh 9:16 Тогда некоторые из фарисеев говорили: не от Бога Этот Человек, потому что не хранит субботы. Другие говорили: как может человек грешный творить такие чудеса? И была между ними распря.
\vs Joh 9:17 Опять говорят слепому: ты что скажешь о Нем, потому что Он отверз тебе очи? Он сказал: это пророк.
\vs Joh 9:18 Тогда Иудеи не поверили, что он был слеп и прозрел, доколе не призвали родителей сего прозревшего
\vs Joh 9:19 и спросили их: это ли сын ваш, о котором вы говорите, что родился слепым? как же он теперь видит?
\vs Joh 9:20 Родители его сказали им в ответ: мы знаем, что это сын наш и что он родился слепым,
\vs Joh 9:21 а как теперь видит, не знаем, или кто отверз ему очи, мы не знаем. Сам в совершенных летах; самого спрос\acc{и}те; пусть сам о себе скажет.
\vs Joh 9:22 Так отвечали родители его, потому что боялись Иудеев; ибо Иудеи сговорились уже, чтобы, кто признает Его за Христа, того отлучать от синагоги.
\vs Joh 9:23 Посему-то родители его и сказали: он в совершенных летах; самого спрос\acc{и}те.
\vs Joh 9:24 Итак, вторично призвали человека, который был слеп, и сказали ему: воздай славу Богу; мы знаем, что Человек Тот грешник.
\vs Joh 9:25 Он сказал им в ответ: грешник ли Он, не знаю; одно знаю, что я был слеп, а теперь вижу.
\vs Joh 9:26 Снова спросили его: что сделал Он с тобою? как отверз твои очи?
\vs Joh 9:27 Отвечал им: я уже сказал вам, и вы не слушали; что еще хотите слышать? или и вы хотите сделаться Его учениками?
\vs Joh 9:28 Они же укорили его и сказали: ты ученик Его, а мы Моисеевы ученики.
\vs Joh 9:29 Мы знаем, что с Моисеем говорил Бог; Сего же не знаем, откуда Он.
\vs Joh 9:30 Человек \bibemph{прозревший} сказал им в ответ: это и удивительно, что вы не знаете, откуда Он, а Он отверз мне очи.
\vs Joh 9:31 Но мы знаем, что грешников Бог не слушает; но кто чтит Бога и творит волю Его, того слушает.
\vs Joh 9:32 От века не слыхано, чтобы кто отверз очи слепорожденному.
\vs Joh 9:33 Если бы Он не был от Бога, не мог бы творить ничего.
\vs Joh 9:34 Сказали ему в ответ: во грехах ты весь родился, и ты ли нас учишь? И выгнали его вон.
\vs Joh 9:35 Иисус, услышав, что выгнали его вон, и найдя его, сказал ему: ты веруешь ли в Сына Божия?
\vs Joh 9:36 Он отвечал и сказал: а кто Он, Господи, чтобы мне веровать в Него?
\vs Joh 9:37 Иисус сказал ему: и видел ты Его, и Он говорит с тобою.
\vs Joh 9:38 Он же сказал: верую, Господи! И поклонился Ему.
\rsbpar\vs Joh 9:39 И сказал Иисус: на суд пришел Я в мир сей, чтобы невидящие видели, а видящие стали слепы.
\vs Joh 9:40 Услышав это, некоторые из фарисеев, бывших с Ним, сказали Ему: неужели и мы слепы?
\vs Joh 9:41 Иисус сказал им: если бы вы были слепы, то не имели бы \bibemph{на себе} греха; но как вы говорите, что видите, то грех остается на вас.
\vs Joh 10:1 Истинно, истинно говорю вам: кто не дверью входит во двор овчий, но перелазит инуде, тот вор и разбойник;
\vs Joh 10:2 а входящий дверью есть пастырь овцам.
\vs Joh 10:3 Ему придверник отворяет, и овцы слушаются голоса его, и он зовет своих овец по имени и выводит их.
\vs Joh 10:4 И когда выведет своих овец, идет перед ними; а овцы за ним идут, потому что знают голос его.
\vs Joh 10:5 За чужим же не идут, но бегут от него, потому что не знают чужого голоса.
\vs Joh 10:6 Сию притчу сказал им Иисус; но они не поняли, что такое Он говорил им.
\vs Joh 10:7 Итак, опять Иисус сказал им: истинно, истинно говорю вам, что Я дверь овцам.
\vs Joh 10:8 Все, сколько их ни приходило предо Мною, суть воры и разбойники; но овцы не послушали их.
\vs Joh 10:9 Я есмь дверь: кто войдет Мною, тот спасется, и войдет, и выйдет, и пажить найдет.
\vs Joh 10:10 Вор приходит только для того, чтобы украсть, убить и погубить. Я пришел для того, чтобы имели жизнь и имели с избытком.
\vs Joh 10:11 Я есмь пастырь добрый: пастырь добрый полагает жизнь свою за овец.
\vs Joh 10:12 А наемник, не пастырь, которому овцы не свои, видит приходящего волка, и оставляет овец, и бежит; и волк расхищает овец, и разгоняет их.
\vs Joh 10:13 А наемник бежит, потому что наемник, и нерадит об овцах.
\vs Joh 10:14 Я есмь пастырь добрый; и знаю Моих, и Мои знают Меня.
\vs Joh 10:15 Как Отец знает Меня, \bibemph{так} и Я знаю Отца; и жизнь Мою полагаю за овец.
\vs Joh 10:16 Есть у Меня и другие овцы, которые не сего двора, и тех надлежит Мне привести: и они услышат голос Мой, и будет одно стадо и один Пастырь.
\vs Joh 10:17 Потому любит Меня Отец, что Я отдаю жизнь Мою, чтобы опять принять ее.
\vs Joh 10:18 Никто не отнимает ее у Меня, но Я Сам отдаю ее. Имею власть отдать ее и власть имею опять принять ее. Сию заповедь получил Я от Отца Моего.
\vs Joh 10:19 От этих слов опять произошла между Иудеями распря.
\vs Joh 10:20 Многие из них говорили: Он одержим бесом и безумствует; что слушаете Его?
\vs Joh 10:21 Другие говорили: это слова не бесноватого; может ли бес отверзать очи слепым?
\rsbpar\vs Joh 10:22 Настал же тогда в Иерусалиме \bibemph{праздник} обновления, и была зима.
\vs Joh 10:23 И ходил Иисус в храме, в притворе Соломоновом.
\vs Joh 10:24 Тут Иудеи обступили Его и говорили Ему: долго ли Тебе держать нас в недоумении? если Ты Христос, скажи нам прямо.
\vs Joh 10:25 Иисус отвечал им: Я сказал вам, и не верите; дела, которые творю Я во имя Отца Моего, они свидетельствуют о Мне.
\vs Joh 10:26 Но вы не верите, ибо вы не из овец Моих, как Я сказал вам.
\vs Joh 10:27 Овцы Мои слушаются голоса Моего, и Я знаю их; и они идут за Мною.
\vs Joh 10:28 И Я даю им жизнь вечную, и не погибнут вовек; и никто не похитит их из руки Моей.
\vs Joh 10:29 Отец Мой, Который дал Мне их, больше всех; и никто не может похитить их из руки Отца Моего.
\vs Joh 10:30 Я и Отец~--- одно.
\vs Joh 10:31 Тут опять Иудеи схватили каменья, чтобы побить Его.
\vs Joh 10:32 Иисус отвечал им: много добрых дел показал Я вам от Отца Моего; за которое из них хотите побить Меня камнями?
\vs Joh 10:33 Иудеи сказали Ему в ответ: не за доброе дело хотим побить Тебя камнями, но за богохульство и за то, что Ты, будучи человек, делаешь Себя Богом.
\vs Joh 10:34 Иисус отвечал им: не написано ли в законе вашем: Я сказал: вы боги?
\vs Joh 10:35 Если Он назвал богами тех, к которым было слово Божие, и не может нарушиться Писание,~---
\vs Joh 10:36 Тому ли, Которого Отец освятил и послал в мир, вы говорите: богохульствуешь, потому что Я сказал: Я Сын Божий?
\vs Joh 10:37 Если Я не творю дел Отца Моего, не верьте Мне;
\vs Joh 10:38 а если творю, то, когда не верите Мне, верьте делам Моим, чтобы узнать и поверить, что Отец во Мне и Я в Нем.
\vs Joh 10:39 Тогда опять искали схватить Его; но Он уклонился от рук их,
\vs Joh 10:40 и пошел опять за Иордан, на то место, где прежде крестил Иоанн, и остался там.
\vs Joh 10:41 Многие пришли к Нему и говорили, что Иоанн не сотворил никакого чуда, но все, что сказал Иоанн о Нем, было истинно.
\vs Joh 10:42 И многие там уверовали в Него.
\vs Joh 11:1 Был болен некто Лазарь из Вифании, из селения, \bibemph{где жили} Мария и Марфа, сестра ее.
\vs Joh 11:2 Мария же, которой брат Лазарь был болен, была \bibemph{та}, которая помазала Господа миром и отерла ноги Его волосами своими.
\vs Joh 11:3 Сестры послали сказать Ему: Господи! вот, кого Ты любишь, болен.
\vs Joh 11:4 Иисус, услышав \bibemph{то}, сказал: эта болезнь не к смерти, но к славе Божией, да прославится через нее Сын Божий.
\vs Joh 11:5 Иисус же любил Марфу и сестру ее и Лазаря.
\vs Joh 11:6 Когда же услышал, что он болен, то пробыл два дня на том месте, где находился.
\vs Joh 11:7 После этого сказал ученикам: пойдем опять в Иудею.
\vs Joh 11:8 Ученики сказали Ему: Равв\acc{и}! давно ли Иудеи искали побить Тебя камнями, и Ты опять идешь туда?
\vs Joh 11:9 Иисус отвечал: не двенадцать ли часов во дне? кто ходит днем, тот не спотыкается, потому что видит свет мира сего;
\vs Joh 11:10 а кто ходит ночью, спотыкается, потому что нет света с ним.
\vs Joh 11:11 Сказав это, говорит им потом: Лазарь, друг наш, уснул; но Я иду разбудить его.
\vs Joh 11:12 Ученики Его сказали: Господи! если уснул, то выздоровеет.
\vs Joh 11:13 Иисус говорил о смерти его, а они думали, что Он говорит о сне обыкновенном.
\vs Joh 11:14 Тогда Иисус сказал им прямо: Лазарь умер;
\vs Joh 11:15 и радуюсь за вас, что Меня не было там, дабы вы уверовали; но пойдем к нему.
\vs Joh 11:16 Тогда Фома, иначе называемый Близнец, сказал ученикам: пойдем и мы умрем с ним.
\vs Joh 11:17 Иисус, придя, нашел, что он уже четыре дня в гробе.
\vs Joh 11:18 Вифания же была близ Иерусалима, стадиях в пятнадцати;
\vs Joh 11:19 и многие из Иудеев пришли к Марфе и Марии утешать их \bibemph{в печали} о брате их.
\vs Joh 11:20 Марфа, услышав, что идет Иисус, пошла навстречу Ему; Мария же сидела дома.
\vs Joh 11:21 Тогда Марфа сказала Иисусу: Господи! если бы Ты был здесь, не умер бы брат мой.
\vs Joh 11:22 Но и теперь знаю, что чего Ты попросишь у Бога, даст Тебе Бог.
\vs Joh 11:23 Иисус говорит ей: воскреснет брат твой.
\vs Joh 11:24 Марфа сказала Ему: знаю, что воскреснет в воскресение, в последний день.
\vs Joh 11:25 Иисус сказал ей: Я есмь воскресение и жизнь; верующий в Меня, если и умрет, оживет.
\vs Joh 11:26 И всякий, живущий и верующий в Меня, не умрет вовек. Веришь ли сему?
\vs Joh 11:27 Она говорит Ему: так, Господи! я верую, что Ты Христос, Сын Божий, грядущий в мир.
\vs Joh 11:28 Сказав это, пошла и позвала тайно Марию, сестру свою, говоря: Учитель здесь и зовет тебя.
\vs Joh 11:29 Она, как скоро услышала, поспешно встала и пошла к Нему.
\vs Joh 11:30 Иисус еще не входил в селение, но был на том месте, где встретила Его Марфа.
\vs Joh 11:31 Иудеи, которые были с нею в доме и утешали ее, видя, что Мария поспешно встала и вышла, пошли за нею, полагая, что она пошла на гроб~--- плакать там.
\vs Joh 11:32 Мария же, придя туда, где был Иисус, и увидев Его, пала к ногам Его и сказала Ему: Господи! если бы Ты был здесь, не умер бы брат мой.
\vs Joh 11:33 Иисус, когда увидел ее плачущую и пришедших с нею Иудеев плачущих, Сам восскорбел духом и возмутился
\vs Joh 11:34 и сказал: где вы положили его? Говорят Ему: Господи! пойди и посмотри.
\vs Joh 11:35 Иисус прослезился.
\vs Joh 11:36 Тогда Иудеи говорили: смотри, как Он любил его.
\vs Joh 11:37 А некоторые из них сказали: не мог ли Сей, отверзший очи слепому, сделать, чтобы и этот не умер?
\vs Joh 11:38 Иисус же, опять скорбя внутренно, приходит ко гробу. То была пещера, и камень лежал на ней.
\vs Joh 11:39 Иисус говорит: отнимите камень. Сестра умершего, Марфа, говорит Ему: Господи! уже смердит; ибо четыре дня, как он во гробе.
\vs Joh 11:40 Иисус говорит ей: не сказал ли Я тебе, что, если будешь веровать, увидишь славу Божию?
\vs Joh 11:41 Итак отняли камень \bibemph{от пещеры}, где лежал умерший. Иисус же возвел очи к небу и сказал: Отче! благодарю Тебя, что Ты услышал Меня.
\vs Joh 11:42 Я и знал, что Ты всегда услышишь Меня; но сказал \bibemph{сие} для народа, здесь стоящего, чтобы поверили, что Ты послал Меня.
\vs Joh 11:43 Сказав это, Он воззвал громким голосом: Лазарь! иди вон.
\vs Joh 11:44 И вышел умерший, обвитый по рукам и ногам погребальными пеленами, и лице его обвязано было платком. Иисус говорит им: развяжите его, пусть идет.
\vs Joh 11:45 Тогда многие из Иудеев, пришедших к Марии и видевших, что сотворил Иисус, уверовали в Него.
\vs Joh 11:46 А некоторые из них пошли к фарисеям и сказали им, что сделал Иисус.
\rsbpar\vs Joh 11:47 Тогда первосвященники и фарисеи собрали совет и говорили: что нам делать? Этот Человек много чудес творит.
\vs Joh 11:48 Если оставим Его так, то все уверуют в Него, и придут Римляне и овладеют и местом нашим и народом.
\vs Joh 11:49 Один же из них, некто Каиафа, будучи на тот год первосвященником, сказал им: вы ничего не знаете,
\vs Joh 11:50 и не подумаете, что лучше нам, чтобы один человек умер за людей, нежели чтобы весь народ погиб.
\vs Joh 11:51 Сие же он сказал не от себя, но, будучи на тот год первосвященником, предсказал, что Иисус умрет за народ,
\vs Joh 11:52 и не только за народ, но чтобы и рассеянных чад Божиих собрать воедино.
\vs Joh 11:53 С этого дня положили убить Его.
\vs Joh 11:54 Посему Иисус уже не ходил явно между Иудеями, а пошел оттуда в страну близ пустыни, в город, называемый Ефраим, и там оставался с учениками Своими.
\rsbpar\vs Joh 11:55 Приближалась Пасха Иудейская, и многие из всей страны пришли в Иерусалим перед Пасхою, чтобы очиститься.
\vs Joh 11:56 Тогда искали Иисуса и, стоя в храме, говорили друг другу: как вы думаете? не придет ли Он на праздник?
\vs Joh 11:57 Первосвященники же и фарисеи дали приказание, что если кто узнает, где Он будет, то объявил бы, дабы взять Его.
\vs Joh 12:1 За шесть дней до Пасхи пришел Иисус в Вифанию, где был Лазарь умерший, которого Он воскресил из мертвых.
\vs Joh 12:2 Там приготовили Ему вечерю, и Марфа служила, и Лазарь был одним из возлежавших с Ним.
\vs Joh 12:3 Мария же, взяв фунт нардового чистого драгоценного мира, помазала ноги Иисуса и отерла волосами своими ноги Его; и дом наполнился благоуханием от мира.
\vs Joh 12:4 Тогда один из учеников Его, Иуда Симонов Искариот, который хотел предать Его, сказал:
\vs Joh 12:5 Для чего бы не продать это миро за триста динариев и не раздать нищим?
\vs Joh 12:6 Сказал же он это не потому, чтобы заботился о нищих, но потому что был вор. Он имел \bibemph{при себе денежный} ящик и носил, что туда опускали.
\vs Joh 12:7 Иисус же сказал: оставьте ее; она сберегла это на день погребения Моего.
\vs Joh 12:8 Ибо нищих всегда имеете с собою, а Меня не всегда.
\vs Joh 12:9 Многие из Иудеев узнали, что Он там, и пришли не только для Иисуса, но чтобы видеть и Лазаря, которого Он воскресил из мертвых.
\vs Joh 12:10 Первосвященники же положили убить и Лазаря,
\vs Joh 12:11 потому что ради него многие из Иудеев приходили и веровали в Иисуса.
\vs Joh 12:12 На другой день множество народа, пришедшего на праздник, услышав, что Иисус идет в Иерусалим,
\rsbpar\vs Joh 12:13 взяли пальмовые ветви, вышли навстречу Ему и восклицали: осанна! благословен грядущий во имя Господне, Царь Израилев!
\vs Joh 12:14 Иисус же, найдя молодого осла, сел на него, как написано:
\vs Joh 12:15 Не бойся, дщерь Сионова! се, Царь твой грядет, сидя на молодом осле.
\vs Joh 12:16 Ученики Его сперва не поняли этого; но когда прославился Иисус, тогда вспомнили, что т\acc{а}к было о Нем написано, и это сделали Ему.
\vs Joh 12:17 Народ, бывший с Ним прежде, свидетельствовал, что Он вызвал из гроба Лазаря и воскресил его из мертвых.
\vs Joh 12:18 Потому и встретил Его народ, ибо слышал, что Он сотворил это чудо.
\vs Joh 12:19 Фарисеи же говорили между собою: видите ли, что не успеваете ничего? весь мир идет за Ним.
\rsbpar\vs Joh 12:20 Из пришедших на поклонение в праздник были некоторые Еллины.
\vs Joh 12:21 Они подошли к Филиппу, который был из Вифсаиды Галилейской, и просили его, говоря: господин! нам хочется видеть Иисуса.
\vs Joh 12:22 Филипп идет и говорит о том Андрею; и потом Андрей и Филипп сказывают о том Иисусу.
\vs Joh 12:23 Иисус же сказал им в ответ: пришел час прославиться Сыну Человеческому.
\vs Joh 12:24 Истинно, истинно говорю вам: если пшеничное зерно, пав в землю, не умрет, то останется одно; а если умрет, то принесет много плода.
\vs Joh 12:25 Любящий душу свою погубит ее; а ненавидящий душу свою в мире сем сохранит ее в жизнь вечную.
\vs Joh 12:26 Кто Мне служит, Мне да последует; и где Я, там и слуга Мой будет. И кто Мне служит, того почтит Отец Мой.
\vs Joh 12:27 Душа Моя теперь возмутилась; и что Мне сказать? Отче! избавь Меня от часа сего! Но на сей час Я и пришел.
\rsbpar\vs Joh 12:28 Отче! прославь имя Твое. Тогда пришел с неба глас: и прославил и еще прославлю.
\vs Joh 12:29 Народ, стоявший и слышавший \bibemph{то}, говорил: это гром; а другие говорили: Ангел говорил Ему.
\vs Joh 12:30 Иисус на это сказал: не для Меня был глас сей, но для народа.
\vs Joh 12:31 Ныне суд миру сему; ныне князь мира сего изгнан будет вон.
\vs Joh 12:32 И когда Я вознесен буду от земли, всех привлеку к Себе.
\vs Joh 12:33 Сие говорил Он, давая разуметь, какою смертью Он умрет.
\vs Joh 12:34 Народ отвечал Ему: мы слышали из закона, что Христос пребывает вовек; как же Ты говоришь, что должно вознесену быть Сыну Человеческому? кто Этот Сын Человеческий?
\vs Joh 12:35 Тогда Иисус сказал им: еще на малое время свет есть с вами; ходите, пока есть свет, чтобы не объяла вас тьма: а ходящий во тьме не знает, куда идет.
\vs Joh 12:36 Доколе свет с вами, веруйте в свет, да будете сынами света. Сказав это, Иисус отошел и скрылся от них.
\vs Joh 12:37 Столько чудес сотворил Он пред ними, и они не веровали в Него,
\vs Joh 12:38 да сбудется слово Исаии пророка: Господи! кто поверил слышанному от нас? и кому открылась мышца Господня?
\vs Joh 12:39 Потому не могли они веровать, что, как еще сказал Исаия,
\vs Joh 12:40 народ сей ослепил глаза свои и окаменил сердце свое, да не видят глазами, и не уразумеют сердцем, и не обратятся, чтобы Я исцелил их.
\vs Joh 12:41 Сие сказал Исаия, когда видел славу Его и говорил о Нем.
\vs Joh 12:42 Впрочем и из начальников многие уверовали в Него; но ради фарисеев не исповедовали, чтобы не быть отлученными от синагоги,
\vs Joh 12:43 ибо возлюбили больше славу человеческую, нежели славу Божию.
\vs Joh 12:44 Иисус же возгласил и сказал: верующий в Меня не в Меня верует, но в Пославшего Меня.
\vs Joh 12:45 И видящий Меня видит Пославшего Меня.
\vs Joh 12:46 Я свет пришел в мир, чтобы всякий верующий в Меня не оставался во тьме.
\vs Joh 12:47 И если кто услышит Мои слова и не поверит, Я не сужу его, ибо Я пришел не судить мир, но спасти мир.
\vs Joh 12:48 Отвергающий Меня и не принимающий слов Моих имеет судью себе: слово, которое Я говорил, оно будет судить его в последний день.
\vs Joh 12:49 Ибо Я говорил не от Себя; но пославший Меня Отец, Он дал Мне заповедь, что сказать и что говорить.
\vs Joh 12:50 И Я знаю, что заповедь Его есть жизнь вечная. Итак, что Я говорю, говорю, как сказал Мне Отец.
\vs Joh 13:1 Перед праздником Пасхи Иисус, зная, что пришел час Его перейти от мира сего к Отцу, \bibemph{явил делом, что}, возлюбив Своих сущих в мире, до конца возлюбил их.
\vs Joh 13:2 И во время вечери, когда диавол уже вложил в сердце Иуде Симонову Искариоту предать Его,
\vs Joh 13:3 Иисус, зная, что Отец все отдал в руки Его, и что Он от Бога исшел и к Богу отходит,
\vs Joh 13:4 встал с вечери, снял \bibemph{с Себя верхнюю} одежду и, взяв полотенце, препоясался.
\vs Joh 13:5 Потом влил воды в умывальницу и начал умывать ноги ученикам и отирать полотенцем, которым был препоясан.
\vs Joh 13:6 Подходит к Симону Петру, и тот говорит Ему: Господи! Тебе ли умывать мои ноги?
\vs Joh 13:7 Иисус сказал ему в ответ: что Я делаю, теперь ты не знаешь, а уразумеешь после.
\vs Joh 13:8 Петр говорит Ему: не умоешь ног моих вовек. Иисус отвечал ему: если не умою тебя, не имеешь части со Мною.
\vs Joh 13:9 Симон Петр говорит Ему: Господи! не только ноги мои, но и руки и голову.
\vs Joh 13:10 Иисус говорит ему: омытому нужно только ноги умыть, потому что чист весь; и вы чисты, но не все.
\vs Joh 13:11 Ибо знал Он предателя Своего, потому \bibemph{и} сказал: не все вы чисты.
\vs Joh 13:12 Когда же умыл им ноги и надел одежду Свою, то, возлегши опять, сказал им: знаете ли, что Я сделал вам?
\vs Joh 13:13 Вы называете Меня Учителем и Господом, и правильно говорите, ибо Я точно то.
\vs Joh 13:14 Итак, если Я, Господь и Учитель, умыл ноги вам, то и вы должны умывать ноги друг другу.
\vs Joh 13:15 Ибо Я дал вам пример, чтобы и вы делали то же, что Я сделал вам.
\vs Joh 13:16 Истинно, истинно говорю вам: раб не больше господина своего, и посланник не больше пославшего его.
\vs Joh 13:17 Если это знаете, блаженны вы, когда исполняете.
\vs Joh 13:18 Не о всех вас говорю; Я знаю, которых избрал. Но да сбудется Писание: ядущий со Мною хлеб поднял на Меня пяту свою.
\vs Joh 13:19 Теперь сказываю вам, прежде нежели \bibemph{то} сбылось, дабы, когда сбудется, вы поверили, что это Я.
\vs Joh 13:20 Истинно, истинно говорю вам: принимающий того, кого Я пошлю, Меня принимает; а принимающий Меня принимает Пославшего Меня.
\vs Joh 13:21 Сказав это, Иисус возмутился духом, и засвидетельствовал, и сказал: истинно, истинно говорю вам, что один из вас предаст Меня.
\vs Joh 13:22 Тогда ученики озирались друг на друга, недоумевая, о ком Он говорит.
\vs Joh 13:23 Один же из учеников Его, которого любил Иисус, возлежал у груди Иисуса.
\vs Joh 13:24 Ему Симон Петр сделал знак, чтобы спросил, кто это, о котором говорит.
\vs Joh 13:25 Он, припав к груди Иисуса, сказал Ему: Господи! кто это?
\vs Joh 13:26 Иисус отвечал: тот, кому Я, обмакнув кусок хлеба, подам. И, обмакнув кусок, подал Иуде Симонову Искариоту.
\vs Joh 13:27 И после сего куска вошел в него сатана. Тогда Иисус сказал ему: что делаешь, делай скорее.
\vs Joh 13:28 Но никто из возлежавших не понял, к чему Он это сказал ему.
\vs Joh 13:29 А как у Иуды был ящик, то некоторые думали, что Иисус говорит ему: купи, что нам нужно к празднику, или чтобы дал что-нибудь нищим.
\vs Joh 13:30 Он, приняв кусок, тотчас вышел; а была ночь.
\rsbpar\vs Joh 13:31 Когда он вышел, Иисус сказал: ныне прославился Сын Человеческий, и Бог прославился в Нем.
\vs Joh 13:32 Если Бог прославился в Нем, то и Бог прославит Его в Себе, и вскоре прославит Его.
\vs Joh 13:33 Дети! недолго уже быть Мне с вами. Будете искать Меня, и, как сказал Я Иудеям, что, куда Я иду, вы не можете прийти, \bibemph{так} и вам говорю теперь.
\vs Joh 13:34 Заповедь новую даю вам, да любите друг друга; как Я возлюбил вас, \bibemph{так} и вы да любите друг друга.
\vs Joh 13:35 По тому узнают все, что вы Мои ученики, если будете иметь любовь между собою.
\vs Joh 13:36 Симон Петр сказал Ему: Господи! куда Ты идешь? Иисус отвечал ему: куда Я иду, ты не можешь теперь за Мною идти, а после пойдешь за Мною.
\vs Joh 13:37 Петр сказал Ему: Господи! почему я не могу идти за Тобою теперь? я душу мою положу за Тебя.
\vs Joh 13:38 Иисус отвечал ему: душу твою за Меня положишь? истинно, истинно говорю тебе: не пропоет петух, как отречешься от Меня трижды.
\vs Joh 14:1 Да не смущается сердце ваше; веруйте в Бога, и в Меня веруйте.
\vs Joh 14:2 В доме Отца Моего обителей много. А если бы не так, Я сказал бы вам: Я иду приготовить место вам.
\vs Joh 14:3 И когда пойду и приготовлю вам место, приду опять и возьму вас к Себе, чтобы и вы были, где Я.
\vs Joh 14:4 А куда Я иду, вы знаете, и путь знаете.
\vs Joh 14:5 Фома сказал Ему: Господи! не знаем, куда идешь; и как можем знать путь?
\vs Joh 14:6 Иисус сказал ему: Я есмь путь и истина и жизнь; никто не приходит к Отцу, как только через Меня.
\vs Joh 14:7 Если бы вы знали Меня, то знали бы и Отца Моего. И отныне знаете Его и видели Его.
\vs Joh 14:8 Филипп сказал Ему: Господи! покажи нам Отца, и довольно для нас.
\vs Joh 14:9 Иисус сказал ему: столько времени Я с вами, и ты не знаешь Меня, Филипп? Видевший Меня видел Отца; как же ты говоришь, покажи нам Отца?
\vs Joh 14:10 Разве ты не веришь, что Я в Отце и Отец во Мне? Слова, которые говорю Я вам, говорю не от Себя; Отец, пребывающий во Мне, Он творит дела.
\vs Joh 14:11 Верьте Мне, что Я в Отце и Отец во Мне; а если не так, то верьте Мне по самым делам.
\vs Joh 14:12 Истинно, истинно говорю вам: верующий в Меня, дела, которые творю Я, и он сотворит, и больше сих сотворит, потому что Я к Отцу Моему иду.
\vs Joh 14:13 И если чего попросите у Отца во имя Мое, то сделаю, да прославится Отец в Сыне.
\vs Joh 14:14 Если чего попросите во имя Мое, Я то сделаю.
\vs Joh 14:15 Если любите Меня, соблюдите Мои заповеди.
\vs Joh 14:16 И Я умолю Отца, и даст вам другого Утешителя, да пребудет с вами вовек,
\vs Joh 14:17 Духа истины, Которого мир не может принять, потому что не видит Его и не знает Его; а вы знаете Его, ибо Он с вами пребывает и в вас будет.
\vs Joh 14:18 Не оставлю вас сиротами; приду к вам.
\vs Joh 14:19 Еще немного, и мир уже не увидит Меня; а вы увидите Меня, ибо Я живу, и вы будете жить.
\vs Joh 14:20 В тот день узнаете вы, что Я в Отце Моем, и вы во Мне, и Я в вас.
\vs Joh 14:21 Кто имеет заповеди Мои и соблюдает их, тот любит Меня; а кто любит Меня, тот возлюблен будет Отцем Моим; и Я возлюблю его и явлюсь ему Сам.
\vs Joh 14:22 Иуда~--- не Искариот~--- говорит Ему: Господи! что это, что Ты хочешь явить Себя нам, а не миру?
\vs Joh 14:23 Иисус сказал ему в ответ: кто любит Меня, тот соблюдет слово Мое; и Отец Мой возлюбит его, и Мы придем к нему и обитель у него сотворим.
\vs Joh 14:24 Нелюбящий Меня не соблюдает слов Моих; слово же, которое вы слышите, не есть Мое, но пославшего Меня Отца.
\vs Joh 14:25 Сие сказал Я вам, находясь с вами.
\vs Joh 14:26 Утешитель же, Дух Святый, Которого пошлет Отец во имя Мое, научит вас всему и напомнит вам все, что Я говорил вам.
\vs Joh 14:27 Мир оставляю вам, мир Мой даю вам; не так, как мир дает, Я даю вам. Да не смущается сердце ваше и да не устрашается.
\vs Joh 14:28 Вы слышали, что Я сказал вам: иду от вас и приду к вам. Если бы вы любили Меня, то возрадовались бы, что Я сказал: иду к Отцу; ибо Отец Мой более Меня.
\vs Joh 14:29 И вот, Я сказал вам \bibemph{о том}, прежде нежели сбылось, дабы вы поверили, когда сбудется.
\vs Joh 14:30 Уже немного Мне говорить с вами; ибо идет князь мира сего, и во Мне не имеет ничего.
\vs Joh 14:31 Но чтобы мир знал, что Я люблю Отца и, как заповедал Мне Отец, так и творю: встаньте, пойдем отсюда.
\vs Joh 15:1 Я есмь истинная виноградная лоза, а Отец Мой~--- виноградарь.
\vs Joh 15:2 Всякую у Меня ветвь, не приносящую плода, Он отсекает; и всякую, приносящую плод, очищает, чтобы более принесла плода.
\vs Joh 15:3 Вы уже очищены через слово, которое Я проповедал вам.
\vs Joh 15:4 Пребудьте во Мне, и Я в вас. Как ветвь не может приносить плода сама собою, если не будет на лозе: так и вы, если не будете во Мне.
\vs Joh 15:5 Я есмь лоза, а вы ветви; кто пребывает во Мне, и Я в нем, тот приносит много плода; ибо без Меня не можете делать ничего.
\vs Joh 15:6 Кто не пребудет во Мне, извергнется вон, как ветвь, и засохнет; а такие \bibemph{ветви} собирают и бросают в огонь, и они сгорают.
\vs Joh 15:7 Если пребудете во Мне и слова Мои в вас пребудут, то, чего ни пожелаете, прос\acc{и}те, и будет вам.
\vs Joh 15:8 Тем прославится Отец Мой, если вы принесете много плода и будете Моими учениками.
\vs Joh 15:9 Как возлюбил Меня Отец, и Я возлюбил вас; пребудьте в любви Моей.
\vs Joh 15:10 Если заповеди Мои соблюдете, пребудете в любви Моей, как и Я соблюл заповеди Отца Моего и пребываю в Его любви.
\vs Joh 15:11 Сие сказал Я вам, да радость Моя в вас пребудет и радость ваша будет совершенна.
\vs Joh 15:12 Сия есть заповедь Моя, да любите друг друга, как Я возлюбил вас.
\vs Joh 15:13 Нет больше той любви, как если кто положит душу свою за друзей своих.
\vs Joh 15:14 Вы друзья Мои, если исполняете то, что Я заповедую вам.
\vs Joh 15:15 Я уже не называю вас рабами, ибо раб не знает, что делает господин его; но Я назвал вас друзьями, потому что сказал вам все, что слышал от Отца Моего.
\vs Joh 15:16 Не вы Меня избрали, а Я вас избрал и поставил вас, чтобы вы шли и приносили плод, и чтобы плод ваш пребывал, дабы, чего ни попросите от Отца во имя Мое, Он дал вам.
\vs Joh 15:17 Сие заповедаю вам, да любите друг друга.
\vs Joh 15:18 Если мир вас ненавидит, знайте, что Меня прежде вас возненавидел.
\vs Joh 15:19 Если бы вы были от мира, то мир любил бы свое; а как вы не от мира, но Я избрал вас от мира, потому ненавидит вас мир.
\vs Joh 15:20 Помните слово, которое Я сказал вам: раб не больше господина своего. Если Меня гнали, будут гнать и вас; если Мое слово соблюдали, будут соблюдать и ваше.
\vs Joh 15:21 Но все то сделают вам за имя Мое, потому что не знают Пославшего Меня.
\vs Joh 15:22 Если бы Я не пришел и не говорил им, то не имели бы греха; а теперь не имеют извинения во грехе своем.
\vs Joh 15:23 Ненавидящий Меня ненавидит и Отца Моего.
\vs Joh 15:24 Если бы Я не сотворил между ними дел, каких никто другой не делал, то не имели бы греха; а теперь и видели, и возненавидели и Меня и Отца Моего.
\vs Joh 15:25 Но да сбудется слово, написанное в законе их: возненавидели Меня напрасно.
\vs Joh 15:26 Когда же приидет Утешитель, Которого Я пошлю вам от Отца, Дух истины, Который от Отца исходит, Он будет свидетельствовать о Мне;
\vs Joh 15:27 а также и вы будете свидетельствовать, потому что вы сначала со Мною.
\vs Joh 16:1 Сие сказал Я вам, чтобы вы не соблазнились.
\vs Joh 16:2 Изгонят вас из синагог; даже наступает время, когда всякий, убивающий вас, будет думать, что он тем служит Богу.
\vs Joh 16:3 Так будут поступать, потому что не познали ни Отца, ни Меня.
\vs Joh 16:4 Но Я сказал вам сие для того, чтобы вы, когда придет то время, вспомнили, что Я сказывал вам о том; не говорил же сего вам сначала, потому что был с вами.
\vs Joh 16:5 А теперь иду к Пославшему Меня, и никто из вас не спрашивает Меня: куда идешь?
\vs Joh 16:6 Но оттого, что Я сказал вам это, печалью исполнилось сердце ваше.
\vs Joh 16:7 Но Я истину говорю вам: лучше для вас, чтобы Я пошел; ибо, если Я не пойду, Утешитель не приидет к вам; а если пойду, то пошлю Его к вам,
\vs Joh 16:8 и Он, придя, обличит мир о грехе и о правде и о суде:
\vs Joh 16:9 о грехе, что не веруют в Меня;
\vs Joh 16:10 о правде, что Я иду к Отцу Моему, и уже не увидите Меня;
\vs Joh 16:11 о суде же, что князь мира сего осужден.
\vs Joh 16:12 Еще многое имею сказать вам; но вы теперь не можете вместить.
\vs Joh 16:13 Когда же приидет Он, Дух истины, то наставит вас на всякую истину: ибо не от Себя говорить будет, но будет говорить, что услышит, и будущее возвестит вам.
\vs Joh 16:14 Он прославит Меня, потому что от Моего возьмет и возвестит вам.
\vs Joh 16:15 Все, что имеет Отец, есть Мое; потому Я сказал, что от Моего возьмет и возвестит вам.
\vs Joh 16:16 Вскоре вы не увидите Меня, и опять вскоре увидите Меня, ибо Я иду к Отцу.
\vs Joh 16:17 Тут \bibemph{некоторые} из учеников Его сказали один другому: что это Он говорит нам: вскоре не увидите Меня, и опять вскоре увидите Меня, и: Я иду к Отцу?
\vs Joh 16:18 Итак они говорили: что это говорит Он: <<вскоре>>? Не знаем, что говорит.
\vs Joh 16:19 Иисус, уразумев, что хотят спросить Его, сказал им: о том ли спрашиваете вы один другого, что Я сказал: вскоре не увидите Меня, и опять вскоре увидите Меня?
\vs Joh 16:20 Истинно, истинно говорю вам: вы восплачете и возрыдаете, а мир возрадуется; вы печальны будете, но печаль ваша в радость будет.
\vs Joh 16:21 Женщина, когда рождает, терпит скорбь, потому что пришел час ее; но когда родит младенца, уже не помнит скорби от радости, потому что родился человек в мир.
\vs Joh 16:22 Так и вы теперь имеете печаль; но Я увижу вас опять, и возрадуется сердце ваше, и радости вашей никто не отнимет у вас;
\vs Joh 16:23 и в тот день вы не спросите Меня ни о чем. Истинно, истинно говорю вам: о чем ни попросите Отца во имя Мое, даст вам.
\vs Joh 16:24 Доныне вы ничего не просили во имя Мое; прос\acc{и}те, и пол\acc{у}чите, чтобы радость ваша была совершенна.
\vs Joh 16:25 Доселе Я говорил вам притчами; но наступает время, когда уже не буду говорить вам притчами, но прямо возвещу вам об Отце.
\vs Joh 16:26 В тот день будете просить во имя Мое, и не говорю вам, что Я буду просить Отца о вас:
\vs Joh 16:27 ибо Сам Отец любит вас, потому что вы возлюбили Меня и уверовали, что Я исшел от Бога.
\vs Joh 16:28 Я исшел от Отца и пришел в мир; и опять оставляю мир и иду к Отцу.
\vs Joh 16:29 Ученики Его сказали Ему: вот, теперь Ты прямо говоришь, и притчи не говоришь никакой.
\vs Joh 16:30 Теперь видим, что Ты знаешь все и не имеешь нужды, чтобы кто спрашивал Тебя. Посему веруем, что Ты от Бога исшел.
\vs Joh 16:31 Иисус отвечал им: теперь веруете?
\vs Joh 16:32 Вот, наступает час, и настал уже, что вы рассеетесь каждый в свою \bibemph{сторону} и Меня оставите одного; но Я не один, потому что Отец со Мною.
\vs Joh 16:33 Сие сказал Я вам, чтобы вы имели во Мне мир. В мире будете иметь скорбь; но мужайтесь: Я победил мир.
\vs Joh 17:1 После сих слов Иисус возвел очи Свои на небо и сказал: Отче! пришел час, прославь Сына Твоего, да и Сын Твой прославит Тебя,
\vs Joh 17:2 так как Ты дал Ему власть над всякою плотью, да всему, что Ты дал Ему, даст Он жизнь вечную.
\vs Joh 17:3 Сия же есть жизнь вечная, да знают Тебя, единого истинного Бога, и посланного Тобою Иисуса Христа.
\vs Joh 17:4 Я прославил Тебя на земле, совершил дело, которое Ты поручил Мне исполнить.
\vs Joh 17:5 И ныне прославь Меня Ты, Отче, у Тебя Самого славою, которую Я имел у Тебя прежде бытия мира.
\vs Joh 17:6 Я открыл имя Твое человекам, которых Ты дал Мне от мира; они были Твои, и Ты дал их Мне, и они сохранили слово Твое.
\vs Joh 17:7 Ныне уразумели они, что все, что Ты дал Мне, от Тебя есть,
\vs Joh 17:8 ибо слова, которые Ты дал Мне, Я передал им, и они приняли, и уразумели истинно, что Я исшел от Тебя, и уверовали, что Ты послал Меня.
\vs Joh 17:9 Я о них молю: не о всем мире молю, но о тех, которых Ты дал Мне, потому что они Твои.
\vs Joh 17:10 И все Мое Твое, и Твое Мое; и Я прославился в них.
\vs Joh 17:11 Я уже не в мире, но они в мире, а Я к Тебе иду. Отче Святый! соблюди их во имя Твое, \bibemph{тех}, которых Ты Мне дал, чтобы они были едино, как и Мы.
\vs Joh 17:12 Когда Я был с ними в мире, Я соблюдал их во имя Твое; тех, которых Ты дал Мне, Я сохранил, и никто из них не погиб, кроме сына погибели, да сбудется Писание.
\vs Joh 17:13 Ныне же к Тебе иду, и сие говорю в мире, чтобы они имели в себе радость Мою совершенную.
\vs Joh 17:14 Я передал им слово Твое; и мир возненавидел их, потому что они не от мира, как и Я не от мира.
\vs Joh 17:15 Не молю, чтобы Ты взял их из мира, но чтобы сохранил их от зла.
\vs Joh 17:16 Они не от мира, как и Я не от мира.
\vs Joh 17:17 Освяти их истиною Твоею; слово Твое есть истина.
\vs Joh 17:18 Как Ты послал Меня в мир, \bibemph{так} и Я послал их в мир.
\vs Joh 17:19 И за них Я посвящаю Себя, чтобы и они были освящены истиною.
\vs Joh 17:20 Не о них же только молю, но и о верующих в Меня по слову их,
\vs Joh 17:21 да будут все едино, как Ты, Отче, во Мне, и Я в Тебе, \bibemph{так} и они да будут в Нас едино,~--- да уверует мир, что Ты послал Меня.
\vs Joh 17:22 И славу, которую Ты дал Мне, Я дал им: да будут едино, как Мы едино.
\vs Joh 17:23 Я в них, и Ты во Мне; да будут совершен\acc{ы} воедино, и да познает мир, что Ты послал Меня и возлюбил их, как возлюбил Меня.
\vs Joh 17:24 Отче! которых Ты дал Мне, хочу, чтобы там, где Я, и они были со Мною, да видят славу Мою, которую Ты дал Мне, потому что возлюбил Меня прежде основания мира.
\vs Joh 17:25 Отче праведный! и мир Тебя не познал; а Я познал Тебя, и сии познали, что Ты послал Меня.
\vs Joh 17:26 И Я открыл им имя Твое и открою, да любовь, которою Ты возлюбил Меня, в них будет, и Я в них.
\vs Joh 18:1 Сказав сие, Иисус вышел с учениками Своими за поток Кедрон, где был сад, в который вошел Сам и ученики Его.
\vs Joh 18:2 Знал же это место и Иуда, предатель Его, потому что Иисус часто собирался там с учениками Своими.
\vs Joh 18:3 Итак Иуда, взяв отряд \bibemph{воинов} и служителей от первосвященников и фарисеев, приходит туда с фонарями и светильниками и оружием.
\vs Joh 18:4 Иисус же, зная все, что с Ним будет, вышел и сказал им: кого ищете?
\vs Joh 18:5 Ему отвечали: Иисуса Назорея. Иисус говорит им: это Я. Стоял же с ними и Иуда, предатель Его.
\vs Joh 18:6 И когда сказал им: это Я, они отступили назад и пали на землю.
\vs Joh 18:7 Опять спросил их: кого ищете? Они сказали: Иисуса Назорея.
\vs Joh 18:8 Иисус отвечал: Я сказал вам, что это Я; итак, если Меня ищете, оставьте их, пусть идут,
\vs Joh 18:9 да сбудется слово, реченное Им: из тех, которых Ты Мне дал, Я не погубил никого.
\vs Joh 18:10 Симон же Петр, имея меч, извлек его, и ударил первосвященнического раба, и отсек ему правое ухо. Имя рабу было Малх.
\vs Joh 18:11 Но Иисус сказал Петру: вложи меч в ножны; неужели Мне не пить чаши, которую дал Мне Отец?
\vs Joh 18:12 Тогда воины и тысяченачальник и служители Иудейские взяли Иисуса и связали Его,
\rsbpar\vs Joh 18:13 и отвели Его сперва к Анне, ибо он был тесть Каиафе, который был на тот год первосвященником.
\vs Joh 18:14 Это был Каиафа, который подал совет Иудеям, что лучше одному человеку умереть за народ.
\rsbpar\vs Joh 18:15 За Иисусом следовали Симон Петр и другой ученик; ученик же сей был знаком первосвященнику и вошел с Иисусом во двор первосвященнический.
\vs Joh 18:16 А Петр стоял вне за дверями. Потом другой ученик, который был знаком первосвященнику, вышел, и сказал придвернице, и ввел Петра.
\vs Joh 18:17 Тут раба придверница говорит Петру: и ты не из учеников ли Этого Человека? Он сказал: нет.
\vs Joh 18:18 Между тем рабы и служители, разведя огонь, потому что было холодно, стояли и грелись. Петр также стоял с ними и грелся.
\rsbpar\vs Joh 18:19 Первосвященник же спросил Иисуса об учениках Его и об учении Его.
\vs Joh 18:20 Иисус отвечал ему: Я говорил явно миру; Я всегда учил в синагоге и в храме, где всегда Иудеи сходятся, и тайно не говорил ничего.
\vs Joh 18:21 Что спрашиваешь Меня? спроси слышавших, что Я говорил им; вот, они знают, что Я говорил.
\vs Joh 18:22 Когда Он сказал это, один из служителей, стоявший близко, ударил Иисуса по щеке, сказав: так отвечаешь Ты первосвященнику?
\vs Joh 18:23 Иисус отвечал ему: если Я сказал худо, покажи, что худо; а если хорошо, что ты бьешь Меня?
\vs Joh 18:24 Анна послал Его связанного к первосвященнику Каиафе.
\rsbpar\vs Joh 18:25 Симон же Петр стоял и грелся. Тут сказали ему: не из учеников ли Его и ты? Он отрекся и сказал: нет.
\vs Joh 18:26 Один из рабов первосвященнических, родственник тому, которому Петр отсек ухо, говорит: не я ли видел тебя с Ним в саду?
\vs Joh 18:27 Петр опять отрекся; и тотчас запел петух.
\rsbpar\vs Joh 18:28 От Каиафы повели Иисуса в преторию. Было утро; и они не вошли в преторию, чтобы не оскверниться, но чтобы \bibemph{можно было} есть пасху.
\vs Joh 18:29 Пилат вышел к ним и сказал: в чем вы обвиняете Человека Сего?
\vs Joh 18:30 Они сказали ему в ответ: если бы Он не был злодей, мы не предали бы Его тебе.
\vs Joh 18:31 Пилат сказал им: возьмите Его вы, и по закону вашему судите Его. Иудеи сказали ему: нам не позволено предавать смерти никого,~---
\rsbpar\vs Joh 18:32 да сбудется слово Иисусово, которое сказал Он, давая разуметь, какою смертью Он умрет.
\vs Joh 18:33 Тогда Пилат опять вошел в преторию, и призвал Иисуса, и сказал Ему: Ты Царь Иудейский?
\vs Joh 18:34 Иисус отвечал ему: от себя ли ты говоришь это, или другие сказали тебе о Мне?
\vs Joh 18:35 Пилат отвечал: разве я Иудей? Твой народ и первосвященники предали Тебя мне; что Ты сделал?
\vs Joh 18:36 Иисус отвечал: Царство Мое не от мира сего; если бы от мира сего было Царство Мое, то служители Мои подвизались бы за Меня, чтобы Я не был предан Иудеям; но ныне Царство Мое не отсюда.
\vs Joh 18:37 Пилат сказал Ему: итак Ты Царь? Иисус отвечал: ты говоришь, что Я Царь. Я на то родился и на то пришел в мир, чтобы свидетельствовать о истине; всякий, кто от истины, слушает гласа Моего.
\vs Joh 18:38 Пилат сказал Ему: что есть истина? И, сказав это, опять вышел к Иудеям и сказал им: я никакой вины не нахожу в Нем.
\vs Joh 18:39 Есть же у вас обычай, чтобы я одного отпускал вам на Пасху; хотите ли, отпущу вам Царя Иудейского?
\vs Joh 18:40 Тогда опять закричали все, говоря: не Его, но Варавву. Варавва же был разбойник.
\vs Joh 19:1 Тогда Пилат взял Иисуса и \bibemph{велел} бить Его.
\vs Joh 19:2 И воины, сплетши венец из терна, возложили Ему на голову, и одели Его в багряницу,
\vs Joh 19:3 и говорили: радуйся, Царь Иудейский! и били Его по ланитам.
\vs Joh 19:4 Пилат опять вышел и сказал им: вот, я вывожу Его к вам, чтобы вы знали, что я не нахожу в Нем никакой вины.
\vs Joh 19:5 Тогда вышел Иисус в терновом венце и в багрянице. И сказал им \bibemph{Пилат}: се, Человек!
\vs Joh 19:6 Когда же увидели Его первосвященники и служители, то закричали: распни, распни Его! Пилат говорит им: возьмите Его вы, и распните; ибо я не нахожу в Нем вины.
\vs Joh 19:7 Иудеи отвечали ему: мы имеем закон, и по закону нашему Он должен умереть, потому что сделал Себя Сыном Божиим.
\vs Joh 19:8 Пилат, услышав это слово, больше убоялся.
\vs Joh 19:9 И опять вошел в преторию и сказал Иисусу: откуда Ты? Но Иисус не дал ему ответа.
\vs Joh 19:10 Пилат говорит Ему: мне ли не отвечаешь? не знаешь ли, что я имею власть распять Тебя и власть имею отпустить Тебя?
\vs Joh 19:11 Иисус отвечал: ты не имел бы надо Мною никакой власти, если бы не было дано тебе свыше; посему более греха на том, кто предал Меня тебе.
\vs Joh 19:12 С этого \bibemph{времени} Пилат искал отпустить Его. Иудеи же кричали: если отпустишь Его, ты не друг кесарю; всякий, делающий себя царем, противник кесарю.
\vs Joh 19:13 Пилат, услышав это слово, вывел вон Иисуса и сел на судилище, на месте, называемом Лиф\acc{о}стротон\fns{Каменный помост.}, а по-еврейски Гаввафа.
\vs Joh 19:14 Тогда была пятница перед Пасхою, и час шестый. И сказал \bibemph{Пилат} Иудеям: се, Царь ваш!
\vs Joh 19:15 Но они закричали: возьми, возьми, распни Его! Пилат говорит им: Царя ли вашего распну? Первосвященники отвечали: нет у нас царя, кроме кесаря.
\rsbpar\vs Joh 19:16 Тогда наконец он предал Его им на распятие. И взяли Иисуса и повели.
\vs Joh 19:17 И, неся крест Свой, Он вышел на место, называемое Лобное, по-еврейски Голгофа;
\vs Joh 19:18 там распяли Его и с Ним двух других, по ту и по другую сторону, а посреди Иисуса.
\vs Joh 19:19 Пилат же написал и надпись, и поставил на кресте. Написано было: Иисус Назорей, Царь Иудейский.
\vs Joh 19:20 Эту надпись читали многие из Иудеев, потому что место, где был распят Иисус, было недалеко от города, и написано было по-еврейски, по-гречески, по-римски.
\vs Joh 19:21 Первосвященники же Иудейские сказали Пилату: не пиши: Царь Иудейский, но что Он говорил: Я Царь Иудейский.
\vs Joh 19:22 Пилат отвечал: что я написал, то написал.
\vs Joh 19:23 Воины же, когда распяли Иисуса, взяли одежды Его и разделили на четыре части, каждому воину по части, и хитон; хитон же был не сшитый, а весь тканый сверху.
\vs Joh 19:24 Итак сказали друг другу: не станем раздирать его, а бросим о нем жребий, чей будет,~--- да сбудется реченное в Писании: разделили ризы Мои между собою и об одежде Моей бросали жребий. Так поступили воины.
\rsbpar\vs Joh 19:25 При кресте Иисуса стояли Матерь Его и сестра Матери Его, Мария Клеопова, и Мария Магдалина.
\vs Joh 19:26 Иисус, увидев Матерь и ученика тут стоящего, которого любил, говорит Матери Своей: Ж\acc{е}но! се, сын Твой.
\vs Joh 19:27 Потом говорит ученику: се, Матерь твоя! И с этого времени ученик сей взял Ее к себе.
\rsbpar\vs Joh 19:28 После того Иисус, зная, что уже все совершилось, да сбудется Писание, говорит: жажду.
\vs Joh 19:29 Тут стоял сосуд, полный уксуса. \bibemph{Воины}, напоив уксусом губку и наложив на иссоп, поднесли к устам Его.
\vs Joh 19:30 Когда же Иисус вкусил уксуса, сказал: совершилось! И, преклонив главу, предал дух.
\rsbpar\vs Joh 19:31 Но так как \bibemph{тогда} была пятница, то Иудеи, дабы не оставить тел на кресте в субботу,~--- ибо та суббота была день великий,~--- просили Пилата, чтобы перебить у них голени и снять их.
\vs Joh 19:32 Итак пришли воины, и у первого перебили голени, и у другого, распятого с Ним.
\vs Joh 19:33 Но, придя к Иисусу, как увидели Его уже умершим, не перебили у Него голеней,
\vs Joh 19:34 но один из воинов копьем пронзил Ему ребра, и тотчас истекла кровь и вода.
\vs Joh 19:35 И видевший засвидетельствовал, и истинно свидетельство его; он знает, что говорит истину, дабы вы поверили.
\vs Joh 19:36 Ибо сие произошло, да сбудется Писание: кость Его да не сокрушится.
\vs Joh 19:37 Также и в другом \bibemph{месте} Писание говорит: воззрят на Того, Которого пронзили.
\rsbpar\vs Joh 19:38 После сего Иосиф из Аримафеи~--- ученик Иисуса, но тайный из страха от Иудеев,~--- просил Пилата, чтобы снять тело Иисуса; и Пилат позволил. Он пошел и снял тело Иисуса.
\vs Joh 19:39 Пришел также и Никодим,~--- приходивший прежде к Иисусу ночью,~--- и принес состав из смирны и алоя, литр около ста.
\vs Joh 19:40 Итак они взяли тело Иисуса и обвили его пеленами с благовониями, как обыкновенно погребают Иудеи.
\vs Joh 19:41 На том месте, где Он распят, был сад, и в саду гроб новый, в котором еще никто не был положен.
\vs Joh 19:42 Там положили Иисуса ради пятницы Иудейской, потому что гроб был близко.
\vs Joh 20:1 В первый же \bibemph{день} недели Мария Магдалина приходит ко гробу рано, когда было еще темно, и видит, что камень отвален от гроба.
\vs Joh 20:2 Итак, бежит и приходит к Симону Петру и к другому ученику, которого любил Иисус, и говорит им: унесли Господа из гроба, и не знаем, где положили Его.
\vs Joh 20:3 Тотчас вышел Петр и другой ученик, и пошли ко гробу.
\vs Joh 20:4 Они побежали оба вместе; но другой ученик бежал скорее Петра, и пришел ко гробу первый.
\vs Joh 20:5 И, наклонившись, увидел лежащие пелены; но не вошел \bibemph{во гроб}.
\vs Joh 20:6 Вслед за ним приходит Симон Петр, и входит во гроб, и видит одни пелены лежащие,
\vs Joh 20:7 и плат, который был на главе Его, не с пеленами лежащий, но особо свитый на другом месте.
\vs Joh 20:8 Тогда вошел и другой ученик, прежде пришедший ко гробу, и увидел, и уверовал.
\vs Joh 20:9 Ибо они еще не знали из Писания, что Ему надлежало воскреснуть из мертвых.
\vs Joh 20:10 Итак ученики опять возвратились к себе.
\rsbpar\vs Joh 20:11 А Мария стояла у гроба и плакала. И, когда плакала, наклонилась во гроб,
\vs Joh 20:12 и видит двух Ангелов, в белом одеянии сидящих, одного у главы и другого у ног, где лежало тело Иисуса.
\vs Joh 20:13 И они говорят ей: жена! что ты плачешь? Говорит им: унесли Господа моего, и не знаю, где положили Его.
\vs Joh 20:14 Сказав сие, обратилась назад и увидела Иисуса стоящего; но не узнала, что это Иисус.
\vs Joh 20:15 Иисус говорит ей: жена! что ты плачешь? кого ищешь? Она, думая, что это садовник, говорит Ему: господин! если ты вынес Его, скажи мне, где ты положил Его, и я возьму Его.
\vs Joh 20:16 Иисус говорит ей: Мария! Она, обратившись, говорит Ему: Раввун\acc{и}!~--- что значит: Учитель!
\vs Joh 20:17 Иисус говорит ей: не прикасайся ко Мне, ибо Я еще не восшел к Отцу Моему; а иди к братьям Моим и скажи им: восхожу к Отцу Моему и Отцу вашему, и к Богу Моему и Богу вашему.
\vs Joh 20:18 Мария Магдалина идет и возвещает ученикам, что видела Господа, и \bibemph{что} Он это сказал ей.
\rsbpar\vs Joh 20:19 В тот же первый день недели вечером, когда двери \bibemph{дома}, где собирались ученики Его, были заперты из опасения от Иудеев, пришел Иисус, и стал посреди, и говорит им: мир вам!
\vs Joh 20:20 Сказав это, Он показал им руки и ноги и ребра Свои. Ученики обрадовались, увидев Господа.
\vs Joh 20:21 Иисус же сказал им вторично: мир вам! как послал Меня Отец, \bibemph{так} и Я посылаю вас.
\vs Joh 20:22 Сказав это, дунул, и говорит им: примите Духа Святаго.
\vs Joh 20:23 Кому простите грехи, тому простятся; на ком оставите, на том останутся.
\vs Joh 20:24 Фома же, один из двенадцати, называемый Близнец, не был тут с ними, когда приходил Иисус.
\vs Joh 20:25 Другие ученики сказали ему: мы видели Господа. Но он сказал им: если не увижу на руках Его ран от гвоздей, и не вложу перста моего в раны от гвоздей, и не вложу руки моей в ребра Его, не поверю.
\rsbpar\vs Joh 20:26 После восьми дней опять были в доме ученики Его, и Фома с ними. Пришел Иисус, когда двери были заперты, стал посреди них и сказал: мир вам!
\vs Joh 20:27 Потом говорит Фоме: подай перст твой сюда и посмотри руки Мои; подай руку твою и вложи в ребра Мои; и не будь неверующим, но верующим.
\vs Joh 20:28 Фома сказал Ему в ответ: Господь мой и Бог мой!
\vs Joh 20:29 Иисус говорит ему: ты поверил, потому что увидел Меня; блаженны невидевшие и уверовавшие.
\rsbpar\vs Joh 20:30 Много сотворил Иисус пред учениками Своими и других чудес, о которых не писано в книге сей.
\vs Joh 20:31 Сие же написано, дабы вы уверовали, что Иисус есть Христос, Сын Божий, и, веруя, имели жизнь во имя Его.
\vs Joh 21:1 После того опять явился Иисус ученикам Своим при море Тивериадском. Явился же так:
\vs Joh 21:2 были вместе Симон Петр, и Фома, называемый Близнец, и Нафанаил из Каны Галилейской, и сыновья Зеведеевы, и двое других из учеников Его.
\vs Joh 21:3 Симон Петр говорит им: иду ловить рыбу. Говорят ему: идем и мы с тобою. Пошли и тотчас вошли в лодку, и не поймали в ту ночь ничего.
\vs Joh 21:4 А когда уже настало утро, Иисус стоял на берегу; но ученики не узнали, что это Иисус.
\vs Joh 21:5 Иисус говорит им: дети! есть ли у вас какая пища? Они отвечали Ему: нет.
\vs Joh 21:6 Он же сказал им: закиньте сеть по правую сторону лодки, и поймаете. Они закинули, и уже не могли вытащить \bibemph{сети} от множества рыбы.
\vs Joh 21:7 Тогда ученик, которого любил Иисус, говорит Петру: это Господь. Симон же Петр, услышав, что это Господь, опоясался одеждою,~--- ибо он был наг,~--- и бросился в море.
\vs Joh 21:8 А другие ученики приплыли в лодке,~--- ибо недалеко были от земли, локтей около двухсот,~--- таща сеть с рыбою.
\vs Joh 21:9 Когда же вышли на землю, видят разложенный огонь и на нем лежащую рыбу и хлеб.
\vs Joh 21:10 Иисус говорит им: принесите рыбы, которую вы теперь поймали.
\vs Joh 21:11 Симон Петр пошел и вытащил на землю сеть, наполненную большими рыбами, \bibemph{которых было} сто пятьдесят три; и при таком множестве не прорвалась сеть.
\vs Joh 21:12 Иисус говорит им: придите, обедайте. Из учеников же никто не смел спросить Его: кто Ты? зная, что это Господь.
\vs Joh 21:13 Иисус приходит, берет хлеб и дает им, также и рыбу.
\vs Joh 21:14 Это уже в третий раз явился Иисус ученикам Своим по воскресении Своем из мертвых.
\rsbpar\vs Joh 21:15 Когда же они обедали, Иисус говорит Симону Петру: Симон Ионин! любишь ли ты Меня больше, нежели они? \bibemph{Петр} говорит Ему: так, Господи! Ты знаешь, что я люблю Тебя. \bibemph{Иисус} говорит ему: паси агнцев Моих.
\vs Joh 21:16 Еще говорит ему в другой раз: Симон Ионин! любишь ли ты Меня? \bibemph{Петр} говорит Ему: так, Господи! Ты знаешь, что я люблю Тебя. \bibemph{Иисус} говорит ему: паси овец Моих.
\vs Joh 21:17 Говорит ему в третий раз: Симон Ионин! любишь ли ты Меня? Петр опечалился, что в третий раз спросил его: любишь ли Меня? и сказал Ему: Господи! Ты все знаешь; Ты знаешь, что я люблю Тебя. Иисус говорит ему: паси овец Моих.
\vs Joh 21:18 Истинно, истинно говорю тебе: когда ты был молод, то препоясывался сам и ходил, куда хотел; а когда состаришься, то прострешь руки твои, и другой препояшет тебя, и поведет, куда не хочешь.
\vs Joh 21:19 Сказал же это, давая разуметь, какою смертью \bibemph{Петр} прославит Бога. И, сказав сие, говорит ему: иди за Мною.
\vs Joh 21:20 Петр же, обратившись, видит идущего за ним ученика, которого любил Иисус и который на вечери, приклонившись к груди Его, сказал: Господи! кто предаст Тебя?
\vs Joh 21:21 Его увидев, Петр говорит Иисусу: Господи! а он что?
\vs Joh 21:22 Иисус говорит ему: если Я хочу, чтобы он пребыл, пока приду, что тебе \bibemph{до того}? ты иди за Мною.
\vs Joh 21:23 И пронеслось это слово между братиями, что ученик тот не умрет. Но Иисус не сказал ему, что не умрет, но: если Я хочу, чтобы он пребыл, пока приду, что тебе \bibemph{до того}?~---
\vs Joh 21:24 Сей ученик и свидетельствует о сем, и написал сие; и знаем, что истинно свидетельство его.
\vs Joh 21:25 Многое и другое сотворил Иисус; но, если бы писать о том подробно, то, думаю, и самому миру не вместить бы написанных книг. Аминь.

\bibbookdescr{Act}{
  inline={Деяния\\\LARGE святых Апостолов},
  toc={Деяния},
  bookmark={Деяния},
  header={Деяния},
  %headerleft={},
  %headerright={},
  abbr={Деян}
}
\vs Act 1:1 Первую книгу написал я \bibemph{к тебе}, Феофил, о всем, что Иисус делал и чему учил от начала
\vs Act 1:2 до того дня, в который Он вознесся, дав Святым Духом повеления Апостолам, которых Он избрал,
\vs Act 1:3 которым и явил Себя живым, по страдании Своем, со многими верными доказательствами, в продолжение сорока дней являясь им и говоря о Царствии Божием.
\rsbpar\vs Act 1:4 И, собрав их, Он повелел им: не отлучайтесь из Иерусалима, но ждите обещанного от Отца, о чем вы слышали от Меня,
\vs Act 1:5 ибо Иоанн крестил водою, а вы, через несколько дней после сего, будете крещены Духом Святым.
\vs Act 1:6 Посему они, сойдясь, спрашивали Его, говоря: не в сие ли время, Господи, восстановляешь Ты царство Израилю?
\vs Act 1:7 Он же сказал им: не ваше дело знать времена или сроки, которые Отец положил в Своей власти,
\vs Act 1:8 но вы примете силу, когда сойдет на вас Дух Святый; и будете Мне свидетелями в Иерусалиме и во всей Иудее и Самарии и даже до края земли.
\vs Act 1:9 Сказав сие, Он поднялся в глазах их, и облако взяло Его из вида их.
\vs Act 1:10 И когда они смотрели на небо, во время восхождения Его, вдруг предстали им два мужа в белой одежде
\vs Act 1:11 и сказали: мужи Галилейские! что вы стоите и смотрите на небо? Сей Иисус, вознесшийся от вас на небо, придет таким же образом, как вы видели Его восходящим на небо.
\rsbpar\vs Act 1:12 Тогда они возвратились в Иерусалим с горы, называемой Елеон, которая находится близ Иерусалима, в расстоянии субботнего пути.
\vs Act 1:13 И, придя, взошли в горницу, где и пребывали, Петр и Иаков, Иоанн и Андрей, Филипп и Фома, Варфоломей и Матфей, Иаков Алфеев и Симон Зилот, и Иуда, \bibemph{брат} Иакова.
\vs Act 1:14 Все они единодушно пребывали в молитве и молении, с \bibemph{некоторыми} женами и Мариею, Материю Иисуса, и с братьями Его.
\rsbpar\vs Act 1:15 И в те дни Петр, став посреди учеников, сказал
\vs Act 1:16 (было же собрание человек около ста двадцати): мужи братия! Надлежало исполниться тому, что в Писании предрек Дух Святый устами Давида об Иуде, бывшем вожде тех, которые взяли Иисуса;
\vs Act 1:17 он был сопричислен к нам и получил жребий служения сего;
\vs Act 1:18 но приобрел землю неправедною мздою, и когда низринулся, расселось чрево его, и выпали все внутренности его;
\vs Act 1:19 и это сделалось известно всем жителям Иерусалима, так что земля та на отечественном их наречии названа Акелдам\acc{а}, то есть земля крови.
\vs Act 1:20 В книге же Псалмов написано: да будет двор его пуст, и да не будет живущего в нем; и: достоинство его да приимет другой.
\vs Act 1:21 Итак надобно, чтобы один из тех, которые находились с нами во всё время, когда пребывал и обращался с нами Господь Иисус,
\vs Act 1:22 начиная от крещения Иоаннова до того дня, в который Он вознесся от нас, был вместе с нами свидетелем воскресения Его.
\vs Act 1:23 И поставили двоих: Иосифа, называемого Варсавою, который прозван Иустом, и Матфия;
\vs Act 1:24 и помолились и сказали: Ты, Господи, Сердцеведец всех, покажи из сих двоих одного, которого Ты избрал
\vs Act 1:25 принять жребий сего служения и Апостольства, от которого отпал Иуда, чтобы идти в свое место.
\vs Act 1:26 И бросили о них жребий, и выпал жребий Матфию, и он сопричислен к одиннадцати Апостолам.
\vs Act 2:1 При наступлении дня Пятидесятницы все они были единодушно вместе.
\vs Act 2:2 И внезапно сделался шум с неба, как бы от несущегося сильного ветра, и наполнил весь дом, где они находились.
\vs Act 2:3 И явились им разделяющиеся языки, как бы огненные, и почили по одному на каждом из них.
\vs Act 2:4 И исполнились все Духа Святаго, и начали говорить на иных языках, как Дух давал им провещевать.
\rsbpar\vs Act 2:5 В Иерусалиме же находились Иудеи, люди набожные, из всякого народа под небом.
\vs Act 2:6 Когда сделался этот шум, собрался народ, и пришел в смятение, ибо каждый слышал их говорящих его наречием.
\vs Act 2:7 И все изумлялись и дивились, говоря между собою: сии говорящие не все ли Галилеяне?
\vs Act 2:8 Как же мы слышим каждый собственное наречие, в котором родились.
\vs Act 2:9 Парфяне, и Мидяне, и Еламиты, и жители Месопотамии, Иудеи и Каппадокии, Понта и Асии,
\vs Act 2:10 Фригии и Памфилии, Египта и частей Ливии, прилежащих к Киринее, и пришедшие из Рима, Иудеи и прозелиты\fns{Обращенные из язычников.},
\vs Act 2:11 критяне и аравитяне, слышим их нашими языками говорящих о великих \bibemph{делах} Божиих?
\vs Act 2:12 И изумлялись все и, недоумевая, говорили друг другу: что это значит?
\vs Act 2:13 А иные, насмехаясь, говорили: они напились сладкого вина.
\rsbpar\vs Act 2:14 Петр же, став с одиннадцатью, возвысил голос свой и возгласил им: мужи Иудейские, и все живущие в Иерусалиме! сие да будет вам известно, и внимайте словам моим:
\vs Act 2:15 они не пьяны, как вы думаете, ибо теперь третий час дня;
\vs Act 2:16 но это есть предреченное пророком Иоилем:
\vs Act 2:17 И будет в последние дни, говорит Бог, излию от Духа Моего на всякую плоть, и будут пророчествовать сыны ваши и дочери ваши; и юноши ваши будут видеть видения, и старцы ваши сновидениями вразумляемы будут.
\vs Act 2:18 И на рабов Моих и на рабынь Моих в те дни излию от Духа Моего, и будут пророчествовать.
\vs Act 2:19 И покажу чудеса на небе вверху и знамения на земле внизу, кровь и огонь и курение дыма.
\vs Act 2:20 Солнце превратится во тьму, и луна~--- в кровь, прежде нежели наступит день Господень, великий и славный.
\vs Act 2:21 И будет: всякий, кто призовет имя Господне, спасется.
\vs Act 2:22 Мужи Израильские! выслушайте слова сии: Иисуса Назорея, Мужа, засвидетельствованного вам от Бога силами и чудесами и знамениями, которые Бог сотворил через Него среди вас, как и сами знаете,
\vs Act 2:23 Сего, по определенному совету и предведению Божию преданного, вы взяли и, пригвоздив руками беззаконных, убили;
\vs Act 2:24 но Бог воскресил Его, расторгнув узы смерти, потому что ей невозможно было удержать Его.
\vs Act 2:25 Ибо Давид говорит о Нем: видел я пред собою Господа всегда, ибо Он одесную меня, дабы я не поколебался.
\vs Act 2:26 Оттого возрадовалось сердце мое и возвеселился язык мой; даже и плоть моя упокоится в уповании,
\vs Act 2:27 ибо Ты не оставишь души моей в аде и не дашь святому Твоему увидеть тления.
\vs Act 2:28 Ты дал мне познать путь жизни, Ты исполнишь меня радостью пред лицем Твоим.
\vs Act 2:29 Мужи братия! да будет позволено с дерзновением сказать вам о праотце Давиде, что он и умер и погребен, и гроб его у нас до сего дня.
\vs Act 2:30 Будучи же пророком и зная, что Бог с клятвою обещал ему от плода чресл его воздвигнуть Христа во плоти и посадить на престоле его,
\vs Act 2:31 он прежде сказал о воскресении Христа, что не оставлена душа Его в аде, и плоть Его не видела тления.
\vs Act 2:32 Сего Иисуса Бог воскресил, чему все мы свидетели.
\vs Act 2:33 Итак Он, быв вознесен десницею Божиею и приняв от Отца обетование Святаго Духа, излил то, что вы ныне видите и слышите.
\vs Act 2:34 Ибо Давид не восшел на небеса; но сам говорит: сказал Господь Господу моему: седи одесную Меня,
\vs Act 2:35 доколе положу врагов Твоих в подножие ног Твоих.
\vs Act 2:36 Итак твердо знай, весь дом Израилев, что Бог соделал Господом и Христом Сего Иисуса, Которого вы распяли.
\rsbpar\vs Act 2:37 Услышав это, они умилились сердцем и сказали Петру и прочим Апостолам: что нам делать, мужи братия?
\vs Act 2:38 Петр же сказал им: покайтесь, и да крестится каждый из вас во имя Иисуса Христа для прощения грехов; и пол\acc{у}чите дар Святаго Духа.
\vs Act 2:39 Ибо вам принадлежит обетование и детям вашим и всем дальним, кого ни призовет Господь Бог наш.
\vs Act 2:40 И другими многими словами он свидетельствовал и увещевал, говоря: спасайтесь от рода сего развращенного.
\vs Act 2:41 Итак охотно принявшие слово его крестились, и присоединилось в тот день душ около трех тысяч.
\vs Act 2:42 И они постоянно пребывали в учении Апостолов, в общении и преломлении хлеба и в молитвах.
\vs Act 2:43 Был же страх на всякой душе; и много чудес и знамений совершилось через Апостолов в Иерусалиме.
\vs Act 2:44 Все же верующие были вместе и имели всё общее.
\vs Act 2:45 И продавали имения и всякую собственность, и разделяли всем, смотря по нужде каждого.
\vs Act 2:46 И каждый день единодушно пребывали в храме и, преломляя по домам хлеб, принимали пищу в веселии и простоте сердца,
\vs Act 2:47 хваля Бога и находясь в любви у всего народа. Господь же ежедневно прилагал спасаемых к Церкви.
\vs Act 3:1 Петр и Иоанн шли вместе в храм в час молитвы девятый.
\vs Act 3:2 И был человек, хромой от чрева матери его, которого носили и сажали каждый день при дверях храма, называемых Красными, просить милостыни у входящих в храм.
\vs Act 3:3 Он, увидев Петра и Иоанна перед входом в храм, просил у них милостыни.
\vs Act 3:4 Петр с Иоанном, всмотревшись в него, сказали: взгляни на нас.
\vs Act 3:5 И он пристально смотрел на них, надеясь получить от них что-нибудь.
\vs Act 3:6 Но Петр сказал: серебра и золота нет у меня; а что имею, то даю тебе: во имя Иисуса Христа Назорея встань и ходи.
\vs Act 3:7 И, взяв его за правую руку, поднял; и вдруг укрепились его ступни и колени,
\vs Act 3:8 и вскочив, стал, и начал ходить, и вошел с ними в храм, ходя и скача, и хваля Бога.
\vs Act 3:9 И весь народ видел его ходящим и хвалящим Бога;
\vs Act 3:10 и узнали его, что это был тот, который сидел у Красных дверей храма для милостыни; и исполнились ужаса и изумления от случившегося с ним.
\rsbpar\vs Act 3:11 И как исцеленный хромой не отходил от Петра и Иоанна, то весь народ в изумлении сбежался к ним в притвор, называемый Соломонов.
\vs Act 3:12 Увидев это, Петр сказал народу: мужи Израильские! что дивитесь сему, или что смотрите на нас, как будто бы мы своею силою или благочестием сделали то, что он ходит?
\vs Act 3:13 Бог Авраама и Исаака и Иакова, Бог отцов наших, прославил Сына Своего Иисуса, Которого вы предали и от Которого отреклись перед лицом Пилата, когда он полагал освободить Его.
\vs Act 3:14 Но вы от Святого и Праведного отреклись, и просили даровать вам человека убийцу,
\vs Act 3:15 а Начальника жизни убили. Сего Бог воскресил из мертвых, чему мы свидетели.
\vs Act 3:16 И ради веры во имя Его, имя Его укрепило сего, которого вы видите и знаете, и вера, которая от Него, даровала ему исцеление сие перед всеми вами.
\vs Act 3:17 Впрочем я знаю, братия, что вы, как и начальники ваши, сделали это по неведению;
\vs Act 3:18 Бог же, как предвозвестил устами всех Своих пророков пострадать Христу, так и исполнил.
\vs Act 3:19 Итак покайтесь и обратитесь, чтобы загладились грехи ваши,
\vs Act 3:20 да придут времена отрады от лица Господа, и да пошлет Он предназначенного вам Иисуса Христа,
\vs Act 3:21 Которого небо должно было принять до времен совершения всего, что говорил Бог устами всех святых Своих пророков от века.
\vs Act 3:22 Моисей сказал отцам: Господь Бог ваш воздвигнет вам из братьев ваших Пророка, как меня, слушайтесь Его во всем, что Он ни будет говорить вам;
\vs Act 3:23 и будет, что всякая душа, которая не послушает Пророка того, истребится из народа.
\vs Act 3:24 И все пророки, от Самуила и после него, сколько их ни говорили, также предвозвестили дни сии.
\vs Act 3:25 Вы сыны пророков и завета, который завещевал Бог отцам вашим, говоря Аврааму: и в семени твоем благословятся все племена земные.
\vs Act 3:26 Бог, воскресив Сына Своего Иисуса, к вам первым послал Его благословить вас, отвращая каждого от злых дел ваших.
\vs Act 4:1 Когда они говорили к народу, к ним приступили священники и начальники стражи при храме и саддукеи,
\vs Act 4:2 досадуя на то, что они учат народ и проповедуют в Иисусе воскресение из мертвых;
\vs Act 4:3 и наложили на них руки и отдали \bibemph{их} под стражу до утра; ибо уже был вечер.
\vs Act 4:4 Многие же из слушавших слово уверовали; и было число таковых людей около пяти тысяч.
\rsbpar\vs Act 4:5 На другой день собрались в Иерусалим начальники их и старейшины, и книжники,
\vs Act 4:6 и Анна первосвященник, и Каиафа, и Иоанн, и Александр, и прочие из рода первосвященнического;
\vs Act 4:7 и, поставив их посреди, спрашивали: какою силою или каким именем вы сделали это?
\vs Act 4:8 Тогда Петр, исполнившись Духа Святаго, сказал им: начальники народа и старейшины Израильские!
\vs Act 4:9 Если от нас сегодня требуют ответа в благодеянии человеку немощному, как он исцелен,
\vs Act 4:10 то да будет известно всем вам и всему народу Израильскому, что именем Иисуса Христа Назорея, Которого вы распяли, Которого Бог воскресил из мертвых, Им поставлен он перед вами здрав.
\vs Act 4:11 Он есть камень, пренебреженный вами зиждущими, но сделавшийся главою угла, и нет ни в ком ином спасения,
\vs Act 4:12 ибо нет другого имени под небом, данного человекам, которым надлежало бы нам спастись.
\rsbpar\vs Act 4:13 Видя смелость Петра и Иоанна и приметив, что они люди некнижные и простые, они удивлялись, между тем узнавали их, что они были с Иисусом;
\vs Act 4:14 видя же исцеленного человека, стоящего с ними, ничего не могли сказать вопреки.
\vs Act 4:15 И, приказав им выйти вон из синедриона, рассуждали между собою,
\vs Act 4:16 говоря: чт\acc{о} нам делать с этими людьми? Ибо всем, живущим в Иерусалиме, известно, что ими сделано явное чудо, и мы не можем отвергнуть \bibemph{сего};
\vs Act 4:17 но, чтобы более не разгласилось это в народе, с угрозою запретим им, чтобы не говорили об имени сем никому из людей.
\vs Act 4:18 И, призвав их, приказали им отнюдь не говорить и не учить о имени Иисуса.
\vs Act 4:19 Но Петр и Иоанн сказали им в ответ: суд\acc{и}те, справедливо ли пред Богом слушать вас более, нежели Бога?
\vs Act 4:20 Мы не можем не говорить того, что видели и слышали.
\vs Act 4:21 Они же, пригрозив, отпустили их, не находя возможности наказать их, по причине народа; потому что все прославляли Бога за происшедшее.
\vs Act 4:22 Ибо лет более сорока было тому человеку, над которым сделалось сие чудо исцеления.
\rsbpar\vs Act 4:23 Быв отпущены, они пришли к своим и пересказали, что говорили им первосвященники и старейшины.
\vs Act 4:24 Они же, выслушав, единодушно возвысили голос к Богу и сказали: Владыко Боже, сотворивший небо и землю и море и всё, что в них!
\vs Act 4:25 Ты устами отца нашего Давида, раба Твоего, сказал Духом Святым: что мятутся язычники, и народы замышляют тщетное?
\vs Act 4:26 Восстали цари земные, и князи собрались вместе на Господа и на Христа Его.
\vs Act 4:27 Ибо поистине собрались в городе сем на Святаго Сына Твоего Иисуса, помазанного Тобою, Ирод и Понтий Пилат с язычниками и народом Израильским,
\vs Act 4:28 чтобы сделать то, чему быть предопределила рука Твоя и совет Твой.
\vs Act 4:29 И ныне, Господи, воззри на угрозы их, и дай рабам Твоим со всею смелостью говорить слово Твое,
\vs Act 4:30 тогда как Ты простираешь руку Твою на исцеления и на соделание знамений и чудес именем Святаго Сына Твоего Иисуса.
\vs Act 4:31 И, по молитве их, поколебалось место, где они были собраны, и исполнились все Духа Святаго, и говорили слово Божие с дерзновением.
\rsbpar\vs Act 4:32 У множества же уверовавших было одно сердце и одна душа; и никто ничего из имения своего не называл своим, но всё у них было общее.
\vs Act 4:33 Апостолы же с великою силою свидетельствовали о воскресении Господа Иисуса Христа; и великая благодать была на всех их.
\vs Act 4:34 Не было между ними никого нуждающегося; ибо все, которые владели землями или домами, продавая их, приносили цену проданного
\vs Act 4:35 и полагали к ногам Апостолов; и каждому давалось, в чем кто имел нужду.
\vs Act 4:36 Так Иосия, прозванный от Апостолов Варнавою, что значит~--- сын утешения, левит, родом Кипрянин,
\vs Act 4:37 у которого была своя земля, продав ее, принес деньги и положил к ногам Апостолов.
\vs Act 5:1 Некоторый же муж, именем Анания, с женою своею Сапфирою, продав имение,
\vs Act 5:2 утаил из цены, с ведома и жены своей, а некоторую часть принес и положил к ногам Апостолов.
\vs Act 5:3 Но Петр сказал: Анания! Для чего \bibemph{ты допустил} сатане вложить в сердце твое \bibemph{мысль} солгать Духу Святому и утаить из цены земли?
\vs Act 5:4 Чем ты владел, не твое ли было, и приобретенное продажею не в твоей ли власти находилось? Для чего ты положил это в сердце твоем? Ты солгал не человекам, а Богу.
\vs Act 5:5 Услышав сии слова, Анания пал бездыханен; и великий страх объял всех, слышавших это.
\vs Act 5:6 И встав, юноши приготовили его к погребению и, вынеся, похоронили.
\vs Act 5:7 Часа через три после сего пришла и жена его, не зная о случившемся.
\vs Act 5:8 Петр же спросил ее: скажи мне, за столько ли продали вы землю? Она сказала: да, за столько.
\vs Act 5:9 Но Петр сказал ей: что это согласились вы искусить Духа Господня? вот, входят в двери погребавшие мужа твоего; и тебя вынесут.
\vs Act 5:10 Вдруг она упала у ног его и испустила дух. И юноши, войдя, нашли ее мертвою и, вынеся, похоронили подле мужа ее.
\vs Act 5:11 И великий страх объял всю церковь и всех слышавших это.
\rsbpar\vs Act 5:12 Руками же Апостолов совершались в народе многие знамения и чудеса; и все единодушно пребывали в притворе Соломоновом.
\vs Act 5:13 Из посторонних же никто не смел пристать к ним, а народ прославлял их.
\vs Act 5:14 Верующих же более и более присоединялось к Господу, множество мужчин и женщин,
\vs Act 5:15 так что выносили больных на улицы и полагали на постелях и кроватях, дабы хотя тень проходящего Петра осенила кого из них.
\vs Act 5:16 Сходились также в Иерусалим многие из окрестных городов, неся больных и нечистыми духами одержимых, которые и исцелялись все.
\rsbpar\vs Act 5:17 Первосвященник же и с ним все, принадлежавшие к ереси саддукейской, исполнились зависти,
\vs Act 5:18 и наложили руки свои на Апостолов, и заключили их в народную темницу.
\vs Act 5:19 Но Ангел Господень ночью отворил двери темницы и, выведя их, сказал:
\vs Act 5:20 идите и, став в храме, говорите народу все сии слова жизни.
\vs Act 5:21 Они, выслушав, вошли утром в храм и учили. Между тем первосвященник и которые с ним, придя, созвали синедрион и всех старейшин из сынов Израилевых и послали в темницу привести \bibemph{Апостолов}.
\vs Act 5:22 Но служители, придя, не нашли их в темнице и, возвратившись, донесли,
\vs Act 5:23 говоря: темницу мы нашли запертою со всею предосторожностью и стражей стоящими перед дверями; но, отворив, не нашли в ней никого.
\vs Act 5:24 Когда услышали эти слова первосвященник, начальник стражи и \bibemph{прочие} первосвященники, недоумевали, что бы это значило.
\vs Act 5:25 Пришел же некто и донес им, говоря: вот, мужи, которых вы заключили в темницу, стоят в храме и учат народ.
\vs Act 5:26 Тогда начальник стражи пошел со служителями и привел их без принуждения, потому что боялись народа, чтобы не побили их камнями.
\vs Act 5:27 Приведя же их, поставили в синедрионе; и спросил их первосвященник, говоря:
\vs Act 5:28 не запретили ли мы вам накрепко учить о имени сем? и вот, вы наполнили Иерусалим учением вашим и хотите навести на нас кровь Того Человека.
\vs Act 5:29 Петр же и Апостолы в ответ сказали: должно повиноваться больше Богу, нежели человекам.
\vs Act 5:30 Бог отцов наших воскресил Иисуса, Которого вы умертвили, повесив на древе.
\vs Act 5:31 Его возвысил Бог десницею Своею в Начальника и Спасителя, дабы дать Израилю покаяние и прощение грехов.
\vs Act 5:32 Свидетели Ему в сем мы и Дух Святый, Которого Бог дал повинующимся Ему.
\rsbpar\vs Act 5:33 Слышав это, они разрывались от гнева и умышляли умертвить их.
\vs Act 5:34 Встав же в синедрионе, некто фарисей, именем Гамалиил, законоучитель, уважаемый всем народом, приказал вывести Апостолов на короткое время,
\vs Act 5:35 а им сказал: мужи Израильские! подумайте сами с собою о людях сих, чт\acc{о} вам с ними делать.
\vs Act 5:36 Ибо незадолго перед сим явился Февда, выдавая себя за кого-то великого, и к нему пристало около четырехсот человек; но он был убит, и все, которые слушались его, рассеялись и исчезли.
\vs Act 5:37 После него во время переписи явился Иуда Галилеянин и увлек за собою довольно народа; но он погиб, и все, которые слушались его, рассыпались.
\vs Act 5:38 И ныне, говорю вам, отстаньте от людей сих и оставьте их; ибо если это предприятие и это дело~--- от человеков, то оно разрушится,
\vs Act 5:39 а если от Бога, то вы не можете разрушить его; \bibemph{берегитесь}, чтобы вам не оказаться и богопротивниками.
\vs Act 5:40 Они послушались его; и, призвав Апостолов, били \bibemph{их} и, запретив им говорить о имени Иисуса, отпустили их.
\vs Act 5:41 Они же пошли из синедриона, радуясь, что за имя Господа Иисуса удостоились принять бесчестие.
\vs Act 5:42 И всякий день в храме и по домам не переставали учить и благовествовать об Иисусе Христе.
\vs Act 6:1 В эти дни, когда умножились ученики, произошел у Еллинистов\fns{Евреи из стран языческих.} ропот на Евреев за то, что вдовицы их пренебрегаемы были в ежедневном раздаянии потребностей.
\vs Act 6:2 Тогда двенадцать \bibemph{Апостолов}, созвав множество учеников, сказали: нехорошо нам, оставив слово Божие, пещись о столах.
\vs Act 6:3 Итак, братия, выберите из среды себя семь человек изведанных, исполненных Святаго Духа и мудрости; их поставим на эту службу,
\vs Act 6:4 а мы постоянно пребудем в молитве и служении слова.
\vs Act 6:5 И угодно было это предложение всему собранию; и избрали Стефана, мужа, исполненного веры и Духа Святаго, и Филиппа, и Прохора, и Никанора, и Тимона, и Пармена, и Николая Антиохийца, обращенного из язычников;
\vs Act 6:6 их поставили перед Апостолами, и \bibemph{сии}, помолившись, возложили на них руки.
\rsbpar\vs Act 6:7 И слово Божие росло, и число учеников весьма умножалось в Иерусалиме; и из священников очень многие покорились вере.
\rsbpar\vs Act 6:8 А Стефан, исполненный веры и силы, совершал великие чудеса и знамения в народе.
\vs Act 6:9 Некоторые из так называемой синагоги Либертинцев и Киринейцев и Александрийцев и некоторые из Киликии и Асии вступили в спор со Стефаном;
\vs Act 6:10 но не могли противостоять мудрости и Духу, Которым он говорил.
\vs Act 6:11 Тогда научили они некоторых сказать: мы слышали, как он говорил хульные слова на Моисея и на Бога.
\vs Act 6:12 И возбудили народ и старейшин и книжников и, напав, схватили его и повели в синедрион.
\vs Act 6:13 И представили ложных свидетелей, которые говорили: этот человек не перестает говорить хульные слова на святое место сие и на закон.
\vs Act 6:14 Ибо мы слышали, как он говорил, что Иисус Назорей разрушит место сие и переменит обычаи, которые передал нам Моисей.
\vs Act 6:15 И все, сидящие в синедрионе, смотря на него, видели лице его, как лице Ангела.
\vs Act 7:1 Тогда сказал первосвященник: так ли это?
\vs Act 7:2 Но он сказал: мужи братия и отцы! послушайте. Бог славы явился отцу нашему Аврааму в Месопотамии, прежде переселения его в Харран,
\vs Act 7:3 и сказал ему: выйди из земли твоей и из родства твоего и из дома отца твоего, и пойди в землю, которую покажу тебе.
\vs Act 7:4 Тогда он вышел из земли Халдейской и поселился в Харране; а оттуда, по смерти отца его, переселил его \bibemph{Бог} в сию землю, в которой вы ныне живете.
\vs Act 7:5 И не дал ему на ней наследства ни на стопу ноги, а обещал дать ее во владение ему и потомству его по нем, когда еще был он бездетен.
\vs Act 7:6 И сказал ему Бог, что потомки его будут переселенцами в чужой земле и будут в порабощении и притеснении лет четыреста.
\vs Act 7:7 Но Я, сказал Бог, произведу суд над тем народом, у которого они будут в порабощении; и после того они выйдут и будут служить Мне на сем месте.
\vs Act 7:8 И дал ему завет обрезания. По сем родил он Исаака и обрезал его в восьмой день; а Исаак \bibemph{родил} Иакова, Иаков же двенадцать патриархов.
\vs Act 7:9 Патриархи, по зависти, продали Иосифа в Египет; но Бог был с ним,
\vs Act 7:10 и избавил его от всех скорбей его, и даровал мудрость ему и благоволение царя Египетского фараона, \bibemph{который} и поставил его начальником над Египтом и над всем домом своим.
\vs Act 7:11 И пришел голод и великая скорбь на всю землю Египетскую и Ханаанскую, и отцы наши не находили пропитания.
\vs Act 7:12 Иаков же, услышав, что есть хлеб в Египте, послал \bibemph{туда} отцов наших в первый раз.
\vs Act 7:13 А когда \bibemph{они пришли} во второй раз, Иосиф открылся братьям своим, и известен стал фараону род Иосифов.
\vs Act 7:14 Иосиф, послав, призвал отца своего Иакова и все родство свое, душ семьдесят пять.
\vs Act 7:15 Иаков перешел в Египет, и скончался сам и отцы наши;
\vs Act 7:16 и перенесены были в Сихем и положены во гробе, который купил Авраам ценою серебра у сынов Еммора Сихемова.
\vs Act 7:17 А по мере, как приближалось время \bibemph{исполниться} обетованию, о котором клялся Бог Аврааму, народ возрастал и умножался в Египте,
\vs Act 7:18 до тех пор, как восстал иной царь, который не знал Иосифа.
\vs Act 7:19 Сей, ухищряясь против рода нашего, притеснял отцов наших, принуждая их бросать детей своих, чтобы не оставались в живых.
\vs Act 7:20 В это время родился Моисей, и был прекрасен пред Богом. Три месяца он был питаем в доме отца своего.
\vs Act 7:21 А когда был брошен, взяла его дочь фараонова и воспитала его у себя, как сына.
\vs Act 7:22 И научен был Моисей всей мудрости Египетской, и был силен в словах и делах.
\vs Act 7:23 Когда же исполнилось ему сорок лет, пришло ему на сердце посетить братьев своих, сынов Израилевых.
\vs Act 7:24 И, увидев одного из них обижаемого, вступился и отмстил за оскорбленного, поразив Египтянина.
\vs Act 7:25 Он думал, поймут братья его, что Бог рукою его дает им спасение; но они не поняли.
\vs Act 7:26 На следующий день, когда некоторые из них дрались, он явился и склонял их к миру, говоря: вы братья; зачем обижаете друг друга?
\vs Act 7:27 Но обижающий ближнего оттолкнул его, сказав: кто тебя поставил начальником и судьею над нами?
\vs Act 7:28 Не хочешь ли ты убить и меня, как вчера убил Египтянина?
\vs Act 7:29 От сих слов Моисей убежал и сделался пришельцем в земле Мадиамской, где родились от него два сына.
\vs Act 7:30 По исполнении сорока лет явился ему в пустыне горы Синая Ангел Господень в пламени горящего тернового куста.
\vs Act 7:31 Моисей, увидев, дивился видению; а когда подходил рассмотреть, был к нему глас Господень:
\vs Act 7:32 Я Бог отцов твоих, Бог Авраама и Бог Исаака и Бог Иакова. Моисей, объятый трепетом, не смел смотреть.
\vs Act 7:33 И сказал ему Господь: сними обувь с ног твоих, ибо место, на котором ты стоишь, есть земля святая.
\vs Act 7:34 Я вижу притеснение народа Моего в Египте, и слышу стенание его, и нисшел избавить его: итак пойди, Я пошлю тебя в Египет.
\vs Act 7:35 Сего Моисея, которого они отвергли, сказав: кто тебя поставил начальником и судьею? сего Бог чрез Ангела, явившегося ему в терновом кусте, послал начальником и избавителем.
\vs Act 7:36 Сей вывел их, сотворив чудеса и знамения в земле Египетской, и в Чермном море, и в пустыне в продолжение сорока лет.
\vs Act 7:37 Это тот Моисей, который сказал сынам Израилевым: Пророка воздвигнет вам Господь Бог ваш из братьев ваших, как меня; Его слушайте.
\vs Act 7:38 Это тот, который был в собрании в пустыне с Ангелом, говорившим ему на горе Синае, и с отцами нашими, и который принял живые слова, чтобы передать нам,
\vs Act 7:39 которому отцы наши не хотели быть послушными, но отринули его и обратились сердцами своими к Египту,
\vs Act 7:40 сказав Аарону: сделай нам богов, которые предшествовали бы нам; ибо с Моисеем, который вывел нас из земли Египетской, не знаем, что случилось.
\vs Act 7:41 И сделали в те дни тельца, и принесли жертву идолу, и веселились перед делом рук своих.
\vs Act 7:42 Бог же отвратился и оставил их служить воинству небесному, как написано в книге пророков: дом Израилев! приносили ли вы Мне заколения и жертвы в продолжение сорока лет в пустыне?
\vs Act 7:43 Вы приняли скинию Молохову и звезду бога вашего Ремфана, изображения, которые вы сделали, чтобы поклоняться им: и Я переселю вас далее Вавилона.
\vs Act 7:44 Скиния свидетельства была у отцов наших в пустыне, как повелел Говоривший Моисею сделать ее по образцу, им виденному.
\vs Act 7:45 Отцы наши с Иисусом, взяв ее, внесли во владения народов, изгнанных Богом от лица отцов наших. \bibemph{Так было} до дней Давида.
\vs Act 7:46 Сей обрел благодать пред Богом и молил, \bibemph{чтобы} найти жилище Богу Иакова.
\vs Act 7:47 Соломон же построил Ему дом.
\vs Act 7:48 Но Всевышний не в рукотворенных храмах живет, как говорит пророк:
\vs Act 7:49 Небо~--- престол Мой, и земля~--- подножие ног Моих. Какой дом созиждете Мне, говорит Господь, или какое место для покоя Моего?
\vs Act 7:50 Не Моя ли рука сотворила всё сие?
\vs Act 7:51 Жестоковыйные! люди с необрезанным сердцем и ушами! вы всегда противитесь Духу Святому, как отцы ваши, так и вы.
\vs Act 7:52 Кого из пророков не гнали отцы ваши? Они убили предвозвестивших пришествие Праведника, Которого предателями и убийцами сделались ныне вы,~---
\vs Act 7:53 вы, которые приняли закон при служении Ангелов и не сохранили.
\rsbpar\vs Act 7:54 Слушая сие, они рвались сердцами своими и скрежетали на него зубами.
\vs Act 7:55 Стефан же, будучи исполнен Духа Святаго, воззрев на небо, увидел славу Божию и Иисуса, стоящего одесную Бога,
\vs Act 7:56 и сказал: вот, я вижу небеса отверстые и Сына Человеческого, стоящего одесную Бога.
\vs Act 7:57 Но они, закричав громким голосом, затыкали уши свои, и единодушно устремились на него,
\vs Act 7:58 и, выведя за город, стали побивать его камнями. Свидетели же положили свои одежды у ног юноши, именем Савла,
\vs Act 7:59 и побивали камнями Стефана, который молился и говорил: Господи Иисусе! приими дух мой.
\vs Act 7:60 И, преклонив колени, воскликнул громким голосом: Господи! не вмени им греха сего. И, сказав сие, почил.
\vs Act 8:1 Савл же одобрял убиение его. В те дни произошло великое гонение на церковь в Иерусалиме; и все, кроме Апостолов, рассеялись по разным местам Иудеи и Самарии.
\vs Act 8:2 Стефана же погребли мужи благоговейные, и сделали великий плач по нем.
\vs Act 8:3 А Савл терзал церковь, входя в домы и влача мужчин и женщин, отдавал в темницу.
\rsbpar\vs Act 8:4 Между тем рассеявшиеся ходили и благовествовали слово.
\vs Act 8:5 Так Филипп пришел в город Самарийский и проповедовал им Христа.
\vs Act 8:6 Народ единодушно внимал тому, что говорил Филипп, слыша и видя, какие он творил чудеса.
\vs Act 8:7 Ибо нечистые духи из многих, одержимых ими, выходили с великим воплем, а многие расслабленные и хромые исцелялись.
\vs Act 8:8 И была радость великая в том городе.
\rsbpar\vs Act 8:9 Находился же в городе некоторый муж, именем Симон, который перед тем волхвовал и изумлял народ Самарийский, выдавая себя за кого-то великого.
\vs Act 8:10 Ему внимали все, от малого до большого, говоря: сей есть великая сила Божия.
\vs Act 8:11 А внимали ему потому, что он немалое время изумлял их волхвованиями.
\vs Act 8:12 Но, когда поверили Филиппу, благовествующему о Царствии Божием и о имени Иисуса Христа, то крестились и мужчины и женщины.
\vs Act 8:13 Уверовал и сам Симон и, крестившись, не отходил от Филиппа; и, видя совершающиеся великие силы и знамения, изумлялся.
\rsbpar\vs Act 8:14 Находившиеся в Иерусалиме Апостолы, услышав, что Самаряне приняли слово Божие, послали к ним Петра и Иоанна,
\vs Act 8:15 которые, придя, помолились о них, чтобы они приняли Духа Святаго.
\vs Act 8:16 Ибо Он не сходил еще ни на одного из них, а только были они крещены во имя Господа Иисуса.
\vs Act 8:17 Тогда возложили руки на них, и они приняли Духа Святаго.
\vs Act 8:18 Симон же, увидев, что через возложение рук Апостольских подается Дух Святый, принес им деньги,
\vs Act 8:19 говоря: дайте и мне власть сию, чтобы тот, на кого я возложу руки, получал Духа Святаго.
\vs Act 8:20 Но Петр сказал ему: серебро твое да будет в погибель с тобою, потому что ты помыслил дар Божий получить за деньги.
\vs Act 8:21 Нет тебе в сем части и жребия, ибо сердце твое неправо пред Богом.
\vs Act 8:22 Итак покайся в сем грехе твоем, и молись Богу: может быть, отпустится тебе помысел сердца твоего;
\vs Act 8:23 ибо вижу тебя исполненного горькой желчи и в узах неправды.
\vs Act 8:24 Симон же сказал в ответ: помолитесь вы за меня Господу, дабы не постигло меня ничто из сказанного вами.
\vs Act 8:25 Они же, засвидетельствовав и проповедав слово Господне, обратно пошли в Иерусалим и во многих селениях Самарийских проповедали Евангелие.
\rsbpar\vs Act 8:26 А Филиппу Ангел Господень сказал: встань и иди на полдень, на дорогу, идущую из Иерусалима в Газу, на ту, которая пуста.
\vs Act 8:27 Он встал и пошел. И вот, муж Ефиоплянин, евнух, вельможа Кандакии, царицы Ефиопской, хранитель всех сокровищ ее, приезжавший в Иерусалим для поклонения,
\vs Act 8:28 возвращался и, сидя на колеснице своей, читал пророка Исаию.
\vs Act 8:29 Дух сказал Филиппу: подойди и пристань к сей колеснице.
\vs Act 8:30 Филипп подошел и, услышав, что он читает пророка Исаию, сказал: разумеешь ли, что читаешь?
\vs Act 8:31 Он сказал: как могу разуметь, если кто не наставит меня? и попросил Филиппа взойти и сесть с ним.
\vs Act 8:32 А место из Писания, которое он читал, было сие: как овца, веден был Он на заклание, и, как агнец пред стригущим его безгласен, так Он не отверзает уст Своих.
\vs Act 8:33 В уничижении Его суд Его совершился. Но род Его кто разъяснит? ибо вземлется от земли жизнь Его.
\vs Act 8:34 Евнух же сказал Филиппу: прошу тебя \bibemph{сказать}: о ком пророк говорит это? о себе ли, или о ком другом?
\vs Act 8:35 Филипп отверз уста свои и, начав от сего Писания, благовествовал ему об Иисусе.
\vs Act 8:36 Между тем, продолжая путь, они приехали к воде; и евнух сказал: вот вода; что препятствует мне креститься?
\vs Act 8:37 Филипп же сказал ему: если веруешь от всего сердца, можно. Он сказал в ответ: верую, что Иисус Христос есть Сын Божий.
\vs Act 8:38 И приказал остановить колесницу, и сошли оба в воду, Филипп и евнух; и крестил его.
\vs Act 8:39 Когда же они вышли из воды, Дух Святый сошел на евнуха, а Филиппа восхитил Ангел Господень, и евнух уже не видел его, и продолжал путь, радуясь.
\vs Act 8:40 А Филипп оказался в Азоте и, проходя, благовествовал всем городам, пока пришел в Кесарию.
\vs Act 9:1 Савл же, еще дыша угрозами и убийством на учеников Господа, пришел к первосвященнику
\vs Act 9:2 и выпросил у него письма в Дамаск к синагогам, чтобы, кого найдет последующих сему учению, и мужчин и женщин, связав, приводить в Иерусалим.
\rsbpar\vs Act 9:3 Когда же он шел и приближался к Дамаску, внезапно осиял его свет с неба.
\vs Act 9:4 Он упал на землю и услышал голос, говорящий ему: Савл, Савл! что ты гонишь Меня?
\vs Act 9:5 Он сказал: кто Ты, Господи? Господь же сказал: Я Иисус, Которого ты гонишь. Трудно тебе идти против рожна.
\vs Act 9:6 Он в трепете и ужасе сказал: Господи! что повелишь мне делать? и Господь \bibemph{сказал} ему: встань и иди в город; и сказано будет тебе, что тебе надобно делать.
\vs Act 9:7 Люди же, шедшие с ним, стояли в оцепенении, слыша голос, а никого не видя.
\vs Act 9:8 Савл встал с земли, и с открытыми глазами никого не видел. И повели его за руки, и привели в Дамаск.
\vs Act 9:9 И три дня он не видел, и не ел, и не пил.
\rsbpar\vs Act 9:10 В Дамаске был один ученик, именем Анания; и Господь в видении сказал ему: Анания! Он сказал: я, Господи.
\vs Act 9:11 Господь же \bibemph{сказал} ему: встань и пойди на улицу, так называемую Прямую, и спроси в Иудином доме Тарсянина, по имени Савла; он теперь молится,
\vs Act 9:12 и видел в видении мужа, именем Ананию, пришедшего к нему и возложившего на него руку, чтобы он прозрел.
\vs Act 9:13 Анания отвечал: Господи! я слышал от многих о сем человеке, сколько зла сделал он святым Твоим в Иерусалиме;
\vs Act 9:14 и здесь имеет от первосвященников власть вязать всех, призывающих имя Твое.
\vs Act 9:15 Но Господь сказал ему: иди, ибо он есть Мой избранный сосуд, чтобы возвещать имя Мое перед народами и царями и сынами Израилевыми.
\vs Act 9:16 И Я покажу ему, сколько он должен пострадать за имя Мое.
\vs Act 9:17 Анания пошел и вошел в дом и, возложив на него руки, сказал: брат Савл! Господь Иисус, явившийся тебе на пути, которым ты шел, послал меня, чтобы ты прозрел и исполнился Святаго Духа.
\vs Act 9:18 И тотчас как бы чешуя отпала от глаз его, и вдруг он прозрел; и, встав, крестился,
\vs Act 9:19 и, приняв пищи, укрепился.\rsbpar И был Савл несколько дней с учениками в Дамаске.
\vs Act 9:20 И тотчас стал проповедовать в синагогах об Иисусе, что Он есть Сын Божий.
\vs Act 9:21 И все слышавшие дивились и говорили: не тот ли это самый, который гнал в Иерусалиме призывающих имя сие? да и сюда за тем пришел, чтобы вязать их и вести к первосвященникам.
\vs Act 9:22 А Савл более и более укреплялся и приводил в замешательство Иудеев, живущих в Дамаске, доказывая, что Сей есть Христос.
\rsbpar\vs Act 9:23 Когда же прошло довольно времени, Иудеи согласились убить его.
\vs Act 9:24 Но Савл узнал об этом умысле их. А они день и ночь стерегли у ворот, чтобы убить его.
\vs Act 9:25 Ученики же ночью, взяв его, спустили по стене в корзине.
\vs Act 9:26 Савл прибыл в Иерусалим и старался пристать к ученикам; но все боялись его, не веря, что он ученик.
\vs Act 9:27 Варнава же, взяв его, пришел к Апостолам и рассказал им, как на пути он видел Господа, и что говорил ему Господь, и как он в Дамаске смело проповедовал во имя Иисуса.
\vs Act 9:28 И пребывал он с ними, входя и исходя, в Иерусалиме, и смело проповедовал во имя Господа Иисуса.
\vs Act 9:29 Говорил также и состязался с Еллинистами; а они покушались убить его.
\vs Act 9:30 Братия, узнав \bibemph{о сем}, отправили его в Кесарию и препроводили в Тарс.
\rsbpar\vs Act 9:31 Церкви же по всей Иудее, Галилее и Самарии были в покое, назидаясь и ходя в страхе Господнем; и, при утешении от Святаго Духа, умножались.
\rsbpar\vs Act 9:32 Случилось, что Петр, обходя всех, пришел и к святым, живущим в Лидде.
\vs Act 9:33 Там нашел он одного человека, именем Энея, который восемь уже лет лежал в постели в расслаблении.
\vs Act 9:34 Петр сказал ему: Эней! исцеляет тебя Иисус Христос; встань с постели твоей. И он тотчас встал.
\vs Act 9:35 И видели его все, живущие в Лидде и в Сароне, которые и обратились к Господу.
\rsbpar\vs Act 9:36 В Иоппии находилась одна ученица, именем Тавифа, что значит: <<серна>>; она была исполнена добрых дел и творила много милостынь.
\vs Act 9:37 Случилось в те дни, что она занемогла и умерла. Ее омыли и положили в горнице.
\vs Act 9:38 А как Лидда была близ Иоппии, то ученики, услышав, что Петр находится там, послали к нему двух человек просить, чтобы он не замедлил прийти к ним.
\vs Act 9:39 Петр, встав, пошел с ними; и когда он прибыл, ввели его в горницу, и все вдовицы со слезами предстали перед ним, показывая рубашки и платья, какие делала Серна, живя с ними.
\vs Act 9:40 Петр выслал всех вон и, преклонив колени, помолился, и, обратившись к телу, сказал: Тавифа! встань. И она открыла глаза свои и, увидев Петра, села.
\vs Act 9:41 Он, подав ей руку, поднял ее, и, призвав святых и вдовиц, поставил ее перед ними живою.
\vs Act 9:42 Это сделалось известным по всей Иоппии, и многие уверовали в Господа.
\vs Act 9:43 И довольно дней пробыл он в Иоппии у некоторого Симона кожевника.
\vs Act 10:1 В Кесарии был некоторый муж, именем Корнилий, сотник из полка, называемого Италийским,
\vs Act 10:2 благочестивый и боящийся Бога со всем домом своим, творивший много милостыни народу и всегда молившийся Богу.
\vs Act 10:3 Он в видении ясно видел около девятого часа дня Ангела Божия, который вошел к нему и сказал ему: Корнилий!
\vs Act 10:4 Он же, взглянув на него и испугавшись, сказал: чт\acc{о}, Господи? \bibemph{Ангел} отвечал ему: молитвы твои и милостыни твои пришли на память пред Богом.
\vs Act 10:5 Итак пошли людей в Иоппию и призови Симона, называемого Петром.
\vs Act 10:6 Он гостит у некоего Симона кожевника, которого дом находится при море; он скажет тебе слова, которыми спасешься ты и весь дом твой.
\vs Act 10:7 Когда Ангел, говоривший с Корнилием, отошел, то он, призвав двоих из своих слуг и благочестивого воина из находившихся при нем
\vs Act 10:8 и, рассказав им все, послал их в Иоппию.
\rsbpar\vs Act 10:9 На другой день, когда они шли и приближались к городу, Петр около шестого часа взошел на верх дома помолиться.
\vs Act 10:10 И почувствовал он голод, и хотел есть. Между тем, как приготовляли, он пришел в исступление
\vs Act 10:11 и видит отверстое небо и сходящий к нему некоторый сосуд, как бы большое полотно, привязанное за четыре угла и опускаемое на землю;
\vs Act 10:12 в нем находились всякие четвероногие земные, звери, пресмыкающиеся и птицы небесные.
\vs Act 10:13 И был глас к нему: встань, Петр, заколи и ешь.
\vs Act 10:14 Но Петр сказал: нет, Господи, я никогда не ел ничего скверного или нечистого.
\vs Act 10:15 Тогда в другой раз \bibemph{был} глас к нему: что Бог очистил, того ты не почитай нечистым.
\vs Act 10:16 Это было трижды; и сосуд опять поднялся на небо.
\vs Act 10:17 Когда же Петр недоумевал в себе, что бы значило видение, которое он видел,~--- вот, мужи, посланные Корнилием, расспросив о доме Симона, остановились у ворот,
\vs Act 10:18 и, крикнув, спросили: здесь ли Симон, называемый Петром?
\vs Act 10:19 Между тем, как Петр размышлял о видении, Дух сказал ему: вот, три человека ищут тебя;
\vs Act 10:20 встань, сойди и иди с ними, нимало не сомневаясь; ибо Я послал их.
\vs Act 10:21 Петр, сойдя к людям, присланным к нему от Корнилия, сказал: я тот, которого вы ищете; за каким делом пришли вы?
\vs Act 10:22 Они же сказали: Корнилий сотник, муж добродетельный и боящийся Бога, одобряемый всем народом Иудейским, получил от святаго Ангела повеление призвать тебя в дом свой и послушать речей твоих.
\vs Act 10:23 Тогда Петр, пригласив их, угостил. А на другой день, встав, пошел с ними, и некоторые из братий Иоппийских пошли с ним.
\rsbpar\vs Act 10:24 В следующий день пришли они в Кесарию. Корнилий же ожидал их, созвав родственников своих и близких друзей.
\vs Act 10:25 Когда Петр входил, Корнилий встретил его и поклонился, пав к ногам его.
\vs Act 10:26 Петр же поднял его, говоря: встань; я тоже человек.
\vs Act 10:27 И, беседуя с ним, вошел \bibemph{в дом}, и нашел многих собравшихся.
\vs Act 10:28 И сказал им: вы знаете, что Иудею возбранено сообщаться или сближаться с иноплеменником; но мне Бог открыл, чтобы я не почитал ни одного человека скверным или нечистым.
\vs Act 10:29 Посему я, будучи позван, и пришел беспрекословно. Итак спрашиваю: для какого дела вы призвали меня?
\vs Act 10:30 Корнилий сказал: четвертого дня я постился до теперешнего часа, и в девятом часу молился в своем доме, и вот, стал предо мною муж в светлой одежде,
\vs Act 10:31 и говорит: Корнилий! услышана молитва твоя, и милостыни твои воспомянулись пред Богом.
\vs Act 10:32 Итак пошли в Иоппию и призови Симона, называемого Петром; он гостит в доме кожевника Симона при море; он придет и скажет тебе.
\vs Act 10:33 Тотчас послал я к тебе, и ты хорошо сделал, что пришел. Теперь все мы предстоим пред Богом, чтобы выслушать все, что повелено тебе от Бога.
\rsbpar\vs Act 10:34 Петр отверз уста и сказал: истинно позна\acc{ю}, что Бог нелицеприятен,
\vs Act 10:35 но во всяком народе боящийся Его и поступающий по правде приятен Ему.
\vs Act 10:36 Он послал сынам Израилевым слово, благовествуя мир чрез Иисуса Христа; Сей есть Господь всех.
\vs Act 10:37 Вы знаете происходившее по всей Иудее, начиная от Галилеи, после крещения, проповеданного Иоанном:
\vs Act 10:38 как Бог Духом Святым и силою помазал Иисуса из Назарета, и Он ходил, благотворя и исцеляя всех, обладаемых диаволом, потому что Бог был с Ним.
\vs Act 10:39 И мы свидетели всего, что сделал Он в стране Иудейской и в Иерусалиме, и что наконец Его убили, повесив на древе.
\vs Act 10:40 Сего Бог воскресил в третий день, и дал Ему являться
\vs Act 10:41 не всему народу, но свидетелям, предъизбранным от Бога, нам, которые с Ним ели и пили, по воскресении Его из мертвых.
\vs Act 10:42 И Он повелел нам проповедовать людям и свидетельствовать, что Он есть определенный от Бога Судия живых и мертвых.
\vs Act 10:43 О Нем все пророки свидетельствуют, что всякий верующий в Него получит прощение грехов именем Его.
\rsbpar\vs Act 10:44 Когда Петр еще продолжал эту речь, Дух Святый сошел на всех, слушавших слово.
\vs Act 10:45 И верующие из обрезанных, пришедшие с Петром, изумились, что дар Святаго Духа излился и на язычников,
\vs Act 10:46 ибо слышали их говорящих языками и величающих Бога. Тогда Петр сказал:
\vs Act 10:47 кто может запретить креститься водою тем, которые, как и мы, получили Святаго Духа?
\vs Act 10:48 И велел им креститься во имя Иисуса Христа. Потом они просили его пробыть у них несколько дней.
\vs Act 11:1 Услышали Апостолы и братия, бывшие в Иудее, что и язычники приняли слово Божие.
\vs Act 11:2 И когда Петр пришел в Иерусалим, обрезанные упрекали его,
\vs Act 11:3 говоря: ты ходил к людям необрезанным и ел с ними.
\vs Act 11:4 Петр же начал пересказывать им по порядку, говоря:
\vs Act 11:5 в городе Иоппии я молился, и в исступлении видел видение: сходил некоторый сосуд, как бы большое полотно, за четыре угла спускаемое с неба, и спустилось ко мне.
\vs Act 11:6 Я посмотрел в него и, рассматривая, увидел четвероногих земных, зверей, пресмыкающихся и птиц небесных.
\vs Act 11:7 И услышал я голос, говорящий мне: встань, Петр, заколи и ешь.
\vs Act 11:8 Я же сказал: нет, Господи, ничего скверного или нечистого никогда не входило в уста мои.
\vs Act 11:9 И отвечал мне голос вторично с неба: что Бог очистил, того ты не почитай нечистым.
\vs Act 11:10 Это было трижды, и опять поднялось всё на небо.
\vs Act 11:11 И вот, в тот самый час три человека стали перед домом, в котором я был, посланные из Кесарии ко мне.
\vs Act 11:12 Дух сказал мне, чтобы я шел с ними, нимало не сомневаясь. Пошли со мною и сии шесть братьев, и мы пришли в дом \bibemph{того} человека.
\vs Act 11:13 Он рассказал нам, как он видел в доме своем Ангела (святого), который стал и сказал ему: пошли в Иоппию людей и призови Симона, называемого Петром;
\vs Act 11:14 он скажет тебе слова, которыми спасешься ты и весь дом твой.
\vs Act 11:15 Когда же начал я говорить, сошел на них Дух Святый, как и на нас вначале.
\vs Act 11:16 Тогда вспомнил я слово Господа, как Он говорил: <<Иоанн крестил водою, а вы будете крещены Духом Святым>>.
\vs Act 11:17 Итак, если Бог дал им такой же дар, как и нам, уверовавшим в Господа Иисуса Христа, то кто же я, чтобы мог воспрепятствовать Богу?
\vs Act 11:18 Выслушав это, они успокоились и прославили Бога, говоря: видно, и язычникам дал Бог покаяние в жизнь.
\rsbpar\vs Act 11:19 Между тем рассеявшиеся от гонения, бывшего после Стефана, прошли до Финикии и Кипра и Антиохии, никому не проповедуя слово, кроме Иудеев.
\vs Act 11:20 Были же некоторые из них Кипряне и Киринейцы, которые, придя в Антиохию, говорили Еллинам, благовествуя Господа Иисуса.
\vs Act 11:21 И была рука Господня с ними, и великое число, уверовав, обратилось к Господу.
\vs Act 11:22 Дошел слух о сем до церкви Иерусалимской, и поручили Варнаве идти в Антиохию.
\vs Act 11:23 Он, прибыв и увидев благодать Божию, возрадовался и убеждал всех держаться Господа искренним сердцем;
\vs Act 11:24 ибо он был муж добрый и исполненный Духа Святаго и веры. И приложилось довольно народа к Господу.
\vs Act 11:25 Потом Варнава пошел в Тарс искать Савла и, найдя его, привел в Антиохию.
\vs Act 11:26 Целый год собирались они в церкви и учили немалое число людей, и ученики в Антиохии в первый раз стали называться Христианами.
\rsbpar\vs Act 11:27 В те дни пришли из Иерусалима в Антиохию пророки.
\vs Act 11:28 И один из них, по имени Агав, встав, предвозвестил Духом, что по всей вселенной будет великий голод, который и был при кесаре Клавдии.
\vs Act 11:29 Тогда ученики положили, каждый по достатку своему, послать пособие братьям, живущим в Иудее,
\vs Act 11:30 что и сделали, послав \bibemph{собранное} к пресвитерам через Варнаву и Савла.
\vs Act 12:1 В то время царь Ирод поднял руки на некоторых из принадлежащих к церкви, чтобы сделать им зло,
\vs Act 12:2 и убил Иакова, брата Иоаннова, мечом.
\vs Act 12:3 Видя же, что это приятно Иудеям, вслед за тем взял и Петра,~--- тогда были дни опресноков,~---
\vs Act 12:4 и, задержав его, посадил в темницу, и приказал четырем четверицам воинов стеречь его, намереваясь после Пасхи вывести его к народу.
\vs Act 12:5 Итак Петра стерегли в темнице, между тем церковь прилежно молилась о нем Богу.
\rsbpar\vs Act 12:6 Когда же Ирод хотел вывести его, в ту ночь Петр спал между двумя воинами, скованный двумя цепями, и стражи у дверей стерегли темницу.
\vs Act 12:7 И вот, Ангел Господень предстал, и свет осиял темницу. \bibemph{Ангел}, толкнув Петра в бок, пробудил его и сказал: встань скорее. И цепи упали с рук его.
\vs Act 12:8 И сказал ему Ангел: опояшься и обуйся. Он сделал так. Потом говорит ему: надень одежду твою и иди за мною.
\vs Act 12:9 \bibemph{Петр} вышел и следовал за ним, не зная, что делаемое Ангелом было действительно, а думая, что видит видение.
\vs Act 12:10 Пройдя первую и вторую стражу, они пришли к железным воротам, ведущим в город, которые сами собою отворились им: они вышли, и прошли одну улицу, и вдруг Ангела не стало с ним.
\vs Act 12:11 Тогда Петр, придя в себя, сказал: теперь я вижу воистину, что Господь послал Ангела Своего и избавил меня из руки Ирода и от всего, чего ждал народ Иудейский.
\vs Act 12:12 И, осмотревшись, пришел к дому Марии, матери Иоанна, называемого Марком, где многие собрались и молились.
\vs Act 12:13 Когда же Петр постучался у ворот, то вышла послушать служанка, именем Рода,
\vs Act 12:14 и, узнав голос Петра, от радости не отворила ворот, но, вбежав, объявила, что Петр стоит у ворот.
\vs Act 12:15 А те сказали ей: в своем ли ты уме? Но она утверждала свое. Они же говорили: это Ангел его.
\vs Act 12:16 Между тем Петр продолжал стучать. Когда же отворили, то увидели его и изумились.
\vs Act 12:17 Он же, дав знак рукою, чтобы молчали, рассказал им, как Господь вывел его из темницы, и сказал: уведомьте о сем Иакова и братьев. Потом, выйдя, пошел в другое место.
\rsbpar\vs Act 12:18 По наступлении дня между воинами сделалась большая тревога о том, что сделалось с Петром.
\vs Act 12:19 Ирод же, поискав его и не найдя, судил стражей и велел казнить их. Потом он отправился из Иудеи в Кесарию и \bibemph{там} оставался.
\rsbpar\vs Act 12:20 Ирод был раздражен на Тирян и Сидонян; они же, согласившись, пришли к нему и, склонив на свою сторону Власта, постельника царского, просили мира, потому что область их питалась от \bibemph{области} царской.
\vs Act 12:21 В назначенный день Ирод, одевшись в царскую одежду, сел на возвышенном месте и говорил к ним;
\vs Act 12:22 а народ восклицал: \bibemph{это} голос Бога, а не человека.
\vs Act 12:23 Но вдруг Ангел Господень поразил его за то, что он не воздал славы Богу; и он, быв изъеден червями, умер.
\rsbpar\vs Act 12:24 Слово же Божие росло и распространялось.
\vs Act 12:25 А Варнава и Савл, по исполнении поручения, возвратились из Иерусалима (в Антиохию), взяв с собою и Иоанна, прозванного Марком.
\vs Act 13:1 В Антиохии, в тамошней церкви были некоторые пророки и учители: Варнава, и Симеон, называемый Нигер, и Луций Киринеянин, и Манаил, совоспитанник Ирода четвертовластника, и Савл.
\vs Act 13:2 Когда они служили Господу и постились, Дух Святый сказал: отделите Мне Варнаву и Савла на дело, к которому Я призвал их.
\vs Act 13:3 Тогда они, совершив пост и молитву и возложив на них руки, отпустили их.
\rsbpar\vs Act 13:4 Сии, быв посланы Духом Святым, пришли в Селевкию, а оттуда отплыли в Кипр;
\vs Act 13:5 и, быв в Саламине, проповедовали слово Божие в синагогах Иудейских; имели же при себе и Иоанна для служения.
\vs Act 13:6 Пройдя весь остров до Пафа, нашли они некоторого волхва, лжепророка, Иудеянина, именем Вариисуса,
\vs Act 13:7 который находился с проконсулом Сергием Павлом, мужем разумным. Сей, призвав Варнаву и Савла, пожелал услышать слово Божие.
\vs Act 13:8 А Елима волхв (ибо т\acc{о} значит имя его) противился им, стараясь отвратить проконсула от веры.
\vs Act 13:9 Но Савл, он же и Павел, исполнившись Духа Святаго и устремив на него взор,
\vs Act 13:10 сказал: о, исполненный всякого коварства и всякого злодейства, сын диавола, враг всякой правды! перестанешь ли ты совращать с прямых путей Господних?
\vs Act 13:11 И ныне вот, рука Господня на тебя: ты будешь слеп и не увидишь солнца до времени. И вдруг напал на него мрак и тьма, и он, обращаясь туда и сюда, искал вожатого.
\vs Act 13:12 Тогда проконсул, увидев происшедшее, уверовал, дивясь учению Господню.
\rsbpar\vs Act 13:13 Отплыв из Пафа, Павел и бывшие при нем прибыли в Пергию, в Памфилии. Но Иоанн, отделившись от них, возвратился в Иерусалим.
\vs Act 13:14 Они же, проходя от Пергии, прибыли в Антиохию Писидийскую и, войдя в синагогу в день субботний, сели.
\vs Act 13:15 После чтения закона и пророков, начальники синагоги послали сказать им: мужи братия! если у вас есть слово наставления к народу, говорите.
\rsbpar\vs Act 13:16 Павел, встав и дав знак рукою, сказал: мужи Израильтяне и боящиеся Бога! послушайте.
\vs Act 13:17 Бог народа сего избрал отцов наших и возвысил сей народ во время пребывания в земле Египетской, и мышцею вознесенною вывел их из нее,
\vs Act 13:18 и около сорока лет времени питал их в пустыне.
\vs Act 13:19 И, истребив семь народов в земле Ханаанской, разделил им в наследие землю их.
\vs Act 13:20 И после сего, около четырехсот пятидесяти лет, давал им судей до пророка Самуила.
\vs Act 13:21 Потом просили они царя, и Бог дал им Саула, сына Кисова, мужа из колена Вениаминова. \bibemph{Так прошло} лет сорок.
\vs Act 13:22 Отринув его, поставил им царем Давида, о котором и сказал, свидетельствуя: нашел Я мужа по сердцу Моему, Давида, сына Иессеева, который исполнит все хотения Мои.
\vs Act 13:23 Из его-то потомства Бог по обетованию воздвиг Израилю Спасителя Иисуса.
\vs Act 13:24 Перед самым явлением Его Иоанн проповедовал крещение покаяния всему народу Израильскому.
\vs Act 13:25 При окончании же поприща своего, Иоанн говорил: за кого почитаете вы меня? я не тот; но вот, идет за мною, у Которого я недостоин развязать обувь на ногах.
\vs Act 13:26 Мужи братия, дети рода Авраамова, и боящиеся Бога между вами! вам послано слово спасения сего.
\vs Act 13:27 Ибо жители Иерусалима и начальники их, не узнав Его и осудив, исполнили слова пророческие, читаемые каждую субботу,
\vs Act 13:28 и, не найдя в Нем никакой вины, достойной смерти, просили Пилата убить Его.
\vs Act 13:29 Когда же исполнили всё написанное о Нем, то, сняв с древа, положили Его во гроб.
\vs Act 13:30 Но Бог воскресил Его из мертвых.
\vs Act 13:31 Он в продолжение многих дней являлся тем, которые вышли с Ним из Галилеи в Иерусалим и которые ныне суть свидетели Его перед народом.
\vs Act 13:32 И мы благовествуем вам, что обетование, данное отцам, Бог исполнил нам, детям их, воскресив Иисуса,
\vs Act 13:33 как и во втором псалме написано: Ты Сын Мой: Я ныне родил Тебя.
\vs Act 13:34 А что воскресил Его из мертвых, так что Он уже не обратится в тление, \bibemph{о сем} сказал так: Я дам вам милости, \bibemph{обещанные} Давиду, верно.
\vs Act 13:35 Посему и в другом \bibemph{месте} говорит: не дашь Святому Твоему увидеть тление.
\vs Act 13:36 Давид, в свое время послужив изволению Божию, почил и приложился к отцам своим, и увидел тление;
\vs Act 13:37 а Тот, Которого Бог воскресил, не увидел тления.
\vs Act 13:38 Итак, да будет известно вам, мужи братия, что ради Него возвещается вам прощение грехов;
\vs Act 13:39 и во всем, в чем вы не могли оправдаться законом Моисеевым, оправдывается Им всякий верующий.
\vs Act 13:40 Берегитесь же, чтобы не пришло на вас сказанное у пророков:
\vs Act 13:41 смотрите, презрители, подивитесь и исчезните; ибо Я делаю дело во дни ваши, дело, которому не поверили бы вы, если бы кто рассказывал вам.
\rsbpar\vs Act 13:42 При выходе их из Иудейской синагоги язычники просили их говорить о том же в следующую субботу.
\vs Act 13:43 Когда же собрание было распущено, то многие Иудеи и чтители \bibemph{Бога}, обращенные из язычников, последовали за Павлом и Варнавою, которые, беседуя с ними, убеждали их пребывать в благодати Божией.
\rsbpar\vs Act 13:44 В следующую субботу почти весь город собрался слушать слово Божие.
\vs Act 13:45 Но Иудеи, увидев народ, исполнились зависти и, противореча и злословя, сопротивлялись тому, что говорил Павел.
\vs Act 13:46 Тогда Павел и Варнава с дерзновением сказали: вам первым надлежало быть проповедану слову Божию, но как вы отвергаете его и сами себя делаете недостойными вечной жизни, то вот, мы обращаемся к язычникам.
\vs Act 13:47 Ибо так заповедал нам Господь: Я положил Тебя во свет язычникам, чтобы Ты был во спасение до края земли.
\vs Act 13:48 Язычники, слыша это, радовались и прославляли слово Господне, и уверовали все, которые были предуставлены к вечной жизни.
\vs Act 13:49 И слово Господне распространялось по всей стране.
\vs Act 13:50 Но Иудеи, подстрекнув набожных и почетных женщин и первых в городе \bibemph{людей}, воздвигли гонение на Павла и Варнаву и изгнали их из своих пределов.
\vs Act 13:51 Они же, отрясши на них прах от ног своих, пошли в Иконию.
\vs Act 13:52 А ученики исполнялись радости и Духа Святаго.
\vs Act 14:1 В Иконии они вошли вместе в Иудейскую синагогу и говорили так, что уверовало великое множество Иудеев и Еллинов.
\vs Act 14:2 А неверующие Иудеи возбудили и раздражили против братьев сердца язычников.
\vs Act 14:3 Впрочем они пробыли \bibemph{здесь} довольно времени, смело действуя о Господе, Который, во свидетельство слову благодати Своей, творил руками их знамения и чудеса.
\vs Act 14:4 Между тем народ в городе разделился: и одни были на стороне Иудеев, а другие на стороне Апостолов.
\vs Act 14:5 Когда же язычники и Иудеи со своими начальниками устремились на них, чтобы посрамить и побить их камнями,
\vs Act 14:6 они, узнав \bibemph{о сем}, удалились в Ликаонские города Листру и Дервию и в окрестности их,
\vs Act 14:7 и там благовествовали.
\rsbpar\vs Act 14:8 В Листре некоторый муж, не владевший ногами, сидел, будучи хром от чрева матери своей, и никогда не ходил.
\vs Act 14:9 Он слушал говорившего Павла, который, взглянув на него и увидев, что он имеет веру для получения исцеления,
\vs Act 14:10 сказал громким голосом: тебе говорю во имя Господа Иисуса Христа: стань на ноги твои прямо. И он тотчас вскочил и стал ходить.
\vs Act 14:11 Народ же, увидев, что сделал Павел, возвысил свой голос, говоря по-ликаонски: боги в образе человеческом сошли к нам.
\vs Act 14:12 И называли Варнаву Зевсом, а Павла Ермием, потому что он начальствовал в слове.
\vs Act 14:13 Жрец же \bibemph{идола} Зевса, находившегося перед их городом, приведя к воротам волов и \bibemph{принеся} венки, хотел вместе с народом совершить жертвоприношение.
\vs Act 14:14 Но Апостолы Варнава и Павел, услышав \bibemph{о сем}, разодрали свои одежды и, бросившись в народ, громогласно говорили:
\vs Act 14:15 мужи! чт\acc{о} вы это делаете? И мы~--- подобные вам человеки, и благовествуем вам, чтобы вы обратились от сих ложных к Богу живому, Который сотворил небо и землю, и море, и все, что в них,
\vs Act 14:16 Который в прошедших родах попустил всем народам ходить своими путями,
\vs Act 14:17 хотя и не переставал свидетельствовать о Себе благодеяниями, подавая нам с неба дожди и времена плодоносные и исполняя пищею и веселием сердца наши.
\vs Act 14:18 И, говоря сие, они едва убедили народ не приносить им жертвы и идти каждому домой. Между тем, как они, оставаясь там, учили,
\vs Act 14:19 из Антиохии и Иконии пришли некоторые Иудеи и, когда \bibemph{Апостолы} смело проповедовали, убедили народ отстать от них, говоря: они не говорят ничего истинного, а все лгут. И, возбудив народ, побили Павла камнями и вытащили за город, почитая его умершим.
\vs Act 14:20 Когда же ученики собрались около него, он встал и пошел в город, а на другой день удалился с Варнавою в Дервию.
\rsbpar\vs Act 14:21 Проповедав Евангелие сему городу и приобретя довольно учеников, они обратно проходили Листру, Иконию и Антиохию,
\vs Act 14:22 утверждая души учеников, увещевая пребывать в вере и \bibemph{поучая}, что многими скорбями надлежит нам войти в Царствие Божие.
\vs Act 14:23 Рукоположив же им пресвитеров к каждой церкви, они помолились с постом и предали их Господу, в Которого уверовали.
\vs Act 14:24 Потом, пройдя через Писидию, пришли в Памфилию,
\vs Act 14:25 и, проповедав слово Господне в Пергии, сошли в Атталию;
\vs Act 14:26 а оттуда отплыли в Антиохию, откуда были преданы благодати Божией на дело, которое и исполнили.
\rsbpar\vs Act 14:27 Прибыв туда и собрав церковь, они рассказали всё, что сотворил Бог с ними и как Он отверз дверь веры язычникам.
\vs Act 14:28 И пребывали там немалое время с учениками.
\vs Act 15:1 Некоторые, пришедшие из Иудеи, учили братьев: если не обрежетесь по обряду Моисееву, не можете спастись.
\vs Act 15:2 Когда же произошло разногласие и немалое состязание у Павла и Варнавы с ними, то положили Павлу и Варнаве и некоторым другим из них отправиться по сему делу к Апостолам и пресвитерам в Иерусалим.
\vs Act 15:3 Итак, быв провожены церковью, они проходили Финикию и Самарию, рассказывая об обращении язычников, и производили радость великую во всех братиях.
\vs Act 15:4 По прибытии же в Иерусалим они были приняты церковью, Апостолами и пресвитерами, и возвестили всё, что Бог сотворил с ними и как отверз дверь веры язычникам.
\vs Act 15:5 Тогда восстали некоторые из фарисейской ереси уверовавшие и говорили, что должно обрезывать \bibemph{язычников} и заповедовать соблюдать закон Моисеев.
\rsbpar\vs Act 15:6 Апостолы и пресвитеры собрались для рассмотрения сего дела.
\vs Act 15:7 По долгом рассуждении Петр, встав, сказал им: мужи братия! вы знаете, что Бог от дней первых избрал из нас \bibemph{меня}, чтобы из уст моих язычники услышали слово Евангелия и уверовали;
\vs Act 15:8 и Сердцеведец Бог дал им свидетельство, даровав им Духа Святаго, как и нам;
\vs Act 15:9 и не положил никакого различия между нами и ими, верою очистив сердца их.
\vs Act 15:10 Что же вы ныне искушаете Бога, \bibemph{желая} возложить на выи учеников иго, которого не могли понести ни отцы наши, ни мы?
\vs Act 15:11 Но мы веруем, что благодатию Господа Иисуса Христа спасемся, как и они.
\vs Act 15:12 Тогда умолкло все собрание и слушало Варнаву и Павла, рассказывавших, какие знамения и чудеса сотворил Бог через них среди язычников.
\vs Act 15:13 После же того, как они умолкли, начал речь Иаков и сказал: мужи братия! послушайте меня.
\vs Act 15:14 Симон изъяснил, как Бог первоначально призрел на язычников, чтобы составить из них народ во имя Свое.
\vs Act 15:15 И с сим согласны слова пророков, как написано:
\vs Act 15:16 Потом обращусь и воссоздам скинию Давидову падшую, и то, что в ней разрушено, воссоздам, и исправлю ее,
\vs Act 15:17 чтобы взыскали Господа прочие человеки и все народы, между которыми возвестится имя Мое, говорит Господь, творящий все сие.
\vs Act 15:18 Ведомы Богу от вечности все дела Его.
\vs Act 15:19 Посему я полагаю не затруднять обращающихся к Богу из язычников,
\vs Act 15:20 а написать им, чтобы они воздерживались от оскверненного идолами, от блуда, удавленины и крови, и чтобы не делали другим того, чего не хотят себе.
\vs Act 15:21 Ибо \bibemph{закон} Моисеев от древних родов по всем городам имеет проповедующих его и читается в синагогах каждую субботу.
\rsbpar\vs Act 15:22 Тогда Апостолы и пресвитеры со всею церковью рассудили, избрав из среды себя мужей, послать их в Антиохию с Павлом и Варнавою, \bibemph{именно}: Иуду, прозываемого Варсавою, и Силу, мужей, начальствующих между братиями,
\vs Act 15:23 написав и вручив им следующее: <<Апостолы и пресвитеры и братия~--- находящимся в Антиохии, Сирии и Киликии братиям из язычников: радоваться.
\vs Act 15:24 Поелику мы услышали, что некоторые, вышедшие от нас, смутили вас \bibemph{своими} речами и поколебали ваши души, говоря, что должно обрезываться и соблюдать закон, чего мы им не поручали,
\vs Act 15:25 то мы, собравшись, единодушно рассудили, избрав мужей, послать их к вам с возлюбленными нашими Варнавою и Павлом,
\vs Act 15:26 человеками, предавшими души свои за имя Господа нашего Иисуса Христа.
\vs Act 15:27 Итак мы послали Иуду и Силу, которые изъяснят вам то же и словесно.
\vs Act 15:28 Ибо угодно Святому Духу и нам не возлагать на вас никакого бремени более, кроме сего необходимого:
\vs Act 15:29 воздерживаться от идоложертвенного и крови, и удавленины, и блуда, и не делать другим того, чего себе не хотите. Соблюдая сие, хорошо сделаете. Будьте здравы>>.
\rsbpar\vs Act 15:30 Итак, отправленные пришли в Антиохию и, собрав людей, вручили письмо.
\vs Act 15:31 Они же, прочитав, возрадовались о сем наставлении.
\vs Act 15:32 Иуда и Сила, будучи также пророками, обильным словом преподали наставление братиям и утвердили их.
\vs Act 15:33 Пробыв там \bibemph{некоторое} время, они с миром отпущены были братиями к Апостолам.
\vs Act 15:34 Но Силе рассудилось остаться там. (А Иуда возвратился в Иерусалим.)
\vs Act 15:35 Павел же и Варнава жили в Антиохии, уча и благовествуя, вместе с другими многими, слово Господне.
\rsbpar\vs Act 15:36 По некотором времени Павел сказал Варнаве: пойдем опять, посетим братьев наших по всем городам, в которых мы проповедали слово Господне, как они живут.
\vs Act 15:37 Варнава хотел взять с собою Иоанна, называемого Марком.
\vs Act 15:38 Но Павел полагал не брать отставшего от них в Памфилии и не шедшего с ними на дело, на которое они были посланы.
\vs Act 15:39 Отсюда произошло огорчение, так что они разлучились друг с другом; и Варнава, взяв Марка, отплыл в Кипр;
\vs Act 15:40 а Павел, избрав себе Силу, отправился, быв поручен братиями благодати Божией,
\vs Act 15:41 и проходил Сирию и Киликию, утверждая церкви.
\vs Act 16:1 Дошел он до Дервии и Листры. И вот, там был некоторый ученик, именем Тимофей, которого мать была Иудеянка уверовавшая, а отец Еллин,
\vs Act 16:2 и о котором свидетельствовали братия, находившиеся в Листре и Иконии.
\vs Act 16:3 Его пожелал Павел взять с собою; и, взяв, обрезал его ради Иудеев, находившихся в тех местах; ибо все знали об отце его, что он был Еллин.
\vs Act 16:4 Проходя же по городам, они предавали \bibemph{верным} соблюдать определения, постановленные Апостолами и пресвитерами в Иерусалиме.
\vs Act 16:5 И церкви утверждались верою и ежедневно увеличивались числом.
\rsbpar\vs Act 16:6 Пройдя через Фригию и Галатийскую страну, они не были допущены Духом Святым проповедовать слово в Асии.
\vs Act 16:7 Дойдя до Мисии, предпринимали идти в Вифинию; но Дух не допустил их.
\vs Act 16:8 Миновав же Мисию, сошли они в Троаду.
\vs Act 16:9 И было ночью видение Павлу: предстал некий муж, Македонянин, прося его и говоря: приди в Македонию и помоги нам.
\vs Act 16:10 После сего видения, тотчас мы положили отправиться в Македонию, заключая, что призывал нас Господь благовествовать там.
\rsbpar\vs Act 16:11 Итак, отправившись из Троады, мы прямо прибыли в Самофракию, а на другой день в Неаполь,
\vs Act 16:12 оттуда же в Филиппы: это первый город в той части Македонии, колония. В этом городе мы пробыли несколько дней.
\vs Act 16:13 В день же субботний мы вышли за город к реке, где, по обыкновению, был молитвенный дом, и, сев, разговаривали с собравшимися \bibemph{там} женщинами.
\vs Act 16:14 И одна женщина из города Фиатир, именем Лидия, торговавшая багряницею, чтущая Бога, слушала; и Господь отверз сердце ее внимать тому, что говорил Павел.
\vs Act 16:15 Когда же крестилась она и домашние ее, то просила нас, говоря: если вы признали меня верною Господу, то войдите в дом мой и живите \bibemph{у меня}. И убедила нас.
\rsbpar\vs Act 16:16 Случилось, что, когда мы шли в молитвенный дом, встретилась нам одна служанка, одержимая духом прорицательным, которая через прорицание доставляла большой доход господам своим.
\vs Act 16:17 Идя за Павлом и за нами, она кричала, говоря: сии человеки~--- рабы Бога Всевышнего, которые возвещают нам путь спасения.
\vs Act 16:18 Это она делала много дней. Павел, вознегодовав, обратился и сказал духу: именем Иисуса Христа повелеваю тебе выйти из нее. И \bibemph{дух} вышел в тот же час.
\vs Act 16:19 Тогда господа ее, видя, что исчезла надежда дохода их, схватили Павла и Силу и повлекли на площадь к начальникам.
\vs Act 16:20 И, приведя их к воеводам, сказали: сии люди, будучи Иудеями, возмущают наш город
\vs Act 16:21 и проповедуют обычаи, которых нам, Римлянам, не следует ни принимать, ни исполнять.
\vs Act 16:22 Народ также восстал на них, а воеводы, сорвав с них одежды, велели бить их палками
\vs Act 16:23 и, дав им много ударов, ввергли в темницу, приказав темничному стражу крепко стеречь их.
\vs Act 16:24 Получив такое приказание, он ввергнул их во внутреннюю темницу и ноги их забил в колоду.
\rsbpar\vs Act 16:25 Около полуночи Павел и Сила, молясь, воспевали Бога; узники же слушали их.
\vs Act 16:26 Вдруг сделалось великое землетрясение, так что поколебалось основание темницы; тотчас отворились все двери, и у всех узы ослабели.
\vs Act 16:27 Темничный же страж, пробудившись и увидев, что двери темницы отворены, извлек меч и хотел умертвить себя, думая, что узники убежали.
\vs Act 16:28 Но Павел возгласил громким голосом, говоря: не делай себе никакого зла, ибо все мы здесь.
\vs Act 16:29 Он потребовал огня, вбежал \bibemph{в темницу} и в трепете припал к Павлу и Силе,
\vs Act 16:30 и, выведя их вон, сказал: государи \bibemph{мои}! что мне делать, чтобы спастись?
\vs Act 16:31 Они же сказали: веруй в Господа Иисуса Христа, и спасешься ты и весь дом твой.
\vs Act 16:32 И проповедали слово Господне ему и всем, бывшим в доме его.
\vs Act 16:33 И, взяв их в тот час ночи, он омыл раны их и немедленно крестился сам и все \bibemph{домашние} его.
\vs Act 16:34 И, приведя их в дом свой, предложил трапезу и возрадовался со всем домом своим, что уверовал в Бога.
\rsbpar\vs Act 16:35 Когда же настал день, воеводы послали городских служителей сказать: отпусти тех людей.
\vs Act 16:36 Темничный страж объявил о сем Павлу: воеводы прислали отпустить вас; итак выйдите теперь и идите с миром.
\vs Act 16:37 Но Павел сказал к ним: нас, Римских граждан, без суда всенародно били и бросили в темницу, а теперь тайно выпускают? нет, пусть придут и сами выведут нас.
\vs Act 16:38 Городские служители пересказали эти слова воеводам, и те испугались, услышав, что это Римские граждане.
\vs Act 16:39 И, придя, извинились перед ними и, выведя, просили удалиться из города.
\vs Act 16:40 Они же, выйдя из темницы, пришли к Лидии и, увидев братьев, поучали их, и отправились.
\vs Act 17:1 Пройдя через Амфиполь и Аполлонию, они пришли в Фессалонику, где была Иудейская синагога.
\vs Act 17:2 Павел, по своему обыкновению, вошел к ним и три субботы говорил с ними из Писаний,
\vs Act 17:3 открывая и доказывая им, что Христу надлежало пострадать и воскреснуть из мертвых и что Сей Христос есть Иисус, Которого я проповедую вам.
\vs Act 17:4 И некоторые из них уверовали и присоединились к Павлу и Силе, как из Еллинов, чтущих \bibemph{Бога}, великое множество, так и из знатных женщин немало.
\vs Act 17:5 Но неуверовавшие Иудеи, возревновав и взяв с площади некоторых негодных людей, собрались толпою и возмущали город и, приступив к дому Иасона, домогались вывести их к народу.
\vs Act 17:6 Не найдя же их, повлекли Иасона и некоторых братьев к городским начальникам, крича, что эти всесветные возмутители пришли и сюда,
\vs Act 17:7 а Иасон принял их, и все они поступают против повелений кесаря, почитая другого царем, Иисуса.
\vs Act 17:8 И встревожили народ и городских начальников, слушавших это.
\vs Act 17:9 Но \bibemph{сии}, получив удостоверение от Иасона и прочих, отпустили их.
\rsbpar\vs Act 17:10 Братия же немедленно ночью отправили Павла и Силу в Верию, куда они прибыв, пошли в синагогу Иудейскую.
\vs Act 17:11 Здешние были благомысленнее Фессалоникских: они приняли слово со всем усердием, ежедневно разбирая Писания, точно ли это так.
\vs Act 17:12 И многие из них уверовали, и из Еллинских почетных женщин и из мужчин немало.
\vs Act 17:13 Но когда Фессалоникские Иудеи узнали, что и в Верии проповедано Павлом слово Божие, то пришли и туда, возбуждая и возмущая народ.
\vs Act 17:14 Тогда братия тотчас отпустили Павла, как будто идущего к морю; а Сила и Тимофей остались там.
\vs Act 17:15 Сопровождавшие Павла проводили его до Афин и, получив приказание к Силе и Тимофею, чтобы они скорее пришли к нему, отправились.
\rsbpar\vs Act 17:16 В ожидании их в Афинах Павел возмутился духом при виде этого города, полного идолов.
\vs Act 17:17 Итак он рассуждал в синагоге с Иудеями и с чтущими \bibemph{Бога}, и ежедневно на площади со встречающимися.
\vs Act 17:18 Некоторые из эпикурейских и стоических философов стали спорить с ним; и одни говорили: <<чт\acc{о} хочет сказать этот суеслов?>>, а другие: <<кажется, он проповедует о чужих божествах>>, потому что он благовествовал им Иисуса и воскресение.
\vs Act 17:19 И, взяв его, привели в ареопаг и говорили: можем ли мы знать, что это за новое учение, проповедуемое тобою?
\vs Act 17:20 Ибо что-то странное ты влагаешь в уши наши. Посему хотим знать, чт\acc{о} это такое?
\vs Act 17:21 Афиняне же все и живущие \bibemph{у них} иностранцы ни в чем охотнее не проводили время, как в том, чтобы говорить или слушать что-нибудь новое.
\rsbpar\vs Act 17:22 И, став Павел среди ареопага, сказал: Афиняне! по всему вижу я, что вы как бы особенно набожны.
\vs Act 17:23 Ибо, проходя и осматривая ваши святыни, я нашел и жертвенник, на котором написано <<неведомому Богу>>. Сего-то, Которого вы, не зная, чтите, я проповедую вам.
\vs Act 17:24 Бог, сотворивший мир и всё, что в нем, Он, будучи Господом неба и земли, не в рукотворенных храмах живет
\vs Act 17:25 и не требует служения рук человеческих, \bibemph{как бы} имеющий в чем-либо нужду, Сам дая всему жизнь и дыхание и всё.
\vs Act 17:26 От одной крови Он произвел весь род человеческий для обитания по всему лицу земли, назначив предопределенные времена и пределы их обитанию,
\vs Act 17:27 дабы они искали Бога, не ощутят ли Его и не найдут ли, хотя Он и недалеко от каждого из нас:
\vs Act 17:28 ибо мы Им живем и движемся и существуем, как и некоторые из ваших стихотворцев говорили: <<мы Его и род>>.
\vs Act 17:29 Итак мы, будучи родом Божиим, не должны думать, что Божество подобно золоту, или серебру, или камню, получившему образ от искусства и вымысла человеческого.
\vs Act 17:30 Итак, оставляя времена неведения, Бог ныне повелевает людям всем повсюду покаяться,
\vs Act 17:31 ибо Он назначил день, в который будет праведно судить вселенную, посредством предопределенного Им Мужа, подав удостоверение всем, воскресив Его из мертвых.
\vs Act 17:32 Услышав о воскресении мертвых, одни насмехались, а другие говорили: об этом послушаем тебя в другое время.
\vs Act 17:33 Итак Павел вышел из среды их.
\vs Act 17:34 Некоторые же мужи, пристав к нему, уверовали; между ними был Дионисий Ареопагит и женщина, именем Дамарь, и другие с ними.
\vs Act 18:1 После сего Павел, оставив Афины, пришел в Коринф;
\vs Act 18:2 и, найдя некоторого Иудея, именем Акилу, родом Понтянина, недавно пришедшего из Италии, и Прискиллу, жену его,~--- потому что Клавдий повелел всем Иудеям удалиться из Рима,~--- пришел к ним;
\vs Act 18:3 и, по одинаковости ремесла, остался у них и работал; ибо ремеслом их было делание палаток.
\vs Act 18:4 Во всякую же субботу он говорил в синагоге и убеждал Иудеев и Еллинов.
\rsbpar\vs Act 18:5 А когда пришли из Македонии Сила и Тимофей, то Павел понуждаем был духом свидетельствовать Иудеям, что Иисус есть Христос.
\vs Act 18:6 Но как они противились и злословили, то он, отрясши одежды свои, сказал к ним: кровь ваша на главах ваших; я чист; отныне иду к язычникам.
\vs Act 18:7 И пошел оттуда, и пришел к некоторому чтущему Бога, именем Иусту, которого дом был подле синагоги.
\vs Act 18:8 Крисп же, начальник синагоги, уверовал в Господа со всем домом своим, и многие из Коринфян, слушая, уверовали и крестились.
\vs Act 18:9 Господь же в видении ночью сказал Павлу: не бойся, но говори и не умолкай,
\vs Act 18:10 ибо Я с тобою, и никто не сделает тебе зла, потому что у Меня много людей в этом городе.
\vs Act 18:11 И он оставался там год и шесть месяцев, поучая их слову Божию.
\rsbpar\vs Act 18:12 Между тем, во время проконсульства Галлиона в Ахаии, напали Иудеи единодушно на Павла и привели его пред судилище,
\vs Act 18:13 говоря, что он учит людей чтить Бога не по закону.
\vs Act 18:14 Когда же Павел хотел открыть уста, Галлион сказал Иудеям: Иудеи! если бы какая-нибудь была обида или злой умысел, то я имел бы причину выслушать вас,
\vs Act 18:15 но когда идет спор об учении и об именах и о законе вашем, то разбирайте сами; я не хочу быть судьею в этом.
\vs Act 18:16 И прогнал их от судилища.
\vs Act 18:17 А все Еллины, схватив Сосфена, начальника синагоги, били его перед судилищем; и Галлион нимало не беспокоился о том.
\rsbpar\vs Act 18:18 Павел, пробыв еще довольно дней, простился с братиями и отплыл в Сирию,~--- и с ним Акила и Прискилла,~--- остригши голову в Кенхреях, по обету.
\vs Act 18:19 Достигнув Ефеса, оставил их там, а сам вошел в синагогу и рассуждал с Иудеями.
\vs Act 18:20 Когда же они просили его побыть у них долее, он не согласился,
\vs Act 18:21 а простился с ними, сказав: мне нужно непременно провести приближающийся праздник в Иерусалиме; к вам же возвращусь опять, если будет угодно Богу. И отправился из Ефеса. (Акила же и Прискилла остались в Ефесе.)
\vs Act 18:22 Побывав в Кесарии, он приходил \bibemph{в Иерусалим}, приветствовал церковь и отошел в Антиохию.
\vs Act 18:23 И, проведя \bibemph{там} несколько времени, вышел, и проходил по порядку страну Галатийскую и Фригию, утверждая всех учеников.
\rsbpar\vs Act 18:24 Некто Иудей, именем Аполлос, родом из Александрии, муж красноречивый и сведущий в Писаниях, пришел в Ефес.
\vs Act 18:25 Он был наставлен в начатках пути Господня и, горя духом, говорил и учил о Господе правильно, зная только крещение Иоанново.
\vs Act 18:26 Он начал смело говорить в синагоге. Услышав его, Акила и Прискилла приняли его и точнее объяснили ему путь Господень.
\vs Act 18:27 А когда он вознамерился идти в Ахаию, то братия послали к \bibemph{тамошним} ученикам, располагая их принять его; и он, прибыв туда, много содействовал уверовавшим благодатью,
\vs Act 18:28 ибо он сильно опровергал Иудеев всенародно, доказывая Писаниями, что Иисус есть Христос.
\vs Act 19:1 Во время пребывания Аполлоса в Коринфе Павел, пройдя верхние страны, прибыл в Ефес и, найдя \bibemph{там} некоторых учеников,
\vs Act 19:2 сказал им: приняли ли вы Святаго Духа, уверовав? Они же сказали ему: мы даже и не слыхали, есть ли Дух Святый.
\vs Act 19:3 Он сказал им: во что же вы крестились? Они отвечали: во Иоанново крещение.
\vs Act 19:4 Павел сказал: Иоанн крестил крещением покаяния, говоря людям, чтобы веровали в Грядущего по нем, то есть во Христа Иисуса.
\vs Act 19:5 Услышав это, они крестились во имя Господа Иисуса,
\vs Act 19:6 и, когда Павел возложил на них руки, нисшел на них Дух Святый, и они стали говорить \bibemph{иными} языками и пророчествовать.
\vs Act 19:7 Всех их было человек около двенадцати.
\rsbpar\vs Act 19:8 Придя в синагогу, он небоязненно проповедовал три месяца, беседуя и удостоверяя о Царствии Божием.
\vs Act 19:9 Но как некоторые ожесточились и не верили, злословя путь Господень перед народом, то он, оставив их, отделил учеников, и ежедневно проповедовал в училище некоего Тиранна.
\vs Act 19:10 Это продолжалось до двух лет, так что все жители Асии слышали проповедь о Господе Иисусе, как Иудеи, так и Еллины.
\rsbpar\vs Act 19:11 Бог же творил немало чудес руками Павла,
\vs Act 19:12 так что на больных возлагали платки и опоясания с тела его, и у них прекращались болезни, и злые духи выходили из них.
\vs Act 19:13 Даже некоторые из скитающихся Иудейских заклинателей стали употреблять над имеющими злых духов имя Господа Иисуса, говоря: заклинаем вас Иисусом, Которого Павел проповедует.
\vs Act 19:14 Это делали какие-то семь сынов Иудейского первосвященника Скевы.
\vs Act 19:15 Но злой дух сказал в ответ: Иисуса знаю, и Павел мне известен, а вы кто?
\vs Act 19:16 И бросился на них человек, в котором был злой дух, и, одолев их, взял над ними такую силу, что они, нагие и избитые, выбежали из того дома.
\vs Act 19:17 Это сделалось известно всем живущим в Ефесе Иудеям и Еллинам, и напал страх на всех их, и величаемо было имя Господа Иисуса.
\vs Act 19:18 Многие же из уверовавших приходили, исповедуя и открывая дела свои.
\vs Act 19:19 А из занимавшихся чародейством довольно многие, собрав книги свои, сожгли перед всеми, и сложили цены их, и оказалось их на пятьдесят тысяч \bibemph{драхм}.
\vs Act 19:20 С такою силою возрастало и возмогало слово Господне.
\rsbpar\vs Act 19:21 Когда же это совершилось, Павел положил в духе, пройдя Македонию и Ахаию, идти в Иерусалим, сказав: побывав там, я должен видеть и Рим.
\vs Act 19:22 И, послав в Македонию двоих из служивших ему, Тимофея и Ераста, сам остался на время в Асии.
\rsbpar\vs Act 19:23 В то время произошел немалый мятеж против пути Господня,
\vs Act 19:24 ибо некто серебряник, именем Димитрий, делавший серебряные храмы Артемиды и доставлявший художникам немалую прибыль,
\vs Act 19:25 собрав их и других подобных ремесленников, сказал: друзья! вы знаете, что от этого ремесла зависит благосостояние наше;
\vs Act 19:26 между тем вы видите и слышите, что не только в Ефесе, но почти во всей Асии этот Павел своими убеждениями совратил немалое число людей, говоря, что делаемые руками человеческими не суть боги.
\vs Act 19:27 А это нам угрожает тем, что не только ремесло наше придет в презрение, но и храм великой богини Артемиды ничего не будет значить, и испровергнется величие той, которую почитает вся Асия и вселенная.
\vs Act 19:28 Выслушав это, они исполнились ярости и стали кричать, говоря: велика Артемида Ефесская!
\vs Act 19:29 И весь город наполнился смятением. Схватив Македонян Гаия и Аристарха, спутников Павловых, они единодушно устремились на зрелище.
\vs Act 19:30 Когда же Павел хотел войти в народ, ученики не допустили его.
\vs Act 19:31 Также и некоторые из Асийских начальников, будучи друзьями его, послав к нему, просили не показываться на зрелище.
\vs Act 19:32 Между тем одни кричали одно, а другие другое, ибо собрание было беспорядочное, и большая часть \bibemph{собравшихся} не знали, зачем собрались.
\vs Act 19:33 По предложению Иудеев, из народа вызван был Александр. Дав знак рукою, Александр хотел говорить к народу.
\vs Act 19:34 Когда же узнали, что он Иудей, то закричали все в один голос, и около двух часов кричали: велика Артемида Ефесская!
\vs Act 19:35 Блюститель же порядка, утишив народ, сказал: мужи Ефесские! какой человек не знает, что город Ефес есть служитель великой богини Артемиды и Диопета?
\vs Act 19:36 Если же в этом нет спора, то надобно вам быть спокойными и не поступать опрометчиво.
\vs Act 19:37 А вы привели этих мужей, которые ни храма Артемидина не обокрали, ни богини вашей не хулили.
\vs Act 19:38 Если же Димитрий и другие с ним художники имеют жалобу на кого-нибудь, то есть судебные собрания и есть проконсулы: пусть жалуются друг на друга.
\vs Act 19:39 А если вы ищете чего-нибудь другого, то это будет решено в законном собрании.
\vs Act 19:40 Ибо мы находимся в опасности~--- за происшедшее ныне быть обвиненными в возмущении, так как нет никакой причины, которою мы могли бы оправдать такое сборище. Сказав это, он распустил собрание.
\vs Act 20:1 По прекращении мятежа Павел, призвав учеников и дав им наставления и простившись с ними, вышел и пошел в Македонию.
\vs Act 20:2 Пройдя же те места и преподав \bibemph{верующим} обильные наставления, пришел в Елладу.
\vs Act 20:3 \bibemph{Там} пробыл он три месяца. Когда же, по случаю возмущения, сделанного против него Иудеями, он хотел отправиться в Сирию, то пришло ему на мысль возвратиться через Македонию.
\vs Act 20:4 Его сопровождали до Асии Сосипатр Пирров, Вериянин, и из Фессалоникийцев Аристарх и Секунд, и Гаий Дервянин и Тимофей, и Асийцы Тихик и Трофим.
\vs Act 20:5 Они, пойдя вперед, ожидали нас в Троаде.
\vs Act 20:6 А мы, после дней опресночных, отплыли из Филипп и дней в пять прибыли к ним в Троаду, где пробыли семь дней.
\rsbpar\vs Act 20:7 В первый же день недели, когда ученики собрались для преломления хлеба, Павел, намереваясь отправиться в следующий день, беседовал с ними и продолжил слово до полуночи.
\vs Act 20:8 В горнице, где мы собрались, было довольно светильников.
\vs Act 20:9 Во время продолжительной беседы Павловой один юноша, именем Евтих, сидевший на окне, погрузился в глубокий сон и, пошатнувшись, сонный упал вниз с третьего жилья, и поднят мертвым.
\vs Act 20:10 Павел, сойдя, пал на него и, обняв его, сказал: не тревожьтесь, ибо душа его в нем.
\vs Act 20:11 Взойдя же и преломив хлеб и вкусив, беседовал довольно, даже до рассвета, и потом вышел.
\vs Act 20:12 Между тем отрока привели живого, и немало утешились.
\rsbpar\vs Act 20:13 Мы пошли вперед на корабль и поплыли в Асс, чтобы взять оттуда Павла; ибо он так приказал нам, намереваясь сам идти пешком.
\vs Act 20:14 Когда же он сошелся с нами в Ассе, то, взяв его, мы прибыли в Митилину.
\vs Act 20:15 И, отплыв оттуда, в следующий день мы остановились против Хиоса, а на другой пристали к Самосу и, побывав в Трогиллии, в следующий \bibemph{день} прибыли в Милит,
\vs Act 20:16 ибо Павлу рассудилось миновать Ефес, чтобы не замедлить ему в Асии; потому что он поспешал, если можно, в день Пятидесятницы быть в Иерусалиме.
\rsbpar\vs Act 20:17 Из Милита же послав в Ефес, он призвал пресвитеров церкви,
\vs Act 20:18 и, когда они пришли к нему, он сказал им: вы знаете, как я с первого дня, в который пришел в Асию, все время был с вами,
\vs Act 20:19 работая Господу со всяким смиренномудрием и многими слезами, среди искушений, приключавшихся мне по злоумышлениям Иудеев;
\vs Act 20:20 как я не пропустил ничего полезного, о чем вам не проповедовал бы и чему не учил бы вас всенародно и по домам,
\vs Act 20:21 возвещая Иудеям и Еллинам покаяние пред Богом и веру в Господа нашего Иисуса Христа.
\vs Act 20:22 И вот, ныне я, по влечению Духа, иду в Иерусалим, не зная, чт\acc{о} там встретится со мною;
\vs Act 20:23 только Дух Святый по всем городам свидетельствует, говоря, что узы и скорби ждут меня.
\vs Act 20:24 Но я ни на что не взираю и не дорожу своею жизнью, только бы с радостью совершить поприще мое и служение, которое я принял от Господа Иисуса, проповедать Евангелие благодати Божией.
\vs Act 20:25 И ныне, вот, я знаю, что уже не увидите лица моего все вы, между которыми ходил я, проповедуя Царствие Божие.
\vs Act 20:26 Посему свидетельствую вам в нынешний день, что чист я от крови всех,
\vs Act 20:27 ибо я не упускал возвещать вам всю волю Божию.
\vs Act 20:28 Итак внимайте себе и всему стаду, в котором Дух Святый поставил вас блюстителями, пасти Церковь Господа и Бога, которую Он приобрел Себе Кровию Своею.
\vs Act 20:29 Ибо я знаю, что, по отшествии моем, войдут к вам лютые волки, не щадящие стада;
\vs Act 20:30 и из вас самих восстанут люди, которые будут говорить превратно, дабы увлечь учеников за собою.
\vs Act 20:31 Посему бодрствуйте, памятуя, что я три года день и ночь непрестанно со слезами учил каждого из вас.
\vs Act 20:32 И ныне предаю вас, братия, Богу и слову благодати Его, могущему назидать \bibemph{вас} более и дать вам наследие со всеми освященными.
\vs Act 20:33 Ни серебра, ни золота, ни одежды я ни от кого не пожелал:
\vs Act 20:34 сами знаете, что нуждам моим и \bibemph{нуждам} бывших при мне послужили руки мои сии.
\vs Act 20:35 Во всем показал я вам, что, так трудясь, надобно поддерживать слабых и памятовать слова Господа Иисуса, ибо Он Сам сказал: <<блаженнее давать, нежели принимать>>.
\vs Act 20:36 Сказав это, он преклонил колени свои и со всеми ими помолился.
\vs Act 20:37 Тогда немалый плач был у всех, и, падая на выю Павла, целовали его,
\vs Act 20:38 скорбя особенно от сказанного им слова, что они уже не увидят лица его. И провожали его до корабля.
\vs Act 21:1 Когда же мы, расставшись с ними, отплыли, то прямо пришли в Кос, на другой день в Родос и оттуда в Патару,
\vs Act 21:2 и, найдя корабль, идущий в Финикию, взошли на него и отплыли.
\vs Act 21:3 Быв в виду Кипра и оставив его слева, мы плыли в Сирию, и пристали в Тире, ибо тут надлежало сложить груз с корабля.
\vs Act 21:4 И, найдя учеников, пробыли там семь дней. Они, по \bibemph{внушению} Духа, говорили Павлу, чтобы он не ходил в Иерусалим.
\vs Act 21:5 Проведя эти дни, мы вышли и пошли, и нас провожали все с женами и детьми даже за город; а на берегу, преклонив колени, помолились.
\vs Act 21:6 И, простившись друг с другом, мы вошли в корабль, а они возвратились домой.
\rsbpar\vs Act 21:7 Мы же, совершив плавание, прибыли из Тира в Птолемаиду, где, приветствовав братьев, пробыли у них один день.
\vs Act 21:8 А на другой день Павел и мы, бывшие с ним, выйдя, пришли в Кесарию и, войдя в дом Филиппа благовестника, одного из семи \bibemph{диаконов}, остались у него.
\vs Act 21:9 У него были четыре дочери девицы, пророчествующие.
\vs Act 21:10 Между тем как мы пребывали у них многие дни, пришел из Иудеи некто пророк, именем Агав,
\vs Act 21:11 и, войдя к нам, взял пояс Павлов и, связав себе руки и ноги, сказал: так говорит Дух Святый: мужа, чей этот пояс, так свяжут в Иерусалиме Иудеи и предадут в руки язычников.
\vs Act 21:12 Когда же мы услышали это, то и мы и тамошние просили, чтобы он не ходил в Иерусалим.
\vs Act 21:13 Но Павел в ответ сказал: что вы делаете? что плачете и сокрушаете сердце мое? я не только хочу быть узником, но готов умереть в Иерусалиме за имя Господа Иисуса.
\vs Act 21:14 Когда же мы не могли уговорить его, то успокоились, сказав: да будет воля Господня!
\rsbpar\vs Act 21:15 После сих дней, приготовившись, пошли мы в Иерусалим.
\vs Act 21:16 С нами шли и некоторые ученики из Кесарии, провожая \bibemph{нас} к некоему давнему ученику, Мнасону Кипрянину, у которого можно было бы нам жить.
\rsbpar\vs Act 21:17 По прибытии нашем в Иерусалим братия радушно приняли нас.
\vs Act 21:18 На другой день Павел пришел с нами к Иакову; пришли и все пресвитеры.
\vs Act 21:19 Приветствовав их, \bibemph{Павел} рассказывал подробно, что сотворил Бог у язычников служением его.
\vs Act 21:20 Они же, выслушав, прославили Бога и сказали ему: видишь, брат, сколько тысяч уверовавших Иудеев, и все они ревнители закона.
\vs Act 21:21 А о тебе наслышались они, что ты всех Иудеев, живущих между язычниками, учишь отступлению от Моисея, говоря, чтобы они не обрезывали детей своих и не поступали по обычаям.
\vs Act 21:22 Итак что же? Верно соберется народ; ибо услышат, что ты пришел.
\vs Act 21:23 Сделай же, что мы скажем тебе: есть у нас четыре человека, имеющие на себе обет.
\vs Act 21:24 Взяв их, очистись с ними, и возьми на себя издержки на \bibemph{жертву} за них, чтобы остригли себе голову, и узнают все, что слышанное ими о тебе несправедливо, но что и сам ты продолжаешь соблюдать закон.
\vs Act 21:25 А об уверовавших язычниках мы писали, положив, чтобы они ничего такого не наблюдали, а только хранили себя от идоложертвенного, от крови, от удавленины и от блуда.
\vs Act 21:26 Тогда Павел, взяв тех мужей и очистившись с ними, в следующий день вошел в храм и объявил окончание дней очищения, когда должно быть принесено за каждого из них приношение.
\rsbpar\vs Act 21:27 Когда же семь дней оканчивались, тогда Асийские Иудеи, увидев его в храме, возмутили весь народ и наложили на него руки,
\vs Act 21:28 крича: мужи Израильские, помогите! этот человек всех повсюду учит против народа и закона и места сего; притом и Еллинов ввел в храм и осквернил святое место сие.
\vs Act 21:29 Ибо перед тем они видели с ним в городе Трофима Ефесянина и думали, что Павел его ввел в храм.
\vs Act 21:30 Весь город пришел в движение, и сделалось стечение народа; и, схватив Павла, повлекли его вон из храма, и тотчас заперты были двери.
\vs Act 21:31 Когда же они хотели убить его, до тысяченачальника полка дошла весть, что весь Иерусалим возмутился.
\vs Act 21:32 Он, тотчас взяв воинов и сотников, устремился на них; они же, увидев тысяченачальника и воинов, перестали бить Павла.
\vs Act 21:33 Тогда тысяченачальник, приблизившись, взял его и велел сковать двумя цепями, и спрашивал: кто он, и что сделал.
\vs Act 21:34 В народе одни кричали одно, а другие другое. Он же, не могши по причине смятения узнать ничего верного, повелел вести его в крепость.
\vs Act 21:35 Когда же он был на лестнице, то воинам пришлось нести его по причине стеснения от народа,
\vs Act 21:36 ибо множество народа следовало и кричало: смерть ему!
\rsbpar\vs Act 21:37 При входе в крепость Павел сказал тысяченачальнику: можно ли мне сказать тебе нечто? А тот сказал: ты знаешь по-гречески?
\vs Act 21:38 Так не ты ли тот Египтянин, который перед сими днями произвел возмущение и вывел в пустыню четыре тысячи человек разбойников?
\vs Act 21:39 Павел же сказал: я Иудеянин, Тарсянин, гражданин небезызвестного Киликийского города; прошу тебя, позволь мне говорить к народу.
\vs Act 21:40 Когда же тот позволил, Павел, стоя на лестнице, дал знак рукою народу; и, когда сделалось глубокое молчание, начал говорить на еврейском языке так:
\vs Act 22:1 Мужи братия и отцы! выслушайте теперь мое оправдание перед вами.
\vs Act 22:2 Услышав же, что он заговорил с ними на еврейском языке, они еще более утихли. Он сказал:
\vs Act 22:3 я Иудеянин, родившийся в Тарсе Киликийском, воспитанный в сем городе при ногах Гамалиила, тщательно наставленный в отеческом законе, ревнитель по Боге, как и все вы ныне.
\vs Act 22:4 Я даже до смерти гнал \bibemph{последователей} сего учения, связывая и предавая в темницу и мужчин и женщин,
\vs Act 22:5 как засвидетельствует о мне первосвященник и все старейшины, от которых и письма взяв к братиям, живущим в Дамаске, я шел, чтобы тамошних привести в оковах в Иерусалим на истязание.
\vs Act 22:6 Когда же я был в пути и приближался к Дамаску, около полудня вдруг осиял меня великий свет с неба.
\vs Act 22:7 Я упал на землю и услышал голос, говоривший мне: Савл, Савл! что ты гонишь Меня?
\vs Act 22:8 Я отвечал: кто Ты, Господи? Он сказал мне: Я Иисус Назорей, Которого ты гонишь.
\vs Act 22:9 Бывшие же со мною свет видели, и пришли в страх; но голоса Говорившего мне не слыхали.
\vs Act 22:10 Тогда я сказал: Господи! что мне делать? Господь же сказал мне: встань и иди в Дамаск, и там тебе сказано будет всё, что назначено тебе делать.
\vs Act 22:11 А как я от славы света того лишился зрения, то бывшие со мною за руку привели меня в Дамаск.
\vs Act 22:12 Некто Анания, муж благочестивый по закону, одобряемый всеми Иудеями, живущими в Дамаске,
\vs Act 22:13 пришел ко мне и, подойдя, сказал мне: брат Савл! прозри. И я тотчас увидел его.
\vs Act 22:14 Он же сказал мне: Бог отцов наших предъизбрал тебя, чтобы ты познал волю Его, увидел Праведника и услышал глас из уст Его,
\vs Act 22:15 потому что ты будешь Ему свидетелем пред всеми людьми о том, что ты видел и слышал.
\vs Act 22:16 Итак, что ты медлишь? Встань, крестись и омой грехи твои, призвав имя Господа Иисуса.
\vs Act 22:17 Когда же я возвратился в Иерусалим и молился в храме, пришел я в исступление,
\vs Act 22:18 и увидел Его, и Он сказал мне: поспеши и выйди скорее из Иерусалима, потому что \bibemph{здесь} не примут твоего свидетельства о Мне.
\vs Act 22:19 Я сказал: Господи! им известно, что я верующих в Тебя заключал в темницы и бил в синагогах,
\vs Act 22:20 и когда проливалась кровь Стефана, свидетеля Твоего, я там стоял, одобрял убиение его и стерег одежды побивавших его.
\vs Act 22:21 И Он сказал мне: иди; Я пошлю тебя далеко к язычникам.
\rsbpar\vs Act 22:22 До этого слова слушали его; а за сим подняли крик, говоря: истреби от земли такого! ибо ему не должно жить.
\vs Act 22:23 Между тем как они кричали, метали одежды и бросали пыль на воздух,
\vs Act 22:24 тысяченачальник повелел ввести его в крепость, приказав бичевать его, чтобы узнать, по какой причине так кричали против него.
\vs Act 22:25 Но когда растянули его ремнями, Павел сказал стоявшему сотнику: разве вам позволено бичевать Римского гражданина, да и без суда?
\vs Act 22:26 Услышав это, сотник подошел и донес тысяченачальнику, говоря: смотри, что ты хочешь делать? этот человек~--- Римский гражданин.
\vs Act 22:27 Тогда тысяченачальник, подойдя к нему, сказал: скажи мне, ты Римский гражданин? Он сказал: да.
\vs Act 22:28 Тысяченачальник отвечал: я за большие деньги приобрел это гражданство. Павел же сказал: а я и родился в нем.
\vs Act 22:29 Тогда тотчас отступили от него хотевшие пытать его. А тысяченачальник, узнав, что он Римский гражданин, испугался, что связал его.
\rsbpar\vs Act 22:30 На другой день, желая достоверно узнать, в чем обвиняют его Иудеи, освободил его от оков и повелел собраться первосвященникам и всему синедриону и, выведя Павла, поставил его перед ними.
\vs Act 23:1 Павел, устремив взор на синедрион, сказал: мужи братия! я всею доброю совестью жил пред Богом до сего дня.
\vs Act 23:2 Первосвященник же Анания стоявшим перед ним приказал бить его по устам.
\vs Act 23:3 Тогда Павел сказал ему: Бог будет бить тебя, стена подбеленная! ты сидишь, чтобы судить по закону, и, вопреки закону, велишь бить меня.
\vs Act 23:4 Предстоящие же сказали: первосвященника Божия поносишь?
\vs Act 23:5 Павел сказал: я не знал, братия, что он первосвященник; ибо написано: начальствующего в народе твоем не злословь.
\vs Act 23:6 Узнав же Павел, что \bibemph{тут} одна часть саддукеев, а другая фарисеев, возгласил в синедрионе: мужи братия! я фарисей, сын фарисея; за чаяние воскресения мертвых меня судят.
\vs Act 23:7 Когда же он сказал это, произошла распря между фарисеями и саддукеями, и собрание разделилось.
\vs Act 23:8 Ибо саддукеи говорят, что нет воскресения, ни Ангела, ни духа; а фарисеи признают и то и другое.
\vs Act 23:9 Сделался большой крик; и, встав, книжники фарисейской стороны спорили, говоря: ничего худого мы не находим в этом человеке; если же дух или Ангел говорил ему, не будем противиться Богу.
\vs Act 23:10 Но как раздор увеличился, то тысяченачальник, опасаясь, чтобы они не растерзали Павла, повелел воинам сойти взять его из среды их и отвести в крепость.
\rsbpar\vs Act 23:11 В следующую ночь Господь, явившись ему, сказал: дерзай, Павел; ибо, как ты свидетельствовал о Мне в Иерусалиме, так надлежит тебе свидетельствовать и в Риме.
\vs Act 23:12 С наступлением дня некоторые Иудеи сделали умысел, и заклялись не есть и не пить, доколе не убьют Павла.
\vs Act 23:13 Было же более сорока сделавших такое заклятие.
\vs Act 23:14 Они, придя к первосвященникам и старейшинам, сказали: мы клятвою заклялись не есть ничего, пока не убьем Павла.
\vs Act 23:15 Итак ныне же вы с синедрионом дайте знать тысяченачальнику, чтобы он завтра вывел его к вам, как будто вы хотите точнее рассмотреть дело о нем; мы же, прежде нежели он приблизится, готовы убить его.
\vs Act 23:16 Услышав о сем умысле, сын сестры Павловой пришел и, войдя в крепость, уведомил Павла.
\vs Act 23:17 Павел же, призвав одного из сотников, сказал: отведи этого юношу к тысяченачальнику, ибо он имеет нечто сказать ему.
\vs Act 23:18 Тот, взяв его, привел к тысяченачальнику и сказал: узник Павел, призвав меня, просил отвести к тебе этого юношу, который имеет нечто сказать тебе.
\vs Act 23:19 Тысяченачальник, взяв его за руку и отойдя с ним в сторону, спрашивал: что такое имеешь ты сказать мне?
\vs Act 23:20 Он отвечал, что Иудеи согласились просить тебя, чтобы ты завтра вывел Павла пред синедрион, как будто они хотят точнее исследовать дело о нем.
\vs Act 23:21 Но ты не слушай их; ибо его подстерегают более сорока человек из них, которые заклялись не есть и не пить, доколе не убьют его; и они теперь готовы, ожидая твоего распоряжения.
\vs Act 23:22 Тогда тысяченачальник отпустил юношу, сказав: никому не говори, что ты объявил мне это.
\rsbpar\vs Act 23:23 И, призвав двух сотников, сказал: приготовьте мне воинов \bibemph{пеших} двести, конных семьдесят и стрелков двести, чтобы с третьего часа ночи шли в Кесарию.
\vs Act 23:24 Приготовьте также ослов, чтобы, посадив Павла, препроводить его к правителю Феликсу.
\vs Act 23:25 Написал и письмо следующего содержания:
\vs Act 23:26 <<Клавдий Лисий достопочтенному правителю Феликсу~--- радоваться.
\vs Act 23:27 Сего человека Иудеи схватили и готовы были убить; я, придя с воинами, отнял его, узнав, что он Римский гражданин.
\vs Act 23:28 Потом, желая узнать, в чем обвиняли его, привел его в синедрион их
\vs Act 23:29 и нашел, что его обвиняют в спорных мнениях, касающихся закона их, но что нет в нем никакой вины, достойной смерти или оков.
\vs Act 23:30 А как до меня дошло, что Иудеи злоумышляют на этого человека, то я немедленно послал его к тебе, приказав и обвинителям говорить на него перед тобою. Будь здоров>>.
\rsbpar\vs Act 23:31 Итак воины, по \bibemph{данному} им приказанию, взяв Павла, повели ночью в Антипатриду.
\vs Act 23:32 А на другой день, предоставив конным идти с ним, возвратились в крепость.
\vs Act 23:33 А те, придя в Кесарию и отдав письмо правителю, представили ему и Павла.
\vs Act 23:34 Правитель, прочитав письмо, спросил, из какой он области, и, узнав, что из Киликии, сказал:
\vs Act 23:35 я выслушаю тебя, когда явятся твои обвинители. И повелел ему быть под стражею в Иродовой претории.
\vs Act 24:1 Через пять дней пришел первосвященник Анания со старейшинами и с некоторым ритором Тертуллом, которые жаловались правителю на Павла.
\vs Act 24:2 Когда же он был призван, то Тертулл начал обвинять его, говоря:
\vs Act 24:3 всегда и везде со всякою благодарностью признаём мы, что тебе, достопочтенный Феликс, обязаны мы многим миром, и твоему попечению благоустроением сего народа.
\vs Act 24:4 Но, чтобы много не утруждать тебя, прошу тебя выслушать нас кратко, со свойственным тебе снисхождением.
\vs Act 24:5 Найдя сего человека язвою \bibemph{общества}, возбудителем мятежа между Иудеями, живущими по вселенной, и представителем Назорейской ереси,
\vs Act 24:6 который отважился даже осквернить храм, мы взяли его и хотели судить его по нашему закону.
\vs Act 24:7 Но тысяченачальник Лисий, придя, с великим насилием взял его из рук наших и послал к тебе,
\vs Act 24:8 повелев и нам, обвинителям его, идти к тебе. Ты можешь сам, разобрав, узнать от него о всем том, в чем мы обвиняем его.
\vs Act 24:9 И Иудеи подтвердили, сказав, что это так.
\rsbpar\vs Act 24:10 Павел же, когда правитель дал ему знак говорить, отвечал: зная, что ты многие годы справедливо судишь народ сей, я тем свободнее буду защищать мое дело.
\vs Act 24:11 Ты можешь узнать, что не более двенадцати дней тому, как я пришел в Иерусалим для поклонения.
\vs Act 24:12 И ни в святилище, ни в синагогах, ни по городу они не находили меня с кем-либо спорящим или производящим народное возмущение,
\vs Act 24:13 и не могут доказать того, в чем теперь обвиняют меня.
\vs Act 24:14 Но в том признаюсь тебе, что по учению, которое они называют ересью, я действительно служу Богу отцов \bibemph{моих}, веруя всему, написанному в законе и пророках,
\vs Act 24:15 имея надежду на Бога, что будет воскресение мертвых, праведных и неправедных, чего и сами они ожидают.
\vs Act 24:16 Посему и сам подвизаюсь всегда иметь непорочную совесть пред Богом и людьми.
\vs Act 24:17 После многих лет я пришел, чтобы доставить милостыню народу моему и приношения.
\vs Act 24:18 При сем нашли меня, очистившегося в храме не с народом и не с шумом.
\vs Act 24:19 \bibemph{Это были} некоторые Асийские Иудеи, которым надлежало бы предстать пред тебя и обвинять меня, если что имеют против меня.
\vs Act 24:20 Или пусть сии самые скажут, какую нашли они во мне неправду, когда я стоял перед синедрионом,
\vs Act 24:21 разве только т\acc{о} одно слово, которое громко произнес я, стоя между ними, что за \bibemph{учение о} воскресении мертвых я ныне судим вами.
\rsbpar\vs Act 24:22 Выслушав это, Феликс отсрочил \bibemph{дело} их, сказав: рассмотрю ваше дело, когда придет тысяченачальник Лисий, и я обстоятельно узнаю об этом учении.
\vs Act 24:23 А Павла приказал сотнику стеречь, но не стеснять его и не запрещать никому из его близких служить ему или приходить к нему.
\rsbpar\vs Act 24:24 Через несколько дней Феликс, придя с Друзиллою, женою своею, Иудеянкою, призвал Павла, и слушал его о вере во Христа Иисуса.
\vs Act 24:25 И как он говорил о правде, о воздержании и о будущем суде, то Феликс пришел в страх и отвечал: теперь пойди, а когда найду время, позову тебя.
\vs Act 24:26 Притом же надеялся он, что Павел даст ему денег, чтобы отпустил его: посему часто призывал его и беседовал с ним.
\vs Act 24:27 Но по прошествии двух лет на место Феликса поступил Порций Фест. Желая доставить удовольствие Иудеям, Феликс оставил Павла в узах.
\vs Act 25:1 Фест, прибыв в область, через три дня отправился из Кесарии в Иерусалим.
\vs Act 25:2 Тогда первосвященник и знатнейшие из Иудеев явились к нему \bibemph{с жалобою} на Павла и убеждали его,
\vs Act 25:3 прося, чтобы он сделал милость, вызвал его в Иерусалим; и злоумышляли убить его на дороге.
\vs Act 25:4 Но Фест отвечал, что Павел содержится в Кесарии под стражею и что он сам скоро отправится туда.
\vs Act 25:5 Итак, сказал он, которые из вас могут, пусть пойдут со мною, и если есть что-нибудь за этим человеком, пусть обвиняют его.
\rsbpar\vs Act 25:6 Пробыв же у них не больше восьми или десяти дней, возвратился в Кесарию, и на другой день, сев на судейское место, повелел привести Павла.
\vs Act 25:7 Когда он явился, стали кругом пришедшие из Иерусалима Иудеи, принося на Павла многие и тяжкие обвинения, которых не могли доказать.
\vs Act 25:8 Он же в оправдание свое сказал: я не сделал никакого преступления ни против закона Иудейского, ни против храма, ни против кесаря.
\vs Act 25:9 Фест, желая сделать угождение Иудеям, сказал в ответ Павлу: хочешь ли идти в Иерусалим, чтобы я там судил тебя в этом?
\vs Act 25:10 Павел сказал: я сто\acc{ю} перед судом кесаревым, где мне и следует быть судиму. Иудеев я ничем не обидел, как и ты хорошо знаешь.
\vs Act 25:11 Ибо, если я неправ и сделал что-нибудь, достойное смерти, то не отрекаюсь умереть; а если ничего того нет, в чем сии обвиняют меня, то никто не может выдать меня им. Требую суда кесарева.
\vs Act 25:12 Тогда Фест, поговорив с советом, отвечал: ты потребовал суда кесарева, к кесарю и отправишься.
\rsbpar\vs Act 25:13 Через несколько дней царь Агриппа и Вереника прибыли в Кесарию поздравить Феста.
\vs Act 25:14 И как они провели там много дней, то Фест предложил царю дело Павлово, говоря: \bibemph{здесь} есть человек, оставленный Феликсом в узах,
\vs Act 25:15 на которого, в бытность мою в Иерусалиме, \bibemph{с жалобою} явились первосвященники и старейшины Иудейские, требуя осуждения его.
\vs Act 25:16 Я отвечал им, что у Римлян нет обыкновения выдавать какого-нибудь человека на смерть, прежде нежели обвиняемый будет иметь обвинителей налицо и получит свободу защищаться против обвинения.
\vs Act 25:17 Когда же они пришли сюда, то, без всякого отлагательства, на другой же день сел я на судейское место и повелел привести того человека.
\vs Act 25:18 Обступив его, обвинители не представили ни одного из обвинений, какие я предполагал;
\vs Act 25:19 но они имели некоторые споры с ним об их Богопочитании и о каком-то Иисусе умершем, о Котором Павел утверждал, что Он жив.
\vs Act 25:20 Затрудняясь в решении этого вопроса, я сказал: хочет ли он идти в Иерусалим и там быть судимым в этом?
\vs Act 25:21 Но как Павел потребовал, чтобы он оставлен был на рассмотрение Августово, то я велел содержать его под стражею до тех пор, как пошлю его к кесарю.
\vs Act 25:22 Агриппа же сказал Фесту: хотел бы и я послушать этого человека. Завтра же, отвечал тот, услышишь его.
\rsbpar\vs Act 25:23 На другой день, когда Агриппа и Вереника пришли с великою пышностью и вошли в судебную палату с тысяченачальниками и знатнейшими гражданами, по приказанию Феста приведен был Павел.
\vs Act 25:24 И сказал Фест: царь Агриппа и все присутствующие с нами мужи! вы видите того, против которого всё множество Иудеев приступали ко мне в Иерусалиме и здесь и кричали, что ему не должно более жить.
\vs Act 25:25 Но я нашел, что он не сделал ничего, достойного смерти; и как он сам потребовал суда у Августа, то я решился послать его \bibemph{к нему}.
\vs Act 25:26 Я не имею ничего верного написать о нем государю; посему привел его пред вас, и особенно пред тебя, царь Агриппа, дабы, по рассмотрении, было мне что написать.
\vs Act 25:27 Ибо, мне кажется, нерассудительно послать узника и не показать обвинений на него.
\vs Act 26:1 Агриппа сказал Павлу: позволяется тебе говорить за себя. Тогда Павел, простерши руку, стал говорить в свою защиту:
\vs Act 26:2 царь Агриппа! почитаю себя счастливым, что сегодня могу защищаться перед тобою во всем, в чем обвиняют меня Иудеи,
\vs Act 26:3 тем более, что ты знаешь все обычаи и спорные мнения Иудеев. Посему прошу тебя выслушать меня великодушно.
\vs Act 26:4 Жизнь мою от юности \bibemph{моей}, которую сначала проводил я среди народа моего в Иерусалиме, знают все Иудеи;
\vs Act 26:5 они издавна знают обо мне, если захотят свидетельствовать, что я жил фарисеем по строжайшему в нашем вероисповедании учению.
\vs Act 26:6 И ныне я сто\acc{ю} перед судом за надежду на обетование, данное от Бога нашим отцам,
\vs Act 26:7 которого исполнение надеются увидеть наши двенадцать колен, усердно служа \bibemph{Богу} день и ночь. За сию-то надежду, царь Агриппа, обвиняют меня Иудеи.
\vs Act 26:8 Что же? Неужели вы невероятным почитаете, что Бог воскрешает мертвых?
\vs Act 26:9 Правда, и я думал, что мне должно много действовать против имени Иисуса Назорея.
\vs Act 26:10 Это я и делал в Иерусалиме: получив власть от первосвященников, я многих святых заключал в темницы, и, когда убивали их, я подавал на то голос;
\vs Act 26:11 и по всем синагогам я многократно мучил их и принуждал хулить \bibemph{Иисуса} и, в чрезмерной против них ярости, преследовал даже и в чужих городах.
\vs Act 26:12 Для сего, идя в Дамаск со властью и поручением от первосвященников,
\vs Act 26:13 среди дня на дороге я увидел, государь, с неба свет, превосходящий солнечное сияние, осиявший меня и шедших со мною.
\vs Act 26:14 Все мы упали на землю, и я услышал голос, говоривший мне на еврейском языке: Савл, Савл! что ты гонишь Меня? Трудно тебе идти против рожна.
\vs Act 26:15 Я сказал: кто Ты, Господи? Он сказал: <<Я Иисус, Которого ты гонишь.
\vs Act 26:16 Но встань и стань на ноги твои; ибо Я для того и явился тебе, чтобы поставить тебя служителем и свидетелем того, что ты видел и что Я открою тебе,
\vs Act 26:17 избавляя тебя от народа Иудейского и от язычников, к которым Я теперь посылаю тебя
\vs Act 26:18 открыть глаза им, чтобы они обратились от тьмы к свету и от власти сатаны к Богу, и верою в Меня получили прощение грехов и жребий с освященными>>.
\vs Act 26:19 Поэтому, царь Агриппа, я не воспротивился небесному видению,
\vs Act 26:20 но сперва жителям Дамаска и Иерусалима, потом всей земле Иудейской и язычникам проповедовал, чтобы они покаялись и обратились к Богу, делая дела, достойные покаяния.
\vs Act 26:21 За это схватили меня Иудеи в храме и покушались растерзать.
\vs Act 26:22 Но, получив помощь от Бога, я до сего дня стою, свидетельствуя малому и великому, ничего не говоря, кроме того, о чем пророки и Моисей говорили, что это будет,
\vs Act 26:23 \bibemph{то есть} что Христос имел пострадать и, восстав первый из мертвых, возвестить свет народу (Иудейскому) и язычникам.
\vs Act 26:24 Когда он так защищался, Фест громким голосом сказал: безумствуешь ты, Павел! большая ученость доводит тебя до сумасшествия.
\vs Act 26:25 Нет, достопочтенный Фест, сказал он, я не безумствую, но говорю слова истины и здравого смысла.
\vs Act 26:26 Ибо знает об этом царь, перед которым и говорю смело. Я отнюдь не верю, чтобы от него было что-нибудь из сего скрыто; ибо это не в углу происходило.
\vs Act 26:27 Веришь ли, царь Агриппа, пророкам? Знаю, что веришь.
\vs Act 26:28 Агриппа сказал Павлу: ты немного не убеждаешь меня сделаться Христианином.
\vs Act 26:29 Павел сказал: молил бы я Бога, чтобы мало ли, много ли, не только ты, но и все, слушающие меня сегодня, сделались такими, как я, кроме этих уз.
\vs Act 26:30 Когда он сказал это, царь и правитель, Вереника и сидевшие с ними встали;
\vs Act 26:31 и, отойдя в сторону, говорили между собою, что этот человек ничего, достойного смерти или уз, не делает.
\vs Act 26:32 И сказал Агриппа Фесту: можно было бы освободить этого человека, если бы он не потребовал суда у кесаря. Посему и решился правитель послать его к кесарю.
\vs Act 27:1 Когда решено было плыть нам в Италию, то отдали Павла и некоторых других узников сотнику Августова полка, именем Юлию.
\vs Act 27:2 Мы взошли на Адрамитский корабль и отправились, намереваясь плыть около Асийских мест. С нами был Аристарх, Македонянин из Фессалоники.
\vs Act 27:3 На другой \bibemph{день} пристали к Сидону. Юлий, поступая с Павлом человеколюбиво, позволил ему сходить к друзьям и воспользоваться их усердием.
\vs Act 27:4 Отправившись оттуда, мы приплыли в Кипр, по причине противных ветров,
\vs Act 27:5 и, переплыв море против Киликии и Памфилии, прибыли в Миры Ликийские.
\vs Act 27:6 Там сотник нашел Александрийский корабль, плывущий в Италию, и посадил нас на него.
\vs Act 27:7 Медленно плавая многие дни и едва поровнявшись с Книдом, по причине неблагоприятного нам ветра, мы подплыли к Криту при Салмоне.
\vs Act 27:8 Пробравшись же с трудом мимо него, прибыли к одному месту, называемому Хорошие Пристани, близ которого был город Ласея.
\vs Act 27:9 Но как прошло довольно времени, и плавание было уже опасно, потому что и пост уже прошел, то Павел советовал,
\vs Act 27:10 говоря им: мужи! я вижу, что плавание будет с затруднениями и с большим вредом не только для груза и корабля, но и для нашей жизни.
\vs Act 27:11 Но сотник более доверял кормчему и начальнику корабля, нежели словам Павла.
\vs Act 27:12 А как пристань не была приспособлена к зимовке, то многие давали совет отправиться оттуда, чтобы, если можно, дойти до Финика, пристани Критской, лежащей против юго-западного и северо-западного ветра, и \bibemph{там} перезимовать.
\vs Act 27:13 Подул южный ветер, и они, подумав, что уже получили желаемое, отправились, и поплыли поблизости Крита.
\vs Act 27:14 Но скоро поднялся против него ветер бурный, называемый эвроклидон.
\vs Act 27:15 Корабль схватило так, что он не мог противиться ветру, и мы носились, отдавшись волнам.
\vs Act 27:16 И, набежав на один островок, называемый Кл\acc{а}вдой, мы едва могли удержать лодку.
\vs Act 27:17 Подняв ее, стали употреблять пособия и обвязывать корабль; боясь же, чтобы не сесть на мель, спустили парус и таким образом носились.
\vs Act 27:18 На другой день, по причине сильного обуревания, начали выбрасывать \bibemph{груз},
\vs Act 27:19 а на третий мы своими руками побросали с корабля вещи.
\vs Act 27:20 Но как многие дни не видно было ни солнца, ни звезд и продолжалась немалая буря, то наконец исчезала всякая надежда к нашему спасению.
\vs Act 27:21 И как долго не ели, то Павел, став посреди них, сказал: мужи! надлежало послушаться меня и не отходить от Крита, чем и избежали бы сих затруднений и вреда.
\vs Act 27:22 Теперь же убеждаю вас ободриться, потому что ни одна душа из вас не погибнет, а только корабль.
\vs Act 27:23 Ибо Ангел Бога, Которому принадлежу я и Которому служу, явился мне в эту ночь
\vs Act 27:24 и сказал: <<не бойся, Павел! тебе должно предстать пред кесаря, и вот, Бог даровал тебе всех плывущих с тобою>>.
\vs Act 27:25 Посему ободритесь, мужи, ибо я верю Богу, что будет так, как мне сказано.
\vs Act 27:26 Нам должно быть выброшенными на какой-нибудь остров.
\rsbpar\vs Act 27:27 В четырнадцатую ночь, как мы носимы были в Адриатическом море, около полуночи корабельщики стали догадываться, что приближаются к какой-то земле,
\vs Act 27:28 и, вымерив глубину, нашли двадцать сажен; потом на небольшом расстоянии, вымерив опять, нашли пятнадцать сажен.
\vs Act 27:29 Опасаясь, чтобы не попасть на каменистые места, бросили с кормы четыре якоря, и ожидали дня.
\vs Act 27:30 Когда же корабельщики хотели бежать с корабля и спускали на море лодку, делая вид, будто хотят бросить якоря с носа,
\vs Act 27:31 Павел сказал сотнику и воинам: если они не останутся на корабле, то вы не можете спастись.
\vs Act 27:32 Тогда воины отсекли веревки у лодки, и она упала.
\vs Act 27:33 Перед наступлением дня Павел уговаривал всех принять пищу, говоря: сегодня четырнадцатый день, как вы, в ожидании, остаетесь без пищи, не вкушая ничего.
\vs Act 27:34 Потому прошу вас принять пищу: это послужит к сохранению вашей жизни; ибо ни у кого из вас не пропадет волос с головы.
\vs Act 27:35 Сказав это и взяв хлеб, он возблагодарил Бога перед всеми и, разломив, начал есть.
\vs Act 27:36 Тогда все ободрились и также приняли пищу.
\vs Act 27:37 Было же всех нас на корабле двести семьдесят шесть душ.
\vs Act 27:38 Насытившись же пищею, стали облегчать корабль, выкидывая пшеницу в море.
\vs Act 27:39 Когда настал день, земли не узнавали, а усмотрели только некоторый залив, имеющий \bibemph{отлогий} берег, к которому и решились, если можно, пристать с кораблем.
\vs Act 27:40 И, подняв якоря, пошли по морю и, развязав рули и подняв малый парус по ветру, держали к берегу.
\vs Act 27:41 Попали на косу, и корабль сел на мель. Нос увяз и остался недвижим, а корма разбивалась силою волн.
\vs Act 27:42 Воины согласились было умертвить узников, чтобы кто-нибудь, выплыв, не убежал.
\vs Act 27:43 Но сотник, желая спасти Павла, удержал их от сего намерения, и велел умеющим плавать первым броситься и выйти на землю,
\vs Act 27:44 прочим же \bibemph{спасаться} кому на досках, а кому на чем-нибудь от корабля; и таким образом все спаслись на землю.
\vs Act 28:1 Спасшись же, бывшие с Павлом узнали, что остров называется Мелит.
\vs Act 28:2 Иноплеменники оказали нам немалое человеколюбие, ибо они, по причине бывшего дождя и холода, разложили огонь и приняли всех нас.
\vs Act 28:3 Когда же Павел набрал множество хвороста и клал на огонь, тогда ехидна, выйдя от жара, повисла на руке его.
\vs Act 28:4 Иноплеменники, когда увидели висящую на руке его змею, говорили друг другу: верно этот человек~--- убийца, когда его, спасшегося от моря, суд \bibemph{Божий} не оставляет жить.
\vs Act 28:5 Но он, стряхнув змею в огонь, не потерпел никакого вреда.
\vs Act 28:6 Они ожидали было, что у него будет воспаление, или он внезапно упадет мертвым; но, ожидая долго и видя, что не случилось с ним никакой беды, переменили мысли и говорили, что он Бог.
\rsbpar\vs Act 28:7 Около того места были поместья начальника острова, именем Публия; он принял нас и три дня дружелюбно угощал.
\vs Act 28:8 Отец Публия лежал, страдая горячкою и болью в животе; Павел вошел к нему, помолился и, возложив на него руки свои, исцелил его.
\vs Act 28:9 После сего события и прочие на острове, имевшие болезни, приходили и были исцеляемы,
\vs Act 28:10 и оказывали нам много почести и при отъезде снабдили нужным.
\rsbpar\vs Act 28:11 Через три месяца мы отплыли на Александрийском корабле, называемом Диоскуры, зимовавшем на том острове,
\vs Act 28:12 и, приплыв в Сиракузы, пробыли там три дня.
\vs Act 28:13 Оттуда отплыв, прибыли в Ригию; и как через день подул южный ветер, прибыли на второй день в Путеол,
\vs Act 28:14 где нашли братьев, и были упрошены пробыть у них семь дней, а потом пошли в Рим.
\vs Act 28:15 Тамошние братья, услышав о нас, вышли нам навстречу до Аппиевой площади и трех гостиниц. Увидев их, Павел возблагодарил Бога и ободрился.
\vs Act 28:16 Когда же пришли мы в Рим, то сотник передал узников военачальнику, а Павлу позволено жить особо с воином, стерегущим его.
\rsbpar\vs Act 28:17 Через три дня Павел созвал знатнейших из Иудеев и, когда они сошлись, говорил им: мужи братия! не сделав ничего против народа или отеческих обычаев, я в узах из Иерусалима предан в руки Римлян.
\vs Act 28:18 Они, судив меня, хотели освободить, потому что нет во мне никакой вины, достойной смерти;
\vs Act 28:19 но так как Иудеи противоречили, то я принужден был потребовать суда у кесаря, впрочем не с тем, чтобы обвинить в чем-либо мой народ.
\vs Act 28:20 По этой причине я и призвал вас, чтобы увидеться и поговорить с вами, ибо за надежду Израилеву обложен я этими узами.
\vs Act 28:21 Они же сказали ему: мы ни писем не получали о тебе из Иудеи, ни из приходящих братьев никто не известил о тебе и не сказал чего-либо худого.
\vs Act 28:22 Впрочем желательно нам слышать от тебя, как ты мыслишь; ибо известно нам, что об этом учении везде спорят.
\vs Act 28:23 И, назначив ему день, очень многие пришли к нему в гостиницу; и он от утра до вечера излагал им \bibemph{учение} о Царствии Божием, приводя свидетельства и удостоверяя их о Иисусе из закона Моисеева и пророков.
\vs Act 28:24 Одни убеждались словами его, а другие не верили.
\vs Act 28:25 Будучи же не согласны между собою, они уходили, когда Павел сказал следующие слова: хорошо Дух Святый сказал отцам нашим через пророка Исаию:
\vs Act 28:26 пойди к народу сему и скажи: слухом услышите, и не уразумеете, и очами смотреть будете, и не увидите.
\vs Act 28:27 Ибо огрубело сердце людей сих, и ушами с трудом слышат, и очи свои сомкнули, да не узрят очами, и не услышат ушами, и не уразумеют сердцем, и не обратятся, чтобы Я исцелил их.
\vs Act 28:28 Итак да будет вам известно, что спасение Божие послано язычникам: они и услышат.
\vs Act 28:29 Когда он сказал это, Иудеи ушли, много споря между собою.
\rsbpar\vs Act 28:30 И жил Павел целых два года на своем иждивении и принимал всех, приходивших к нему,
\vs Act 28:31 проповедуя Царствие Божие и уча о Господе Иисусе Христе со всяким дерзновением невозбранно.
\vs Act 29:1 И Павел, исполненный благословений Христа и преизобилующий в духе, удалился из Рима, решив идти в Испанию; ибо он давно имел намерение на путешествие туда, и задумал также идти оттуда в Британию.\fns{Гл.~29: Оригинальная греческая рукопись была найдена во второй половине XVIII столетия в архивах Константинополя и подарена султаном Абдул Ахметом французскому учёному К.С.~Соннини, который перевёл её на английский язык и издал в 1801 году.}
\vs Act 29:2
Ибо он слышал в Финикии, что некоторые дети Израилевы,
со времени Ассирийского плена, бежали морем
на острова отдалённые, как сказано пророком,
и называвшиеся римлянами Британией.
\vs Act 29:3
И Господь повелел, чтобы евангелие было проповедано
далеко отсюда народам и заблудшим овцам дома Израилева.
\vs Act 29:4
И никто не препятствовал Павлу;
ибо он смело свидетельствовал об Исусе
пред трибунами и среди людей;
и он взял с собою некоторых из братьев,
которые находились с ним в Риме, и они,
погрузившись в Остриуме и имея попутные ветры,
благополучно прибыли в пристань Испании.
\vs Act 29:5
И множество людей собралось вместе из городов
и селений и горной местности;
ибо они услышали об обращении к апостолу и многих чудесах,
которые он совершал.
\vs Act 29:6
И Павел дерзновенно проповедывал в Испании,
и великое множество уверовало и было обращено;
ибо они уразумели, что он~--- апостол, посланный от Бога.
\vs Act 29:7
И они удалились из Испании, и Павел и его спутники,
найдя судно, отплывающее в Арморику, к Британии,
куда они желали, и продвигаясь вдоль южного берега,
они достигли порта, называемого Рафин. 
\vs Act 29:8
Ныне, когда об этом было возвещено повсюду,
что апостол высадится на их берег,
великое множество обитателей встретило его,
и они обходились с Павлом учтиво и он вошёл
в восточные ворота их города и поселился
в доме еврея и одного из его племени.

\vs Act 29:9
И назавтра он пришёл и стал на горе Луд;
и народ толпился у прохода, и собрались на дороге,
и он проповедовал им Христа,
и они поверили слову и свидетельству об Исусе.
\vs Act 29:10
И к вечеру Святой Дух сошёл на Павла,
и он пророчествовал, говоря:
<<Вот, в последние дни Бог мира пребудет
в городах и поэтому жители будут исчислены:
и в 7-м исчислении людей их глаза откроются,
и слава их наследия впредь возсияет перед ними.
Народы пойдут поклоняться на гору,
свидетельствовавшую о страдании
и долготерпении раба Господня. 
\vs Act 29:11
И в последние дни новая весть
о евангелии произойдет из Иерусалима,
и сердца людей возрадуются,
и вот, источники разверзнутся,
и больше не будет безпокойства.
\vs Act 29:12
В те дни будут в\acc{о}йны и слухи войн;
и царь возстанет, и его меч будет для исцеления народов,
и его миротворчество будет неизменно,
и слава его царства~--- удивление среди князей.>>
\vs Act 29:13
И вот пришли передать, что некоторые из друидов
пришли к Павлу лично; и показали свои обряды и церемонии,
унаследованные ими от иудеев, которые бежали из рабства
в земле Египетской; и апостол поверил их словам,
и он дал им целование мира.
\vs Act 29:14
И Павел пребывал в своём жилище 3 месяца,
утверждая в вере и непрестанно проповедуя Христа.
\vs Act 29:15
И после этих деяний Павел и его братья удалились из Рафина,
и поплыли на Атиум в Галлию.
\vs Act 29:16
И Павел проповедовал в Римском гарнизоне и среди народа,
увещевая всех мужчин раскаяться и исповедовать их грехи.

\vs Act 29:17
И вот пришли к нему некоторые из бельгов разузнать
о его новом учении и человеке Иисусе;
и Павел открыл им своё сердце, и поведал им всё то,
что произошло с ним, в том числе, как Иисус Христос
пришёл в мир, чтобы спасти грешников;
и они ушли, размышляя между собою о том,
что они услышали.

\vs Act 29:18
И после длительной проповеди и труда Павел
и его соработники перешли в Гельветию,
и пришли к горе Понтия Пилата, где тот,
кто осудил Господа Иисуса,
бросился вниз головой, и так жалко погиб.
\vs Act 29:19
И тотчас поток хлынул из скалы и смыл его тело,
разбившееся на части, в озеро.
\vs Act 29:20
И Павел простёр свои руки на воду, 
и молился Господу, говоря:
<<О Господи Боже, дай знамение всем народам,
что здесь Понтий Пилат, который осудил твоего единородного сына,
низвергся вниз головой в бездну.>>
\vs Act 29:21
И пока Павел ещё говорил, вот,
там случилось сильное землетрясение,
и лицо вод изменилось, и вид озера стал подобен
сыну человеческому, висящему в мучении на кресте.
\vs Act 29:22
И голос изшёл с небес, говоря:
<<Даже Пилат избегает грядущего гнева,
ибо он умыл свои руки перед толпой
при пролитии крови Господа Иисуса.>>
\vs Act 29:23
Поэтому когда Павел и те, что были с ним,
увидeли землетрясение и услышали голос ангела,
они прославили Бога и сильно укрепились в духе.

\vs Act 29:24
И они путешествовали и пришли к горе Юлия,
где стояли 2 столпа, один справа, а другой слева,
воздвигнутые кесарем Августом.
\vs Act 29:25
И Павел, исполнившись Святым Духом,
стал между 2-я столпами, говоря:
<<Мужи и братья, эти камни, которые вы видите сегодня,
будут свидетельствовать о моём путешествии отсюда;
и поистине я скажу: они останутся до излияния Духа на все народы,
никакой путь не будет препятствовать этому во всех поколениях.>>

\vs Act 29:26
И они отправились дальше и пришли в Иллирик,
собираясь идти через Македонию в Асию,
и благодать обреталась во всех церквях,
и они преуспевали и имели мир. Аминь!
\newbookpage
\bibbookdescr{Jam}{
  inline={Соборное Послание\\\LARGE Святого Апостола Иакова},
  toc={Иакова},
  bookmark={Иакова},
  header={Иакова},
  %headerleft={},
  %headerright={},
  abbr={Иак}
}
\vs Jam 1:1 Иаков, раб Бога и Господа Иисуса Христа, двенадцати коленам, находящимся в рассеянии,~--- радоваться.
\vs Jam 1:2 С великою радостью принимайте, братия мои, когда впадаете в различные искушения,
\vs Jam 1:3 зная, что испытание вашей веры производит терпение;
\vs Jam 1:4 терпение же должно иметь совершенное действие, чтобы вы были совершенны во всей полноте, без всякого недостатка.
\vs Jam 1:5 Если же у кого из вас недостает мудрости, да просит у Бога, дающего всем просто и без упреков,~--- и дастся ему.
\vs Jam 1:6 Но да просит с верою, нимало не сомневаясь, потому что сомневающийся подобен морской волне, ветром поднимаемой и развеваемой.
\vs Jam 1:7 Да не думает такой человек получить что-нибудь от Господа.
\vs Jam 1:8 Человек с двоящимися мыслями не тверд во всех путях своих.
\rsbpar\vs Jam 1:9 Да хвалится брат униженный высотою своею,
\vs Jam 1:10 а богатый~--- унижением своим, потому что он прейдет, как цвет на траве.
\vs Jam 1:11 Восходит солнце, \bibemph{настает} зной, и зноем иссушает траву, цвет ее опадает, исчезает красота вида ее; так увядает и богатый в путях своих.
\rsbpar\vs Jam 1:12 Блажен человек, который переносит искушение, потому что, быв испытан, он получит венец жизни, который обещал Господь любящим Его.
\vs Jam 1:13 В искушении никто не говори: Бог меня искушает; потому что Бог не искушается злом и Сам не искушает никого,
\vs Jam 1:14 но каждый искушается, увлекаясь и обольщаясь собственною похотью;
\vs Jam 1:15 похоть же, зачав, рождает грех, а сделанный грех рождает смерть.
\rsbpar\vs Jam 1:16 Не обманывайтесь, братия мои возлюбленные.
\vs Jam 1:17 Всякое даяние доброе и всякий дар совершенный нисходит свыше, от Отца светов, у Которого нет изменения и ни тени перемены.
\vs Jam 1:18 Восхотев, родил Он нас словом истины, чтобы нам быть некоторым начатком Его созданий.
\rsbpar\vs Jam 1:19 Итак, братия мои возлюбленные, всякий человек да будет скор на слышание, медлен на слова, медлен на гнев,
\vs Jam 1:20 ибо гнев человека не творит правды Божией.
\vs Jam 1:21 Посему, отложив всякую нечистоту и остаток злобы, в кротости примите насаждаемое слово, могущее спасти ваши души.
\vs Jam 1:22 Будьте же исполнители слова, а не слышатели только, обманывающие самих себя.
\vs Jam 1:23 Ибо, кто слушает слово и не исполняет, тот подобен человеку, рассматривающему природные черты лица своего в зеркале:
\vs Jam 1:24 он посмотрел на себя, отошел и тотчас забыл, каков он.
\vs Jam 1:25 Но кто вникнет в закон совершенный, \bibemph{закон} свободы, и пребудет в нем, тот, будучи не слушателем забывчивым, но исполнителем дела, блажен будет в своем действии.
\vs Jam 1:26 Если кто из вас думает, что он благочестив, и не обуздывает своего языка, но обольщает свое сердце, у того пустое благочестие.
\vs Jam 1:27 Чистое и непорочное благочестие пред Богом и Отцем есть то, чтобы призирать сирот и вдов в их скорбях и хранить себя неоскверненным от мира.
\vs Jam 2:1 Братия мои! имейте веру в Иисуса Христа нашего Господа славы, не взирая на лица.
\vs Jam 2:2 Ибо, если в собрание ваше войдет человек с золотым перстнем, в богатой одежде, войдет же и бедный в скудной одежде,
\vs Jam 2:3 и вы, смотря на одетого в богатую одежду, скажете ему: тебе хорошо сесть здесь, а бедному скажете: ты стань там, или садись здесь, у ног моих,~---
\vs Jam 2:4 то не пересуживаете ли вы в себе и не становитесь ли судьями с худыми мыслями?
\vs Jam 2:5 Послушайте, братия мои возлюбленные: не бедных ли мира избрал Бог быть богатыми верою и наследниками Царствия, которое Он обещал любящим Его?
\vs Jam 2:6 А вы презрели бедного. Не богатые ли притесняют вас, и не они ли влекут вас в суды?
\vs Jam 2:7 Не они ли бесславят доброе имя, которым вы называетесь?
\vs Jam 2:8 Если вы исполняете закон царский, по Писанию: возлюби ближнего твоего, как себя самого,~--- хорошо делаете.
\vs Jam 2:9 Но если поступаете с лицеприятием, то грех делаете, и перед законом оказываетесь преступниками.
\vs Jam 2:10 Кто соблюдает весь закон и согрешит в одном чем-нибудь, тот становится виновным во всем.
\vs Jam 2:11 Ибо Тот же, Кто сказал: не прелюбодействуй, сказал и: не убей; посему, если ты не прелюбодействуешь, но убьешь, то ты также преступник закона.
\vs Jam 2:12 Т\acc{а}к говорите и т\acc{а}к поступайте, как имеющие быть судимы по закону свободы.
\vs Jam 2:13 Ибо суд без милости не оказавшему милости; милость превозносится над судом.
\rsbpar\vs Jam 2:14 Чт\acc{о} пользы, братия мои, если кто говорит, что он имеет веру, а дел не имеет? может ли эта вера спасти его?
\vs Jam 2:15 Если брат или сестра наги и не имеют дневного пропитания,
\vs Jam 2:16 а кто-нибудь из вас скажет им: <<идите с миром, грейтесь и питайтесь>>, но не даст им потребного для тела: что пользы?
\vs Jam 2:17 Так и вера, если не имеет дел, мертва сама по себе.
\vs Jam 2:18 Но скажет кто-нибудь: <<ты имеешь веру, а я имею дела>>: покажи мне веру твою без дел твоих, а я покажу тебе веру мою из дел моих.
\vs Jam 2:19 Ты веруешь, что Бог един: хорошо делаешь; и бесы веруют, и трепещут.
\vs Jam 2:20 Но хочешь ли знать, неосновательный человек, что вера без дел мертва?
\vs Jam 2:21 Не делами ли оправдался Авраам, отец наш, возложив на жертвенник Исаака, сына своего?
\vs Jam 2:22 Видишь ли, что вера содействовала делам его, и делами вера достигла совершенства?
\vs Jam 2:23 И исполнилось слово Писания: <<веровал Авраам Богу, и это вменилось ему в праведность, и он наречен другом Божиим>>.
\vs Jam 2:24 Видите ли, что человек оправдывается делами, а не верою только?
\vs Jam 2:25 Подобно и Раав блудница не делами ли оправдалась, приняв соглядатаев и отпустив их другим путем?
\vs Jam 2:26 Ибо, как тело без духа мертво, так и вера без дел мертва.
\vs Jam 3:1 Братия мои! не многие делайтесь учителями, зная, что мы подвергнемся большему осуждению,
\vs Jam 3:2 ибо все мы много согрешаем. Кто не согрешает в слове, тот человек совершенный, могущий обуздать и все тело.
\vs Jam 3:3 Вот, мы влагаем удила в рот коням, чтобы они повиновались нам, и управляем всем телом их.
\vs Jam 3:4 Вот, и корабли, как ни велики они и как ни сильными ветрами носятся, небольшим рулем направляются, куда хочет кормчий;
\vs Jam 3:5 так и язык~--- небольшой член, но много делает. Посмотри, небольшой огонь как много вещества зажигает!
\vs Jam 3:6 И язык~--- огонь, прикраса неправды; язык в таком положении находится между членами нашими, что оскверняет все тело и воспаляет круг жизни, будучи сам воспаляем от геенны.
\vs Jam 3:7 Ибо всякое естество зверей и птиц, пресмыкающихся и морских животных укрощается и укрощено естеством человеческим,
\vs Jam 3:8 а язык укротить никто из людей не может: это~--- неудержимое зло; он исполнен смертоносного яда.
\vs Jam 3:9 Им благословляем Бога и Отца, и им проклинаем человеков, сотворенных по подобию Божию.
\vs Jam 3:10 Из тех же уст исходит благословение и проклятие: не должно, братия мои, сему так быть.
\vs Jam 3:11 Течет ли из одного отверстия источника сладкая и горькая \bibemph{вода}?
\vs Jam 3:12 Не может, братия мои, смоковница приносить маслины или виноградная лоза смоквы. Также и один источник не \bibemph{может} изливать соленую и сладкую воду.
\rsbpar\vs Jam 3:13 Мудр ли и разумен кто из вас, докажи это на самом деле добрым поведением с мудрою кротостью.
\vs Jam 3:14 Но если в вашем сердце вы имеете горькую зависть и сварливость, то не хвалитесь и не лгите на истину.
\vs Jam 3:15 Это не есть мудрость, нисходящая свыше, но земная, душевная, бесовская,
\vs Jam 3:16 ибо где зависть и сварливость, там неустройство и всё худое.
\vs Jam 3:17 Но мудрость, сходящая свыше, во-первых, чиста, потом мирна, скромна, послушлива, полна милосердия и добрых плодов, беспристрастна и нелицемерна.
\vs Jam 3:18 Плод же правды в мире сеется у тех, которые хранят мир.
\vs Jam 4:1 Откуда у вас вражды и распри? не отсюда ли, от вожделений ваших, воюющих в членах ваших?
\vs Jam 4:2 Желаете~--- и не имеете; убиваете и завидуете~--- и не можете достигнуть; препираетесь и враждуете~--- и не имеете, потому что не пр\acc{о}сите.
\vs Jam 4:3 Пр\acc{о}сите, и не получаете, потому что пр\acc{о}сите не на добро, а чтобы употребить для ваших вожделений.
\vs Jam 4:4 Прелюбодеи и прелюбодейцы! не знаете ли, что дружба с миром есть вражда против Бога? Итак, кто хочет быть другом миру, тот становится врагом Богу.
\vs Jam 4:5 Или вы думаете, что напрасно говорит Писание: <<до ревности любит дух, живущий в нас>>?
\vs Jam 4:6 Но тем б\acc{о}льшую дает благодать; посему и сказано: Бог гордым противится, а смиренным дает благодать.
\rsbpar\vs Jam 4:7 Итак покоритесь Богу; противостаньте диаволу, и убежит от вас.
\vs Jam 4:8 Приблизьтесь к Богу, и приблизится к вам; очистите руки, грешники, исправьте сердца, двоедушные.
\vs Jam 4:9 Сокрушайтесь, плачьте и рыдайте; смех ваш да обратится в плач, и радость~--- в печаль.
\vs Jam 4:10 Смиритесь пред Господом, и вознесет вас.
\rsbpar\vs Jam 4:11 Не злословьте друг друга, братия: кто злословит брата или судит брата своего, тот злословит закон и судит закон; а если ты судишь закон, то ты не исполнитель закона, но судья.
\vs Jam 4:12 Един Законодатель и Судия, могущий спасти и погубить; а ты кто, который судишь другого?
\rsbpar\vs Jam 4:13 Теперь послушайте вы, говорящие: <<сегодня или завтра отправимся в такой-то город, и проживем там один год, и будем торговать и получать прибыль>>;
\vs Jam 4:14 вы, которые не знаете, что случится завтра: ибо что такое жизнь ваша? пар, являющийся на малое время, а потом исчезающий.
\vs Jam 4:15 Вместо того, чтобы вам говорить: <<если угодно будет Господу и живы будем, то сделаем то или другое>>,~---
\vs Jam 4:16 вы, по своей надменности, тщеславитесь: всякое такое тщеславие есть зло.
\vs Jam 4:17 Итак, кто разумеет делать добро и не делает, тому грех.
\vs Jam 5:1 Послушайте вы, богатые: плачьте и рыдайте о бедствиях ваших, находящих на вас.
\vs Jam 5:2 Богатство ваше сгнило, и одежды ваши изъедены молью.
\vs Jam 5:3 Золото ваше и серебро изоржавело, и ржавчина их будет свидетельством против вас и съест плоть вашу, как огонь: вы собрали себе сокровище на последние дни.
\vs Jam 5:4 Вот, плата, удержанная вами у работников, пожавших поля ваши, вопиет, и вопли жнецов дошли до слуха Господа Саваофа.
\vs Jam 5:5 Вы роскошествовали на земле и наслаждались; напитали сердца ваши, как бы на день заклания.
\vs Jam 5:6 Вы осудили, убили Праведника; Он не противился вам.
\rsbpar\vs Jam 5:7 Итак, братия, будьте долготерпеливы до пришествия Господня. Вот, земледелец ждет драгоценного плода от земли и для него терпит долго, пока получит дождь ранний и поздний.
\vs Jam 5:8 Долготерп\acc{и}те и вы, укрепите сердца ваши, потому что пришествие Господне приближается.
\vs Jam 5:9 Не сетуйте, братия, друг на друга, чтобы не быть осужденными: вот, Судия стоит у дверей.
\vs Jam 5:10 В пример злострадания и долготерпения возьмите, братия мои, пророков, которые говорили именем Господним.
\vs Jam 5:11 Вот, мы ублажаем тех, которые терпели. Вы слышали о терпении Иова и видели конец \bibemph{оного} от Господа, ибо Господь весьма милосерд и сострадателен.
\rsbpar\vs Jam 5:12 Прежде же всего, братия мои, не клянитесь ни небом, ни землею, и никакою другою клятвою, но да будет у вас: <<да, да>> и <<нет, нет>>, дабы вам не подпасть осуждению.
\rsbpar\vs Jam 5:13 Злостраждет ли кто из вас, пусть молится. Весел ли кто, пусть поет псалмы.
\vs Jam 5:14 Болен ли кто из вас, пусть призовет пресвитеров Церкви, и пусть помолятся над ним, помазав его елеем во имя Господне.
\vs Jam 5:15 И молитва веры исцелит болящего, и восставит его Господь; и если он соделал грехи, простятся ему.
\rsbpar\vs Jam 5:16 Признавайтесь друг пред другом в проступках и молитесь друг за друга, чтобы исцелиться: много может усиленная молитва праведного.
\vs Jam 5:17 Илия был человек, подобный нам, и молитвою помолился, чтобы не было дождя: и не было дождя на землю три года и шесть месяцев.
\vs Jam 5:18 И опять помолился: и небо дало дождь, и земля произрастила плод свой.
\rsbpar\vs Jam 5:19 Братия! если кто из вас уклонится от истины, и обратит кто его,
\vs Jam 5:20 пусть тот знает, что обративший грешника от ложного пути его спасет душу от смерти и покроет множество грехов.

\bibbookdescr{1Pe}{
  inline={Первое Соборное Послание\\\LARGE Святого Апостола Петра},
  toc={1-е Петра},
  bookmark={1-е Петра},
  header={1-е Петра},
  %headerleft={},
  %headerright={},
  abbr={1~Пет}
}
\vs 1Pe 1:1 Петр, Апостол Иисуса Христа, пришельцам, рассеянным в Понте, Галатии, Каппадокии, Асии и Вифинии, избранным,
\vs 1Pe 1:2 по предведению Бога Отца, при освящении от Духа, к послушанию и окроплению Кровию Иисуса Христа: благодать вам и мир да умножится.
\rsbpar\vs 1Pe 1:3 Благословен Бог и Отец Господа нашего Иисуса Христа, по великой Своей милости возродивший нас воскресением Иисуса Христа из мертвых к упованию живому,
\vs 1Pe 1:4 к наследству нетленному, чистому, неувядаемому, хранящемуся на небесах для вас,
\vs 1Pe 1:5 силою Божиею через веру соблюдаемых ко спасению, готовому открыться в последнее время.
\vs 1Pe 1:6 О сем радуйтесь, поскорбев теперь немного, если нужно, от различных искушений,
\vs 1Pe 1:7 дабы испытанная вера ваша оказалась драгоценнее гибнущего, хотя и огнем испытываемого золота, к похвале и чести и славе в явление Иисуса Христа,
\vs 1Pe 1:8 Которого, не видев, любите, и Которого доселе не видя, но веруя в Него, радуетесь радостью неизреченною и преславною,
\vs 1Pe 1:9 достигая наконец верою вашею спасения душ.
\vs 1Pe 1:10 К сему-то спасению относились изыскания и исследования пророков, которые предсказывали о назначенной вам благодати,
\vs 1Pe 1:11 исследуя, на которое и на какое время указывал сущий в них Дух Христов, когда Он предвозвещал Христовы страдания и последующую за ними славу.
\vs 1Pe 1:12 Им открыто было, что не им самим, а нам служило то, что ныне проповедано вам благовествовавшими Духом Святым, посланным с небес, во что желают проникнуть Ангелы.
\rsbpar\vs 1Pe 1:13 Посему, (возлюбленные,) препоясав чресла ума вашего, бодрствуя, совершенно уповайте на подаваемую вам благодать в явлении Иисуса Христа.
\vs 1Pe 1:14 Как послушные дети, не сообразуйтесь с прежними похотями, бывшими в неведении вашем,
\vs 1Pe 1:15 но, по примеру призвавшего вас Святаго, и сами будьте святы во всех поступках.
\vs 1Pe 1:16 Ибо написано: будьте святы, потому что Я свят.
\vs 1Pe 1:17 И если вы называете Отцем Того, Который нелицеприятно судит каждого по делам, то со страхом провод\acc{и}те время странствования вашего,
\vs 1Pe 1:18 зная, что не тленным серебром или золотом искуплены вы от суетной жизни, преданной вам от отцов,
\vs 1Pe 1:19 но драгоценною Кровию Христа, как непорочного и чистого Агнца,
\vs 1Pe 1:20 предназначенного еще прежде создания мира, но явившегося в последние времена для вас,
\vs 1Pe 1:21 уверовавших чрез Него в Бога, Который воскресил Его из мертвых и дал Ему славу, чтобы вы имели веру и упование на Бога.
\rsbpar\vs 1Pe 1:22 Послушанием истине чрез Духа, очистив души ваши к нелицемерному братолюбию, постоянно люб\acc{и}те друг друга от чистого сердца,
\vs 1Pe 1:23 \bibemph{как} возрожденные не от тленного семени, но от нетленного, от слова Божия, живаго и пребывающего вовек.
\vs 1Pe 1:24 Ибо всякая плоть~--- как трава, и всякая слава человеческая~--- как цвет на траве: засохла трава, и цвет ее опал;
\vs 1Pe 1:25 но слово Господне пребывает вовек; а это есть то слово, которое вам проповедано.
\vs 1Pe 2:1 Итак, отложив всякую злобу и всякое коварство, и лицемерие, и зависть, и всякое злословие,
\vs 1Pe 2:2 как новорожденные младенцы, возлюб\acc{и}те чистое словесное молоко, дабы от него возрасти вам во спасение;
\vs 1Pe 2:3 ибо вы вкусили, что благ Господь.
\vs 1Pe 2:4 Приступая к Нему, камню живому, человеками отверженному, но Богом избранному, драгоценному,
\vs 1Pe 2:5 и сами, как живые камни, устрояйте из себя дом духовный, священство святое, чтобы приносить духовные жертвы, благоприятные Богу Иисусом Христом.
\vs 1Pe 2:6 Ибо сказано в Писании: вот, Я полагаю в Сионе камень краеугольный, избранный, драгоценный; и верующий в Него не постыдится.
\vs 1Pe 2:7 Итак Он для вас, верующих, драгоценность, а для неверующих камень, который отвергли строители, но который сделался главою угла, камень претыкания и камень соблазна,
\vs 1Pe 2:8 о который они претыкаются, не покоряясь слову, на что они и оставлены.
\vs 1Pe 2:9 Но вы~--- род избранный, царственное священство, народ святой, люди, взятые в удел, дабы возвещать совершенства Призвавшего вас из тьмы в чудный Свой свет;
\vs 1Pe 2:10 некогда не народ, а ныне народ Божий; \bibemph{некогда} непомилованные, а ныне помилованы.
\vs 1Pe 2:11 Возлюбленные! прошу вас, как пришельцев и странников, удаляться от плотских похотей, восстающих на душу,
\vs 1Pe 2:12 и провождать добродетельную жизнь между язычниками, дабы они за то, за что злословят вас, как злодеев, увидя добрые дела ваши, прославили Бога в день посещения.
\vs 1Pe 2:13 Итак будьте покорны всякому человеческому начальству, для Господа: царю ли, как верховной власти,
\vs 1Pe 2:14 правителям ли, как от него посылаемым для наказания преступников и для поощрения делающих добро,~---
\vs 1Pe 2:15 ибо такова есть воля Божия, чтобы мы, делая добро, заграждали уста невежеству безумных людей,~---
\vs 1Pe 2:16 как свободные, не как употребляющие свободу для прикрытия зла, но как рабы Божии.
\vs 1Pe 2:17 Всех почитайте, братство любите, Бога бойтесь, царя чтите.
\rsbpar\vs 1Pe 2:18 Слуги, со всяким страхом повинуйтесь господам, не только добрым и кротким, но и суровым.
\vs 1Pe 2:19 Ибо то угодно Богу, если кто, помышляя о Боге, переносит скорби, страдая несправедливо.
\vs 1Pe 2:20 Ибо что за похвала, если вы терпите, когда вас бьют за проступки? Но если, делая добро и страдая, терпите, это угодно Богу.
\vs 1Pe 2:21 Ибо вы к тому призваны, потому что и Христос пострадал за нас, оставив нам пример, дабы мы шли по следам Его.
\vs 1Pe 2:22 Он не сделал никакого греха, и не было лести в устах Его.
\vs 1Pe 2:23 Будучи злословим, Он не злословил взаимно; страдая, не угрожал, но предавал то Судии Праведному.
\vs 1Pe 2:24 Он грехи наши Сам вознес телом Своим на древо, дабы мы, избавившись от грехов, жили для правды: ранами Его вы исцелились.
\vs 1Pe 2:25 Ибо вы были, как овцы блуждающие (не имея пастыря), но возвратились ныне к Пастырю и Блюстителю душ ваших.
\vs 1Pe 3:1 Также и вы, жены, повинуйтесь своим мужьям, чтобы те из них, которые не покоряются слову, житием жен своих без слова приобретаемы были,
\vs 1Pe 3:2 когда увидят ваше чистое, богобоязненное житие.
\vs 1Pe 3:3 Да будет украшением вашим не внешнее плетение волос, не золотые уборы или нарядность в одежде,
\vs 1Pe 3:4 но сокровенный сердца человек в нетленной \bibemph{красоте} кроткого и молчаливого духа, что драгоценно пред Богом.
\vs 1Pe 3:5 Так некогда и святые жены, уповавшие на Бога, украшали себя, повинуясь своим мужьям.
\vs 1Pe 3:6 Так Сарра повиновалась Аврааму, называя его господином. Вы~--- дети ее, если делаете добро и не смущаетесь ни от какого страха.
\rsbpar\vs 1Pe 3:7 Также и вы, мужья, обращайтесь благоразумно с женами, как с немощнейшим сосудом, оказывая им честь, как сонаследницам благодатной жизни, дабы не было вам препятствия в молитвах.
\rsbpar\vs 1Pe 3:8 Наконец будьте все единомысленны, сострадательны, братолюбивы, милосерды, дружелюбны, смиренномудры;
\vs 1Pe 3:9 не воздавайте злом за зло или ругательством за ругательство; напротив, благословляйте, зная, что вы к тому призваны, чтобы наследовать благословение.
\vs 1Pe 3:10 Ибо, кто любит жизнь и хочет видеть добрые дни, тот удерживай язык свой от зла и уста свои от лукавых речей;
\vs 1Pe 3:11 уклоняйся от зла и делай добро; ищи мира и стремись к нему,
\vs 1Pe 3:12 потому что очи Господа \bibemph{обращены} к праведным и уши Его к молитве их, но лице Господне против делающих зло (чтобы истребить их с земли).
\vs 1Pe 3:13 И кто сделает вам зло, если вы будете ревнителями доброго?
\vs 1Pe 3:14 Но если и страдаете за правду, то вы блаженны; а страха их не бойтесь и не смущайтесь.
\rsbpar\vs 1Pe 3:15 Господа Бога святите в сердцах ваших; \bibemph{будьте} всегда готовы всякому, требующему у вас отчета в вашем уповании, дать ответ с кротостью и благоговением.
\vs 1Pe 3:16 Имейте добрую совесть, дабы тем, за что злословят вас, как злодеев, были постыжены порицающие ваше доброе житие во Христе.
\vs 1Pe 3:17 Ибо, если угодно воле Божией, лучше пострадать за добрые дела, нежели за злые;
\vs 1Pe 3:18 потому что и Христос, чтобы привести нас к Богу, однажды пострадал за грехи наши, праведник за неправедных, быв умерщвлен по плоти, но ожив духом,
\vs 1Pe 3:19 которым Он и находящимся в темнице духам, сойдя, проповедал,
\vs 1Pe 3:20 некогда непокорным ожидавшему их Божию долготерпению, во дни Ноя, во время строения ковчега, в котором немногие, то есть восемь душ, спаслись от воды.
\vs 1Pe 3:21 Так и нас ныне подобное сему образу крещение, не плотской нечистоты омытие, но обещание Богу доброй совести, спасает воскресением Иисуса Христа,
\vs 1Pe 3:22 Который, восшед на небо, пребывает одесную Бога и Которому покорились Ангелы и Власти и Силы.
\vs 1Pe 4:1 Итак, как Христос пострадал за нас плотию, то и вы вооружитесь тою же мыслью; ибо страдающий плотию перестает грешить,
\vs 1Pe 4:2 чтобы остальное во плоти время жить уже не по человеческим похотям, но по воле Божией.
\vs 1Pe 4:3 Ибо довольно, что вы в прошедшее время жизни поступали по воле языческой, предаваясь нечистотам, похотям (мужеложству, скотоложству, помыслам), пьянству, излишеству в пище и питии и нелепому идолослужению;
\vs 1Pe 4:4 почему они и дивятся, что вы не участвуете с ними в том же распутстве, и злословят вас.
\vs 1Pe 4:5 Они дадут ответ Имеющему вскоре судить живых и мертвых.
\vs 1Pe 4:6 Ибо для того и мертвым было благовествуемо, чтобы они, подвергшись суду по человеку плотию, жили по Богу духом.
\vs 1Pe 4:7 Впрочем близок всему конец.\rsbpar Итак будьте благоразумны и бодрствуйте в молитвах.
\vs 1Pe 4:8 Более же всего имейте усердную любовь друг ко другу, потому что любовь покрывает множество грехов.
\vs 1Pe 4:9 Будьте страннолюбивы друг ко другу без ропота.
\vs 1Pe 4:10 Служите друг другу, каждый тем даром, какой получил, как добрые домостроители многоразличной благодати Божией.
\vs 1Pe 4:11 Говорит ли кто, \bibemph{говори} как слова Божии; служит ли кто, \bibemph{служи} по силе, какую дает Бог, дабы во всем прославлялся Бог через Иисуса Христа, Которому слава и держава во веки веков. Аминь.
\rsbpar\vs 1Pe 4:12 Возлюбленные! огненного искушения, для испытания вам посылаемого, не чуждайтесь, как приключения для вас странного,
\vs 1Pe 4:13 но как вы участвуете в Христовых страданиях, радуйтесь, да и в явление славы Его возрадуетесь и восторжествуете.
\vs 1Pe 4:14 Если злословят вас за имя Христово, то вы блаженны, ибо Дух Славы, Дух Божий почивает на вас. Теми Он хулится, а вами прославляется.
\vs 1Pe 4:15 Только бы не пострадал кто из вас, как убийца, или вор, или злодей, или как посягающий на чужое;
\vs 1Pe 4:16 а если как Христианин, то не стыдись, но прославляй Бога за такую участь.
\vs 1Pe 4:17 Ибо время начаться суду с дома Божия; если же прежде с нас \bibemph{начнется}, то какой конец непокоряющимся Евангелию Божию?
\vs 1Pe 4:18 И если праведник едва спасается, то нечестивый и грешный где явится?
\vs 1Pe 4:19 Итак страждущие по воле Божией да предадут Ему, как верному Создателю, души свои, делая добро.
\vs 1Pe 5:1 Пастырей ваших умоляю я, сопастырь и свидетель страданий Христовых и соучастник в славе, которая должна открыться:
\vs 1Pe 5:2 пасите Божие стадо, какое у вас, надзирая за ним не принужденно, но охотно и богоугодно, не для гнусной корысти, но из усердия,
\vs 1Pe 5:3 и не господствуя над наследием \bibemph{Божиим}, но подавая пример стаду;
\vs 1Pe 5:4 и когда явится Пастыреначальник, вы получите неувядающий венец славы.
\vs 1Pe 5:5 Также и младшие, повинуйтесь пастырям; все же, подчиняясь друг другу, облекитесь смиренномудрием, потому что Бог гордым противится, а смиренным дает благодать.
\rsbpar\vs 1Pe 5:6 Итак смиритесь под крепкую руку Божию, да вознесет вас в свое время.
\vs 1Pe 5:7 Все заботы ваши возлож\acc{и}те на Него, ибо Он печется о вас.
\vs 1Pe 5:8 Трезвитесь, бодрствуйте, потому что противник ваш диавол ходит, как рыкающий лев, ища, кого поглотить.
\vs 1Pe 5:9 Противостойте ему твердою верою, зная, что такие же страдания случаются и с братьями вашими в мире.
\vs 1Pe 5:10 Бог же всякой благодати, призвавший нас в вечную славу Свою во Христе Иисусе, Сам, по кратковременном страдании вашем, да совершит вас, да утвердит, да укрепит, да соделает непоколебимыми.
\vs 1Pe 5:11 Ему слава и держава во веки веков. Аминь.
\rsbpar\vs 1Pe 5:12 Сие кратко написал я вам чрез Силуана, верного, как думаю, вашего брата, чтобы уверить вас, утешая и свидетельствуя, что это истинная благодать Божия, в которой вы стоите.
\rsbpar\vs 1Pe 5:13 Приветствует вас избранная, подобно \bibemph{вам, церковь} в Вавилоне и Марк, сын мой.
\vs 1Pe 5:14 Приветствуйте друг друга лобзанием любви. Мир вам всем во Христе Иисусе. Аминь.

\bibbookdescr{2Pe}{
  inline={Второе Соборное Послание\\\LARGE Святого Апостола Петра},
  toc={2-е Петра},
  bookmark={2-е Петра},
  header={2-е Петра},
  %headerleft={},
  %headerright={},
  abbr={2~Пет}
}
\vs 2Pe 1:1 Симон Петр, раб и Апостол Иисуса Христа, принявшим с нами равно драгоценную веру по правде Бога нашего и Спасителя Иисуса Христа:
\vs 2Pe 1:2 благодать и мир вам да умножится в познании Бога и Христа Иисуса, Господа нашего.
\rsbpar\vs 2Pe 1:3 Как от Божественной силы Его даровано нам все потребное для жизни и благочестия, через познание Призвавшего нас славою и благостию,
\vs 2Pe 1:4 которыми дарованы нам великие и драгоценные обетования, дабы вы через них соделались причастниками Божеского естества, удалившись от господствующего в мире растления похотью:
\vs 2Pe 1:5 то вы, прилагая к сему все старание, покажите в вере вашей добродетель, в добродетели рассудительность,
\vs 2Pe 1:6 в рассудительности воздержание, в воздержании терпение, в терпении благочестие,
\vs 2Pe 1:7 в благочестии братолюбие, в братолюбии любовь.
\vs 2Pe 1:8 Если это в вас есть и умножается, то вы не останетесь без успеха и плода в познании Господа нашего Иисуса Христа.
\vs 2Pe 1:9 А в ком нет сего, тот слеп, закрыл глаза, забыл об очищении прежних грехов своих.
\vs 2Pe 1:10 Посему, братия, более и более старайтесь делать твердым ваше звание и избрание; так поступая, никогда не преткнетесь,
\vs 2Pe 1:11 ибо так откроется вам свободный вход в вечное Царство Господа нашего и Спасителя Иисуса Христа.
\rsbpar\vs 2Pe 1:12 Для того я никогда не перестану напоминать вам о сем, хотя вы то и знаете, и утверждены в настоящей истине.
\vs 2Pe 1:13 Справедливым же почитаю, доколе нахожусь в этой \bibemph{телесной} храмине, возбуждать вас напоминанием,
\vs 2Pe 1:14 зная, что скоро должен оставить храмину мою, как и Господь наш Иисус Христос открыл мне.
\vs 2Pe 1:15 Буду же стараться, чтобы вы и после моего отшествия всегда приводили это на память.
\vs 2Pe 1:16 Ибо мы возвестили вам силу и пришествие Господа нашего Иисуса Христа, не хитросплетенным басням последуя, но быв очевидцами Его величия.
\vs 2Pe 1:17 Ибо Он принял от Бога Отца честь и славу, когда от велелепной славы принесся к Нему такой глас: Сей есть Сын Мой возлюбленный, в Котором Мое благоволение.
\vs 2Pe 1:18 И этот глас, принесшийся с небес, мы слышали, будучи с Ним на святой горе.
\vs 2Pe 1:19 И притом мы имеем вернейшее пророческое слово; и вы хорошо делаете, что обращаетесь к нему, как к светильнику, сияющему в темном месте, доколе не начнет рассветать день и не взойдет утренняя звезда в сердцах ваших,
\vs 2Pe 1:20 зная прежде всего то, что никакого пророчества в Писании нельзя разрешить самому собою.
\vs 2Pe 1:21 Ибо никогда пророчество не было произносимо по воле человеческой, но изрекали его святые Божии человеки, будучи движимы Духом Святым.
\vs 2Pe 2:1 Были и лжепророки в народе, как и у вас будут лжеучители, которые введут пагубные ереси и, отвергаясь искупившего их Господа, навлекут сами на себя скорую погибель.
\vs 2Pe 2:2 И многие последуют их разврату, и через них путь истины будет в поношении.
\vs 2Pe 2:3 И из любостяжания будут уловлять вас льстивыми словами; суд им давно готов, и погибель их не дремлет.
\vs 2Pe 2:4 Ибо, если Бог ангелов согрешивших не пощадил, но, связав узами адского мрака, предал блюсти на суд для наказания;
\vs 2Pe 2:5 и если не пощадил первого мира, но в восьми душах сохранил семейство Ноя, проповедника правды, когда навел потоп на мир нечестивых;
\vs 2Pe 2:6 и если города Содомские и Гоморрские, осудив на истребление, превратил в пепел, показав пример будущим нечестивцам,
\vs 2Pe 2:7 а праведного Лота, утомленного обращением между людьми неистово развратными, избавил
\vs 2Pe 2:8 (ибо сей праведник, живя между ними, ежедневно мучился в праведной душе, видя и слыша дела беззаконные)~---
\vs 2Pe 2:9 то, конечно, знает Господь, как избавлять благочестивых от искушения, а беззаконников соблюдать ко дню суда, для наказания,
\vs 2Pe 2:10 а наипаче тех, которые идут вслед скверных похотей плоти, презирают начальства, дерзки, своевольны и не страшатся злословить высших,
\vs 2Pe 2:11 тогда как и Ангелы, превосходя их крепостью и силою, не произносят на них пред Господом укоризненного суда.
\vs 2Pe 2:12 Они, как бессловесные животные, водимые природою, рожденные на уловление и истребление, злословя то, чего не понимают, в растлении своем истребятся.
\vs 2Pe 2:13 Они получат возмездие за беззаконие, ибо они полагают удовольствие во вседневной роскоши; срамники и осквернители, они наслаждаются обманами своими, пиршествуя с вами.
\vs 2Pe 2:14 Глаза у них исполнены любострастия и непрестанного греха; они прельщают неутвержденные души; сердце их приучено к любостяжанию: это сыны проклятия.
\vs 2Pe 2:15 Оставив прямой путь, они заблудились, идя по следам Валаама, сына Восорова, который возлюбил мзду неправедную,
\vs 2Pe 2:16 но был обличен в своем беззаконии: бессловесная ослица, проговорив человеческим голосом, остановила безумие пророка.
\vs 2Pe 2:17 Это безводные источники, облака и мглы, гонимые бурею: им приготовлен мрак вечной тьмы.
\vs 2Pe 2:18 Ибо, произнося надутое пустословие, они уловляют в плотские похоти и разврат тех, которые едва отстали от находящихся в заблуждении.
\vs 2Pe 2:19 Обещают им свободу, будучи сами рабы тления; ибо, кто кем побежден, тот тому и раб.
\vs 2Pe 2:20 Ибо если, избегнув скверн мира чрез познание Господа и Спасителя нашего Иисуса Христа, опять запутываются в них и побеждаются ими, то последнее бывает для таковых хуже первого.
\vs 2Pe 2:21 Лучше бы им не познать пути правды, нежели, познав, возвратиться назад от преданной им святой заповеди.
\vs 2Pe 2:22 Но с ними случается по верной пословице: пес возвращается на свою блевотину, и: вымытая свинья \bibemph{идет} валяться в грязи.
\vs 2Pe 3:1 Это уже второе послание пишу к вам, возлюбленные; в них напоминанием возбуждаю ваш чистый смысл,
\vs 2Pe 3:2 чтобы вы помнили слова, прежде реченные святыми пророками, и заповедь Господа и Спасителя, преданную Апостолами вашими.
\vs 2Pe 3:3 Прежде всего знайте, что в последние дни явятся наглые ругатели, поступающие по собственным своим похотям
\vs 2Pe 3:4 и говорящие: где обетование пришествия Его? Ибо с тех пор, как стали умирать отцы, от начала творения, всё остается так же.
\vs 2Pe 3:5 Думающие так не знают, что вначале словом Божиим небеса и земля составлены из воды и водою:
\vs 2Pe 3:6 потому тогдашний мир погиб, быв потоплен водою.
\vs 2Pe 3:7 А нынешние небеса и земля, содержимые тем же Словом, сберегаются огню на день суда и погибели нечестивых человеков.
\rsbpar\vs 2Pe 3:8 Одно т\acc{о} не должно быть сокрыто от вас, возлюбленные, что у Господа один день, как тысяча лет, и тысяча лет, как один день.
\vs 2Pe 3:9 Не медлит Господь \bibemph{исполнением} обетования, как некоторые почитают то медлением; но долготерпит нас, не желая, чтобы кто погиб, но чтобы все пришли к покаянию.
\vs 2Pe 3:10 Придет же день Господень, как тать ночью, и тогда небеса с шумом прейдут, стихии же, разгоревшись, разрушатся, земля и все дела на ней сгорят.
\vs 2Pe 3:11 Если так всё это разрушится, то какими должно быть в святой жизни и благочестии вам,
\vs 2Pe 3:12 ожидающим и желающим пришествия дня Божия, в который воспламененные небеса разрушатся и разгоревшиеся стихии растают?
\vs 2Pe 3:13 Впрочем мы, по обетованию Его, ожидаем нового неба и новой земли, на которых обитает правда.
\rsbpar\vs 2Pe 3:14 Итак, возлюбленные, ожидая сего, потщитесь явиться пред Ним неоскверненными и непорочными в мире;
\vs 2Pe 3:15 и долготерпение Господа нашего почитайте спасением, как и возлюбленный брат наш Павел, по данной ему премудрости, написал вам,
\vs 2Pe 3:16 как он говорит об этом и во всех посланиях, в которых есть нечто неудобовразумительное, что невежды и неутвержденные, к собственной своей погибели, превращают, как и прочие Писания.
\vs 2Pe 3:17 Итак вы, возлюбленные, будучи предварены о сем, берегитесь, чтобы вам не увлечься заблуждением беззаконников и не отпасть от своего утверждения,
\vs 2Pe 3:18 но возрастайте в благодати и познании Господа нашего и Спасителя Иисуса Христа. Ему слава и ныне и в день вечный. Аминь.
\newbookpage
\bibbookdescr{1Jo}{
  inline={Первое Соборное Послание\\\LARGE Святого Апостола Иоанна Богослова},
  toc={1-е Иоанна},
  bookmark={1-е Иоанна},
  header={1-е Иоанна},
  %headerleft={},
  %headerright={},
  abbr={1~Ин}
}
\vs 1Jo 1:1 О том, что было от начала, что мы слышали, что видели своими очами, что рассматривали и что осязали руки наши, о Слове жизни,~---
\vs 1Jo 1:2 ибо жизнь явилась, и мы видели и свидетельствуем, и возвещаем вам сию вечную жизнь, которая была у Отца и явилась нам,~---
\vs 1Jo 1:3 о том, что мы видели и слышали, возвещаем вам, чтобы и вы имели общение с нами: а наше общение~--- с Отцем и Сыном Его, Иисусом Христом.
\vs 1Jo 1:4 И сие пишем вам, чтобы радость ваша была совершенна.
\rsbpar\vs 1Jo 1:5 И вот благовестие, которое мы слышали от Него и возвещаем вам: Бог есть свет, и нет в Нем никакой тьмы.
\vs 1Jo 1:6 Если мы говорим, что имеем общение с Ним, а ходим во тьме, то мы лжем и не поступаем по истине;
\vs 1Jo 1:7 если же ходим во свете, подобно как Он во свете, то имеем общение друг с другом, и Кровь Иисуса Христа, Сына Его, очищает нас от всякого греха.
\vs 1Jo 1:8 Если говорим, что не имеем греха,~--- обманываем самих себя, и истины нет в нас.
\vs 1Jo 1:9 Если исповедуем грехи наши, то Он, будучи верен и праведен, простит нам грехи наши и очистит нас от всякой неправды.
\vs 1Jo 1:10 Если говорим, что мы не согрешили, то представляем Его лживым, и сл\acc{о}ва Его нет в нас.
\vs 1Jo 2:1 Дети мои! сие пишу вам, чтобы вы не согрешали; а если бы кто согрешил, то мы имеем ходатая пред Отцем, Иисуса Христа, праведника;
\vs 1Jo 2:2 Он есть умилостивление за грехи наши, и не только за наши, но и за \bibemph{грехи} всего мира.
\rsbpar\vs 1Jo 2:3 А что мы познали Его, узнаём из того, что соблюдаем Его заповеди.
\vs 1Jo 2:4 Кто говорит: <<я познал Его>>, но заповедей Его не соблюдает, тот лжец, и нет в нем истины;
\vs 1Jo 2:5 а кто соблюдает слово Его, в том истинно любовь Божия совершилась: из сего узнаём, что мы в Нем.
\vs 1Jo 2:6 Кто говорит, что пребывает в Нем, тот должен поступать так, как Он поступал.
\rsbpar\vs 1Jo 2:7 Возлюбленные! пишу вам не новую заповедь, но заповедь древнюю, которую вы имели от начала. Заповедь древняя есть слово, которое вы слышали от начала.
\vs 1Jo 2:8 Но притом и новую заповедь пишу вам, чт\acc{о} есть истинно и в Нем и в вас: потому что тьма проходит и истинный свет уже светит.
\vs 1Jo 2:9 Кто говорит, что он во свете, а ненавидит брата своего, тот еще во тьме.
\vs 1Jo 2:10 Кто любит брата своего, тот пребывает во свете, и нет в нем соблазна.
\vs 1Jo 2:11 А кто ненавидит брата своего, тот находится во тьме, и во тьме ходит, и не знает, куда идет, потому что тьма ослепила ему глаза.
\rsbpar\vs 1Jo 2:12 Пишу вам, дети, потому что прощены вам грехи ради имени Его.
\vs 1Jo 2:13 Пишу вам, отцы, потому что вы познали Сущего от начала. Пишу вам, юноши, потому что вы победили лукавого. Пишу вам, отроки, потому что вы познали Отца.
\vs 1Jo 2:14 Я написал вам, отцы, потому что вы познали Безначального. Я написал вам, юноши, потому что вы сильны, и слово Божие пребывает в вас, и вы победили лукавого.
\vs 1Jo 2:15 Не люб\acc{и}те мира, ни того, что в мире: кто любит мир, в том нет любви Отчей.
\vs 1Jo 2:16 Ибо всё, что в мире: похоть плоти, похоть очей и гордость житейская, не есть от Отца, но от мира сего.
\vs 1Jo 2:17 И мир проходит, и похоть его, а исполняющий волю Божию пребывает вовек.
\rsbpar\vs 1Jo 2:18 Дети! последнее время. И как вы слышали, что придет антихрист, и теперь появилось много антихристов, то мы и познаём из того, что последнее время.
\vs 1Jo 2:19 Они вышли от нас, но не были наши: ибо если бы они были наши, то остались бы с нами; но \bibemph{они вышли, и} через т\acc{о} открылось, что не все наши.
\vs 1Jo 2:20 Впрочем, вы имеете помазание от Святаго и знаете всё.
\vs 1Jo 2:21 Я написал вам не потому, чтобы вы не знали истины, но потому, что вы знаете ее, \bibemph{равно как} и т\acc{о}, что всякая ложь не от истины.
\vs 1Jo 2:22 Кто лжец, если не тот, кто отвергает, что Иисус есть Христос? Это антихрист, отвергающий Отца и Сына.
\vs 1Jo 2:23 Всякий, отвергающий Сына, не имеет и Отца; а исповедующий Сына имеет и Отца.
\vs 1Jo 2:24 Итак, что вы слышали от начала, то и да пребывает в вас; если пребудет в вас то, что вы слышали от начала, то и вы пребудете в Сыне и в Отце.
\vs 1Jo 2:25 Обетование же, которое Он обещал нам, есть жизнь вечная.
\rsbpar\vs 1Jo 2:26 Это я написал вам об обольщающих вас.
\vs 1Jo 2:27 Впрочем, помазание, которое вы получили от Него, в вас пребывает, и вы не имеете нужды, чтобы кто учил вас; но как самое сие помазание учит вас всему, и оно истинно и неложно, то, чему оно научило вас, в том пребывайте.
\rsbpar\vs 1Jo 2:28 Итак, дети, пребывайте в Нем, чтобы, когда Он явится, иметь нам дерзновение и не постыдиться пред Ним в пришествие Его.
\vs 1Jo 2:29 Если вы знаете, что Он праведник, знайте и т\acc{о}, что всякий, делающий правду, рожден от Него.
\vs 1Jo 3:1 Смотр\acc{и}те, какую любовь дал нам Отец, чтобы нам называться и быть детьми Божиими. Мир потому не знает нас, что не познал Его.
\rsbpar\vs 1Jo 3:2 Возлюбленные! мы теперь дети Божии; но еще не открылось, чт\acc{о} будем. Знаем только, что, когда откроется, будем подобны Ему, потому что увидим Его, как Он есть.
\vs 1Jo 3:3 И всякий, имеющий сию надежду на Него, очищает себя так, как Он чист.
\vs 1Jo 3:4 Всякий, делающий грех, делает и беззаконие; и грех есть беззаконие.
\vs 1Jo 3:5 И вы знаете, что Он явился для того, чтобы взять грехи наши, и что в Нем нет греха.
\vs 1Jo 3:6 Всякий, пребывающий в Нем, не согрешает; всякий согрешающий не видел Его и не познал Его.
\rsbpar\vs 1Jo 3:7 Дети! да не обольщает вас никто. Кто делает правду, тот праведен, подобно как Он праведен.
\vs 1Jo 3:8 Кто делает грех, тот от диавола, потому что сначала диавол согрешил. Для сего-то и явился Сын Божий, чтобы разрушить дела диавола.
\vs 1Jo 3:9 Всякий, рожденный от Бога, не делает греха, потому что семя Его пребывает в нем; и он не может грешить, потому что рожден от Бога.
\vs 1Jo 3:10 Дети Божии и дети диавола узна\acc{ю}тся так: всякий, не делающий правды, не есть от Бога, равно и не любящий брата своего.
\vs 1Jo 3:11 Ибо таково благовествование, которое вы слышали от начала, чтобы мы любили друг друга,
\vs 1Jo 3:12 не т\acc{а}к, к\acc{а}к Каин, \bibemph{который} был от лукавого и убил брата своего. А за что убил его? За то, что дела его были злы, а дела брата его праведны.
\vs 1Jo 3:13 Не дивитесь, братия мои, если мир ненавидит вас.
\vs 1Jo 3:14 Мы знаем, что мы перешли из смерти в жизнь, потому что любим братьев; не любящий брата пребывает в смерти.
\vs 1Jo 3:15 Всякий, ненавидящий брата своего, есть человекоубийца; а вы знаете, что никакой человекоубийца не имеет жизни вечной, в нем пребывающей.
\vs 1Jo 3:16 Любовь познали мы в том, что Он положил за нас душу Свою: и мы должны полагать души свои за братьев.
\vs 1Jo 3:17 А кто имеет достаток в мире, но, видя брата своего в нужде, затворяет от него сердце свое,~--- как пребывает в том любовь Божия?
\rsbpar\vs 1Jo 3:18 Дети мои! станем любить не словом или языком, но делом и истиною.
\vs 1Jo 3:19 И вот по чему узнаём, что мы от истины, и успокаиваем пред Ним сердца наши;
\vs 1Jo 3:20 ибо если сердце наше осуждает нас, то \bibemph{кольми паче Бог}, потому что Бог больше сердца нашего и знает всё.
\vs 1Jo 3:21 Возлюбленные! если сердце наше не осуждает нас, то мы имеем дерзновение к Богу,
\vs 1Jo 3:22 и, чего ни попросим, получим от Него, потому что соблюдаем заповеди Его и делаем благоугодное пред Ним.
\vs 1Jo 3:23 А заповедь Его та, чтобы мы веровали во имя Сына Его Иисуса Христа и любили друг друга, как Он заповедал нам.
\vs 1Jo 3:24 И кто сохраняет заповеди Его, тот пребывает в Нем, и Он в том. А что Он пребывает в нас, узнаём по духу, который Он дал нам.
\vs 1Jo 4:1 Возлюбленные! не всякому духу верьте, но испытывайте духов, от Бога ли они, потому что много лжепророков появилось в мире.
\vs 1Jo 4:2 Духа Божия (и духа заблуждения) узнавайте так: всякий дух, который исповедует Иисуса Христа, пришедшего во плоти, есть от Бога;
\vs 1Jo 4:3 а всякий дух, который не исповедует Иисуса Христа, пришедшего во плоти, не есть от Бога, но это дух антихриста, о котором вы слышали, что он придет и теперь есть уже в мире.
\rsbpar\vs 1Jo 4:4 Дети! вы от Бога, и победили их; ибо Тот, Кто в вас, больше того, кто в мире.
\vs 1Jo 4:5 Они от мира, потому и говорят по-мирски, и мир слушает их.
\vs 1Jo 4:6 Мы от Бога; знающий Бога слушает нас; кто не от Бога, тот не слушает нас. По сему-то узнаём духа истины и духа заблуждения.
\rsbpar\vs 1Jo 4:7 Возлюбленные! будем любить друг друга, потому что любовь от Бога, и всякий любящий рожден от Бога и знает Бога.
\vs 1Jo 4:8 Кто не любит, тот не познал Бога, потому что Бог есть любовь.
\vs 1Jo 4:9 Любовь Божия к нам открылась в том, что Бог послал в мир Единородного Сына Своего, чтобы мы получили жизнь через Него.
\vs 1Jo 4:10 В том любовь, что не мы возлюбили Бога, но Он возлюбил нас и послал Сына Своего в умилостивление за грехи наши.
\rsbpar\vs 1Jo 4:11 Возлюбленные! если так возлюбил нас Бог, то и мы должны любить друг друга.
\vs 1Jo 4:12 Бога никто никогда не видел. Если мы любим друг друга, то Бог в нас пребывает, и любовь Его совершенна есть в нас.
\vs 1Jo 4:13 Что мы пребываем в Нем и Он в нас, узнаём из того, что Он дал нам от Духа Своего.
\vs 1Jo 4:14 И мы видели и свидетельствуем, что Отец послал Сына Спасителем миру.
\vs 1Jo 4:15 Кто исповедует, что Иисус есть Сын Божий, в том пребывает Бог, и он в Боге.
\vs 1Jo 4:16 И мы познали любовь, которую имеет к нам Бог, и уверовали в нее. Бог есть любовь, и пребывающий в любви пребывает в Боге, и Бог в нем.
\vs 1Jo 4:17 Любовь до того совершенства достигает в нас, что мы имеем дерзновение в день суда, потому что поступаем в мире сем, как Он.
\vs 1Jo 4:18 В любви нет страха, но совершенная любовь изгоняет страх, потому что в страхе есть мучение. Боящийся несовершен в любви.
\vs 1Jo 4:19 Будем любить Его, потому что Он прежде возлюбил нас.
\vs 1Jo 4:20 Кто говорит: <<я люблю Бога>>, а брата своего ненавидит, тот лжец: ибо не любящий брата своего, которого видит, как может любить Бога, Которого не видит?
\vs 1Jo 4:21 И мы имеем от Него такую заповедь, чтобы любящий Бога любил и брата своего.
\vs 1Jo 5:1 Всякий верующий, что Иисус есть Христос, от Бога рожден, и всякий, любящий Родившего, любит и Рожденного от Него.
\vs 1Jo 5:2 Что мы любим детей Божиих, узнаём из того, когда любим Бога и соблюдаем заповеди Его.
\vs 1Jo 5:3 Ибо это есть любовь к Богу, чтобы мы соблюдали заповеди Его; и заповеди Его нетяжки.
\vs 1Jo 5:4 Ибо всякий, рожденный от Бога, побеждает мир; и сия есть победа, победившая мир, вера наша.
\vs 1Jo 5:5 Кто побеждает мир, как не тот, кто верует, что Иисус есть Сын Божий?
\vs 1Jo 5:6 Сей есть Иисус Христос, пришедший водою и кровию и Духом, не водою только, но водою и кровию, и Дух свидетельствует о \bibemph{Нем}, потому что Дух есть истина.
\vs 1Jo 5:7 Ибо три свидетельствуют на небе: Отец, Слово и Святый Дух; и Сии три суть едино.
\vs 1Jo 5:8 И три свидетельствуют на земле: дух, вода и кровь; и сии три об одном.
\vs 1Jo 5:9 Если мы принимаем свидетельство человеческое, свидетельство Божие~--- больше, ибо это есть свидетельство Божие, которым Бог свидетельствовал о Сыне Своем.
\vs 1Jo 5:10 Верующий в Сына Божия имеет свидетельство в себе самом; не верующий Богу представляет Его лживым, потому что не верует в свидетельство, которым Бог свидетельствовал о Сыне Своем.
\vs 1Jo 5:11 Свидетельство сие состоит в том, что Бог даровал нам жизнь вечную, и сия жизнь в Сыне Его.
\vs 1Jo 5:12 Имеющий Сына (Божия) имеет жизнь; не имеющий Сына Божия не имеет жизни.
\rsbpar\vs 1Jo 5:13 Сие написал я вам, верующим во имя Сына Божия, дабы вы знали, что вы, веруя в Сына Божия, имеете жизнь вечную.
\vs 1Jo 5:14 И вот какое дерзновение мы имеем к Нему, что, когда просим чего по воле Его, Он слушает нас.
\vs 1Jo 5:15 А когда мы знаем, что Он слушает нас во всем, чего бы мы ни просили,~--- знаем и то, что получаем просимое от Него.
\vs 1Jo 5:16 Если кто видит брата своего согрешающего грехом не к смерти, то пусть молится, и \bibemph{Бог} даст ему жизнь, \bibemph{то есть} согрешающему \bibemph{грехом} не к смерти. Есть грех к смерти: не о том говорю, чтобы он молился.
\vs 1Jo 5:17 Всякая неправда есть грех; но есть грех не к смерти.
\rsbpar\vs 1Jo 5:18 Мы знаем, что всякий, рожденный от Бога, не грешит; но рожденный от Бога хранит себя, и лукавый не прикасается к нему.
\vs 1Jo 5:19 Мы знаем, что мы от Бога и что весь мир лежит во зле.
\vs 1Jo 5:20 Знаем также, что Сын Божий пришел и дал нам свет и разум, да позн\acc{а}ем Бога истинного и да будем в истинном Сыне Его Иисусе Христе. Сей есть истинный Бог и жизнь вечная.
\rsbpar\vs 1Jo 5:21 Дети! храните себя от идолов. Аминь.

\bibbookdescr{2Jo}{
  inline={Второе Соборное Послание\\\LARGE Святого Апостола Иоанна Богослова},
  toc={2-е Иоанна},
  bookmark={2-е Иоанна},
  header={2-е Иоанна},
  %headerleft={},
  %headerright={},
  abbr={2~Ин}
}
\vs 2Jo 1:1 Старец~--- избранной госпоже и детям ее, которых я люблю по истине, и не только я, но и все, познавшие истину,
\vs 2Jo 1:2 ради истины, которая пребывает в нас и будет с нами вовек.
\vs 2Jo 1:3 Да будет с вами благодать, милость, мир от Бога Отца и от Господа Иисуса Христа, Сына Отчего, в истине и любви.
\rsbpar\vs 2Jo 1:4 Я весьма обрадовался, что нашел из детей твоих, ходящих в истине, как мы получили заповедь от Отца.
\vs 2Jo 1:5 И ныне прошу тебя, госпожа, не как новую заповедь предписывая тебе, но ту, которую имеем от начала, чтобы мы любили друг друга.
\vs 2Jo 1:6 Любовь же состоит в том, чтобы мы поступали по заповедям Его. Это та заповедь, которую вы слышали от начала, чтобы поступали по ней.
\vs 2Jo 1:7 Ибо многие обольстители вошли в мир, не исповедующие Иисуса Христа, пришедшего во плоти: такой \bibemph{человек} есть обольститель и антихрист.
\vs 2Jo 1:8 Наблюдайте за собою, чтобы нам не потерять того, над чем мы трудились, но чтобы получить полную награду.
\vs 2Jo 1:9 Всякий, преступающий учение Христово и не пребывающий в нем, не имеет Бога; пребывающий в учении Христовом имеет и Отца и Сына.
\vs 2Jo 1:10 Кто приходит к вам и не приносит сего учения, того не принимайте в дом и не приветствуйте его.
\vs 2Jo 1:11 Ибо приветствующий его участвует в злых делах его.
\rsbpar\vs 2Jo 1:12 Многое имею писать вам, но не хочу на бумаге чернилами, а надеюсь прийти к вам и говорить устами к устам, чтобы радость ваша была полна.
\vs 2Jo 1:13 Приветствуют тебя дети сестры твоей избранной. Аминь.

\bibbookdescr{3Jo}{
  inline={Третье Соборное Послание\\\LARGE Святого Апостола Иоанна Богослова},
  toc={3-е Иоанна},
  bookmark={3-е Иоанна},
  header={3-е Иоанна},
  %headerleft={},
  %headerright={},
  abbr={3~Ин}
}
\vs 3Jo 1:1 Старец~--- возлюбленному Гаию, которого я люблю по истине.
\rsbpar\vs 3Jo 1:2 Возлюбленный! молюсь, чтобы ты здравствовал и преуспевал во всем, как преуспевает душа твоя.
\vs 3Jo 1:3 Ибо я весьма обрадовался, когда пришли братия и засвидетельствовали о твоей верности, как ты ходишь в истине.
\vs 3Jo 1:4 Для меня нет б\acc{о}льшей радости, как слышать, что дети мои ходят в истине.
\rsbpar\vs 3Jo 1:5 Возлюбленный! ты как верный поступаешь в том, что делаешь для братьев и для странников.
\vs 3Jo 1:6 Они засвидетельствовали перед церковью о твоей любви. Ты хорошо поступишь, если отпустишь их, как должно ради Бога,
\vs 3Jo 1:7 ибо они ради имени Его пошли, не взяв ничего от язычников.
\vs 3Jo 1:8 Итак мы должны принимать таковых, чтобы сделаться споспешниками истине.
\rsbpar\vs 3Jo 1:9 Я писал церкви; но любящий первенствовать у них Диотреф не принимает нас.
\vs 3Jo 1:10 Посему, если я приду, то напомню о делах, которые он делает, понося нас злыми словами, и не довольствуясь тем, и сам не принимает братьев, и запрещает желающим, и изгоняет из церкви.
\vs 3Jo 1:11 Возлюбленный! не подражай злу, но добру. Кто делает добро, тот от Бога; а делающий зло не видел Бога.
\vs 3Jo 1:12 О Димитрии засвидетельствовано всеми и самою истиною; свидетельствуем также и мы, и вы знаете, что свидетельство наше истинно.
\rsbpar\vs 3Jo 1:13 Многое имел я писать; но не хочу писать к тебе чернилами и тростью,
\vs 3Jo 1:14 а надеюсь скоро увидеть тебя и поговорить устами к устам.
\vs 3Jo 1:15 Мир тебе. Приветствуют тебя друзья; приветствуй друзей поименно. Аминь.
\newbookpage
\bibbookdescr{Jud}{
  inline={Соборное Послание\\\LARGE Святого Апостола Иуды},
  toc={Иуды},
  bookmark={Иуды},
  header={Иуды},
  %headerleft={},
  %headerright={},
  abbr={Иуд}
}
\vs Jud 1:1 Иуда, раб Иисуса Христа, брат Иакова, призванным, которые освящены Богом Отцем и сохранены Иисусом Христом:
\vs Jud 1:2 милость вам и мир и любовь да умножатся.
\rsbpar\vs Jud 1:3 Возлюбленные! имея все усердие писать вам об общем спасении, я почел за нужное написать вам увещание~--- подвизаться за веру, однажды преданную святым.
\vs Jud 1:4 Ибо вкрались некоторые люди, издревле предназначенные к сему осуждению, нечестивые, обращающие благодать Бога нашего в \bibemph{повод к} распутству и отвергающиеся единого Владыки Бога и Господа нашего Иисуса Христа.
\rsbpar\vs Jud 1:5 Я хочу напомнить вам, уже знающим это, что Господь, избавив народ из земли Египетской, потом неверовавших погубил,
\vs Jud 1:6 и ангелов, не сохранивших своего достоинства, но оставивших свое жилище, соблюдает в вечных узах, под мраком, на суд великого дня.
\vs Jud 1:7 Как Содом и Гоморра и окрестные города, подобно им блудодействовавшие и ходившие за иною плотию, подвергшись казни огня вечного, поставлены в пример,~---
\vs Jud 1:8 так точно будет и с сими мечтателями, которые оскверняют плоть, отвергают начальства и злословят высокие власти.
\vs Jud 1:9 Михаил Архангел, когда говорил с диаволом, споря о Моисеевом теле, не смел произнести укоризненного суда, но сказал: <<да запретит тебе Господь>>.
\vs Jud 1:10 А сии злословят то, чего не знают; что же по природе, как бессловесные животные, знают, тем растлевают себя.
\vs Jud 1:11 Горе им, потому что идут путем Каиновым, предаются обольщению мзды, как Валаам, и в упорстве погибают, как Корей.
\vs Jud 1:12 Таковые бывают соблазном на ваших вечерях любви; пиршествуя с вами, без страха утучняют себя. Это безводные облака, носимые ветром; осенние деревья, бесплодные, дважды умершие, исторгнутые;
\vs Jud 1:13 свирепые морские волны, пенящиеся срамотами своими; звезды блуждающие, которым блюдется мрак тьмы на веки.
\vs Jud 1:14 О них пророчествовал и Енох, седьмой от Адама, говоря: <<се, идет Господь со тьмами святых Ангелов Своих~---
\vs Jud 1:15 сотворить суд над всеми и обличить всех между ними нечестивых во всех делах, которые произвело их нечестие, и во всех жестоких словах, которые произносили на Него нечестивые грешники>>.
\vs Jud 1:16 Это ропотники, ничем не довольные, поступающие по своим похотям (нечестиво и беззаконно); уста их произносят надутые слова; они оказывают лицеприятие для корысти.
\vs Jud 1:17 Но вы, возлюбленные, помните предсказанное Апостолами Господа нашего Иисуса Христа.
\vs Jud 1:18 Они говорили вам, что в последнее время появятся ругатели, поступающие по своим нечестивым похотям.
\vs Jud 1:19 Это люди, отделяющие себя (от единства веры), душевные, не имеющие духа.
\vs Jud 1:20 А вы, возлюбленные, назидая себя на святейшей вере вашей, молясь Духом Святым,
\vs Jud 1:21 сохраняйте себя в любви Божией, ожидая милости от Господа нашего Иисуса Христа, для вечной жизни.
\vs Jud 1:22 И к одним будьте милостивы, с рассмотрением,
\vs Jud 1:23 а других страхом спасайте, исторгая из огня, обличайте же со страхом, гнушаясь даже одеждою, которая осквернена плотью.
\rsbpar\vs Jud 1:24 Могущему же соблюсти вас от падения и поставить пред славою Своею непорочными в радости,
\vs Jud 1:25 Единому Премудрому Богу, Спасителю нашему чрез Иисуса Христа Господа нашего, слава и величие, сила и власть прежде всех веков, ныне и во все веки. Аминь.

\bibbookdescr{Rom}{
  inline={Послание к Римлянам\\\LARGE Святого Апостола Павла},
  toc={к Римлянам},
  bookmark={к Римлянам},
  header={к Римлянам},
  %headerleft={},
  %headerright={},
  abbr={Рим}
}
\vs Rom 1:1 Павел, раб Иисуса Христа, призванный Апостол, избранный к благовестию Божию,
\vs Rom 1:2 которое Бог прежде обещал через пророков Своих, в святых писаниях,
\vs Rom 1:3 о Сыне Своем, Который родился от семени Давидова по плоти
\vs Rom 1:4 и открылся Сыном Божиим в силе, по духу святыни, через воскресение из мертвых, о Иисусе Христе Господе нашем,
\vs Rom 1:5 через Которого мы получили благодать и апостольство, чтобы во имя Его покорять вере все народы,
\vs Rom 1:6 между которыми находитесь и вы, призванные Иисусом Христом,~---
\vs Rom 1:7 всем находящимся в Риме возлюбленным Божиим, призванным святым: благодать вам и мир от Бога Отца нашего и Господа Иисуса Христа.
\rsbpar\vs Rom 1:8 Прежде всего благодарю Бога моего через Иисуса Христа за всех вас, что вера ваша возвещается во всем мире.
\vs Rom 1:9 Свидетель мне Бог, Которому служу духом моим в благовествовании Сына Его, что непрестанно воспоминаю о вас,
\vs Rom 1:10 всегда прося в молитвах моих, чтобы воля Божия когда-нибудь благопоспешила мне прийти к вам,
\vs Rom 1:11 ибо я весьма желаю увидеть вас, чтобы преподать вам некое дарование духовное к утверждению вашему,
\vs Rom 1:12 то есть утешиться с вами верою общею, вашею и моею.
\vs Rom 1:13 Не хочу, братия, \bibemph{оставить} вас в неведении, что я многократно намеревался прийти к вам (но встречал препятствия даже доныне), чтобы иметь некий плод и у вас, как и у прочих народов.
\vs Rom 1:14 Я должен и Еллинам и варварам, мудрецам и невеждам.
\vs Rom 1:15 Итак, что до меня, я готов благовествовать и вам, находящимся в Риме.
\vs Rom 1:16 Ибо я не стыжусь благовествования Христова, потому что \bibemph{оно} есть сила Божия ко спасению всякому верующему, во-первых, Иудею, \bibemph{потом} и Еллину.
\vs Rom 1:17 В нем открывается правда Божия от веры в веру, как написано: праведный верою жив будет.
\rsbpar\vs Rom 1:18 Ибо открывается гнев Божий с неба на всякое нечестие и неправду человеков, подавляющих истину неправдою.
\vs Rom 1:19 Ибо, чт\acc{о} можно знать о Боге, явно для них, потому что Бог явил им.
\vs Rom 1:20 Ибо невидимое Его, вечная сила Его и Божество, от создания мира через рассматривание творений видимы, так что они безответны.
\vs Rom 1:21 Но как они, познав Бога, не прославили Его, как Бога, и не возблагодарили, но осуетились в умствованиях своих, и омрачилось несмысленное их сердце;
\vs Rom 1:22 называя себя мудрыми, обезумели,
\vs Rom 1:23 и славу нетленного Бога изменили в образ, подобный тленному человеку, и птицам, и четвероногим, и пресмыкающимся,~---
\vs Rom 1:24 то и предал их Бог в похотях сердец их нечистоте, так что они сквернили сами свои тела.
\vs Rom 1:25 Они заменили истину Божию ложью, и поклонялись, и служили твари вместо Творца, Который благословен во веки, аминь.
\vs Rom 1:26 Потому предал их Бог постыдным страстям: женщины их заменили естественное употребление противоестественным;
\vs Rom 1:27 подобно и мужчины, оставив естественное употребление женского пола, разжигались похотью друг на друга, мужчины на мужчинах делая срам и получая в самих себе должное возмездие за свое заблуждение.
\vs Rom 1:28 И как они не заботились иметь Бога в разуме, то предал их Бог превратному уму~--- делать непотребства,
\vs Rom 1:29 так что они исполнены всякой неправды, блуда, лукавства, корыстолюбия, злобы, исполнены зависти, убийства, распрей, обмана, злонравия,
\vs Rom 1:30 злоречивы, клеветники, богоненавистники, обидчики, самохвалы, горды, изобретательны на зло, непослушны родителям,
\vs Rom 1:31 безрассудны, вероломны, нелюбовны, непримиримы, немилостивы.
\vs Rom 1:32 Они знают праведный \bibemph{суд} Божий, что делающие такие \bibemph{дела} достойны смерти; однако не только \bibemph{их} делают, но и делающих одобряют.
\vs Rom 2:1 Итак, неизвинителен ты, всякий человек, судящий \bibemph{другого}, ибо тем же судом, каким судишь другого, осуждаешь себя, потому что, судя \bibemph{другого}, делаешь т\acc{о} же.
\vs Rom 2:2 А мы знаем, что поистине есть суд Божий на делающих такие \bibemph{дела}.
\vs Rom 2:3 Неужели думаешь ты, человек, что избежишь суда Божия, осуждая делающих такие \bibemph{дела} и (сам) делая т\acc{о} же?
\vs Rom 2:4 Или пренебрегаешь богатство благости, кротости и долготерпения Божия, не разумея, что благость Божия ведет тебя к покаянию?
\vs Rom 2:5 Но, по упорству твоему и нераскаянному сердцу, ты сам себе собираешь гнев на день гнева и откровения праведного суда от Бога,
\vs Rom 2:6 Который воздаст каждому по делам его:
\vs Rom 2:7 тем, которые постоянством в добром деле ищут славы, чести и бессмертия,~--- жизнь вечную;
\vs Rom 2:8 а тем, которые упорствуют и не покоряются истине, но предаются неправде,~--- ярость и гнев.
\vs Rom 2:9 Скорбь и теснота всякой душе человека, делающего злое, во-первых, Иудея, \bibemph{потом} и Еллина!
\vs Rom 2:10 Напротив, слава и честь и мир всякому, делающему доброе, во-первых, Иудею, \bibemph{потом} и Еллину!
\vs Rom 2:11 Ибо нет лицеприятия у Бога.
\rsbpar\vs Rom 2:12 Те, которые, не \bibemph{имея} закона, согрешили, вне закона и погибнут; а те, которые под законом согрешили, по закону осудятся
\vs Rom 2:13 (потому что не слушатели закона праведны пред Богом, но исполнители закона оправданы будут,
\vs Rom 2:14 ибо когда язычники, не имеющие закона, по природе законное делают, то, не имея закона, они сами себе закон:
\vs Rom 2:15 они показывают, что дело закона у них написано в сердцах, о чем свидетельствует совесть их и мысли их, то обвиняющие, то оправдывающие одна другую)
\vs Rom 2:16 в день, когда, по благовествованию моему, Бог будет судить тайные \bibemph{дела} человеков через Иисуса Христа.
\rsbpar\vs Rom 2:17 Вот, ты называешься Иудеем, и успокаиваешь себя законом, и хвалишься Богом,
\vs Rom 2:18 и знаешь волю \bibemph{Его}, и разумеешь лучшее, научаясь из закона,
\vs Rom 2:19 и уверен о себе, что ты путеводитель слепых, свет для находящихся во тьме,
\vs Rom 2:20 наставник невежд, учитель младенцев, имеющий в законе образец ведения и истины:
\vs Rom 2:21 как же ты, уча другого, не учишь себя самого?
\vs Rom 2:22 Проповедуя не красть, крадешь? говоря: <<не прелюбодействуй>>, прелюбодействуешь? гнушаясь идолов, святотатствуешь?
\vs Rom 2:23 Хвалишься законом, а преступлением закона бесчестишь Бога?
\vs Rom 2:24 Ибо ради вас, как написано, имя Божие хулится у язычников.
\vs Rom 2:25 Обрезание полезно, если исполняешь закон; а если ты преступник закона, то обрезание твое стало необрезанием.
\vs Rom 2:26 Итак, если необрезанный соблюдает постановления закона, то его необрезание не вменится ли ему в обрезание?
\vs Rom 2:27 И необрезанный по природе, исполняющий закон, не осудит ли тебя, преступника закона при Писании и обрезании?
\vs Rom 2:28 Ибо не тот Иудей, кто \bibemph{таков} по наружности, и не то обрезание, которое наружно, на плоти;
\vs Rom 2:29 но \bibemph{тот} Иудей, кто внутренно \bibemph{таков}, и \bibemph{то} обрезание, \bibemph{которое} в сердце, по духу, \bibemph{а} не по букве: ему и похвала не от людей, но от Бога.
\vs Rom 3:1 Итак, какое преимущество \bibemph{быть} Иудеем, или какая польза от обрезания?
\vs Rom 3:2 Великое преимущество во всех отношениях, а наипаче \bibemph{в том}, что им вверено слово Божие.
\vs Rom 3:3 Ибо что же? если некоторые и неверны были, неверность их уничтожит ли верность Божию?
\vs Rom 3:4 Никак. Бог верен, а всякий человек лжив, как написано: Ты праведен в словах Твоих и победишь в суде Твоем.
\vs Rom 3:5 Если же наша неправда открывает правду Божию, то что скажем? не будет ли Бог несправедлив, когда изъявляет гнев? (говорю по человеческому \bibemph{рассуждению}).
\vs Rom 3:6 Никак. Ибо \bibemph{иначе} как Богу судить мир?
\vs Rom 3:7 Ибо, если верность Божия возвышается моею неверностью к славе Божией, за что еще меня же судить, как грешника?
\vs Rom 3:8 И не делать ли нам зло, чтобы вышло добро, как некоторые злословят нас и говорят, будто мы так учим? Праведен суд на таковых.
\rsbpar\vs Rom 3:9 Итак, что же? имеем ли мы преимущество? Нисколько. Ибо мы уже доказали, что как Иудеи, так и Еллины, все под грехом,
\vs Rom 3:10 как написано: нет праведного ни одного;
\vs Rom 3:11 нет разумевающего; никто не ищет Бога;
\vs Rom 3:12 все совратились с пути, до одного негодны; нет делающего добро, нет ни одного.
\vs Rom 3:13 Гортань их~--- открытый гроб; языком своим обманывают; яд аспидов на губах их.
\vs Rom 3:14 Уста их полны злословия и горечи.
\vs Rom 3:15 Ноги их быстры на пролитие крови;
\vs Rom 3:16 разрушение и пагуба на путях их;
\vs Rom 3:17 они не знают пути мира.
\vs Rom 3:18 Нет страха Божия перед глазами их.
\rsbpar\vs Rom 3:19 Но мы знаем, что закон, если что говорит, говорит к состоящим под законом, так что заграждаются всякие уста, и весь мир становится виновен пред Богом,
\vs Rom 3:20 потому что делами закона не оправдается пред Ним никакая плоть; ибо законом познаётся грех.
\vs Rom 3:21 Но ныне, независимо от закона, явилась правда Божия, о которой свидетельствуют закон и пророки,
\vs Rom 3:22 правда Божия через веру в Иисуса Христа во всех и на всех верующих, ибо нет различия,
\vs Rom 3:23 потому что все согрешили и лишены славы Божией,
\vs Rom 3:24 получая оправдание даром, по благодати Его, искуплением во Христе Иисусе,
\vs Rom 3:25 которого Бог предложил в жертву умилостивления в Крови Его через веру, для показания правды Его в прощении грехов, соделанных прежде,
\vs Rom 3:26 во \bibemph{время} долготерпения Божия, к показанию правды Его в настоящее время, да \bibemph{явится} Он праведным и оправдывающим верующего в Иисуса.
\vs Rom 3:27 Где же то, чем бы хвалиться? уничтожено. Каким законом? \bibemph{законом} дел? Нет, но законом веры.
\vs Rom 3:28 Ибо мы признаём, что человек оправдывается верою, независимо от дел закона.
\vs Rom 3:29 Неужели Бог \bibemph{есть Бог} Иудеев только, а не и язычников? Конечно, и язычников,
\vs Rom 3:30 потому что один Бог, Который оправдает обрезанных по вере и необрезанных через веру.
\vs Rom 3:31 Итак, мы уничтожаем закон верою? Никак; но закон утверждаем.
\vs Rom 4:1 Чт\acc{о} же, скажем, Авраам, отец наш, приобрел по плоти?
\vs Rom 4:2 Если Авраам оправдался делами, он имеет похвалу, но не пред Богом.
\vs Rom 4:3 Ибо чт\acc{о} говорит Писание? Поверил Авраам Богу, и это вменилось ему в праведность.
\vs Rom 4:4 Воздаяние делающему вменяется не по милости, но по долгу.
\vs Rom 4:5 А не делающему, но верующему в Того, Кто оправдывает нечестивого, вера его вменяется в праведность.
\vs Rom 4:6 Так и Давид называет блаженным человека, которому Бог вменяет праведность независимо от дел:
\vs Rom 4:7 Блаженны, чьи беззакония прощены и чьи грехи покрыты.
\vs Rom 4:8 Блажен человек, которому Господь не вменит греха.
\vs Rom 4:9 Блаженство сие \bibemph{относится} к обрезанию, или к необрезанию? Мы говорим, что Аврааму вера вменилась в праведность.
\vs Rom 4:10 Когда вменилась? по обрезании или до обрезания? Не по обрезании, а до обрезания.
\vs Rom 4:11 И знак обрезания он получил, \bibemph{как} печать праведности через веру, которую \bibemph{имел} в необрезании, так что он стал отцом всех верующих в необрезании, чтобы и им вменилась праведность,
\vs Rom 4:12 и отцом обрезанных, не только \bibemph{принявших} обрезание, но и ходящих по следам веры отца нашего Авраама, которую \bibemph{имел он} в необрезании.
\vs Rom 4:13 Ибо не законом \bibemph{даровано} Аврааму, или семени его, обетование~--- быть наследником мира, но праведностью веры.
\vs Rom 4:14 Если утверждающиеся на законе суть наследники, то тщетна вера, бездейственно обетование;
\vs Rom 4:15 ибо закон производит гнев, потому что, где нет закона, нет и преступления.
\vs Rom 4:16 Итак по вере, чтобы \bibemph{было} по милости, дабы обетование было непреложно для всех, не только по закону, но и по вере потомков Авраама, который есть отец всем нам
\vs Rom 4:17 (как написано: Я поставил тебя отцом многих народов) пред Богом, Которому он поверил, животворящим мертвых и называющим несуществующее, как существующее.
\vs Rom 4:18 Он, сверх надежды, поверил с надеждою, через что сделался отцом многих народов, по сказанному: <<так \bibemph{многочисленно} будет семя твое>>.
\vs Rom 4:19 И, не изнемогши в вере, он не помышлял, что тело его, почти столетнего, уже омертвело, и утроба Саррина в омертвении;
\vs Rom 4:20 не поколебался в обетовании Божием неверием, но пребыл тверд в вере, воздав славу Богу
\vs Rom 4:21 и будучи вполне уверен, что Он силен и исполнить обещанное.
\vs Rom 4:22 Потому и вменилось ему в праведность.
\vs Rom 4:23 А впрочем не в отношении к нему одному написано, что вменилось ему,
\vs Rom 4:24 но и в отношении к нам; вменится и нам, верующим в Того, Кто воскресил из мертвых Иисуса Христа, Господа нашего,
\vs Rom 4:25 Который предан за грехи наши и воскрес для оправдания нашего.
\vs Rom 5:1 Итак, оправдавшись верою, мы имеем мир с Богом через Господа нашего Иисуса Христа,
\vs Rom 5:2 через Которого верою и получили мы доступ к той благодати, в которой стоим и хвалимся надеждою славы Божией.
\vs Rom 5:3 И не сим только, но хвалимся и скорбями, зная, что от скорби происходит терпение,
\vs Rom 5:4 от терпения опытность, от опытности надежда,
\vs Rom 5:5 а надежда не постыжает, потому что любовь Божия излилась в сердца наши Духом Святым, данным нам.
\vs Rom 5:6 Ибо Христос, когда еще мы были немощны, в определенное время умер за нечестивых.
\vs Rom 5:7 Ибо едва ли кто умрет за праведника; разве за благодетеля, может быть, кто и решится умереть.
\vs Rom 5:8 Но Бог Свою любовь к нам доказывает тем, что Христос умер за нас, когда мы были еще грешниками.
\vs Rom 5:9 Посему тем более ныне, будучи оправданы Кровию Его, спасемся Им от гнева.
\vs Rom 5:10 Ибо если, будучи врагами, мы примирились с Богом смертью Сына Его, то тем более, примирившись, спасемся жизнью Его.
\vs Rom 5:11 И не довольно сего, но и хвалимся Богом чрез Господа нашего Иисуса Христа, посредством Которого мы получили ныне примирение.
\rsbpar\vs Rom 5:12 Посему, как одним человеком грех вошел в мир, и грехом смерть, так и смерть перешла во всех человеков, \bibemph{потому что} в нем все согрешили.
\vs Rom 5:13 Ибо \bibemph{и} до закона грех был в мире; но грех не вменяется, когда нет закона.
\vs Rom 5:14 Однако же смерть царствовала от Адама до Моисея и над несогрешившими подобно преступлению Адама, который есть образ будущего.
\vs Rom 5:15 Но дар благодати не как преступление. Ибо если преступлением одного подверглись смерти многие, то тем более благодать Божия и дар по благодати одного Человека, Иисуса Христа, преизбыточествуют для многих.
\vs Rom 5:16 И дар не как \bibemph{суд} за одного согрешившего; ибо суд за одно \bibemph{преступление}~--- к осуждению; а дар благодати~--- к оправданию от многих преступлений.
\vs Rom 5:17 Ибо если преступлением одного смерть царствовала посредством одного, то тем более приемлющие обилие благодати и дар праведности будут царствовать в жизни посредством единого Иисуса Христа.
\vs Rom 5:18 Посему, как преступлением одного всем человекам осуждение, так правдою одного всем человекам оправдание к жизни.
\vs Rom 5:19 Ибо, как непослушанием одного человека сделались многие грешными, так и послушанием одного сделаются праведными многие.
\vs Rom 5:20 Закон же пришел после, и таким образом умножилось преступление. А когда умножился грех, стала преизобиловать благодать,
\vs Rom 5:21 дабы, как грех царствовал к смерти, так и благодать воцарилась через праведность к жизни вечной Иисусом Христом, Господом нашим.
\vs Rom 6:1 Что же скажем? оставаться ли нам в грехе, чтобы умножилась благодать? Никак.
\vs Rom 6:2 Мы умерли для греха: как же нам жить в нем?
\vs Rom 6:3 Неужели не знаете, что все мы, крестившиеся во Христа Иисуса, в смерть Его крестились?
\vs Rom 6:4 Итак мы погреблись с Ним крещением в смерть, дабы, как Христос воскрес из мертвых славою Отца, так и нам ходить в обновленной жизни.
\vs Rom 6:5 Ибо если мы соединены с Ним подобием смерти Его, то должны быть \bibemph{соединены} и \bibemph{подобием} воскресения,
\vs Rom 6:6 зная то, что ветхий наш человек распят с Ним, чтобы упразднено было тело греховное, дабы нам не быть уже рабами греху;
\vs Rom 6:7 ибо умерший освободился от греха.
\vs Rom 6:8 Если же мы умерли со Христом, то веруем, что и жить будем с Ним,
\vs Rom 6:9 зная, что Христос, воскреснув из мертвых, уже не умирает: смерть уже не имеет над Ним власти.
\vs Rom 6:10 Ибо, что Он умер, то умер однажды для греха; а что живет, то живет для Бога.
\vs Rom 6:11 Так и вы почитайте себя мертвыми для греха, живыми же для Бога во Христе Иисусе, Господе нашем.
\rsbpar\vs Rom 6:12 Итак да не царствует грех в смертном вашем теле, чтобы вам повиноваться ему в похотях его;
\vs Rom 6:13 и не предавайте членов ваших греху в орудия неправды, но представьте себя Богу, как оживших из мертвых, и члены ваши Богу в орудия праведности.
\vs Rom 6:14 Грех не должен над вами господствовать, ибо вы не под законом, но под благодатью.
\rsbpar\vs Rom 6:15 Что же? станем ли грешить, потому что мы не под законом, а под благодатью? Никак.
\vs Rom 6:16 Неужели вы не знаете, что, кому вы отдаете себя в рабы для послушания, того вы и рабы, кому повинуетесь, или \bibemph{рабы} греха к смерти, или послушания к праведности?
\vs Rom 6:17 Благодарение Богу, что вы, быв прежде рабами греха, от сердца стали послушны тому образу учения, которому предали себя.
\vs Rom 6:18 Освободившись же от греха, вы стали рабами праведности.
\vs Rom 6:19 Говорю по \bibemph{рассуждению} человеческому, ради немощи плоти вашей. Как предавали вы члены ваши в рабы нечистоте и беззаконию на \bibemph{дела} беззаконные, так ныне представьте члены ваши в рабы праведности на \bibemph{дела} святые.
\vs Rom 6:20 Ибо, когда вы были рабами греха, тогда были свободны от праведности.
\vs Rom 6:21 Какой же плод вы имели тогда? \bibemph{Такие дела}, каких ныне сами стыдитесь, потому что конец их~--- смерть.
\vs Rom 6:22 Но ныне, когда вы освободились от греха и стали рабами Богу, плод ваш есть святость, а конец~--- жизнь вечная.
\vs Rom 6:23 Ибо возмездие за грех~--- смерть, а дар Божий~--- жизнь вечная во Христе Иисусе, Господе нашем.
\vs Rom 7:1 Разве вы не знаете, братия (ибо говорю знающим закон), что закон имеет власть над человеком, пока он жив?
\vs Rom 7:2 Замужняя женщина привязана законом к живому мужу; а если умрет муж, она освобождается от закона замужества.
\vs Rom 7:3 Посему, если при живом муже выйдет за другого, называется прелюбодейцею; если же умрет муж, она свободна от закона, и не будет прелюбодейцею, выйдя за другого мужа.
\vs Rom 7:4 Так и вы, братия мои, умерли для закона телом Христовым, чтобы принадлежать другому, Воскресшему из мертвых, да приносим плод Богу.
\vs Rom 7:5 Ибо, когда мы жили по плоти, тогда страсти греховные, \bibemph{обнаруживаемые} законом, действовали в членах наших, чтобы приносить плод смерти;
\vs Rom 7:6 но ныне, умерши для закона, которым были связаны, мы освободились от него, чтобы нам служить Богу в обновлении духа, а не по ветхой букве.
\rsbpar\vs Rom 7:7 Что же скажем? Неужели \bibemph{от} закона грех? Никак. Но я не иначе узнал грех, как посредством закона. Ибо я не понимал бы и пожелания, если бы закон не говорил: не пожелай.
\vs Rom 7:8 Но грех, взяв повод от заповеди, произвел во мне всякое пожелание: ибо без закона грех мертв.
\vs Rom 7:9 Я жил некогда без закона; но когда пришла заповедь, то грех ожил,
\vs Rom 7:10 а я умер; и таким образом заповедь, \bibemph{данная} для жизни, послужила мне к смерти,
\vs Rom 7:11 потому что грех, взяв повод от заповеди, обольстил меня и умертвил ею.
\vs Rom 7:12 Посему закон свят, и заповедь свята и праведна и добра.
\vs Rom 7:13 Итак, неужели доброе сделалось мне смертоносным? Никак; но грех, оказывающийся грехом потому, что посредством доброго причиняет мне смерть, так что грех становится крайне грешен посредством заповеди.
\vs Rom 7:14 Ибо мы знаем, что закон духовен, а я плотян, продан греху.
\vs Rom 7:15 Ибо не понимаю, что делаю: потому что не то делаю, что хочу, а что ненавижу, то делаю.
\vs Rom 7:16 Если же делаю то, чего не хочу, то соглашаюсь с законом, что он добр,
\vs Rom 7:17 а потому уже не я делаю то, но живущий во мне грех.
\vs Rom 7:18 Ибо знаю, что не живет во мне, то есть в плоти моей, доброе; потому что желание добра есть во мне, но чтобы сделать оное, того не нахожу.
\vs Rom 7:19 Доброго, которого хочу, не делаю, а злое, которого не хочу, делаю.
\vs Rom 7:20 Если же делаю т\acc{о}, чего не хочу, уже не я делаю то, но живущий во мне грех.
\vs Rom 7:21 Итак я нахожу закон, что, когда хочу делать доброе, прилежит мне злое.
\vs Rom 7:22 Ибо по внутреннему человеку нахожу удовольствие в законе Божием;
\vs Rom 7:23 но в членах моих вижу иной закон, противоборствующий закону ума моего и делающий меня пленником закона греховного, находящегося в членах моих.
\vs Rom 7:24 Бедный я человек! кто избавит меня от сего тела смерти?
\vs Rom 7:25 Благодарю Бога моего Иисусом Христом, Господом нашим. Итак тот же самый я умом моим служу закону Божию, а плотию закону греха.
\vs Rom 8:1 Итак нет ныне никакого осуждения тем, которые во Христе Иисусе живут не по плоти, но по духу,
\vs Rom 8:2 потому что закон духа жизни во Христе Иисусе освободил меня от закона греха и смерти.
\vs Rom 8:3 Как закон, ослабленный плотию, был бессилен, то Бог послал Сына Своего в подобии плоти греховной \bibemph{в жертву} за грех и осудил грех во плоти,
\vs Rom 8:4 чтобы оправдание закона исполнилось в нас, живущих не по плоти, но по духу.
\vs Rom 8:5 Ибо живущие по плоти о плотском помышляют, а живущие по духу~--- о духовном.
\vs Rom 8:6 Помышления плотские суть смерть, а помышления духовные~--- жизнь и мир,
\vs Rom 8:7 потому что плотские помышления суть вражда против Бога; ибо закону Божию не покоряются, да и не могут.
\vs Rom 8:8 Посему живущие по плоти Богу угодить не могут.
\vs Rom 8:9 Но вы не по плоти живете, а по духу, если только Дух Божий живет в вас. Если же кто Духа Христова не имеет, тот \bibemph{и} не Его.
\vs Rom 8:10 А если Христос в вас, то тело мертво для греха, но дух жив для праведности.
\vs Rom 8:11 Если же Дух Того, Кто воскресил из мертвых Иисуса, живет в вас, то Воскресивший Христа из мертвых оживит и ваши смертные тела Духом Своим, живущим в вас.
\rsbpar\vs Rom 8:12 Итак, братия, мы не должники плоти, чтобы жить по плоти;
\vs Rom 8:13 ибо если живете по плоти, то умрете, а если духом умерщвляете дела плотские, то живы будете.
\vs Rom 8:14 Ибо все, водимые Духом Божиим, суть сыны Божии.
\vs Rom 8:15 Потому что вы не приняли духа рабства, \bibemph{чтобы} опять \bibemph{жить} в страхе, но приняли Духа усыновления, Которым взываем: <<Авва, Отче!>>
\vs Rom 8:16 Сей самый Дух свидетельствует духу нашему, что мы~--- дети Божии.
\vs Rom 8:17 А если дети, то и наследники, наследники Божии, сонаследники же Христу, если только с Ним страдаем, чтобы с Ним и прославиться.
\rsbpar\vs Rom 8:18 Ибо думаю, что нынешние временные страдания ничего не стоят в сравнении с тою славою, которая откроется в нас.
\vs Rom 8:19 Ибо тварь с надеждою ожидает откровения сынов Божиих,
\vs Rom 8:20 потому что тварь покорилась суете не добровольно, но по воле покорившего ее, в надежде,
\vs Rom 8:21 что и сама тварь освобождена будет от рабства тлению в свободу славы детей Божиих.
\vs Rom 8:22 Ибо знаем, что вся тварь совокупно стенает и мучится доныне;
\vs Rom 8:23 и не только \bibemph{она}, но и мы сами, имея начаток Духа, и мы в себе стенаем, ожидая усыновления, искупления тела нашего.
\vs Rom 8:24 Ибо мы спасены в надежде. Надежда же, когда видит, не есть надежда; ибо если кто видит, то чего ему и надеяться?
\vs Rom 8:25 Но когда надеемся того, чего не видим, тогда ожидаем в терпении.
\rsbpar\vs Rom 8:26 Также и Дух подкрепляет нас в немощах наших; ибо мы не знаем, о чем молиться, как должно, но Сам Дух ходатайствует за нас воздыханиями неизреченными.
\vs Rom 8:27 Испытующий же сердца знает, какая мысль у Духа, потому что Он ходатайствует за святых по \bibemph{воле} Божией.
\vs Rom 8:28 Притом знаем, что любящим Бога, призванным по \bibemph{Его} изволению, все содействует ко благу.
\vs Rom 8:29 Ибо кого Он предузнал, тем и предопределил быть подобными образу Сына Своего, дабы Он был первородным между многими братиями.
\vs Rom 8:30 А кого Он предопределил, тех и призвал, а кого призвал, тех и оправдал; а кого оправдал, тех и прославил.
\vs Rom 8:31 Что же сказать на это? Если Бог за нас, кто против нас?
\vs Rom 8:32 Тот, Который Сына Своего не пощадил, но предал Его за всех нас, как с Ним не дарует нам и всего?
\vs Rom 8:33 Кто будет обвинять избранных Божиих? Бог оправдывает \bibemph{их}.
\vs Rom 8:34 Кто осуждает? Христос Иисус умер, но и воскрес: Он и одесную Бога, Он и ходатайствует за нас.
\vs Rom 8:35 Кто отлучит нас от любви Божией: скорбь, или теснота, или гонение, или голод, или нагота, или опасность, или меч? как написано:
\vs Rom 8:36 за Тебя умерщвляют нас всякий день, считают нас за овец, \bibemph{обреченных} на заклание.
\vs Rom 8:37 Но все сие преодолеваем силою Возлюбившего нас.
\vs Rom 8:38 Ибо я уверен, что ни смерть, ни жизнь, ни Ангелы, ни Начала, ни Силы, ни настоящее, ни будущее,
\vs Rom 8:39 ни высота, ни глубина, ни другая какая тварь не может отлучить нас от любви Божией во Христе Иисусе, Господе нашем.
\vs Rom 9:1 Истину говорю во Христе, не лгу, свидетельствует мне совесть моя в Духе Святом,
\vs Rom 9:2 что великая для меня печаль и непрестанное мучение сердцу моему:
\vs Rom 9:3 я желал бы сам быть отлученным от Христа за братьев моих, родных мне по плоти,
\vs Rom 9:4 то есть Израильтян, которым принадлежат усыновление и слава, и заветы, и законоположение, и богослужение, и обетования;
\vs Rom 9:5 их и отцы, и от них Христос по плоти, сущий над всем Бог, благословенный во веки, аминь.
\vs Rom 9:6 Но не т\acc{о}, чтобы слово Божие не сбылось: ибо не все те Израильтяне, которые от Израиля;
\vs Rom 9:7 и не все дети Авраама, которые от семени его, но сказано: в Исааке наречется тебе семя.
\vs Rom 9:8 То есть не плотские дети суть дети Божии, но дети обетования признаются за семя.
\vs Rom 9:9 А слово обетования таково: в это же время приду, и у Сарры будет сын.
\vs Rom 9:10 И не одно это; но \bibemph{так было} и с Ревеккою, когда она зачала в одно время \bibemph{двух сыновей} от Исаака, отца нашего.
\vs Rom 9:11 Ибо, когда они еще не родились и не сделали ничего доброго или худого (дабы изволение Божие в избрании происходило
\vs Rom 9:12 не от дел, но от Призывающего), сказано было ей: больший будет в порабощении у меньшего,
\vs Rom 9:13 как и написано: Иакова Я возлюбил, а Исава возненавидел.
\rsbpar\vs Rom 9:14 Чт\acc{о} же скажем? Неужели неправда у Бога? Никак.
\vs Rom 9:15 Ибо Он говорит Моисею: кого миловать, помилую; кого жалеть, пожалею.
\vs Rom 9:16 Итак \bibemph{помилование зависит} не от желающего и не от подвизающегося, но от Бога милующего.
\vs Rom 9:17 Ибо Писание говорит фараону: для того самого Я и поставил тебя, чтобы показать над тобою силу Мою и чтобы проповедано было имя Мое по всей земле.
\vs Rom 9:18 Итак, кого хочет, милует; а кого хочет, ожесточает.
\rsbpar\vs Rom 9:19 Ты скажешь мне: <<за что же еще обвиняет? Ибо кто противостанет воле Его?>>
\vs Rom 9:20 А ты кто, человек, что споришь с Богом? Изделие скажет ли сделавшему его: <<зачем ты меня так сделал?>>
\vs Rom 9:21 Не властен ли горшечник над глиною, чтобы из той же смеси сделать один сосуд для почетного \bibemph{употребления}, а другой для низкого?
\vs Rom 9:22 Что же, если Бог, желая показать гнев и явить могущество Свое, с великим долготерпением щадил сосуды гнева, готовые к погибели,
\vs Rom 9:23 дабы вместе явить богатство славы Своей над сосудами милосердия, которые Он приготовил к славе,
\vs Rom 9:24 над нами, которых Он призвал не только из Иудеев, но и из язычников?
\vs Rom 9:25 Как и у Осии говорит: не Мой народ назову Моим народом, и не возлюбленную~--- возлюбленною.
\vs Rom 9:26 И на том месте, где сказано им: вы не Мой народ, там названы будут сынами Бога живаго.
\vs Rom 9:27 А Исаия провозглашает об Израиле: хотя бы сыны Израилевы были числом, как песок морской, \bibemph{только} остаток спасется;
\vs Rom 9:28 ибо дело оканчивает и скоро решит по правде, дело решительное совершит Господь на земле.
\vs Rom 9:29 И, как предсказал Исаия: если бы Господь Саваоф не оставил нам семени, то мы сделались бы, как Содом, и были бы подобны Гоморре.
\rsbpar\vs Rom 9:30 Что же скажем? Язычники, не искавшие праведности, получили праведность, праведность от веры.
\vs Rom 9:31 А Израиль, искавший закона праведности, не достиг до закона праведности.
\vs Rom 9:32 Почему? потому что \bibemph{искали} не в вере, а в делах закона. Ибо преткнулись о камень преткновения,
\vs Rom 9:33 как написано: вот, полагаю в Сионе камень преткновения и камень соблазна; но всякий, верующий в Него, не постыдится.
\vs Rom 10:1 Братия! желание моего сердца и молитва к Богу об Израиле во спасение.
\vs Rom 10:2 Ибо свидетельствую им, что имеют ревность по Боге, но не по рассуждению.
\vs Rom 10:3 Ибо, не разумея праведности Божией и усиливаясь поставить собственную праведность, они не покорились праведности Божией,
\vs Rom 10:4 потому что конец закона~--- Христос, к праведности всякого верующего.
\vs Rom 10:5 Моисей пишет о праведности от закона: исполнивший его человек жив будет им.
\vs Rom 10:6 А праведность от веры так говорит: не говори в сердце твоем: кто взойдет на небо? то есть Христа свести.
\vs Rom 10:7 Или кто сойдет в бездну? то есть Христа из мертвых возвести.
\vs Rom 10:8 Но что говорит Писание? Близко к тебе слово, в устах твоих и в сердце твоем, то есть слово веры, которое проповедуем.
\vs Rom 10:9 Ибо если устами твоими будешь исповедовать Иисуса Господом и сердцем твоим веровать, что Бог воскресил Его из мертвых, то спасешься,
\vs Rom 10:10 потому что сердцем веруют к праведности, а устами исповедуют ко спасению.
\vs Rom 10:11 Ибо Писание говорит: всякий, верующий в Него, не постыдится.
\vs Rom 10:12 Здесь нет различия между Иудеем и Еллином, потому что один Господь у всех, богатый для всех, призывающих Его.
\vs Rom 10:13 Ибо всякий, кто призовет имя Господне, спасется.
\rsbpar\vs Rom 10:14 Но как призывать \bibemph{Того}, в Кого не уверовали? как веровать \bibemph{в Того}, о Ком не слыхали? как слышать без проповедующего?
\vs Rom 10:15 И как проповедовать, если не будут посланы? как написано: как прекрасны ноги благовествующих мир, благовествующих благое!
\vs Rom 10:16 Но не все послушались благовествования. Ибо Исаия говорит: Господи! кто поверил слышанному от нас?
\vs Rom 10:17 Итак вера от слышания, а слышание от слова Божия.
\vs Rom 10:18 Но спрашиваю: разве они не слышали? Напротив, по всей земле прошел голос их, и до пределов вселенной слова их.
\vs Rom 10:19 Еще спрашиваю: разве Израиль не знал? Но первый Моисей говорит: Я возбужу в вас ревность не народом, раздражу вас народом несмысленным.
\vs Rom 10:20 А Исаия смело говорит: Меня нашли не искавшие Меня; Я открылся не вопрошавшим о Мне.
\vs Rom 10:21 Об Израиле же говорит: целый день Я простирал руки Мои к народу непослушному и упорному.
\vs Rom 11:1 Итак, спрашиваю: неужели Бог отверг народ Свой? Никак. Ибо и я Израильтянин, от семени Авраамова, из колена Вениаминова.
\vs Rom 11:2 Не отверг Бог народа Своего, который Он наперед знал. Или не знаете, что говорит Писание в \bibemph{повествовании об} Илии? как он жалуется Богу на Израиля, говоря:
\vs Rom 11:3 Господи! пророков Твоих убили, жертвенники Твои разрушили; остался я один, и моей души ищут.
\vs Rom 11:4 Что же говорит ему Божеский ответ? Я соблюл Себе семь тысяч человек, которые не преклонили колени перед Ваалом.
\vs Rom 11:5 Так и в нынешнее время, по избранию благодати, сохранился остаток.
\vs Rom 11:6 Но если по благодати, то не по делам; иначе благодать не была бы уже благодатью. А если по делам, то это уже не благодать; иначе дело не есть уже дело.
\vs Rom 11:7 Что же? Израиль, чего искал, того не получил; избранные же получили, а прочие ожесточились,
\vs Rom 11:8 как написано: Бог дал им дух усыпления, глаза, которыми не видят, и уши, которыми не слышат, даже до сего дня.
\vs Rom 11:9 И Давид говорит: да будет трапеза их сетью, тенетами и петлею в возмездие им;
\vs Rom 11:10 да помрачатся глаза их, чтобы не видеть, и хребет их да будет согбен навсегда.
\rsbpar\vs Rom 11:11 Итак спрашиваю: неужели они преткнулись, чтобы \bibemph{совсем} пасть? Никак. Но от их падения спасение язычникам, чтобы возбудить в них ревность.
\vs Rom 11:12 Если же падение их~--- богатство миру, и оскудение их~--- богатство язычникам, то тем более полнота их.
\rsbpar\vs Rom 11:13 Вам говорю, язычникам. Как Апостол язычников, я прославляю служение мое.
\vs Rom 11:14 Не возбужу ли ревность в \bibemph{сродниках} моих по плоти и не спасу ли некоторых из них?
\vs Rom 11:15 Ибо если отвержение их~--- примирение мира, то что \bibemph{будет} принятие, как не жизнь из мертвых?
\vs Rom 11:16 Если начаток свят, то и целое; и если корень свят, то и ветви.
\vs Rom 11:17 Если же некоторые из ветвей отломились, а ты, дикая маслина, привился на место их и стал общником корня и сока маслины,
\vs Rom 11:18 то не превозносись перед ветвями. Если же превозносишься, \bibemph{то вспомни, что} не ты корень держишь, но корень тебя.
\vs Rom 11:19 Скажешь: <<ветви отломились, чтобы мне привиться>>.
\vs Rom 11:20 Хорошо. Они отломились неверием, а ты держишься верою: не гордись, но бойся.
\vs Rom 11:21 Ибо если Бог не пощадил природных ветвей, то смотри, пощадит ли и тебя.
\vs Rom 11:22 Итак видишь благость и строгость Божию: строгость к отпадшим, а благость к тебе, если пребудешь в благости \bibemph{Божией}; иначе и ты будешь отсечен.
\vs Rom 11:23 Но и те, если не пребудут в неверии, привьются, потому что Бог силен опять привить их.
\vs Rom 11:24 Ибо если ты отсечен от дикой по природе маслины и не по природе привился к хорошей маслине, то тем более сии природные привьются к своей маслине.
\rsbpar\vs Rom 11:25 Ибо не хочу оставить вас, братия, в неведении о тайне сей,~--- чтобы вы не мечтали о себе,~--- что ожесточение произошло в Израиле отчасти, \bibemph{до времени}, пока войдет полное \bibemph{число} язычников;
\vs Rom 11:26 и так весь Израиль спасется, как написано: придет от Сиона Избавитель, и отвратит нечестие от Иакова.
\vs Rom 11:27 И сей завет им от Меня, когда сниму с них грехи их.
\vs Rom 11:28 В отношении к благовестию, они враги ради вас; а в отношении к избранию, возлюбленные \bibemph{Божии} ради отцов.
\vs Rom 11:29 Ибо дары и призвание Божие непреложны.
\vs Rom 11:30 Как и вы некогда были непослушны Богу, а ныне помилованы, по непослушанию их,
\vs Rom 11:31 так и они теперь непослушны для помилования вас, чтобы и сами они были помилованы.
\vs Rom 11:32 Ибо всех заключил Бог в непослушание, чтобы всех помиловать.
\rsbpar\vs Rom 11:33 О, бездна богатства и премудрости и ведения Божия! Как непостижимы судьбы Его и неисследимы пути Его!
\vs Rom 11:34 Ибо кто познал ум Господень? Или кто был советником Ему?
\vs Rom 11:35 Или кто дал Ему наперед, чтобы Он должен был воздать?
\vs Rom 11:36 Ибо все из Него, Им и к Нему. Ему слава во веки, аминь.
\vs Rom 12:1 Итак умоляю вас, братия, милосердием Божиим, представьте тела ваши в жертву живую, святую, благоугодную Богу, \bibemph{для} разумного служения вашего,
\vs Rom 12:2 и не сообразуйтесь с веком сим, но преобразуйтесь обновлением ума вашего, чтобы вам познавать, чт\acc{о} есть воля Божия, благая, угодная и совершенная.
\rsbpar\vs Rom 12:3 По данной мне благодати, всякому из вас говорю: не думайте \bibemph{о себе} более, нежели должно думать; но думайте скромно, по мере веры, какую каждому Бог уделил.
\vs Rom 12:4 Ибо, как в одном теле у нас много членов, но не у всех членов одно и то же дело,
\vs Rom 12:5 так мы, многие, составляем одно тело во Христе, а порознь один для другого члены.
\vs Rom 12:6 И как, по данной нам благодати, имеем различные дарования, \bibemph{то, имеешь ли} пророчество, \bibemph{пророчествуй} по мере веры;
\vs Rom 12:7 \bibemph{имеешь ли} служение, \bibemph{пребывай} в служении; учитель ли,~--- в учении;
\vs Rom 12:8 увещатель ли, увещевай; раздаватель ли, \bibemph{раздавай} в простоте; начальник ли, \bibemph{начальствуй} с усердием; благотворитель ли, \bibemph{благотвори} с радушием.
\vs Rom 12:9 Любовь \bibemph{да будет} непритворна; отвращайтесь зла, прилепляйтесь к добру;
\vs Rom 12:10 будьте братолюбивы друг к другу с нежностью; в почтительности друг друга предупреждайте;
\vs Rom 12:11 в усердии не ослабевайте; духом пламенейте; Господу служите;
\vs Rom 12:12 утешайтесь надеждою; в скорби \bibemph{будьте} терпеливы, в молитве постоянны;
\vs Rom 12:13 в нуждах святых принимайте участие; ревнуйте о странноприимстве.
\vs Rom 12:14 Благословляйте гонителей ваших; благословляйте, а не проклинайте.
\vs Rom 12:15 Радуйтесь с радующимися и плачьте с плачущими.
\vs Rom 12:16 Будьте единомысленны между собою; не высокомудрствуйте, но последуйте смиренным; не мечтайте о себе;
\vs Rom 12:17 никому не воздавайте злом за зло, но пекитесь о добром перед всеми человеками.
\vs Rom 12:18 Если возможно с вашей стороны, будьте в мире со всеми людьми.
\vs Rom 12:19 Не мстите за себя, возлюбленные, но дайте место гневу \bibemph{Божию}. Ибо написано: Мне отмщение, Я воздам, говорит Господь.
\vs Rom 12:20 Итак, если враг твой голоден, накорми его; если жаждет, напой его: ибо, делая сие, ты соберешь ему на голову горящие уголья.
\vs Rom 12:21 Не будь побежден злом, но побеждай зло добром.
\vs Rom 13:1 Всякая душа да будет покорна высшим властям, ибо нет власти не от Бога; существующие же власти от Бога установлены.
\vs Rom 13:2 Посему противящийся власти противится Божию установлению. А противящиеся сами навлекут на себя осуждение.
\vs Rom 13:3 Ибо начальствующие страшны не для добрых дел, но для злых. Хочешь ли не бояться власти? Делай добро, и получишь похвалу от нее,
\vs Rom 13:4 ибо \bibemph{начальник} есть Божий слуга, тебе на добро. Если же делаешь зло, бойся, ибо он не напрасно носит меч: он Божий слуга, отмститель в наказание делающему злое.
\vs Rom 13:5 И потому надобно повиноваться не только из \bibemph{страха} наказания, но и по совести.
\vs Rom 13:6 Для сего вы и подати платите, ибо они Божии служители, сим самым постоянно занятые.
\vs Rom 13:7 Итак отдавайте всякому должное: кому п\acc{о}дать, подать; кому оброк, оброк; кому страх, страх; кому честь, честь.
\rsbpar\vs Rom 13:8 Не оставайтесь должными никому ничем, кроме взаимной любви; ибо любящий другого исполнил закон.
\vs Rom 13:9 Ибо заповеди: не прелюбодействуй, не убивай, не кради, не лжесвидетельствуй, не пожелай \bibemph{чужого} и все другие заключаются в сем слове: люби ближнего твоего, как самого себя.
\vs Rom 13:10 Любовь не делает ближнему зла; итак любовь есть исполнение закона.
\rsbpar\vs Rom 13:11 Так \bibemph{поступайте}, зная время, что наступил уже час пробудиться нам от сна. Ибо ныне ближе к нам спасение, нежели когда мы уверовали.
\vs Rom 13:12 Ночь прошла, а день приблизился: итак отвергнем дела тьмы и облечемся в оружия света.
\vs Rom 13:13 Как днем, будем вести себя благочинно, не \bibemph{предаваясь} ни пированиям и пьянству, ни сладострастию и распутству, ни ссорам и зависти;
\vs Rom 13:14 но облекитесь в Господа нашего Иисуса Христа, и попечения о плоти не превращайте в похоти.
\vs Rom 14:1 Немощного в вере принимайте без споров о мнениях.
\vs Rom 14:2 Ибо иной уверен, \bibemph{что можно} есть все, а немощный ест овощи.
\vs Rom 14:3 Кто ест, не уничижай того, кто не ест; и кто не ест, не осуждай того, кто ест, потому что Бог принял его.
\vs Rom 14:4 Кто ты, осуждающий чужого раба? Перед своим Господом сто\acc{и}т он, или падает. И будет восставлен, ибо силен Бог восставить его.
\vs Rom 14:5 Иной отличает день от дня, а другой судит о всяком дне \bibemph{равно}. Всякий \bibemph{поступай} по удостоверению своего ума.
\vs Rom 14:6 Кто различает дни, для Господа различает; и кто не различает дней, для Господа не различает. Кто ест, для Господа ест, ибо благодарит Бога; и кто не ест, для Господа не ест, и благодарит Бога.
\vs Rom 14:7 Ибо никто из нас не живет для себя, и никто не умирает для себя;
\vs Rom 14:8 а живем ли~--- для Господа живем; умираем ли~--- для Господа умираем: и потому, живем ли или умираем,~--- \bibemph{всегда} Господни.
\vs Rom 14:9 Ибо Христос для того и умер, и воскрес, и ожил, чтобы владычествовать и над мертвыми и над живыми.
\vs Rom 14:10 А ты чт\acc{о} осуждаешь брата твоего? Или и ты, чт\acc{о} унижаешь брата твоего? Все мы предстанем на суд Христов.
\vs Rom 14:11 Ибо написано: живу Я, говорит Господь, предо Мною преклонится всякое колено, и всякий язык будет исповедовать Бога.
\vs Rom 14:12 Итак каждый из нас за себя даст отчет Богу.
\rsbpar\vs Rom 14:13 Не станем же более судить друг друга, а лучше судите о том, как бы не подавать брату \bibemph{случая к} преткновению или соблазну.
\vs Rom 14:14 Я знаю и уверен в Господе Иисусе, что нет ничего в себе самом нечистого; только почитающему что-либо нечистым, тому нечисто.
\vs Rom 14:15 Если же за пищу огорчается брат твой, то ты уже не по любви поступаешь. Не губи твоею пищею того, за кого Христос умер.
\vs Rom 14:16 Да не хулится ваше доброе.
\vs Rom 14:17 Ибо Царствие Божие не пища и питие, но праведность и мир и радость во Святом Духе.
\vs Rom 14:18 Кто сим служит Христу, тот угоден Богу и \bibemph{достоин} одобрения от людей.
\vs Rom 14:19 Итак будем искать того, что служит к миру и ко взаимному назиданию.
\vs Rom 14:20 Ради пищи не разрушай д\acc{е}ла Божия. Все чисто, но худо человеку, который ест на соблазн.
\vs Rom 14:21 Лучше не есть мяса, не пить вина и не \bibemph{делать} ничего \bibemph{такого}, отчего брат твой претыкается, или соблазняется, или изнемогает.
\vs Rom 14:22 Ты имеешь веру? имей ее сам в себе, пред Богом. Блажен, кто не осуждает себя в том, чт\acc{о} избирает.
\vs Rom 14:23 А сомневающийся, если ест, осуждается, потому что не по вере; а все, что не по вере, грех.
\vs Rom 14:24 Могущему же утвердить вас, по благовествованию моему и проповеди Иисуса Христа, по откровению тайны, о которой от вечных времен было умолчано,
\vs Rom 14:25 но которая ныне явлена, и через писания пророческие, по повелению вечного Бога, возвещена всем народам для покорения их вере,
\vs Rom 14:26 Единому Премудрому Богу, через Иисуса Христа, слава во веки. Аминь.
\vs Rom 15:1 Мы, сильные, должны сносить немощи бессильных и не себе угождать.
\vs Rom 15:2 Каждый из нас должен угождать ближнему, во благо, к назиданию.
\vs Rom 15:3 Ибо и Христос не Себе угождал, но, как написано: злословия злословящих Тебя пали на Меня.
\vs Rom 15:4 А все, что писано было прежде, написано нам в наставление, чтобы мы терпением и утешением из Писаний сохраняли надежду.
\vs Rom 15:5 Бог же терпения и утешения да дарует вам быть в единомыслии между собою, по \bibemph{учению} Христа Иисуса,
\vs Rom 15:6 дабы вы единодушно, едиными устами славили Бога и Отца Господа нашего Иисуса Христа.
\vs Rom 15:7 Посему принимайте друг друга, как и Христос принял вас в славу Божию.
\rsbpar\vs Rom 15:8 Разумею то, что Иисус Христос сделался служителем для обрезанных~--- ради истины Божией, чтобы исполнить обещанное отцам,
\vs Rom 15:9 а для язычников~--- из милости, чтобы славили Бога, как написано: за т\acc{о} буду славить Тебя, (Господи,) между язычниками, и буду петь имени Твоему.
\vs Rom 15:10 И еще сказано: возвеселитесь, язычники, с народом Его.
\vs Rom 15:11 И еще: хвал\acc{и}те Господа, все язычники, и прославляйте Его, все народы.
\vs Rom 15:12 Исаия также говорит: будет корень Иессеев, и восстанет владеть народами; на Него язычники надеяться будут.
\vs Rom 15:13 Бог же надежды да исполнит вас всякой радости и мира в вере, дабы вы, силою Духа Святаго, обогатились надеждою.
\rsbpar\vs Rom 15:14 И сам я уверен о вас, братия мои, что и вы полны благости, исполнены всякого познания и можете наставлять друг друга;
\vs Rom 15:15 но писал вам, братия, с некоторою смелостью, отчасти как бы в напоминание вам, по данной мне от Бога благодати
\vs Rom 15:16 быть служителем Иисуса Христа у язычников и \bibemph{совершать} священнодействие благовествования Божия, дабы сие приношение язычников, будучи освящено Духом Святым, было благоприятно \bibemph{Богу}.
\vs Rom 15:17 Итак я могу похвалиться в Иисусе Христе в том, чт\acc{о} \bibemph{относится} к Богу,
\vs Rom 15:18 ибо не осмелюсь сказать что-нибудь такое, чего не совершил Христос через меня, в покорении язычников \bibemph{вере}, словом и делом,
\vs Rom 15:19 силою знамений и чудес, силою Духа Божия, так что благовествование Христово распространено мною от Иерусалима и окрестности до Иллирика.
\vs Rom 15:20 Притом я старался благовествовать не там, где \bibemph{уже} было известно имя Христово, дабы не созидать на чужом основании,
\vs Rom 15:21 но как написано: не имевшие о Нем известия увидят, и не слышавшие узн\acc{а}ют.
\vs Rom 15:22 Сие-то много раз и препятствовало мне прийти к вам.
\vs Rom 15:23 Ныне же, не имея \bibemph{такого} места в сих странах, а с давних лет имея желание прийти к вам,
\vs Rom 15:24 как только предприму путь в Испанию, приду к вам. Ибо надеюсь, что, проходя, увижусь с вами и что вы проводите меня туда, как скоро наслажусь \bibemph{общением} с вами, хотя отчасти.
\vs Rom 15:25 А теперь я иду в Иерусалим, чтобы послужить святым,
\vs Rom 15:26 ибо Македония и Ахаия усердствуют некоторым подаянием для бедных между святыми в Иерусалиме.
\vs Rom 15:27 Усердствуют, да и должники они перед ними. Ибо если язычники сделались участниками в их духовном, то должны и им послужить в телесном.
\vs Rom 15:28 Исполнив это и верно доставив им сей плод \bibemph{усердия}, я отправлюсь через ваши \bibemph{места} в Испанию,
\vs Rom 15:29 и уверен, что когда приду к вам, то приду с полным благословением благовествования Христова.
\rsbpar\vs Rom 15:30 Между тем умоляю вас, братия, Господом нашим Иисусом Христом и любовью Духа, подвизаться со мною в молитвах за меня к Богу,
\vs Rom 15:31 чтобы избавиться мне от неверующих в Иудее и чтобы служение мое для Иерусалима было благоприятно святым,
\vs Rom 15:32 дабы мне в радости, если Богу угодно, прийти к вам и успокоиться с вами.
\vs Rom 15:33 Бог же мира да будет со всеми вами, аминь.
\vs Rom 16:1 Представляю вам Фиву, сестру нашу, диакониссу церкви Кенхрейской.
\vs Rom 16:2 Примите ее для Господа, как прилично святым, и помогите ей, в чем она будет иметь нужду у вас, ибо и она была помощницею многим и мне самому.
\rsbpar\vs Rom 16:3 Приветствуйте Прискиллу и Акилу, сотрудников моих во Христе Иисусе
\vs Rom 16:4 (которые голову свою полагали за мою душу, которых не я один благодарю, но и все церкви из язычников), и домашнюю их церковь.
\vs Rom 16:5 Приветствуйте возлюбленного моего Епенета, который есть начаток Ахаии для Христа.
\vs Rom 16:6 Приветствуйте Мариам, которая много трудилась для нас.
\vs Rom 16:7 Приветствуйте Андроника и Юнию, сродников моих и узников со мною, прославившихся между Апостолами и прежде меня еще уверовавших во Христа.
\vs Rom 16:8 Приветствуйте Амплия, возлюбленного мне в Господе.
\vs Rom 16:9 Приветствуйте Урбана, сотрудника нашего во Христе, и Стахия, возлюбленного мне.
\vs Rom 16:10 Приветствуйте Апеллеса, испытанного во Христе. Приветствуйте \bibemph{верных} из дома Аристовулова.
\vs Rom 16:11 Приветствуйте Иродиона, сродника моего. Приветствуйте из домашних Наркисса тех, которые в Господе.
\vs Rom 16:12 Приветствуйте Трифену и Трифосу, трудящихся о Господе. Приветствуйте Персиду возлюбленную, которая много потрудилась о Господе.
\vs Rom 16:13 Приветствуйте Руфа, избранного в Господе, и матерь его и мою.
\vs Rom 16:14 Приветствуйте Асинкрита, Флегонта, Ерма, Патрова, Ермия и других с ними братьев.
\vs Rom 16:15 Приветствуйте Филолога и Юлию, Нирея и сестру его, и Олимпана, и всех с ними святых.
\vs Rom 16:16 Приветствуйте друг друга с целованием святым. Приветствуют вас все церкви Христовы.
\rsbpar\vs Rom 16:17 Умоляю вас, братия, остерегайтесь производящих разделения и соблазны, вопреки учению, которому вы научились, и уклоняйтесь от них;
\vs Rom 16:18 ибо такие \bibemph{люди} служат не Господу нашему Иисусу Христу, а своему чреву, и ласкательством и красноречием обольщают сердца простодушных.
\vs Rom 16:19 Ваша покорность \bibemph{вере} всем известна; посему я радуюсь за вас, но желаю, чтобы вы были мудры на добро и просты на зло.
\vs Rom 16:20 Бог же мира сокрушит сатану под ногами вашими вскоре. Благодать Господа нашего Иисуса Христа с вами! Аминь.
\rsbpar\vs Rom 16:21 Приветствуют вас Тимофей, сотрудник мой, и Луций, Иасон и Сосипатр, сродники мои.
\vs Rom 16:22 Приветствую вас в Господе и я, Тертий, писавший сие послание.
\vs Rom 16:23 Приветствует вас Гаий, странноприимец мой и всей церкви. Приветствует вас Ераст, городской казнохранитель, и брат Кварт.
\rsbpar\vs Rom 16:24 Благодать Господа нашего Иисуса Христа со всеми вами. Аминь.

\bibbookdescr{1Co}{
  inline={Первое Послание\\к Коринфянам\\\LARGE Святого Апостола Павла},
  toc={1-е Коринфянам},
  bookmark={1-е Коринфянам},
  header={1-е Коринфянам},
  %headerleft={},
  %headerright={},
  abbr={1~Кор}
}
\vs 1Co 1:1 Павел, волею Божиею призванный Апостол Иисуса Христа, и Сосфен брат,
\vs 1Co 1:2 церкви Божией, находящейся в Коринфе, освященным во Христе Иисусе, призванным святым, со всеми призывающими имя Господа нашего Иисуса Христа, во всяком месте, у них и у нас:
\vs 1Co 1:3 благодать вам и мир от Бога Отца нашего и Господа Иисуса Христа.
\rsbpar\vs 1Co 1:4 Непрестанно благодарю Бога моего за вас, ради благодати Божией, дарованной вам во Христе Иисусе,
\vs 1Co 1:5 потому что в Нем вы обогатились всем, всяким словом и всяким познанием,~---
\vs 1Co 1:6 ибо свидетельство Христово утвердилось в вас,~---
\vs 1Co 1:7 так что вы не имеете недостатка ни в каком даровании, ожидая явления Господа нашего Иисуса Христа,
\vs 1Co 1:8 Который и утвердит вас до конца, \bibemph{чтобы вам быть} неповинными в день Господа нашего Иисуса Христа.
\vs 1Co 1:9 Верен Бог, Которым вы призваны в общение Сына Его Иисуса Христа, Господа нашего.
\rsbpar\vs 1Co 1:10 Умоляю вас, братия, именем Господа нашего Иисуса Христа, чтобы все вы говорили одно, и не было между вами разделений, но чтобы вы соединены были в одном духе и в одних мыслях.
\vs 1Co 1:11 Ибо от \bibemph{домашних} Хлоиных сделалось мне известным о вас, братия мои, что между вами есть споры.
\vs 1Co 1:12 Я разумею то, что у вас говорят: <<я Павлов>>; <<я Аполлосов>>; <<я Кифин>>; <<а я Христов>>.
\vs 1Co 1:13 Разве разделился Христос? разве Павел распялся за вас? или во имя Павла вы крестились?
\vs 1Co 1:14 Благодарю Бога, что я никого из вас не крестил, кроме Криспа и Гаия,
\vs 1Co 1:15 дабы не сказал кто, что я крестил в мое имя.
\vs 1Co 1:16 Крестил я также Стефанов дом; а крестил ли еще кого, не знаю.
\vs 1Co 1:17 Ибо Христос послал меня не крестить, а благовествовать, не в премудрости слова, чтобы не упразднить креста Христова.
\vs 1Co 1:18 Ибо слово о кресте для погибающих юродство есть, а для нас, спасаемых,~--- сила Божия.
\vs 1Co 1:19 Ибо написано: погублю мудрость мудрецов, и разум разумных отвергну.
\vs 1Co 1:20 Где мудрец? где книжник? где совопросник века сего? Не обратил ли Бог мудрость мира сего в безумие?
\vs 1Co 1:21 Ибо когда мир \bibemph{своею} мудростью не познал Бога в премудрости Божией, то благоугодно было Богу юродством проповеди спасти верующих.
\vs 1Co 1:22 Ибо и Иудеи требуют чудес, и Еллины ищут мудрости;
\vs 1Co 1:23 а мы проповедуем Христа распятого, для Иудеев соблазн, а для Еллинов безумие,
\vs 1Co 1:24 для самих же призванных, Иудеев и Еллинов, Христа, Божию силу и Божию премудрость;
\vs 1Co 1:25 потому что немудрое Божие премудрее человеков, и немощное Божие сильнее человеков.
\rsbpar\vs 1Co 1:26 Посмотрите, братия, кто вы, призванные: не много \bibemph{из вас} мудрых по плоти, не много сильных, не много благородных;
\vs 1Co 1:27 но Бог избрал немудрое мира, чтобы посрамить мудрых, и немощное мира избрал Бог, чтобы посрамить сильное;
\vs 1Co 1:28 и незнатное мира и уничиженное и ничего не значащее избрал Бог, чтобы упразднить значащее,~---
\vs 1Co 1:29 для того, чтобы никакая плоть не хвалилась пред Богом.
\vs 1Co 1:30 От Него и вы во Христе Иисусе, Который сделался для нас премудростью от Бога, праведностью и освящением и искуплением,
\vs 1Co 1:31 чтобы \bibemph{было}, как написано: хвалящийся хвались Господом.
\vs 1Co 2:1 И когда я приходил к вам, братия, приходил возвещать вам свидетельство Божие не в превосходстве слова или мудрости,
\vs 1Co 2:2 ибо я рассудил быть у вас незнающим ничего, кроме Иисуса Христа, и притом распятого,
\vs 1Co 2:3 и был я у вас в немощи и в страхе и в великом трепете.
\vs 1Co 2:4 И слово мое и проповедь моя не в убедительных словах человеческой мудрости, но в явлении духа и силы,
\vs 1Co 2:5 чтобы вера ваша \bibemph{утверждалась} не на мудрости человеческой, но на силе Божией.
\rsbpar\vs 1Co 2:6 Мудрость же мы проповедуем между совершенными, но мудрость не века сего и не властей века сего преходящих,
\vs 1Co 2:7 но проповедуем премудрость Божию, тайную, сокровенную, которую предназначил Бог прежде веков к славе нашей,
\vs 1Co 2:8 которой никто из властей века сего не познал; ибо если бы познали, то не распяли бы Господа славы.
\vs 1Co 2:9 Но, как написано: не видел того глаз, не слышало ухо, и не приходило то на сердце человеку, что приготовил Бог любящим Его.
\vs 1Co 2:10 А нам Бог открыл \bibemph{это} Духом Своим; ибо Дух все проницает, и глубины Божии.
\vs 1Co 2:11 Ибо кто из человеков знает, чт\acc{о} в человеке, кроме духа человеческого, живущего в нем? Т\acc{а}к и Божьего никто не знает, кроме Духа Божия.
\vs 1Co 2:12 Но мы приняли не духа мира сего, а Духа от Бога, дабы знать дарованное нам от Бога,
\vs 1Co 2:13 что и возвещаем не от человеческой мудрости изученными словами, но изученными от Духа Святаго, соображая духовное с духовным.
\vs 1Co 2:14 Душевный человек не принимает того, чт\acc{о} от Духа Божия, потому что он почитает это безумием; и не может разуметь, потому что о сем \bibemph{надобно} судить духовно.
\vs 1Co 2:15 Но духовный судит о всем, а о нем судить никто не может.
\vs 1Co 2:16 Ибо кто познал ум Господень, чтобы \bibemph{мог} судить его? А мы имеем ум Христов.
\vs 1Co 3:1 И я не мог говорить с вами, братия, как с духовными, но как с плотскими, как с младенцами во Христе.
\vs 1Co 3:2 Я питал вас молоком, а не \bibemph{твердою} пищею, ибо вы были еще не в силах, да и теперь не в силах,
\vs 1Co 3:3 потому что вы еще плотские. Ибо если между вами зависть, споры и разногласия, то не плотские ли вы? и не по человеческому ли \bibemph{обычаю} поступаете?
\vs 1Co 3:4 Ибо когда один говорит: <<я Павлов>>, а другой: <<я Аполлосов>>, то не плотские ли вы?
\vs 1Co 3:5 Кто Павел? кто Аполлос? Они только служители, через которых вы уверовали, и притом поскольку каждому дал Господь.
\vs 1Co 3:6 Я насадил, Аполлос поливал, но возрастил Бог;
\vs 1Co 3:7 посему и насаждающий и поливающий есть ничто, а \bibemph{все} Бог возращающий.
\vs 1Co 3:8 Насаждающий же и поливающий суть одно; но каждый получит свою награду по своему труду.
\vs 1Co 3:9 Ибо мы соработники у Бога, \bibemph{а} вы Божия нива, Божие строение.
\rsbpar\vs 1Co 3:10 Я, по данной мне от Бога благодати, как мудрый строитель, положил основание, а другой строит на \bibemph{нем}; но каждый смотри, к\acc{а}к строит.
\vs 1Co 3:11 Ибо никто не может положить другого основания, кроме положенного, которое есть Иисус Христос.
\vs 1Co 3:12 Строит ли кто на этом основании из золота, серебра, драгоценных камней, дерева, сена, соломы,~---
\vs 1Co 3:13 каждого дело обнаружится; ибо день покажет, потому что в огне открывается, и огонь испытает дело каждого, каково оно есть.
\vs 1Co 3:14 У кого дело, которое он строил, устоит, тот получит награду.
\vs 1Co 3:15 А у кого дело сгорит, тот потерпит урон; впрочем сам спасется, но т\acc{а}к, как бы из огня.
\rsbpar\vs 1Co 3:16 Разве не знаете, что вы храм Божий, и Дух Божий живет в вас?
\vs 1Co 3:17 Если кто разорит храм Божий, того покарает Бог: ибо храм Божий свят; а этот \bibemph{храм}~--- вы.
\rsbpar\vs 1Co 3:18 Никто не обольщай самого себя. Если кто из вас думает быть мудрым в веке сем, тот будь безумным, чтобы быть мудрым.
\vs 1Co 3:19 Ибо мудрость мира сего есть безумие пред Богом, как написано: уловляет мудрых в лукавстве их.
\vs 1Co 3:20 И еще: Господь знает умствования мудрецов, что они суетны.
\vs 1Co 3:21 Итак никто не хвались человеками, ибо все ваше:
\vs 1Co 3:22 Павел ли, или Аполлос, или Кифа, или мир, или жизнь, или смерть, или настоящее, или будущее,~--- все ваше;
\vs 1Co 3:23 вы же~--- Христовы, а Христос~--- Божий.
\vs 1Co 4:1 Итак каждый должен разуметь нас, как служителей Христовых и домостроителей таин Божиих.
\vs 1Co 4:2 От домостроителей же требуется, чтобы каждый оказался верным.
\vs 1Co 4:3 Для меня очень мало значит, к\acc{а}к судите обо мне вы или \bibemph{к\acc{а}к судят} другие люди; я и сам не сужу о себе.
\vs 1Co 4:4 Ибо \bibemph{хотя} я ничего не знаю за собою, но тем не оправдываюсь; судия же мне Господь.
\vs 1Co 4:5 Посему не суд\acc{и}те никак прежде времени, пока не придет Господь, Который и осветит скрытое во мраке и обнаружит сердечные намерения, и тогда каждому будет похвала от Бога.
\rsbpar\vs 1Co 4:6 Это, братия, приложил я к себе и Аполлосу ради вас, чтобы вы научились от нас не мудрствовать сверх того, что написано, и не превозносились один перед другим.
\vs 1Co 4:7 Ибо кто отличает тебя? Что ты имеешь, чего бы не получил? А если получил, что хвалишься, как будто не получил?
\vs 1Co 4:8 Вы уже пресытились, вы уже обогатились, вы стали царствовать без нас. О, если бы вы \bibemph{и в самом деле} царствовали, чтобы и нам с вами царствовать!
\vs 1Co 4:9 Ибо я думаю, что нам, последним посланникам, Бог судил быть как бы приговоренными к смерти, потому что мы сделались позорищем для мира, для Ангелов и человеков.
\vs 1Co 4:10 Мы безумны Христа ради, а вы мудры во Христе; мы немощны, а вы крепки; вы в славе, а мы в бесчестии.
\vs 1Co 4:11 Даже доныне терпим голод и жажду, и наготу и побои, и скитаемся,
\vs 1Co 4:12 и трудимся, работая своими руками. Злословят нас, мы благословляем; гонят нас, мы терпим;
\vs 1Co 4:13 хулят нас, мы молим; мы как сор для мира, \bibemph{как} прах, всеми \bibemph{попираемый} доныне.
\rsbpar\vs 1Co 4:14 Не к постыжению вашему пишу сие, но вразумляю вас, как возлюбленных детей моих.
\vs 1Co 4:15 Ибо, хотя у вас тысячи наставников во Христе, но не много отцов; я родил вас во Христе Иисусе благовествованием.
\vs 1Co 4:16 Посему умоляю вас: подражайте мне, как я Христу.
\vs 1Co 4:17 Для сего я послал к вам Тимофея, моего возлюбленного и верного в Господе сына, который напомнит вам о путях моих во Христе, как я учу везде во всякой церкви.
\vs 1Co 4:18 Как я не иду к вам, то некоторые \bibemph{у вас} возгордились;
\vs 1Co 4:19 но я скоро приду к вам, если угодно будет Господу, и испытаю не слова возгордившихся, а силу,
\vs 1Co 4:20 ибо Царство Божие не в слове, а в силе.
\vs 1Co 4:21 Чего вы хотите? с жезлом прийти к вам, или с любовью и духом кротости?
\vs 1Co 5:1 Есть верный слух, что у вас \bibemph{появилось} блудодеяние, и притом такое блудодеяние, какого не слышно даже у язычников, что некто \bibemph{вместо жены} имеет жену отца своего.
\vs 1Co 5:2 И вы возгордились, вместо того, чтобы лучше плакать, дабы изъят был из среды вас сделавший такое дело.
\vs 1Co 5:3 А я, отсутствуя телом, но присутствуя \bibemph{у вас} духом, уже решил, как бы находясь у вас: сделавшего такое дело,
\vs 1Co 5:4 в собрании вашем во имя Господа нашего Иисуса Христа, обще с моим духом, силою Господа нашего Иисуса Христа,
\vs 1Co 5:5 предать сатане во измождение плоти, чтобы дух был спасен в день Господа нашего Иисуса Христа.
\vs 1Co 5:6 Нечем вам хвалиться. Разве не знаете, что малая закваска квасит все тесто?
\vs 1Co 5:7 Итак очистите старую закваску, чтобы быть вам новым тестом, так как вы бесквасны, ибо Пасха наша, Христос, заклан за нас.
\vs 1Co 5:8 Посему станем праздновать не со старою закваскою, не с закваскою порока и лукавства, но с опресноками чистоты и истины.
\rsbpar\vs 1Co 5:9 Я писал вам в послании~--- не сообщаться с блудниками;
\vs 1Co 5:10 впрочем не вообще с блудниками мира сего, или лихоимцами, или хищниками, или идолослужителями, ибо иначе надлежало бы вам выйти из мира \bibemph{сего}.
\vs 1Co 5:11 Но я писал вам не сообщаться с тем, кто, называясь братом, остается блудником, или лихоимцем, или идолослужителем, или злоречивым, или пьяницею, или хищником; с таким даже и не есть вместе.
\vs 1Co 5:12 Ибо чт\acc{о} мне судить и внешних? Не внутренних ли вы судите?
\vs 1Co 5:13 Внешних же судит Бог. Итак, извергните развращенного из среды вас.
\vs 1Co 6:1 Как смеет кто у вас, имея дело с другим, судиться у нечестивых, а не у святых?
\vs 1Co 6:2 Разве не знаете, что святые будут судить мир? Если же вами будет судим мир, то неужели вы недостойны судить маловажные \bibemph{дела}?
\vs 1Co 6:3 Разве не знаете, что мы будем судить ангелов, не тем ли более \bibemph{дела} житейские?
\vs 1Co 6:4 А вы, когда имеете житейские тяжбы, поставляете \bibemph{своими судьями} ничего не значащих в церкви.
\vs 1Co 6:5 К стыду вашему говорю: неужели нет между вами ни одного разумного, который мог бы рассудить между братьями своими?
\vs 1Co 6:6 Но брат с братом судится, и притом перед неверными.
\vs 1Co 6:7 И то уже весьма унизительно для вас, что вы имеете тяжбы между собою. Для чего бы вам лучше не оставаться обиженными? для чего бы вам лучше не терпеть лишения?
\vs 1Co 6:8 Но вы \bibemph{сами} обижаете и отнимаете, и притом у братьев.
\vs 1Co 6:9 Или не знаете, что неправедные Царства Божия не наследуют? Не обманывайтесь: ни блудники, ни идолослужители, ни прелюбодеи, ни малакии, ни мужеложники,
\vs 1Co 6:10 ни воры, ни лихоимцы, ни пьяницы, ни злоречивые, ни хищники~--- Царства Божия не наследуют.
\vs 1Co 6:11 И такими были некоторые из вас; но омылись, но освятились, но оправдались именем Господа нашего Иисуса Христа и Духом Бога нашего.
\rsbpar\vs 1Co 6:12 Все мне позволительно, но не все полезно; все мне позволительно, но ничто не должно обладать мною.
\vs 1Co 6:13 Пища для чрева, и чрево для пищи; но Бог уничтожит и то и другое. Тело же не для блуда, но для Господа, и Господь для тела.
\vs 1Co 6:14 Бог воскресил Господа, воскресит и нас силою Своею.
\rsbpar\vs 1Co 6:15 Разве не знаете, что тел\acc{а} ваши суть члены Христовы? Итак отниму ли члены у Христа, чтобы сделать \bibemph{их} членами блудницы? Да не будет!
\vs 1Co 6:16 Или не знаете, что совокупляющийся с блудницею становится одно тело \bibemph{с нею}? ибо сказано: два будут одна плоть.
\vs 1Co 6:17 А соединяющийся с Господом есть один дух с Господом.
\vs 1Co 6:18 Бегайте блуда; всякий грех, какой делает человек, есть вне тела, а блудник грешит против собственного тела.
\vs 1Co 6:19 Не знаете ли, что тел\acc{а} ваши суть храм живущего в вас Святаго Духа, Которого имеете вы от Бога, и вы не свои?
\vs 1Co 6:20 Ибо вы куплены \bibemph{дорогою} ценою. Посему прославляйте Бога и в телах ваших и в душах ваших, которые суть Божии.
\vs 1Co 7:1 А о чем вы писали ко мне, то хорошо человеку не касаться женщины.
\vs 1Co 7:2 Но, \bibemph{во избежание} блуда, каждый имей свою жену, и каждая имей своего мужа.
\vs 1Co 7:3 Муж оказывай жене должное благорасположение; подобно и жена мужу.
\vs 1Co 7:4 Жена не властна над своим телом, но муж; равно и муж не властен над своим телом, но жена.
\vs 1Co 7:5 Не уклоняйтесь друг от друга, разве по согласию, на время, для упражнения в посте и молитве, а \bibemph{потом} опять будьте вместе, чтобы не искушал вас сатана невоздержанием вашим.
\vs 1Co 7:6 Впрочем это сказано мною как позволение, а не как повеление.
\vs 1Co 7:7 Ибо желаю, чтобы все люди были, как и я; но каждый имеет свое дарование от Бога, один так, другой иначе.
\rsbpar\vs 1Co 7:8 Безбрачным же и вдовам говорю: хорошо им оставаться, как я.
\vs 1Co 7:9 Но если не \bibemph{могут} воздержаться, пусть вступают в брак; ибо лучше вступить в брак, нежели разжигаться.
\vs 1Co 7:10 А вступившим в брак не я повелеваю, а Господь: жене не разводиться с мужем,~---
\vs 1Co 7:11 если же разведется, то должна оставаться безбрачною, или примириться с мужем своим,~--- и мужу не оставлять жены \bibemph{своей}.
\vs 1Co 7:12 Прочим же я говорю, а не Господь: если какой брат имеет жену неверующую, и она согласна жить с ним, то он не должен оставлять ее;
\vs 1Co 7:13 и жена, которая имеет мужа неверующего, и он согласен жить с нею, не должна оставлять его.
\vs 1Co 7:14 Ибо неверующий муж освящается женою верующею, и жена неверующая освящается мужем верующим. Иначе дети ваши были бы нечисты, а теперь святы.
\vs 1Co 7:15 Если же неверующий \bibemph{хочет} развестись, пусть разводится; брат или сестра в таких \bibemph{случаях} не связаны; к миру призвал нас Господь.
\vs 1Co 7:16 Почему ты знаешь, жена, не спасешь ли мужа? Или ты, муж, почему знаешь, не спасешь ли жены?
\vs 1Co 7:17 Только каждый поступай так, как Бог ему определил, и каждый, как Господь призвал. Так я повелеваю по всем церквам.
\vs 1Co 7:18 Призван ли кто обрезанным, не скрывайся; призван ли кто необрезанным, не обрезывайся.
\vs 1Co 7:19 Обрезание ничто и необрезание ничто, но \bibemph{всё} в соблюдении заповедей Божиих.
\vs 1Co 7:20 Каждый оставайся в том звании, в котором призван.
\vs 1Co 7:21 Рабом ли ты призван, не смущайся; но если и можешь сделаться свободным, то лучшим воспользуйся.
\vs 1Co 7:22 Ибо раб, призванный в Господе, есть свободный Господа; равно и призванный свободным есть раб Христов.
\vs 1Co 7:23 Вы куплены \bibemph{дорогою} ценою; не делайтесь рабами человеков.
\vs 1Co 7:24 В каком \bibemph{звании} кто призван, братия, в том каждый и оставайся пред Богом.
\rsbpar\vs 1Co 7:25 Относительно девства я не имею повеления Господня, а даю совет, как получивший от Господа милость быть \bibemph{Ему} верным.
\vs 1Co 7:26 По настоящей нужде за лучшее призна\acc{ю}, что хорошо человеку оставаться т\acc{а}к.
\vs 1Co 7:27 Соединен ли ты с женой? не ищи развода. Остался ли без жены? не ищи жены.
\vs 1Co 7:28 Впрочем, если и женишься, не согрешишь; и если девица выйдет замуж, не согрешит. Но таковые будут иметь скорби по плоти; а мне вас жаль.
\rsbpar\vs 1Co 7:29 Я вам сказываю, братия: время уже коротко, так что имеющие жен должны быть, как не имеющие;
\vs 1Co 7:30 и плачущие, как не плачущие; и радующиеся, как не радующиеся; и покупающие, как не приобретающие;
\vs 1Co 7:31 и пользующиеся миром сим, как не пользующиеся; ибо проходит образ мира сего.
\vs 1Co 7:32 А я хочу, чтобы вы были без забот. Неженатый заботится о Господнем, как угодить Господу;
\vs 1Co 7:33 а женатый заботится о мирском, как угодить жене. Есть разность между замужнею и девицею:
\vs 1Co 7:34 незамужняя заботится о Господнем, как угодить Господу, чтобы быть святою и телом и духом; а замужняя заботится о мирском, как угодить мужу.
\vs 1Co 7:35 Говорю это для вашей же пользы, не с тем, чтобы наложить на вас узы, но чтобы вы благочинно и непрестанно \bibemph{служили} Господу без развлечения.
\vs 1Co 7:36 Если же кто почитает неприличным для своей девицы то, чтобы она, будучи в зрелом возрасте, оставалась так, тот пусть делает, как хочет: не согрешит; пусть \bibemph{таковые} выходят замуж.
\vs 1Co 7:37 Но кто непоколебимо тверд в сердце своем и, не будучи стесняем нуждою, но будучи властен в своей воле, решился в сердце своем соблюдать свою деву, тот хорошо поступает.
\vs 1Co 7:38 Посему выдающий замуж свою девицу поступает хорошо; а не выдающий поступает лучше.
\vs 1Co 7:39 Жена связана законом, доколе жив муж ее; если же муж ее умрет, свободна выйти, за кого хочет, только в Господе.
\vs 1Co 7:40 Но она блаженнее, если останется так, по моему совету; а думаю, и я имею Духа Божия.
\vs 1Co 8:1 О идоложертвенных \bibemph{яствах} мы знаем, потому что мы все имеем знание; но знание надмевает, а любовь назидает.
\vs 1Co 8:2 Кто думает, что он знает что-нибудь, тот ничего еще не знает так, как должно знать.
\vs 1Co 8:3 Но кто любит Бога, тому дано знание от Него.
\vs 1Co 8:4 Итак об употреблении в пищу идоложертвенного мы знаем, что идол в мире ничто, и что нет иного Бога, кроме Единого.
\vs 1Co 8:5 Ибо хотя и есть так называемые боги, или на небе, или на земле, так как есть много богов и господ много,~---
\vs 1Co 8:6 но у нас один Бог Отец, из Которого все, и мы для Него, и один Господь Иисус Христос, Которым все, и мы Им.
\vs 1Co 8:7 Но не у всех \bibemph{такое} знание: некоторые и доныне с совестью, \bibemph{признающею} идолов, едят \bibemph{идоложертвенное} как жертвы идольские, и совесть их, будучи немощна, оскверняется.
\vs 1Co 8:8 Пища не приближает нас к Богу: ибо, едим ли мы, ничего не приобретаем; не едим ли, ничего не теряем.
\vs 1Co 8:9 Берегитесь однако же, чтобы эта свобода ваша не послужила соблазном для немощных.
\vs 1Co 8:10 Ибо если кто-нибудь увидит, что ты, имея знание, сидишь за столом в капище, то совесть его, как немощного, не расположит ли и его есть идоложертвенное?
\vs 1Co 8:11 И от знания твоего погибнет немощный брат, за которого умер Христос.
\vs 1Co 8:12 А согрешая таким образом против братьев и уязвляя немощную совесть их, вы согрешаете против Христа.
\vs 1Co 8:13 И потому, если пища соблазняет брата моего, не буду есть мяса вовек, чтобы не соблазнить брата моего.
\vs 1Co 9:1 Не Апостол ли я? Не свободен ли я? Не видел ли я Иисуса Христа, Господа нашего? Не мое ли дело вы в Господе?
\vs 1Co 9:2 Если для других я не Апостол, то для вас \bibemph{Апостол}; ибо печать моего апостольства~--- вы в Господе.
\vs 1Co 9:3 Вот мое защищение против осуждающих меня.
\vs 1Co 9:4 Или мы не имеем власти есть и пить?
\vs 1Co 9:5 Или не имеем власти иметь спутницею сестру жену, как и прочие Апостолы, и братья Господни, и Кифа?
\vs 1Co 9:6 Или один я и Варнава не имеем власти не работать?
\vs 1Co 9:7 Какой воин служит когда-либо на своем содержании? Кто, насадив виноград, не ест плодов его? Кто, пася стадо, не ест молока от стада?
\vs 1Co 9:8 По человеческому ли только \bibemph{рассуждению} я это говорю? Не то же ли говорит и закон?
\vs 1Co 9:9 Ибо в Моисеевом законе написано: не заграждай рта у вола молотящего. О волах ли печется Бог?
\vs 1Co 9:10 Или, конечно, для нас говорится? Так, для нас это написано; ибо, кто пашет, должен пахать с надеждою, и кто молотит, \bibemph{должен молотить} с надеждою получить ожидаемое.
\vs 1Co 9:11 Если мы посеяли в вас духовное, велико ли то, если пожнем у вас телесное?
\vs 1Co 9:12 Если другие имеют у вас власть, не паче ли мы? Однако мы не пользовались сею властью, но все переносим, дабы не поставить какой преграды благовествованию Христову.
\vs 1Co 9:13 Разве не знаете, что священнодействующие питаются от святилища? что служащие жертвеннику берут долю от жертвенника?
\vs 1Co 9:14 Т\acc{а}к и Господь повелел проповедующим Евангелие жить от благовествования.
\vs 1Co 9:15 Но я не пользовался ничем таковым. И написал это не для того, чтобы т\acc{а}к было для меня. Ибо для меня лучше умереть, нежели чтобы кто уничтожил похвалу мою.
\vs 1Co 9:16 Ибо если я благовествую, то нечем мне хвалиться, потому что это необходимая \bibemph{обязанность} моя, и горе мне, если не благовествую!
\vs 1Co 9:17 Ибо если делаю это добровольно, то \bibemph{буду} иметь награду; а если недобровольно, то \bibemph{исполняю только} вверенное мне служение.
\vs 1Co 9:18 За чт\acc{о} же мне награда? За т\acc{о}, что, проповедуя Евангелие, благовествую о Христе безмездно, не пользуясь моею властью в благовествовании.
\vs 1Co 9:19 Ибо, будучи свободен от всех, я всем поработил себя, дабы больше приобрести:
\vs 1Co 9:20 для Иудеев я был как Иудей, чтобы приобрести Иудеев; для подзаконных был как подзаконный, чтобы приобрести подзаконных;
\vs 1Co 9:21 для чуждых закона~--- как чуждый закона,~--- не будучи чужд закона пред Богом, но подзаконен Христу,~--- чтобы приобрести чуждых закона;
\vs 1Co 9:22 для немощных был как немощный, чтобы приобрести немощных. Для всех я сделался всем, чтобы спасти по крайней мере некоторых.
\vs 1Co 9:23 Сие же делаю для Евангелия, чтобы быть соучастником его.
\vs 1Co 9:24 Не знаете ли, что бегущие на ристалище бегут все, но один получает награду? Так бегите, чтобы получить.
\vs 1Co 9:25 Все подвижники воздерживаются от всего: те для получения венца тленного, а мы~--- нетленного.
\vs 1Co 9:26 И потому я бегу не так, как на неверное, бьюсь не так, чтобы только бить воздух;
\vs 1Co 9:27 но усмиряю и порабощаю тело мое, дабы, проповедуя другим, самому не остаться недостойным.
\vs 1Co 10:1 Не хочу оставить вас, братия, в неведении, что отцы наши все были под облаком, и все прошли сквозь море;
\vs 1Co 10:2 и все крестились в Моисея в облаке и в море;
\vs 1Co 10:3 и все ели одну и ту же духовную пищу;
\vs 1Co 10:4 и все пили одно и то же духовное питие: ибо пили из духовного последующего камня; камень же был Христос.
\vs 1Co 10:5 Но не о многих из них благоволил Бог, ибо они поражены были в пустыне.
\vs 1Co 10:6 А это были образы для нас, чтобы мы не были похотливы на злое, как они были похотливы.
\vs 1Co 10:7 Не будьте также идолопоклонниками, как некоторые из них, о которых написано: народ сел есть и пить, и встал играть.
\vs 1Co 10:8 Не станем блудодействовать, как некоторые из них блудодействовали, и в один день погибло их двадцать три тысячи.
\vs 1Co 10:9 Не станем искушать Христа, как некоторые из них искушали и погибли от змей.
\vs 1Co 10:10 Не ропщите, как некоторые из них роптали и погибли от истребителя.
\vs 1Co 10:11 Все это происходило с ними, \bibemph{как} образы; а описано в наставление нам, достигшим последних веков.
\vs 1Co 10:12 Посему, кто думает, что он сто\acc{и}т, берегись, чтобы не упасть.
\vs 1Co 10:13 Вас постигло искушение не иное, как человеческое; и верен Бог, Который не попустит вам быть искушаемыми сверх сил, но при искушении даст и облегчение, так чтобы вы могли перенести.
\rsbpar\vs 1Co 10:14 Итак, возлюбленные мои, убегайте идолослужения.
\vs 1Co 10:15 Я говорю \bibemph{вам} как рассудительным; сами рассуд\acc{и}те о том, что говорю.
\vs 1Co 10:16 Чаша благословения, которую благословляем, не есть ли приобщение Крови Христовой? Хлеб, который преломляем, не есть ли приобщение Тела Христова?
\vs 1Co 10:17 Один хлеб, и мы многие одно тело; ибо все причащаемся от одного хлеба.
\vs 1Co 10:18 Посмотрите на Израиля по плоти: те, которые едят жертвы, не участники ли жертвенника?
\vs 1Co 10:19 Что же я говорю? То ли, что идол есть что-нибудь, или идоложертвенное значит что-нибудь?
\vs 1Co 10:20 \bibemph{Нет}, но что язычники, принося жертвы, приносят бесам, а не Богу. Но я не хочу, чтобы вы были в общении с бесами.
\vs 1Co 10:21 Не можете пить чашу Господню и чашу бесовскую; не можете быть участниками в трапезе Господней и в трапезе бесовской.
\vs 1Co 10:22 Неужели мы \bibemph{решимся} раздражать Господа? Разве мы сильнее Его?
\rsbpar\vs 1Co 10:23 Все мне позволительно, но не все полезно; все мне позволительно, но не все назидает.
\vs 1Co 10:24 Никто не ищи своего, но каждый \bibemph{пользы} другого.
\vs 1Co 10:25 Все, что продается на торгу, ешьте без всякого исследования, для \bibemph{спокойствия} совести;
\vs 1Co 10:26 ибо Господня земля, и чт\acc{о} наполняет ее.
\vs 1Co 10:27 Если кто из неверных позовет вас, и вы захотите пойти, то все, предлагаемое вам, ешьте без всякого исследования, для \bibemph{спокойствия} совести.
\vs 1Co 10:28 Но если кто скажет вам: это идоложертвенное,~--- то не ешьте ради того, кто объявил вам, и ради совести. Ибо Господня земля, и чт\acc{о} наполняет ее.
\vs 1Co 10:29 Совесть же разумею не свою, а другого: ибо для чего моей свободе быть судимой чужою совестью?
\vs 1Co 10:30 Если я с благодарением принимаю \bibemph{пищу}, то для чего порицать меня за то, за что я благодарю?
\vs 1Co 10:31 Итак, едите ли, пьете ли, или иное что делаете, все делайте в славу Божию.
\vs 1Co 10:32 Не подавайте соблазна ни Иудеям, ни Еллинам, ни церкви Божией,
\vs 1Co 10:33 так, как и я угождаю всем во всем, ища не своей пользы, но \bibemph{пользы} многих, чтобы они спаслись.
\vs 1Co 11:1 Будьте подражателями мне, как я Христу.
\rsbpar\vs 1Co 11:2 Хвалю вас, братия, что вы все мое помните и держите предания так, как я передал вам.
\vs 1Co 11:3 Хочу также, чтобы вы знали, что всякому мужу глава Христос, жене глава~--- муж, а Христу глава~--- Бог.
\vs 1Co 11:4 Всякий муж, молящийся или пророчествующий с покрытою головою, постыжает свою голову.
\vs 1Co 11:5 И всякая жена, молящаяся или пророчествующая с открытою головою, постыжает свою голову, ибо \bibemph{это} то же, как если бы она была обритая.
\vs 1Co 11:6 Ибо если жена не хочет покрываться, то пусть и стрижется; а если жене стыдно быть остриженной или обритой, пусть покрывается.
\vs 1Co 11:7 Итак муж не должен покрывать голову, потому что он есть образ и слава Божия; а жена есть слава мужа.
\vs 1Co 11:8 Ибо не муж от жены, но жена от мужа;
\vs 1Co 11:9 и не муж создан для жены, но жена для мужа.
\vs 1Co 11:10 Посему жена и должна иметь на голове своей \bibemph{знак} власти \bibemph{над нею}, для Ангелов.
\vs 1Co 11:11 Впрочем ни муж без жены, ни жена без мужа, в Господе.
\vs 1Co 11:12 Ибо как жена от мужа, так и муж через жену; все же~--- от Бога.
\vs 1Co 11:13 Рассудите сами, прилично ли жене молиться Богу с непокрытою \bibemph{головою}?
\vs 1Co 11:14 Не сама ли природа учит вас, что если муж растит волосы, то это бесчестье для него,
\vs 1Co 11:15 но если жена растит волосы, для нее это честь, так как волосы даны ей вместо покрывала?
\vs 1Co 11:16 А если бы кто захотел спорить, то мы не имеем такого обычая, ни церкви Божии.
\rsbpar\vs 1Co 11:17 Но, предлагая сие, не хвалю \bibemph{вас}, что вы собираетесь не на лучшее, а на худшее.
\vs 1Co 11:18 Ибо, во-первых, слышу, что, когда вы собираетесь в церковь, между вами бывают разделения, чему отчасти и верю.
\vs 1Co 11:19 Ибо надлежит быть и разномыслиям между вами, дабы открылись между вами искусные.
\vs 1Co 11:20 Далее, вы собираетесь, \bibemph{так, что это} не значит вкушать вечерю Господню;
\vs 1Co 11:21 ибо всякий поспешает прежде \bibemph{других} есть свою пищу, \bibemph{так что} иной бывает голоден, а иной упивается.
\vs 1Co 11:22 Разве у вас нет домов на то, чтобы есть и пить? Или пренебрегаете церковь Божию и унижаете неимущих? Чт\acc{о} сказать вам? похвалить ли вас за это? Не похвалю.
\vs 1Co 11:23 Ибо я от \bibemph{Самого} Господа принял т\acc{о}, что и вам передал, что Господь Иисус в ту ночь, в которую предан был, взял хлеб
\vs 1Co 11:24 и, возблагодарив, преломил и сказал: приимите, ядите, сие есть Тело Мое, за вас ломимое; сие творите в Мое воспоминание.
\vs 1Co 11:25 Также и чашу после вечери, и сказал: сия чаша есть новый завет в Моей Крови; сие творите, когда только будете пить, в Мое воспоминание.
\vs 1Co 11:26 Ибо всякий раз, когда вы едите хлеб сей и пьете чашу сию, смерть Господню возвещаете, доколе Он придет.
\vs 1Co 11:27 Посему, кто будет есть хлеб сей или пить чашу Господню недостойно, виновен будет против Тела и Крови Господней.
\vs 1Co 11:28 Да испытывает же себя человек, и таким образом пусть ест от хлеба сего и пьет из чаши сей.
\vs 1Co 11:29 Ибо, кто ест и пьет недостойно, тот ест и пьет осуждение себе, не рассуждая о Теле Господнем.
\vs 1Co 11:30 Оттого многие из вас немощны и больны и немало умирает.
\vs 1Co 11:31 Ибо если бы мы судили сами себя, то не были бы судимы.
\vs 1Co 11:32 Будучи же судимы, наказываемся от Господа, чтобы не быть осужденными с миром.
\vs 1Co 11:33 Посему, братия мои, собираясь на вечерю, друг друга ждите.
\vs 1Co 11:34 А если кто голоден, пусть ест дома, чтобы собираться вам не на осуждение. Прочее устрою, когда приду.
\vs 1Co 12:1 Не хочу оставить вас, братия, в неведении и о \bibemph{дарах} духовных.
\vs 1Co 12:2 Знаете, что когда вы были язычниками, то ходили к безгласным идолам, так, как бы вели вас.
\vs 1Co 12:3 Потому сказываю вам, что никто, говорящий Духом Божиим, не произнесет анафемы на Иисуса, и никто не может назвать Иисуса Господом, как только Духом Святым.
\vs 1Co 12:4 Дары различны, но Дух один и тот же;
\vs 1Co 12:5 и служения различны, а Господь один и тот же;
\vs 1Co 12:6 и действия различны, а Бог один и тот же, производящий все во всех.
\vs 1Co 12:7 Но каждому дается проявление Духа на пользу.
\vs 1Co 12:8 Одному дается Духом слово мудрости, другому слово знания, тем же Духом;
\vs 1Co 12:9 иному вера, тем же Духом; иному дары исцелений, тем же Духом;
\vs 1Co 12:10 иному чудотворения, иному пророчество, иному различение духов, иному разные языки, иному истолкование языков.
\vs 1Co 12:11 Все же сие производит один и тот же Дух, разделяя каждому особо, как Ему угодно.
\vs 1Co 12:12 Ибо, как тело одно, но имеет многие члены, и все члены одного тела, хотя их и много, составляют одно тело,~--- так и Христос.
\vs 1Co 12:13 Ибо все мы одним Духом крестились в одно тело, Иудеи или Еллины, рабы или свободные, и все напоены одним Духом.
\vs 1Co 12:14 Тело же не из одного члена, но из многих.
\vs 1Co 12:15 Если нога скажет: я не принадлежу к телу, потому что я не рука, то неужели она потому не принадлежит к телу?
\vs 1Co 12:16 И если ухо скажет: я не принадлежу к телу, потому что я не глаз, то неужели оно потому не принадлежит к телу?
\vs 1Co 12:17 Если все тело глаз, то где слух? Если все слух, то где обоняние?
\vs 1Co 12:18 Но Бог расположил члены, каждый в \bibemph{составе} тела, как Ему было угодно.
\vs 1Co 12:19 А если бы все были один член, то где \bibemph{было бы} тело?
\vs 1Co 12:20 Но теперь членов много, а тело одно.
\vs 1Co 12:21 Не может глаз сказать руке: ты мне не надобна; или также голова ногам: вы мне не нужны.
\vs 1Co 12:22 Напротив, члены тела, которые кажутся слабейшими, гораздо нужнее,
\vs 1Co 12:23 и которые нам кажутся менее благородными в теле, о тех более прилагаем попечения;
\vs 1Co 12:24 и неблагообразные наши более благовидно покрываются, а благообразные наши не имеют \bibemph{в том} нужды. Но Бог соразмерил тело, внушив о менее совершенном большее попечение,
\vs 1Co 12:25 дабы не было разделения в теле, а все члены одинаково заботились друг о друге.
\vs 1Co 12:26 Посему, страдает ли один член, страдают с ним все члены; славится ли один член, с ним радуются все члены.
\vs 1Co 12:27 И вы~--- тело Христово, а порознь~--- члены.
\vs 1Co 12:28 И иных Бог поставил в Церкви, во-первых, Апостолами, во-вторых, пророками, в-третьих, учителями; далее, \bibemph{иным дал} силы \bibemph{чудодейственные}, также дары исцелений, вспоможения, управления, разные языки.
\vs 1Co 12:29 Все ли Апостолы? Все ли пророки? Все ли учители? Все ли чудотворцы?
\vs 1Co 12:30 Все ли имеют дары исцелений? Все ли говорят языками? Все ли истолкователи?
\vs 1Co 12:31 Ревнуйте о дарах б\acc{о}льших, и я покажу вам путь еще превосходнейший.
\vs 1Co 13:1 Если я говорю языками человеческими и ангельскими, а любви не имею, то я~--- медь звенящая или кимвал звучащий.
\vs 1Co 13:2 Если имею \bibemph{дар} пророчества, и знаю все тайны, и имею всякое познание и всю веру, так что \bibemph{могу} и горы переставлять, а не имею любви,~--- то я ничто.
\vs 1Co 13:3 И если я раздам все имение мое и отдам тело мое на сожжение, а любви не имею, нет мне в том никакой пользы.
\vs 1Co 13:4 Любовь долготерпит, милосердствует, любовь не завидует, любовь не превозносится, не гордится,
\vs 1Co 13:5 не бесчинствует, не ищет своего, не раздражается, не мыслит зла,
\vs 1Co 13:6 не радуется неправде, а сорадуется истине;
\vs 1Co 13:7 все покрывает, всему верит, всего надеется, все переносит.
\vs 1Co 13:8 Любовь никогда не перестает, хотя и пророчества прекратятся, и языки умолкнут, и знание упразднится.
\vs 1Co 13:9 Ибо мы отчасти знаем, и отчасти пророчествуем;
\vs 1Co 13:10 когда же настанет совершенное, тогда то, что отчасти, прекратится.
\vs 1Co 13:11 Когда я был младенцем, то по-младенчески говорил, по-младенчески мыслил, по-младенчески рассуждал; а как стал мужем, то оставил младенческое.
\vs 1Co 13:12 Теперь мы видим как бы сквозь \bibemph{тусклое} стекло, гадательно, тогда же лицем к лицу; теперь знаю я отчасти, а тогда позн\acc{а}ю, подобно как я познан.
\vs 1Co 13:13 А теперь пребывают сии три: вера, надежда, любовь; но любовь из них больше.
\vs 1Co 14:1 Достигайте любви; ревнуйте о \bibemph{дарах} духовных, особенно же о том, чтобы пророчествовать.
\vs 1Co 14:2 Ибо кто говорит на \bibemph{незнакомом} языке, тот говорит не людям, а Богу; потому что никто не понимает \bibemph{его}, он тайны говорит духом;
\vs 1Co 14:3 а кто пророчествует, тот говорит людям в назидание, увещание и утешение.
\vs 1Co 14:4 Кто говорит на \bibemph{незнакомом} языке, тот назидает себя; а кто пророчествует, тот назидает церковь.
\vs 1Co 14:5 Желаю, чтобы вы все говорили языками; но лучше, чтобы вы пророчествовали; ибо пророчествующий превосходнее того, кто говорит языками, разве он притом будет и изъяснять, чтобы церковь получила назидание.
\vs 1Co 14:6 Теперь, если я приду к вам, братия, и стану говорить на \bibemph{незнакомых} языках, то какую принесу вам пользу, когда не изъяснюсь вам или откровением, или познанием, или пророчеством, или учением?
\vs 1Co 14:7 И бездушные \bibemph{вещи}, издающие звук, свирель или гусли, если не производят раздельных тонов, как распознать т\acc{о}, чт\acc{о} играют на свирели или на гуслях?
\vs 1Co 14:8 И если труба будет издавать неопределенный звук, кто станет готовиться к сражению?
\vs 1Co 14:9 Так если и вы языком произносите невразумительные слова, то как узн\acc{а}ют, чт\acc{о} вы говорите? Вы будете говорить на ветер.
\vs 1Co 14:10 Сколько, например, различных слов в мире, и ни одного из них нет без значения.
\vs 1Co 14:11 Но если я не разумею значения слов, то я для говорящего чужестранец, и говорящий для меня чужестранец.
\vs 1Co 14:12 Так и вы, ревнуя о \bibemph{дарах} духовных, старайтесь обогатиться \bibemph{ими} к назиданию церкви.
\vs 1Co 14:13 А потому, говорящий на \bibemph{незнакомом} языке, молись о даре истолкования.
\vs 1Co 14:14 Ибо когда я молюсь на \bibemph{незнакомом} языке, то хотя дух мой и молится, но ум мой остается без плода.
\vs 1Co 14:15 Что же делать? Стану молиться духом, стану молиться и умом; буду петь духом, буду петь и умом.
\vs 1Co 14:16 Ибо если ты будешь благословлять духом, то стоящий на месте простолюдина к\acc{а}к скажет: <<аминь>> при твоем благодарении? Ибо он не понимает, чт\acc{о} ты говоришь.
\vs 1Co 14:17 Ты хорошо благодаришь, но другой не назидается.
\vs 1Co 14:18 Благодарю Бога моего: я более всех вас говорю языками;
\vs 1Co 14:19 но в церкви хочу лучше пять слов сказать умом моим, чтобы и других наставить, нежели тьму слов на \bibemph{незнакомом} языке.
\rsbpar\vs 1Co 14:20 Братия! не будьте дети умом: на злое будьте младенцы, а по уму будьте совершеннолетни.
\vs 1Co 14:21 В законе написано: иными языками и иными устами буду говорить народу сему; но и тогда не послушают Меня, говорит Господь.
\vs 1Co 14:22 Итак языки суть знамение не для верующих, а для неверующих; пророчество же не для неверующих, а для верующих.
\vs 1Co 14:23 Если вся церковь сойдется вместе, и все станут говорить \bibemph{незнакомыми} языками, и войдут к вам незнающие или неверующие, то не скажут ли, что вы беснуетесь?
\vs 1Co 14:24 Но когда все пророчествуют, и войдет кто неверующий или незнающий, то он всеми обличается, всеми судится.
\vs 1Co 14:25 И таким образом тайны сердца его обнаруживаются, и он падет ниц, поклонится Богу и скажет: истинно с вами Бог.
\rsbpar\vs 1Co 14:26 Итак чт\acc{о} же, братия? Когда вы сходитесь, и у каждого из вас есть псалом, есть поучение, есть язык, есть откровение, есть истолкование,~--- все сие да будет к назиданию.
\vs 1Co 14:27 Если кто говорит на \bibemph{незнакомом} языке, \bibemph{говорите} двое, или много трое, и т\acc{о} порознь, а один изъясняй.
\vs 1Co 14:28 Если же не будет истолкователя, то молчи в церкви, а говори себе и Богу.
\vs 1Co 14:29 И пророки пусть говорят двое или трое, а прочие пусть рассуждают.
\vs 1Co 14:30 Если же другому из сидящих будет откровение, то первый молчи.
\vs 1Co 14:31 Ибо все один за другим можете пророчествовать, чтобы всем поучаться и всем получать утешение.
\vs 1Co 14:32 И духи пророческие послушны пророкам,
\vs 1Co 14:33 потому что Бог не есть \bibemph{Бог} неустройства, но мира. Т\acc{а}к \bibemph{бывает} во всех церквах у святых.
\vs 1Co 14:34 Жены ваши в церквах да молчат, ибо не позволено им говорить, а быть в подчинении, как и закон говорит.
\vs 1Co 14:35 Если же они хотят чему научиться, пусть спрашивают \bibemph{о том} дома у мужей своих; ибо неприлично жене говорить в церкви.
\vs 1Co 14:36 Разве от вас вышло слово Божие? Или до вас одних достигло?
\rsbpar\vs 1Co 14:37 Если кто почитает себя пророком или духовным, тот да разумеет, чт\acc{о} я пишу вам, ибо это заповеди Господни.
\vs 1Co 14:38 А кто не разумеет, пусть не разумеет.
\vs 1Co 14:39 Итак, братия, ревнуйте о том, чтобы пророчествовать, но не запрещайте говорить и языками;
\vs 1Co 14:40 только всё должно быть благопристойно и чинно.
\vs 1Co 15:1 Напоминаю вам, братия, Евангелие, которое я благовествовал вам, которое вы и приняли, в котором и утвердились,
\vs 1Co 15:2 которым и спасаетесь, если преподанное удерживаете так, как я благовествовал вам, если только не тщетно уверовали.
\vs 1Co 15:3 Ибо я первоначально преподал вам, что и \bibemph{сам} принял, \bibemph{то есть}, что Христос умер за грехи наши, по Писанию,
\vs 1Co 15:4 и что Он погребен был, и что воскрес в третий день, по Писанию,
\vs 1Co 15:5 и что явился Кифе, потом двенадцати;
\vs 1Co 15:6 потом явился более нежели пятистам братий в одно время, из которых б\acc{о}льшая часть доныне в живых, а некоторые и почили;
\vs 1Co 15:7 потом явился Иакову, также всем Апостолам;
\vs 1Co 15:8 а после всех явился и мне, как некоему извергу.
\vs 1Co 15:9 Ибо я наименьший из Апостолов, и недостоин называться Апостолом, потому что гнал церковь Божию.
\vs 1Co 15:10 Но благодатию Божиею есмь то, что есмь; и благодать Его во мне не была тщетна, но я более всех их потрудился: не я, впрочем, а благодать Божия, которая со мною.
\vs 1Co 15:11 Итак я ли, они ли, мы так проповедуем, и вы так уверовали.
\rsbpar\vs 1Co 15:12 Если же о Христе проповедуется, что Он воскрес из мертвых, то к\acc{а}к некоторые из вас говорят, что нет воскресения мертвых?
\vs 1Co 15:13 Если нет воскресения мертвых, то и Христос не воскрес;
\vs 1Co 15:14 а если Христос не воскрес, то и проповедь наша тщетна, тщетна и вера ваша.
\vs 1Co 15:15 Притом мы оказались бы и лжесвидетелями о Боге, потому что свидетельствовали бы о Боге, что Он воскресил Христа, Которого Он не воскрешал, если, \bibemph{то есть}, мертвые не воскресают;
\vs 1Co 15:16 ибо если мертвые не воскресают, то и Христос не воскрес.
\vs 1Co 15:17 А если Христос не воскрес, то вера ваша тщетна: вы еще во грехах ваших.
\vs 1Co 15:18 Поэтому и умершие во Христе погибли.
\vs 1Co 15:19 И если мы в этой только жизни надеемся на Христа, то мы несчастнее всех человеков.
\vs 1Co 15:20 Но Христос воскрес из мертвых, первенец из умерших.
\vs 1Co 15:21 Ибо, как смерть через человека, \bibemph{так} через человека и воскресение мертвых.
\vs 1Co 15:22 Как в Адаме все умирают, так во Христе все оживут,
\vs 1Co 15:23 каждый в своем порядке: первенец Христос, потом Христовы, в пришествие Его.
\vs 1Co 15:24 А затем конец, когда Он предаст Царство Богу и Отцу, когда упразднит всякое начальство и всякую власть и силу.
\vs 1Co 15:25 Ибо Ему надлежит царствовать, доколе низложит всех врагов под ноги Свои.
\vs 1Co 15:26 Последний же враг истребится~--- смерть,
\vs 1Co 15:27 потому что все покорил под ноги Его. Когда же сказано, что \bibemph{Ему} все покорено, то ясно, что кроме Того, Который покорил Ему все.
\vs 1Co 15:28 Когда же все покорит Ему, тогда и Сам Сын покорится Покорившему все Ему, да будет Бог все во всем.
\vs 1Co 15:29 Иначе, что делают крестящиеся для мертвых? Если мертвые совсем не воскресают, то для чего и крестятся для мертвых?
\vs 1Co 15:30 Для чего и мы ежечасно подвергаемся бедствиям?
\vs 1Co 15:31 Я каждый день умираю: свидетельствуюсь в том похвалою вашею, братия, которую я имею во Христе Иисусе, Господе нашем.
\vs 1Co 15:32 По \bibemph{рассуждению} человеческому, когда я боролся со зверями в Ефесе, какая мне польза, если мертвые не воскресают? Станем есть и пить, ибо завтра умрем!
\vs 1Co 15:33 Не обманывайтесь: худые сообщества развращают добрые нравы.
\vs 1Co 15:34 Отрезвитесь, как должно, и не грешите; ибо, к стыду вашему скажу, некоторые из вас не знают Бога.
\rsbpar\vs 1Co 15:35 Но скажет кто-нибудь: как воскреснут мертвые? и в каком теле придут?
\vs 1Co 15:36 Безрассудный! то, что ты сеешь, не оживет, если не умрет.
\vs 1Co 15:37 И когда ты сеешь, то сеешь не тело будущее, а голое зерно, какое случится, пшеничное или другое какое;
\vs 1Co 15:38 но Бог дает ему тело, как хочет, и каждому семени свое тело.
\vs 1Co 15:39 Не всякая плоть такая же плоть; но иная плоть у человеков, иная плоть у скотов, иная у рыб, иная у птиц.
\vs 1Co 15:40 Есть тела небесные и тела земные; но иная слава небесных, иная земных.
\vs 1Co 15:41 Иная слава солнца, иная слава луны, иная звезд; и звезда от звезды разнится в славе.
\vs 1Co 15:42 Так и при воскресении мертвых: сеется в тлении, восстает в нетлении;
\vs 1Co 15:43 сеется в уничижении, восстает в славе; сеется в немощи, восстает в силе;
\vs 1Co 15:44 сеется тело душевное, восстает тело духовное. Есть тело душевное, есть тело и духовное.
\vs 1Co 15:45 Так и написано: первый человек Адам стал душею живущею; а последний Адам есть дух животворящий.
\vs 1Co 15:46 Но не духовное прежде, а душевное, потом духовное.
\vs 1Co 15:47 Первый человек~--- из земли, перстный; второй человек~--- Господь с неба.
\vs 1Co 15:48 Каков перстный, таковы и перстные; и каков небесный, таковы и небесные.
\vs 1Co 15:49 И как мы носили образ перстного, будем носить и образ небесного.
\rsbpar\vs 1Co 15:50 Но то скажу \bibemph{вам}, братия, что плоть и кровь не могут наследовать Царствия Божия, и тление не наследует нетления.
\vs 1Co 15:51 Говорю вам тайну: не все мы умрем, но все изменимся
\vs 1Co 15:52 вдруг, во мгновение ока, при последней трубе; ибо вострубит, и мертвые воскреснут нетленными, а мы изменимся.
\vs 1Co 15:53 Ибо тленному сему надлежит облечься в нетление, и смертному сему облечься в бессмертие.
\vs 1Co 15:54 Когда же тленное сие облечется в нетление и смертное сие облечется в бессмертие, тогда сбудется слово написанное: поглощена смерть победою.
\vs 1Co 15:55 Смерть! где твое жало? ад! где твоя победа?
\vs 1Co 15:56 Жало же смерти~--- грех; а сила греха~--- закон.
\vs 1Co 15:57 Благодарение Богу, даровавшему нам победу Господом нашим Иисусом Христом!
\vs 1Co 15:58 Итак, братия мои возлюбленные, будьте тверды, непоколебимы, всегда преуспевайте в деле Господнем, зная, что труд ваш не тщетен пред Господом.
\vs 1Co 16:1 При сборе же для святых поступайте так, как я установил в церквах Галатийских.
\vs 1Co 16:2 В первый день недели каждый из вас пусть отлагает у себя и сберегает, сколько позволит ему состояние, чтобы не делать сборов, когда я приду.
\vs 1Co 16:3 Когда же приду, то, которых вы изберете, тех отправлю с письмами, для доставления вашего подаяния в Иерусалим.
\vs 1Co 16:4 А если прилично будет и мне отправиться, то они со мной пойдут.
\rsbpar\vs 1Co 16:5 Я приду к вам, когда пройду Македонию; ибо я иду через Македонию.
\vs 1Co 16:6 У вас же, может быть, поживу, или и перезимую, чтобы вы меня проводили, куда пойду.
\vs 1Co 16:7 Ибо я не хочу видеться с вами теперь мимоходом, а надеюсь пробыть у вас несколько времени, если Господь позволит.
\vs 1Co 16:8 В Ефесе же я пробуду до Пятидесятницы,
\vs 1Co 16:9 ибо для меня отверста великая и широкая дверь, и противников много.
\rsbpar\vs 1Co 16:10 Если же придет к вам Тимофей, смотр\acc{и}те, чтобы он был у вас безопасен; ибо он делает дело Господне, как и я.
\vs 1Co 16:11 Посему никто не пренебрегай его, но провод\acc{и}те его с миром, чтобы он пришел ко мне, ибо я жду его с братиями.
\vs 1Co 16:12 А что до брата Аполлоса, я очень просил его, чтобы он с братиями пошел к вам; но он никак не хотел идти ныне, а придет, когда ему будет удобно.
\rsbpar\vs 1Co 16:13 Бодрствуйте, стойте в вере, будьте мужественны, тверды.
\vs 1Co 16:14 Все у вас да будет с любовью.
\rsbpar\vs 1Co 16:15 Прошу вас, братия (вы знаете семейство Стефаново, что оно есть начаток Ахаии и что они посвятили себя на служение святым),
\vs 1Co 16:16 будьте и вы почтительны к таковым и ко всякому содействующему и трудящемуся.
\vs 1Co 16:17 Я рад прибытию Стефана, Фортуната и Ахаика: они восполнили для меня отсутствие ваше,
\vs 1Co 16:18 ибо они мой и ваш дух успокоили. Почитайте таковых.
\rsbpar\vs 1Co 16:19 Приветствуют вас церкви Асийские; приветствуют вас усердно в Господе Акила и Прискилла с домашнею их церковью.
\vs 1Co 16:20 Приветствуют вас все братия. Приветствуйте друг друга святым целованием.
\rsbpar\vs 1Co 16:21 Мое, Павлово, приветствие собственноручно.
\vs 1Co 16:22 Кто не любит Господа Иисуса Христа, анафема, мар\acc{а}н-аф\acc{а}\fns{Да будет отлучен до пришествия Господа.}.
\vs 1Co 16:23 Благодать Господа нашего Иисуса Христа с вами,
\vs 1Co 16:24 и любовь моя со всеми вами во Христе Иисусе. Аминь.

\bibbookdescr{2Co}{
  inline={Второе Послание\\к Коринфянам\\\LARGE Святого Апостола Павла},
  toc={2-е Коринфянам},
  bookmark={2-е Коринфянам},
  header={2-е Коринфянам},
  %headerleft={},
  %headerright={},
  abbr={2~Кор}
}
\vs 2Co 1:1 Павел, волею Божиею Апостол Иисуса Христа, и Тимофей брат, церкви Божией, находящейся в Коринфе, со всеми святыми по всей Ахаии:
\vs 2Co 1:2 благодать вам и мир от Бога Отца нашего и Господа Иисуса Христа.
\rsbpar\vs 2Co 1:3 Благословен Бог и Отец Господа нашего Иисуса Христа, Отец милосердия и Бог всякого утешения,
\vs 2Co 1:4 утешающий нас во всякой скорби нашей, чтобы и мы могли утешать находящихся во всякой скорби тем утешением, которым Бог утешает нас самих!
\vs 2Co 1:5 Ибо по мере, как умножаются в нас страдания Христовы, умножается Христом и утешение наше.
\vs 2Co 1:6 Скорбим ли мы, \bibemph{скорбим} для вашего утешения и спасения, которое совершается перенесением тех же страданий, какие и мы терпим.
\vs 2Co 1:7 И надежда наша о вас тверда. Утешаемся ли, \bibemph{утешаемся} для вашего утешения и спасения, зная, что вы участвуете как в страданиях наших, так и в утешении.
\rsbpar\vs 2Co 1:8 Ибо мы не хотим оставить вас, братия, в неведении о скорби нашей, бывшей с нами в Асии, потому что мы отягчены были чрезмерно и сверх силы, так что не надеялись остаться в живых.
\vs 2Co 1:9 Но сами в себе имели приговор к смерти, для того, чтобы надеяться не на самих себя, но на Бога, воскрешающего мертвых,
\vs 2Co 1:10 Который и избавил нас от столь \bibemph{близкой} смерти, и избавляет, и на Которого надеемся, что и еще избавит,
\vs 2Co 1:11 при содействии и вашей молитвы за нас, дабы за дарованное нам, по ходатайству многих, многие возблагодарили за нас.
\rsbpar\vs 2Co 1:12 Ибо похвала наша сия есть свидетельство совести нашей, что мы в простоте и богоугодной искренности, не по плотской мудрости, но по благодати Божией, жили в мире, особенно же у вас.
\vs 2Co 1:13 И мы пишем вам не иное, как то, что вы читаете или разумеете, и что, как надеюсь, до конца уразумеете,
\vs 2Co 1:14 так как вы отчасти и уразумели уже, что мы будем вашею похвалою, равно и вы нашею, в день Господа нашего Иисуса Христа.
\vs 2Co 1:15 И в этой уверенности я намеревался прийти к вам ранее, чтобы вы вторично получили благодать,
\vs 2Co 1:16 и через вас пройти в Македонию, из Македонии же опять прийти к вам; а вы проводили бы меня в Иудею.
\vs 2Co 1:17 Имея такое намерение, легкомысленно ли я поступил? Или, чт\acc{о} я предпринимаю, по плоти предпринимаю, так что у меня то <<да, да>>, то <<нет, нет>>?
\vs 2Co 1:18 Верен Бог, что слово наше к вам не было то <<да>>, то <<нет>>.
\vs 2Co 1:19 Ибо Сын Божий, Иисус Христос, проповеданный у вас нами, мною и Силуаном и Тимофеем, не был <<да>> и <<нет>>; но в Нем было <<да>>,~---
\vs 2Co 1:20 ибо все обетования Божии в Нем <<да>> и в Нем <<аминь>>,~--- в славу Божию, через нас.
\vs 2Co 1:21 Утверждающий же нас с вами во Христе и помазавший нас \bibemph{есть} Бог,
\vs 2Co 1:22 Который и запечатлел нас и дал залог Духа в сердца наши.
\rsbpar\vs 2Co 1:23 Бога призываю во свидетели на душу мою, что, щадя вас, я доселе не приходил в Коринф,
\vs 2Co 1:24 не потому, будто мы берем власть над верою вашею; но мы споспешествуем радости вашей: ибо верою вы тверды.
\vs 2Co 2:1 Итак я рассудил сам в себе не приходить к вам опять с огорчением.
\vs 2Co 2:2 Ибо если я огорчаю вас, то кто обрадует меня, как не тот, кто огорчен мною?
\vs 2Co 2:3 Это самое и писал я вам, дабы, придя, не иметь огорчения от тех, о которых мне надлежало радоваться: ибо я во всех вас уверен, что моя радость есть \bibemph{радость} и для всех вас.
\vs 2Co 2:4 От великой скорби и стесненного сердца я писал вам со многими слезами, не для того, чтобы огорчить вас, но чтобы вы познали любовь, какую я в избытке имею к вам.
\vs 2Co 2:5 Если же кто огорчил, то не меня огорчил, но частью,~--- чтобы не сказать много,~--- и всех вас.
\vs 2Co 2:6 Для такого довольно сего наказания от многих,
\vs 2Co 2:7 так что вам лучше уже простить его и утешить, дабы он не был поглощен чрезмерною печалью.
\vs 2Co 2:8 И потому прошу вас оказать ему любовь.
\vs 2Co 2:9 Ибо я для того и писал, чтобы узнать на опыте, во всем ли вы послушны.
\vs 2Co 2:10 А кого вы в чем прощаете, того и я; ибо и я, если в чем простил кого, простил для вас от лица Христова,
\vs 2Co 2:11 чтобы не сделал нам ущерба сатана, ибо нам не безызвестны его умыслы.
\rsbpar\vs 2Co 2:12 Придя в Троаду для благовествования о Христе, хотя мне и отверста была дверь Господом,
\vs 2Co 2:13 я не имел покоя духу моему, потому что не нашел \bibemph{там} брата моего Тита; но, простившись с ними, я пошел в Македонию.
\vs 2Co 2:14 Но благодарение Богу, Который всегда дает нам торжествовать во Христе и благоухание познания о Себе распространяет нами во всяком месте.
\vs 2Co 2:15 Ибо мы Христово благоухание Богу в спасаемых и в погибающих:
\vs 2Co 2:16 для одних запах смертоносный на смерть, а для других запах живительный на жизнь. И кто способен к сему?
\vs 2Co 2:17 Ибо мы не повреждаем слова Божия, как многие, но проповедуем искренно, как от Бога, пред Богом, во Христе.
\vs 2Co 3:1 Неужели нам снова знакомиться с вами? Неужели нужны для нас, как для некоторых, одобрительные письма к вам или от вас?
\vs 2Co 3:2 Вы~--- наше письмо, написанное в сердцах наших, узнаваемое и читаемое всеми человеками;
\vs 2Co 3:3 вы показываете собою, что вы~--- письмо Христово, через служение наше написанное не чернилами, но Духом Бога живаго, не на скрижалях каменных, но на плотяных скрижалях сердца.
\vs 2Co 3:4 Такую уверенность мы имеем в Боге через Христа,
\vs 2Co 3:5 не потому, чтобы мы сами способны были помыслить чт\acc{о} от себя, как бы от себя, но способность наша от Бога.
\vs 2Co 3:6 Он дал нам способность быть служителями Нового Завета, не буквы, но духа, потому что буква убивает, а дух животворит.
\vs 2Co 3:7 Если же служение смертоносным буквам, начертанное на камнях, было так славно, что сыны Израилевы не могли смотреть на лице Моисеево по причине славы лица его преходящей,~---
\vs 2Co 3:8 то не гораздо ли более должно быть славно служение духа?
\vs 2Co 3:9 Ибо если служение осуждения славно, то тем паче изобилует славою служение оправдания.
\vs 2Co 3:10 То прославленное даже не оказывается славным с сей стороны, по причине преимущественной славы \bibemph{последующего}.
\vs 2Co 3:11 Ибо, если преходящее славно, тем более славно пребывающее.
\vs 2Co 3:12 Имея такую надежду, мы действуем с великим дерзновением,
\vs 2Co 3:13 а не так, как Моисей, \bibemph{который} полагал покрывало на лице свое, чтобы сыны Израилевы не взирали на конец преходящего.
\vs 2Co 3:14 Но умы их ослеплены: ибо то же самое покрывало доныне остается неснятым при чтении Ветхого Завета, потому что оно снимается Христом.
\vs 2Co 3:15 Доныне, когда они читают Моисея, покрывало лежит на сердце их;
\vs 2Co 3:16 но когда обращаются к Господу, тогда это покрывало снимается.
\vs 2Co 3:17 Господь есть Дух; а где Дух Господень, там свобода.
\vs 2Co 3:18 Мы же все открытым лицем, как в зеркале, взирая на славу Господню, преображаемся в тот же образ от славы в славу, как от Господня Духа.
\vs 2Co 4:1 Посему, имея по милости \bibemph{Божией} такое служение, мы не унываем;
\vs 2Co 4:2 но, отвергнув скрытные постыдные \bibemph{дела}, не прибегая к хитрости и не искажая слова Божия, а открывая истину, представляем себя совести всякого человека пред Богом.
\vs 2Co 4:3 Если же и закрыто благовествование наше, то закрыто для погибающих,
\vs 2Co 4:4 для неверующих, у которых бог века сего ослепил умы, чтобы для них не воссиял свет благовествования о славе Христа, Который есть образ Бога невидимого.
\vs 2Co 4:5 Ибо мы не себя проповедуем, но Христа Иисуса, Господа; а мы~--- рабы ваши для Иисуса,
\vs 2Co 4:6 потому что Бог, повелевший из тьмы воссиять свету, озарил наши сердца, дабы просветить \bibemph{нас} познанием славы Божией в лице Иисуса Христа.
\rsbpar\vs 2Co 4:7 Но сокровище сие мы носим в глиняных сосудах, чтобы преизбыточная сила была \bibemph{приписываема} Богу, а не нам.
\vs 2Co 4:8 Мы отовсюду притесняемы, но не стеснены; мы в отчаянных обстоятельствах, но не отчаиваемся;
\vs 2Co 4:9 мы гонимы, но не оставлены; низлагаемы, но не погибаем.
\vs 2Co 4:10 Всегда носим в теле мертвость Господа Иисуса, чтобы и жизнь Иисусова открылась в теле нашем.
\vs 2Co 4:11 Ибо мы живые непрестанно предаемся на смерть ради Иисуса, чтобы и жизнь Иисусова открылась в смертной плоти нашей,
\vs 2Co 4:12 так что смерть действует в нас, а жизнь в вас.
\vs 2Co 4:13 Но, имея тот же дух веры, как написано: я веровал и потому говорил, и мы веруем, потому и говорим,
\vs 2Co 4:14 зная, что Воскресивший Господа Иисуса воскресит через Иисуса и нас и поставит перед \bibemph{Собою} с вами.
\vs 2Co 4:15 Ибо всё для вас, дабы обилие благодати тем б\acc{о}льшую во многих произвело благодарность во славу Божию.
\vs 2Co 4:16 Посему мы не унываем; но если внешний наш человек и тлеет, то внутренний со дня на день обновляется.
\vs 2Co 4:17 Ибо кратковременное легкое страдание наше производит в безмерном преизбытке вечную славу,
\vs 2Co 4:18 когда мы смотрим не на видимое, но на невидимое: ибо видимое временно, а невидимое вечно.
\vs 2Co 5:1 Ибо знаем, что, когда земной наш дом, эта хижина, разрушится, мы имеем от Бога жилище на небесах, дом нерукотворенный, вечный.
\vs 2Co 5:2 Оттого мы и воздыхаем, желая облечься в небесное наше жилище;
\vs 2Co 5:3 только бы нам и одетым не оказаться нагими.
\vs 2Co 5:4 Ибо мы, находясь в этой хижине, воздыхаем под бременем, потому что не хотим совлечься, но облечься, чтобы смертное поглощено было жизнью.
\vs 2Co 5:5 На сие самое и создал нас Бог и дал нам залог Духа.
\vs 2Co 5:6 Итак мы всегда благодушествуем; и как знаем, что, водворяясь в теле, мы устранены от Господа,~---
\vs 2Co 5:7 ибо мы ходим верою, а не в\acc{и}дением,~---
\vs 2Co 5:8 то мы благодушествуем и желаем лучше выйти из тела и водвориться у Господа.
\vs 2Co 5:9 И потому ревностно стараемся, водворяясь ли, выходя ли, быть Ему угодными;
\vs 2Co 5:10 ибо всем нам должно явиться пред судилище Христово, чтобы каждому получить \bibemph{соответственно тому}, чт\acc{о} он делал, живя в теле, доброе или худое.
\rsbpar\vs 2Co 5:11 Итак, зная страх Господень, мы вразумляем людей, Богу же мы открыты; надеюсь, что открыты и вашим совестям.
\vs 2Co 5:12 Не снова представляем себя вам, но даем вам повод хвалиться нами, дабы имели вы \bibemph{чт\acc{о} сказать} тем, которые хвалятся лицем, а не сердцем.
\vs 2Co 5:13 Если мы выходим из себя, то для Бога; если же скромны, то для вас.
\vs 2Co 5:14 Ибо любовь Христова объемлет нас, рассуждающих так: если один умер за всех, то все умерли.
\vs 2Co 5:15 А Христос за всех умер, чтобы живущие уже не для себя жили, но для умершего за них и воскресшего.
\vs 2Co 5:16 Потому отныне мы никого не знаем по плоти; если же и знали Христа по плоти, то ныне уже не знаем.
\vs 2Co 5:17 Итак, кто во Христе, \bibemph{тот} новая тварь; древнее прошло, теперь все новое.
\vs 2Co 5:18 Все же от Бога, Иисусом Христом примирившего нас с Собою и давшего нам служение примирения,
\vs 2Co 5:19 потому что Бог во Христе примирил с Собою мир, не вменяя \bibemph{людям} преступлений их, и дал нам слово примирения.
\vs 2Co 5:20 Итак мы~--- посланники от имени Христова, и как бы Сам Бог увещевает через нас; от имени Христова просим: примиритесь с Богом.
\vs 2Co 5:21 Ибо не знавшего греха Он сделал для нас \bibemph{жертвою за} грех, чтобы мы в Нем сделались праведными пред Богом.
\vs 2Co 6:1 Мы же, как споспешники, умоляем вас, чтобы благодать Божия не тщетно была принята вами.
\vs 2Co 6:2 Ибо сказано: во время благоприятное Я услышал тебя и в день спасения помог тебе. Вот, теперь время благоприятное, вот, теперь день спасения.
\vs 2Co 6:3 Мы никому ни в чем не полагаем претыкания, чтобы не было порицаемо служение,
\vs 2Co 6:4 но во всем являем себя, как служители Божии, в великом терпении, в бедствиях, в нуждах, в тесных обстоятельствах,
\vs 2Co 6:5 под ударами, в темницах, в изгнаниях, в трудах, в бдениях, в постах,
\vs 2Co 6:6 в чистоте, в благоразумии, в великодушии, в благости, в Духе Святом, в нелицемерной любви,
\vs 2Co 6:7 в слове истины, в силе Божией, с оружием правды в правой и левой руке,
\vs 2Co 6:8 в чести и бесчестии, при порицаниях и похвалах: нас почитают обманщиками, но мы верны;
\vs 2Co 6:9 мы неизвестны, но нас узнают; нас почитают умершими, но вот, мы живы; нас наказывают, но мы не умираем;
\vs 2Co 6:10 нас огорчают, а мы всегда радуемся; мы нищи, но многих обогащаем; мы ничего не имеем, но всем обладаем.
\rsbpar\vs 2Co 6:11 Уста наши отверсты к вам, Коринфяне, сердце наше расширено.
\vs 2Co 6:12 Вам не тесно в нас; но в сердцах ваших тесно.
\vs 2Co 6:13 В равное возмездие,~--- говорю, как детям,~--- распространитесь и вы.
\rsbpar\vs 2Co 6:14 Не преклоняйтесь под чужое ярмо с неверными, ибо какое общение праведности с беззаконием? Что общего у света с тьмою?
\vs 2Co 6:15 Какое согласие между Христом и Велиаром? Или какое соучастие верного с неверным?
\vs 2Co 6:16 Какая совместность храма Божия с идолами? Ибо вы храм Бога живаго, как сказал Бог: вселюсь в них и буду ходить \bibemph{в них}; и буду их Богом, и они будут Моим народом.
\vs 2Co 6:17 И потому выйдите из среды их и отделитесь, говорит Господь, и не прикасайтесь к нечистому; и Я прииму вас.
\vs 2Co 6:18 И буду вам Отцем, и вы будете Моими сынами и дщерями, говорит Господь Вседержитель.
\vs 2Co 7:1 Итак, возлюбленные, имея такие обетования, очистим себя от всякой скверны плоти и духа, совершая святыню в страхе Божием.
\rsbpar\vs 2Co 7:2 Вместите нас. Мы никого не обидели, никому не повредили, ни от кого не искали корысти.
\vs 2Co 7:3 Не в осуждение говорю; ибо я прежде сказал, что вы в сердцах наших, так чтобы вместе и умереть и жить.
\vs 2Co 7:4 Я много надеюсь на вас, много хвалюсь вами; я исполнен утешением, преизобилую радостью, при всей скорби нашей.
\vs 2Co 7:5 Ибо, когда пришли мы в Македонию, плоть наша не имела никакого покоя, но мы были стеснены отовсюду: отвне~--- нападения, внутри~--- страхи.
\vs 2Co 7:6 Но Бог, утешающий смиренных, утешил нас прибытием Тита,
\vs 2Co 7:7 и не только прибытием его, но и утешением, которым он утешался о вас, пересказывая нам о вашем усердии, о вашем плаче, о вашей ревности по мне, так что я еще более обрадовался.
\vs 2Co 7:8 Посему, если я опечалил вас посланием, не жалею, хотя и пожалел было; ибо вижу, что послание т\acc{о} опечалило вас, впрочем на время.
\vs 2Co 7:9 Теперь я радуюсь не потому, что вы опечалились, но что вы опечалились к покаянию; ибо опечалились ради Бога, так что нисколько не понесли от нас вреда.
\vs 2Co 7:10 Ибо печаль ради Бога производит неизменное покаяние ко спасению, а печаль мирская производит смерть.
\vs 2Co 7:11 Ибо то самое, что вы опечалились ради Бога, смотр\acc{и}те, какое произвело в вас усердие, какие извинения, какое негодование \bibemph{на виновного}, какой страх, какое желание, какую ревность, какое взыскание! По всему вы показали себя чистыми в этом деле.
\vs 2Co 7:12 Итак, если я писал к вам, то не ради оскорбителя и не ради оскорбленного, но чтобы вам открылось попечение наше о вас пред Богом.
\vs 2Co 7:13 Посему мы утешились утешением вашим; а еще более обрадованы мы радостью Тита, что вы все успокоили дух его.
\vs 2Co 7:14 Итак я не остался в стыде, если чем-либо о вас похвалился перед ним, но как вам мы говорили все истину, так и перед Титом похвала наша оказалась истинною;
\vs 2Co 7:15 и сердце его весьма расположено к вам, при воспоминании о послушании всех вас, как вы приняли его со страхом и трепетом.
\vs 2Co 7:16 Итак радуюсь, что во всем могу положиться на вас.
\vs 2Co 8:1 Уведомляем вас, братия, о благодати Божией, данной церквам Македонским,
\vs 2Co 8:2 ибо они среди великого испытания скорбями преизобилуют радостью; и глубокая нищета их преизбыточествует в богатстве их радушия.
\vs 2Co 8:3 Ибо они доброхотны по силам и сверх сил~--- я свидетель:
\vs 2Co 8:4 они весьма убедительно просили нас принять дар и участие \bibemph{их} в служении святым;
\vs 2Co 8:5 и не только то, чего мы надеялись, но они отдали самих себя, во-первых, Господу, \bibemph{потом} и нам по воле Божией;
\vs 2Co 8:6 поэтому мы просили Тита, чтобы он, как начал, так и окончил у вас и это доброе дело.
\rsbpar\vs 2Co 8:7 А к\acc{а}к вы изобилуете всем: верою и словом, и познанием, и всяким усердием, и любовью вашею к нам,~--- т\acc{а}к изобилуйте и сею добродетелью.
\vs 2Co 8:8 Говорю это не в виде повеления, но усердием других испытываю искренность и вашей любви.
\vs 2Co 8:9 Ибо вы знаете благодать Господа нашего Иисуса Христа, что Он, будучи богат, обнищал ради вас, дабы вы обогатились Его нищетою.
\vs 2Co 8:10 Я даю на это совет: ибо это полезно вам, которые не только начали делать сие, но и желали того еще с прошедшего года.
\vs 2Co 8:11 Совершите же теперь самое дело, дабы, чего усердно желали, то и исполнено было по достатку.
\vs 2Co 8:12 Ибо если есть усердие, то оно принимается смотря по тому, кто что имеет, а не по тому, чего не имеет.
\vs 2Co 8:13 Не \bibemph{требуется}, чтобы другим \bibemph{было} облегчение, а вам тяжесть, но чтобы была равномерность.
\vs 2Co 8:14 Ныне ваш избыток в \bibemph{восполнение} их недостатка; а после их избыток в \bibemph{восполнение} вашего недостатка, чтобы была равномерность,
\vs 2Co 8:15 как написано: кто собрал много, не имел лишнего; и кто мало, не имел недостатка.
\rsbpar\vs 2Co 8:16 Благодарение Богу, вложившему в сердце Титово такое усердие к вам.
\vs 2Co 8:17 Ибо, хотя и я просил его, впрочем он, будучи очень усерден, пошел к вам добровольно.
\vs 2Co 8:18 С ним послали мы также брата, во всех церквах похваляемого за благовествование,
\vs 2Co 8:19 и притом избранного от церквей сопутствовать нам для сего благотворения, которому мы служим во славу Самого Господа и \bibemph{в соответствие} вашему усердию,
\vs 2Co 8:20 остерегаясь, чтобы нам не подвергнуться от кого нареканию при таком обилии приношений, вверяемых нашему служению;
\vs 2Co 8:21 ибо мы стараемся о добром не только пред Господом, но и пред людьми.
\vs 2Co 8:22 Мы послали с ними и брата нашего, которого усердие много раз испытали во многом и который ныне еще усерднее по великой уверенности в вас.
\vs 2Co 8:23 Что касается до Тита, это~--- мой товарищ и сотрудник у вас; а что до братьев наших, это~--- посланники церквей, слава Христова.
\vs 2Co 8:24 Итак перед лицем церквей дайте им доказательство любви вашей и того, что мы \bibemph{справедливо} хвалимся вами.
\vs 2Co 9:1 Для меня впрочем излишне писать вам о вспоможении святым,
\vs 2Co 9:2 ибо я знаю усердие ваше и хвалюсь вами перед Македонянами, что Ахаия приготовлена еще с прошедшего года; и ревность ваша поощрила многих.
\vs 2Co 9:3 Братьев же послал я для того, чтобы похвала моя о вас не оказалась тщетною в сем случае, но чтобы вы, как я говорил, были приготовлены,
\vs 2Co 9:4 \bibemph{и} чтобы, когда придут со мною Македоняне и найдут вас неготовыми, не остались в стыде мы,~--- не говорю <<вы>>,~--- похвалившись с такою уверенностью.
\vs 2Co 9:5 Посему я почел за нужное упросить братьев, чтобы они наперед пошли к вам и предварительно озаботились, дабы возвещенное уже благословение ваше было готово, как благословение, а не как побор.
\rsbpar\vs 2Co 9:6 При сем скажу: кто сеет скупо, тот скупо и пожнет; а кто сеет щедро, тот щедро и пожнет.
\vs 2Co 9:7 Каждый \bibemph{уделяй} по расположению сердца, не с огорчением и не с принуждением; ибо доброхотно дающего любит Бог.
\vs 2Co 9:8 Бог же силен обогатить вас всякою благодатью, чтобы вы, всегда и во всем имея всякое довольство, были богаты на всякое доброе дело,
\vs 2Co 9:9 как написано: расточил, раздал нищим; правда его пребывает в век.
\vs 2Co 9:10 Дающий же семя сеющему и хлеб в пищу подаст обилие посеянному вами и умножит плоды правды вашей,
\vs 2Co 9:11 так чтобы вы всем богаты были на всякую щедрость, которая через нас производит благодарение Богу.
\vs 2Co 9:12 Ибо дело служения сего не только восполняет скудость святых, но и производит во многих обильные благодарения Богу;
\vs 2Co 9:13 ибо, видя опыт сего служения, они прославляют Бога за покорность исповедуемому вами Евангелию Христову и за искреннее общение с ними и со всеми,
\vs 2Co 9:14 молясь за вас, по расположению к вам, за преизбыточествующую в вас благодать Божию.
\vs 2Co 9:15 Благодарение Богу за неизреченный дар Его!
\vs 2Co 10:1 Я же, Павел, который лично между вами скромен, а заочно против вас отважен, убеждаю вас кротостью и снисхождением Христовым.
\vs 2Co 10:2 Прошу, чтобы мне по пришествии моем не прибегать к той твердой смелости, которую думаю употребить против некоторых, помышляющих о нас, что мы поступаем по плоти.
\vs 2Co 10:3 Ибо мы, ходя во плоти, не по плоти воинствуем.
\vs 2Co 10:4 Оружия воинствования нашего не плотские, но сильные Богом на разрушение твердынь: \bibemph{ими} ниспровергаем замыслы
\vs 2Co 10:5 и всякое превозношение, восстающее против познания Божия, и пленяем всякое помышление в послушание Христу,
\vs 2Co 10:6 и готовы наказать всякое непослушание, когда ваше послушание исполнится.
\vs 2Co 10:7 На личность ли см\acc{о}трите? Кто уверен в себе, что он Христов, тот сам по себе суди, что, как он Христов, так и мы Христовы.
\vs 2Co 10:8 Ибо если бы я и более стал хвалиться нашею властью, которую Господь дал нам к созиданию, а не к расстройству вашему, то не остался бы в стыде.
\vs 2Co 10:9 Впрочем, да не покажется, что я устрашаю вас \bibemph{только} посланиями.
\vs 2Co 10:10 Так как \bibemph{некто} говорит: в посланиях он строг и силен, а в личном присутствии слаб, и речь \bibemph{его} незначительна,~---
\vs 2Co 10:11 такой пусть знает, что, каковы мы на словах в посланиях заочно, таковы и на деле лично.
\vs 2Co 10:12 Ибо мы не смеем сопоставлять или сравнивать себя с теми, которые сами себя выставляют: они измеряют себя самими собою и сравнивают себя с собою неразумно.
\vs 2Co 10:13 А мы не без меры хвалиться будем, но по мере удела, какой назначил нам Бог в такую меру, чтобы достигнуть и до вас.
\vs 2Co 10:14 Ибо мы не напрягаем себя, как не достигшие до вас, потому что достигли и до вас благовествованием Христовым.
\vs 2Co 10:15 Мы не без меры хвалимся, не чужими трудами, но надеемся, с возрастанием веры вашей, с избытком увеличить в вас удел наш,
\vs 2Co 10:16 так чтобы и далее вас проповедовать Евангелие, а не хвалиться готовым в чужом уделе.
\vs 2Co 10:17 Хвалящийся хвались о Господе.
\vs 2Co 10:18 Ибо не тот достоин, кто сам себя хвалит, но кого хвалит Господь.
\vs 2Co 11:1 О, если бы вы несколько были снисходительны к моему неразумию! Но вы и снисходите ко мне.
\vs 2Co 11:2 Ибо я ревную о вас ревностью Божиею; потому что я обручил вас единому мужу, чтобы представить Христу чистою девою.
\vs 2Co 11:3 Но боюсь, чтобы, как змий хитростью своею прельстил Еву, так и ваши умы не повредились, \bibemph{уклонившись} от простоты во Христе.
\vs 2Co 11:4 Ибо если бы кто, придя, начал проповедовать другого Иисуса, которого мы не проповедовали, или если бы вы получили иного Духа, которого не получили, или иное благовестие, которого не принимали,~--- то вы были бы очень снисходительны \bibemph{к тому}.
\vs 2Co 11:5 Но я думаю, что у меня ни в чем нет недостатка против высших Апостолов:
\vs 2Co 11:6 хотя я и невежда в слове, но не в познании. Впрочем мы во всем совершенно известны вам.
\vs 2Co 11:7 Согрешил ли я тем, что унижал себя, чтобы возвысить вас, потому что безмездно проповедовал вам Евангелие Божие?
\vs 2Co 11:8 Другим церквам я причинял издержки, получая \bibemph{от них} содержание для служения вам; и, будучи у вас, хотя терпел недостаток, никому не докучал,
\vs 2Co 11:9 ибо недостаток мой восполнили братия, пришедшие из Македонии; да и во всем я старался и постараюсь не быть вам в тягость.
\vs 2Co 11:10 По истине Христовой во мне \bibemph{скажу}, что похвала сия не отнимется у меня в странах Ахаии.
\vs 2Co 11:11 Почему же \bibemph{так поступаю}? Потому ли, что не люблю вас? Богу известно! Но как поступаю, так и буду поступать,
\vs 2Co 11:12 чтобы не дать повода ищущим повода, дабы они, чем хвалятся, в том оказались \bibemph{такими же}, как и мы.
\vs 2Co 11:13 Ибо таковые лжеапостолы, лукавые делатели, принимают вид Апостолов Христовых.
\vs 2Co 11:14 И неудивительно: потому что сам сатана принимает вид Ангела света,
\vs 2Co 11:15 а потому не великое дело, если и служители его принимают вид служителей правды; но конец их будет по делам их.
\rsbpar\vs 2Co 11:16 Еще скажу: не почти кто-нибудь меня неразумным; а если не так, то примите меня, хотя как неразумного, чтобы и мне сколько-нибудь похвалиться.
\vs 2Co 11:17 Чт\acc{о} скажу, т\acc{о} скажу не в Господе, но как бы в неразумии при такой отважности на похвалу.
\vs 2Co 11:18 Как многие хвалятся по плоти, то и я буду хвалиться.
\vs 2Co 11:19 Ибо вы, люди разумные, охотно терпите неразумных:
\vs 2Co 11:20 вы терпите, когда кто вас порабощает, когда кто объедает, когда кто обирает, когда кто превозносится, когда кто бьет вас в лицо.
\vs 2Co 11:21 К стыду говорю, что \bibemph{на это} у нас недоставало сил. А если кто смеет \bibemph{хвалиться} чем-либо, то (скажу по неразумию) смею и я.
\vs 2Co 11:22 Они Евреи? и я. Израильтяне? и я. Семя Авраамово? и я.
\vs 2Co 11:23 Христовы служители? (в безумии говорю:) я больше. Я гораздо более \bibemph{был} в трудах, безмерно в ранах, более в темницах и многократно при смерти.
\vs 2Co 11:24 От Иудеев пять раз дано мне было по сорока \bibemph{ударов} без одного;
\vs 2Co 11:25 три раза меня били палками, однажды камнями побивали, три раза я терпел кораблекрушение, ночь и день пробыл во глубине \bibemph{морской};
\vs 2Co 11:26 много раз \bibemph{был} в путешествиях, в опасностях на реках, в опасностях от разбойников, в опасностях от единоплеменников, в опасностях от язычников, в опасностях в городе, в опасностях в пустыне, в опасностях на море, в опасностях между лжебратиями,
\vs 2Co 11:27 в труде и в изнурении, часто в бдении, в голоде и жажде, часто в посте, на стуже и в наготе.
\vs 2Co 11:28 Кроме посторонних \bibemph{приключений}, у меня ежедневно стечение \bibemph{людей}, забота о всех церквах.
\vs 2Co 11:29 Кто изнемогает, с кем бы и я не изнемогал? Кто соблазняется, за кого бы я не воспламенялся?
\vs 2Co 11:30 Если должно мне хвалиться, то буду хвалиться немощью моею.
\vs 2Co 11:31 Бог и Отец Господа нашего Иисуса Христа, благословенный во веки, знает, что я не лгу.
\vs 2Co 11:32 В Дамаске областной правитель царя Ареты стерег город Дамаск, чтобы схватить меня; и я в корзине был спущен из окна по стене и избежал его рук.
\vs 2Co 12:1 Не полезно хвалиться мне, ибо я приду к видениям и откровениям Господним.
\vs 2Co 12:2 Знаю человека во Христе, который назад тому четырнадцать лет (в теле ли~--- не знаю, вне ли тела~--- не знаю: Бог знает) восхищен был до третьего неба.
\vs 2Co 12:3 И знаю о таком человеке (\bibemph{только} не знаю~--- в теле, или вне тела: Бог знает),
\vs 2Co 12:4 что он был восхищен в рай и слышал неизреченные слова, которых человеку нельзя пересказать.
\vs 2Co 12:5 Таким \bibemph{человеком} могу хвалиться; собою же не похвалюсь, разве только немощами моими.
\vs 2Co 12:6 Впрочем, если захочу хвалиться, не буду неразумен, потому что скажу истину; но я удерживаюсь, чтобы кто не подумал о мне более, нежели сколько во мне видит или слышит от меня.
\vs 2Co 12:7 И чтобы я не превозносился чрезвычайностью откровений, дано мне жало в плоть, ангел сатаны, удручать меня, чтобы я не превозносился.
\vs 2Co 12:8 Трижды молил я Господа о том, чтобы удалил его от меня.
\vs 2Co 12:9 Но \bibemph{Господь} сказал мне: <<довольно для тебя благодати Моей, ибо сила Моя совершается в немощи>>. И потому я гораздо охотнее буду хвалиться своими немощами, чтобы обитала во мне сила Христова.
\vs 2Co 12:10 Посему я благодушествую в немощах, в обидах, в нуждах, в гонениях, в притеснениях за Христа, ибо, когда я немощен, тогда силен.
\rsbpar\vs 2Co 12:11 Я дошел до неразумия, хвалясь; вы меня \bibemph{к сему} принудили. Вам бы надлежало хвалить меня, ибо у меня ни в чем нет недостатка против высших Апостолов, хотя я и ничто.
\vs 2Co 12:12 Признаки Апостола оказались перед вами всяким терпением, знамениями, чудесами и силами.
\vs 2Co 12:13 Ибо чего у вас недостает перед прочими церквами, разве только того, что сам я не был вам в тягость? Простите мне такую вину.
\vs 2Co 12:14 Вот, в третий раз я готов идти к вам, и не буду отягощать вас, ибо я ищу не вашего, а вас. Не дети должны собирать имение для родителей, но родители для детей.
\vs 2Co 12:15 Я охотно буду издерживать \bibemph{свое} и истощать себя за души ваши, несмотря на то, что, чрезвычайно любя вас, я менее любим вами.
\vs 2Co 12:16 Положим, \bibemph{что} сам я не обременял вас, но, будучи хитр, лукавством брал с вас.
\vs 2Co 12:17 Но пользовался ли я \bibemph{чем} от вас через кого-нибудь из тех, кого посылал к вам?
\vs 2Co 12:18 Я упросил Тита и послал с ним одного из братьев: Тит воспользовался ли чем от вас? Не в одном ли духе мы действовали? Не одним ли путем ходили?
\rsbpar\vs 2Co 12:19 Не думаете ли еще, что мы \bibemph{только} оправдываемся перед вами? Мы говорим пред Богом, во Христе, и все это, возлюбленные, к вашему назиданию.
\vs 2Co 12:20 Ибо я опасаюсь, чтобы мне, по пришествии моем, не найти вас такими, какими не желаю, также чтобы и вам не найти меня таким, каким не желаете: чтобы \bibemph{не найти у вас} раздоров, зависти, гнева, ссор, клевет, ябед, гордости, беспорядков,
\vs 2Co 12:21 чтобы опять, когда приду, не уничижил меня у вас Бог мой и \bibemph{чтобы} не оплакивать мне многих, которые согрешили прежде и не покаялись в нечистоте, блудодеянии и непотребстве, какое делали.
\vs 2Co 13:1 В третий уже раз иду к вам. При устах двух или трех свидетелей будет твердо всякое слово.
\vs 2Co 13:2 Я предварял и предваряю, как бы находясь \bibemph{у вас} во второй раз, и теперь, отсутствуя, пишу прежде согрешившим и всем прочим, что, когда опять приду, не пощажу.
\vs 2Co 13:3 Вы ищете доказательства на то, Христос ли говорит во мне: Он не бессилен для вас, но силен в вас.
\vs 2Co 13:4 Ибо, хотя Он и распят в немощи, но жив силою Божиею; и мы также, \bibemph{хотя} немощны в Нем, но будем живы с Ним силою Божиею в вас.
\rsbpar\vs 2Co 13:5 Испытывайте самих себя, в вере ли вы; самих себя исследуйте. Или вы не знаете самих себя, что Иисус Христос в вас? Разве только вы не т\acc{о}, чем должны быть.
\vs 2Co 13:6 О нас же, надеюсь, узн\acc{а}ете, что мы т\acc{о}, чем быть должны.
\vs 2Co 13:7 Молим Бога, чтобы вы не делали никакого зла, не для того, чтобы нам показаться, чем должны быть; но чтобы вы делали добро, хотя бы мы казались и не тем, чем должны быть.
\vs 2Co 13:8 Ибо мы не сильны против истины, но сильны за истину.
\vs 2Co 13:9 Мы радуемся, когда мы немощны, а вы сильны; о сем-то и молимся, о вашем совершенстве.
\vs 2Co 13:10 Для того я и пишу сие в отсутствии, чтобы в присутствии не употребить строгости по власти, данной мне Господом к созиданию, а не к разорению.
\rsbpar\vs 2Co 13:11 Впрочем, братия, радуйтесь, усовершайтесь, утешайтесь, будьте единомысленны, мирны,~--- и Бог любви и мира будет с вами.
\vs 2Co 13:12 Приветствуйте друг друга лобзанием святым. Приветствуют вас все святые.
\rsbpar\vs 2Co 13:13 Благодать Господа нашего Иисуса Христа, и любовь Бога Отца, и общение Святаго Духа со всеми вами. Аминь.

\bibbookdescr{Gal}{
  inline={Послание к Галатам\\\LARGE Святого Апостола Павла},
  toc={к Галатам},
  bookmark={к Галатам},
  header={к Галатам},
  %headerleft={},
  %headerright={},
  abbr={Гал}
}
\vs Gal 1:1 Павел Апостол, \bibemph{избранный} не человеками и не через человека, но Иисусом Христом и Богом Отцем, воскресившим Его из мертвых,
\vs Gal 1:2 и все находящиеся со мною братия~--- церквам Галатийским:
\vs Gal 1:3 благодать вам и мир от Бога Отца и Господа нашего Иисуса Христа,
\vs Gal 1:4 Который отдал Себя Самого за грехи наши, чтобы избавить нас от настоящего лукавого века, по воле Бога и Отца нашего;
\vs Gal 1:5 Ему слава во веки веков. Аминь.
\rsbpar\vs Gal 1:6 Удивляюсь, что вы от призвавшего вас благодатью Христовою так скоро переходите к иному благовествованию,
\vs Gal 1:7 которое \bibemph{впрочем} не иное, а только есть люди, смущающие вас и желающие превратить благовествование Христово.
\vs Gal 1:8 Но если бы даже мы или Ангел с неба стал благовествовать вам не то, чт\acc{о} мы благовествовали вам, да будет анафема.
\vs Gal 1:9 Как прежде мы сказали, \bibemph{так} и теперь еще говорю: кто благовествует вам не то, чт\acc{о} вы приняли, да будет анафема.
\vs Gal 1:10 У людей ли я ныне ищу благоволения, или у Бога? людям ли угождать стараюсь? Если бы я и поныне угождал людям, то не был бы рабом Христовым.
\rsbpar\vs Gal 1:11 Возвещаю вам, братия, что Евангелие, которое я благовествовал, не есть человеческое,
\vs Gal 1:12 ибо и я принял его и научился не от человека, но через откровение Иисуса Христа.
\vs Gal 1:13 Вы слышали о моем прежнем образе жизни в Иудействе, что я жестоко гнал Церковь Божию, и опустошал ее,
\vs Gal 1:14 и преуспевал в Иудействе более многих сверстников в роде моем, будучи неумеренным ревнителем отеческих моих преданий.
\vs Gal 1:15 Когда же Бог, избравший меня от утробы матери моей и призвавший благодатью Своею, благоволил
\vs Gal 1:16 открыть во мне Сына Своего, чтобы я благовествовал Его язычникам,~--- я не стал тогда же советоваться с плотью и кровью,
\vs Gal 1:17 и не пошел в Иерусалим к предшествовавшим мне Апостолам, а пошел в Аравию, и опять возвратился в Дамаск.
\vs Gal 1:18 Потом, спустя три года, ходил я в Иерусалим видеться с Петром и пробыл у него дней пятнадцать.
\vs Gal 1:19 Другого же из Апостолов я не видел \bibemph{никого}, кроме Иакова, брата Господня.
\vs Gal 1:20 А в том, чт\acc{о} пишу вам, пред Богом, не лгу.
\vs Gal 1:21 После сего отошел я в страны Сирии и Киликии.
\vs Gal 1:22 Церквам Христовым в Иудее лично я не был известен,
\vs Gal 1:23 а только слышали они, что гнавший их некогда ныне благовествует веру, которую прежде истреблял,~---
\vs Gal 1:24 и прославляли за меня Бога.
\vs Gal 2:1 Потом, через четырнадцать лет, опять ходил я в Иерусалим с Варнавою, взяв с собою и Тита.
\vs Gal 2:2 Ходил же по откровению, и предложил там, и особо знаменитейшим, благовествование, проповедуемое мною язычникам, не напрасно ли я подвизаюсь или подвизался.
\vs Gal 2:3 Но они и Тита, бывшего со мною, хотя и Еллина, не принуждали обрезаться,
\vs Gal 2:4 а вкравшимся лжебратиям, скрытно приходившим подсмотреть за нашею свободою, которую мы имеем во Христе Иисусе, чтобы поработить нас,
\vs Gal 2:5 мы ни на час не уступили и не покорились, дабы истина благовествования сохранилась у вас.
\vs Gal 2:6 И в знаменитых чем-либо, какими бы ни были они когда-либо, для меня нет ничего особенного: Бог не взирает на лице человека. И знаменитые не возложили на меня ничего более.
\vs Gal 2:7 Напротив того, увидев, что мне вверено благовестие для необрезанных, как Петру для обрезанных
\vs Gal 2:8 (ибо Содействовавший Петру в апостольстве у обрезанных содействовал и мне у язычников),
\vs Gal 2:9 и, узнав о благодати, данной мне, Иаков и Кифа и Иоанн, почитаемые столпами, подали мне и Варнаве руку общения, чтобы нам \bibemph{идти} к язычникам, а им к обрезанным,
\vs Gal 2:10 только чтобы мы помнили нищих, что и старался я исполнять в точности.
\rsbpar\vs Gal 2:11 Когда же Петр пришел в Антиохию, то я лично противостал ему, потому что он подвергался нареканию.
\vs Gal 2:12 Ибо, до прибытия некоторых от Иакова, ел вместе с язычниками; а когда те пришли, стал таиться и устраняться, опасаясь обрезанных.
\vs Gal 2:13 Вместе с ним лицемерили и прочие Иудеи, так что даже Варнава был увлечен их лицемерием.
\vs Gal 2:14 Но когда я увидел, что они не прямо поступают по истине Евангельской, то сказал Петру при всех: если ты, будучи Иудеем, живешь по-язычески, а не по-иудейски, то для чего язычников принуждаешь жить по-иудейски?
\vs Gal 2:15 Мы по природе Иудеи, а не из язычников грешники;
\vs Gal 2:16 однако же, узнав, что человек оправдывается не делами закона, а только верою в Иисуса Христа, и мы уверовали во Христа Иисуса, чтобы оправдаться верою во Христа, а не делами закона; ибо делами закона не оправдается никакая плоть.
\vs Gal 2:17 Если же, ища оправдания во Христе, мы и сами оказались грешниками, то неужели Христос есть служитель греха? Никак.
\vs Gal 2:18 Ибо если я снова созидаю, что разрушил, то сам себя делаю преступником.
\vs Gal 2:19 Законом я умер для закона, чтобы жить для Бога. Я сораспялся Христу,
\vs Gal 2:20 и уже не я живу, но живет во мне Христос. А что ныне живу во плоти, то живу верою в Сына Божия, возлюбившего меня и предавшего Себя за меня.
\vs Gal 2:21 Не отвергаю благодати Божией; а если законом оправдание, то Христос напрасно умер.
\vs Gal 3:1 О, несмысленные Галаты! кто прельстил вас не покоряться истине, \bibemph{вас}, у которых перед глазами предначертан был Иисус Христос, \bibemph{как бы} у вас распятый?
\vs Gal 3:2 Сие только хочу знать от вас: через дела ли закона вы получили Духа, или через наставление в вере?
\vs Gal 3:3 Так ли вы несмысленны, что, начав духом, теперь оканчиваете плотью?
\vs Gal 3:4 Столь многое потерпели вы неужели без пользы? О, если бы только без пользы!
\vs Gal 3:5 Подающий вам Духа и совершающий между вами чудеса через дела ли закона \bibemph{сие производит}, или через наставление в вере?
\vs Gal 3:6 Так Авраам поверил Богу, и это вменилось ему в праведность.
\vs Gal 3:7 Познайте же, что верующие суть сыны Авраама.
\vs Gal 3:8 И Писание, провидя, что Бог верою оправдает язычников, предвозвестило Аврааму: в тебе благословятся все народы.
\vs Gal 3:9 Итак верующие благословляются с верным Авраамом,
\vs Gal 3:10 а все, утверждающиеся на делах закона, находятся под клятвою. Ибо написано: проклят всяк, кто не исполняет постоянно всего, что написано в книге закона.
\vs Gal 3:11 А что законом никто не оправдывается пред Богом, это ясно, потому что праведный верою жив будет.
\vs Gal 3:12 А закон не по вере; но кто исполняет его, тот жив будет им.
\vs Gal 3:13 Христос искупил нас от клятвы закона, сделавшись за нас клятвою (ибо написано: проклят всяк, висящий на древе),
\vs Gal 3:14 дабы благословение Авраамово через Христа Иисуса распространилось на язычников, чтобы нам получить обещанного Духа верою.
\rsbpar\vs Gal 3:15 Братия! говорю по \bibemph{рассуждению} человеческому: даже человеком утвержденного завещания никто не отменяет и не прибавляет \bibemph{к нему}.
\vs Gal 3:16 Но Аврааму даны были обетования и семени его. Не сказано: и потомкам, как бы о многих, но как об одном: и семени твоему, которое есть Христос.
\vs Gal 3:17 Я говорю то, что завета о Христе, прежде Богом утвержденного, закон, явившийся спустя четыреста тридцать лет, не отменяет т\acc{а}к, чтобы обетование потеряло силу.
\vs Gal 3:18 Ибо если по закону наследство, то уже не по обетованию; но Аврааму Бог даровал \bibemph{оное} по обетованию.
\rsbpar\vs Gal 3:19 Для чего же закон? Он дан после по причине преступлений, до времени пришествия семени, к которому \bibemph{относится} обетование, и преподан через Ангелов, рукою посредника.
\vs Gal 3:20 Но посредник при одном не бывает, а Бог один.
\vs Gal 3:21 Итак закон противен обетованиям Божиим? Никак! Ибо если бы дан был закон, могущий животворить, то подлинно праведность была бы от закона;
\vs Gal 3:22 но Писание всех заключило под грехом, дабы обетование верующим дано было по вере в Иисуса Христа.
\vs Gal 3:23 А до пришествия веры мы заключены были под стражею закона, до того \bibemph{времени}, как надлежало открыться вере.
\vs Gal 3:24 Итак закон был для нас детоводителем ко Христу, дабы нам оправдаться верою;
\vs Gal 3:25 по пришествии же веры, мы уже не под \bibemph{руководством} детоводителя.
\vs Gal 3:26 Ибо все вы сыны Божии по вере во Христа Иисуса;
\vs Gal 3:27 все вы, во Христа крестившиеся, во Христа облеклись.
\vs Gal 3:28 Нет уже Иудея, ни язычника; нет раба, ни свободного; нет мужеского пола, ни женского: ибо все вы одно во Христе Иисусе.
\vs Gal 3:29 Если же вы Христовы, то вы семя Авраамово и по обетованию наследники.
\vs Gal 4:1 Еще скажу: наследник, доколе в детстве, ничем не отличается от раба, хотя и господин всего:
\vs Gal 4:2 он подчинен попечителям и домоправителям до срока, отцом \bibemph{назначенного}.
\vs Gal 4:3 Так и мы, доколе были в детстве, были порабощены вещественным началам мира;
\vs Gal 4:4 но когда пришла полнота времени, Бог послал Сына Своего (Единородного), Который родился от жены, подчинился закону,
\vs Gal 4:5 чтобы искупить подзаконных, дабы нам получить усыновление.
\vs Gal 4:6 А как вы~--- сыны, то Бог послал в сердца ваши Духа Сына Своего, вопиющего: <<Авва, Отче!>>
\vs Gal 4:7 Посему ты уже не раб, но сын; а если сын, то и наследник Божий через Иисуса Христа.
\vs Gal 4:8 Но тогда, не знав Бога, вы служили \bibemph{богам}, которые в существе не боги.
\vs Gal 4:9 Ныне же, познав Бога, или, лучше, получив познание от Бога, для чего возвращаетесь опять к немощным и бедным вещественным началам и хотите еще снова поработить себя им?
\vs Gal 4:10 Наблюдаете дни, месяцы, времена и годы.
\vs Gal 4:11 Боюсь за вас, не напрасно ли я трудился у вас.
\rsbpar\vs Gal 4:12 Прошу вас, братия, будьте, как я, потому что и я, как вы. Вы ничем не обидели меня:
\vs Gal 4:13 знаете, что, \bibemph{хотя} я в немощи плоти благовествовал вам в первый раз,
\vs Gal 4:14 но вы не презрели искушения моего во плоти моей и не возгнушались \bibemph{им}, а приняли меня, как Ангела Божия, как Христа Иисуса.
\vs Gal 4:15 Как вы были блаженны! Свидетельствую о вас, что, если бы возможно было, вы исторгли бы очи свои и отдали мне.
\vs Gal 4:16 Итак, неужели я сделался врагом вашим, говоря вам истину?
\vs Gal 4:17 Ревнуют по вас нечисто, а хотят вас отлучить, чтобы вы ревновали по них.
\vs Gal 4:18 Хорошо ревновать в добром всегда, а не в моем только присутствии у вас.
\vs Gal 4:19 Дети мои, для которых я снова в м\acc{у}ках рождения, доколе не изобразится в вас Христос!
\vs Gal 4:20 Хотел бы я теперь быть у вас и изменить голос мой, потому что я в недоумении о вас.
\rsbpar\vs Gal 4:21 Скажите мне вы, желающие быть под законом: разве вы не слушаете закона?
\vs Gal 4:22 Ибо написано: Авраам имел двух сынов, одного от рабы, а другого от свободной.
\vs Gal 4:23 Но который от рабы, тот рожден по плоти; а который от свободной, тот по обетованию.
\vs Gal 4:24 В этом есть иносказание. Это два завета: один от горы Синайской, рождающий в рабство, который есть Агарь,
\vs Gal 4:25 ибо Агарь означает гору Синай в Аравии и соответствует нынешнему Иерусалиму, потому что он с детьми своими в рабстве;
\vs Gal 4:26 а вышний Иерусалим свободен: он~--- матерь всем нам.
\vs Gal 4:27 Ибо написано: возвеселись, неплодная, нерождающая; воскликни и возгласи, не мучившаяся родами; потому что у оставленной гораздо более детей, нежели у имеющей мужа.
\vs Gal 4:28 Мы, братия, дети обетования по Исааку.
\vs Gal 4:29 Но, как тогда рожденный по плоти гнал \bibemph{рожденного} по духу, так и ныне.
\vs Gal 4:30 Что же говорит Писание? Изгони рабу и сына ее, ибо сын рабы не будет наследником вместе с сыном свободной.
\vs Gal 4:31 Итак, братия, мы дети не рабы, но свободной.
\vs Gal 5:1 Итак стойте в свободе, которую даровал нам Христос, и не подвергайтесь опять игу рабства.
\vs Gal 5:2 Вот, я, Павел, говорю вам: если вы обрезываетесь, не будет вам никакой пользы от Христа.
\vs Gal 5:3 Еще свидетельствую всякому человеку обрезывающемуся, что он должен исполнить весь закон.
\vs Gal 5:4 Вы, оправдывающие себя законом, остались без Христа, отпали от благодати,
\vs Gal 5:5 а мы духом ожидаем и надеемся праведности от веры.
\vs Gal 5:6 Ибо во Христе Иисусе не имеет силы ни обрезание, ни необрезание, но вера, действующая любовью.
\vs Gal 5:7 Вы шли хорошо: кто остановил вас, чтобы вы не покорялись истине?
\vs Gal 5:8 Такое убеждение не от Призывающего вас.
\vs Gal 5:9 Малая закваска заквашивает все тесто.
\vs Gal 5:10 Я уверен о вас в Господе, что вы не будете мыслить иначе; а смущающий вас, кто бы он ни был, понесет на себе осуждение.
\vs Gal 5:11 За что же гонят меня, братия, если я и теперь проповедую обрезание? Тогда соблазн креста прекратился бы.
\vs Gal 5:12 О, если бы удалены были возмущающие вас!
\rsbpar\vs Gal 5:13 К свободе призваны вы, братия, только бы свобода ваша не была поводом к \bibemph{угождению} плоти, но любовью служ\acc{и}те друг другу.
\vs Gal 5:14 Ибо весь закон в одном слове заключается: люби ближнего твоего, как самого себя.
\vs Gal 5:15 Если же друг друга угрызаете и съедаете, берегитесь, чтобы вы не были истреблены друг другом.
\rsbpar\vs Gal 5:16 Я говорю: поступайте по духу, и вы не будете исполнять вожделений плоти,
\vs Gal 5:17 ибо плоть желает противного духу, а дух~--- противного плоти: они друг другу противятся, так что вы не то делаете, что хотели бы.
\vs Gal 5:18 Если же вы духом водитесь, то вы не под законом.
\vs Gal 5:19 Дела плоти известны; они суть: прелюбодеяние, блуд, нечистота, непотребство,
\vs Gal 5:20 идолослужение, волшебство, вражда, ссоры, зависть, гнев, распри, разногласия, (соблазны,) ереси,
\vs Gal 5:21 ненависть, убийства, пьянство, бесчинство и тому подобное. Предваряю вас, как и прежде предварял, что поступающие так Царствия Божия не наследуют.
\vs Gal 5:22 Плод же духа: любовь, радость, мир, долготерпение, благость, милосердие, вера,
\vs Gal 5:23 кротость, воздержание. На таковых нет закона.
\vs Gal 5:24 Но те, которые Христовы, распяли плоть со страстями и похотями.
\vs Gal 5:25 Если мы живем духом, то по духу и поступать должны.
\vs Gal 5:26 Не будем тщеславиться, друг друга раздражать, друг другу завидовать.
\vs Gal 6:1 Братия! если и впадет человек в какое согрешение, вы, духовные, исправляйте такового в духе кротости, наблюдая каждый за собою, чтобы не быть искушенным.
\vs Gal 6:2 Нос\acc{и}те бремена друг друга, и таким образом исполните закон Христов.
\vs Gal 6:3 Ибо кто почитает себя чем-нибудь, будучи ничто, тот обольщает сам себя.
\vs Gal 6:4 Каждый да испытывает свое дело, и тогда будет иметь похвалу только в себе, а не в другом,
\vs Gal 6:5 ибо каждый понесет свое бремя.
\rsbpar\vs Gal 6:6 Наставляемый словом, делись всяким добром с наставляющим.
\vs Gal 6:7 Не обманывайтесь: Бог поругаем не бывает. Что посеет человек, то и пожнет:
\vs Gal 6:8 сеющий в плоть свою от плоти пожнет тление, а сеющий в дух от духа пожнет жизнь вечную.
\vs Gal 6:9 Делая добро, да не унываем, ибо в свое время пожнем, если не ослабеем.
\vs Gal 6:10 Итак, доколе есть время, будем делать добро всем, а наипаче своим по вере.
\rsbpar\vs Gal 6:11 Видите, как много написал я вам своею рукою.
\vs Gal 6:12 Желающие хвалиться по плоти принуждают вас обрезываться только для того, чтобы не быть гонимыми за крест Христов,
\vs Gal 6:13 ибо и сами обрезывающиеся не соблюдают закона, но хотят, чтобы вы обрезывались, дабы похвалиться в вашей плоти.
\vs Gal 6:14 А я не желаю хвалиться, разве только крестом Господа нашего Иисуса Христа, которым для меня мир распят, и я для мира.
\vs Gal 6:15 Ибо во Христе Иисусе ничего не значит ни обрезание, ни необрезание, а новая тварь.
\vs Gal 6:16 Тем, которые поступают по сему правилу, мир им и милость, и Израилю Божию.
\vs Gal 6:17 Впрочем никто не отягощай меня, ибо я ношу язвы Господа Иисуса на теле моем.
\rsbpar\vs Gal 6:18 Благодать Господа нашего Иисуса Христа со духом вашим, братия. Аминь.

\bibbookdescr{Eph}{
  inline={Послание к Ефесянам\\\LARGE Святого Апостола Павла},
  toc={к Ефесянам},
  bookmark={к Ефесянам},
  header={к Ефесянам},
  %headerleft={},
  %headerright={},
  abbr={Еф}
}
\vs Eph 1:1 Павел, волею Божиею Апостол Иисуса Христа, находящимся в Ефесе святым и верным во Христе Иисусе:
\vs Eph 1:2 благодать вам и мир от Бога Отца нашего и Господа Иисуса Христа.
\rsbpar\vs Eph 1:3 Благословен Бог и Отец Господа нашего Иисуса Христа, благословивший нас во Христе всяким духовным благословением в небесах,
\vs Eph 1:4 так как Он избрал нас в Нем прежде создания мира, чтобы мы были святы и непорочны пред Ним в любви,
\vs Eph 1:5 предопределив усыновить нас Себе чрез Иисуса Христа, по благоволению воли Своей,
\vs Eph 1:6 в похвалу славы благодати Своей, которою Он облагодатствовал нас в Возлюбленном,
\vs Eph 1:7 в Котором мы имеем искупление Кровию Его, прощение грехов, по богатству благодати Его,
\vs Eph 1:8 каковую Он в преизбытке даровал нам во всякой премудрости и разумении,
\vs Eph 1:9 открыв нам тайну Своей воли по Своему благоволению, которое Он прежде положил в Нем,
\vs Eph 1:10 в устроении полноты времен, дабы все небесное и земное соединить под главою Христом.
\vs Eph 1:11 В Нем мы и сделались наследниками, быв предназначены \bibemph{к тому} по определению Совершающего все по изволению воли Своей,
\vs Eph 1:12 дабы послужить к похвале славы Его нам, которые ранее уповали на Христа.
\vs Eph 1:13 В Нем и вы, услышав слово истины, благовествование вашего спасения, и уверовав в Него, запечатлены обетованным Святым Духом,
\vs Eph 1:14 Который есть залог наследия нашего, для искупления удела \bibemph{Его}, в похвалу славы Его.
\rsbpar\vs Eph 1:15 Посему и я, услышав о вашей вере во Христа Иисуса и о любви ко всем святым,
\vs Eph 1:16 непрестанно благодарю за вас \bibemph{Бога}, вспоминая о вас в молитвах моих,
\vs Eph 1:17 чтобы Бог Господа нашего Иисуса Христа, Отец славы, дал вам Духа премудрости и откровения к познанию Его,
\vs Eph 1:18 и просветил очи сердца вашего, дабы вы познали, в чем состоит надежда призвания Его, и какое богатство славного наследия Его для святых,
\vs Eph 1:19 и как безмерно величие могущества Его в нас, верующих по действию державной силы Его,
\vs Eph 1:20 которою Он воздействовал во Христе, воскресив Его из мертвых и посадив одесную Себя на небесах,
\vs Eph 1:21 превыше всякого Начальства, и Власти, и Силы, и Господства, и всякого имени, именуемого не только в сем веке, но и в будущем,
\vs Eph 1:22 и все покорил под ноги Его, и поставил Его выше всего, главою Церкви,
\vs Eph 1:23 которая есть Тело Его, полнота Наполняющего все во всем.
\vs Eph 2:1 И вас, мертвых по преступлениям и грехам вашим,
\vs Eph 2:2 в которых вы некогда жили, по обычаю мира сего, по \bibemph{воле} князя, господствующего в воздухе, духа, действующего ныне в сынах противления,
\vs Eph 2:3 между которыми и мы все жили некогда по нашим плотским похотям, исполняя желания плоти и помыслов, и были по природе чадами гнева, как и прочие,
\vs Eph 2:4 Бог, богатый милостью, по Своей великой любви, которою возлюбил нас,
\vs Eph 2:5 и нас, мертвых по преступлениям, оживотворил со Христом,~--- благодатью вы спасены,~---
\vs Eph 2:6 и воскресил с Ним, и посадил на небесах во Христе Иисусе,
\vs Eph 2:7 дабы явить в грядущих веках преизобильное богатство благодати Своей в благости к нам во Христе Иисусе.
\vs Eph 2:8 Ибо благодатью вы спасены через веру, и сие не от вас, Божий дар:
\vs Eph 2:9 не от дел, чтобы никто не хвалился.
\vs Eph 2:10 Ибо мы~--- Его творение, созданы во Христе Иисусе на добрые дела, которые Бог предназначил нам исполнять.
\rsbpar\vs Eph 2:11 Итак помните, что вы, некогда язычники по плоти, которых называли необрезанными так называемые обрезанные плотским \bibemph{обрезанием}, совершаемым руками,
\vs Eph 2:12 что вы были в то время без Христа, отчуждены от общества Израильского, чужды заветов обетования, не имели надежды и были безбожники в мире.
\vs Eph 2:13 А теперь во Христе Иисусе вы, бывшие некогда далеко, стали близки Кровию Христовою.
\vs Eph 2:14 Ибо Он есть мир наш, соделавший из обоих одно и разрушивший стоявшую посреди преграду,
\vs Eph 2:15 упразднив вражду Плотию Своею, а закон заповедей учением, дабы из двух создать в Себе Самом одного нового человека, устрояя мир,
\vs Eph 2:16 и в одном теле примирить обоих с Богом посредством креста, убив вражду на нем.
\vs Eph 2:17 И, придя, благовествовал мир вам, дальним и близким,
\vs Eph 2:18 потому что через Него и те и другие имеем доступ к Отцу, в одном Духе.
\rsbpar\vs Eph 2:19 Итак вы уже не чужие и не пришельцы, но сограждане святым и свои Богу,
\vs Eph 2:20 быв утверждены на основании Апостолов и пророков, имея Самого Иисуса Христа краеугольным \bibemph{камнем},
\vs Eph 2:21 на котором все здание, слагаясь стройно, возрастает в святый храм в Господе,
\vs Eph 2:22 на котором и вы устрояетесь в жилище Божие Духом.
\vs Eph 3:1 Для сего-то я, Павел, \bibemph{сделался} узником Иисуса Христа за вас язычников.
\vs Eph 3:2 Как вы слышали о домостроительстве благодати Божией, данной мне для вас,
\vs Eph 3:3 потому что мне через откровение возвещена тайна (о чем я и выше писал кратко),
\vs Eph 3:4 то вы, читая, можете усмотреть мое разумение тайны Христовой,
\vs Eph 3:5 которая не была возвещена прежним поколениям сынов человеческих, как ныне открыта святым Апостолам Его и пророкам Духом Святым,
\vs Eph 3:6 чтобы и язычникам быть сонаследниками, составляющими одно тело, и сопричастниками обетования Его во Христе Иисусе посредством благовествования,
\vs Eph 3:7 которого служителем сделался я по дару благодати Божией, данной мне действием силы Его.
\vs Eph 3:8 Мне, наименьшему из всех святых, дана благодать сия~--- благовествовать язычникам неисследимое богатство Христово
\vs Eph 3:9 и открыть всем, в чем состоит домостроительство тайны, сокрывавшейся от вечности в Боге, создавшем все Иисусом Христом,
\vs Eph 3:10 дабы ныне соделалась известною через Церковь начальствам и властям на небесах многоразличная премудрость Божия,
\vs Eph 3:11 по предвечному определению, которое Он исполнил во Христе Иисусе, Господе нашем,
\vs Eph 3:12 в Котором мы имеем дерзновение и надежный доступ через веру в Него.
\vs Eph 3:13 Посему прошу вас не унывать при моих ради вас скорбях, которые суть ваша слава.
\rsbpar\vs Eph 3:14 Для сего преклоняю колени мои пред Отцем Господа нашего Иисуса Христа,
\vs Eph 3:15 от Которого именуется всякое отечество на небесах и на земле,
\vs Eph 3:16 да даст вам, по богатству славы Своей, крепко утвердиться Духом Его во внутреннем человеке,
\vs Eph 3:17 верою вселиться Христу в сердца ваши,
\vs Eph 3:18 чтобы вы, укорененные и утвержденные в любви, могли постигнуть со всеми святыми, чт\acc{о} широта и долгота, и глубина и высота,
\vs Eph 3:19 и уразуметь превосходящую разумение любовь Христову, дабы вам исполниться всею полнотою Божиею.
\rsbpar\vs Eph 3:20 А Тому, Кто действующею в нас силою может сделать несравненно больше всего, чего мы просим, или о чем помышляем,
\vs Eph 3:21 Тому слава в Церкви во Христе Иисусе во все роды, от века до века. Аминь.
\vs Eph 4:1 Итак я, узник в Господе, умоляю вас поступать достойно звания, в которое вы призваны,
\vs Eph 4:2 со всяким смиренномудрием и кротостью и долготерпением, снисходя друг ко другу любовью,
\vs Eph 4:3 стараясь сохранять единство духа в союзе мира.
\vs Eph 4:4 Одно тело и один дух, как вы и призваны к одной надежде вашего звания;
\vs Eph 4:5 один Господь, одна вера, одно крещение,
\vs Eph 4:6 один Бог и Отец всех, Который над всеми, и через всех, и во всех нас.
\vs Eph 4:7 Каждому же из нас дана благодать по мере дара Христова.
\vs Eph 4:8 Посему и сказано: восшед на высоту, пленил плен и дал дары человекам.
\vs Eph 4:9 А <<восшел>> чт\acc{о} означает, как не то, что Он и нисходил прежде в преисподние места земли?
\vs Eph 4:10 Нисшедший, Он же есть и восшедший превыше всех небес, дабы наполнить все.
\vs Eph 4:11 И Он поставил одних Апостолами, других пророками, иных Евангелистами, иных пастырями и учителями,
\vs Eph 4:12 к совершению святых, на дело служения, для созидания Тела Христова,
\vs Eph 4:13 доколе все придем в единство веры и познания Сына Божия, в мужа совершенного, в меру полного возраста Христова;
\vs Eph 4:14 дабы мы не были более младенцами, колеблющимися и увлекающимися всяким ветром учения, по лукавству человеков, по хитрому искусству обольщения,
\vs Eph 4:15 но истинною любовью все возращали в Того, Который есть глава Христос,
\vs Eph 4:16 из Которого все тело, составляемое и совокупляемое посредством всяких взаимно скрепляющих связей, при действии в свою меру каждого члена, получает приращение для созидания самого себя в любви.
\rsbpar\vs Eph 4:17 Посему я говорю и заклинаю Господом, чтобы вы более не поступали, как поступают прочие народы, по суетности ума своего,
\vs Eph 4:18 будучи помрачены в разуме, отчуждены от жизни Божией, по причине их невежества и ожесточения сердца их.
\vs Eph 4:19 Они, дойдя до бесчувствия, предались распутству так, что делают всякую нечистоту с ненасытимостью.
\vs Eph 4:20 Но вы не так познали Христа;
\vs Eph 4:21 потому что вы слышали о Нем и в Нем научились,~--- так как истина во Иисусе,~---
\vs Eph 4:22 отложить прежний образ жизни ветхого человека, истлевающего в обольстительных похотях,
\vs Eph 4:23 а обновиться духом ума вашего
\vs Eph 4:24 и облечься в нового человека, созданного по Богу, в праведности и святости истины.
\rsbpar\vs Eph 4:25 Посему, отвергнув ложь, говорите истину каждый ближнему своему, потому что мы члены друг другу.
\vs Eph 4:26 Гневаясь, не согрешайте: солнце да не зайдет во гневе вашем;
\vs Eph 4:27 и не давайте места диаволу.
\vs Eph 4:28 Кто крал, вперед не кради, а лучше трудись, делая своими руками полезное, чтобы было из чего уделять нуждающемуся.
\vs Eph 4:29 Никакое гнилое слово да не исходит из уст ваших, а только доброе для назидания в вере, дабы оно доставляло благодать слушающим.
\vs Eph 4:30 И не оскорбляйте Святаго Духа Божия, Которым вы запечатлены в день искупления.
\vs Eph 4:31 Всякое раздражение и ярость, и гнев, и крик, и злоречие со всякою злобою да будут удалены от вас;
\vs Eph 4:32 но будьте друг ко другу добры, сострадательны, прощайте друг друга, как и Бог во Христе простил вас.
\vs Eph 5:1 Итак, подражайте Богу, как чада возлюбленные,
\vs Eph 5:2 и живите в любви, как и Христос возлюбил нас и предал Себя за нас в приношение и жертву Богу, в благоухание приятное.
\vs Eph 5:3 А блуд и всякая нечистота и любостяжание не должны даже именоваться у вас, как прилично святым.
\vs Eph 5:4 Также сквернословие и пустословие и смехотворство не приличны \bibemph{вам}, а, напротив, благодарение;
\vs Eph 5:5 ибо знайте, что никакой блудник, или нечистый, или любостяжатель, который есть идолослужитель, не имеет наследия в Царстве Христа и Бога.
\vs Eph 5:6 Никто да не обольщает вас пустыми словами, ибо за это приходит гнев Божий на сынов противления;
\vs Eph 5:7 итак, не будьте сообщниками их.
\vs Eph 5:8 Вы были некогда тьма, а теперь~--- свет в Господе: поступайте, как чада света,
\vs Eph 5:9 потому что плод Духа состоит во всякой благости, праведности и истине.
\vs Eph 5:10 Испытывайте, чт\acc{о} благоугодно Богу,
\vs Eph 5:11 и не участвуйте в бесплодных делах тьмы, но и обличайте.
\vs Eph 5:12 Ибо о том, чт\acc{о} они делают тайно, стыдно и говорить.
\vs Eph 5:13 Все же обнаруживаемое делается явным от света, ибо все, делающееся явным, свет есть.
\vs Eph 5:14 Посему сказано: <<встань, спящий, и воскресни из мертвых, и осветит тебя Христос>>.
\rsbpar\vs Eph 5:15 Итак, смотр\acc{и}те, поступайте осторожно, не как неразумные, но как мудрые,
\vs Eph 5:16 дорожа временем, потому что дни лукавы.
\vs Eph 5:17 Итак, не будьте нерассудительны, но познавайте, чт\acc{о} есть воля Божия.
\vs Eph 5:18 И не упивайтесь вином, от которого бывает распутство; но исполняйтесь Духом,
\vs Eph 5:19 назидая самих себя псалмами и славословиями и песнопениями духовными, поя и воспевая в сердцах ваших Господу,
\vs Eph 5:20 благодаря всегда за все Бога и Отца, во имя Господа нашего Иисуса Христа,
\vs Eph 5:21 повинуясь друг другу в страхе Божием.
\rsbpar\vs Eph 5:22 Жены, повинуйтесь своим мужьям, как Господу,
\vs Eph 5:23 потому что муж есть глава жены, как и Христос глава Церкви, и Он же Спаситель тела.
\vs Eph 5:24 Но как Церковь повинуется Христу, так и жены своим мужьям во всем.
\rsbpar\vs Eph 5:25 Мужья, любите своих жен, как и Христос возлюбил Церковь и предал Себя за нее,
\vs Eph 5:26 чтобы освятить ее, очистив банею водною посредством слова;
\vs Eph 5:27 чтобы представить ее Себе славною Церковью, не имеющею пятна, или порока, или чего-либо подобного, но дабы она была свята и непорочна.
\vs Eph 5:28 Так должны мужья любить своих жен, как свои тела: любящий свою жену любит самого себя.
\vs Eph 5:29 Ибо никто никогда не имел ненависти к своей плоти, но питает и греет ее, как и Господь Церковь,
\vs Eph 5:30 потому что мы члены тела Его, от плоти Его и от костей Его.
\vs Eph 5:31 Посему оставит человек отца своего и мать и прилепится к жене своей, и будут двое одна плоть.
\vs Eph 5:32 Тайна сия велика; я говорю по отношению ко Христу и к Церкви.
\vs Eph 5:33 Так каждый из вас да любит свою жену, как самого себя; а жена да боится своего мужа.
\vs Eph 6:1 Дети, повинуйтесь своим родителям в Господе, ибо сего \bibemph{требует} справедливость.
\vs Eph 6:2 Почитай отца твоего и мать, это первая заповедь с обетованием:
\vs Eph 6:3 да будет тебе благо, и будешь долголетен на земле.
\vs Eph 6:4 И вы, отцы, не раздражайте детей ваших, но воспитывайте их в учении и наставлении Господнем.
\rsbpar\vs Eph 6:5 Рабы, повинуйтесь господам своим по плоти со страхом и трепетом, в простоте сердца вашего, как Христу,
\vs Eph 6:6 не с видимою только услужливостью, как человекоугодники, но как рабы Христовы, исполняя волю Божию от души,
\vs Eph 6:7 служа с усердием, как Господу, а не как человекам,
\vs Eph 6:8 зная, что каждый получит от Господа по мере добра, которое он сделал, раб ли, или свободный.
\vs Eph 6:9 И вы, господа, поступайте с ними так же, умеряя строгость, зная, что и над вами самими и над ними есть на небесах Господь, у Которого нет лицеприятия.
\rsbpar\vs Eph 6:10 Наконец, братия мои, укрепляйтесь Господом и могуществом силы Его.
\vs Eph 6:11 Облекитесь во всеоружие Божие, чтобы вам можно было стать против козней диавольских,
\vs Eph 6:12 потому что наша брань не против крови и плоти, но против начальств, против властей, против мироправителей тьмы века сего, против духов злобы поднебесных.
\vs Eph 6:13 Для сего приимите всеоружие Божие, дабы вы могли противостать в день злой и, все преодолев, устоять.
\vs Eph 6:14 Итак станьте, препоясав чресла ваши истиною и облекшись в броню праведности,
\vs Eph 6:15 и обув ноги в готовность благовествовать мир;
\vs Eph 6:16 а паче всего возьмите щит веры, которым возможете угасить все раскаленные стрелы лукавого;
\vs Eph 6:17 и шлем спасения возьмите, и меч духовный, который есть Слово Божие.
\vs Eph 6:18 Всякою молитвою и прошением молитесь во всякое время духом, и старайтесь о сем самом со всяким постоянством и молением о всех святых
\vs Eph 6:19 и о мне, дабы мне дано было слово~--- устами моими открыто с дерзновением возвещать тайну благовествования,
\vs Eph 6:20 для которого я исполняю посольство в узах, дабы я смело проповедовал, к\acc{а}к мне должно.
\rsbpar\vs Eph 6:21 А дабы и вы знали о моих обстоятельствах и делах, обо всем известит вас Тихик, возлюбленный брат и верный в Господе служитель,
\vs Eph 6:22 которого я и послал к вам для того самого, чтобы вы узнали о нас и чтобы он утешил сердца ваши.
\rsbpar\vs Eph 6:23 Мир братиям и любовь с верою от Бога Отца и Господа Иисуса Христа.
\vs Eph 6:24 Благодать со всеми, неизменно любящими Господа нашего Иисуса Христа. Аминь.

\bibbookdescr{Phi}{
  inline={Послание к Филиппийцам\\\LARGE Святого Апостола Павла},
  toc={к Филиппийцам},
  bookmark={к Филиппийцам},
  header={к Филиппийцам},
  %headerleft={},
  %headerright={},
  abbr={Флп}
}
\vs Phi 1:1 Павел и Тимофей, рабы Иисуса Христа, всем святым во Христе Иисусе, находящимся в Филиппах, с епископами и диаконами:
\vs Phi 1:2 благодать вам и мир от Бога Отца нашего и Господа Иисуса Христа.
\rsbpar\vs Phi 1:3 Благодарю Бога моего при всяком воспоминании о вас,
\vs Phi 1:4 всегда во всякой молитве моей за всех вас принося с радостью молитву мою,
\vs Phi 1:5 за ваше участие в благовествовании от первого дня даже доныне,
\vs Phi 1:6 будучи уверен в том, что начавший в вас доброе дело будет совершать его даже до дня Иисуса Христа,
\vs Phi 1:7 как и должно мне помышлять о всех вас, потому что я имею вас в сердце в узах моих, при защищении и утверждении благовествования, вас всех, как соучастников моих в благодати.
\vs Phi 1:8 Бог~--- свидетель, что я люблю всех вас любовью Иисуса Христа;
\vs Phi 1:9 и молюсь о том, чтобы любовь ваша еще более и более возрастала в познании и всяком чувстве,
\vs Phi 1:10 чтобы, познавая лучшее, вы были чисты и непреткновенны в день Христов,
\vs Phi 1:11 исполнены плодов праведности Иисусом Христом, в славу и похвалу Божию.
\rsbpar\vs Phi 1:12 Желаю, братия, чтобы вы знали, что обстоятельства мои послужили к большему успеху благовествования,
\vs Phi 1:13 так что узы мои о Христе сделались известными всей претории и всем прочим,
\vs Phi 1:14 и б\acc{о}льшая часть из братьев в Господе, ободрившись узами моими, начали с большею смелостью, безбоязненно проповедовать слово Божие.
\vs Phi 1:15 Некоторые, правда, по зависти и любопрению, а другие с добрым расположением проповедуют Христа.
\vs Phi 1:16 Одни по любопрению проповедуют Христа не чисто, думая увеличить тяжесть уз моих;
\vs Phi 1:17 а другие~--- из любви, зная, что я поставлен защищать благовествование.
\vs Phi 1:18 Но что до того? Как бы ни проповедали Христа, притворно или искренно, я и тому радуюсь и буду радоваться,
\vs Phi 1:19 ибо знаю, что это послужит мне во спасение по вашей молитве и содействием Духа Иисуса Христа,
\vs Phi 1:20 при уверенности и надежде моей, что я ни в чем посрамлен не буду, но при всяком дерзновении, и ныне, как и всегда, возвеличится Христос в теле моем, жизнью ли то, или смертью.
\vs Phi 1:21 Ибо для меня жизнь~--- Христос, и смерть~--- приобретение.
\vs Phi 1:22 Если же жизнь во плоти \bibemph{доставляет} плод моему делу, то не знаю, что избрать.
\vs Phi 1:23 Влечет меня то и другое: имею желание разрешиться и быть со Христом, потому что это несравненно лучше;
\vs Phi 1:24 а оставаться во плоти нужнее для вас.
\vs Phi 1:25 И я верно знаю, что останусь и пребуду со всеми вами для вашего успеха и радости в вере,
\vs Phi 1:26 дабы похвала ваша во Христе Иисусе умножилась через меня, при моем вторичном к вам пришествии.
\vs Phi 1:27 Только живите достойно благовествования Христова, чтобы мне, приду ли я и увижу вас, или не приду, слышать о вас, что вы стоите в одном духе, подвизаясь единодушно за веру Евангельскую,
\vs Phi 1:28 и не страшитесь ни в чем противников: это для них есть предзнаменование погибели, а для вас~--- спасения. И сие от Бога,
\vs Phi 1:29 потому что вам дано ради Христа не только веровать в Него, но и страдать за Него
\vs Phi 1:30 таким же подвигом, какой вы видели во мне и ныне слышите о мне.
\vs Phi 2:1 Итак, если \bibemph{есть} какое утешение во Христе, если \bibemph{есть} какая отрада любви, если \bibemph{есть} какое общение духа, если \bibemph{есть} какое милосердие и сострадательность,
\vs Phi 2:2 то дополните мою радость: имейте одни мысли, имейте ту же любовь, будьте единодушны и единомысленны;
\vs Phi 2:3 ничего \bibemph{не делайте} по любопрению или по тщеславию, но по смиренномудрию почитайте один другого высшим себя.
\vs Phi 2:4 Не о себе \bibemph{только} каждый заботься, но каждый и о других.
\vs Phi 2:5 Ибо в вас должны быть те же чувствования, какие и во Христе Иисусе:
\vs Phi 2:6 Он, будучи образом Божиим, не почитал хищением быть равным Богу;
\vs Phi 2:7 но уничижил Себя Самого, приняв образ раба, сделавшись подобным человекам и по виду став как человек;
\vs Phi 2:8 смирил Себя, быв послушным даже до смерти, и смерти крестной.
\vs Phi 2:9 Посему и Бог превознес Его и дал Ему имя выше всякого имени,
\vs Phi 2:10 дабы пред именем Иисуса преклонилось всякое колено небесных, земных и преисподних,
\vs Phi 2:11 и всякий язык исповедал, что Господь Иисус Христос в славу Бога Отца.
\rsbpar\vs Phi 2:12 Итак, возлюбленные мои, как вы всегда были послушны, не только в присутствии моем, но гораздо более ныне во время отсутствия моего, со страхом и трепетом совершайте свое спасение,
\vs Phi 2:13 потому что Бог производит в вас и хотение и действие по \bibemph{Своему} благоволению.
\vs Phi 2:14 Всё делайте без ропота и сомнения,
\vs Phi 2:15 чтобы вам быть неукоризненными и чистыми, чадами Божиими непорочными среди строптивого и развращенного рода, в котором вы сияете, как светила в мире,
\vs Phi 2:16 содержа слово жизни, к похвале моей в день Христов, что я не тщетно подвизался и не тщетно трудился.
\vs Phi 2:17 Но если я и соделываюсь жертвою за жертву и служение веры вашей, то радуюсь и сорадуюсь всем вам.
\vs Phi 2:18 О сем самом и вы радуйтесь и сорадуйтесь мне.
\rsbpar\vs Phi 2:19 Надеюсь же в Господе Иисусе вскоре послать к вам Тимофея, дабы и я, узнав о ваших обстоятельствах, утешился духом.
\vs Phi 2:20 Ибо я не имею никого равно усердного, кто бы столь искренно заботился о вас,
\vs Phi 2:21 потому что все ищут своего, а не того, что \bibemph{угодно} Иисусу Христу.
\vs Phi 2:22 А его верность вам известна, потому что он, как сын отцу, служил мне в благовествовании.
\vs Phi 2:23 Итак я надеюсь послать его тотчас же, как скоро узн\acc{а}ю, что будет со мною.
\vs Phi 2:24 Я уверен в Господе, что и сам скоро приду к вам.
\vs Phi 2:25 Впрочем я почел нужным послать к вам Епафродита, брата и сотрудника и сподвижника моего, а вашего посланника и служителя в нужде моей,
\vs Phi 2:26 потому что он сильно желал видеть всех вас и тяжко скорбел о том, что до вас дошел слух о его болезни.
\vs Phi 2:27 Ибо он был болен при смерти; но Бог помиловал его, и не его только, но и меня, чтобы не прибавилась мне печаль к печали.
\vs Phi 2:28 Посему я скорее послал его, чтобы вы, увидев его снова, возрадовались, и я был менее печален.
\vs Phi 2:29 Примите же его в Господе со всякою радостью, и таких имейте в уважении,
\vs Phi 2:30 ибо он за дело Христово был близок к смерти, подвергая опасности жизнь, дабы восполнить недостаток ваших услуг мне.
\vs Phi 3:1 Впрочем, братия мои, радуйтесь о Господе. Писать вам о том же для меня не тягостно, а для вас назидательно.
\rsbpar\vs Phi 3:2 Берегитесь псов, берегитесь злых делателей, берегитесь обрезания,
\vs Phi 3:3 потому что обрезание~--- мы, служащие Богу духом и хвалящиеся Христом Иисусом, и не на плоть надеющиеся,
\vs Phi 3:4 хотя я могу надеяться и на плоть. Если кто другой думает надеяться на плоть, то более я,
\vs Phi 3:5 обрезанный в восьмой день, из рода Израилева, колена Вениаминова, Еврей от Евреев, по учению фарисей,
\vs Phi 3:6 по ревности~--- гонитель Церкви Божией, по правде законной~--- непорочный.
\vs Phi 3:7 Но что для меня было преимуществом, то ради Христа я почел тщетою.
\vs Phi 3:8 Да и все почитаю тщетою ради превосходства познания Христа Иисуса, Господа моего: для Него я от всего отказался, и все почитаю за сор, чтобы приобрести Христа
\vs Phi 3:9 и найтись в Нем не со своею праведностью, которая от закона, но с тою, которая через веру во Христа, с праведностью от Бога по вере;
\vs Phi 3:10 чтобы познать Его, и силу воскресения Его, и участие в страданиях Его, сообразуясь смерти Его,
\vs Phi 3:11 чтобы достигнуть воскресения мертвых.
\vs Phi 3:12 \bibemph{Говорю так} не потому, чтобы я уже достиг, или усовершился; но стремлюсь, не достигну ли я, как достиг меня Христос Иисус.
\vs Phi 3:13 Братия, я не почитаю себя достигшим; а только, забывая заднее и простираясь вперед,
\vs Phi 3:14 стремлюсь к цели, к почести вышнего звания Божия во Христе Иисусе.
\vs Phi 3:15 Итак, кто из нас совершен, так должен мыслить; если же вы о чем иначе мыслите, то и это Бог вам откроет.
\vs Phi 3:16 Впрочем, до чего мы достигли, так и должны мыслить и по тому правилу жить.
\rsbpar\vs Phi 3:17 Подражайте, братия, мне и смотрите на тех, которые поступают по образу, какой имеете в нас.
\vs Phi 3:18 Ибо многие, о которых я часто говорил вам, а теперь даже со слезами говорю, поступают как враги креста Христова.
\vs Phi 3:19 Их конец~--- погибель, их бог~--- чрево, и слава их~--- в сраме, они мыслят о земном.
\vs Phi 3:20 Наше же жительство~--- на небесах, откуда мы ожидаем и Спасителя, Господа нашего Иисуса Христа,
\vs Phi 3:21 Который уничиженное тело наше преобразит так, что оно будет сообразно славному телу Его, силою, \bibemph{которою} Он действует и покоряет Себе всё.
\vs Phi 4:1 Итак, братия мои возлюбленные и вожделенные, радость и венец мой, стойте так в Господе, возлюбленные.
\rsbpar\vs Phi 4:2 Умоляю Еводию, умоляю Синтихию мыслить то же о Господе.
\vs Phi 4:3 Ей, прошу и тебя, искренний сотрудник, помогай им, подвизавшимся в благовествовании вместе со мною и с Климентом и с прочими сотрудниками моими, которых имена~--- в книге жизни.
\rsbpar\vs Phi 4:4 Радуйтесь всегда в Господе; и еще говорю: радуйтесь.
\vs Phi 4:5 Кротость ваша да будет известна всем человекам. Господь близко.
\vs Phi 4:6 Не заботьтесь ни о чем, но всегда в молитве и прошении с благодарением открывайте свои желания пред Богом,
\vs Phi 4:7 и мир Божий, который превыше всякого ума, соблюдет сердца ваши и помышления ваши во Христе Иисусе.
\rsbpar\vs Phi 4:8 Наконец, братия мои, чт\acc{о} только истинно, чт\acc{о} честно, чт\acc{о} справедливо, чт\acc{о} чисто, чт\acc{о} любезно, чт\acc{о} достославно, чт\acc{о} только добродетель и похвала, о том помышляйте.
\vs Phi 4:9 Чему вы научились, чт\acc{о} приняли и слышали и видели во мне, т\acc{о} исполняйте,~--- и Бог мира будет с вами.
\rsbpar\vs Phi 4:10 Я весьма возрадовался в Господе, что вы уже вновь начали заботиться о мне; вы и прежде заботились, но вам не благоприятствовали обстоятельства.
\vs Phi 4:11 Говорю это не потому, что нуждаюсь, ибо я научился быть довольным тем, что у меня есть.
\vs Phi 4:12 Умею жить и в скудости, умею жить и в изобилии; научился всему и во всем, насыщаться и терпеть голод, быть и в обилии и в недостатке.
\vs Phi 4:13 Все могу в укрепляющем меня Иисусе Христе.
\vs Phi 4:14 Впрочем вы хорошо поступили, приняв участие в моей скорби.
\vs Phi 4:15 Вы знаете, Филиппийцы, что в начале благовествования, когда я вышел из Македонии, ни одна церковь не оказала мне участия подаянием и принятием, кроме вас одних;
\vs Phi 4:16 вы и в Фессалонику и раз и два присылали мне на нужду.
\vs Phi 4:17 \bibemph{Говорю это} не потому, чтобы я искал даяния; но ищу плода, умножающегося в пользу вашу.
\vs Phi 4:18 Я получил все, и избыточествую; я доволен, получив от Епафродита посланное вами, \bibemph{как} благовонное курение, жертву приятную, благоугодную Богу.
\vs Phi 4:19 Бог мой да восполнит всякую нужду вашу, по богатству Своему в славе, Христом Иисусом.
\vs Phi 4:20 Богу же и Отцу нашему слава во веки веков! Аминь.
\rsbpar\vs Phi 4:21 Приветствуйте всякого святого во Христе Иисусе. Приветствуют вас находящиеся со мною братия.
\vs Phi 4:22 Приветствуют вас все святые, а наипаче из кесарева дома.
\rsbpar\vs Phi 4:23 Благодать Господа нашего Иисуса Христа со всеми вами. Аминь.

\bibbookdescr{Col}{
  inline={Послание к Колоссянам\\\LARGE Святого Апостола Павла},
  toc={к Колоссянам},
  bookmark={к Колоссянам},
  header={к Колоссянам},
  %headerleft={},
  %headerright={},
  abbr={Кол}
}
\vs Col 1:1 Павел, волею Божиею Апостол Иисуса Христа, и Тимофей брат,
\vs Col 1:2 находящимся в Колоссах святым и верным братиям во Христе Иисусе:
\vs Col 1:3 благодать вам и мир от Бога Отца нашего и Господа Иисуса Христа. Благодарим Бога и Отца Господа нашего Иисуса Христа, всегда молясь о вас,
\vs Col 1:4 услышав о вере вашей во Христа Иисуса и о любви ко всем святым,
\vs Col 1:5 в надежде на уготованное вам на небесах, о чем вы прежде слышали в истинном слове благовествования,
\vs Col 1:6 которое пребывает у вас, как и во всем мире, и приносит плод, и возрастает, как и между вами, с того дня, как вы услышали и познали благодать Божию в истине,
\vs Col 1:7 как и научились от Епафраса, возлюбленного сотрудника нашего, верного для вас служителя Христова,
\vs Col 1:8 который и известил нас о вашей любви в духе.
\rsbpar\vs Col 1:9 Посему и мы с того дня, как \bibemph{о сем} услышали, не перестаем молиться о вас и просить, чтобы вы исполнялись познанием воли Его, во всякой премудрости и разумении духовном,
\vs Col 1:10 чтобы поступали достойно Бога, во всем угождая \bibemph{Ему}, принося плод во всяком деле благом и возрастая в познании Бога,
\vs Col 1:11 укрепляясь всякою силою по могуществу славы Его, во всяком терпении и великодушии с радостью,
\vs Col 1:12 благодаря Бога и Отца, призвавшего нас к участию в наследии святых во свете,
\vs Col 1:13 избавившего нас от власти тьмы и введшего в Царство возлюбленного Сына Своего,
\vs Col 1:14 в Котором мы имеем искупление Кровию Его и прощение грехов,
\vs Col 1:15 Который есть образ Бога невидимого, рожденный прежде всякой твари;
\vs Col 1:16 ибо Им создано всё, что на небесах и что на земле, видимое и невидимое: престолы ли, господства ли, начальства ли, власти ли,~--- все Им и для Него создано;
\vs Col 1:17 и Он есть прежде всего, и все Им сто\acc{и}т.
\vs Col 1:18 И Он есть глава тела Церкви; Он~--- начаток, первенец из мертвых, дабы иметь Ему во всем первенство,
\vs Col 1:19 ибо благоугодно было \bibemph{Отцу}, чтобы в Нем обитала всякая полнота,
\vs Col 1:20 и чтобы посредством Его примирить с Собою все, умиротворив через Него, Кровию креста Его, и земное и небесное.
\vs Col 1:21 И вас, бывших некогда отчужденными и врагами, по расположению к злым делам,
\vs Col 1:22 ныне примирил в теле Плоти Его, смертью Его, \bibemph{чтобы} представить вас святыми и непорочными и неповинными пред Собою,
\vs Col 1:23 если только пребываете тверды и непоколебимы в вере и не отпадаете от надежды благовествования, которое вы слышали, которое возвещено всей твари поднебесной, которого я, Павел, сделался служителем.
\rsbpar\vs Col 1:24 Ныне радуюсь в страданиях моих за вас и восполняю недостаток в плоти моей скорбей Христовых за Тело Его, которое есть Церковь,
\vs Col 1:25 которой сделался я служителем по домостроительству Божию, вверенному мне для вас, \bibemph{чтобы} исполнить слово Божие,
\vs Col 1:26 тайну, сокрытую от веков и родов, ныне же открытую святым Его,
\vs Col 1:27 которым благоволил Бог показать, какое богатство славы в тайне сей для язычников, которая есть Христос в вас, упование славы,
\vs Col 1:28 Которого мы проповедуем, вразумляя всякого человека и научая всякой премудрости, чтобы представить всякого человека совершенным во Христе Иисусе;
\vs Col 1:29 для чего я и тружусь и подвизаюсь силою Его, действующею во мне могущественно.
\vs Col 2:1 Желаю, чтобы вы знали, какой подвиг имею я ради вас и ради тех, которые в Лаодикии и Иераполе, и ради всех, кто не видел лица моего в плоти,
\vs Col 2:2 дабы утешились сердца их, соединенные в любви для всякого богатства совершенного разумения, для познания тайны Бога и Отца и Христа,
\vs Col 2:3 в Котором сокрыты все сокровища премудрости и ведения.
\vs Col 2:4 Это говорю я для того, чтобы кто-нибудь не прельстил вас вкрадчивыми словами;
\vs Col 2:5 ибо хотя я и отсутствую телом, но духом нахожусь с вами, радуясь и видя ваше благоустройство и твердость веры вашей во Христа.
\vs Col 2:6 Посему, как вы приняли Христа Иисуса Господа, \bibemph{так} и ход\acc{и}те в Нем,
\vs Col 2:7 будучи укоренены и утверждены в Нем и укреплены в вере, как вы научены, преуспевая в ней с благодарением.
\rsbpar\vs Col 2:8 Смотрите, братия, чтобы кто не увлек вас философиею и пустым обольщением, по преданию человеческому, по стихиям мира, а не по Христу;
\vs Col 2:9 ибо в Нем обитает вся полнота Божества телесно,
\vs Col 2:10 и вы имеете полноту в Нем, Который есть глава всякого начальства и власти.
\vs Col 2:11 В Нем вы и обрезаны обрезанием нерукотворенным, совлечением греховного тела плоти, обрезанием Христовым;
\vs Col 2:12 быв погребены с Ним в крещении, в Нем вы и совоскресли верою в силу Бога, Который воскресил Его из мертвых,
\vs Col 2:13 и вас, которые были мертвы во грехах и в необрезании плоти вашей, оживил вместе с Ним, простив нам все грехи,
\vs Col 2:14 истребив учением бывшее о нас рукописание, которое было против нас, и Он взял его от среды и пригвоздил ко кресту;
\vs Col 2:15 отняв силы у начальств и властей, властно подверг их позору, восторжествовав над ними Собою.
\rsbpar\vs Col 2:16 Итак никто да не осуждает вас за пищу, или питие, или за какой-нибудь праздник, или новомесячие, или субботу:
\vs Col 2:17 это есть тень будущего, а тело~--- во Христе.
\vs Col 2:18 Никто да не обольщает вас самовольным смиренномудрием и служением Ангелов, вторгаясь в то, чего не видел, безрассудно надмеваясь плотским своим умом
\vs Col 2:19 и не держась главы, от которой все тело, составами и связями будучи соединяемо и скрепляемо, растет возрастом Божиим.
\vs Col 2:20 Итак, если вы со Христом умерли для стихий мира, то для чего вы, как живущие в мире, держитесь постановлений:
\vs Col 2:21 <<не прикасайся>>, <<не вкушай>>, <<не дотрагивайся>>
\vs Col 2:22 (что все истлевает от употребления), по заповедям и учению человеческому?
\vs Col 2:23 Это имеет только вид мудрости в самовольном служении, смиренномудрии и изнурении тела, в некотором небрежении о насыщении плоти.
\vs Col 3:1 Итак, если вы воскресли со Христом, то ищите горнего, где Христос сидит одесную Бога;
\vs Col 3:2 о горнем помышляйте, а не о земном.
\vs Col 3:3 Ибо вы умерли, и жизнь ваша сокрыта со Христом в Боге.
\vs Col 3:4 Когда же явится Христос, жизнь ваша, тогда и вы явитесь с Ним во славе.
\rsbpar\vs Col 3:5 Итак, умертвите земные члены ваши: блуд, нечистоту, страсть, злую похоть и любостяжание, которое есть идолослужение,
\vs Col 3:6 за которые гнев Божий грядет на сынов противления,
\vs Col 3:7 в которых и вы некогда обращались, когда жили между ними.
\vs Col 3:8 А теперь вы отложите все: гнев, ярость, злобу, злоречие, сквернословие уст ваших;
\vs Col 3:9 не говорите лжи друг другу, совлекшись ветхого человека с делами его
\vs Col 3:10 и облекшись в нового, который обновляется в познании по образу Создавшего его,
\vs Col 3:11 где нет ни Еллина, ни Иудея, ни обрезания, ни необрезания, варвара, Скифа, раба, свободного, но все и во всем Христос.
\rsbpar\vs Col 3:12 Итак облекитесь, как избранные Божии, святые и возлюбленные, в милосердие, благость, смиренномудрие, кротость, долготерпение,
\vs Col 3:13 снисходя друг другу и прощая взаимно, если кто на кого имеет жалобу: как Христос простил вас, так и вы.
\vs Col 3:14 Более же всего \bibemph{облекитесь} в любовь, которая есть совокупность совершенства.
\vs Col 3:15 И да владычествует в сердцах ваших мир Божий, к которому вы и призваны в одном теле, и будьте дружелюбны.
\vs Col 3:16 Слово Христово да вселяется в вас обильно, со всякою премудростью; научайте и вразумляйте друг друга псалмами, славословием и духовными песнями, во благодати воспевая в сердцах ваших Господу.
\vs Col 3:17 И всё, что вы делаете, словом или делом, всё \bibemph{делайте} во имя Господа Иисуса Христа, благодаря через Него Бога и Отца.
\rsbpar\vs Col 3:18 Жены, повинуйтесь мужьям своим, как прилично в Господе.
\vs Col 3:19 Мужья, любите своих жен и не будьте к ним суровы.
\vs Col 3:20 Дети, будьте послушны родителям вашим во всем, ибо это благоугодно Господу.
\vs Col 3:21 Отцы, не раздражайте детей ваших, дабы они не унывали.
\vs Col 3:22 Рабы, во всем повинуйтесь господам вашим по плоти, не в глазах только служа \bibemph{им}, как человекоугодники, но в простоте сердца, боясь Бога.
\vs Col 3:23 И всё, что делаете, делайте от души, как для Господа, а не для человеков,
\vs Col 3:24 зная, что в воздаяние от Господа получите наследие, ибо вы служите Господу Христу.
\vs Col 3:25 А кто неправо поступит, тот получит по своей неправде, \bibemph{у Него} нет лицеприятия.
\vs Col 4:1 Господ\acc{а}, оказывайте рабам должное и справедливое, зная, что и вы имеете Господа на небесах.
\rsbpar\vs Col 4:2 Будьте постоянны в молитве, бодрствуя в ней с благодарением.
\vs Col 4:3 Молитесь также и о нас, чтобы Бог отверз нам дверь для слова, возвещать тайну Христову, за которую я и в узах,
\vs Col 4:4 дабы я открыл ее, как должно мне возвещать.
\vs Col 4:5 Со внешними обходитесь благоразумно, пользуясь временем.
\vs Col 4:6 Слово ваше \bibemph{да будет} всегда с благодатию, приправлено солью, дабы вы знали, как отвечать каждому.
\rsbpar\vs Col 4:7 О мне всё скажет вам Тихик, возлюбленный брат и верный служитель и сотрудник в Господе,
\vs Col 4:8 которого я для того послал к вам, чтобы он узнал о ваших \bibemph{обстоятельствах} и утешил сердца ваши,
\vs Col 4:9 с Онисимом, верным и возлюбленным братом нашим, который от вас. Они расскажут вам о всем здешнем.
\rsbpar\vs Col 4:10 Приветствует вас Аристарх, заключенный вместе со мною, и Марк, племянник Варнавы (о котором вы получили приказания: если придет к вам, примите его),
\vs Col 4:11 также Иисус, прозываемый Иустом, оба из обрезанных. Они~--- единственные сотрудники для Царствия Божия, бывшие мне отрадою.
\vs Col 4:12 Приветствует вас Епафрас ваш, раб Иисуса Христа, всегда подвизающийся за вас в молитвах, чтобы вы пребыли совершенны и исполнены всем, что угодно Богу.
\vs Col 4:13 Свидетельствую о нем, что он имеет великую ревность и заботу о вас и о находящихся в Лаодикии и Иераполе.
\vs Col 4:14 Приветствует вас Лука, врач возлюбленный, и Димас.
\vs Col 4:15 Приветствуйте братьев в Лаодикии, и Нимфана, и домашнюю церковь его.
\rsbpar\vs Col 4:16 Когда это послание прочитано будет у вас, то распорядитесь, чтобы оно было прочитано и в Лаодикийской церкви; а то, которое из Лаодикии, прочитайте и вы.
\rsbpar\vs Col 4:17 Скажите Архиппу: смотри, чтобы тебе исполнить служение, которое ты принял в Господе.
\rsbpar\vs Col 4:18 Приветствие моею рукою, Павловою. Помните мои узы. Благодать со всеми вами. Аминь.
\newbookpage
\bibbookdescr{1Th}{
  inline={Первое Послание\\к Фессалоникийцам\\\LARGE Святого Апостола Павла},
  toc={1-е Фессалоникийцам},
  bookmark={1-е Фессалоникийцам},
  header={1-е Фессалоникийцам},
  %headerleft={},
  %headerright={},
  abbr={1~Фес}
}
\vs 1Th 1:1 Павел и Силуан и Тимофей~--- церкви Фессалоникской в Боге Отце и Господе Иисусе Христе: благодать вам и мир от Бога Отца нашего и Господа Иисуса Христа.
\rsbpar\vs 1Th 1:2 Всегда благодарим Бога за всех вас, вспоминая о вас в молитвах наших,
\vs 1Th 1:3 непрестанно памятуя ваше дело веры и труд любви и терпение упования на Господа нашего Иисуса Христа пред Богом и Отцем нашим,
\vs 1Th 1:4 зная избрание ваше, возлюбленные Богом братия;
\vs 1Th 1:5 потому что наше благовествование у вас было не в слове только, но и в силе и во Святом Духе, и со многим удостоверением, как вы \bibemph{сами} знаете, каковы были мы для вас между вами.
\vs 1Th 1:6 И вы сделались подражателями нам и Господу, приняв слово при многих скорбях с радостью Духа Святаго,
\vs 1Th 1:7 так что вы стали образцом для всех верующих в Македонии и Ахаии.
\vs 1Th 1:8 Ибо от вас пронеслось слово Господне не только в Македонии и Ахаии, но и во всяком месте прошла \bibemph{слава} о вере вашей в Бога, так что нам ни о чем не нужно рассказывать.
\vs 1Th 1:9 Ибо сами они сказывают о нас, какой вход имели мы к вам, и как вы обратились к Богу от идолов, \bibemph{чтобы} служить Богу живому и истинному
\vs 1Th 1:10 и ожидать с небес Сына Его, Которого Он воскресил из мертвых, Иисуса, избавляющего нас от грядущего гнева.
\vs 1Th 2:1 Вы сами знаете, братия, о нашем входе к вам, что он был не бездейственный;
\vs 1Th 2:2 но, прежде пострадав и быв поруганы в Филиппах, как вы знаете, мы дерзнули в Боге нашем проповедать вам благовестие Божие с великим подвигом.
\vs 1Th 2:3 Ибо в учении нашем нет ни заблуждения, ни нечистых \bibemph{побуждений}, ни лукавства;
\vs 1Th 2:4 но, как Бог удостоил нас того, чтобы вверить \bibemph{нам} благовестие, так мы и говорим, угождая не человекам, но Богу, испытующему сердца наши.
\vs 1Th 2:5 Ибо никогда не было у нас перед вами ни слов ласкательства, как вы знаете, ни видов корысти: Бог свидетель!
\vs 1Th 2:6 Не ищем славы человеческой ни от вас, ни от других:
\vs 1Th 2:7 мы могли явиться с важностью, как Апостолы Христовы, но были тихи среди вас, подобно как кормилица нежно обходится с детьми своими.
\vs 1Th 2:8 Так мы, из усердия к вам, восхотели передать вам не только благовестие Божие, но и души наши, потому что вы стали нам любезны.
\vs 1Th 2:9 Ибо вы помните, братия, труд наш и изнурение: ночью и днем работая, чтобы не отяготить кого из вас, мы проповедовали у вас благовестие Божие.
\vs 1Th 2:10 Свидетели вы и Бог, как свято и праведно и безукоризненно поступали мы перед вами, верующими,
\vs 1Th 2:11 потому что вы знаете, как каждого из вас, как отец детей своих,
\vs 1Th 2:12 мы просили и убеждали и умоляли поступать достойно Бога, призвавшего вас в Свое Царство и славу.
\rsbpar\vs 1Th 2:13 Посему и мы непрестанно благодарим Бога, что, приняв от нас слышанное слово Божие, вы приняли не \bibemph{к\acc{а}к} слово человеческое, но \bibemph{как} слово Божие,~--- каково оно есть по истине,~--- которое и действует в вас, верующих.
\vs 1Th 2:14 Ибо вы, братия, сделались подражателями церквам Божиим во Христе Иисусе, находящимся в Иудее, потому что и вы то же претерпели от своих единоплеменников, что и те от Иудеев,
\vs 1Th 2:15 которые убили и Господа Иисуса и Его пророков, и нас изгнали, и Богу не угождают, и всем человекам противятся,
\vs 1Th 2:16 которые препятствуют нам говорить язычникам, чтобы спаслись, и через это всегда наполняют меру грехов своих; но приближается на них гнев до конца.
\rsbpar\vs 1Th 2:17 Мы же, братия, быв разлучены с вами на короткое время лицем, а не сердцем, тем с б\acc{о}льшим желанием старались увидеть лице ваше.
\vs 1Th 2:18 И потому мы, я Павел, и раз и два хотели прийти к вам, но воспрепятствовал нам сатана.
\vs 1Th 2:19 Ибо кто наша надежда, или радость, или венец похвалы? Не и вы ли пред Господом нашим Иисусом Христом в пришествие Его?
\vs 1Th 2:20 Ибо вы~--- слава наша и радость.
\vs 1Th 3:1 И потому, не терпя более, мы восхотели остаться в Афинах одни,
\vs 1Th 3:2 и послали Тимофея, брата нашего и служителя Божия и сотрудника нашего в благовествовании Христовом, чтобы утвердить вас и утешить в вере вашей,
\vs 1Th 3:3 чтобы никто не поколебался в скорбях сих: ибо вы сами знаете, что так нам суждено.
\vs 1Th 3:4 Ибо мы и тогда, как были у вас, предсказывали вам, что будем страдать, как и случилось, и вы знаете.
\vs 1Th 3:5 Посему и я, не терпя более, послал узнать о вере вашей, чтобы как не искусил вас искуситель и не сделался тщетным труд наш.
\vs 1Th 3:6 Теперь же, когда пришел к нам от вас Тимофей и принес нам добрую весть о вере и любви вашей, и что вы всегда имеете добрую память о нас, желая нас видеть, как и мы вас,
\vs 1Th 3:7 то мы, при всей скорби и нужде нашей, утешились вами, братия, ради вашей веры;
\vs 1Th 3:8 ибо теперь мы живы, когда вы стоите в Господе.
\vs 1Th 3:9 Какую благодарность можем мы воздать Богу за вас, за всю радость, которою радуемся о вас пред Богом нашим,
\vs 1Th 3:10 ночь и день всеусердно молясь о том, чтобы видеть лице ваше и дополнить, чего недоставало вере вашей?
\vs 1Th 3:11 Сам же Бог и Отец наш и Господь наш Иисус Христос да управит путь наш к вам.
\vs 1Th 3:12 А вас Господь да исполнит и преисполнит любовью друг к другу и ко всем, какою мы исполнены к вам,
\vs 1Th 3:13 чтобы утвердить сердца ваши непорочными во святыне пред Богом и Отцем нашим в пришествие Господа нашего Иисуса Христа со всеми святыми Его. Аминь.
\vs 1Th 4:1 За сим, братия, просим и умоляем вас Христом Иисусом, чтобы вы, приняв от нас, как должно вам поступать и угождать Богу, более в том преуспевали,
\vs 1Th 4:2 ибо вы знаете, какие мы дали вам заповеди от Господа Иисуса.
\vs 1Th 4:3 Ибо воля Божия есть освящение ваше, чтобы вы воздерживались от блуда;
\vs 1Th 4:4 чтобы каждый из вас умел соблюдать свой сосуд в святости и чести,
\vs 1Th 4:5 а не в страсти похотения, как и язычники, не знающие Бога;
\vs 1Th 4:6 чтобы вы ни в чем не поступали с братом своим противозаконно и корыстолюбиво: потому что Господь~--- мститель за все это, как и прежде мы говорили вам и свидетельствовали.
\vs 1Th 4:7 Ибо призвал нас Бог не к нечистоте, но к святости.
\vs 1Th 4:8 Итак непокорный непокорен не человеку, но Богу, Который и дал нам Духа Своего Святаго.
\rsbpar\vs 1Th 4:9 О братолюбии же нет нужды писать к вам; ибо вы сами научены Богом любить друг друга,
\vs 1Th 4:10 ибо вы так и поступаете со всеми братиями по всей Македонии. Умоляем же вас, братия, более преуспевать
\vs 1Th 4:11 и усердно стараться о том, чтобы жить тихо, делать свое \bibemph{дело} и работать своими собственными руками, как мы заповедовали вам;
\vs 1Th 4:12 чтобы вы поступали благоприлично перед внешними и ни в чем не нуждались.
\rsbpar\vs 1Th 4:13 Не хочу же оставить вас, братия, в неведении об умерших, дабы вы не скорбели, как прочие, не имеющие надежды.
\vs 1Th 4:14 Ибо, если мы веруем, что Иисус умер и воскрес, то и умерших в Иисусе Бог приведет с Ним.
\vs 1Th 4:15 Ибо сие говорим вам словом Господним, что мы живущие, оставшиеся до пришествия Господня, не предупредим умерших,
\vs 1Th 4:16 потому что Сам Господь при возвещении, при гласе Архангела и трубе Божией, сойдет с неба, и мертвые во Христе воскреснут прежде;
\vs 1Th 4:17 потом мы, оставшиеся в живых, вместе с ними восхищены будем на облаках в сретение Господу на воздухе, и так всегда с Господом будем.
\vs 1Th 4:18 Итак утешайте друг друга сими словами.
\vs 1Th 5:1 О временах же и сроках нет нужды писать к вам, братия,
\vs 1Th 5:2 ибо сами вы достоверно знаете, что день Господень так придет, как тать ночью.
\vs 1Th 5:3 Ибо, когда будут говорить: <<мир и безопасность>>, тогда внезапно постигнет их пагуба, подобно как мука родами \bibemph{постигает} имеющую во чреве, и не избегнут.
\vs 1Th 5:4 Но вы, братия, не во тьме, чтобы день застал вас, как тать.
\vs 1Th 5:5 Ибо все вы~--- сыны света и сыны дня: мы~--- не \bibemph{сыны} ночи, ни тьмы.
\vs 1Th 5:6 Итак, не будем спать, как и прочие, но будем бодрствовать и трезвиться.
\vs 1Th 5:7 Ибо спящие спят ночью, и упивающиеся упиваются ночью.
\vs 1Th 5:8 Мы же, будучи \bibemph{сынами} дня, да трезвимся, облекшись в броню веры и любви и в шлем надежды спасения,
\vs 1Th 5:9 потому что Бог определил нас не на гнев, но к получению спасения через Господа нашего Иисуса Христа,
\vs 1Th 5:10 умершего за нас, чтобы мы, бодрствуем ли, или спим, жили вместе с Ним.
\vs 1Th 5:11 Посему увещавайте друг друга и назидайте один другого, как вы и делаете.
\rsbpar\vs 1Th 5:12 Просим же вас, братия, уважать трудящихся у вас, и предстоятелей ваших в Господе, и вразумляющих вас,
\vs 1Th 5:13 и почитать их преимущественно с любовью за дело их; будьте в мире между собою.
\vs 1Th 5:14 Умоляем также вас, братия, вразумляйте бесчинных, утешайте малодушных, поддерживайте слабых, будьте долготерпеливы ко всем.
\vs 1Th 5:15 Смотрите, чтобы кто кому не воздавал злом за зло; но всегда ищите добра и друг другу и всем.
\vs 1Th 5:16 Всегда радуйтесь.
\vs 1Th 5:17 Непрестанно молитесь.
\vs 1Th 5:18 За все благодарите: ибо такова о вас воля Божия во Христе Иисусе.
\vs 1Th 5:19 Духа не угашайте.
\vs 1Th 5:20 Пророчества не уничижайте.
\vs 1Th 5:21 Все испытывайте, хорошего держитесь.
\vs 1Th 5:22 Удерживайтесь от всякого рода зла.
\vs 1Th 5:23 Сам же Бог мира да освятит вас во всей полноте, и ваш дух и душа и тело во всей целости да сохранится без порока в пришествие Господа нашего Иисуса Христа.
\vs 1Th 5:24 Верен Призывающий вас, Который и сотворит \bibemph{сие}.
\vs 1Th 5:25 Братия! молитесь о нас.
\rsbpar\vs 1Th 5:26 Приветствуйте всех братьев лобзанием святым.
\vs 1Th 5:27 Заклинаю вас Господом прочитать сие послание всем святым братиям.
\rsbpar\vs 1Th 5:28 Благодать Господа нашего Иисуса Христа с вами. Аминь.

\bibbookdescr{2Th}{
  inline={Второе Послание\\к Фессалоникийцам\\\LARGE Святого Апостола Павла},
  toc={2-е Фессалоникийцам},
  bookmark={2-е Фессалоникийцам},
  header={2-е Фессалоникийцам},
  %headerleft={},
  %headerright={},
  abbr={2~Фес}
}
\vs 2Th 1:1 Павел и Силуан и Тимофей~--- Фессалоникской церкви в Боге Отце нашем и Господе Иисусе Христе:
\vs 2Th 1:2 благодать вам и мир от Бога Отца нашего и Господа Иисуса Христа.
\rsbpar\vs 2Th 1:3 Всегда по справедливости мы должны благодарить Бога за вас, братия, потому что возрастает вера ваша, и умножается любовь каждого друг ко другу между всеми вами,
\vs 2Th 1:4 так что мы сами хвалимся вами в церквах Божиих, терпением вашим и верою во всех гонениях и скорбях, переносимых вами
\vs 2Th 1:5 в доказательство того, что будет праведный суд Божий, чтобы вам удостоиться Царствия Божия, для которого и страдаете.
\vs 2Th 1:6 Ибо праведно пред Богом~--- оскорбляющим вас воздать скорбью,
\vs 2Th 1:7 а вам, оскорбляемым, отрадою вместе с нами, в явление Господа Иисуса с неба, с Ангелами силы Его,
\vs 2Th 1:8 в пламенеющем огне совершающего отмщение не познавшим Бога и не покоряющимся благовествованию Господа нашего Иисуса Христа,
\vs 2Th 1:9 которые подвергнутся наказанию, вечной погибели, от лица Господа и от славы могущества Его,
\vs 2Th 1:10 когда Он приидет прославиться во святых Своих и явиться дивным в день оный во всех веровавших, так как вы поверили нашему свидетельству.
\vs 2Th 1:11 Для сего и молимся всегда за вас, чтобы Бог наш соделал вас достойными звания и совершил всякое благоволение благости и дело веры в силе,
\vs 2Th 1:12 да прославится имя Господа нашего Иисуса Христа в вас, и вы в Нем, по благодати Бога нашего и Господа Иисуса Христа.
\vs 2Th 2:1 Молим вас, братия, о пришествии Господа нашего Иисуса Христа и нашем собрании к Нему,
\vs 2Th 2:2 не спешить колебаться умом и смущаться ни от духа, ни от слова, ни от послания, как бы нами посланного, будто уже наступает день Христов.
\vs 2Th 2:3 Да не обольстит вас никто никак: \bibemph{ибо день тот не придет}, доколе не придет прежде отступление и не откроется человек греха, сын погибели,
\vs 2Th 2:4 противящийся и превозносящийся выше всего, называемого Богом или святынею, так что в храме Божием сядет он, как Бог, выдавая себя за Бога.
\vs 2Th 2:5 Не помните ли, что я, еще находясь у вас, говорил вам это?
\vs 2Th 2:6 И ныне вы знаете, чт\acc{о} не допускает открыться ему в свое время.
\vs 2Th 2:7 Ибо тайна беззакония уже в действии, только \bibemph{не совершится} до тех пор, пока не будет взят от среды удерживающий теперь.
\vs 2Th 2:8 И тогда откроется беззаконник, которого Господь Иисус убьет духом уст Своих и истребит явлением пришествия Своего
\vs 2Th 2:9 того, которого пришествие, по действию сатаны, будет со всякою силою и знамениями и чудесами ложными,
\vs 2Th 2:10 и со всяким неправедным обольщением погибающих за то, что они не приняли любви истины для своего спасения.
\vs 2Th 2:11 И за сие пошлет им Бог действие заблуждения, так что они будут верить лжи,
\vs 2Th 2:12 да будут осуждены все, не веровавшие истине, но возлюбившие неправду.
\rsbpar\vs 2Th 2:13 Мы же всегда должны благодарить Бога за вас, возлюбленные Господом братия, что Бог от начала, через освящение Духа и веру истине, избрал вас ко спасению,
\vs 2Th 2:14 к которому и призвал вас благовествованием нашим, для достижения славы Господа нашего Иисуса Христа.
\rsbpar\vs 2Th 2:15 Итак, братия, стойте и держите предания, которым вы научены или словом или посланием нашим.
\vs 2Th 2:16 Сам же Господь наш Иисус Христос и Бог и Отец наш, возлюбивший нас и давший утешение вечное и надежду благую во благодати,
\vs 2Th 2:17 да утешит ваши сердца и да утвердит вас во всяком слове и деле благом.
\vs 2Th 3:1 Итак молитесь за нас, братия, чтобы слово Господне распространялось и прославлялось, как и у вас,
\vs 2Th 3:2 и чтобы нам избавиться от беспорядочных и лукавых людей, ибо не во всех вера.
\vs 2Th 3:3 Но верен Господь, Который утвердит вас и сохранит от лукавого.
\vs 2Th 3:4 Мы уверены о вас в Господе, что вы исполняете и будете исполнять то, что мы вам повелеваем.
\vs 2Th 3:5 Господь же да управит сердца ваши в любовь Божию и в терпение Христово.
\rsbpar\vs 2Th 3:6 Завещеваем же вам, братия, именем Господа нашего Иисуса Христа, удаляться от всякого брата, поступающего бесчинно, а не по преданию, которое приняли от нас,
\vs 2Th 3:7 ибо вы сами знаете, как должны вы подражать нам; ибо мы не бесчинствовали у вас,
\vs 2Th 3:8 ни у кого не ели хлеба даром, но занимались трудом и работою ночь и день, чтобы не обременить кого из вас,~---
\vs 2Th 3:9 не потому, чтобы мы не имели власти, но чтобы себя самих дать вам в образец для подражания нам.
\vs 2Th 3:10 Ибо когда мы были у вас, то завещевали вам сие: если кто не хочет трудиться, тот и не ешь.
\vs 2Th 3:11 Но слышим, что некоторые у вас поступают бесчинно, ничего не делают, а суетятся.
\vs 2Th 3:12 Таковых увещеваем и убеждаем Господом нашим Иисусом Христом, чтобы они, работая в безмолвии, ели свой хлеб.
\vs 2Th 3:13 Вы же, братия, не унывайте, делая добро.
\vs 2Th 3:14 Если же кто не послушает слова нашего в сем послании, того имейте на замечании и не сообщайтесь с ним, чтобы устыдить его.
\vs 2Th 3:15 Но не считайте его за врага, а вразумляйте, как брата.
\vs 2Th 3:16 Сам же Господь мира да даст вам мир всегда во всем. Господь со всеми вами!
\rsbpar\vs 2Th 3:17 Приветствие моею рукою, Павловою, что служит знаком во всяком послании; пишу я так:
\vs 2Th 3:18 благодать Господа нашего Иисуса Христа со всеми вами. Аминь.

\bibbookdescr{1Ti}{
  inline={Первое Послание к Тимофею\\\LARGE Святого Апостола Павла},
  toc={1-е Тимофею},
  bookmark={1-е Тимофею},
  header={1-е Тимофею},
  %headerleft={},
  %headerright={},
  abbr={1~Тим}
}
\vs 1Ti 1:1 Павел, Апостол Иисуса Христа по повелению Бога, Спасителя нашего, и Господа Иисуса Христа, надежды нашей,
\vs 1Ti 1:2 Тимофею, истинному сыну в вере: благодать, милость, мир от Бога, Отца нашего, и Христа Иисуса, Господа нашего.
\rsbpar\vs 1Ti 1:3 Отходя в Македонию, я просил тебя пребыть в Ефесе и увещевать некоторых, чтобы они не учили иному
\vs 1Ti 1:4 и не занимались баснями и родословиями бесконечными, которые производят больше споры, нежели Божие назидание в вере.
\vs 1Ti 1:5 Цель же увещания есть любовь от чистого сердца и доброй совести и нелицемерной веры,
\vs 1Ti 1:6 от чего отступив, некоторые уклонились в пустословие,
\vs 1Ti 1:7 желая быть законоучителями, но не разумея ни того, о чем говорят, ни того, что утверждают.
\vs 1Ti 1:8 А мы знаем, что закон добр, если кто законно употребляет его,
\vs 1Ti 1:9 зная, что закон положен не для праведника, но для беззаконных и непокоривых, нечестивых и грешников, развратных и оскверненных, для оскорбителей отца и матери, для человекоубийц,
\vs 1Ti 1:10 для блудников, мужеложников, человекохищников, (клеветников, скотоложников,) лжецов, клятвопреступников, и для всего, что противно здравому учению,
\vs 1Ti 1:11 по славному благовестию блаженного Бога, которое мне вверено.
\rsbpar\vs 1Ti 1:12 Благодарю давшего мне силу, Христа Иисуса, Господа нашего, что Он признал меня верным, определив на служение,
\vs 1Ti 1:13 меня, который прежде был хулитель и гонитель и обидчик, но помилован потому, что \bibemph{так} поступал по неведению, в неверии;
\vs 1Ti 1:14 благодать же Господа нашего (Иисуса Христа) открылась \bibemph{во мне} обильно с верою и любовью во Христе Иисусе.
\vs 1Ti 1:15 Верно и всякого принятия достойно слово, что Христос Иисус пришел в мир спасти грешников, из которых я первый.
\vs 1Ti 1:16 Но для того я и помилован, чтобы Иисус Христос во мне первом показал все долготерпение, в пример тем, которые будут веровать в Него к жизни вечной.
\vs 1Ti 1:17 Царю же веков нетленному, невидимому, единому премудрому Богу честь и слава во веки веков. Аминь.
\rsbpar\vs 1Ti 1:18 Преподаю тебе, сын \bibemph{мой} Тимофей, сообразно с бывшими о тебе пророчествами, такое завещание, чтобы ты воинствовал согласно с ними, как добрый воин,
\vs 1Ti 1:19 имея веру и добрую совесть, которую некоторые отвергнув, потерпели кораблекрушение в вере;
\vs 1Ti 1:20 таковы Именей и Александр, которых я предал сатане, чтобы они научились не богохульствовать.
\vs 1Ti 2:1 Итак прежде всего прошу совершать молитвы, прошения, моления, благодарения за всех человеков,
\vs 1Ti 2:2 за царей и за всех начальствующих, дабы проводить нам жизнь тихую и безмятежную во всяком благочестии и чистоте,
\vs 1Ti 2:3 ибо это хорошо и угодно Спасителю нашему Богу,
\vs 1Ti 2:4 Который хочет, чтобы все люди спаслись и достигли познания истины.
\vs 1Ti 2:5 Ибо един Бог, един и посредник между Богом и человеками, человек Христос Иисус,
\vs 1Ti 2:6 предавший Себя для искупления всех. \bibemph{Таково было} в свое время свидетельство,
\vs 1Ti 2:7 для которого я поставлен проповедником и Апостолом,~--- истину говорю во Христе, не лгу,~--- учителем язычников в вере и истине.
\rsbpar\vs 1Ti 2:8 Итак желаю, чтобы на всяком месте произносили молитвы мужи, воздевая чистые руки без гнева и сомнения;
\vs 1Ti 2:9 чтобы также и жены, в приличном одеянии, со стыдливостью и целомудрием, украшали себя не плетением \bibemph{волос}, не золотом, не жемчугом, не многоценною одеждою,
\vs 1Ti 2:10 но добрыми делами, как прилично женам, посвящающим себя благочестию.
\vs 1Ti 2:11 Жена да учится в безмолвии, со всякою покорностью;
\vs 1Ti 2:12 а учить жене не позволяю, ни властвовать над мужем, но быть в безмолвии.
\vs 1Ti 2:13 Ибо прежде создан Адам, а потом Ева;
\vs 1Ti 2:14 и не Адам прельщен; но жена, прельстившись, впала в преступление;
\vs 1Ti 2:15 впрочем спасется через чадородие, если пребудет в вере и любви и в святости с целомудрием.
\vs 1Ti 3:1 Верно слово: если кто епископства желает, доброго дела желает.
\vs 1Ti 3:2 Но епископ должен быть непорочен, одной жены муж, трезв, целомудрен, благочинен, честен, страннолюбив, учителен,
\vs 1Ti 3:3 не пьяница, не бийца, не сварлив, не корыстолюбив, но тих, миролюбив, не сребролюбив,
\vs 1Ti 3:4 хорошо управляющий домом своим, детей содержащий в послушании со всякою честностью;
\vs 1Ti 3:5 ибо, кто не умеет управлять собственным домом, тот будет ли пещись о Церкви Божией?
\vs 1Ti 3:6 Не \bibemph{должен быть} из новообращенных, чтобы не возгордился и не подпал осуждению с диаволом.
\vs 1Ti 3:7 Надлежит ему также иметь доброе свидетельство от внешних, чтобы не впасть в нарекание и сеть диавольскую.
\vs 1Ti 3:8 Диаконы также \bibemph{должны быть} честны, не двоязычны, не пристрастны к вину, не корыстолюбивы,
\vs 1Ti 3:9 хранящие таинство веры в чистой совести.
\vs 1Ti 3:10 И таких надобно прежде испытывать, потом, если беспорочны, \bibemph{допускать} до служения.
\vs 1Ti 3:11 Равно и жены \bibemph{их должны быть} честны, не клеветницы, трезвы, верны во всем.
\vs 1Ti 3:12 Диакон должен быть муж одной жены, хорошо управляющий детьми и домом своим.
\vs 1Ti 3:13 Ибо хорошо служившие приготовляют себе высшую степень и великое дерзновение в вере во Христа Иисуса.
\rsbpar\vs 1Ti 3:14 Сие пишу тебе, надеясь вскоре прийти к тебе,
\vs 1Ti 3:15 чтобы, если замедлю, ты знал, как должно поступать в доме Божием, который есть Церковь Бога живаго, столп и утверждение истины.
\vs 1Ti 3:16 И беспрекословно~--- великая благочестия тайна: Бог явился во плоти, оправдал Себя в Духе, показал Себя Ангелам, проповедан в народах, принят верою в мире, вознесся во славе.
\vs 1Ti 4:1 Дух же ясно говорит, что в последние времена отступят некоторые от веры, внимая духам обольстителям и учениям бесовским,
\vs 1Ti 4:2 через лицемерие лжесловесников, сожженных в совести своей,
\vs 1Ti 4:3 запрещающих вступать в брак \bibemph{и} употреблять в пищу то, что Бог сотворил, дабы верные и познавшие истину вкушали с благодарением.
\vs 1Ti 4:4 Ибо всякое творение Божие хорошо, и ничто не предосудительно, если принимается с благодарением,
\vs 1Ti 4:5 потому что освящается словом Божиим и молитвою.
\rsbpar\vs 1Ti 4:6 Внушая сие братиям, будешь добрый служитель Иисуса Христа, питаемый словами веры и добрым учением, которому ты последовал.
\vs 1Ti 4:7 Негодных же и бабьих басен отвращайся, а упражняй себя в благочестии,
\vs 1Ti 4:8 ибо телесное упражнение мало полезно, а благочестие на все полезно, имея обетование жизни настоящей и будущей.
\vs 1Ti 4:9 Слово сие верно и всякого принятия достойно.
\vs 1Ti 4:10 Ибо мы для того и трудимся и поношения терпим, что уповаем на Бога живаго, Который есть Спаситель всех человеков, а наипаче верных.
\vs 1Ti 4:11 Проповедуй сие и учи.
\vs 1Ti 4:12 Никто да не пренебрегает юностью твоею; но будь образцом для верных в слове, в житии, в любви, в духе, в вере, в чистоте.
\vs 1Ti 4:13 Доколе не приду, занимайся чтением, наставлением, учением.
\vs 1Ti 4:14 Не неради о пребывающем в тебе даровании, которое дано тебе по пророчеству с возложением рук священства.
\vs 1Ti 4:15 О сем заботься, в сем пребывай, дабы успех твой для всех был очевиден.
\vs 1Ti 4:16 Вникай в себя и в учение; занимайся сим постоянно: ибо, так поступая, и себя спасешь и слушающих тебя.
\vs 1Ti 5:1 Старца не укоряй, но увещевай, как отца; младших, как братьев;
\vs 1Ti 5:2 стариц, как матерей; молодых, как сестер, со всякою чистотою.
\vs 1Ti 5:3 Вдовиц почитай, истинных вдовиц.
\vs 1Ti 5:4 Если же какая вдовица имеет детей или внучат, то они прежде пусть учатся почитать свою семью и воздавать должное родителям, ибо сие угодно Богу.
\vs 1Ti 5:5 Истинная вдовица и одинокая надеется на Бога и пребывает в молениях и молитвах день и ночь;
\vs 1Ti 5:6 а сластолюбивая заживо умерла.
\vs 1Ti 5:7 И сие внушай им, чтобы были беспорочны.
\vs 1Ti 5:8 Если же кто о своих и особенно о домашних не печется, тот отрекся от веры и хуже неверного.
\vs 1Ti 5:9 Вдовица должна быть избираема не менее, как шестидесятилетняя, бывшая женою одного мужа,
\vs 1Ti 5:10 известная по добрым делам, если она воспитала детей, принимала странников, умывала ноги святым, помогала бедствующим и была усердна ко всякому доброму делу.
\vs 1Ti 5:11 Молодых же вдовиц не принимай, ибо они, впадая в роскошь в противность Христу, желают вступать в брак.
\vs 1Ti 5:12 Они подлежат осуждению, потому что отвергли прежнюю веру;
\vs 1Ti 5:13 притом же они, будучи праздны, приучаются ходить по домам и \bibemph{бывают} не только праздны, но и болтливы, любопытны, и говорят, чего не должно.
\vs 1Ti 5:14 Итак я желаю, чтобы молодые вдовы вступали в брак, рождали детей, управляли домом и не подавали противнику никакого повода к злоречию;
\vs 1Ti 5:15 ибо некоторые уже совратились вслед сатаны.
\vs 1Ti 5:16 Если какой верный или верная имеет вдов, то должны их довольствовать и не обременять Церкви, чтобы она могла довольствовать истинных вдовиц.
\rsbpar\vs 1Ti 5:17 Достойно начальствующим пресвитерам должно оказывать сугубую честь, особенно тем, которые трудятся в слове и учении.
\vs 1Ti 5:18 Ибо Писание говорит: не заграждай рта у вола молотящего; и: трудящийся достоин награды своей.
\vs 1Ti 5:19 Обвинение на пресвитера не иначе принимай, как при двух или трех свидетелях.
\vs 1Ti 5:20 Согрешающих обличай перед всеми, чтобы и прочие страх имели.
\rsbpar\vs 1Ti 5:21 Пред Богом и Господом Иисусом Христом и избранными Ангелами заклинаю тебя сохранить сие без предубеждения, ничего не делая по пристрастию.
\vs 1Ti 5:22 Рук ни на кого не возлагай поспешно, и не делайся участником в чужих грехах. Храни себя чистым.
\vs 1Ti 5:23 Впредь пей не \bibemph{одну} воду, но употребляй немного вина, ради желудка твоего и частых твоих недугов.
\vs 1Ti 5:24 Грехи некоторых людей явны и прямо ведут к осуждению, а некоторых \bibemph{открываются} впоследствии.
\vs 1Ti 5:25 Равным образом и добрые дела явны; а если и не таковы, скрыться не могут.
\vs 1Ti 6:1 Рабы, под игом находящиеся, должны почитать господ своих достойными всякой чести, дабы не было хулы на имя Божие и учение.
\vs 1Ti 6:2 Те, которые имеют господами верных, не должны обращаться с ними небрежно, потому что они братья; но тем более должны служить им, что они верные и возлюбленные и благодетельствуют \bibemph{им}. Учи сему и увещевай.
\rsbpar\vs 1Ti 6:3 Кто учит иному и не следует здравым словам Господа нашего Иисуса Христа и учению о благочестии,
\vs 1Ti 6:4 тот горд, ничего не знает, но заражен \bibemph{страстью} к состязаниям и словопрениям, от которых происходят зависть, распри, злоречия, лукавые подозрения.
\vs 1Ti 6:5 Пустые споры между людьми поврежденного ума, чуждыми истины, которые думают, будто благочестие служит для прибытка. Удаляйся от таких.
\rsbpar\vs 1Ti 6:6 Великое приобретение~--- быть благочестивым и довольным.
\vs 1Ti 6:7 Ибо мы ничего не принесли в мир; явно, что ничего не можем и вынести \bibemph{из него}.
\vs 1Ti 6:8 Имея пропитание и одежду, будем довольны тем.
\vs 1Ti 6:9 А желающие обогащаться впадают в искушение и в сеть и во многие безрассудные и вредные похоти, которые погружают людей в бедствие и пагубу;
\vs 1Ti 6:10 ибо корень всех зол есть сребролюбие, которому предавшись, некоторые уклонились от веры и сами себя подвергли многим скорбям.
\vs 1Ti 6:11 Ты же, человек Божий, убегай сего, а преуспевай в правде, благочестии, вере, любви, терпении, кротости.
\vs 1Ti 6:12 Подвизайся добрым подвигом веры, держись вечной жизни, к которой ты и призван, и исповедал доброе исповедание перед многими свидетелями.
\vs 1Ti 6:13 Пред Богом, все животворящим, и пред Христом Иисусом, Который засвидетельствовал пред Понтием Пилатом доброе исповедание, завещеваю тебе
\vs 1Ti 6:14 соблюсти заповедь чисто и неукоризненно, даже до явления Господа нашего Иисуса Христа,
\vs 1Ti 6:15 которое в свое время откроет блаженный и единый сильный Царь царствующих и Господь господствующих,
\vs 1Ti 6:16 единый имеющий бессмертие, Который обитает в неприступном свете, Которого никто из человеков не видел и видеть не может. Ему честь и держава вечная! Аминь.
\rsbpar\vs 1Ti 6:17 Богатых в настоящем веке увещевай, чтобы они не высоко думали \bibemph{о себе} и уповали не на богатство неверное, но на Бога живаго, дающего нам всё обильно для наслаждения;
\vs 1Ti 6:18 чтобы они благодетельствовали, богатели добрыми делами, были щедры и общительны,
\vs 1Ti 6:19 собирая себе сокровище, доброе основание для будущего, чтобы достигнуть вечной жизни.
\rsbpar\vs 1Ti 6:20 О, Тимофей! храни преданное тебе, отвращаясь негодного пустословия и прекословий лжеименного знания,
\vs 1Ti 6:21 которому предавшись, некоторые уклонились от веры. Благодать с тобою. Аминь.
\newbookpage
\bibbookdescr{2Ti}{
  inline={Второе Послание к Тимофею\\\LARGE Святого Апостола Павла},
  toc={2-е Тимофею},
  bookmark={2-е Тимофею},
  header={2-е Тимофею},
  %headerleft={},
  %headerright={},
  abbr={2~Тим}
}
\vs 2Ti 1:1 Павел, волею Божиею Апостол Иисуса Христа, по обетованию жизни во Христе Иисусе,
\vs 2Ti 1:2 Тимофею, возлюбленному сыну: благодать, милость, мир от Бога Отца и Христа Иисуса, Господа нашего.
\rsbpar\vs 2Ti 1:3 Благодарю Бога, Которому служу от прародителей с чистою совестью, что непрестанно вспоминаю о тебе в молитвах моих днем и ночью,
\vs 2Ti 1:4 и желаю видеть тебя, вспоминая о слезах твоих, дабы мне исполниться радости,
\vs 2Ti 1:5 приводя на память нелицемерную веру твою, которая прежде обитала в бабке твоей Лоиде и матери твоей Евнике; уверен, что она и в тебе.
\vs 2Ti 1:6 По сей причине напоминаю тебе возгревать дар Божий, который в тебе через мое рукоположение;
\vs 2Ti 1:7 ибо дал нам Бог духа не боязни, но силы и любви и целомудрия.
\vs 2Ti 1:8 Итак, не стыдись свидетельства Господа нашего Иисуса Христа, ни меня, узника Его; но страдай с благовестием Христовым силою Бога,
\vs 2Ti 1:9 спасшего нас и призвавшего званием святым, не по делам нашим, но по Своему изволению и благодати, данной нам во Христе Иисусе прежде вековых времен,
\vs 2Ti 1:10 открывшейся же ныне явлением Спасителя нашего Иисуса Христа, разрушившего смерть и явившего жизнь и нетление через благовестие,
\vs 2Ti 1:11 для которого я поставлен проповедником и Апостолом и учителем язычников.
\vs 2Ti 1:12 По сей причине я и страдаю так; но не стыжусь. Ибо я знаю, в Кого уверовал, и уверен, что Он силен сохранить залог мой на оный день.
\vs 2Ti 1:13 Держись образца здравого учения, которое ты слышал от меня, с верою и любовью во Христе Иисусе.
\vs 2Ti 1:14 Храни добрый залог Духом Святым, живущим в нас.
\rsbpar\vs 2Ti 1:15 Ты знаешь, что все Асийские оставили меня; в числе их Фигелл и Ермоген.
\vs 2Ti 1:16 Да даст Господь милость дому Онисифора за то, что он многократно покоил меня и не стыдился уз моих,
\vs 2Ti 1:17 но, быв в Риме, с великим тщанием искал меня и нашел.
\vs 2Ti 1:18 Да даст ему Господь обрести милость у Господа в оный день; а сколько он служил мне в Ефесе, ты лучше знаешь.
\vs 2Ti 2:1 Итак укрепляйся, сын мой, в благодати Христом Иисусом,
\vs 2Ti 2:2 и что слышал от меня при многих свидетелях, то передай верным людям, которые были бы способны и других научить.
\vs 2Ti 2:3 Итак переноси страдания, как добрый воин Иисуса Христа.
\vs 2Ti 2:4 Никакой воин не связывает себя делами житейскими, чтобы угодить военачальнику.
\vs 2Ti 2:5 Если же кто и подвизается, не увенчивается, если незаконно будет подвизаться.
\vs 2Ti 2:6 Трудящемуся земледельцу первому должно вкусить от плодов.
\vs 2Ti 2:7 Разумей, что я говорю. Да даст тебе Господь разумение во всем.
\rsbpar\vs 2Ti 2:8 Помни Господа Иисуса Христа от семени Давидова, воскресшего из мертвых, по благовествованию моему,
\vs 2Ti 2:9 за которое я страдаю даже до уз, как злодей; но для слова Божия нет уз.
\vs 2Ti 2:10 Посему я все терплю ради избранных, дабы и они получили спасение во Христе Иисусе с вечною славою.
\vs 2Ti 2:11 Верно слово: если мы с Ним умерли, то с Ним и оживем;
\vs 2Ti 2:12 если терпим, то с Ним и царствовать будем; если отречемся, и Он отречется от нас;
\vs 2Ti 2:13 если мы неверны, Он пребывает верен, ибо Себя отречься не может.
\rsbpar\vs 2Ti 2:14 Сие напоминай, заклиная пред Господом не вступать в словопрения, что нимало не служит к пользе, а к расстройству слушающих.
\vs 2Ti 2:15 Старайся представить себя Богу достойным, делателем неукоризненным, верно преподающим слово истины.
\vs 2Ti 2:16 А непотребного пустословия удаляйся; ибо они еще более будут преуспевать в нечестии,
\vs 2Ti 2:17 и слово их, как рак, будет распространяться. Таковы Именей и Филит,
\vs 2Ti 2:18 которые отступили от истины, говоря, что воскресение уже было, и разрушают в некоторых веру.
\vs 2Ti 2:19 Но твердое основание Божие сто\acc{и}т, имея печать сию: <<познал Господь Своих>>; и: <<да отступит от неправды всякий, исповедующий имя Господа>>.
\vs 2Ti 2:20 А в большом доме есть сосуды не только золотые и серебряные, но и деревянные и глиняные; и одни в почетном, а другие в низком употреблении.
\vs 2Ti 2:21 Итак, кто будет чист от сего, тот будет сосудом в чести, освященным и благопотребным Владыке, годным на всякое доброе дело.
\vs 2Ti 2:22 Юношеских похотей убегай, а держись правды, веры, любви, мира со всеми призывающими Господа от чистого сердца.
\vs 2Ti 2:23 От глупых и невежественных состязаний уклоняйся, зная, что они рождают ссоры;
\vs 2Ti 2:24 рабу же Господа не должно ссориться, но быть приветливым ко всем, учительным, незлобивым,
\vs 2Ti 2:25 с кротостью наставлять противников, не даст ли им Бог покаяния к познанию истины,
\vs 2Ti 2:26 чтобы они освободились от сети диавола, который уловил их в свою волю.
\vs 2Ti 3:1 Знай же, что в последние дни наступят времена тяжкие.
\vs 2Ti 3:2 Ибо люди будут самолюбивы, сребролюбивы, горды, надменны, злоречивы, родителям непокорны, неблагодарны, нечестивы, недружелюбны,
\vs 2Ti 3:3 непримирительны, клеветники, невоздержны, жестоки, не любящие добра,
\vs 2Ti 3:4 предатели, наглы, напыщенны, более сластолюбивы, нежели боголюбивы,
\vs 2Ti 3:5 имеющие вид благочестия, силы же его отрекшиеся. Таковых удаляйся.
\vs 2Ti 3:6 К сим принадлежат те, которые вкрадываются в домы и обольщают женщин, утопающих во грехах, водимых различными похотями,
\vs 2Ti 3:7 всегда учащихся и никогда не могущих дойти до познания истины.
\vs 2Ti 3:8 Как Ианний и Иамврий противились Моисею, так и сии противятся истине, люди, развращенные умом, невежды в вере.
\vs 2Ti 3:9 Но они не много успеют; ибо их безумие обнаружится перед всеми, как и с теми случилось.
\vs 2Ti 3:10 А ты последовал мне в учении, житии, расположении, вере, великодушии, любви, терпении,
\vs 2Ti 3:11 в гонениях, страданиях, постигших меня в Антиохии, Иконии, Листрах; каковые гонения я перенес, и от всех избавил меня Господь.
\vs 2Ti 3:12 Да и все, желающие жить благочестиво во Христе Иисусе, будут гонимы.
\vs 2Ti 3:13 Злые же люди и обманщики будут преуспевать во зле, вводя в заблуждение и заблуждаясь.
\vs 2Ti 3:14 А ты пребывай в том, чему научен и что тебе вверено, зная, кем ты научен.
\vs 2Ti 3:15 Притом же ты из детства знаешь священные писания, которые могут умудрить тебя во спасение верою во Христа Иисуса.
\vs 2Ti 3:16 Все Писание богодухновенно и полезно для научения, для обличения, для исправления, для наставления в праведности,
\vs 2Ti 3:17 да будет совершен Божий человек, ко всякому доброму делу приготовлен.
\vs 2Ti 4:1 Итак заклинаю тебя пред Богом и Господом нашим Иисусом Христом, Который будет судить живых и мертвых в явление Его и Царствие Его:
\vs 2Ti 4:2 проповедуй слово, настой во время и не во время, обличай, запрещай, увещевай со всяким долготерпением и назиданием.
\vs 2Ti 4:3 Ибо будет время, когда здравого учения принимать не будут, но по своим прихотям будут избирать себе учителей, которые льстили бы слуху;
\vs 2Ti 4:4 и от истины отвратят слух и обратятся к басням.
\vs 2Ti 4:5 Но ты будь бдителен во всем, переноси скорби, совершай дело благовестника, исполняй служение твое.
\rsbpar\vs 2Ti 4:6 Ибо я уже становлюсь жертвою, и время моего отшествия настало.
\vs 2Ti 4:7 Подвигом добрым я подвизался, течение совершил, веру сохранил;
\vs 2Ti 4:8 а теперь готовится мне венец правды, который даст мне Господь, праведный Судия, в день оный; и не только мне, но и всем, возлюбившим явление Его.
\rsbpar\vs 2Ti 4:9 Постарайся прийти ко мне скоро.
\vs 2Ti 4:10 Ибо Димас оставил меня, возлюбив нынешний век, и пошел в Фессалонику, Крискент в Галатию, Тит в Далматию; один Лука со мною.
\vs 2Ti 4:11 Марка возьми и приведи с собою, ибо он мне нужен для служения.
\vs 2Ti 4:12 Тихика я послал в Ефес.
\vs 2Ti 4:13 Когда пойдешь, принеси фелонь, который я оставил в Троаде у Карпа, и книги, особенно кожаные.
\vs 2Ti 4:14 Александр медник много сделал мне зла. Да воздаст ему Господь по делам его!
\vs 2Ti 4:15 Берегись его и ты, ибо он сильно противился нашим словам.
\rsbpar\vs 2Ti 4:16 При первом моем ответе никого не было со мною, но все меня оставили. Да не вменится им!
\vs 2Ti 4:17 Господь же предстал мне и укрепил меня, дабы через меня утвердилось благовестие и услышали все язычники; и я избавился из львиных челюстей.
\vs 2Ti 4:18 И избавит меня Господь от всякого злого дела и сохранит для Своего Небесного Царства, Ему слава во веки веков. Аминь.
\rsbpar\vs 2Ti 4:19 Приветствуй Прискиллу и Акилу и дом Онисифоров.
\vs 2Ti 4:20 Ераст остался в Коринфе; Трофима же я оставил больного в Милите.
\vs 2Ti 4:21 Постарайся прийти до зимы. Приветствуют тебя Еввул, и Пуд, и Лин, и Клавдия, и все братия.
\rsbpar\vs 2Ti 4:22 Господь Иисус Христос со духом твоим. Благодать с вами. Аминь.

\bibbookdescr{Tit}{
  inline={Послание к Титу\\\LARGE Святого Апостола Павла},
  toc={к Титу},
  bookmark={к Титу},
  header={к Титу},
  %headerleft={},
  %headerright={},
  abbr={Тит}
}
\vs Tit 1:1 Павел, раб Божий, Апостол же Иисуса Христа, по вере избранных Божиих и познанию истины, \bibemph{относящейся} к благочестию,
\vs Tit 1:2 в надежде вечной жизни, которую обещал неизменный в слове Бог прежде вековых времен,
\vs Tit 1:3 а в свое время явил Свое слово в проповеди, вверенной мне по повелению Спасителя нашего, Бога,~---
\vs Tit 1:4 Титу, истинному сыну по общей вере: благодать, милость и мир от Бога Отца и Господа Иисуса Христа, Спасителя нашего.
\rsbpar\vs Tit 1:5 Для того я оставил тебя в Крите, чтобы ты довершил недоконченное и поставил по всем городам пресвитеров, как я тебе приказывал:
\vs Tit 1:6 если кто непорочен, муж одной жены, детей имеет верных, не укоряемых в распутстве или непокорности.
\vs Tit 1:7 Ибо епископ должен быть непорочен, как Божий домостроитель, не дерзок, не гневлив, не пьяница, не бийца, не корыстолюбец,
\vs Tit 1:8 но страннолюбив, любящий добро, целомудрен, справедлив, благочестив, воздержан,
\vs Tit 1:9 держащийся истинного слова, согласного с учением, чтобы он был силен и наставлять в здравом учении и противящихся обличать.
\rsbpar\vs Tit 1:10 Ибо есть много и непокорных, пустословов и обманщиков, особенно из обрезанных,
\vs Tit 1:11 каковым должно заграждать уста: они развращают целые домы, уча, чему не должно, из постыдной корысти.
\vs Tit 1:12 Из них же самих один стихотворец сказал: <<Критяне всегда лжецы, злые звери, утробы ленивые>>.
\vs Tit 1:13 Свидетельство это справедливо. По сей причине обличай их строго, дабы они были здравы в вере,
\vs Tit 1:14 не внимая Иудейским басням и постановлениям людей, отвращающихся от истины.
\vs Tit 1:15 Для чистых все чисто; а для оскверненных и неверных нет ничего чистого, но осквернены и ум их и совесть.
\vs Tit 1:16 Они говорят, что знают Бога, а делами отрекаются, будучи гнусны и непокорны и не способны ни к какому доброму делу.
\vs Tit 2:1 Ты же говори то, что сообразно с здравым учением:
\vs Tit 2:2 чтобы старцы были бдительны, степенны, целомудренны, здравы в вере, в любви, в терпении;
\vs Tit 2:3 чтобы старицы также одевались прилично святым, не были клеветницы, не порабощались пьянству, учили добру;
\vs Tit 2:4 чтобы вразумляли молодых любить мужей, любить детей,
\vs Tit 2:5 быть целомудренными, чистыми, попечительными о доме, добрыми, покорными своим мужьям, да не порицается слово Божие.
\vs Tit 2:6 Юношей также увещевай быть целомудренными.
\vs Tit 2:7 Во всем показывай в себе образец добрых дел, в учительстве чистоту, степенность, неповрежденность,
\vs Tit 2:8 слово здравое, неукоризненное, чтобы противник был посрамлен, не имея ничего сказать о нас худого.
\vs Tit 2:9 Рабов \bibemph{увещевай} повиноваться своим господам, угождать им во всем, не прекословить,
\vs Tit 2:10 не красть, но оказывать всю добрую верность, дабы они во всем были украшением учению Спасителя нашего, Бога.
\vs Tit 2:11 Ибо явилась благодать Божия, спасительная для всех человеков,
\vs Tit 2:12 научающая нас, чтобы мы, отвергнув нечестие и мирские похоти, целомудренно, праведно и благочестиво жили в нынешнем веке,
\vs Tit 2:13 ожидая блаженного упования и явления славы великого Бога и Спасителя нашего Иисуса Христа,
\vs Tit 2:14 Который дал Себя за нас, чтобы избавить нас от всякого беззакония и очистить Себе народ особенный, ревностный к добрым делам.
\rsbpar\vs Tit 2:15 Сие говори, увещевай и обличай со всякою властью, чтобы никто не пренебрегал тебя.
\vs Tit 3:1 Напоминай им повиноваться и покоряться начальству и властям, быть готовыми на всякое доброе дело,
\vs Tit 3:2 никого не злословить, быть не сварливыми, но тихими, и оказывать всякую кротость ко всем человекам.
\vs Tit 3:3 Ибо и мы были некогда несмысленны, непокорны, заблуждшие, были рабы похотей и различных удовольствий, жили в злобе и зависти, были гнусны, ненавидели друг друга.
\vs Tit 3:4 Когда же явилась благодать и человеколюбие Спасителя нашего, Бога,
\vs Tit 3:5 Он спас нас не по делам праведности, которые бы мы сотворили, а по Своей милости, банею возрождения и обновления Святым Духом,
\vs Tit 3:6 Которого излил на нас обильно через Иисуса Христа, Спасителя нашего,
\vs Tit 3:7 чтобы, оправдавшись Его благодатью, мы по упованию соделались наследниками вечной жизни.
\vs Tit 3:8 Слово это верно; и я желаю, чтобы ты подтверждал о сем, дабы уверовавшие в Бога старались быть прилежными к добрым делам: это хорошо и полезно человекам.
\vs Tit 3:9 Глупых же состязаний и родословий, и споров и распрей о законе удаляйся, ибо они бесполезны и суетны.
\vs Tit 3:10 Еретика, после первого и второго вразумления, отвращайся,
\vs Tit 3:11 зная, что таковой развратился и грешит, будучи самоосужден.
\rsbpar\vs Tit 3:12 Когда пришлю к тебе Артему или Тихика, поспеши прийти ко мне в Никополь, ибо я положил там провести зиму.
\vs Tit 3:13 Зину законника и Аполлоса позаботься отправить так, чтобы у них ни в чем не было недостатка.
\vs Tit 3:14 Пусть и наши учатся упражняться в добрых делах, \bibemph{в удовлетворении} необходимым нуждам, дабы не были бесплодны.
\rsbpar\vs Tit 3:15 Приветствуют тебя все находящиеся со мною. Приветствуй любящих нас в вере. Благодать со всеми вами. Аминь.

\bibbookdescr{Phm}{
  inline={Послание к Филимону\\\LARGE Святого Апостола Павла},
  toc={к Филимону},
  bookmark={к Филимону},
  header={к Филимону},
  %headerleft={},
  %headerright={},
  abbr={Флм}
}
\vs Phm 1:1 Павел, узник Иисуса Христа, и Тимофей брат, Филимону возлюбленному и сотруднику нашему,
\vs Phm 1:2 и Апфии, (сестре) возлюбленной, и Архиппу, сподвижнику нашему, и домашней твоей церкви:
\vs Phm 1:3 благодать вам и мир от Бога Отца нашего и Господа Иисуса Христа.
\rsbpar\vs Phm 1:4 Благодарю Бога моего, всегда вспоминая о тебе в молитвах моих,
\vs Phm 1:5 слыша о твоей любви и вере, которую имеешь к Господу Иисусу и ко всем святым,
\vs Phm 1:6 дабы общение веры твоей оказалось деятельным в познании всякого у вас добра во Христе Иисусе.
\vs Phm 1:7 Ибо мы имеем великую радость и утешение в любви твоей, потому что тобою, брат, успокоены сердца святых.
\rsbpar\vs Phm 1:8 Посему, имея великое во Христе дерзновение приказывать тебе, что должно,
\vs Phm 1:9 по любви лучше прошу, не иной кто, как я, Павел старец, а теперь и узник Иисуса Христа;
\vs Phm 1:10 прошу тебя о сыне моем Онисиме, которого родил я в узах моих:
\vs Phm 1:11 он был некогда негоден для тебя, а теперь годен тебе и мне; я возвращаю его;
\vs Phm 1:12 ты же прими его, как мое сердце.
\vs Phm 1:13 Я хотел при себе удержать его, дабы он вместо тебя послужил мне в узах \bibemph{за} благовествование;
\vs Phm 1:14 но без твоего согласия ничего не хотел сделать, чтобы доброе дело твое было не вынужденно, а добровольно.
\vs Phm 1:15 Ибо, может быть, он для того на время отлучился, чтобы тебе принять его навсегда,
\vs Phm 1:16 не как уже раба, но выше раба, брата возлюбленного, особенно мне, а тем больше тебе, и по плоти и в Господе.
\vs Phm 1:17 Итак, если ты имеешь общение со мною, то прими его, как меня.
\vs Phm 1:18 Если же он чем обидел тебя, или должен, считай это на мне.
\vs Phm 1:19 Я, Павел, написал моею рукою: я заплач\acc{у}; не говорю тебе о том, что ты и самим собою мне должен.
\vs Phm 1:20 Так, брат, дай мне воспользоваться от тебя в Господе; успокой мое сердце в Господе.
\vs Phm 1:21 Надеясь на послушание твое, я написал к тебе, зная, что ты сделаешь и более, нежели говорю.
\vs Phm 1:22 А вместе приготовь для меня и помещение; ибо надеюсь, что по молитвам вашим я буду дарован вам.
\rsbpar\vs Phm 1:23 Приветствует тебя Епафрас, узник вместе со мною ради Христа Иисуса,
\vs Phm 1:24 Марк, Аристарх, Димас, Лука, сотрудники мои.
\rsbpar\vs Phm 1:25 Благодать Господа нашего Иисуса Христа со духом вашим. Аминь.

\bibbookdescr{Heb}{
  inline={Послание к Евреям\\\LARGE Святого Апостола Павла},
  toc={к Евреям},
  bookmark={к Евреям},
  header={к Евреям},
  %headerleft={},
  %headerright={},
  abbr={Евр}
}
\vs Heb 1:1 Бог, многократно и многообразно говоривший издревле отцам в пророках,
\vs Heb 1:2 в последние дни сии говорил нам в Сыне, Которого поставил наследником всего, чрез Которого и веки сотворил.
\vs Heb 1:3 Сей, будучи сияние славы и образ ипостаси Его и держа все словом силы Своей, совершив Собою очищение грехов наших, воссел одесную (престола) величия на высоте,
\vs Heb 1:4 будучи столько превосходнее Ангелов, сколько славнейшее пред ними наследовал имя.
\vs Heb 1:5 Ибо кому когда из Ангелов сказал \bibemph{Бог}: Ты Сын Мой, Я ныне родил Тебя? И еще: Я буду Ему Отцем, и Он будет Мне Сыном?
\vs Heb 1:6 Также, когда вводит Первородного во вселенную, говорит: и да поклонятся Ему все Ангелы Божии.
\vs Heb 1:7 Об Ангелах сказано: Ты творишь Ангелами Своими духов и служителями Своими пламенеющий огонь.
\vs Heb 1:8 А о Сыне: престол Твой, Боже, в век века; жезл царствия Твоего~--- жезл правоты.
\vs Heb 1:9 Ты возлюбил правду и возненавидел беззаконие, посему помазал Тебя, Боже, Бог Твой елеем радости более соучастников Твоих.
\vs Heb 1:10 И: в начале Ты, Господи, основал землю, и небеса~--- дело рук Твоих;
\vs Heb 1:11 они погибнут, а Ты пребываешь; и все обветшают, как риза,
\vs Heb 1:12 и как одежду свернешь их, и изменятся; но Ты тот же, и лета Твои не кончатся.
\vs Heb 1:13 Кому когда из Ангелов сказал \bibemph{Бог}: седи одесную Меня, доколе положу врагов Твоих в подножие ног Твоих?
\vs Heb 1:14 Не все ли они суть служебные духи, посылаемые на служение для тех, которые имеют наследовать спасение?
\vs Heb 2:1 Посему мы должны быть особенно внимательны к слышанному, чтобы не отпасть.
\vs Heb 2:2 Ибо, если через Ангелов возвещенное слово было твердо, и всякое преступление и непослушание получало праведное воздаяние,
\vs Heb 2:3 то как мы избежим, вознерадев о толиком спасении, которое, быв сначала проповедано Господом, в нас утвердилось слышавшими \bibemph{от Него},
\vs Heb 2:4 при засвидетельствовании от Бога знамениями и чудесами, и различными силами, и раздаянием Духа Святаго по Его воле?
\rsbpar\vs Heb 2:5 Ибо не Ангелам Бог покорил будущую вселенную, о которой говорим;
\vs Heb 2:6 напротив некто негде засвидетельствовал, говоря: что значит человек, что Ты помнишь его? или сын человеческий, что Ты посещаешь его?
\vs Heb 2:7 Не много Ты унизил его пред Ангелами; славою и честью увенчал его, и поставил его над делами рук Твоих,
\vs Heb 2:8 все покорил под ноги его. Когда же покорил ему все, то не оставил ничего непокоренным ему. Ныне же еще не видим, чтобы все было ему покорено;
\vs Heb 2:9 но видим, что за претерпение смерти увенчан славою и честью Иисус, Который не много был унижен пред Ангелами, дабы Ему, по благодати Божией, вкусить смерть за всех.
\vs Heb 2:10 Ибо надлежало, чтобы Тот, для Которого все и от Которого все, приводящего многих сынов в славу, вождя спасения их совершил через страдания.
\vs Heb 2:11 Ибо и освящающий и освящаемые, все~--- от Единого; поэтому Он не стыдится называть их братиями, говоря:
\vs Heb 2:12 возвещу имя Твое братиям Моим, посреди церкви воспою Тебя.
\vs Heb 2:13 И еще: Я буду уповать на Него. И еще: вот Я и дети, которых дал Мне Бог.
\vs Heb 2:14 А как дети причастны плоти и крови, то и Он также воспринял оные, дабы смертью лишить силы имеющего державу смерти, то есть диавола,
\vs Heb 2:15 и избавить тех, которые от страха смерти через всю жизнь были подвержены рабству.
\vs Heb 2:16 Ибо не Ангелов восприемлет Он, но восприемлет семя Авраамово.
\vs Heb 2:17 Посему Он должен был во всем уподобиться братиям, чтобы быть милостивым и верным первосвященником пред Богом, для умилостивления за грехи народа.
\vs Heb 2:18 Ибо, как Сам Он претерпел, быв искушен, то может и искушаемым помочь.
\vs Heb 3:1 Итак, братия святые, участники в небесном звании, уразумейте Посланника и Первосвященника исповедания нашего, Иисуса Христа,
\vs Heb 3:2 Который верен Поставившему Его, как и Моисей во всем доме Его.
\vs Heb 3:3 Ибо Он достоин тем большей славы пред Моисеем, чем б\acc{о}льшую честь имеет в сравнении с домом тот, кто устроил его,
\vs Heb 3:4 ибо всякий дом устрояется кем-либо; а устроивший всё \bibemph{есть} Бог.
\vs Heb 3:5 И Моисей верен во всем доме Его, как служитель, для засвидетельствования того, что надлежало возвестить;
\vs Heb 3:6 а Христос~--- как Сын в доме Его; дом же Его~--- мы, если только дерзновение и упование, которым хвалимся, твердо сохраним до конца.
\vs Heb 3:7 Почему, как говорит Дух Святый, ныне, когда услышите глас Его,
\vs Heb 3:8 не ожесточите сердец ваших, как во время ропота, в день искушения в пустыне,
\vs Heb 3:9 где искушали Меня отцы ваши, испытывали Меня, и видели дела Мои сорок лет.
\vs Heb 3:10 Посему Я вознегодовал на оный род и сказал: непрестанно заблуждаются сердцем, не познали они путей Моих;
\vs Heb 3:11 посему Я поклялся во гневе Моем, что они не войдут в покой Мой.
\rsbpar\vs Heb 3:12 Смотрите, братия, чтобы не было в ком из вас сердца лукавого и неверного, дабы вам не отступить от Бога живаго.
\vs Heb 3:13 Но наставляйте друг друга каждый день, доколе можно говорить: <<ныне>>, чтобы кто из вас не ожесточился, обольстившись грехом.
\vs Heb 3:14 Ибо мы сделались причастниками Христу, если только начатую жизнь твердо сохраним до конца,
\vs Heb 3:15 доколе говорится: <<ныне, когда услышите глас Его, не ожесточите сердец ваших, как во время ропота>>.
\vs Heb 3:16 Ибо некоторые из слышавших возроптали; но не все вышедшие из Египта с Моисеем.
\vs Heb 3:17 На кого же негодовал Он сорок лет? Не на согрешивших ли, которых кости пали в пустыне?
\vs Heb 3:18 Против кого же клялся, что не войдут в покой Его, как не против непокорных?
\vs Heb 3:19 Итак видим, что они не могли войти за неверие.
\vs Heb 4:1 Посему будем опасаться, чтобы, когда еще остается обетование войти в покой Его, не оказался кто из вас опоздавшим.
\vs Heb 4:2 Ибо и нам оно возвещено, как и тем; но не принесло им пользы слово слышанное, не растворенное верою слышавших.
\vs Heb 4:3 А входим в покой мы уверовавшие, так как Он сказал: <<Я поклялся в гневе Моем, что они не войдут в покой Мой>>, хотя дела \bibemph{Его} были совершены еще в начале мира.
\vs Heb 4:4 Ибо негде сказано о седьмом \bibemph{дне} так: и почил Бог в день седьмый от всех дел Своих.
\vs Heb 4:5 И еще здесь: <<не войдут в покой Мой>>.
\vs Heb 4:6 Итак, как некоторым остается войти в него, а те, которым прежде возвещено, не вошли в него за непокорность,
\vs Heb 4:7 \bibemph{то} еще определяет некоторый день, <<ныне>>, говоря через Давида, после столь долгого времени, как выше сказано: <<ныне, когда услышите глас Его, не ожесточите сердец ваших>>.
\vs Heb 4:8 Ибо если бы Иисус \bibemph{Навин} доставил им покой, то не было бы сказано после того о другом дне.
\vs Heb 4:9 Посему для народа Божия еще остается субботство.
\vs Heb 4:10 Ибо, кто вошел в покой Его, тот и сам успокоился от дел своих, как и Бог от Своих.
\rsbpar\vs Heb 4:11 Итак постараемся войти в покой оный, чтобы кто по тому же примеру не впал в непокорность.
\vs Heb 4:12 Ибо слово Божие живо и действенно и острее всякого меча обоюдоострого: оно проникает до разделения души и духа, составов и мозгов, и судит помышления и намерения сердечные.
\vs Heb 4:13 И нет твари, сокровенной от Него, но все обнажено и открыто перед очами Его: Ему дадим отчет.
\rsbpar\vs Heb 4:14 Итак, имея Первосвященника великого, прошедшего небеса, Иисуса Сына Божия, будем твердо держаться исповедания \bibemph{нашего}.
\vs Heb 4:15 Ибо мы имеем не такого первосвященника, который не может сострадать нам в немощах наших, но Который, подобно \bibemph{нам}, искушен во всем, кроме греха.
\vs Heb 4:16 Посему да приступаем с дерзновением к престолу благодати, чтобы получить милость и обрести благодать для благовременной помощи.
\vs Heb 5:1 Ибо всякий первосвященник, из человеков избираемый, для человеков поставляется на служение Богу, чтобы приносить дары и жертвы за грехи,
\vs Heb 5:2 могущий снисходить невежествующим и заблуждающим, потому что и сам обложен немощью,
\vs Heb 5:3 и посему он должен как за народ, так и за себя приносить \bibemph{жертвы} о грехах.
\vs Heb 5:4 И никто сам собою не приемлет этой чести, но призываемый Богом, как и Аарон.
\vs Heb 5:5 Так и Христос не Сам Себе присвоил славу быть первосвященником, но Тот, Кто сказал Ему: Ты Сын Мой, Я ныне родил Тебя;
\vs Heb 5:6 как и в другом \bibemph{месте} говорит: Ты священник вовек по чину Мелхиседека.
\vs Heb 5:7 Он, во дни плоти Своей, с сильным воплем и со слезами принес молитвы и моления Могущему спасти Его от смерти; и услышан был за \bibemph{Свое} благоговение;
\vs Heb 5:8 хотя Он и Сын, однако страданиями навык послушанию,
\vs Heb 5:9 и, совершившись, сделался для всех послушных Ему виновником спасения вечного,
\vs Heb 5:10 быв наречен от Бога Первосвященником по чину Мелхиседека.
\rsbpar\vs Heb 5:11 О сем надлежало бы нам говорить много; но трудно истолковать, потому что вы сделались неспособны слушать.
\vs Heb 5:12 Ибо, \bibemph{судя} по времени, вам надлежало быть учителями; но вас снова нужно учить первым началам слова Божия, и для вас нужно молоко, а не твердая пища.
\vs Heb 5:13 Всякий, питаемый молоком, несведущ в слове правды, потому что он младенец;
\vs Heb 5:14 твердая же пища свойственна совершенным, у которых чувства навыком приучены к различению добра и зла.
\vs Heb 6:1 Посему, оставив начатки учения Христова, поспешим к совершенству; и не станем снова полагать основание обращению от мертвых дел и вере в Бога,
\vs Heb 6:2 учению о крещениях, о возложении рук, о воскресении мертвых и о суде вечном.
\vs Heb 6:3 И это сделаем, если Бог позволит.
\vs Heb 6:4 Ибо невозможно~--- однажды просвещенных, и вкусивших дара небесного, и соделавшихся причастниками Духа Святаго,
\vs Heb 6:5 и вкусивших благого глагола Божия и сил будущего века,
\vs Heb 6:6 и отпадших, опять обновлять покаянием, когда они снова распинают в себе Сына Божия и ругаются \bibemph{Ему}.
\vs Heb 6:7 Земля, пившая многократно сходящий на нее дождь и произращающая злак, полезный тем, для которых и возделывается, получает благословение от Бога;
\vs Heb 6:8 а производящая терния и волчцы негодна и близка к проклятию, которого конец~--- сожжение.
\rsbpar\vs Heb 6:9 Впрочем о вас, возлюбленные, мы надеемся, что вы в лучшем \bibemph{состоянии} и держитесь спасения, хотя и говорим так.
\vs Heb 6:10 Ибо не неправеден Бог, чтобы забыл дело ваше и труд любви, которую вы оказали во имя Его, послужив и служа святым.
\vs Heb 6:11 Желаем же, чтобы каждый из вас, для совершенной уверенности в надежде, оказывал такую же ревность до конца,
\vs Heb 6:12 дабы вы не обленились, но подражали тем, которые верою и долготерпением наследуют обетования.
\vs Heb 6:13 Бог, давая обетование Аврааму, как не мог никем высшим клясться, клялся Самим Собою,
\vs Heb 6:14 говоря: истинно благословляя благословлю тебя и размножая размножу тебя.
\vs Heb 6:15 И так Авраам, долготерпев, получил обещанное.
\vs Heb 6:16 Люди клянутся высшим, и клятва во удостоверение оканчивает всякий спор их.
\vs Heb 6:17 Посему и Бог, желая преимущественнее показать наследникам обетования непреложность Своей воли, употребил в посредство клятву,
\vs Heb 6:18 дабы в двух непреложных вещах, в которых невозможно Богу солгать, твердое утешение имели мы, прибегшие взяться за предлежащую надежду,
\vs Heb 6:19 которая для души есть как бы якорь безопасный и крепкий, и входит во внутреннейшее за завесу,
\vs Heb 6:20 куда предтечею за нас вошел Иисус, сделавшись Первосвященником навек по чину Мелхиседека.
\vs Heb 7:1 Ибо Мелхиседек, царь Салима, священник Бога Всевышнего, тот, который встретил Авраама и благословил его, возвращающегося после поражения царей,
\vs Heb 7:2 которому и десятину отделил Авраам от всего,~--- во-первых, по знаменованию \bibemph{имени} царь правды, а потом и царь Салима, то есть царь мира,
\vs Heb 7:3 без отца, без матери, без родословия, не имеющий ни начала дней, ни конца жизни, уподобляясь Сыну Божию, пребывает священником навсегда.
\rsbpar\vs Heb 7:4 Видите, как велик тот, которому и Авраам патриарх дал десятину из лучших добыч своих.
\vs Heb 7:5 Получающие священство из сынов Левииных имеют заповедь~--- брать по закону десятину с народа, то есть со своих братьев, хотя и сии произошли от чресл Авраамовых.
\vs Heb 7:6 Но сей, не происходящий от рода их, получил десятину от Авраама и благословил имевшего обетования.
\vs Heb 7:7 Без всякого же прекословия меньший благословляется б\acc{о}льшим.
\vs Heb 7:8 И здесь десятины берут человеки смертные, а там~--- имеющий о себе свидетельство, что он живет.
\vs Heb 7:9 И, так сказать, сам Левий, принимающий десятины, в \bibemph{лице} Авраама дал десятину:
\vs Heb 7:10 ибо он был еще в чреслах отца, когда Мелхиседек встретил его.
\rsbpar\vs Heb 7:11 Итак, если бы совершенство достигалось посредством левитского священства,~--- ибо с ним сопряжен закон народа,~--- то какая бы еще нужда была восставать иному священнику по чину Мелхиседека, а не по чину Аарона именоваться?
\vs Heb 7:12 Потому что с переменою священства необходимо быть перемене и закона.
\vs Heb 7:13 Ибо Тот, о Котором говорится сие, принадлежал к иному колену, из которого никто не приступал к жертвеннику.
\vs Heb 7:14 Ибо известно, что Господь наш воссиял из колена Иудина, о котором Моисей ничего не сказал относительно священства.
\vs Heb 7:15 И это еще яснее видно \bibemph{из того}, что по подобию Мелхиседека восстает Священник иной,
\vs Heb 7:16 Который таков не по закону заповеди плотской, но по силе жизни непрестающей.
\vs Heb 7:17 Ибо засвидетельствовано: Ты священник вовек по чину Мелхиседека.
\vs Heb 7:18 Отменение же прежде бывшей заповеди бывает по причине ее немощи и бесполезности,
\vs Heb 7:19 ибо закон ничего не довел до совершенства; но вводится лучшая надежда, посредством которой мы приближаемся к Богу.
\vs Heb 7:20 И как \bibemph{сие было} не без клятвы,~---
\vs Heb 7:21 ибо те были священниками без клятвы, а Сей с клятвою, потому что о Нем сказано: клялся Господь, и не раскается: Ты священник вовек по чину Мелхиседека,~---
\vs Heb 7:22 то лучшего завета поручителем соделался Иисус.
\vs Heb 7:23 Притом тех священников было много, потому что смерть не допускала пребывать одному;
\vs Heb 7:24 а Сей, как пребывающий вечно, имеет и священство непреходящее,
\vs Heb 7:25 посему и может всегда спасать приходящих чрез Него к Богу, будучи всегда жив, чтобы ходатайствовать за них.
\rsbpar\vs Heb 7:26 Таков и должен быть у нас Первосвященник: святой, непричастный злу, непорочный, отделенный от грешников и превознесенный выше небес,
\vs Heb 7:27 Который не имеет нужды ежедневно, как те первосвященники, приносить жертвы сперва за свои грехи, потом за грехи народа, ибо Он совершил это однажды, принеся \bibemph{в жертву} Себя Самого.
\vs Heb 7:28 Ибо закон поставляет первосвященниками человеков, имеющих немощи; а слово клятвенное, после закона, \bibemph{поставило} Сына, на веки совершенного.
\vs Heb 8:1 Главное же в том, о чем говорим, есть то: мы имеем такого Первосвященника, Который воссел одесную престола величия на небесах
\vs Heb 8:2 и \bibemph{есть} священнодействователь святилища и скинии истинной, которую воздвиг Господь, а не человек.
\vs Heb 8:3 Всякий первосвященник поставляется для приношения даров и жертв; а потому нужно было, чтобы и Сей также имел, что принести.
\vs Heb 8:4 Если бы Он оставался на земле, то не был бы и священником, потому что \bibemph{здесь} такие священники, которые по закону приносят дары,
\vs Heb 8:5 которые служат образу и тени небесного, как сказано было Моисею, когда он приступал к совершению скинии: смотри, сказано, сделай все по образу, показанному тебе на горе.
\vs Heb 8:6 Но Сей \bibemph{Первосвященник} получил служение тем превосходнейшее, чем лучшего Он ходатай завета, который утвержден на лучших обетованиях.
\vs Heb 8:7 Ибо, если бы первый \bibemph{завет} был без недостатка, то не было бы нужды искать места другому.
\vs Heb 8:8 Но \bibemph{пророк}, укоряя их, говорит: вот, наступают дни, говорит Господь, когда Я заключу с домом Израиля и с домом Иуды новый завет,
\vs Heb 8:9 не такой завет, какой Я заключил с отцами их в то время, когда взял их за руку, чтобы вывести их из земли Египетской, потому что они не пребыли в том завете Моем, и Я пренебрег их, говорит Господь.
\vs Heb 8:10 Вот завет, который завещаю дому Израилеву после тех дней, говорит Господь: вложу законы Мои в мысли их, и напишу их на сердцах их; и буду их Богом, а они будут Моим народом.
\vs Heb 8:11 И не будет учить каждый ближнего своего и каждый брата своего, говоря: познай Господа; потому что все, от малого до большого, будут знать Меня,
\vs Heb 8:12 потому что Я буду милостив к неправдам их, и грехов их и беззаконий их не воспомяну более.
\vs Heb 8:13 Говоря <<новый>>, показал ветхость первого; а ветшающее и стареющее близко к уничтожению.
\vs Heb 9:1 И первый завет имел постановление о богослужении и святилище земное:
\vs Heb 9:2 ибо устроена была скиния первая, в которой был светильник, и трапеза, и предложение хлебов, и которая называется Святое.
\vs Heb 9:3 За второю же завесою была скиния, называемая Святое Святых,
\vs Heb 9:4 имевшая золотую кадильницу и обложенный со всех сторон золотом ковчег завета, где были золотой сосуд с манною, жезл Ааронов расцветший и скрижали завета,
\vs Heb 9:5 а над ним херувимы славы, осеняющие очистилище; о чем не нужно теперь говорить подробно.
\vs Heb 9:6 При таком устройстве, в первую скинию всегда входят священники совершать богослужение;
\vs Heb 9:7 а во вторую~--- однажды в год один только первосвященник, не без крови, которую приносит за себя и за грехи неведения народа.
\vs Heb 9:8 \bibemph{Сим} Дух Святый показывает, что еще не открыт путь во святилище, доколе сто\acc{и}т прежняя скиния.
\vs Heb 9:9 Она есть образ настоящего времени, в которое приносятся дары и жертвы, не могущие сделать в совести совершенным приносящего,
\vs Heb 9:10 и которые с яствами и питиями, и различными омовениями и обрядами, \bibemph{относящимися} до плоти, установлены были только до времени исправления.
\rsbpar\vs Heb 9:11 Но Христос, Первосвященник будущих благ, придя с большею и совершеннейшею скиниею, нерукотворенною, то есть не такового устроения,
\vs Heb 9:12 и не с кровью козлов и тельцов, но со Своею Кровию, однажды вошел во святилище и приобрел вечное искупление.
\vs Heb 9:13 Ибо если кровь тельцов и козлов и пепел телицы, через окропление, освящает оскверненных, дабы чисто было тело,
\vs Heb 9:14 то кольми паче Кровь Христа, Который Духом Святым принес Себя непорочного Богу, очистит совесть нашу от мертвых дел, для служения Богу живому и истинному!
\vs Heb 9:15 И потому Он есть ходатай нового завета, дабы вследствие смерти \bibemph{Его}, бывшей для искупления от преступлений, сделанных в первом завете, призванные к вечному наследию получили обетованное.
\vs Heb 9:16 Ибо, где завещание, там необходимо, чтобы последовала смерть завещателя,
\vs Heb 9:17 потому что завещание действительно после умерших: оно не имеет силы, когда завещатель жив.
\vs Heb 9:18 Почему и первый \bibemph{завет} был утвержден не без крови.
\vs Heb 9:19 Ибо Моисей, произнеся все заповеди по закону перед всем народом, взял кровь тельцов и козлов с водою и шерстью червленою и иссопом, и окропил как самую книгу, так и весь народ,
\vs Heb 9:20 говоря: это кровь завета, который заповедал вам Бог.
\vs Heb 9:21 Также окропил кровью и скинию и все сосуды богослужебные.
\vs Heb 9:22 Да и все почти по закону очищается кровью, и без пролития крови не бывает прощения.
\rsbpar\vs Heb 9:23 Итак образы небесного должны были очищаться сими, самое же небесное лучшими сих жертвами.
\vs Heb 9:24 Ибо Христос вошел не в рукотворенное святилище, по образу истинного \bibemph{устроенное}, но в самое небо, чтобы предстать ныне за нас пред лице Божие,
\vs Heb 9:25 и не для того, чтобы многократно приносить Себя, как первосвященник входит во святилище каждогодно с чужою кровью;
\vs Heb 9:26 иначе надлежало бы Ему многократно страдать от начала мира; Он же однажды, к концу веков, явился для уничтожения греха жертвою Своею.
\vs Heb 9:27 И как человекам положено однажды умереть, а потом суд,
\vs Heb 9:28 так и Христос, однажды принеся Себя в жертву, чтобы подъять грехи многих, во второй раз явится не \bibemph{для очищения} греха, а для ожидающих Его во спасение.
\vs Heb 10:1 Закон, имея тень будущих благ, а не самый образ вещей, одними и теми же жертвами, каждый год постоянно приносимыми, никогда не может сделать совершенными приходящих \bibemph{с ними}.
\vs Heb 10:2 Иначе перестали бы приносить \bibemph{их}, потому что приносящие жертву, быв очищены однажды, не имели бы уже никакого сознания грехов.
\vs Heb 10:3 Но жертвами каждогодно напоминается о грехах,
\vs Heb 10:4 ибо невозможно, чтобы кровь тельцов и козлов уничтожала грехи.
\vs Heb 10:5 Посему \bibemph{Христос}, входя в мир, говорит: жертвы и приношения Ты не восхотел, но тело уготовал Мне.
\vs Heb 10:6 Всесожжения и \bibemph{жертвы} за грех неугодны Тебе.
\vs Heb 10:7 Тогда Я сказал: вот, иду, \bibemph{как} в начале книги написано о Мне, исполнить волю Твою, Боже.
\vs Heb 10:8 Сказав прежде, что <<ни жертвы, ни приношения, ни всесожжений, ни \bibemph{жертвы} за грех,~--- которые приносятся по закону,~--- Ты не восхотел и не благоизволил>>,
\vs Heb 10:9 потом прибавил: <<вот, иду исполнить волю Твою, Боже>>. Отменяет первое, чтобы постановить второе.
\vs Heb 10:10 По сей-то воле освящены мы единократным принесением тела Иисуса Христа.
\vs Heb 10:11 И всякий священник ежедневно сто\acc{и}т в служении, и многократно приносит одни и те же жертвы, которые никогда не могут истребить грехов.
\vs Heb 10:12 Он же, принеся одну жертву за грехи, навсегда воссел одесную Бога,
\vs Heb 10:13 ожидая затем, доколе враги Его будут положены в подножие ног Его.
\vs Heb 10:14 Ибо Он одним приношением навсегда сделал совершенными освящаемых.
\vs Heb 10:15 \bibemph{О сем} свидетельствует нам и Дух Святый; ибо сказано:
\vs Heb 10:16 Вот завет, который завещаю им после тех дней, говорит Господь: вложу законы Мои в сердца их, и в мыслях их напишу их,
\vs Heb 10:17 и грехов их и беззаконий их не воспомяну более.
\vs Heb 10:18 А где прощение грехов, там не нужно приношение за них.
\rsbpar\vs Heb 10:19 Итак, братия, имея дерзновение входить во святилище посредством Крови Иисуса Христа, путем новым и живым,
\vs Heb 10:20 который Он вновь открыл нам через завесу, то есть плоть Свою,
\vs Heb 10:21 и \bibemph{имея} великого Священника над домом Божиим,
\vs Heb 10:22 да приступаем с искренним сердцем, с полною верою, кроплением очистив сердца от порочной совести, и омыв тело водою чистою,
\vs Heb 10:23 будем держаться исповедания упования неуклонно, ибо верен Обещавший.
\vs Heb 10:24 Будем внимательны друг ко другу, поощряя к любви и добрым делам.
\vs Heb 10:25 Не будем оставлять собрания своего, как есть у некоторых обычай; но будем увещевать \bibemph{друг друга}, и тем более, чем более усматриваете приближение дня оного.
\rsbpar\vs Heb 10:26 Ибо если мы, получив познание истины, произвольно грешим, то не остается более жертвы за грехи,
\vs Heb 10:27 но некое страшное ожидание суда и ярость огня, готового пожрать противников.
\vs Heb 10:28 \bibemph{Если} отвергшийся закона Моисеева, при двух или трех свидетелях, без милосердия \bibemph{наказывается} смертью,
\vs Heb 10:29 то сколь тягчайшему, думаете, наказанию повинен будет тот, кто попирает Сына Божия и не почитает за святыню Кровь завета, которою освящен, и Духа благодати оскорбляет?
\vs Heb 10:30 Мы знаем Того, Кто сказал: у Меня отмщение, Я воздам, говорит Господь. И еще: Господь будет судить народ Свой.
\vs Heb 10:31 Страшно впасть в руки Бога живаго!
\rsbpar\vs Heb 10:32 Вспомните прежние дни ваши, когда вы, быв просвещены, выдержали великий подвиг страданий,
\vs Heb 10:33 то сами среди поношений и скорбей служа зрелищем \bibemph{для других}, то принимая участие в других, находившихся в таком же \bibemph{состоянии};
\vs Heb 10:34 ибо вы и моим узам сострадали и расхищение имения вашего приняли с радостью, зная, что есть у вас на небесах имущество лучшее и непреходящее.
\vs Heb 10:35 Итак не оставляйте упования вашего, которому предстоит великое воздаяние.
\vs Heb 10:36 Терпение нужно вам, чтобы, исполнив волю Божию, получить обещанное;
\vs Heb 10:37 ибо еще немного, очень немного, и Грядущий придет и не умедлит.
\vs Heb 10:38 Праведный верою жив будет; а если \bibemph{кто} поколеблется, не благоволит к тому душа Моя.
\vs Heb 10:39 Мы же не из колеблющихся на погибель, но \bibemph{сто\acc{и}м} в вере к спасению души.
\vs Heb 11:1 Вера же есть осуществление ожидаемого и уверенность в невидимом.
\vs Heb 11:2 В ней свидетельствованы древние.
\vs Heb 11:3 Верою познаём, что веки устроены словом Божиим, так что из невидимого произошло видимое.
\vs Heb 11:4 Верою Авель принес Богу жертву лучшую, нежели Каин; ею получил свидетельство, что он праведен, как засвидетельствовал Бог о дарах его; ею он и по смерти говорит еще.
\vs Heb 11:5 Верою Енох переселен был так, что не видел смерти; и не стало его, потому что Бог переселил его. Ибо прежде переселения своего получил он свидетельство, что угодил Богу.
\vs Heb 11:6 А без веры угодить Богу невозможно; ибо надобно, чтобы приходящий к Богу веровал, что Он есть, и ищущим Его воздает.
\vs Heb 11:7 Верою Ной, получив откровение о том, что еще не было видимо, благоговея приготовил ковчег для спасения дома своего; ею осудил он (весь) мир, и сделался наследником праведности по вере.
\vs Heb 11:8 Верою Авраам повиновался призванию идти в страну, которую имел получить в наследие, и пошел, не зная, куда идет.
\vs Heb 11:9 Верою обитал он на земле обетованной, как на чужой, и жил в шатрах с Исааком и Иаковом, сонаследниками того же обетования;
\vs Heb 11:10 ибо он ожидал города, имеющего основание, которого художник и строитель Бог.
\vs Heb 11:11 Верою и сама Сарра (будучи неплодна) получила силу к принятию семени, и не по времени возраста родила, ибо знала, что верен Обещавший.
\vs Heb 11:12 И потому от одного, и притом омертвелого, родилось так много, как \bibemph{много} звезд на небе и как бесчислен песок на берегу морском.
\rsbpar\vs Heb 11:13 Все сии умерли в вере, не получив обетований, а только издали видели оные, и радовались, и говорили о себе, что они странники и пришельцы на земле;
\vs Heb 11:14 ибо те, которые так говорят, показывают, что они ищут отечества.
\vs Heb 11:15 И если бы они в мыслях имели то \bibemph{отечество}, из которого вышли, то имели бы время возвратиться;
\vs Heb 11:16 но они стремились к лучшему, то есть к небесному; посему и Бог не стыдится их, называя Себя их Богом: ибо Он приготовил им город.
\rsbpar\vs Heb 11:17 Верою Авраам, будучи искушаем, принес в жертву Исаака и, имея обетование, принес единородного,
\vs Heb 11:18 о котором было сказано: в Исааке наречется тебе семя.
\vs Heb 11:19 Ибо он думал, что Бог силен и из мертвых воскресить, почему и получил его в предзнаменование.
\vs Heb 11:20 Верою в будущее Исаак благословил Иакова и Исава.
\vs Heb 11:21 Верою Иаков, умирая, благословил каждого сына Иосифова и поклонился на верх жезла своего.
\vs Heb 11:22 Верою Иосиф, при кончине, напоминал об исходе сынов Израилевых и завещал о костях своих.
\vs Heb 11:23 Верою Моисей по рождении три месяца скрываем был родителями своими, ибо видели они, что дитя прекрасно, и не устрашились царского повеления.
\vs Heb 11:24 Верою Моисей, придя в возраст, отказался называться сыном дочери фараоновой,
\vs Heb 11:25 и лучше захотел страдать с народом Божиим, нежели иметь временное греховное наслаждение,
\vs Heb 11:26 и поношение Христово почел б\acc{о}льшим для себя богатством, нежели Египетские сокровища; ибо он взирал на воздаяние.
\vs Heb 11:27 Верою оставил он Египет, не убоявшись гнева царского, ибо он, как бы видя Невидимого, был тверд.
\vs Heb 11:28 Верою совершил он Пасху и пролитие крови, дабы истребитель первенцев не коснулся их.
\vs Heb 11:29 Верою перешли они Чермное море, как по суше,~--- на что покусившись, Египтяне потонули.
\vs Heb 11:30 Верою пали стены Иерихонские, по семидневном обхождении.
\vs Heb 11:31 Верою Раав блудница, с миром приняв соглядатаев (и проводив их другим путем), не погибла с неверными.
\vs Heb 11:32 И что еще скажу? Недостанет мне времени, чтобы повествовать о Гедеоне, о Вараке, о Самсоне и Иеффае, о Давиде, Самуиле и (других) пророках,
\vs Heb 11:33 которые верою побеждали царства, творили правду, получали обетования, заграждали уста львов,
\vs Heb 11:34 угашали силу огня, избегали острия меча, укреплялись от немощи, были крепки на войне, прогоняли полки чужих;
\vs Heb 11:35 жены получали умерших своих воскресшими; иные же замучены были, не приняв освобождения, дабы получить лучшее воскресение;
\vs Heb 11:36 другие испытали поругания и побои, а также узы и темницу,
\vs Heb 11:37 были побиваемы камнями, перепиливаемы, подвергаемы пытке, умирали от меча, скитались в м\acc{и}лотях и козьих кожах, терпя недостатки, скорби, озлобления;
\vs Heb 11:38 те, которых весь мир не был достоин, скитались по пустыням и горам, по пещерам и ущельям земли.
\vs Heb 11:39 И все сии, свидетельствованные в вере, не получили обещанного,
\vs Heb 11:40 потому что Бог предусмотрел о нас нечто лучшее, дабы они не без нас достигли совершенства.
\vs Heb 12:1 Посему и мы, имея вокруг себя такое облако свидетелей, свергнем с себя всякое бремя и запинающий нас грех и с терпением будем проходить предлежащее нам поприще,
\vs Heb 12:2 взирая на начальника и совершителя веры Ии\-су\-са, Который, вместо предлежавшей Ему радости, претерпел крест, пренебрегши посрамление, и воссел одесную престола Божия.
\vs Heb 12:3 Помыслите о Претерпевшем такое над Собою поругание от грешников, чтобы вам не изнемочь и не ослабеть душами вашими.
\vs Heb 12:4 Вы еще не до крови сражались, подвизаясь против греха,
\vs Heb 12:5 и забыли утешение, которое предлагается вам, как сынам: сын мой! не пренебрегай наказания Господня, и не унывай, когда Он обличает тебя.
\vs Heb 12:6 Ибо Господь, кого любит, того наказывает; бьет же всякого сына, которого принимает.
\vs Heb 12:7 Если вы терпите наказание, то Бог поступает с вами, как с сынами. Ибо есть ли какой сын, которого бы не наказывал отец?
\vs Heb 12:8 Если же остаетесь без наказания, которое всем обще, то вы незаконные дети, а не сыны.
\vs Heb 12:9 Притом, \bibemph{если} мы, будучи наказываемы плотскими родителями нашими, боялись их, то не гораздо ли более должны покориться Отцу духов, чтобы жить?
\vs Heb 12:10 Те наказывали нас по своему произволу для немногих дней; а Сей~--- для пользы, чтобы нам иметь участие в святости Его.
\vs Heb 12:11 Всякое наказание в настоящее время кажется не радостью, а печалью; но после наученным через него доставляет мирный плод праведности.
\vs Heb 12:12 Итак укрепите опустившиеся руки и ослабевшие колени
\vs Heb 12:13 и ходите прямо ногами вашими, дабы хромлющее не совратилось, а лучше исправилось.
\vs Heb 12:14 Старайтесь иметь мир со всеми и святость, без которой никто не увидит Господа.
\vs Heb 12:15 Наблюдайте, чтобы кто не лишился благодати Божией; чтобы какой горький корень, возникнув, не причинил вреда, и чтобы им не осквернились многие;
\vs Heb 12:16 чтобы не было \bibemph{между вами} какого блудника, или нечестивца, который бы, как Исав, за одну снедь отказался от своего первородства.
\vs Heb 12:17 Ибо вы знаете, что после того он, желая наследовать благословение, был отвержен; не мог переменить мыслей \bibemph{отца}, хотя и просил о том со слезами.
\rsbpar\vs Heb 12:18 Вы приступили не к горе, осязаемой и пылающей огнем, не ко тьме и мраку и буре,
\vs Heb 12:19 не к трубному звуку и гласу глаголов, который слышавшие просили, чтобы к ним более не было продолжаемо слово,
\vs Heb 12:20 ибо они не могли стерпеть того, что заповедуемо было: если и зверь прикоснется к горе, будет побит камнями (или поражен стрелою);
\vs Heb 12:21 и столь ужасно было это видение, \bibemph{что и} Моисей сказал: <<я в страхе и трепете>>.
\vs Heb 12:22 Но вы приступили к горе Сиону и ко граду Бога живаго, к небесному Иерусалиму и тьмам Ангелов,
\vs Heb 12:23 к торжествующему собору и церкви первенцев, написанных на небесах, и к Судии всех Богу, и к духам праведников, достигших совершенства,
\vs Heb 12:24 и к Ходатаю нового завета Иисусу, и к Крови кропления, говорящей лучше, нежели Авелева.
\vs Heb 12:25 Смотрите, не отвратитесь и вы от говорящего. Если те, не послушав глаголавшего на земле, не избегли \bibemph{наказания}, то тем более \bibemph{не избежим} мы, если отвратимся от \bibemph{Глаголющего} с небес,
\vs Heb 12:26 Которого глас тогда поколебал землю, и Который ныне дал такое обещание: еще раз поколеблю не только землю, но и небо.
\vs Heb 12:27 Слова: <<еще раз>> означают изменение колеблемого, как сотворенного, чтобы пребыло непоколебимое.
\vs Heb 12:28 Итак мы, приемля царство непоколебимое, будем хранить благодать, которою будем служить благоугодно Богу, с благоговением и страхом,
\vs Heb 12:29 потому что Бог наш есть огнь поядающий.
\vs Heb 13:1 Братолюбие \bibemph{между вами} да пребывает.
\vs Heb 13:2 Страннолюбия не забывайте, ибо через него некоторые, не зная, оказали гостеприимство Ангелам.
\vs Heb 13:3 Помните узников, как бы и вы с ними были в узах, и страждущих, как и сами находитесь в теле.
\rsbpar\vs Heb 13:4 Брак у всех \bibemph{да будет} честен и ложе непорочно; блудников же и прелюбодеев судит Бог.
\vs Heb 13:5 Имейте нрав несребролюбивый, довольствуясь тем, что есть. Ибо Сам сказал: не оставлю тебя и не покину тебя,
\vs Heb 13:6 так что мы смело говорим: Господь мне помощник, и не убоюсь: что сделает мне человек?
\rsbpar\vs Heb 13:7 Поминайте наставников ваших, которые проповедовали вам слово Божие, и, взирая на кончину их жизни, подражайте вере их.
\vs Heb 13:8 Иисус Христос вчера и сегодня и во веки Тот же.
\vs Heb 13:9 Учениями различными и чуждыми не увлекайтесь; ибо хорошо благодатью укреплять сердца, а не яствами, от которых не получили пользы занимающиеся ими.
\vs Heb 13:10 Мы имеем жертвенник, от которого не имеют права питаться служащие скинии.
\vs Heb 13:11 Так как тела животных, которых кровь для \bibemph{очищения} греха вносится первосвященником во святилище, сжигаются вне стана,~---
\vs Heb 13:12 то и Иисус, дабы освятить людей Кровию Своею, пострадал вне врат.
\rsbpar\vs Heb 13:13 Итак выйдем к Нему за стан, нося Его поругание;
\vs Heb 13:14 ибо не имеем здесь постоянного града, но ищем будущего.
\vs Heb 13:15 Итак будем через Него непрестанно приносить Богу жертву хвалы, то есть плод уст, прославляющих имя Его.
\vs Heb 13:16 Не забывайте также благотворения и общительности, ибо таковые жертвы благоугодны Богу.
\rsbpar\vs Heb 13:17 Повинуйтесь наставникам вашим и будьте покорны, ибо они неусыпно пекутся о душах ваших, как обязанные дать отчет; чтобы они делали это с радостью, а не воздыхая, ибо это для вас неполезно.
\rsbpar\vs Heb 13:18 Молитесь о нас; ибо мы уверены, что имеем добрую совесть, потому что во всем желаем вести себя честно.
\vs Heb 13:19 Особенно же прошу делать это, дабы я скорее возвращен был вам.
\vs Heb 13:20 Бог же мира, воздвигший из мертвых Пастыря овец великого Кровию завета вечного, Господа нашего Иисуса \bibemph{Христа},
\vs Heb 13:21 да усовершит вас во всяком добром деле, к исполнению воли Его, производя в вас благоугодное Ему через Иисуса Христа. Ему слава во веки веков! Аминь.
\rsbpar\vs Heb 13:22 Прошу вас, братия, примите сие слово увещания; я же не много и написал вам.
\vs Heb 13:23 Знайте, что брат наш Тимофей освобожден, и я вместе с ним, если он скоро придет, увижу вас.
\rsbpar\vs Heb 13:24 Приветствуйте всех наставников ваших и всех святых. Приветствуют вас Италийские.
\rsbpar\vs Heb 13:25 Благодать со всеми вами. Аминь.

\bibbookdescr{Rev}{
  inline={Откровение\fns{Апокалипсис (греч.).}\\\LARGE Святого Иоанна Богослова},
  toc={Откровение},
  bookmark={Откровение},
  header={Откровение},
  %headerleft={},
  %headerright={},
  abbr={Откр}
}
\vs Rev 1:1 Откровение Иисуса Христа, которое дал Ему Бог, чтобы показать рабам Своим, чему надлежит быть вскоре. И Он показал, послав \bibemph{оное} через Ангела Своего рабу Своему Иоанну,
\vs Rev 1:2 который свидетельствовал слово Божие и свидетельство Иисуса Христа и что он видел.
\vs Rev 1:3 Блажен читающий и слушающие слова пророчества сего и соблюдающие написанное в нем; ибо время близко.
\rsbpar\vs Rev 1:4 Иоанн семи церквам, находящимся в Асии: благодать вам и мир от Того, Который есть и был и грядет, и от семи духов, находящихся перед престолом Его,
\vs Rev 1:5 и от Иисуса Христа, Который есть свидетель верный, первенец из мертвых и владыка царей земных. Ему, возлюбившему нас и омывшему нас от грехов наших Кровию Своею
\vs Rev 1:6 и соделавшему нас царями и священниками Богу и Отцу Своему, слава и держава во веки веков, аминь.
\vs Rev 1:7 Се, грядет с облаками, и узрит Его всякое око и те, которые пронзили Его; и возрыдают пред Ним все племена земные. Ей, аминь.
\rsbpar\vs Rev 1:8 Я есмь Альфа и Омега, начало и конец, говорит Господь, Который есть и был и грядет, Вседержитель.
\rsbpar\vs Rev 1:9 Я, Иоанн, брат ваш и соучастник в скорби и в царствии и в терпении Иисуса Христа, был на острове, называемом Патмос, за слово Божие и за свидетельство Иисуса Христа.
\vs Rev 1:10 Я был в духе в день воскресный, и слышал позади себя громкий голос, как бы трубный, который говорил: Я есмь Альфа и Омега, Первый и Последний;
\vs Rev 1:11 то, что видишь, напиши в книгу и пошли церквам, находящимся в Асии: в Ефес, и в Смирну, и в Пергам, и в Фиатиру, и в Сардис, и в Филадельфию, и в Лаодикию.
\vs Rev 1:12 Я обратился, чтобы увидеть, чей голос, говоривший со мною; и обратившись, увидел семь золотых светильников
\vs Rev 1:13 и, посреди семи светильников, подобного Сыну Человеческому, облеченного в подир\fns{Подир~--- длинная одежда Иудейских первосвященников и царей.} и по персям опоясанного золотым поясом:
\vs Rev 1:14 глава Его и волосы белы, как белая в\acc{о}лна, как снег; и очи Его, как пламень огненный;
\vs Rev 1:15 и ноги Его подобны халколивану, как раскаленные в печи, и голос Его, как шум вод многих.
\vs Rev 1:16 Он держал в деснице Своей семь звезд, и из уст Его выходил острый с обеих сторон меч; и лице Его, как солнце, сияющее в силе своей.
\vs Rev 1:17 И когда я увидел Его, то пал к ногам Его, как мертвый. И Он положил на меня десницу Свою и сказал мне: не бойся; Я есмь Первый и Последний,
\vs Rev 1:18 и живый; и был мертв, и се, жив во веки веков, аминь; и имею ключи ада и смерти.
\vs Rev 1:19 Итак напиши, что ты видел, и что есть, и что будет после сего.
\vs Rev 1:20 Тайна семи звезд, которые ты видел в деснице Моей, и семи золотых светильников \bibemph{есть сия}: семь звезд суть Ангелы семи церквей; а семь светильников, которые ты видел, суть семь церквей.
\vs Rev 2:1 Ангелу Ефесской церкви напиши: так говорит Держащий семь звезд в деснице Своей, Ходящий посреди семи золотых светильников:
\vs Rev 2:2 знаю дела твои, и труд твой, и терпение твое, и то, что ты не можешь сносить развратных, и испытал тех, которые называют себя апостолами, а они не таковы, и нашел, что они лжецы;
\vs Rev 2:3 ты много переносил и имеешь терпение, и для имени Моего трудился и не изнемогал.
\vs Rev 2:4 Но имею против тебя то, что ты оставил первую любовь твою.
\vs Rev 2:5 Итак вспомни, откуда ты ниспал, и покайся, и твори прежние дела; а если не так, скоро приду к тебе, и сдвину светильник твой с места его, если не покаешься.
\vs Rev 2:6 Впрочем то в тебе \bibemph{хорошо}, что ты ненавидишь дела Николаитов, которые и Я ненавижу.
\vs Rev 2:7 Имеющий ухо да слышит, что Дух говорит церквам: побеждающему дам вкушать от древа жизни, которое посреди рая Божия.
\rsbpar\vs Rev 2:8 И Ангелу Смирнской церкви напиши: так говорит Первый и Последний, Который был мертв, и се, жив:
\vs Rev 2:9 знаю твои дела, и скорбь, и нищету (впрочем ты богат), и злословие от тех, которые говорят о себе, что они Иудеи, а они не таковы, но сборище сатанинское.
\vs Rev 2:10 Не бойся ничего, что тебе надобно будет претерпеть. Вот, диавол будет ввергать из среды вас в темницу, чтобы искусить вас, и будете иметь скорбь дней десять. Будь верен до смерти, и дам тебе венец жизни.
\vs Rev 2:11 Имеющий ухо (слышать) да слышит, что Дух говорит церквам: побеждающий не потерпит вреда от второй смерти.
\rsbpar\vs Rev 2:12 И Ангелу Пергамской церкви напиши: так говорит Имеющий острый с обеих сторон меч:
\vs Rev 2:13 знаю твои дела, и что ты живешь там, где престол сатаны, и что содержишь имя Мое, и не отрекся от веры Моей даже в те дни, в которые у вас, где живет сатана, умерщвлен верный свидетель Мой Антипа.
\vs Rev 2:14 Но имею немного против тебя, потому что есть у тебя там держащиеся учения Валаама, который научил Валака ввести в соблазн сынов Израилевых, чтобы они ели идоложертвенное и любодействовали.
\vs Rev 2:15 Так и у тебя есть держащиеся учения Николаитов, которое Я ненавижу.
\vs Rev 2:16 Покайся; а если не так, скоро приду к тебе и сражусь с ними мечом уст Моих.
\vs Rev 2:17 Имеющий ухо (слышать) да слышит, что Дух говорит церквам: побеждающему дам вкушать сокровенную манну, и дам ему белый камень и на камне написанное новое имя, которого никто не знает, кроме того, кто получает.
\rsbpar\vs Rev 2:18 И Ангелу Фиатирской церкви напиши: так говорит Сын Божий, у Которого очи, как пламень огненный, и ноги подобны халколивану:
\vs Rev 2:19 знаю твои дела и любовь, и служение, и веру, и терпение твое, и то, что последние дела твои больше первых.
\vs Rev 2:20 Но имею немного против тебя, потому что ты попускаешь жене Иезавели, называющей себя пророчицею, учить и вводить в заблуждение рабов Моих, любодействовать и есть идоложертвенное.
\vs Rev 2:21 Я дал ей время покаяться в любодеянии ее, но она не покаялась.
\vs Rev 2:22 Вот, Я повергаю ее на одр и любодействующих с нею в великую скорбь, если не покаются в делах своих.
\vs Rev 2:23 И детей ее поражу смертью, и уразумеют все церкви, что Я есмь испытующий сердца и внутренности; и воздам каждому из вас по делам вашим.
\vs Rev 2:24 Вам же и прочим, находящимся в Фиатире, которые не держат сего учения и которые не знают так называемых глубин сатанинских, сказываю, что не наложу на вас иного бремени;
\vs Rev 2:25 только то, что имеете, держите, пока приду.
\vs Rev 2:26 Кто побеждает и соблюдает дела Мои до конца, тому дам власть над язычниками,
\vs Rev 2:27 и будет пасти их жезлом железным; как сосуды глиняные, они сокрушатся, как и Я получил \bibemph{власть} от Отца Моего;
\vs Rev 2:28 и дам ему звезду утреннюю.
\vs Rev 2:29 Имеющий ухо (слышать) да слышит, что Дух говорит церквам.
\vs Rev 3:1 И Ангелу Сардийской церкви напиши: так говорит Имеющий семь духов Божиих и семь звезд: знаю твои дела; ты носишь имя, будто жив, но ты мертв.
\vs Rev 3:2 Бодрствуй и утверждай прочее близкое к смерти; ибо Я не нахожу, чтобы дела твои были совершенны пред Богом Моим.
\vs Rev 3:3 Вспомни, что ты принял и слышал, и храни и покайся. Если же не будешь бодрствовать, то Я найду на тебя, как тать, и ты не узнаешь, в который час найду на тебя.
\vs Rev 3:4 Впрочем у тебя в Сардисе есть несколько человек, которые не осквернили одежд своих, и будут ходить со Мною в белых \bibemph{одеждах}, ибо они достойны.
\vs Rev 3:5 Побеждающий облечется в белые одежды; и не изглажу имени его из книги жизни, и исповедаю имя его пред Отцем Моим и пред Ангелами Его.
\vs Rev 3:6 Имеющий ухо да слышит, что Дух говорит церквам.
\rsbpar\vs Rev 3:7 И Ангелу Филадельфийской церкви напиши: так говорит Святый, Истинный, имеющий ключ Давидов, Который отворяет~--- и никто не затворит, затворяет~--- и никто не отворит:
\vs Rev 3:8 знаю твои дела; вот, Я отворил перед тобою дверь, и никто не может затворить ее; ты не много имеешь силы, и сохранил слово Мое, и не отрекся имени Моего.
\vs Rev 3:9 Вот, Я сделаю, что из сатанинского сборища, из тех, которые говорят о себе, что они Иудеи, но не суть таковы, а лгут,~--- вот, Я сделаю то, что они придут и поклонятся пред ногами твоими, и познают, что Я возлюбил тебя.
\vs Rev 3:10 И как ты сохранил слово терпения Моего, то и Я сохраню тебя от годины искушения, которая придет на всю вселенную, чтобы испытать живущих на земле.
\vs Rev 3:11 Се, гряду скоро; держи, что имеешь, дабы кто не восхитил венца твоего.
\vs Rev 3:12 Побеждающего сделаю столпом в храме Бога Моего, и он уже не выйдет вон; и напишу на нем имя Бога Моего и имя града Бога Моего, нового Иерусалима, нисходящего с неба от Бога Моего, и имя Мое новое.
\vs Rev 3:13 Имеющий ухо да слышит, что Дух говорит церквам.
\rsbpar\vs Rev 3:14 И Ангелу Лаодикийской церкви напиши: так говорит Аминь, свидетель верный и истинный, начало создания Божия:
\vs Rev 3:15 знаю твои дела; ты ни холоден, ни горяч; о, если бы ты был холоден, или горяч!
\vs Rev 3:16 Но, как ты тепл, а не горяч и не холоден, то извергну тебя из уст Моих.
\vs Rev 3:17 Ибо ты говоришь: <<я богат, разбогател и ни в чем не имею нужды>>; а не знаешь, что ты несчастен, и жалок, и нищ, и слеп, и наг.
\vs Rev 3:18 Советую тебе купить у Меня золото, огнем очищенное, чтобы тебе обогатиться, и белую одежду, чтобы одеться и чтобы не видна была срамота наготы твоей, и глазною мазью помажь глаза твои, чтобы видеть.
\vs Rev 3:19 Кого Я люблю, тех обличаю и наказываю. Итак будь ревностен и покайся.
\vs Rev 3:20 Се, стою у двери и стучу: если кто услышит голос Мой и отворит дверь, войду к нему, и буду вечерять с ним, и он со Мною.
\vs Rev 3:21 Побеждающему дам сесть со Мною на престоле Моем, как и Я победил и сел с Отцем Моим на престоле Его.
\vs Rev 3:22 Имеющий ухо да слышит, что Дух говорит церквам.
\vs Rev 4:1 После сего я взглянул, и вот, дверь отверста на небе, и прежний голос, который я слышал как бы звук трубы, говоривший со мною, сказал: взойди сюда, и покажу тебе, чему надлежит быть после сего.
\vs Rev 4:2 И тотчас я был в духе; и вот, престол стоял на небе, и на престоле был Сидящий;
\vs Rev 4:3 и Сей Сидящий видом был подобен камню яспису и сардису; и радуга вокруг престола, видом подобная смарагду.
\vs Rev 4:4 И вокруг престола двадцать четыре престола; а на престолах видел я сидевших двадцать четыре старца, которые облечены были в белые одежды и имели на головах своих золотые венцы.
\vs Rev 4:5 И от престола исходили молнии и громы и гласы, и семь светильников огненных горели перед престолом, которые суть семь духов Божиих;
\vs Rev 4:6 и перед престолом море стеклянное, подобное кристаллу; и посреди престола и вокруг престола четыре животных, исполненных очей спереди и сзади.
\vs Rev 4:7 И первое животное было подобно льву, и второе животное подобно тельцу, и третье животное имело лице, как человек, и четвертое животное подобно орлу летящему.
\vs Rev 4:8 И каждое из четырех животных имело по шести крыл вокруг, а внутри они исполнены очей; и ни днем, ни ночью не имеют покоя, взывая: свят, свят, свят Господь Бог Вседержитель, Который был, есть и грядет.
\vs Rev 4:9 И когда животные воздают славу и честь и благодарение Сидящему на престоле, Живущему во веки веков,
\vs Rev 4:10 тогда двадцать четыре старца падают пред Сидящим на престоле, и поклоняются Живущему во веки веков, и полагают венцы свои перед престолом, говоря:
\vs Rev 4:11 достоин Ты, Господи, приять славу и честь и силу: ибо Ты сотворил все, и \bibemph{все} по Твоей воле существует и сотворено.
\vs Rev 5:1 И видел я в деснице у Сидящего на престоле книгу, написанную внутри и отвне, запечатанную семью печатями.
\vs Rev 5:2 И видел я Ангела сильного, провозглашающего громким голосом: кто достоин раскрыть сию книгу и снять печати ее?
\vs Rev 5:3 И никто не мог, ни на небе, ни на земле, ни под землею, раскрыть сию книгу, ни посмотреть в нее.
\vs Rev 5:4 И я много плакал о том, что никого не нашлось достойного раскрыть и читать сию книгу, и даже посмотреть в нее.
\vs Rev 5:5 И один из старцев сказал мне: не плачь; вот, лев от колена Иудина, корень Давидов, победил, \bibemph{и может} раскрыть сию книгу и снять семь печатей ее.
\vs Rev 5:6 И я взглянул, и вот, посреди престола и четырех животных и посреди старцев стоял Агнец как бы закланный, имеющий семь рогов и семь очей, которые суть семь духов Божиих, посланных во всю землю.
\vs Rev 5:7 И Он пришел и взял книгу из десницы Сидящего на престоле.
\vs Rev 5:8 И когда Он взял книгу, тогда четыре животных и двадцать четыре старца пали пред Агнцем, имея каждый гусли и золотые чаши, полные фимиама, которые суть молитвы святых.
\vs Rev 5:9 И поют новую песнь, говоря: достоин Ты взять книгу и снять с нее печати, ибо Ты был заклан, и Кровию Своею искупил нас Богу из всякого колена и языка, и народа и племени,
\vs Rev 5:10 и соделал нас царями и священниками Богу нашему; и мы будем царствовать на земле.
\vs Rev 5:11 И я видел, и слышал голос многих Ангелов вокруг престола и животных и старцев, и число их было тьмы тем и тысячи тысяч,
\vs Rev 5:12 которые говорили громким голосом: достоин Агнец закланный принять силу и богатство, и премудрость и крепость, и честь и славу и благословение.
\vs Rev 5:13 И всякое создание, находящееся на небе и на земле, и под землею, и на море, и все, что в них, слышал я, говорило: Сидящему на престоле и Агнцу благословение и честь, и слава и держава во веки веков.
\vs Rev 5:14 И четыре животных говорили: аминь. И двадцать четыре старца пали и поклонились Живущему во веки веков.
\vs Rev 6:1 И я видел, что Агнец снял первую из семи печатей, и я услышал одно из четырех животных, говорящее как бы громовым голосом: иди и смотри.
\vs Rev 6:2 Я взглянул, и вот, конь белый, и на нем всадник, имеющий лук, и дан был ему венец; и вышел он \bibemph{как} победоносный, и чтобы победить.
\rsbpar\vs Rev 6:3 И когда он снял вторую печать, я слышал второе животное, говорящее: иди и смотри.
\vs Rev 6:4 И вышел другой конь, рыжий; и сидящему на нем дано взять мир с земли, и чтобы убивали друг друга; и дан ему большой меч.
\rsbpar\vs Rev 6:5 И когда Он снял третью печать, я слышал третье животное, говорящее: иди и смотри. Я взглянул, и вот, конь вороной, и на нем всадник, имеющий меру в руке своей.
\vs Rev 6:6 И слышал я голос посреди четырех животных, говорящий: хиникс\fns{Хиникс~--- малая хлебная мера.} пшеницы за динарий\fns{Динарий~--- монета, соответствующая дневной плате поденщику.}, и три хиникса ячменя за динарий; елея же и вина не повреждай.
\rsbpar\vs Rev 6:7 И когда Он снял четвертую печать, я слышал голос четвертого животного, говорящий: иди и смотри.
\vs Rev 6:8 И я взглянул, и вот, конь бледный, и на нем всадник, которому имя <<смерть>>; и ад следовал за ним; и дана ему власть над четвертою частью земли~--- умерщвлять мечом и голодом, и мором и зверями земными.
\rsbpar\vs Rev 6:9 И когда Он снял пятую печать, я увидел под жертвенником души убиенных за слово Божие и за свидетельство, которое они имели.
\vs Rev 6:10 И возопили они громким голосом, говоря: доколе, Владыка Святый и Истинный, не судишь и не мстишь живущим на земле за кровь нашу?
\vs Rev 6:11 И даны были каждому из них одежды белые, и сказано им, чтобы они успокоились еще на малое время, пока и сотрудники их и братья их, которые будут убиты, как и они, дополнят число.
\rsbpar\vs Rev 6:12 И когда Он снял шестую печать, я взглянул, и вот, произошло великое землетрясение, и солнце стало мрачно как власяница, и луна сделалась как кровь.
\vs Rev 6:13 И звезды небесные пали на землю, как смоковница, потрясаемая сильным ветром, роняет незрелые смоквы свои.
\vs Rev 6:14 И небо скрылось, свившись как свиток; и всякая гора и остров двинулись с мест своих.
\vs Rev 6:15 И цари земные, и вельможи, и богатые, и тысяченачальники, и сильные, и всякий раб, и всякий свободный скрылись в пещеры и в ущелья гор,
\vs Rev 6:16 и говорят горам и камням: падите на нас и сокройте нас от лица Сидящего на престоле и от гнева Агнца;
\vs Rev 6:17 ибо пришел великий день гнева Его, и кто может устоять?
\vs Rev 7:1 И после сего видел я четырех Ангелов, стоящих на четырех углах земли, держащих четыре ветра земли, чтобы не дул ветер ни на землю, ни на море, ни на какое дерево.
\vs Rev 7:2 И видел я иного Ангела, восходящего от востока солнца и имеющего печать Бога живаго. И воскликнул он громким голосом к четырем Ангелам, которым дано вредить земле и морю, говоря:
\vs Rev 7:3 не делайте вреда ни земле, ни морю, ни деревам, доколе не положим печати на челах рабов Бога нашего.
\vs Rev 7:4 И я слышал число запечатленных: запечатленных было сто сорок четыре тысячи из всех колен сынов Израилевых.
\vs Rev 7:5 Из колена Иудина запечатлено двенадцать тысяч; из колена Рувимова запечатлено двенадцать тысяч; из колена Гадова запечатлено двенадцать тысяч;
\vs Rev 7:6 из колена Асирова запечатлено двенадцать тысяч; из колена Неффалимова запечатлено двенадцать тысяч; из колена Манассиина запечатлено двенадцать тысяч;
\vs Rev 7:7 из колена Симеонова запечатлено двенадцать тысяч; из колена Левиина запечатлено двенадцать тысяч; из колена Иссахарова запечатлено двенадцать тысяч;
\vs Rev 7:8 из колена Завулонова запечатлено двенадцать тысяч; из колена Иосифова запечатлено двенадцать тысяч; из колена Вениаминова запечатлено двенадцать тысяч.
\vs Rev 7:9 После сего взглянул я, и вот, великое множество людей, которого никто не мог перечесть, из всех племен и колен, и народов и языков, стояло пред престолом и пред Агнцем в белых одеждах и с пальмовыми ветвями в руках своих.
\vs Rev 7:10 И восклицали громким голосом, говоря: спасение Богу нашему, сидящему на престоле, и Агнцу!
\vs Rev 7:11 И все Ангелы стояли вокруг престола и старцев и четырех животных, и пали перед престолом на лица свои, и поклонились Богу,
\vs Rev 7:12 говоря: аминь! благословение и слава, и премудрость и благодарение, и честь и сила и крепость Богу нашему во веки веков! Аминь.
\vs Rev 7:13 И, начав речь, один из старцев спросил меня: сии облеченные в белые одежды кто, и откуда пришли?
\vs Rev 7:14 Я сказал ему: ты знаешь, господин. И он сказал мне: это те, которые пришли от великой скорби; они омыли одежды свои и убелили одежды свои Кровию Агнца.
\vs Rev 7:15 За это они пребывают \bibemph{ныне} перед престолом Бога и служат Ему день и ночь в храме Его, и Сидящий на престоле будет обитать в них.
\vs Rev 7:16 Они не будут уже ни алкать, ни жаждать, и не будет палить их солнце и никакой зной:
\vs Rev 7:17 ибо Агнец, Который среди престола, будет пасти их и водить их на живые источники вод; и отрет Бог всякую слезу с очей их.
\vs Rev 8:1 И когда Он снял седьмую печать, сделалось безмолвие на небе, как бы на полчаса.
\vs Rev 8:2 И я видел семь Ангелов, которые стояли пред Богом; и дано им семь труб.
\vs Rev 8:3 И пришел иной Ангел, и стал перед жертвенником, держа золотую кадильницу; и дано было ему множество фимиама, чтобы он с молитвами всех святых возложил его на золотой жертвенник, который перед престолом.
\vs Rev 8:4 И вознесся дым фимиама с молитвами святых от руки Ангела пред Бога.
\vs Rev 8:5 И взял Ангел кадильницу, и наполнил ее огнем с жертвенника, и поверг на землю: и произошли голоса и громы, и молнии и землетрясение.
\rsbpar\vs Rev 8:6 И семь Ангелов, имеющие семь труб, приготовились трубить.
\rsbpar\vs Rev 8:7 Первый Ангел вострубил, и сделались град и огонь, смешанные с кровью, и пали на землю; и третья часть дерев сгорела, и вся трава зеленая сгорела.
\rsbpar\vs Rev 8:8 Второй Ангел вострубил, и как бы большая гора, пылающая огнем, низверглась в море; и третья часть моря сделалась кровью,
\vs Rev 8:9 и умерла третья часть одушевленных тварей, живущих в море, и третья часть судов погибла.
\rsbpar\vs Rev 8:10 Третий Ангел вострубил, и упала с неба большая звезда, горящая подобно светильнику, и пала на третью часть рек и на источники вод.
\vs Rev 8:11 Имя сей звезде <<полынь>>; и третья часть вод сделалась полынью, и многие из людей умерли от вод, потому что они стали горьки.
\rsbpar\vs Rev 8:12 Четвертый Ангел вострубил, и поражена была третья часть солнца и третья часть луны и третья часть звезд, так что затмилась третья часть их, и третья часть дня не светла была~--- так, как и ночи.
\vs Rev 8:13 И видел я и слышал одного Ангела, летящего посреди неба и говорящего громким голосом: горе, горе, горе живущим на земле от остальных трубных голосов трех Ангелов, которые будут трубить!
\vs Rev 9:1 Пятый Ангел вострубил, и я увидел звезду, падшую с неба на землю, и дан был ей ключ от кладязя бездны.
\vs Rev 9:2 Она отворила кладязь бездны, и вышел дым из кладязя, как дым из большой печи; и помрачилось солнце и воздух от дыма из кладязя.
\vs Rev 9:3 И из дыма вышла саранча на землю, и дана была ей власть, какую имеют земные скорпионы.
\vs Rev 9:4 И сказано было ей, чтобы не делала вреда траве земной, и никакой зелени, и никакому дереву, а только одним людям, которые не имеют печати Божией на челах своих.
\vs Rev 9:5 И дано ей не убивать их, а только мучить пять месяцев; и мучение от нее подобно мучению от скорпиона, когда ужалит человека.
\vs Rev 9:6 В те дни люди будут искать смерти, но не найдут ее; пожелают умереть, но смерть убежит от них.
\vs Rev 9:7 По виду своему саранча была подобна коням, приготовленным на войну; и на головах у ней как бы венцы, похожие на золотые, лица же ее~--- как лица человеческие;
\vs Rev 9:8 и волосы у ней~--- как волосы у женщин, а зубы у ней были, как у львов.
\vs Rev 9:9 На ней были брони, как бы брони железные, а шум от крыльев ее~--- как стук от колесниц, когда множество коней бежит на войну;
\vs Rev 9:10 у ней были хвосты, как у скорпионов, и в хвостах ее были жала; власть же ее была~--- вредить людям пять месяцев.
\vs Rev 9:11 Царем над собою она имела ангела бездны; имя ему по-еврейски Аваддон, а по-гречески Аполлион\fns{Губитель.}.
\rsbpar\vs Rev 9:12 Одно горе прошло; вот, идут за ним еще два горя.
\rsbpar\vs Rev 9:13 Шестой Ангел вострубил, и я услышал один голос от четырех рогов золотого жертвенника, стоящего пред Богом,
\vs Rev 9:14 говоривший шестому Ангелу, имевшему трубу: освободи четырех Ангелов, связанных при великой реке Евфрате.
\vs Rev 9:15 И освобождены были четыре Ангела, приготовленные на час и день, и месяц и год, для того, чтобы умертвить третью часть людей.
\vs Rev 9:16 Число конного войска было две тьмы тем; и я слышал число его.
\vs Rev 9:17 Так видел я в видении коней и на них всадников, которые имели на себе брони огненные, гиацинтовые и серные; головы у коней~--- как головы у львов, и изо рта их выходил огонь, дым и сера.
\vs Rev 9:18 От этих трех язв, от огня, дыма и серы, выходящих изо рта их, умерла третья часть людей;
\vs Rev 9:19 ибо сила коней заключалась во рту их и в хвостах их; а хвосты их были подобны змеям, и имели головы, и ими они вредили.
\vs Rev 9:20 Прочие же люди, которые не умерли от этих язв, не раскаялись в делах рук своих, так чтобы не поклоняться бесам и золотым, серебряным, медным, каменным и деревянным идолам, которые не могут ни видеть, ни слышать, ни ходить.
\vs Rev 9:21 И не раскаялись они в убийствах своих, ни в чародействах своих, ни в блудодеянии своем, ни в воровстве своем.
\vs Rev 10:1 И видел я другого Ангела сильного, сходящего с неба, облеченного облаком; над головою его была радуга, и лице его как солнце, и ноги его как столпы огненные,
\vs Rev 10:2 в руке у него была книжка раскрытая. И поставил он правую ногу свою на море, а левую на землю,
\vs Rev 10:3 и воскликнул громким голосом, как рыкает лев; и когда он воскликнул, тогда семь громов проговорили голосами своими.
\vs Rev 10:4 И когда семь громов проговорили голосами своими, я хотел было писать; но услышал голос с неба, говорящий мне: скрой, что говорили семь громов, и не пиши сего.
\vs Rev 10:5 И Ангел, которого я видел стоящим на море и на земле, поднял руку свою к небу
\vs Rev 10:6 и клялся Живущим во веки веков, Который сотворил небо и все, что на нем, землю и все, что на ней, и море и все, что в нем, что времени уже не будет;
\vs Rev 10:7 но в те дни, когда возгласит седьмой Ангел, когда он вострубит, совершится тайна Божия, как Он благовествовал рабам Своим пророкам.
\vs Rev 10:8 И голос, который я слышал с неба, опять стал говорить со мною, и сказал: пойди, возьми раскрытую книжку из руки Ангела, стоящего на море и на земле.
\vs Rev 10:9 И я пошел к Ангелу, и сказал ему: дай мне книжку. Он сказал мне: возьми и съешь ее; она будет горька во чреве твоем, но в устах твоих будет сладка, как мед.
\vs Rev 10:10 И взял я книжку из руки Ангела, и съел ее; и она в устах моих была сладка, как мед; когда же съел ее, то горько стало во чреве моем.
\vs Rev 10:11 И сказал он мне: тебе надлежит опять пророчествовать о народах и племенах, и языках и царях многих.
\vs Rev 11:1 И дана мне трость, подобная жезлу, и сказано: встань и измерь храм Божий и жертвенник, и поклоняющихся в нем.
\vs Rev 11:2 А внешний двор храма исключи и не измеряй его, ибо он дан язычникам: они будут попирать святый город сорок два месяца.
\vs Rev 11:3 И дам двум свидетелям Моим, и они будут пророчествовать тысячу двести шестьдесят дней, будучи облечены во вретище.
\vs Rev 11:4 Это суть две маслины и два светильника, стоящие пред Богом земли.
\vs Rev 11:5 И если кто захочет их обидеть, то огонь выйдет из уст их и пожрет врагов их; если кто захочет их обидеть, тому надлежит быть убиту.
\vs Rev 11:6 Они имеют власть затворить небо, чтобы не шел дождь на землю во дни пророчествования их, и имеют власть над водами, превращать их в кровь, и поражать землю всякою язвою, когда только захотят.
\vs Rev 11:7 И когда кончат они свидетельство свое, зверь, выходящий из бездны, сразится с ними, и победит их, и убьет их,
\vs Rev 11:8 и трупы их оставит на улице великого города, который духовно называется Содом и Египет, где и Господь наш распят.
\vs Rev 11:9 И \bibemph{многие} из народов и колен, и языков и племен будут смотреть на трупы их три дня с половиною, и не позволят положить трупы их во гробы.
\vs Rev 11:10 И живущие на земле будут радоваться сему и веселиться, и пошлют дары друг другу, потому что два пророка сии мучили живущих на земле.
\vs Rev 11:11 Но после трех дней с половиною вошел в них дух жизни от Бога, и они оба стали на ноги свои; и великий страх напал на тех, которые смотрели на них.
\vs Rev 11:12 И услышали они с неба громкий голос, говоривший им: взойдите сюда. И они взошли на небо на облаке; и смотрели на них враги их.
\vs Rev 11:13 И в тот же час произошло великое землетрясение, и десятая часть города пала, и погибло при землетрясении семь тысяч имен человеческих; и прочие объяты были страхом и воздали славу Богу небесному.
\rsbpar\vs Rev 11:14 Второе горе прошло; вот, идет скоро третье горе.
\rsbpar\vs Rev 11:15 И седьмой Ангел вострубил, и раздались на небе громкие голоса, говорящие: царство мира соделалось \bibemph{царством} Господа нашего и Христа Его, и будет царствовать во веки веков.
\vs Rev 11:16 И двадцать четыре старца, сидящие пред Богом на престолах своих, пали на лица свои и поклонились Богу,
\vs Rev 11:17 говоря: благодарим Тебя, Господи Боже Вседержитель, Который еси и был и грядешь, что Ты приял силу Твою великую и воцарился.
\vs Rev 11:18 И рассвирепели язычники; и пришел гнев Твой и время судить мертвых и дать возмездие рабам Твоим, пророкам и святым и боящимся имени Твоего, малым и великим, и погубить губивших землю.
\rsbpar\vs Rev 11:19 И отверзся храм Божий на небе, и явился ковчег завета Его в храме Его; и произошли молнии и голоса, и громы и землетрясение и великий град.
\vs Rev 12:1 И явилось на небе великое знамение: жена, облеченная в солнце; под ногами ее луна, и на главе ее венец из двенадцати звезд.
\vs Rev 12:2 Она имела во чреве, и кричала от болей и мук рождения.
\vs Rev 12:3 И другое знамение явилось на небе: вот, большой красный дракон с семью головами и десятью рогами, и на головах его семь диадим.
\vs Rev 12:4 Хвост его увлек с неба третью часть звезд и поверг их на землю. Дракон сей стал перед женою, которой надлежало родить, дабы, когда она родит, пожрать ее младенца.
\vs Rev 12:5 И родила она младенца мужеского пола, которому надлежит пасти все народы жезлом железным; и восхищено было дитя ее к Богу и престолу Его.
\vs Rev 12:6 А жена убежала в пустыню, где приготовлено было для нее место от Бога, чтобы питали ее там тысячу двести шестьдесят дней.
\rsbpar\vs Rev 12:7 И произошла на небе война: Михаил и Ангелы его воевали против дракона, и дракон и ангелы его воевали \bibemph{против них},
\vs Rev 12:8 но не устояли, и не нашлось уже для них места на небе.
\vs Rev 12:9 И низвержен был великий дракон, древний змий, называемый диаволом и сатаною, обольщающий всю вселенную, низвержен на землю, и ангелы его низвержены с ним.
\vs Rev 12:10 И услышал я громкий голос, говорящий на небе: ныне настало спасение и сила и царство Бога нашего и власть Христа Его, потому что низвержен клеветник братий наших, клеветавший на них пред Богом нашим день и ночь.
\vs Rev 12:11 Они победили его кровию Агнца и словом свидетельства своего, и не возлюбили души своей даже до смерти.
\vs Rev 12:12 Итак веселитесь, небеса и обитающие на них! Горе живущим на земле и на море! потому что к вам сошел диавол в сильной ярости, зная, что немного ему остается времени.
\rsbpar\vs Rev 12:13 Когда же дракон увидел, что низвержен на землю, начал преследовать жену, которая родила младенца мужеского пола.
\vs Rev 12:14 И даны были жене два крыла большого орла, чтобы она летела в пустыню в свое место от лица змия и там питалась в продолжение времени, времен и полвремени.
\vs Rev 12:15 И пустил змий из пасти своей вслед жены воду как реку, дабы увлечь ее рекою.
\vs Rev 12:16 Но земля помогла жене, и разверзла земля уста свои, и поглотила реку, которую пустил дракон из пасти своей.
\vs Rev 12:17 И рассвирепел дракон на жену, и пошел, чтобы вступить в брань с прочими от семени ее, сохраняющими заповеди Божии и имеющими свидетельство Иисуса Христа.
\vs Rev 13:1 И стал я на песке морском, и увидел выходящего из моря зверя с семью головами и десятью рогами: на рогах его было десять диадим, а на головах его имена богохульные.
\vs Rev 13:2 Зверь, которого я видел, был подобен барсу; ноги у него~--- как у медведя, а пасть у него~--- как пасть у льва; и дал ему дракон силу свою и престол свой и великую власть.
\vs Rev 13:3 И видел я, что одна из голов его как бы смертельно была ранена, но эта смертельная рана исцелела. И дивилась вся земля, следя за зверем, и поклонились дракону, который дал власть зверю,
\vs Rev 13:4 и поклонились зверю, говоря: кто подобен зверю сему? и кто может сразиться с ним?
\vs Rev 13:5 И даны были ему уста, говорящие гордо и богохульно, и дана ему власть действовать сорок два месяца.
\vs Rev 13:6 И отверз он уста свои для хулы на Бога, чтобы хулить имя Его, и жилище Его, и живущих на небе.
\vs Rev 13:7 И дано было ему вести войну со святыми и победить их; и дана была ему власть над всяким коленом и народом, и языком и племенем.
\vs Rev 13:8 И поклонятся ему все живущие на земле, которых имена не написаны в книге жизни у Агнца, закланного от создания мира.
\vs Rev 13:9 Кто имеет ухо, да слышит.
\vs Rev 13:10 Кто ведет в плен, тот сам пойдет в плен; кто мечом убивает, тому самому надлежит быть убиту мечом. Здесь терпение и вера святых.
\rsbpar\vs Rev 13:11 И увидел я другого зверя, выходящего из земли; он имел два рога, подобные агнчим, и говорил как дракон.
\vs Rev 13:12 Он действует перед ним со всею властью первого зверя и заставляет всю землю и живущих на ней поклоняться первому зверю, у которого смертельная рана исцелела;
\vs Rev 13:13 и творит великие знамения, так что и огонь низводит с неба на землю перед людьми.
\vs Rev 13:14 И чудесами, которые дано было ему творить перед зверем, он обольщает живущих на земле, говоря живущим на земле, чтобы они сделали образ зверя, который имеет рану от меча и жив.
\vs Rev 13:15 И дано ему было вложить дух в образ зверя, чтобы образ зверя и говорил и действовал так, чтобы убиваем был всякий, кто не будет поклоняться образу зверя.
\vs Rev 13:16 И он сделает то, что всем, малым и великим, богатым и нищим, свободным и рабам, положено будет начертание на правую руку их или на чело их,
\vs Rev 13:17 и что никому нельзя будет ни покупать, ни продавать, кроме того, кто имеет это начертание, или имя зверя, или число имени его.
\vs Rev 13:18 Здесь мудрость. Кто имеет ум, тот сочти число зверя, ибо это число человеческое; число его шестьсот шестьдесят шесть.
\vs Rev 14:1 И взглянул я, и вот, Агнец стоит на горе Сионе, и с Ним сто сорок четыре тысячи, у которых имя Отца Его написано на челах.
\vs Rev 14:2 И услышал я голос с неба, как шум от множества вод и как звук сильного грома; и услышал голос как бы гуслистов, играющих на гуслях своих.
\vs Rev 14:3 Они поют как бы новую песнь пред престолом и пред четырьмя животными и старцами; и никто не мог научиться сей песни, кроме сих ста сорока четырех тысяч, искупленных от земли.
\vs Rev 14:4 Это те, которые не осквернились с женами, ибо они девственники; это те, которые следуют за Агнцем, куда бы Он ни пошел. Они искуплены из людей, как первенцы Богу и Агнцу,
\vs Rev 14:5 и в устах их нет лукавства; они непорочны пред престолом Божиим.
\rsbpar\vs Rev 14:6 И увидел я другого Ангела, летящего по средине неба, который имел вечное Евангелие, чтобы благовествовать живущим на земле и всякому племени и колену, и языку и народу;
\vs Rev 14:7 и говорил он громким голосом: убойтесь Бога и воздайте Ему славу, ибо наступил час суда Его, и поклонитесь Сотворившему небо и землю, и море и источники вод.
\vs Rev 14:8 И другой Ангел следовал за ним, говоря: пал, пал Вавилон, город великий, потому что он яростным вином блуда своего напоил все народы.
\vs Rev 14:9 И третий Ангел последовал за ними, говоря громким голосом: кто поклоняется зверю и образу его и принимает начертание на чело свое, или на руку свою,
\vs Rev 14:10 тот будет пить вино ярости Божией, вино цельное, приготовленное в чаше гнева Его, и будет мучим в огне и сере пред святыми Ангелами и пред Агнцем;
\vs Rev 14:11 и дым мучения их будет восходить во веки веков, и не будут иметь покоя ни днем, ни ночью поклоняющиеся зверю и образу его и принимающие начертание имени его.
\vs Rev 14:12 Здесь терпение святых, соблюдающих заповеди Божии и веру в Иисуса.
\rsbpar\vs Rev 14:13 И услышал я голос с неба, говорящий мне: напиши: отныне блаженны мертвые, умирающие в Господе; ей, говорит Дух, они успокоятся от трудов своих, и дела их идут вслед за ними.
\rsbpar\vs Rev 14:14 И взглянул я, и вот светлое облако, и на облаке сидит подобный Сыну Человеческому; на голове его золотой венец, и в руке его острый серп.
\vs Rev 14:15 И вышел другой Ангел из храма и воскликнул громким голосом к сидящему на облаке: пусти серп твой и пожни, потому что пришло время жатвы, ибо жатва на земле созрела.
\vs Rev 14:16 И поверг сидящий на облаке серп свой на землю, и земля была пожата.
\vs Rev 14:17 И другой Ангел вышел из храма, находящегося на небе, также с острым серпом.
\vs Rev 14:18 И иной Ангел, имеющий власть над огнем, вышел от жертвенника и с великим криком воскликнул к имеющему острый серп, говоря: пусти острый серп твой и обрежь гроздья винограда на земле, потому что созрели на нем ягоды.
\vs Rev 14:19 И поверг Ангел серп свой на землю, и обрезал виноград на земле, и бросил в великое точило гнева Божия.
\vs Rev 14:20 И истоптаны \bibemph{ягоды} в точиле за городом, и потекла кровь из точила даже до узд конских, на тысячу шестьсот стадий.
\vs Rev 15:1 И увидел я иное знамение на небе, великое и чудное: семь Ангелов, имеющих семь последних язв, которыми оканчивалась ярость Божия.
\vs Rev 15:2 И видел я как бы стеклянное море, смешанное с огнем; и победившие зверя и образ его, и начертание его и число имени его, стоят на этом стеклянном море, держа гусли Божии,
\vs Rev 15:3 и поют песнь Моисея, раба Божия, и песнь Агнца, говоря: велики и чудны дела Твои, Господи Боже Вседержитель! праведны и истинны пути Твои, Царь святых!
\vs Rev 15:4 Кто не убоится Тебя, Господи, и не прославит имени Твоего? ибо Ты един свят. Все народы придут и поклонятся пред Тобою, ибо открылись суды Твои.
\rsbpar\vs Rev 15:5 И после сего я взглянул, и вот, отверзся храм скинии свидетельства на небе.
\vs Rev 15:6 И вышли из храма семь Ангелов, имеющие семь язв, облеченные в чистую и светлую льняную одежду и опоясанные по персям золотыми поясами.
\vs Rev 15:7 И одно из четырех животных дало семи Ангелам семь золотых чаш, наполненных гневом Бога, живущего во веки веков.
\vs Rev 15:8 И наполнился храм дымом от славы Божией и от силы Его, и никто не мог войти в храм, доколе не окончились семь язв семи Ангелов.
\vs Rev 16:1 И услышал я из храма громкий голос, говорящий семи Ангелам: идите и вылейте семь чаш гнева Божия на землю.
\vs Rev 16:2 Пошел первый Ангел и вылил чашу свою на землю: и сделались жестокие и отвратительные гнойные раны на людях, имеющих начертание зверя и поклоняющихся образу его.
\rsbpar\vs Rev 16:3 Второй Ангел вылил чашу свою в море: и сделалась кровь, как бы мертвеца, и все одушевленное умерло в море.
\rsbpar\vs Rev 16:4 Третий Ангел вылил чашу свою в реки и источники вод: и сделалась кровь.
\vs Rev 16:5 И услышал я Ангела вод, который говорил: праведен Ты, Господи, Который еси и был, и свят, потому что так судил;
\vs Rev 16:6 за то, что они пролили кровь святых и пророков, Ты дал им пить кровь: они достойны того.
\vs Rev 16:7 И услышал я другого от жертвенника говорящего: ей, Господи Боже Вседержитель, истинны и праведны суды Твои.
\rsbpar\vs Rev 16:8 Четвертый Ангел вылил чашу свою на солнце: и дано было ему жечь людей огнем.
\vs Rev 16:9 И жег людей сильный зной, и они хулили имя Бога, имеющего власть над сими язвами, и не вразумились, чтобы воздать Ему славу.
\rsbpar\vs Rev 16:10 Пятый Ангел вылил чашу свою на престол зверя: и сделалось царство его мрачно, и они кусали языки свои от страдания,
\vs Rev 16:11 и хулили Бога небесного от страданий своих и язв своих; и не раскаялись в делах своих.
\rsbpar\vs Rev 16:12 Шестой Ангел вылил чашу свою в великую реку Евфрат: и высохла в ней вода, чтобы готов был путь царям от восхода солнечного.
\vs Rev 16:13 И видел я \bibemph{выходящих} из уст дракона и из уст зверя и из уст лжепророка трех духов нечистых, подобных жабам:
\vs Rev 16:14 это~--- бесовские духи, творящие знамения; они выходят к царям земли всей вселенной, чтобы собрать их на брань в оный великий день Бога Вседержителя.
\vs Rev 16:15 Се, иду как тать: блажен бодрствующий и хранящий одежду свою, чтобы не ходить ему нагим и чтобы не увидели срамоты его.
\vs Rev 16:16 И он собрал их на место, называемое по-еврейски Армагеддон.
\rsbpar\vs Rev 16:17 Седьмой Ангел вылил чашу свою на воздух: и из храма небесного от престола раздался громкий голос, говорящий: совершилось!
\vs Rev 16:18 И произошли молнии, громы и голоса, и сделалось великое землетрясение, какого не бывало с тех пор, как люди на земле. Такое землетрясение! Так великое!
\vs Rev 16:19 И город великий распался на три части, и города языческие пали, и Вавилон великий воспомянут пред Богом, чтобы дать ему чашу вина ярости гнева Его.
\vs Rev 16:20 И всякий остров убежал, и гор не стало;
\vs Rev 16:21 и град, величиною в талант, пал с неба на людей; и хулили люди Бога за язвы от града, потому что язва от него была весьма тяжкая.
\vs Rev 17:1 И пришел один из семи Ангелов, имеющих семь чаш, и, говоря со мною, сказал мне: подойди, я покажу тебе суд над великою блудницею, сидящею на водах многих;
\vs Rev 17:2 с нею блудодействовали цари земные, и вином ее блудодеяния упивались живущие на земле.
\vs Rev 17:3 И повел меня в духе в пустыню; и я увидел жену, сидящую на звере багряном, преисполненном именами богохульными, с семью головами и десятью рогами.
\vs Rev 17:4 И жена облечена была в порфиру и багряницу, украшена золотом, драгоценными камнями и жемчугом, и держала золотую чашу в руке своей, наполненную мерзостями и нечистотою блудодейства ее;
\vs Rev 17:5 и на челе ее написано имя: тайна, Вавилон великий, мать блудницам и мерзостям земным.
\vs Rev 17:6 Я видел, что жена упоена была кровью святых и кровью свидетелей Иисусовых, и видя ее, дивился удивлением великим.
\vs Rev 17:7 И сказал мне Ангел: что ты дивишься? я скажу тебе тайну жены сей и зверя, носящего ее, имеющего семь голов и десять рогов.
\vs Rev 17:8 Зверь, которого ты видел, был, и нет его, и выйдет из бездны, и пойдет в погибель; и удивятся те из живущих на земле, имена которых не вписаны в книгу жизни от начала мира, видя, что зверь был, и нет его, и явится.
\vs Rev 17:9 Здесь ум, имеющий мудрость. Семь голов суть семь гор, на которых сидит жена,
\vs Rev 17:10 и семь царей, из которых пять пали, один есть, а другой еще не пришел, и когда придет, не долго ему быть.
\vs Rev 17:11 И зверь, который был и которого нет, есть восьмой, и из числа семи, и пойдет в погибель.
\vs Rev 17:12 И десять рогов, которые ты видел, суть десять царей, которые еще не получили царства, но примут власть со зверем, как цари, на один час.
\vs Rev 17:13 Они имеют одни мысли и передадут силу и власть свою зверю.
\vs Rev 17:14 Они будут вести брань с Агнцем, и Агнец победит их; ибо Он есть Господь господствующих и Царь царей, и те, которые с Ним, суть званые и избранные и верные.
\vs Rev 17:15 И говорит мне: в\acc{о}ды, которые ты видел, где сидит блудница, суть люди и народы, и племена и языки.
\vs Rev 17:16 И десять рогов, которые ты видел на звере, сии возненавидят блудницу, и разорят ее, и обнажат, и плоть ее съедят, и сожгут ее в огне;
\vs Rev 17:17 потому что Бог положил им на сердце~--- исполнить волю Его, исполнить одну волю, и отдать царство их зверю, доколе не исполнятся слова Божии.
\vs Rev 17:18 Жена же, которую ты видел, есть великий город, царствующий над земными царями.
\vs Rev 18:1 После сего я увидел иного Ангела, сходящего с неба и имеющего власть великую; земля осветилась от славы его.
\vs Rev 18:2 И воскликнул он сильно, громким голосом говоря: пал, пал Вавилон, великая \bibemph{блудница}, сделался жилищем бесов и пристанищем всякому нечистому духу, пристанищем всякой нечистой и отвратительной птице; ибо яростным вином блудодеяния своего она напоила все народы,
\vs Rev 18:3 и цари земные любодействовали с нею, и купцы земные разбогатели от великой роскоши ее.
\rsbpar\vs Rev 18:4 И услышал я иной голос с неба, говорящий: выйди от нее, народ Мой, чтобы не участвовать вам в грехах ее и не подвергнуться язвам ее;
\vs Rev 18:5 ибо грехи ее дошли до неба, и Бог воспомянул неправды ее.
\vs Rev 18:6 Воздайте ей так, как и она воздала вам, и вдвое воздайте ей по делам ее; в чаше, в которой она приготовляла вам вино, приготовьте ей вдвое.
\vs Rev 18:7 Сколько славилась она и роскошествовала, столько воздайте ей мучений и горестей. Ибо она говорит в сердце своем: <<сижу царицею, я не вдова и не увижу горести!>>
\vs Rev 18:8 За то в один день придут на нее казни, смерть и плач и голод, и будет сожжена огнем, потому что силен Господь Бог, судящий ее.
\vs Rev 18:9 И восплачут и возрыдают о ней цари земные, блудодействовавшие и роскошествовавшие с нею, когда увидят дым от пожара ее,
\vs Rev 18:10 стоя издали от страха мучений ее \bibemph{и} говоря: горе, горе \bibemph{тебе}, великий город Вавилон, город крепкий! ибо в один час пришел суд твой.
\vs Rev 18:11 И купцы земные восплачут и возрыдают о ней, потому что товаров их никто уже не покупает,
\vs Rev 18:12 товаров золотых и серебряных, и камней драгоценных и жемчуга, и виссона и порфиры, и шелка и багряницы, и всякого благовонного дерева, и всяких изделий из слоновой кости, и всяких изделий из дорогих дерев, из меди и железа и мрамора,
\vs Rev 18:13 корицы и фимиама, и мира и ладана, и вина и елея, и муки и пшеницы, и скота и овец, и коней и колесниц, и тел и душ человеческих.
\vs Rev 18:14 И плодов, угодных для души твоей, не стало у тебя, и все тучное и блистательное удалилось от тебя; ты уже не найдешь его.
\vs Rev 18:15 Торговавшие всем сим, обогатившиеся от нее, станут вдали от страха мучений ее, плача и рыдая
\vs Rev 18:16 и говоря: горе, горе \bibemph{тебе}, великий город, одетый в виссон и порфиру и багряницу, украшенный золотом и камнями драгоценными и жемчугом,
\vs Rev 18:17 ибо в один час погибло такое богатство! И все кормчие, и все плывущие на кораблях, и все корабельщики, и все торгующие на море стали вдали
\vs Rev 18:18 и, видя дым от пожара ее, возопили, говоря: какой город подобен городу великому!
\vs Rev 18:19 И пос\acc{ы}пали пеплом головы свои, и вопили, плача и рыдая: горе, горе \bibemph{тебе}, город великий, драгоценностями которого обогатились все, имеющие корабли на море, ибо опустел в один час!
\vs Rev 18:20 Веселись о сем, небо и святые Апостолы и пророки; ибо совершил Бог суд ваш над ним.
\rsbpar\vs Rev 18:21 И один сильный Ангел взял камень, подобный большому жернову, и поверг в море, говоря: с таким стремлением повержен будет Вавилон, великий город, и уже не будет его.
\vs Rev 18:22 И г\acc{о}лоса играющих на гуслях, и поющих, и играющих на свирелях, и трубящих трубами в тебе уже не слышно будет; не будет уже в тебе никакого художника, никакого художества, и шума от жерновов не слышно уже будет в тебе;
\vs Rev 18:23 и свет светильника уже не появится в тебе; и г\acc{о}лоса жениха и невесты не будет уже слышно в тебе: ибо купцы твои были вельможи земли, и волшебством твоим введены в заблуждение все народы.
\vs Rev 18:24 И в нем найдена кровь пророков и святых и всех убитых на земле.
\vs Rev 19:1 После сего я услышал на небе громкий голос как бы многочисленного народа, который говорил: аллилуия! спасение и слава, и честь и сила Господу нашему!
\vs Rev 19:2 Ибо истинны и праведны суды Его: потому что Он осудил ту великую любодейцу, которая растлила землю любодейством своим, и взыскал кровь рабов Своих от руки ее.
\vs Rev 19:3 И вторично сказали: аллилуия! И дым ее восходил во веки веков.
\vs Rev 19:4 Тогда двадцать четыре старца и четыре животных пали и поклонились Богу, сидящему на престоле, говоря: аминь! аллилуия!
\vs Rev 19:5 И голос от престола исшел, говорящий: хвалите Бога нашего, все рабы Его и боящиеся Его, малые и великие.
\vs Rev 19:6 И слышал я как бы голос многочисленного народа, как бы шум вод многих, как бы голос громов сильных, говорящих: аллилуия! ибо воцарился Господь Бог Вседержитель.
\vs Rev 19:7 Возрадуемся и возвеселимся и воздадим Ему славу; ибо наступил брак Агнца, и жена Его приготовила себя.
\vs Rev 19:8 И дано было ей облечься в виссон чистый и светлый; виссон же есть праведность святых.
\vs Rev 19:9 И сказал мне \bibemph{Ангел}: напиши: блаженны званые на брачную вечерю Агнца. И сказал мне: сии суть истинные слова Божии.
\vs Rev 19:10 Я пал к ногам его, чтобы поклониться ему; но он сказал мне: смотри, не делай сего; я сослужитель тебе и братьям твоим, имеющим свидетельство Иисусово; Богу поклонись; ибо свидетельство Иисусово есть дух пророчества.
\rsbpar\vs Rev 19:11 И увидел я отверстое небо, и вот конь белый, и сидящий на нем называется Верный и Истинный, Который праведно судит и воинствует.
\vs Rev 19:12 Очи у Него как пламень огненный, и на голове Его много диадим. \bibemph{Он} имел имя написанное, которого никто не знал, кроме Его Самого.
\vs Rev 19:13 \bibemph{Он был} облечен в одежду, обагренную кровью. Имя Ему: <<Слово Божие>>.
\vs Rev 19:14 И воинства небесные следовали за Ним на конях белых, облеченные в виссон белый и чистый.
\vs Rev 19:15 Из уст же Его исходит острый меч, чтобы им поражать народы. Он пасет их жезлом железным; Он топчет точило вина ярости и гнева Бога Вседержителя.
\vs Rev 19:16 На одежде и на бедре Его написано имя: <<Царь царей и Господь господствующих>>.
\vs Rev 19:17 И увидел я одного Ангела, стоящего на солнце; и он воскликнул громким голосом, говоря всем птицам, летающим по средине неба: летите, собирайтесь на великую вечерю Божию,
\vs Rev 19:18 чтобы пожрать трупы царей, трупы сильных, трупы тысяченачальников, трупы коней и сидящих на них, трупы всех свободных и рабов, и малых и великих.
\rsbpar\vs Rev 19:19 И увидел я зверя и царей земных и воинства их, собранные, чтобы сразиться с Сидящим на коне и с воинством Его.
\vs Rev 19:20 И схвачен был зверь и с ним лжепророк, производивший чудеса пред ним, которыми он обольстил принявших начертание зверя и поклоняющихся его изображению: оба живые брошены в озеро огненное, горящее серою;
\vs Rev 19:21 а прочие убиты мечом Сидящего на коне, исходящим из уст Его, и все птицы напитались их трупами.
\vs Rev 20:1 И увидел я Ангела, сходящего с неба, который имел ключ от бездны и большую цепь в руке своей.
\vs Rev 20:2 Он взял дракона, змия древнего, который есть диавол и сатана, и сковал его на тысячу лет,
\vs Rev 20:3 и низверг его в бездну, и заключил его, и положил над ним печать, дабы не прельщал уже народы, доколе не окончится тысяча лет; после же сего ему должно быть освобожденным на малое время.
\rsbpar\vs Rev 20:4 И увидел я престолы и сидящих на них, которым дано было судить, и души обезглавленных за свидетельство Иисуса и за слово Божие, которые не поклонились зверю, ни образу его, и не приняли начертания на чело свое и на руку свою. Они ожили и царствовали со Христом тысячу лет.
\vs Rev 20:5 Прочие же из умерших не ожили, доколе не окончится тысяча лет. Это~--- первое воскресение.
\vs Rev 20:6 Блажен и свят имеющий участие в воскресении первом: над ними смерть вторая не имеет власти, но они будут священниками Бога и Христа и будут царствовать с Ним тысячу лет.
\rsbpar\vs Rev 20:7 Когда же окончится тысяча лет, сатана будет освобожден из темницы своей и выйдет обольщать народы, находящиеся на четырех углах земли, Гога и Магога, и собирать их на брань; число их как песок морской.
\vs Rev 20:8 И вышли на широту земли, и окружили стан святых и город возлюбленный.
\vs Rev 20:9 И ниспал огонь с неба от Бога и пожрал их;
\vs Rev 20:10 а диавол, прельщавший их, ввержен в озеро огненное и серное, где зверь и лжепророк, и будут мучиться день и ночь во веки веков.
\rsbpar\vs Rev 20:11 И увидел я великий белый престол и Сидящего на нем, от лица Которого бежало небо и земля, и не нашлось им места.
\vs Rev 20:12 И увидел я мертвых, малых и великих, стоящих пред Богом, и книги раскрыты были, и иная книга раскрыта, которая есть книга жизни; и судимы были мертвые по написанному в книгах, сообразно с делами своими.
\vs Rev 20:13 Тогда отдало море мертвых, бывших в нем, и смерть и ад отдали мертвых, которые были в них; и судим был каждый по делам своим.
\vs Rev 20:14 И смерть и ад повержены в озеро огненное. Это смерть вторая.
\vs Rev 20:15 И кто не был записан в книге жизни, тот был брошен в озеро огненное.
\vs Rev 21:1 И увидел я новое небо и новую землю, ибо прежнее небо и прежняя земля миновали, и моря уже нет.
\vs Rev 21:2 И я, Иоанн, увидел святый город Иерусалим, новый, сходящий от Бога с неба, приготовленный как невеста, украшенная для мужа своего.
\vs Rev 21:3 И услышал я громкий голос с неба, говорящий: се, скиния Бога с человеками, и Он будет обитать с ними; они будут Его народом, и Сам Бог с ними будет Богом их.
\vs Rev 21:4 И отрет Бог всякую слезу с очей их, и смерти не будет уже; ни плача, ни вопля, ни болезни уже не будет, ибо прежнее прошло.
\vs Rev 21:5 И сказал Сидящий на престоле: се, творю все новое. И говорит мне: напиши; ибо слова сии истинны и верны.
\vs Rev 21:6 И сказал мне: совершилось! Я есмь Альфа и Омега, начало и конец; жаждущему дам даром от источника воды живой.
\vs Rev 21:7 Побеждающий наследует все, и буду ему Богом, и он будет Мне сыном.
\vs Rev 21:8 Боязливых же и неверных, и скверных и убийц, и любодеев и чародеев, и идолослужителей и всех лжецов участь в озере, горящем огнем и серою. Это смерть вторая.
\rsbpar\vs Rev 21:9 И пришел ко мне один из семи Ангелов, у которых было семь чаш, наполненных семью последними язвами, и сказал мне: пойди, я покажу тебе жену, невесту Агнца.
\vs Rev 21:10 И вознес меня в духе на великую и высокую гору, и показал мне великий город, святый Иерусалим, который нисходил с неба от Бога.
\vs Rev 21:11 Он имеет славу Божию. Светило его подобно драгоценнейшему камню, как бы камню яспису кристалловидному.
\vs Rev 21:12 Он имеет большую и высокую стену, имеет двенадцать ворот и на них двенадцать Ангелов; на воротах написаны имена двенадцати колен сынов Израилевых:
\vs Rev 21:13 с востока трое ворот, с севера трое ворот, с юга трое ворот, с запада трое ворот.
\vs Rev 21:14 Стена города имеет двенадцать оснований, и на них имена двенадцати Апостолов Агнца.
\vs Rev 21:15 Говоривший со мною имел золотую трость для измерения города и ворот его и стен\acc{ы} его.
\vs Rev 21:16 Город расположен четвероугольником, и длина его такая же, как и широта. И измерил он город тростью на двенадцать тысяч стадий; длина и широта и высота его равны.
\vs Rev 21:17 И стену его измерил во сто сорок четыре локтя, мерою человеческою, какова мера и Ангела.
\vs Rev 21:18 Стена его построена из ясписа, а город был чистое золото, подобен чистому стеклу.
\vs Rev 21:19 Основания стены города украшены всякими драгоценными камнями: основание первое яспис, второе сапфир, третье халкидон, четвертое смарагд,
\vs Rev 21:20 пятое сардоникс, шестое сердолик, седьмое хризолит, восьмое вирилл, девятое топаз, десятое хризопрас, одиннадцатое гиацинт, двенадцатое аметист.
\vs Rev 21:21 А двенадцать ворот~--- двенадцать жемчужин: каждые ворота были из одной жемчужины. Улица города~--- чистое золото, как прозрачное стекло.
\vs Rev 21:22 Храма же я не видел в нем, ибо Господь Бог Вседержитель~--- храм его, и Агнец.
\vs Rev 21:23 И город не имеет нужды ни в солнце, ни в луне для освещения своего, ибо слава Божия осветила его, и светильник его~--- Агнец.
\vs Rev 21:24 Спасенные народы будут ходить во свете его, и цари земные принесут в него славу и честь свою.
\vs Rev 21:25 Ворота его не будут запираться днем; а ночи там не будет.
\vs Rev 21:26 И принесут в него славу и честь народов.
\vs Rev 21:27 И не войдет в него ничто нечистое и никто преданный мерзости и лжи, а только те, которые написаны у Агнца в книге жизни.
\vs Rev 22:1 И показал мне чистую реку воды жизни, светлую, как кристалл, исходящую от престола Бога и Агнца.
\vs Rev 22:2 Среди улицы его, и по ту и по другую сторону реки, древо жизни, двенадцать \bibemph{раз} приносящее плоды, дающее на каждый месяц плод свой; и листья дерева~--- для исцеления народов.
\vs Rev 22:3 И ничего уже не будет проклятого; но престол Бога и Агнца будет в нем, и рабы Его будут служить Ему.
\vs Rev 22:4 И узрят лице Его, и имя Его будет на челах их.
\vs Rev 22:5 И ночи не будет там, и не будут иметь нужды ни в светильнике, ни в свете солнечном, ибо Господь Бог освещает их; и будут царствовать во веки веков.
\rsbpar\vs Rev 22:6 И сказал мне: сии слова верны и истинны; и Господь Бог святых пророков послал Ангела Своего показать рабам Своим то, чему надлежит быть вскоре.
\vs Rev 22:7 Се, гряду скоро: блажен соблюдающий слова пророчества книги сей.
\rsbpar\vs Rev 22:8 Я, Иоанн, видел и слышал сие. Когда же услышал и увидел, пал к ногам Ангела, показывающего мне сие, чтобы поклониться \bibemph{ему};
\vs Rev 22:9 но он сказал мне: смотри, не делай сего; ибо я сослужитель тебе и братьям твоим пророкам и соблюдающим слова книги сей; Богу поклонись.
\vs Rev 22:10 И сказал мне: не запечатывай слов пророчества книги сей; ибо время близко.
\vs Rev 22:11 Неправедный пусть еще делает неправду; нечистый пусть еще сквернится; праведный да творит правду еще, и святый да освящается еще.
\vs Rev 22:12 Се, гряду скоро, и возмездие Мое со Мною, чтобы воздать каждому по делам его.
\vs Rev 22:13 Я есмь Альфа и Омега, начало и конец, Первый и Последний.
\vs Rev 22:14 Блаженны те, которые соблюдают заповеди Его, чтобы иметь им право на древо жизни и войти в город воротами.
\vs Rev 22:15 А вне~--- псы и чародеи, и любодеи, и убийцы, и идолослужители, и всякий любящий и делающий неправду.
\rsbpar\vs Rev 22:16 Я, Иисус, послал Ангела Моего засвидетельствовать вам сие в церквах. Я есмь корень и потомок Давида, звезда светлая и утренняя.
\rsbpar\vs Rev 22:17 И Дух и невеста говорят: прииди! И слышавший да скажет: прииди! Жаждущий пусть приходит, и желающий пусть берет воду жизни даром.
\rsbpar\vs Rev 22:18 И я также свидетельствую всякому слышащему слова пророчества книги сей: если кто приложит что к ним, на того наложит Бог язвы, о которых написано в книге сей;
\vs Rev 22:19 и если кто отнимет что от слов книги пророчества сего, у того отнимет Бог участие в книге жизни и в святом граде и в том, что написано в книге сей.
\rsbpar\vs Rev 22:20 Свидетельствующий сие говорит: ей, гряду скоро! Аминь. Ей, гряди, Господи Иисусе!
\rsbpar\vs Rev 22:21 Благодать Господа нашего Иисуса Христа со всеми вами. Аминь.

\bibpart{Апокрифа}{Апокрифа}{Apo}
\bibbookdescr{1En}{
  inline={Первая книга Еноха},
  toc={1-я Еноха},
  bookmark={1-я Еноха},
  header={1-я Еноха},
  abbr={1~Ено}
}
\vs 1En 1:1
Слова благословения Еноха, которыми он благословил избранных и
праведных, которые будут жить в день скорби, когда все злые и нечестивые
будут отвержены.
И отвечал и сказал Енох,~--- праведный муж, которому были открыты
Богом очи,~--- что он видел на небесах святое видение: Его показали мне
ангелы, и от них я слышал всё, и уразумел, что видел, но не для этого рода,
а для родов отдалённых, которые явятся.
Об избранных говорил я и о них беседовал со Святым и Великим, с
Богом мира, Который выйдет из Своего жилища.
И оттуда Он придёт на гору Синай, и явится со Своими воинствами, и в
силе Своего могущества явится с неба.
И всё устрашится, и стражи содрогнутся, и великий страх и трепет
обоймёт их до пределов земли.
Поколеблются возвышенные горы, и высокие холмы опустятся, и растают,
как сотовый мёд от пламени.
Земля погрузится, и всё, что на земле, погибнет, и совершится суд
над всем и над всеми праведными.
Но праведным Он уготовит мир и будет охранять избранных, и милость
будет господствовать над ними; они все будут Божии, и хорошо им будет, и они
будут благословлены, и свет Божий будет светить им.
И вот Он идёт с мириадами святых,  чтобы совершить суд над ними,  и
Он уничтожит нечестивых, и будет судиться со всякою плотью относительно
всего, что грешники и нечестивые сделали и совершили против Него.
Я наблюдал всё, что происходит на небе: как светила, которые на
небе, не изменяют своих путей, как все они восходят и заходят по порядку,
каждое в своё время, и не преступают своих законов.
Взгляните на землю и обратите внимание на вещи, которые на ней, от
первой до последней, как каждое произведение Божие правильно обнаруживает
себя!
Взгляните на лето и зиму, как тогда (зимою) вся земля изобилует
водою, и тучи, и роса, и дождь стелются над нею!
Я наблюдал и видел, как зимою все деревья кажутся, будто они
высохли, и все листья их опали, кроме четырнадцати деревьев, которые не
обнажаются, но ожидают, оставаясь со старой листвой, появления новой в
течение двух--трёх лет.
И опять я наблюдал дни летние, как тогда солнце стоит над нею
(землёю), прямо против неё, а вы ищете прохладных мест и тени от солнечной
жары, и как тогда даже земля горит от зноя, а вы не можете ступить ни на
землю, ни на скалу (камень) вследствие их жара.
Я наблюдал, как деревья покрываются зеленью листьев и приносят
плоды; и вы обратите внимание на всё и узнайте, что всё это для вас сотворил
Тот, Который живёт вечно; посмотрите, как Его произведения существуют пред
Ним в каждом новом году и все Его произведения служат Ему и не изнемогают,
но как установил Бог, так всё и происходит!
И посмотрите, как моря и реки все вместе выполняют своё дело!
А вы не претерпели до конца и не выполнили закона Господня; но
преступили его и надменными, хульными словами поносили Его величие из своих
нечестивых уст; вы, жестокосердые, не обретёте никакого мира!
И посему вы проклянёте ваши дни, и годы вашей жизни прекратятся;
велико будет вечное осуждение, и вы не обретёте никакой милости.
В те дни вы лишитесь мира, чтобы быть вечным проклятием для всех
праведных, и они будут всегда проклинать вас как грешников,~--- вас вместе со
всеми грешниками.
Для избранных же настанет свет, и радость, и мир, и они наследуют
землю; а для вас, нечестивые, наступит проклятие.
Тогда избранным будет дана мудрость и они все будут жить и не
согрешат опять ни по небрежности, ни по надменности, но будут смирёнными, не
согрешая опять, так как имеют мудрость.
И они будут наказаны в продолжение своей жизни, и не умрут в муках
и в гневном осуждении, но окончат число дней своей жизни, а состареются в
мире, и годы их счастья будут многими: они будут пребывать в вечном
наслаждении и в мире в продолжение всей своей жизни.
\vs 1En 2:1
И случилось,~--- после того как сыны человеческие умножились в те
дни, у них родились красивые и прелестные дочери.
И ангелы, сыны неба, увидели их, и возжелали их, и сказали друг
другу: "давайте выберем себе жён в среде сынов человеческих и родим себе
детей"!
И Семъйяза,  начальник их,  сказал им: "Я боюсь, что вы не захотите
привести в исполнение это дело и тогда я  один  должен  буду  искупать
этот великий грех".
Тогда все они ответили ему и сказали: "Мы все поклянёмся клятвою и
обяжемся друг другу заклятиями не оставлять этого намерения, но привести его в
исполнение".
Тогда  поклялись  все  они вместе и обязались в этом все друг другу
заклятиями: было же их всего двести.
И они спустились на Ардис,  который есть вершина горы Ермон;  и они
назвали  её  горою  Ермон,  потому что поклялись на ней и изрекли друг другу
заклятия.
И вот имена их начальников:  Семъйяза, их начальник, Уракибарамеел,
Акибеел,  Тамиел,  Рамуел,  Данел, Езекеел, Саракуйял, Азаел, Батраал, Анани,
Цакебе, Самсавеел, Сартаел, Турел, Иомъйяел, Аразъйял. Это управители двухсот
ангелов, и другие все были с ними.
И они взяли себе жён, и каждый выбрал для себя одну; и они начали
входить к ним и смешиваться с ними, и научили их волшебству и заклятиям, и
открыли им срезывания корней и деревьев.
Они зачали и родили великих исполинов, рост которых был в три тысячи
локтей.
Они поели всё приобретение людей, так что люди уже не могли
прокармливать их.
Тогда исполины обратились против самих людей, чтобы пожирать их.
И они стали согрешать по отношению к птицам и зверям, и тому, что
движется, и рыбам, и стали пожирать друг с другом их мясо и пить из него кровь.
Тогда сетовала земля на нечестивых.
И Азазел научил людей делать мечи, и ножи, и щиты, и панцири, и
научил их видеть, что было позади них, и научил их искусствам: запястьям, и
предметам украшения, и употреблению белил и румян, и украшению бровей, и
украшению драгоценнейших и превосходнейших камней, и всяких цветных материй и
металлов земли.
И явилось великое нечестие и много непотребств, и люди согрешали, и
все пути их развратились.
Амезарак научил всяким заклинаниям и срезыванию корней, Армарос~---
расторжению заклятий, Баракал~--- наблюдению над звёздами, Кокабел~--- знамениям;
и Темел научил наблюдению над звёздами, и Астрадел научил движению Луны.
И когда люди погибли, они возопили и голос их проник к небу.
Тогда взглянули Михаил, Гавриил, Суръйян и Уръйян с неба и
увидели много крови, которая текла на земле, и всю неправду, которая
совершалась на земле.
И они сказали друг другу: "Голос вопля их (людей) достиг от
опустошённой земли до врат неба.
И ныне к вам, о святые неба, обращаются с мольбою души людей, говоря:
испросите нам правду у Всевышнего".
И они сказали своему Господу Царю: "Господь господей, Бог богов, Царь
царей!
Престол Твоей славы существует во все роды мира: Ты прославлен и
восхвалён!
Ты всё сотворил, и владычество над всем Тебе принадлежит: всё пред
Тобою обнаружено и открыто, и Ты видишь всё, и ничто не могло сокрыться пред
Тобою.
Так посмотри же, что сделал Азазел, как он научил на земле всякому
нечестию и открыл небесные тайны мира.
И заклинания открыл Семъйяза, которому ты дал власть быть вождём его
сообщников.
И пришли они (стражи) друг с другом к дочерям человеческими переспали
с ними, с этими жёнами, и осквернились, и открыли им эти грехи.
Жёны же родили исполинов, и чрез это вся земля наполнилась кровью и
нечестием.
И вот теперь разлученные души сетуют и вопиют к вратам неба и их
воздыхание возносится: они не могут убежать от нечестия, которое совершается
на земле.
И Ты знаешь всё, прежде чем это случилось, и Ты знаешь это и их дела,
и, однако же, ничего не говоришь нам.
Что мы теперь должны сделать с ними за это?
Тогда стал говорить Всевышний, Великий и Святый, и послал
Арсъйялалйюра к сыну Лемеха (Ною) и сказал ему: "Скажи ему Моим именем:
"скройся"!
и объяви ему предстоящий конец!
Ибо вся земля погибнет, и вода потопа готовится прийти на всю землю,
и то, что есть на ней, погибнет.
И теперь научи его, чтобы он спасся и его семя сохранилось для всей
земли"!
И сказал опять Господь Рафуилу: "Свяжи Азазела по рукам и ногам и
положи его во мрак; сделай отверстие в пустыне, которая находится в Дудаеле, и
опусти его туда.
И положи на него грубый и острый камень, и покрой его мраком, чтобы
он оставался там навсегда, и закрой ему лицо, чтобы он не смотрел на свет!
И в великий день суда он будет брошен в жар (в геенну).
И исцели землю, которую развратили ангелы, и возвести земле
исцеление, что Я исцелю её и что не все сыны человеческие погибнут чрез тайну
всего того, что сказали стражи и чему научили сыновей своих; и вся земля
развратилась чрез научения делам Азазела: ему припиши все грехи"!
И Гавриилу Бог сказал: "Иди к незаконным детям, и любодейцам, и к
детям любодеяния и уничтожь детей любодеяния и детей стражей из среды людей;
выведи их и выпусти, чтобы они сами погубили себя чрез избиения друг друга:
ибо они не должны иметь долгой жизни.
И  все они будут просить тебя, но отцы их (исполинов) ничего не
добьются для них (в пользу их), хотя они и надеются на вечную жизнь и на то,
что каждый из них проживёт пятьсот лет".
И Михаилу Бог сказал: "Извести Семъйязу и его соучастников, которые
соединились с жёнами, чтобы развратиться с ними во всей их нечистоте.
Когда все сыны их взаимно будут избивать друг друга и они увидят
погибель своих любимцев, то крепко свяжи их под холмами земли на семьдесят
родов до дня суда над ними и до окончания родов, пока не совершится последний
суд над всею вечностью.
В те дни их бросят в огненную бездну; на муку и в узы они будут
заключены на всю вечность.
И немедленно Семъйяза сгорит и отныне погибнет с ними; они будут
связаны друг с другом до окончания всех родов.
И уничтожь все сладострастные души и детей стражей, ибо они дурно
поступили с людьми.
Уничтожь всякое насилие с лица земли, и всякое злое деяние должно
прекратиться; и явится растение справедливости и правды, и всякое дело будет
сопровождаться благословением; справедливость и правда будут насаждать полную
радость в века.
И теперь в смирении будут поклоняться все праведные и будут пребывать
в жизни, пока не родят тысячу детей, и все дни своей юности и свои субботы они
окончат в мире.
В те дни вся земля будет обработана в справедливости, и будет вся
обсажена деревьями, и исполнятся благословения.
Всякие деревья веселия насадятся на ней, и виноградники насадят на
ней; виноградник, который будет насажен на ней, принесёт плод в изобилии, и от
всякого семени, которое будет на ней посеяно, одна мера принесёт десять тысяч,
и мера маслин даст десять пресов елея.
И ты очисть землю от всякого насилия, и от всякой неправды, и от
всякого греха, и от всякой нечистоты, какая совершается на земле, уничтожь их
с земли.
И все сыны человеческие должны сделаться праведными, и все народы
будут оказывать Мне почесть и прославлять Меня, и все будут поклоняться Мне.
И земля будет очищена от всякого развращения, и от всякого греха, и
от всякого наказания, и от всякого мучения; И Я никогда не пошлю опять на неё
потопа, от рода до рода вовек.
В те дни Я открою сокровищницы благословения, которые на небе,
чтобы низвести их на землю, на произведение и на труд сынов человеческих.
Мир и правда соединятся тогда на все дни мира и на все роды земли.
\vs 1En 3:1
И прежде чем всё это случилось, Енох был сокрыт, и никто из
людей не знал, где он сокрыт, и где он пребывает, и что с ним стало.
И вся его деятельность в течение земной жизни была со святыми и со
стражами.
--- И едва я, Енох, прославил великого Господа и Царя мира, как меня
призвали стражи,~--- меня, Еноха, писца,~--- и сказали мне: "Енох, писец правды!
Иди, возвести стражам неба, которые оставили вышнее небо и святые
вечные места, и развратились с жёнами, и поступили так, как делают сыны
человеческие, и взяли себе жён,  и погрузились на земле в великое
развращение: они не будут иметь на земле ни мира, ни прощение грехов: ибо они
не могут радоваться своим детям.
Избиение своих любимцев увидят они, и о погибели своих детей будут
воздыхать; и будут умолять, но милосердия и мира не будет для них".
И Енох пошёл и сказал Азазелу: "Ты не будешь иметь мира; тяжкий
суд учинён над тобою, чтобы взять тебя, связать тебя, и облегчение, ходатайство
и милосердие не будут долею для тебя за то насилие, которому ты научил, и за
все дела хулы, насилия и греха, которые ты показал сынам человеческим".
Тогда я пошёл далее и сказал всем им вместе; и они устрашились все,
страх и трепет объял их.
И они просили меня написать за них просьбу, чтобы чрез это они обрели
прощение, и вознести их просьбу на небо к Богу.
Ибо сами они не могли отныне ни говорить с Ним, ни поднять очей своих
к небу от стыда за свою греховную вину, за которую они были наказаны.
Тогда я составил им письменную просьбу и мольбу относительно
состояния их духа и их отдельных поступков и относительно того, о чём они
просили, чтобы чрез это получили они прощение и долготерпение.
И я пошёл, и сел при водах Дана в области Дан (т.е. к югу) от
западной стороны Ермона, и читал их просьбу, пока не заснул.
И вот нашёл на меня сон, и напало на меня видение; и я видел видение
суда, которое я должен был возвестить сынам неба и сделать им порицание.
И как только я пробудился от сна, то пришёл к ним; и все они сидели
печальные с закрытыми лицами, собравшись в Ублес-йяеле, который лежит между
Ливаном и Сенезером.
И я рассказал им все видения, которые видел во время своего сна, и
начал говорить те слова правды и порицать стражей неба.
То, что здесь далее написано, есть слово правды и наставления,
данное мне вечными стражами, как повелел им Святый и Великий в том видении.
Я видел во время видения моего сна то, что я буду теперь рассказывать
моим плотским языком и моим дыханием, которое Великий вложил в уста людям,
чтобы они говорили им и понимали это сердцем (мыслию).
Как сотворил Он всех людей и даровал им понимание слова благоразумия,
так Он сотворил и меня и дал мне право порицать стражей~--- сынов неба.
"Я написал вашу просьбу, и мне было открыто в видении, что именно
ваша просьба не будет для вас исполнена до всей вечности, дабы совершился над
вами суд, и ничто не будет для вас исполнено.
И отныне вы не взойдёте уже на небо до всей вечности и на земле вас
должны связать на все дни мира: такой произнесён приговор.
Но прежде этого вы увидите уничтожение ваших возлюбленных сынов, и вы
будете обладать ими, но они падут пред вами от меча.
Ваша просьба за них не будет исполнена для вас, как и та (моя)
просьба за вас; вы не можете даже в плаче и воздыхании произносить устами ни
одного слова из писания, которое я написал".
И видение мне явилось таким образом: вот тучи звали меня в видении и
облако звало меня; движение звёзд и молний гнало и влекло меня; и ветры в
видении дали мне крылья и гнали меня.
Они вознесли меня на небо, и я приблизился к одной стене, которая
была устроена из кристалловых камней и окружена огненным пламенем; и она стала
устрашать меня.
И я вошёл в огненное пламя, и приблизился к великому дому, который
был устроен из кристалловых камней; стены этого дома были подобны наборному
полу (паркет или мозаика) из кристалловых камней, и почвою его был кристалл.
Его крыша была подобна пути звёзд и молний с огненными херувимами
между нею (крышею) и водным небом.
Пылающий огонь окружал стены дома, и дверь его горела огнём.
И я вступил в тот дом, который был горяч как огонь и холоден как лёд;
не было в нём ни веселия, ни жизни~--- страх покрыл меня и трепет объял меня.
И так как я был потрясён и трепетал, то упал на своё лицо; и я видел
в видении.
И вот там был другой дом, больший, нежели тот; все врата его стояли
предо мной отворёнными, и он был выстроен из огненного пламени.
И во всём было так преизобильно: во славе, в великолепии и величии,
что я не могу дать описания вам его величия и его славы.
Почвою же дома был огонь, а поверх его была молния и путь звёзд, и
даже его крышею был пылающий огонь.
И я взглянул и увидел в нём возвышенный престол; его вид был как
иней, и вокруг него было как бы блистающее солнце и херувимские голоса.
И из-под великого престола выходили реки пылающего огня, так что
нельзя было смотреть на него.
И Тот, Кто велик во славе, сидел на нём; одежда Его была блестящее,
чем само солнце, и белее чистого снега.
Ни ангел не мог вступить сюда, ни смертный созерцать вид лица самого
Славного и Величественного.
Пламень пылающего огня был вокруг Него, и великий огонь находился
пред Ним, и никто не мог к Нему приблизиться из тех, которые находились около
Него: тьмы тем были пред Ним, но Он не нуждался в святом совете.
И святые, которые были вблизи Его, не удалялись ни днём, ни ночью и
никогда не отходили от Него.
И  я с тех пор имел покрывало на своём челе, потому что трепетал;
тогда позвал меня Господь собственными устами и сказал мне: "Пойди, Енох, сюда
и к Моему святому слову"!
И Он повелел подняться мне и подойти к вратам~--- я же опустил своё
лицо.
И Он отвечал и сказал мне Своим словом: Слушай!
Не страшись, Енох, ты праведный муж и писец правды; подойди сюда и
выслушай Моё слово!
И ступай, скажи стражам неба, которые послали тебя, чтобы ты просил
за них: вы должны попросить за людей, а не люди за вас.
Зачем вы оставили вышнее, святое, вечное небо, и преспали с жёнами,
и осквернились с дочерьми человеческими, и взяли себе жён, и поступали как сыны
земли, и родили сынов-исполинов?
Будучи духовными, святыми, в наслаждении вечной жизни, вы
осквернились с жёнами, кровию плотской родили детей, возжелали крови людей и
произвели плоть и кровь, как производят те, которые смертны и тленны.
Ради того-то Я им и дал жён, чтобы они оплодотворяли их, и чрез них
рождали бы детей, как это обыкновенно происходит на земле.
Но вы были прежде духовны, призваны к наслаждению вечной, бессмертной
жизни на все роды мира.
Посему Я не сотворил для вас жён, ибо духовные имеют своё жилище на
небе.
И теперь исполины, которые родились от тела и плоти, будут называться
на земле злыми духами и на земле будет их жилище.
Злые существа выходят из тела их; так как они сотворены свыше и их
начало и первое происхождение было от святых стражей, то они будут на земле
злыми духами, и будут называться злыми духами.
А духи неба имеют своё жилище на небе, а духи земли, родившиеся на
земле, имеют своё жилище на земле.
И духи исполинов, которые устремляются на облака, погибнут, и будут
низринуты, и станут совершать насилие, и производить разрушения на земле, и
причинять бедствия; они не будут принимать пищи, и не будут жаждать, и будут
невидимы.
И те существа не восстанут против сынов человеческих и против жён,
так как они произошли от них.
В дни избиения и погибели и смерти исполинов, лишь только души
выйдут из тел, их тело должно предаться тлению без суда; так будут погибать
они до того дня, когда великий суд совершится над великим миром,~--- над стражами
и нечестивыми людьми.
И теперь скажи стражам, которые послали тебя, чтобы ты просил за них,
и которые жили прежде на небе, теперь скажи им: "Вы были на небе, и хотя
сокровенные вещи не были ещё открыты вам, однако вы узнали незначительную тайну
и рассказали её в своём жестокосердии жёнам, и чрез эту тайну жёны и мужья
причиняют земле много зла".
Скажи им: "Для вас нет мира".
\vs 1En 4:1
И они (ангелы) унесли меня в одно место, где были фигуры, как
пылающий огонь, и когда они хотели, то казались людьми.
И они привели меня к месту бури и на одну гору, конец вершины которой
доходил до неба.
И я увидел ярко блестящие места и гром на краях их; в глубине этого
огненный лук стрелы и колчан для них, и огненный меч, и все молнии.
И они донесли меня до так называемой воды и до огня запада, который
принимает в себя каждый вечер заходящее солнце.
И я пришёл к огненной реке, огонь которой жидкий, как вода, и которая
впадает в великое море к западу.
И я видел все великие реки, и дошёл до великого мрака, и пришёл туда,
где шествуют все смертные.
И  я  видел горы мрачных туч зимнего времени и место, куда впадает
вода целой бездны.
И я видел устье всех рек земли и устье бездны.
И я видел хранилища всех ветров, и видел, как Он изукрасил этим
всё творение, и видел основание земли.
И я видел краеугольный камень земли, и видел четыре ветра, которые
носят землю и основание неба.
И я видел, как ветры растягивают высоты неба, и они носятся между
небом и землёю~--- это столпы неба.
И я видел ветры, которые кружат небо, которые несут солнечный круг и
все звёзды к заходу.
И я видел ветры на земле, которые носят тучи; и видел пути ангелов,
и видел в конце земли вверху основание неба.
И я пошёл далее к югу, который горит день и ночь,~--- туда, где
находятся семь гор из драгоценных камней,~--- три к востоку и три к югу: и
именно те, которые к востоку, одна из цветных камней, и одна из перловых
камней, и одна из сурьмы; а те, которые к югу, из красных камней.
Средняя же, достигавшая до неба, как престол Божий, была из
алебастра, и вершина престола из сапфира.
И я видел пылающий огонь, который был во всех горах.
И я видел там одно место по ту сторону великой земли: там собирались
воды.
И я видел глубокую расселину в земле со столбами небесного огня; и я
видел между ними ниспадающие столбы небесного огня, которые нельзя было
сосчитать ни в направлении к верху, ни к низу.
И над тою расселиной я видел одно место, которое не имело ни небесной
тверди над собою, ни земного основания под собою; на нём не было ни воды, ни
птиц, но это было пустое место.
И было ужасно то, что я видел там: семь звёзд, как великие горящие
горы и как духи, которые просили меня.
Ангел сказал мне: "Это то место, где оканчивается небо и земля; оно
служит темницей для звёзд небесных и для воинства небесного.
И эти звёзды, которые катятся над огнём, суть те самые, которые
преступили повеление Божие пред своим восходом, так как они пришли не в своё
определённое время.
И Он разгневался на них и связал их до времени, когда окончится их
вина,~--- в год тайны".
И Уриил сказал мне: "Здесь будут находиться духи ангелов,
которые соединились с жёнами и, принявши различные виды, осквернили людей и
соблазнили их, чтобы они приносили жертвы демонам, как богам,~--- будут
находиться именно в тот день, когда над ними будет произведён великий суд, пока
не постигнет их конечная участь.
Так же и с жёнами их, которые соблазнили ангелов неба, будет
поступлено точно так же, как и с друзьями их.
И только я, Енох, созерцал пределы всего, и ни один человек не видел
их так, как видел их я.
И вот имена святых ангелов, которые стерегут: Уриил, один из
святых ангелов, ангел грома и колебания; Руфаил, один из святых ангелов, ангел
духов людей; Рагуил, один из святых ангелов, который карает мир и светила;
Михаил, один из святых ангелов, поставленный над лучшею частью людей,~--- над
избранным народом; Саракаел, один из святых ангелов, который поставлен над
душами сынов человеческих, склонивших духов к греху; Гавриил, один из святых
ангелов, который поставлен над змеями, и над раем, и над херувимами.
И я обошёл кругом до одного места, где не было никакой вещи.
И я видел там нечто страшное, ни небо возвышенное и ни землю
утверждённую, но одно пустое (пустынное) место, величественное и страшное.
И здесь я видел семь звёзд небесных, вместе связанных в этом месте,
подобным великим горам и пылающим как бы огнём.
На этот раз я сказал: "За какой грех они связаны и за что они сюда
изгнаны"?
Тогда мне сказал Уриил, один из святых ангелов, который был при мне
как мой путеводитель: "Енох, для чего ты разведываешь, и для чего разузнаёшь,
и спрашиваешь, и любопытствуешь?
Это те звёзды, которые преступили повеление Всевышнего Бога, и они
связаны здесь до тех пор, пока не окончится тьма миров,~--- число дней их вины".
И отсюда я пошёл в другое место, которое было ещё страшнее, чем это,
и увидел нечто страшное: там был великий огонь, который пылал и горел, и он
имел разделения; он был ограничен (окружён) совершенною пропастью; великие
огненные столбы низвергались туда; но его (огня) протяжения и величины я не мог
рассмотреть, и не в состоянии был даже взглянуть, откуда он происходит.
Тогда я сказал: "Как страшно это место и как мучительно осматривать
его!"
Тогда отвечал мне Уриил, один из святых ангелов, который был при мне;
он отвечал мне и сказал: "Енох, к чему такой страх и трепет в тебе на этом
ужасном месте и при виде этого мучения?"
И он сказал мне: "Это место~--- темница ангелов, и здесь они будут
содержаться заключёнными до вечности".
\vs 1En 5:1
Отсюда я пошёл в другое место, и он (Руфаил) показал мне на
западе большой высокий горный хребет, твёрдые скалы и четыре прекрасных места.
И между ними (последними) были глубокие, и обширные, и совершенно
выглаженные настолько гладко, как нечто, что катится, и глубокое, и мрачное
на вид.
На этот раз ответил мне Руфаил, один из святых ангелов, который был
со мною, и сказал мне: "Эти прекрасные места назначены для того, чтобы на них
собирались духи,~--- души умерших; для них они созданы, чтобы все души сынов
человеческих собирались здесь.
Места эти созданы для них местами жилища до дня их суда и до
определённого для них срока, и срок этот велик: он продолжится дотоле, пока не
совершится над ними великий суд".
И я видел духов сынов человеческих, которые умерли, и их голос
проникал до неба и сетовал.
На этот раз я спросил ангела Руфаила, который был со мною, и сказал
ему: "Чей это там дух, голос которого так проникает вверх и сетует?"
И он отвечал мне и сказал мне так: "Это дух, который вышел из Авеля,
убитого своим братом Каином; и он жалуется на него, пока семя его (Каина) не
будет изглажено с лица земли и из семени людей не будет уничтожено его семя".
И поэтому я спросил тогда о нём (об Авеле) и о суде над всеми и
сказал: "Почему одно место отделено от другого?"
И он отвечал мне и сказал мне: "Эти три остальные отделения сделаны
для того, чтобы разделять души умерших.
И души праведных отделены таким образом: там есть источник воды, над
которым свет.
Точно также сделано такое отделение и для грешников, когда они
умирают и погребаются на земле без того, что суд над ними ещё не произведён при
их жизни.
Здесь отделены их души, в этом великом мучении, пока не наступит
великий день суда и наказания, и мучения для хулителей до вечности, и мщения
для их душ; и он (ангел наказания) связал их здесь до вечности.
И если это было пред вечностью, тогда это (последнее) отделение
сделано для душ тех, которые сетуют и возвещают о своей погибели, так как они
были умерщвлены в дни грешников.
Таким образом, это отделение сделано для душ людей, которые были не
праведными, а грешниками, скончавшись в вине; они будут находиться возле
виновных и подобны им, но их души не умрут до дня суда и не выйдут отсюда.
Тогда я прославил Господа славы и сказал: "Будь прославлен, Господь
мой, Господь славы и справедливости, всё направляющий в вечность!"
Оттуда я пошёл в другое место к западу, к пределам земли.
И я видел здесь горящий огонь, который тёк беспрерывно, и ни днём, ни
ночью не прекращал своего течения, но равномерно тёк.
И я спросил Рагуила, говоря: "Что это такое там, что не имеет покоя?"
На этот раз отвечал мне Рагуил, один из святых ангелов, который был
со мною, и сказал мне: "Этот горящий огонь на западе, течение которого ты
видел, есть огонь всех светил небесных".
Оттуда я пошёл в другое место земли, и он (Михаил) показал мне
там горный хребет огненный, который горел день и ночь.
И я взошёл на него и увидел семь великолепных гор, из которых каждая
отделена от другой, и великолепные (драгоценные), прекрасные камни; всё было
великолепно и славного вида и прекрасной видимости; три горы расположены к
востоку, одна над другой укреплена, и три к югу, одна над другой укреплена;
здесь были и глубокие вьющиеся долины, из которых ни одна не примыкала к
другой.
И седьмая гора была между ними; в своей же вышине они все были
подобны тронному седалищу, которое было окружено благовонными деревьями.
И  между ними было одно дерево с благоуханием, которого я ещё никогда
не обонял ни от тех, ни от других деревьев; и никакой другой запах не был похож
на его запах; его листья, цветы, ствол не гниют вечно, и плод его прекрасен; а
его плод подобен грозду пальмы.
На этот раз я сказал: "Посмотри на это прекрасное дерево: прекрасны
на вид и приятны его листья (ветви), и его плод очень приятен для ока".
Тогда отвечал мне Михаил, один из святых и почитаемых ангелов, бывший
со мною, который был поставлен над этим.
И он сказал мне: "Енох, что ты спрашиваешь меня о запахе этого
дерева и стремишься узнать?"
Тогда я, Енох, отвечал ему, говоря: "Обо всём желал бы я узнать, но
особенно об этом дереве".
И он отвечал мне, говоря: "Эта высокая гора, которую ты видел, и
вершина, которая подобна престолу Господа, есть Его престол, где воссядет
Святый и Великий, Господь славы, вечный Царь, когда Он сойдёт, чтобы посетить
землю с милостью.
И к этому дереву с драгоценным запахом не позволено прикасаться ни
одному из смертных до времени великого суда; когда всё будет искуплено и
окончено для вечности, оно будет отдано праведным и смиренным.
От его плода будет дана жизнь избранным; оно будет пересажено на
север к святому месту,~--- к храму Господа, великого Царя.
Тогда они будут радоваться полною радостью и ликовать в Святом; они
будут воспринимать запах его в свои кости, и продолжительную жизнь они будут
жить на земле, как жили их отцы; и в дни их жизни не коснётся их ни печаль, ни
горе, ни труд, ни мучение".
Тогда я прославил Господа славы, вечного Царя, за то, что Он уготовал
это для праведных людей, и создал такое, и обещал дать им.
И оттуда я пошёл в средину земли, и видел благословенное и
плодородное место, где были ветви, которые укоренялись и вырастали из
срубленного дерева.
И там я видел святую гору, и под горой~--- к востоку от неё~--- воду,
которая текла к югу.
И я видел к востоку другую гору такой же вышины, и между ними обоими
глубокую долину, но неширокую; в ней также текла вода возле горы.
И на западе от неё была другая гора, ниже той и невысокая, и внизу
её, между ними (горами) обоими, была долина; и другие долины глубокие и сухие
были в конце всех трёх.
И все долины были глубокие, но не широкие, из твёрдого скалистого
камня; и деревья были насажены на них.
И я удивился скалам, и удивился долине, и удивился чрезвычайно.
Тогда я сказал: "Для чего эта благословенная страна, которая
вся наполнена деревьями, и в промежутке (между горами) эта проклятая долина?"
Тогда отвечал мне Уриил, один из святых ангелов, который был со мною,
и сказал мне: "Эта проклятая долина для тех, которые прокляты до вечности;
здесь должны собраться все те, которые говорят своими устами непристойные речи
против Бога, и дерзко говорят о Его славе; здесь соберут их, и здесь место их
наказания.
И в последнее время будет зрелище праведного суда над ними пред лицом
праведных навсегда в вечности; за это те, которые обрели милосердие, будут
прославлять Господа славы, вечного Царя.
И в дни суда над ними (грешниками) они (праведные) прославят Его за
милосердие, по которому он назначил им такой жребий".
Тогда и я прославил Господа славы, и говорил к Нему, и вспоминал Его
величие, как подобает.
Оттуда я пошёл к востоку, в самую средину горного хребта,
(находящегося в) пустыне, и здесь я не видел ничего, кроме одной равнины.
Но она была наполнена деревьями тех же семян, и вода струилась на неё
сверху.
Можно было видеть, насколько орошение, которое она поглощала, было
обильное, можно было видеть и то, что как на севере, так и на западе и как
повсюду, так и здесь поднималась вода и роса.
И я пошёл в другое место, прочь от пустыни, приближаясь к
горному хребту на востоке.
И там я видел деревья суда, особенно же такие, которые издают запах
ладана и мирры и которые были не похожи на обыкновенные деревья.
И над этим, высоко над ними (деревьями), над восточною горою и
недалеко от неё, видел я другое место, именно~--- долины с водой, которые не
иссякают.
И я видел прекрасное дерево, запах которого, как запах мастикса.
И по сторонам тех долин я видел благовонную корицу.
И я поднялся вверх над ними (долинами или деревьями), направляясь
ближе к востоку.
И я видел другую гору с деревьями, из которой текла вода и из
которой выходило нечто подобное нектару, что называют сарира и гальбан.
И  над той горой я видел другую гору, на которой были алойные
деревья; и те деревья изобиловали миндалеподобным твёрдым веществом.
И если взять тот плод, то он был лучше, чем всякие благовония.
И после этих благовоний, как только я взглянул к северу выше тех
гор, я увидел там ещё семь гор, изобиловавших драгоценными нардами и
благовонными деревьями, корицей и перцем.
Оттуда я пошёл на вершину тех гор далеко к востоку, и подвинулся
далее, пройдя над Эритрейским морем, и ушёл далеко от него, и прошёл над
ангелом Цутелем.
И я пришёл в сад правды и увидел разнообразное множество тех
деревьев; там росло много больших деревьев,~--- благовонных, великих, очень
прекрасных и великолепных,~--- и дерево мудрости, доставляющее великую мудрость
тем, которые вкушают от него.
И оно похоже на кератонию; его плод, подобный виноградной кисти,
очень прекрасен; запах дерева распространяется и проникает далеко.
И я сказал: "Как прекрасно это дерево и как прекрасен и прелестен его
вид!"
И святой ангел Руфаил, который был со мною, отвечал мне и сказал:
"Это то самое дерево мудрости, от которого твои предки, твой старый отец и
старая мать вкусили и обрели познание мудрости, и у них открылись очи, и они
узнали, что были наги и были изгнаны из сада".
Оттуда я пошёл к пределам земли и увидел там великих зверей, из
которых каждый был отличен от другого, а также птиц, разнообразных по наружной
красоте и по голосу, из которых каждая была отлична от другой.
И на востоке от тех зверей я видел пределы земли, на которых покоится
небо, и открытые врата неба.
И я видел, как выходят звёзды небесные, и сосчитал врата, из которых
они выходят, и записал все выходы их,~--- о каждой из них особо, по числу их, их
именам, их связи, их положению, их времени и их месяцам,~--- так, как показал мне
это ангел Уриил, который был со мною.
Всё показал он мне и записал мне; их имена он также записал для меня,
и их законы и их отправления.
Оттуда я пошёл к северу к пределам земли, и там я видел великое
и славное чудо на пределах всей земли.
Здесь я видел трое открытых небесных врат на небе; из них выходят
северные ветры; если там (из них) дует, то бывает холод, град, иней, снег, роса
и дождь.
Из одних врат (средних) дует ко благу; но если они (ветры) дуют чрез
двое других врат, то бывает бурно и на землю приносится бедствие, и они дуют
тогда бурно.
Оттуда я пошёл к западу к пределам земли и увидел тогда трое
открытых врат подобно тому, как я видел их на востоке,~--- одинаковые врата и
одинаковые выходы.
Оттуда я пошёл на юг к пределам земли и видел там двое открытых
врат неба; из них выходит южный ветер, а с ним~--- роса, дождь и ветер.
Оттуда я пошёл к востоку к пределам неба и видел здесь трое восточных
небесных врат открытых и над ними маленькие врата.
Чрез каждые маленькие врата проходят звёзды небесные и бегут к вечеру
(к западу) на колеснице, которая им назначена.
И как только я увидел это, то прославил Господа, и таким образом я
всякий раз прославлял Господа славы, который сотворил великие и славные чудеса,
чтобы показать величие Своего творения ангелам и душам людей, дабы они
восхваляли Его творение и дабы все Его твари видели дело Его могущества,
восхваляли великое дело Его рук и славили Его довеку.
\vs 1En 6:1
Второе видение мудрости, которое видел Енох, сын Иареда, сына
Малелеила, сына Каинана, сына Еноса, сына Сифа, сына Адама.
И вот начало речи мудрости,  которую я начал говорить и высказывать
живущим на земле; слушайте вы, древнейшие, и обратите внимание, потомки, на
святые слова, которые я буду говорить пред Господом духов.
Справедливо назвать тех (древних) прежде всего, но и потомков мы не
будем удерживать от начала премудрости.
И до сегодня никогда не была дарована от Господа духов кому-либо та
мудрость, которую я получил по моему разумению, по благоволению
Господа духов, от которого мне назначен жребий вечной жизни.
Три притчи были долею для меня, и я начал их рассказывать тем, которые
населяют твердь.
\vs 1En 7:1
Первая притча.
Когда откроется общество праведных, и грешники будут судимы за свои
грехи, и будут изгнаны с лица земли, и когда Праведный явится пред очами
избранных праведников, дела которых взвешены Господом духов, и свет откроется
праведным и избранным, живущим на земле,~--- то где тогда будет жилище грешников
и убежище тех, которые отвергли Господа духов?
Было бы лучше для них, если б они никогда не рождались.
И когда тайны праведных будут открыты, тогда грешники будут судимы и
нечестивые будут отвергнуты от праведных и избранных.
И отныне не будут более сильными и вознесёнными те, которые владеют
землёю, и не будут в состоянии видеть лицо святых, ибо свет Господа духов будет
сиять на лица святых, и праведных, и избранных.
И сильные цари погибнут в то время и будут преданы в руки праведных и
святых.
И с тех пор никто не будет (иметь возможности) молить Господа духов о
милости, ибо жизнь их (людей) окончится.
И это случится в те дни, когда избранные и святые дети сойдут с
высоких небес и их семя соединится с сынами человеческими.
В те дни Енох получил книги гнева и ярости и книги беспокойства и
смятения, и в это самое время меня унесла прочь от земли туча и буря, и
принесла меня к пределам неба.
И здесь я видел другое видение, именно~--- жилища праведных и ложа
святых.
Здесь мои очи видели жилища возле ангелов и их ложа возле святых,
видел, как они молились, и просили, и умоляли за сынов человеческих, и правда
текла пред ними, как вода, и милосердие, как роса на земле: так бывает между
ними от века до века.
И в те дни мои очи видели место избранных правды и веры, и как правда
господствует в те дни, и как неисчислимо велико множество праведных и избранных
пред Ним от века до века.
И я видел жилища их под крыльями Господа духов, и видел, как все
праведные и избранные украшены пред Ним как бы огненным сиянием, и их уста
полны славословия, и их губы хвалят имя Господа духов, и правда не преходит
пред Ним.
Здесь желал я жить, и моя душа стремилась к тому жилищу; здесь уже
прежде была уготована мне участь, ибо так постановлено относительно меня у
Господа духов.
В те дни я хвалил и превозносил имя Господа духов благословениями и
славословиями, ибо Он определил мне благословение и славу.
Долго рассматривали мои очи то место, и я прославил Его (Господа),
говоря: Хвала Ему и да прославится Он от начала до вечности!
Пред Ним нет прехождения; Он знает, прежде чем создан мир, что он
такое и что будет от рода до рода.
Тебя славят те, которые не спят они стоят пред Тобою славою, и
прославляют, хвалят и превозносят Тебя, говоря: "свят, свят, свят
Господь духов, Он наполняет землю духами!"
И здесь мои очи видели всех тех, которые не спят, как они стоят пред
Ним, и прославляют, и говорят: "Будь прославлен Ты и да будет прославлено имя
Господа от века до века!"
И моё лицо изменилось, так что я не мог больше видеть.
И после этого я видел тысячу тысяч и тьму тем, несметно и
неисчислимо многих, стоящих пред славою Господа духов.
Я видел, и на четырёх сторонах престола Господа духов я заметил
четыре лица, отличные от тех, которые стояли там (1 ст.), и я узнал имена их,
так как ангел, пришедший со мною (или ко мне), открыл мне имена их и показал
мне все сокровенные вещи.
И я слышал глас тех четырёх лиц, как они пели хвалу пред Господом
славы.
Первый голос прославляет Господа духов от века и до века.
И другой голос слышал я, прославляет Избранного и избранных, которые
взвешены Господом духов.
И третий голос слышал я, просит, и молится за живущих на земле, и
умоляет во имя Господа духов.
И слышал я четвёртый голос, как он отражал врагов и не дозволял им
приступить к Господу духов, чтобы клеветать или жаловаться на живущих на земле.
После этого я спросил ангела мира, шедшего со мною, который показал
мне всё, что сокрыто, и сказал ему: "Кто эти четыре лица, которых я видел и
глас, которых я слышал и записал?"
И он сказал мне: "Этот первый~--- есть милосердный и долготерпеливый
святой Михаил; и другой, поставленный над всеми болезнями и над всеми ранами
сынов человеческих, есть Руфаил; и третий, поставленный над всеми силами, есть
святой Гавриил; и четвёртый, поставленный над покаянием и надеждою тех,
которые получат в наследие вечную жизнь, есть Фануил".
И вот четыре ангела всевышнего Бога, и четыре голоса их я слышал в те
дни.
И после этого я видел все тайны неба, и как разделено царство, и
как дела людей взвешены на весах.
Там видел я жилища избранных и жилища святых; и мои очи видели там,
как изгоняются оттуда все грешники, которые отвергли имя Господа духов, как
отражают их, и для них там нет места вследствие наказания, которое исходит от
Господа духов.
И там мои очи видели тайны молний и грома, и тайны ветров, как они
распределены,  чтобы дуть на землю, и тайны туч и росы; и там видел я,
откуда они выходят в том самом месте и как отсюда насыщается пыль
земная.
И там видел я замкнутые хранилища, из которых распределяются ветра,
и хранилища града, и хранилища тумана и туч, и Его туча, которая носится над
землёю до вечности.
И я видел хранилища Солнца и Луны, откуда они выходят и куда
возвращаются, и их славное возвращение; и я видел, как одно (т.е. Солнце) имеет
преимущество перед другой, видел и их определённое движение, как они не
преступают пути, ничего не прибавляя к своему пути и ничего не убавляя от него,
и соблюдают верность между собой, сохраняя клятву.
Прежде всего, выходит Солнце и совершает свой путь по Вселенной
Господа духов, и могущественно имя Его от века до века; за ним следует видимый
и невидимый путь Луны; и я видел, как она оканчивает движение по своему пути в
том месте днём и ночью, одно (светило, т.е. Луна), противостоя другому
(Солнцу), пред Господом духов; и они благодарят, и прославляют, и не
успокаиваются, так как их благодарение служит для них покоем.
Ибо сияющее Солнце совершает много обращений для благословения и для
проклятия; и движение Луны по её пути есть свет для праведных и для грешников
во имя Господа, Который положил разделение между светом и тьмою, и разделил
души людей, и утвердил души праведных во имя Своей правды.
Ибо ни ангел не нарушает этого, и никакая сила не может нарушить
этого (установленного Богом), но Судья видит их все (души людей) и судит их все
пред Собою.
Мудрость не нашла на земле места, где бы ей жить, и потому
жилище её стало на небесах.
Пришла мудрость, чтобы жить между сынами человеческими, не нашла себе
места; тогда мудрость возвратилась назад в своё место, и заняла своё положение
между ангелами.
И неправда вышла из своих хранилищ: не искавшая его (приёма), она
нашла его и жила между людьми, как дождь в пустыне и как роса в земле жаждущей.
И видел я опять молнии и звёзды небесные, как Он призывал их
всех отдельно по именам и они внимали Ему.
И я видел, как они взвешены правильными весами по мере их света, по
обширности их мест и времени их появления и обращения (видел, как одна молния
рождает другую), и их обращение по числу ангелов, и как они сохраняют между
собой верность.
И я спросил ангела, который шёл со мною и показал мне, что сокрыто:
"Кто это"?
И он сказал мне: "Образ их показал тебе Господь духов: это имена
праведных, которые живут на земле и веруют во имя Господа духов во всю
вечность".
И иное также видел я относительно молний, как они возникают из
звёзд, и становятся молниями, и ничего не могут удержать при себе.
\vs 1En 8:1
И  вот вторая притча относительно тех, которые отвергают имя
жилища святых и имя Господа духов.
Они не взойдут на небо, и на землю не придут они: таков будет жребий
грешников, которые отвергают имя Господа духов и которые сохраняются, таким
образом, на день страданий и скорби.
В тот день Избранный сядет на престоле славы и произведёт выбор между
делами их (людей) и местами без числа, и дух их сделается сильным в их
внутренности, ибо они увидят моего Избранного и тех, которые умаляли Моё святое
и славное имя.
И в тот день Я пошлю Моего Избранного жить между ними, и преобразую
небо, и приготовлю его для вечного благословения и света.
И я изменю землю, и приготовлю её для благословения,  и поселю  на
ней Моих избранных; грех же и преступления исчезнут на ней,~--- они не появятся.
Ибо Я увидел и насытил миром Моих праведных, и поставил их пред Собою;
для грешников же у Меня предстоит суд, дабы уничтожить их с лица земли.
И там я видел Единого, имевшего главу дней (престарелую  главу), и
Его глава была бела, как руно; и при Нём был другой, лице которого было подобно
виду человека, и Его лице полно было прелести и подобно одному из святых
ангелов.
И я спросил одного из ангелов, который шёл со мною и показывал мне все
сокровенные вещи, о том Сыне человеческом, кто Он, и откуда Он, и почему Он
идёт с Главою дней?
И он отвечал мне и сказал: "Это Сын человеческий, Который имеет правду,
при Котором живёт правда, и Который открывает все сокровища того, что сокрыто,
ибо Господь духов избрал Его, и жребий Его пред Господом духов превзошёл всё,
благодаря праведности.
И этот Сын человеческий, Которого ты видел, поднимет царей и
могущественных с их лож и сильных с их престолов, и развяжет узы сильных, и
зубы грешников сокрушит.
И Он изгонит царей с их престолов и из их царств, ибо они не
превозносят Его, и не прославляют Его, и не признают с благодарностью, откуда
досталось им царство.
И лицо сильных Он отвергнет, и краска стыда покроет их; мрак будет их
жилищем, и слёзы их ложем, и они не будут иметь надежды встать со своих лож,
так как они не превозносят имя Господа духов.
И это те, которые осуждают звёзды небесные и возвышают свои руки
против Всевышнего, и попирают землю и на ней живут; все дела их неправда, и они
открывают неправду; сила их основывается на богатстве, и вера их относится к
богам, сделанным их же руками; и они отвергли Господа духов.
И они изгоняются из домов их общественного собрания и из домов
верующих, которые взвешены во имя Господа духов.
И в те дни восходит молитва праведных и кровь праведного от земли
к Господу духов.
В те дни святые ангелы, живущие вверху на небесах, соединившись
вместе,
будут единым гласом просить, и молить, и прославлять, и благодарить, и
восхвалять имя Господа духов ради крови праведных, которая пролита, и ради
молитв праведных, что она не может быть тщетной пред Господом духов и что
совершён суд для них, и им не нужно терпеть (или дожидаться суда) вечного.
И в те дни я видел Главу дней как Он восседал на престоле своей славы
и книги живых были раскрыты пред Ним, и видели всё Его воинство, которое
находится вверху и на небесах и окружает Его, предстоя пред Ним.
И сердца святых были полны радостью, ибо исполнилось число правды, и
молитвы праведных услышана, и кровь праведного искуплена (или отомщена) пред
Господом духов.
И в том месте я видел источник правды, который был неисчерпаем;
его окружали вокруг многие источники мудрости, и все жаждущие пили из них и
исполнялись мудростью, и имели свои жилища около праведных, и святых, и
избранных.
И в тот час был назван тот Сын человеческий возле Господа духов и Его
имя пред Главою дней.
И прежде чем Солнце и знамения были сотворены, прежде чем звёзды
небесные были созданы, Его имя было названо пред Господом духов.
Он будет жезлом для праведных и святых, чтобы они опёрлись на Него и
не падали; и Он будет светом народов и чаянием тех, которые опечалены в своём
сердце.
Пред Ним упадут и поклонятся все живущие на земле, и будут хвалить и
прославлять, и петь хвалу имени Господа духов.
И посему Он был избран и сокрыт пред Ним, прежде даже чем создан мир;
и Он будет пред Ним до вечности.
И премудрость Господа духов открыла Его святым и избранным, ибо Он
охраняет жребий праведных, так как они возненавидели и презрели этот мир
неправды, и все его произведения и пути возненавидели во имя Господа духов; ибо
во имя Его они спасаются, и Он становится мстителем за их жизнь.
И в те дни потупили взор цари земли и сильные, владеющие твердью,
страшась за дела своих рук, ибо в день своей печали и бедствия они не спасут
своих душ.
И Я предал их в руки Моих избранных: как солома в огне и как свинец в
воде, они сгорят пред лицом праведных и потонут пред лицом святых, и никакого
следа более не останется от них.
И в день их бедствия водворится покой на земле; они падут пред Ним и
не восстанут опять; не будет никого, кто бы взял их в свои руки и поднял: ибо
они отвергли Господа духов и Его Помазанника.
Имя Господа духов будет прославлено.
Ибо мудрость излилась на Сына человеческого, как вода, и слава не
прекращается пред Ним от века до века.
Ибо Он силён во всех тайнах правды, и неправда прейдёт пред Ним, как
тень, и не будет иметь постоянства, так как Избранный восстал пред Господом
духов; и Его слава от века до века, и Его могущество от рода до рода.
В Нём живёт дух мудрости и дух Того, Кто даёт проницательность, и дух
учения и силы, и дух тех, которые почили в правде.
И Он будет судить сокровенные вещи, и никто не осмелится вести пред
Ним пустую речь, ибо Он избран пред Господом духов, и Его благоволению.
И в те дни совершится перемена со святыми и избранными: свет дней
будет обитать пред ними, и слава, и честь будут дарованы святым.
И в день бедствия соберётся нечестие на грешников, праведные же
победят во имя Господа духов; и Он покажет это другим, чтобы они принесли
покаяние и оставили дела своих рук.
Они не будут иметь чести пред Господом духов, но будут спасены во имя
Его; И Господь духов умилосердится над ними, ибо Его милосердие велико.
И праведен Он в Своём суде и пред Его славою, и на Его суде не устоит
неправда: кто не принесёт покаяния пред Ним, тот погибнет.
Но отныне Я не буду более милосердным к ним, говорит Господь духов.
И в те дни земля возвратит вверенное ей и царство мёртвых
возвратит вверенное ему, что оно получило, и преисподняя отдаст назад то, что
обязана отдать.
И Он изберёт между ними праведных и святых, ибо пришёл день, чтобы
спастись им.
И Избранный в те дни сядет на престоле Своём, и все тайны мудрости
будут истекать из мыслей Его уст, ибо Господь духов даровал Ему это и прославил
Его.
И в те дни горы будут скакать, как овны, и холмы будут прыгать, как
агнцы, насытившиеся молоком; и все они сделаются ангелами на небе.
Их лицо будет сиять от радости, так как в те дни восстанет Избранный;
и земля возрадуется, и на ней будут жить праведные, и избранные будут ходить и
шествовать по ней.
И после тех дней, в том месте, где я видел все видения
относительно того, что сокрыто,~--- я был восхищён в вихре ветра и приведён к
западу,~--- там очи мои видели сокровенные предметы неба, всё, что произойдёт на
земле: одну гору из железа, одну из меди, одну из серебра, одну из золота,
одну из жидкого металла и одну из свинца.
И я спросил ангела, который шёл со мною, говоря: "Что это за предметы,
которые я видел в сокровенном месте?"
И он сказал мне: "Все эти предметы, которые ты видел, служат
владычеству Его Помазанника, дабы Он был сильным и могущественным на земле".
И отвечал мне тот ангел мира, говоря: "Подожди немного, тогда ты
увидишь и тебе будет открыто всё, что сокровенно и что насадил Господь духов.
И те горы, которые ты видел: гора из железа, гора из меди, гора из
серебра, гора из золота, гора из жидкого металла и гора из свинца~--- все они
будут пред Избранным, как сотовый мёд пред огнём и как та вода, которая стекает
сверху на эти горы, и они окажутся слабыми под Его ногами.
И случится в те дни, что нельзя будет спастись ни золотом, ни
серебром: нельзя будет тогда ни спастись, ни убежать.
И не будет дано тогда для битвы ни железа, ни панцирной одежды; руда
не будет пригодна ни на что, и олово не будет пригодным ни на что и не пойдёт в
прок, и свинец не будет добываться.
Все эти вещи исчезнут и уничтожатся с поверхности земли, когда
появится Избранный пред лицом Господа духов.
И там мои очи видели глубокую долину, устье которой было открыто;
и все живущие на тверди, и в море, и на островах принесут Ему дары, и подарки,
и знаки верности, но та глубокая долина не наполнится.
И они совершают преступление своими руками, и всё, что они, грешники,
добывают, то преступным образом пожирают сами, так они, грешники, погибнут пред
лицом Господа, и будут изгнаны с лица Его земли без прекращения на всю
вечность.
Ибо я видел ангелов наказания, как они шли и готовили сатане все
орудия.
И я спросил ангела мира, шедшего со мною: "Те орудия,~--- для кого они
их готовят?"
И он сказал мне: "Они готовят их для царей и для сильных земли сей,
чтобы уничтожить их чрез это.
И после этого Праведный и Избранный откроет дом Своего общественного
собрания, которое отныне не должно быть стесняемо, во имя Господа духов.
И эти горы будут пред Его лицом, как земля и холмы будут, как водный
источник; и праведники будут иметь покой при унижении грешников".
И я взглянул и обратился к другой стороне земли, и увидел там
глубокую долину с пылающим огнём.
И они (ангелы наказания) принесли царей и сильных и положили их в
глубокую долину.
И там мои очи видели, как сделали для них орудия,~--- железные цепи
безмерного веса.
И я спросил ангела мира, говоря: "Эти цепи-орудия,~--- для кого они
приготовлены?"
И он сказал мне: "Они приготовлены для отрядов Азазела, чтобы взять их
и бросить в преисподний ад: и челюсти их будут покрыты грубыми камнями, как
повелел Господь духов.
Михаил и Гавриил, Руфаил и Фануил схватят их в тот великий день суда и
бросят в этот день в печь с пылающим огнём, дабы Господь духов отмстил им за их
неправду,~--- за то, что они покорились сатане и прельстили живущих на земле.
И в те дни наступит осуждение Господа духов, и откроется хранилище
вод, которые сверху на небесах, и, кроме них, те источники, которые под
небесами и внизу в земле.
И все воды на земле соединятся с водами, которые в верху на небесах;
вода же, которая вверху на небе, есть мужская, и вода, которая внизу на земле,
есть женская.
И тогда будут уничтожены все, которые живут на земле и которые живут
между пределами неба.
И чрез это они узнают всю неправду, которую они совершили на земле и
за которую погибают.
И после этого раскаялся Глава дней и сказал: "Напрасно Я погубил
всех живущих на земле".
И Он поклялся Своим великим именем: "Отныне Я не буду более поступать
так с живущими на земле; и Я положу знамение на небе: оно будет залогом между
Мною и ими до вечности, пока существует небо над землёю.
И тогда произойдёт по Моему повелению: когда Я в Моём гневе и в Моём
осуждении решу схватить их рукою ангелов в день скорби и печали, то Мой гнев и
Моё осуждение будут оставаться над ними навсегда,~--- говорит Бог, Господь духов.
Вы, могущественные цари, которые будете жить на земле, вы должны
увидеть Моего Избранного, как Он сидит на престоле Моей славы и судит Азазела,
и всё его сообщество, и все его отряды, во имя Господа духов".
И я видел там воинство идущих ангелов наказания, которые держали
верёвки из железа и руды.
И я спросил ангела мира, шедшего со мною, говоря: "К кому идут те,
которые держат верёвки?"
И он сказал мне: "Каждый к своим избранным и возлюбленным, чтобы
бросить их в глубокую пропасть долины.
И тот час та долина наполниться избранными и возлюбленными, и день их
жизни окончится, и день их обольщения не будет с тех пор более считаться".
И в те дни соберутся ангелы, и их начальники направятся к востоку к
Парфеянам и Мидянам~--- приготовить там возмущение между царями, чтобы нашёл на
них дух возмущения; и они поднимутся со своих престолов, чтобы выступить в
середину их стада, как львы из своих логовищ и как голодные волки.
И они поднимутся и обступят землю их избранных, и земля Его избранных
будет пред ними гумном и тропою.
Но город Моих праведных будет преградой для их коней; и они начнут
борьбу друг с другом, и их правая рука будет сильна против них самих, и никто
не будет знать своего ближнего и брата, ни сын своего отца и своей матери, пока
не будет достаточно трупов вследствие из смерти, и осуждение над ними не будет
тщетным.
И в те дни царство мёртвых откроет свою пасть, и они будут опущены в
него; и вот их погибель: царство мёртвых поглотит грешников пред лицом
избранных.
И случилось после этого: там опять я увидел отряд колесниц, на
которых ехали люди, и они шли на крыльях ветра от восхода и захода к полудню.
И был слышен шум их колесниц; и как только это смятение произошло,
святые ангелы заметили это с неба; и столпы земли подвинулись со своих мест, и
это было слышно от пределов земли до пределов неба, в один день.
И они все упадут и поклонятся Господу духов.
И это конец второй притчи.
\vs 1En 9:1
И я начал говорить третью притчу о праведных и избранных.
Будьте блаженными вы, праведные и избранные, ибо жребий ваш будет
славен!
И праведные будут жить в свете солнца и избранные в свете вечной жизни;
дни вечной жизни их не кончаются, и дни святых бесчисленны.
И они будут искать света и обретут правду у Господа духов: мир будут
иметь праведные у Господа мира.
И после этого будет сказано святым, чтобы они искали на небе тайны
справедливости и наследие веры, ибо оно стало ясно, как сияние солнца на земле,
и мрак исчез.
И не прекращаемый свет будет существовать, и дни, в которые они будут
жить, бесчисленны, ибо мрак заранее будет уничтожен, и силен будет свет пред
Господом духов, и свет праведности будет силен во век пред Господом духов.
И в те дни очи мои видели тайны молний и массы света, и их правду;
и они блестят для благословения и для проклятия, как желает этого Господь
духов.
И там я видел тайны грома, и слышал, как раздаётся глас его, когда он
гремит вверху на небе, и они (ангелы проводники) показали мне места жилищ на
земле и глас грома, как он служит для благополучия и благословения или для
проклятия, по слову Господа духов.
И после этого мне были показаны все тайны масс света и молний, как они
блестят для благословения и для насыщения.
\vs 1En 10:1
В пятисотый год, в седьмой месяц, в четырнадцатый день месяца
жизни Еноха.
В той притче я видел, как небо небес поколебалось от сильного трепета,
и воинство Всевышнего, и тысячи тысяч и тьмы тем ангелов были потрясены
вследствие сильного волнения.
И тот час я увидел Главу дней, сидящего на престоле Своей славы, и
ангелов и праведных, стоящих вокруг Него.
И меня объял сильный трепет, и страх охватил меня; моё бедро согнулось
и ослабело, всё моё существо сплавилось, и я упал на своё лицо.
Тогда святой Михаил послал другого святого ангела~--- одного из святых
ангелов~--- и он поднял меня; и как только он меня поднял, мой дух обратился
назад, ибо я не мог вынести вида этого воинства, и колебания и трепета неба.
И сказал мне святой Михаил: "что за вид так взволновал тебя?
До сего дня был день Его милосердия, ибо Он был милосерден и
долготерпелив к населяющим почву земную.
Но вот придет день, и власть, и наказание, и суд, что приготовил
Господь духов для тех, которые преклоняются пред праведным судом, и для тех,
которые отвергают праведный суд, и для тех, которые напрасно употребляют Его
имя; и тот день будет для избранных защитою, а для грешников расследованием.
И в тот день будут распределены два чудовища: женское чудовище,
называемое Левияфа, чтобы оно жило в бездне моря над источниками вод, мужеское
же называется Бегемотом, который своею грудью занимает необитаемую пустыню,
называемую Дендаин, находящуюся на востоке сада, где живут избранные и
праведные и куда взят мой дед, седьмой от Адама~--- первого человека, которого
сотворил Господь духов.
И я молил того другого ангела, чтобы он показал мне власть тех
чудовищ, как они разделены в один день, и одно было поставлено в глубину моря,
а другое на твердую почву пустыни.
И он сказал мне: "ты, сын человеческий,~--- ты добиваешься здесь узнать,
что сокрыто".
И сказал мне другой ангел, который шел со мною и показал мне, что
находится в сокровенных местах, первое и последнее, что на небе в высоте и на
земле в глубине, и что на пределах неба, и в хранилищах при основании неба, и в
хранилищах ветров; и он показал, как распределены духи, и как возвышаются
(явления в природе), и как исчислены источники и ветры по силе духа, и какова
сила лунного света, и как все это есть сила правды, и (показал) отделения звезд
по их именам, и как все отделения разделены; и он показал громы по местам их
падения, и все отделения; которые сделаны между молниям, чтобы они сверкали и
их отряды тотчас бы повиновались (следовали за ними); ибо гром имеет места
отдыха и ему определено выжидать свой удар; и они оба~--- гром и молния~---
неотделимы; и хотя они не одно, однако оба чрез посредство духа идут вместе и
не разделяются.
Ибо, когда сверкает молния, то и гром дает свой глас, и дух
задерживает во время удара и одинаково делает разделение между ними; ибо запас
их ударов, как песок, и каждый в отдельности из них удерживается при своем
ударе уздою, и силою духа они возвращаются назад, и таким образом посылаются
далее соразмерно с множеством стран земли.
И дух моря есть мужественный и сильный; и соразмерно с крепостью своей
силы он притягивает его (море) назад уздою; и таким образом оно перегоняется
вперед и разливается во все горы земли.
И дух инея есть его (собственный, особенный) ангел, и дух града есть
добрый ангел.
И духа снега Он назначил ради его силы, и он (снег) имеет особенного
духа; и то, что поднимается из него, есть как бы дым и его имя мороз.
Но дух облака не соединён с ними (духами инея, града и снега) в их
хранилищах, а имеет особое хранилище; ибо его движение бывает при ясности и
свете и при мраке, и зимой и летом, и его хранилище есть свет; и он (дух
облака) есть его ангел.
И дух росы имеет свое жилище на пределах неба, и оно связано с
хранилищем дождя, и его движение бывает зимою и летом; и его тучи и тучи
дождевого облака находятся в связи и сообщаются друг с другом.
И когда дух дождя выходит из своего хранилища, приходят ангелы, и
открывают хранилище, и выпускают его, и тогда он рассевается по всей суше и
таким образом соединяется с водою на земле.
Ибо воды существуют для живущих на земле, так как они составляют пищу
для земли от Всевышнего, Который существует на небе; посему дождь имеет меру,
и ангелы владеют им.
Я видел все эти вещи вплоть до сада праведных.
И ангел мира, который был со мною, сказал мне: "эти два чудовища
приготовлены сообразно с величием Божиим для того, чтобы быть накормленными,
дабы осуждение Божие не было тщетным; и будут умерщвлены сыны со своими
матерями и дети со своими отцами.
Когда осуждение Господа духов будет пребывать над ними, то будет
пребывать для того, чтобы осуждение Господа духов не сделалось тщетным по
отношению к ним; после этого будет суд по Его милосердию и терпению.
И я видел в те самые дни, как даны были тем ангелам длинные
веревки, и они подняли крылья и полетели, и достигли севера.
И я спросил ангела, говоря: "для чего они держали те длинные веревки и
удалились?"
И он сказал мне: "они ушли, чтобы измерять".
И ангел, шедший со мною, сказал мне: "они несут меры праведных и
канаты праведных, чтобы они опирались на имя Господа духов навсегда и навеки.
И начнут и будут жить избранные с избранными, и эти меры будут даны
вере и будут укреплять слова правды.
И эти меры откроют всё сокровенное в глубине земли, и погибших по
пустыням, и пожранных рыбами морскими и зверями, чтобы они возвратились и
оперлись на день Избранного; ибо никто не погибнет пред Господом духов, и никто
не может погибнуть.
И сохранили повеления все те, которые вверху на небе, и одна сила,
один голос и один свет, подобный огню, был дан им.
И Того, прежде всего, прославили, и возвеличили, и восхвалили они с
мудростью, и показали себя мудрыми в слове и духе жизни.
И Господь духов посадил Избранного на престол Своей славы, и Он будет
судить все деяния святых ангелов на небе и взвесит их поступки на весах.
И когда Он поднимает Свое лицо, чтобы судить их сокрытые пути по слову
имени Господа духов и их стезю по пути праведного суда всевышнего Бога, тогда
все они возглаголят одним гласом, и прославят, и восхвалят, и вознесут, и будут
хвалить имя Господа духов.
И будет взывать все воинство небесное и все святые, которые вверху, и
воинство Божие,~--- херувимы и серафимы, и офанимы, и все ангелы власти, и все
ангелы господства, и Избранный, и другие силы, которые на тверди и над водою,~---
все они будут взывать в тот день и будут возносить одним гласом, и прославлять,
и восхвалят, и хвалить, и превозносить в духе веры, и в духе мудрости и
терпения, и в духе милосердия, и в духе правды и мира, и в духе благости; и
будут все говорить одним гласом: "славь Его, и да будет прославлено имя Господа
духов во веки и до века!"
Его будут хвалить все, которые не спят вверху на небе; Его будут
прославлять все Его святые, которые на небе, и все избранные, живущие в саду
жизни, и каждый дух света, способный прославлять и восхвалять, и превозносить,
и святить Твое имя, и всякая плоть, которая будет чрезмерно прославлять и
восхвалять Твое имя вовеки.
Ибо велико милосердие Господа духов, и Он долготерпелив, и все Свои
творения и всю Свою силу,~--- так много Он сотворил,~--- Он открыл праведным и
избранным, во имя Господа духов.
И Господь духов так повелел царям, и сильным, и вознесенным, и
населяющим землю, и сказал: "откройте свои глаза и вознесите ваши роги, ибо вы
можете узнать Избранного!"
И Господь духов сел на престол славы, и дух правды изливался на Него,
и слово уст Его умертвило всех грешников и всех неправедных, и они погибли
перед лицом Его.
И будут стоять в тот день все цари, и сильные, и вознесенные, и
владеющие твердью, и увидят Его и узнают, как Он сидит на престоле Своей славы,
и пред Ним судятся праведные в правде и никакая пустая речь не говорится пред
Ним.
Тогда постигнет их боль, как жену, которая в родильных потугах и
которой трудно бывает родить, когда ее сын входит в проход утробы, и которая
имеет боли при родах.
И одна часть из них будет смотреть на другую, и они устрашатся и
потупят свой взор, и боль обоймёт их, когда они увидят того Сына жены, сидящим
на престоле Своей славы.
И цари, и сильные, и все владеющие землею будут восхвалять, и
прославлять, и превозносить Владычествующего над всем, Который был сокрыт.
Ибо прежде Сын человеческий был сокрыт, и Всевышний сохранял Его пред
Своим могуществом, и открыл Его избранным; и будет посеяно общество святых и
избранных, и будут стоять пред Ним в тот день все избранные.
И все могущественные цари, и вознесенные, и господствующие над
твердью, упадут пред Ним на свое лицо, и поклонятся, и возложат на того Сына
человеческого свою надежду, и будут умолять Его и просить у Него милосердия.
И Господь духов будет теперь теснить их, чтобы они немедленно
удалились прочь от Его лица; и их лица исполнятся стыдом, и мрак соберется на
них.
И ангелы наказания возьмут их, чтобы совершить над ними возмездие за
то, что они притеснили Его детей и избранных.
И они сделаются зрелищем для праведных и избранных Его: они
(праведные) будут радоваться, взирая на них, ибо гнев Господа духов будет
пребывать на них, и меч Господа духов упьется ими.
И праведные и избранные будут спасены в тот день, и не будут более
видеть отныне лица грешников и неправедных.
И Господь духов будет обитать над ними, и они будут жить вместе с тем
Сыном человеческим, и есть, и ложиться, и вставать, от века до века.
И праведные и избранные будут вознесены от земли, и перестанут
опускать свой взор, и будут облечены в одежду жизни.
И это будет одежда жизни у Господа духов.
В те дни могущественные цари, владеющие твердью, будут вымаливать
у Его ангелов наказания, которым они преданы,~--- даровать им немного успокоения,
и просить, чтобы им можно было пасть ниц перед Господом духов и поклониться, и
сознаться перед Ним в своих грехах.
И они будут прославлять и восхвалять Господа духов, и говорить: "да
будет прославлен Он, Господь духов и Господь царей, Господь сильных и Господь
властителей, Господь славы и Господь мудрости, пред которым всякая тайна ясна.
И Твое могущество от рода до рода, и Твоя слава от века до века;
глубоки все Твои тайны и бесчисленны, и слава Твоя неисчислима.
Теперь узнали мы, что нам нужно восхвалять и прославлять Господа царей
и Того, Кто царь над всеми царями".
И они скажут: "о, если бы нам дали успокоение, чтобы мы восхвалили
Его, и возблагодарили Его, и прославили Его, и уверовали пред Его славой!
И теперь мы домогаемся небольшого успокоения, но не находим его: мы
прогнаны, и не получили его, свет исчез пред нами, и мрак служит нашим жилищем
навсегда и навеки.
Ибо мы не уверовали в Него, и не восхвалили имя Господа царей за
всякое Его дело, и наша надежда была бы на скипетр нашего владычества и на наше
величие.
И в тот день нашего страдания и нашей печали Он не спасет нас, и мы не
найдем успокоения, дабы уверовать, что Господь наш истинен во всяком Своем
деле, и во всех Своих судах, и в Своей правде, и суды Его не лицеприятны.
И мы погибнем пред Его лицом за свои дела, и все грехи наши исчислены
по справедливости".
Теперь они скажут себе: "душа наша насытилась неправедным стяжанием,
но оно не отвратит того, что мы будем низвергнуты в пламя адского мучения".
И после этого их лицо исполнится мраком перед тем Сыном человеческим и
они будут отвергнуты от Его лица, и меч будет жить между ними перед Его лицом.
И Господь духов так сказал: "вот повеление и суд над сильными, и
царями, и вознесёнными, и владеющими твердью, пред Господом духов".
Также и другие виды я видел в том сокровенном месте.
Я слышал глас ангела, как он сказал: "это ангелы, которые сошли с неба
на землю и открыли сынам человеческим то, что было сокрыто, и соблазнили сынов
человеческих совершать грехи".
\vs 1En 11:1
И в те дни Ной увидел землю, как она согнулась, и ее гибель была
близка.
И он направил оттуда свои стоны и пришел к пределам земли, и вскликнул
к своему деду Еноху; и Ной трижды сказал опечаленным голосом: "послушай меня,
послушай меня, послушай меня!"
И он (Ной) сказал ему: "скажи мне, что это такое происходит на земле
что земля так ослабела и поколебалась?
как бы я не погиб вместе с нею!"
И после этого мгновения было великое колебание на земле, и голос был
слышен с неба, и я упал на свое лицо.
И пришел мой дед Енох, и встал около меня, и сказал мне: "Почему ты
восклицал ко мне опечаленным криком и плачем?
От лица Господа вышло повеление относительно живущих на тверди, что
должен наступить их конец, так как они знают все тайны ангелов, и всю власть
дьяволов, и всю их сокровенную силу, и всю силу тех, которые совершают
волшебства, и силу заклинаний, и силу тех, которые льют для всей земли
изображения идолов; и хорошо также знают, как серебро производится из праха
земли, и как жидкий металл образуется на земле.
Ибо свинец и олово не так производятся из земли, как первое (серебро):
существует особый источник, производящий их, и ангел, стоящий в нем, и он
преимущественно тот ангел".
И после этого дед Енох обнял меня своей рукою, поднял меня и сказал
мне: "иди, ибо я спрашивал Господа духов об этом колебании на земле.
И он сказал мне: за их нечестие над ними совершен суд, и он уже не
вычисляется предо Мною ради месяцев, которые они расследовали и через это
узнали, что земля и живущие на ней погибнут.
И для них (ангелов) не будет убежища вовеки, так как они показали им
(людям) то, что сокрыто, и они осуждены; но не так ты, мой сын: Господь духов
знает, что ты чист и свободен от этой укоризны за тайны.
И Он утвердил твое имя между святыми, и сохранит тебя между живущими
на тверди; и Он определил в правде твое семя для царей и для великой славы, и
из твоего семени произойдет источник праведных и святых без числа во веки".
И после этого он показал мне ангелов наказания, готовых идти и
выпустить все силы воды, которая внизу на земле, чтобы принести суд и погибель
всем, покоящимся и живущим на тверди.
И Господь духов дал повеление ангелам, вышедшим теперь, чтобы они не
простирали рук, а дожидались: ибо те ангелы были поставлены над силами вод.
И я удалился от лица Еноха.
И в те дни было слово Господа ко мне, и Он сказал мне: "Ной!
вот твой жребий предстал предо Мною, жребий без порока, жребий любви
и милосердия.
И теперь ангелы делают деревянное здание; и так как они вышли на это
дело, то и Я приложу к нему Свою руку и буду охранять его (ковчег); и выйдет из
него семя жизни, и земля должна подвергнуться превращению, чтобы ей не остаться
пустою.
И Я укреплю твое семя предо Мною на всю вечность, и живущие с тобою
распространятся по поверхности земли, и оно (семя) будет благословенно и
умножится на земле во имя Господа".
И они заключат тех ангелов, показавших неправду, в ту пылающую долину
на западе, которую показал мне прежде дед Енох, возле гор золота, и серебра, и
железа, и жидкого металла, и свинца.
И я видел ту долину, в которой было великое колебание и волнение вод.
И когда всё это случилось, то из той огненной металлической лавы и от
колебания, которое их (воды) колебало, в том месте (в долине) явился серный
запах, и он соединился с теми водами; и та долина ангелов, которые прельстили
людей, разгоралась все далее под тою землею.
И через долины этой самой земли проходят реки огня,~--- именно там, где
осуждены пребывать те ангелы, которые соблазнили живущих на тверди.
Но те воды будут служить в те дни для царей, и сильных, и вознесённых,
и для живущих на тверди к исцелению души и тела и к наказанию духа, так как
дух их исполнен сладострастием, чтобы они были наказаны со своим телом, ибо
они отвергли Господа духов; и они изо дня в день видят свое будущее наказание и
однако не веруют в Его имя.
И в той самой мере, насколько становятся сильными жар их тела, будет
происходить изменение и в их духе (от века до века), ибо не может быть сказано
пред Господом духов пустое слово.
Ибо придет суд на них, так как они веруют в сладострастии своего тела
и отвергают дух Господа.
И те воды сами в те дни претерпят изменение: ибо, когда те ангелы
будут наказаны в те дни, будет изменяться жар тех водных источников, и когда
ангелы будут подниматься, та вода источников будет изменяться и охлаждаться.
И я слышал святого Михаила, когда он отвечал и говорил: "этот суд,
которым осуждены ангелы, есть свидетельство для царей, и сильных, и владеющих
твердью.
Ибо эти воды суда служат к исцелению ангелов и для смерти их тела; но
они (владыки) не увидят того и не уверуют, что те воды изменятся и превратятся
в огонь, который горит вовек".
И после этого мой дед Енох дал мне в книге знамения всех тайн и
притч, которые ему были даны, и собрал их для меня в словах книги притчей.
И в тот день отвечал святой Михаил Руфаилу, говоря: "сила духа увлекает
меня и возбуждает меня, и строгость суда тайн, суда над ангелами, поражает
меня; кто может вынести строгость суда, который совершен и до сих пор пребывает
и от которого они распаляются?"
И опять отвечал и сказал святой Михаил Руфаилу: "есть ли кто такой,
который не размягчился бы сердцем и почки которого не содрогнулись бы от этого
слова?
суд вышел относительно них, относительно тех, которых выгнали они
таким образом".
И случилось, когда святой Михаил стоял пред Господом духов, то он
сказал Руфаилу так: "и я не буду представительствовать за них пред очами
Господа, ибо Господь духов разгневался на них, потому что они действуют так,
как если бы были равны Богу.
Посему на них грядет суд, который сокрыт, от века до века; ибо ни
ангел, ни человек не получат своей доли, но только они получат свой суд, от
века до века".
И после этого суда они навлекут на них гнев и ярость, так как они
показали это живущим на тверди.
И вот имена тех ангелов, и эти имена их: первый из них Семъяйза,
второй Арестикифа, третий Армен, четвертый Кокабаел, пятый Тураел, шестой
Румъйял, седьмой Данел, восьмой Нукаел, девятый Баракел, десятый Азазел:
одиннадцатый Армерс, двенадцатый Батаръйял, тринадцатый Базазаел, четырнадцатый
Ананел, пятнадцатый Турхйял, шестнадцатый Симанизиел, семнадцатый Иетарел,
восемнадцатый Тумаел, девятнадцатый Тарел, двадцатый Румаел, двадцать первый
Изезеел.
И это главы их ангелов и имена их предводителей над сотнею,
пятьюдесятью и десятью.
Имя первому Иекун; это тот, который соблазнил всех детей святых
ангелов, и свел их на землю, и соблазнил их чрез дочерей человеческих.
И имя другому Асбеел: этот внушил детям святых ангелов злой совет, и
соблазнил их, чтобы они осквернили свои тела с дочерьми человеческими.
И имя третьему Гадрел: это тот, который показал сынам человеческим все
смертоносные удары, и он соблазнил Еву, и показал сынам человеческим орудия
смерти, и панцырь, и щит, и меч для битвы, и показал сынам человеческим все
орудия смерти.
И из его руки они перешли к живущим на тверди, от того часа до века.
И имя четвертому Пенемуэ: этот показал сынам человеческим горькое и
сладкое, и показал им все тайны их мудрости.
Он научил людей письму чернилами и употреблению бумаги, и чрез это
многие согрешили от века до века и до сего дня.
Ибо люди сотворены не для того, чтобы они, таким образом, тростью и
чернилами закрепляли свою верность (свое слово).
Ибо люди сотворены не иначе, чем ангелы, чтобы им пребывать праведными
и чистыми, и смерть, которая губит всех, не касалась бы их, но они погибают
чрез это свое знание, и чрез эту силу она пожирает меня.
И имя пятому Касдейя: этот показал людям все злые удары духов и
демонов, и улары рождения в утробе матери, дабы устранить его, и удары души,
укушения змей, и удары, случающиеся в полдень,~--- сына змеи, именуемого Табает.
И это число Кесбеела, который показал святым главу клятвы, когда он
жил высоко вверху во славе, и имя ее (клятвы) Бека.
И этот ангел сказал святому Михаилу, чтобы он показал им сокровенное
имя Божие, дабы они видели то сокровенное имя и упоминали его при клятве, чтобы
содрогались пред тем именем и клятвой те, которые показали сынам человеческим
все, что было сокрыто.
И такова сила той клятвы, ибо она сильна и могущественна, и Он положил
эту клятву Акаэ в руку святого Михаила.
И таковы тайны клятвы, и они (тайны мира) утверждены чрез его клятву,
и силою его небо повешено, прежде чем был создан мир, и до века.
И чрез нее была основана земля на воде, и силою ее
выходит из сокровищ гор прекрасная вода для живущих от сотворения мира
до века.
И чрез ту клятву было сотворено море, и, как его основание, Он положил
ему на время ярости песок, и оно не должно преступать его от сотворения мира до
века.
И чрез ту клятву основания земли утверждены, и стоят и не движутся со
своего места от века до века.
И чрез ту клятву совершают свое движение солнце и луна, и не отступают
от предписанного им от века до века.
И чрез клятву звезды совершают свое движение, и Он зовет их по именам
и они отвечают Ему от века до века; и точно также духи воды, ветров и всего
воздуха, и их пути по всем соединениям духов.
И в ней (силою клятв) сберегаются хранилища гласов грома и света
молний; и в ней сберегаются хранилища града и инея, и хранилища дождя и росы.
И они все веруют и воссылают благодарение Господу духов, и восхваляют
всею своею силою, и их пища состоит в громких благодарениях; они благодарят, и
прославляют, и превозносят имя Господа духов от века до века.
И могущественна над ними эта клятва, и они сохраняются чрез неё, и их
пути сохраняются, и их движения не нарушаются.
И было для них (для праведников) великою радостью, и прославляли, и
восхваляли за то, что им было открыто имя того Сына человеческого.
И Он сел на престол Своей славы и весь суд был предан Ему, Сыну
человеческому, и Он допустил прийти и погибнуть с лица земли грешникам и тем,
которые соблазнили мир.
Они связаны ценою и заключены в своих сборных местах разврата, и все
дела их исчезают с земли.
И отныне не будет более там ничего тленного, ибо Он, Сын мужа, явился
и сел на престоле Своей славы: и всякое зло исчезнет и прейдет пред Его лицом;
слово же того Сына мужа будет иметь силу пред Господом духов.
Это третья притча Еноха.
\vs 1En 12:1
И случилось после этого: вот его Еноха имя было вознесено при
жизни к тому Сыну человеческому, к Господу духов, от живущих на тверди.
И оно было вознесено на колесницах духа, и имя его вышло среди людей.
И с того дня я не входил в их среду; и Он посадил меня между двумя
ветрами, между севером и западом,~--- там, где ангелы взяли веревки, чтобы
измерить около меня место для избранных и праведных.
И там я видел первых отцов праведных, от древнейшего времени живущих в
том месте.
И после того случилось, что мой дух был сокрыт (восхищен) и
вознесен на небеса; там я видел сынов ангелов, как они ходят по огненному
пламени; и их одежды и их одеяния белы, и свет лица их как кристалл.
И я видел две реки из огня, и свет того огня блистал, как гиацинт: и я
пал на свое лицо пред Господом духов.
И ангел Михаил, один из архангелов, взял меня за правую руку и поднял
меня, и привел меня ко всем тайнам милосердия и правды.
И он показал мне все тайны пределов неба и все хранилища всех звезд и
светил, откуда они выходят пред святых.
И дух восхитил Еноха на небо небес, и я видел там, в средине того
света, нечто такое, что было устроено из кристалловых камней, и между теми
камнями было пламя живого огня.
И мой дух видел, как вокруг того дома обходил огонь, на четырех же
сторонах его реки, наполненные живым огнем, и видел, как они окружают тот дом.
И вокруг были серафимы, херувимы и офанимы: это те которые не спят и
охраняют престол его славы.
И я видел ангелов, которые не могут быть исчислены, тысячу тысяч и
тьму тем, окружающих тот дом: и Михаил и Руфаил, Гавриил и Фануил, и святые
ангелы, которые вверху на небесах, выходят и входят в тот дом.
И вышли из того дома Михаил и Гавриил, Руфаил и Фануил, и многие
святые ангелы без числа, и с ними Глава дней; Его глава была чиста как волна
(руно) И Его одежда неописуема.
И я упал на свое лицо, и все мое тело сплавилось, и мой дух изменился:
и я воскликнул громким голосом, духом силы, и прославил и восхвалил и
превознес.
И эти прославления, которые вышли из моих уст, были приятны для того
Главы дней.
И сам Глава дней шел с Михаилом и Гавриилом, Руфаилом и Фануилом, и с
тысячами и со тьмами тысяч, с ангелами без числа.
И тот ангел пришел ко мне, и приветствовал меня своим гласом, и
сказал: "ты~--- сын человеческий, рожденный для правды", и правда обитает над
тобою, и правда Главы дней не оставляет тебя".
И он сказал мне: "Он призывает тебе мира, во имя будущего мира, ибо
оттуда исходит мир со времени сотворения вселенной, и таким образом ты будешь
иметь его во веки и от века до века.
И все, которые в будущем пойдут по твоему пути,~--- ты, которого правда
не оставляет вовек, жилища тех будут возле тебя и наследие их около тебя, и они
не будут отделены от тебя во век и от века до века.
И таким образом возле того Сына человеческого будет долгая жизнь, и
мир наступит для праведных, во имя Господа духов от века до века.
\vs 1En 13:1
Книга об обращении светил небесных, как это обращение происходит
с каждым из них, по их классам, по их господству и их времени, по их именам и
местам происхождения, и по их месяцам, которые показал мне их путеводитель,
святой ангел Уриил, бывший при мне; и он показал мне все их описание, что с
ними происходит со всеми годами мира и до века, пока не создано новое творение,
которое продолжится во век.
И вот первый закон светил: светило солнце имеет свой восход в восточных
вратах неба и свой заход в западных вратах неба.
И я видел шесть врат, в которых солнце заходит; луна также восходит и
заходит чрез те же врата, путеводители звёзд вместе со своими путеводными
восходят и заходят там же: шесть врат на востоке и шесть на западе, следующих
друг за другом в строго соответствующем порядке, а также много окон направо и
налево от тех врат.
Прежде всего, выходит великое светило, называемое солнцем, его
окружность как окружность неба, и оно совершенно наполнено блистающим и
согревающим огнем.
Колесницы, в которых оно поднимается, гонит ветер, и солнце, заходя,
исчезает с неба и возвращается назад через север, чтобы достигнуть востока; и
оно направляется таким образом, что приходит к соответствующим вратам и светит
на небе.
Таким образом, оно восходит в первый месяц в великих вратах и
именно оно восходит через четвёртые из тех шести восточных врат.
И при тех четвёртых вратах, через которые солнце восходит в первый
месяц, находятся двенадцать оконных отверстий, из которых выходит пламя, когда
они в свое время открываются.
Когда солнце поднимается на небо, то оно выходит чрез те четвёртые
врата в продолжение тридцати утров, и заходит прямо, напротив, в четвёртых
вратах на западе неба.
И в этот период день становится день за днем длиннее, и ночь становится
ночь за ночью короче до тридцатого утра.
И в тот день, день бывает длиннее на две части, чем ночь, и день
включает ровно десять частей и ночь восемь частей.
И солнце восходит из тех четвёртых врат и заходит в четвёртых, и
возвращается к пятым вратам востока в продолжение тридцати утров, и восходит из
них, и заходит в пятых вратах.
Тогда день становится длиннее на две части и заключает одиннадцать
частей, и ночь становится короче и заключает семь частей.
И солнце возвращается к востоку, и вступает в шестые врата, и восходит
и заходит в шестых вратах в продолжение тридцати одного утра ради их знака.
И в тот день, день становится длиннее ночи настолько, что заключает
двойное число частей ночи,~--- именно двенадцать частей, и ночь делается короче и
заключает шесть частей.
И поднимается солнце, чтобы день стал короче и ночь длиннее, и солнце
возвращается к востоку и вступает в шестые врата, и восходит из них и заходит в
продолжение тридцати утров.
И когда пройдет тридцать утров, день уменьшается ровно на одну часть,
и заключает одиннадцать частей и ночь семь частей.
И солнце выступает на западе их тех шести врат и идёт к востоку, и
восходит в пятых вратах в продолжение тридцати утров, и опять заходит на западе
в пятых западных вратах.
В тот день, день уменьшится на две части и заключает десять частей и
ночь восемь частей.
И солнце выходит из тех пятых врат запада, и поднимается в четвёртых
вратах ради их знака тридцать одно утро, и заходит на западе.
В тот день сравнивается день с ночью, и они становятся одинаково
длинными, и ночь заключает девять частей и день девять частей.
И солнце восходит из тех врат и заходит на западе, возвращается к
востоку и восходит в третьих вратах тридцать утров, и заходит на западе в
третьих вратах.
И в тот день ночь становится длиннее дня до тридцатого утра, и день
становится ежедневно короче до тридцатого дня, и ночь заключает ровно десять
частей и день восемь частей.
И солнце восходит из тех третьих врат, и заходит в третьих вратах на
западе, возвращается к востоку и восходит во вторых вратах востока в
продолжение тридцати утров, и точно также заходит во вторых вратах на западе
неба.
И в тот день ночь заключает одиннадцать частей и день из тех вторых
врат и заходит на западе во вторых вратах, и возвращается к востоку в первые
врата в продолжение тридцати одного утра и заходит на западе в первых вратах.
И в тот день ночь становится настолько длинною, что включает двойное
число частей дня; ночь заключает ровно двенадцать частей и день шесть частей.
Этим солнце закончило свои путевые становища, и оно опять поворачивает
на эти же становища, и вступает в те первые врата в продолжение тридцати утров,
и заходит также на западе напротив них.
И в тот день ночь уменьшается в продолжительности на одну часть, и она
заключает одиннадцать частей в день семь частей.
И солнце возвращается и вступает во вторые врата востока, и
возвращается на те свои путевые становища в продолжение тридцати утров, восходя
и заходя.
И в тот день ночь уменьшается в продолжительности, и ночь заключает
десять частей и день восемь частей.
И в тот день солнце восходит из тех вторых врат и заходит на западе,
потом возвращается к востоку поднимается в третьих вратах в продолжение
тридцати одного утра, и заходит на западе неба.
В тот день ночь уменьшается и заключает десять частей и день девять
частей, и ночь сравнивается с днем, и год заключает ровно триста шестьдесят
четыре дня, и продолжительность дня и ночи, и краткость дня и ночи в следствие
движения солнца становятся различными.
По причине этого дневное движение ежедневно становится длиннее, и его
ночное движение становится каждоночно короче.
И таков закон и движение солнца и его возвращение, насколько оно часто
возвращается: шестьдесят раз возвращается и восходит оно, именно то великое
вечное светило, которое навеки именуется солнцем.
И то, что таким образом восходит, есть великое светило, как оно
называется по своему появлению в силу повеления Господа.
И таким образом оно восходит и заходит, и не уменьшается и не
покоится, но движется день и ночь в колеснице, и его свет в семь раз светлее
лунного, но по величине они оба одинаковы.
\vs 1En 14:1
И после этого закона я видел другой закон, касающийся малого
светила, который называется луною.
Ее окружность подобна окружности неба, и ее колесница, в которой она
идет, гонится ветром; и ей дается свет по определённой мере.
В каждый месяц изменяется ее восход и заход; её дни как дни солнца; и
если ее свет равномерен (полон), то она содержит седьмую часть солнечного
света.
И она восходит таким образом: и ее начало на востоке выступает в
тридцатое утро; в тот день она становится видимою, и тогда бывает для всех
начало луны, в тридцатое утро, одинаково с солнцем в тех же вратах, где
восходит солнце.
И одна половина ее выступает на одну седьмую часть, и весь ее круг
бывает пуст, без света, кроме одной седьмой части из ее четырнадцати частей
света.
И когда она получает одну седьмую часть с половиной от своего света, то
ее свет заключает одну седьмую и седьмую часть с половиной.
Она заходит в новолуние вместе с солнцем, и когда солнце восходит,
восходит и луна вместе с ним, и получает половину одной седьмой части света, и
в ту ночь, в начале ее утра, луна заходит в первый день месяца вместе с
солнцем, и бывает невидима в ту ночь семью и семью частями с половиной.
И она в тот день становится видимою ровно одной седьмой частью, и
восходит и отклоняется от солнца, и дает света в остальные дни семь и семь (14)
частей.
И я видел другой закон и движение её, как она по тому закону
совершает своё месячное обращение.
И всё показал мне святой ангел Уриил, который служит вождём всех их
(светил); и я описал все её (луны) положения, и показал их мне, и описал ее
месяцы, как они бывают, и появление её света до истечения пятнадцати дней.
В каждых семи частях весь ее свет делается полным на востоке, и в
каждых седьми частях весь ее мрак делается полным на западе.
И в определенные месяцы она изменяет свой заход, и в определенные
месяцы она идет своим особенным (от солнца) движением.
И в двоих вратах луна заходит вместе с солнцем,~--- в тех двоих вратах,
в третьих и четвертых вратах.
Именно,~--- она выходит в продолжение семи дней и поворачивает, и
возвращается опять через врата, где восходит солнце; и в них ее свет делается
полным; и она отклоняется от солнца, и вступает в течение 8 дней в шестые
врата, из которых выходит солнце.
И когда солнце выходит из четвертых врат, она выходит семь дней, так
что она выходит из пятых, и возвращается опять в течение семи дней в четвёртые
врата, и весь ее свет делается полным, и она отклоняется и вступает в первые
врата в течение восьми дней.
И опять она возвращается в течение семи дней в четвертые врата, из
которых выходит солнце.
Так видел я их положения, как солнце восходит и заходит по порядку
своих месяцев.
И между теми днями, если взять вместе пять лет, солнце имеет излишку
тридцать дней; которые приходятся на один из тех пяти лет, если они полны,
составляют триста шестьдесят четыре дня.
И излишек солнца и звезд простирается до шести дней; а в пять лет, в
каждый по шести, до тридцати дней, и луна отстает от солнца и звезд на тридцать
дней.
И луна точно ведет все года, так что их положение вовек ни поспешает,
ни запаздывает ни на один день, но действительно правильно совершает годовую
смену в триста шестьдесят четыре дня.
Три года имеют тысячу девяносто два дня и пять лет тысячу восемьсот
двадцать дней, так что на восемь лет приходится две тысячи девятьсот двенадцать
дней.
На луну же приходится в три года тысяча шестьдесят два дня, и в пять
лет она отстаёт на пятьдесят дней; именно с суммою этого нужно прибавить к
шестидесяти двум дням.
И на пять лет приходится тысяча семьсот семьдесят пять дней, так что
лунные дни в восемь лет составляют две тысячи восемьсот тридцать два дня.
Именно ее отставание образует в восемь лет восемьдесят дней, и всех
дней, на которые она отстает в восемь лет, восемьдесят.
И правильный год достигает конца сообразно с положением их (фаз луны?)
и с положением солнца, так как оно восходит из врат, из которых оно восходит и
заходит тридцать дней.
И путеводители глав тысячей, которые поставлены над всем
творением и над всеми звёздами, существует с четырьмя добавочными днями,
которые не могут быть отделены от своего места сообразно со всем исчислением
года; и эти путеводители служат для четырёх дней, которые не считаются при
исчисление года.
И из-за них люди ошибаются в том (в исчислении), ибо те светила
действительно служат для положения мира, одно в первых, одно в третьих, одно в
четвертых и одно в шестых вратах; и точность движения мира оканчивается всегда
чрез триста шестьдесят четыре положения его.
Ибо знаки, времена, и годы и дни показал мне ангел Уриил, которого
вечный Господь славы поставил над всеми небесными светилами на небе и в мире,
чтобы они управляли на поверхности неба, и явились над землею, и были
путеводителями для дня и ночи, именно солнце, луна и звезды, и все служебные
творения, которые совершают свое обращение во всех колесницах неба.
Точно так же Уриил дал мне увидеть двенадцать дверных отверстие в
кругу солнечных колесниц на небе, из которых пробиваются лучи солнца; и от них
исходит теплота на землю, когда они открываются в определённые времена.
Такие же отверстия есть также для ветров для духа росы, когда они по
временам открываются, стоя открытыми в небесах на пределах.
И я видел двенадцать врат на небе на приделах земли, из которых
солнце, луна и звёзды, и все произведения неба выходят на востоке и на западе.
И много оконных отверстий находится направо и налево от них, и каждое
окно выбрасывает в свое время тепло, соответствуя тем вратам, из которых
выходят звезды по повелению, которое Он дал им, и в которые они заходят,
соответствуя их числу.
И я видел на небе колесницы, как они неслись в мире,~--- вверху и внизу
от тех врат,~--- в которых обращаются никогда не заходящие звезды.
И одна их них больше всех их, и она проходит чрез весь мир.
\vs 1En 15:1
И на пределах земли я видел открытыми для всех ветров двенадцать
врат, из которых выходят ветры и дуют на землю.
Трое из них открыты на лице неба (на востоке), и трое на заходе, и трое
на правой стороне неба и трое на левой.
И трое первых лежат к востоку, и трое к северу, и трое, противостоящих
им налево, к югу, и трое на западе.
Чрез четверо из них выходят ветры благословения и благополучия, а из
тех (из остальных) восьми выходят ветры бедствия; когда они посылаются, то
производят разрушение на всей земле, и в воде, существующей на ней, и во всех
тварях, живущих на ней, и во всем, что находится в воде и на суше.
И первый ветер, дующий из тех врат и называющийся восточным,
выходит в первых восточных вратах, склоняющихся к югу; из них выходит
разрушение, сухость, зной и гибель.
И чрез вторые врата, что лежат в средине, выходит правильное смещение,
и именно~--- из них выходит дождь и плодородие, и благополучие, и роса; и чрез
третьи врата, которые лежат к северу, выходит холод и сухость.
И после этих, выходят южные ветры чрез трое врат: во-первых, через
первые из них, которые склоняются к востоку, выходит жгучий ветер.
И через прилежащие к ним средние врата выходят благовония, и роса, и
дождь, и благополучие, и здоровье.
И чрез третьи врата, лежащие к западу, выходит роса, и дождь, и саранча,
и разрушение.
И после этих северные ветры: из седьмых врат, которые на восточной
стороне склоняются к югу выходит роса и дождь, саранча и разрушение.
И из средних врат на прямом направлении выходит дождь, и роса, и
здоровье, и благополучие; и через третьи врата на северо-западной стороне
выходит туман, и иней, и снег, и дождь, и роса, и саранча.
И после этих западные ветры: чрез первые врата, склоняющиеся к северу,
выходит роса, и дождь, и иней, и холод, и снег, и мороз.
И из средних врат выходит роса и дождь, благополучие и благословение;
и чрез последние врата, лежащие к югу, выходит сухость и разрушение, жар и
гибель.
Этим оканчиваются двенадцать врат четырех небесных стран; и все их
законы, и все их бедствия, и все их благодеяния я показал тебе, мой сын
Мафусаил.
Первый ветер называют восточным, так как он передний (первый); и
второй ветер называется южным ветром, так как там нисходит Всевышний; и там
предпочтительнее всего сходит Тот, Который да будет прославлен вовеки.
И западный ветер называется ветром уменьшения, так как там небесные
светила уменьшаются и опускаются.
И четвертый ветер называется северным: он разделяется на три части:
первая из них назначена для жилища людей, вторая для водных морей и с долинами,
и лесами, и реками, и мраком, и туманом; и третья часть с садом правды.
Я видел семь высоких гор, выше всех гор, находящихся на земле; оттуда
выходит иней; и приходят и исчезают дни, времена и годы.
Семь рек видел я на земле, больше всех других; одна из них, текущая с
запада, изливает свою воду в великое море.
И две из них текут с севера к морю, и изливают свою воду в эритрейское
море на востоке.
И четыре остальные вытекают на северной стране к своему морю, две к
эритрейскому морю, и две имеют устье в великом море, по другим~--- в пустыне.
Семь великих островов я видел на море и на суше: два на суше и пять на
великом море.
Имена солнца следующие: первое Оререс, второе Томас.
И луна имеет четыре имени: первое Азонъйя, второе Эбла, третье Беназэ,
и четвертое Эраэ.
Это оба великие светила: их окружность, как окружность неба, и по
величине они оба равны.
В кругу солнца находится одна седьмая часть света, в которою
прибавляется свет луне, и именно~--- в определенной мере прибавляется он, пока не
истощится седьмая часть солнца.
И они заходят и входят в западные врата, и совершают обращение через
север, и через восточные врата они выходят на поверхность неба.
И когда луна поднимается, то она появляется на небе имея в себе света
половину одной седьмой части; и в течение четырнадцати дней весь ее свет
делается полным.
В нее прибавляется также трижды пять (15) частей света, так что к
пятнадцатому дню свет ее становится полным по знаку года, и составляется трижды
пять частей, и луна рождается чрез половину одной седьмой части.
И при своем ущербе она уменьшается в первый день до четырнадцати своих
частей света, во второй до тринадцати, в третий до двенадцати, в четвертый до
одиннадцати, в пятый до десяти, в шестой до девяти, в седьмой до восьми, в
восьмой до семи, в девятый до шести, в десятый до пяти, в одиннадцатый до
четырех, в двенадцатый до трех, в тринадцатый до двух, в четырнадцатый до
половины одной седьмой части: и ее свет, который оставался от целого,
совершенно исчезает в пятнадцатый день.
И в определенные месяцы месяц имеет по двадцати девяти дней и один раз
двадцать восемь.
Также и другое установление показал мне Уриил относительно того, когда
прибавляется луне свет и на которой стороне он прибавляется ей от солнца.
Во все время, когда луна усиливается в своем свете, она лежит по
отношению к солнцу напротив; к четырнадцатому дню ее свет становится полным на
небе; и когда она вся освещена, ее свет бывает полным на небе.
И в первый день она называется новолунием, ибо в тот день начинается в
ней свет.
И она становится полною ровно в тот день (в 15-ый), когда солнце
заходит на западе, а она ночью восходит с востока и светит целую ночь, пока
солнце не взойдет напротив нее, и она бывает видима напротив солнца.
На той стороне, где бывает свет луны, она также опять уменьшается,
пока не исчезнет весь ее свет, и дни месяца оканчиваются, и ее круг остается
пустым без света.
И в продолжение трех месяцев она делает тридцать дней в свое время, и
в продолжение трех месяцев она делает по двадцати девяти дней, в которых
происходит ее ущерб в первое время и в первых вратах в течение ста семидесяти
семи дней.
И во время своего восхода она показывается в продолжение трех месяцев
по тридцать дней, и в продолжение трех месяцев по двадцати девяти дней.
Ночью она показывается приблизительно в течение двадцати дней как муж,
и днем как небо, ибо нет ничего другого в ней, кроме ее света.
И теперь, мой сын Мафусаил, я показал тебе все, и весь закон
звезд (светил) небесных окончен.
И он (Уриил) показал мне весь закон их для каждого дня, для каждого
времени (года), для каждого господства, и для каждого года, и его выход по Его
предписанию для каждого месяца и каждой недели; и он показал ущерб луны,
который происходит в шестых вратах; именно~--- в этих шестых вратах оканчивается
весь ее свет, и после этого там бывает начало месяца; и он показал ущерб,
который происходит в первых вратах в свое время, пока не пройдет сто семьдесят
семь дней, а по исчислению по неделям~--- двадцать недель и два дня; и он
показал, как она отстает от солнца и от порядка звезд ровно на пять дней в одно
время, и когда это место, которое ты видишь, оканчивается.
Таков образ, и описание каждого светила, как их показал мне вождь их~---
великий ангел Уриил.
И в те дни отвечал мне Уриил и сказал мне: "вот я показал тебе
все, о Енох, и открыл тебе все, чтобы ты увидел это, это солнце, и эту луну, и
путеводителей звезд небесных, и всех тех, которые вращают их, их соотношения, и
времена, и выходы.
И в дни грешников годы будут укорочены, и их посев будет запаздывать в
их странах и на их пастбищах (полях), и все вещи на земле изменятся и не будут
являться в свое время; дождь будет задержан, и небо удержит его.
И в те времена плоды земли будут запаздывать и не будут вырастать в
свое время; и плоды деревьев будут задержаны от созревания в свое время.
И луна изменит свой порядок и не будет являться в свое время.
И в те дни будет видимо на небе, как приходит великое неплодородие, на
самой крайней колеснице на западе; и оно (небо или солнце) будет светить ярче,
чем по обыкновенному порядку света.
И многие главы начальственных звезд будут ошибаться и они нарушат свои
пути и отправления, и подчинённые им не будут появляться в свои времена.
И весь порядок звезд сокрыт для грешников, и мысли тех, которые живут
на земле, будут ошибаться из-за них, и они уклонятся от всех своих путей, и
будут грешить и станут считать их (звезды) за богов.
И много зол придет на них, и осуждение придет на них, чтобы уничтожить
их всех".
И он сказал мне: "о Энох, рассмотри писание небесных скрижалей и
прочитай, что на них написано, и заметь для себя все в отдельности".
И я рассмотрел все на небесных скрижалях, и прочитал все, что на них,
и заметил для себя все, и прочитал книгу и все, что было на ней, все дела людей
и всех телесно-рожденных, которые будут на земле до самых отдаленных родов.
И после этого я тотчас прославил Господа, вечного Царя славы, за то,
что Он сотворил все произведения мира и восхвалил Господа за Его терпение, и
благословил Его за детей мира.
И в тот час я сказал: "блажен муж, который умирает как праведный и
благой, о котором не написано никакое писание неправды и против которого не
найдено вины"!
И те трое святых ангелов принесли меня и поставили меня на землю пред
дверями моего дома, и сказали мне: "возвести все своему сыне Мафусаилу и открой
всем своим детям, что ни один из смертных не праведен пред Господом, ибо Он
Творец их.
На один год мы оставим тебя при твоих детях,~--- пока ты не укрепишься
снова,~--- чтобы ты научил своих детей, и записал им это, и засвидетельствовал
им, всем твоим детям, и на другой год ты будешь взят из среды их.
Ибо добрые будут возвещать правду; праведный будет радоваться с
праведными, и они будут благожелать друг другу.
Грешник же умрет с грешником и отпадший потонет с отпадшим.
И те, которые сохранят справедливость, умрут ради дел людей и будут
соединены ради деяния нечестивых".
И в те дни они перестали говорить со мною.
И я пришел к своим домочадцам, прославляя Господа мира.
И теперь, сын мой Мафусаил, я рассказываю тебе все эти вещи и
записываю тебе; и я открыл тебе все и дал тебе писание обо всех них (светилах);
итак, сохрани же, мой сын Мафусаил, писания ради твоего отца, и передай их
грядущим родам.
Мудрость я дал тебе и твоим детям, и тем твоим детям, которые еще
придут, чтобы они передали ее своим детям и грядущим родам до вечности,~---
именно эту мудрость, превышающую их мысли.
И разумеющие ее не будут спать, и будут прислушиваться своим ухом,
чтобы научиться этой мудрости, ибо она понравится тем, которые кушают от неё,
лучше приятной пищи.
Блаженны все праведные, блаженны все, ходящие по пути правды, и не
погрешающие, подобно грешникам, в исчислении всех своих дней, в течение которых
солнце ходит на небе, входя и выходя через врата по тридцати дней вместе с
главами над тысячью этого порядка звезд, именно~--- вместе с четырьмя, которые
прибавляются и разделяют четыре части года, которые их направляют, и с ними
входят четыре дня.
И из-за них люди будут ошибаться, и не будут считать их при исчислении
целого движения мира; напротив люди будут ошибаться в них и не узнают их в
точности.
Ибо они (добавочные дни) относятся к исчислению года и действительно
отмечены навсегда~--- один в первых вратах, и один в третьих, и один в четвертых,
и один в шестых; и год завершается в 364 дня.
И рассказ об этом праведен, и точно указано исчисление этого
(т.е.года) ибо светила, и месяцы, и праздники, и годы, и дни мне показал и
внушил Уриил, которому Господь всего мироздания дал повеление ради меня
относительно воинства небесного; и он имеет власть над ночью и днем на небе,
чтобы заставлять свет светить над людьми,~--- солнце, луну и звёзды, и все силы
небесные, которые вращаются в своих кругах.
И таковы порядки звезд, которые заходят в своих местах и в свое время,
и праздники и месяцы.
И таковы имена тех, которые путеводят их (звезды) и которые
бодрствуют, чтобы он вступил в определенные им времена, в своих порядках, в
свои сроки, и месяцы, и времена господства, и по своим местам.
Четыре их путеводителя, которые разделяют четыре части года, вступают
прежде всех и после них двенадцать путеводителей порядков, которые разделяют
месяцы и год на 364 дня, рядом с главами над тысячью (хилиархами), которые
делают дни; и для четырех добавочных дней существуют те же путеводители,
которые разделяют четыре части года.
И из тех начальников над тысячью один расположен между путеводителем и
путеводимым позади мест, но только путеводители их делают разделение.
И вот имена путеводителей, разделяющих четыре установленные части
года: Мелкеел, и Гелеммелех, и Мелейял, и Нарел, И имена тех, которых они
ведут: Аднарел, и Ийязузаел, и Ийелумиел.
Эти трое следуют за путеводителями порядков, и один следует за троими
путеводителями порядков, следующими за теми место начальниками (топархами),
которые разделяют четыре части года.
В начале года первым восходит и управляет Мелкейял, который называется
Таммани и солнцем; и всего времени его господства, в продолжение которого он
управляет, девяносто один день.
И вот признаки дней, которые должны появляться на земле во время его
господства: пот, и жар, и тоска; все деревья тогда производят плоды, и листва
появляется на всех деревьях, и бывает жатва пшеницы и расцвет роз, и все цветы
тогда цветут на поле, но зимние деревья становятся сухими.
И вот имена подчиненных им (топархам) путеводителей: Беркеел,
Цалбезаел и еще другой, который присоединяется,~--- глава над тысячью, называемый
Голойязеф, и дни господства этого заканчиваются.
Другой путеводитель (топарх), который следует за ними, есть
Гелеммелек, которого называют также светящим солнцем; и все время его света
девяносто один день, и вот признаки дней на земле в то время: жар и сухость, и
плоды деревьев становятся зрелыми и спелыми, и плоды их сохнут; и овцы тогда
спариваются и становятся суягными; и тогда собираются все плоды земли и все,
что есть на полях, и бывает выжимание винограда: все это происходит во дни его
господства.
И вот имена, и порядки, и подчиненные им путеводители тех глав над
тысячью: Гедаел, и Кеел, и Геел, и имя начальника над тысячью, который
присоединяется к ним, Асфаел; и оканчиваются дни его господства.
\vs 1En 16:1
И теперь, мой сын Мафусаил, я хочу открыть тебе все видения,
которые я видел, рассказавши тебе их.
Два видения видел я, прежде чем взял жену, и они не похожи одно на
другое; в первый раз, когда я изучал писание, и во второй раз, прежде чем взять
твою мать, я видел страшные видения: и из-за них я молил Господа.
Я лег в доме моего деда Малелеила, и тогда я увидел в видении, как небо
опустилось и уменьшилось, и упало к земле.
И когда оно упало к земле.
И когда оно упало на землю, я увидел землю, как она была поглощена
великою бездною, и горы опустились на горы, и холмы погрузились на холмы, и
высокие деревья оторвались от своих стволов (корней), и низверглись и потонули
в бездне.
И от этого в моих устах обрелась речь, и я начал восклицать и сказал: "
погибла земля"!
И мой дед Малелеил разбудил меня, ибо я лежал около него, и сказал мне:
"отчего ты восклицаешь так, мой сын, и от чего ты так сетуешь"?
Тогда я рассказал ему видение, которые видел, и он сказал мне: "ужасно
то, что ты видел, мой сын!
И твое сновидение обнимает тайну всех грехов земли: она должна
погрузиться в бездну и потерпеть насильственную гибель.
И теперь, сын мой, встань и молись Господу славы,~--- ибо ты верующий,~---
чтобы остаток сохранился на земле целым и чтобы Он истребил не всю землю.
Сын мой!
С неба все это придет на землю, и на земле совершится насильственная
гибель".
После этого я встал, и просил, и умолял, и записал свою молитву для
грядущих родов, и я все покажу тебе, сын мой Мафусаил.
И когда я вышел вниз (т.е.из дому), и увидел небо и солнце, восходящее
на востоке, и луну, опускающуюся на западе, и все, как Он узнал это в начале,
то я прославил Господа суда, и превознес Его, ибо Он повелел солнцу выходить из
окон востока, чтобы оно поднималось, и восходило на плоскости неба, и
возносилось, и проходило теперь путь, который ему указан.
И я воздвиг руки свои в правде, и прославил Святого и Великого, и
говорил дыханием моих уст и телесным языком, который сотворил Бог для сынов
человеческих, чтобы они говорили им, и дал им дыхание, и язык, и уста, чтобы
они говорили благодаря этому.
"Будь прославлен Ты, о Господи, Царь Великий и Могущественный в Своем
величии, Господь всего небесного творения, Царь царей, и Бог всего мира!
И Твое божество, и царство, и величие пребывает во век и от века до
века, и Твое господство~--- чрез все роды, и все небеса служат Тебе престолом
вовек, и вся земля~--- подножием Твоих ног вовек и от века до века.
Ибо Ты сотворил и господствуешь над всем, и для Тебя совершенно ничего
нет трудного, и никакая мудрость не ускользнет от Тебя; она не отвращается от
своего престола,~--- Твоего престола,~--- ни от Твоего лица; и Ты знаешь, и видишь,
и слышишь все, и нет ничего, чтобы было сокровенно для Тебя, ибо Ты видишь все.
И теперь ангелы Твоего неба беззаконнуют, и гнев Твой пребывает на
плоти людей до дня великого суда.
И теперь, о Боже и Господи, и великий Царь, я молю и прошу, чтобы Ты
исполнил для меня мою просьбу, прошу оставить мне на земле потомство целым, и
не истреблять всю плоть человеческую, и не делать землю безлюдною, чтобы была
вечная гибель, и теперь, Господь мой, истреби от земли плоть, которая
разгневала Тебя, но плоть правды и праведности утверди как растение семени
навсегда, и не отвращай Твоего лица от молитвы раба Твоего, о Господи"!
\vs 1En 17:1
И после этого я видел другой сон, и я вполне открою его тебе, мой
сын.
И Енох начал и сказал своему сыну Мафусаилу: "тебе я буду говорить,
мой сын; слушай речь мою и приклони ухо свое к сновидению твоего отца!
Прежде чем я взял твою мать Едну, я видел в видении на своем ложе, и
вот телец вышел из земли, и тот телец был белый; и за ним вышло женское рогатое
животное, и вместе с ним вышли другие рогатые животные: одно из них было
черное, другое красное.
И то черное рогатое животное бодало красное и преследовало его на
земле; и скоро я не мог более видеть того красного рогатого животного.
Но то черное рогатое животное выросло и к нему пришло женское рогатое
животное, и я видел, как многие тельцы, которые были похожи на него и следовали
за ним, вышли от него.
И та корова,~--- та первая,~--- вышла от лица того первого тельца, чтобы
искать то красное животное, но не нашла его, и тотчас подняла великий жалобный
вопль, и искала его.
И я видел, как пришел к ней тот первый телец и успокоил ее, и с того
часа она более не ревела.
После этого она родила другого белого тельца, а после него родила
многих других тельцов и черных коров.
И я видел в моем сновидении, как тот белый вол также вырос и сделался
большим белым волом, и от него произошло много белых тельцов, которые были
похожи на него, И они стали производить белых тельцов, которые были похожи на
них, следуя один за другим.
И я опять видел своими очами, в то время как спал, и увидел
вверху небо, и вот одна звезда упала с неба, и она поднялась, и ела, и паслась
между теми тельцами.
И после этого я видел больших и черных тельцов, и вот они все
переменили свои загороди, и пастбища, и своих рогатых животных, и начали
сетовать друг с другом.
И я опять видел в видении, и посмотрел на небо, и вот я увидел много
звезд, как они упали и были низвергнуты с неба к той первой звезде и в среду
тех рогатых животных и тельцов; и вот они были с теми и паслись в среде их.
И я посмотрел на них и увидел, и вот все они обнаружили свои срамные
члены, как кони, и начали подниматься на тельцовых коров; и все они стали
стельными, и родили слонов, верблюдов и ослов.
И все тельцы устрашились и испугались их; и они начали кусаться своими
зубами и пожирать, и бодать своими рогами.
И они начали теперь поедать тех тельцов; и вот все дети земли начали
трепетать пред ними, и дрожать, и спасаться бегством.
И я опять видел их, как они начали бодаться сами между собою и
пожирать друг друга, и земля стала взывать.
И я опять поднял свои очи к небу и увидел в видении: и вот там вышли
из неба имевшие вид белых людей; из того места вышел один и вместе с ним трое.
И те трое, которые вышли после, взяли меня за руку и подняли меня
прочь от рода земли, и вознесли меня на высокое место, и показали мне башню,
высоко стоящую над землей, и все холмы были ниже ее.
И они сказали мне: "оставайся здесь, чтобы видеть всё, что произойдет
со всеми теми слонами, и верблюдами, и ослами, со звездами, и со всем
тельцами"!
И я видел одного из тех четверых, которые вышли прежде, как он
схватил звезду, прежде всех ниспадшую с неба, связал ей руки и ноги, и положил
ее в пропасть; пропасть же та была тесна и глубока, ужасна и мрачна.
И один из них обнажил свой меч и отдал его тем слонам, и верблюдам, и
ослам; тогда они начали поражать друг друга, так что вся земля дрожала
вследствие этого.
И когда я видел в видении,~--- вот там бросился теперь с неба вниз один
из тех четверых, которые спустились, и собрал и взял великие звезды, срамные
члены которых были как срамные члены коней, и связал их всех по рукам и ногам,
и положил их в ущелье земли.
И один из тех четверых пришел к тем белым тельцам, и научал его
(одного из них) тайне, в то время как он трепетал: он был рожден подобно тельцу
и сделался человеком, и выстроил себе большое судно и поселился в нем; вместе с
ним расположились также в том судне трое тельцов; и оно было закрыто над ними.
И я опять поднял свои очи к небу и увидел высокую крышу с семью
шлюзами на ней, и те шлюзы изливали много воды во двор.
И я видел опять, и вот, тогда открылись источники на почве в том
великом дворе, и эта самая вода начала волноваться и подниматься выше почвы, и
сделала тот двор невидимым, так что вся почва его закрылась водою.
И выростала на нем (дворе) вода, мрак и облако; и тогда я посмотрел на
высоту той воды, как она поднялась выше того двора, и текла поверх него, и
остановилась на земле.
И все тельцы того двора столпились вместе, так что я тотчас увидел,
как они потонули, и были поглощены и погибли в той воде.
Само же судно плавало по воде, между тем как все тельцы, и слоны, и
верблюды, и ослы на земле погрузились вместе со всем скотом, так что я не мог
более видеть их, и они не могли выйти, но потонули и погрузились в бездне.
И я опять видел в видении, как те шлюзы отложились от той высокой
крыши, и источники земли иссякли, и другие бездны открылись.
Тогда вода начала стекать в них, пока земля не сделалась видимою; а то
судно твердо встало на земле, и отступил мрак, и просиял свет.
А тот белый телец, который стал мужем, вышел из того судна и три
тельца с ним; и один из трех был белый, подобно тому тельцу, и один из них был
красный, как кровь, и один черный; и этот самый,~--- тот белый телец, отошел от
них.
И они начали рождать диких зверей и птиц, так что от всех их вместе
произошло разнообразное множество видов,~--- львы, тигры, псы, волки, шакалы,
дикие свиньи, соколы, коршуны, ястребы, орлы и вороны; и в среде их родился
белый телец.
И они начали грызться друг с другом: но тот белый телец, родившийся в
среде их, произвел дикого осла и вместе с ним белого вола; и дикий осел
умножился.
А тот телец, родившийся от него, произвел черную дикую свинью и белую
овцу; и та дикая свинья произвела многих свиней, та овца произвела двенадцать
овец.
И когда те двенадцать овец выросли, они передали одну из своей среды
ослам, и эти ослы опять передали ту овцу волкам, и та овца росла между волками.
И Господь привел одиннадцать овец~--- жить вместе с нею и пастись при
ней среди волков, и они размножились и выросли во многие овечьи стада.
И волки начали бояться их, и притесняли их, так что, наконец, стали
лишать жизни их агнцев; и они бросали их агнцев в многоводную реку; а те овцы
начали кричать о своих агнцах и жаловаться своему Господу.
И одна овца, которая была спасена от волков, убежала и ушла к диким
ослам; и я видел овец, как они сетовали, и кричали, и просили своего Господа
изо всех сил, пока тот Господь не сошел из высокого покоя на зов овец, и не
пошёл к ним и не посетил их.
И Он позвал ту овцу, удалившуюся от волков, и говорил с нею
относительно волков, чтобы она уговорила их не трогать овец.
И овца пошла к волкам по слову Господа, и другая овца сошлась с той
овцой и пошла с нею, и они обе вместе одна с другой пришли на сборище тех
волков, и говорили с ними, и увещевали их отныне не трогать впредь более овец.
При этом я видел волков, и как они стали еще более смирять овец всею
своею силою; и овцы кричали.
И Господь их пришел к овцам и начал бить тех волков; тогда волки
начали сетовать, овцы же сделались спокойными и тотчас не стали более кричать.
И я видел овец, как они ушли от волков; у волков же глаза были
ослеплены, и те волки вышли для преследования овец со всею своею силою.
И Господь овец шел с ними, предводительствуя ими, и все Его овцы
следовали за Ним; лицо же Его было блестящее, и вид Его страшен и величествен.
А волки стали преследовать тех овец, пока не настигли их при водном
озере.
И это самое водное озеро разделилось, и вода остановилась пред ними по
обеим сторонам; и их Господь, Который вел их, встал между ними и волками.
И так как те волки не стали уже видеть овец, то они вошли в средину
того водного озера и преследовали овец, и те волки погнались за ними в водном
озере.
И когда они увидели Господа овец, то воротились, чтобы убежать от
Него, но то водное озеро соединилось, внезапно приняло свою природу, и вода
поднялась и возвысилась, так что покрыла тех волков.
И я видел, как все волки, преследовавшие тех овец, погибли и потонули.
Но овцы вышли из той воды и перешли пустыню, где не было воды и травы;
и они начали открывать свои глаза и видеть; и я видел Господа овец, как Он пас
их и дал им воды и травы, и как та овца шла и вела их.
И та овца поднялась на вершину высокой скалы; и Господь овец послал ее
к ним.
И после этого я видел Господа овец, стоящего пред ними; и Его вид был
величествен и чрезмерно велик, и все те овцы видели Его и устрашились пред Его
лицом.
И все они устрашились и трепетали пред Ним, и кричали после ухода той
овцы, которая была при Нем, к другой овце, находившейся между ними: "мы не
можем вынести этого пред нашим Господом и взирать на Него".
И та овца, которая вела их, возвратилась и поднялась на вершину той
скалы; но овцы начали слепнуть и уклоняться от пути, который она показала им;
между тем та овца ничего не знала об этом.
И Господь овец сильно разгневался на них, и та овца узнала это и
спустилась с вершины скалы, и пришла к овцам, и нашла самую большую часть из
них ослепленною и уклонившеюся от своего пути.
И как только они увидели ее, устрашились и затрепетали пред ее лицом,
и пожелали возвратиться в свои загороди.
И та овца взяла с собою других овец и пришла к тем уклонившимся овцам,
и при этом начала умерщвлять их, и овцы устрашились пред ее лицом; и таким
образом та овца направила уклонившихся овец, и они возвратились в свои
загороди.
И я видел там видение, как та овца сделалась мужем, и выстроила
Господу овец дом, и повелела всем овцам стоять в том доме.
И я видел, как овца, сошедшаяся с той овцою, которая вела их, заснула;
и я видел, как все большие овцы погибли, и малые направились к своему месту, и
они пошли на пастбище и приблизились к водной реке.
Тогда отделилась от них та овца, которая вела их и которая сделалась
мужем, и заснула; и все овцы искали ее и подняли по ней великий вопль.
И я видел, как они прекратили вопль по той овце и переправились через
ту водную реку; и стояли всегда овцы, ведшие их, на месте тех, которые заснули
и которые вели их.
И я видел, как овцы пришли в хорошее место и в вожделенную и
великолепную страну, и видел, как те овцы насытились; а тот дом стоял между
ними в вожделенной стране.
И глаза их то открывались, то ослеплялись, пока не восстала другая
овца, и не повела их, и не направила их всех, и глаза их открылись.
И псы, и лисицы, и дикие свиньи начали пожирать тех овец, пока не
восстала другая овца,~--- баран из их среды, который вел их.
И тот баран начал бодать на обе стороны тех псов, лисиц и диких
свиней, пока не уничтожил их всех.
И у той овцы раскрылись глаза, и она увидела того барана, бывшего
между овцами, как он отрекся от своего достоинства и начал бодать тех овец, и
попирал их, и действовал непристойно.
И Господь овец послал овцу к другой овце, и возвысил ее (последнюю),
чтобы она была бараном и вела овец вместо той овцы, которая оказалась неверной
в своем достоинстве.
И она пошла к ней и говорила только с ней, и поставила ее бараном, и
сделала ее царем и вождем овец; а между всем этим псы притесняли овец.
И первый баран преследовал того второго барана, и тот второй баран
встал и убежал от него; и я увидел, как те псы низвергли того первого барана.
И тот второй баран возвысился и вел малых овец; и тот баран родил
многих овец и заснул; и малая овца сделалась бараном вместо него, и стала
вождём и царем тех овец.
И выросли и размножились те овцы, и все псы, и лисицы, и дикие свиньи
устрашились и разбежались от него; и тот баран бодал и убивал диких зверей; и
те дикие звери не могли уже осилить овец, и никогда уже не похищали у них
ничего.
И тот дом стал великим и широким, и тем овцам была выстроена высокая
башня над тем домом для Господа овец; и тот дом был низок, а башня была
возвышена и высока; и Господь овец стоял на той башне, и пред Ним поставили
полный стол.
И я видел опять тех овец, как они опять заблудились и пошли
многоразличными путями, и оставили тот свой дом; и Господь овец призвал
некоторых из овец и послал их к овцам, но овцы начали умерщвлять их.
И одна из них спаслась и не была умерщвлена, и она убежала и кричала
об овцах, и он хотел ее умертвить; но Господь овец спас ее из рук их и возвел
ее ко мне, и позволил ей жить там.
И многих других овец Он посылал к тем овцам, чтобы свидетельствовать
(или увещевать) и сетовать о них.
И после этого я видел: вот они оставили дом Господа овец и его башню;
они уклонились совершенно и их глаза ослепли; и я видел Господа овец, как он
допустил много поражений над ними в их отдельных стадах, так что те овцы начали
жаловаться на такие поражения и переменили место.
И Он предал их в руки львов и тигров, и волков, и шакалов, и в руки
лисиц и всех диких зверей; и дикие звери стали разрывать тех овец.
И я видел, что Он оставил тот дом их и их башню, и предал их всех в
руки львов, в руки всех диких зверей, чтобы они разрывали их и пожирали.
И я начал кричать изо всех сил, и призывать Господа овец, и
представлять Ему относительно овец, что они пожираются всеми дикими зверями.
Но Он оставался спокойным, когда видел это, и радовался, что они
пожираются, и истребляются и расхищаются; и Он оставил их в руках всех диких
зверей на съедение.
И Он призвал семьдесят пастырей,~--- и отверг тех овец,~--- чтобы они
пасли их, и сказал пастырям и их товарищам: "каждый из вас должен отныне пасти
овец, и все, что Я вам прикажу, то делайте!
И Я передаю их вам по числу, и буду вам объявлять: кто из их должен
погибнуть, тех истребляйте"!
И Он предал им тех овец.
И Он призвал другого и сказал ему: "замечай и смотри за всем, что
будут делать пастыри с этими овцами: ибо они будут губить их более, чем Я им
повелел.
И всякий излишек и уничтожение, которое будет совершаемо пастухами,
запиши,~--- именно сколько губят они по Моему повелению и сколько по своей
собственной воле; и запиши о каждом пастыре в отдельности все, что они губят.
И прочитай это предо Мною по числу (с указанием числа), сколько они
погубили по собственной воле и сколько предано им на погибель, чтобы это было
для Меня свидетельством против них, дабы я знал всякое действие пастырей, чтобы
передать их суду; и смотри, что они делают,~--- пребывают ли в Моем повелении,
которое Я им дал, или нет.
Но они не должны открывать им этого и наставлять их на путь, но запиши
только все, что они погубят, всякий раз о каждом в отдельности, и представь все
Мне"!
И я видел, как те пастыри пасли в определенное им время, и они начали
умерщвлять и погубят более чем им было повелено, и предали тех овец в руки
львов.
И львы и тигры пожирали и истребляли большую часть тех овец, и дикие
свиньи пожирали вместе с ними; и они сожгли ту башню и разрушили тот дом.
И я сильно опечалился из-за башни, так как самый дом овец был
разрушен; и после этого я не мог уже видеть тех овец, входили ли они в тот дом.
И пастыри и их товарищи предали тех овец всем диким зверям, чтобы они
пожирали их; и каждый в отдельности из них получил в своё время определенное
число, и о каждом в отдельности записал другой в книгу, сколько он погубил.
И каждый из них умертвил и погубил гораздо более чем ему было
позволено; и я начал плакать и сильно сетовать о тех овцах.
И я видел в видении того писца, как он записал каждую овцу, погибшую
от тех пастырей, день за днем, и всю книгу вознес к Господу овец, и представил
и показал, что они сделали, и всех, которых каждый из них уничтожил, и всех,
которых они предали погибели.
И книга была прочитана пред Господом овец, и он взял книгу в свои
руки, и прочитал ее, и сложил ее.
И тотчас я увидел, как пастыри пасли в продолжение двенадцати часов; и
вот три из тех овец возвратились, и пришли, и приступили, и начали строить все,
что было разрушено в том доме; но дикие свиньи помешали им, так что они не
могли продолжать этого.
И они начали опять строить, как прежде, и возвели ту башню, и она была
названа высокой башней; и они начали опять ставить стол пред башнею, но весь
хлеб на нем был скверен и нечист.
И по отношению ко всему глаза у тех овец были ослеплены, так что они
не видели, а также и у пастырей их, весьма многие из них были преданы пастырям
на погибель, и они попирали овец своими ногами и пожирали их.
И Господь овец оставался спокойным, пока все овцы не рассеялись по
полю и не перемешались с ними (диким зверями), и они (пастыри) не спасли их от
рук зверей.
И тот, который писал книгу, вознес ее к обителям Господа овец, и
показал ее, и умолял Его за их, и просил Его, показав Ему всю деятельность
пастырей их, и представил Ему свидетельство против всех пастырей.
И он взял книгу, сложил ее у Него и вышел.
И я смотрел до тех пор, пока таким образом не приняли паству
тридцать семь пастырей, и они все окончили каждый свое время, как первые; и
другие приняли их (овец) в свою власть, чтобы каждый пас их по определенному им
времени,~--- каждый пастырь в свое время.
И после этого я видел в видении, как пришли птицы небесные,~--- орлы,
коршуны, ястреба, вороны; орлы же предводительствовали всеми птицами; и они
начали пожирать тех овец, и выклевывать им глаза, и пожирать их мясо.
И овцы кричали, так как их мясо было пожираемо птицами, и я восклицал
и жаловался во время моего сна на того пастыря, который пас овец.
И я видел, как те овцы были пожраны псами, и орлами, и ястребами, и
они не оставили им ни мяса, ни кожи, ни сухожилий, так что от них остался
только остов, но и остов их упал на землю, и овец стало мало.
И я смотрел до тех пор, пока не приняли паству двадцать три пастыря,
и окончили, управляя каждый по определенному им времени, пятьдесят восемь
времен.
Но от тех белых овец родились малые агнцы, и они стали открывать свои
глаза, и видеть, и кричать овцам.
И овцы не кричали им и не слышали, что и сказали им, но были
чрезвычайно глухи, и их глаза были слишком и чрезмерно ослеплены.
И я видел в видении, как вороны налетели на тех агнцев и взяли одного
из тех агнцев, овец же разорвали и пожрали.
И я видел, как у тех агнцев выросли рога, и вороны низвергли их рога;
и я видел, как выскочил один великий рог,~--- одна из тех овец; и их глаза
открылись.
И я смотрел за ним, и глаза их раскрылись; и она кричала к овцам, и
юнцы увидели ее и все побежали к ней.
И между всем тем те орлы, и коршуны, и вороны, и ястреба все еще
разрывали овец беспрестанно, и слетались, на них и пожирали их; но овцы
оставались покойными, и юнцы сетовали и кричали.
И те вороны сражались и боролись с ними, и хотели сломить его рог, но
ничего не могли сделать с ним.
И я видел их, пока не пришли пастыри, и орлы, и те коршуны и ястреба,
и они кричали воронам, чтобы они сломили рог того юнца; и они боролись и
сражались с ними, и он боролся с ним, и кричал, чтобы пришла к нему помощь.
И я видел, как пришел тот муж, который записывал имена пастырей и
представлял Господу овец, и он помог тому юнцу, и показал ему все, чтобы пришла
его помощь.
И я видел, как тот Господь овец пришел к ним во гневе, и все видевшие
Его убежали, и упали все в Его тени пред лицом Его.
Все орлы, и коршуны, и вороны, и ястребы собрались и привели с собою
всех полевых овец, и все они сошлись и помогали друг другу, сломить тот рог
юнца.
И я видел того мужа, который писал книгу по повелению Господа, как он
развернул ту книгу умертвления, которое совершили те двенадцать последних
пастырей, и он показал пред Господом овец, что они умертвили гораздо более, чем
предшествовавшие.
И я видел, как пришел к ним (к хищным птицам и зверям) Господь овец,
и взял в Свою руку посох гнева, и ударил в землю, так что она расторгалась, и
все звери и птицы небесные упали с овец, и погрузились в землю, и она
замкнулась над ними.
И я видел, как овцам дан был великий меч: тогда овцы выступили против
тех полевых зверей, чтобы умертвить их, и все звери и птицы небесные
разбежались от их лица.
И я видел, как был воздвигнут престол в любимой земле, и Господь овец
воссел на нем; и он взял все запечатанные книги и раскрыл их пред Господом
овец.
И Господь призвал тех шесть (или семь) первых белых, чтобы они
принесли к Нему, начиная от первой звезды, пришедшей вперёд, все звезды, у
которых срамные члены были как срамные члены коней, и первую звезду, которая
ниспала прежде всех; и они принесли их все к Нему.
И Он сказал тому мужу, который писал пред Ним и который был одним из
тех шести (или семи) белых, и сказал ему: "возьми тех семьдесят пастырей,
которым Я предал овец, и которые взяли их и умертвили из них более, чем Я им
повелел, самовластно"!
И вот я видел их всех связанными, и они все стояли пред Ним.
И суд совершился, прежде всего, над звездами, и они были судимы и
оказались виновными, и пришли к месту осуждения, и их бросили в глубокое место,
наполненное огнем, пылающее и наполненное огненными столбами.
И те семьдесят пастырей были судимы и оказались виновными, и точно
также были брошены в ту огненную пропасть.
И я видел тогда, как была открыта подобная пропасть в средине земли,
наполненная огнем, и как принесли тех ослепленных овец, и они все были судимы и
найдены виновными, и брошены в ту огненную пропасть, и они сгорели: а пропасть
эта была направо от того дома.
И я видел, как сгорели те овцы, и кости их сгорели.
И я встал, чтобы видеть, как Он украшал тот древний дом: и выломали в
нем все столбы, и все балки и украшения этого дома были завернуты вместе с
ними; и выломали их совсем, и положили их в одно место на юге страны.
И я видел Господа овец, как он принес новый дом больше и выше того
первого, и поставил его на месте первого, который был завернут; все его столбы
были новы и больше, чем украшения первого древнего, который Он выломал; и все
овцы были в нем.
И я видел всех овец, которые остались целыми, и всех зверей на земле
и всех птиц небесных, как они пали ниц и воздавали честь тем овцам, и умоляли
их, и слушались их в каждом слове.
И после этого меня взяли те трое, одетые в белом, которые подняли
меня прежде, за мою руку, и в то время, как рука того юнца взяла меня, они
подняли меня и посадили меня посреди тех овец, прежде чем совершился суд.
А те овцы были все белы и их шерсть была большая и чистая.
И все и все погибшие и рассеянные овцы, и все звери полевые, и все
птицы небесные собрались в том доме, и у Господа овец была великая радость, так
как все они были добры и возвратились к Его дому.
И я видел, как они сложили тот меч, который был дан овцам, и принесли
назад в Его дом, и он был запечатан пред лицом Господа; и все овцы были
заключены в тот дом, и он не вмещал их.
И у них у всех были открыты глаза, так что они видели доброе, и не
было между ними ни одной, которая бы не сделалась видящею.
И я видел, что тот дом был велик и широк, и весьма наполнен.
И я видел, что родился белый телец с большими рогами, и все звери
полевые и все птицы небесные устрашились его и умоляли его все время.
И я видел, как весь род их изменился, и все они стали белыми
тельцами; и первый между ними (был Слово, и это Слово сделалось) сделался
великим зверем, и имел большие черные рога на своей голове; и Господь овец
радовался, взирая на них и на всех тельцов.
И я спал в среде их, затем пробудился и видел все.
Таково видение, которое я видел в то время, как спал, и я пробудился
и прославил Господа правды, и воздал Ему хвалу.
И после этого я поднял великий вопль, и мои слезы не останавливались,
так как я не мог более удержаться; когда я смотрел, то у меня лились слезы по
поводу того, что я видел, ибо все придет и исполнится; и всякое деяние людей
мне было показано по порядку.
И в ту ночь я вспомнил о моем первом сне; также и из-за этого я
плакал и трепетал, ибо я видел то видение.
\vs 1En 18:1
И теперь, мой сын Мафусаил, призови ко мне всех своих братьев, и
собери ко мне всех сыновей твоей матери; ибо слово побуждает меня и дух излился
на меня, чтобы я открыл вам все, что придет на вас до вечности.
После этого Мафусаил пошел и призвал всех своих братьев к себе, и
собрал своих родственников.
И он (Енох) говорил со всеми своими детьми о правде, и сказал:
"вслушайтесь, сыны мои, каждую речь вашего отца и должным образом внемли гласу
моих уст, ибо я увещеваю вам, возлюбленные мои: любите праведность и ходите в
ней.
И не приближайтесь к праведности с двояким сердцем, и не
присоединяйтесь к тем, у которых двоякое сердце, но ходите в правде, сыны мои;
и она приведет вас на добрые пути, и правда будет вашей помощницей.
Ибо я знаю, что дела насилия возьмут верх на земле, и великое осуждение
совершится на земле; и всякая неправда прекратится и будет отделена от своих
корней, и все здание ее исчезнет.
И неправда опять повторится, и все дела неправды и все дела насилия и
беззакония вторично совершатся на земле.
И так как тогда усилится неправда, и грех, и хула, и насилие, и другого
рода действия, и увеличится отпадение, и беззаконие, и нечистота, то придет
великое осуждение с неба на всех них, и святой Господь выйдет с гневом и
наказанием, чтобы совершить суд на земле.
В те дни насилие будет отделено от своих корней, и корни неправды
погибнут вместе с ложью, и они исчезнут из-под неба.
И все идолы язычников будут преданы погибели; башни будут сожжены
огнем, и их уберут со всей земли; и они будут брошены по осуждению в огнь, и
погибнут в гневе и жестоком осуждении, которое продолжится вовек.
И восстанет тогда праведный от сна, и мудрость восстанет и будет дана
им.
И после того корни неправды будет отделены, и грешники погибнут от
меча, у клеветников будут отделены корни во всяком месте, и те, которые
замышляют насилие и произносят хулу, погибнут от острия меча.
И после этого будет другая седмина~--- восьмая, седмина правды; и будет
дан ей меч, чтобы судить и справедливость исполнить над теми, которые поступают
насильственно, и грешники будут преданы в руки праведных.
И в конце ее они приобретут домы своею спаведливостю, и создастся дом
великому Царю в прославление навсегда и навечно.
И после этого в девятую седмину откроется всему миру праведный суд, и
все деяния нечестивых исчезнут со всей земли; и мир будет присужден к погибели,
и все люди будут взирать на путь праведности.
И после этого в десятую седьмину, в седьмую ее часть, будет суд на
вечность, который совершится над стражами, и явится великое небо,
произрастающее из среды ангелов.
И прежнее небо уменьшится и исчезнет, и явится новое небо, и все силы
небесные седмерицею будут светить вовек.
И после этого будет много седьмин без числа в вечность во благо и в
правду, и с тех пор грех не будет более именоваться до вечности.
И теперь я говорю вам, мои сыны, и указываю вам пути правды и пути
насилия, и я укажу вам их опять, чтобы вы знали, что придет.
И теперь послушайте, мои сыны, и ходите в путях правды, и не ходите по
путям насилия, ибо навеки погибнут все, ходящие путями неправды.
\vs 1En 19:1
Написанное Енохом писцом пространное учение мудрости,~---
которое заслуживает прославления от всех людей и есть судья всей земли,~---
для всех моих детей, которые будут жить на земле, и для будущих родов,
которые будут ходить в праведности и мире.
Да не смущается дух ваш из-за времен, ибо Святой и Великий всему
положил дни.
И праведный восстанет от сна, восстанет и пойдет по пути правды, и весь
его путь и стезя будут в вечном благе и милости для праведного, и даст
господство, и он будет жить во благе и правде, будет ходить в вечном свете.
И погибнет грех во мраке навсегда и навечно, и более уже не появится от
того дня до вечности.
И после этого Енох начал возвещать из книг.
И сказал Енох: "о детях правды, и об избранных мира, и о растении
справедливости и праведности, говорю я это вам, мои сыны,~--- я Енох,~--- согласно с
тем, что мне открыто в небесном видении, и что я знаю чрез слово святых ангелов
и что узнал из скрижалей небесных".
И Енох начал теперь повествовать из книг и сказал: "я родился седьмым
в первую седьмину, когда суд и правда еще медлили.
И после меня во вторую седьмину восстанет великая злоба и произрастет
обман, и во время нее будет первый конец, и во время ее спасется один муж; и
после того, как он (конец) совершится, возрастет неправда, и Он даст закон
грешникам.
И после этого в третью седьмину, в конце ее, будет избран в растение
праведного суда один муж, и после него явится растение правды навсегда и
навечно.
И после этого в четвертую седьмину, в конце ее, будут видимы видения
святых и праведных, и закон для всех будущих родов и двор будет сделан (дан)
им, И после этого в пятую седьмину, в конце ее, будет устроен дом славы и
господства навсегда и навечно.
И после этого в шестую седмину все, которые будут жить во время ее,
будут ослеплены, и все они погрузятся своею мыслью в неразумие, забыв мудрость;
и во время нее будет взят вверх один муж; и в конце его господства будет сожжен
огнем, и весь род избранного корня будет рассеян.
И после этого в седьмую седьмину восстанет отпадший (или развращенный)
род, и много будет деяний его, и все его деяния будут отпадением.
И в конце ее будут награждены избранные и праведные от вечного
растения правды, между тем как им будет дано седьмикратное наставление обо всем
Его творении.
Ибо есть ли где-нибудь сын человеческий, который услышал бы голос
Святого и не был бы потрясен?
И есть ли где-нибудь такой, кто мог бы мыслить его мысли?
И где есть такой, кто мог бы видеть все произведения неба?
И как мог бы существовать тот, кто узнал бы произведения неба, и
увидел бы Его дыхание, и Его дух, и подсказал бы о том, или вошел бы наверх и
увидел все концы (буквально~--- крылья) их (небес), и мог бы придумать их, или
сделать что подобное им?
И есть ли где-нибудь такой муж, который мог бы знать, какова широта и
длина земли, и кому открыта мера всего этого?
И найдется ли кто-нибудь, который мог бы знать длину неба, и какова
его высота, и на чем оно утверждено, и как велико число звезд, и где покоятся
все светила?
И теперь я говорю вам, мои сыны, любите правду и ходите в ней,
ибо пути правды достойны, чтобы принять их; а пути неправды исчезают внезапно и
погибают.
И некоторым людям из грядущих родов будут открыты пути насилия и
смерти, и они будут держать себя далеко от них, и не будут им следовать.
И теперь я говорю вам~--- праведным: ходите не по злому пути и не в
насилии, и не по путям смерти, и не приближайтесь к ним, чтобы вам не
погибнуть.
Но ищите и изберите себе правду и приятную для Бога жизнь, и ходите по
путям мира, чтобы выжили и имели радость.
И держите в мыслях вашего сердца и не допускайте, чтобы речь моя
искоренилась из вашего сердца, ибо я знаю, что грешники соблазнят людей~---
унижать мудрость, и она не приобретет нигде места, и искушения всякого рода не
уменьшатся.
Горе тем, которые созидают неправду и насилие, и полагают основание
обману; ибо они внезапно будут искоренены и не будут иметь мира.
Горе тем, которые строят свои дома грехом, ибо они будут искоренены до
основания и падут от меча; и приобретающие золото и серебро внезапно погибнут
на суде.
Горе вам, богатые, ибо вы положитесь на ваше богатство, и вы лишитесь
своего богатства, так как вы не думали о Всевышнем в дни своего богатства.
Вы творили хулу и неправду, и приготовили себя ко дню кровопролития, и
ко дню мрака, и ко дню великого суда.
Это я говорю вам, что вас истребит до основания Тот, Кто сотворил вас:
и не будет никакого сострадания к вашему падению; и ваш Творец будет радоваться
вашей погибели.
И ваши праведники в те дни будут служить поношением для грешников и
нечестивых.
О, если бы мои очи были водной тучей, чтобы плакать о вас, и
излить мои слезы как водную тучу, дабы я получил успокоение для своего сердца
от печали!
Кто позволил вам совершать ненависть и злобу?
Так пусть же постигнет вас, грешники, суд!
Не страшитесь грешников, вы~--- праведные, ибо Господь опять предаст их
в ваши руки, чтобы вы совершили над ними суд, как желаете.
Горе вам, изрекающим проклятие, чтобы проклинать неразрешимо; и ваше
исцеление должно быть далеко от вас вследствие ваших грехов!
Горе вам, воздающим своему ближнему злом, ибо вам будет уготовано по
вашим делам!
Горе вам лжесвидетелям и тем, которые показывают неправду, ибо вы
внезапно погибнете!
Горе вам, грешникам, так как вы преследуете праведных; ибо вы будете
преданы и преследуемы, вы~--- люди неправды, и тяжело будет на вас их (праведных)
ярмо.
Вы, праведные, надейтесь, ибо грешники внезапно погибнут пред
вами, и вы будете господствовать над ними, как желаете!
И в день страдания грешников ваши юнцы вознесутся и взлетят, как орлы,
и выше, чем у коршуна, будет ваше гнездо, вознесетесь; и как кролики вы
проникнете в ущелье земли и в расселины скал навсегда пред праведными; а они
будут воздыхать из-за вас и плакать, как лесные духи.
Но и вы не бойтесь,~--- вы страдающие, ибо для вас будет исцеление, и
будет светить вам блестящий свет, и призыв к покою вы услышите с неба.
Горе вам, вы~--- грешники, ибо ваше богатство позволяет вам казаться
праведным, но ваше сердце изобличает вас, что вы грешники, и эта речь будет
свидетельствовать против вас для напоминания о ваших злодеяниях.
Горе вам, которые едите тук пшеницы и пьете силу корня источника, и
попираете своею силою приниженных!
Горе вам, которые всегда пьете воду, ибо вам внезапно будет воздано, и
вы завянете и иссохнете, так как вы оставили источник жизни!
Горе вам, совершающим неправду, и обман, и хулу: это будет памятью
против вас к вашему злу!
Горе вам, сильные, поражающие своею силою праведного, ибо придет день
вашей погибели, в то время много хороших дней придет для праведных день~---
вашего суда.
Веруйте вы, праведные, ибо грешники будут позором для вас и
погибнут в день неправды.
Да будет вам (грешникам) известно, что Всевышний думает о вашей
погибели, и ангелы радуются вашей погибели.
Что будете вы делать, грешники, и куда убежите в тот день суда, когда
услышите голос молитвы праведных?
И для вас не будет того, что для них,~--- для вас, против которых будет
свидетельством это слово: "вы сделались союзниками грешников".
И в те дни молитва праведных проникнет к Господу, и для вас наступят
дни вашего суда.
И все ваши неправедные речи будут прочитаны пред Великим и Святым, и
ваше лицо пристыдится, и всякое дело, основанное на неправде, будет отринуто.
Горе вам, грешникам, в средине моря и на суше, воспоминание которых о вас
недоброе!
Горе вам, приобретающим себе серебро и золото не по правде и
говорящим: "мы сделались богатыми и имеем сокровища, и владеем всем, чего
хотим; и теперь мы исполним все то, что нам думается, ибо мы собрали серебра и
наполнили наши кладовые, и как воды много у нас оберегающих наши дома".
как вода, разольется ваша ложь, ибо богатство не сохранится у вас, но
внезапно будет у вас отнято, так как вы все приобрели неправдою, и вы сами
подвергнетесь великому осуждению.
И теперь я клянусь вам, мудрым и безумным; ибо вы много
переживете (или увидите) на земле.
Ибо вы, мужи, будете возлагать на себя украшений более, нежели жены, и
разноцветного более, чем дева, в царском достоинстве и величии и власти, и в
серебре, и в золоте, и в пурпуре, и в почести, и в пище они разольются, как
вода.
Посему им не достает учения и мудрости, и чрез то они погибнут вместе
со своими сокровищами, и со всею своею силою и почестью; и в позоре, и в
умертвлении, и в великой бедности их дух будет брошен в огненную печь.
Я клянусь вам, грешники: как гора не была и не будет рабой, ни
возвышенность служанкой жены, так точно и грех не был послан на землю, но люди
произвели его из своей головы; и великому осуждению подпадут те, которые
совершают его.
И неплодие не дано было жене, но ради дела своих рук она умирает без
детей, Я клянусь вам, грешники, Святым и Великим, что всякое злое дело ваше
открыто на небесах, и ни одно из ваших деяний насилия не утаено или прикрыто.
И не думайте в своем духе и не говорите в своем сердце,~--- вы не знаете
и не видите, что каждый грех записывается ежедневно на небе пред Всевышним.
Отныне вы знайте, что все ваше насилие, которое вы совершаете,
записывается каждый день до дня вашего суда, Горе вам, безумные, ибо вы
погибнете чрез ваше безумие; и так как вы не слушаетесь мудрых, то ничто доброе
не будет вашим уделом.
И теперь знайте, что вы приготовлены на день погибели, и не надейтесь,
что вы будете жить,~--- вы грешники,~--- но вы погибнете и умрете, так как вы не
знаете никакого выкупа: ибо вы приготовлены на день великого суда, и на день
страдания и великого позора для вашего духа.
Горе вам,~--- вы ожесточенные, которые делаете зло и едите кровь!
Откуда у вас хорошая пища, и питье, и насыщение?
От всякого блага, которое наш Господь, Всевышний в изобилии послал на
землю: посему вы не должны иметь мира.
Горе вам, любящим свои деяния неправды!
Почему вы чаете блага для себя?
Знайте, что вы будете преданы в руки праведных; они перережут ваши шеи
и умертвят вас, и не будут иметь сострадания к вам.
Горе вам, радующимся страданию праведных, ибо для вас не будет вырыта
могила!
Горе вам, для которых слова праведных только пустые речи, ибо для вас
не будет надежды на жизнь!
Горе вам, записывающим лживые речи и беззаконные слова; ибо они
записывают свою ложь, чтобы их слушали и не забывали их безумия; так не будет
же для них мира, но они умрут внезапной смертью!
Горе тем, которые совершают нечестие, и похваляют и сохраняют в
уважении лживые речи: вы погибнете чрез это и для вас нет хорошей жизни!
Горе вам, искажающим слова праведности!
И они отпадут от вечного закона и сами себя делают тем, чем не были,
именно~--- грешниками; они будут попираемы на земле.
В те дни вы, праведные, приготовьтесь вознести свои мысленные молитвы,
вы представите их как свидетельство ангелам, чтобы они представили беззакония
грешников Всевышнему в напоминание.
В те дни народы придут в смятение, и поколения народов восстанут ко
дню погибели.
И в те дни выйдет плод материного чрева, и они (матери) растерзают
своих собственных детей; они оттолкнут от себя своих детей, и у них выпадет
недоношенный плод; грудных детей они оттолкнут от себя, и не возвратятся опять
к ним, и не сжалятся над своими любимцами.
Опять клянусь вам, грешники, что грех уготован на день беспрерывного
кровопролития.
И они будут поклоняться камням, и другие будут делать изображения из
золота и серебра, и из дерева и глины; и другие будут поклоняться нечистым
духам, и демонам, и разного рода идолам в идольских капищах: между тем у них
(идолов) нельзя найти никакой помощи.
И они погрузятся в неведение вследствие безумия своего сердца, и их
очи будут ослеплены страхом их сердца и сновидениями.
Чрез них они впадут в неведенье и страх, ибо они все свои дела
совершают во лжи, и поклоняются камням; и они погибнут все разом.
Но в те дни блаженны все те, которые принимают слова мудрости, и знают
ее, и исполняют пути Всевышнего, и ходят по пути правды, и с безбожными: ибо
они будут спасены.
Горе вам, распространяющим зло между своими ближними, ибо вы будете
умерщвленны в геенне.
Горе вам, полагающим основание греху и лжи, и вызывающим ожесточение
на земле: ибо за это их постигнет конец.
Горе вам, которые строите свои дома потом других и у которых
строительный материал есть не что иное, как черепица и камень греха; я говорю
вам, что для вас нет мира.
Горе тем, которые отвергают меру и наследие своих отцов, пребывающее
вечно, и прилепляют свои души к идолам: ибо для них не будет покоя.
Горе тем, которые делают неправду, и помогают насилию, и умерщвляют
своих ближних, в день великого суда: ибо Он низринет вашу славу, и положит вам
злобу на сердце, и возбудит дух Своего гнева, чтобы погубить вас всех мечом; и
все праведные и святые припомнят ваши грехи.
И в те дни будут умерщвлены в одном месте отцы вместе со своими
сынами, и братья друг с другом упадут от смерти, пока их кровь не потечет
подобно потоку.
Ибо муж не будет из сострадания удерживать свою руку от своих сынов и
от своих внуков, убивая их; и грешник не будет сдерживать своей руки от своего
почетнейшего брата; от утренней зари до солнечного захода они будут умерщвлять
друг друга.
И конь будет по самую грудь ходить в крови грешников, и колесница
потонет до своего верха.
И в те дни ангелы сойдут в убежища грешников и соберут в одно место
всех тех, которые помогали греху; и Всевышний восстанет в тот день, чтобы
произвести великий суд над всеми грешниками.
Но над всеми праведными и святыми Он поставит стражами святых ангелов,
чтобы они берегли их, как зеницу ока, пока не наступит конец всякой злобе и
всякого греха; и если даже праведные спят продолжительным сном, то и тогда они
не должны ничего бояться.
И кто мудр между людьми, тот увидит истину, и дети земли поймут все
слова этой книги, и узнают, что их богатство не может спасти их при погибели их
греха.
Горе вам, грешники, если вы мучите праведных,~--- в день жестокого
страдания,~--- и сжигаете их огнем: вам будет воздано по вашим делам.
Горе вам, развращенные сердцем, заботящиеся о том, чтобы измышлять
злое; на вас нападет страх, и никто не поможет вам.
Горе вам, грешники, ибо вы будете гореть в озере огненного пламени за
слова своих уст и за дела своих рук, которыми вы действуете нечестиво.
И теперь знайте, что ангелы на небе будут выведывать о ваших делах у
солнца, и луны, и звезд,~--- выведывать о ваших греховных делах, ибо вы
совершаете на земле суд над праведными.
И Он сделает свидетелями против вас каждую тучу, и облако, и росу, и
дождь, ибо все они задерживаются вами, чтобы не ходить на вас; и не должны ли
они думать о ваших грехах?
И теперь дайте дары дождю, чтобы он не был задержан от снисхождения
на вас, а также не была бы задержана роса, если она получила от вас золото и
серебро.
Когда будут падать на вас иней и снег вместе с их холодом, и все
снежные ветры со всеми своими бедствиями, то вы не устоите в те дни против них.
Рассмотрите небо, все дети земли, и каждое произведение
Всевышнего, и устрашайтесь пред Ним, и не делайте пред Ним ничего злого!
Если бы Он закрыл окна небесные и задержал из-за вас дождь и росу,
чтобы они не падали на землю, то что вы стали бы тогда делать?
И если Он посылает Свой гнев на вас и на все ваши произведения, то
можете ли вы не поклоняться Ему, так как вы высказываете надменные и бесстыдные
речи против Его правды, и для вас не будет мира.
И не видите ли вы управителей кораблей, как их корабли бросаются
волнами, качаются ветрами, и подвергаются опасности; и они вследствие этого
впадают в страх, так как они взяли с собою в море самое лучшее из своего
имения, и они беспокоятся в своем сердце, как бы море не поглотило их и как бы
они не погибли в нем?
Все море, и все его воды, и все его движение~--- не есть ли творение
Всевышнего, и не запечатал ли Он все Свое дело и не заключил ли его совсем в
песок?
Оно засыхает от Его угроз и устрашается, и все его рабы и все, что
есть в нем, умирают: и вы, грешники, живущие на земле, не боитесь Его.
Не сотворил ли Он небо, и землю, и все, что есть на них?
И кто дал учение и мудрость всем, которые движутся на земле и которые
живут в море?
Не боятся ли моря все цари кораблей?
А грешники не боятся Всевышнего.
В те дни, когда Он пошлет на вас мучительный огонь, куда вы
убежите, и где спасетесь?
И когда Он пошлет на вас Свое слово, не будете ли вы поражены тогда и
не устрашитесь ли?
Все светила потрясутся тогда от великого страха, и вся земля будет
поражена, и она задрожит и устрашится.
И все ангелы выполнят данные им повеления и будут стараться укрыться
пред Тем, Кто велик во славе, и дети земли задрожат и затрепещут; и вы, о
грешники, будете прокляты навеки, и пусть не будет для вас мира!
--- Не страшитесь вы, души праведных, и уповайте на день своей смерти в
правде!
И не печальтесь, что ваша душа нисходит в царство мертвых в великой
скорби, в горе, и воздыхании, и печали, и что ваше тело не обрело в вашей жизни
того, чего заслужила ваша благость, скорее теперь в день, когда вы стали
одинаковыми с грешниками, и в день проклятия и осуждения.
И когда вы умираете, грешники говорят над вами: "праведники умирают,
как и мы, и какая для них польза от их дел?
Вот они, как и мы, умерли в печали и мраке, и какое преимущество они
имеют пред нами?
отныне мы одинаковы.
И чего они достигнут этим, и что они увидят в вечности?
Ибо вот они также умерли и отныне не увидят света до века".
Я говорю вам, грешники: для вас достаточно есть, и пить, и обнажать
человека, и расхищать, и согрешать, и приобретать силу, и видеть хорошие дни.
Видели ли вы праведных, как конец их был мирен, ибо никакого рода
насилия не было в них по день их смерти, "И они погибли, как бы и не
существовали, и их души в печали сошли в царство мертвых".
И теперь я клянусь вам праведным Его великою славою и честью, и
Его достохвальным царством, и Его владычеством я клянусь вам: я знаю эту тайну
и прочитал ее на небесных скрижалях, и видел книгу святых, и нашел написанное и
отмеченное в ней относительно них, что для них уготовано всякое благо, и
радость и почесть; и я нашел записанное относительно духов тех, которые умерли
в правде; и узнали, что вам будет воздано многими благами за ваши труды, и ваша
участь лучше, чем участь живущих.
И будут жить ваши духи,~--- вы, умершие в правде; и будут радоваться и
ликовать их духи, и память о них будет пред лицом Великого на все роды мира:
так не страшитесь же их поношения!
Горе вам, грешники, когда вы умираете в своих грехах и подобные вам
говорят о вас: "блаженны грешники, они видели все свои дни; и теперь они умерли
в счастье и в богатстве, и не видели в своей жизни ни горести, ни убийства; в
славе они умерли, и во время их жизни суд не совершился над ними".
Но знаете ли вы, что души их должны сойти в царство мертвых, и они
найдут его невыносимым, и велика будет печаль их?
И во время великого суда ваш дух сойдет во мраке, и в сети, и в
плавающее пламя, и великий суд будет для всех родов до века: горе вам, ибо для
вас нет мира!
Не говорите праведным и добрым, которые еще живут: "в дни нашего
бедствия мы трудились, и побеждали всякую нужду, и встречались со всякими
бедствиями; мы не могли ничего сделать против врагов ни словом, ни делом, и
совершенно ничего не достигли; мы мучились и погибали, и не могли надеяться
видеть жизнь день за днем.
Мы надеялись быть главою, а сделались хвостом; мы измучились в
работах и не получили плодов своего труда, мы сделались пищею для грешников,
неправедные сделали для нас тяжким свое ярмо.
Владыками над нами были те, которые ненавидели нас и били нас: и мы
должны были склонять свои головы пред ненавидящими нас, и они не имели
сострадания к нам.
Мы старались ускользнуть от них, чтобы убежать и получить успокоение,
но мы не находили, куда бежать нам и спастись от них.
Мы жаловались на них в своей горести властителям, и сетовали на тех,
которые поедали нас; но они не взирали на наш вопль и не хотели слышать нашего
голоса.
И они помогали тем, которые обкрадывали нас и поедали, и тем, которые
принижали нас; и они утаивали их притеснения, так что не снимали с нас их ярма,
но поедали нас, и прогоняли, и убивали: и они утаивали умерщвление нас, и не
думали о том, что они подняли свои руки против нас".
Я клянусь вам, праведные, что ангелы на небе напоминают о вас
пред славою Великого к вашему благу, и ваши имена записаны пред славою
Великого.
Надейтесь вы, праведные, ибо прежде вы были в позоре, и несчастии, и
бедствии, а теперь вы будете светить, как светила небесные, и будете видимы, и
врата небесные отверзнутся для вас.
И ваш вопль о суде продолжается: он откроется для вас, ибо
властителям отомстится за ваше страдание, и всем помощникам тех, которые
обкрадывали вас.
Надейтесь и не покидайте свои надежды: ибо вы будете иметь великую
радость, как ангелы небесные, Так как вам предстоит таковое, то вы не будете
скрываться в день великого суда, и не будете найдены подобными грешникам, и от
вас далеко будет вечное осуждение, на все роды мира.
И теперь вы не бойтесь, праведные, когда видите грешников
усиливающимися и услаждающимися в своем веселие, и не имейте никакого общения с
ними, но держитесь в отдалении, ибо вы должны быть союзниками небесных воинств.
Вы грешники, хотя и говорите: "вы не можете разузнать этого и наши
грехи не записаны все", однако же они (ангелы) каждый день записывают ваши
грехи.
И теперь я открываю вам, что свет и мрак, день и ночь видят все ваши
грехи.
Не будьте нечестивыми в своем сердце, и не лгите, не изменяйте слов
праведности (или истины), и не выдавайте за ложь слов Святого и Великого, и не
прославляйте своих идолов; ибо вся ваша ложь и ваше нечестие служит не к
правде, а к великому греху.
И теперь я знаю эту тайну, что многие грешники изменят слова
праведности (или истины) и отпадут от них, и будут говорить двойные речи, и
говорить ложь, и творить великие (греховные) дела, и писать книги о своих
речах.
Но когда они все мои слова пишут правильно на своих языках, и ничего
не изменяют и не пропускают из моих слов, но все пишут правильно,~--- все, что я
прежде утверждал относительно них; то я знаю другую тайну, что именно только
праведным и мудрым даны книги к радости, и к праведности, и к великой мудрости,
и им даны книги, и они уверуют в них и возрадуются о них; и получат награду все
праведные, научившиеся из них знать все пути праведности.
"И в те дни, говорит Господь, они (праведные) должны воззвать к
сынам земли и представить свидетельство относительно мудрости их (книг);
покажите им их,~--- ибо вы их вожди,~--- и награды для всей земли.
Ибо Я и Мой Сын соединимся с ними навсегда и навечно на путях
праведности в их жизни.
И мир будет с вами: радуйтесь, вы~--- дети праведности, воистину"!
\vs 1En 20:1
И после некоторого времени мой сын Мафусаил взял своему сыну
Лемеху жену, и она зачала от него и родила сына.
Тело его было бело, как снег, и красно, как роза, и его волосы головные
и темянные были, как волна (руно), и его глаза были прекрасны; и когда он
открыл свои глаза, то они осветили весь дом подобно солнцу, так что весь дом
сделался необычайно светлым.
И как только он был взят из руки повивальной бабки, то открыл свои уста
и начал говорить к Господу правды.
И его отец Ламех устрашился этого, и удалился, и пришел к своему отцу
Мафусаилу.
И он сказал ему: "я родил необыкновенного сына; он не как человек, а
похож на детей небесных ангелов, ибо он родился иначе, нежели мы: его глаза
подобны лучам солнца и его лицо блестящее.
И мне кажется, что он происходит не от меня, а от ангелов; и я боюсь,
как бы в его дни не произошло на земле чудо.
И теперь, мой отец, я здесь с неотступною просьбою к тебе о том, чтобы
ты отправился к нашему отцу Еноху и выведал от него истину, ибо он имеет свое
жилище возле ангелов".
И когда Мафусаил слушал речь своего сына, то пришел ко мне к пределам
земли,~--- ибо он слышал, что я там,~--- и воскликнул; и я услышал его голос, и
пришел к нему, и сказал ему: "вот я здесь, мой сын, ибо ты пришел ко мне".
И он отвечал мне и сказал: "ради важного дела я пришел к тебе, и из-за
тревожного случая я приблизился сюда.
И теперь, отец мой, выслушай меня: у моего сына Ламеха родился сын,
образ и вид которого не как вид человека; его цвет белее, нежели снег, и
краснее розы, и его головные волосы белее, чем белое руно, и его глаза, как
лучи солнца; и он открыл свои глаза, и вот они осветили весь дом.
И взятый из руки повивальной бабки он открыл свои уста и прославил
Господа неба.
Тогда устрашился отец его Ламех и прибежал ко мне; и он не верит, что
он произошел от него, но что будто он подобие ангелов небесных; и вот я пришел
к тебе, чтобы ты открыл мне истину".
И я, Енох, отвечал и сказал ему: "Господь совершит на земле новое, и
это я знаю, и я видел в видении, и открыл тебе, что в век моего отца Иареда
некоторые ангелы, сошедшие с высоты неба преступили слово Господне.
И вот они совершили грех, и преступили закон, и соединились с женами,
и совершили с ними грех, и взяли жен из них, и родили с ними детей.
И великая погибель придет на всю землю, придет потоп, и будет великая
погибель в продолжение года.
Этот сын, родившийся у вас, останется на земле, и три его сына
спасутся вместе с ним; когда все люди живущие на земле, умрут, он и его сыновья
спасутся.
[Они рождают на земле исполинов не по духу, а по плоти, и за это
придет великое наказание на землю, земля будет вполне омыта от всей нечистоты].
И теперь извести сына своего Ламеха, что родившийся есть действительно
его сын, и нареки ему имя Ной, ибо он будет для вас остатком; и он и его
сыновья спасутся от уничтожения, которое придет на землю за все грехи и за
всякую неправду, которые совершаются на земле в его дни.
И после того неправда будет еще гораздо больше, чем та, которая
совершалась на земле прежде, ибо я знаю тайны святых,
так как Он~--- Господь~--- дозволил мне видеть их и открыл их мне, и я почитал их на скрижалях небесных.
И я видел написанное на них, что род за родом будет
беззаконовать, пока не восстанет род правды, и беззаконие будет обречено на
погибель, и грех исчезнет с земли, и все доброе появится на ней.
И теперь, мой сын, иди и возвести своему сыну Ламеху, что этот
родившийся сын есть действительно его сын, и это не ложь".
И когда Мафусаил выслушал речь своего отца Еноха,~--- ибо все тайные
вещи он открыл ему,~--- то возвратился, увидевшись с ним (Енохом), назад, и нарёк
тому сыну имя Ной, ибо он утешит землю в вознаграждение за всю погибель.
Другое писание, которое Енох написал для своего сына Мафусаила и
для всех, которые придут после него и будут сохранять закон в последние дни.
Вы, исполнившие его и теперь ожидающие, как в те дни совершится конец
над теми, которые делают злое, и сила беззаконников окончится,~--- вы ожидайте
только, когда минует грех, ибо имя их (грешников) будет изглажено из книг
святых и семя их погибнет навсегда и навечно, и их духи будут умерщвлены, и они
будут восклицать и взывать в пустом необитаемом месте и гореть в огне, где нет
земли.
И я видел там нечто похожее на облако, чего нельзя было узнать, ибо
вследствие глубины его (этого места) я не мог взглянуть на него; и я увидел там
ярко~--- пылающее пламя огня, и там кружились предметы, как блестящие горы, и
двигались туда и сюда.
И я спросил одного из святых ангелов, бывших при мне, и сказал ему:
"что это такое блестящее?
ибо это не небо, а только пламя пылающего огня и звуки вопля, и плача,
и сетования, и жестокого страдания".
28 И он сказал мне: "в это место, которое ты видишь,~--- сюда приносятся
духи грешников, и хулителей, и тех, которые делают злое и изменяют всё, что Бог
сказал устами пророков о будущем.
Ибо об этом есть писания и начертания вверху на небе, чтобы ангелы
читали их и знали, что случится с грешниками и духами покорных и тех, которые
умерщвляли свою плоть и за это получили от Бога награду, и тех, которые были
обесчещены злыми людьми; которые любили Бога, не любили ни золота, ни серебра,
ни всех благ мира, но предавали свое тело мучению; и которые, со времени своего
бытия, домогались не земных явлений, а считали самих себя за преходящее дыхание
и сообразно с этим жили, и были многократно испытываемы Господом, но их души
были обретены в чистоте, чтобы прославить Его имя.
Все благословения, которые они получают, я представил в книгах; и Он
назначал им за это награду, ибо они обрелись возлюбившими более вечное небо,
чем свою жизнь, и в то время, как были попираемы злыми людьми, и должны были
выслушывать от них оскорбления и хуления, и были обесчещиваемы, они прославили
Меня".
И теперь Я призову духов добрых людей из поколения света, и произведу
перемену с теми, которые родились во тьме и которые в своей плоти не были
награждены почестью, как надлежало за их верность.
И Я введу в блистающий свет любивших Мое святое имя, и посажу каждого
из них отдельно на престоле почести,~--- его почести.
И они будут блистать в продолжение бесчисленных времен, ибо
справедливость есть суд Божий и верным Он даст верность в жилище праведных
путей.
И они (праведные) увидят, как родившиеся во тьме будут брошены во
тьму, между тем как праведные будут блистать.
И грешники воскликнут и увидят, как они блистают: и они также пойдут
туда, где им написаны дни и времена.

\bibbookdescr{2En}{
  inline={Вторая книга Еноха},
  toc={2-я Еноха},
  bookmark={2-я Еноха},
  header={2-я Еноха},
  abbr={2~Ено}
}
\vs 2En 1:1
Мужа мудрого, великого книжника, которого взял Господь, дабы он увидел и возлюбил высшее житие, непреходящее царство премудрого и великого Бога Вседержителя,
\vs 2En 1:2
дабы стал он свидетелем превеликого, многоочитого и непоколебимого престола Господа, пресветлого предстояния слуг Господа и степеней их господства,
\vs 2En 1:3
геенны огненной, неисчислимого состава воинства небесного, многого множества стихий и различных видений, несказанного пения воинства Херувов, и света безмерного.
\vs 2En 1:4
Когда мне исполнилось сто шестьдесят пять лет, сказал Енох, у меня родился сын Мафусаил. Затем я прожил еще двести лет. А вся моя жизнь продолжалась триста шестьдесят пять лет.
\vs 2En 1:5
В первый месяц, в известный день первого месяца я, Енох, был в доме своем один.
\vs 2En 1:6
И когда лежал я на ложе своем и спал, обильная скорбь охватила сердце мое, и я сказал: Вот, очи мои испускают слезы (ибо во сне я не мог понять, что означает сия скорбь). Что будет со мною?"
\vs 2En 1:7
И явились мне два мужа столь великих, каких никогда не видел я на земле: лица их сияли подобно солнцу, а очи их были словно свечи горящие,
\vs 2En 1:8
из уст их исходил как бы огонь, и одеяния их были как струящаяся пена, светлее золота крылья их, и руки их белее снега.
\vs 2En 1:9
И стали они у изголовья моего и позвали меня по имени.
\vs 2En 1:10
И пробудился я от сна моего, и вот, мужи те стоят предо мною наяву.
\vs 2En 1:11
И встал я поспешно и поклонился им, и вспыхнуло лицо мое от страха перед увиденным.
\vs 2En 1:12
И сказали мне мужи те: Ободрись, Енох, не бойся, Господь вечный послал нас к тебе, в день сей восходишь ты с нами на небо.
\vs 2En 1:13
Скажи же сынам своим все, что нужно им сделать на земле; и пусть никто из дома твоего не ищет тебя до тех пор, пока не возвратит тебя к ним Господь".

\vs 2En 2:1
И послушался я их, и пошел, и призвал сыновей своих Мафуселу и Ригима, и поведал им то, что сказали мне мужи те:
\vs 2En 2:2
И вот я знаю, дети, что я не знаю, куда иду и что встретит меня.
\vs 2En 2:3
Вы же, дети мои, не отступайте от Бога, и пред лицем Господним ходите, и соблюдайте суды Его,
\vs 2En 2:4
не отвергайте жертв спасения вашего и не отвергнет Господь труд рук ваших;
\vs 2En 2:5
не лишайте даров Господа и не лишит Господь приращений Своих в хранилищах ваших!
\vs 2En 2:6
Благословляйте Господа первенцами от стад скота вашего и будете благословенны перед Господом во веки.
\vs 2En 2:7
Не отступайте от Господа и не поклоняйтесь богам суетным, не сотворившим ни небес, ни земли;
\vs 2En 2:8
и они погибнут, и те, что поклонятся им.
\vs 2En 2:9
Да утвердит Господь сердца ваши в страхе своем!
\vs 2En 2:10
И ныне, дети мои, пусть никто не ищет меня, доколе Господь не возвратит меня к вам.

\vs 2En 3:1
И было, когда говорил я сыновьям своим, позвали меня мужи те и взяли на крылья свои.
\vs 2En 3:2
И вознесли меня на первое небо, и поставили меня там.
\vs 2En 3:3
И привели пред лице мое верховных владык чинов звездных, и показали мне путь и движение их от года до года.
\vs 2En 3:4
И показали мне двести ангелов, которые управляют звездами и составом небес.
\vs 2En 3:5
И показали мне там море огромное, большее моря земного.
\vs 2En 3:6
И вокруг ангелы летали на крыльях своих.
\vs 2En 3:7
И показали мне хранилища снега и льда и грозных ангелов, стражей хранилищ тех.
\vs 2En 3:8
И показали мне там хранилища облаков, откуда они выходят и куда входят.
\vs 2En 3:9
И показали мне хранилища росы, подобной елею масличному; и ангелов, стерегущих сокровища те, и вид их как все цветы земные.

\vs 2En 4:1
И взяли меня мужи те, и поставили меня на втором небе, и показали мне узников, соблюдаемых для суда безмерного.
\vs 2En 4:2
И там видел я ангелов осужденных, плачущих, и спросил я мужей, которые были со мною: За что они мучимы?
\vs 2En 4:3
И отвечали мне мужи те: Это отступники от Господа, не послушавшиеся гласа Господня, но своею волею державшие совет.
\vs 2En 4:4
И опечалился я о них. И ангелы те поклонились мне, и сказали: Муж Божий, помолись бы о нас ко Господу.
\vs 2En 4:5
И я отвечал им, и сказал: Кто я, человек смертный, чтобы молиться об ангелах; кто знает, куда иду или что встретит меня, или кто помолится обо мне?

\vs 2En 5:1
И взяли меня оттуда мужи те, и возвели на третье небо, и поставили меня посреди рая.
\vs 2En 5:2
И место то невыразимо красотою вида его: всякое дерево цветами украшено, и всякий плод зрел, и всякие яства вечно изобилуют, всякое дуновение благовонно.
\vs 2En 5:3
И четыре реки протекают там покойным течением.
\vs 2En 5:4
И всякий злак, который рождается в пищу, прекрасен.
\vs 2En 5:5
И древо жизни на месте том, и на нем почивает Господь, когда входит Господь в рай, и древо то несказанно прекрасно благоуханием.
\vs 2En 5:6
И рядом другое древо масличное, постоянно источающее елей.
\vs 2En 5:7
И всякое дерево благоплодно, и нет там дерева безплодного; и все место то благовонно.
\vs 2En 5:8
И Ангелы, охраняющие рай, светлы весьма, непрестанным гласом сладкопения своего служат Богу во все дни.
\vs 2En 5:9
И сказал я: Сколь благо это место весьма!
\vs 2En 5:10
Отвечали мне мужи те: Место это, Енох, уготовано праведникам, которые претерпят напасти в этой жизни, и душам которых причинят зло, и которые отвратят очи свои от неправды и сотворят суд праведный
\vs 2En 5:11
чтобы дать хлеб алчущим и покрыть нагого одеждой, и поднять падшего, и помочь обиженным;
\vs 2En 5:12
которые пред лицем Господа ходят и Ему одному служат, тем уготовано сие в наследие вечное.

\vs 2En 6:1
И взяли меня оттуда мужи те, и вознесли меня на север неба, и показали мне там место весьма страшное:
\vs 2En 6:2
всякое томление и мучение на месте том, и тьма, и мгла, и нет там света, но огонь мрачный разгорается всегда на месте том, и река огненная растекается на все места те;
\vs 2En 6:3
лед холодный, и темницы, и ангелы лютые и неистовые, носящие оружие и мучающие без милости.
\vs 2En 6:4
И сказал я: Как страшно место это весьма!
\vs 2En 6:5
И отвечали мне мужи те: Это место, Енох, уготовано нечестивым, творящим безбожное на земле,
\vs 2En 6:6
тем, которые творят колдовство и волхвование, и похваляются делами своими, и тайно крадут души, и разрешают бремя связанное,
\vs 2En 6:7
тем, которые богатеют в ущерб имуществу чужому, и умерщвляют голодом алчущего, дабы самим насытиться; и имея возможность одеть нагих, раздевают;
\vs 2En 6:8
тем, которые не познали Творца своего, но поклонялись богам суетным, создавая идолов и поклоняясь творению рук своих.
\vs 2En 6:9
И всем тем уготовано это место в удел вечный.

\vs 2En 7:1
И взяли меня оттуда мужи те и подняли на четвертое небо, и показали мне там все движение солнца и луны и все лучи их.
\vs 2En 7:2
И измерил я путь их, и рассчитал свет их,
\vs 2En 7:3
и видел я: солнце имеет свет, в семь раз больший луны.
\vs 2En 7:4
И видел я круг их и колесницы, на которых ездит каждый из них, как ходит ветер,
\vs 2En 7:5
и нет им покоя, день и ночь ходящим и возвращающимся.
\vs 2En 7:6
И я видел четыре звезды великих, висящих справа от колесницы солнца, и четыре слева от солнца всегда.
\vs 2En 7:7
И ангелы движутся перед колесницей солнечной, духи летающие;
\vs 2En 7:8
двенадцать крыльев у каждого ангела, что мчат колесницу солнца, неся росу и зной, когда повелит им Господь сойти на землю с лучами солнечными.

\vs 2En 8:1
И отнесли меня мужи те на восток неба, и показали мне врата, из которых выходит солнце в положенные времена и по обращениям луны всего года,
\vs 2En 8:2
и при убавлении, и при возрастании дня, сообразно уменьшению и возрастанию дня и ночи:
\vs 2En 8:3
шесть ворот одинаковых, отверстых в тридцать одну стадию ровно, и я измерил величину их, и не мог постичь величины их.
\vs 2En 8:4
И те врата ими восходит солнце и идет на запад:
\vs 2En 8:5
первыми вратами выходит оно сорок два дня, вторыми тридцать пять дней,
\vs 2En 8:6
третьими тридцать пять дней, четвертыми тридцать пять дней,
\vs 2En 8:7
пятыми тридцать пять дней, шестыми сорок два дня,
\vs 2En 8:8
и снова возвращается шестыми вратами по истечению срока своего.
\vs 2En 8:9
И оно входит пятыми воротами тридцать пять дней, четвертыми воротами тридцать пять дней,
\vs 2En 8:10
третьими воротами тридцать пять дней, вторыми тридцать пять дней,
\vs 2En 8:11
и заканчиваются дни года по обращению времен.

\vs 2En 9:1
И возвели меня мужи те на запад неба, и показали мне там шесть великих ворот отверстых, поставленных против ворот восточных.
\vs 2En 9:2
И ими заходит солнце по обращении по небу из восточных ворот: по восходу из восточных ворот и по числу дней так же заходит в западные ворота.
\vs 2En 9:3
И когда изойдет оно из западных ворот, берут четыре ангела венец его и возносят ко Господу,
\vs 2En 9:4
а солнце поворачивает колесницу свою и идет без света, и там возлагают на него венец.
\vs 2En 9:5
И вот, показали они мне порядок солнца и ворот, которыми оно восходит и заходит, ибо эти ворота сотворил Господь. И солнце смену времен года указывает.
\vs 2En 9:7
А лунный порядок другой. И показали они мне весь путь ее и все обращения ее,
\vs 2En 9:8
и показали мне мужи те врата, и показали мне двенадцать ворот на востоке и двенадцать ворот на западе,
\vs 2En 9:9
и круги, по которым восходит и заходит луна по установленному времени:
\vs 2En 9:10
первыми воротами на востоке тридцать один день точно, вторыми тридцать пять дней точно,
\vs 2En 9:11
третьими тридцать один день ровно, четвертыми тридцать дней точно,
\vs 2En 9:12
пятыми тридцать один день особо, шестыми тридцать один день точно,
\vs 2En 9:13
седьмыми тридцать дней точно, восьмыми тридцать один день особо,
\vs 2En 9:14
девятыми тридцать один день определенно, десятыми тридцать дней точно,
\vs 2En 9:15
одиннадцатыми тридцать один день ровно, двенадцатыми воротами входит двадцать два дня точно.
\vs 2En 9:16
Также и западными воротамим по обращению и по числу восточных ворот.
\vs 2En 9:17
Так входит она и западными воротами и совершает год в триста шестьдесят пять дней \ldots
\vs 2En 9:18
Когда заканчиваются западные ворота, возвращается и идет к восточным со светом своим.
\vs 2En 9:19
И так ходит кругом день и ночь, и круг ее подобен небу.
\vs 2En 9:20
И колесницу, на которую она восходит, влечет ветер.
\vs 2En 9:21
И движется колесница ее летящими духами, у каждого из ангелов по шесть крыльев. Таков порядок лунный.
\vs 2En 9:22
И видел я посреди неба воинов вооруженных, служащих Богу непрестанным гласом тимпанов и органов, и я наслаждался, слушая их.

\vs 2En 10:1
И взяли меня оттуда мужи те, и вознесли меня на пятое небо.
\vs 2En 10:2
И я видел там множество Бодрствующих, видел я двести.
\vs 2En 10:3
Видом своим как люди, величиной же больше чудес великих, лица их печальны, уста их молчат, и не было там служения.
\vs 2En 10:4
И сказал я у мужам, бывшим со мною: Почему столь печальны и унылы лица их, и уста их молчат, и нет службы на небе этом?
\vs 2En 10:5
И отвечали мне мужи те: Это Бодрствующие, которые отпали от Господа: двое князей и двести ходящих вслед князей тех;
\vs 2En 10:6
и сошли они на землю и исполнили обет свой на хребте горы Ермон, чтобы оскверниться с женами человеческими.
\vs 2En 10:7
И за осквернение то осудил их Господь, и вот, рыдают они о братии своей, и о бывшей укоризне.
\vs 2En 10:8
И сказал я Бодрствующим: Я видел братию вашу и дела их познал, и вот мольбу их видел, и молился о них.
\vs 2En 10:9
И вот осудил их Господь под землю, доколе не придет конец неба и земли.
\vs 2En 10:10
Зачем же ждете вы братьев своих и не служите пред лицем Господа?
\vs 2En 10:11
Восстановите прежнее служение, служите во имя Господне! Ведь если разгневаете Господа, Бога вашего, свергнет Он вас с места этого.
\vs 2En 10:12
И вняли они увещанию моему и стали на небе по четырем чинам;
\vs 2En 10:13
и пока я стоял там, вострубили одновременно в четыре трубы и стали служить Бодрствующие, и поднялся голос их единым гласом к лицу Господа.

\vs 2En 11:1
И подняли меня оттуда мужи те, и вознесли меня на шестое небо.
\vs 2En 11:2
И увидел я там семь ангелов, стоящих вместе, светлых и славных весьма: лица их как лучи солнечные блистают, и нет различия ни в лицах, ни во власти, ни в содержании власти их.
\vs 2En 11:3
Они устрояют и преподают благой порядок миру: движению звезд, солнца и луны, и ангелов, возящих их, и небесным гласам, и умиротворяют все бытие небес;
\vs 2En 11:4
и устрояют заповеди и поучения, и сладкогласное пение, и всякую хвалу и славу.
\vs 2En 11:5
И есть ангелы над временами и годами, и ангелы над реками и морями, и ангелы над плодами и травою, и над всем прозябающим;
\vs 2En 11:6
и ангелы всех народов, управляющие всею жизнью их и записывающие ее перед лицом Господа.
\vs 2En 11:7
И среди них семь Фениксов, и семь Херувов и семь Серафов, единогласных голосами своими и пением своим, и неизъяснимо пение их.
\vs 2En 11:8
И радуется Господь подножию Своему.

\vs 2En 12:1
И подняли меня оттуда мужи те и вознесли меня на седьмое небо.
\vs 2En 12:2
И увидел я свет великий и все огненное воинство безплотных: архангелов, ангелов, и светозарное стояние Офанов.
\vs 2En 12:3
И я устрашился, и вострепетал, и взяли меня иужи те, и поставили среди них, говоря мне: Ободрись, Енох, не бойся!
\vs 2En 12:4
И показали мне издалека Господа, сидящего на престоле Своем, и все воинства небесные, соединенные по чину, приступали и кланялись Господу,
\vs 2En 12:5
и снова отходили, и шли на места свои в радости и веселии, и в свете безмерном.
\vs 2En 12:6
И Славные служат ему неотступно день и ночь, стоя перед лицем Господа и творя волю Его.
\vs 2En 12:7
И все воинство Херувов вокруг престола Его неотступно, и Серафы покрывают престол Его, воспевая пред лицем Господа.
\vs 2En 12:8
И когда я увидел все это, то отошли от меня мужи те, и больше я не видел их.
\vs 2En 12:9
И они поставили меня одного на краю неба, и испугался я, и пал на лице свое.
\vs 2En 12:10
И послал Господь одного из Славных своих ко мне, Гавриила, и он сказал мне. Дерзай, Енох, не бойся! Встань и пойди со мной, и стань перед лицем Господним во веки.
\vs 2En 12:11
Я же отвечал ему, говоря: Увы мне, господин, душа моя отступила из меня от страха.
\vs 2En 12:12
Позови ко мне мужей, приведших меня на место сие, ибо к ним я имею доверие и с ними пришел я пред лице Господне.

\vs 2En 13:1
И восхитил меня Гавриил, как восхищается лист ветром, и погнал меня, и поставил меня пред лицем Господним.
\vs 2En 13:2
И увидел я Господа, и лицо его могущественно, и преславно и страшно.
\vs 2En 13:3
Но кто я, чтобы поведать, объять подлинное лице Господа, могущественное и весьма страшное,
\vs 2En 13:4
или изречь о хорах окрест Его, многоочитых и многогласных, о престоле Его, весьма великом и нерукотворенном,
\vs 2En 13:5
или стояние, которое пред Ним, воинства Херувов и Серафов, или неизменное, неисповедимое, неумолкающее славное служение Ему?
\vs 2En 13:6
И пал я на лице свое, и поклонился Господу.
\vs 2En 13:7
И Господь устами Своими воззвал ко мне: Дерзай, Енох, не бойся! Восстань и стань пред лицем Моим во веки.
\vs 2En 13:8
И поднял меня Михаил, архангел Господень, и привел меня пред лице Господа.
\vs 2En 13:9
И испытал Господь слуг своих, сказав им: Да вступит Енох, чтобы стоять пред лицем Моим во веки.
\vs 2En 13:10
И Славные поклонились Ему и сказали: Да вступит.
\vs 2En 13:11
И сказал Господь Михаилу: Возьми Еноха, и сними с него земные одежды, и помажь елеем многоценным, и облеки в ризы славы.
\vs 2En 13:12
И снял Михаил одежды мои с меня, и помазал меня елеем благим.
\vs 2En 13:13
И вид елея ярче света великого, и умащение им словно роса добрая, и благоухание его подобно мирре, и лучи его как солнечные.
\vs 2En 13:14
И оглядел я всего себя: и стал я, как один из Славных, и не было на вид различия.

\vs 2En 14:1
И призвал Господь Веревеила, одного из архангелов Своих, который был мудр и записывал все дела Господни.
\vs 2En 14:2
И сказал Господь Веревеилу: Возьми книги из хранилищ и дай Еноху трость и прочти ему книги.
\vs 2En 14:3
И поспешил Веревеил и принес мне книги, изукрашенные смирной.
\vs 2En 14:4
И дал мне трость из руки своей, и стал рассказывать мне все дела Господни: о земле, о море, о всех стихиях, о движении всех планет и бытии их,
\vs 2En 14:5
о смене лет и движении дней, о земных заповедях и наставлениях, о сладкогласном пении, о входах облаков и исходах ветров,
\vs 2En 14:6
о всяком народе, и о новой песне вооруженного воинства все, чему следовало меня научить, поведал мне Веревеил.
\vs 2En 14:7
Тридцать дней и тридцать ночей говорили уста его, не умолкая.
\vs 2En 14:8
И я не спал тридцать дней и тридцать ночей, записывая все знаками.
\vs 2En 14:9
Когда же закончил, сказал мне Веревеил: Сядь, напиши то, что я поведал тебе.
\vs 2En 14:10
И я, просидев еще тридцать дней и тридцать ночей, записал все подробно и отчетливо, и поведал это в трехсот и шестидесяти книгах.

\vs 2En 15:1
И призвал меня Господь, и поставил меня слева от Себя рядом с Гавриилом, и я поклонился Господу.
\vs 2En 15:2
И сказал мне Господь: Все, что ты видел, Енох, неподвижное и движущееся, сотворено Мною, и Я возвещу тебе о том от начала.
\vs 2En 15:3
Прежде, когда не было всего, что Я привел из небытия в бытие, и из невидимого в видимое, и ангелам Моим не возвестил Я тайны Моей, и не поведал им, как сотворил их, и не постигли они бесконечного Моего и непостижимого творения, тебе же Я возвещаю сегодня.
\vs 2En 15:4
Прежде, когда не было всего видимого, открылся свет, и Я среди света, будучи невидим, один проезжал, как ездит солнце от востока до запада и от запада на восток,
\vs 2En 15:5
но солнце находит покой, Я же не обрел покоя, поскольку все было несотворенным.
\vs 2En 15:6
И помыслил Я поставить основание, сотворить тварь видимую. И повелел Я в глубине, да взойдет одно из невидимых в видимое.
\vs 2En 15:7
И вышла Божественная вечность, весьма великая, и вот, имела она во чреве своем великий век.
\vs 2En 15:8
И Я сказал ей: Разрешись от бремени, о Божественная вечность, и да будет видимо разрешаемое из тебя.
\vs 2En 15:9
И разрешилась она, и вышел из нее великий век, и таким образом изнесло все творение, которое Я восхотел сотворить. И увидел Я, что это хорошо.
\vs 2En 15:10
И поставил Я Себе престол, и сел на нем, свету же сказал: Взойди ввысь, и утвердись, и будь основанием горнему. И нет превыше света ничего иного.
\vs 2En 15:11
И увидев это, Я восклонился с престола Моего, и воззвал в глубине во второй раз, и сказал: Да произыдет из невидимого твердь, и да станет видима!
\vs 2En 15:12
И произошло основание тверди, тяжелое и весьма мрачное. И увидел Я, что это хорошо.
\vs 2En 15:13
И Я сказал ему: Сойди вниз, и утвердись, и будь основанием дольнему.
\vs 2En 15:14
И сошло оно, и утвердилось, и стало основанием дольнему. И ниже тьмы нет ничего иного.
\vs 2En 15:15
И облек Я эфир светом, уплотнил Я его и простер Я его над тьмою, а из вод утвердил камни великие.
\vs 2En 15:16
Водам же бездны повелел Я высохнуть, а впадины Я назвал безднами.
\vs 2En 15:17
Море Я собрал в одно место, связал его узами, и дал морю границу вечную, и не исторгнется из вод.
\vs 2En 15:18
И поставил Я твердь и основал ее поверх вод.
\vs 2En 15:19
И помимо всего воинства небесного, образовал Я на небесах солнце от света великого, и поставил его на небе, дабы светило оно на землю.
\vs 2En 15:20
Из камня Я высек огонь великий и из огня сотворил все воинства безплотные и все воинства звезд. И Херувов, и Серафов, и Офанов и это все Я высек из огня.
\vs 2En 15:21
Земле же Я повелел произрастить всякое дерево, и всякую гору, и всякую траву живую, и всякое семя живое, сеющее семя, прежде, чем сотворить души живые, Я приготовил пищу им.
\vs 2En 15:22
И морю повелел Я породить в себе рыб и всяких гадов, ползающих по земле, и всякую птицу парящую.
\vs 2En 15:23
И когда закончил все, повелел Я Премудрости Своей сотворить человека.

\vs 2En 16:1
И ныне, то, что Я рассказал тебе, и то, что ты видел на небесах, и то, что видел на земле, и то, что ты написал в книгах, создал Я Премудростью Своею и искусством Своим, и сотворил от нижнего основания до высшего.
\vs 2En 16:2
И до скончания их нет Мне советника, ни помощника.
\vs 2En 16:3
Сам Я вечен, нерукотворен. Неизменна мысль моя, советник Мой, и слово Мое есть дело, и очи Мои следят за всем: если отверну лице Мое все погибнут, если же призираю утверждаются.
\vs 2En 16:4
Положи, Енох, ум свой, и познай Говорящего с тобою, и возьми книги, которые ты написал.
\vs 2En 16:5
И даю тебе Семеила и Рагуила, возведших тебя ко Мне, и сойди на землю и расскажи сынам своим то, что Я говорил тебе, и то, что видел ты от нижнего неба и до престола Моего,
\vs 2En 16:6
все воинства Я сотворил, и нет противящегося Мне или не покоряющегося: все покоряются Моему единовластию и работают одной Моей власти.
\vs 2En 16:7
И дай им книги, соделанные рукою твоею, и прочтут они и познают Творца своего, и уразумеют и они, что нет иного Творца, кроме Меня.
\vs 2En 16:8
И передай книги, соделанные рукою твоею, детям и детям детей твоих, и дай наставления ближним из рода в род.
\vs 2En 16:9
Ибо Я дам тебе ходатая, о Енох, архистратига моего Михаила, чтобы написанное рукою твоею и написанное рукою отцов твоих, Адамом и Сифом, не погибло до века последнего, 10 Ибо Я заповедал ангелам Моим, Ариоху и Мариоху, которых поставил Я над землею, дабы хранили ее и повелевали временами,
\vs 2En 16:11
дабы соблюли они написанное рукою твоею и написанное рукою отцов твоих, и не погибло это в грядущий потоп, который Я сотворю в роде твоем.
\vs 2En 16:12
Я знаю злобу человеческую, что они не вынесут бремени, которое Я возложил на них, и не будут сеять семя, которое Я дал им, но отвергнут бремя Мое, и иное бремя примут,
\vs 2En 16:13
и посеют семена пустые, и поклонятся богам суетным, и отринут единовластие Мое, и вся земля согрешит неправдами, и обидами, и прелюбодейством, и идолослужением.
\vs 2En 16:14
Тогда наведу Я потоп на землю, и земля сама сокрушится в бездну великую.
\vs 2En 16:15
И Я оставлю мужа праведного из племени твоего, со всем домом его, который сотворит по воле Моей.
\vs 2En 16:16
И от семени их востанет род последний, многочисленный и весьма ненасытный.
\vs 2En 16:17
Тогда при исходе рода того явятся книги, написанные рукою твоею и отцов твоих, которые стражи земные покажут мужам верным, и расскажут роду тому и будут они почитаемы впоследствии больше, чем прежде.

\vs 2En 17:1
Ныне же, Енох, даю тебе срок ожидания тридцать дней, чтобы сотворил ты в доме твоем и рассказал сыновьям твоим и домочадцам твоим от Меня.
\vs 2En 17:2
И всякий, кто блюдет сердце свое, да прочтет и уразумеет, что нет никого, кроме Меня.
\vs 2En 17:3
И спустя тридцать дней Я пошлю ангелов за тобою, и возьмут тебя ко Мне от земли и от сыновей твоих;
\vs 2En 17:4
возьмут тебя ко Мне, ибо уготовано тебе место, и ты будешь перед лицем Моим отныне и до века.
\vs 2En 17:5
И будешь созерцать тайны Мои, и будешь книжником над рабами Моими,
\vs 2En 17:6
ибо будешь записывать все дела земные и о пребывающих на земле и на небесах, и будешь Мне свидетелем Суда Великого Века.
\vs 2En 17:7
Все сие говорил мне Господь, как говорит муж ближнему своему.

\vs 2En 18:1
И ныне, чада мои, услышьте голос отца вашего и то, что я заповедаю вам сегодня:
\vs 2En 18:2
ходите пред лицем Господним, и все, чему должно произойти по воле Господа.
\vs 2En 18:3
Ибо я послан от уст Господа к вам, дабы сказать вам, что есть и что будет до Дня Судного.
\vs 2En 18:4
И ныне, чада мои, не от уст моих вещаю вам сегодня, но от уст Господа, пославшего меня к вам.
\vs 2En 18:5
И вы слышите слова мои из уст моих, подобно вам созданного человека; я же слышал из уст Господа, огненных, ибо уста Господа как печь огненная, и слова его как пламя огненное исходят.
\vs 2En 18:6
Вы, чада мои, видите лице мое, подобно вам созданного человека, я же видел лице Господа, как железо, раскаленное огнем, испускающее искры.
\vs 2En 18:7
И вы видите глаза, подобно вам созданного человека, я же видел очи Господа, светящиеся, как лучи солнца, ужасающие глаза человеческие.
\vs 2En 18:8
И вы, дети, видите десницу мою, подающую вам знаки, подобно вам сотворенного человека, я же видел руку Господа, подающую знак мне, наполняющую небо.
\vs 2En 18:9
И вы видите охват тела моего, подобного вашему, я же видел объятие Господа, безграничное и несравненное, которому нет конца.
\vs 2En 18:10
И вы слышите слова из уст моих, я же слышал глаголы Господа, как гром великий, приводящие в непрестанное движение облака.
\vs 2En 18:11
Теперь, чада мои, услышьте беседующего о царе земном страшно и трепетно стоять перед лицем царя земного, страшно, потому что воля царя смерть, и воля царя жизнь.
\vs 2En 18:12
Каково же стоять перед лицем Царя Небесного? кто выдержит этот безмерный страх и жар великий?
\vs 2En 18:13
Но призвал Господь одного из ангелов своих верховных, грозного, и поставил рядом со мною,
\vs 2En 18:14
видом же ангел был, как снег, а руки его как лед, и он остудил лице мое, потому что не стерпел бы я страха и жара огненного.
\vs 2En 18:15
И тогда сказал мне Господь все слова Свои.

\vs 2En 19:1
И ныне, чада мои, я знаю все: одно из уст Господа, другое глаза мои видели; от начала и до конца, и от конца до нового обращения все я узнал.
\vs 2En 19:2
И записал я в книгах обо всем наполняющем небеса до краев их, я измерил путь их, и воинство их я узнал, и записал звезд многое множество бесчисленное.
\vs 2En 19:3
Кто из людей знает о круговом движении их и пути их, и обращении их, или о том, кто ведет их, или о ведомых?
\vs 2En 19:4
Ангелы не знают числа их, я же имена их записал.
\vs 2En 19:5
И солнечный круг я измерил, и лучи сосчитал, и весь путь его, и входы его и исходы его, и имена их записал.
\vs 2En 19:6
И лунный круг я измерил, и движение ее во все дни исчислил, и свет ее на всякий день и час, и в книгах имена ее записал.
\vs 2En 19:7
И жилища облаков, и устав их, и крылья их, и дождь их, и капли их я изследовал.
\vs 2En 19:8
И описал грохот грома и блеск молнии, и показали мне хранителей ключей их, и восходы их;
\vs 2En 19:9
ходят же по мере: узами поднимаются и узами опускаются, дабы тяжелым напором не обрушили облака, и не погубили то, что на земле.
\vs 2En 19:10
Я написал о сокровищницах снега и хранилищах льда и воздуха холодного;
\vs 2En 19:11
наблюдал я, как время от времени хранители ключей их наполняют ими облака, но не истощаются сокровищницы.
\vs 2En 19:12
Написал я о ложе ветров, смотрел я и увидел, как хранители ключей их, носящие весы с собой и меры, на одну чашу весов кладут сокровища, во вторую же меру, и по мере отпускают их на всю землю, дабы чрезмерным ветром не поколебать землю.

\vs 2En 20:1
Оттуда сведен я был вниз, и пришел на место судное, и я видел ад отверстый, и видел там тех, кому хуже, чем узникам, суд безмерный.
\vs 2En 20:2
И спустился я, и написал о всяком суде осужденных, и все вопрошения их увидел, и воздохнул, и заплакал о погибели нечестивых.
\vs 2En 20:3
И сказал я в сердце своем: Блажен, кто еще не родился, или родившийся, но не согрешивший перед лицем Господа, дабы не попал на место это и дабы не понес бремени места этого.
\vs 2En 20:4
И видел я хранителей ключей ада, стоящих у весьма великих ворот, подобных аспидам огромным,
\vs 2En 20:5
лица их как свечи потухшие, глаза их как пламя померкшее, и зубы их обнажены до персей их.
\vs 2En 20:6
И сказал я в лице их: Ушел бы я и не видел вас, ибо вы здесь за деяния ваши. И да не придет никто из племени моего к вам.

\vs 2En 21:1
И оттуда взошел я в рай праведных, и видел там место благословенное, и вся тварь благословенна, и все живут в радости и веселии, и в свете безмерном, и в жизни вечной.
\vs 2En 21:2
Тогда сказал я: Чада мои, говорю вам: блажен, кто боится имени Господа, и перед лицем Его будет служить всегда, и приготовит дары и приношения \ldots и жизнию поживет, и умрет.
\vs 2En 21:3
Блажен, кто будет творить суд праведный: нагого оденет в одежды и голодному даст хлеб.
\vs 2En 21:4
Блажен, кто судит суд праведный: сироте, и вдовице, и всякому обиженному поможет.
\vs 2En 21:5
Блажен, кто сойдет с пути временного и пойдет путями праведными.
\vs 2En 21:6
Блажен, кто сеет семена праведные, ибо и пожнет их седмерицею.
\vs 2En 21:7
Блажен, в ком есть истина, да говорит истину ближнему своему.
\vs 2En 21:8
Блажен, у кого в устах его милость истинная и кротость.
\vs 2En 21:9
Блажен, кто разумеет дела Господа, ибо по делам Его познается Создатель.
\vs 2En 21:10
И вот, дети мои, я обозрел землю до краев ее, и записал я все: все года сложил, и из лет разделил месяцы, и в месяце рассчитал дни, дни разделил на часы, часы же измерил.
\vs 2En 21:11
И описал всякое семя на земле, и разделил каждую меру, и каждые весы правильные я измерил, и описал.
\vs 2En 21:12
И как год от года разнится в достоинстве, так и человек от человека в чести:
\vs 2En 21:13
кто благодаря большому богатству, кто благодаря сердечной мудрости, а кто благодаря остроте ума или молчанию уст.
\vs 2En 21:14
Но нет никого более боящегося Господа, ибо боящиеся Господа славны будут во век.
\vs 2En 21:15
Господь руками Своими создал человека в подобие лица Своего, малого и великого сотворил Господь.
\vs 2En 21:16
Оскорбляющий лице человеческое оскорбляет лице Господа, гнушающийся лица человеческого гнушается лица Господа,
\vs 2En 21:17
презирающий лице человека презирает лице Господа; гнев и Суд Великий тем, кто плюет в лицо человеку.

\vs 2En 22:1
Блажен, кто приготовит себя всякому человеку: кто помогает осуждаемому, и кто поднимает упавшего, и кто подает просящему,
\vs 2En 22:2
ибо в день Суда Великого всякое дело человека обновится писанием.
\vs 2En 22:3
Блажен, у кого будет мера праведная, и весы праведные, и гири праведные,
\vs 2En 22:4
так как в день Суда Великого каждая мера, и каждые весы, и каждая гиря словно при покупке приложатся, и узнает каждый меру свою и по ней примет мзду.
\vs 2En 22:5
Тому, кто всегда творит пред лицем Господа, управит Господь приобретения его.
\vs 2En 22:6
Тому, кто умножает светильники пред лицем Господа, умножит Господь хранилища его.
\vs 2En 22:7
Разве нужны Господу хлеб или свеча, или овен, или телец? но этим испытывает Господь сердце человека.
\vs 2En 22:8
Ибо когда Господь пошлет свой свет великий во тьму и будет Суд, кто тогда утаиться?
\vs 2En 22:9
Ныне же, чада мои, положите помышление на сердцах ваших и внемлите словам отца вашего, ибо то, что я вещаю вам от уст Господних.
\vs 2En 22:10
Возьмите книги эти книги, написанные рукою отца вашего, и прочтите их, и из них узнаете дела Господа, и что нет никого, кроме Господа единого,
\vs 2En 22:11
Который поставил основания на неведомом, и простер небеса на невидимом, землю поставил, на водах ее основав непостоянных,
\vs 2En 22:12
Который безчисленную тварь сотворил один (а кто исчислил прах земной или песок морской, или капли облаков?),
\vs 2En 22:13
Который землю и море соединил нерушимыми узами,
\vs 2En 22:14
Который немыслимую красоту из огня высек и украсил небо,
\vs 2En 22:15
Который из невидимого в видимое все сотворил, Сам будучи невидимым.
\vs 2En 22:16
И раздайте эти книги детям вашим и детям детей ваших.
\vs 2En 22:17
И все ближние ваши, и все сродники ваши, которые знают и боятся Господа, да примут их, и да будут они им нужнее всякой пищи доброй, и да прочтут и прилепятся к ним!
\vs 2En 22:18
А неразумные, не знающие Господа, не примут их, но отвергнут, ибо отягчат они бремя их.
\vs 2En 22:19
Блажен, кто понесет бремя их, примет его, ибо обретет его в день Суда Великого.
\vs 2En 22:20
Ибо я клянусь вам, чада мои, что еще прежде, чем быть человеку, место Судное уготовано ему, и мера, и весы, которыми будет испытан человек, там прежде уготованы.
\vs 2En 22:21
Я же дело всякого человека в письменах изложу, и никто не сможет укрыться.

\vs 2En 23:1
И ныне, чада мои, пребывайте в терпении и кротости число дней ваших, да наследуете безконечный век будущий.
\vs 2En 23:2
И всякое бедствие, и всякое страдание, и зной, и всякое слово злое, если найдет на вас, потерпите Господа ради.
\vs 2En 23:3
И хотя можете отплатить расплатой, не воздавайте ближнему, ибо один Господь воздает, и будет отмщающим за вас в День Суда Великого.
\vs 2En 23:4
Золотом и серебром пожертвуйте ради брата, дабы принять сокровище плоти в День Судный.
\vs 2En 23:5
И к сироте и вдове прострите руки ваши, и по силе помогите бедному, и обретете покровительство во время всякого труда.
\vs 2En 23:6
Если найдет на вас скорбь и печаль, ради Господа отриньте, и обретете воздаяние в День Судный.
\vs 2En 23:7
Утром, и в полдень, и вечером благо есть ходить в дом Господень прославлять Творца всего.
\vs 2En 23:8
Блажен, кто раскрывает сердце свое для хвалы и хвалит Господа.
\vs 2En 23:9
Проклят раскрывающий сердце свое для хулы и клеветы на ближнего.
\vs 2En 23:10
Блажен, кто раскрывает уста свои, благословляя и прославляя Господа.
\vs 2En 23:11
Проклят раскрывающий уста свои для проклятия и хулы в лице Господа.
\vs 2En 23:12
Блажен прославляющий все дела Господни.
\vs 2En 23:13
Проклят оскорбляющий творение Господа.
\vs 2En 23:14
Блажен созидающий и воздвигающий трудом рук своих.
\vs 2En 23:15
Проклят стремящийся уничтожить труды чужие.
\vs 2En 23:16
Блажен хранящий устои отцов до конца.
\vs 2En 23:17
Проклят нарушающий установления и законы отцов своих.
\vs 2En 23:18
Благословен насаждающий мир.
\vs 2En 23:19
Проклят уничтожающий мирное.
\vs 2En 23:20
Благословен говорящий мир и имеющий мир в сердце своем.
\vs 2En 23:21
Проклят говорящий то, но не имеющий мира в сердце своем.
\vs 2En 23:22
Все это на весах и в книгах изобличится в День Суда Высшего.

\vs 2En 24:1
И ныне, чада мои, блюдите сердца ваши от всякой неправды, да унаследуете подножие света во веки.
\vs 2En 24:2
Не говорите, дети мои: отец наш с Господом и умолит нас о грехе.
\vs 2En 24:3
Знайте, что все дела всякого человека я записываю, и никто не может уничтожить написанного рукою моею, потому что Господь все видит.
\vs 2En 24:4
И теперь, чада мои, усвойте все слова отца вашего, которые я говорю вам, да будут они вам в достояние покоя.
\vs 2En 24:5
И книги, которые я дал вам, не отриньте их, но всем желающим растолкуйте их, может быть узнают дела Господа.
\vs 2En 24:6
И вот, чада мои, приближается назначенный день года, и время подходит установленное, и ангелы, которые идут со мною, стоят пред лицем моим.
\vs 2En 24:7
И утром я поднимусь на небо высшее, в вечное мое достояние, и потому заповедаю вам, дети мои, делайте все то, что благословенно пред лицем Господа.

\vs 2En 25:1
И отвечал Мафусела отцу своему Еноху: Что угодно очам твоим, отец? Да приготовим пищу пред лицем твоим;
\vs 2En 25:2
да благословишь дома наши и сыновей своих, и всех домочадцев своих, и прославишь народ свой, и после этого уйдешь.
\vs 2En 25:3
И ответил Енох сыну своему, говоря: Слушай, сын мой, от того дня, как помазал меня Господь елеем славы Своей, вострепетал я, и не услаждает меня пища, и не хочется мне ничего из земных яств.
\vs 2En 25:4
Но позови братьев своих и всех домочадцев наших, и старейшин народа, дабы я говорил с ними, и тогда отойду.
\vs 2En 25:5
И поспешил Мафусела, и позвал братьев своих Регима, и Ариима, и Ахазухана, и Харимиона, и старейшин народа, и привел их пред лице отца своего Еноха.
\vs 2En 25:6
И поклонились ему, и принял их Енох, и благословил их, и отвечал ним, говоря:
\vs 2En 25:7
Послушайте, чада! Во дни отца вашего Адама сошел Господь на землю, чтобы посетить ее и все сотворенное Им, которое Сам создал.
\vs 2En 25:8
И призвал Господь всех зверей земных и всех гадов земных, и всех птиц пернатых, и привел их пред лице отца вашего Адама, чтобы нарек он имена всем на земле.
\vs 2En 25:9
И оставил их Господь у него, и подчинил ему всех, сделав вторыми по меньшинству, и притупил весь разум их, дабы повиновались человеку.
\vs 2En 25:10
Ибо Господь сотворил человека над всем владением Своим, и за это не будет Суда никакой душе живой, но одному человеку.
\vs 2En 25:11
Всем душам скотов уготовано одно место, и предел один, и пастбище одно в Веке Великом.
\vs 2En 25:12
И не укроется ни одна душа живая, которую сотворил Господь, до Суда, и все души, которых оклевещут до Суда.
\vs 2En 25:13
И тот, кто плохо заботится о душе своей, сделает свою душу беззаконною.
\vs 2En 25:14
А приносящий жертву из чистого скота, имеет исцеление, он исцеляет душу свою.
\vs 2En 25:15
Умерщвляющий же скот всякий, не связав его, преступает закон, он предает душу свою беззаконию.
\vs 2En 25:16
И творящий злое животным в тайне, преступает закон, он предает душу свою беззаконию.
\vs 2En 25:17
Творящий злое душе человеческой, творит злое душе своей, и нет ему исцеления во веки.
\vs 2En 25:18
Толкающий человека в сеть, сам в ней увязнет, и нет ему исцеления во веки.
\vs 2En 25:19
И подвергающий человека осуждению, непременно будет осужден во веки.

\vs 2En 26:1
И ныне, чада мои, храните сердца ваши от всякой неправды, которую возненавидит Господь, более же всего по отношению ко всякой душе живой, которую создал Господь.
\vs 2En 26:2
И что просит человек для своей души от Господа, пусть тоже сотворит всякой душе живой.
\vs 2En 26:3
Потому что в Веке Великом многие обители уготованы человеку: обители добрые весьма и бесчисленные обители злые.
\vs 2En 26:4
Блажен, кто отойдет в благословенные обители, ибо из злых нет возвращения.
\vs 2En 26:5
Когда положит человек слово в сердце своем принести дар пред лицем Господа, а руки его того не сделают, тогда отвергнет Господь труд рук его, и не обретет ничего.
\vs 2En 26:6
И если сотворят руки его, но сердце его будет роптать, то не прекратится болезнь сердца его, ибо роптание его поспешно.
\vs 2En 26:7
Блажен человек, который в терпении своем принесет дар пред лицем Господа, ибо обретет воздаяние.
\vs 2En 26:8
И если человек назначит устами своими определенное время для принесения дара пред лицем Господа и совершит это то обретет воздаяние;
\vs 2En 26:9
если же пройдет назначенное время, и возвратит слова свои, то даже если раскается, не будет ему благословения.
\vs 2En 26:10
Потому что всякое промедление порождает искушение.
\vs 2En 26:11
Человек, который прикроет нагого и алчущему даст хлеб, обретет воздаяние.
\vs 2En 26:12
Если же станет роптать сердце его, то погубит себя и не будет ему воздаяния.
\vs 2En 26:13
И если нищий, когда насытится сердце его, возгордится, то погубит все добрые дела свои и не обретет воздаяния, ибо мерзок Господу всякий муж возгордившийся.

\vs 2En 27:1
И было, когда говорил Енох сыновьям своим и князьям народа, услышал его весь народ и все близкие его, что призывает Еноха Господь,
\vs 2En 27:2
и, посовещавшись, сказали: Идем и приветствуем Еноха.
\vs 2En 27:3
И собралось около двух тысяч мужей, и пришли на место Азухань, где был Енох и сыновья его, и старейшины народа, и приветствовали Еноха, говоря:
\vs 2En 27:4
Благословен ты у Господа, Царя Вечного, ныне же благослови народ свой и прославь его пред лицем Господа, ибо тебя избрал Господь в пророки, чтобы ты отнял грехи наши.
\vs 2En 27:5
И отвечал Енох народу своему, говоря: Слушайте, чада мои. Сначала, когда ничего не было, прежде, чем появилось все творение, создал Господь век сотворенный,
\vs 2En 27:6
и после этого сотворил все творение Свое, видимое и невидимое, и после всего этого создал человека по образу Своему,
\vs 2En 27:7
и вложил ему глаза, чтобы видеть, и уши, чтобы слышать, и сердце, чтобы разуметь, и ум, чтобы размышлять.
\vs 2En 27:8
Тогда освободил Господь век ради человека, и разделил его на времена и часы,
\vs 2En 27:9
да размышляет человек о смене времен, и о конце и начале лет, и окончании месяцев, и дней, и часов, да предаст ему свою жизнь и смерть.
\vs 2En 27:10
Когда же перестанет существовать все творение, которое сотворил Господь, и всякий человек придет на Суд Господа Великий,
\vs 2En 27:11
тогда исчезнут времена, и лет больше не будет, и ни месяцы, ни дни, ни часы более не будут сосчитываться, но настанет век единый.
\vs 2En 27:12
И все праведники, которые избегнут Суда Господня Великого, соединятся в Веке Великом, вместе соединятся праведники, и будут пребывать вечно.
\vs 2En 27:13
И более не будет у них ни труда, ни болезни, ни скорби, ни ожидания невзгод, ни тягот, ни ночи, ни тьмы; но свет великий будет для них всегда.
\vs 2En 27:14
И стена неразрушимая в раю великом будет защитой их жилища вечного.
\vs 2En 27:15
Блаженны праведники, которые избегнут Суда Господня Великого, ибо озарятся лица их подобно солнцу.
\vs 2En 27:16
Ныне же, чада мои, оберегайте души ваши от всякой неправды, которую возненавидел Господь;
\vs 2En 27:17
пред лицем Господа ходите и Ему одному служите, и всякое приношение приносите пред лице Господа.
\vs 2En 27:18
Если и посмотрит человек ввысь то там Господь, ибо Господь сотворил небеса;
\vs 2En 27:19
если посмотрит на землю и на море и подумает о том, что под землей, то и там Господь, ибо Господь сотворил все, и не скроется ни какое дело от лица Господня.
\vs 2En 27:20
Вы же с долготерпением и с кротостью, сквозь страдания и мучения, пройдете болезненный век сей.

\vs 2En 28:1
И когда говорил это Енох народу своему, послал Господь мрак на землю, и была тьма, и покрыла тьма стоящих с Енохом мужей.
\vs 2En 28:2
И поспешили ангелы, и взяли ангелы Еноха, и вознесли его на небо вышнее.
\vs 2En 28:3
И принял его Господь, и поставил его пред лицем Своим во веки.
\vs 2En 28:4
И отступила тьма от земли, и стал свет.
\vs 2En 28:5
И увидел народ, и понял, что взят был Енох, и, прославив Бога, пошли в дома свои.
\vs 2En 28:6
И поспешил Мафусела и братья его, сыновья Еноха, и сделали жертвенник на месте Азухань, откуда взят был Енох, и, взяв овнов и тельцов, принесли их в жертву пред лицем Господа.
\vs 2En 28:7
И созвали всех людей, дабы пришли к ним на пир.
\vs 2En 28:8
И принесли люди дары сыновьям Еноха.
\vs 2En 28:9
И радовались и веселились три дня.

\vs 2En 29:1
И в третий день, во время вечернее, сказали старейшины народа Мафуселе, говоря:
\vs 2En 29:2
Иди и встань пред лицем Господа и пред лицем народа своего пред алтарем Господним и будешь славен в народе твоем.
\vs 2En 29:3
И отвечал Мафусела народу своему: Господь Бог отца моего Еноха, Сам изберет священника над народом Своим.
\vs 2En 29:4
И ждал народ всю ночь ту на месте Азухань.
\vs 2En 29:5
И пребывал Мафусела у алтаря, и молился Господу, говоря: Господь всего века, сего ли сына отца нашего Еноха избрал ты?
\vs 2En 29:6
Господи, яви священника народу Своему, да в неразумии сердца их боятся славы Твоей, и сотвори все по воле Твоей!
\vs 2En 29:7
И уснул Мафусела, и явился ему Господь в видении ночном, и сказал ему:
\vs 2En 29:8
Слушай Мафусела, Я Господь Бог отца твоего Еноха, услышал Я глас народа Своего.
\vs 2En 29:9
Встань же пред ними и перед алтарем Моим, и прославлю тебя перед этим народом Моим во все дни жизни твоей.
\vs 2En 29:10
И востал Мафусела от сна своего, и благословил Явившегося ему.
\vs 2En 29:11
И утром пришли старейшины народа к Мафуселе, и направил Господь Бог сердце Мафуселы послушаться голоса народа,
\vs 2En 29:12
и сказал им: Да сотворит Господь Бог наш благое в глазах Его для этого народа Своего.
\vs 2En 29:13
И поспешили Сарсан, и Хармий, и Заза, старейшины народа, и облекли Мафуселу в одежду превосходную, и возложили венец светлый на голову его.
\vs 2En 29:14
И поспешно привел народ овнов и тельцов, и из птиц все, что положено, чтобы принес Мафусела жертву пред лицем Господа и пред лицем народа.
\vs 2En 29:15
И поднялся Мафусела к жертвеннику Господню, подобно восходящей деннице, и весь народ шел за ним.
\vs 2En 29:16
И стал Мафусела у алтаря, и весь народ вокруг алтаря.
\vs 2En 29:17
И старейшины народа, взяв овнов и тельцов, связали их по четыре ноги и положили на возглавие алтаря.
\vs 2En 29:18
И сказал народ Мафуселе: Возьми нож и заколи назначенное перед лицем Господа.
\vs 2En 29:19
И простер Мафусела руки свои к небу и призвал Господа, говоря:
\vs 2En 29:20
Увы мне, Господи, кто я, чтобы стоять у возглавия жертвенника Твоего, во главе всего народа Твоего, и для всего познания?
\vs 2En 29:21
Яви благодать рабу Твоему пред лицем народа сего, да знают, что это Ты! Назначь священника народу Своему!
\vs 2En 29:22
И было, когда молился Мафусела, сотрясся алтарь, и поднялся нож с алтаря, и вскочил нож в руки Мафуселе перед лицем всего народа.
\vs 2En 29:23
И объял всех людей трепет, и прославили Господа.
\vs 2En 29:24
И был почитаем Мафусела в глазах Господа и в глазах всего народа с того дня.
\vs 2En 29:25
И взял Мафусела нож, и совершил заклание всех приношений народа.
\vs 2En 29:26
И возрадовался народ, и возвеселился пред лицем Господа и пред лицем Мафуселы в тот день, и после этого разошлись по домам своим.
\vs 2En 29:27
И стоял Мафусела у возглавия алтаря и во главе всего народа с того дня триста девяносто два года.
\vs 2En 29:28
И благословил Господь Мафуселу в жертвах, и в дарах его, и во всем служении, которое совершал он пред лицем Господа.

\vs 2En 30:1
И когда приблизились к концу дни Мафуселы, явился ему Господь в видении ночном и сказал ему:
\vs 2En 30:2
Слушай, Мафусела, Я Бог отца твоего Еноха. Познай волю Мою, ибо кончились дни жизни твоей, и приблизился день отдохновения твоего.
\vs 2En 30:3
Ибо приблизились времена погибели всей земли, и всякого человека, и всего, что движется по земле, ибо во дни сии велико нестроение на земле.
\vs 2En 30:4
Ибо возненавидел человек ближнего своего, и люди на людей нападают, и народ против народа возбуждает брань, и наполнилась земля кровью и пагубным безпорядком.
\vs 2En 30:5
И ко всему этому они оставили Творца Своего, и стали поклоняться тверди небесной, и ходящим по земле, и волнам морским.
\vs 2En 30:6
И возвеличился Сатана, и радуется делам их.
\vs 2En 30:7
И к негодованию Моему, вся земля восприняла перемену устройства своего, и всякий плод, и всякая трава изменили пору свою, ибо предчувствуют время погибели.
\vs 2En 30:8
И все народы изменились на земле, к сожалению Моему.
\vs 2En 30:9
И тогда Я повелю бездне низринуться на землю, и запасы вод небесных устремятся на землю.
\vs 2En 30:10
И погибнет весь состав земной, и сотрясется вся земля, и лишится силы своей с того дня.
\vs 2En 30:11
Тогда Я сохраню Ноя, первородного сына Ламеха, сына твоего, и воссоздам от семени его иной мир, и семя его пребудет во веки.
\vs 2En 30:12
И пробудился Мафусела от сна своего, и весьма опечалился о сне этом;
\vs 2En 30:13
и призвал всех старейшин народа и поведал им все, что сказал ему Господь, и о видении, явившемся ему от Господа.
\vs 2En 30:14
И опечалились люди из-за видения его, и ответили ему: Господь властен творить по воле Своей.
\vs 2En 30:15
И когда говорил Мафусела народу, взволновался дух его, и преклонил он колени свои, и простер руки свои к небу, и молился Господу, и когда молился он, отошел дух его.

\bibbookdescr{Jub}{
  inline={Книга Юбилеев},
  toc={Книга Юбилеев},
  bookmark={Книга Юбилеев},
  header={Книга Юбилеев},
  abbr={Юбл}
}
\vs Jub 0:0
Вот слова деления дней по закону и свидетельству,
по событиям годов,
по их седминам, по их юбилеям, на все годы мира,
согласно с тем, что говорил он с Моисеем на горе Синай.

\vs Jub 1:1
Случилось в первый год по выходе сынов Израиля из Египта, в третий месяц, в
шестнадцатый день его, тогда сказал Бог Моисею, говоря: <<Взойди ко Мне на
гору, чтобы Я дал тебе две каменные скрижали закона и все заповеди, которые Я
написал, дабы ты возвестил их им (сынам Израиля)!>> И Моисей взошел на гору
Господню, и слава Господня обитала на горе Синай, и облако осеняло ее шесть
дней. И Он воззвал Моисея в седьмой день среди облака. И он видел славу Божию,
как пылающий огонь, на горе Синай, когда взошел, чтобы получить каменные
скрижали закона и заповедей, по слову Господа, как Он сказал ему:
<<Поднимись на вершину горы!>> И Моисей был на горе сорок дней и сорок
ночей, и Господь научил его относительно того, что было прежде и что случится в
будущем; Он изъяснил ему деление дней закона и свидетельства и сказал:
<<Внимай всем словам, которые Я тебе говорю, и запиши их в книгу, дабы их
роды (потомки) видели, как Я оставил их за все зло, какое сделали они,
уклонившись от завета, который Я утверждаю ныне между Мною и тобою на горе
Синай для будущих родов их. И будет это слово, когда придут все наказания,
свидетельствовать против них, и они познают, что Я справедливее, нежели они во
всей их правде и во всяком их деле, и узнают, что Я был с ними. И ты запиши
себе все слова, которые Я тебе возвещаю ныне,~--- ибо Я знаю их противление
и жестоковыйность,~--- прежде чем приведу их в землю, о которой клялся
Аврааму, Исааку и Иакову, когда сказал: <<Вашему семени Я дам землю,
текущую молоком и медом>>. И они будут есть, и насыщаться, и уклоняться к
чуждым богам, к тем, которые их не спасли от всей их тяготы. И будет это
свидетельство услышано во свидетельство им: ибо они будут забывать Мои
заповеди, все, что Я заповедаю им, и пойдут вослед язычников и за их нечистотою
и мерзостию, и будут служить их богам, и эти (боги) сделаются для них
претыканием в бедствие, и страдание, и сетию. И многие погибнут, и будут
пленены, и впадут в руки врага, так как они забудут Мои постановления, и Мои
заповеди, и Мои праздники, Мой завет, и Мои субботы, и Мою святыню, которую Я
освящу Себе между ними, и Мою скинию, и Мое святилище, которое Я освящу Себе в
стране, чтобы положить на нем Свое имя, дабы оно обитало там. И они будут
делать себе изображения из камня и из дерева, и будут преклоняться пред ними,
чтобы впадать в грехи, и будут приносить своих сыновей в жертву демонам и
предаваться всем делам заблуждения своего сердца. И Я буду посылать к
ним свидетелей, чтобы дать им свидетельство, но они не послушают их и будут
убивать Моих свидетелей; и также тех, которые следуют закону, они будут убивать
и преследовать, и отвергнут его (закон) совершенно, и начнут делать то, что
есть зло пред Моими очами. Тогда Я сокрою Свое лице от них, и предам их
язычникам в пленение, и в узы, и на истребление, и изгоню их из земли
(Ханаанской), и рассею между язычниками. И они забудут весь Мой закон, и
все Мои заповеди, и всю Мою правду, и не будут более хранить ни новолуния, ни
субботы, и никакого праздника и юбилейного года, и никакого установления. После
сего они опять обратятся ко Мне из среды язычников всем сердцем и всею
душою и всеми своими силами. И Я соберу их всех из среды язычников; и они опять
будут искать Меня, чтобы Я явил им Себя. Когда же они будут искать Меня всем
сердцем и всею душою, Я открою им великий мир с правдою и восставлю их как
растение праведности от всего Моего сердца и от всей души; и они будут
во благословение, а не в проклятие, и соделаются главою, а не хвостом.

И Я воссоздам Мое святилище между ними и буду обитать с ними, и буду их
Богом, и они будут Моим народом воистину и вправду; и Я не оставлю их, не
отрекусь от них, ибо Я Господь, Бог их>>.

И Моисей пал на свое лице, и молился, и говорил: <<Господи, Боже мой! не
оставляй Твоего народа и Твоего наследия, чтобы не ходить им в заблуждении
своего сердца, и не предавай их в руки врагов-язычников, чтобы они
владычествовали над ними; не допусти их до сего, чтобы им не потерять Тебя!
Простри, Господи, Свое милосердие над народом Своим, и дух правый соделай в
них, и не допусти духа Велиара владычествовать над ними, чтобы он
клеветал на них пред Тобою и совращал их со всех путей правды, дабы они
погибли пред лицем Твоим! Ибо они Твой народ и наследие, который Ты великою
силою освободил из рук египтян; соделай в них чистое сердце и святой дух и не
допусти их, чтобы они были доведены до падения чрез свои грехи, отныне и до
века!>>

И Бог сказал Моисею: <<Я знаю их противление, и их помышления, и их
жестоковыйность; они не покорятся, пока не познают своих грехов и грехов
отцов их. И после сего они обратятся ко Мне во всей праведности, и от всего
сердца, и от всей души, и Я обрежу крайнюю плоть их сердца и крайнюю плоть
сердца их семени, и соделаю в них дух святой, и очищу их, чтобы они более не
отвращались от Меня с того дня и до века. И их душа прилепится ко Мне и ко всем
Моим заповедям, и они будут исполнять Мои повеления, и Я буду их отцом, и они
будут Моим сыном, и все будут именоваться сынами Божиими и все сынами
Духа. И тогда откроется, что они сыны Мои и Я отец их в праведность и
правду, и что Я люблю их. Ты же запиши себе все эти слова, которые Я возвещаю
тебе на этой горе, первое и последнее, и грядущее, согласно со всем делением
времени под законом и свидетельством и по седминам юбилейных годов, до века,
пока Я не сойду и не буду жить с ними от века до века>>.

И Он сказал Ангелу лица: <<Запиши для Моисея события с первого
творения до того времени, когда Мое святилище будет устроено между ними,
навсегда и навечно, и Бог откроется для очей каждого, и всякий познает, что Я
Бог Израиля, и Отец всех детей Иакова, и Царь на горе Сионе, от века до века. И
Сион Иерусалим будет святым>>. И Ангел лица, который шел пред станом
израильтян, взял скрижали деления лет от творения, седмин и юбилеев, закона и
свидетельства, каждый год по его числу и юбилеи по годам, со дня нового
творения, когда были сотворены небо и земля и все их произведения, равно как и
небесные силы и все творение земли, до того времени, когда будет создано
святилище Господа в Иерусалиме на горе Сионе, и все светила будут обновлены к
освящению, и к миру, и к благословению для всех избранных Израиля, чтобы сие
пребывало так от того дня в продолжение всех дней земли.

\vs Jub 2:1
И Ангел лица сказал Моисею по слову Господа, говоря: <<Напиши все
повествование о творении, как Господь Бог совершил в шесть дней все Свои
произведения, которые Он сотворил, и в седьмой день соблюдал субботу, и освятил
ее на все века, и утвердил ее в знамение для всех Своих творений>>.

Ибо в первый день Он сотворил небеса, которые вверху, и землю, и воды, и
всех духов, которые Ему служат, и Ангелов лица, и Ангелов прославления, и
Ангелов духа огня, и Ангелов духа ветров, и Ангелов облачных духов мрака, и
града и инея, и Ангелов долин, и громов и молний, и Ангелов духов холода и
зноя, зимы и весны, осени и лета, и Ангелов всех духов Его творений на небе, и
на земле, и во всех долинах, и духов мрака и света, и утренней зари, и вечера,
которые Он приготовил по предвидению Своей премудрости. И тогда мы увидели Его
произведения, и прославили Его, и восхвалили Его за все произведения Его, ибо
семь великих произведений Он сотворил в первый день.

И во второй день Он сотворил твердь между водами; и разделились воды в тот
день: половина их поднялась вверх, и половина опустилась вниз под твердь,
которая в середине, на поверхность всей земли. И это единственное произведение,
которое Он сотворил во второй день.

И в третий день сотворил Он, как сказал водам: да стекут они с поверхности
всей земли в одно место и да явится суша. И Он сделал таким образом с водами,
как сказал им. И они стекли с поверхности земли в одно место, вне тверди, и
явилась суша. И в тот день Он создал для нее (воды) бездны морей по их
отдельным вместилищам, и все реки, и вместилища вод в горах и во всей земле, и
все озера, и всякую росу земную, и семя, которое сеется по роду своему, и все,
что употребляется в пищу, и плодовые и лесные деревья, и сад Едем для веселия.
Все эти четыре великие творения Он сотворил в третий день.

И в четвертый день Он сотворил солнце, и луну, и звезды, и поставил их на
тверди небесной, чтобы они светили на всю землю, и повелел им управлять днем и
ночью, и разделять между светом и между тьмою. И Бог сделал солнце великим
знамением на земле для дней, и суббот, и годов, и юбилеев, и для всех времен
года, и повелел ему разделять между светом и между тьмою, и
предназначил его для роста, чтобы росло все, что прозябает и
произрастает на земле. Эти три рода творения Он создал в четвертый
день.

И в пятый день Он сотворил больших морских животных в глубинах вод,~---
ибо они были созданы Его рукою прежде всего,~--- всякую плоть, и все, что
движется в водах, рыб, и все, что летает, птиц и весь их род. И солнце взошло
над ними для развития, и над всем, что существует на земле, и над всем, что
прозябает из земли, и над всеми плодовыми деревьями, и над всякою плотью. Все
эти три рода Он сотворил именно в пятый день.

И в шестой день Он создал всех зверей земных, и всякий скот, и все, что
движется на земле. И после всего этого Он сотворил человека, одного, мужа и
жену сотворил их, и поставил его владыкою над всем, что на земле и что в морях,
и над тем, что летает, и над зверями, и над скотом, и над всем, что движется на
земле, и над всею землею: надо всем этим Он сделал его господином. И эти четыре
рода творений Он сотворил в шестой день.

И было всего сотворено в шесть дней двадцать два рода. И Он закончил все
Свои произведения в шестой день~--- все, что на небе, и на земле, и в морях,
и в долинах, во свете и во тьме и всюду. И Он дал нам (Ангелам) великое
знамение~--- день субботний, чтобы мы в продолжение шести дней делали дела и
в седьмой день соблюдали субботу ото всех дел, все Ангелы лица и все Ангелы
прославления. Нам, этим двум великим родам, сказал Он, чтобы мы хранили с Ним
субботу на небе и на земле. И Он сказал нам: <<Вот Я выделю Себе народ из
среды всех народов, чтобы и они (он?) праздновали субботу; и Я освящу
его Себе в Свой народ, и благословлю его, как Я освятил день субботний и
посвятил их (субботы) Себе; так благословлю Я его; и они будут Моим народом, и
Я буду их Богом. И Я избрал семя Иакова между всеми из тех, которых Я увидел, и
написал Его у Себя перворожденным сыном, и освятил его для Себя навсегда и
навечно. И Я возвещу им о субботнем дне, чтобы они хранили в него субботу ото
всех дел своих>>. Так положил Он знамение в нем, дабы и они праздновали с
нами субботу в седьмой день, чтобы есть и пить, и прославлять Того, Кто
сотворил все, как и Он благословил сие и освятил Себя для Своего народа, чтобы
это было явлено пред всеми народами и чтобы они (потомки Иакова?) одинаково с
нами праздновали субботу. И Он установил, чтобы Его повеления возносились пред
Него, как благовоние, которое было бы приятно Ему, во все дни двадцати двух
глав людей от Адама до Иакова. И двадцать два рода произведений были сотворены
до сего седьмого дня. Этот благословлен и освящен, и тот (Иаков) также
благословлен и освящен. И этот вместе с тем служит к освящению и благословению.
И сему (Иакову и его потомкам) даровано было, чтобы они были всегда
благословенными и святыми в свидетельстве и законе, как прежде седьмой день Он
освятил и благословил быть седьмым днем (т.е. субботою). Он сотворил небо
и землю и все, что создано в шесть дней, и Господь установил святой праздник
для всех Своих тварей. Посему Он дал повеление относительно него всем Своим
творениям, что нарушители седьмого дня должны умереть: если кто
осквернит его, тот да умрет смертию. И ты с своей стороны скажи сынам
Израилевым, чтобы они соблюдали этот день, святили его, и никакого дела не
делали в него, и не оскверняли его; ибо он святее, нежели все другие
дни, и всякий, кто оскверняет его, должен умереть смертию; и всякий, кто делает
в него какое-либо дело, должен умереть смертию, навсегда и навечно чтобы сыны
Израиля соблюдали этот день в своих родах и не были истреблены на земле. Ибо
это святой день и благословенный день, и всякий человек, соблюдающий его и
празднующий в него субботу от всякого своего дела, будет свят и благословен
всегда, как мы (Ангелы). И ты возвести и изъясни сынам Израилевым закон этого
дня, чтобы они праздновали в него субботу и не забывали бы его в заблуждении
своего сердца, чтобы они не делали в него ничего из своих нужд, не приготовляли
в него чего-либо из пищи и питья, ни воды не черпали, ни какой-либо ноши не
вносили и не выносили в него чрез свои врата, если бы им не пришлось
приготовить себе чего-нибудь в течение шести дней в своих домах. И они не
должны ничего выносить и вносить в этот день из одного дома в другой, ибо он
святее и благословеннее всех юбилейных дней юбилейного года. В него мы
праздновали субботу, прежде чем кому-либо из смертных сделалось известным
празднование в него на земле субботы. И Творец всех вещей благословил его; но
Он освятил не всех людей и не все народы праздновать в него субботу, а
только Израиля; ему только предназначил Он есть и пить, и праздновать в него
субботу на земле. И благословил его Творец всех вещей, создавший этот день к
благословению, и к освящению, и к прославлению пред всеми другими днями.
Этот закон и свидетельство даны сынам Израилевым как вечный закон для
всех родов.

\vs Jub 3:1
И в шесть дней второй субботы (седмицы) мы по повелению Господа привели к
Адаму всех зверей, и всякий скот, и всех птиц, и все, что движется на земле, и
все, что движется в воде, по их родам и видам, именно~--- зверей в первый
день, скот во второй, птиц в третий, все, что движется по земле, в четвертый,
все, что движется в воде, в пятый день; и Адам дал им всем имена, и как он
назвал их, так и было им имя. И в продолжение этих пяти дней Адам видел все
это, самца и самку в каждом роде, что есть на земле, между тем как только он
был одинок, и он не мог найти себе никого подобного, кто был бы ему
помощником.

И Господь сказал мне: <<Нехорошо быть человеку одному: создадим ему
помощника, подобного ему>>. И Господь Бог наш навел на него усыпление,
чтобы он заснул. И Он взял для жены одно из ребер его, как вещество для жены, и
создал плоть вместо него; и Он создал жену и пробудил Адама от сна. И когда
Адам пробудился, поднялся в шестой день, и взял ее к себе, и узнал ее, и сказал
ей: <<Это кость от моей кости, и плоть от моей плоти; она назовется моею
женою; ибо от своего мужа она взята. Посему муж и жена да будут одно, и посему
он оставит отца своего и матерь свою и прилепится к жене своей, и будут они
одною плотью>>. И в первую седмицу был создан Адам и его жена, и во
вторую седмицу Он (Бог) поставил ее пред ним. И ради сего дана
заповедь~--- семь дней для мальчика, а для девочки дважды семь дней пребывать
женщине в ее нечистоте.

И после того как Адам пробыл сорок дней в стране, где он был сотворен, мы
привели его в сад Едем. Ради сего на небесных скрижалях рожденных предписано:
<<Если она родила дитя мужеского пола, то должна оставаться в своей
нечистоте семь дней, соответственно первой неделе, и тридцать три дня должна
оставаться в крови своего очищения, и не должна прикасаться ни к чему святому,
ни вступать в святилище, пока она не окончит этих дней, та, которая родила
мужеского пола. А родившая младенца женского пола должна две недели,
соответственно двум первым неделям, пребывать в своей нечистоте и шестьдесят
шесть дней в крови очищения; и будет для нее всего восемьдесят дней. И когда
жена (Ева) окончила восемьдесят дней, мы привели ее в сад Едем, ибо он свят во
всей земле, и каждое дерево, которое насаждено в нем, свято. Ради сего для
рождающей мальчика или девочку установлен закон этих дней, чтобы она не
прикасалась ни к чему святому, ни в святилище не входила, пока не окончатся эти
дни для мальчика или девочки. Это закон и свидетельство, написанное для
израильтян, чтобы они соблюдали это всегда.

И в начале первого юбилея Адам и жена его были в саду Едем семь лет,
возделывая и храня его. И мы дали ему занятие и научили его все видимое
употреблять в дело, и он трудился. Он же был наг, не зная сего и не стыдясь. И
он охранял сад от птиц, и зверей, и скота, и собирал плоды сада и ел, и
сберегал остаток для себя и своей жены, и делал запас.

И по истечении семи лет, которые он там провел, ровно семи лет, во второй
месяц в семнадцатый день его пришел змий и приблизился к жене. И сказал змий
жене: <<Разве Бог запретил вам все плоды деревьев, которые в раю, чтобы вы
не ели от них?>> И она сказала ему: <<От всех плодов деревьев, которые
в раю, сказал нам Бог, можно нам есть, но от плода дерева, которое в средине
рая, сказал нам Бог, мы не должны есть и прикасаться к нему, дабы нам не
умереть>>. И змий сказал жене: <<Вы не умрете смертию; напротив, Бог
знает, что в день, в который вы вкусите от него, откроются очи ваши, и
вы будете как боги и будете знать доброе и злое>>. И вот жена увидела
дерево, что оно было хорошо и приятно для глаза, и его плод хорош для пищи,
тотчас взяла его и ела. И она первая покрыла свою срамоту смоковничным листом;
и она дала его (плод) Адаму, и он ел, и его глаза открылись, и он увидел, что
был наг, и взял смоковничных листьев, и сшил их, и сделал себе препоясание, и
покрыл свою срамоту. И Господь проклял змия и разгневался на него навсегда. И
на жену также разгневался Он, так как она послушалась голоса змия и ела. И Он
сказал ей: <<Я умножу твои болезни и твое страдание; с болезнями ты будешь
рождать детей, и у своего мужа будешь находить свою защиту, и он будет
твоим господином>>. И Адаму Он сказал: <<Так как ты послушался гласа
жены своей и ел от того дерева, от которого Я запретил тебе есть, то земля
будет проклята из-за тебя; тернии и волчцы будут произрастать тебе, и свой хлеб
ты будешь есть в поте лица твоего, пока не возвратишься в землю, из которой ты
взят. Ибо ты на земле и в землю возвратишься>>. И Он сделал им кожаные
одежды и одел их ими, и изгнал их из рая Едем. И в тот день, когда Адам вышел
из рая Едем, он принес в приятное благоухание жертву благовонную: ладан, и
халван, и стакти, и Сенегал, утром с восходом солнца, в день, когда он покрыл
свою срамоту. И в тот день заключились уста всех зверей, и скота, и птиц, и
того, что ходит (ногами), и того, что движется, так что они не могли более
говорить, ибо до сего все они говорили между собою одними устами и одним
языком. И Он изгнал из сада Едем всякую плоть, которая была в саду Едем; и
рассеялась всякая плоть по своим породам и видам в места, которые для них были
созданы (удобны). Только Адаму Он повелел покрывать свою срамоту~--- ему
одному между всеми зверями и скотом. Ради сего Он на скрижалях повелел всем,
знающим правду закона, покрывать свою срамоту и не обнажаться, как обнажаются
язычники.

И в новолуние четвертого месяца Адам и его жена вышли из рая Едем, и жили в
земле Елдад, в той земле, где они были созданы. И Адам дал имя жене своей Ева.
И у них не было ни одного сына до первого юбилейного года. И после сего он
познал ее. Он же обрабатывал свою землю, как был научен в саду Едем.

\vs Jub 4:1
И в третью седмину во второй юбилей родила она Каина, и в четвертую родила
Авеля, и в пятую родила дочь свою Аван. И в первую седмину третьего юбилея Каин
убил Авеля, ибо Он (Бог) принял дар от руки его милостиво, а от руки Каина
жертву плодов не милостиво. И он убил его на поле, и его кровь вопиет от земли
к небу, восклицая, что он убит. И Бог наказал Каина за Авеля, которого он убил,
и сделал его проклятым на земле за кровь его брата, и проклял его на
земле, ради чего на небесных скрижалях написано так: <<Да будет проклят,
кто убивает своего ближнего по злобе, и все видящие это должны говорить: да
будет так! И человек, который видит и не объявит сего, да будет проклят, как
он!>> Ради сего мы являемся к Господу, Богу нашему, возвещать все грехи,
которые совершаются на небе и на земле, во свете и во тьме, и всюду.

И Адам и его жена скорбели об Авеле четыре седмины. И в четвертый год пятой
седмины он утешился, и опять познал жену свою, и она родила ему сына, и он
нарек ему имя~--- Сиф; ибо он сказал: <<Господь восставил нам другое
семя на земле вместо Авеля, ибо Каин убил его>>. В шестую седмину он родил
свою дочь Азуру. И Каин взял себе свою сестру Аван в жены, и она родила ему
Еноха в конце четвертого юбилея. И в первый год первой седмины пятого, юбилея
были построены дома на земле, и Каин построил город и назвал его по имени сына
своего Енох. И Адам познал свою жену Еву, и она родила еще девять сыновей. И в
пятую седмину сего юбилея Сиф взял себе в жены свою сестру Азуру, и она родила
ему в четвертый год Эноса. И он первый начал призывать имя Господне на земле. И
в седьмой юбилей в третью седмину Энос взял свою сестру Ноаму в жены, и она
родила ему сына в третий год пятой седмины, и он нарек ему имя Каинан. И в
восьмой юбилей в конце его Каинан взял свою сестру Муалелиту в жены, и
она родила ему сына в девятый юбилей, в первую седмину, в третий год той
седмины, и он нарек ему имя Малалел. И во вторую седмину десятого юбилея
Малалел взял себе в жены Дину, дочь Боракиэла, дочь сестры его отца,~---
себе в жены,~--- и она родила ему сына в третью седмину в шестой год, и он
нарек ему имя Иаред; ибо в его дни сошли на землю Ангелы Господни, которые
назывались стражами, чтобы научить сынов человеческих совершать на земле правду
и справедливость.

И в одиннадцатый юбилей Иаред взял себе жену, по имени Барака, дочь
Разузаила, дочь сестры его отца, в четвертую седмину сего юбилея. И она родила
ему сына в пятую седмину, в четвертый год юбилея, и он нарек ему имя
Енох. Он был первый из сынов человеческих, рожденных на земле, который научился
письму, и знанию, и мудрости; и он описал знамения неба по порядку их месяцев в
книге, чтобы сыны человеческие могли знать время годов в порядке их отдельных
месяцев. Он прежде всех записал свидетельство, и дал сынам человеческим
свидетельство о родах земли, и изъяснил им седмины юбилеев, и возвестил им дни
годов, и распределил в порядке месяцы, и изъяснил субботние годы, как мы ему
возвестили их. И что было, и что будет, он видел в своем сне, как произойдет
это с сынами детей человеческих в их поколениях до дня суда. Все видел и узнал
он, и записал во свидетельство, и положил сие, как свидетельство, на земле для
всех сынов детей человеческих и для их родов. И в двенадцатый юбилей в седьмую
седмину взял он себе жену именем Адни, дочь Даниала, дочь сестры его отца. И в
шестой год этой седмины она родила ему сына, и он нарек ему имя Мефусалаг. И
вот он был с Ангелами Божиими в продолжение шести лет, и они показали ему все,
что на земле и на небесах, господство солнца; и он записал все. И он дал
свидетельство стражам, которые согрешили с дочерьми человеческими. Ибо они
стали смешиваться, чтобы оскверняться с дочерьми человеческими. И Енох дал
свидетельство против всех них. И он был взят из среды сынов детей человеческих,
и мы привели его в рай Едем к славе и почести. И вот здесь он записывает суд и
вечное наказание, и всякое зло сынов детей человеческих. И ради него Он (Бог)
послал потоп на землю; ибо он был поставлен в знамение, и чтобы дать
свидетельство против всех сынов детей человеческих, чтобы объявлять все деяния
родов до дня суда. И он принес в жертву курение..., которое было приятно Богу,
на горе полудня; ибо четыре места Божий существуют на земле: рай Едем, и гора
востока, и эта гора, на которой ты теперь,~--- гора Синай, и гора Сион,
которая будет освящена в новом творении для освящения земли; чрез нее земля
освятится от всей своей вины и нечистоты навсегда и навечно.

И в четырнадцатый юбилей взял Мефусалаг Адину, дочь Азраела, дочь сестры его
отца, себе в жены, в третью седмину в первый год, и он родил сына и нарек ему
имя Ламех. И в пятнадцатый юбилей в третью седмину взял себе Ламех жену, по
имени Битанос, дочь Баракела, дочь сестры его отца, себе в жены; и в эту
седмину она родила ему сына, и он назвал его Ноем, говоря: <<Он утешит меня
о всех моих трудах и о земле, которую проклял Бог>>.

И в конце девятнадцатого юбилея в седьмую седмину в шестой год ее умер Адам,
и все сыны его погребли его в стране, где он был сотворен. И он был погребен
прежде всех в земле. И он жил на семьдесят лет меньше тысячи лет, ибо тысяча
лет как один день по небесному свидетельству. Ради сего о древе познания
написано: <<В день, когда вы вкусите от него, вы умрете>>. Посему он не
окончил годы этого дня, но умер в этот день.

В конце этого юбилея был убит Каин после него (Адама) в том же году. Его дом
упал на него, и он умер посреди своего дома, и погиб под его камнями. Ибо
камнем он убил Авеля, и камнем был убит по праведному суду. Сего ради на
небесных скрижалях предписано: <<Орудием, которым муж убил своего ближнего,
должен быть и он убит; как ранил он его, так должны они сделать и
ему>>.

И в двадцать пятый юбилей Ной взял себе жену, по имени Емзараг, дочь
Ракиела, дочь его сестры (?), себе в жены, в первый год в пятую седмину. И в
третий год ее она родила ему Сима, и в пятый год родила ему Хама, и в первый
год в шестую седмину родила ему Иафета.

\vs Jub 5:1
И случилось, когда сыны детей человеческих начали умножаться на поверхности
всей земли и у них родились дочери, Ангелы Господни увидели в один год этого
юбилея, что они были прекрасны на вид. И они взяли их себе в жены, выбрав их из
всех; и они родили им сыновей, которые сделались исполинами. И неправда
усилилась на земле, и всякая плоть извратила свой путь, от людей до скота, и до
зверей, и до птиц, и до всего, что ходит по земле. Все извратили свой путь и
свой порядок, и начали пожирать друг друга. И неправда усилилась на земле, и
все помышления разума сынов человеческих сделались столь злыми во всякое время.
И Господь воззрел на землю, и вот она извратилась, и всякая плоть извратила
свой порядок, и они совершали всякое зло пред Его очами~--- все, что было на
земле. И Он сказал, что Он уничтожит людей и всякую плоть, которую Он сотворил
на земле.

И только Ной обрел милость пред Его очами. И на Ангелов Своих, которых Он
посылал на землю, Он весьма разгневался, так что решил истребить их. И Он
сказал нам, чтобы мы связали их в пропастях земли. И вот они были связаны в них
и разобщены. И относительно детей их вышло повеление от Его лица, чтобы
поразить их мечом и умертвить их под небом. И Он сказал: <<Моему духу не
вечно пребывать на людях, ибо они плоть, и дней их пусть будет сто двадцать
лет!>> И Он послал Свой меч в среду их, чтобы они умертвили друг друга. И
они начали убивать друг друга, пока не пали все от меча и не были уничтожены с
земли на глазах своих отцов. После сего они (их отцы) были связаны в пропастях
земных до дня великого суда, когда придет наказание на всех, извративших свои
пути и свои дела пред Господом. И Он уничтожил все их пристанища, и ни один из
них не остался, которого Он не уничтожил бы за все их зло. И Он соделал для
всех Своих творений новое и праведное естество, чтобы они не согрешали вовек по
всему своему естеству, и каждый был бы праведен чрез свою отрасль. И наказание
всех их определено и записано на небесных скрижалях без неправды. И все,
преступившие путь, который им определен, чтобы ходить по нему, если не ходят по
нему, то наказание написано для каждого естества и для каждого рода. И ничто
не избежит его, что на небе и на земле, во свете и во тьме, в царстве
мертвых, и в пропасти, и в мрачном месте. Все наказания их определены, и
записаны, и начертаны для всех. Великого Он будет судить по его величию, и
малого~--- по его малости, и каждого отдельно~--- по его пути. И Он не
примет никаких даров, ибо говорит, что будет совершать суд над каждым отдельно.
И если бы кто-нибудь дал Ему все, что есть на земле, то Он не обратит лица
Своего и не примет этого от него: ибо Он Судия. И о сынах Израиля написано и
определено: если они обратятся к Нему в справедливости, то Он отпустит им
всякую вину и все грехи их простит. Написано и определено, что милосердие будет
оказано всем, которые обратятся от всякого своего злодеяния, однажды в год. Но
всем тем, которые свои пути и свое стремление извратили пред потопом, не дано
снисхождения, кроме только Ноя, ибо Господь призрел на лице его ради
сыновей, которых Он спас из-за него от потопа. Ибо сердце его было праведно во
всех путях его, как было повелено ему. И он ничего не преступил из того, что
было ему предписано.

И Господь сказал: <<Да будет истреблено все, что на суше, от всякого
скота до диких зверей и птиц и до всего, что движется на земле!>> И Он
повелел Ною сделать себе ковчег, чтобы спастись в нем от потопа. И Ной сделал
ковчег для всех тварей, как Он повелел ему, в (двадцать седьмой)
юбилей, в пятую седмину, в пятый год. И он вошел в него в шестой год ее, в
другой месяц, в новолуние другого месяца. До шестнадцатого дня его вошел в
ковчег он и все, что мы привели к нему. И Господь затворил его снаружи в
семнадцатый день вечером. И Бог открыл семь окон небесных, чтобы они
изливали воду с неба на землю в продолжение сорока дней и сорока ночей. И
источники бездны также изливали воду, так что весь мир наполнился водою. И
поднялись воды на земле: на пятнадцать локтей поднялась вода над всеми высокими
горами. И ковчег носился над землею и плавал на поверхности воды. И вода стояла
на поверхности земли пять месяцев, сто пятьдесят дней. И он (ковчег) пришел и
остановился на вершине Любара, одной из гор Арарата. И в четвертый месяц
замкнулись источники великой бездны и хляби небесные затворились. И в новолуние
седьмого месяца все отверстия пропастей земли открылись, и вода стала стекать в
преисподнюю бездну. И в новолуние десятого месяца показались вершины гор. И в
новолуние первого месяца обнаружилась земля, и вода стекла с земли в пятую
седмину в седьмой год ее. И в семнадцатый день второго месяца просохла земля. И
в двадцать седьмой день его он отворил ковчег и выпустил из него зверей,
птиц и что двигалось.

\vs Jub 6:1
И в новолуние третьего месяца вышел он из ковчега, и устроил жертвенник на
этой горе, и показался на земле. И он взял молодого козла и пролил кровь его в
искупление за всю вину земли, ибо все, что существовало на ней, было
истреблено, кроме тех, которые были в ковчеге с Ноем. И он положил тук его на
жертвенник, и взял тельца, и овна, и овцу, и козлов, и соли, и горлицу, и
молодого голубя, и принес всесожжение на жертвеннике, и примешал к сему
испеченные в масле жертвенные плоды, и возлил кровь и вино, и положил на все
фимиам, и вознес приятное благоухание, которое было приятно Господу. И Господь
обонял приятное благоухание и заключил с ним завет, что не придет более потоп,
который погубил бы землю, что во все дни земли сеяние и жатва не прекратятся,
мороз и жар, лето и зима, день и ночь не изменят своего порядка и не
прекратятся. <<И вы растите и плодитесь на земле, и размножайтесь на ней, и
будьте во благословение на ней! Ваш страх и трепет Я положу на все, что на
земле и в море. И вот Я всех диких зверей, и всякий скот, и все, что летает, и
все, что движется на земле, и рыб в водах, и все~--- дал вам в пищу, как
зелень травную дал Я вам все, чтобы вы ели. Только плоть, в которой живая душа,
вы не должны вкушать с кровию, ибо душа всякой плоти есть кровь, да не взыщется
кровь вашей души. От каждого человека, от каждого Я взыщу кровь человека; кто
проливает человеческую кровь, того кровь пусть прольется от руки человеческой,
ибо по образу Божию Он сотворил Адама. А вы раститесь и умножайтесь на
земле!>> И дети его поклялись, что они не будут есть крови, которая в
какой-либо плоти. И он заключил завет пред Господом Богом навечно, на все роды
земли, в этом месяце.

Ради сего Он говорил с тобою, чтобы и ты с сынами Израиля в этом месяце на
горе заключил завет с клятвою и окропил их кровию ради всех слов завета,
который Господь заключил с ними на все время. И это свидетельство предписано
им, дабы и вы соблюдали это во все дни, чтобы вам никогда не есть крови
зверей (...). И человек, который ест кровь дикого зверя, и скота, и птиц, пока
стоит земля, будет истреблен на земле~--- он и его семя. И Он повелел сынам
Израиля не есть крови, дабы они и их семя существовали пред Господом, Богом
нашим, всегда. И для сего закона нет конца времени; вечно они должны соблюдать
его вместе с потомками, чтобы непрерывно кровию за вас испрашивалось прощение
пред жертвенником; ежедневно, утром и вечером, должно испрашивать у Господа
прощение за них, чтобы они соблюдали это и не были истреблены.

И Он дал Ною и его сыновьям знамение, что не придет опять потоп на землю. Он
поставил Свою радугу в облаках в знамение вечного завета, что потоп более не
придет на землю для истребления ее, пока стоит земля. Посему определено и
написано на небесных скрижалях, чтобы они соблюдали праздник седмиц в этом
месяце однажды в год, чтобы возобновлять завет каждый год. И всего времени, в
течение которого праздновался этот праздник на небе, от дней творения до дней
Ноя было двадцать семо юбилеев и пять седмин. И Ной праздновал его в
продолжение семи юбилеев и одной седмины до дня своей смерти; а сыны Ноя
оскверняли его до дней Авраама и ели кровь. Только Авраам соблюдал его, и
сыновья его Исаак и Иаков соблюдали его до твоих дней. И в твои дни сыны
Израиля забыли его, пока я не обновил их при этой горе. И ты сделай также
повеление сынам Израиля, чтобы они соблюдали этот праздник во всех своих родах,
как закон для себя. Один день в году в этом месяце пусть празднуют они
праздник. Ибо это праздник седмиц, и это праздник первого творения; праздник
этот имеет двоякого рода значение и установлен для двух родов
сообразно тому, что об этом написано и начертано. Ибо я записал это в книге
первого закона, в той, которую я написал тебе, да празднуешь ты всякий раз по
одному дню в году. Я изъяснил тебе и жертвенные дары в него, дабы они хранились
в памяти, и сыны Израиля праздновали бы его в своих родах в этом месяце по
одному дню в год.

И новолуния первого, четвертого, седьмого и десятого месяцев суть дни
воспоминания и праздничные дни в четыре времени года. Они записаны и
установлены к ежегодному свидетельству. И Ной назначил их себе в праздники для
будущих родов, чтобы иметь в них праздник воспоминания. В новолуние первого
месяца было сказано ему, чтобы он сделал ковчег; и в этот день земля стала
сухою, и он отворил ковчег и увидел землю. В новолуние четвертого месяца
заключилось отверстие преисподней глубины земли. И в новолуние седьмого месяца
все отверстия и глубины бездны открылись и воды стали стекать в них. И в
новолуние десятого месяца показались вершины гор, и Ной возрадовался. Посему он
определил их себе в праздники воспоминания навек, и так они утверждены. И они
внесены на небесные скрижали: двенадцать (?) суббот имеет каждое из них, от
одного новолуния до другого (т.е. от первого до четвертого) идет
их воспоминание, от первого до второго, от второго до третьего, от третьего до
четвертого. И всех дней, которые предписаны, пятьдесят две субботы дней; этим
весь год исполняется. Так начертано и установлено на небесных скрижалях, и не
бывает пропуска, ежегодно, из года в год.

И ты скажи сынам Израилевым, чтобы они содержали годы по сему числу, триста
шестьдесят четыре дня: и это будет полный год, и определенное время дней и
праздники года не будут извращены; ибо все совершается в нем (в году) согласно
тому, что утверждено относительно сего, и они не должны опускать ни одного дня
и не должны нарушать ни одного праздника. Если же они преступят и не будут
поступать по его повелениям, то они враз все определенные времена извратят и
годы будут подвинуты с мест. И они будут преступать свой порядок; и все сыны
Израиля забудут путь годов, и не обретут более, и забудут новолуние и его время
и субботы, и заблудятся относительно всего порядка годов. Ибо я знаю это и
отныне возвещаю тебе сие, и это не по моему разумению, но так, как написано в
книге у меня, и на небесных скрижалях определено деление дней, ибо они не
должны забывать праздников завета, и не должны соблюдать праздников язычников,
и ходить по их заблуждениям и по их мыслям. И это будет с людьми, которые будут
наблюдать над луною, они именно извратят времена, и каждый год уйдет вперед на
десять дней. И из-за этого они извратят будущий год, и сделают мнимый день за
день свидетельства и нечистый день за день праздничный. И каждый будет
смешивать святой день с нечистым и нечистый со святым; ибо они будут
заблуждаться в месяцах, и субботах, и праздниках, и юбилейных годах. Посему я
повелеваю и подтверждаю тебе, чтобы ты засвидетельствовал им,~--- так как
после твоей смерти твои дети (?) извратят это,~--- что они должны считать
год только в триста шестьдесят четыре дня. Из-за сего они будут заблуждаться в
новолунии, и субботе, и в дне торжества и праздника и будут всегда есть плоть в
крови.

\vs Jub 7:1
И в седьмую седмину в первый год ее в этом
юбилее Ной насадил виноградные деревья на горе,
на которой остановился ковчег, называемой Лубар,
на одной из гор Арарата. И они принесли плод на
четвертом году. И он берег свои плоды, и снял их в
том году в седьмом месяце, и сделал из них вино, и
влил его в сосуд, и держал его даже до пятого года,
до первого дня, т.е. до новолуния первого месяца.
И он принес всесожжение для Господа, молодого
тельца, и овна, и семь однолетних агнцев, и
молодого козла, чтобы испросить прощение себе и
своим сыновьям. И он приготовил прежде всего
козла, и принес его кровь к (...) алтаря, который он
сделал, и весь тук его положил на алтарь, где он
приготовил всесожжение, и от тельца, и от овна, и
от агнцев он взял все мясо на жертвенник и
возложил на него все плодовые жертвы, какие
принадлежали к ней, смешанные с елеем. Тогда он
возлил прежде всего вино в огонь на жертвеннике,
и положил фимиам на жертвенник, и вознес доброе,
приятное благоухание, чтобы оно вознеслось пред
Господа, Бога его. И он возрадовался, и испил от
этого вина~--- он и его дети, исполненные радости. И
настал вечер; тогда он вошел в свой шатер, и лег
опьяненный и заснул, обнажился во время сна в
своем шатре. И Хам увидел своего отца Ноя нагого,
и вышел, и рассказал своим двум братьям. И Сим
взял свою одежду и поднялся вместе с Иафетом, и
они сняли свою одежду с своих плеч, обратив лицо
назад, и покрыли срамоту своего отца, обративши лицо
назад. И когда Ной пробудился от своего сна, то
узнал все, что сделал с ним его младший сын. И он
проклял его сына и сказал: <<Проклят Ханаан,
послушнейшим рабом да будет он своим братьям!>>
И он благословил Сима: <<Да будет прославлен
Господь Бог Сима, и Ханаан да будет его рабом! Да
распространит Господь Иафета, и да живет Господь
в жилище Сима, и Ханаан будет его рабом!>>

И Хам узнал, что его отец проклял его сына, и
отделился от своего отца, он и его сыновья с ним, в
Хуш, и Мистрем, и Фуд, и Ханаан. И он выстроил себе
город и назвал его по имени своей жены
Неелатамек. И Иафет увидел это, и позавидовал
своему брату, и также выстроил город, и назвал его
по имени своей жены Адотанелек. Но Сим жил со
своим отцом Ноем, и выстроил город близ города
своего отца при горе, и он также назвал его по
имени своей жены Седукательбаб. Вот три города
близ горы Лубар: Седукательбаб пред горою на ее
восточной стороне, и Неелтамаук на южной стороне,
Адатанезес (?) к западу. И вот сыновья Сима: Елам,
Асур, Арфаскад [...].

В двадцать восьмой юбилей Ной начал учить своих
внуков всем постановлениям и заповедям, которые
он знал, и закону; и дал свидетельство своим
сыновьям, чтобы они делали справедливость, и
покрывали срамоту своего тела, и прославляли
Того, Кто сотворил их, и почитали отца и матерь,
чтобы любили друг друга и ограждали свои души от
всякого любодеяния и нечистоты и от всякой
несправедливости. Ибо за эти три вины пришел на
землю потоп, именно~--- за любодеяние, которым
стражи вопреки предписаниям их закона блудили с
дочерьми человеческими и взяли себе жен из всех,
которые им понравились: они положили начало
нечистоте. И их сыны, Нефилимы и все другие стали
разногласить друг с другом, и один пожирал
другого: исполин убивал Нефила, и Нефил убивал
Елъйо, и Елъйо сынов человеческих, и один человек
другого. И каждый был [...], чтобы делать неправду и
проливать неповинную кровь; и земля наполнилась
нечестием. И за ними последовали все дикие звери,
и птицы, и что движется, и что ходит по земле; и
пролилось много крови на земле. И все помышление
и стремление людей было пустое и злое. И Господь
истребил все с поверхности земли; за лукавство их
дел и за кровь, которую они пролили на земле, Он
истребил все. И я, и вы, мои сыны, и все, что с нами
вошло в кочег, сохранилось целым. И вот я вижу
прежде всего ваши дела, как вы ходите не в
справедливости, но начали ходить по пути
развращения, и отделяться друг от друга, и быть
завистливыми Друг к другу, один к другому, и как
вы не единодушны, мои сыны, брат с своим братом.
Ибо я вижу, что демоны начали обольщать вас и
ваших сыновей. И теперь я страшусь за вас, чтобы
вы, когда я умру, не стали проливать на земле
кровь человеческую, а чтобы и вы не были
истреблены с поверхности земли. Ибо каждый, кто
проливает человеческую кровь, и каждый, кто ест
кровь какой-либо плоти, будет истреблен из среды
всех с лица земли, и ни одного человека не
останется на земле, который ест кровь и проливает
кровь на земле; и не останется у него семени и
потомства под небом; но они пойдут в царство
мертвых и сойдут в место осуждения; все они
погрузятся в мрак бездны через мучительную
смерть~--- каждый из вас, кто от всякой крови не
принесет за себя для очищения; т.е. как только вы
заколете зверя, или скот, или что летает на земле,
то делайте доброе дело за себя кровию, где только
она проливается на земле. И никто из вас не должен
есть плоть с кровию; удерживайте, чтобы не ели
кровь пред вами. Закапывайте кровь, ибо так было
заповедано мне; я свидетельствую о сем как вам,
так и вашим сыновьям, вместе со всякою плотию. И
не ешьте душу с плотию, да не взыщется ваша кровь,
которая есть ваша душа, от всякой плоти, которая
проливает ее на земле. Ибо земля будет нечиста от
крови со времени ее пролития на ней, но только через
кровь того, кто пролил ее, земля будет чистою в
продолжение всех своих родов. И теперь, мои
сыновья, послушайте меня, творите правду и
справедливость, чтобы вы были насаждены в
справедливости на всем лице земли, и да
вознесется ваша слава к Богу моему, Который спас
меня от потопа. И вот вы пойдете и выстроите себе
города, и разведете в них всякие растения,
которые на земле. И теперь от всех плодовых
деревьев в продолжение трех лет не должен
собираться плод ни от какого дерева, чтобы
есть его, и в четвертый год их плод должен быть
освящен, и начаток плодов [...] должен быть
принесен пред Господа, Всевышнего, Который
создал небо и землю и все, чтобы с лучшим начатком
плодов принести вино и елей на жертвенник
Господа, который Он изберет; и что останется,
слуги дома Божия должны съесть пред
жертвенником, который Он изберет. И в пятый год
сделайте обнародование, чтобы вы обнародовали
это в справедливости и праведности, и вы будете
праведными, и все ваши растения умножатся. Ибо
так заповедал Енох, отец вашего отца Мефусалага,
своему сыну, и Мефусалаг своему сыну Ламеху, и
Ламех заповедал мне все, что заповедали ему отцы
его. И я также заповедую вам это, мои сыны, как
Енох заповедал своему сыну в его первый юбилей,
когда он был еще жив, седьмой в своем роде, он
заповедовал и свидетельствовал это своему сыну и
сыновьям его сыновей до дня своей смерти.

\vs Jub 8:1
И в двадцать девятый юбилей в первую седмину в
первый год Арфаскад взял себе жену, по имени
Разуйю, дочь Сусаны, дочери Елама, себе в жены, и
она родила ему сына в третий год этой седмины, и
он нарек ему имя Каинам. И его сын возрос, и его
отец научил его писанию, и он пошел искать себе
место, где бы основать себе город. И он нашел
надписание, которое праотцы начертали на скале; и
он прочитал, что было на ней, и перевел это, и
нашел, что на ней было знание, которому научили
стражи, о колесницах солнца, и луны, и звезд, и обо
всех замениях неба. И он записал это, но ничего о
сем не рассказал, ибо он боялся рассказать о сем
Ною, чтобы он не разгневался на него за это.

И в тридцатый юбилей во вторую седмину в первый
год ее взял он себе жену, по имени Мелку, дочь
Абадая, сына Иафета. И в четвертый год она родила
ему сына, и он нарек ему имя Сала, ибо сказал: <<Я
отпущен>>. В четвертый год родился Сала, и он
возрос и взял себе жену по имени Муак, дочь
Кеседа, брата его отца, себе в жены. И в тридцать
первый юбилей в пятую седмину в первый год она
родила ему сына [...], и он нарек ему имя Ебор. И он
взял ему жену по имени Ацурад, дочь Неброда, и
именно в тридцать второй юбилей в седьмую
седмину в третий год. И в шестой год она родила
ему сына, и он нарек ему имя Фалек. Ибо во Дни,
когда он родился, дети Ноя начали делить землю
между собою; и ради этого он нарек ему имя Фалек. А
они делили между собою лукаво, и об этом было
сказано Ною.

И в начале тридцать третьего юбилея они
разделили землю на три части~--- Симу, Хаму и Иафету,
по их наследственным частям в первый год первой
седмины; в то время Ангел, один из нас, посланных к
ним, был при этом. И он (Ной) призвал своих сыновей,
и они приблизились к нему~--- они со своими
сыновьями~--- и он разделил землю по жребию, что
должны были получить три его сына, и они
распростерли руки, и взяли жребий из пазухи
своего отца Ноя.

И на жребий Сима вышла средина земли, которую он
должен был получить как наследие для своих
сыновей и потомков вовек, от средины горы Рафу,
где изливается вода из реки Тоны; и идет его
наследие к западу чрез средину той реки, и идет,
пока не подойдешь к водному бассейну, из которого
выходит эта река, и река эта вытекает и изливает
свою воду в море Миот, и идет эта река до великого
моря. И все, что к югу от него, принадлежит Симу; и
идет его наследие, пока не подойдешь к Карасо,
т.е. до залива перешейка, который смотрит к югу. И
идет его наследие к великому морю и выходит
прямо, пока не подойдешь к западу перешейка,
который смотрит к югу. Ибо это море называется
египетским морским заливом. И оттуда
направляется на юг к устью великого моря до
берегов воды, и идет к Аравии в Офру, и идет, пока
не достигнет воды потока Гигон, и на юг от воды
Гигон, вдоль берега этой реки, и идет на юг, пока
не подойдет к раю Едем на юг от него и на восток от
всей страны Едем [...]; и обращается на восток от
него, и идет, так что подходит к востоку горы,
которая называется Рафа, и спускается к берегу
устья реки Тины. Это наследие досталось по жребию
Симу и его детям, чтобы владеть им (наследием), и
его потомкам до века. И Ной возрадовался, что это
наследие досталось Симу и его детям, и он
размышлял обо всем, что он сказал своими устами в
своем пророчестве, когда говорил: <<Да будет
прославлен Господь, Бог Сима, и да вселится
Господь в жилищах Сима!>> И он знал, что рай Едем
есть святейшая из святынь и жилище Господа и что
гора Сион, центр пустыни, и гора Синай, центр пупа
земли, эти три, одна против другой, созданы были
святынями земли. И он прославил Бога богов,
который вложил речь Господа в уста его [...]. И он
познал, что блаженное и благословенное наследие
Симу и его детям будет уделом для вечных родов;
именно~--- вся страна Эритрейского моря, и вся
страна востока и Индия (и при Эритрейском море) и
горы ее, и вся страна Бала, и вся страна Либанос, и
острова Кафтор, и весь горный хребет Санер и Амар,
и горный хребет Ассур на севере, и вся страна
Елам, Ассур, и Бабель, и Сузан, и Мадай, и вся
страна Арарат, и вся страна по ту сторону горного
хребта Ассур к северу~--- благословенная и обширная
страна, и все, что в ней, очень хорошо.

И Хаму досталась вторая наследственная часть,
по ту сторону Гигона, к югу, направо от рая, и она
идет к югу. И направляется она к огненным горам
и к западу к морю Атил, и направляется на запад,
пока не подойдет к морю бассейна, того, в котором
погибает все, что бы ни стекало, и идет к северу к
пределу Гадит, и идет до берегов моря по ту
сторону великого моря, пока не подойдет к потоку
Гигон, [...], направо от рая Едем. И эта страна,
которая досталась Хаму как наследственная часть,
которой он должен владеть вовек,~--- ему и его
сыновьям в их родах вовек.

И Иафету вышла третья наследственная часть, по
ту сторону реки Тины, к северным странам истока
ее воды, и идет к северо-востоку вся область Лага
и все восточные страны ее; и идет на крайний
север, и простирается до гор Кильта к северу, и к
морю Маук, и идет на восток Гадира, до берегов
моря; и направляется, пока не подойдет к западу
Пары, и обращается назад к Аферагу, и
направляется к востоку, к воде моря Миот, и
направляется вдоль реки Тины, к востоку севера,
пока не подойдет к границе ее воды, к горе Рафы, и
обходит кругом к северу. Это страна, доставшаяся
Иафету и его сыновьям как его наследие, которым
он должен владеть вовек,~--- ему и его сыновьям в их
родах до века: пять великих островов и великая
страна на севере, только она холодная, а страна
Хама жаркая. Но земля Сима не имеет ни жары, ни
мороза, а в ней холод и тепло смешаны.

\vs Jub 9:1
И Хам разделил свою часть между своими
сыновьями. И вышла первая наследственная часть
для всех к востоку и западу Фуду, и запад ее
Ханаану, и к западу моря. И Сим также разделил
между сыновьями. И вышла первая наследственная
часть Еламу и его сыновьям, к востоку от реки
Тигр, пока не подойдешь к стране востока, вся
страна Индия и страна при Эритрейском море, и
воды Дудина, и все горы и Ила (Ела), и вся страна
Сузан, и все, что находится к стороне Фарнака, до
Эритрейского моря и до реки Тины. И Ассуру вышла
вторая наследственная часть: страна Ассур, и
Ниневе, и Синаар, и до границ Индии, и она идет
вверх к реке. И Арфаскаду вышла третья
наследственная часть: вся страна владения
Халдеев, к востоку от Евфрата, вблизи
Эритрейского моря, и все воды пустыни, пока не
придешь к морскому заливу, который смотрит к
Египту, вся страна Либаноса, и Сапера, и Амано, до
соседства с Евфратом. И Араму вышла четвертая
наследственная часть: вся страна Месопотамия,
между Тигром и Евфратом, на север от Халдеев, пока
не придешь к горному хребту Ассур, и все
отдельные страны, до великого моря, и
приближается к востоку к своему брату Ассуру.

И Иафет также разделил страну наследия между
своими сыновьями. И вышел первый жребий Гомеру к
востоку, от севера до реки Тины. И на севере
Магогу досталась вся внутренность севера, пока
не придешь к морю Миот. И Мадаю вышел удел, чтобы
он владел им, на запад от обоих его братьев, до
островов и до границ островов. И Ийоайону вышел
четвертый удел~--- весь остров и острова к Адлуду. И
Толбелу вышел пятый удел, между перешейком,
который подходит к Уда, уделу Луда, до другого
перешейка, внутрь в третий перешеек. И Месеку
вышел шестой удел, и все по ту сторону третьего
перешейка, пока не придешь к востоку Гадира. И
Терасу вышел седьмой удел: он имел великие
острова в средине моря, которые принадлежали к
наследию Хама, и острова Каматури. И детям
Арфаскада вышло блаженное наследие.

Так разделили дети Ноя уделы своим сыновьям
пред Ноем, своим отцом, и он велел им поклясться,
заклиная клятвою каждого, который пытался бы
получить удел, не доставшийся ему по жребию. И все
сказали: <<Да будет так!>> И да будет это так
для них и их сыновей до века, в их родах, до дня
суда, в который Господь Бог будет судить их мечом
и огнем за все лукавство и нечистоту их деяний,
так как они наполнили землю злодеянием,
нечестием, блудодеянием и грехом.

\vs Jub 10:1
И в третью седмину этого юбилея начали нечистые
демоны обольщать сыновей Ноя, чтобы ослеплять их
и губить. И дети Ноя пришли к своему отцу и
рассказали ему о демонах, которые соблазняют
сыновей их сыновей, ослепляют и умерщвляют их. И
он молился Господу Богу своему и сказал: <<Боже
духов всякой плоти, являющий Свое милосердие, и
спасший меня и моих детей от воды потопа, и не
допустивший меня погибнуть, как поступил Ты с
сынами погибели, ибо велика милость Твоя ко мне и
велико Твое милосердие к моей душе: яви милость
Твою на сынах Твоих, не допусти злых духов
господствовать над ними, дабы они не истребили их
от земли! Вот Ты благословил меня и моих сыновей,
чтобы мы возрастали, и умножались, и наполняли
землю. Ты знаешь, как Твои стражи, отцы этих духов,
поступили в мои дни. И этих духов, которые живы,
также заключи и свяжи в месте осуждения, чтобы
они не производили развращения между сынами
Твоего раба, Боже мой, ибо они злобны и созданы на
погибель! Не допусти их господствовать над
духами живущих, ибо Ты один знаешь суд их; и не
допусти их иметь власть над детьми
справедливости отныне и до века!>>

И Бог наш сказал нам, чтобы мы связали всех.
Тогда пришел высший из духов Мастема и сказал:
<<Господи, нельзя ли некоторым из них остаться у
меня, чтобы они слушались моего голоса и делали
все, что я скажу им? Ибо если ни одного из них не
останется у меня, то я не могу являть могущества
своей воли над сынами человеческими; ибо они
существуют для того, чтоб развращать и обольщать
по моему повелению под моим господством, так как
злоба людей велика>>. И он сказал: <<Десятая
часть их пусть останется у меня, и девять частей
пусть сойдут в место суда!>> И один из нас
сказал: <<Мы научим Ноя всем целебным
средствам>>; ибо он знал, что они ходят не в
справедливости, и будут вести борьбу не в
праведности. И мы сделали по Его повелению: всех
злых, лютых духов мы связали в месте
наказания, и десятую часть из них мы оставили,
чтобы они предстали пред Сатаною на земле. И
целебные средства от всех их (т.е. причиняемых
демонами) болезней вместе с их способами
обольщения мы сказали Ною, как излечивать себя
растениями земли. И Ной записал все, как мы
научили его, в книгу, о каждом роде лекарств. Так
злые духи были отделены в заключение от детей
Ноя.

И он дал все писания, которые написал, своему
старейшему сыну Симу, ибо он любил его больше из
всех своих сыновей. И Ной почил с своими отцами и
был погребен на горе Лубар в земле Арарат.
Девятьсот пятьдесят лет он окончил в своей жизни,
девятнадцать юбилеев, две седмины, пять лет. И его
жизнь на земле была знаменитее, чем жизнь остальных сынов человеческих,
ради его справедливости, в которой он усовершился, кроме только Еноха; ибо
история Еноха была предназначена во свидетельство для родов вечности, чтобы
показать все, что случится с родами родов до дня суда.

И в тридцать четвертый юбилей, в первый год
второй седмины, Фалек взял себе жену по имени
Ломна, дочь Синаара. И она родила ему сына в
четвертый год этой седмины, и он нарек ему имя
Рагев, ибо сказал: <<Вот сыны человеческие
сделались дурными через гнусный замысел, что они
построят себе город и башню в земле Синаар, ибо
они переселились от Арарата к востоку в
Синаар>>. Ибо в его дни они построили город и
башню, говоря: <<Мы поднимемся по ней на небо>>.
И они начали строить в четвертую седмину, и
обжигали огнем (кирпичи), и кирпичи служили им
вместо камня, и цементом, которым они укрепляли
промежутки, был асфальт из моря и из водных
источников в стране Синаар. И они строили это в
продолжение сорока трех лет. И Господь Бог наш
сказал нам: <<Вот, это один народ, и он начал
делать это! И ныне Я не отступлю от них! Вот, мы
сойдем и смешаем языки их, чтобы они не понимали
друг друга и рассеялись в страны и народы, и да не
осуществится никогда их замысел до дня суда!>> И
Господь сошел, и мы сошли с Ним, видеть город и
башню, которую строили сыны человеческие; и Он
расторг каждое слово их языка, и никто уже не
понимал слово другого. И вот они отказались
строить город и башню. Ради сего вся страна
Синаар была названа Бабель (Вавилон). Ибо так
расторг Бог все языки сынов человеческих; и
оттуда они рассеялись в свои города по их языкам
и народам. И Бог послал сильный ветер на их башню
и поверг ее на землю. И вот она стояла между
страной Ассур и Вавилоном в земле Синаар; и
нарекли ей имя развалины.

В первый год четвертой седмины тридцать пятого
юбилея они рассеялись в стране Синаар. И Хам с
своими сыновьями ушел в страну, которая стала его
собственностью и которая досталась ему при
разделе, в страну юга. А Ханаан увидел страну
Либаноса, до ручья Египетского, что она очень
хороша, и пошел не в страну своего наследия, на
запад от моря, но жил в стране Либанос, на востоке
и на западе от сынов народа Либаноса и вдоль моря.
И отец его Хам, и Куш, и Мицраим, его братья,
сказали ему: <<Ты поселился в стране, которая не
принадлежит тебе и которая по жребию не
досталась нам. Ты не должен так делать. Ибо если
ты сделаешь так, то погибнешь, так же как и твои
сыновья, в стране, как подвергшийся проклятию,
силой оружия; ибо вы силой оружия поселились, и
силой оружия падут твои сыновья, и ты будешь
истреблен вовек. Не живи в месте обитания Сима,
ибо оно досталось по жребию Симу и его детям.
Проклят ты и проклят будешь пред всеми сыновьями
Ноя проклятием, которым мы обязались в клятве
пред святым Судиею и пред нашим отцом Ноем>>. Но
он не послушал их и жил в стране Либанос, от
Гамафа до начала Египта, он и его сыновья до
нынешнего дня. И посему та страна была названа
Ханаан. Но Иафет и его сыновья пошли на запад и
жили в стране своего наследия. И Мадай увидел
страну моря, и она понравилась ему, и он выпросил
ее себе у Елама и Ассура и Арфаскада, брата его
жены, и жил в стране Мидакин (Мидийской стране)
вблизи брата своей жены до сего дня; и он назвал
свое место обитания и место обитания своих детей Медекин, по имени их отца
Мадая.

\vs Jub 11:1
И в тридцать пятый юбилей в третью седмину в
первый год ее Рагев взял жену по имени Ара, дочь
сына Кеседа. И она родила ему сына, и он нарек ему
имя Серуг в седьмой год этой седмины и этого
юбилея. И сыны Ноя начали вести борьбу друг с
другом; они стали друг друга брать и убивать,
проливать кровь человеческую на земле, и есть
кровь, и строить укрепленные города, и стены и
башни, и помимо того возноситься над народом, и
повсюду основывать царство, и вести войну~--- один
народ против другого, и народы против народов, и
город против города, и подвергать все порче, и
делать себе оружие, и учить своих детей войне. И
они начали покорять города, и продавать
невольников и невольниц.

И Ур, сын Кеседа, построил город Ару Халдейскую
и назвал его по имени себя и по имени отца своего.
И он делал им звезды и поклонялся каждому идолу,
которого он лил себе. И они начали делать
изваяния, и статуи, и нечистое, и духи нечистые
помогали в этом и обольщали их совершать грех и
нечистоту. И князь Мастема прилагал свою власть,
чтобы делали это, и побуждал чрез духов, которые
были отданы в его руки, совершать различного рода
злодеяния, и грехи и всякое развращение, чтобы
развращать, и губить, и проливать на земле кровь.
Посему ему было наречено имя Серух, ибо он
удалился, чтобы свободнее совершать грех и
злодеяние. И он сделался великим, и жил в Уре
Халдейском, вблизи родителей (своей матери), и
поклонялся идолам. И он взял себе жену в тридцать
шестой юбилей, в пятую седмину, в первый год, по
имени Мелка, дочь Кгебера, дочь (сестры) его отца.
И она родила ему Накгора в первый год этой
седмины; и он возрос и жил в Уре, в Уре Халдейском.
И его отец, мудрец Халдейский, научил его
предсказанию и гаданию по знамениям неба.

И в тридцать седьмой юбилей, в шестую седмину, в
первый год взял он себе жену по имени Ийосака,
дочь Нестега Халдейского, и она родила ему сына
Фарага в седьмой год этой седмины. И князь
Мастема послал воронов и птиц, чтобы они пожирали
семя, посеянное на земле, чтобы произвести порчу
на земле, чтобы они расхищали у сынов
человеческих их произведения. Ибо прежде чем они
запахивали семя, вороны подбирали его с
поверхности земли. Посему он нарек ему имя Фараг,
так как вороны и птицы обкрадывали их и пожирали
у них семя их. И годы стали делаться неурожайными
от птиц; и все древесные плоды они пожирали с
деревьев [...]. Только с великим трудом можно было в
их дни спасти кое-что от всех плодов земли.

И в тридцать девятый юбилей во вторую седмину, в
первый год, взял себе Фараг жену, по имени Една,
дочь Арема, сестрину дочь его отца, себе в жены. И
в седьмой год этой седмины она родила сына, и он
нарек ему имя Аврам, по имени отца его матери; ибо
он умер, прежде чем приобретен был ее и его сын. И
дитя начало замечать греховность земли, как она
была соблазнена к греху чрез изваяния и
нечистоту. И его отец научил его писать. И когда
он был двух седмин, то отдалился от своего отца,
чтобы не поклоняться вместе с ним идолам. И он
начал молиться Творцу всех вещей, чтобы Он спас
его от обольщения сынов человеческих и чтобы его
наследие, после того как он стал праведным, не
впало в греховность и нечестие.

И пришло время посева для тех, кто засевает
землю. И они вышли все вместе, чтобы стеречь свои
семена от воронов. И Аврам вышел с другими, будучи
дитею четырнадцати лет. И налетело облако (стая)
воронов, чтобы пожирать семена. Но Аврам побежал
к ним, прежде чем они сели на землю, и закричал на
них, прежде нежели они сели на землю, чтобы
пожирать семена, и сказал: <<Не смейте
спускаться, воротитесь в то место, откуда
прилетели!>> И они воротились. И они (?) сделали
так в тот день с семью стаями воронов. И из всех
воронов ни один не сел где-либо на пашню, где был
сам Аврам,~--- даже ни один. И все, бывшие около него
на той пашне, видели, как он закричал и сказал:
<<Воротитесь, вороны!>> И его имя сделалось
великим во всей стране Халдейской. К нему
приходили в этот год все, которые сеяли; и он
ходил с ними, пока не прошло время сеяния. И они
засеяли свою землю и собрали в том году хлеб, так
что ели и были сыты.

И в первый год пятой седмины Аврам научил тех,
которые делают воловью упряжь,~--- плотников, и они
сделали прибор над землею против деревянной дуги
плуга, чтобы класть на него семена и выбрасывать
их оттуда в семенную борозду, чтобы они
скрывались в земле. И они не боялись более
воронов и делали так у всех дуг плугов нечто над
землею. И они засеяли и обработали всю страну
вполне так, как им велел Аврам; и они не боялись
более воронов.

\vs Jub 12:1
И случилось в шестую седмину в седьмой год ее,
сказал Аврам отцу своему Фарагу, говоря: <<Отец,
отец мой!>> И он сказал: <<Вот я здесь, мой
сын!>> И он сказал: <<Что нам за помощь и
услаждение от всех идолов [...], что ты
поклоняешься им? Ибо в них совсем нет духа (души);
но они, которых вы почитаете, суть проклятие и
соблазн сердца. Почитайте Бога небесного,
Который низводит на землю дождь и росу, и все
совершает на земле, и все сотворил Своим словом, и
вся жизнь пред Его лицем! Зачем вы почитаете тех,
которые не имеют духа? ибо они нечто сделанное, и
на своих плечах вы носите их, и не имеете от них
никакой помощи, но они служат великим поношением
для тех, которые делают их к соблазну сердца и
почитают их. Не почитайте их!>> И отец его сказал
ему: <<И я знаю это, сын мой. Но что я сделаю с
моим родством, которое заставило меня служить им?
Если я скажу им истину, то они убьют меня, ибо их
душа прилепилась к ним, чтобы почитать и
прославлять их. Молчи, сын мой, чтобы они не убили
тебя!>> И он сказал эту речь своим двум братьям,
и они разгневались на него. Тогда он замолчал.

И в сороковой юбилей во вторую седмину в
седьмой год ее Аврам взял жену по имени Сора, дочь
его отца (?), и она сделалась его женою. Аран, брат
его, взял себе жену в [...] год третьей седмины,
и она родила ему сына в седьмой год этой седмины;
и он нарек ему имя Лот. И его брат Накгор также
взял себе жену.

И (в шестидесятый) год жизни Аврама, т.е. в
четвертый год четвертой седмины, встал Аврам
ночью, и сожег капище идолов и все, что было в нем,
так что люди ничего не знали об этом. И они встали
ночью и хотели спасти своих идолов из огня. И Аран
поспешил сюда, чтобы спасти их; тогда пламя
бросилось на него, и он сгорел в огне и умер в Уре.
Халдейском, прежде своего отца Фарага; и они
погребли его в Уре Халдейском.

И Фараг вышел из Ура Халдейского, он и дети его,
чтобы идти в страну Либаноса и в страну Ханаан; и
он жил в стране Харран . И Аврам жил со своим отцом
Фарагом в Харране две седмины.

И в шестую седмину в пятый год ее встал Аврам и
сидел в течение ночи, в новолуние седьмого
месяца, чтобы наблюдать звезды, от вечера до утра,
чтобы видеть, что будет с погодою в этот год. И он
был один, когда сидел и наблюдал. И пришло на его
мысль слово, и он сказал: <<Все знамения звезд и
знамения солнца и луны в руке Господа. Зачем мне
исследовать их? Когда Он хочет, то посылает дождь
рано и поздно, и когда хочет, то изливает потоки
(дождя), и все в Его руке>>. И он молился в эту
ночь и сказал: <<Боже мой, Боже мой! Ты всевышний
Бог, Ты один только Бог мой, и Ты все сотворил и
все есть дело рук Твоих; и Тебя, Твое божество
избрал я. Спаси меня от руки злых духов, которые
сильны над помышлениями человеческого сердца,
чтобы они не отвратили меня от Тебя, Боже мой! И
соделай, чтобы я и мое семя вовек не отвращались от
Тебя, отныне и до века!>> И он сказал:
<<Возвратиться ли мне в Ур Халдеев, которые
ищут моего лица, чтобы я возвратился к ним, или
оставаться мне здесь в этом месте? Укажи рабу
Твоему правый путь пред Тобою, чтобы исполнять
его, и чтобы я не ходил в обольщении моего сердца,
Боже мой!>> И когда он окончил речь и молитву,
вот тогда было послано чрез меня слово Господа к
нему, говоря: <<Поднимись из земли твоей и из
рода твоего и из дома отца твоего в землю, которую
Я тебе покажу! И Я произведу от тебя великий и
бесчисленный народ, и благословлю тебя, и сделаю
твое имя великим. И ты будешь благословлен на
земле, и в тебе благословятся все народы земли;
благословляющих тебя Я благословлю и
проклинающих тебя прокляну; и Я буду Богом тебе, и
твоим сыновьям, и сыновьям сынов твоих, и всему
твоему семени; и буду за тобою, Я Бог твой. Не
бойся, отныне до всех родов земли я Бог твой>>. И
Господь Бог сказал мне: <<Открой его уста, и его
уши, и его губы!>> И я начал говорить по-еврейски
на его коренном языке. И он взял книги своего
отца, которые были написаны по-еврейски, и списал
их. Тогда он начал изучать их, и я объяснял ему
все, чего он не понимал, и он изучал их в
продолжение шести дождливых месяцев.

И был седьмой год шестой седмины. Тогда говорил
он со своим отцом и возвестил ему, что он выйдет
из Харрана, чтобы идти в землю Ханаан, что он
осмотрит ее и возвратится к нему. И отец Фараг
сказал ему: <<Иди в мире, Бог мира да соделает
путь твой правый, и да будет Господь с тобою, и
хранит тебя от всех зол, и да даст тебе милость, и
благоволение, и милосердие пред теми, которые
увидят тебя, чтобы никакой человек не возымел
силы над тобою, чтобы предпринять что-либо против
тебя! Иди в мире! И если ты найдешь страну угодною
очам твоим, чтобы жить там, то возьми и меня с
собою; и возьми с собою Лота, сына Арана, брата
твоего, как своего сына! И Господь да будет с
тобою!>>

\vs Jub 13:1
И Аврам вышел из Харрана и взял с собою жену
свою Сору и Лота, сына брата своего Харрана, в
страну Ханаан. И он пошел [...] и прошел до Сикимона,
близ высокого дуба. И Господь сказал ему: <<Тебе
и твоему семени Я дам эту страну!>> И он устроил
там жертвенник и принес на нем Господу, Который
явился ему, всесожжение, И оттуда он поднялся к
горному хребту Бетель (Вефиль), который был от
него на запад и Ай (Гай) на восток, и разбил там
шатер свой. И он увидел, что земля была очень
обширна и хороша, и что в ней росло все:
виноградные лозы, смоквы, гранаты, дубы, и твердые
деревья, и теревинфы, и масличные деревья, и
кедры, и кипарисы, и ливанские деревья, и все
деревья полевые, и что она имела воду на горах. И
он благословил Господа, Который привел его из Ура
Халдейского на эту гору.

И случилось, в первый год, в седьмую седмину, в
новолуние первого месяца он устроил на этой горе
жертвенник и призвал имя Господа: <<Ты, Боже мой,
вечный Бог>>. И он принес на жертвеннике Божием
всесожжение, чтобы Он был с ним и не оставлял его
в течение его жизни. И он поднялся оттуда и пошел
(на юг), и достиг Хеврона; и Хеврон был тогда
построен. И он оставался там два года [...]. Тогда
пошел Аврам в Египет в третий год седмины, и жил
в Египте пять лет, прежде чем у него была похищена
жена. Санай же был тогда построен в Египте чрез
семь лет после Хеврона. И случилось, когда Фараон
похитил Сору, жену Аврама, Господь поразил
Фараона и весь дом его тяжкими бедствиями за
Сору, жену Аврама. И Аврам был очень обогащен
овцами, и рогатым скотом, и ослами, и конями, и
верблюдами, и рабами, и служанками, и серебром, и
золотом вполне; и Лот, сын его брата, также был
обогащен. И когда Фараон возвратил Сору, его жену,
он переселился из земли Египетской, и пришел в
одно место, на восток от Вефиля, и прославил
Господа Бога своего, который вывел его обратно с
миром.

И случилось, в (сорок первый) юбилей в
третий год первой седмины он возвратился в это
место, и принес там всесожжение, и призвал имя
Господне и сказал: <<Ты Господь, Бог всевышний,
Бог мой вовек!>> И в четвертый год седмины Лот
отделился от него. И Лот жил в Содоме, но жители
содомские были очень злы. И он (Аврам) опечалился
в сердце своем, что его племянник отделился от
него, ибо он не имел детей. В тот год этой седмины,
когда Лот был пленен, Господь говорил Авраму,
после того как Лот отделился от него, и сказал
ему: <<Возведи очи твои от места, где ты живешь, к
(северу), и к югу, и к утру, ибо всю страну,
которую ты видишь, Я дам тебе и дам семени твоему
вовек. И Я сделаю твое семя, как песок при море, (и
как человек не может сосчитать песок при море),
так нельзя исчислить и твоего семени. Встань и
пройди ее по длине и широте, и посмотри все, ибо Я
дам ее твоему семени>>.

И Аврам пошел в Хеврон и жил там. И в этот год
пришли Колодогомер, царь еламский, и Амалфал,
царь синаарский, и Ариох, царь селасарский, и
Тергал, языческий царь; и они поразили царя
Гоморры, и царь Содома бежал, и многие,
обратившиеся в Сиддин, солончатую страну, в
Содом, Адом и Севоим, пали. И они взяли в плен Лота,
племянника Аврама, со всем его имуществом, и
отвели в Дан. И пришел один, который спасся
бегством, и рассказал Авраму, что племянник его
взят в плен [...]. И его раб принес в умилостивление
за Аврама и его семя десятину начатков Господу. И
Господь сделал отсюда постановление навсегда,
чтобы давать ее (десятину) священникам, которые
служат пред Его лицем, дабы они пользовались ею
вовек. И это установление не на день, но Он
утвердил его на вечные роды, чтобы давать Господу
десятину от семян, и вина, и масла, и рогатого
скота, и овец. И Он дал ее Своим священникам, чтобы
они ели от нее с радостью и пили пред Ним.

И вышел к нему царь содомский, и пал пред ним
ниц, и сказал: <<Господин мой Аврам, отдай мне
людей, которых ты освободил; добыча же пусть
будет твоя!>> И Аврам сказал ему: <<Я воздвигаю
руки мои к всевышнему Богу: ни нитки, ни
башмачного ремня я не возьму из всего, что
принадлежит тебе, дабы ты не сказал: <<Я сделал
Аврама богатым>>, кроме того, что съели отроки. И
мужи, ходившие со мною, Аунан, и Ескол, и Мамре,
должны взять свою долю>>.

\vs Jub 14:1
И после сего события, в четвертый год этой
седмины, в новолуние третьего месяца, было слово
Господне к Авраму в сновидении, говорящее: <<Не
бойся, Аврам, Я твоя защита, и награда твоя будет
чрезмерна>>. И он сказал: <<Господи, Господи,
что Ты дашь мне? Вот я иду туда без детей, и сын
Месек мой раб тот Дамаск Елиезер, будет
наследником мне; а мне Ты не дал семени>>. И Он
сказал ему: <<Он не наследит тебе, но
происшедший от плоти твоей будет тебе
наследником>>. И Он вывел его и сказал ему:
<<Взгляни на небо и сосчитай звезды небесные:
можешь ли ты сосчитать их?>> И он взглянул на
небо и увидел звезды. И Он сказал ему: <<Так
будет твое семя>>. И он поверил Господу, и это
было вменено ему в праведность. И Он сказал: <<Я
Господь Бог твой, выведший тебя из Ура Халдейского,
чтобы дать тебе в вечное владение землю
Ханаанитов, и чтобы Я был твоим Богом и Богом
твоего семени>>. И он сказал: <<Господи,
Господи!>> И он сказал: <<Господи, по чему я
узнаю, что наследую ее?>> И Он сказал ему:
<<Принеси Мне трехлетнюю телицу, и трехлетнюю
козу, и трехлетнюю овцу, и трехлетнюю горлицу, и
голубя>>. И он взял все это в средине месяца. И он
жил при дубе Мамре, который близ Хеврона. Там
устроил он жертвенник, и заколол все, возлил
кровь их на жертвенник, и разделил их пополам, и
положил их друг против друга; но птиц он не
касался. И птицы спустились на куски, но Аврам
отгонял их, и не давал птицам прикасаться к ним. И
было, когда солнце зашло, бессилие напало на
Аврама, и вот сильный страх мрака напал на него. И
было сказано: <<Аврам, знай, что твое семя будет
странником в чужой земле и его будут порабощать и
угнетать в продолжение четырехсот лет. Но Я
произведу суд над народом, которому они будут
служить; после того они выйдут оттуда с большим
имуществом. И ты в мире отойдешь к своим отцам, и
будешь погребен в доброй старости. И в четвертом
роде оно (твое семя) возвратится сюда, ибо грех
Аморреев доселе еще не наполнился>>.

И он пробудился от своего сна и встал, и солнце
было зашедшим. Тогда появилось пламя, и вот~--- печь
дымилась, и огненное пламя прошло между кусками.
И в ту ночь Бог заключил завет с Аврамом, сказав:
<<Твоему семени Я отдам эту землю, от реки
Египетской до великой реки, реки Евфрат,~--- Кенеев,
Кенезеев, Ферезеев, Рафейн, [...], Евеев, Аморреев,
Канаанеев, Гергесеев>>. И Он отошел. И Аврам
принес куски, и птиц, и жертву плодовую, и жертву
возлияния, которые принадлежали к сему, и огонь
пожрал их.

И в эту ночь Он заключил завет с Аврамом,
согласно завету, который мы заключили в этом
месяце с Ноем. И Аврам возобновил его в праздник и
в постановление для себя, до века. И Аврам
возрадовался и рассказал все это происшествие
своей жене Соре. И он поверил, что у него будет
семя, но она не рождала. Тогда Сора посоветовала
своему мужу Авраму и сказала ему: <<Войди к моей
служанке Агари, египтянке; быть может, я
произведу тебе от нее семя>>. И Аврам послушался
голоса жены своей Соры и сказал ей: <<Сделай
это>>. Тогда Сора взяла египетскую служанку
Агарь и дала ее своему мужу Авраму, чтобы она была
его женою. И он вошел к ней, и она сделалась
беременной, и родила сына, и он нарек ему имя
Измаил в пятый год этой седмины. В том году был
восемьдесят шестой год жизни Аврама.

\vs Jub 15:1
И в пятый год четвертой седмины этого юбилея, в
третий месяц, в средине месяца, Аврам праздновал
праздник начатков жатвы хлеба и принес свежую
хлебную жертву; к жертвам начатков хлеба для
Господа (он присоединил) тельца, и овна, и овцу на
жертвенник вместе с благовонным курением. И
Господь явился ему и сказал Авраму: <<Я~--- Бог
Владыка, благоугождай предо Мною и будь
благочестив. И Я заключу завет между Мною и тобою
и сделаю тебя весьма великим>>. И Аврам пал на
свое лице. И Господь говорил с ним и сказал:
<<Вот завет Мой с тобою, и Я сделаю тебя отцом
многих народов, и ты не будешь более называться
Аврам отныне до века; но Авраам будет тебе имя, ибо
Я сделал тебя отцом многих народов, и сделаю тебя
весьма великим, и произведу от тебя народов и
царей. И я поставлю завет Мой между тобою и Мною, и
между твоим семенем после тебя, в их родах, в
вечное установление, чтобы Я был твоим Богом и
Богом твоего семени после тебя во всех родах. И
Я дам тебе и семени твоему после тебя землю~---
ибо ты пришлец в ней~--- землю Ханаанскую, чтобы ты
был господином над нею навсегда. И Я буду им
Богом>>. И Господь сказал Аврааму: <<И храни
Мой завет ты и твое семя после тебя, и обрезывайте
все ваши крайние плоти. И это будет знамением
Моего вечного установления между Мною и тобою и
для родов (потомков). В осьмой день вы должны
обрезывать все мужеское, в ваших родах,
рожденного дома и купленного вами за золото у
всех сыновей чужеземцев, что приобрели вы. Кто от
твоего семени, тот да будет обрезан, рожденный
дома и купленный за золото да будет обрезан. И Мой
завет на теле вашем пусть будет в вечное
установление.

И кто не обрезан, всякий мужеского пола между
вами, крайняя плоть которого не обрезана в
восьмой день, душа та да истребится из рода
вашего, ибо она нарушила завет Мой>>. И Господь
сказал Аврааму: <<Сора, жена твоя, не будет более
называться Сорою, но Сара~--- имя ее; и Я благословлю
ее, и дам тебе от нее сына; и Я благословлю его, и
произведу от него народ, и цари над народами
произойдут от него>>.

И Авраам пал на лице свое, и возрадовался, и
сказал в сердце своем: <<У меня ли, имеющего сто
лет, родится сын, и Сара девяноста лет родит ли
сына?>> И Авраам сказал Господу: <<Хотя бы
Измаил остался жив пред Тобою!>> И Господь
сказал: <<Да! но и Сара родит тебе сына, и ты
наречешь ему имя Исаак. И Я восстановлю завет Мой
с ними, завет Мой вечный, и с его семенем после
него. И о Измаиле Я услышал тебя, и вот я
благословлю и умножу его, и сделаю его весьма
многочисленным. И двенадцать царей произведет
он; и Я произведу от него великий народ; но завет
Мой Я поставлю с Исааком, которого родит тебе
Сара около сего времени на другой год>>. И после
того, как Он кончил говорить с ним, Господь
восшел.

И он (Авраам) взял своего сына Измаила и всех
своих рожденных дома (рабов) и купленных за
золото, весь мужеский пол, который был в его доме,
и обрезал плоть их члена. И в этот день был
обрезан Авраам, и люди его дома были обрезаны, и
также все, которых он купил за золото у сынов
иноплеменников, были обрезаны вместе с ним. И
этот закон~--- для всех родов вовек. И нельзя
изменять дней, ни пропускать одного из восьми
дней, ибо это вечное благословение, утвержденное
и записанное на небесных скрижалях. И каждый
рожденный, крайняя плоть которого не обрезана до
восьмого дня, не принадлежит к сынам завета,
который Господь заключил с Авраамом, но к сынам
погибели, и вот он не имеет знака на себе, что он
Господень; он предназначен к погибели, и
уничтожению, и истреблению от земли, ибо он
нарушил завет Господа нашего Бога. Ибо Он освятил
Израиля, чтобы он был со всеми Его Ангелами лица,
и со всеми Ангелами прославления, и со святыми
Его Ангелами. И ты повели также сынам Израиля,
чтобы они хранили знак сего завета в своих родах,
как вечное установление, чтобы не быть им
истребленными от земли. Ибо постановление cue
утверждено для завета, чтобы оно соблюдалось
навсегда между всеми сынами Израиля. Ибо Измаила,
и сыновей его и братьев, и Исава не приблизил
Господь и не избрал их; но сынов Авраама познал Он
и избрал Израиля, чтобы они были Его народом, и
освятил его, и собрал его из всех сынов
человеческих. Ибо много народов, и бесчисленны
люди, и все принадлежат Ему, и над всеми Он
поставил духов вместо Господа, чтобы они
отвращали их от Него. Над Израилем же Он никого не
поставил господом~--- ни Ангела, ни духа, но Он
единый их Владыка, и Он охраняет их и ведет тяжбы
их против Своих Ангелов, и Своих духов, и против
всего. И если они будут хранить все Его повеления,
то Он благословит их, и они будут Его сынами, и Он
будет их Отцом отныне до века. И теперь я
предсказываю тебе, что сыны Израиля будут
поступать вопреки этому установлению, и их сыны
не будут обрезываться согласно всему этому
закону. Ибо на плоти своего обрезания они не
будут совершать оного обрезания своих сыновей, и
они все, сыны Велиара, будут оставлять своих
сыновей необрезанными, как они родились. И гнев
Господа на детей Израиля будет велик, ибо они
оставили завет Его, и уклонились от Его слова, и
возбудили Его на гнев, и восхулили Его, и не
сделали сего знака по их закону, но оставили свою
плоть необрезанною подобно язычникам, чтобы
быть уничтоженными и истребленными с земли. И они
впредь не обретут прощения и помилования, чтобы
быть прощенными и помилованными во всех своих
грехах за сие отступление вовек.

\vs Jub 16:1
И в новолуние четвертого месяца явились мы
Аврааму при дубе Мамврийском и беседовали с ним.
И мы также возвестили ему, что у него родится сын
от жены его Сары. Тогда Сара рассмеялась, ибо она
слышала, что мы говорили эту речь Аврааму. И мы
заметили ей; но она испугалась и стала отрицать,
что она смеялась над нашими словами. И мы
сказали ей имя его сына, как определено и
написано было на небесных скрижалях, именно
Исаак. И когда мы возвратимся к ней в
определенное время, тогда она будет беременной
сыном.

И в этот месяц Господь совершил суд над Содомом,
и Гоморрою, и Севоимом, и всею страною Иорданскою,
и сожег их огнем и серой, и предал их погибели до
сего дня; согласно тому, как мы рассказывали тебе
о всех их делах, что они были гнусными и весьма
греховными и что они осквернялись, и
блудодействовали, и делали мерзость на земле~---
согласно сему Бог совершил суд; во гневе и ярости
за нечистоту Содома совершил Он суд над Содомом.
И мы спасли Лота, ибо Господь вспомнил об Аврааме
и вывел его (Лота) из разрушения. Но и он, и дочери
его совершили на земле грех, какого не было на
земле от Адама до того времени; ибо муж переспал с
своею дочерью. И вот, относительно всего его
семени определено и начертано на скрижалях,
чтобы уничтожить и истребить его, и совершить суд
над ним, как над Содомом, и не оставить ему семени
на земле ко дню осуждения.

И в этот месяц поднялся Авраам от Хеврона и
пошел и жил между Кадетом и Суром на горах
Герарона. И в средине пятого месяца он поднялся
оттуда и жил при клятвенном колодезе. И в средине
шестого месяца Господь посетил Сару, и сотворил
ей, как сказал, и она сделалась беременною. И она
родила ему сына в третий месяц, в средине месяца,
как сказал Бог Аврааму. В праздник начатков жатвы
родился Исаак, и Авраам обрезал своего сына в
восьмой день. Он первый был обрезан согласно
завету, как определено навечно.

И в шестой год четвертой седмины пришли мы к
Аврааму к клятвенному колодезю и явились ему, как
сказали Саре, что придем к ней. А она сделалась
беременною сыном, и мы возвратились в седьмой
месяц, и нашли Сару беременною пред нами, и
благословили Сару, и рассказали Саре все, что
было повелено нам относительно него (т.е.
Авраама), что он не умрет, пока не родит шесть
сыновей, и что он увидит их, прежде чем умрет, но
что в Исааке будет наречено имя его и семя, и что
все семя его сыновей будет язычниками и
причтется к язычникам; но только семя от сыновей
Исаака будет святым, и не причтется к язычникам;
ибо оно будет наследием Всевышнего, и все его
семя будет между теми, которые почитают Бога,
чтобы быть для Господа драгоценным украшением
пред всеми народами и быть царством и народом
святым. И мы прошли наш путь, и передали Саре все,
что мы сказали ему (Аврааму). И они оба друг с
другом были в великой радости. И он устроил там
жертвенник Господу, Который спас его и
возвеселил его в стране его странствования, и
праздновал торжество в этом месяце в течение
семи дней близ жертвенника, который он устроил
при клятвенном колодезе, и устроил кущи для себя
и своих рабов к этому празднику. И он праздновал этот
праздник в первый раз на земле; и в эти семь
дней он приносил каждодневно на жертвеннике
Господу всесожжение: семь волов, двух молодых
козлов, двух овнов, семь овец; одного козла в
жертву за грех, чтобы искупить ею себя и свое
семя; и в жертву благодарения семь овнов, семь
молодых козлов, семь овец, семь тельцов вместе с
плодовою жертвою и возлиянием, которые
относились к сему. Над всем их туком он воскурял
на жертвеннике избранное всесожжение в приятное
благовоние. Утром и вечером он воскурял ладан, и
халван, и стакти, и нард, и мирру, и Сенегал и кост;
все эти семь веществ он приносил
истолченными, смешанными между собою по равной
части и очищенными. И он праздновал этот праздник
в течение семи дней, радуясь в своем сердце и всею
душою,~--- он и все, бывшие в его доме; и ни одного
чужеземца не было с ним, и ни одного
незаконнорожденного. И он прославлял своего
Творца, Который создал его в его роде, ибо Он по
Своему благоволению создал его. Ибо он знал и
уразумел, что от него придет растение
праведности для будущих родов и что равным
образом от Него придет святое семя, от Него,
который все создал. И он прославил Его, и нарек
имя этому празднику~--- праздник Господень, и
радовался радостию, которая была приятна
Всевышнему Богу. И мы благословили его вовек и
все его семя после него на все роды земли, ибо он
праздновал тогда этот праздник по свидетельству
небесных скрижалей. Посему на небесных скрижалях
определено для Израиля, чтобы они праздновали
праздник кущей в течение семи дней с радостию, в
седьмой месяц, дабы это было приятно Господу, в
вечный закон для родов их, на все века и годы; и
нет для сего установления конца дней, но
навек определено относительно Израиля, чтобы они
праздновали его, и жили в кущах, и полагали венки
на свои головы. И как они берут от ручья покрытую
листьями ивовую ветвь, так брал и Авраам сережки
от пальмовых ветвей и хорошие древесные плоды, и
обходил каждый день с ветвями вокруг жертвенника
семь раз в день, и утром он восхвалял и благодарил
Бога своего за все с радостию.

\vs Jub 17:1
И в первый год пятой седмины этого юбилея Исаак
был отнят от груди, и Авраам сделал большой пир на
третий месяц, в день, когда сын его Исаак был
отнят от груди. И Измаил, сын египтянки Агари, был
пред лицем отца своего Авраама на своем месте. И
Авраам радовался и прославлял Бога, что он увидел
от себя сыновей и не умер без сыновей. И он
вспомнил слово, как Он говорил с ним в тот день,
когда Лот отделился от него. И он радовался, что
Бог дал ему семя на земле, чтобы получить в
наследие страну. И он прославил громким голосом
Творца всех вещей. И когда Сара увидела Измаила,
как он был весел и плясал и что даже Авраам
радовался при этом, то почувствовала зависть при
взгляде на Измаила и сказала Аврааму: <<Выгони
эту служанку и ее сына; сын этой служанки не
должен наследовать с моим сыном Исааком>>. И это
показалось неприятным Аврааму ради его служанки
и его сына, что он должен выгнать их от себя. И
Господь сказал Аврааму: <<Не нужно тебе
огорчаться из-за отрока и рабыни; все, что сказала
тебе Сара, послушайся ее слова и исполни его, ибо
в Исааке наречется тебе имя и семя. Сына же этой
рабыни Я сделаю великим народом, ибо он~--- твой
род>>. И Авраам собрался рано утром, взял хлеба и
мех с водою, и положил их на плечи Агари вместе с
отроком, и отослал их. И она пошла, блуждая в
пустыне Вирсавии. И не стало воды в мехе; и отрок
истомился от жажды, и не мог идти и упал. И мать
взяла его, и пошла и бросила под масличное дерево.
И она пошла дальше, и села против него, удалившись
на выстрел из лука, ибо сказала: <<Я не могу
смотреть на смерть моего сына>>. И вот она села и
плакала. Тогда Ангел Божий, один из святых, сказал
ей: <<Что ты плачешь, Агарь? встань, подними
отрока и возьми его своею рукою, ибо Господь
услышал твой голос>>. И когда она увидела
отрока, подняла свои глаза и увидела колодезь с
водою, и пошла туда, наполнила свой мех водою и
напоила свое дитя. И она встала и пошла к Фараону.
И отрок вырос и сделался стрелком из лука, и
Господь был с ним. И мать его взяла ему жену из
дочерей Египетских, и она родила ему сына. И он
нарек ему имя Навайвоф, ибо она сказала: <<Бог
был близ меня, когда я призывала его>>.

И случилось в седьмую седмину в первый год в
первый месяц этого юбилея, в двенадцатый день
сего месяца, были сказаны на небесах некоторые
слова об Аврааме, что он верен во всем, что
Господь говорит ему, и что он любит Его и верен во
всяком искушении. Тогда пришел начальный Мастема
и сказал пред Богом: <<Вот Авраам любит и
дорожит своим сыном Исааком больше всего; скажи
ему, чтобы он принес его во всесожжение на
жертвеннике, и Ты увидишь, исполнит ли он это
повеление, чтобы узнать Тебе, верен ли он во всем,
чем Ты его испытываешь>>. И Бог знал, что Авраам
верен во всех испытаниях, которые Он назначает
ему, ибо Он искушал его царством царей, и затем
женою его, когда она была похищена у него, и далее
Измаилом и Агарью, его служанкою, когда он
отослал их, и во всем, чем Он искушал его, он
оказался верным, и его душа не была мятежною, и не
медлил он исполнять сие, ибо был верен и любил
Бога.

\vs Jub 18:1
И Господь сказал Аврааму: <<Авраам!>> И он
сказал: <<Вот я!>> И Он сказал ему: <<Возьми
возлюбленного твоего сына Исаака, и пойди на
высокую гору, и принеси его в жертву на одной
из гор, которую Я тебе покажу>>. И он собрался
оттуда утром на рассвете, и оседлал свою ослицу, и
взял двух своих рабов с собою и своего сына
Исаака, и наколол дров для жертвы. И он шел к назначенному
месту три дня и увидел то место издали. И он
пришел к колодезю с водою и сказал своим рабам:
<<Останьтесь здесь с ослицею; я и отрок пойдем и,
когда помолимся, возвратимся к вам>>. И он взял
дрова для жертвы и возложил на плечи сыну своему
Исааку, и взял в руки огонь и нож, и они пошли оба
вместе к тому месту. И Исаак сказал своему отцу:
<<Отец!>> И он сказал: <<Вот я, сын мой>>.
И он сказал: <<Вот здесь нож и дрова, где же овца
для всесожжения, отец мой?>> И он сказал:
<<Господь усмотрит себе овцу для всесожжения,
сын мой>>. И он пошел к месту горы Божией, и
устроил жертвенник, и положил дрова на
жертвенник, и поднял сына своего Исаака, и
положил его на дрова на жертвенник, и простер
руку свою взять нож, чтобы заколоть сына своего
Исаака. И я (Ангел) стал пред ним (пред Богом?) и
пред высшим Мастемой. И Господь сказал: <<Скажи
ему, чтобы он не возлагал руки своей на отрока и
не делал ему никакого вреда, ибо Я знаю, что он
богобоязнен>>. И я воззвал к нему с неба и
сказал: <<Авраам, Авраам!>> И он убоялся и
сказал: <<Вот я>>. И Он сказал ему: <<Не
возлагай руки своей на отрока и не делай ему
никакого вреда, ибо теперь Я знаю, что ты
богобоязнен и сам не пожалел твоего
перворожденного сына предо Мною>>. И посрамился
высший Мастема. И Авраам возвел очи свои и увидел,
и вот там был овен, зацепившийся своими рогами. И
Авраам пошел, и взял овна, и принес его во
всесожжение вместо сына своего. И Авраам назвал
то место: <<Господь усмотрел сие>>, так что
говорят: <<Господь усмотрел сие>>, т.е.
гора Сион.

И Господь вторично воззвал Авраама по имени с
неба, как Он возвестил мне, чтобы я говорил с ним
во имя Господа. И Он сказал: <<Моею главою Я
поклялся, говорит Господь: так как ты сделал это,
и твоего перворожденного сына, которого любишь,
ты не пожалел предо Мною, то Я поистине
благословлю тебя и умножу семя твое, как звезды
небесные и как песок на берегу моря. Твое семя
получит в наследие города врагов своих, и
благословятся в семени твоем все народы земли за
то, что ты послушался гласа Моего и показал всем,
что ты верен Мне во всем, что Я возложил на тебя.
Иди в мире!>>

И Авраам пошел к своим рабам, и встали, и пошли
они вместе в Вирсавию, и Авраам жил при
клятвенном колодезе. И он соблюдал сей праздник
ежегодно в течение семи дней с радостию, и назвал
его праздником 1Ъсподним соответственно семи
дням, в продолжение которых он ходил и
возвратился в мире. И так утверждено сие и
записано на небесных скрижалях относительно
Израиля и его семени, чтобы они праздновали этот
праздник в течение семи дней с радостию.

\vs Jub 19:1
И в первый год первой седмины сорок второго
юбилея возвратился Авраам и жил против Хеврона,
т.е. Каръяфарбока. Во вторую седмину в третий год
этого юбилея окончились дни жизни Сары, и она
умерла в Хевроне. И пришел Авраам оплакать и
погребсти ее. И мы испытывали его, покорен ли дух
его и не произнесет ли он устами своими мятежного
слова, но он и здесь оказался покорным и не
возмущался, а с спокойным духом говорил с детьми
Киту (т.е. Хета), чтобы они дали ему место, на
котором он похоронил бы свою умершую. И Господь
наградил его благоволением пред всеми, которые
видели его, и он просил, полный смирения, детей
Хета, и они дали ему землю двойной пещеры против
Мамре, т.е. Хеврона, за сорок серебреников. Но они
просили его, говоря: <<Мы отдадим вам это
даром>>. Но он не взял у них даром, а отдал им
цену за место~--- хорошее серебро, и поклонился им
дважды. И после сего он похоронил свою умершую в
двойной пещере. И всех дней жизни Сары было сто
двадцать семь лет, т.е. два юбилея четыре седмины
и один год. Это годы жизни Сары. И это было десятое
испытание, которым был искушаем Авраам; и он
обнаружил верный и покорный дух. И он не сказал
никакого слова о том, что Бог обещал ему дать
страну ему и его семени после него, но он просил
там только о местах, чтобы похоронить свою
умершую. Так оказался он верным и покорным, и
записан был, как друг Господа, на небесных
скрижалях.

И в четвертый год ее (второй седмины) взял он
сыну своему Исааку жену по имени Ревекка, дочь
Вафуила, сына Нахорова, брата Авраама. И Авраам
взял себе третью жену по имени Кетура, из дочерей
своих домашних рабов; ибо Агарь умерла прежде
Сары; и она родила ему шесть сыновей: Ценбари, и
Якзана, и Мадая, и Ийясбока, и Зигийю.

Во второй год шестой седмины Ревекка родила
Исааку двух сыновей~--- Иакова и Исава. И Иаков был
благочестив, а Исав~--- муж грубый, земледелец и
волосатый; и Иаков жил в шатрах. И юноши подросли:
и Исав научился, так как он был земледельцем и
охотником, войне и всякому грубому затятию. И
Иакова любил Авраам, а Исава Исаак. И Авраам видел
занятие Исава и уразумел, что в Иакове будет
наречено ему имя и семя. И он призвал Ревекку и
дал ей повеление относительно Иакова, ибо он
видел, что она также любила гораздо более Иакова,
нежели Исава. И он сказал ей: <<Дочь моя! береги
сына моего Иакова, ибо он будет вместо меня на
земле в благословление между сынами
человеческими и всему своему семени имя его будет
во славу. Ибо я знаю, что Господь произведет от
него народ и он будет предпочтен пред всеми,
которые на лице земли. И вот, сын мой Исаак любит
Исава более, нежели Иакова, и я вижу, что ты
действительно любишь Иакова. Так сделай ему еще
больше добра, и да будет он твоим возлюбленным сыном,
ибо он будет мне в благословение на земле
отныне до всех родов века. Да укрепятся руки твои
и да возрадуешься ты о сыне твоем Иакове, ибо я
люблю его более всех моих сыновей; ибо он будет
благословен вовек, и семя его наполнит всю землю.
Ибо как не может человек сосчитать пыль земную,
так не может быть исчислено и семя его. И все
благословения, которыми Господь благословил
меня и мое семя, будут также уделом Иакову и его
семени во все дни. И в его семени будет
благословлено мое имя, и имя моих отцов Сима, и
Ноя, и Еноха, и Малалела, и Сифа, и Адама. Да
послужат они к тому, чтобы основать небо, и
утвердить землю, и обновить светила, которые на
тверди небесной>>.

И он призвал Иакова пред очи матери его Ревекки,
и поцеловал его, и благословил его, и сказал:
<<Возлюбленный сын мой Иаков, которого
возлюбила душа моя! Да благословит тебя Бог с
высоты тверди небесной, и да даст тебе все
благословения, которыми Он благословил Адама, и
Еноха, и Ноя, и Сима; и все, что Он говорил со мною,
и все, что он обещал дать только мне, да пошлет Он
на тебя и на твое семя до века, пока небо
существует над землею. И да не владычествуют над
тобою и над твоим семенем духи Мастемы, чтобы
отвращать тебя от Господа, Который есть Бог твой,
отныне до века! И да будет Господь твоим Богом и
твоим отцом, а ты Его первородным сыном и Его
народом во все дни! Иди, сын мой, в мире!>> И они
все (?) вместе вышли от Авраама. И Ревекка любила
Иакова всем сердцем и всею душою и гораздо
больше, чем Исава. И Исаак любил гораздо больше
Исава, нежели Иакова.

\vs Jub 20:1
И в сорок второй юбилей в первый год седьмой
седмины призвал Авраам Измаила и двенадцать его
сыновей, и Исаака и обоих его сыновей, и шесть
сыновей Кетуры и детей их, и заповедал им хранить
пути Господа, чтобы поступали по справедливости
и любили друг друга, чтобы поступали таким же
образом во всякой войне, чтобы против каждого,
кто будет против них, они выходили все вместе, и
совершали правду и справедливость на земле,
чтобы они своих сыновей обрезывали по завету,
который Он заключил с ними, и не уклонялись бы ни
направо, ни налево от всех путей, <<которые
Господь заповедал нам>>, и соблюдали бы себя от
всякой мерзости, и избегали бы всякой мерзости и
блуда. <<И если какая-либо женщина или девица
совершит прелюбодеяние между вами, то сожгите ее
огнем, и не блудите вслед за нею очами и
сердцем>>. И пусть они не берут себе жен из
дочерей Ханаанских, ибо семя Ханаана будет
истреблено на земле. И он говорил им о суде над
исполинами и суде над Содомом, как они были
наказаны за их порочность, и блудодеяние, и
нечистоту, и взаимное развращение. За
блудодеяние погибли они, но вы воздерживайтесь
от всякого любодеяния и мерзости, и от всякого
осквернения грехами и мерзостию их, чтобы вам не
сделать имя наше проклятием и всю жизнь вашу позором,
и не предать бы всех сыновей ваших погибели
от меча, и чтобы проклятие ваше не было как Содом
и остаток ваш как сыны Гоморры. Я свидетельствую
вам, сыны мои: любите Бога небес и покоряйтесь
всем Его заповедям, и не обращайтесь к идолам и
мерзостям их (язычников); и не делайте себе ни
литых идолов, ни изваяний, ибо они ничтожны, и не
имеют души, но они суть дело рук, и все, которые
полагаются на них, не получают помощи,~--- все,
которые положились на них. Не почитайте их и не
поклоняйтесь им, а почитайте Бога Всевышнего, и
поклоняйтесь всегда Ему, и надейтесь на Твое
лице, о Господи, (?) во всякое время, и совершайте
правду, и справедливость, и праведность пред Ним,
чтобы Он имел благоволение к вам, и являл вам Свое
милосердие, и ниспосылал дождь утром и вечером, и
благословлял всякий труд ваш и все, над чем вы
трудитесь на земле; и ваш посев, и твою (?) воду, и
семя твоей плоти, и семя твоей земли, и твои стада
и овец благословит Он, и ты будешь во
благословение на земле, и все народы земли будут
иметь благоволение к вам и благословлять сынов
ваших именем моим, дабы они были благословлены,
как я>>.

И он дал Измаилу и его сыновьям и сыновьям
Кетуры подарки, и отослал их от своего сына
Исаака. И Измаил с сыновьями своими и сыновья
Кетуры с сыновьями их пошли вместе, и жили от
Фармона (вероятно, Фаран), пока не придешь к
Вавилону, во всей области, которая лежит к
востоку против пустыни, И они соединились вместе,
и были названы арабами и измаильтянами.

\vs Jub 21:1
И в шестой год седьмой седмины этого юбилея
Авраам призвал сына своего Исаака и заповедал
ему, говоря: <<Я стар и не знаю, когда умру, ибо я
пресытился днями своими. И вот мне сто семьдесят
пять лет, и в продолжение всей жизни моей я
помышлял о Господе, и от всего сердца стремился
исполнять волю Бога моего и ходить право по всем
Его путям. Идолов ненавидела душа моя, дабы быть
внимательным к исполнению воли Того, Кто
сотворил меня; ибо Он~--- Бог живый, и свят, и верен, и
праведен во всем, и нет неправды в Нем, чтобы
взирать на лице и принимать дары; но Он есть Бог
правды, совершающий наказание над всеми, которые
преступают его заповеди и нарушают Его завет. И
ты также, сын мой, соблюдай Его заповеди, Его
установление и правду, и не ходите (?) вслед за
мерзостию язычников, и изваяниями и литыми
изображениями, и не ешьте крови ни<b> </b>зверей, ни
скота, ни различных птиц, которые летают на небе.
И если ты закалываешь, то закалывай в жертву мира,
которая приятна Богу: закалывай ее, и кровь ее
выливай к жертвеннику, с мукою и плодовыми
жертвами, смешанными с маслом, вместе с жертвою
возлияния. Принеси все эта на жертвеннике
всесожжения в приятное благоухание пред
Господом. Как при жертве благодарения, положи
куски тука на огонь жертвенника, именно~--- тук
чрева, и тук внутренностей, и обе почки и весь тук
на них, и тук на стегнах, и печень вместе с
прилежащими к ней почками, И ты принесешь все в
доброе благоухание, которое приятно Господу, с
первыми жертвами и возлияниями, которые к сему
относятся, в доброе благоухание, как хлеб
всесожжения для Господа. Мясо же сей жертвы ешь
в этот и в следующий дни, и не дай солнцу во второй
день зайти над ним, пока оно не съедено. И ничего
не должно оставлять на третий день, ибо это
неприятно и неугодно Господу и его нельзя уже
съедать. Все, которые будут есть его, понесут на
себе грех; ибо так нашел я написанным о сем в
книге моих праотцев, в словах Еноха и Ноя. На свою
плодовую жертву ты должен положить соли, и без
соли завета не должны быть оставляемы все твои
плодовые жертвы пред Господом.

И в отношении к жертвенным дровам ты должен
остерегаться, чтобы не принести какое-нибудь
другое жертвенное дерево, как только кипарис, и
ель, и миндаль, и сосна, и пихта, и кедр, и
можжевельник, и лимон, и маслина, и мирт, и лавр, и
кедр, называемый арбот, и бальзамовый кустарник.
Из этих пород деревьев полагай под всесожжение
на жертвеннике, после того как ты рассмотришь их
наружность, и не клади [...] разрушенного дерева; но
твердое и безукоризненное, лучшее и
новорастущее, и не старое, ибо запах у него исчез
и его нет уже в нем, как прежде. Кроме этих дров не
клади других, ибо они не имеют запаха. И да
вознесется от тебя воня благоухания их к небу.
Соблюдай сию заповедь и исполняй ее, сын мой,
чтобы поступать право во всяком своем деле.

И всякий раз будь чист своим телом и омывайся
водою, прежде чем приступишь принести жертву на
жертвеннике; омой руки и ноги, прежде чем
приблизиться к жертвеннику. И когда ты
приготовишь жертвоприношение, то опять омой руки
и ноги, чтобы не оказалось следов крови ни на вас,
ни на ваших одеждах. Будь очень осторожен, сын
мой, с кровью, будь очень осторожен. Закопай ее в
землю, и не ешьте крови, ибо она есть душа; совсем
не ешь крови.

И не бери выкупа за кровь какого-либо человека,
чтобы она не была пролита даром без наказания;
ибо эта кровь, которая проливается, делает землю
греховною, и она не может быть очищена от крови,
как только кровью того, кто пролил ее. И не
принимай выкупа и дара за человеческую кровь:
кровь за кровь; тогда она вас сделает угодными
Господу, всевышнему Богу, и он будет хранителем
блага, чтобы сохранять тебя от всякого зла и
спасать тебя от всякой смерти. Я вижу, сын мой, все
дела сынов человеческих, что они~--- грех и зло; и
всякое дело их~--- мерзость, и жестоковыйность, и
осквернение, и нет правды в нем. Берегись, не ходи
по их путям, и не следуй по стезям их, и не
совершай смертного греха пред всевышним Богом, а
не то отвратит Он лице Свое от тебя, и вменит тебе
вину твою, и истребит тебя в сей стране и твое
семя под небом, чтобы имя твое и семя твое исчезли
на всей земле. Удаляйся от всех дел их и всякой
мерзости их, и храни защиту Бога всевышнего, и
исполняй волю Его и поступай право во всем. Тогда
Он благословит тебя во всех твоих делах, и
произведет от тебя растение правды для всей
земли, на все роды земли. И будут знать имя мое и
имя твое под небом во все дни. Иди, сын мой, в мире;
да укрепит тебя всевышний Бог, Бог мой и Бог твой,
исполнять Его волю; да благословит Он все семя
твое и остаток твоего семени на вечные роды всеми
благословениями правды, дабы ты был благословен
на всей земле!>> И он вышел от него, исполненный
радости.

\vs Jub 22:1
И было в первую седмину сорок третьего юбилея
во второй год, т.е. в тот год, когда умер Авраам,
пришли Исаак и Измаил от клятвенного колодезя,
чтобы праздновать семидневный праздник, т.е.
праздник начатков жатвы, со своим отцом Авраамом.
И Авраам обрадовался, что пришли два его сына.
Именно, Исаак имел много имущества в Вирсавии, и
ходил туда, чтобы осмотреть свое имущество, и
возвратился теперь к своему отцу. И в эти дни
пришел Измаил, чтобы видеть своего отца; и они
пришли оба вместе. Тогда Исаак заколол жертву во
всесожжение, и принес ее на жертвеннике своего
отца, устроенном им в Хевроне, и принес жертву, и
сделал торжественный пир своему брату Измаилу. И
Ревекка приготовила новый хлеб из нового жита; и
она дала его Иакову, своему предпочтенному
сыну, чтобы он отнес своему отцу Аврааму первый
плод земли, дабы он ел и благословил Творца всех
вещей, прежде чем умрет. И Исаак также послал чрез
Иакова, предпочтенного, Аврааму от
благодарственной жертвы, чтобы он ел и пил. И он
ел и пил, и благословлял всевышнего Бога, Который
создал и небо и землю, и распростер всю землю, и
дал сынам человеческим пищу и питие. И он благословил
своего Творца: <<И ныне благодарю Тебя, Боже
мой, что Ты удостоил меня видеть сей день. Вот я
теперь ста семидесяти пяти лет, седой и
престарелый. И все мои дни~--- суд мира: меч
ненавистника не победил меня; [...] и во всем, что Ты
давал мне и моим детям во все дни жизни моей до
сего дня. Боже мой, да будет милость Твоя на рабе
Твоем и на семени его сыновей, чтобы оно было для
Тебя избранным народом и наследием пред всеми
народами земли, отныне до всех дней родов земли,
во все века>>.

И он подозвал Иакова и сказал ему: <<Сын мой
Иаков! да благословит тебя Бог всех вещей, и да
укрепит тебя~--- совершать правду и волю Его [...], и
изберет тебя и семя твое, чтобы вы были Ему
народом, как наследие Его, согласно Его воле! И
подойди сюда, сын мой Иаков, и поцелуй меня!>> И
он подошел и поцеловал его.

Тогда он сказал: <<Да будут благословлены
Иаков и все сыны его Господом, Всевышним, во все
века! Да даст тебе Господь семя правды от сынов
твоих, которое святило бы Его по всей земле!
Да послужат тебе и падут пред семенем твоим все
народы! Будь силен пред людьми! И так как ты
уподобишься во всем семени Сифа, то да будут пути
твои и пути сыновей твоих правыми, чтобы народ
твой был свят. Бог, Всевышний, да даст тебе все те
благословения, которыми Он благословил меня и
которыми благословил Ноя и Адама! Да покоятся они
на священном темени (главе) твоего потомства на
все роды и до всей вечности! И да сохранит тебя
Господь чистым от всякого мерзкого осквернения,
чтобы получить тебе прощение во всякой вине,
которую ты по неведению совершишь; и да укрепит
Он тебя и да благословит тебя, чтобы ты
наследовал всю землю. Да восстановит Он завет
Свой с тобою, чтобы ты был Ему народом наследия
Его во все века! И да будет Он тебе и семени твоему
Богом, в действительность и истину, во все дни
земли! Помни же, сын мой Иаков, слово мое и храни
заповедь Авраама, отца твоего! Не сообщайся с
народами, и не ешь с ними, и не поступай по делам
их, и не вступай в родство с ними, ибо (всякое) дело
их нечисто, и все пути их осквернены и суть
мерзость. Свои жертвы они закалают мертвым, и
почитают демонов, и едят на могилах; они лишены
мудрости, чтобы разуметь, и очи их ничего не
видят; как еще погрешать им, если они говорят
дереву. <<Ты бог мой>> и камню: <<Ты господь
мой и спаситель мой>>, тогда как они (дерево и
камень) не имеют разума? И ты, сын мой Иаков,~--- Бог,
Всевышний, да вспомоществует тебе, и Бог небесный
да благословит тебя и да удалит тебя от нечистоты
их и от всей греховности их! Берегись, сын мой
Иаков, чтобы не брать жены из всего семени
дочерей Ханаана; ибо семя его предназначено к
истреблению на земле; ибо за вину Хама и за
проступок Ханаана будет уничтожено и все семя
его, и весь остаток его, и что избегло гибели. И
все поклоняющиеся идолам и все упорствующие не
имеют надежды в земле живых, но они сойдут в
царство мертвых, и пойдут к месту осуждения, и не
оставят по себе памяти на земле. Как сыны Содома
были истреблены на земле, так будут истреблены
все, поклоняющиеся идолам. Не бойся, сын мой
Иаков, и не страшись! Бог, Всевышний, будет
охранять тебя от погибели, и от всякого пути
греховного Он спасет тебя. Здесь в этой стране построишь
мне дом, чтобы я положил имя мое на нем,
предназначено тебе и семени твоему вовек, и он
будет называться домом Авраама. Это
предназначено тебе и семени твоему вовек, ибо ты
построишь дом мой и имя мое поддержишь пред
Богом. Вовек будет пребывать семя твое и имя твое
во все роды земли>>. И он перестал изрекать
заповеди и благословения.

И они легли оба вместе на одно ложе, и Иаков
заснул при персях своего деда Авраама. И его душа
семь раз прижимала его к сердцу, и любовь его и
сердце его радовались о нем, и он благословил его
от всего сердца и сказал: <<Бог, Всевышний, Бог
всех вещей и Творец всего, изведший меня из Ура
Халдейского, чтобы дать мне эту страну, дабы я
владел ею вовек и воскресил святое семя, чтобы
оно было благословенно вовек! Благослови и сына
моего Иакова, о котором я радуюсь всем сердцем
моим и любовию моею! Твоя милость и Твоя благость
да пребудут на нем и на семени его всегда! Не
оставляй его и не покидай его отныне до века! И да
будут очи Твои открытыми на него и на его семя,
чтобы охранять его, и благослови его и освяти его
в народ наследия Твоего! Благослови его всеми
благословениями Твоими, отныне до всех дней
вечности; и восстанови завет Твой и милость Твою
с ним и семенем его, и всю волю Твою восстанови с
ним на все роды земли!>>

\vs Jub 23:1
И он положил два перста Иакова на свои очи, и
прославил Бога богов, и закрыл свои глаза. И он
простер ноги свои, и уснул вечным сном, и
приложился к отцам своим. И во все это время Иаков
лежал при его персях, не зная, что отец его Авраам
умер. И Иаков пробудился от своего сна, и вот~---
Авраам был холоден как лед; и он сказал: <<Отец,
отец!>>, но он не говорил. Тогда он узнал, что
Авраам умер, и он встал, и побежал, и известил о
сем мать свою Ревекку. И Ревекка пошла ночью к
Исааку, и известила его о сем. И они пошли вместе с
Иаковом, который имел светильник в руке своей; и
когда они вошли, то нашли лежащее тело Авраама. И
Исаак пал на лице отца своего, и плакал, и
благословлял его, и лобызал его; и вопль раздался
в доме Авраама. Тогда встал сын его Измаил, и
пришел к отцу своему Аврааму, и оплакивал отца
своего Авраама, он и весь дом Авраама; и они
подняли великий плач. И сыновья его Исаак и
Измаил погребли его в двойной пещере вместе с
женою его Сарой. И оплакивали его в продолжение
сорока дней все домочадцы его, Исаак и Измаил со
всеми сыновьями своими и сыновья Кетуры в местах
своего поселения. Потом окончился плач по
Аврааме.

И он жил три юбилея и четыре седмины, сто
семьдесят пять лет, и дни его окончились. Ибо дни
предков простирались до девятнадцати юбилеев; но
после дней потопа они начали уменьшаться и
становиться короче девятнадцати юбилеев. И они
(люди) стали скоро достигать старости и
пресыщаться жизнью вследствие многих несчастий
и вследствие неправды своих путей, исключая
Авраама; ибо Авраам был совершенным во всяком
своем деле с Богом и благоугоден, и (ходил) в
правде в продолжение своей жизни. И вот он не
окончил четырех юбилеев в своей жизни, так что
состарился ради неправды и пресытился жизнью. И
все роды, которые явятся отныне до дня великого
суда, будут скоро достигать старости, прежде чем
достигнут двух юбилеев. И так как и знание их
будет оставлять их по причине их престарелости,
то уменьшится все знание их. И в те дни, если кто
проживет полтора юбилея, то об нем будут
говорить: <<Он жил долго>>, но большая часть
его жизни пройдет в несчастии, и труде, и
страдании, и без мира; ибо наказание последует за
наказанием, мучение за мучением, нужда за нуждой,
зло за злом, болезнь за болезнью, и все таковые
злые наказания одно за другим: болезнь, и резь в
животе, и град, и лед, и снег [...], и несчастие, и
оцепенение, и неплодородие, и смерть, и меч, и
пленение, и всякие наказания и несчастия. Все это
придет на злой род, который наполнит беззаконием
землю чрез нечистоту блудодеяния и скверноты и
чрез мерзость своих деяний. И тогда будут
говорить: <<Жизнь предков продолжалась до
тысячи лет, и она была хороша; а дней нашей жизни,
если человек проживет долго, семьдесят лет, и
если они сильны, восемьдесят лет, и вся она
нехороша>>. И не будет мира во дни того злого
рода. И в том роде дети будут злословить своих
отцов и старцев за греховность и нечистоту, и за
речи их уст, и за великие нечестия, которые они
совершают, и за то, что они оставили завет,
который Господь заключил между ними и Собою, дабы
они соблюдали и хранили все Его заповеди, и
постановления, и весь закон Его, не уклоняясь ни
налево, ни направо; так что все они совершают
злое, и все уста их говорят беззаконное, и всякое
дело их нечистота и мерзость, и все пути их
осквернение, и нечистота, и погибель. Вот земля
погибнет ради всех дел их; и не будет более семени
от вина и елея, так как все дела их~--- полное
нечестие; и все вместе погибнут: дикие звери, и
скот, и птицы, и все морские рыбы~--- из-за сынов
человеческих. И они будут враждовать друг с
другом, эти с теми, юноши со старейшинами, и
старейшины с юношами, бедные с богатыми, и
униженные с великими, и нищий с князьями~--- именно,
относительно закона и завета; ибо они забыли Его
заповеди, и завет, и праздники, и новолуния, и
субботы, и юбилеи, и всякую правду. И они будут
восставать с мечами и луком, чтобы привести их
обратно на путь, но они не возвратятся, пока не
прольется много крови на земле; один будет против
другого, и те, которые останутся, не возвратятся
на путь правды от своего нечестия. Ибо все они
будут восставать, чтобы расхищать богатство, и
брать, что принадлежит другому, и приобретать
себе великое имя, но не для правды и истины; и
святое святых осквернят они мерзостью своего
осквернения. И придет великое осуждение ради дел
того рода от Господа, и Он предаст их мечу, и на
осуждение, и пленение, и расхищение, и пожрание. И
Он возбудит на них грешников~--- которые не знают
сострадания и милости и не взирают на лицо ни
старого, ни юного, ни другого кого-либо, но на злых
и могущественных людей,~--- чтобы они поступали
яростнее, чем все сыны человеческие, и совершали
насилие против Израиля и делали беззаконие
Иакову. И много крови прольется на земле. И не
будет никого собирающего и никого ближнего. В те
дни будут они издавать вопль, и взывать, и
умолять, чтобы их освободили от руки греховных
язычников, но не будет никого спасающего. И
головы детей будут белыми от седых волос, и
трехнедельное дитя будет казаться старым, как
столетний; и их существование будет приведено к
погибели чрез страдание и бедствие.

И в те дни дети начнут оставлять свои
(греховные) законы, и стремиться к заповедям, и
возвращаться на путь правды. И дни начнут
возрастать, и сыны человеческие будут достигать
большей старости, от рода до рода и от дня до дня,
так что время жизни их продлится тысячу лет. И не
будет старого и пресыщенного жизнью, но все они
будут как дети и отроки, и скончаются все дни их в
мире и радости, и будут они жить так, что тогда не
будет сатаны и какого-либо губителя; ибо все дни
их будут днями благословения и спасения. В то
время Господь исцелит своих слуг; и они
вознесутся и будут наслаждаться глубоким миром,
и опять преследовать своих врагов. И они увидят
это и будут благодарить и радоваться радостью до
века. И они увидят на своих врагах все наказания
их и все проклятие их; и хотя кости их будут
покоиться в земле, но для духа их будет многая
радость, и они познают, что это Господь,
совершающий суд и являющий милость на сотнях и
тысячах, и на всех, которые любят Его. И ты, Моисей,
запиши сие слово; ибо так начертано на
свидетельстве небесных скрижалей для вечных
родов.

\vs Jub 24:1
И было, после того как Авраам умер, Бог
благословил сына его Исаака. И он поднялся от
Хеврона и пошел дальше, и жил при кладезе видения,
в первый год третьей седмины этого юбилея, в
продолжение семи лет.

И в первый год четвертой седмины начался голод
в стране, сверх прежнего, который был во время
Авраама. И Иаков сварил чечевичное кушанье; тогда
пришел Исав с поля голодный. И он сказал брату
своему Иакову: <<Дай мне от этого кушанья
плод!>> И Иаков сказал ему: <<Передай мне твое
первородство, тогда я дам тебе хлеба и также плод
от этого кушанья>>. И Исав сказал в сердце своем:
<<Я должен умереть: что мне за польза быть
первородным?>> И он сказал Иакову: <<Я отдаю
тебе его>>. И Иаков сказал: <<Поклянись мне!>>
И он поклялся ему. И Иаков дал брату своему Исаву
хлеба и кушанья; и он ел, пока не насытился. Так
пренебрег Исав первородством; посему Исав
называется также Едомом ради плода кушанья,
которое Иаков дал ему за его первородство. И
Иаков сделался старшим; Исав же потерял свое
преимущество.

И был голод в стране; тогда Исаак пошел, чтобы
спуститься в Египет, во второй год этой седмины. И
он пошел к царю Филистимскому в Герарон к
Авимелеху. И Господь явился ему и сказал ему:
<<Не ходи в Египет, оставайся в стране, которую Я
указываю, и будь чужеземцем в оной стране; Я буду
с тобою и благословлю тебя. Ибо тебе и твоему
семени Я дам всю эту землю, и исполню клятву Свою,
которою Я поклялся отцу твоему Аврааму, и умножу
семя твое, как звезды небесные, и дам всю эту
землю твоему семени. И благословятся в твоем
семени все народы земли~--- за то, что отец твой
слушался гласа Моего, и соблюдал слово Мое, и Мои
заповеди, и законы, и установления, и завет. И
теперь и ты слушайся гласа Моего и Моих заповедей
и живи в этой стране!>>

И он жил в Герароне три седмины. И Авимелех
отдал приказание относительно него и
относительно всего имущества его и сказал:
<<Всякий, кто прикоснется к нему или к
чему-нибудь из его имения, умрет смертию>>. И
Исаак стал великим в Филистимской земле, и
приобрел много волов и овец, и верблюдов, и много
имущества. И они сеяли в земле филистимлян, и
получили прибыли во сто крат. И Исаак сделался
очень великим. И филистимляне стали завидовать
ему; и все колодези, которые отроки Авраама
выкопали при его жизни, филистимляне засыпали
после смерти Авраама и завалили их землею. И
Авимелех сказал Исааку: <<Дались от нас, ибо ты
стал великим для нас!>> И Исаак вышел оттуда в
первый год седьмой седмины и странствовал в
долинах Герарона. И они опять выкопали колодези,
которые отроки отца его Авраама выкопали, и
филистимляне после смерти отца его Авраама
засыпали. И он назвал их именем, которое нарек им
отец его Авраам. И отроки Исаака выкопали
колодези в долине и нашли источник воды. И
пастухи герарские спорили с пастухами Исаака,
говоря: <<Вода принадлежит нам>>. И Исаак
назвал место сего колодезя: <<противный>>,
<<ибо они враждовали с нами>>. И они выкопали
другой колодезь и спорили из-за него, и Исаак дал
ему имя~--- <<теснота>>. И он вышел оттуда, и они
выкопали другой колодезь, и о нем они не спорили;
тогда он дал ему имя~--- <<пространный>>. И Исаак
сказал: <<Теперь Господь распространил нас>>.
И он усилился в земле той и пришел к кладезю
клятвенному в первый год первой седмины в сорок
четвертый юбилей.

И Господь явился ему в ту ночь, в новолуние
первого месяца, и сказал ему: <<Я Бог Авраама,
отуа твоего; не бойся, ибо Я с тобою. И Я
благословлю тебя и умножу семя твое, как песок
морской, ради раба Моего Авраама>>. И он устроил
там жертвенник, где прежде устроил отец его
Авраам, и призвал имя Господа, и принес жертву
Богу отца своего Авраама. И они выкопали колодезь
и нашли источник воды. И отроки Исаака выкопали
еще колодезь и не нашли воды. И они пришли и
сказали Исааку, что не нашли воды. И Исаак сказал:
<<Я поклялся ныне филистимлянам, и это есть
причина (безводности кладезя)>>. И он дал имя
месту сему <<клятвенный колодезь>>. Ибо здесь
он поклялся Авимелеху и Акофу, другу его, и
Фиколу, надзирателю его. И Исаак познал в тот
день, что они ложно поклялись~--- хранить с ними мир.
И Исаак проклял в тот день филистимлян и сказал:
<<Да будут прокляты филистимляне в день гнева и
ярости всеми народами! Пусть сделает их Господь
посмешищем, и проклятием, и гневом, и яростью в
руках грешных язычников и истребит их рукою
Хеттеев. И что избегнет меча врагов и Хеттеев, то
да истребит народ праведных судом праведным под
небом. Ибо врагами и ненавистниками будут они для
сынов моих во дни их и в земле их. И никто из
них не останется и никто не спасется в день
гневного суда. Но погибнет, и истребится, и будет
уничтожено в стране сей все семя филистимлян, и
от них не останется более и потомства их на земле.
Если бы оно взошло даже на небо, то да
низвергнется оттуда; и если бы оно утвердилось на
земле, то да будет исторгнуто, и если бы укрылось
между народами, то да будет истреблено и отсюда, и
если бы оно взошло в царство мертвых, то и там да
будет велико его наказание, и пусть не будет ему
там мира, и если бы оно странствовало в плену, то
да будет умерщвлено теми, которые подстерегают
на пути его душу. Не оставляй ему Ты, Который да
будешь прославлен, имени и семени на всей земле, и
да сопровождает его вечное проклятие!>> И
относительно него написано и начертано на
небесных скрижалях, чтобы так было поступлено с
ним в день суда, дабы оно было истреблено на
земле.

\vs Jub 25:1
И во второй год этой седмины в этом юбилее
Ревекка призвала сына своего Иакова и беседовала
с ним, говоря: <<Сын мой, не бери себе жену из
дочерей Ханаана, как брат твой Исав, который взял
себе двух жен из семени Ханаана; и они поразили
мой дух всеми делами своими, нечистотою блуда и
брака, и нет правды в них, но злы дела их. И я
весьма люблю тебя, сын мой; моя нежность
благословляет тебя каждый час и стражу нощную. И
ныне послушай гласа моего, и исполни волю матери
твоей, и не бери себе жену из дочерей сей страны, а
только из дома отца твоего и из рода отца твоего.
Возьми себе жену из дома отца моего; и Бог
всевышний благословит тебя и сынов твоих сделает
праведным родом и семя твое святым>>. После сего
Иаков говорил с матерью своей Ревеккою и сказал
ей: <<Вот, мать моя, мне девять седмин, и я не знаю
жены; ни одна не прикасалась ко мне и не
обручалась мне, и я не думаю брать себе жену из
какого-либо семени дочерей Ханаана, ибо я помню
слова отца нашего Авраама и что он заповедал мне,
что я не должен брать жены из всего семени дома
Ханаана. Но я хочу взять себе жену из семени дома
отца моего и из рода моего. Я слышал прежде, что
брат твой Лаван имеет в потомстве дочерей. На них
обратил я свое сердце, чтобы из них взять жену. И
посему я остерегался в своем духе, чтобы не
согрешить и не развратиться на всех путях моих,
во все дни жизни моей. Ибо относительно брака и
блуда отец мой Авраам дал мне много заповедей. И
вопреки всему тому, что он заповедал мне, брат мой
теперь прекословит мне в продолжение двадцати
двух лет и говорит часто мне, говоря: <<Брат мой,
возьми в жены сестру моих двух жен!>> Но я не
хочу делать так, как делает брат мой. Я клянусь
пред тобою, что я в продолжение всей моей жизни не
возьму себе жену из семени всех дочерей Ханаана и
не поступлю худо, как поступил брат мой. Не бойся,
мать моя! Поверь мне, что я исполню волю твою, и
буду ходить в праведности, и не извращу моих
путей вовек!>>

После сего она возвела лицо свое на небо, и
распростерла персты своей руки, и открыла уста
свои, и прославила Бога всевышнего, сотворившего
небо и землю, и принесла Ему благодарение и хвалу,
и сказала: <<Да будет прославлен Господь, Бог
мой, и да прославится имя Его вовек, что Он дал мне
Иакова, невинного сына и святое семя; ибо он Твой,
и семя его всегда Твое во все роды вовек.
Благослови его, Владыка, и вложи благословение
правды в уста мои, чтобы я благословила его!>> В
тот самый час дух святой сошел в уста ее, и она
положила руки свои на главу Иакова и сказала:
<<Будь прославлен Ты, Господь правды и Бог
миров, и да прославляют Тебя люди всех родов! Да
дарует Он тебе, сын мой, путь правды, и да откроет
семени твоему правду, и умножит сынов твоих во
время жизни твоей, и восставит их по числу
месяцев года! И да умножатся сыны их, и будут
бесчисленны, как звезды небесные, и да будет
число их больше, чем песок морской! Да даст Он тебе
эту плодоносную землю, как сказал Он, что Он
даст ее Аврааму и семени его после него навсегда
и что они вечно будут владеть ею. И да увижу я в
тебе, сын мой, благословенного сына во время
жизни моей; и все семя твое да будет святым
семенем! И как покоил тебя дух твоей матери во
время жизни ее на лоне родившей тебя, так
благословляет тебя моя нежность, и перси мои
благословляют тебя, и уста мои, и язык мой
прославляют тебя. Умножайся, и возрастай, и
распространяйся на земле, и да будет семя твое
совершенным в небесной и земной радости во всю
вечность! И да ликует семя твое, и в великие дни
мира да будет ему уделом мир твоего имени! И да
пребывает семя твое во всю вечность; и Бог
всевышний да будет их Богом, и Бог всевышний да
обитает с ними, и да будет устроено между ними
святилище Его на все века! Благословляющий тебя
да будет благословен, и всякая плоть,
проклинающая тебя напрасно, да будет
проклята!>> И она поцеловала его и сказала ему:
<<Господь мира да возлюбит тебя, как сердце
матери твоей и нежность ее; да возрадуется Он о
тебе и да благословит тебя!>> И вот она
перестала благословлять его.

\vs Jub 26:1
И в седьмой год этой седмины Исаак призвал
старшего сына своего Исава и сказал ему: <<Сын
мой, я стар, и вот очи мои притупились для зрения,
и я не знаю, когда умру. И теперь возьми свои
охотничьи орудия, свой колчан и лук, и выйди в
поле, и добудь дичи для меня, и налови мне
что-нибудь, сын мой, и приготовь мне кушанье, как
любит душа моя, и принеси его мне, чтобы я ел, и
душа моя благословила тебя, прежде чем я умру>>.
И Ревекка услышала речь его, когда Исаак говорил
Исаву. И Исав вышел рано в поле, чтобы добыть дичи,
и наловить, и принести своему отцу.

И Ревекка позвала сына своего Иакова и сказала
ему: <<Вот я слышала, как отец твой Исаак говорил
с братом твоим Исавом, говоря: <<Налови мне
какой-нибудь дичи, и приготовь мне кушанье, и
принеси его мне, чтобы я благословил тебя пред
Господом, прежде чем я умру>>. И теперь выслушай
слово мое, сын мой, что я тебе велю! Ступай в свое
стадо и принеси мне двух хороших козлят, я
приготовлю из них кушанье, как он любит. И ты
отнесешь его отцу твоему поесть, дабы он
благословил тебя пред Господом, прежде чем умрет,
и ты будешь благословен!>> И Иаков сказал матери
своей Ревекке: <<Мать, я ничего не жалею, что
отец мой может есть н что ему угодно. Только я
боюсь, мать моя, как бы он не узнал моего голоса и
не ощупал меня; ты знаешь ведь, что я гладкий, а
брат мой Исав волосат; и как бы мне не явиться в
его очах преступником, и не сделать чего-либо
такого, чего он не повелел мне, и как бы он не
разгневался на меня, и я навлеку на себя
проклятие, а не благословение>>. И мать его
Ревекка сказала ему: <<Пусть на меня придет твое
проклятие, сын мой; скорее послушайся гласа
моего!>>

И Иаков послушался гласа матери своей Ревекки,
и пошел, и сходил за двумя хорошими тучными
козлятами, и принес их матери своей, и мать его
приготовила их, как он любил. И Ревекка взяла
одежды старшего сына своего Исава, самые дорогие,
какие были у нее в доме, и одела у себя в них
Иакова, и кожею козлят обложила его руки и
открытые части его тела. И она дала кушанье и
обед, который приготовила, в руки сыну своему
Иакову; и он вошел к своему отцу и сказал: <<Я,
сын твой, сделал, что ты сказал мне; встань и сядь,
и поешь того, что я наловил, отец, чтобы душа твоя
благословила меня>>. И Исаак сказал сыну своему:
<<Как это ты так скоро наловил дичи, сын мой?>>
И Иаков сказал: <<Пославший мне это, Бог твой,
был со мною>>. И Исаак сказал ему: <<Подойди
сюда, чтобы я тебя ощупал, сын мой, сын ли ты мой
Исав или нет>>. И Иаков подошел к отцу своему
Исааку, и он ощупал его и сказал: <<Голос~--- голос
Иакова, но руки Исава>>; и он не узнал его; ибо
было соизволение (послание) с неба, которое
отняло чувство его. И Исаак не узнал его, ибо руки
его были, как руки того, и волосаты, как руки
Исава, дабы он благословил его. И он сказал:
<<Сын ли ты мой?>> И он сказал: <<Я сын твой>>.
И он сказал: <<Подай мне поесть того, что наловил
ты, сын мой, дабы душа моя благословила тебя!>> И
он поднес ему, и он ел; и он подал ему вина, и он
пил. И отец его Исаак сказал: <<Подойди и поцелуй
меня, сын мой!>> И он подошел и поцеловал его; и
он ощутил запах одежды его. И он благословил его и
сказал: <<Вот запах от сына моего, как запах от
поля, которое благословил Господь. Да даст тебе
Господь и сделает жребием твоим много росы
небесной и плодородия земли, и много хлеба; и
масла да даст тебе Он в изобилии! И да послужат
тебе народы, и люди (племена) да поклонятся тебе!
Ты будешь господином над братьями своими, и сыны
матери твоей да поклонятся тебе! И все
благословения, которыми Господь благословил
меня и отца моего Авраама, да будут на тебе и
семени твоем до века! Проклинающий тебя да будет
проклят, и благословляющий тебя да будет
благословен!>>

И после того, как Исаак кончил благословлять
сына своего Иакова, и Иаков вышел от отца своего
Исаака и скрылся, пришел его брат Исав с охоты; и
он также приготовил кушанье, и принес его отцу
своему, и сказал отцу своему: <<Отец мой, встань
и поешь моей дичи, чтобы душа твоя благословила
меня!>> И отец его Исаак сказал ему: <<Кто
ты?>> И он сказал: <<Я первенец твой Исав, я
сделал, как ты повелел мне>>. И Исаак
вострепетал великим трепетом и сказал: <<Кто же
тот, который мне добыл дичи, и наловил, и принес,
чтобы я ел от всего, прежде чем ты пришел, и я
благословил его? Да будет благословен он и все
семя его вовек!>> И когда Исав услышал речь отца
своего Исаака, то поднял громкий вопль, горько
сетуя, и сказал отцу своему: <<Благослови и меня,
отец!>> И он сказал ему: <<Твой брат пришел
хитростью и восхитил благословения твои>>. И
Исав сказал: <<Теперь я знаю, почему он
называется Иаковом; дважды он теперь запнул меня:
в первый раз он взял мое первородство, а теперь он
взял у меня мое благословение>>. И он сказал:
<<Разве ты не оставил для меня благословения,
отец?>> И Исаак отвечал и сказал Исаву: <<Вот я
сделал его господином над тобою и над всеми его
братьями, и дал их ему в рабы; изобилием хлеба и
масла я сделал его сильным; что я теперь сделаю
тебе, сын мой?>> И Исав сказал отцу своему
Исааку: <<Разве у тебя только одно
благословение, отец? Благослови и меня, отец!>> И
Исав возвысил голос свой и заплакал. И Исаак
отвечал и сказал ему: <<Вот от тучности земли
будет благословение твое и от росы небесной
свыше; своим мечом будешь питаться ты и будешь
служить твоему брату. И будет, если ты сделаешься
великим и свергнешь ярмо его с выи твоей, то
совершишь смертный грех, и все твое семя будет
истреблено под небом>>. И Исав разгневался на
Иакова за благословение, которым отец его
благословил его, и сказал в сердце своем:
<<Теперь придут дни плача по отце моем, и я убью
брата моего Иакова>>.

\vs Jub 27:1
Тогда было открыто во сне Ревекке слово Исава,
старшего ее сына. И Ревекка послала и призвала
Иакова, старшего сына своего, и сказала ему:
<<Вот брат твой Исав замышляет мщение, чтобы
убить тебя. И ныне послушайся гласа моего, встань
и беги к брату моему Лавану, и оставайся у него
некоторое время, пока не переменится гнев брата
твоего, и он не оставит гнева своего против тебя,
и забудет все, что ты сделал ему, и тогда я пошлю и
вызову тебя оттуда>>. И Иаков сказал: <<Я не
боюсь; если он хочет убить меня, то я сам убью
его>>. И она сказала: <<Так я лишилась бы в один
день обоих моих сыновей>>. И Иаков сказал своей
матери Ревекке: <<Вот ты знаешь, что мой отец
стар, и я вижу, что очи его ослабели; и если я
покину его, то это будет злом пред очами его, что я
оставлю его и уйду от вас; и отец мой разгневается
и проклянет меня. Я не могу идти; только если он
меня отошлет, чтобы я шел, то я пойду>>. И Ревекка
сказала Иакову: <<Я войду и скажу ему это, чтобы
он отпустил тебя>>. И Ревекка вошла и сказала
Исааку: <<Мне стала противною моя жизнь из-за
обеих дочерей Хета, которых Исав взял себе в жены,
и если Иаков возьмет себе жену между дочерями этой
страны, которые такие же, как и те, то зачем
мне еще жить? ибо дочери земли Ханаанской злы>>.
И Исаак призвал своего сына Иакова, и благословил
его, и увещевал его, и сказал ему: <<Не бери себе
жену из всех дочерей Ханаана; соберись, иди в
Месопотамию, в дом Бефуела, отца твоей матери, и
возьми себе оттуда жену из дочерей Лавана, брата
твоей матери. И Бог небесный да благословит тебя,
и возрастит тебя, и умножит тебя, чтобы ты
сделался обществом народов. И да даст Он тебе
благословения отца моего Авраама, тебе и твоему
семени после тебя, дабы ты наследовал землю
твоего странствования и всю землю, которую
Господь дал Аврааму. Иди, сын мой, с миром!>> И
Исаак отпустил Иакова, и он пошел в Месопотамию к
Лавану, сыну Бефуела, сирийцу, брату Ревекки,
матери Иакова.

И было, когда Иаков собрался идти в Месопотамию,
Ревекка опечалилась о своем сыне и плакала. И
Исаак сказал Ревекке: <<Сестра моя! не плачь о
моем сыне Иакове, ибо с миром он пойдет и с миром
возвратится. Бог всевышний охранит его от
всякого зла, и будет с ним, и не оставит его во все
дни; ибо я знаю, что Господь даст успех в путях
его, повсюду, где он пойдет, пока не возвратится к
нам с миром, и мы увидим его в мире. Не бойся за
него, сестра моя, ибо путь его прямой, и он муж
благочестивый и верный, и посему не погибнет.
Не плачь!>> И Исаак утешал Ревекку о сыне ее
Иакове и благословил его.

И Иаков вышел от клятвенного колодезя, чтобы
идти в Харран, в первый год второй седмины сорок
четвертого юбилея, и пришел в Лозу на горе, т.е. в
Вефиль, в новолуние первого месяца, в эту седмину;
и дошел до некоторого места, когда был вечер.
И он уклонился несколько к западу от дороги в ту
ночь и заснул здесь, ибо солнце зашло. И он взял
один из камней того места и положил его под
дерево,~--- ибо он странствовал один~--- и заснул, и
видел сон в ту ночь. И вот лестница была
утверждена на земле, вершина которой досягала до
неба. И вот Ангелы Господни поднимались и
опускались по ней, и сам Господь стоял на ней. И
Господь говорил с Иаковом и сказал: <<Я Бог отца
твоего Авраама и Бог Исаака; землю, на которой ты
стоишь, Я дам тебе и твоему семени после тебя; и
твое семя будет как пыль земная, и ты
размножишься к западу и востоку, и югу, и северу, и
благословятся в тебе и твоем семени все страны
народов. И вот Я буду с тобою, и буду охранять тебя
повсюду, где ты будешь ходить, и возвращу тебя в
мире в эту землю. Ибо Я не оставлю тебя, пока не
исполню все, что Я сказал тебе>>. И Иаков спал
(пробудился) и сказал: <<Точно, это место~--- дом
Господа, и я не знал сего>>. И он убоялся и
сказал: <<Священно это место, на котором ничего
нет иного, как только дом Господень; и это врата
небесные>>. И утром рано встал Иаков и взял
камень, который он положил себе в изголовье, и
поставил его памятником в знамение на этом месте,
и возлил на него сверху елей, и нарек имя тому
месту Вефиль. Прежде же оно называлось Луз, как
страна. И Иаков дал Богу обет, говоря: <<Если
Господь будет со мною, и сохранит меня на том
пути, в который я иду, и даст мне хлеб в пищу и
одежду для одеяния, и я возвращусь в мире в
дом отца моего, то да будет Господь моим Богом, и
также камень этот, который я поставил в этом
месте памятником в знамение, да будет домом
Господним! И из всего, что Ты дашь мне, я дам
десятую часть Тебе, Боже мой!>>

\vs Jub 28:1
И он встал и пошел в Месопотамию, в землю Лавана,
брата Ревекки, лежащую к востоку. И он оставался у
него и служил ему за Рахиль, одну из дочерей его. И
в первый год третьей седмины сказал он ему:
<<Дай мне мою жену, за которую я служил тебе семь
лет>>. И Лаван сказал Иакову: <<Я отдам тебе
твою жену>>. И Лаван устроил пир, и взял Лию, свою
старшую дочь, и отдал ее Иакову в жены, и дал ей
свою рабу Залафу в служанки. И Иаков не заметил
этого, ибо он думал, что это Рахиль. И он вошел к
ней, и вот это была Лия. Тогда Иаков разгневался
на Лавана и сказал ему: <<Зачем ты сделал так? Не
служил ли я тебе за Рахиль, а не за Лию? Зачем ты
обидел меня? Возьми свою дочь и отпусти меня, ибо
ты нехорошо поступил со мною>>. А Иаков любил
Рахиль больше, чем Лию. Ибо глаза Лии были слабы,
но лицом она была очень красива. Рахиль же имела
прекрасные глаза, и лицом она была очень красива
и привлекательна. И Лаван сказал Иакову: <<В
нашей стране нет такого обычая, чтобы выдавать
младшую дочь прежде старшей, и несправедливо
делать это. Ибо так сие определено и написано в
небесных скрижалях, и неправеден тот, кто делает
это, ибо это нехорошее дело пред Господом. И ты
также с своей стороны скажи сынам Израиля, чтобы
они не делали этого, и не позволяли брать и
выдавать младшую, прежде чем выдадут старшую; ибо
это очень нехорошо>>. И Лаван сказал Иакову;
<<Пусть пройдут семь дней пиршества, тогда я дам
тебе Рахиль, чтобы ты служил мне другие семь лет,
чтобы ты пас моих овец, как ты служил в течение
первой седмины (семилетия). [Когда же семь дней
пиршества Лии прошли], Лаван дал Иакову Рахиль,
чтобы он служил ему другие семь лет. И Рахили он
дал в служанки Баллу, сестру Залафы. И он служил
еще семь лет за Рахиль. [...].

И Господь отверз утробу Лии, и она зачала, и
родила Иакову сына, и он дал ему имя Робел, в
четырнадцатый день девятого месяца в первый год
третьей седмины. Утроба же Рахили была заключена,
ибо Господь видел, что Лия была ненавидима, а
Рахиль любима. И Иаков опять вошел к Лии, и она
зачала, и родила Иакову второго сына, и он дал ему
имя Симеон в двадцать первый день десятого
месяца в третий год этой седмины. И Иаков опять
вошел к Лии, и она зачала, и родила ему третьего
сына, и он дал ему имя Левий в новолуние первого
месяца в шестой год этой седмины. И Иаков опять
вошел к Лии, и она зачала, и родила ему четвертого
сына, и он дал ему имя Иуда в пятнадцатый день
третьего месяца в первый год четвертой седмины. И
ради всего этого Рахиль позавидовала Лии, так как
сама она не рождала. И она сказала Иакову: <<Дай
мне сына!>> И Иаков сказал: <<Разве я задержал
плод тебе, плод утробы твоей, разве я покинул
тебя?>> И когда Рахиль увидела, что Лия родила
Иакову четверых детей~--- Робела, Симеона, Левия и
Иуду,~--- то Рахиль сказала ему: <<Войди к моей
служанке Балле, чтобы она зачала и родила мне
сына!>> И он вошел, и она зачала и родила ему
сына, и она нарекла ему имя Дан в девятый день
шестого месяца в шестой год третьей седмины. И
Иаков опять вошел к Балле, и она зачала и родила
Иакову второго сына, и Рахиль дала ему имя
Наффали в пятый день седьмого месяца во второй
год четвертой седмины. И когда Лия увидела, что
она стала неплодною и не рождала более, то
позавидовала, и дала точно так же свою служанку
Залафу Иакову в жены; и она зачала и родила сына, и
она дала ему имя Асер в двенадцатый день восьмого
месяца в третий год четвертой седмины. И опять он
вошел к ней, и она зачала и родила ему второго
сына; и Лия дала ему имя Исашар во второй день
одиннадцатого месяца в пятый год четвертой
седмины. И Иаков вошел к Лии, и она зачала и родила
Иакову сына, и он дал ему имя Заблон в четвертый
день пятого месяца в четвертый год четвертой
седмины; и она передала его няньке. И Иаков опять
вошел к ней, и она зачала и родила двоих детей,
сына и дочь, и дала имя ему Заблон и дочери Дина в
седьмой день седьмого месяца в шестой год
четвертой седмины. И Господь умилостивился над
Рахилью и отверз утробу ее, и она зачала и родила
сына, и дала ему имя Иосиф в новолуние четвертого
месяца в шестой год этой четвертой седмины.

И когда родился Иосиф, Иаков сказал Лавану:
<<Дай мне моих жен и детей, чтобы идти мне к отцу
моему Исааку и чтобы он (?) сделал мне дом; ибо я
кончил годы, которые должен был служить тебе за
двух твоих дочерей, и я хочу идти в дом моего
отца>>. И Лаван сказал Иакову: <<Останься у
меня за вознаграждение, и паси опять у меня мои
стада, и возьми себе вознаграждение>>. И они
договорились друг с другом, чтобы он дал ему в
вознаграждение из овец и коз всех, которые [...]. И
овцы опять родили других, подобных себе, и все
были со знаком Иакова, и ни одна со знаком Лавана.
И имущество Иакова очень увеличилось. И он
приобрел себе рогатого скота, и овец, и ослов, и
верблюдов, и рабов, и служанок. И Лаван вместе с
своими сыновьями стали завидовать Иакову. И
Лаван отнял у него овец и замышлял злое против
него.

\vs Jub 29:1
И случилось, когда Рахиль родила Иосифа, пошел
Лаван стричь своих овец на расстояние трех дней
пути. И Иаков увидел, что Лаван пошел стричь своих
овец, и призвал Баллу и Рахиль, и уговаривал их
идти с ним в землю Ханаанскую; он рассказал им
именно все, что он видел во сне, и все, что Он
говорил с ним, чтобы он возвратился в дом своего
отца. И они сказали: <<Мы пойдем в то место; куда пойдешь
ты, пойдем и мы с тобою>>. И Иаков прославил
Бога отца своего Исаака и Бога деда своего
Авраама, и собрался, и посадил на верблюдов своих
жен и детей, и взял все свое имущество, и
переправился через реку, и пришел в землю
Гилеадскую. Но Иаков скрыл свой замысел от Лавана
и ничего не сказал ему об этом. В седьмой год
четвертой седмины Иаков возвратился в Гилеад, в
двадцать первый день первого месяца. И Лаван
преследовал его и настиг Иакова на горе Гилеад в
тринадцатый день третьего месяца. Но Господь не
допустил, чтобы он причинил вред Иакову; ибо Он
явился ему во сне ночью. И Лаван говорил с
Иаковом. И в пятнадцатый день того месяца сделал
Иаков Лавану и всем, которые пришли с ним,
пиршество. И Иаков поклялся Лавану в тот день, и
Лаван Иакову, что они не перейдут гору Гилеад с
злым умыслом друг против друга. И он устроил там
большой каменный холм во свидетельство; посему
дано имя тому месту~--- <<каменный холм
свидетельства>>. [...]. Прежде же звали землю
Гилеад землею Рефаил, ибо она была страною
Рефаимов, и рождались там Рефаимы-исполины,
которые были высотою до десяти, девяти, восьми,
семи локтей, и жилища которых простирались от
земли Аммонитян до горы Гермон, и главные города
которых были Хоронаим, и Астарос, и Эдрао, и Мисур,
и Беон. И Господь истребил их за нечестие их дел,
ибо они были очень мерзкими. И они оставили ее
(страну) вместо себя Аморреям, злому и греховному
народу; и нет ныне никакого другого народа, который
совершил бы все их грехи; посему они не имеют
долгой жизни на земле.

И Иаков отпустил Лавана в Месопотамию, в
восточную страну; и Иаков с своей стороны
направился в землю Гилеадскую и перешел Иаббок в
девятый месяц в одиннадцатый день его. И в этот
день пришел к нему брат его Исав, и они прекратили
свою распрю. И он ушел от него в землю Сеир, а
Иаков жил в шатрах. И в первый год в пятую седмину
в этот юбилей перешел он Иордан, и жил по ту
сторону Иордана, и пас свои стада от моря [...] до
Бефазона, и Дафаама, и Акрабита. И он посылал отцу
своему Исааку от всего своего имения одеяние, и
пищи, и мяса, и питья, и молока, и масла, и сыра, и
плодов от всяких пальм долины; и также матери своей
Ревекке он посылал четыре раза в год, между
месячными периодами, между пашней и жатвой, между
весной и дождем, между зимой и летом. И он (Исаак)
жил в башне Авраама; ибо Исаак возратился от
клятвенного колодезя и пошел в башню своего отца
Авраама, и жил здесь без (далеко от) Исава, своего
сына. К тому времени, когда Иаков отправился в
Месопотамию, Исав взял себе Маалиф, дочь Измаила,
в жены, и собрал все стада своего отца и своих жен,
и поднялся, и жил в горе Сеир, и оставил отца
своего Исаака одного при клятвенном колодезе. И
Исаак поднялся от клятвенного колодезя, и жил в
башне Авраама, отца своего, в горе Хеврон. И сюда
посылал Иаков все, что он от времени до времени
посылал своему отцу и своей матери, чтобы
облегчить всякую их скорбь. И они
благословляли Иакова от всего сердца и от всей
души.

\vs Jub 30:1
И в первый год шестой седмины поднялся он в
Салем, который находится на востоке от Сихема, с
миром, в четвертый месяц. И там увезли они Дину,
дочь Иакова, в дом Сихема, сына Емора, Гевитянина,
владетеля страны; и он спал с нею и обесчестил ее.
И она была маленькая девушка двенадцати лет. И он
просил ее отца, чтобы она была отдана ему в жены, и
у ее братьев он просил ее себе. Но Иаков и его
сыновья разгневались на сихемских мужей, которые
обесчестили их сестру Дину. И они замыслили между
собою нечто злое, и перехитрили, и обманули их. И
Симеон и Левий пришли тайно в Сихем, и совершили
наказание над всеми сихемскими мужьями, и убили
всех мужей, которых нашли в нем, и не оставили в
нем ни одного. Всех предали они мучительной
смерти, так как они обесчестили сестру их Дину.

И вы не должны отныне более делать так~---
бесчестить дочерей Израиля! Ибо на небе было
определено против них наказание, чтобы истребить
мечом всех сихемских мужей, так как они причинили
дочери Израиля бесчестие. И Господь предал их в
руки сыновей Иакова, чтобы они истребили их мечом
и совершили над ними наказание. И пусть никогда
не случится более в Израиле что-либо подобное
тому, чтобы бесчестили израильскую девицу. И если
муж в Израиле отдаст свою дочь или сестру
какому-либо мужу от семени язычников, или отдал,
то да умрет он смертию, и его должно побить
камнями, ибо он совершил грех и бесчестие
Израилю. И жену должно сожечь огнем, ибо она
осквернила имя дома своего отца, и она должна
быть истреблена из Израиля. И да не обретется
мерзость и блуд во Израиле во все роды земли, ибо
Израиль свят Господу. И каждый человек, который
совершит мерзость, должен умереть смертию и быть
побитым камнями. Ибо так утверждено это и
написано на небесных скрижалях относительно
всего семени Израиля: кто совершит мерзость, тот
должен умереть, смертию и быть побитым камнями. И
для сего закона нет конца дней, и прекращения, и
послабления, но непременно должен быть истреблен
тот муж, который осквернил свою дочь, во всем
Израиле, ибо он от своего семени дал Молоху и
совершил вину, осквернив его (семя). И ты, Моисей,
скажи сынам Израиля и положи свидетельство
против них, чтобы они не отдавали дочерей своих
язычникам и не брали дочерей языческих; ибо это
преступно пред Господом. Посему я записал тебе во
всех словах закона все деяние Сихемлян, что они
сделали с Диной, и как сыновья Иакова совещались,
говоря: <<Мы не отдадим нашу дочь (?) мужам
необрезанным, ибо это~--- поношение для нас и для
Израиля, если отдать ее или если взять из дочерей
языческих; ибо это нечисто и преступно для
Израиля, и Израиль не был бы чистым>>. И за эту
нечистоту, что один имеет жену из дочерей
языческих или что один отдает из своих дочерей
мужу из разного рода язычников, придут мучение за
мучением, и проклятие за проклятием, и все
наказания, и мучения, и проклятия. И если ты это
сделаешь, а он (народ) будет закрывать свои глаза, чтобы
не видеть тех, которые совершают мерзость, и
делают нечистым храм Господень, и оскверняют
святое имя, то весь народ должен быть наказан за
всю эту мерзость и осквернение, и не должно быть
допускаемо никакого лицеприятия и никакого
снисхождения, и не должны быть принимаемы от его
рук плоды, и плодовые жертвы, и всесожжения, и тук,
и жертвы курения в добрую воню, чтобы он
(согрешивший) был угоден. Так да будет с каждым
мужем и женщиною во Израиле, которые оскверняют
храм Его. Ради сего я дал тебе повеление, говоря:
<<Засвидетельствуй Израилю, что было
засвидетельствовано: вот как поступлено с
Сихемлянами и их сыновьями, как они были преданы
в руки двоих сыновей Иакова и были преданы
мучительной смерти. И это послужило им к правде, и
семя Левиино было избрано во священники и левиты,
чтобы они служили пред Господом, как мы, во все
дни. И да будет благословен Левий с его сыновьями
вовек, ибо они возревновали, чтобы совершить
правду, и суд, и мщение в отношении ко всем,
которые восстают против Израиля. И таким образом
отмечаются мужу в свидетельстве небесных
скрижалей благословение и справедливость пред
Ним, Богом всех вещей; и мы также будем вспоминать
правду, которую он совершил в своей жизни, во все
времена до тысячи родов. Благословение
предначертано ему, и оно придет на него~--- на него и
его род после него; и он будет записан, как друг и
праведник, на небесных скрижалях. И все это
событие я записал тебе, и дал тебе повеление,
чтобы ты открыл его сынам Израиля, дабы они не
совершали вины и не преступали закона, и не
разрушали завета, заключенного с ними, дабы они
хранили его, и были записаны друзьями. Если же они
преступят и будут ходить по всем путям нечистоты,
то будут написаны на небесных скрижалях врагами,
чтобы быть изглаженными из книги живых и
записанными в книгу тех, которые будут
уничтожены, и вместе с теми, которые будут
истреблены в стране>>. В тот день, когда сыновья
Иакова умертвили Сихемлян, было начертано им это
в книге на небе, что они совершили
справедливость, и правду, и мщение в отношении к
грешникам, и было записано им в благословение.

И они увели сестру свою Дину из дома Сихема. И
они увели в добычу все, что было в Сихеме: овец их,
и рогатый скот, и ослов, и все их имущество, и все
стада их~--- и привели все это к своему отцу Иакову.
И он совещался с ними о разрушении города; ибо они
страшились жителей страны~--- Хананеев и Ферезеев.
Но пришел страх Божий на все города вокруг
Сихемлян, так что они не решились изгнать сыновей
Иакова, ибо напало на них смущение.

\vs Jub 31:1
И в новолуние [...] месяца говорил Иаков со всеми
своими домочадцами, говоря: <<Очиститесь и
смените ваши одежды; соберемтесь и пойдемте в
Вефиль, где я дал обет в тот день, когда бежал от
моего брата Исава, ибо Он был со мною и возвратил
меня в мире в эту страну. И удалите чуждых богов,
которые у вас, и бросьте чуждых богов, и что
имеете на шеях и в ушах, и идола, которого Рахиль
украла у своего отца Лавана!>> И она (?) отдала
все Иакову и [...]. И он разбил и уничтожил их, и
оставил их под теревинфом в стране Сихемлян.

И в новолуние седьмого месяца пошел он в Вефиль
и устроил жертвенник на том месте, где он спал, и
поставил памятник. И он послал к отцу своему
Исааку, чтобы он пришел к нему к его жертве, а
также и к своей матери Ревекке. И Исаак сказал:
<<Пусть придет сын мой Иаков, чтобы я увидел его,
прежде чем умру>>. И Иаков пошел к своему отцу
Исааку и к своей матери Ревекке в дом отца его
Авраама, и взял с собою двоих из своих сыновей~---
Левия и Иуду, и пришел к отцу своему Исааку и к
своей матери Ревекке. И Ревекка вышла из башни к
нему, чтобы поцеловать и обнять Иакова; ибо ожил
дух ее, как только она услышала: <<Вот пришел сын
твой Иаков>>. И она поцеловала его, и увидела
двоих сыновей его, и узнала их, и сказала ему:
<<Это твои сыновья, сын мой?>> И она обняла их, и
поцеловала их, и благословила их, говоря: <<Да
будет чрез вас честь семени Авраама, и да будете
вы во благословение на земле!>> И Иаков вошел к
отцу своему Исааку в покой его, где он спал, и двое
сыновей его с ним. И он взял руку отца своего, и
наклонился, и поцеловал его; и Исаак пал на шею
сыну своему Иакову и плакал на шее его. И мрак сошел
с очей Исаака, и он увидел обоих сыновей Иакова
- Левия и Иуду, и сказал: <<Это твои сыновья, сын
мой? ибо они похожи на тебя>>. И он сказал ему,
что они действительно его сыновья, и
<<действительно я видел, что они истинно мои
сыновья>>. И они подошли и обернулись к нему, и
он поцеловал их и обнял их обоих вместе. И дух
пророчества нисшел в уста его. И он взял Левия за
правую руку, и обратился к Левию, и начал прежде
благословлять его, и сказал: <<Да благословит
прежде всего тебя Господь миров~--- тебя и твоих
сыновей во весь век! И да прославит Господь тебя и
твое семя великой честью; и да благоволит Он,
чтобы из всякой плоти ты и твое семя приступали к
Нему для служения Ему в Его святилище; как Ангелы
лица и как святые, так да будет семя твоих сыновей
в честь, и достоинство, и освящение! И да соделает
Он их великими во все века; и владыками, и
князьями, и начальниками да будут они над всем
семенем сыновей Иакова; да изрекают они слово
Господне с искренностью, и Его правду да
исполняют они по всей справедливости, и да
повествуют они о моих путях Иакову и об
откровении Израиля; благословение Господне да
будет вложено в уста их, дабы все семя их
благословляло тебя, возлюбленный! И мать твоя
нарекла тебе имя Левий, и справедливо она назвала
тебя так: ты будешь стоять ближе всех к Господу и
будешь иметь долю у всех детей Иакова; его стол да
будет твоим, и ты и сыновья твои должны питаться
от него; во все роды да изобилует стол твой, и твое
пропитание да не умалится никогда во все века!
Все ненавидящие тебя за что-либо погибнут пред
тобою, и все твои враги да будут истреблены и да
погибнут! Благословляющие тебя да будут
благословлены, и все люди, проклинающие тебя, да
будут прокляты!>>

И Иуде сказал он: <<Да даст тебе Господь силу и
крепость низложить всех, ненавидящих тебя! Будь
господином, ты и один из сыновей твоих, над сынами
Иакова! Твое имя и имя сыновей твоих да пойдет и
распространится по всей земле и по городам! Тогда
устрашатся язычники пред лицем твоим, и все
народы будут поражены, и все люди будут поражены.
Да придет Иакову чрез тебя помощь Его, и да
обретет Израиль чрез тебя избавление! И когда ты
будешь восседать на престоле славы, да
возвеличится справедливость твоя! Мир да будет
всему семени сыновей возлюбленного!
Благословляющий тебя да будет благословлен, и
все ненавидящие тебя, и гнетущие, и проклинающие
тебя да истребятся и погибнут от земли, и да будут
прокляты!>> И он обратился, и поцеловал его
опять, и обнял его, и очень радовался, что увидел
сыновей Иакова, который был его истинным сыном.

И он (Иаков) отошел от его лона, и пал ниц, и
склонился пред ним, и так он благословил их. И он
оставался там у своего отца Исаака в ту ночь, и
они ели и пили, исполненные радости. И он поставил
обоих сыновей Иакова, одного направо, другого
налево от себя, и это было вменено ему в
праведность. И Иаков рассказал своему отцу все в
ту ночь, как являл Господь к нему великую милость
и давал ему на всех его путях счастие и охранял
его от всякого зла. И Исаак благословил Бога отца
своего Авраама, Который Своим милосердием и
справедливостью не отступил от сына раба Своего
Исаака. И утром Иаков открыл отцу своему Исааку
обет, какой он дал Господу, и видение, какое он
видел, и сказал, что он устроил жертвенник и что
все приготовлено к жертве, чтобы совершить ее
пред Господом, как он дал обет, и что он пришел,
чтобы посадить его на осла. И Исаак сказал Иакову:
<<Я не могу идти с тобою, ибо я стар, и не могу
перенести путешествия. Иди, сын мой, с миром; ибо
мне теперь сто шестьдесят пять лет, я не могу
путешествовать. Посади на осла твою мать,
чтобы она шла с тобою. Но я знаю, сын мой, что ты
пришел ради меня; и да будет благословен этот
день, в который ты увидел меня живым и я тебя, сын
мой! Будь счастлив и исполни обет, который ты дал,
и не откладывай своего обета, (ибо это радостный
обет). И теперь поспеши исполнить его, и да примет
его Сотворивший все, Которому ты дал обет!>> И он
сказал Ревекке: <<Иди ты с своим сыном
Иаковом!>> И Ревекка пошла с Иаковом и Девора с
ней; и они пришли в Вефиль.

И Иаков вспомнил о молитве, которою отец
благословил его и двоих его сыновей~--- Левия и
Иуду, и возрадовался, и прославил Бога отцов
своих~--- Авраама и Исаака,~--- и сказал: <<Теперь я
знаю, что у меня есть вечная надежда~--- у меня и
моих сыновей~--- пред Богом всех вещей; и это
определено относительно них обоих, и начертано
это для них в вечное свидетельство па небесных
скрижалях~--- так, как благословил Исаак>>.

\vs Jub 32:1
И он оставался в ту ночь в Вефиле. И Левий видел
во сне, что он и его сыновья поставлены и
определены навек ко священству для Бога
всевышнего. И он пробудился от сна и прославил
Бога. И Иаков собрался рано утром в четырнадцатый
день этого месяца и дал десятину от всего, что
прибыло с ним, от человека до скота, от золота до
сосудов и одежд, и дал десятину от всего. И в те
дни Рахиль была беременною Вениамином, своим
сыном; и Иаков исчислил, начиная с него, своих
сыновей; и жребий Господа пал на Левия. Тогда он
одел его в священнические одежды и наполнил руки
его. И в пятнадцатый день этого месяца он принес
на жертвеннике четырнадцать тельцов из рогатого
скота, и двадцать восемь овнов, и сорок девять
овец, и шестьдесят агнцев, и двадцать девять
молодых козлят, как всесожжение на жертвеннике и
как благоприятный дар в добрую воню пред
Господом. Это была дань его ради обета, который он
дал~--- отделить десятину~--- вместе с плодовыми
жертвами и жертвами возлияния, которые
относились сюда. И когда огонь пожрал их, он
воскурил над ними фимиам на огне; и в жертву
благодарения он принес двух тельцов, и
четырех овнов, и двух годовых ягнят, и десять
телят, и четырех овец и двух молодых козлят. Так
делал он, давая свою дань, в продолжение семи
дней. И он ел там со всеми своими сыновьями и
людьми в радости в течение семи дней, и прославил
и благодарил при сем Господа, Который спас его от
всякого зла и исполнил на нем Свое обетование. И
он отделил десятину от всего чистого скота и
совершил всесожжение. А нечистый скот он отдал
сыну своему Левию, и людей отдал он ему. И Левий
исполнил в Вефиле священнические обязанности
пред своим отцом Иаковом, будучи предпочтен
своим десяти братьям, и был там священником. И
Иаков отдал ему свой обет. И таким образом он
отдал вторую десятину Господу и посвятил ее, и
она стала посвященною ему. И посему определено
это как закон на небе~--- давать вторую десятину,
чтобы есть ее пред Господом в том месте, которое
избрано, чтобы имя Его обитало там, во все годы. И
для сего закона нет конца дней; навечно записано
то постановление, чтобы делать это ежегодно, именно~---
есть вторую десятину пред Господом в том
месте, которое избрано. И ничего не должно быть
оставляемо от нее на следующий год, но в том же
году должно быть съедаемо семя до следующего
года, от дней начатков года, семени, и вина, и
масла, опять до этих же дней. И все, что останется
от нее и сделается устаревшим, должно считать
оскверненным и сожечь огнем, ибо это нечистое. И
так они должны вместе есть ее во святом доме и не
давать ей залеживаться. И все десятины от
рогатого скота и овец суть святы Господу, и Его
священникам должны принадлежать они, чтобы они
ели пред Ним из года в год. Ибо так это определено
и начертано на небесных скрижалях относительно
десятины.

И в следующую ночь, в двадцать второй день этого
месяца, Иаков решил обстроить то место, и обнести
место стенами, и посвятить, и сделать его святым
навечно для себя и своих детей после себя до века.
И Господь явился ему ночью, и благословил его, и
сказал ему: <<Твое имя не должно быть только
Иаков, но должно быть наречено имя тебе
Израиль>>. И Он опять сказал ему: <<Я Господь
Бог твой, сотворивший небо и землю. Я возращу
тебя, и весьма умножу тебя, и цари произойдут от
тебя, и будут они господствовать всюду, где
только ступит нога сынов человеческих. И Я дам
твоему семени всю землю, которая под небом, и они
будут по своей воле господствовать над всеми
народами; и после этого они завладеют всею землею
и наследуют ее навеки>>. И Он окончил Свою
беседу с ним и поднялся от него. И Иаков видел, как
Он вознесся на небо; и он видел ночью в видении, и
вот Ангел сошел с неба с семью скрижалями в своих
руках, и он дал их Иакову, и он читал их и прочитал
все, что было написано на них, что случится с ним и
с его сыновьями во все века. И он показал ему все,
что было написано на скрижалях, и сказал ему:
<<Ты не должен строить на этом месте и делать
святыню навечно, и Он не хочет обитать здесь, ибо
не это Его место. Иди в дом Авраама, отца
твоего, и живи в доме отца твоего Исаака до дня
смерти твоего отца. Ибо в Египте ты умрешь в мире,
и будешь погребен в этой стране с честью в гробах
твоих отцов с Авраамом и Исааком. Не бойся! ибо
как ты видел и прочитал, так все и случится. И
запиши все, как ты видел это и прочитал>>. И
Иаков сказал: <<Как я упомню все так, как видел
это и прочитал?>> И он сказал ему: <<Я опять
приведу тебе все на память>>. И он поднялся от
него.

И он пробудился от сна своего, и вспомнил все,
что читал и видел, и записал всю речь, которую
читал и видел. И он остался там еще на один день и
принес в этот день жертву совершенно так же, как в
прежние дни, и назвал его~--- <<прибавление>>. Ибо
тот день прибавлен. И прежние дни он назвал
праздником. И так ему было открыто, что должно
случиться, и это написано на небесных скрижалях.
И ради того было ему это открыто, чтобы он хранил
его, и прибавлял его таким образом к семи
праздничным дням. И он назван был прибавлением,
как заканчивающий в мире праздничные дни по
числу дней года.

И в ночь на двадцать третий день этого месяца
умерла Девора, нянька Ревекки, и они похоронили
ее внизу города под дубом реки, и он нарек имя той
реке~--- <<река Деворы>> и дубу~--- <<дуб плача
Деворы>>. И Ревекка пошла и возвратилась в дом к
его отцу Исааку. И Иаков послал ему чрез нее
барана, и телят, и овец, чтобы она приготовила его
отцу кушанье, как он любил. И после отправления своей
матери он пошел дальше, пока не пришел в страну
Кебрафан, и жил там. И Рахиль родила ночью сына и
назвала его: <<мой сын болезни>>, ибо она имела
тяжелые роды. А отец его назвал его Вениамином в
десятый день восьмого месяца в первый год шестой
седмины этого юбилея. И Рахиль умерла там и была
погребена в стране Ефрафе, т.е. Вифлееме. И Иаков
устроил на могиле Рахили памятник при дороге, над
ее могилою.

\vs Jub 33:1
И Иаков пошел дальше и жил к северу в Магд-Ладре
Еф(рафа). И он пошел к своему отцу Исааку, он и его
жена Лия, в новолуние десятого месяца, и Робел
увидел Баллу, служанку Рахили, наложницу своего
отца, когда она купалась в воде в уединенном
месте, возымел любовь к ней, и спрятался ночью, и
вошел в жилище Баллы, и нашел ее одну ночью
лежащей на своей постели и спящей в своем жилище.
И он лег к ней на ложе, и открыл покрывало ее; и она
схватила его и вскрикнула. И когда она узнала его,
что это был Робел, застыдилась его, и отняла свою
руку от него, и убежала, и очень скорбела о
случившемся, но не сказала ничего ни одному
человеку. И когда Иаков пришел и отыскивал ее, она
сказала ему: <<Я не чиста для тебя, но обесчещена
для тебя, ибо Робел обесчестил меня, и лег со мною
ночью, когда я спала у себя, и я не узнала его, пока
он не открыл моего покрывала, и он спал со
мною>>. И Иаков сильно разгневался на Робела,
что он спал с Валлою и открыл покров своего
отца. И Иаков не приближался более к ней, так как
Робел обесчестил ее, и пред всеми людьми открыл
покров своего отца. Ибо поступок его был очень
нехорош; это постыдно пред Господом.

Посему написано и определено на небесных
скрижалях, что муж не должен лежать с женою
своего отца и открывать покров своего отца, ибо
это мерзость. Смертию должен умереть преступный
муж, который ляжет с женою своего отца, а также и
жена: ибо они мерзость совершили на земле. И пред
нашим Господом не должно быть ничего мерзкого в
народе, который Он избрал Себе в царское
достояние. И еще написано: да будет проклят, если
кто лежит с женою своего отца, за то, что он
открывает срамоту своего отца. И все святые
Господа пусть скажут: <<Аминь, аминь!>> И ты,
Моисей, скажи сынам Израиля, чтобы они соблюдали
сие слово, ибо за него угрожает наказание
смертию, и это мерзость, и нет за это прощения,
чтобы можно было искупить мужа, который совершит
сие зло, кроме наказания смертию, и умерщвления, и
побиения камнями, и истребления из народа нашего
Бога. Ни одного дня не должен жить на земле муж,
который совершит это во Израиле, ибо это
преступно и мерзко. И не должно говорить, что
Робел остался в живых и получил прощение, хотя он
лежал с наложницей своего отца, в то время как муж
ее, отец его Иаков, еще был жив. Ибо Он тогда же
вполне открыл всем постановление, и правду, и
закон, который существует вовек. Но во все дни
твои он должен иметь силу закона, с его дней и
есть вечный закон для вечных родов. И этот закон
не прекратится, и никакое прощение не будет
уделом такового, кроме того, что оба вместе будут
истреблены из народа; в тот день, когда они
совершили это, должно умертвить их. И ты, Моисей,
напиши это для Израиля, чтобы они соблюдали сие и
поступали по сему слову и не совершали смертного
греха, ибо Господь Бог наш есть судия
нелицеприятный и неподкупный. И скажи им это
постановление, чтобы они слушались, и
оберегались, и внимали сему, и не погибли бы, и не
были бы истреблены на земле. Ибо нечисты, и
мерзки, и преступны, и скверны все они,
совершающие сие на земле пред нашим Господом. И
нет большего греха на земле, как любодеяние,
которым они любодействуют; ибо Израиль есть
народ, святый Господу, и народ наследия для
своего Бога, и народ священства и царства, и
достояние Божие. И никто не должен
существовать, кто является столь нечистым среди
святого народа.

И в третий год этой шестой седмины вышел Иаков и
все его сыновья, и жили в доме Авраама у своего
отца Исаака и своей матери Ревекки. И вот имена
сыновей Иакова: его первенец Робел, Симеон, Левий,
Иуда, Исашар, Завулон~--- сыновья Лии; и сыновья
Рахили: Иосиф и Вениамин; и сыновья Баллы: Дан и
Наффали; и сыновья Залафы: Гад и Асер; и Дина, дочь
Лии, единственная дочь Иакова. И они пошли и
поклонились Исааку и Ревекке. И когда последние
увидели их, благословили Иакова и всех его детей.
И Исаак очень обрадовался, что увидел сыновей
Иакова, своего младшего сына, и благословил их.

\vs Jub 34:1
И в шестой год этой седмины сорок четвертого
юбилея Иаков отослал своих сыновей~--- пасти его
овец~--- и своих рабов с ними на поля Сихемские. И
собрались против них семь царей аморрейских,
чтобы умертвить их, укрывшись под деревьями, и
увести их скот. Но жены их, и Иаков, и Левий, и Иуда,
и Иосиф оставались дома у своего отца Исаака, ибо
дух его был прискорбен, и он не хотел отпустить
их; Вениамин, как юнейший, оставался с своим
отцом. И пришли цари Фафы и Арезы, и Сарагана, и
Село, и Гаиза, и царь Бефорона, и Маанизакира, и
все, живущие на тех горах и обитающие в лесах
страны Ханаанской. И известили Иакова: <<Цари
аморрейские окружили твоих сыновей и похитили их
стада>>. И отправился из своего дома он, и его
три сына, и все рабы его отца, и его рабы, и вышли
против них в числе восьмисот мужей, носивших
мечи; и они поразили их на поле Сихемском, и
преследовали бегущих, и убили Арезу, и Фафу, и
Сарагана, и Аманискино, и Гаганиса. И он собрал
свои стада, и был могущественнее их, и наложил на
них дань, чтобы они давали плоды своей страны. И
они построили Робел и Фамнафарес. И он
возвратился благополучно, и заключил с ним мир, и
они сделались его рабами, пока он не ушел с своими
сыновьями в Египет.

И в седьмой год этой седмины послал он Иосифа,
чтобы он осведомился о состоянии своих братьев,
из своего дома в Сихем. И он нашел их в стране
Дуфаим. И они подстерегали его, и сделали против
него умысел убить его. И когда они изменили свое
намерение, то продали его измаильским
странствующим купцам. И они отвели его в Египет и
продали его Питфаре, евнуху Фараона, главному
повару, жрецу Гелиопольскому. И сыновья Иакова
закололи козленка, и омочили одежду Иосифа в
его крови, и послали ее Иакову в десятый день
седьмого месяца. [...]. И они принесли ее ему, и он
занемог болезнию от печали о его смерти. И он
сказал: <<Дикий зверь пожрал Иосифа>>. И все
его домочадцы были при нем в этот день; и его
сыновья, и его дочери собрались утешать его; но он
оставался безутешным о своем сыне. И в тот день
услышала Балла, что Иосиф потерян, и умерла от
печали по нем, в то время как она была в Караффифе.
И дочь его Дина также умерла, после того как Иосиф
был потерян. Эта тройная скорбь пришла на Израиля
в один месяц. И они похоронили Баллу напротив
могилы Рахили, а также и дочь его Дину похоронили
там. И он скорбел об Иосифе год и не переставал
печалиться; ибо он сказал: <<Я сойду в могилу,
печалясь об Иосифе>>. Ради сего определено
между сынами Израиля, чтобы скорбеть в десятый
день седьмого месяца, в тот день, когда пришло
печальное известие об Иосифе к его отцу Иакову,
чтобы испрашивать в оный день прощение чрез
козла, в десятый день седьмого месяца, один раз в
год, в своих грехах; ибо они превратили любовь
своего отца к его сыну Иосифу в печаль о нем. И
этот день установлен, чтобы они в течение его
скорбели о своих грехах, и о всякой своей вине, и о
своем проступке, дабы очищаться в этот день
однажды в год.

И после того как Иосиф был потерян, сыновья
Иакова взяли себе жен. 1)Имя жены Робела~--- Ада;
2)жены Симеона~--- Адиба, хананеянка;
3)жены Левия~--- Мелха, из дочерей Аррама, из семени сыновей
Фарана; 4)жены Иуды~--- Бефазуел, хананеянка; 5)жены
Исашара~--- Гизека; 6)жены Дана~--- Эгла; 7)жены
Завулона~--- Нииман; 8)жены Наффалима~--- Разуу из
Месопотамии; 9)жены Гада~--- Михи; 10)жены Асера~---
Ийона; 11)жены Иосифа~--- Асанеф, египтянка; 12)жены
Витамина~--- Ийоска. И Симеон изменил намерение, и
взял вторую жену из Месопотамии, как и его братья.

\vs Jub 35:1
И в первый год первой седмины сорок пятого
юбилея призвала Ревекка сына своего Иакова и
дала ему повеление относительно его отца и брата,
чтобы он почитал их во все дни жизни своей. Он
сказал: <<Я буду поступать так, как ты повелела
мне, ибо это будет для меня честью, и
достоинством, и праведностью пред Господом, что я
почитаю их. Ты же знаешь от дня моего рождения до
сего дня каждое мое деяние и всякое мое
помышление, что я всегда благожелаю всем. Как же
мне не исполнять того, что ты заповедала мне,~--- именно
почитать моего отца и брата? Скажи мне, мать
моя, какое зло ты заметила во мне? Я же и далек от
него (от Исава), и между нами существует
доброе согласие>>. И она сказала ему: <<Сын мой,
в продолжение всей своей жизни я не видела в тебе
ничего предосудительного, а только
справедливое. Я говорю тебе, сын мой: в этом году я
кончу свою жизнь. Ибо я видела во сне день
моей смерти, что я не проживу более ста
пятидесяти лет. И вот я кончила все дни своей
жизни, которые надлежало мне прожить>>. И Иаков
усмехнулся над словами своей матери, что мать
сказала ему, будто она умрет, между тем как она
сидела против него в полной силе, не ослабевшая;
ибо она входила и выходила, и видела, и зубы ее
были здоровы, и никакая болезнь не коснулась ее в
течение всей ее жизни. И Иаков сказал ей: <<Я
буду счастлив, мать моя, если моя жизнь
сравняется по продолжительности с твоей жизнью и
если я так же сохранюсь в полной своей силе, как
ты. Ты не умрешь, и напрасно говоришь со мною о
своей смерти>>.

И она вошла к Исааку и сказала ему: <<Я имею к
тебе просьбу: заставь поклясться Исава, что он не
причинит обиды Иакову и никогда не будет
преследовать его. Ибо ты знаешь нрав Исава, что он
груб от юности, и нет в нем добродушия; ибо он
замышляет после твоей смерти убить его. И ты
знаешь все, что совершил он во все дни от того дня,
когда брат его Иаков пошел в Харран, до сего дня;
как он оставил нас всем своим сердцем и сделал
нам злое; как он присвоил себе твои стада и все
достояние твое похитил пред лицем твоим. И когда
мы умоляли и просили о нашем достоянии, он
действовал подобно человеку, как бы оказывающему
нам свою милость. И он гневается на тебя, ибо ты
благословил своего благочестивого и праведного
сына Иакова; ибо в нем нет ничего злого, но одно
только доброе. И с того времени, как он
возвратился из Харрана, до сего дня он не обидел
нас ни в малейшем; но мы все получаем от него вовремя
и всегда; и он радуется от всего сердца, если мы
что-нибудь принимаем от него, и благословляет
нас; и он не отделился от нас с того времени, как
пришел из Харрана, до сего дня. И он живет всегда с
нами в доме, почитая нас>>. И Исаак сказал ей:
<<Знаю и я, и вижу отношение Иакова к нам, что он
нас почитает во всем. Я раньше любил более Исава,
чем Иакова, ибо он родился прежде; но теперь я
люблю Иакова больше, чем Исава, так как он
оказался в своих делах весьма дурным и в нем нет
никакой справедливости. Ибо все пути его~---
несправедливость и насилие, и нет в нем
справедливости. Мое сердце также потрясено
теперь из-за всех его дел, и ему и семени его не
будет счастия, но они погибнут на земле и будут
истреблены под небом. Ибо он оставил Бога Авраама
и последовал за своими женами, за мерзостию и
соблазном их~--- он и его сыновья. И ты говоришь мне,
чтобы я заставил его поклясться, что он не убьет
Иакова; но если он и поклялся бы, то это будет
бесполезно, и он будет совершать не добро, а
только зло. И если он захочет умертвить своего
брата Иакова, то будет предан в руки Иакова, и не
избегнет рук его, но впадет в руки его. И ты не
бойся за Иакова: ибо хранитель Иакова~---
могущественный, и досточтимый, и
достопоклоняемый всеми>>. [...]

И Ревекка послала и призвала Исава; и он пришел
к ней. И она сказала ему: <<У меня есть просьба к
тебе, сын мой, и ты обещай, что исполнишь то, что я
скажу тебе, сын мой!>> И он сказал: <<Я сделаю
все, что ты скажешь мне, и не откажу в твоей
просьбе>>. И она сказала ему: <<Я прошу тебя,
чтобы ты, когда я умру, перенес меня и похоронил с
Сарой, матерью отца твоего, и чтобы вы любили друг
друга, ты и брат твой Иаков, и никто не
предпринимал бы никакого зла против своего
брата, а оказывал бы только взаимную любовь, дабы
вы были счастливы, сыновья мои, и были почитаемы
на земле, и никакой враг не восторжествовал бы
над вами, и вы были бы достойными милосердия пред
очами тех, которые любят вас>>. И он сказал: <<Я
все исполню, что ты сказала мне, и похороню тебя,
когда ты умрешь, вместе с Сарой, матерью отца
моего, так как ты любишь кости ее, чтобы они были с
твоими костями. И также брата моего Иакова я буду
любить больше, чем всякую плоть; у меня на всей
земле нет брата, кроме его одного; и нет ничего
великого (трудного) для меня в том, чтобы любить
его, ибо он брат мой, и мы вместе были посеяны в
твоем чреве и вместе вышли из твоих недр. И если
не любить мне своего брата, то кого же мне любить?
И я также прошу тебя, чтобы ты сделала увещание
Иакову относительно меня и моих детей, так как я
знаю, что он как царь будет господствовать надо
мною и над моими сыновьями. Ибо в тот день, когда
мой отец благословил его, он сделал его высшим, а
меня подчиненным. И я клянусь тебе, что я буду
любить его и ничего злого не замыслю против него
в продолжение всей моей жизни, а только
доброе>>. И он подтвердил клятвою все эти слова.
И она призвала Иакова пред очи Исава и дала Иакову
повеление согласно беседе, какую она вела с
Исавом, и он сказал: <<Я исполню твою волю,
ручаясь за то, что от меня и моих сыновей не
выйдет ничего злого против моего брата Исава, и
только лишь одну любовь встретит он>>. И они ели
и пили, она и ее сыновья, в эту ночь. И она умерла,
трех юбилеев одной седмины и одного года, в эту
ночь. И оба ее сына Исав и Иаков похоронили ее в
пещере около Сары, матери их отца.

\vs Jub 36:1
И в шестой год этой седмины призвал Исаак обоих
своих сыновей~--- Исава и Иакова, и они пришли к
нему, и он сказал им: <<Сыны мои, я иду по пути
моих отцов в вечное жилище, где отцы мои.
Похороните меня с моим отцом Авраамом в двойной
пещере на полях Эфрона Хеттеянина, которые
Авраам купил для могилы; там похороните меня! И я
заповедую вам, сыны мои, совершать на земле
справедливость и правду, чтобы Господь послал
вам все, что обещал сделать Аврааму и семени его.
И любите друг друга, как братья, сыны мои, так, как
каждый любит самого себя, и стараясь сделать
лучшее для другого, действуя единодушно на земле
и каждый любя другого, как самого себя. И
относительно идолов я заповедую вам, чтобы вы
отвергали их, и ненавидели, и не любили их; ибо они
исполнены соблазна для тех, которые почитают их,
и для тех, которые поклоняются им. Памятуйте, сыны
мои, о Господе, Боге Авраама, отца вашего, как и я
после него почитал Его и служил Ему воистину,
дабы Он умножил вас в радости и возрастил семя
ваше~--- умножил вас в числе, как звезды небесные, и
насадал бы на земле вас и всякое растение правды,
которое не будет истреблено во все роды века. И
ныне я заклинаю вас великою клятвою~--- ибо нет
большей клятвы, как клятва славным, и честнейшим,
и великим именем Того, Кто сотворил небо и землю и
все в совокупности,~--- чтобы вы страшились Его и
почитали и чтобы каждый любил своего брата нежно
и искренно, и не желал бы своему брату зла отныне
до века, во все дни вашей жизни, дабы вы были
счастливы во всех своих делах и не погибли. И если
кто из вас предпримет что-либо злое против своего
брата, то знайте отныне, что всякий, замышляющий
что-либо злое против своего брата, падет от его
руки и будет истреблен из страны живых, и семя его
также погибнет под небом. И в день проклятия и
власти Он сожжет пылающим и поедающим огнем и
его страну, и город, и все принадлежащее ему,
подобно тому как Он сожег Содом; и он будет
изглажен из книги наставления сынов
человеческих и не будет записан в книге жизни. Но
он погибнет и подпадет вечному осуждению, чтобы
их наказание беспрерывно возобновлялось чрез
ненависть, и проклятие, и гнев, и мучение, и злобу,
и муки, и болезнь, вовек. Я говорю и возвещаю вам,
сыны мои, суд, как он придет на мужа, который
захочет сделать что-нибудь дурное против своего
брата>>.

И он разделил все свое имущество между ними
обоими в тот день. И он дал преимущество тому, кто
был рожден прежде, и отдал ему башню, и все
кругом ее, и все, что Авраам приобрел у
клятвенного колодезя. И он сказал:
<<Преимущество это должен иметь тот, кто рожден
прежде>>. И Исав сказал: <<Я продал и передал
свое старшинство Иакову; пусть будет отдано это
Иакову! И я не буду более говорить ему об этом, ибо
так случилось это>>. И Исаак сказал: <<Да
покоится, сыны мои, благословение на вас и на
вашем семени в этот день, что вы остались
спокойными и не огорчили меня из-за старшинства,
что вы не допускаете ничего постыдного из-за
него! Господь, Всевышний, да благословит того
мужа, который совершает справедливость, его и
семя его вовек!>> И он перестал давать заповеди
и благословлять их. И они ели и пили вместе пред
ним, и он радовался, что между ними совершилось
примирение. И они вышли от него, и отдыхали в тот
день, и спали.

И Исаак почил в тот день на своем ложе, полный
радости, и почил вечным сном, и умер ста
восьмидесяти лет, окончив двадцать пять седмин и
пять лет. И оба сына его, Исав и Иаков, похоронили
его. И Исав пошел в страну Едом, на горе Сеир, и
оставался там. И Иаков жил на горе Хеврон в башне
страны странствования отца своего Авраама; и он
почитал Господа от всего сердца и по Его заповеди
[...].

И жена его Лия умерла в четвертый год второй
седмины сорок пятого юбилея; и он похоронил ее в
двойной пещере возле своей матери Ревекки,
налево от могилы Сары, матери отца его. И все ее и
его сыновья пришли оплакивать вместе с ним Лию,
жену его, и утешать его в скорби по ней. Ибо он
скорбел об ней, так как любил ее еще более после
того, как умерла сестра ее Рахиль. Ибо она была
благочестива и праведна во всех путях своих и
почитала Иакова. И в течение всего времени, как
она жила с ним, он не слышал из уст ее никакого
грубого слова; ибо она была кротка, и миролюбива,
и праведна, и досточтима. И он вспомнил ее дела,
какие она делала во время своей жизни, и очень
оплакивал ее, ибо он чрезвычайно любил ее от
всего сердца и от всей души.

\vs Jub 37:1
И когда Исаак, отец Иакова и Исава, умер,
услышали сыновья Исава, что Исаак отдал
первенство своему младшему сыну Иакову, и
разгневались чрезмерно, и препирались с своим
отцом, говоря: <<Почему, когда ты старший, а
Иаков~--- младший, твой отец отдал первенство
Иакову, и тебя поставил ниже?>> И он сказал им:
<<Потому что я свое первородство продал за
немногое~--- за чечевичное кушанье. И в тот день,
когда мой отец послал меня на охоту~--- наловить
чего-нибудь и принести к нему, чтобы он ел и
благословил меня, пришел он (Иаков) хитростью и
принес моему отцу есть и пить, и мой отец
благословил его, а меня отдал в его руки. И вот
отец наш заставил нас поклясться, меня и его, что
мы ничего злого не замыслим друг против друга, и
будем жить друг с другом в любви и мире, и не
извратим наших путей>>. И они сказали ему: <<Мы
не послушаемся тебя в том, чтобы поддерживать с
ним мир, ибо мы сильнее, нежели он, и мы преодолеем
его. Мы выйдем против него, и умертвим его, и
истребим его сыновей. И если ты не пойдешь с нами,
мы причиним зло и тебе. Послушай же нас теперь: в
Араме, и Филистее, и Моаве, и Аммоне мы наберем
себе отборных людей, которые способны к войне, и
пойдем против него, и сразимся с ним, и истребим
его в стране, прежде чем он приобретет силу>>. И
отец их сказал им: <<Не ходите, и не начинайте с
ним войны, дабы вам не пасть от него>>. И они
сказали ему: <<Неужели тебе от юности и до сего
дня только и делать, чтобы склонять свою выю под
его ярмо? Мы не послушаемся сих слов>>. И они
послали в Арам и к Адураму, другу своего отца, и
наняли себе у них тысячу способных к войне мужей
и отборных воинов. И пришли к ним от Моава и от
сынов Аммона нанятых тысяча отборных воинов, и от
филистимлян тысяча отборных воинов, и от Эдома и
хореев тысяча отборных ратников, и от хетитов
сильные, способные к войне мужи. И они сказали
своему отцу: <<Выходи, веди нас! а иначе мы убьем
тебя>>. И он разгневался и пришел в ярость, когда
увидел, как сыновья употребляли в отношении к
нему насилие, чтобы он был предводителем их и вел
их против своего брата Иакова.

После сего ему вспомнилось все то зло, которое
лежало сокрытым внутри его против его брата
Иакова, и он не вспомнил о клятве, которую он дал
своему отцу и своей матери, что он не предпримет
ничего злого против своего брата Иакова во всю
свою жизнь.

И в продолжение всего этого времени Иаков
ничего не знал о том, что они выступают против
него войною,~--- он сильно скорбел о своей жене Лии,~---
пока они не подошли к башне против него~--- четыре
тысячи способных к войне, сильных, воинственных,
отборных мужей. И жители Хеврона послали к нему
сказать: <<Вот брат твой пришел на тебя, чтобы
победить тебя, с четырьмя тысячами мужей,
препоясанных мечами и носящих щит и оружие>>.
Они любили Иакова более, чем Исава, поэтому и
сказали ему это; ибо Иаков был муж милостивый и
более любвеобильный, чем Исав. И Иаков не поверил
этому, пока они не приблизились к самой башне. И
он взошел на башню, и говорил с своим братом
Исавом, и сказал: <<Приносишь ли ты мне доброе
утешение? Пришел ли ты ко мне ради моей умершей
жены? Это ли клятва, которою ты дважды поклялся
твоим родителям пред их смертию? Ты нарушил
клятву, и тем, чем ты поклялся своему отцу, ты
осужден>>. Тогда Исав отвечал и сказал ему:
<<Никогда не клянутся между сынами
человеческими и между зверями земли истинною
клятвою до века; но в тот самый день они уже
замышляют злое друг против друга, и враг ищет
убить своего врага. И ты также ненавидишь меня и
моих сыновей до века, и с тобою нельзя сохранять братской
любви. Слушай эти слова мои, которые я скажу
тебе. Если бы я мог изменить кожу и щетину свиньи,
чтобы она (щетина) стала шерстью, и если бы на ее
голове выросли рога, подобно рогам овец, тогда я
поддерживал бы с тобою братскую любовь. И если
грудь у матери отделится~--- ибо ты отселе мне не
брат,~--- и если волки заключат мир с ягнятами, что
они не будут пожирать и похищать их, и если сердце
их склонится к тому, чтобы делать друг другу
добро, тогда я буду иметь в своем сердце мир с
тобою. И если лев сделается другом вола, и будет
запрягаться с ним в одно ярмо, и будет пахать с
ним, тогда я заключу мир с тобою. И если вороны
сделаются белыми, как рис, тогда я узнаю, что я
люблю тебя и храню мир с тобою. Ты должен быть
истреблен, и сыновья твои должны быть истреблены,
и да не будет с тобою мира!>> И Иаков увидел в тот
час, что он замыслил против него злое [...], чтобы
убить его, и что он пришел, стремясь как дикий
зверь, бросающийся на копье, которое пронзает и
убивает его самого, и он не отступает от него.
Тогда он сказал домочадцам и своим рабам, чтобы
они напали на него~--- на него и на всех его
соучастников.

\vs Jub 38:1
И после сего Иуда говорил со своим отцом
Иаковом и сказал ему: <<Отец! Натяни лук свой, и
пусти стрелу свою, и порази злодея, и убей врага.
Да будет у тебя сила на это, ибо мы не хотим
убивать твоего брата!>> [...]. И Иаков тотчас
натянул лук свой, и пустил он стрелу свою, и
поразил брата своего Исава, и убил его. И еще
пустил он стрелу свою и попал в арамеянина Адрона
в его левый грудной сосок, и обратил его в
бегство, и убил его. После сего сыновья Иакова
выступили со своими рабами и распределились на
четырех сторонах башни. Вперед вышел Иуда с
Наффали, и Гадом, и пятьюдесятью рабами на
северной стороне башни, и они умертвили все, что
было пред ними, и никто не спасся от них, даже ни
один. И Левий, и Дан, и Асер выступили на восточной
башне с пятьюдесятью мужами и убили ратников
моавитян и аммонитян. И Робел с Исашаром и
Завулоном выступили на южной стороне башни с
пятьюдесятью мужами и убили воинов филистимлян.
И Симеон, и Вениамин, и Енох, сын Робела, выступили
на западной стороне башни с пятьюдесятью мужами
и перебили из едомитян и хореев (триста) сильных
воинственных мужей; и семьсот убежали. И четыре
сына Исава бежали с ними, и оставили отца своего
убитого, как он пал на холме, который находится в
Адураме. И сыновья Иакова преследовали их до горы
Сеир; а Иаков похоронил своего брата на холме,
который находится в Адураме, и возвратился в свой
дом. И сыновья Иакова стеснили сыновей Исава на
горе Сеир, и согнули их выю, так что они стали
рабами сыновей Иакова. И они послали к своему
отцу спросить, заключить ли мир с ними или
умертвить их. И Иаков велел сказать своим
сыновьям, чтобы они заключили мир. И они
заключили мир с ними и наложили на них ярмо
рабства, чтобы они платили Иакову и его сыновьям
дань всякий год. И они, не переставая, платили
Иакову дань до того дня, когда он ушел в Египет
[...].

И вот цари, которые владычествовали над Едомом,
- прежде чем стал владычествовать царь над сынами
Израиля,~--- до сего дня в стране Едом. И был царем в
Едоме Балак, сын Беора, и имя его города было
Динаба. И Балак умер, и вместо него стал царем
Иобаб, сын Зары из Базуры. И вместо него стал
царем Адафа, сын Барада, который поразил
Мидианитян на поле Моав; и имя его города Авуф. И
Адафа умер, и вместо него стал царем Салман из
Амелека. И Салман умер, и вместо него стал царем
Суал из Робаофа при реке. И Суал умер, и вместо
него стал царем Беулуман, сын Акбура. И Беулуман,
сын Акбура, умер, и вместо него стал царем Адафа, и
имя жены его было Майя-Тобиф, дочь Матрифы, дочери
Мимифбид-Цаобы. Вот цари, которые управляли в
стране Едом.

\vs Jub 39:1
И Иаков жил в земле странствования отца своего,
в стране Ханаанской. Вот роды Иакова. Иосиф был
семнадцати лет, когда они отвели его в Египет, и
Питфаран, евнух Фараона, главный повар, купил его.
И он поставил Иосифа над всем своим домом. И
благословение Господа было над домом египтянина
ради Иосифа, и во всем, что он делал, Господь давал
ему успех. И египтянин предоставил Иосифу все,
что было у него, ибо видел, что Господь был с
ним, и во всем, что он делал, давал ему успех. Иосиф
же был красив и весьма миловиден лицом. И жена
господина его обратила на него свои взоры, и
увидела Иосифа, и почувствовала любовь к нему, и
просила его, чтобы он лег с нею. Но он не предал ей
свою душу, и вспомнил о Господе и о словах,
которые отец его Иаков читал в словах Авраама,
что никто не должен прелюбодействовать с женою,
имеющей мужа, и что для такового определено
наказание смертию на небесах пред Господом
всевышним, и что грех будет записан за ним в
книгах, которые до века всегда существуют пред
Господом. И Иосиф вспомнил эти слова, и не хотел
лечь с нею. И она просила его в течение года, но он
отказывал ей, и не хотел слушаться ее. Но она
обняла его и схватила его в доме, чтобы принудить
его лечь с нею, и заперла двери дома. Но он
вырвался из рук ее, и оставил в руках ее свою
одежду, и разломал запор, и выбежал от нее. И когда
та жена увидела, что он не хочет лечь с нею,
очернила его пред своим господином, говоря:
<<Твой еврейский раб, которого ты любишь, хотел
причинить мне насилие, чтобы лечь со мною; но
когда я возвысила голос свой, он убежал, и оставил
свою одежду в моих руках, как только я схватила
его, и разломал запор>>. И египтянин увидел
одежду Иосифа и также запор, который был
разломан; и послушался слов жены своей, и посадил
Иосифа в темницу в одно место, где сидели
заключенные, которых царь велел заключить в
темницу. И он оставался там в темнице. И Господь
дал Иосифу милость в глазах главного темничного
стража и милосердие в глазах его. Ибо он видел,
что Господь был с ним и во всем, что он делал,
давал ему успех. И он передал ему все, и главный
темничный страж не смотрел ни за чем; ибо все, что
делал Иосиф, совершал Господь. И он оставался там
два года.

И в те дни Фараон, царь египетский, разгневался
на двух своих евнухов, на главного кравчего и на
главного хлебника, и посадил их в темницу в доме
главного повара~--- в темницу, где был заключен
Иосиф. И главный темничный страж приказал Иосифу,
чтобы он служил им; и он служил им. И они оба
видели сон~--- кравчий и хлебник, и рассказали его
Иосифу. И как он истолковал его им, так с ними и
случилось. Главного кравчего Фараон опять
приставил к его должности, а главного хлебника
предал смерти~--- как он истолковал им. И главный
кравчий забыл Иосифа в темнице, хотя он возвестил
ему, что с ним случится; и он не думал о том, чтобы
объявить Фараону, как Иосиф сказал ему; но он
забыл о нем.

\vs Jub 40:1
И в те дни Фараон видел два сна в одну ночь о
голоде, который придет на всю землю. И он
пробудился от сна своего, и призвал всех
снотолкователей, которые были в Египте, и
волхвов, и рассказал им оба свои сна; но они не
могли ничего узнать. После этого главный кравчий
вспомнил об Иосифе и сказал о нем царю. И он велел
привести его из темницы и рассказал ему оба свои
сна. И он сказал Фараону: <<Два сна означают одно
и то же>>. И он сказал ему: <<В продолжение семи
лет будет изобилие во всем Египте, и после этого в
продолжение семи лет голод, подобного которому
не было на всей земле. Теперь назначь, Фараон, во
всей земле Египетской житницы, чтобы в них
собирали пищу в каждом городе в продолжение лет
изобилия, чтобы иметь пищу на семь лет голода, ибо
он будет весьма велик>>. И Господь дал Иосифу
милость и милосердие в очах Фараона. И Фараон
сказал своим слугам: <<Мы не найдем столь мудрого
и разумного мужа, как он, ибо дух Господа с ним>>.
И он поставил его вторым над всем своим царством,
и сделал его господином над всем Египтом, и велел
везти на своей второй колеснице, и одел его в
виссонную одежду, и возложил ему золотую цепь на
шею, и велел возвещать впереди него: <<Ел Ел
Ваабрир>>. И он надел (кольцо) на руку его, и
сделал его господином над всем своим домом, и
возвеличил его, и сказал ему: <<Только престолом
одним я буду больше тебя>>. И Иосиф был
господином над всею Египетскою страною. И любили
его все князья Фараона, и все слуги его, и все
исполнявшие царские дела, ибо он ходил в
праведности и без гордости и надменности и был
нелицеприятным и неподкупным, но по
справедливости судил все народы страны. И страна
была хорошо управляема Фараоном благодаря
Иосифу, ибо Господь был с ним и дал ему милость и
благоволение на весь его род в глазах всех,
которые его знали и о нем слышали. И царство
Фараона было благоустроено: ни злоумышленника,
ни злодея не было там. И царь нарек имя Иосифу
Сафанфи-фанс и дал Иосифу в жены дочь Патифарана,
дочь жреца Гелиопольского, главного повара. И в
тот день, когда Иосиф стоял пред Фараоном, ему
было (тридцать) лет, когда он стоял пред
Фараоном. И в тот самый год умер Исаак. И сбылось
так, как Иосиф сказал в толковании его сна: и
пришли семь лет изобилия на всю Египетскую
страну~--- на одну меру тысяча восемьсот мер. И
Иосиф собирал пищу в каждом городе, пока они
не наполнились хлебом, так что нельзя было уже
считать его и мерить по причине великого
изобилия.

\vs Jub 41:1
И в сорок пятый юбилей во вторую седмину во
второй год взял Иуда своему первенцу Еру жену из
дочерей Арама, по имени Фамарь. Но он ненавидел
ее, и не спал с нею, так как мать его была из
дочерей Ханаанских, и он хотел взять себе жену из
родства своей матери, но отец его Иуда не
позволил ему этого. И этот первенец его был
дурной, и Господь лишил его жизни. И Иуда сказал
сыну своему Онану: <<Войди к жене брата твоего, и
соверши с нею брак ужичества, и восстанови свое
семя брату твоему!>> И Онан знал, что это было бы
семя не его, а его брата, и вошел к жене своего
брата, и излил свое семя на землю. И это было злом
пред очами Господа, и Он лишил его жизни. И Иуда
сказал своей невестке Фамари: <<Оставайся в
доме отца твоего вдовою, пока сын мой Шела не
подрастет; тогда я отдам тебя ему в жены>>. И он
подрос. Но Бефзуел, жена Иуды, не допускала, чтобы
сын ее Шела женился на ней. И Бефзуел, жена Иуды,
умерла в пятый год этой седмины.

И в шестой год отправился Иуда стричь своих
овец в Фимнафу. И она сняла вдовьи одежды, и
надела покрывало, и нарядилась, и села при
воротах на дороге в Фимнафу. И когда Иуда вошел,
он встретил ее, и принял ее за блудницу, и сказал
ей: <<Я войду к тебе>>. И она сказала:
<<Войди!>> И он вошел. И она сказала: <<Дай мне
плату блудницы>>. И он сказал: <<Я ничего не
имею при себе, кроме кольца на пальце, и серег, и
трости, которая у меня в руке>>. И она сказала
ему: <<Дай их мне, пока ты не пришлешь мне плату
блудницы>>. И он сказал ей: <<Я пришлю тебе
козленка>>, и отдал их ей. И она зачала от него; и
Иуда пошел к своим овцам, а она в дом отца своего.
И Иуда послал чрез пастуха едолламского
козленка, но он не нашел ее. И он спрашивал людей той
местности, и сказал им: <<Где блудница,
которая была там?>> И они сказали: <<У нас
нет блудницы>>. И он возвратился и известил его,
что он не встретил ее, и сказал ему: <<Я
спрашивал людей того места, и они сказали мне:
<<Нет там блудницы>>>>. И он сказал:
<<Пойдемте, чтобы не быть осмеянными>>. И когда
прошло три месяца, она узнала, что зачала; и они
известили об этом Иуду, говоря: <<Вот твоя
невестка Фамарь сделалась беременной от
блуда>>. И Иуда пошел в дом отца ее, и сказал ее
родителям и братьям: <<Выведите ее, чтобы она
была сожжена, ибо она совершила нечто нечистое в
Израиле>>. И вот, когда они вывели ее, чтобы
сжечь, она послала своему свекру кольцо, и серьгу,
и трость, говоря: <<Узнай, кому принадлежит
это: ибо от того я зачала>>. И Иуда узнал, и
сказал: <<Фамарь правее меня>>. И они не сожгли
ее. И посему она не была отдана Шеле. И он уже не
приближался больше к ней. И после сего она родила
двоих сыновей~--- Фареса и Зару, в седьмой год этой
второй седмины. И тогда окончились семь лет
плодородия, о которых Иосиф сказал Фараону.

И Иуда сознал, что это было дурное дело, которое
он совершил, так как он преспал с своею невесткою,
и нашел это неправым пред своими очами, и сознал,
что он совершил вину и согрешил, так как открыл
покров своего сына. И он стал скорбеть и умолять
Господа о милосердии к своей вине. И мы сказали
ему в сновидении, что она прощена ему, ибо он
неотступно просил о милости, и скорбел, и вновь не
совершил сего. И он получил прощение, ибо он
обратился от своего греха и неразумия. Ибо велика
эта вина пред нашим Господом; всякого, кто делает
так, и всякого, кто преспит с своею тещею, должно
сожечь огнем, чтобы он сгорел в нем. Ибо мерзость
и осквернение лежит на них; огнем должно сожечь
их. И ты также скажи сынам Израиля, чтобы не было
между ними мерзости; огнем должно сожечь мужа,
который преспит с нею, и также жену, дабы Он
отвратил Свой гнев и Свое наказание от Израиля. И
Иуде мы сказали, что так как два его сына не
сочетались браком, то семя его восстановлено для
другого рода, и оно не будет истреблено; ибо он
пришел по своему неведению и желал наказания;
именно по закону Авраама, который он заповедал
своим детям, Иуда хотел сожечь ее огнем.

\vs Jub 42:1
И в первый год третьей седмины сорок пятого юбилея
настал в стране голод; и на земле не было дождя,
так что совсем ничего не падало, и земля
сделалась бесплодною. И только в стране
Египетской была пища, так как Иосиф собрал, чтобы
можно было давать им пищу. И Иосиф собирал в
течение семи лет плодородия семя в стране и
сберегал его. И египтяне пришли к Иосифу, чтобы он
дал им пищи; и он открыл житницы, где был хлеб от
первого года, и продавал его жителям страны за
золото.

И Иаков услышал, что в Египте была пища; (тогда
он послал своих сыновей в Египет приобрести
хлеба), но Вениамина не послал. И они пришли
вместе с сопровождавшими их; и Иосиф узнал их,
но они его не узнали. И он беседовал с ними, и
спрашивал их, и говорил им: <<Не соглядатаи ли
вы, и не пришли ли, чтобы разузнать след
страны?>> И он заключил их; потом он освободил
их, и оставил одного только Симеона, и его девять
братьев отпустил, и наполнил мешки их хлебом; а их
золото он положил им в их мешки, но так, что они не
знали. И он повелел им привести своего младшего
брата, ибо они сказали ему, что их отец и младший
брат живы. И они вышли из страны Египетской, и
пришли в землю Ханаанскую, и рассказали своему
отцу все, что с ними случилось, и как правитель
страны говорил с ними, и как он посадил Симеона в
заключение, пока они не привезут к нему
Вениамина. И Иаков сказал: <<Вы похитили у меня
моих детей; Иосифа нет более, и Симеона нет, и
Вениамина еще хотите взять? Ваши дурные действия
ложатся тяготою на мне>>. И он сказал: <<Сын мой
не пойдет с вами; он может заболеть во время
пути. Ибо мать их родила только двоих; один
из них потерян, и еще этого хотите у меня взять? С
ним может случиться в путешествии болезнь, и вы
доведете до смерти мою седую старость от горя>>.
Ибо он видел, что золото их принесено назад в их
мешках, и посему он боялся послать его с ними.

И усилился голод, и сделался великим в стране
Ханаанской и во всех странах, кроме только земли
Египетской. Ибо многие из египтян собирали себе
семена в пищу, после того как увидели, что Иосиф
собирает семена, и кладет их в житницы, и
сберегает на голодные годы. И жители Египта
прокармливались этим в первый год голода. И когда
Израиль увидел, что голод в стране очень
усилился, и не было более спасения, он сказал
своим сыновьям: <<Идите опять, и приобретите
себе пищи, чтобы нам не умереть>>. И они сказали:
<<Мы не пойдем; если наш младший брат не пойдет с
нами, то мы не пойдем>>. И (Иаков) увидел, что
если он не пошлет его с ними, то все они погибнут
от голода. И Робел сказал: <<Передай мне его в
мои руки, и если я его не приведу к тебе назад, то
умертви двух моих сыновей за его душу>>. Но он
сказал: <<Он не пойдет с тобою>>. И Иуда подошел
и сказал: <<Отпусти его со мною, и если я его не
приведу к тебе назад, то буду пред тобою
преступником во все дни моей жизни>>. И он
отпустил его с ними во второй год седмины в
новолуние, и они пришли в Египетскую страну
вместе со всеми другими, шедшими туда, с дарами в
своих руках, с стираксой (стакти), и миндальными
орехами, и фисташками, и чистым медом. И они
пришли и предстали пред Иосифа, и он увидел
Вениамина, своего брата, и узнал их, и сказал им:
<<Это ваш младший брат?>> И они сказали ему:
<<Это он>>. И он сказал: <<Да будет милость
Господня с тобою, сын мой!>> И он послал их в свой
дом, и выдал им также Симеона, и приготовил им
обед. И они передали ему дар, который они привезли
для него. И они ели пред ним, и он дал каждому из
них по части, но часть Вениамина была в семь раз
больше, чем часть остальных. И они ели, и пили, и
встали, и оставались у своих ослов. И Иосиф
придумал способ, посредством которого он мог бы
узнать их помышления, господствуют ли в них
человеческие помышления. И он сказал мужу,
который управлял его домом: <<Наполни им все их
мешки хлебом, положи им также назад их золото в их
хранилища, и мою серебряную чашу, из которой я
пью, положи в мешок младшего, и отпусти их>>.

\vs Jub 43:1
И он сделал, как сказал ему Иосиф, и наполнил
мешки их пищею, и золото их положил также в их
мешки, и чашу в мешок Вениамина. И рано утром они
отправились. И когда они выехали оттуда, Иосиф
сказал мужу: <<Гонись за ними, беги и обличи их,
говоря: <<Вы отплатили злом за добро, и похитили
серебряную чашу, из которой пьет господин мой>>.
И приведи назад ко мне их младшего брата, и
приведи его немедленно, прежде чем я займусь
своими делами>>. И он побежал за ними и сказал им
по его словам. И они сказали ему: <<Да будет это
далеко от рабов твоих; они не сделают ничего
подобного, и не украдут никакого имущества из
дома твоего господина. И даже золото, которое мы в
первый раз нашли в наших мешках, мы, рабы твои,
принесли назад из земли Ханаанской. Украдем ли мы
какое-нибудь имущество? Вот мы здесь и мешки наши:
ищи, и тот из нас, в мешке которого ты найдешь
чашу, пусть будет наказан смертию, и мы с своими
ослами будем в подчинении у твоего господина>>.
И он сказал им: <<Нет; мужа, у которого я найду,
его одного только возьму я в рабы; а вы идите с
миром>>. И когда он искал в их сосудах, он начал
со старшего и кончил младшим, и она была найдена в
мешке Вениамина, младшего. И они пришли в ужас, и
разорвали свои одежды, и навьючили своих ослов, и
возвратились назад в город. И они пришли в дом
Иосифа, и пали все пред ним на свое лице на землю.
И Иосиф сказал им: <<Вы сделали это>>. И они
сказали: <<Что нам сказать и как оправдаться,
когда наш господин нашел вину за своими рабами?
Вот мы рабы господина нашего вместе с нашими
ослами>>. И Иосиф сказал им: <<Я страшусь
Господа, и вы пойдете домой; но ваш брат будет
принадлежать мне, ибо вы сделали злое. Вы не
знаете, что муж, как я, пьющий из этой чаши,
дорожит своею чашею? И вы похитили ее у меня>>. И
Иуда сказал: <<Да будет позволено мне, господин
мой, сказать слово в уши господина моего. Двоих
братьев мать моя родила нашему отцу, рабу твоему:
один ушел и погиб, так что его уже не нашли; и
только тот один остался от своей матери, и раб
твой, отец наш, любит его, и душа его привязалась к
этой душе. И будет, что если мы возвратимся к рабу
твоему, отцу нашему, и младшего не будет с нами, то
он умрет, и мы погубим нашего отца, и он умрет от
печали. Лучше я буду рабом твоим вместо дитяти,
рабом моего господина; но юноше позволь идти с
его братьями, ибо я поручился за него пред рабом
твоим, отцом нашим; и если ты не отдашь его, то раб
твой будет всегда виновным пред нашим отцом>>.

И Иосиф увидел, что все они были единодушными и
благожелательными друг к другу; и он не мог более
удерживаться, и сказал им, что он~--- Иосиф, и
разговаривал с ними по-еврейски, и пал им на шею, и
плакал; и они не узнали его. Теперь и они начали
плакать. И он сказал им: <<Не плачьте из-за меня.
Поспешите и приведите ко мне отца моего, чтобы я
увидел моего отца, прежде чем умру [...]. Ибо вот это
второй год голода, и еще предстоят пять лет, когда
не будет жатвы, и плода с деревьев, и никаких
растений. Поспешите с вашими домочадцами, чтобы
вам не погибнуть от голода и не быть в
беспокойстве за себя и за свое имущество. Ибо
Господь послал меня вам как вашего питателя,
чтобы остались в живых многие. И расскажите отцу
моему, что я жив еще. Вы сами видите, что Господь
поставил меня отцом Фараону и господином в доме
его и над всею страною Египетскою. И расскажите
отцу моему о всей моей славе и о всем богатстве и
славе, которые дал мне Господь>>. И он дал им по
повелению Фараона колесницы и пищу на дорогу и
дал им цветные одежды и серебро; и отцу их также
Фараон послал одежд, и серебра, и десять ослов,
которые везли хлеб. И он отпустил их, и они пошли и
рассказали своему отцу, что он жив, и что он всем
народам земли отпускает хлеб, и что он поставлен
господином над всею Египетскою землею. И отец их
не поверил этому, ибо он был поражен в своей душе.
И после сего он увидел колесницы, которые прислал
Иосиф; тогда опять ожил вновь дух его. И он сказал:
<<Довольно для меня, что Иосиф жив; я пойду и
увижу его, прежде чем умру>>.

\vs Jub 44:1
И Израиль пошел из своего жилища Хеврона в
новолуние третьего месяца, и зашел к клятвенному
колодезю, и принес жертву Богу отца своего Исаака
в седьмой день этого месяца. И Иаков вспомнил сон,
который он видел в Вефиле, и убоялся идти в
Египет. И, подумав, он хотел известить Иосифа,
чтобы он пришел к нему, и что он сам не пойдет; он
оставался там семь дней, ожидая, не увидит ли
он, быть может, видение о том, оставаться ли ему
или идти. И он совершил праздник жатвы~--- начатков
хлеба~--- со старым хлебом, ибо во всей стране
Ханаанской не было и пригоршни семян, но был
голод для всех зверей, и скота, и птиц, и людей. И в
шестнадцатый день явился ему Господь и сказал:
<<Иаков, Иаков!>> И он сказал: <<Вот я
здесь>>. И Он сказал ему: <<Я Бог отцов твоих,
Авраама и Исаака; не бойся и иди в Египет! Ибо Я
сделаю тебя там великим народом; Я пойду с тобою,
и приведу (возвращу) тебя в эту землю, чтобы ты был
погребен здесь. И Иосиф закроет своими руками
твои глаза. Не бойся, иди в Египет!>>

И они собрались, его дети и дети его детей, и
посадили своего отца и положили свое имущество
на колесницы. И Израиль пошел от клятвенного
колодезя в шестнадцатый день этого третьего
месяца и отправился в страну Египет. И Израиль
послал сына своего Иуду вперед себя к Иосифу,
чтобы осмотреть страну Гесем. Ибо сюда~--- так
сказал Иосиф братьям~--- они должны были прийти,
чтобы жить здесь, дабы быть им вблизи его. И это
хорошая страна в земле Египте; и она была
недалеко от него.

Вот имена сыновей Израиля, которые пошли с
своим отцом Иаковом в Египет. Иаков, отец их.
Робел, перворожденный Израиля. И вот имена его
сыновей: Енох, Фалус, Есером, Карами~--- пятеро.
Симеон и его сыновья; и вот имена его сыновей:
Иямуел, Иямин, Аод, Ияким, Саар, Саул, сын
Сефенсеянки~--- семеро. Левий и его сыновья; вот
имена сыновей его: Гедеон, Кааф и Мерари~--- четверо.
Иуда и его сыновья; и вот имена его сыновей: Селом,
Фарес, Зара~--- [четверо]. Исашар и его сыновья; и
вот имена его сыновей: Фола, Фуа, Иясоб, и Сам~---
пятеро. Заблон и его сыновья; и вот имена его
сыновей: Саор, и Елом, и Иялиел~--- четверо. И вот
сыновья Иакова, которых родила Иакову Лия в
Месопотамии, шесть сыновей и одна сестра их Дина.

И всех душ детей Лии и их детей, которые пошли со
своим отцом Иаковом в Египет, было двадцать
девять; с отцом их Иаковом было тридцать. И дети
Залафы, служанки Лии, жены Иакова, которых она
родила Иакову, суть Гад и Асер. И вот имена их
детей, которые пошли с ними в Египет. Дети Гада:
Сафион, Агафи, Суни, Асон, Араби, Аради~--- восьмеро.
И дети Асера: Иямна, Иесуа, Баръя и Сара, их сестра.
Всего четырнадцать душ. И всех детей Лии было
сорок четыре. И дети Рахили, жены Иакова,~--- Иосиф и
Вениамин. И у Иосифа родились в Египте, прежде чем
отец его пришел в Египет, сыновья, которых родила
ему Ассенеф, дочь Питфары Гелиопольского,~---
Манассе и Ефрем~--- трое. Дети Вениамина: Лаубаел,
Асбел, Гуав, Нееман, Абродио, Раифес, Ианини, Афим,
Яам, Гаам~--- одиннадцать. И всех детей Рахили было
четырнадцать. И дети Баллы, служанки Рахили, жены
Иакова, которых она родила Иакову,~--- Дан и
Неффалим. И вот имена их детей, которые пошли с
ними в Египет. Дети Дана: Куси, Самой, Асуд, Иясек,
Саломон~--- шестеро. И они умерли в Египте в тот год,
в который пришли, и у Дана остался только Куси. И
вот имена детей Неффалима: Иязиел, Гахан, Асаар,
Якум, Ау~--- шестеро. И умер Ау, родившийся после
первого голодного года. И всех детей Рахили
вместе было двадцать шесть. И всех душ Иакова,
пришедших в Египет, было семьдесят душ. Вот его
дети и дети его детей~--- всего семьдесят. И пятеро
из них умерли в Египте при Иосифе, не имея детей. И
в стране Ханаанской умерли у Иуды два его сына~---
Ер и Онан, не имея детей. И сыновья Израиля
похоронили тех, которые умерли, и они входят в
число семидесяти человек.

\vs Jub 45:1
И Израиль пришел в Египетскую землю, в страну
Гесем, в новолуние четвертого месяца во второй
год третьей седмины сорок пятого юбилея. Иосиф
вышел навстречу своему отцу Иакову, в страну
Гесем, и пал отцу на шею, и плакал. И Израиль
сказал Иосифу: <<Теперь я умру спокойно, так
как увидел тебя. И ныне да будет прославлен
Господь, Бог Израилев, Бог Авраама, и Бог Исаака,
Который не отвратил Своего милосердия и
благоволения от раба Своего Иакова! Довольно для
меня, что я увидел лицо твое, пока я жив. Да,
истинно видение, которое я видел в Вефиле. Да
будет прославлен Господь, Бог мой, во весь век!>>
И Иосиф и братья его ели пред очами своего отца
хлеб, и пили вино; и Иаков был исполнен великой
радости, что видел Иосифа, как он с братьями
своими пред его глазами ел и пил. И он прославил
Творца всех вещей, Который сохранил его, и
сохранил ему двенадцать его сыновей. И Иосиф дал
своему отцу и своим братьям в дар страну Гесем,
чтобы они жили в ней и в Рамизифино и во всей ее
области, чтобы они владели ею пред глазами
Фараона. И Израиль жил с своими сыновьями в
стране Гесем, лучшей части земли Египетской.
Израиль же был ста тридцати лет, когда он пришел в
Египет. И Иосиф снабжал своего отца, и своих
братьев, и их домочадцев съестными припасами,
насколько они нуждались в них, в продолжение семи
лет голода. И земля Египетская страдала от
голода. И Иосиф подчинил всю страну Египет
Фараону за хлеб, и также людей и скот; все
приобрел Фараон.

И кончились неурожайные годы, и Иосиф дал
народам, жившим в стране, семян и съестных
продуктов, чтобы они посеяли их в восьмой год; ибо
река наводнила всю страну Египет. Именно в семь
лет неурожая она орошала только отдельные места
около берега реки; теперь же она переполнилась. И
египтяне засеяли страну, и она принесла в том
году много хлеба, и это был первый год четвертой
седмины сорок литого юбилея. И Иосиф взял из
хлеба, который они засевали, пятую часть для царя,
и четыре (части) оставил им в пищу и для посева. И
Иосиф сделал это законом для Египетской земли до
сего дня.

И Израиль жил в стране Египте семнадцать лет, и
всей его жизни было три юбилея, сто сорок семь
лет. И он умер в четвертый год пятой седмины сорок
пятого юбилея. И Израиль благословил своих
сыновей пред своею смертью, и сказал им все, что
случится с ними в последние дни; все возвестил он
им, н благословил их. И он дал Иосифу две части в
стране. И он почил с своими отцами, и был погребен
в двойной пещере в земле Ханаанской, рядом с
своим отцом Авраамом, в могиле, которую он
выкопал для себя, в двойной пещере, в стране
Хеврон. И он отдал все свои книги и книги своих
отцов сыну своему Левию, чтобы он хранил их и
возобновлял их для своих детей до сего дня.

\vs Jub 46:1
И было, после того как Иаков умер, умножились
сыны Израиля в стране Египетской и сделались
многочисленными; и они были все единодушными в
своих мыслях, так что брат любил своего брата, и
каждый помогал своему брату; и они умножились
чрезмерно. И было всей жизни Иосифа десять
седмин, которые он прожил после своего отца. И
Иосиф не имел зложелателя, и не случилось с ним
чего-либо худого во все время его жизни, которую
он прожил после отца своего Иакова. Ибо все
Египтяне почитали сынов Израиля в продолжение
всего времени, пока жил Иосиф. И Иосиф умер ста
десяти лет; семнадцать лет он пробыл в стране
Ханаанской, и десять лет был слугою, и три гада
пробыл в темнице, и восемьдесят лет у царя
управлял всею страною Египетскою, И он умер, и все
его братья, и весь тот род.

И он завещал сынам Израиля перед смертью, чтобы
они взяли с собою его кости, когда они выйдут из
Египта. И он взял с них клятву относительно
костей своих; ибо он знал, что Египтяне не отнесут
его тело и не похоронят его в свое время в
стране Ханаанской, так как Ханаанский царь
Мемкерон, владевший страною Ассур, сражался в
долине с царем Египетским, и убил там его, и
преследовал Египтян до ворот Эромона. Но он не
мог вступить в Египет, ибо восстал другой
новый царь над Египтом для управления, и был
могущественнее его. И он возвратился в страну
Ханаанскую, а ворота Египта были заперты, и никто
не приходил в Египет.

И Иосиф умер в сорок шестой юбилей в шестую
седмину во второй год, и они похоронили его в
стране Египетской. И все братья его умерли после
него. И царь Египетский выступил, чтобы сразиться
с царем Ханаанским, в сорок седьмой юбилей во
вторую седмину во второй год. И сыны Израиля
вынесли кости всех сыновей Иакова, кроме Иосифа,
и похоронили их на поле, в двойной пещере на горе.
И большинство возвратилось в Египет; и только
немногие из них остались на горе Хеврон, и твой
отец [Амрам] остался с ними. И царь Ханаанский
победил царя Египетского, и запер ворота Египта.

И он (царь Египетский) замыслил недоброе дело
против сынов Израиля~--- притеснять их, и сказал
египтянам: <<Вот народ сынов Израиля возрос и
сделался многочисленнее нас; употребим же против
них хитрость, прежде чем они слишком размножатся,
и будем притеснять их рабскою работою, прежде чем
нас постигнет поражение и они победят нас в
битве. А не то они вступят в союз с врагами и
выйдут из нашей страны; ибо их сердце и лицо
обращено к стране Ханаанской>>. И он поставил
над ними смотрителей за работами, чтобы они
притесняли их рабскою работою. И они должны были
строить крепкие города для Фараона~--- Питофо и
Рамзе, и должны были строить всякие стены и
оплоты, которые обрушились в городах египетских,
и они сильно притесняли их. Но чем хуже поступали
они с ними, тем больше умножались и увеличивались
они. И египтяне считали сынов Израиля нечистыми.

\vs Jub 47:1
И в седьмую седмину в седьмой год сорок
седьмого юбилея пришел отец твой из страны
Ханаанской, и ты родился в четвертую седмину, в
шестой год, в сорок восьмой юбилей, когда были дни
преследования сынов Израиля. И царь Фараон
Египетский дал повеление относительно них, чтобы
детей их~--- всякое дитя мужеского пола, которое
родится,~--- бросали в реку. И они бросали их в
течение семи месяцев до того месяца, когда ты был
рожден. И твоя мать скрывала тебя три месяца, и на
нее донесли. Тогда она сделала для тебя корзину, и
осмолила ее смолою и асфальтом, и положила ее в
траву на берегу реки, и клала тебя в нее в течение
семи дней. И мать твоя приходила ночью и кормила
тебя грудью; и днем тебя стерегла от птиц сестра
твоя Мария.

И в те дни пришла дочь Фараона Фармуф
искупаться в реке. И она услышала твой голос,
когда ты плакал, и сказала своей служанке, чтобы
она принесла тебя. И она принесла тебя к ней. И она
вынула тебя из корзинки, и сжалилась над тобою. А
сестра твоя сказала ей: <<Не пойти ли мне, и не
призвать ли к тебе одну из евреек, чтобы она
воспитала это дитя для тебя и кормила грудью?>>
И она пошла, и призвала твою мать Ийокабиф, и она
дала ей плату, чтобы она ходила за тобою. И после
сего ты возрос, и тебя привели в дом Фараона, и ты
сделался отроком. И твой отец (Амбран) научил тебя
писанию. И после того как ты окончил три седмины,
он привел тебя в царский дворец, и ты был при
дворе три седмины до того времени, когда ты вышел
из царского дворца и увидел египтянина, который
бил твоего друга из сынов Израиля. И ты убил его и
скрыл его в песке. И в следующий день ты встретил
двоих из сынов Израиля, которые ссорились, и
сказал обидчику: <<Зачем ты бьешь своего
брата?>> И он разгневался, и озлобился, и сказал:
<<Кто поставил тебя начальником и судьею над
нами, разве ты хочешь убить меня, как ты убил
египтянина?>> И ты испугался и убежал
вследствие этих слов.

\vs Jub 48:1
И в шестой год третьей седмины сорок девятого
юбилея ты ушел и оставался (вне Египта) шесть
седмин и один год. И ты возвратился в Египет во
вторую седмину во второй год в пятидесятый
юбилей. И ты знаешь, что Он говорил с тобою у горы
Синай, и что высший Мастема хотел сделать с тобою
на пути, когда ты возвращался в Египет, в праздник
кущей. Не хотел ли он всеми силами умертвить тебя
и спасти египтян от рук твоих, когда увидел, как
ты был послан совершить над египтянами суд и
мщение? И я спас тебя от руки его и совершил
знамения и чудеса, которые ты был послан
совершить в Египте пред Фараоном, и всем его
домом, и рабами его, и его народом. И Господь
совершил мщение над ними, тяжкое мщение за
Израиля, и поражал, и умерщвлял их чрез кровь, и
чрез жаб, и мошек, и песьих мух, и злокачественные
воспалительные нарывы, и их скот Он поразил смертию,
и градом~--- чрез это Он истребил все, что росло у
них,~--- и саранчой, которая поела остаток,
оставшийся от града, и тьмою; и их первенцев из
людей и скота Он истребил. И всем их идолам
отметил Господь и сожег их огнем. И все это
послано было чрез твою руку, чтобы ты совершил
это, [...]. И ты говорил с царем египетским, и пред
всеми его служителями, и пред его народом; и все
случилось по твоему слову; десять великих и
страшных наказаний пришли на страну Египетскую,
чтобы чрез них отметить за Израиля. И все это
совершил Господь за Израиля и согласно завету,
который Он заключил с Авраамом, чтобы отметить
им, ибо они жестоко притесняли их. И высший
Мастема восстал против тебя, и хотел предать тебя
в руки Фараона, и содействовал египетским
волхвам, и помогал им, чтобы и они сделали это
пред твоими глазами. Хотя мы и допустили их
произвести зло, но, однако, не позволили им
врачебных средств, чтобы они воспользовались ими
своими руками. И Господь поразил их (волхвов)
злокачественными нарывами, чтобы они не могли
противостоять ему; ибо мы погубили их, чтобы они
не могли совершить ни одного знамения. Но
несмотря на все знамения и чудеса, высший Мастема
не смутился, ибо он приложил все силы и воззвал к
египтянам, чтобы они преследовали тебя всеми
силами Египта, с своими колесницами и конями, и со
всем множеством народов Египта. И я встал между
тобою и ими, между египтянами и израильтянами, и
спас израильтян от руки их, от руки египтян. И
Господь провел их чрез море, как по сухой земле; и
всех людей, которые выступали для преследования
Израиля, Господь Бог наш ввергнул в море, в
глубину бездны, вместо детей Израиля, за то, что
египтяне бросали их в реку сотня за сотней; за это
совершено над ними мщение, и тысяча сильных мужей
[...] была истреблена за одного погибшего младенца
из детей твоего народа, брошенного ими в реку. В
четырнадцатый, и в пятнадцатый, шестнадцатый,
семнадцатый и восемнадцатый дни высший Мастема
был связан и заключен позади сынов Израиля, чтобы
он не мог обвинять их (пред египтянами). А в
девятнадцатый день мы освободили его, чтобы он
помогал египтянам и чтобы они преследовали сынов
Израиля. И он очерствил сердце их, и ожесточил их,
и стал могущественным над ними по воле Господа,
Бога нашего, чтобы поразить египтян и ввергнуть
их в море. И в пятнадцатый день мы связали его,
чтобы он не обвинял сынов Израиля, в тот день,
когда они требовали у египтян утварь и одежды,
утварь серебряную, золотую и медную, чтобы
обобрать египтян за то, что когда они служили им,
они сильно притесняли их; и мы не допустили, чтобы
сыны Израиля вышли из Египта с пустыми руками.

\vs Jub 49:1
Вспомни заповедь, которую дал тебе Господь
относительно пасхи, чтобы ты соблюдал ее в свое
время, в четырнадцатый день первого месяца, чтобы
ты заколол его (агнца), прежде чем наступит вечер,
и чтобы ели его ночью, в вечер пятнадцатого дня, с
солнечного захода. Ибо день этот есть первый
праздник и первый день пасхи. И вы ели пасху в
Египте, в то время как все силы Мастемы были
освобождены, чтобы умерщвлять всякого первенца в
стране Египетской, от первенца фараонова до
первенца пленной рабыни на мельнице и до екота. И
вот знамение, которое дал им Бог. В каждый дом, у
которого дверной косяк был обрызган кровью
агнца, в этот дом они не должны были входить для
избиения находящихся в нем, так что все, бывшие в
этом доме, спаслись, потому что на двери был знак
крови. И силы Господин сделали все, что только
Господь повелел им, и прошли мимо всех сынов
Израиля. И на них не простерлось бедствие, чтобы
погубить из них чью-либо душу, ни на скот, ни на
человека, ни даже на собаку. В Египте же бедствие
было очень велико, и не было дома, в котором не
было бы умершего, и плача, и сетования. И весь
Израиль спокойно вкушал пасхальное мясо, и пил
вино, и хвалил, и благодарил, и прославлял
Господа, Бога отцов своих, и приготовлялся к
исходу из-под ига. рабства и из злого Египта. И ты
помни этот день во все дни твоей жизни, раз в год,
в свой (определенный) день, согласно со всем
законом относительно сего, и не смешивай этого
дня с другими и этого месяца с другим. Ибо это~---
вечное установление, и оно начертано на небесных
скрижалях для сынов Израиля, чтобы они каждый год
соблюдали праздник, один раз в год, во все свои
роды; и нет предела времени сему, но он (праздник)
утвержден навек. И муж, если он чист и не придет
совершить его в назначенный день, чтобы принести
дар, угодный Господу, в день Его праздника и чтобы
есть и пить пред Господом в день Его праздника,
тот муж должен быть истреблен, если он чист и
находится недалеко, ибо не принес дар Господень в
назначенное время. И грех примет на себя тот муж.
Сыны Израиля, грядущие, должны праздновать пасху
в назначенное для нее время, в четырнадцатый день
первого месяца вечером, в третью часть дня до
третьей части ночи. Ибо две части дня назначены
для света и третья~--- для вечера. Вот то, что
повелел Господь, чтобы ты совершал это в исходе
вечера. И не должно совершаться это утром в
какой-либо час света, но в вечернее время. И они
должны вкушать его в вечернее время до третьей
части ночи, и что останется от всего мяса после
третьей части ночи, они опять должны сожечь
огнем. И они не должны варить его в воде, и не
должны его есть сырым, но тщательно испекши на
огне и изжарив на огне. Его голову, со
внутренностями и ногами его, они должны изжарить
на огне и не раздроблять ему костей. Ради сего
Господь повелел сынам Израиля, чтобы они
праздновали пасху в назначенный для нее день и не
раздробляли у него (агнца) костей; ибо это
праздничный день и назначенный для празднования
день, и нельзя уклоняться от него на день или на
месяц, но в свой праздничный день он должен
праздноваться. И ты скажи сынам Израиля, чтобы
они соблюдали пасху в ее день, ежегодно, один раз
в год, в определенный день, чтобы это было
воспоминанием, которое будет приятно для
Господа, и чтобы не случилось с ними в том году
никакого бедствия и они не были бы умерщвлены и
поражены. Если они будут праздновать пасху в свое
время, соблюдая все, как заповедано, то они не
должны вкушать ее вне святилища Господня; пред
всем народом общества Израилева должны
соблюдать ее в свое время все люди, которые
явились в день ее, чтобы вкушать пред Господом в
святилище вашего Бога, кто имеет двадцать лет и
выше. Ибо так написано и определено, чтобы ели ее
в доме святилища Господня. И когда сыны Израиля
придут в страну, которою они будут владеть, в
страну Ханаанскую, и устроят скинию Господню в
сей стране, в одном из своих отрядов (колен), так
что святилище Господа будет устроено в стране, то
они должны приходить и праздновать пасху среди
скинии Господней, и закалать ее пред Господом из
года в год. И во дни, когда будет устроен дом во
имя Господне в стране их наследия, они должны
ходить туда и закалать пасху вечером, когда
зайдет солнце, в третью часть дня, и должны
окропить кровью порог алтаря, и тук положить на
огонь, который на жертвеннике, мясо же его,
изжаренное на огне, есть в преддверии дома
святилища во имя Господне. И они не должны
совершать пасху в своих городах и в своих местах,
а только пред скиниею Господнею, или пред Его
домом, так как имя Его живет в нем, дабы им не
согрешить пред Господом. И ты, Моисей, скажи сынам
Израиля, чтобы они соблюдали постановление о
пасхе, как повелено тебе, что вы должны соблюдать
ее ежегодно в день ее и также праздник
опресноков, чтобы они ели пресное в продолжение
семи дней, соблюдая праздник Его и принося для
Него ежедневно дар, в те семь пасхальных дней,
пред Господом, на жертвеннике вашего Бога. Ибо
этот праздник вы праздновали с боязливою
робостию, когда вы вышли из Египта, пока не
перешли чрез море в пустыню Сур; ибо на берегу
моря вы окончили его.

\vs Jub 50:1
И потом после сего закона я возвестил тебе о
субботних днях в пустыне Синая, которая
находится между Еломом и Синаем. И также о
субботах земли я сказал тебе на горе Синай и о
юбилейных годах вместе с субботними годами.

Но год его мы не сказали тебе, пока ты не придешь
в страну, которою вы будете владеть. Тогда и
страна должна праздновать свои субботы, когда
они будут жить в ней, и они узнают год юбилея.
Посему я определил тебе седмины и юбилейные годы:
сорок девять юбилейных годов от дней Адама до
сего дня и одна седмина и два года. И еще
предлежат тебе сорок лет, чтобы узнать заповеди
Господа, пока они не переправятся чрез Иордан к
западу. И юбилеи прекратятся, когда Израиль
очистится от всякого блуда, и вины, и нечистоты, и
осквернения, и греха, и злодеяния, и спокойно
будет жить во всей стране, и против него не
восстанет более ни сатана, ни какой-либо
ненавистник, и земля будет с тех пор чистою
всегда.

И вот я записал тебе также повеление
относительно суббот, и все установления законов
относительно них: шесть дней делай дела, и в
седьмой день суббота для Господа Бога вашего. Вы
не должны делать в нее никакого дела, вы, и ваши
сыновья, и ваши рабы, и служанки, и весь ваш скот, и
чужеземец, который у тебя. И человек, который
делает какое-либо дело, должен умереть. Всякий,
кто оскверняет этот день, кто спит с своею женою,
и кто говорит о том, что он хочет предпринять в
нее (в субботу) путешествие или о разного рода
купле и продаже, и кто черпает воду, не приготовив
ее себе в шестой день, и кто поднимает ношу, чтобы
перенести ее из своего шатра или из своего дома,
тот должен умереть. Вы не должны делать никакого
дела в субботу, которого вы не приготовили себе в
шестой день, чтобы есть, и пить, и покоиться, и
соблюдать субботу от всякого дела в этот день, и
прославлять Господа Бога вашего, Который дал ее
вам в праздник. И днем святым, и даем святого
царства для всего Израиля должен быть этот день в
вашей жизни непрестанно. Ибо велика честь,
которой Господь удостоил Израиля, чтобы они ели,
и пили, и насыщались в этот праздничный день, и
отдыхали от всякого дела, которое относится к
человеческим делам, кроме воскурения фимиама и
принесения даров и жертв пред Господом в субботы.
Только это дело пусть совершается в субботы, во
дни дома святилища Господа Бога вашего, чтобы
приносить в умилостивление за Израиля
непрестанно и ежедневно дары в воспоминание,
которое приятно и которое делает их угодными
пред Господом, каждый день года, как повелено
тебе. Но каждый человек, который совершает дело, и
предпринимает путешествие, и ухаживает за своим
скотом, будь это дома или в другом месте, и кто
зажигает огонь, или едет верхом на каком-нибудь
животном, или путешествует на корабле по морю, и
каждый, кто убивает и умерщвляет кого-либо, и кто
закалывает животное или птицу, и кто ловит зверя,
или птицу, или рыбу, и кто постится, и кто ведет
войну в субботний день; всякий, кто делает
что-нибудь из этого в субботний день, тот должен
умереть, чтобы дети Израиля хранили субботу по
заповедям о субботах земли, как это списано с
небесных скрижалей, которые Он дал мне в мои руки,
дабы я написал тебе законы времени и время по
делению его дней.

\chhdr{Отрывки из Книги Юбилеев, сохранившиеся у греческих церковных писателей}
\chhdr{1. Св. Епифаний Кипрский}
Но в Книге Юбилеев, называемой также и Малым
Бытием, можно найти, что эта книга содержит имена
жен Каиновой и Сифовой, чтобы всяким образом были
посрамлены эти слагатели басен для жизни (т.е.
Сифияне). Когда Адам родил сынов и дочерей, было
необходимостью в то время, чтобы его сыновья
вступили в брак с собственными сестрами; ибо это
не было беззаконным, потому что иного рода не
было. Да и сам Адам, можно сказать, был в
супружестве почти с собственной дочерью,
образованною из его тела и созданною Богом для
супружества с ним, и это не было беззаконным. Так
и сыновья его вступили в брак~--- Каин с старшею
сестрою, так называемой Савою, а Сиф, третий сын,
рожденный после Авеля, с сестрою своею,
называемой Азурою. У Адама родились, как
описывает Малое Бытие, и другие девять сыновей,
после тех трех, так что у него было две дочери, а
детей мужеского пола двенадцать: один убитый, а
одиннадцать оставшихся в живых. Ты имеешь
указание на это в Бытии мира и первой книге
Моисеевой, где говорится так: <<и поживе Адам
лет девятьсот тридесять, и роди сыны и дщери, и
умре>>. (Ср. Кн. Юбил., IV).

\chhdr{2. Иоанн Зонара}
Действительно я знаю записанное в Малом Бытии,
что в первый день и небесные силы прежде прочего
были созданы Творцом вселенной; но так как это
Малое Бытие не отнесено к книгам еврейской
мудрости, написанным божественными отцами, то я
ничего, что в ней написано, не считаю достаточно
твердым и не соглашаюсь с этим учением (ср. Кн. Юбил., II).

\chhdr{3. Георгий Синкелл}
В первосозданные сутки, по-еврейскому, в первый
день первого месяца Нисана, как указано прежде,
по-римскому, в двадцать пятый день месяца марта и,
по-египетскому, в двадцать девятый день Фаменофа,
в день божественный, именно в первую неделю, Бог
сотворил небо и землю, мрак и воды, дух и свет, и
сутки, всего семь творений. Во вторые сутки
явилась твердь~--- одно творение. В третьи сутки
было четыре творения~--- появление земли и осушение
ее, рай, многоразличные деревья, травы и семена. В
четвертый день Бог сотворил солнце, и луну, и
звезды. В пятый день Бог сотворил пресмыкающихся
и всех плавающих, великих морских животных и рыб,
и все, что в водах, а также пернатых~--- всего три
творения. В шестой день Бог сотворил
четвероногих и пресмыкающихся на земле, зверей и
человека~--- четыре творения. Вместе всех творений~---
двадцать два, соответственно двадцати двум
еврейским буквам, затем двадцати двум еврейским
книгам и наконец двадцати двум генерациям от
Адама до Иакова, как говорится в Малом Бытии,
которое называют иные откровением Моисея. Эта же
книга говорит, что небесные силы были сотворены в
первый день (Кн.Юбил.,II).

Необходимость побудила меня сообщить нечто и
из того, что и другими историками, записавшими
иудейские древности и христианские
повествования, заимствуется о сем из Малого
Бытия и так называемой Жизни Адама~--- хотя она и не
считается божественною,~--- чтобы исследующие это
не впали в нелепейшие вымыслы. Итак, в известной
под именем Жизни Адама указывается число дней,
когда было наименование животных, и образование
жены, и вход Адама в рай, и заповедь Божия к нему о
пище с дерева, и вход Евы после сего в рай, также
обстоятельства преступления заповеди и
последствия преступления, как далее следует.

В первый день недели, который был третьим от
сотворения Адама, восьмой первого месяца Нисана,
первый месяца апреля и, по-египетскому, шестой
месяца Фармуфи, Адам по некоему божественному
благоизволению наименовал диких зверей; во
второй день второй недели он дал имена скотам; в
третий день второй недели он наименовал
пернатых; в четвертый день второй недели он
наименовал пресмыкающихся; в пятый день второй
недели он наименовал плавающих. В шестой день
второй недели, который, по-римскому, был шестой
день апреля, а по-египетскому, одиннадцатый
месяца Фармуфи, Бог, взявши некую часть ребра
Адамова, образовал жену. В сорок шестой день от
сотворения мира, в четвертый день седьмой недели,
четырнадцатого Пахона, девятого мая, когда
Солнце было в знаке Тельца и Луна против
созвездия Скорпиона, в восход Плеяд, Бог ввел
Адама в рай в сороковой день после его
сотворения. В пятидесятый день от сотворения
мира, в сорок четвертый от сотворения Адама, день
божественный, восемнадцатого Пахона,
тринадцатого мая, через три дня после входа его в
рай, когда Солнце было в знаке Тельца и Луна в
знаке Козерога, Бог заповедал Адаму не вкушать от
древа познания.

В девяносто третий день творения, во второй
день четырнадцатой недели, во время летнего
поворота Солнца, когда и Солнце и Луна были в
созвездии Рака, в двадцать пятый день месяца
июня, первого Епифи, введена была Богом в рай
помощница Адама Ева, в восьмидесятый день по
сотворении ее. Взяв ее, Адам дал ей имя~--- Ева, что
значит жизнь. Посему Бог повелел чрез Моисея в
книге Левит, именно, ради дней пребывания их вне
рая по сотворении, чтобы она (женщина) оставалась
нечистою при рождении мальчика сорок, а при
рождении девочки восемьдесят дней; посему и Адам
в сороковой день по сотворении введен был в рай,
ради чего и новорожденных в сороковой день
приносят в храм по закону. При рождении же
девочки она должна быть нечистою восемьдесят
дней, ради того, что она (Ева) вошла в рай в
восьмидесятый день, и ради женской нечистоты в
отношении к мужу; даже и находящаяся в месячном
очищении не входит в храм до семи дней по
божественному закону.

Это я ради любознания в сокращении заимствовал
из так называемой Жизни Адама (Кн.Юбил.,III).

Из Малого Бытия:

В седьмой год он согрешил, и в осьмой они были
изгнаны из рая, как говорит (Малое Бытие), чрез
сорок пять дней после падения, в восход Плеяд.

Пробыл же Адам в раю седмину трехсот
шестидесяти пяти дней; и изгнан был с женою Евою
за преступление заповеди в десятый день месяца
мая.

Звери, и четвероногие и пресмыкающиеся, говорит
Иосиф и Малое Бытие, до падения говорили одним
языкам с первосотворенными; посему, говорит, змей
беседовал с Евою человеческим голосом, что,
кажется, невозможно.

В восьмой год (говорит) Адам познал Еву, жену
свою.

В восьмидесятый род родился у них первородньм
сын Каин.

В семьдесят седьмой год, говорят, родился
праведный Авель.

В восемьдесят пятый год родилась у них дочь, и
они дали ей имя Асуам.

В девяносто седьмой год Каин принес жертву.

В девяносто девятый год Авель принес жертву
Богу, имея от роду двадцать два года, в полнолуние
седьмого еврейского месяца, то есть в праздник
кущей.

Достойно примечания, что Писание называет
жертву Каина принесением плодов, а жертву Авеля
дарами, обозначая сим настроение каждого.

В тот же девяносто девятый год Каин убил Авеля,
и первозданные оплакивали его четыре седмины, то
есть двадцать восемь лет.

В сто двадцать седьмой год Адам и Ева
прекратили свой плач. В сто тридцать пятый год
Каин взял собственную сестру Асавнан, которой
было пятьдесят лет; а сам он был шестидесяти пяти
лет (Кн.Юбил.,III,IV).

В двести тридцать четвертом году он родил дочь,
которой дал имя Азуран (Кн.Юбил.,IV).

В четыреста двадцать пятом году Сиф взял в жены
собственную сестру Азуран; Сиф же был девяносто
одного года (Кн.Юбил.,IV).

В том же девятьсот тридцатом году умер и Каин от
обрушившегося на него дома; ибо и сам он камнями
убил Авеля (Кн.Юбил.,IV).

В этом 2251 году, как говорят, Ной насадил
виноградник на горе Лувар в Армении (Кн.Юбил.,VII).

Ангел, говоривший с Моисеем, сказал ему: я
научил Авраама еврейскому языку, каким он был от
начала творения, чтобы он говорил на нем, как на
природном, о чем говорится в Малом Бытии (XII).

В 3373 году от сотворения мира, когда Аврааму был
61 год, сожег Авраам идолов отца своего, и вместе с
ними сожжен был Арран, хотевший тушить огонь
ночью. И вышел Фарра с Авраамом, чтобы идти в
землю Ханаанскую, и, переменив намерение, жил в
Харране, предаваясь идолопоклонству до своей
смерти (Кн.Юбил.,XII).

В сто пятьдесят третьем году жизни Исаака,
Иаков возвратился к нему из Месопотамии. И Исаак,
возведя очи и увидя сыновей Иакова, благословил
Левия, как первосвященника, и Иуду, как царя и
начальника. Ревекка побудила Исаака, уже бывшего
в старости, чтобы он внушил Исаву и Иакову любить
друг друга. И он, увещевая их, предсказал, что если
Исав восстанет на Иакова, то впадет в руки его. И
вот после смерти Исаака, Исав, возмущаемый своими
сыновьями, собрав людей, вышел войною против
Иакова и его сыновей. Иаков, заперев ворота башни,
увещевал Исава вспомнить родительские
завещания. Когда же он не склонился на увещания, а
напротив, стал оскорблять и поносить его, Иаков,
побуждаемый Иудой, натянул лук и поверг Исава,
поразив его в правый сосок груди. После его
смерти сыновья Иакова, открыв ворота, перебили
весьма многих. Это говорится в Малом Бытии
(XXXVII,XXXVIII).

\chhdr{4. Михаил Глика}
Не потому, что он (змий) имел прежде ноги, как
говорит Иосиф и так называемое Малое Бытие,
теперь Бог объявляет, что он будет ходить на
чреве; но, как объясняет Златоустый Иоанн, прежде
он благодаря прямому положению имел такую
смелость, что приблизился к самому уху Евы и
разговаривал с ней, а теперь осужден, и
совершенно справедливо, ползать по земле
(Кн.Юбил.,III).
Малое Бытие говорит, что Адам неосмотрительно
взял от древа и ел и не обратил полного внимания
на слова Евы, потому что изнемог от труда и
голода. Но об этом, возлюбленный, лучше умолчать,
ибо, как сказано выше, бывает нечто достойное и
молчания; разве только и ты хочешь говорить, что
Адам взял жену, чтобы не обратиться на других
животных. Змий стал пресмыкающимся из скота, и
имел руки и ноги; но это было отнято ради того, что
он дерзновенно вошел в рай и посему первый взял
от древа и ел. Адам отгонял птиц и пресмыкающихся,
собирал плод в раю и ел его с своею женою. Вот
это-то, чтобы не сказать, и еще гораздо большее из
подобного, содержит Малое Бытие. Но оставь это;
ибо иначе относящимся к Священному Писанию (это)
покажется, напротив, смешным и забавным (Кн.Юбил.,III).

\chhdr{5. Георгий Кедрин}
В Малом Бытии говорится, что Мастифат,
начальник демонов, приблизясь к Богу, сказал Ему:
если Авраам любит Тебя, пусть принесет Тебе в
жертву сына своего (Кн.Юбил.,XVII).
Ревекка, приготовив кушанье, отдала его Иакову
и ввела его вместе с другими дарами для Исаака к
Аврааму; взяв его на свое лоно и многообразно
благословив его, Авраам, почивши, умер, на
пятнадцатом году жизни Иакова (Кн.Юбил.,IXX).
В Малом Бытии говорится, что израильские дети
были бросаемы в реку только в течение десяти
месяцев, пока Моисей не был поднят царицею.
Посему на египтян были посланы десять казней в
течение десяти месяцев, и наконец они были
ввергнуты в море, по образу того, как они погубили
в реке еврейских детей,~--- за одного израильского
мальчика тысяча погубленных сильных мужей из
египтян. Самого же Моисея дочь Фараона усыновила
в царском достоинстве, но, конечно, не освободила
израильтян от порученной им работы
(ср.Кн.Юбил.,XLVII,XLVIII).
Моисей первый написал законы для иудеев.
Оставив занятия, соответственные Египту, Моисей
в пустыне изучал мудрость, получая откровения от
архангела Гавриила о происхождении мира и
первого человека, о бывшем после него, о потопе, о
смешении и многообразии языков, о событиях из
жизни первого человека, о происшествиях до его
времени, о законе, который он должен был дать
народу иудейскому, также о положении звезд, о
стихиях, арифметике, геометрии и всякой мудрости,
как говорится в Малом Бытии (ср.Кн.Юбил.,I).

\bibbookdescr{Asn}{
  inline={Книга Иосифа и Асенефи},
  toc={Иосиф и Асенефь},
  bookmark={Иосиф и Асенефь},
  header={Иосиф и Асенефь},
  abbr={Асн}
}
\vs Asn 1:1
И было в 1-ый год 7-ми лет изобилия, в 3-ий месяц, в 5-ый день месяца.
\vs Asn 1:2
И послал фараон Иосифа обойти всю страну Египетскую.
\vs Asn 1:3
И в 4-ом месяце 1-го года, в 18-ый день месяца
он прибыл в пределы Илиополя, и он собрал
пшеницы полей края того, как песок морской.
\vs Asn 1:4
И был муж в том городе, сатрап фараона;
и был он поставлен над всеми сатрапами, и превосходил разумом
всех вельмож фараоновых.
\vs Asn 1:5
И был муж тот весьма богат,
и был он советником фараона,
и было имя ему Потифер, жрец илиопольский.
\vs Asn 1:6
У него была дочь, около
18-ти лет от роду, дева высокого роста и прекрасная лицом,
превосходившая бывших на земле.
\vs Asn 1:7
В ней не было никакого
сходства с дщерями египтян; она во всём походила на дочерей Еврейских:
была она высока, как Сарра, и благообразна,
как Ревекка, и прекрасна, как Рахиль.
И было имя девы той Асенефь.
\vs Asn 1:8
И слава о красоте её прошла по всей земле той,
и даже до пределов земли той; искали её руки и сыновья всех
сатрапов, и сыновья вельмож, и все царственные юноши, и военачальники;
\vs Asn 1:9
и разделяла их всех ревность и вражда из-за Асенефи,
и они готовы были из-за неё воевать между собою.
\vs Asn 1:10
И первородный сын фараона,
услышав о ней, стал просить отца своего дать её ему в жёны,
и говорил отцу:
дай мне в жёны Асенефь, дочь Потифера, жреца илиопольского.
\vs Asn 1:11
И ответил ему отец его, фараон:
зачем домогаешься ты жены ниже тебя?
не ты ли царь всей вселенной?
Ведь обручена уже с тобою дочь моавитского царя,
царевна красоты отменной; её и бери в жёны.

\vs Asn 2:1
И Асенефь уничижала и презирала всякого мужа
и была очень горда и надменна в отношении всех.
Никакой муж никогда не видел её.
\vs Asn 2:2
При доме Потифера была башня,
весьма великая и высокая,
и в ней горница, имевшая 10 комнат,
где она и жила, никем не видимая.
\vs Asn 2:3
И была 1-ая комната велика и благолепна:
пол её был выложен каменьями порфировыми;
\vs Asn 2:4
и была та горница убрана мрамором;
её стены были унизаны драгоценными блестящими камнями;
под кровом её поставлены были боги египетские,
золотые и серебряные, без числа.
\vs Asn 2:5
И всех их почитала Асенефь,
боялась их,
всегда приносила им жертвы всесожжения и фимиам.
\vs Asn 2:6
И 2-ая комната была хранилищем всего убранства
Асенефи и всех ларцов её с золотом, серебром,
златоткаными ризами,
превосходными дорогими камнями,
всем девичьим её убранством.
\vs Asn 2:7
И 3-ья комната содержала все блага земные
и служила Асенефи кладовой.
\vs Asn 2:8
Остальные же 7 комнат отданы были 7-ми девам,
по одной каждой.
И девы эти служили Асенефи, одного года,
родившиеся в одну ночь с нею.
\vs Asn 2:9
И все они были прелестны, как звёзды небесные.
С ними никогда не говорил муж, ни даже дитя мужского пола.
\vs Asn 2:10
В комнате, где охранялось девство Асенефи,
были 3 больших окна.
И 1-ое, самое большое, выходившее на двор,
было обращено на восток;
2-ое глядело на юг,
а 3-ье на север, где прямая дорога.
\vs Asn 2:11
В комнате, выходившей на восток,
утверждено было золотое ложе,
убранное золотой пурпуровой тканью,
украшенное иакинфом и виссоном.
\vs Asn 2:12
На ложе том почивала Асенефь,
и не сидел на ложе том ни один муж с женой,
кроме одной Асенефи.
\vs Asn 2:13
Обширный двор окружал комнаты,
а двор высокие четырёхугольные стены из больших камней.
\vs Asn 2:14
Входили во двор 3-мя железными воротами,
которые охранялись 8-ью сильными вооруженными мужами.
\vs Asn 2:15
На дворе, вдоль стены,
росли различные красивые
плодовые деревья со спелыми на них плодами,
ибо наступила пора урожая.
\vs Asn 2:16
На правой стороне двора был большой источник,
в\acc{о}ды которого, стекались в водоём; от водоёма исходил ручей,
бегущий посреди двора, орошая находившиеся там деревья.

\vs Asn 3:1
И было на 1-ом году 7-ми лет изобилия,
в 4-ый месяц, в 18-ый день месяца,
когда Иосиф вступил в пределы илиопольские
для собирания хлеба во время изобилия.
\vs Asn 3:2
Приблизившись к тому городу,
Иосиф послал перед лицом своим 12 мужей к жрецу Потиферу сказать:
\vs Asn 3:3
Сегодня я остановлюсь у тебя,
ибо вот полдень, час трапезы,
и солнечный жар усиливается,
и отдохну под сенью твоего дома.
\vs Asn 3:4
Потифер, услышав это,
возрадовался радостью великой, и сказал:
Да будет благословен Бог Иосифов,
внушивший ему посетить нас!
\vs Asn 3:5
И Потифер, призвав
домоправителя своего, сказал ему:
Поспеши, и устрой дом мой, и приготовить
большой обед, ибо Иосиф, сильный бог, ныне придёт к нам,.
\vs Asn 3:6
Асенефь, услышав, что отец
её и мать возвратились с поля наследия её, обрадовалась и сказала:
\vs Asn 3:7
пойду и увижу отца моего и
матерь, возвратившихся с поля наследия моего; то было время жатвы.
\vs Asn 3:8
И Асенефь поспешно надела на себя виссонную ризу златотканную,
шитую нитями иакинфовыми; опоясалась золотым поясом,
надела обручи на руки и обручи на ноги, дорогое ожерелье на шею и
усыпанную различными камнями обувь на ноги.
\vs Asn 3:9
И на всём её убранстве были начертаны
имена богов египетских, на ожерелье же её и на драгоценных камнях
вырезаны были лица идолов.
\vs Asn 3:10
И возложила она на голову венец,
и замкнула повязку вокруг висков своих,
а сверху покрылась летним покрывалом.

\vs Asn 4:1
И поспешила она, и спустилась
по лестнице из своей горницы навстречу отцу и матери
и поклонилась им с приветствием.
\vs Asn 4:2
И возрадовались Потифер и жена его радостью великой,
глядя на дочь свою; ибо видели её родители
её нарядившейся как невесту бога.
\vs Asn 4:3
И вынесли они всё добро,
что принесли они с поля наследия их, и дали дочери своей.
\vs Asn 4:4
И возрадовалась Асенефь о добре том,
при виде всех плодов винограда, смоквы и финика, и о гранатовых
яблоках, ибо всё было в поре той.
\vs Asn 4:5
И сказал Потифер дочери своей Асенефи: Дитя моё!
\vs Asn 4:6
---~Вот я, господин мой!
\vs Asn 4:7
И он сказал: Пойди, сядь между нами и скажу тебе слова мои.
\vs Asn 4:8
И села Асенефь между отцом своим и матерью.
\vs Asn 4:9
И взял Потифер правую руку дочери и, поцеловав её, сказал: Дитя моё!
\vs Asn 4:10
---~Да говорит господин мой и отец мой!
\vs Asn 4:11
И сказал Потифер:
Вот, Иосиф, сильный бог, сегодня придёт к нам: он повелитель всей страны
Египетской, ибо фараон поставил его над всеми своими владениями,
\vs Asn 4:12
и он спаситель всей нашей земли,
ибо доставляет хлеб всей стране нашей,
чем и избавит людей от предстоящего голода.
\vs Asn 4:13
Иосиф муж благочестивый, целомудренный,
скромный, и девственник, как ты ныне, муж, сильный в премудрости
и знании, ибо с ним дух Божий и благодать Господня.
\vs Asn 4:14
Итак, дитя моё,
приди и я отдам тебя ему в жёны и он будет тебе мужем навсегда.
\vs Asn 4:15
Асенефь, услышав слова отца
своего, побледнела и разлился по ней пот кровавый, обильный.
\vs Asn 4:16
С гневом посмотрев на отца, она сказала:
отец, господин мой!
Неужели по этим словам ты, как рабу,
отдашь меня человеку чужому, беглому, проданному в рабство?
\vs Asn 4:17
Не сын ли он пастуха из земли ханаанской?
Не он ли был уличён в том, что лёг с госпожою своею,
за что господин его бросил его в мрачную темницу,
откуда вывел его царь,
потому что тот истолковал его сон,
как толкуют старицы египетские?
\vs Asn 4:18
Нет, но я сочетаюсь с первородным сыном фараона,
ибо он царь всей земли.
\vs Asn 4:19
Услышав это, не стал
Потифер продолжать разговор с своею дочерью об Иосифе,
так как она ответила ему дерзко и гневно.

\vs Asn 5:1
И пришёл к Потиферу один из
отроков его, и говорит: вот, Иосиф у ворот двора нашего!
\vs Asn 5:2
И убежала Асенефь от лица
отца своего и матери, как только услышала, что они хотят отдать её за Иосифа,
взошла в горницу и вступила в свою комнату.
\vs Asn 5:3
И стала она у большого своего окна,
выходящего на восток, чтобы видеть Иосифа, входящего в дом отца её.
\vs Asn 5:4
И вышли Потифер, и жена его,
и все рабы его, и все слуги дома его Иосифу навстречу, и отверзли восточные
ворота двора.
\vs Asn 5:5
И въехал Иосиф, восседая на 2-ой колеснице фараоновой,
запряжённой 4-мя белоснежными конями,
все в золотых удилах; и вся колесница была из цельного золота.
\vs Asn 5:6
И Иосиф был облачён в белую прекрасную одежду
с пурпуровой накидкой из златотканого виссона,
с золотым венцом на главе.
\vs Asn 5:7
Вокруг венца вделаны были 12 драгоценных камней,
и на камнях 12 блестящих лучей из золота.
\vs Asn 5:8
В левой руке у Иосифа был жезл,
а в правой масленичные ветви с тучными плодами.
\vs Asn 5:9
И он вступил во двор,
и затворены были за ним все ворота.
\vs Asn 5:10
И муж\acc{и} и жёны остались за
воротами, ибо привратники заложили их и никому не дали входить.
\vs Asn 5:11
И пришли Потифер, и жена его,
и все сродники его, кроме дочери его Асенефи,
и пали на лицо своё и поклонились Иосифу.
\vs Asn 5:12
И Иосиф сошел с колесницы
своей, и они приняли его в свои объятия.
\vs Asn 6:1
И увидела Асенефь Иосифа и полюбила его сильною любовью:
и сокрушилось сердце её, и подкосились колени её,
и дрожь напала на всё тело её,
и великий страх напал на Асенефь,
и ужас овладел ею, и она сказала со вздохом:
\vs Asn 6:2
куда пойду я и куда сокроюсь от лица его?
или как взглянет на меня Иосиф, сын Божий?
ибо худое говорила я о нём.
куда бегу и укроюсь?
\vs Asn 6:3
Ибо всё сокрытое видит он
и ничто тайное не утаится от него
по причине великого света, пребывающего в нём.
\vs Asn 6:4
И ныне милостив будь ко мне, Бог Иосифа,
ибо в неведении говорила я слова лукавые.
\vs Asn 6:5
Что сделаю теперь я, несчастная?
Давно ли с презрением говорили о нём со мною отец мой и мать,
что идёт к нам сын пастуха из земли ханаанской~--- так
они отзывались об Иосифе!
\vs Asn 6:6
Ныне же само солнце с неба приходит
к нам в колеснице его, и вступает в наш дом.
\vs Asn 6:7
И я, неразумная, дерзкая,
негодная, с презрением дурно говорила о нём,
не ведая, что Иосиф сын богов;
\vs Asn 6:8
ибо невозможно родиться человеку с такой красотой,
и какая утроба произведёт такого светозарного человека!
\vs Asn 6:9
Я же, злополучная и неразумная,
худое говорила о нём со своим отцом!
\vs Asn 6:10
И теперь господин мой удалил меня от него;
ибо я по неведению худо отозвалась о нём;
пусть теперь мой отец отдаст меня
к нему в рабы в вечное услужение.

\vs Asn 7:1
И вступил Иосиф в дом Потифера и сел на седалище.
\vs Asn 7:2
И омыли ноги его, и приготовили ему трапезу особо:
ибо Иосиф не ел с египтянами,
считая осквернением вкушать с ними.
\vs Asn 7:3
И говорит Иосиф Потиферу и всем его сродникам:
кто эта женщина, которая стоит в горнице у окна?
пусть она удалится отсюда, из этого дома.
\vs Asn 7:4
Ибо Иосиф опасался беспокойства от неё;
ибо досаждали ему все жёны и дочери вельмож египетских,
желавшие возлечь с ним.
\vs Asn 7:5
При виде его они воспламенялись страстью к нему;
но Иосиф презирал их; и посланцев,
которых жёны египетские посылали
к нему с золотом и серебром и богатыми дарами,
он отсылал с бранью и угрозой.
\vs Asn 7:6
И говорил он перед Господом:
нет, не сотворю греха перед лицом Бога Израилева.
\vs Asn 7:7
И он всегда имел перед глазами образ отца своего,
Иакова, и не забывал заповедей отца своего,
который говорил Иосифу и всем сыновьям своим:
\vs Asn 7:8
берегитесь, сыны мои, жён иноплемённых,
не имейте с ними общения;
ибо общение с ними гибель для вас и осквернение.
\vs Asn 7:9
Вот почему Иосиф сказал:
Пусть та женщина удалится из этого дома.
\vs Asn 7:10
---~Господин! Та, которую ты видел в горнице,
не чужая женщина, но дочь наша и раба твоя:
\vs Asn 7:11
она дева, не видевшая мужа,
и никто из мужей ещё не видел её, кроме тебя сегодня.
\vs Asn 7:12
Если желаешь, она придёт поклониться тебе,
ибо дочь наша тебе сестра.
\vs Asn 7:13
И возрадовался Иосиф радостью великой,
когда Потифер сказал, что она дева
и что она ещё не видела мужа.
\vs Asn 7:14
Он подумал в мыслях своих, сказав сам себе:
если она дева, то должна ненавидеть всякого мужа
и не будет обременять меня.
\vs Asn 7:15
И говорит Иосиф Потиферу и всем сродникам его:
если дочь твоя дева, пусть она придёт, и так как она
сестра мне, то отныне я готов любить её как сестру свою.
\vs Asn 8:1
И взошла мать её в горницу,
и привела Асенефь, и поставила её перед Иосифа.
\vs Asn 8:2
И сказал Потифер Асенефи:
Дочь моя!
Приветствуй брата твоего; ибо он подобно тебе целомудрен по сей день
и ненавидит всякую жену чужую, как и ты всякого чужого мужа.
\vs Asn 8:3
И Асенефь сказала Иосифу:
радуйся, господин, благословенный Всевышнего Бога!
\vs Asn 8:4
И говорит Иосиф Асенефи:
да благословит тебя Господь, дающий жизнь всему!
\vs Asn 8:5
И сказал Потифер: Дочь моя!
Подойди и поцелуй брата своего.
\vs Asn 8:6
И когда подошла Асенефь поцеловать Иосифа,
простёр Иосиф десницу свою и, положив её на грудь её, сказал:
\vs Asn 8:7
Не подобает мужу богобоязненному,
который благословляет Бога живого устами своими,
который вкушает хлеб благословенный и животворящий,
который пьёт благословенную чашу бессмертия,
помазуется помазанием нетления,
\vs Asn 8:8
лобызать жену иноплемённую,
благословляющую своими устами мёртвых и немых идолов,
вкушающую с жертвенников их удавленину,
и пьющую на возлияниях их из чаши вино обмана,
и помазующуюся помазанием погибели.
\vs Asn 8:9
Но мужу богобоязненному надлежит лобызать
своих благочестивых, возлюбленных мать и сестру,
и всех из своего племени и народа,
и жену, делящую с ним ложе,
устами своими благословляющих Бога Живого.
\vs Asn 8:10
Так же и жене богобоязненной не подобает
лобызать чужого мужа, ибо это скверна перед Богом.
\vs Asn 8:11
И когда услышала Асенефь слова Иосифа,
сильно опечалилась; и стала она воздыхать,
и смотрела на Иосифа со страхом, и глаза её наполнились слезами.
\vs Asn 8:12
При виде этого Иосиф сжалился над нею,
ибо был Иосиф кроток и милостив, и боялся Бога.
\vs Asn 8:13
И он поднял десницу свою и
возложил её на голову её, и сказал:
\vs Asn 8:14
Господь, Бог отца моего
Израиля, сильный и Вышний Бог Иакова!
\vs Asn 8:15
Ты, который из мрака вызвал всё существующее к свету!
\vs Asn 8:16
Ты, который вывел из заблуждения к истине, из смерти к жизни,
\vs Asn 8:17
Господи, животвори и благослови деву сию, и обнови её духом твоим,
\vs Asn 8:18
и воссоздай её невидимой твоею рукою, и сообщи ей новую жизнь.
\vs Asn 8:19
И да вкушает она хлеб жизни, и да пьёт она от чаши благословения:
\vs Asn 8:20
приобщи её к народу твоему,
избранному тобою прежде мироздания,
\vs Asn 8:21
и да войдёт она в покой твой, уготованный тобою твоим возлюбленным,
\vs Asn 8:22
и да живёт она жизнью вечною!

\vs Asn 9:1
И возрадовалась Асенефь радостью великой
при благословении Иосифа,
и поспешно возвратилась в уединённую свою горницу,
и пала на своё ложе с воздыханиями.
\vs Asn 9:2
Ибо нашли на неё и радость,
и печаль, и страх, и трепет,
и сильный пот, когда услышала она те слова Иосифа,
что говорил он ей во имя Бога Всевышнего,
\vs Asn 9:3
и плакала она плачем великим и горьким:
раскаяние объяло её сердце при мысли о своих богах,
которым она служила; и она возненавидела всех своих идолов.
\vs Asn 9:4
Так она пробыла до наступления вечера.
\vs Asn 9:5
И Иосиф ел и пил, и по окончании трапезы приказал своим отрокам:
\vs Asn 9:6
Запрягите коней в колесницу:
вот, отхожу в путь и обойду город и страну эту.
\vs Asn 9:7
И сказал Потифер Иосифу:
Отдохни здесь, господин мой,
под кровом сим этот день, завтра поедешь в путь свой.
\vs Asn 9:8
И отвечал Иосиф:
Нет, отойду сегодня же,
ибо в сей день Бог начал творить свои создания.
\vs Asn 9:9
В 7-ой же день, когда снова наступит этот день,
возвращусь и я к вам и отдохну под кровом этим.

\vs Asn 9:10
И Иосиф отправился в путь,
а Потифер со всем своим семейством отправился в поле наследия своего.
\vs Asn 10:1
И в доме осталась одна Асенефь с 7-ью девицами:
тосковала она и плакала до заката солнца, не ела хлеба, и воды не пила.
\vs Asn 10:2
И когда наступила ночь, и все бывшие в доме заснули,
не спала одна только Асенефь.
\vs Asn 10:3
И вспоминая Иосифа, она плакала и сильно била себя в перси;
великий страх напал на неё, и начала она сильно дрожать.

\vs Asn 10:4
И когда всюду водворилась тишина,
Асенефь открыла дверь свою и спустилась с ложа своего, и сошла из
горницы тихонько по лестнице,
\vs Asn 10:5
и пришла к мельнице, и нашла
мельника спящим вместе с своими сыновьями, и поспешно сняла с дверей шерстяную
завесу,
\vs Asn 10:6
и насыпала в неё пепел из печи,
и понесла в горницу, и положила на пол,
и заперла дверь железным запором.
\vs Asn 10:7
И она начала громко рыдать и плакать.
\vs Asn 10:8
И услышали кормилица и сверстница её,
которую она любила больше всех дев, стенание госпожи своей,
\vs Asn 10:9
и пробудились от сна прочие девы,
и подошли к дверям Асенефи и нашли их запертыми.
\vs Asn 10:10
До них доходили плач и рыдания,
и они спросили: Что с тобою, госпожа наша Асенефь?
Чем ты огорчена?
 Отвори нам, чтоб мы увидели, что с тобою случилось.
\vs Asn 10:11
И Асенефь, не отпирая дверей, отвечала им изнутри, говоря:
Голова моя отяжелела и не нахожу покоя на ложе своём,
\vs Asn 10:12
нет силы отворить вам,
ослабели все члены мои, ни встать не могу, ни отпереть;
\vs Asn 10:13
разойдитесь по своим комнатам,
успокойтесь и дайте мне также успокоиться
и отдохнуть немного.
\vs Asn 10:14
И девы по слову её разошлись по своим комнатам.
\vs Asn 10:15
И встала Асенефь, тихонько отперла дверь
и пошла в другую комнату, где хранились ларцы с убранством её;
\vs Asn 10:16
и открыла ковчежец, и вынула из него
чёрное траурное платье,
которое она надевала,
оплакивая смерть первородного брата своего.
\vs Asn 10:17
И принесла Асенефь то траурное платье
в свою комнату и, положив его, заперла дверь запором.
\vs Asn 10:18
И поспешно совлекла Асенефь с себя
царственное своё одеяние и виссон,
и златотканную порфиру, и облеклась в чёрное;
\vs Asn 10:19
и развязала золотой пояс, и препоясалась вервием;
\vs Asn 10:20
и сложила венец и повязку с головы своей,
и запястья с рук и ног.
\vs Asn 10:21
И взяла она всё это и выбросила в окно,
выходившее на север.
\vs Asn 10:22
И поспешила Асенефь,
и взяла также золотых и серебряных богов,
которым не было числа, и разбила их намелко,
и выбросила их в окно из горницы нищим.
\vs Asn 10:23
И взяла Асенефь царственный свой ужин,
хлеб и рыбу, и мясо тельца,
и жертвы для богов своих, и чаши для вина,
в которых она совершала возлияния, и выбросила в окно.
\vs Asn 10:24
И бросила она всю пищу чужим собакам на съедение,
дабы ужин её, мясо агнцев, приготовленный для
идолов, не сделался пищею её собственных собак.
\vs Asn 10:25
Тогда Асенефь распорола шерстяную завесу,
наполненную пеплом, и посыпала им пол;
\vs Asn 10:26
и взяла вретище, и препоясала чресла свои;
и сняла покрывало с главы своей и расплела свои волосы,
и посыпала главу пеплом,
лежавшим на полу, и накрылась им, и пала ниц в пепел.
\vs Asn 10:27
И начала часто бить себя в перси руками,
рыдать и проливать горькие слёзы всю ночь до утра.

\vs Asn 10:28
И когда рассвело, восстала Асенефь и увидела,
и вот пепел под нею стал от слёз её как грязь болотная.
\vs Asn 10:29
И снова она пала на лицо своё
в пепел и пролежала до вечера, до заката солнца.
\vs Asn 10:30
И так делала Асенефь 7 дней,
в продолжение которых она
не переставала мучить и терзать себя:
7 дней она не вкусила хлеба и не пила воды.

\vs Asn 11:1
И было в 8-ой день, на рассвете,
когда начали кричать петухи
и собаки лаять на проходящих,
она подняла голову свою от пола
\vs Asn 11:2
(ибо члены её расслабели от
непринятия пищи в продолжение 7-ми дней),
и пала на колена и, опершись рукой о пол,
поникла головой.
\vs Asn 11:3
Волосы на голове у неё были
распущены, взъерошены, покрыты густым пеплом.
\vs Asn 11:4
Сложив руки, Асенефь оплакивала свою голову,
била себя в грудь, издавала глубокие вздохи, рвала себе
волосы, посыпая их пеплом.
\vs Asn 11:5
Таким-то образом Асенефь,
утруждая себя, изнемогла, лишилась сил
\vs Asn 11:6
и, обратившись к стене, села у окна,
выходящего на восток, и наклонила голову на грудь,
и положила руки на колена и оставалась безмолвною,
\vs Asn 11:7
ибо не нашлось слова на устах её:
в тесноте своей она в продолжение 7-ми дней не раскрывала уст.
\vs Asn 11:8
И сказала Асенефь в сердце своём:
Что мне делать?
Кто будет моим прибежищем?
К кому я обращусь?
\vs Asn 11:9
Я дева и сирота, всеми покинутая:
отец и мать меня возненавидели, потому что я возненавидела их богов,
я уничтожила, я бросила их на попрание людям;
за это возненавидели меня отец, и мать, и все мои сродники.
\vs Asn 11:10
Отец мой сказал:
Отныне Асенефь не назовётся нашей дочерью,
потому что она уничтожила золотых и серебряных богов наших.
\vs Asn 11:11
И вот, я стала ненавистной
в глазах людей, ибо надмевалась над всеми,
за коих сватали меня.
И теперь все обрадовались моему горю.
\vs Asn 11:12
Об этом только она думала и говорила:
\vs Asn 11:13
Господь Всевышний, Бог Иосифа!
Ты ненавидишь чествующих идолов мёртвых, немых и бездыханных;
ибо ты Бог мстительный и страшный богам чуждым.
\vs Asn 11:14
За это и меня возненавидел Бог,
что я чествовала идолов немых, бездыханных,
за то, что я восхваляла их, что я ела от жертвенного их мяса,
\vs Asn 11:15
уста мои осквернены их трапезой,
и я не имею права взывать к Господу, Богу неба и земли,
к Всевышнему Избавителю Иосифа.
\vs Asn 11:16
Ибо душа моя осквернена
жертвоприношениями и всесожжениями идолам.
\vs Asn 11:17
Слышала я, как говорили,
что Бог евреев Бог истинный, Бог живой,
Бог милостивый, долготерпеливый, многомилостивый,
не вменяющий человеку грехи, терпеливый к кающемуся,
не обличающий человека в тесноте его.
\vs Asn 11:18
Итак, дерзну, обращусь к нему,
сделаю его своим прибежищем,
исповедую ему все грехи мои,
изолью мольбы свои пред ним,
и он помилует меня.
\vs Asn 11:19
Быть может, он взглянет на горе
моё и сжалится надо мною, покинутою;
быть может, он, видя мои рыдания,
поможет мне,
\vs Asn 11:20
ибо он отец сирот и помощник угнетённых,
дерзну и я воззвать к нему~--- быть может, простит меня.

\vs Asn 11:21
И отвернулась Асенефь от стены
и обратилась к окну,
выходящему на восток, и стала на колена свои,
и подняла руки к небу;
\vs Asn 11:22
но страх напал на Асенефь,
и она не могла раскрыть уста свои и произнести имя Бога.
\vs Asn 11:23
И снова обратилась к стене
и села, и стала бить себя руками
в грудь и в голову неоднократно.
\vs Asn 11:24
И говорила она в сердце своём,
не раскрывая уст:
Несчастная я сирота уста мои осквернены жертвенным
мясом идолов и хвалением богов египетских.
\vs Asn 11:25
И хотя проливаю теперь слёзы
и покрываю голову пеплом,
но не могу устами своими хвалить
святое и страшное имя Бога,
боясь гнева его за призывание его имени.
\vs Asn 11:26
Итак, что делать мне, злосчастной?
Дерзну, обращусь к нему:
\vs Asn 11:27
если он в гневе своём низвергнет меня,
то он властен восстановить; если накажет, то может утешить;
при наказании может возобновить меня своею милостью;
\vs Asn 11:28
если грехи мои огорчат его,
то примириться со мною и отпустить все мои грехи.
\vs Asn 11:29
Итак, дерзну, открою уста свои,
обращусь к нему, может быть, сжалится и простит мои прегрешения.

\vs Asn 12:1
И встала Асенефь, отвернулась от стены,
стала на колена, воздела руки свои к востоку,
взглянула на небо и произнесла:
\vs Asn 12:2
Господь, Бог веков!
Ты создал всё, ты оживил всех тварей,
\vs Asn 12:3
ты вывел всё из небытия,
всё видимое из невидимого на свет,
\vs Asn 12:4
ты поднял небо и основал его
на ветрах, и землю утвердил на водах;
\vs Asn 12:5
ты поставил над бездною великие горы,
которые не тонут, но держатся на водах, как дубовый лист:
\vs Asn 12:6
горы те живые, ибо внимают гласу твоему,
Господи, ибо ты сообщаешь жизнь всем созданиям твоим.
\vs Asn 12:7
Господь, Бог мой!
На тебя уповаю, к тебе простираю мольбы мои,
тебе исповедаю грехи мои
и пред тобою открою беззакония мои:
\vs Asn 12:8
пощади меня, Господи,
ибо я во всём согрешила,
совершила преступления перед тобою, Господи!
\vs Asn 12:9
Я произносила недостойные речи,
оскверняла уста свои жертвенным мясом
и от трапезы богов египетских.
\vs Asn 12:10
Согрешила я, Господи,
согрешила и лукавое сотворила,
почитая идолов глухих и мёртвых по неведению;
\vs Asn 12:10
поэтому я недостойна к тебе обратиться с мольбами,
по причине прегрешений моих.
\vs Asn 12:11
Согрешила я, Господи, перед лицом твоим,
я, Асенефь, дочь жреца Потифера,
некогда гордая, надменная,
стоявшая выше всех богатством,
теперь стою как сирота, покинутая всеми.
\vs Asn 12:12
Тебе приношу, Господи, моление моё,
и к тебе взываю: спаси меня от гонящих меня
\vs Asn 12:14
Как испуганное дитя бежит к отцу,
тянется к нему, чтобы тот поднял его с пола,
и, раз уже в его объятиях,
оно крепко обхватывает руками
его шею и тут успокаивается;
\vs Asn 12:15
так и я, преследуемая со всех сторон,
к тебе, Господи, прибегаю:
\vs Asn 12:16
простри руку твою надо мною,
как отец чадолюбивый и милостивый,
и возьми меня с лица земли.
\vs Asn 12:17
Ибо вот старый свирепый лев преследует меня,
ибо он отец богов египетских, а идолы народов~--- дети львов;
\vs Asn 12:18
я же выбросила всех богов,
уничтожила их, а лев отец их,
разгневанный, хочет поглотить меня.
\vs Asn 12:19
Избавь меня, Господи, от когтей его
и спаси из пасти его, дабы он не схватил меня как волк,
\vs Asn 12:20
и не растерзал, и не бросил в огонь печи,
из огня в вихрь, который, охватив,
лишит меня зрения и низвергнет в бездну морскую,
\vs Asn 12:21
где поглотит меня великое чудовище морское,
существующее изначала, и где я погибну на вечные времена.
\vs Asn 12:22
Господи!
Спаси меня, прежде чем постигнет меня всё это;
\vs Asn 12:23
спаси и укрепи меня покинутую,
ибо отец и мать отреклись от меня, сказав:
Асенефь не дочь нам, она уничтожила богов наших,
отвергла их.
\vs Asn 12:24
Ты один, Господи, надежда моя, на тебя уповаю,
ибо ты отец сирот и защитник гонимых, и притесняемых
покровитель.
\vs Asn 12:25
Помилуй меня, деву.
Ты милостив, как отец;
ты жалостлив, как мать;
ты долготерпелив как никто.
\vs Asn 12:26
Ибо вот, всё наследие,
данное мне отцом моим Потифером,
тленно и скоротечно;
твои же дары непреходящи, вечны.
\vs Asn 12:27
Теперь я отреклась от всего
и ото всех, покинула все блага земные.
\vs Asn 12:28
Тебя одного сделала своей надеждой.
\vs Asn 12:29
Одевшись во вретище,
покрывшись пеплом, оплакиваю грехи мои.
\vs Asn 12:30
Призри на сиротство моё,
Господи, ибо к тебе прибегла я.
\vs Asn 12:31
Вот, бросила я царственную
ризу свою златотканную,
виссон и серьги драгоценные и облеклась в хитон чёрный.
\vs Asn 12:32
Вот, сняла с себя золотой пояс
и препоясалась вервием и вретищем.
\vs Asn 12:33
Вот, бросила с головы венец
и покрыла главу мою пеплом.
\vs Asn 12:34
Вот, мраморные полы моего жилища,
прежде убранные разноцветными каменьями и пурпуром,
блестящие чистотой,
теперь омочены моими слезами и омрачены пеплом.
\vs Asn 12:35
Вот, Господь мой, от пепла и слёз плача моего
грязь великая соделалась в чертоге моём,
как на пути проезжем.
\vs Asn 12:36
Вот, я отказалась от царского моего ужина
и бросила его чужим собакам на съедение.
\vs Asn 12:37
И вот, 7 дней и 7 ночей не ела я хлеба, не пила воды;
\vs Asn 12:38
уста у меня высохли,
как кожа тимпана;
язык мой стал как рог;
губы мои сделались как черепица,
лицо моё осунулось;
очи опухли от беспрерывных слёз,
все силы оставили меня.
\vs Asn 12:39
Теперь, узнав, что боги,
чествуемые мною по неведению,
были немые и глухие идолы,
я отдала их на попрание,
серебро и золото растаскали воры и унесли с глаз моих;
\vs Asn 12:40
потому что надежду свою я положила на тебя, Господь, Бог Иосифа!
\vs Asn 12:41
Прости меня, ибо всё сделала я по неведению,
я порицала господина моего Иосифа, не зная,
что он сын возлюбленный у тебя.
\vs Asn 12:42
Мне сказали легкомысленные люди про него,
что он сын пастуха земли ханаанской, и я поверила им.
\vs Asn 12:43
И заблуждалась, и с презрением отнеслась
к избраннику твоему и стала говорить о нём непочтительно,
не зная, что он сын твой.
\vs Asn 12:45
Ибо кто из людей породил такую красоту,
и кто есть другой столь мудрый и сильный, как Иосиф?
\vs Asn 12:46
Ты одарил его дивной красотой, великой мудростью и добродетелью.
\vs Asn 12:45
Но, Господь мой, тебе
поручаю его, ибо возлюбила его я больше души моей.
\vs Asn 12:46
Сохрани его премудростью и благодатью твоею
и предай меня в рабы ему, и буду я омывать ноги его и оправлять
ему ложе, и служить ему во все дни жизни моей.

\chhdr{Исповедь Асенефи перед Богом.}
\vs Asn 13:1
Согрешила я перед тобою, согрешила, Господи!
Совершила беззаконие я, Асенефь, дочь Потифера,
жреца илиопольского, главного смотрителя над всеми богами.
\vs Asn 13:2
Согрешила я перед тобой, Господи, согрешила,
совершила беззаконие, почитала богов, коим нет числа,
и ела от жертвенного их мяса.
\vs Asn 13:3
Согрешила я, согрешила, Господи, перед тобой,
совершила беззаконие; ибо я была дева гордая, надменная.
\vs Asn 13:4
Согрешила я, Господи, согрешила перед тобой,
совершила беззаконие: я ела хлеб удушающий,
пила чашу сетей, вкушая от стола смерти.
\vs Asn 13:5
Согрешила я, Господи, согрешила перед тобой,
совершила беззаконие: не знала я Господа, Бога небес, не
надеялась на Всевышнего живого Бога веков.
\vs Asn 13:6
Согрешила я, Господи, согрешила перед тобой,
совершила беззаконие: я надеялась на величие своей
славы, на красоту свою; была я горда и надменна.
\vs Asn 13:7
Согрешила я, Господи, согрешила перед тобой,
совершила беззаконие, презирая всех людей, из которых ни
одного не считала я человеком.
\vs Asn 13:8
Согрешила я, Господи, согрешила перед тобой,
совершила беззаконие; много раз говорила, что нет на земле
князя, достойного развязать девственный мой пояс.
\vs Asn 13:9
Согрешила я, Господи, согрешила перед тобой,
совершила беззаконие: я ненавидела всех желавших взять
меня в жёны, я презирала их и порицала.
\vs Asn 13:10
Согрешила я, Господи, согрешила перед тобой,
совершила беззаконие, но по твоему милосердию я
сделаюсь невестой сына великого государя~--- отпусти мне грехи мои.
\vs Asn 13:11
Когда пришёл Божий воин Иосиф,
он низложил мою гордость, он усмирил меня,
и уловил красотою своею,
и мудростью своею поймал меня, как рыбку в сети,
\vs Asn 13:12
душою своею предложил мне лекарство жизни,
силою своею утвердил меня и посвятил меня Богу веков.
\vs Asn 13:13
Он дал мне есть хлеб жизни и пить чашу бессмертия.
\vs Asn 13:14
Он напоил меня, и я стала его невестой вовеки.

\vs Asn 14:1
И когда Асенефь прекратила беседу с Господом,
вот, взошла денница на небеса от страны восточной.
\vs Asn 14:2
И увидела её Асенефь, и возрадовавшись, сказала:
Внял Бог молитве моей; ибо вот, светило,
предвестник великого дня, явилось.
\vs Asn 14:3
И вот видит Асенефь подле денницы разверзлось небо,
и показался свет неизреченный.
\vs Asn 14:4
Видя это, она пала лицом на пепел.
\vs Asn 14:5
И сошло с неба подобие мужа,
и стал он перед головой Асенефи, и стал звать её: Асенефь!
\vs Asn 14:6
И сказала она: Кто это звал меня?
Двери моей горницы заперты, башня моя высока,
кто это осмелился войти в мой чертог!
\vs Asn 14:7
И тот муж вторично позвал её, и сказал: Асенефь, Асенефь!
\vs Asn 14:8
И спросила Асенефь: Возвести мне, кто ты?
\vs Asn 14:9
---~Я князь Израилев и архистратиг
всего воинства Всевышнего.
Встань, становись на ноги, и я поведаю слово.
\vs Asn 14:10
И Асенефь, подняв голову, увидела мужа,
и вот, во всём подобен он Иосифу:
и одеждою, и венцом, и царским жезлом;
\vs Asn 14:11
лицо же его словно молния,
глаза как солнечные лучи,
волосы на голове как пламя,
а руки как раскалённое железо,
от рук и ног его сыпались искры,
как от пламенеющего огня.
\vs Asn 14:12
При виде этого Асенефь пала на лицо своё на землю,
и ужас объял её, и дрожь проникла до костей членов её.
\vs Asn 14:13
И тот муж сказал ей:
Мужайся, Асенефь, не бойся,
но встань на ноги свои, и я обращу к тебе мои слова.
\vs Asn 14:14
И сказал тот муж:
Пойди, сними с себя хитон чёрный,
выражение печали, и отложи вретище с чресл твоих,
и отряхни прах с головы твоей;
\vs Asn 14:15
и омой лицо своё живою водою,
и облекись в ризу новую, нетронутую,
и опояшь чресла твои двойным
золотым поясом девства твоего,
и приди, и тогда я скажу тебе слово.

\vs Asn 14:16
И Асенефь поспешно
удалилась во 2-ую свою комнату,
где были корзины с её убранствами.
\vs Asn 14:17
И открыла она ковчежец свой,
и вынула полотняное дорогое платье,
до которого никто ещё не касался,
и сняла с себя чёрное траурное платье
и надела платье новое.
\vs Asn 14:18
И сняла она вервие и вретище с чресл своих,
и опоясала 2-мя поясами своего девства:
одним стан свой, а другим грудь свою с сосцами.
\vs Asn 14:19
И отряхнула прах с головы своей,
и умыла руки и лицо своё водою чистою;
и взяла она чистый полотняный покров
и покрыла им свою голову.

\vs Asn 15:1
И пришла она к мужу тому в 1-ую комнату,
и муж тот, увидев её, сказал:
\vs Asn 15:2
Сними с головы своей покрывало,
зачем ты надела его сегодня?
ибо ты до сей поры чистая и скромная дева,
и голова твоя как голова юноши.
\vs Asn 15:3
И сняла Асенефь покрывало с головы своей.
И сказал ей тот муж:
\vs Asn 15:4
Мужайся, чистая дева Асенефь!
Вот я внял исповеди твоей и молитвам,
вот я увидел 7-дневные тяжкие твои лишения;
\vs Asn 15:5
теперь я своими глазами вижу тебя,
стоящую на пепле и проливающую слёзы:
мужайся, чистая дева Асенефь!
\vs Asn 15:6
Ибо имя твоё вот уже написано
на небесах Божьими перстами
в книги живых с именами изначала вписанных
и пребудет оно там неизгладимо во веки веков.
\vs Asn 15:7
Отныне ты обновишься:
ты вкусишь от животворящего,
благословенного хлеба
и пить будешь от чаши благословения
и бессмертия;
и умастишь себя чистым елеем.
\vs Asn 15:8
Мужайся, чистая дева Асенефь!
Вот я отдаю ныне тебя Иосифу в невесты навсегда,
и отныне имя тебе будет не Асенефь, но Город убежища;
\vs Asn 15:9
ибо через тебя многие народы прибегнут к Господу,
Богу небес, и под сенью крыл твоих укроются
уповающие на Господа Бога;
\vs Asn 15:10
и за стенами твоими будут защищены,
притекающие к Всевышнему через покаяние,
ибо покаяние есть дщерь Всевышнего,
и она предстательствует на всякий час
пред Всевышним за тебя и за всех кающихся.
\vs Asn 15:11
Ибо он есть Отец покаяния,
она же есть матерь дев, и на всякий час молит его о кающихся;
\vs Asn 15:12
ибо возлюбленных своих
вознесет Бог, который есть податель даров, подкрепитель всех дев.
\vs Asn 15:13
Ему угодно девство, он ищет его,
он заботится о нём всегда.
\vs Asn 15:14
Кающихся он принимает под сень свою,
готовит для них на небесах место покоя, и они навеки будут под
покровом его.
\vs Asn 15:15
И есть покаяние весьма прекрасная дева,
чистая, и непорочная, и кроткая, и Бог Всевышний любит её, и
все ангелы почитают её.
\vs Asn 15:16
И вот, я иду к Иосифу,
и буду говорить с ним о тебе, и он войдёт к тебе сегодня, и увидит тебя, и
возлюбит тебя, и будет женихом твоим, ты же будешь ему невестой.
\vs Asn 15:17
Итак, внимай, о ты, дева Асенефь!
Облекись в брачную ризу, в ризу древнюю, приготовленную для тебя из
начала в покое твоем;
\vs Asn 15:18
и надень на себя всякий убор твой избранный,
и укрась себя нарядами, как добрую невесту, и ты пойдёшь навстречу Иосифу.
\vs Asn 15:19
Ибо вот, он сегодня приблизится к тебе,
и увидит тебя, и возрадуется.
\vs Asn 15:20
И когда тот муж кончил речь свою,
Асенефь возрадовалась радостью великою,
и пала она на лицо своё,
поклонилась ему и сказала:
\vs Asn 15:21
Благословен Бог Всевышний,
пославший тебя избавить меня от мрака и возвести к свету, и
извлёкший меня из глубины бездны.
\vs Asn 15:22
Скажи имя твое, господин!
Поведай мне, дабы я могла благословлять его навеки.
\vs Asn 15:23
Отвечает ей муж тот:
Имя моё написано на небесах в книге
Всевышнего Божьими перстами прежде,
чем были написаны все имена.
\vs Asn 15:24
Я Князь Всевышнего,
и имена, вписанные в книгу Всевышнего,
не подлежат ни исследованию, ни слуху, ни
зрению человека в этом мире.
\vs Asn 15:25
Асенефь сказала: Если я
обрела благодать пред тобою и разумею все слова, сказанные мне тобою, то
позволь рабе твоей сказать пред тобою слово.
\vs Asn 15:26
И сказал тот муж: Говори.
\vs Asn 15:27
И говорит Асенефь: Прошу тебя, господин.
\vs Asn 15:28
Произнося эти слова, она приблизилась к его руке
и с такой мольбой: Сядь на малое время на этом ложе:
оно чисто и не осквернено, ибо на нём не сидели ещё муж и жена.
\vs Asn 15:29
Я поставлю пред тобою стол
и принесу из моей кладовой хлеб, и вкусишь ты от него;
\vs Asn 15:30
и старое вино, благовоние
которого до небес, и будешь ты пить от него,
и отправишься в путь свой.
\vs Asn 16:1
И сказал ей муж тот: Иди и принеси скорее.
\vs Asn 16:2
И Асенефь поспешно принесла
и поставила перед ним пустой стол,
и выйдя от него, она готова была уже войти в
кладовую за хлебом,
\vs Asn 16:3
когда он сказал: Принеси мне мёд сотовый.
\vs Asn 16:4
Асенефь остановилась, и она опечалилась,
потому что не было у неё сотового мёда в кладовой.
\vs Asn 16:5
И спросил её тот муж: Что же ты остановилась?
\vs Asn 16:6
И отвечала Асенефь: Пошлю отрока за город,
недалеко отсюда поле наследия нашего, он тотчас принесет оттуда
медовые соты, и я поставлю их перед тобою, господин.
\vs Asn 16:7
И сказал ей тот муж: Войди в свою кладовую
и ты найдешь медовые соты на столе, возьми их и принеси.
\vs Asn 16:8
И взошла Асенефь в свою кладовую и нашла на столе соты.
\vs Asn 16:9
И были ячейки тех сотов большие,
и белые как снег, и полные мёдом.
И ячейки эти подобны были небесной
росе, и запах от них благоухание жизни.
\vs Asn 16:10
Асенефь изумилась и подумала:
уж не из уст ли этого мужа вышли эти соты,
так как запах от них, как от этого мужа?
\vs Asn 16:11
И взяла Асенефь медовые соты,
принесла и поставила их на пустой стол перед тем мужем.
\vs Asn 16:12
И спросил тот муж: Как же это ты сказала,
что нет у меня в кладовой медовых сотов, а между тем,
вот, принесены оттуда эти соты?
\vs Asn 16:13
И, смутившись, сказала Асенефь:
У меня, господин, не было сотов медовых в кладовой; но в то время,
как ты заговорил, быть может, из уст твоих они изошли;
ибо запах от них как запах от уст твоих.
\vs Asn 16:14
И улыбнулся тот муж, видя разумность Асенефи.
\vs Asn 16:15
И подозвав её к себе, он простёр правую руку свою к голове её.
\vs Asn 16:16
И Асенефь испугалась руки того мужа,
ибо искры сыпались из уст его, как от раскалённого железа,
\vs Asn 16:17
она, устремив глаза,
смотрела на его руку, а он при виде этого, улыбнувшись, сказал:
\vs Asn 16:18
Блаженна ты, Асенефь;
ибо неизречённые тайны Всевышнего Бога открылись тебе!
Блаженны и те, кои предстанут пред Господа с покаянием;
\vs Asn 16:19
они вкусят от медовых этих
сотов, дающих жизнь; ибо их приготовили пчёлы рая места сладости;
\vs Asn 16:20
они приготовили их из росы
живых райских роз, и ангелы Божьи вкушают от него, и все сыны Всевышнего, и
вкусивший от этих сотов не умрёт вовеки.
\vs Asn 16:21
И простёр тот муж правую руку,
отломил частицу от сотов,
и вкусил сам, и частицу рукою своею вложил ей в уста,
\vs Asn 16:22
говоря: Вот ты, Асенефь,
вкусила хлеб жизни, и пила чашу бессмертия,
и умастилась елеем непорочности.
\vs Asn 16:23
Отныне тело твоё
распускаться будет подобно цветку, выросшему на земле Всевышнего; кости твои
утучнятся подобно кедрам, растущим в раю сладости,
\vs Asn 16:24
и сила проникнет всю тебя,
и молодость твоя не увидит старости, и красота не покинет тебя вовеки,
\vs Asn 16:25
и будешь ты как город,
окружённый бойницами во имя Господа Бога, царя веков.
\vs Asn 16:26
И простёр тот муж руку к
отломанной части сотов, и соты сделались целы, как прежде.
\vs Asn 16:27
И он снова протянул правую
руку свою и перстом коснулся края сотов,
обращенного на восток, и обратил его на сторону,
выходящую на запад; и путь перста его принял кровавый вид.
\vs Asn 16:28
И он, простерши руку в другой раз,
коснулся ею края сотов, обращенного на север,
и обратил его на сторону, выходящую на юг,
и путь перста его имел вид кровавый.
\vs Asn 16:29
И стояла Асенефь слева, и
смотрела и видела всё, что делал он.
\vs Asn 16:30
И сказал тот муж медовым сотам: Приблизьтесь сюда.
\vs Asn 16:31
И вот из твердых сотовых
ячеек поднялись тысячи и тьмы белоснежных
пчёл с длинными пурпуровыми крыльями,
\vs Asn 16:32
у иных же крылья были как
виссон, унизанный дорогими камнями,
и как гиацинт, и как нити златые; их головки
были украшены золотыми венцами.
\vs Asn 16:33
Пчёлы эти были прекрасны на
вид и жала их были изострены.
\vs Asn 16:34
И покрыли Асенефь все пчёлы
роем с головы до ног; и иные пчёлы,
большие, как бы царицы роя, присели к устам Асенефь.
\vs Asn 16:35
Поднялись они из ячеек
своих, облепили всё лицо Асенефи
и стали работать на её лице, и отверстия ячеек
приходились к устам Асенефи.
\vs Asn 16:36
И тот муж сказал пчёлам:
Ступайте по своим местам.
\vs Asn 16:37
И поднялись все пчёлы и
улетели по направлению к небу.
\vs Asn 16:38
И те из них, которые жалить
хотели Асенефь, падали мёртвые.
\vs Asn 16:39
И тот муж, жезлом своим
прикоснувшись к мёртвым пчелам, сказал:
И вы восстаньте и ступайте на свои места.
\vs Asn 16:40
И встрепенулись они, и
полетели перед домом Асенефи,
и уселись на плодовых деревьях.
\vs Asn 17:1
И сказал тот муж Асенефи:
Видела ли слово сие?
\vs Asn 17:2
И отвечала она: Так, господин: всё сие я видела.
\vs Asn 17:3
И он сказал: Таковы будут слова мои ныне.
\vs Asn 17:4
И он в 3-ий раз простёр правую руку свою
к части медовых сотов и съел её, не повредив.
\vs Asn 17:5
И поднялся огонь от стола, и пожрал соты,
и запах от сжигаемых сотов наполнил собою горницу,
и запах был весьма приятен.
\vs Asn 17:6
И сказала Асенефь мужу:
Есть у меня 7 девиц однолеток, служащих мне, воспитанных со мною от
младенчества моего, родившихся в одну ночь со мною: я их люблю как сестёр;
позову их сюда, чтобы ты благословил их, как ты благословил меня.
\vs Asn 17:7
И сказал муж: Зови.
И, будучи позваны, они стали перед ним.
\vs Asn 17:8
И тот муж сказал:
Да благословит вас Бог Всевышний;
да будете вы 7-ью столпами для этого города,
и пусть почиет на вас Господне благословение вовеки.
\vs Asn 17:9
И сказал он Асенефи:
Переставь отсюда этот стол.
\vs Asn 17:10
И когда обратилась Асенефь,
чтобы поставить стол на его прежнее место,
муж тот сделался невидим.
\vs Asn 17:11
И увидела Асенефь подобие
колесницы, несущейся на восток; и колесница подобна была огню, кони её как
молнии, и на колеснице стоял муж тот.
\vs Asn 17:12
И сказала Асенефь:
Я, неразумная и дерзкая, позволила себе сказать смело,
что человек пришёл в мою горницу, не ведая,
что пришедший сегодня ко мне был
Господь небесный, который вот возвращается на своё место,
\vs Asn 17:13
и она присовокупила:
Помилуй, Господи, и сжалься надо мною рабою твоею, ибо в неведении я говорила
перед лицом твоим дерзновенно и неразумно.

\vs Asn 18:1
И между тем как Асенефь
погружена была в эти размышления, прибежал отрок из числа рабов Потифера и
сказал:
\vs Asn 18:2
Вот Иосиф, бог сильный,
едет к нам: его колесница уже перед нашим двором.
\vs Asn 18:3
Асенефь поспешно позвала
кормилицу свою, заведовавшую всем её имуществом, и сказала:
\vs Asn 18:4
Пойди, займись поскорее
убранством нашего дома и приготовь лучшую вечерю для бога сильного Иосифа, ныне
едущего к нам.
\vs Asn 18:5
Тут кормилица заметила, что
у Асенефи ланиты впали по причине 7-дневного воздержания от пищи, ей грустно
стало, и она заплакала, и, взяв её за правую руку, поцеловала,
\vs Asn 18:6
и спросила: Что с тобой, дитя?
Отчего такие впалые у тебя ланиты?
\vs Asn 18:7
Отвечала Асенефь:
Сильная головная боль посетила меня,
ночь провела без сна, вот отчего изменилась в лице.
\vs Asn 18:8
И пошла её кормилица убирать дом и готовить вечерю;
Асенефь же вспомнила слова того мужа, и поспешила
во 2-ую свою комнату, где в хранилищах лежали её наряды.
\vs Asn 18:9
И открыв большой ковчежец,
она вынула из него брачные свои одежды, превосходные, блестящие, и надела их.
\vs Asn 18:10
И опоясалась она золотым царским поясом,
украшенным различными камнями многоценными;
\vs Asn 18:11
и надела на руки и ноги
золотые запястья, на шею дорогие ожерелья, унизанные бесчисленными дорогими
каменьями;
\vs Asn 18:12
и надела на голову золотой
венец, унизанный с обеих сторон у чела 12-ью большими камнями;
\vs Asn 18:13
и набросила на голову
лёгкое покрывало, как подобает невесте; и взяла царский жезл в руку.
\vs Asn 18:14
И вспомнила Асенефь слова
своей кормилицы, что печально выражение лица твоего, воздохнула и с грустью
сказала: лицо моё горит, если Иосиф заметит это, ему не понравится.
\vs Asn 18:15
И, обратившись к подругам
своим, сказала: Принесите мне чистой, ключевой воды, умою лицо своё.
Принесли и налили воды в рукомойницу.
\vs Asn 18:16
И она наклонилась и
увидела в воде лицо своё, подобное солнцу, глаза свои как восходящую утреннюю
звезду, прелестные ланиты свои как части граната, уста свои как
распустившуюся розу и зубы, блестящие белизной.
\vs Asn 18:17
И Асенефь, созерцая себя в
воде в таком виде, возрадовалась радостью великой и стала умывать лицо своё.
\vs Asn 18:18
И когда пришла кормилица с
донесением об исполнении данных ей приказаний, взглянув на Асенефь, удивилась
так, что не могла опомниться в продолжение 2-ух часов:
так велико было её изумление!
\vs Asn 18:19
Она, став на колени, спросила:
Откуда эта великая, дивная красота, госпожа моя?
Вижу сам Господь, Бог небесный, избрал тебя быть невестой Иосифа.

\vs Asn 19:1
Между тем как они
беседовали, пришёл отрок из рабов и сказал Асенефи: Вот, Иосиф стоит у врат
двора нашего.
\vs Asn 19:2
Асенефь поспешно спустилась
по лестнице в сопровождении 7-ми дев навстречу Иосифу и стала в проходе дома.
\vs Asn 19:3
Иосиф вступил на двор;
ворота затворились, и чужой народ остался за воротами.
\vs Asn 19:4
Тогда Асенефь вышла из
прохода навстречу Иосифу.
\vs Asn 19:5
И увидев её, Иосиф был
поражен великой её красотой и спросил: Скажи, кто ты?
\vs Asn 19:6
И она ответила: Я раба
твоя, Асенефь, которая по повелению твоему выбросила всех своих идолов и
уничтожила.
\vs Asn 19:7
Сегодня приходил ко мне
некий муж, который дал мне хлеб жизни и вино благословения, сказав: вот Я отдаю
тебя как вечную невесту Иосифу, который будет твоим женихом навсегда;
\vs Asn 19:8
к этому он прибавил: отныне
ты будешь называться не Асенефь, а Городом прибежища; ибо через тебя
прибегнут многие народы к Богу Всевышнему.
\vs Asn 19:9
Он прибавил: я пойду к
Иосифу и поведаю в слух его слова мои о тебе.
\vs Asn 19:10
Ты знаешь уже, господин
мой; ибо тот муж приходил к тебе и говорил обо мне.
\vs Asn 19:11
И сказал Иосиф Асенефи:
Всевышний Бог благословил тебя; ибо Господь Бог утвердил стены твои на высоте,
стены же твои адамантовые, они стены жизни;
\vs Asn 19:12
ибо многие сыны
человеческие жить будут в твоем Городе прибежища, и Господь Бог воцарится в
нём вовеки.
\vs Asn 19:13
Итак, приди ко мне, дева
чистая; зачем так далеко стоишь от меня! Ибо благую весть о тебе принёс мне от
небес муж, поведавший мне всё, что было с тобою.
\vs Asn 19:14
И он, подняв руку, подозвал
Асенефь. И она подошла к Иосифу и пала ему в объятия.
\vs Asn 19:15
И оживились души у них и
исполнились радостью; и Иосиф, дав Асенефи лобзание, сообщил ей дух жизни, дух
премудрости, и дух истины.
\vs Asn 19:16
И, обнявшись, они долго
лобызали друг друга.
\vs Asn 20:1
Наконец, Асенефь сказала:
Пойди сюда, господин мой, взойди в наш дом; ибо я убрала наш дом и приготовила
великолепную вечерю.
\vs Asn 20:2
Она взяла его за руку правую
и ввела в дом свой, посадила на седалище отца своего и принесла воды, чтобы
омыть ноги его.
\vs Asn 20:3
И Иосиф сказал ей: Пусть
придёт одна из дев и умоет мои ноги.
\vs Asn 20:4
И отвечала Асенефь: Нет,
господин мой, отныне я раба твоя. С чего ты взял, что другая будет умывать
ноги твои? Ноги твои мои ноги, тело твоё моё тело.
\vs Asn 20:5
И она настояла на своём и
омыла ему ноги.
\vs Asn 20:6
И посмотрел Иосиф на её руки
и не мог наглядеться на их жизненность: пальцы у неё ходили, как у скорописца.
\vs Asn 20:7
Затем Асенефь, взяв его за
правую руку, облобызала её; а он поцеловал её в голову. И она села по правую
руку его.
\vs Asn 20:8
И пришли отец её и мать с
поля наследия своего, пришли и все сродники её и увидели Асенефь, как бы
окружённую светом (красота её была словно небесная), сидящую с Иосифом и
одетую в ризу брачную.
\vs Asn 20:9
При виде этого они
ужаснулись, поражённые её красотой, и они воздали славу Богу, который
животворит все.
\vs Asn 20:10
После этого они ели, и пили, и веселились.
\vs Asn 20:11
И сказал Потифер Иосифу:
Завтра ты пригласишь сановников и вельмож египетских, и я устрою свадьбу
вашу, и ты возьмёшь дочь мою Асенефь себе в жёны.
\vs Asn 20:12
И ответил Иосиф: Нет,
прежде я отправлюсь к фараону; ибо он для меня как отец, он меня поставил
князем над этой страной. Поведаю его слуху об Асенефи,
и он отдаст мне Асенефь в жёны.
\vs Asn 20:13
На это Потифер ответил:
Ступай с миром.
\vs Asn 20:14
Иосиф остался тот день у
Потифера и не вошёл к Асенефи;
\vs Asn 20:15
ибо говорил: Не подобает
мужу богобоязненному прежде брака почивать с женою своею.

\vs Asn 21:1
На утро Иосиф отправился к
фараону и сказал ему:
Отдай мне Асенефь, дочь Потифера,
илиопольского жреца, в жёны.
\vs Asn 21:2
Фараон сказал: Ведь она
твоя невеста и с давних пор обручена.
\vs Asn 21:3
И он послал вестников за
Потифером, который пришёл с Асенефью, и представил её перед фараона.
\vs Asn 21:4
И изумился фараон при виде
красоты её и сказал:
Да благословит тебя, дитя моё,
Бог Иосифа, избравшего тебя в невесту себе!
Да не покинет тебя красота твоя!
\vs Asn 21:5
Справедлив Господь,
избравший тебя для Иосифа, и как говорится, отныне наречёшься ты дочерью
Всевышнего, и будет тебе Иосиф женихом навеки.
\vs Asn 21:6
И фараон возложил на Иосифа
и Асенефь золотые венцы, взятые из царской сокровищницы.
\vs Asn 21:7
И фараон, поставив Асенефь
по правую руку Иосифа, возложил на их головы свои руки, правую руку на голову
Асенефь,
\vs Asn 21:8
и сказал: Да благословит
вас Всевышний Бог, и да прославит на вечные времена.
\vs Asn 21:9
И фараон повернул их лицом к
лицу, подвинул их близко и принудил лобызаться.
\vs Asn 21:10
После сего фараон сотворил брак,
устроил роскошную вечерю и пир, и винопитие великое в продолжение 7-ми
дней; и были приглашены все князья египетские, вельможи,
все цари соседних народов.
\vs Asn 21:11
И приказал царь возвестить
по всей земле Египетской, что если кто в продолжение 7-ми дней бракосочетания
Иосифа и Асенефи будет работать, тот умрёт горькою смертью.
\vs Asn 21:12
И было после сего, когда
совершилось торжество брачное и закончилось пиршество, Иосиф вошёл к Асенефи,
она зачала и родила Манассию в доме Иосифа.

\vs Asn 22:1
После этого прошло 7 лет изобилия, и настало 7 лет голода.
\vs Asn 22:2
И услышал Иаков о сыне своём Иосифе,
и прибыл в Египет со всеми сродниками своими на 2-ом году голода, в
21-ый день месяца нисана, и поселился в земле Гесем.
\vs Asn 22:3
И сказала Асенефь Иосифу:
Пойду я посетить отца твоего, ибо отец твой Израиль для меня как Бог.
\vs Asn 22:4
И сказал Иосиф: Ты пойдёшь со мною и увидишь отца моего.
\vs Asn 22:5
И пришли Иосиф и Асенефь в
страну Гесем; и встретились ему братья его,
и пали на лицо своё и поклонились
ему, в особенности же Асенефи.
\vs Asn 22:6
И вошли они к Иакову,
который сидел на ложе своём: и был он сед и очень стар.
\vs Asn 22:7
И Асенефь сильно поразил вид
его; ибо Иаков, несмотря на седину, был благовиден, как прекрасный юноша;
\vs Asn 22:8
глава его была бела как
снег, кудрявые волосы его были густы, как у хушитянина;
белая красивая борода его покрывала всю грудь его;
\vs Asn 22:9
глаза его веселы,
блестящие и красивые; грудь его, и плечи, и мышцы, и пальцы на руках как у
сильного ангела; бёдра и голени его как у нефилима.
\vs Asn 22:10
И представлялся Иаков как
муж боговидный.
\vs Asn 22:11
Асенефь, видя его, пришла в
ужас и пала пред ним на землю на лицо своё.
\vs Asn 22:12
И спросил Иаков: Эта ли
моя невестка, жена твоя? Да благословит её Всевышний Бог!
\vs Asn 22:13
И, подозвав её к себе,
облобызал и благословил её; также и Асенефь простёрла руки свои, и обвила ими
шею Иакова и повисла на раменах отца мужа своего, как бы возвратившегося с войны
целым и невредимым, и лобызала его.
\vs Asn 22:14
После того ели они и пили.
\vs Asn 22:15
И отправились Иосиф и
Асенефь в дом свой, и взяли с собою Симеона и Левия.
\vs Asn 22:16
Но не все сыновья Зелфы и
Валлы, Лии и Рахили провожали их, потому что завидовали им и были их врагами.
\vs Asn 22:17
С правой стороны Асенефи
шел Левий, а с левой Симеон. И Асенефь держала за руку Левия,
\vs Asn 22:18
ибо более всех братьев
Иосифа Асенефь возлюбила Левия, как мужа пророчествующего, и благочестивого, и
богобоязненного.
\vs Asn 22:19
Потому что он читал
письмена, написанные на небесах, и разумел их, и знал тайны Всевышнего, которые
он открывал ей в словах таинственных.
\vs Asn 22:20
И Левий сильно любил
Асенефь. Он видел место её упокоения в вышних, и окружавшие его стены были
словно адамантовые, и основания его как каменные основы 3-го неба.
\vs Asn 23:1
И было, когда возвращались
Иосиф и Асенефь путем своим, первородный сын фараона увидел её с высоты стены и
сильно возмутился духом при виде великой красоты её.
\vs Asn 23:2
И сказал он: Нет, не быть этому!
\vs Asn 23:3
И сын фараона отправил
гонцов призвать к себе Симеона и Левия, которые, пришедши, предстали
пред лицом его.
\vs Asn 23:4
И сказал им первородный сын
фараона: Я знаю, что вы силою превосходите всех людей на земле: вашею десницею
был сокрушён город Сихем, и 2-мя мечами вашими были истреблены 30000
мужей воинственных.
\vs Asn 23:5
И вот призываю я вас,
придите на помощь мне! Вот я приму вас в товарищи себе: дам я вам много золота
и серебра, и рабов, и рабынь, и дома, и большие уделы, и богатства;
только помогите мне, исполните это моё слово.
\vs Asn 23:6
Окажите мне милость, ибо я
поруган братом вашим Иосифом: он отнял у меня Асенефь, которая первоначально
была обручена со мною.
\vs Asn 23:7
И ныне будьте со мною,
воздвигните брань на брата вашего Иосифа, тогда я убью его мечом своим и возьму
Асенефь себе в жену;
\vs Asn 23:8
этим вы докажете верность
вашу и будете мне братьями и друзьями даже до конца, исполните только это моё
слово немедленно.
\vs Asn 23:9
Но если, выслушав моё
предложение, вы пренебрежёте им~--- знайте, что вас ожидает этот меч.
\vs Asn 23:10
Говоря это, обнажил он меч свой и показал им.
\vs Asn 23:11
Услышав такую надменную
речь из уст сына фараонова, Симеон и Левий пришли в негодование.
\vs Asn 23:12
Симеон же был муж смелый и
решительный; он готов был схватиться за рукоятку своего меча, вынуть его из
ножен и поразить им сына фараонова за оскорбительные слова;
\vs Asn 23:13
но Левий провидел его
намерение и, как муж, одарённый даром пророчества, духовным оком прозрел, что
было изображено у него на сердце, и ногою своею наступил ему на правую ногу и
тем дал ему знак, чтобы тот укротил гнев свой.
\vs Asn 23:14
И сказал Левий сыну
фараона, и сказал смело, и без гнева, и с сердцем кротким и ликом ясным:
\vs Asn 23:15
Зачем ты, господин наш,
произносишь такие речи? Мы мужи богобоязненные, отец наш раб Бога
Всевышнего, и брат наш Иосиф возлюблен Богом;
\vs Asn 23:16
как же мы можем творить
злое дело и грешить перед Богом, перед отцом нашим Иаковом и братом нашим
Иосифом?
\vs Asn 23:17
Итак, слушай: мужу
богобоязненному ни в каком случае не подобает творить беззаконие потому только,
что имеет он в руках меч, поэтому воздержись говорить недоброе о нашем брате
Иосифе;
\vs Asn 23:18
ибо если ты будешь
упорствовать в злом твоём намерении, то вот готовы обнаженные мечи наши в
правых руках наших перед лицом твоим.
\vs Asn 23:19
С этими словами Симеон и
Левий извлекли мечи свои из ножен, и сказали: Видишь ли ты в наших руках эти
мечи?
\vs Asn 23:20
посредством этих мечей
отомстил Господь сихемлянам за оскорбление, нанесённое сынам израилевым в лице
сестры нашей Дины, которую обесчестил Сихем, сын Еммора.
\vs Asn 23:21
При виде меча обоюдоострого
сын фараонов испугался, и задрожали кости его; ибо мечи те блистали, как пламя
огня.
\vs Asn 23:22
Потемнело в глазах у сына
фараонова, и упал он на лицо своё под ноги их.
\vs Asn 23:23
Тогда Левий, протянув руки,
ухватил его и сказал: Встань, не бойся; смотри за собой; впредь остерегайся
говорить злое слово о брате нашем.
\vs Asn 23:24
И сказав это, вышли от сына
фараонова Симеон и Левий. Ужас и печаль овладели сыном фараона; ибо он боялся
Симеона.

\vs Asn 24:1
Тяжело было ему от любви к Асенефи; великая, безмерная тоска
напала на него.
\vs Asn 24:2
Тогда рабы его стали нашептывать ему, говоря: Вот, сыны Валлы
и сыны Зелфы, рабынь Лии и Рахили, жен иаковлевых, враги Иосифу и Асенефи,
которым они завидуют: они-то будут тебе покорны и исполнят твою волю.
\vs Asn 24:3
И сын фараонов послал вестников призвать их к себе. И пришли
они ночью и стали перед ним.
\vs Asn 24:4
И сказал им сын фараонов: Речь свою обращаю к вам, как к
мужам сильным.
\vs Asn 24:5
И говорят ему старшие братья Дан и Гад: Пусть говорит теперь
господин наш, что хочет, и мы, рабы твои, услышим и исполним волю твою.
\vs Asn 24:6
И сын фараонов возрадовался радостью великой и сказал своим
рабам: Отойдите от меня немного, ибо этим мужам я имею сообщить нечто втайне,
и все они отошли.
\vs Asn 24:7
И сказал сын фараонов Дану и Гаду: Перед лицом вашим
благословение и смерть; выбирайте скорее благословение, нежели смерть; вы
мужи сильные, вы не должны умереть как женщина, мужайтесь и отмстите врагам
вашим.
\vs Asn 24:8
Ибо я сам слышал, как брат ваш Иосиф говорил о вас отцу моему
фараону, что вы сыновья рабыни его матери, а не братья его: не дождусь смерти
моего отца, чтобы их истребить вместе с их родом, дабы они, дети служанки, не
могли участвовать в наследии.
\vs Asn 24:9
Это они продали меня измаильтянам, и я отплачу им за зло, мне
причинённое: пусть только умрёт отец мой.
\vs Asn 24:10
И похвалил его отец мой, фараон, говоря: Ты хорошо сказал,
возьми у меня 1000 человек воинов, и выйди на них тайно, и сотвори им, как
сотворили они тебе, и я буду твоим помощником.
\vs Asn 24:11
Когда же те мужи услышали
слова сына фараонова, возмутились духом, и опечалились, и сказали сыну
фараонову: Просим тебя, господин, помоги нам.
\vs Asn 24:12
И тот в ответ: Я помогу
вам, если вы послушаетесь меня.
\vs Asn 24:13
И они сказали: Стоим перед
тобою мы, рабы твои; прикажи, и мы исполним твою волю.
\vs Asn 24:14
И говорит им фараонов сын:
Нынешнею ночью я убью отца моего, потому что фараон стал отцом Иосифа;
\vs Asn 24:15
вы же убейте Иосифа, и я
возьму себе в жёны Асенефь, а вы и братья ваши будете моими сонаследниками
исполните только моё слово.
\vs Asn 24:16
И сказали ему Дан и Гад:
мы рабы твои: сегодня же исполним твоё приказание;
\vs Asn 24:17
ибо ныне мы слышали, как
Иосиф говорил Асенефи: пойди завтра в поле наследия нашего, ибо настало время
сбора винограда.
\vs Asn 24:18
600 сильных воинов
будут сопровождать её и 60 ей предшествовать. Теперь послушай, что мы
тебе скажем. И открыли они ему свои мысли.
\vs Asn 24:19
И сын фараонов дал каждому
из 4-ёх мужей по 500 воинов, назначив их князьями и начальниками.
\vs Asn 24:20
И говорят ему Дан и Гад:
Нынешнею ночью мы пойдем и сядем в засаде у потока в зарослях тростника;
\vs Asn 24:21
ты же возьми с собой 50 стрелков всадников и поезжай вперед.
\vs Asn 24:22
Как только покажется
Асенефь, мы предадим мечу воинов, сопровождающих её, тогда бросится она на
колеснице вперёд, и попадёт тебе в руки Асенефь, и сотворишь с ней, как хочет
душа твоя.
\vs Asn 24:23
После этого мы умертвим
Иосифа, погружённого в печаль, и сыновей его пред глазами его.
\vs Asn 24:24
И обрадовался сын фараонов,
услышав слова сии, и дал им 2000 воинов.
\vs Asn 24:25
И пришли они к потоку, и
укрылись в зарослях тростника, и 500 засели впереди, и заняли широкую
переправу с той и с другой стороны потока.

\vs Asn 25:1
И сын фараонов встал и отправился в ту ночь в дом отца своего.
\vs Asn 25:2
И пришёл сын фараонов к ложу
отца своего, чтобы убить его мечом; но телохранители не позволили ему доступ к
отцу его и спросили: Какая тебе надобность, господин?
\vs Asn 25:3
И отвечал сын фараона: Хочу
видеть отца моего, так как отправляюсь на сбор винограда в новый виноградник.
\vs Asn 25:4
И сказали ему телохранители:
Страданием страдает отец твой, всю ночь не мог заснуть, и ныне немного
успокоился. Приказал он никого не впускать.
\vs Asn 25:5
И отошёл он в ярости, и
отправился сын фараонов к своим воинам и при рассвете засел в засаде, как
посоветовали ему Дан и Гад.
\vs Asn 25:6
Услышав об этом, сыновья
Иакова, Неффалим и Асир, младшие братья, сказали Дану и Гаду: За что вы
задумываете ещё злые козни против нашего отца, Израиля, и брата нашего, Иосифа?
\vs Asn 25:7
Разве Господь не бережёт его
как зеницу ока? Не вы ли некогда продали его?
\vs Asn 25:8
А ныне он царствует над
страной, он раздаёт по доброй воле пшеницу, которой питается народ, он спасает
многим жизнь.
\vs Asn 25:9
Если вы сегодня попытаетесь
причинить ему зло, то умолит он Бога Израилева и появится на небе и настигнет
вас огонь, который пожрёт вас, и ангелы Божьи будут сражаться за него и явятся к
нему на помощь.
\vs Asn 25:10
Дан и Гад разгневались на
своих братьев и сказали им: В противном случае мы умрём как женщины!
Да не будет того.
\vs Asn 25:11
И вышли навстречу Иосифу и Асенефь.

\vs Asn 26:1
На рассвете Асенефь встала и
сказала Иосифу: Как ты сказал, я отправлюсь в поле наследия нашего для сбора
винограда; но я боюсь, как бы кто-нибудь, придя, не похитил бы меня у тебя.
\vs Asn 26:2
И сказал ей Иосиф: Мужайся,
не бойся ничего, но спеши идти; Господь будет с тобою, и он убережёт тебя как
зеницу ока и сохранит тебя от всякого зла.
\vs Asn 26:3
Я же пойду на труд свой и
раздавать буду хлебные припасы в городе, чтобы кормить народ и принять меры,
дабы от голода никто не погиб в стране.
\vs Asn 26:4
И отошла Асенефь своим путем;
и Иосиф отошёл на труд свой и раздавал хлеб.
\vs Asn 26:5
И приблизилась Асенефь в
сопровождении 600 воинов к месту, где была ложбина.
\vs Asn 26:6
И внезапно вышли из засады воины Дана и Гада
и напали на воинов Асенефи, и завязался бой с сильными Асенефи,
\vs Asn 26:7
и убили из них около 50 всадников,
ехавших впереди, а Асенефь бежала на своей колеснице.
\vs Asn 26:8
И Левий узнал всё сие, и
известил своих братьев, сынов Лии, об измене.
\vs Asn 26:9
И каждый из них обнажил меч
свой при бедре своём, и щит свой на плече своём, и взял копье в правую руку. И
побежали они поспешно вслед Асенефи.

\vs Asn 26:10
И Асенефь бежала, и вот, сын фараона, сопровождаемый
50-ью всадниками, навстречу ей.
\vs Asn 26:11
И увидела его Асенефь, и испугалась, и вострепетала. Тогда
призвала она имя Господа, Бога Всевышнего.
\vs Asn 27:1
И Вениамин был с нею в её колеснице; и был он отрок сильный,
красивый, богобоязненный и весьма храбрый.
\vs Asn 27:2
И он сошел с колесницы, и
набрал у потока полные руки гладких камней, и бросил их в сына фараонова, и
поразил его в левый висок, и причинил ему жестокую рану, и сын фараона упал с
коня своего и лежал на земле.
\vs Asn 27:3
И после того Вениамин
поднялся поспешно на высокую скалу и сказал вознице Асенефи: Достань мне
гладких камней из потока.
\vs Asn 27:4
И тот достал ему 48 гладких камней.
И бросил те камни Вениамин, и убил 48 мужей,
сопровождавших сына фараонова.

\vs Asn 27:5
И сыны Лии Рувим, Симеон,
Левий, Иуда, Иссахар и Завулон погнались за мужами, сидевшими в засаде в
кустах, и напали на них неожиданно; и шестеро их убили их всех.
\vs Asn 27:6
И братья их, Дан и Гад,
сыновья Валлы и Зелфы, убежали при виде их, говоря:
\vs Asn 27:7
Мы не устояли перед нашими
братьями, и сын фараона побеждён и ранен смертельно Вениамином,
и все, бывшие с ним, погибли от руки его.
Пойдём же, убьём Асенефь, и скроемся в зарослях тростника.
\vs Asn 27:8
И пришли они, держа в руке
мечи свои обнаженными и полными крови.
\vs Asn 27:9
И увидела их Асенефь и
сказала: Господь, Бог мой, ты, который спас меня от смерти и который сказал
мне: живи вовеки! избавь меня от меча этих нечестивых мужей.
\vs Asn 27:10
И внял Бог гласу её, и
тотчас мечи их выпали из рук их на землю и рассыпались в прах.
\vs Asn 28:1
Видя это, сыны Валлы и Зелфы
устрашились и сказали: Истинно Господь воюет против нас за Асенефь.
\vs Asn 28:2
И пали на лица свои на землю
и бросились к ногам Асенефи и сказали ей:
\vs Asn 28:3
Ты наша госпожа и царица;
мы согрешили пред тобою, и Бог воздал нам по делам нашим.
\vs Asn 28:4
Мы, рабы твои, умоляем тебя,
помилуй нас и спаси нас от рук братьев наших, ибо они идут отмстить нам и мечи
их изострены на нас.
\vs Asn 28:5
И сказала Асенефь:
Мужайтесь и не страшитесь братьев ваших, ибо они мужи богобоязненные; идите в
эти тростниковые заросли, пока не умолю я за вас и не усмирю гнева их; ибо
велика дерзость ваша против них.
\vs Asn 28:6
Мужайтесь и не бойтесь, и да
рассудит Господь между мною и вами! И убежали в заросли тростника Дан и Гад.
\vs Asn 28:7
И вот, прибежали сыновья
Левия, подобно стаду оленей, и сошла Асенефь с закрытой колесницы своей и
встретила их со слезами.
\vs Asn 28:8
Они же, пав на землю,
поклонились ей, и плакали громко, и искали братьев своих.
\vs Asn 28:9
И сказала Асенефь: Пощадите
братьев ваших и не делайте им зла, ибо Господь явился мне защитником против них
и сокрушил мечи их, и, как воск от огня, растаяли они на земле.
\vs Asn 28:10
И этого будет с них, ибо
Господь воюет против них, а вы пощадите их, ведь они братья ваши и кровь отца
вашего Израиля.
\vs Asn 28:11
И сказал ей Симеон: Зачем
госпожа наша говорит доброе о врагах наших? Нет, мы перебьём их мечами нашими,
так как они замышляли об отце нашем Израиле,
и о брате нашем Иосифе, уже 2-жды;
а ныне и на тебя
\vs Asn 28:12
И, простёрши руку свою,
Асенефь коснулась бороды его и поцеловала её, говоря: Ты никогда этого не
сделаешь, брат мой, и не отплатишь злом за зло ближнему своему; ибо Господь
судит обиду сию; а ведь они братья ваши и чада отца вашего и убежали от лица
вашего.
\vs Asn 28:13
И преклонился Симеон пред
Асенефь. И подошёл к ней Левий, и облобызал ей руки, и благословил её; и понял
он, что она желает спасти братьев его.
\vs Asn 28:14
И находились они в зарослях
тростника; и узнал он это от братьев их, но не дал знать им о том, ибо опасался,
как бы они в пылу не перебили их.
\vs Asn 29:1
И сын фараонов поднялся от
земли и сел, выплевывая кровь из уст своих, ибо кровь текла из виска его в уста
его.
\vs Asn 29:2
И подбежал к нему Вениамин,
и взял меч его, и вынул его из ножен (ибо не носил Вениамин меча при бедре
своём) и хотел убить его и поразить сына фараона в грудь.
\vs Asn 29:3
И подошел к нему Левий, и
взял его за руку, и сказал: Брат мой, не делай этого, ведь мы мужи
богобоязненные и не подобает мужу богобоязненному воздавать злом за зло, ни
попирать поверженного или добивать до смерти попавшего в руки врага.
\vs Asn 29:4
И теперь вложи свой меч в
ножны и помоги мне обвязать раны его, и если жив будет, сделается нашим другом;
как и фараон нам как отец.
\vs Asn 29:5
И поднял Левий сына
фараонова, и отер кровь с лица его, и наложил повязку на рану его, и принял его
на коня своего, и повёз его к отцу его, и рассказал ему обо всем случившемся.

\vs Asn 29:6
И поднялся фараон с престола своего и поклонился Левию.
\vs Asn 29:7
И на 3-ий день умер сын
фараонов от раны, причинённой камнем Вениаминовым.
\vs Asn 29:8
И оплакивал фараон сына
своего первородного, и впал от печали в недуг.
\vs Asn 29:9
И умер фараон, имея 109 лет от роду, и оставил диадему свою Иосифу.

\vs Asn 29:10
И царствовал Иосиф в земле
египетской 48 лет, а после того предал Иосиф диадему внуку фараона.
\vs Asn 29:11
И был Иосиф в земле
египетской, как отец его.
\vs Asn 29:12
И так хранил его Бог от
нежной юности даже до конца жизни его, ибо был он семенем избранных мужей
праведных, Авраама, Исаака и Иакова, и молитвы их шли перед ним;
\vs Asn 29:13
и солнце и звёзды
преклонились пред Иосифом, знаменуя, что быть ему царем.

\include{tex/Vis}
\bibbookdescr{2Ba}{
  inline={Вторая Книга Пророка Варуха\fns{Переведена с сирийского.}},
  toc={2-я Варуха},
  bookmark={2-я Варуха},
  header={2-я Варуха},
  abbr={2~Вар}
}
\vs 2Ba 1:1
В 25-й год Иехонии, царя Иуды, слово ЯХВЕ было обращено к Варуху, сыну Нерии, и ему было сказано:
\vs 2Ba 1:2
Ты видел всё, что сделал Мне этот народ. Зло, совершенное оставшимися двумя коленами, превосходит зло, совершенное десятью коленами, которые были уведены в плен.
\vs 2Ba 1:3
Ибо те первые колена были вовлечены в грех своими царями, эти же два увлекли и вынудили ко греху своих царей.
\vs 2Ba 1:4
Поэтому вот, Я посылаю злое на этот город и на живущих в нём, и он отнимется от Меня на некоторое время. Я разсею Мой народ среди народов, дабы он был посланием для них.
\vs 2Ba 1:5
И Мой народ будет наказан, и придет время, когда они пожалеют о бывшем благоденствии.

\vs 2Ba 2:1
Я сказал тебе эти слова, чтобы ты передал их Иеремии и всем, кто подобен вам, с тем, чтобы вы удалились из этого города.
\vs 2Ba 2:2
Ибо ваши дела для этого города словно прочный столп, и ваши молитвы словно укрепленная стена.

\vs 2Ba 3:1
И я сказал: ЯХВЕ, Бог мой, для того ли я пришел в этот мир, чтобы видеть грехи моей матери? Нет, ЯХВЕ!
\vs 2Ba 3:2
Если я обрел милость в Твоих глазах, отними сперва мой дух, дабы мне отойти к моим отцам и не быть при погибели моей матери.
\vs 2Ba 3:3
Ибо я весьма страшусь двух этих вещей: Тебе противиться не могу, но и видеть грехи моей матери также не вынесет душа моя.
\vs 2Ba 3:4
Одно только я скажу пред Тобою, ЯХВЕ:
\vs 2Ba 3:5
Что же будет после всего этого? Ибо если Ты разрушишь Твой город и предашь Твою землю ненавидящим нас, как еще останется память имени Израиля?
\vs 2Ba 3:6
Или как вознесётся слава Тебе? Кому будут объяснять то, что содержится в Твоём Законе?
\vs 2Ba 3:7
Или мир возвратится к своей изначальной природе и век вернется к молчанию, бывшему искони? Или множество душ будет унесено, и человеческая природа не будет более называться своим именем? И что станет со всем, что сказано Тобою о нас Мойсею?

\vs 2Ba 4:1
И ЯХВЕ сказал мне: Этот город будет предан на время, и народ на время наказан, но мир не будет предан забвению.
\vs 2Ba 4:2
Или ты думаешь, что об этом городе Я сказал: Я вырезал тебя на ладонях рук Моих?
\vs 2Ba 4:3
Нет, это здание, что возвышается сейчас посреди вас, не то, которое будет открыто у Меня, уготованное здесь от времени, когда Мне на мысль пришло создать Рай. Я показал его Адаму до того, как он согрешил. Когда же он преступил повеление, он утратил его, также, как и Рай.
\vs 2Ba 4:4
Я показал его также и Аврааму ночью между разсеченными жертвами.
\vs 2Ba 4:5
И Моисею Я тоже показал его на горе Синай, когда Я открыл ему образ Скинии и всех её сосудов.
\vs 2Ba 4:6
И вот теперь Я храню его у Себя, равно как и Рай.
\vs 2Ba 4:7
Итак, иди и исполни то, что Я повелел тебе.

\vs 2Ba 5:1
И я отвечал, говоря: Так, на меня возложена скорбь о Сионе оттого, что Твои враги придут на это место и осквернят Твоё Святилище, и уведут в плен Твоё наследие. Они воцарятся над теми, кого Ты возлюбил. И они возвратятся к местам своих идолов и будут хвалиться перед ними. Но что Ты сделаешь ради Твоего великого Имени?
\vs 2Ba 5:2
И ЯХВЕ сказал мне: Мое Имя и Моя слава во веки веков; и Мой суд удерживает Свою праведность до времени.
\vs 2Ba 5:3
И ты увидишь это своими глазами: не вражеское войско разрушит Сион и предаст огню Иеросалим, но слуги Судьи в свое время.
\vs 2Ba 5:4
Ты же иди и исполни всё, что Я сказал тебе.
\vs 2Ba 5:5
И я пошел и взял с собою Иеремию, Аддо и Серайю, и Иавеша и Годолию, а также всех почтенных мужей из народа. Я привел их к потоку Кедронскому и рассказал им всё, что было сказано мне.
\vs 2Ba 5:6
И они возвысили свой голос и все заплакали.
\vs 2Ba 5:7
И мы сидели там, соблюдая пост до вечера.

\vs 2Ba 6:1
Назавтра войско Халдеев окружило город. К вечеру я, Варух, оставил народ и ушел, и стал у дуба.
\vs 2Ba 6:2
Я плакал о Сионе и сокрушался о плене, который выпал народу,
\vs 2Ba 6:3
и вдруг сильный дух поднял меня и перенес через стену Иеросалима.
\vs 2Ba 6:4
И я увидел: вот, четыре ангела стоят в четырех углах города, и каждый из них держал в руках горящий факел.
\vs 2Ba 6:5
Еще один ангел сошел с неба и сказал им: Держите факелы, но не зажигайте пожар, доколе я не скажу вам.
\vs 2Ba 6:6
Ибо я послан сказать прежде слово земле и передать ей то, что повелено мне от ЯХВЕ Элиона.
\vs 2Ba 6:7
И я видел, как он сошел на Святое Святых и взял завесу, святой ефод, седалище искупления, две скрижали и священное облачение священников, алтарь для курений, сорок восемь драгоценных камней, что носят священники, и святые сосуды Скинии.
\vs 2Ba 6:8
И он сказал земле громким голосом: Земля, земля, земля, слушай слово Эл Шаддаи, прими предметы, которые я вручаю тебе, и храни их до последних времен. Тогда лишь, когда ты получишь повеление, ты вернешь их. И так чужие не овладеют ими.
\vs 2Ba 6:9
Ибо настало время, когда Иеросалим будет предан ненадолго, пока не придет повеление возстановить его навечно.
\vs 2Ba 6:10
И земля разверзла свою пасть и поглотила их.

\vs 2Ba 7:1
И после этого я услышал, как ангел сказал тем ангелам, что держали факелы: Теперь сокрушите и уничтожьте эту стену до её основания, чтобы враги не могли хвалиться, говоря: Это мы разрушили стены Сионские, это мы сожгли место Эл Шаддаи.
\vs 2Ba 7:2
Возьмите то место, на котором я стоял прежде.

\vs 2Ba 8:1
Ангелы сделали так, как он им велел. И когда они уничтожили углы стен, и когда рухнула стена, из Храма раздался голос, и он сказал:
\vs 2Ba 8:2
Входите, враги, и вы, ненавистники, идите сюда. Охранявший этот Дом покинул его.
\vs 2Ba 8:3
И я, Варух, ушел.
\vs 2Ba 8:4
И после этого вошло войско Халдеев и заняло Дом и всё, что вокруг него.
\vs 2Ba 8:5
И оно увело народ в плен, убив иных из него; они связали цепью царя Седекию и отправили его к царю Вавилона.

\vs 2Ba 9:1
И я, Варух, возвратился вместе с Иеремией, сердце которого было найдено чистым от греха и который не был взят в плен при взятии города.
\vs 2Ba 9:2
И мы разодрали наши одежды и плакали, и оделись во вретище, и постились семь дней.

\vs 2Ba 10:1
Семь дней спустя, слово Божие было обращено ко мне, и мне было сказано:
\vs 2Ba 10:2
Скажи Иеремии, чтобы он шел в Вавилон укреплять пленный народ.
\vs 2Ba 10:3
Сам же ты оставайся здесь, на развалинах Сиона, и после этих дней Я покажу тебе то, что случится в конце дней.
\vs 2Ba 10:4
Я передал Иеремии повеление ЯХВЕ,
\vs 2Ba 10:5
и он ушел вместе с народом. Я же, Варух, возвратился и сел у дверей Храма. И я оплакал Сион плачем, говоря так:
\vs 2Ba 10:6
Счастлив тот, кто никогда не рождался, или родился для того, чтобы умереть тотчас же.
\vs 2Ba 10:7
Но горе нам, живым, видевшим муки Сиона и участь Иеросалима.
\vs 2Ba 10:8
Я буду взывать к морским сиренам и вы, лилиты, спешите сюда из пустыни; демоны и драконы, поспешайте из леса, пробудитесь и препояшьтесь вретищем. Воспойте вместе со мною погребальную песнь, стенайте вместе со мною.
\vs 2Ba 10:9
Вы, земледельцы перестаньте сеять; земля, к чему тебе приносить плоды урожаев? Храни в своем чреве сладость твоей пищи.
\vs 2Ba 10:10
К чему, виноградная лоза, продолжать тебе дарить вино? Ибо никогда более его не принесут на Сион, никогда больше здесь не предложат в жертву начатков.
\vs 2Ba 10:11
Вы, небеса, удерживайте росу и не открывайте дождевые хранилища.
\vs 2Ba 10:12
Солнце, удержи свет твоих лучей, и ты, луна, погаси яркость твоего сияния. Зачем вновь рождаться дню, когда затмился свет Сиона?
\vs 2Ba 10:13
Женихи, не входите в брачный чертог и не давайте невестам украшаться венками. Женщины, не молитесь о том, чтобы зачать.
\vs 2Ba 10:14
Ибо безплодная весьма возрадуется, и те, у которых нет сыновей, будут считать себя счастливыми, а имеющие сыновей возскорбят.
\vs 2Ba 10:15
Зачем рожать в муках и затем хоронить в скорби?
\vs 2Ba 10:16
Или зачем мужам рождать потомство, или зачем их семя будет вновь получать имя, когда мать брошена в одиночестве, а сыновья уведены в плен?
\vs 2Ba 10:17
Не говорите отныне о красоте и не разсуждайте об изяществе.
\vs 2Ba 10:18
Но вы, священники, возьмите ключи от Святилища и забросьте их высоко в небо, верните их ЯХВЕ, говоря: Охраняй Сам Свой Дом, ибо мы были найдены неверными управителями.
\vs 2Ba 10:19
И вы, девы, вплетающие в виссон и шелк Офирское золото, поспешите, возьмите всё это и бросьте в огонь, дабы он возвратил эти вещи Сотворившему их, и пламя вернуло их Творцу, дабы враги не овладели ими.

\vs 2Ba 11:1
Но и против тебя, Вавилон, буду говорить я, Варух: даже если ты благоденствовал, и Сион жил в своей славе, велика была бы наша скорбь видеть тебя равным Сиону.
\vs 2Ba 11:2
Но теперь наша скорбь безгранична и стенания наши безмерны, ибо ты благоденствуешь, а Сион опустошен.
\vs 2Ba 11:3
Кто будет судьею, видя это? Или кому мы пожалуемся на то, что отяготило нас? Как ты вынес всё это, ЯХВЕ?
\vs 2Ba 11:4
Отцы наши уснули без страданий, и все праведники покоятся в земле с миром.
\vs 2Ba 11:5
Они не познали нынешних тягот, они не слышали о горе, выпавшем нам.
\vs 2Ba 11:6
Земля, если бы ты имела уши; пыль, если бы ты имела сердце, вы бы могли пойти возвестить Шеолу и сказать мёртвым: Вы гораздо блаженнее нас, живущих.

\vs 2Ba 12:1
Но я выскажу мои мысли, и возвышу голос против тебя, земля благоденствующая:
\vs 2Ba 12:2
Не всегда пылает жара полуденная, и лучи солца не постоянно дают свет.
\vs 2Ba 12:3
Не думай и не надейся, что ты во всякое время будешь благоденствовать и радоваться. Не возвышайся чрезмерно и не надмевайся, повергая в рабство.
\vs 2Ba 12:4
Ибо воистину, в свою пору проснется гнев на тебя, пока что удерживаемый как уздою долготерпением.
\vs 2Ba 12:5
И закончив говорить, я постился семь дней.

\vs 2Ba 13:1
И после этого я, Варух, стоял на горе Сион, и вот, голос пришел с высоты и сказал мне:
\vs 2Ba 13:2
Встань, Варух, и слушай слово Эл Шаддаи.
\vs 2Ba 13:3
Поскольку ты изумлен участью Сиона, ты останешься до конца времен свидетельствовать о ней.
\vs 2Ba 13:4
И когда вдруг эти процветающие города спросят: Почему Эл Шаддаи навлек на нас эту кару?
\vs 2Ba 13:5
скажи им, ты и подобные тебе, вынесшие эту катастрофу и кары, которые обрушились на вас и ваш народ в определенное время, скажи им, что народы будут тяжко наказаны.
\vs 2Ba 13:6
И они будут упорствовать во зле.
\vs 2Ba 13:7
И если они тогда скажут: Когда это будет?
\vs 2Ba 13:8
ответь им: Вы пили оцеженное вино, выпейте и осадок. Ибо таков суд Элиона, Который не взирает на лица.
\vs 2Ba 13:9
Поэтому он не пощадил Своих сыновей, но прежде поразил их, как Своих врагов, ибо они согрешили.
\vs 2Ba 13:10
Поэтому они подверглись карам лишь для того, чтобы снискать прощение.
\vs 2Ba 13:11
Но вы, народы и языки, вы поистине виновны, ибо вы всегда попирали ногами землю и неправедно обходились с творением.
\vs 2Ba 13:12
Я всегда творил для вас добро, и всегда вы противились Моему благосердию.

\vs 2Ba 14:1
И я отвечал, говоря: Вот, Ты показал мне последовательность времён и то, что придёт после. Ты сказал мне воздаяние, о котором Ты возвестил, что оно постигнет народы.
\vs 2Ba 14:2
Теперь я знаю, сколь многие согрешили. Пожив в благоденствии, они покинули этот мир, где в эти времена останется мало народов, которые услышат слова, сказанные Тобою.
\vs 2Ba 14:3
Какой в этом прок? Или какое зло должны мы ожидать более того, что мы видели, поразившим нас?
\vs 2Ba 14:4
Но я снова буду говорить пред Тобою.
\vs 2Ba 14:5
Какое преимущество получат знающие пред Тобою, не ходившие в тщеславии, подобно остальным народам? Они не сказали тому, в чем нет жизни: Дай нам жизнь, но всегда боялись Тебя и никогда не удалялись от Твоих путей.
\vs 2Ba 14:6
И вот, несмотря на их ревность, Ты не пощадил Сион.
\vs 2Ba 14:7
И если одни творили злые дела, надо было простить Сион, ради добрых дел, которые сотворили другие и не губить его ради дел неправедных.
\vs 2Ba 14:8
Но кто, ЯХВЕ, Бог мой, проникнет в Твой суд, кто изследует глубину Твоего пути или возвестит вес Твоей стези?
\vs 2Ba 14:9
Или же кто способен понять Твой непостижимый совет? Или кто из тех, что рожден, когда-нибудь найдет начало и конец Твоей премудрости?
\vs 2Ba 14:10
Ибо все мы стали подобны дыханию.
\vs 2Ba 14:11
Ибо так же, как дыхание поднимается само собою и умирает во вне, так и природа человека: они не отходят по своей воле и не знают, какова будет их участь в конце.
\vs 2Ba 14:12
Праведные справедливо ожидают конца и без страха покидают свои жилища, ибо они у Тебя обладают могучим хранилищем своих дел.
\vs 2Ba 14:13
Поэтому они без страха оставляют этот мир и с радостью доверяются надежде стяжать мир, обещанный Тобою.
\vs 2Ba 14:14
Но горе нам, уже покрытым поношением и ожидающим лишь злого в будущем.
\vs 2Ba 14:15
Но Ты точно знаешь, что Ты сделал с Твоими рабами, ибо мы неспособны понять, что есть добро так, как Ты, наш Создатель.
\vs 2Ba 14:16
Но я буду опять говорить пред Тобою, ЯХВЕ, Бог мой.
\vs 2Ba 14:17
Прежде чем мир начал быть с его обитателями, Ты разсудил и произнес слово, и сейчас же творения стали пред Тобою.
\vs 2Ba 14:18
И Ты сказал, что создашь человека управителем Твоих дел в Твоем мире, дабы каждый знал, что не он был создан для мира, но мир для него.
\vs 2Ba 14:19
На деле же я вижу, что творение, созданное ради нас, пребывает, тогда как мы, для которых оно создано, погибаем.

\vs 2Ba 15:1
ЯХВЕ отвечал мне, говоря: Справедливо ты изумлен тем, что люди преходящи, но ты неверно разсудил о зле, поражающем грешников,
\vs 2Ba 15:2
когда ты сказал: Праведные отняты, а нечестивые благоденствуют;
\vs 2Ba 15:3
и когда ты прибавил: Никто не познал Твои суды.
\vs 2Ba 15:4
Посему слушай Меня, и Я буду говорить тебе; будь внимателен и услышишь от Меня Мои слова.
\vs 2Ba 15:5
Человек по праву мог бы пренебречь Моим судом, если бы не получил от Меня Закон, и если бы Я не увещевал его к уразумению.
\vs 2Ba 15:6
Но поскольку он грешил сознательно, так же сознательно он будет наказан.
\vs 2Ba 15:7
Что же до сказанного тобою о праведниках, будто бы мир начался для них, мир грядущий будет еще более принадлежать им.
\vs 2Ba 15:8
Ибо этот мир есть борьба и труд с великою скорбью, но мир грядущий будет венцом, соединенным с великою славой.

\vs 2Ba 16:1
Я отвечал, говоря: ЯХВЕ, Бог мой, вот нынешние годы кратки и злы. Кто же сможет за столь краткое время приобрести то, что не имеет меры?

\vs 2Ba 17:1
ЯХВЕ отвечал мне, говоря: У Элиона ничего не значит ни долгота, ни краткость времени.
\vs 2Ba 17:2
Что дала Адаму жизнь длиною в девятьсот тридцать лет, когда он преступил данную заповедь?
\vs 2Ba 17:3
Безполезным было прожитое им долгое время. Он ввел в мир смерть и сократил годы родившихся от него.
\vs 2Ba 17:4
Оттого ли и Моисей не потерпел урон, прожив только сто двадцать лет в послушании своему Создателю? Он передал Закон потомству Иакова и зажег светоч для народа Израилева.

\vs 2Ba 18:1
Я отвечал, говоря: Тот, кто зажег свет, принял его от Света, однако мало таких, кто последовал ему, и многие из тех, кого он осветил, взяли от тьмы Адама и не возрадовались свету Светоча.

\vs 2Ba 19:1
И Он отвечал мне, говоря: Посему в то время он заключил с ними завет и сказал: Вот, я поставил перед вами жизнь и смерть. И он взял в свидетели перед ними небо и землю,
\vs 2Ba 19:2
ибо он знал, что его время кратко, но что небеса и земля пребудут вовек.
\vs 2Ba 19:3
Они, однако, после его смерти, согрешили и пренебрегли, хорошо зная, что у них есть Закон, который будет их обвинять, Свет, который ничто не обманет, сферы, которые будут свидетельствовать,
\vs 2Ba 19:4
и Я Сам, Судящий всё, что есть. Ты же не заботься об этом и не печалься о прошлом.
\vs 2Ba 19:5
Вот, теперь надлежит придти исполнению времен, дел и благоденствия, а также унижению, а не началу всего этого.
\vs 2Ba 19:6
Если человек, преуспевавший в начале, в старости покрывается поношением, он забывает о своем бывшем благоденствии.
\vs 2Ba 19:7
И напротив, преуспев позднее, после того, как в начале был в ничтожестве, он уже не помнит о своем унижении.
\vs 2Ba 19:8
Слушай еще: если всё это время, с того дня, когда смерть была суждена людям, преходящим в этом мире, каждый в начале преуспевал бы, дабы погибнуть в конце, всё было бы тщетою.

\vs 2Ba 20:1
Поэтому вот, придут дни, когда времена будут проходить быстрее, чем в прошлом. И все сроки будут истекать скорее, чем сейчас, быстрее нынешних будут течь годы.
\vs 2Ba 20:2
Поэтому Я и уничтожил Сион, дабы ускорить время Моего посещения этого мира.
\vs 2Ba 20:3
Ныне же сохраняй всем сердцем всё, что Я повелеваю тебе, и запечатай это в самой глубине ума.
\vs 2Ba 20:4
Тогда Я открою Мой всевластный суд и Мои непроницаемые пути.
\vs 2Ba 20:5
Иди же и освятись в течение семи дней: ни хлеба не ешь, ни воды не пей, и не говори ни с кем.
\vs 2Ba 20:6
И по окончании этого срока возвращайся на это место, и Я явлюсь тебе, и Я скажу тебе истины и дам тебе наставления о последовании времён; ибо они приближаются и не замедлят.

\vs 2Ba 21:1
Молитва Варуха, сына Нерии. И я покинул это место, и ушел, и сел у потока Кедронского в подземной пещере. Там я освятился: хлеба не ел и не был голоден, воды не пил и не испытывал жажды. Там я оставался до седьмого дня, как Он заповедал мне.
\vs 2Ba 21:2
Затем я пришел на то место, где Он разговаривал со мною.
\vs 2Ba 21:3
И на закате солнца душа моя была охвачена множеством мыслей, и я обратился к Шаддаи
\vs 2Ba 21:4
и сказал: Ты, сотворивший землю, выслушай меня; Ты, утвердивший небесный свод Твоим словом и укрепивший высоту неба духом; Ты, от начала мира призывающий то, что еще не существует, и всё тебе повинуется;
\vs 2Ba 21:5
Ты, одним мановением повелевающий воздуху и видящий будущее словно прошлое;
\vs 2Ba 21:6
Ты, великим советом управляющий силами, которые предстоят Тебе, и в гневе направляющий безчисленные святые создания, сотворенные Тобою из огня и пламени, окружающие Твой престол;
\vs 2Ba 21:7
Ты Единый, имеющий власть исполнить во всякое время всё, что Ты пожелаешь;
\vs 2Ba 21:8
Ты, Кто пролил каждую каплю дождя, упавшую на землю; Ты, Единый, знающий конец времен прежде их начала, внемли моей молитве.
\vs 2Ba 21:9
Ты Один можешь поддерживать тех, кто есть, тех, кто ушел и тех, кто придет, грешников и тех, кто оправдан, ибо Ты Живой и Непостижимый.
\vs 2Ba 21:10
Ты Единый, Живой, Безсмертный и Непостижимый; Ты знаешь число людей,
\vs 2Ba 21:11
многие ли согрешили в своё время, и были ли оправданы другие, не менее многочисленные.
\vs 2Ba 21:12
Ты знаешь место, которое Ты приготовил в конце одним, тем, кто согрешил; а также место свершения других, тех, что были оправданы.
\vs 2Ba 21:13
Ибо если бы для всех людей была лишь одна жизнь здесь, не было бы ничего горше.
\vs 2Ba 21:14
Ибо к чему сила, которая обращается в слабость, пища вдоволь, если она обращается в голод, красота, если она становится ненавистною?
\vs 2Ba 21:15
Ибо без конца изменяется природа человека.
\vs 2Ba 21:16
Мы уже не те, чем мы были прежде, и в будущем мы не останемся тем, что мы есть сегодня.
\vs 2Ba 21:17
Если бы всему этому не надлежало окончиться, тщетно было бы и самое начало.
\vs 2Ba 21:18
Но дай мне знать, что исходит от Тебя, и просвети меня во всём, что я спрошу у Тебя.
\vs 2Ba 21:19
Доколе будет пребывать растленное и доколе будет процветающим время смертных? Будут ли еще сильнее оскверняться преходящие в этом мире?
\vs 2Ba 21:20
Дай же повеление в Твоём милосердии и соверши то, чему Ты обещал нам исполниться, дабы явилось Твоё могущество тем, кто думает, будто Твоё долготерпение от слабости.
\vs 2Ba 21:21
Покажи тем, кто видел несчастье, падшее на нас и на наш город, не разумеющим того, что в долготерпении Твоей мощи Ты назвал нас ради Твоего Имени Своим возлюбленным народом.
\vs 2Ba 21:22
Всякое существо ныне смертно по природе.
\vs 2Ba 21:23
Посему удержи ангела смерти, и да возсияет Слава Твоя, да явится величие Твоей красоты. И пусть Шеол будет запечатан, пусть отныне он не принимает более умерших, и хранилища душ отпустят тех, кто в них заключен.
\vs 2Ba 21:24
Ибо многочисленны для нас года со дней Авраама, Исаака и Иакова и всех, кто подобен им и спит в земле. И Ты сказал, что ради них Ты сотворил мир.
\vs 2Ba 21:25
Ныне поспеши явить Твою Славу и не замедли исполнением Твоих обетований.
\vs 2Ba 21:26
И закончив на этом слова молитвы, я был в изнеможении.

\vs 2Ba 22:1
И после этого небеса открылись, и мне было видение. Мне дана была сила, и голос раздался свыше и сказал мне:
\vs 2Ba 22:2
Варух, Варух, почему ты в смятении?
\vs 2Ba 22:3
Отправившийся в путешествие не завершает ли его? И мореплаватель может ли утешиться, прежде чем достигнет гавани?
\vs 2Ba 22:4
Или обещавший сделать дар, но не сделавший, не крадет ли?
\vs 2Ba 22:5
Также и человек, засеявший землю, не теряет ли всё, если не соберет урожай в благоприятное время?
\vs 2Ba 22:6
И насадивший готовится ли собирать плоды раньше, чем они созреют?
\vs 2Ba 22:7
Или женщина, зачав и родив не в срок, не причиняет ли смерть своему ребёнку?
\vs 2Ba 22:8
Тот же, кто строит дом, может ли назвать своё строение домом прежде, чем покроет его крышей и закончит? Ответь Мне вначале на этот вопрос.

\vs 2Ba 23:1
И я отвечал, говоря: Нет, ЯХВЕ, Бог мой.
\vs 2Ba 23:2
И Он отвечал, говоря: Почему тогда ты волнуешься о том, чего не знаешь и возмущаешься тем, о чем тебе ничего не известно?
\vs 2Ba 23:3
Ибо так же, как Я не забыл о людях, которые живут сейчас и которые отошли, так же Я вспомню о тех, кто придет.
\vs 2Ba 23:4
Когда Адам согрешил, и смерть стала приговором для всех, кто родится, множество имеющих родиться было сочтено, и для этого числа были уготованы места, где будут жить живые и храниться мёртвые.
\vs 2Ba 23:5
И пока предустановленное число не исполнится, творение не будет спасено, ибо Мой дух созидает жизнь, и Шеол принимает умерших.
\vs 2Ba 23:6
Послушай же еще о том, что придет после этих времен.
\vs 2Ba 23:7
Ибо пришествие Моего искупления поистине близко и не отстоит уже далеко, как некогда.

\vs 2Ba 24:1
Вот, грядут дни, и откроются книги, в которых записаны грехи всех грешников, а также собраны все сокровища праведности оправданных в творении.
\vs 2Ba 24:2
В эти времена ты узнаешь, ты и многие с тобою, каково было долготерпение Элиона из поколения в поколение, и насколько Он был терпелив ко всем рожденным, как грешникам, так и праведным.
\vs 2Ba 24:3
И я отвечал, говоря: Но, ЯХВЕ, нет никого, кто бы знал число того, что произошло и того, что будет. И я сам, хотя и знаю, что было с нами, я не ведаю того, что будет с нашими врагами, и времени, когда Ты посетишь Твоих рабов.

\vs 2Ba 25:1
И Он отвечал, говоря: Ты также пребудешь до того времени, ради знамения, которое Элион сделает для населяющих землю в конце дней.
\vs 2Ba 25:2
Знамение будет таким:
\vs 2Ba 25:3
когда оцепенение охватит населяющих землю, они впадут во многие безпокойства и снова в жестокие муки.
\vs 2Ba 25:4
И будет, что от великих безпокойств они станут думать: Шаддаи не помнит более о земле. И когда они утратят надежду, тогда и настанет это время.

\vs 2Ba 26:1
И я отвечал, говоря: Долго ли продлятся эти безпокойства и растянется ли нужда на многие годы?

\vs 2Ba 27:1
И Он отвечал мне, говоря: То время разделено на двенадцать частей, и каждая предназначается для предписанного ей.
\vs 2Ba 27:2
В первой части начало волнений,
\vs 2Ba 27:3
во второй убийство знатных,
\vs 2Ba 27:4
в третьей падение великого множества,
\vs 2Ba 27:5
в четвертой послан меч,
\vs 2Ba 27:6
в пятой голод и засуха,
\vs 2Ba 27:7
в седьмой землетрясения и ужасы,
\vs 2Ba 27:9
в восьмой множество призраков и нападения демонов,
\vs 2Ba 27:10
в девятой падение огня,
\vs 2Ba 27:11
в десятой кражи и великое притеснение,
\vs 2Ba 27:12
в одиннадцатой нечестие и страсти,
\vs 2Ba 27:13
а в двенадцатой смута от смешения всего перечисленного.
\vs 2Ba 27:14
Ибо эти части времени соблюдаются для того, чтобы быть смешанными друг с другом и послужить одна другой,
\vs 2Ba 27:15
ибо одни из них превзойдут свой удел и возьмут у других, иные исполнят свои свойства, а также свойства других, чтобы находящиеся на земле не поняли, что с этими днями настал конец времен.

\vs 2Ba 28:1
И, однако, будет премудр тот, кто уразумеет.
\vs 2Ba 28:2
Мерою же счета будут две части: недели по семь недель.
\vs 2Ba 28:3
И я отвечал, говоря: Хорошо человеку достичь этого времени и быть его зрителем. Лучше, однако, не достигать ему этого, дабы не пасть.
\vs 2Ba 28:4
Но я прибавлю еще: нетленный презрит ли тленных и их удел, взирая лишь на нетленное?
\vs 2Ba 28:6
ЯХВЕ, если поистине события, предсказанные мне Тобою, должны совершиться, и если я обрел благодать в Твоих глазах, открой мне еще и это: в одном ли месте или в одной ли части земли они совершатся, или весь мир испытает их?

\vs 2Ba 29:1
И Он отвечал мне, говоря: То, что произойдет тогда, произойдет по всей земле.
\vs 2Ba 29:2
Поэтому их испытают все живущие. Я пощажу лишь тех, кто в эти дни окажется на этой земле.
\vs 2Ba 29:3
Когда исполнится предусмотренное для этих частей земли, Мессия начнет открываться.
\vs 2Ba 29:4
И Бегемот появится из своего места, и Левиафан возстанет из моря оба эти гигантских чудовища, которых Я создал в пятый день творения и которых Я сберегал для этого времени в пищу всем, кто останется.
\vs 2Ba 29:5
И земля принесёт свои плоды мириадократно. Каждый виноградник принесет тысячу лоз, каждая лоза принесет тысячу гроздей, каждая гроздь будет насчитывать по тысяче ягод, и каждая ягода даст кор вина.
\vs 2Ba 29:6
И те, кто голоден, возрадуются и каждый день будут видеть чудеса.
\vs 2Ba 29:7
Ветры изойдут от Моего лица, обдувая каждое утро благоуханием ароматных плодов, а вечерами будут поднимать облака, которые испустят целительную росу.
\vs 2Ba 29:8
В это время манна, хранящаяся в хранилище, будет падать вновь, и они будут есть её в эти годы, ибо они достигли конца времен.

\vs 2Ba 30:1
И после этого, когда исполнится время пришествия Мессии, и Он вернется в славе, все, кто почил в надежде на Него, воскреснут.
\vs 2Ba 30:2
В это время будут распечатаны хранилища, содержащие души праведников; они выйдут, и множество душ явится в единомысленном собрании. Первые возрадуются, последние не познают тревоги.
\vs 2Ba 30:3
Они истинно узнают, что пришел день, о котором было предсказано, как о скончании времен.
\vs 2Ba 30:4
Но души злых, увидев всё это, истлеют полностью, ибо они знают, что их ожидают мучения, и что настала их погибель.

\vs 2Ba 31:1
И после этого я пошел к народу и сказал им: Соберите ко мне всех старейшин, и я буду говорить с вами.
\vs 2Ba 31:2
Все собрались у потока Кедронского.
\vs 2Ba 31:3
И я отвечал, говоря: Слушай, Израиль, и я буду говорить тебе; и ты, семя Иакова, открой слух, и я буду учить тебя.
\vs 2Ba 31:4
Не забывайте Сион, но помните о скорби Иеросалима.
\vs 2Ba 31:5
Ибо вот, грядут дни, когда всё станет добычей тления и сделается как не бывшее вовсе.

\vs 2Ba 32:1
Вы же, если приготовите ваши сердца к тому, чтобы посеять в них плоды Закона, Шаддай пощадит вас в то время, когда Он потрясет всё создание.
\vs 2Ba 32:2
Ибо через краткое время Здание Сиона будет потрясено, а затем возстановлено.
\vs 2Ba 32:3
Это новое Здание будет также временным и после некоторого времени также будет разрушено и останется в развалинах до времени.
\vs 2Ba 32:4
Затем ему должно будет обновиться в славе и придти к вечному исполнению.
\vs 2Ba 32:5
Не надо скорбеть чрезмерно, видя бедствие, постигшее нас ныне, как и то, которое еще будет.
\vs 2Ba 32:6
Ибо еще более ужасным, чем эти два потрясения, будет испытание, в котором Шаддай обновит творение.
\vs 2Ba 32:7
А теперь не приближайтесь ко мне несколько дней и не приходите ко мне, прежде чем я приду к вам.
\vs 2Ba 32:8
И когда я сказал им эти слова, я, Варух, пошел своим путем, и когда народ увидел, что я ухожу, все возвысили голос и возстенали, говоря: Куда ты уходишь от нас, Варух? Неужто и ты покидаешь нас, словно отец, бросающий своих детей, оставляющий их сиротами?

\vs 2Ba 33:1
Таково ли повеление, данное тебе твоим спутником пророком Иеремией, когда он говорил тебе:
\vs 2Ba 33:2
Смотри за этим народом, тогда как я ухожу укреплять в Вавилон остаток наших братьев, приговоренных идти в плен?
\vs 2Ba 33:3
Теперь если и ты нас оставишь, лучше было бы всем нам погибнуть прежде, а потом уж уходи от нас.

\vs 2Ba 34:1
Я отвечал народу, говоря: Далека от меня мысль покинуть вас или скрыться от вас. Я лишь иду к Святому Святых, дабы просить у Шаддаи большего просвещения ради вас и ради Сиона. И после этого я вернусь к вам.

\vs 2Ba 35:1
И я, Варух, пошел на Святое место и сел на его развалинах. Я заплакал и сказал:
\vs 2Ba 35:2
Да будут мои глаза источником, и мои ресницы ключом слёз,
\vs 2Ba 35:3
ибо доколе мне скорбеть о Сионе и доколе мне оплакивать Иеросалим?
\vs 2Ba 35:4
Ибо на этом месте, где я ныне простерся, некогда первосвященник приносил святые жертвы, возлагая на них благовонные курения.
\vs 2Ba 35:5
Но ныне наша слава обратилась в прах, и наслаждение наших душ в пепел.

\vs 2Ba 36:1
И сказав эти слова, я заснул там, и ночью мне было видение: Вот, это был лес деревьев, выросший в долине, окруженной высокими горами и крутыми скалами. Лес простирался на обширном пространстве.
\vs 2Ba 36:3
И вот напротив него вырос виноградник, из под которого мирно струился источник.
\vs 2Ba 36:4
Потом этот источник достиг леса и стал большим потоком, и этот поток затопил лес и в одно мгновение вырвал с корнем все деревья и ниспроверг все горы вокруг.
\vs 2Ba 36:5
Так был унижен возвысившийся лес, и вершины гор опустились: источник был столь силен, что остался только один кедр.
\vs 2Ba 36:6
Когда он сломал и его и вырвал с корнем и уничтожил весь лес так, что от него не осталось ничего, и место его стало неузнаваемо, тогда виноградник вместе с источником поднялся в великом спокойствии, и он подошел близко к месту кедра. И ему принесли кедр, который был сломлен.
\vs 2Ba 36:7
И мне было видение: виноградник открыл уста свои и сказал кедру: Ты, кедр, единственный уцелевший из злого леса. Из-за тебя нечестие утвердилось и распространялось все эти годы, но никогда ничего доброго. Ты завладел тем, что тебе не принадлежало, и никогда не жалел твой собственный удел.
\vs 2Ba 36:8
Ты простирал свою власть на тех, кто был далеко от тебя; ты завлекал в сети твоего лукавства тех, кто был рядом с тобою. Каждый миг ты превозносился, будто бы тебя нельзя вырвать с корнем.
\vs 2Ba 36:9
Но теперь твой исход приблизился быстро, и твой час настал.
\vs 2Ba 36:10
Иди же, кедр, вслед за лесом, который исчез перед тобою, стань, как и он, пеплом, и пусть смешается ваш прах. Спите ныне в муках, покойтесь в страдании до пришествия последних времен, когда ты возвратишься и будешь наказан еще тяжелее.

\vs 2Ba 37:1
После этого я видел кедр в огне, а виноградник возросшим, долина же вокруг покрылась неувядающими цветами. И я проснулся и встал.

\vs 2Ba 38:1
И я помолился, говоря: ЯХВЕ, Бог мой, Ты во всякое время просвещаешь тех, кто водится разумом.
\vs 2Ba 38:2
Твой закон есть жизнь, и Твоя премудрость правота.
\vs 2Ba 38:3
Открой же мне истолкование этого видения.
\vs 2Ba 38:4
Ибо Ты знаешь, что моя душа всегда ходила в Твоем Законе, и от рождения я не удалялся от Твоей Премудрости.

\vs 2Ba 39:1
И Он отвечал мне, говоря: Варух, вот истолкование видения, которое ты видел.
\vs 2Ba 39:2
Так же, как ты видел большой лес, окруженный высокими и крутыми горами вот значение этого
\vs 2Ba 39:3
точно так же наступают дни, когда царство, некогда уничтожившее Сион, само будет уничтожено и покорено пришедшему вслед за ним.
\vs 2Ba 39:4
Оно также будет скоро разрушено, и возстанет третье, которе будет главенствовать в свое время и тоже исчезнет.
\vs 2Ba 39:5
После этого возстанет четвёртое царство, власть которого будет суровей и злее власти предыдущих царств, и оно будет править долго, как лес в долине, и побеждать века и вознесётся как кедр Ливанский.
\vs 2Ba 39:6
Истина скроется от него, а те, кто запятнан нечестием, будут прибегать к нему, как злые звери прибегают в лес и прячутся в нем.
\vs 2Ba 39:7
И когда приблизится время его конца и время его падения, будет открыта власть Моего Мессии. Она будет подобна источнику и винограднику, и открывшись, выдернет с корнем их собравшееся множество.
\vs 2Ba 39:8
Что же до высокого кедра, уцелевшего в одиночестве из леса, и слов, которые ему были сказаны виноградником и которые ты слышал, вот значение их:

\vs 2Ba 40:1
Последний вождь, который уцелеет в то время уничтожения множества его собраний, будет связан и приведен на гору Сион. Мой Мессия обличит его во всяком нечестиии, соберёт и поставит перед ним всякое деяние его воинств.
\vs 2Ba 40:2
И потом Он казнит его и будет покровителем для остатка Моего народа, который обретётся в месте, избранном Мною.
\vs 2Ba 40:3
И Его власть пребудет вовек, доколе не прекратится этот мир тления, и не исполнятся предсказанные времена.
\vs 2Ba 40:4
Вот твое видение, и вот его истолкование.

\vs 2Ba 41:1
И я отвечал, говоря: Для кого и для скольких будут эти события? Кто из них удостоится быть спасенным в это время?
\vs 2Ba 41:2
Я выскажу пред Тобою все мои мысли и спрошу у Тебя о том, над чем я размышляю.
\vs 2Ba 41:3
Вот, я вижу многих, удалившихся от Твоего завета, и отбросивших ярмо Твоего Закона.
\vs 2Ba 41:4
Я видел и других, кто, напротив, прибег под сень Твоих крыльев.
\vs 2Ba 41:5
Какова будет их участь? И что готовит им последнее время?
\vs 2Ba 41:6
Будет ли тщательно взвешена долгота их жизни, и будут ли они осуждены по тому, куда склонятся весы?

\vs 2Ba 42:1
Он отвечал мне, говоря: И это Я также покажу тебе.
\vs 2Ba 42:2
Ты спрашиваешь, для кого предназначено это, и сколько их будет? Те, кто уверовал, получать обещанные блага; презревшим же будет противное.
\vs 2Ba 42:3
Ты также говорил о некоторых, котороые приблизились и удалились. Вот слово о них.
\vs 2Ba 42:4
Для тех, кто вначале послушался, а потом отдалился и смешался с семенем перемешавшихся народов, первая часть их жизни была [прежде] и будет сочтена за нечто [возвышенное].
\vs 2Ba 42:5
Тем же, кто прежде не знал, а после познал жизнь и примешался к семени, отделенному от народов, первая часть их времени [была позже и] будет сочтена за нечто [возвышенное].
\vs 2Ba 42:6
Одни времена сменяют другие, эпохи сменяются эпохами. Они не примут одни от других и в конце уравняются по мере времени и часов эпох.
\vs 2Ba 42:7
Тление возьмет своих, а жизнь своих.
\vs 2Ba 42:8
И прах будет призван, и ему будет сказано: Верни то, что тебе не принадлежит; яви то, что ты сохранял до времени.

\vs 2Ba 43:1
Но ты, Варух, укрепи твое сердце ради сказанного тебе и уразумей то, что тебе показала благодать в видениях, ибо вот ты был весьма утешен навек.
\vs 2Ba 43:2
Сейчас ты покинешь это место, и видимые его окрестности; ты забудешь всё тленное и никогда не вспомнишь о том, что бывает среди смертных.
\vs 2Ba 43:3
Иди же, отдай повеления твоему народу, а после возвращайся сюда, и постись семь дней, а потом Я приду к тебе и буду говорить с тобою.

\vs 2Ba 44:1
И я, Варух, ушел оттуда и пришел к моему народу и призвал своего первородного сына, Годолию, [моих друзей] и семерых из старейшин народа,
\vs 2Ba 44:2
и я сказал им: Вот, я иду к моим отцам, по пути всей земли,
\vs 2Ba 44:3
вы же не удаляйтесь от пути Закона, но и сохраняйте и увещевайте уцелевший народ, дабы они не удалялись от заповедей Элиона.
\vs 2Ba 44:4
Ибо вы видите, что Тот, Кому мы служим, справедлив; Тот, Кто образовал нас, не взирает на лица.
\vs 2Ba 44:5
Посмотрите, что было с Сионом, что произошло с Иеросалимом;
\vs 2Ba 44:6
суд Шаддая познается так же, как Его пути, неизследимые и прямые.
\vs 2Ba 44:7
Если вы будете терпеливы и пребудете в Его страхе, и не будете забывать Его Закона, времена для вас изменятся к лучшему, и вы увидите утешение для Сиона.
\vs 2Ba 44:8
То, что существует ныне, есть ничто, грядущее же будет весьма великим.
\vs 2Ba 44:9
Тленное проходит, и смертное уносится, и ни о чем из настоящего уже не вспомнят, не вспомнят более об этом времени, оскверненном злом.
\vs 2Ba 44:10
Бегущий ныне бежит тщетно; и успевающий скоро падет и будет унижен.
\vs 2Ba 44:11
Ибо желанно будущее, и в грядущем наша надежда.
\vs 2Ba 44:12
И есть час, который не прейдет, грядет эпоха, которая пребудет вовек, новый мир, который не приведет к тлению идущих под его властью и не поведет к гибели тех, кто спасается в нем.
\vs 2Ba 44:13
Для этих будет наследством возвещенное время, они унаследуют обетованную эпоху.
\vs 2Ba 44:14
Приготовившим хранилища премудрости и тем, у кого были найдены сокровищницы разума, не удалившимся от милосердия и сохранившим истину Закона будет отдан грядущий мир; жилище же других, весьма многих, будет в огне.

\vs 2Ba 45:1
Вы же, насколько сможете, наставляйте народ, ибо в этом ваш труд. 2 Ибо научая их Закону, вы спасёте их.

\vs 2Ba 46:1
Мой сын и старейшины народа отвечали мне, говоря: Неужели Шаддаи унизит нас настолько, что скоро отнимет тебя у нас?
\vs 2Ba 46:2
Воистину, мы будем во тьме; не будет более света для уцелевшего народа.
\vs 2Ba 46:3
Где еще будем мы искать Закон и кто различит для нас смерть от жизни?
\vs 2Ba 46:4
И я сказал им: Я не могу противиться престолу Шаддаи. Однако Израиль не останется без премудрого, и без Закона семя Иакова.
\vs 2Ba 46:5
Вы лишь приготовьте ваши сердца к слушанию Закона и подчинению тем, кто в страхе обладает премудростью и разумом. Расположите ваши души так, чтобы вам не уклониться от этого.
\vs 2Ba 46:6
Если вы будете делать это, благо, которое я предсказал вам, сбудется вам, и вы не подвергнетесь мучению, которое я возвещал вам прежде.
\vs 2Ba 46:7
Но о слове, которое возвестило мне удалиться, я не открыл никому, даже моему сыну.

\vs 2Ba 47:1
С тем я ушел и отослал их, и покинул их, говоря: Вот, я иду к Хеврону, туда посылает меня Шаддаи.
\vs 2Ba 47:2
И я пришел на то место, о котором мне было сказано, и я сел там и постился семь дней.

\chhdr{Молитва Варуха.}
\vs 2Ba 48:1
И после седьмого дня я помолился Шаддаи, говоря:
\vs 2Ba 48:2
О ЯХВЕ, Ты призываешь приход времён, и они встают перед Тобою. Мощь столетий исчезает от Тебя. Ты располагаешь череды эпох, и они повинуются Тебе.
\vs 2Ba 48:3
Ты Один знаешь счет поколений, и не многим Ты открываешь Твои тайны.
\vs 2Ba 48:4
Ты показываешь изобилие огня, и Ты взвешиваешь легкость ветра.
\vs 2Ba 48:5
Ты изследуешь пределы высот; Ты проницаешь глубины тьмы.
\vs 2Ba 48:6
Ты определяешь число тех, кто преходят, и тех, кого надо сохранить; и Ты готовишь жилище для тех, кто будет.
\vs 2Ba 48:7
Ты помнишь начала, которые Ты создал; и о будущем уничтожении Ты не забываешь.
\vs 2Ba 48:8
Страшными и яростными знамениями Ты повелеваешь пламени, и оно испаряется. Словом Ты вызываешь то, чего нет, и могучею властью Ты удерживаешь то, что еще не пришло.
\vs 2Ba 48:9
Ты научаешь творения Твоему разуму, Ты наделяешь Сферы мудростью служения Тебе.
\vs 2Ba 48:10
Безчисленные воинства предстоят Тебе и служат в покое по своему чину по одному Твоему мановению.
\vs 2Ba 48:11
Послушай раба Твоего и приклони ухо к моему молению.
\vs 2Ba 48:12
Ибо мы рождены на краткое время и скоро уходим.
\vs 2Ba 48:13
У Тебя же часы как эпохи, и дни как поколения.
\vs 2Ba 48:14
Посему не гневайся на человека, ибо он ничто; не изследуй наши дела.
\vs 2Ba 48:15
Ибо что мы такое? По Твоему дару мы пришли в мир, и вошли в него без нашего согласия.
\vs 2Ba 48:16
Мы не говорили нашим родителям: Зачните нас!, ни посылали к Шеолу с вестью: Прими нас!
\vs 2Ba 48:17
Сильны ли мы выдержать Твой гнев? Способны ли мы вынести Твой суд?
\vs 2Ba 48:18
Прикрой нас Твоим милосердием и Твоею благостью помоги нам.
\vs 2Ba 48:19
Взгляни на смиренных, служащих Тебе, и спаси всех, приближающихся к Тебе. Не разрушай надежду нашего народа и не сокращай времени Твоей помощи.
\vs 2Ba 48:20
Ибо это народ, избранный Тобою, эти люди народ, подобного которому нет в очах Твоих.
\vs 2Ba 48:21
Но и теперь я буду говорить прямо пред Тобою и высказывать мысли моего сердца.
\vs 2Ba 48:22
Мы уповаем на Тебя, ибо вот, с нами Твой Закон. Мы знаем, что покуда мы соблюдаем Твои заповеди, мы не падём.
\vs 2Ba 48:23
На всё время и во всяком деле на нас благословение, доколе мы не смешиваемся с народами.
\vs 2Ba 48:24
Ибо мы единственный прославленный народ, получивший Закон от Единого. И этот Закон, который посреди нас, поможет нам; его превосходная премудрость поддержит нас.
\vs 2Ba 48:25
И когда я произнес слова этой молитвы, я был в полном изнеможении.
\vs 2Ba 48:26
И Он отвечал мне, говоря: Ты молился в простоте, Варух, и все твои слова были услышаны.
\vs 2Ba 48:27
Но Мой суд взыскует своё, и Мой Закон взывает к своим правам.
\vs 2Ba 48:28
По твоим словам Я отвечу тебе и по твоей молитве Я буду говорить с тобою.
\vs 2Ba 48:29
Ибо так и есть: подлежащий тлению есть ничто. Он творит зло, как если бы он мог делать что-либо, и ни о Моей благости не помнит, ни постигает Моего долготерпения.
\vs 2Ba 48:30
Поэтому он отнимется, как Я говорил тебе прежде. Время, о котором я предсказал тебе, придет; время потрясения исполнится.
\vs 2Ba 48:31
Оно придет и пройдет в силе и ярости, посреди смятения, в буйстве и возмущении.
\vs 2Ba 48:32
В эти дни все живущие на земле поднимутся друг против друга, не зная, что Мой суд приблизился.
\vs 2Ba 48:33
Ибо в эти дни не найдется много мудрых; и разумные будут редки. Более того, те, кто знают больше всего, будут молчать.
\vs 2Ba 48:34
Поднимется много слухов и немало новостей; будут наблюдать явления призрачные, сообщать много пророчеств, иные из которых окажутся тщетными, иные подтвердятся.
\vs 2Ba 48:35
Честь обратится в позор, и сила будет унижена в презрение, уверенность ослабнет, красота станет отвратительной.
\vs 2Ba 48:36
И многие скажут друг другу в это время: Где спряталось изобилие разума? Куда сокрылось множество премудрости?
\vs 2Ba 48:37
И пока они будут размышлять об этом, ревность проявится против тех, кто об этом не думал. Спокойный будет обуреваем страстями, и многие будут в гневе вредить многим. Они поднимут войска для кровопролития, и в конце все они погибнут.
\vs 2Ba 48:38
И в это самое время каждому станет ясно, что времена меняются. Потому что во всё это время они осквернялись, творя жестокие дела, и каждый ходил по своим деяниям, и не вспоминал о Законе Элиона.
\vs 2Ba 48:39
Поэтому огонь пожрет их помыслы, и пламя испытает помышления их печени. Судья придет и не замедлит.
\vs 2Ba 48:40
Ибо каждый из живущих на земле творит зло сознательно; и по своей гордыне они пренебрегли Моим Законом.
\vs 2Ba 48:41
Многие тогда искренне заплачут о живых сильнее, чем о мертвых.
\vs 2Ba 48:42
И я отвечал, говоря: О Адам, что ты сделал для всех, кто родился от тебя? Что будет сказано Евве, которая первою послушала змея?
\vs 2Ba 48:43
Всё это множество идет к своей погибели; и нет счета тем, кого пожрет огонь.
\vs 2Ba 48:44
Но я буду еще говорить пред Тобою,
\vs 2Ba 48:45
ЯХВЕ, Бог мой. Ты знаешь, что есть Твое создание,
\vs 2Ba 48:46
ибо Ты некогда повелел праху образовать Адама, и Ты знаешь число родившихся от него, и то, сколь многие из бывших некогда, согрешили пред Твоим лицем, отказавшись признать Тебя своим Создателем.
\vs 2Ba 48:47
И их конец станет их обвинением, и Твой Закон, который они преступили, будет для них отмщением в Твой день.
\vs 2Ba 48:48
Но теперь оставим нечестивых и спросим о праведных.
\vs 2Ba 48:49
И я расскажу об их благополучии и не устану превозносить уготованную им славу.
\vs 2Ba 48:50
Ибо поистине, как в то краткое время, что вы живете в преходящем мире, вы понесли множество трудов, так и в мире, который не скончается, вы получите великий свет.

\vs 2Ba 49:1
И всё же я буду еще просить Тебя, о Шаддаи, и умолять Твоё милосердие, Создавший всё.
\vs 2Ba 49:2
В каком обличье будут жить те, кто увидит Твой День? Или что станет с блеском тех, кто переживет его?
\vs 2Ba 49:3
Обретут ли они вновь свой нынешний вид? Примут ли снова эти члены пленения, ныне преданные злу и через которые совершается зло? Или же Ты изменишь бывших в мире настолько же, насколько и самый мир?

\vs 2Ba 50:1
Он мне отвечал, говоря: Выслушай это слово, Варух, и запечатлей в памяти сердца всё, что узнаешь.
\vs 2Ba 50:2
Тогда земля возвратит мёртвых, которых она ныне принимает на хранение. Она не изменит ничего в их обличьи и вернет их такими, какими приняла, и какими Я их отдаю ей, такими она их воскресит.
\vs 2Ba 50:3
Ибо тогда надо будет показать живым, что мёртвые ожили вновь, и что те, кто ушел, возвратился вновь.
\vs 2Ba 50:4
И когда ныне знакомые узнaют друг друга, тогда суд обретёт силу, и предсказанное состоится.

\vs 2Ba 51:1
И когда пройдет этот назначенный день, тогда лишь изменятся обличья тех, кто будет осужден, и слава тех, кто будет оправдан.
\vs 2Ba 51:2
Обличье тех, кто ныне делает злое, явится худшим, чем было, ибо им надлежит испытывать мучения.
\vs 2Ba 51:3
Также и слава тех, кого ныне оправдывает Мой Закон, кто явит себя разумным в жизни и укоренил в своём сердце корень премудрости, их блеск прославится во время преображения, и образ их лиц изменится светлою красотой, чтобы они могли получить и принять мир, который не умирает и который обетован им на то время.
\vs 2Ba 51:4
Те, кто придут тогда, весьма возстенают оттого, что презрели Мой Закон и заткнули свой слух, чтобы не слышать премудрость и не принять разум.
\vs 2Ba 51:5
Когда же они увидят вознесенными и прославленными тех, над которыми они ныне так возносятся, а также то, что изменятся те и другие, одни в ангельское сияние, а они сами в страшные привидения, они будут полностью подавлены.
\vs 2Ba 51:6
Сначала они претерпят это зрелище, а после пойдут на муки.
\vs 2Ba 51:7
Но тем, кто будет спасен своими делами, и чаяния которых в Законе, надежды на разум, чья вера в премудрости, тем в их время будут явлены чудеса.
\vs 2Ba 51:8
Они увидят тот мир, что незрим ныне, они узрят время, которое ныне скрыто от них,
\vs 2Ba 51:9
время, которое уже не будет старить их.
\vs 2Ba 51:10
Они будут жить на вершинах этого мира, они будут подобны ангелам и сравниваемы со звездами. И они смогут принимать по желанию любое обличье, от красоты до сияния, и от света до торжества славы.
\vs 2Ba 51:11
Ибо пред ними откроются просторы Рая, и им будет явлена величественная красота живых существ под престолом, а также все ангельские воинства, ныне удерживаемые Моим словом от проявления и привязанные Моим повелением к своим местам, доколе им не придти.
\vs 2Ba 51:12
Но тогда праведникам будет принадлежать преимущество перед ангелами.
\vs 2Ba 51:13
Ибо первые примут последних, которых они ждали, а последние тех, о ком они привыкли слышать, как о предваривших их.
\vs 2Ba 51:14
Они освободились от этого мира скорби и отложили груз страданий.
\vs 2Ba 51:15
Почему же те люди привели свою жизнь к погибели? На что они, бывшие на земле, променяли свою душу?
\vs 2Ba 51:16
Ибо они тогда предпочли выбрать время, которое не может пройти без скорби и исход которого в печали и зле. Они отвергли мир, в котором не стареют те, кто в нём пребывает; они презрели время славы. Поэтому они и не достигли чести, которую Я предсказал тебе.

\vs 2Ba 52:1
И я отвечал, говоря: Намного ли они заблудили, те, кому уготовано проклятие?
\vs 2Ba 52:2
И почему мы носим еще траур по тем, кто умер, и оплакиваем тех, кто идет в Шеол?
\vs 2Ba 52:3
Лучше отложить плач до начала кары и удержать слезы до пришествия погибели.
\vs 2Ba 52:4
Но я скажу пред лицем этого:
\vs 2Ba 52:5
что ныне будут делать праведные?
\vs 2Ba 52:6
Радуйтесь в страданиях, которые вы испытываете ныне: зачем вам ожидать падения ваших врагов?
\vs 2Ba 52:7
Готовьтесь к тому, что уготовано вам и расположите ваши души для награды, вам предназначенной.
\vs 2Ba 52:8
И сказав так, я заснул на этом месте.

\vs 2Ba 53:1
И мне было видение. Вот, из моря поднялось огромное облако. Я смотрел на него, и оно было полно водою, черною и светлою, и подобие сильной молнии было видно над его вершиной.
\vs 2Ba 53:2
Я увидел, как быстро и скоро это облако подошло и покрыло всю землю.
\vs 2Ba 53:3
И потом это облако пролило на землю воды, бывшие в нём.
\vs 2Ba 53:4
И я увидел, что воды, падая на землю, были непохожи одна на другую.
\vs 2Ba 53:5
Сперва они были все черными до времени [и более плотными]. Затем я увидел, что они стали светлыми, но не плотными. Затем я увидел их черными, потом светлыми, потом черными и снова светлыми.
\vs 2Ba 53:6
Так повторялось двенадцать раз, но всякий раз черные воды были более плотными, чем светлые.
\vs 2Ba 53:7
И вот облако истаяло, пролившись дождем черных вод, еще более темных, чем прежде и смешавшихся с огнем. И куда падали воды, там они причиняли опустошение и разрушение.
\vs 2Ba 53:8
И потом я увидел, как молния, которую я видел на вершине облака, схватила его и спустила на землю.
\vs 2Ba 53:9
Эта молния была настолько яркою, что осветила всю землю и возстановила те места, что разрушили, падая, последние воды.
\vs 2Ba 53:10
И она заняла всю землю и овладела ею.
\vs 2Ba 53:11
И после этого я увидел: вот, двенадцать рек поднялись из моря и окружили молнию и служили ей.
\vs 2Ba 53:12
и я проснулся, охваченный страхом.

\vs 2Ba 54:1
Молитва Варуха. И я взмолился к Шаддаи, говоря: Ты один, ЯХВЕ, знающий заранее глубины мира. И Ты повелеваешь, и в своё время бывает то или иное по Твоему слову. Против дел тех, кто живет на земле, Ты ускоряешь начала времён, и Ты один знаешь концы эпох.
\vs 2Ba 54:2
Ты, для Кого ничто не трудно, совершаешь всё с легкостью, одним мановением.
\vs 2Ba 54:3
Пропасти и высоты обращаются к Тебе, и начала веков повинуются Твоему слову.
\vs 2Ba 54:4
Ты открываешь боящимся Тебя то, что ожидает их; и так Ты утешаешь их.
\vs 2Ba 54:5
Ты показываешь великие деяния тем, кто не знает; Ты отнимаешь занавес перед невеждами; Ты освещаешь тьму, и Ты открываешь тайное непорочным, тем, кто в вере послушен Тебе и Твоему Закону.
\vs 2Ba 54:6
Ты показал это видение Твоему рабу; дай мне также и истолкование.
\vs 2Ba 54:7
Ибо я знаю истинно, что о тех вещах, о которых я взыскивал ответ у Тебя, Ты открывал мне, как и то, каким гласом хвалить Тебя и какими членами возносить похвалу и аллилуйю.
\vs 2Ba 54:8
Ибо если все мои члены были бы устами, и все волосы на моей голове имели бы голос, даже и тогда я не смог бы воздать Тебе хвалу и возславить Тебя как должно, и не смог бы произнести Тебе хваления, ни прославить величие Твоей красоты.
\vs 2Ba 54:9
Ибо что я среди людей? И чем мне сочетаться с теми, кто превосходит меня? Ведь я слышал столько чудесного из уст Элиона и безчисленные вести от Того, Кто создал меня.
\vs 2Ba 54:10
Блаженна мать моя среди зачинающих! Да прославится между женами та, что произвела меня на свет!
\vs 2Ba 54:11
Я же не перестану восхвалять Шаддаи; я поведаю Его чудеса гласом хваления.
\vs 2Ba 54:12
Ибо кто может сравниться с Тобою в чудесах, как Твои, о Боже? Кто изследует глубину Твоего помысла о жизни?
\vs 2Ba 54:13
Ибо Ты управляешь в Твоей премудрости всем сотворенным Твоею десницею. Ты поставил рядом с Собою всякий источник света, и Ты уготовал ниже Твоего престола хранилища премудрости.
\vs 2Ba 54:14
Праведно погибают те, кто не возлюбил Твой Закон, и мука осуждения ожидает непокорившихся Твоему могуществу.
\vs 2Ba 54:15
Ибо хотя первым согрешил Адам и навел смерть на всех, кого еще не было в его время, однако и каждый из рожденых от него приготовил для себя грядущую муку или же избрал вечную славу.
\vs 2Ba 54:16
Ибо поистине, кто верит, получит свою награду.
\vs 2Ba 54:17
Но вы, творящие нечестивое ныне, возвращайтесь в истление: вы будете тяжко наказаны за то, что некогда отвергли познание Элиона.
\vs 2Ba 54:18
Его деяния ничему не научили вас, ни неизменное искусство Его творения не убедило вас.
\vs 2Ba 54:19
Ибо Адам повинен только в своей душе, но каждый из нас Адам для своей собственной души.
\vs 2Ba 54:20
Но Ты, ЯХВЕ, объясни мне открытое Тобою; наставь меня в вопросах, которые я задал Тебе.
\vs 2Ba 54:21
Ибо в конце света будет возмездие нечестивым по их нечестию, и Ты прославишь тех, кто будет верен в Твоей вере.
\vs 2Ba 54:22
Ибо Ты ведешь тех, кто принадлежит Тебе, и Ты вырываешь грешников из среды Твоего владения.

\vs 2Ba 55:1
И когда я окончил эту молитву, я сел под дерево отдохнуть в тени его ветвей.
\vs 2Ba 55:2
Я был поражен и изумлен, возвращаясь мыслию к величию благости, которое грешники на земле удалили от себя, и к тяжести муки, которую они презирают, зная, что они будут мучаться за свои грехи.
\vs 2Ba 55:3
И когда я думал об этом, и о другом, подобном этому, вот ангел Ремиил, вождь истинных видений, был послан ко мне. И он сказал мне:
\vs 2Ba 55:4
Отчего волнуется твое сердце, Варух? Почему смущаются твои мысли?
\vs 2Ba 55:5
Если одно лишь предвозвещение суда так тревожит тебя, то что будет, когда ты увидишь его совершающимся открыто на твоих глазах?
\vs 2Ba 55:6
Если ты столь подавлен ожиданием Дня Шаддаи, то что будет, когда ты достигнешь его пришествия?
\vs 2Ba 55:7
Если слово, возвещающее казнь падших, так тяготит тебя, каково же будет, когда чудеса начнут сбываться?
\vs 2Ba 55:8
Если, услышав имя благ и зол, грядущих в то время, ты затрепетал, то что будет, когда ты увидишь то, что откроет Величие, Которое обвинит одних и возвеселит других?

\vs 2Ba 56:1
Однако, поскольку ты просил Элиона открыть тебе истолкование увиденного тобою видения, я был послан говорить с тобою.
\vs 2Ba 56:2
Ибо Шаддаи особо показал тебе последовательность времен, которые прошли и еще имеют пройти в мире от начала его сотворения и до его завершения: среди них иные ложь, и иные истина.
\vs 2Ba 56:3
Ты видел большое облако, поднявшееся из моря и затем покрывшее землю, которому подобна долгота века, созданного Шаддаи, когда Он решил сотворить мир.
\vs 2Ba 56:4
И когда слово вышло от Него, долгота века пришла в бытие, и была крайне малою, и устроилась согласно изобилию разума Пославшего её.
\vs 2Ba 56:5
Подобно черным водам, которые ты видел вначале наверху облака и которые первыми пали на землю, было прегрешение Адама, первого человека.
\vs 2Ba 56:6
Ибо когда он преступил, безвременно явилась смерть, скорбь получила имя, боль была уготована, страдание было создано, труд достиг своего предела. И гордыня начала утверждаться. Шеол потребовал своего обновления в крови и похищал детей. И страсть родителей была создана. Величие человечества было унижено, и доброта зачахла.
\vs 2Ba 56:7
Что могло оказаться чернее и мрачнее этого?
\vs 2Ba 56:8
Таково начало черных вод, которые ты видел.
\vs 2Ba 56:9
Из этих черных вод в свой черед рождались другие черные воды, и тьма явилась над тьмою.
\vs 2Ba 56:10
Тот, кто был опасен для своей души, стал опасностью и для ангелов.
\vs 2Ba 56:11
Ибо когда он был создан, те наслаждались свободою.
\vs 2Ba 56:12
И когда он стал опасен, некоторые из них сошли и смешались с женщинами.
\vs 2Ba 56:13
И те, кто сделал это, были мучимы в цепях.
\vs 2Ba 56:14
Но остаток неисчислимого множества ангелов уцелел.
\vs 2Ba 56:15
А те, кто населял землю, погибли вместе в водах потопа. Это первые черные воды.

\vs 2Ba 57:1
После этого ты видел светлые воды. Это корень Авраамов и его потомство, пришествие его сына, и сына его сына, и тех, кто подобен им.
\vs 2Ba 57:2
Ибо в их время среди них был призываем неписанный Закон и исполнялись заповеди. Тогда родилась вера в будущий суд, и утвердилось упование на обновление мира, и обетование грядущей жизни укоренилось в сердцах.
\vs 2Ba 57:3
Это светлые воды, которые ты видел.

\vs 2Ba 58:1
Черные воды, которые ты видел в-третьих смесь всех грехов, соделанных народами после смерти этих праведников: нечестие земли Мицрейской, зло, которое они совершили, обратив в рабство их сынов.
\vs 2Ba 58:2
И, однако, в конце они также погибли.

\vs 2Ba 59:1
Светлые воды, которые ты видел в-четвертых, это пришествие Мойсея, Аарона, Мариами, Иошуа, сына Нунова, Халева и всех, подобных им.
\vs 2Ba 59:2
В эти дни светоч Вечного Закона осветил всех, пребывавших во тьме, означая для уверовавших обетование награды и для отступников уготованную им огненную кару.
\vs 2Ba 59:3
Но в эти времена также небеса закрылись для всякой земли, и стоявшие ниже престола Шаддаи пошатнулись, когда Он принял к Себе Мойсея.
\vs 2Ba 59:4
Он показал ему, как и тебе, множество наставлений, и вместе с ними Закон и исполнение времён, а также образ и размеры Сиона, который будет построен по образу нынешнего Святилища.
\vs 2Ba 59:5
Он показал ему также меру огня и глубину бездны, вес ветров, число капель дождя,
\vs 2Ba 59:6
повеление гневом, изобилие долготерпения, праведность суда,
\vs 2Ba 59:7
корень премудрости, богатство разума, источник знания;
\vs 2Ba 59:8
высоту воздуха, величие Рая, конец веков, начало Судного Дня,
\vs 2Ba 59:9
число приношений земли, что еще не возникли,
\vs 2Ba 59:10
пасть геенны, союз с местью, место веры, жилище надежды,
\vs 2Ba 59:11
образ будущей муки, неисчисимое множество ангелов, мощь пламени, блеск молний, голос грома, перемены времён и глубокое изучение Закона.
\vs 2Ba 59:12
Это светлые воды, которые ты видел в-четвертых.

\vs 2Ba 60:1
Черные воды, которые ты видел нисшедшими дождем на мир в-пятых, это дела Аморреев и их колдовские взывания, нечестие их таинств и осквернение их нечистотою.
\vs 2Ba 60:2
Но Израиль также осквернился грехом в дни Судий, несмотря на то, что видел множество знамений, совершенных Тем, Кто создал его.

\vs 2Ba 61:1
Светлые воды, которые ты видел в-шестых, это время, в которое родились Давыд и Соломон.
\vs 2Ba 61:2
В это время был возведен Сион и было освящено Святилище, была пролита кровь многих народов, грешивших тогда, и много приношений было сделано во время освящения Святилища.
\vs 2Ba 61:3
И мир и покой были в это время, и премудрость была слышна в собрании,
\vs 2Ba 61:4
и богатые разумом были прославляемы на сходбищах.
\vs 2Ba 61:5
И святые празднества соблюдались с великим приличием и радостью.
\vs 2Ba 61:6
И суд правителей выносился без коварства, и праведность заповедей исполнялась в истине.
\vs 2Ba 61:7
И земля была тогда возлюбленною ЯХВЕ, и, поскольку жившие на ней не грешили, она была прославлена над всеми землями и странами.
\vs 2Ba 61:8
Это светлые воды, которые ты видел.

\vs 2Ba 62:1
Черные воды, которые ты видел в-седьмых, это извращенность духа Иеровоама, решившегося воздвигнуть двух золотых тельцов,
\vs 2Ba 62:2
а также всё нечестие, сотворённое бывшими после него царями,
\vs 2Ba 62:3
проклятие Иезавели и идолопоклонство Израиля в то время,
\vs 2Ba 62:4
засуха и голод, столь великие, что женщины ели плод чрева своего;
\vs 2Ba 62:5
наконец, время плена, наставшее для девяти с половиною колен, которые предавались множеству грехов.
\vs 2Ba 62:6
Тогда пришел Салманасар, царь Ассирийцев, и увел их в плен.
\vs 2Ba 62:7
Но и о народах говорить тягостно, столь велики всегда были их нечестие и лукавство, и никогда они не явили праведности.
\vs 2Ba 62:8
Это черные воды, которые ты видел в-седьмых.

\vs 2Ba 63:1
Светлые воды, которые ты видел в-восьмых, это праведность и правда Езекии, царя Иуды, и милость, которая была ему дарована,
\vs 2Ba 63:2
когда Синнахериб был увлечен к своей погибели и ослеплён гневом настолько, что повел на гибель великое множество бывших у него народов.
\vs 2Ba 63:3
Когда Езекия узнал о намерениях царя Ассура захватить его, погубить оставшиеся два с половиною колена, а также разрушить Сион, Езекия положился на свои дела и понадеялся на свою праведность. Он сказал такое слово к Шаддаи:
\vs 2Ba 63:4
Смотри, вот Синнахериб готов нас погубить. Он будет хвалиться и превозноситься, разрушив Сион.
\vs 2Ba 63:5
И Шаддаи услышал его, ибо Езекия был премудр, и Он внял его молитве, ибо Езекия был праведен.
\vs 2Ba 63:6
Шаддаи дал тогда повеление Своему ангелу Ремиилу, который говорит с тобою.
\vs 2Ba 63:7
И я пошел погубить множество, одних вождей которого было сто восемьдесят тысяч, и каждый из них имел столько же воинов.
\vs 2Ba 63:8
В это время я спалил их тела изнури, оставляя нетронутым всё снаружи, вплоть до одежды и оружия. И так более явными открылись великие деяния Шаддаи, Имя Которого произносится по всей земле.
\vs 2Ba 63:9
И Сион был спасен, и Иеросалим освобожден, и Израиль избавлен от тревоги.
\vs 2Ba 63:10
Все, бывшие в святой земле, возрадовались, и Имя Шаддаи было прославлено и наречено по всей земле.
\vs 2Ba 63:11
Это светлые воды, которые ты видел.

\vs 2Ba 64:1
Черные воды, которые ты видел в-девятых, это всеобщее нечестие во времена Манассии, сына Езекии.
\vs 2Ba 64:2
Он содеял зло в изобилии. Он убивал праведников, он извращал суды, он проливал невинную кровь, осквернял и насиловал замужних женщин, ниспроверг жертвенники, прекратил жертвоприношения и изгнал священников, дабы они не служили более в Святилище.
\vs 2Ba 64:3
Он поставил идола с пятью лицами. Четыре из них смотрели на четыре ветра; пятое же венчало идола, как противника ревности Шаддаи.
\vs 2Ba 64:4
Тогда гнев вышел от Шаддаи, сильный искоренить Сион, как это было в ваши дни.
\vs 2Ba 64:5
И против двух с половиною колен вышло определение быть уведёнными в плен в точности так, как ты видел.
\vs 2Ba 64:6
Нечестие Манассии достигло таких пределов, что Слава Шаддаи удалилась от Святилища.
\vs 2Ba 64:7
Вот почему Манассия был назван нечестивым и в конце концов его жилищем стал огонь.
\vs 2Ba 64:8
Ибо, хотя его молитва и достигла Шаддаи, когда он был брошен в бронзового коня, и бронзовый конь раскололся, и в тот час ему было знамение,
\vs 2Ba 64:9
он не жил в совершенстве и не оказался достоин. Но с этого дня он будет знать, Кем в конце концов он будет мучим.
\vs 2Ba 64:10
Ибо могущий благотворить, может и карать.

\vs 2Ba 65:1
Так Манассия творил злое и всю свою жизнь думал, что Шаддаи не взыщет ни за что.
\vs 2Ba 65:2
Это черные воды, которые ты видел в-девятых.

\vs 2Ba 66:1
Светлые воды, которые ты видел в-десятых, это чистота поколений Иосии, царя Иуды, единственного в свое время явившегося покорным Шаддаи всем сердцем своим и всею душею своей.
\vs 2Ba 66:2
Он очистил землю от идолов и освятил все сосуды, которые были осквернены. Он возобновил приношения на жертвенниках, он вознёс рог святых, возвысил праведных, прославил всех премудрых за их разум; он вернул священникам их служение; он искоренил и уничтожил с земли волхвов, прорицателей и гадателей.
\vs 2Ba 66:3
Он не только умертивил нечестивцев, которые жили, но и вынул из гробов кости умерших и предал их огню.
\vs 2Ba 66:4
Он устроил празднества и субботы в их святости. Он сжег огнём осквернившихся. И лжепророков, обманывавших народ, он также предал огню. И толпу, покорную тем, он сбросил живьем в поток Кедронский и собрал камни над ними.
\vs 2Ba 66:5
Всею душею он предался ревности Шаддаи. Единственный в свое время, он был тверд в Законе, не терпя никого необрезанного, ни одного, кто творил бы нечестивое по всей земле, во все дни своей жизни.
\vs 2Ba 66:6
Таков тот, кто стяжал вечную награду; он будет прославлен у Шаддаи над многими в последнее время.
\vs 2Ba 66:7
Для него и для подобных ему были созданы и уготованы прекраснейшие почести; они были указаны тебе прежде.
\vs 2Ba 66:8
Это светлые воды, которые ты видел.

\vs 2Ba 67:1
Черные воды, которые ты видел в-одиннадцатых, это бич, поразивший ныне Сион.
\vs 2Ba 67:2
Ты думаешь, ангелы не опечалены пред Элионом, видя Сион преданным так, на виду у народов, которые похваляются в своём сердце и собираются перед своими идолами, говоря: Вот, ныне попираема ногами та, что так часто попирала других и низвергнута в рабство покорявшая?
\vs 2Ba 67:3
Ты думаешь, Элион радуется этому, или что Его Имя прославляется от этого?
\vs 2Ba 67:4
Но что тогда было бы с Его праведным судом?
\vs 2Ba 67:5
Однако после этого волнение охватит тех, кто разсеян среди народов; на всяком месте они будут жить в позоре.
\vs 2Ba 67:6
Ибо в то время, когда Сион предан, Иеросалим разрушен, а идолы благоденствуют в городах народов, ароматный дым благовоний затух на Сионе, и вот, повсюду в области Сионской возносится дым нечестия.
\vs 2Ba 67:7
И еще также царь Вавилонский, разрушивший ныне Сион, возвысится; он будет похваляться над народом. В своём сердце он произнесёт слова гордыни пред Элионом.
\vs 2Ba 67:8
Но и он падёт в конце.
\vs 2Ba 67:9
Это черные воды.

\vs 2Ba 68:1
Светлые воды, которые ты видел двенадцатыми, вот слово о них.
\vs 2Ba 68:2
Настанет время, когда на твой народ найдет столь великое испытание, что он едва не погибнет весь сразу.
\vs 2Ba 68:3
Однако, он будет спасен, и его враги падут перед ним.
\vs 2Ba 68:4
Некоторое время он пребудет в великой радости.
\vs 2Ba 68:5
В это время после краткого промежутка Сион будет отстроен вновь; и в нём снова принесут жертву, священники вернутся к своему служению, и народы придут воздать ему почести, но не так единодушно, как в первые времена.
\vs 2Ba 68:7
Но после этого падение поразит многие народы.
\vs 2Ba 68:8
Это светлые воды, которые ты видел.

\vs 2Ba 69:1
Последние воды, которые ты видел, более черные, чем все предыдущие, пришедшие после первых двенадцати вместе взятых, относятся ко всему миру.
\vs 2Ba 69:2
Ибо Элион отделил их с самого начала, и Он один знает, что будет.
\vs 2Ba 69:3
Он предвидел, что нечестивые злодейства пройдут пред Ним в шести образах, не считая того, что совершится в конце света.
\vs 2Ba 69:5
Вот почему Он не примешал черные воды к черным водам и светлые воды к светлым. Ибо это конец.

\vs 2Ba 70:1
Выслушай значение последних черных вод, которым надлежит придти после черных же, вот слово о них.
\vs 2Ba 70:2
Вот, грядут дни, и когда время достигнет зрелости, и придет пора собирать урожай злых и добрых семян. Элион приведет на землю, на живущих на земле и на её правителей, смятение духа и оцепенение сердца.
\vs 2Ba 70:3
Они будут ненавидеть друг друга и вызывать друг друга на бой, и низкие будут подчинять себе достойных, презренные возвысятся над почтенными.
\vs 2Ba 70:4
Великое множество будет предано малому числу. Бывшие ничем подчинят себе сильных, бедные превзойдут богатых в изобилии, нечестивые возобладают над героями.
\vs 2Ba 70:5
Мудрецы умолкнут, и глупцы будут говорить. Ни мысли человеческие, ни совет могучих не упрочатся, ни надежда уповающих не утвердится.
\vs 2Ba 70:6
Когда придет то, что предсказано, смятение распространится среди людей: одни падут на войне, другие умрут в скорбях, иные будут обманом захвачены своими ближними.
\vs 2Ba 70:7
Элион откроет народам уготованное Им заранее, и они придут сразиться с вождями, которые уцелеют тогда.
\vs 2Ba 70:8
И тот, кто убежит от войны, погибнет от мятежа; кого пощадит мятеж, тот сгорит в огне, а убежавший от огня, погибнет от голода.
\vs 2Ba 70:9
Те же, кто в конце выйдут целыми и невредимыми из предсказанных бед, победители и побежденные, будут живыми преданы в руки Моего Мессии. Ибо вся земля пожрет живущих на ней.

\vs 2Ba 71:1
Но святая земля помилует своих и пощадит живущих на ней в это время.
\vs 2Ba 71:2
Вот виденное тобою видение, и вот его истолкование.
\vs 2Ba 71:3
Я же пришел рассказать тебе это, ибо твоя молитва услышана ЯХВЕ.

\vs 2Ba 72:1
Выслушай еще о светлых водах, которые появятся в конце после этих черных вод.
\vs 2Ba 72:2
Как только явятся преждереченные знамения, народы смятутся, и придёт Мой Мессия. Он созовёт все народы, из них же иные спасет, а иные погубит.
\vs 2Ba 72:3
Вот что будет для народов, которые Он спасёт.
\vs 2Ba 72:4
Всякий народ, не познавший Израиля и не попиравший ногами семя Иакова, будет спасён.
\vs 2Ba 72:5
Так будет потому, что среди всех народов они повиновались твоему народу.
\vs 2Ba 72:6
Но те, кто правил вами или кто познал вас, будут преданы мечу.

\vs 2Ba 73:1
И когда Он унизит весь мир и возсядет в мире навеки на свой царственный трон, тогда откроется радость, и явится спокойствие;
\vs 2Ba 73:2
тогда здоровье снизойдет как роса, и болезнь удалится. Заботы, скорби и стоны уйдут далеко от людей; радость распространится по всей земле.
\vs 2Ba 73:3
Не будут более умирать преждевременно; никакая беда не поразит внезапно.
\vs 2Ba 73:4
Суды, обвинения, борьба, месть, преступление, ревность, ненависть и всё, что подобно им, будет выкорчевано и пойдёт получать своё осуждение.
\vs 2Ba 73:5
Ибо это они исполнили землю злом, и через них жизнь человеческая была весьма поколеблена.
\vs 2Ba 73:6
Дикие звери выйдут из леса и будут служить людям. Змеи и драконы вылезут из своих нор и будут послушны младенцу.
\vs 2Ba 73:7
Женщины не будут больше страдать вынашивая и не будут мучаться, давая жизнь плоду чрева своего.

\vs 2Ba 74:1
В эти дни жнецы не будут более знать усталости, ни утомления строители. Работа будет совершаться сама собою, скоро и спокойно, в согласии с теми, кто трудится.
\vs 2Ba 74:2
Ибо в это время прекратится тлен и начнется нетление.
\vs 2Ba 74:3
Поэтому исполнение предсказанного совершится в нём, и поэтому оно удалено от злых и совсем близко для тех, кто никогда не умрет.
\vs 2Ba 74:4
Это последние светлые воды, явившиеся за последними черными водами.

\vs 2Ba 75:1
И я отвечал, говоря: Кто сравнится с Тобою в благости, ЯХВЕ? Ибо она непревосходима.
\vs 2Ba 75:2
Кто изследует Твоё безконечное милосердие?
\vs 2Ba 75:3
Или кто постигнет Твой разум?
\vs 2Ba 75:4
Или кто может поведать мысли Твоего ума?
\vs 2Ba 75:5
Есть ли один среди рождённых, кто может надеяться на это, кроме того, над кем Ты сжалишься и кому Ты покровительствуешь?
\vs 2Ba 75:6
Не будь этой жалости Твоей к людям, которых прикрывает Твоя десница, они не смогут этого, кроме тех, кто назван в числе избранных по имени.
\vs 2Ba 75:7
Но мы, живые, если мы знаем причину, по которой мы пришли, и если мы повинуемся Тому, Кто вывел нас из Мицры, при нашем возвращении мы вспомним о прошлом и возрадуемся тому, что было.
\vs 2Ba 75:8
Если же, напротив, мы не знаем, почему мы пришли и не признаём власть Того, Кто поднял нас из Мицры, при нашем возвращении мы будем сожалеть о том, что происходит ныне и мы будем тяжко страдать о прошлом.

\vs 2Ba 76:1
Он отвечал мне, говоря: Поскольку по твоей молитве тебе изъяснилось откровение этого видения, выслушай слова Шаддаи и узнай, чтo будет после этих событий.
\vs 2Ba 76:2
Ты покинешь эту землю, но не умрешь, а будешь сохранен до [скончания] времён.
\vs 2Ba 76:3
Взойди же на вершину этой горы. Все места земли пройдут перед тобою: образ мира, вершины гор, глубины долин, бездны моря и великое множество рек. Так ты увидишь то, что ты оставляешь и то, куда ты идешь.
\vs 2Ba 76:4
Это случится через сорок дней.
\vs 2Ba 76:5
Теперь же, в эти дни, иди и научи народ, насколько ты сможешь, дабы они знали, что они не умрут в эти последние времена, но что они доживут до скончания времён.

\vs 2Ba 77:1
И я, Варух, оставил это место и ушел к народу. Я собрал их всех, от самого великого до самого малого, и сказал им:
\vs 2Ba 77:2
Слушайте, сыны Израиля, посмотрите, сколько вас уцелело из двенадцати колен Израилевых.
\vs 2Ba 77:3
Вам и вашим отцам ЯХВЕ дал Закон преимущественно перед всеми народами.
\vs 2Ba 77:4
Но поскольку ваши братья преступили заповеди Элиона, Он навлёк на вас, как и на них, отмщение. Он не пощадил первых, но последних Он также отдал в плен и не оставил остатка.
\vs 2Ba 77:5
И вот вы здесь, на этом месте со мною.
\vs 2Ba 77:6
Если вы исправите ваши пути, вы не уйдёте, как ушли ваши братья; скорее, они вернутся к вам.
\vs 2Ba 77:7
Ибо Он милосерд, Тот, Кому вы служите; Он Покровитель, Тот, на Кого вы надеетесь; Он верен, творящий доброе, а не злое.
\vs 2Ba 77:8
Не видели ли вы, что было с Сионом?
\vs 2Ba 77:9
Не думаете ли вы, что это место было разграблено потому, что оно согрешило, и земля сотворила неразумное и за это была предана?
\vs 2Ba 77:10
Не знаете ли вы, что из-за вас, грешники, она была разграблена и из-за нечестивых она была предана врагам, не сотворившая неразумного?
\vs 2Ba 77:11
И весь народ отвечал, говоря: Насколько мы в силах, мы помним благодеяния, оказанные нам Элионом. Те же, о которых мы забыли, Он помнит в Своем милосердии.
\vs 2Ba 77:12
Однако сделай милость для нас, твоего народа. Напиши и нашим братьям в Вавилон письмо учения, свиток надежды, дабы утвердить их, прежде, чем ты покинешь нас.
\vs 2Ba 77:13
Ибо пастыри Израиля погибли, и светочи, просвещавшие его, погасли. Изсяк источник, из которого мы пили. Мы остались во тьме, в густом лесу, в жажду пустыни.
\vs 2Ba 77:15
И я отвечал им, говоря: Пастыри, светочи и источники происходят от Закона. И если мы преходим, то Закон пребывает.
\vs 2Ba 77:16
Если разсматривая Закон, вы соблюдаете благоразумие в премудрости, светоч не отнимется от вас, ни пастырь не падёт, ни источник не изсякнет.
\vs 2Ba 77:17
Всё же я напишу вашим братьям в Вавилон, как вы просили меня, и пошлю с людьми, и напишу даже и девяти с половиною коленам и отправлю к ним с птицею.
\vs 2Ba 77:18
В двадцать восьмой день восьмого месяца я, Варух, пришел и сел под дубом в тени его ветвей. Никого не было со мною; я был один.
\vs 2Ba 77:19
И я написал эти два письма. Я отправил одно из них с орлом девяти с половиною коленам, и я послал другое с тремя людьми тем, кто был в Вавилоне.
\vs 2Ba 77:20
Я призвал орла и сказал ему такие слова:
\vs 2Ba 77:21
Элион создал тебя над прочими птицами.
\vs 2Ba 77:22
Ныне лети, не садясь нигде, ни в твоём гнезде и ни на каком дереве, прежде чем пролетишь над широкими водами Ефрата. Лети к народу, что проживает там. Оставь им это письмо.
\vs 2Ba 77:23
И помни, что во время потопа оливковую ветвь Ною принес голубь, которого он трижды выпускал из ковчега.
\vs 2Ba 77:24
Также и вoроны служили Илии, принося ему пищу, как он им приказывал.
\vs 2Ba 77:25
Даже Соломон во время своего царствования, когда хотел передать послание или спросить что-нибудь, посылал птицу, и та повиновалась ему неукоснительно.
\vs 2Ba 77:26
Теперь же не противься, не отклоняйся ни направо, ни налево. Но лети самым прямым путем и так исполни повеление Шаддаи, как я сказал тебе.

\vs 2Ba 78:1
Письмо, которое Варух, сын Нерии написал девяти с половиною коленам. Вот слова письма, которое Варух, сын Нерии, послал девяти с половиною коленам в изгнании за рекою, в котором было написано:
\vs 2Ba 78:2
Так говорит Варух, сын Нерии, братьям, уведённым в плен: милосердие и мир да будут с вами.
\vs 2Ba 78:3
Я храню в памяти любовь Создавшего нас, Возлюбившего нас искони; никогда мы не были ненавистны Ему, но превыше всего Он наставлял нас.
\vs 2Ba 78:4
Я истинно знаю, что мы, все двенадцать колен, связаны одними узами, так же как мы рождены от одного отца.
\vs 2Ba 78:5
Поэтому я весьма забочусь о том, чтобы оставить вам слова этого послания, прежду чем я умру: будьте утешены во зле, которое выпало вам, испытайте новую скорбь пред лицем испытаний, поразивших ваших братьев и воздайте справедливость приговору Того, Кто осудил вас на плен. Ибо то, что вы вынесли ничто по сравнению с тем, что вы сделали, лишь бы в последние времена вам оказаться достойными ваших отцов.
\vs 2Ba 78:6
Вот почему, если вы размыслите о том, что претерпели ныне для вашего блага, дабы не подвергнуться конечному осуждению и каре, вы обретёте вечное в надежде, если только вы упраздните из ваших сердец тщетное заблуждение, из-за которого вам пришлось уйти отсюда.
\vs 2Ba 78:7
Ибо если вы так сделаете, Он вспомнит верно о вас, Тот, Кто искони обещал ради нас лучшим, чем мы, никогда не забывать и не оставлять наше семя, но в Своём великом милосердии собрать вновь разсеянных.

\vs 2Ba 79:1
Теперь, братья мои, узнайте, чтo случилось с Сионом. Навуходресар, царь Вавилона, поднялся против нас.
\vs 2Ba 79:2
Поистине мы согрешили против Того, Кто создал нас, и не сохранили заповеди, которые Он дал нам. Но Он не наказал нас так, как мы заслужили.
\vs 2Ba 79:3
Ибо выпавшее вам претерпели и мы, и сильнее, ибо это же выпало и нам.

\vs 2Ba 80:1
И ныне, братья, я открываю вам, что когда город был окружен врагами, ангелы Элиона были посланы низвергнуть его стены и укрепления и уничтожить прочные углы стен, которые были выдернуты с корнем.
\vs 2Ba 80:2
Однако, они скрыли священные сосуды, дабы уберечь их от вражеского осквернения.
\vs 2Ba 80:3
Совершив это, они отдали врагам скрытые укрепления, Дом ограбленный, Храм сожженный и народ побеждённый, ибо отданный. И так враги не смогли сказать в своей гордыне: Наша мощь была такова, что мы разрушили войною самый Дом Элиона.
\vs 2Ba 80:4
И ваших братьев они также заковали в цепи и отвели их в Вавилон и поселили там.
\vs 2Ba 80:5
Нас осталось здесь весьма мало.
\vs 2Ba 80:6
Вот потрясение, о котором я пишу вам.
\vs 2Ba 80:7
Ибо я знаю поистине, каким утешением было для вас то, что Сион был заселен, и знание о том, что он благоденствует, было для вас более печали, которую вы испытывали далеко от него.

\vs 2Ba 81:1
Но выслушайте и слово облегчения.
\vs 2Ba 81:2
Ибо я оплакивал Сион и просил о милосердии Шаддаи, говоря:
\vs 2Ba 81:3
Будет ли это для нас безконечно длительным? Навечно ли постигли нас эти беды?
\vs 2Ba 81:4
И Шаддаи соделал в изобилии Своего милосердия, Элион по величию Своей благости. Он открыл мне слово для утешения, Он показал мне ведение, дабы не печалиться больше. Он явил мне таинства времён, и Он дал мне знать приходы эпох.

\vs 2Ba 82:1
Итак, братья, я написал вам, как обрести утешение во множестве ваших скорбей.
\vs 2Ba 82:2
Знайте, что Создавший нас отмстит за нас всем врагам по тому, что они нам сделали. И особенно знайте, что конец, дело Шаддаи, близок, так же, как и грядущее милосердие. Исполнение Его суда недалеко.
\vs 2Ba 82:3
Ибо мы стоим ныне при изобилии и процветании народов, в то время как они творят нечестивое.
\vs 2Ba 82:4
Мы видим величие их мощи и вместе их гнусности, но они будут подобны капле.
\vs 2Ba 82:5
Мы видим уверенность их силы и их непрестающее противление Шаддаи, но они будут сочтены за плевок.
\vs 2Ba 82:6
Мы думаем о славе их величия, в то время как они не соблюдают заповеди Элиона, и однако они пройдут как дым.
\vs 2Ba 82:7
Мы размышляем о блеске их красоты, тогда как они живут в скверне, но они изсохнут, как увядшая трава.
\vs 2Ba 82:8
Мы глядим на их силу и жестокость, хотя они не вспоминают о конце, но они исчезнут, как пробежавшая волна.
\vs 2Ba 82:9
Мы ясно видим гордыню их мощи, но они отвергают милость Того, Кто дал им эту мощь, и как проходит облако, пройдут и они.

\vs 2Ba 83:1
Ибо Элион ускорит явственно Своё время, и наведёт Свой век,
\vs 2Ba 83:2
и Он будет судить всех, кто есть в Его мире, Он поистине посетит всё и вся, потому как все их дела греховны.
\vs 2Ba 83:3
Он изследует сокровенные мысли и всё, что хранится в тайниках всех членов человеческих, и Он явит всё это перед всеми во время исправления.
\vs 2Ba 83:4
Пусть ничто из этого нынешнего не всходит вам на сердце, но будем ждать, ибо обетованное грядет.
\vs 2Ba 83:5
Не будем смотреть на нынешнее наслаждение народов, но будем помнить о том, что нам обетовано в конце.
\vs 2Ba 83:6
Ибо рубежи времен преходят, эпохи и всё, что в них, вместе.
\vs 2Ba 83:7
Конец мира откроет великое могущество Того, Кто управляет им, тогда как всякая вещь идет на суд.
\vs 2Ba 83:8
Вы же утверждайте ваши сердца в ожидании того, во что вы верили искони, дабы не оказаться вам далеко от обоих миров: здесь вы были уведены в плен, и будете мучимы там.
\vs 2Ba 83:9
Во всём, что есть ныне, во всём, что прошло, во всём, что будет, во всём этом ни зло не есть полностью зло, ни добро не есть полностью добро.
\vs 2Ba 83:10
Ибо всё, что ныне есть здоровье, становится болезнью,
\vs 2Ba 83:11
всякая сила ныне становится слабостью, всякая уверенность сегодня обращается в ничтожество.
\vs 2Ba 83:12
Крепость молодости обращается в старость и истление. Красота, ныне блестящая, увядает и делается отвратительною.
\vs 2Ba 83:13
Набухшая гордыня скоро обращается в уничижение и позор.
\vs 2Ba 83:14
Слово всякого превозношения ныне превращается в смятение и немоту, всякая нынешняя похвальба и наглость ведут к падению и нищете.
\vs 2Ba 83:15
Всякое нынешнее удовольствие и всякое удовлетворение превращаются в червей и тлен.
\vs 2Ba 83:16
Крики гордыни изменяются в прах и безмолвие.
\vs 2Ba 83:17
Всякое приобретение богатств ныне возвращает в одиночестве к Шеолу.
\vs 2Ba 83:18
Всё похищенное из жадности ныне ведет к нежеланной смерти, всякое вожделение приводит к осуждению на муку.
\vs 2Ba 83:19
Всякое лукавое притворство будет призвано на суд во имя Истины.
\vs 2Ba 83:20
Всякая нынешняя сладость умащений уступает место суду и осуждению.
\vs 2Ba 83:21
Всякая притворная дружба падёт в безмолвный стыд.
\vs 2Ba 83:22
И неужто вы думаете, что всё то, что творится у нас на глазах, останется без отмщения?
\vs 2Ba 83:23
Всякая вещь, достигая своего исполнения, приводит к Истине.

\vs 2Ba 84:1
Я же учил вас этому, пока я жил. Я сказал вам познавать заповеди Шаддаи, которым я наставлял вас, и прежде чем умереть, я представлю перед вашими глазами некоторые из заповедей Его судилища.
\vs 2Ba 84:2
Помните, что некогда Мойсей взял небо и землю в свидетели против вас: Если вы нарушите Закон, вы будете разсеяны, и если вы будете его соблюдать, вы укрепитесь.
\vs 2Ba 84:3
И он сказал вам и другие слова, когда вы были двенадцатью коленами, собранными вместе в пустыне.
\vs 2Ba 84:4
А после его смерти вы отвергли их. И так пророчества исполнились против вас.
\vs 2Ba 84:5
То, что Мойсей объяснил вам некогда еще до того, как это сбылось, вот, совершилось теперь, ибо вы оставили Закон.
\vs 2Ba 84:6
Я тоже говорю вам, после ваших страданий, что если вы уверуете в то, что сказано вам, вы получите от Шаддаи всё, что отложено и сохранено для вас.
\vs 2Ba 84:7
Пусть это письмо будет свидетелем между мною и вами, дабы вы помнили о заповедях Шаддаи. И так я смогу оправдаться перед Пославшим меня.
\vs 2Ba 84:8
Помните о Законе, о Сионе, а также о святой земле и ваших братьях. Не забывайте Завета, заключенного с вашими отцами, праздников и суббот.
\vs 2Ba 84:9
Передавайте это письмо и предания Закона вашим детям после вас, как вам их передали ваши отцы.
\vs 2Ba 84:10
Во всякое время верно взывайте и ревностно молитесь всею душею, чтобы Шаддаи сжалился над вами и не смотрел на множество ваших грехов, но, напротив, вспомнил о праведности ваших отцов. Ибо если Он не станет судить нас по изобилию Своего милосердия, горе нам, рожденным.

\vs 2Ba 85:1
Затем знайте, что в прежние времена и в прежние роды наши отцы имели поддержкою праведников и святых пророков.
\vs 2Ba 85:2
Но и мы были на нашей земле, и они помогали нам, когда мы грешили. Они молились за нас Создавшему нас, ибо они полагались на свои дела. Шаддаи внимал им и являл нам Свою милость.
\vs 2Ba 85:4
Но теперь праведники умерли, пророки упокоились и мы также покинули нашу землю; Сион был отнят у нас. У нас есть лишь Шаддаи и его Закон.
\vs 2Ba 85:4
Если же мы исправимся и настроим наши сердца, мы вновь обретём то, что мы утратили, и намного более, гораздо более того, что мы утратили.
\vs 2Ba 85:5
Ибо то, что мы утратили, было подвержено истлению; то, что мы получим, нетленно.
\vs 2Ba 85:6
И такие же слова я написал нашим братьям в Вавилон, чтобы засвидетельствать им всё это.
\vs 2Ba 85:7
Пусть все эти предсказания остаются перед вашими глазами на всякое время, ибо до сих пор мы живы и владеем нашей свободой.
\vs 2Ba 85:8
Шаддаи также терпелив к нам; Он открыл нам будущее и не скрыл грядущего в конце.
\vs 2Ba 85:9
И прежде чем Его суд не потребовал своего, и истина должного ей по праву, мы приготовим наши души к тому, чтобы принять и не быть отнятыми, уповать и не быть посрамлёнными, упокоиться вместе с нашими отцами и не быть мучимыми вместе с теми, кто нас ненавидит.
\vs 2Ba 85:10
Ибо юность мира прошла; сила творения ныне потребилась. Мало что остается еще до свершения проходящих времен. Кувшин у колодца, корабль близ гавани. Дорога завершается у города, а жизнь приближается к концу.
\vs 2Ba 85:11
Снова приготовьте ваши души к тому, чтобы, окончив плавание и сойдя с корабля, вам отдохнуть, и, прибыв на место, вам не оказаться осуждёнными.
\vs 2Ba 85:12
Ибо вот, Элион наведёт эти события. Тогда не будет больше места покаянию, рубежей времени, долготы веков, перемен к облегчению. Не будет более места мольбе, дара любви, [покаяния душе], ни ходатайства за прегрешения, ни умолений отцов, ни молитв пророков, ни помощи от праведных.
\vs 2Ba 85:13
Но будет только приговор к истязанию, путь в огонь, стезя в пекло.
\vs 2Ba 85:14
Вот почему един Закон, данный Единым, и един мир. И для всех, кто в нём, настаёт конец.
\vs 2Ba 85:15
Тогда Он спасёт тех, кого Он обретёт, и Он простит их. И тогда же Он погубит тех, кто оскверняется грехом.

\vs 2Ba 86:1
И когда вы получите письмо, читайте его прилежно в синагогах
\vs 2Ba 86:2
и размышляйте над ним, главным образом, в дни поста.
\vs 2Ba 86:3
И вспоминайте обо мне ради этого письма, так же как и я держу вас в памяти сейчас, когда и пишу, и каждую минуту.

\vs 2Ba 87:1
И когда я исполнил все слова этого письма и внимательно дописал его до конца, я свернул его, тщательно запечатал и укрепил на шее орла. Я освободил его и отправил.
\chhdr{Конец книги Варуха, сына Нерии.}

\bibbookdescr{3Ba}{
  inline={Третья Книга Пророка Варуха\fns{Переведена с греческого.}},
  toc={3-я Варуха},
  bookmark={3-я Варуха},
  header={3-я Варуха},
  abbr={3~Вар}
}
\vs 3Ba 1:1
Откровение Варуха, который стал у реки Гел, плача о пленении Иерусалима, когда Авимелех сохранен был рукой Божией в садах Агриппы. И так сидел он у Красных дверей, где пребывало Святое святых. Я, Варух, плакал в помышлении моем о народе, как же позволил Бог царю Навуходресару разрушить город Его, говоря:
\vs 3Ba 1:2
"Яхве, зачем выжег Ты виноградник Твой и опустошил его? Зачем сделал Ты это? И зачем, Яхве, не воздал Ты нам другим наказанием, но предал нас язычникам, чтобы надругались они, говоря: Где Бог их?"
\vs 3Ba 1:3
И вот, когда плакал я и говорил это, вижу я, Ангел Яхве пришел и говорит мне:
\vs 3Ba 1:4
"Слушай, муж желаний, не тревожься так о спасении Иеросалима, ибо вот что говорит Яхве, Бог Вседержитель: послал Он меня пред лице твое, чтобы возвестил я и явил тебе все Божественное, ибо молитва твоя услышана пред Ним и вошла в уши Адонаи Яхве".
\vs 3Ba 1:5
И когда он сказал мне это, успокоился я.
\vs 3Ba 1:6
И говорит мне Ангел: "Перестань раздражать Бога, и покажу я тебе другие тайны, большие этих".
\vs 3Ba 1:7
И сказал я, Варух: "Жив Адонаи Яхве, если покажешь мне и услышу я слова твои, уже не буду я больше говорить; да умножит Бог в день суда суд надо мною, если скажу что-либо впредь".
\vs 3Ba 1:8
И сказал мне Ангел сил: "Идем, покажу я тебе тайны Божии".

\vs 3Ba 2:1
И взяв меня, отнес он меня туда, где утверждено небо и где была река, которую никому не пересечь ни одному странствующему дуновению, из всех, что создал Бог.
\vs 3Ba 2:2
И взяв меня, отнес он меня к первому небу и показал мне превеликие врата. И сказал мне: "Войдем через них".
\vs 3Ba 2:3
И вошли мы словно на крыльях, преодолев расстояние примерно в тридцать дней пути.
\vs 3Ba 2:4
И показал он мне равнину, бывшую внутри этого неба, и были люди, жившие на ней: лица бычьи, рога оленьи, ноги козьи, а чресла бараньи.
\vs 3Ba 2:5
И вопросил я, Варух, Ангела: "Возвести мне, прошу тебя, какова толщина неба, где мы держим путь, и каково расстояние его от земли и что это за равнина, чтобы и я мог возвестить это сынам человеческим".
\vs 3Ba 2:6
И сказал мне Ангел, которому имя было Фамаил: "Врата, которые видишь ты, ведут на небо, и сколько от земли до него, такова и толщина его, и каково расстояние от севера до юга, такова длина равнины, которую ты увидел".
\vs 3Ba 2:7
И снова говорит мне Ангел сил: "Се, покажу я тебе и большие тайны".
\vs 3Ba 2:8
Сказал же я: "Прошу тебя, объясни мне, что это за люди?"
\vs 3Ba 2:9
И сказал он мне: "Это те, кто построили богопротивную башню, и за то удалил их Яхве".

\vs 3Ba 3:1
И взяв меня, отнес меня Ангел Яхве ко второму небу.
\vs 3Ba 3:2
И показал он мне и там врата, подобные первым, и сказал: "Войдем через них".
\vs 3Ba 3:3
И вошли мы, поднятые на крыльях, преодолев расстояние в шестьдесят дней пути, и показал мне он там равнину, и была она полна людей, видом же они походили на собак, а ноги оленьи.
\vs 3Ba 3:4
И вопросил я Ангела: "Прошу тебя, господин, скажи мне, кто эти люди?"
\vs 3Ba 3:5
И сказал он: "Это те, кто дали совет построить башню.
\vs 3Ba 3:6
Сами они, кого ты видишь, выгнали множество мужчин и женщин для изготовления кирпичей.
\vs 3Ba 3:7
Женщине одной, делавшей кирпичи, когда пришло ей время родить, не позволили они уйти, но, делая кирпичи, родила она и ребенка своего носила в полотенце, и делала кирпичи.
\vs 3Ba 3:8
И явившись им, Яхве изменил языки их, когда башня достигала высоты в триста шестьдесят три локтя.
\vs 3Ba 3:9
И взяв бурав, стали они стараться пробуравить небо, говоря: "Посмотрим, глиняное небо, медное или железное".
\vs 3Ba 3:10
Увидев это, Бог не позволил им, но поразил их слепотой и разноязычием и оставил их как ты их видишь".

\vs 3Ba 4:1
И сказал я, Варух: "Се, господин, великое и чудесное показал ты мне. И сейчас покажи мне все ради Яхве".
\vs 3Ba 4:2
И сказал мне Ангел: "Отправимся дальше".
\vs 3Ba 4:3
И отправился я с Ангелом дальше от места этого примерно на сто восемьдесят пять дней пути, и показал он мне равнину и змея, длиной, как мне показалось, около двух сотен плетров.
\vs 3Ba 4:4
И показал он мне ад, и вид его был мрачный и непотребный.
\vs 3Ba 4:5
И сказал я: "Что это за дракон, и что за дикость вокруг него?"
\vs 3Ba 4:6
И сказал Ангел: "Дракон этот есть пожирающий тела живущих неправедной жизнью, ими же он питается, то же, что вокруг него ад, который сам подобен ему, где он пьет из моря примерно с локоть, а воды нисколько не убывает".
\vs 3Ba 4:7
Сказал Варух: "Как же так?"
\vs 3Ba 4:8
И сказал Ангел: "Слушай: Яхве Бог сотворил триста шестьдесят рек, из которых из всех первые Алфей, Авир и Гирик. И, беря от них, не убывает вода в море".

\vs 3Ba 5:1
И сказал я, Варух, Ангелу: "Спрошу я тебя об одном, господин: когда уж сказал ты мне, что выпивает дракон из моря локоть, скажи мне: какова глубина чрева его?"
\vs 3Ba 5:2
И сказал Ангел: "Чрево его ад, и сколько пролетает свинец, пущенный тремястами мужами, таково и чрево его. Идем, и я покажу тебе дела еще большие этих".

\vs 3Ba 6:1
И взяв меня, отнес он меня туда, где начинает свой путь Солнце.
\vs 3Ba 6:2
И показал он мне колесницу с четверной упряжью, и вырывался из-под нее огонь, и сидел на колеснице муж в огненном венце, и влекли ту колесницу сорок Ангелов.
\vs 3Ba 6:3
И вот, впереди Солнца кружила птица величиной с девять гор.
\vs 3Ba 6:4
И сказал я Ангелу: "Что это за птица?" И говорит он мне: "Она хранитель вселенной".
\vs 3Ba 6:5
И сказал я: "Господин, как это хранитель вселенной? Объясни мне".
\vs 3Ba 6:6
И сказал мне Ангел: "Птица эта летит вместе с Солнцем и, раскинув крылья, принимает лучи его, которые подобны языкам пламени. И если бы не принимала она их, не уцелел бы род человеческий, и вообще ничто живое, но приставил Бог эту птицу".
\vs 3Ba 6:7
И раскинула она крылья свои, и увидел я на правом крыле ее буквы весьма великие, каждая словно гумно, величиной около четырех тысяч модиев, и были те буквы золотые.
\vs 3Ba 6:8
И сказал мне Ангел: "Прочти их".
\vs 3Ba 6:9
И прочел я, и гласили они: "Не земля рождает меня и не небо, а рождают меня крылья огненные".
\vs 3Ba 6:10
И сказал я: "Господин, что это за птица, и как имя ее?" И сказал мне Ангел: "Феникс имя ее".
\vs 3Ba 6:11
И сказал я: "А что ест она?" И сказал он мне: "Манну небесную и росу земную".
\vs 3Ba 6:12
И сказал я: "Испражняется ли птица эта?" И сказал он мне: "Испражняется червяком, а из испражнений червяка получается корица та, что употребляют цари и правители. Но помедли, и увидишь славу Божию".
\vs 3Ba 6:13
И пока говорил он, послышался словно бы раскат грома, и поколебалось место, на котором мы стояли.
\vs 3Ba 6:14
И вопросил я Ангела: "Господин мой, что это за шум?" И сказал мне Ангел: "Сейчас открывают Ангелы триста шестьдесят пять врат небесных, и выходит через них свет из тьмы".
\vs 3Ba 6:15
И пришел голос, говорящий: "Податель света, дай миру свет!"
\vs 3Ba 6:16
И услышав звук, издаваемый птицей, сказал я: "Господин, что это за звук?"
\vs 3Ba 6:17
И сказал он: "Звук, который пробуждает на земле петухов, ибо петух, подобно вторым устам, оповещает мир о наступлении утра своей песней. Ангелы приготовили Солнце вот и кричит петух".

\vs 3Ba 7:1
И сказал я: "А где пребывает Солнце после того, как кричит петух?"
\vs 3Ba 7:2
И сказал мне Ангел: "Слушай, Варух: все, что показал я тебе, находится на первом и втором небе. И проходит Солнце по третьему небу, и дает миру свет. Но подожди, и увидишь славу Божию".
\vs 3Ba 7:3
И в тот самый миг, когда говорил он, вижу я птицу, и вновь появилась она предо мною, и мало-помалу увеличивалась она и вырастала, а позади нее блистающее Солнце и с ним Ангелы, несущие венец над головой его, вида и лицезрения которого я не мог вынести.
\vs 3Ba 7:4
И только засияло Солнце, как раскинул Феникс крылья свои. Я же, увидев подобную славу, умалился от великого страха и бежал, и спрятался среди крыльев Ангела.
\vs 3Ba 7:5
И сказал мне Ангел: "Не бойся, Варух, но подожди, и тогда увидишь еще и заход их".

\vs 3Ba 8:1
И взяв меня, отнес он меня к Западу.
\vs 3Ba 8:2
И когда пришло время захода Солнца, снова вижу я впереди летящую птицу.
\vs 3Ba 8:3
И только приблизилась она, вижу я Ангелов, и убрали они венец от головы Солнца.
\vs 3Ba 8:4
Птица же стала, присмирев, и сложила крылья свои.
\vs 3Ba 8:5
И увидев это, сказал я: "Господин, зачем убрали они венец от головы Солнца, и отчего так присмирела птица?"
\vs 3Ba 8:6
И сказал мне Ангел: "Венец Солнца, после того, как прошло оно свой дневной путь, забирают четыре Ангела и уносят на небо, и обновляют его, ибо осквернился он и лучи его на земле. Да и вовсе каждый день обновляется он подобным образом".
\vs 3Ba 8:7
И сказал я, Варух: "Господин, а из-за чего оскверняются лучи его на земле?"
\vs 3Ba 8:8
И сказал мне Ангел: "Взирая на людские беззакония и неправедности: блуд, разврат, кражи, разбой, идолопоклонство, пьянство, убийства, вражду, ревность, злословие, ропот, наушничество, гадание и другие вещи, неугодные Богу.
\vs 3Ba 8:9
Из-за них оно оскверняется и поэтому обновляется.
\vs 3Ba 8:10
О птице же, почему она так присмирела: это потому, что сдерживает лучи Солнца, из-за огня и жара в течение всего дня вот из-за чего она присмирела.
\vs 3Ba 8:11
Ведь если бы ее крылья, о чем уже говорил я тебе, не прикрывали кругом солнечных лучей, не уцелело бы ни одно дыхание".

\vs 3Ba 9:1
И когда сложила она крылья, настала ночь с Луной и со звездами.
\vs 3Ba 9:2
И сказал я, Варух: "Господин, покажи мне и Луну, прошу тебя, то, как восходит она и как заходит, и в каком виде идет по небу".
\vs 3Ba 9:3
И сказал Ангел: "Дождись увидишь и ее спустя короткое время".
\vs 3Ba 9:4
И день спустя вижу я и ее в виде женщины, сидящей на колесе колесницы.
\vs 3Ba 9:5
И были впереди нее тельцы и агнцы запряжены в колесницу, и сонм Ангелов также.
\vs 3Ba 9:6
И сказал я: "Господин, кто эти тельцы и агнцы?" И сказал он мне: "И они тоже Ангелы".
\vs 3Ba 9:7
И сказал я: "А почему не светит она всегда, но только ночью?"
\vs 3Ba 9:8
И сказал Ангел: "Слушай: как челядь не может смело говорить в присутствии царя, так пред лицом Солнца не могут воссиять Луна и звезды. Ибо звезды висят всегда, однако прикрыты Солнцем.
\vs 3Ba 9:9
И Луна, оставаясь целой и невредимой, истощается солнечным жаром".

\vs 3Ba 10:1
И когда узнал я это все от Архангела, взяв, отнес он меня на четвертое небо.
\vs 3Ba 10:2
И увидел я плоскую равнину, и посреди нее озеро вод.
\vs 3Ba 10:3
И были там сонмища птиц всех родов, и не были птицы эти похожи на тех, что здесь, но увидел я журавля величиной с больших тельцов, превосходящего всех крупных животных, какие существуют в мире.
\vs 3Ba 10:4
И вопросил я Ангела: "Что это за равнина, и озеро, и что за множество птиц вокруг него?"
\vs 3Ba 10:5
И сказал Ангел: "Слушай, Варух: равнина эта вмещает озеро и прочие чудеса, которые есть на ней. Тут, водя бесконечные хороводы, ходят и беседуют друг с другом души праведников.
\vs 3Ba 10:6
Вода же та, беря которую, облака проливаются на землю дождем, и возрастают плоды".
\vs 3Ba 10:7
И снова сказал я Ангелу Яхве: "А птицы?"
\vs 3Ba 10:8
И сказал он мне: "Они те, которые ежечасно прославляют Яхве".
\vs 3Ba 10:9
И сказал я: "Господин, как же люди говорят, что из моря вода, проливающаяся на землю?"
\vs 3Ba 10:10
И сказал Ангел: "Дождевая вода та, что из моря, и из земных вод, и вот эта. Та же ее часть, которая дает рост плодам от этой воды. И еще узнай: от нее то, что люди называют росой небесной".

\vs 3Ba 11:1
И после того, взяв меня, отнес меня Ангел на пятое небо. И были заперты врата.
\vs 3Ba 11:2
И сказал я: "Господин, разве не отворятся ворота эти, чтобы мы вошли?"
\vs 3Ba 11:3
И сказал мне Ангел: "Не можем мы войти, пока не пришел Михаил, хранитель ключей от Царства Небесного. Подожди же, и увидишь славу Божию".
\vs 3Ba 11:4
И раздался голос громкий, как раскат грома. И сказал я: "Господин, что это за голос?"
\vs 3Ba 11:5
И сказал он мне: "Сейчас сойдет архистратиг Михаил, чтобы принять молитвы людей".
\vs 3Ba 11:6
И вот, пришел голос: "Да отворятся врата!" И отворились они, и раздался скрежет громкий, как при ударе грома.
\vs 3Ba 11:7
И пришел Михаил, и выступил навстречу ему Ангел, бывший со мной, и поклонился ему, и сказал: "Радуйся, архистратиг мой и всего нашего войска!"
\vs 3Ba 11:8
И сказал архистратиг Михаил: "Радуйся, брат наш и тот, кто толкует откровения живущим праведной жизнью".
\vs 3Ba 11:9
И так приветствовав друг друга встали они.
\vs 3Ba 11:10
И увидел я архистратига Михаила держащим великую чашу: глубина ее сколько есть расстояния от неба до земли, ширина ее сколько от севера и до юга.
\vs 3Ba 11:11
И сказал я: "Господин, что это держит Михаил архангел?"
\vs 3Ba 11:12
И сказал он мне: "Это чаша, куда приходят добродетели праведников и все благие поступки, совершаемые ими, которые затем доставляются пред лицем Бога Небесного".

\vs 3Ba 12:1
И еще говорил я с ними, как вот, пришли Ангелы, неся корзины, наполненные цветами. И отдали они их Михаилу.
\vs 3Ba 12:2
И вопросил я Ангела: "Господин, кто они и что есть приносимое ими?" И сказал он мне: "Это Ангелы-власти". И взяв, опрокинул архангел корзины в чашу.
\vs 3Ba 12:3
И говорит мне Ангел: "Цветы эти добродетели праведников".
\vs 3Ba 12:4
И увидел я других Ангелов, несущих корзины пустые и ненаполненные.
\vs 3Ba 12:5
И шли они печальные, и не осмелились приблизиться, потому что не имели наград совершенных.
\vs 3Ba 12:6
И воззвал Михаил, говоря: "Ну же и вы, Ангелы, несите, что принесли".
\vs 3Ba 12:7
И огорчился Михаил и Ангел, бывший со мной, потому что не наполнили они чашу.

\vs 3Ba 13:1
И так же затем пришли другие Ангелы, плача и сетуя, и со страхом говорили: "Взгляни, как почернели мы, господин, ибо преданы мы дурным людям и желаем уйти от них".
\vs 3Ba 13:2
И сказал Михаил: "Не можете вы уйти от них, чтобы враг не завладел ими окончательно. Но скажите мне, чего вы просите?"
\vs 3Ba 13:3
И сказали они: "Просим тебя, Михаил, архистратиг наш, переместить нас от них, ибо не в силах мы оставаться при людях дурных и безрассудных, ибо нет в них ничего доброго, но всяческая неправедность и корыстолюбие.
\vs 3Ba 13:4
Ведь ни разу не видели мы их входящими в собрание, или чтобы сделать что-то во благо, но где убийство они тут как тут, и где блуд, разврат, кражи, злословие, клятвопреступления, зависть, пьянство, вражда, ревность, ропот, наушничество, идолопоклонство, гадание и все тому подобное, там и они делатели таких вот дел и других, еще худших. Потому просим мы позволения уйти от них".
\vs 3Ba 13:5
И сказал Михаил Ангелам: "Подождите, пока я узнаю у Яхве, чему быть".

\vs 3Ba 14:1
И в тот самый миг отошел Михаил, и затворились врата. И был голос словно раскат грома.
\vs 3Ba 14:2
И вопросил я Ангела: "Что это за голос?"
\vs 3Ba 14:3
И сказал он мне: "Сейчас приносит Михаил добродетели человеческие Богу".

\vs 3Ba 15:1
И в то самое мгновение пришел Михаил, и отворились врата. И принес он елей.
\vs 3Ba 15:2
И Ангелам, принесшим полные корзины, наполнил он их елеем, говоря: "Отнесите, воздайте стократ друзьям нашим и тем, кто в трудах совершил благие дела. Ибо посеявшие как должно, собирают должную жатву".
\vs 3Ba 15:3
И говорит он и тем, что держат пустые корзины: "Давайте же и вы, заберите мзду по тому, что принесли, и воздайте сынам человеческим".
\vs 3Ba 15:4
И говорит он потом принесшим полные и принесшим пустые: "Пойдите и воздайте хвалу друзьям нашим и скажите им так:
\vs 3Ba 15:5
Вот что говорит Яхве: в малом верны вы Ему, надо многими поставит Он вас, войдите в радость Господа вашего".

\vs 3Ba 16:1
И повернувшись, говорит он и тем, которые ничего не принесли: "Вот что говорит Яхве: Не будьте унылы и не плачьте, не оставьте же и сынов человеческих, но когда прогневили они Меня делами своими, пойдя, не дайте им покоя, и прогневите их, и огорчите народ неразумный.
\vs 3Ba 16:2
Еще вдобавок к этому нашлите гусеницу, ливень, ржавчину, саранчу и град с молниями и громом, и рассеките их надвое мечом и смертью, и детей их демонами, ибо не услышали они голоса Моего, не соблюли заповедей Моих и не сделали по ним, но презрели заповеди Мои и оскорбили священников, возвещающих им слова Мои".

\vs 3Ba 17:1
И при этих словах затворились врата, и мы отступили.
\vs 3Ba 17:2
И взяв меня, опустил он меня на то место, с какого отправились мы в путь.
\vs 3Ba 17:3
И придя в себя, принялся я возносить славу Богу, удостоившему меня столь великой чести.

\include{tex/Mis}
\bibbookdescr{Azp}{
  inline={Апокалипсис Софонии},
  toc={Апокалипсис Софонии},
  bookmark={Апокалипсис Софонии},
  header={Апокалипсис Софонии},
  abbr={Ап~Соф}
}
\chhdr{Явление на 5-м небе.}
\vs Azp 0:0
И Дух взял меня и вознес меня на 5-е небо.
И я увидел ангелов, которых называют Господствами.
И диадема была возложена на них в Святом Духе,
и трон каждого из них блестел в 7 раз больше,
чем свет восходящего солнца.
И они жили в храмах спасения и пели гимны невыразимому Богу.
\chhdr{Саидский фрагмент\\Пророческое видение души в мучении.}
\vs Azp 1:1
Я увидел душу, которую 5000 ангелов наказывали и стерегли.
\vs Azp 1:2
Они взяли её на Востоке и принесли её на Запад.
Они били её, они давали ей 100 ударов плетью, каждый ежедневно.
\vs Azp 1:3
Я испугался, и я пал на лицо своё,
так что мои составы распались.
\vs Azp 1:4
Ангел помог мне.
Он сказал мне:
<<Крепись, о тот, кто победит и восторжествует,
потому что ты восторжествуешь над обвинителем и взойдёшь из Шеола.>>
\vs Azp 1:5
И после того, как я поднялся, я сказал:
<<Кто это, кого они наказывают?>>
\vs Azp 1:6
Он сказал мне:
<<Это душа, которая была найдена в её беззаконии.
И прежде, чем она пришла в раскаяние,
она была посещена и взята из её тела.>>
\vs Azp 1:7
Истинно, я, Цефания, видел эти вещи в моём видении.

\chhdr{Явление на широком месте.}
\vs Azp 1:8
И ангел Яхве пошёл со мной.
Я видел большое широкое место,
тысячи тысяч окружали его на его левой стороне,
и тьмы тем на его правой стороне.
Вид каждого был различным.
\vs Azp 1:9
Их волосы были распущены как это свойственно женщинам.
Их зубы были похожи на зубы \ldots
\chhdr{Ахмимский текст\\Фрагмент о погребении.}
\vs Azp 1:10
<<\ldots\ мёртвый.
Мы похороним его подобно всякому человеку.
\vs Azp 1:11
Каждый раз, когда он умирает, мы будем выносить его,
играя на кифаре перед ним и воспевая псалмы и оды над его телом.>>
\chhdr{Явления сверху города пророка.}
\vs Azp 2:1
Вот, я пошёл с ангелом Яхве,
и он вознёс меня над всем моим городом.
Сначала ничего не было перед моими глазами.
\vs Azp 2:2
Потом я увидел двух мужей, шедших вместе по одной дороге.
Я наблюдал за ними как они разговаривали.
\vs Azp 2:3
И, кроме того, я также увидел двух женщин,
мелющих вместе на мельнице.
И я наблюдал за ними как они разговаривали.
\vs Azp 2:4
И я также увидел двоих на постели,
каждого из них делавшего себе взаимно~--- на постели.
\vs Azp 2:5
И я увидел всю вселенную, висящую как капля воды,
которая свешивается с ведра, когда оно поднимается из колодца.
\vs Azp 2:6
Я сказал ангелу Яхве:
<<Разве тьмы или мрака нет в этом месте?>>
\vs Azp 2:7
Он сказал мне:
<<Нет, потому что нет тьмы в том месте,
где праведные и святые, а вернее они всегда пребывают во свете.>>
\vs Azp 2:8
И я увидел все души человеческие, как они пребывали в наказании.
\vs Azp 2:9
И я воскликнул к Господу Всемогущему:
<<O Боже, если ты обитаешь со святыми, ты сожалеешь о мире и о душах,
которые в этом наказании!>>
\chhdr{Ангелы с горы Сеир, делающие запись.}
\vs Azp 3:1
Ангел Яхве сказал мне:
<<Приди, я покажу тебе место праведности.>>
\vs Azp 3:2
И он вознёс меня на гору Сеир, и он показал мне 3-х мужей,
как 2 ангела шли с ними, радуясь и ликуя о них.
\vs Azp 3:3
Я сказал ангелу:
<<Какого рода они?>>
\vs Azp 3:4
Он сказал мне:
<<Они~--- 3 сына Йотама священника,
которые не хранят заповедь своего отца,
ни соблюдают повелений Яхве.>>
\vs Azp 3:5
Тогда я увидел, что 2 других ангела плакали
по 3-м сыновьям Йотама священника.
\vs Azp 3:6
Я сказал:
<<O ангел, кто они?>>
Он сказал:
<<Они~--- ангелы Господа Всемогущего.
Они записывают все хорошие дела праведных в свои свитки,
так как они бодрствуют у врат небесных.
\vs Azp 3:7
И я беру их из их рук и возношу их перед Господом Всемогущим.
Он записывает их имя в Книгу Жизни.
\vs Azp 3:8
Подобно ангелы обвинителя, который на земле:
они также записывают все грехи людей в свои свитки.
\vs Azp 3:9
Они также сидят у врат небесных.
Они сообщают обвинителю и он записывает их в свой свиток,
для того чтобы обвинить их, когда они выйдут из мира туда.
\chhdr{Отвратительные ангелы уносят души безбожников.}
\vs Azp 4:1
Потом я шёл с ангелом Яхве.
Я взглянул перед собой и я увидел там место.
\vs Azp 4:2
Тысячи тысяч и тьмы тем ангелов проходили через него.
\vs Azp 4:3
Их лица были подобны леопарду,
их клыки были снаружи их пасти~--- как у диких кабанов.
\vs Azp 4:4
Их глаза были смешаны с кровью.
Их волосы были распущены как волосы у женщин,
и огненные бичи были в их руках.
\vs Azp 4:5
Когда я увидел их, я испугался.
Я сказал тому ангелу, который шёл со мной:
<<Какого рода они?>>
\vs Azp 4:6
Он сказал мне:
<<Они~--- служители всей твари, которые приходят
к душам безбожников и приносят их,
и оставляют их в этом месте.
\vs Azp 4:7
Прежде чем они приносят их и бросают их
на их вечное наказание, они проводят с ними 3 дня,
ходя всюду в воздухе.>>
\vs Azp 4:8
Я сказал:
<<Я умоляю тебя, о господин, не дай им власти прийти ко мне.
\vs Azp 4:9
Ангел сказал:
<<Не бойся; я не позволю им прийти к тебе,
потому что ты чист перед Яхве.
Я не позволю им прийти к тебе,
потому что Господь Всемогущий послал меня к тебе,
потому что ты чист перед ним.
\vs Azp 4:10
Тогда он дал знак им, и они удалились, и они бежали от меня.
\chhdr{Небесный город.}
\vs Azp 5:1
Однако я пошёл с ангелом Яхве,
и я взглянул перед собой, и я увидел ворота.
\vs Azp 5:2
Вот, когда я приблизился к ним, я обнаружил,
что эти ворота были бронзовыми.
\vs Azp 5:3
Ангел коснулся их и они открылись перед ним.
Я вошёл с ним и нашёл целый квартал,
подобный красивому городу, и я шёл посреди него.
\vs Azp 5:4
Тогда ангел Яхве преобразился возле меня в том месте.
\vs Azp 5:5
\ldots\ Вот, я взглянул на них, и я обнаружил,
что эти ворота были бронзовые, и замки бронзовые,
а засовы железные.
\vs Azp 5:6
Вот, мои уста были заключены в них.
Я заметил бронзовые ворота впереди меня
как бы извергающие огонь приблизительно на 50 стадий.
\chhdr{Еремиэл~--- ангел и обвинитель в Шеоле.}
\vs Azp 6:1
Опять я возвратился и шёл,
и я увидел великое море.
\vs Azp 6:2
Но я думал, что это было море водяное.
Я обнаружил, что это море было полностью из огня,
подобно слизи, которая изливает множество пламени
и чьи волны жгут кал и асфальт.
\vs Azp 6:3
Они начали приближаться ко мне.
\vs Azp 6:4
Тогда я подумал, что Господь Всемогущий пришёл посетить меня.
\vs Azp 6:5
Вот, когда я увидел, я пал на лицо моё перед ним,
чтобы поклониться ему.
\vs Azp 6:6
Я был очень сильно напуган, и я умолял его,
чтобы он спас меня от этого бедствия.
\vs Azp 6:7
Я воскликнул, говоря:
<<Элои, Яхве, Адонай, Цебаот!
Я молю тебя, спаси меня от этого бедствия,
потому что это происходит со мной.>>
\vs Azp 6:8
В тот же самый момент я встал,
и я увидел великого ангела передо мной.
Его волосы были распростёрты как у львиц.
Его зубы были снаружи его пасти как у медведя.
Его волосы были распростёрты как у женщин.
Его тело было подобно змеиному,
когда он хотел поглотить меня.
\vs Azp 6:9
И когда я увидел его, я так испугался его,
что все части моего тела ослабли, и я пал на лицо моё.
\vs Azp 6:10
Я не мог стоять, и я молился перед Господом Всемогущим:
<<Спаси меня от этого бедствия.
Ты тот, кто спас Израиля от руки Фараона, царя Египетского.
Ты спас Сусанну от руки старцев неправедности.
Ты спас 3-х святых мужей, Шадрака, Мешака, Абед-Него,
из печи, горящей огнем.
Я прошу тебя спасти меня от этого бедствия.>>

\vs Azp 6:11
Тогда я поднялся и стал, и я увидел,
что другой великий ангел стоял передо мной своим лицом,
сияющим подобно лучам солнца в его славе,
так как его лицо похоже на то,
которое совершенно в своей славе.
\vs Azp 6:12
И он был опоясан, как если бы золотой пояс был на его груди.
Его ноги были подобны бронзе, которая расплавляется в огне.
\vs Azp 6:13
И когда я увидел его, я обрадовался, ибо я подумал,
что Господь Всемогущий пришёл посетить меня.
\vs Azp 6:14
Я пал на лицо моё, и я поклонился ему.
\vs Azp 6:15
Он сказал мне:
<<<Берегись. Не поклоняйся мне.
Я~--- не Господь Всемогущий, но великий ангел Еремиэл,
который поставлен над Абаддоном и Шеолом, тем,
в котором все души заключены с конца Потопа,
нашедшего на землю, до сего дня.>>
\vs Azp 6:16
Тогда я спросил у ангела:
<<Что это за место, куда я пришел?>>
Он сказал мне:
<<Это Шеол.>>
\vs Azp 6:17
Тогда я спросил его:
<<Кто тот великий ангел, стоящий так, которого я видел?>>
Он сказал:
<<Это тот, кто обвиняет людей перед лицом Яхве.>>
\chhdr{Два свитка.}
\vs Azp 7:1
Тогда я взглянул, и я увидел его со свитком в его руке.
Он начал разворачивать его.
\vs Azp 7:2
Вот, после того, как он развернул его,
я прочитал его на моём собственном языке.
Я нашёл, что все мои грехи, которые я сделал,
были написаны в нём,~--- те,
которые я сделал от моей юности до сего дня.
\vs Azp 7:3
Все они были написаны на том свитке моём,
не было ложного слова в нём.
\vs Azp 7:4
Если я не ходил посетить больного или вдову,
я нашёл это записанным в мою рукопись как проступок.
\vs Azp 7:5
Если я не посещал сироту, это было найдено записанным
в мой свиток как проступок.
\vs Azp 7:6
День, в который я не постился или не умолял во время молитвы,
я нашёл записанным в мой свиток как падение.
\vs Azp 7:7
И день, когда я не обращался
к сынам Израиля~--- так как это проступок~--- я нашёл
записанным в мой свиток,
\vs Azp 7:8
так что я пал на лицо моё и молился перед Господом Всемогущим:
<<Простри на меня твою милость и изгладь мой свиток,
потому что милость твоя приходит, чтобы пребывать
во всяком месте, и наполняет всякое место.>>
\vs Azp 7:9
Потом я поднялся и встал,
и я увидел великого ангела передо мной,
говорящего мне:
<<Торжествуй, побеждай, потому что ты превозмог
и восторжествовал над обвинителем,
и ты поднимешься из Шеола и Абаддона.
Ныне ты перейдёшь место перехода.>>
\vs Azp 7:10
Опять он принёс другой свиток,
который был написан рукой.
\vs Azp 7:11
Он начал разворачивать его,
и я читал его,
и нашёл его написанным на моём собственном языке \ldots
\chhdr{Оставление Ада.}
\vs Azp 8:1
Они помогли мне и поставили меня на ту лодку.
\vs Azp 8:2
Тысячи тысяч и тьмы тем ангелов воздали хвалу передо мной.
\vs Azp 8:3
Сам я облёкся ангельским одеянием.
Я видел всех тех ангелов молящимися.
\vs Azp 8:4
Сам я молился вместе с ними.
\vs Azp 8:5
Я понимал их язык, на котором они говорили со мной \ldots
\vs Azp 8:6
<<Вот, более того, сыновья мои, это искушение, потому что нужно,
чтобы добро и зло были взвешены на весах.>>
\chhdr{Первая труба: торжество и посещение праведников.}
\vs Azp 9:1
Потом явился великий ангел,
имея золотой рог в руке своей,
и он протрубил им 3 раза над моей головой, сказав:
<<Мужайся, о тот, кто торжествует!
Превозмогай, о тот, кто превозмогает!
Ибо ты торжествуешь над обвинителем,
и ты избегаешь Абаддона и Шеола.
\vs Azp 9:2
Hыне ты перейдёшь место перехода.
Ибо твоё имя написано в Книге Жизни.
\vs Azp 9:3
Я хотел обнять его, но я не мог обнять великого ангела,
потому что его слава велика.
\vs Azp 9:4
Потом он побежал ко всем праведникам,
то есть Аврааму и Исааку, и Иакову, и Еноху, и Илии, и Давиду.
\vs Azp 9:5
Он беседовал с ними как друг говорит с другом.
\chhdr{Вторая труба: открытие небес и д\acc{у}ши в мучении.}
\vs Azp 10:1
После этого великий ангел пришёл ко мне
с золотым рогом в своей руке,
и он затрубил в него к небесам.
\vs Azp 10:2
Небеса открылись с того места, где солнце восходит,
до места, где оно заходит, и с севера на юг.
\vs Azp 10:3
Я увидел море, которое я видел на дне Шеола.
Его волны поднимались к облакам.
\vs Azp 10:4
Я увидел все души, тонущие в нём.
Я увидел некоторых,
руки которых были привязаны к их шее,
их руки и ноги были скованы.
\vs Azp 10:5
Я сказал:
<<Кто они?>>
Он сказал мне:
<<Они те, кто был подкуплен,
и им давали золото и серебро,
пока души людей были введены в заблуждение.>>
\vs Azp 10:6
И я увидел других, покрытых огненными рогожами.
\vs Azp 10:7
Я сказал:
<<Кто они?>>
Он сказал мне:
<<Они те, кто давал деньги в рост, и они получали лихву за лихву.>>
\vs Azp 10:8
И я также увидел некоторых слепых вопиющих.
И я был изумлен, когда я увидел все эти дела Божьи.
\vs Azp 10:9
Я сказал:
<<Кто они?>>
Он сказал мне:
<<Они обученные, которые услышали слово Божье,
но не усовершенствовались в деле, о котором они услышали.>>
\vs Azp 10:10
И я сказал ему:
<<Значит они не имеют здесь покаяния?>>
Он сказал:
<<Да>>.
\vs Azp 10:11
Я сказал:
<<Как долго?>>
Он сказал мне:
<<До того дня, когда Яхве будет судить.>>
\vs Azp 10:12
И я увидел иных с их волосами на них.
\vs Azp 10:13
Я сказал:
<<Зачем волосы и тело в этом месте?>>
\vs Azp 10:14
Он сказал:
<<Да, Яхве даёт тело и волосы им, как он хочет.>>
\chhdr{Заступничество святых за тех, кто в мучении.}
\vs Azp 11:1
И я также увидел множества.
Он породил их.
\vs Azp 11:2
Так как они смотрели на все мучения,
они взывали, молясь перед Господом Всемогущим,
говоря:
<<Мы молимся тебе за тех, кто пребывает
во всех этих мучениях, чтобы ты пощадил всех их.>>
\vs Azp 11:3
И когда я увидел их, я сказал ангелу, который говорил со мной:
<<Кто они?>>
\vs Azp 11:4
Он сказал:
<<Они те, кто умоляет Яхве,~--- Авраам, Иссак и Иаков.
\vs Azp 11:5
Тогда на некоторое время они ежедневно являются с великим ангелом.
Он даёт трубный глас к небесам, а другой возглашает на землю.
\vs Azp 11:6
Все праведники слышат глас.
Они сбегаются, ежедневно молясь к Господу Всемогущему
за тех, кто пребывает во всех этих мучениях.
\chhdr{Другой рог: грядущий гнев Божий.}
\vs Azp 12:1
И опять великий ангел явился с золотым рогом в своей руке,
трубя над землёй.
\vs Azp 12:2
Они услышали его с места восхода солнца до места заката,
и от южных областей до северных областей.
\vs Azp 12:3
И опять он трубит им к небесам, и слышен его глас.
\vs Azp 12:4
Я сказал:
<<O господин, почему бы тебе не оставить меня,
пока я не увижу их всех?>>
\vs Azp 12:5
Он сказал мне:
<<Я не имею власти показать их тебе,
пока Господь Всемогущий не восстанет в своём гневе,
чтобы уничтожить землю и небо.
\vs Azp 12:6
Они увидят и будут встревожены,
и все они воскликнут, говоря:
<<Всякую плоть, которая предана тебе,
мы вверим тебе в день Яхве.>>
\vs Azp 12:7
Кто устоит перед лицом его,
когда он восстанет в своём гневе,
чтобы уничтожить небо и землю?
\vs Azp 12:8
Всякое дерево, которое растёт на земле,
будет исторгнуто с его корнями и падёт.
И всякая высокая башня и птицы летающие падут \ldots

\bibbookdescr{Ars}{
  inline={Письмо Аристея},
  toc={Письмо Аристея},
  bookmark={Письмо Аристея},
  header={Письмо Аристея},
  abbr={Арист}
}
\chhdr{Аристей Филократу.}
\vs Ars 1:1
Так как у нас имеется заслуживающее внимания повествование о посольстве к иудейскому первосвященнику Елеазару, а ты, Филократ, при всяком случае напоминал, что считаешь важным знать, для чего и почему мы были посланы, то я, зная твою любознательность, попытался изобразить тебе.
\vs Ars 1:2
Самое важное для человека это всегда учиться и приобретать что-либо новое, путем ли исторических повествований, или путем собственного опыта. Ибо чистое настроение души приобретается в том случаe, если она, усвоив прекраснейшее и одобрив то, что важнее всего, устраивает благочестие при помощи твердого правила.
\vs Ars 1:3
Имея склонность к тщательному размышлению о божественном, мы посвятили себя посольству к выше упомянутому мужу, который своим благородством и славой снискал особую честь как у сограждан, так и иноземцев, и принес величайшую пользу иудеям Палестины и других мест переводом божественного Закона, потому что написан у них на пергаменте еврейскими буквами.
\vs Ars 1:4
Это-то мы и выполнили со всяким тщанием. Следует сообщить тебе и о том, что мы говорили царю получив удобный случай о переселенных в Египет из Иудеи отцом царя, прежнем владельце столицы и владыке Египта.
\vs Ars 1:5
Я убежден, что ты, имея значительное расположениe к нравственной чистоте и душевной настроенности мужей, живущих согласно священному законодательству, охотно услышишь о том, что мы желаем сообщить, так как ты недавно приходил к нам с острова и выражал желание узнать о том, что способствует исправлению души.
\vs Ars 1:6
И ранее я отправил тебе описание замечательного, по моему мнению, об иудейском народе, полученное нами от ученейших жрецов в Египте.
\vs Ars 1:7
А так как ты любознателен в том, что может принести пользу душе, то необходимо передать по преимуществу всем единомышленникам, а тем более тебе, ибо у тебя подлинное расположение; ты брат не только по родству, но и по настроенности, влечение к прекрасному у нас одно и то же.
\vs Ars 1:8
Ведь удовольствие от золота, или какое-либо иное имущество, почитаемое пустыми, не приносит столько пользы, как образование и забота о нем. Дабы не впасть в мнoгocлoвиe, удлиняя предисловие, мы вернемся к дальнейшему ходу повествования.
\vs Ars 1:9
Димитрий Фалирей, заведующий царской библиотекой, получил крупные суммы на то, чтобы собрать, по возможности, все книги мира. Скупая и снимая копии, он, по мере сил, довел до конца желание царя.
\vs Ars 1:10
Однажды в нашем присутствии он был спрошен, сколько у него тысяч книг, и ответил: свыше двухсот тысяч, царь, а в непродолжительном времени я позабочусь об остальных, чтобы довести до пятисот тысяч. Но мне сообщают, что и законы иудеев заслуживают того, чтобы их переписать и иметь в твоей библиотеке.
\vs Ars 1:11
Что же препятствует тeбе, спросил, сделать это? Ведь в твоем распоряжении есть всё, касающееся этого дела!. Димитрий ответил: необходим еще перевод, так как среди иудеев пользуются особым письмом, подобно тому как египтяне своим расположением букв, почему имеют и особый язык. Предполагают, что говорят на сирийском, но их не этот, а иного типа. Узнав обо всем, царь повелел написать иудейскому первосвященнику, чтобы привел в исполнение этот план.
\vs Ars 1:12
А я, считая настоящий момент удобным, просил начальников телохранителей Сосивия тарентинца и Андрея о том же, о чем часто об освобождении переселенных из иудеи отцом царя действительно, он столь же удачно, как и храбро напал на всю территорию нижней Сирии и Финикии и одних переселил, а других взял в плен, всё подчинив, благодаря страху. В это время он и переселил около ста тысяч из Иудеи в Египет.
\vs Ars 1:13
Около тридцати тысяч из них, лучших воинов, он, вооружив, поселил в крепостях своей страны (хотя много и раньше прибыло с персидским царем, а до этого и иные были отправлены на помощь Псаммитиху, чтобы сражаться против эфиопского царя, но их прибыло не так много, как переселил Птолемей, сын Лага).
\vs Ars 1:14
Выбрав, как мы сказали, цветущих возрастом и отличающихся силою, он вооружил, а остальную массу старцев, юношей, а также женщин, он обратил в рабство, не столько по собственному желанию, сколько по требованию воинов, за услуги, которые они оказали на войнe. А так как мы, о чем ранее сказано, получили известный предлог к освобождению их, то обратились к царю с такими словами:
\vs Ars 1:15
Царь, не будь настолько безразсуден, чтобы тебя обличали сами факты. Ведь законодательство, которое мы намереваемся не только переписать, но и перевести, имеет силу для всех иудеев; какое же основание у нас будет для отправления, если в твоем царстве огромная масса находится в рабстве? Освободи же, по совершенству и богатству души, угнетаемых бедствиями, ибо, как я тщательно изследовал, Бог, управляющий твоим царством, даровал закон и им.
\vs Ars 1:16
Они, царь, чтут Зрителя всяческих и создателя Бога, Которого почитают и все, а мы иначе называем Его Зевсом и Дием. Древние дали это удачное наименование Тому, Кем оживотворяется и создано всё; Он же управляет и владычествует над всем. А так как величием души ты превосходишь всех людей, то освободи находящихся в рабстве.
\vs Ars 1:17
Подождав немного, когда мы в душе молились Богу, чтобы Он внушил ему мысль об освобождении всех, ибо человеческий род, творение Бога, Он переменяет и снова изменяет. Поэтому я часто и многообразно призывал Владыку сердец, чтобы Он побудил исполнить то, чего я просил.
\vs Ars 1:18
A выступая с речью о спасении людей, я твердо надеялся, что Бог исполнит просимое; ибо, если люди делают по благочестию то, что, по их мнению относится к справедливости и попечению о прекрасном, то владычествующий над всем Бог руководит их действиями и намерениями он, подняв голову и милостиво взглянув, спросил:
\vs Ars 1:19
сколько будет тысяч, по твоему мнению? Присутствовавший Андрей ответил: немногим более ста тысяч. Царь сказал: немногого же просит у нас Аристей. А Сосивий и некоторые из присутствующих ответили это: действительно, величия твоей души достойно принести величайшему Богу в качестве благодарственной жертвы освобождение их. Так как Владыкой всяческих ты удостоен высочайшей чести и прославлен более твоих предков, то тебе следует принести в благодарность и величайшую жертву.
\vs Ars 1:20
Сильно обрадованный, приказал добавить к жалованью и за каждого человека получать по двадцать драхм; издать об этом указ, а списки изготовить немедленно. Он обнаружил величайшее расположение, ибо Бог исполнил все наши желания и побудил его освободить не только тех, которые пришли с войском его отца, но и тех, которые жили ранее, или впоследствие были приведены в государство, хотя ему и заявляли, что дар обойдется более четырехсот талантов.
\vs Ars 1:21
A копия указа была сделана, по моему мнению, не напрасно, ведь великодушие царя будет гораздо яснее и очевиднее, ибо Бог дал ему возможность послужить спасению множества. Содержание же указа таково:
\vs Ars 1:22
По приказанию царя, те из соратников нашего отца в Сирии и Финикии, которые при нападении на Иудею захватили пленников иудеев и переселили их в столицу и страну, или продали иным, точно также, если некоторые жили ранее, или впоследствие были приведены оттуда, владеющие должны немедленно отпустить на свободу, получив тотчас же по двадцать драхм за человека: воины при выдаче жалованья, а остальные из царской казны.
\vs Ars 1:23
Ибо по нашему мнению они были взяты в плен и вопреки воле отца нашего и вопреки благородству, а страна их была опустошена и иудеи были переселены в Египет вследствие запальчивости воинов. Ведь добыча, захваченная воинами на поле брани, была достаточно велика, почему и порабощение этих людей совершенно несправедливо.
\vs Ars 1:24
Итак, воздавая, по общему мнению, справедливое всем людям, а особенно угнетаемым неразумно, и во всём стремясь к полному coглacию со справедливостью и благочестием в отношении всех, мы определяем всех иудеев нашего государства, каким бы то ни было образом в рабстве, отпустить на свободу, уплатив владельцам назначенную сумму. Никто не должен медлить исполнением этого; а списки доставить назначенным для этого в течение трех дней со времени издания настоящего указа, предъявляя вместе с тем и самих людей.
\vs Ars 1:25
Ибо мы решили, что осуществление этого полезно и нам и государству. А о неповинующихся должен доносить всякий желающий, с условием, что он станет господином того, кто окажется виновным, а имущество таковых будет взято в царскую казну.
\vs Ars 1:26
Когда этот указ, в котором находилось всё, кроме: и если некоторые жили ранее, или впоследствие были приведены оттуда, был подан царю для просмотра, то он, по своему благородству и великодушию, это добавил сам и приказал дать назначение казначеям легионов и царским менялам на всю сумму издержек.
\vs Ars 1:27
В таком виде это постановление было утверждено в течение семи дней, а выкупная сумма достигла более шестисот талантов, ибо много и грудных детей было освобождено вместе с их матерями. Когда же к царю обратились с запросом, выдавать ли и за них по двадцать драхм, то царь приказал делать и это, выполняя во всём его волю полностью.
\vs Ars 1:28
Когда это было исполнено, приказал Димитрию сделать доклад относительно копии иудейских книг. (Ибо эти цари управляли всем при посредстве указов и с великой осмотрительностью. Вот почему я помещаю копии доклада и писем, также количество отправленного и устройство их, так как все они отличались роскошью и искусством.) Копия доклада такова:
\vs Ars 1:29
Великому царю от Димитрия.
Так как ты, царь, для пополнения отсутствующих в твоей библиотеке книг приказал собрать, а распавшиеся надлежащим образом исправить, то я, тщательно потрудившись над этим, доношу тебе следующее:
\vs Ars 1:30
отсутствуют, наряду с немногими другими, книги иудейского закона. Они написаны еврейскими буквами и языком, но, как сообщают знающие, слишком небрежно и не так, как должно, ибо не привлекали царского внимания.
\vs Ars 1:31
Teбе необходимо иметь у себя и эти, но тщательно исправив, ибо это законодательство, как божественное, чисто и исполнено мудрости? Поэтому, прозаики, поэты и многие историки были далеки от упоминания о названных книгах и о мужах, которые управлялись на основании их, так как, по словам Экатея Авдиритского, учение в них чисто и священно.
\vs Ars 1:32
Итак, если, царь, угодно, пусть напишут первосвященнику в Иерусалиме, чтобы он прислал старцев особенно добродетельной жизни, сведущих в своём Законе, по шести от каждого колена, чтобы, достигнув coгласия по большинству и получив точный перевод, мы положили на видном месте, достойно и самого дела и твоего намерения. Будь счастлив всегда.
\vs Ars 1:33
После этого доклада царь приказал написать об этом Елеазару, сообщив и об освобождении пленников. А для изготовления сосудов, бокалов, трапезы и чаш для возлияния он дал золота весом пятьдесят талантов, серебра семьдесят талантов и достаточное количество драгоценных камней (ибо он приказал хранителям сокровищ, чтобы они предоставили мастерам выбирать, что те пожелают) и денег для жертв и на остальное около ста талантов.
\vs Ars 1:34
Об изготовлении мы скажем после того, как передадим копии писем. Письмо царя было такого содержания:
\vs Ars 1:35
Царь Птолемей первосвященнику Елеазару радоваться и здравствовать!
Так как в нашу страну было переселено много иудеев, силою уведенных из Иеросалима персами во время их господства, а кроме того пленников прибыло в Египет и вместе с отцом нашим
\vs Ars 1:36
(большинство их он зачислил в войско на большое жалованье, равным образом и тем, которые жили paнee, он, по доверию к ним, поручил охрану построенных им крепостей, чтобы, таким образом, египтяне были в безопасности; а мы, получив царскую власть, проявили большее человеколюбие по отношению ко всем, а особенно твоим согражданам),
\vs Ars 1:37
то мы освободили более ста тысяч пленников, уплатив их господам следуемую денежную плату и исправляя вместе с тем зло, причиненное им яростью черни. Мы решили, что этим поступаем благочестиво и приносим благодарственную жертву величайшему Богу, Который сохраняет наше царство в мире и величайшей славе во всей вселенной. Зрелых возрастом мы зачислили в войско, а пригодных для нашей службы и заслуживающих доверия при дворе мы определили на должности.
\vs Ars 1:38
Желая сделать угодное и им, и иудеям всего миpa и последующим, мы предрешили перевести ваш Закон греческими буквами с букв, называемых у вас еврейскими, чтобы в нашей библиотеке, наряду с другими царскими книгами, находились и эти.
\vs Ars 1:39
Поэтому ты поступишь прекрасно и согласно нашему желанию, если выберешь старцев добродетельной жизни, сведущих в Законе и сильных в переводе, по шести от каждого колена, чтобы достигнуть согласия по большинству, ибо изследование касается очень важных предметов. Мы полагаем, что, исполнив это, ты приобретешь себе великую славу.
\vs Ars 1:40
Для этого мы посылаем Андрея, начальника телохранителей, и Аристея, которые пользуются у нас почетом; они будут вести с тобою переговоры и доставят начатки приношений в храм, а для жертв и на остальное сто талантов серебра. А сообщив нам о желаниях, ты приобретешь благосклонность и поступишь согласно дружбе, так как мы возможно скорее исполним то, что тебе угодно. Будь здоров.
\vs Ars 1:41
На это письмо Елеазар тотчас же ответил следующее:
Первосвященник Елеазар царю Птолемею, истинному другу радоваться!
Нам приятно было бы, если бы ты, царица Арсиноя, твоя сестра и дети были здоровы; этого мы и желаем, а мы здоровы.
\vs Ars 1:42
Получив от тебя письмо, мы весьма возрадовались твоему намерению и прекрасному желанию; собрав весь народ, мы прочли ему, чтобы он знал о твоем благоговении к нашему Богу. Мы показали и присланные тобою бокалы, двадцать золотых и тридцать серебряных, пять сосудов, трапезу для возношения и сто талантов серебра для принесения жертв и необходимых исправлений в храме.
\vs Ars 1:43
Это доставили пользующиеся у тебя почетом Андрей и Аристей, мужи добрые, прекрасные, отличающиеся образованием и во всём достойные твоего настроения и справедливости. Они-то и передали нам твоё, на что и с нашей стороны услышали соответствующее твоему письму.
\vs Ars 1:44
Мы исполним всё, что полезно для тебя, даже если бы это было противно природе (ведь это свидетельствует о дружбе и любви), ибо и ты оказал нашим согражданам великие, разнообразные и никогда не забываемые благодеяния.
\vs Ars 1:45
Поэтому мы тотчас же принесли жертвы за тебя, твою сестру, детей и любезных), и весь народ молился, чтобы исполнилось всё, что тебе угодно, чтобы владычествующий над всем Бог сохранил твоё царство в мире и славе и чтобы перевод святого Закона был сделан с пользою для тебя и тщательно.
\vs Ars 1:46
В присутствии всех мы избрали старцев, мужей добрых и благородных, из каждого колена по шести; их мы отправили вместе с Законом. А ты, праведный царь, прекрасно поступишь, приказав, чтобы эти мужи, по окончании перевода книг, снова безпрепятственно вернулись к нам. Будь здоров.
\vs Ars 1:47
Из первого колена: Иосиф, Езекия, Захария, Иоанн, Езекия, Елисей.
Из второго: Иуда, Симон, Самуил, Адей, Матафия, Есхлемия.
Из третьего: Неемия, Иосиф, Феодосий, Васея, Орния, Дакис.
\vs Ars 1:48
Из четвертого: Ионафан, Аврей, Елисей, Анания, Хаврий, 3ахария.
Из пятого: Исаак, Иаков, Иесуа, Савватий, Симон, Левий.
Из шестого: Иуда, Иосиф, Симон, Захария, Самуил, Шелемия.
\vs Ars 1:49
Из седьмого: Савватий, Седекия, Иаков, Исайя, Иесия, Натфей.
Из восьмого: Феодосий, Иасон, Иесуа, Феодот, Иоанн, Ионафан.
Из десятого: Феофил, Авраам, Арсам, Иасон, Эндемия, Даниил.
\vs Ars 1:50
Из десятого: Иеремия, Елеазар, Захария, Ванея, Елисей, Дафей.
Из одиннадцатого: Самуил, Иосиф, Иуда, Ионафан, Хавев, Досифей.
Из двенадцатого: Исаил, Иоанн, Феодосий, Арсам, Авиит, Иезекииль.
Всего семьдесят два.
\vs Ars 1:51
Таков был ответ Елеазара и его приближенных на письмо царя.
Согласно обещанию, я опишу тебе также приготовленное. Они были сделаны с необыкновенным искусством, так как царь отпустил большие средства и всегда наблюдал за мастерами. Поэтому они ничего не могли упустить из виду и сделать небрежно.
\vs Ars 1:52
Прежде всего я опишу тебе устройство трапезы. Царь желал сделать это сооружение огромных размеров; но он приказал справиться у местных, какова величина уже существующей и стоящей в Иеросалимском храме.
\vs Ars 1:53
Когда же сообщили её размеры, он снова спросил, можно ли делать большую? Некоторые из священников и другие говорили, что нет препятствий, но он сказал, что желает сделать в пять раз большую, однако опасается, что она окажется непригодной для богослужения.
\vs Ars 1:54
А он, конечно, не хотел, чтобы приготовленная им только стояла на месте; ему будет гораздо приятнее, если соответствующие службы будут совершаться, как и должно, назначенными для этого на приготовленной им.
\vs Ars 1:55
Для прежней трапезы были указаны меньшие размеры не по недостатку золота, но, сказал, она была сделана таких размеров по известным, как кажется, основаниям. А если бы оказалось необходимым увеличить её, то ни в чем не было бы недостатка. Поэтому не следует ни уменьшать, ни увеличивать удачно избранные.
\vs Ars 1:56
Итак приказал широко пользоваться различными искусствами, ибо он всё замышлял в величественных чертах и обладал природной способностью представлять предметы в их готовом виде. Что не было записано, он приказал делать согласно с красотой, а что было указано, в этом следовать размерам.
\vs Ars 1:57
Из чистого золота было сделано массивное сооружение длиною в два локтя, шириною в один локоть и высотою в полтора локтя; говорю же не о накладном золоте, но о том, что была положена массивная доска.
\vs Ars 1:58
Вокруг был сделан ободок, шириною в ладонь, с витыми бортами, украшенными рельефной плетеной резьбой, удивительно искуссно выгравированной с трех сторон, ибо был треугольным.
\vs Ars 1:59
На каждой стороне работа была выполнена одинаково, так что, в какую бы сторону ни поворачивать, вид был один и тот же. Но в то время, как художественная работа под ободком была обращена к трапезе, наружная поверхность была видима приходящему.
\vs Ars 1:60
Поэтому верхний край с обоих сторон был острым, ибо, как сказано ранее, был сделан треугольным (в какую бы сторону его ни поворачивать). Посредине плетения в него были вставлены различные драгоценные камни, прикрепленные один к другому с неподражаемым искусством.
\vs Ars 1:61
Все они для безопасности были укреплены в отверстиях золотыми гвоздями, а на углах для прочности оправы связывались вместе.
\vs Ars 1:62
По бокам у ободка в верхней части кругом было сделано всё усеянное драгоценными камнями подобие яиц), изображенное выступами при помощи сплошного барельефа в виде полос, плотно прилегающих одна к другой вокруг всей трапезы.
\vs Ars 1:63
A под изображенным из драгоценных камней подобием яиц художники превосходно и очень отчетливо сделали венок, изобилующий всякими плодами: виноградными кистями, колосьями, финиками, масличными ягодами, гранатовыми яблоками и т. п. Для изображения этих плодов они употребили камни, соответствующие цвету каждого плода и прикрепили их к золотому кольцу вокруг всей трапезы, сбоку её.
\vs Ars 1:64
Украсив ободок, они внизу под изображением подобия яиц устроили таким же образом и остальные части рельефных украшений и резьбы, так что трапеза была сделана для пользования с обоих сторон, с какой угодно. Были сделаны и борты и ободок в нижней части у ножек.
\vs Ars 1:65
По всей ширине трапезы они сделали массивную доску в четыре пальца толщиною, в неё были вставлены ножки и под ободком укреплены шипами, находящимися в углублениях, чтобы можно было пользоваться с какой угодно стороны. Это можно было видеть на верхней доске, так как это произведение было устроено для употребления с обоих сторон.
\vs Ars 1:66
На самой же трапезе превосходно и рельефно изобразили мэандр), посредине которого находилось множество драгоценных камней разных пород: рубины, смарагды, ониксы и другие породы камней превосходного качества.
\vs Ars 1:67
Под изображением мэандра находилась сделанная удивительно искусно сетка, посредине имеющая узор в форме ромба. В него были вставлены горный хрусталь и так называемый янтарь, производя необычайное впечатление на зрителей.
\vs Ars 1:68
Ножки были сделаны в форме головок лилий, под трапезой лилии загибались, а с лицевой стороны имели прямые листья.
\vs Ars 1:69
Основание ножек на полу из рубина и всюду имело четыре пальца, с лицевой стороны имея форму башмака в восемь пальцев ширины. На нем и удерживалась вся тяжесть ножек.
\vs Ars 1:70
Вырезали из камня плющ, обвитый тернием и виноградной лозой, которая вместе с виноградными кистями, вытесанными из камня, окружала ножки до верху. Таково было устройство четырех ножек. Всё было сделано и выступало ясно; опытность и искусство неизменно превосходили действительность, так что при дуновении воздуха листья начинали двигаться, ибо всё было сделано так, чтобы изображать действительность.
\vs Ars 1:71
Переднюю сторону трапезы сделали из трех частей, как бы в форме триптиха, и по толщине сооружения скрепили одну с другой шипами в форме гусиной лапки, так что соединение скреп не было видно и нельзя было найти. А толщина всей трапезы была не меньше полулоктя, так что на всё сооружение пошло много талантов.
\vs Ars 1:72
А так как царь не запрещал увеличивать размеров, то, если нужно было издержать больше приготовленного, царь отпускал на это и больше. Согласно его желанию всё было исполнено удивительно и достойным образом, безподобно со стороны искусства и безукоризненно в отношении красоты.
\vs Ars 1:73
Два сосуда были сделаны из золота; от основания и до середины они были покрыты чешуйчатой резьбой, а между чешуей были весьма искуссно устроены скрепы из драгоценных камней.
\vs Ars 1:74
Далее лежал мэандр высотою в локоть; он был рельефно изображен при помощи драгоценных камней различного цвета, свидетельствуя как о зрелости искусства, так и о тщательности. За ним рельефное украшение в виде ромба, которое до отверстия имело форму сети.
\vs Ars 1:75
Впечатление красоты дополняли небольшие щиты величиною не меньше четырех пальцев из различных драгоценных камней и расположенные посредине один подле другого. А по краю отверстия, кругом, были изображены лилии с цветками и виноградные лозы, переплетающиеся с виноградными кистями.
\vs Ars 1:76
Таково было устройство золотых сосудов, которые вмещали более двух метритов). Что касается серебряных, то они были сделаны гладкими, вроде зеркала, и уже это было удивительно, так как в них гораздо яснее, чем в зеркале, отражалось всё, что подносили.
\vs Ars 1:77
По сравнению с отражаемой ими действительностью их действие описать невозможно. Когда всё было окончено и предметы были поставлены один подле другого, то есть сначала серебряный сосуд, затем золотой, снова серебряный и золотой, то действие их вида было совершенно неописуемо, так что приходившие посмотреть на них не могли уйти вследствие их необычайного блеска и прелести для взоров.
\vs Ars 1:78
Впечатление от их наружного вида было разнообразно. Когда смотрели на работу из золота, являлась радость с удивлением, так как внимание непрерывно устремлялось на каждое из этих художественных произведений. А если кто, напротив, хотел взглянуть на серебряные сосуды, то они всюду и кругом начинали блестеть, где бы кто ни стоял, и доставляли зрителю еще большее удовольствие. Таким образом, изящество их совершенно нельзя описать.
\vs Ars 1:79
На золотых бокалах посредине были выгравированы венки виноградной лозы, а по краям вырезали венок, сплетенный из плюща, мирты и маслины, вставив различные драгоценные камни. Остальные части граверной работы они сделали из различных узоров, усердно стремясь всё сделать для большей славы царя.
\vs Ars 1:80
Вообще, таких роскошных и художественных предметов нет не только в царских сокровищницах, но и в ничьих других. Ибо славолюбивый царь не мало подумал над тем, чтобы всё было исполнено прекрасно.
\vs Ars 1:81
Часто он оставлял публичные аудиенции и внимательно следил за художниками, чтобы они выполняли свою работу достойно того места, куда отправлялись их произведения. Поэтому всё делалось великолепно и достойно, как царя, отправляющего, так и первосвященника, управляющего этим местом.
\vs Ars 1:82
На работу пошло великое множество драгоценных камней, притом большой величины, не менее пяти тысяч и всё отличалось художественностью исполнения, так что количество драгоценных камней и работа ювелиров стоили в пять раз дороже золота.
\vs Ars 1:83
Я сообщил тебe описание их, предполагая, что это необходимо. Дальнейшее содержит наше путешествие к Елеазару. Сначала я опишу устройство всей страны. Когда мы прибыли на место, то увидели город, лежащий посредине всех иудеев, на очень высокой гopе.
\vs Ars 1:84
На краю был построен храм превосходного вида, три стены, высотой более семидесяти локтей, а их ширина и длина соответствовали устройству храма, так как всё было построено с необыкновенными во всех отношениях великолепием и роскошью.
\vs Ars 1:85
Очевидно было, что на двери, на прикрепление их к косякам и укрепление притолоков затрачены были также огромные суммы.
\vs Ars 1:86
Устройство завесы во всем было совершено подобно дверям. Особенно приятный вид, от которого с трудом можно было оторваться, она получала при дуновении ветра, когда ткань приходила в непрерывное движение, так как дуновение от основания передавалось по складкам до верхнего края.
\vs Ars 1:87
Устройство жертвенника отвечало месту и сожигаемым на огне жертвам, точно также и подъем к нему; место это, вследствие необходимого благоприличия, имело наклон, так как священники совершали служение одетыми до пят в льняные хитоны.
\vs Ars 1:88
Храм лицом был обращен к востоку, а задней стороной на запад. Весь пол был вымощен камнем, а для стока воды от замывания жертвенной крови имел в соответствующих местах наклон; ибо в праздничные дни приводились для жертв тысячи скота.
\vs Ars 1:89
Скопление же воды было неисчерпаемо, так как внутри протекал обильный естественный источник, а под землею, кроме того, находились удивительные и неописуемые водоемы. И показывали на пять стадий вокруг основания храма безчисленные галереи каждого из них, так как потоки в каждой части соединялись друг с другом.
\vs Ars 1:90
Всё это на полу и стенах было обложено свинцом, а поверх этого покрыто толстым слоем штукатурки, так что всё было сделано прочно. Частые отверстия в полу не были известны никому, кроме служащих, как будто всё множество жертвенной крови очищалось одним движением и мановением.
\vs Ars 1:91
Объясню, насколько я, по моему убеждению, сам удостоверился, и устройство водоемов. Меня вывели за город дальше, чем на четыре стадии, и приказали, наклонившись в известном месте, прислушаться к шуму от встречи вод. Вследствие этого мне, как сказано, ясной стала величина водоемов.
\vs Ars 1:92
Служение священников по силе, а также настроению благоприличия и тишины несравненно. Все усердно трудятся по доброй воле и с великим напряжением; каждый же заботится о порученном. Они непрерывно работают: одни доставляют дрова, другие масло, иные крупинчатую муку; иные ароматы; другие сожигают части жертвенного мяса, обнаруживая необыкновенную силу.
\vs Ars 1:93
Взяв обоими руками ноги телят, каждая из которых весит почти более двух талантов, они удивительно ловко и без промаха бросают их обоими руками на значительную высоту, точно также и овец и коз, отличающихся значительным весом и тучностью. Назначенные для этого выбирают безпорочных и особенно тучных и совершается вышеуказанное.
\vs Ars 1:94
Для отдыха им назначено место, где сидят отдыхающие. В это время пробуждаются те из отдыхавших, которые имеют желание, хотя никто не приказывает им служить.
\vs Ars 1:95
А тишина такова, что можно подумать, будто здесь нет никого, хотя служащих находится около семи тысяч (количество приносящих жертвы также очень велико), но всё совершается в страхе и достойно великого Божества.
\vs Ars 1:96
Нас охватило великое изумление, когда мы увидели Елеазара в служении, его облачение и славу, которая обнаруживалась в носимом им хитоне и камнях на нем. Вокруг его подира были золотые позвонки, которые издавали своеобразные гармонические звуки, а около каждого из них разноцветные гранатовые яблочки поразительной окраски.
\vs Ars 1:97
Он быль опоясан превосходным и великолепным поясом, вытканным из красивейших цветов. На груди он носит так называемый наперсник судный, в который были вставлены оправленные в золото двенадцать камней различной породы, с расположенными согласно первоначальному порядку именами начальников колен. Каждый сверкал своим характерным и неописуемым природным блеском.
\vs Ars 1:98
На голове имеет так называемый кидар, а на нем безподобная митра, то есть святая диадема, на которой над бровями священным шрифтом на золотом листке было вырезано имя Божие, полное славы. В таком виде выходит тот, кто был признан быть достойным этого, на служение.
\vs Ars 1:99
Всё это вместе вызывало страх и трепет, так что казалось, будто приходишь в иное место, вне этого миpa. И я утверждаю, что каждый человек, приходя посмотреть на это, повергался в изумление и невыразимое удивление, так как мысль его изменялась вследствие святого устройства во всём.
\vs Ars 1:100
Чтобы узнать всё, мы производили осмотр, поднявшись на лежащую около города крепость. Она расположена на самом высоком месте и укреплена множеством башен, так как они доверху выстроены из больших каменных плит, для охраны, как мы понимаем, мест около храма
\vs Ars 1:101
(чтобы, в случаe какого-либо нападения, бунта, или вторжения неприятелей, никто не мог проникнуть за стены, окружающие храм, так как на башнях крепости есть метательные машины и разные снаряды, а место это лежит выше упомянутых ранее стен),
\vs Ars 1:102
так как эти башни охраняются наиболее надежными мужами, давшими отечеству великие доказательства. Им разрешается выходить из крепости только по праздникам, притом по частям. Точно также никого не разрешается и впускать.
\vs Ars 1:103
Большую осторожность соблюдают они, если начальник дал разрешение впустить кого-либо для осмотра. Это случилось и с нами. С трудом безоружных нас двух впустили, чтобы посмотреть на принесение жертв.
\vs Ars 1:104
Говорили, что они обязались в этом клятвою. Bcе они, числом пятьсот, поклялись (конечно, при клятве дело по необходимости выполняется по-божески) не впускать в крепость болee пяти человек одновременно. Ведь крепость является единственной защитой храма и строитель укрепил её так для охраны указанного ранее.
\vs Ars 1:105
Город средней величины, так как стена, насколько можно догадываться, имеет около сорока стадий. Башни расположены в нем в форме театра; в нижних входы не видны, а в верхних заметны; в них и выходы. Местность имеет подъем, так как город построен на горе.
\vs Ars 1:106
Ко входам лестницы; одни вверху, а другие внизу, и очень удалены от дороги, чтобы те, которые живут в чистоте, не соприкасались с недозволенным.
\vs Ars 1:107
И начальники города неслучайно построили его симметрично, но по мудром размышлении. Так как эта страна обширна и прекрасна, и одни части её ровны, как например по направлению к Самарии и граничащие с Идумеей, а другие, как например посредине страны, гористы, то необходимо постоянно возделывать и обрабатывать землю, чтобы таким путем и эти стали плодородными. Если делать это, то во всей указанной стране всё приносит обильные плоды.
\vs Ars 1:108
В городах, отличающихся своей величиной и соответствующим благоденствием, население многочисленно, а страна оставляется в пренебрежении, так как все склоняются к жизненным радостям, ибо все люди склонны к yдoвольcтвиям.
\vs Ars 1:109
Это имеет место и в Александрии, превосходящей все города своей величиной и благоденствием. Именно, те из поселян, которые, прибыв в неё погостить, остаются надолго, отвыкают от земледельческого труда.
\vs Ars 1:110
Поэтому, чтобы они не задерживались, царь разрешил оставаться не более двадцати дней. Соответственно этому он сделал письменное распоряжение чиновникам: если необходимо вызвать кого-либо, то разбираться в течение пяти дней.
\vs Ars 1:111
В виду важности дела он назначил для каждого округа судей и их помощников, чтобы земледельцы и поверенные, получая доходы, не уменьшали городских кладовых; говорю же я о земледельческих налогах.
\vs Ars 1:112
Мы уклонились в сторону, потому что Елеазар прекрасно на примерах разъяснил нам вышеизложенное. Действительно, труд при обработке земли велик. И страна их изобилует масличными деревьями, хлебными плодами, овощами, а кроме того виноградом и массою меда (других плодовых деревьев и финиковых пальм у них нет), множеством различного скота и обилием пастбищ для него.
\vs Ars 1:113
Поэтому они прекрасно обратили внимание на то, что эта страна требует многочисленного населения, и установили надлежащее отношение между городом и деревнями.
\vs Ars 1:114
Кроме того, сюда доставляется арабами масса благовоний, различных драгоценных камней и золота. Страна эта, удобная для земледелия, пригодна и для торговли, а город для: занятия различными ремеслами. Она не имеет недостатка ни в чем, что доставляется морем.
\vs Ars 1:115
Есть в ней и удобные гавани доставляющие: у Аскалона, Яфы, Газы, а также у основанной царем Птолемаиды, которая находится посредине первых, на небольшом от них разстоянии. Страна эта имеет всё в изобилии, так как всюду хорошо орошается и прочно защищена.
\vs Ars 1:116
Её окружает река, называемая Иорданом, которая никогда не пересыхает (первоначально страна эта была не менее шестидесяти миллионов арур), но впоследствие, когда соседи были вытеснены из неё, шестьсот тысяч мужей получили в удел сто арур). Разливаясь, подобно Нилу, она около времени жатвы увлажняет большую часть страны.
\vs Ars 1:117
Он впадает в другую реку, в стране Птолемеев, а эта выходит в море. Текут и иные горные потоки, охватывающие окрестности Газы и местность Азота.
\vs Ars 1:118
Охраняется самою природой, будучи недоступна для вторжения и непроходима для больших масс, так как дороги узки, её окружают утесы и глубокие ущелья, a кромe того все горы вокруг этой страны скалисты.
\vs Ars 1:119
Говорили также, что ранее в соседних горах Аравии существовали медные и железные рудники, но во время господства персов они были оставлены, так как начальники того времени пустили ложные слухи, будто разработка их безполезна и дорого обходится,
\vs Ars 1:120
чтобы вследствие добывания оных не погибла страна и, при их тирании, не отпала, тогда как путем этой клеветы они получили предлог к доступу в эту местность. Итак, брат Филократ, я указал тебе главное и сколько нужно было об этом; далее же мы изложим то, что касается перевода.
\vs Ars 1:121
Елеазар выбрал лучших мужей, отличающихся образованием и знатностью рода, которые приобрели навык не только в иудейской литературе, но тщательно позаботились и об изучении греческой.
\vs Ars 1:122
Поэтому они были пригодны для посольства и в необходимых случаях исполняли его; они обладали большими дарованиями к беседам и изследованию в области Закона, стремясь к среднему положению (ибо оно прекраснее всего); они оставили грубость и неотделанность мысли, а также пренебрегли самомнением и своим превосходством над другими; в беседах они были примером для других, как своим умением слушать, так и отвечать каждому должное; все они соблюдают это, желая в этом более всего превосходить друг друга, все быть достойными своего начальника и его добродетели.
\vs Ars 1:123
А что они любили Елеазара, видно было, как они с неохотой покидали его. И сам он не только царю написал о возвращении их, но настойчиво просил Андрея и нас содействовать, насколько можем.
\vs Ars 1:124
И хотя мы обещали внимательно заботиться о них, он говорил, что сильно безпокоится. Действительно, он знал, что любящий доброе и добрых царь выше всего ставит приглашеше таких людей, которые где-либо признаются, как отличающиеся от других своим образованием и разумом.
\vs Ars 1:125
Царь, я полагаю, прекрасно говорит, что, имея около себя мужей праведных и мудрых, он приобретет лучшую охрану для своего царства, так как друзья с полной откровенностью советуют ему полезное. А этим именно и обладали посланные Елеазара.
\vs Ars 1:126
И он клятвенно уверял, что он не отпустил бы этих людей, если бы того требовало какое-либо иное личное его дело, но он отправляет их для общего исправления всех граждан.
\vs Ars 1:127
Ибо добродетельная жизнь заключается в соблюдении законов, а это гораздо лучше достигается путем слушания, чем путем чтения. Итак, предлагая их и подобное им, Елеазар ясно показывал свое расположение к ним.
\vs Ars 1:128
Следует вкратце упомянуть и о том, что Елеазар ответил нам на наши вопросы (ибо я полагаю, что многиe серьезно интересуются некоторыми из законов о пище и питье, а также о животных, признаваемых нечистыми).
\vs Ars 1:129
Итак, когда мы спросили, почему, несмотря на одинаковое происхождение, одни считаются нечистыми для еды, а другие и для прикосновения (ибо большинство из Закона отличается суеверием, а в этих частях полным), то он начал на это следующее.
\vs Ars 1:130
Ты видишь, сказал он, как влияют образ жизни и знакомства; поддерживая знакомство с порочными, люди совращаются и становятся несчастными на всю жизнь; а если они живут в обществе мудрых и разумных, то из неведения вступают в жизнь лучшую.
\vs Ars 1:131
Поэтому наш законодатель, определив прежде всего то, что относится к благочестию и справедливости, научив всему этому не только в форме запрещений, но и путем разъяснений, показав вредные последствия, а также наказания, посылаемые Богом виновникам этого
\vs Ars 1:132
(ибо прежде всего он и указал, что Бог един и сила Его очевидна всюду, так как всякое место полно Его господства и от Него не скроется ни одно из тайных дел людей на земле, но Ему видно всё, что делает и намеревается делать человек),
\vs Ars 1:133
тщательно выполнив это и сделав вполне очевидным, он показал, что, если бы кто и замыслил сделать дурное, то не не скрыл, но даже и не сделал бы, на протяжении всего законодательства указывая могущество Божие.
\vs Ars 1:134
Итак, положив такое начало и показав, что все остальные люди, кроме нас, почитают многих богов, хотя сами гораздо сильнее тех, кого безразсудно чтут
\vs Ars 1:135
(действительно, сделав из дерева и камней статуи, они говорят, что это образы тех, которые изобрели нечто полезное для их жизни; им они покланяются, сразу обнаруживая глупость.
\vs Ars 1:136
Разве они не [обнаружили бы] полное неразумие, если бы на этом основании, вследствие изобретения, кто-либо был обоготворен? Действительно, взяв одну из тварей, они лишь заметили и указали пользу, но не создали её устройства.
\vs Ars 1:137
Поэтому тщетно и безразсудно обоготворять подобных. И теперь ведь еще есть много людей более ученых и изобретательных, чем прежде, и не доходят до того, чтобы поклоняться им. Создавшие эти образы и составители мифов считают себя самыми мудрыми из греков.
\vs Ars 1:138
Что же говорить о других, более безразсудных, о египтянах и подобных им? Они останавливаются на зверях, на большинстве гадов и животных, покланяются им и приносят жертвы, как живым, так и павшим),
\vs Ars 1:139
и вот, имея в виду всё это, мудрый законодатель, которого Бог наделил способностями к познанию всего, огородил нас частоколом, которого нельзя прорубить, и железными стенами, чтобы мы ни в чем не смешивались с другими народами, пребывая чистыми по телу и душе, свободными от пустых учений, выше всех тварей почитая единого и могущественного Бoгa.
\vs Ars 1:140
На этом основании начальники египтян, их жрецы, постигшие многое и знакомые с письменами, называют нас людьми Божиими. А это неприложимо к остальным, если они не почитают истинного Бога, но являются людьми пищи, питья и одежды.
\vs Ars 1:141
Действительно, к этому направлено всё настроение их души, а у наших это вменяется ни во что; мы в течение всей жизни занимаемся изследованием божественного правления.
\vs Ars 1:142
Поэтому, чтобы мы ни с кем не смешивались и, имея общение с порочными, не испортились, всюду оградил нас законами о чистоте, в пище, питье, прикосновениях, в том, что мы слышим и видим.
\vs Ars 1:143
Вообще, всё подобное согласно с естественным разумом, так как установлено одной Силой и каждое в отдельности о том, почему мы воздерживаемся и пользуемся, имеет глубокое основание.
\vs Ars 1:144
Для примера я кратко объясню тебе одно или два, чтобы ты не впал в опровергнутое мнение, будто Моисей заповедует это, заботясь о мышах, куницах и тому подобных. Напротив, все важные определения сделаны ради справедливости, с целью чистых размышлений и образования нравов.
\vs Ars 1:145
Все птицы, которыми мы питаемся, ручные, чистые, питающиеся овощами и пшеницей, как например голуби, горлицы, куры, куропатки, гуси и прочие, подобные им.
\vs Ars 1:146
А в ряду запрещенных птиц ты найдешь диких, плотоядных, порабощающих благодаря своей силе остальных и несправедливо пожирающих названных выше ручных. Да и не только этих; они похищают даже ягнят, козлят и причиняют вред людям, как мертвым, так и живым.
\vs Ars 1:147
Поэтому, назвав еще нечистыми, обозначил этим, что те, для кого назначено законодательство, должны быть справедливыми по душе, никого не угнетать, полагаясь на свою силу, и ничего не похищать, но управлять своею жизнью согласно справедливости, подобно тому, как названные выше ручные птицы питаются овощами, растущими на земле, и не пользуются своей силой для угнетения более слабых и родственных.
\vs Ars 1:148
И вот, посредством такого законодатель дал знамение разумным, чтобы они были справедливыми, не насильничали и, полагаясь на свою силу, не угнетали других.
\vs Ars 1:149
А если этих, вследствие их природных особенностей, не должно даже и касаться, то как же не охранять себя всячески от того, чтобы наши нравы не извратились в этом.
\vs Ars 1:150
Итак, всё о разрешении этих и животных изложены нам в форме символов. Так, раздвоенность копыт и разделение когтей является символом того, чтобы разграничивать каждое из дел, стремясь к прекрасному.
\vs Ars 1:151
Ибо сила всего тела и деятельность его опору имеет в плечах и бедрах. Поэтому, этим символом принуждает всё направлять к справедливости с разделением, а также, что мы отличаемся от всех людей.
\vs Ars 1:152
Действительно, большинство остальных людей оскверняют себя совокуплениями, совершая тяжкую несправедливость, и этим хвалятся целые страны и города; они не только вступают в сношения с мужчинами, но оскверняют матерей и дочерей. Мы же воздерживаемся от этого.
\vs Ars 1:153
Но кто владеет указанным выше способом различения, тот, как он указал, владеет им и в отношении памяти. Ибо все, раздвояющие копыта, отрыгают и жвачку, ясно показывая размышляющим свойства памяти;
\vs Ars 1:154
ведь жвачность есть ничто иное, как воспоминание о жизни и устройстве, и полагает, что жизнь поддерживается благодаря питанию.
\vs Ars 1:155
Поэтому и чрез писание он предписывает следующее: помни ЯХВЕ, Бога твоего, сотворившего среди тебя великое и чудное. Действительно, если подумать, то великим и славным окажется прежде всего скрепление тела, потребление пищи и разделение каждого члена.
\vs Ars 1:156
Еще более безконечной мудрости заключает устройство чувств, деятельность мысли и невидимое движение, а также быстрота действия в каждом и изобретение искусств.
\vs Ars 1:157
Поэтому предписывает помнить, что указанное выше вместе с устройством хранится божественной силой. Всякое время и место он определил для постоянного памятования о Боге, владыке и хранителе.
\vs Ars 1:158
Поэтому он повелевает, чтобы при пище и питье сначала принести начатки Богу и затем пользоваться. Далее он дал нам знак воспоминания и на покровах); точно также, для памяти о том, что Бог есть, он приказал нам поместить изречения на дверях и воротах).
\vs Ars 1:159
И на руках он ясно приказал привесить знак), ясно показывая, что всякое действие должно совершать справедливо, памятуя о своем устройстве, а более всего питая страх к Богу.
\vs Ars 1:160
Призывает также, отходя ко сну, вставая и путешествуя, изучать творения Божии и не только на словах, но и в мыслях рассматривать свои движения и представления, когда мы отходим ко сну и когда пробуждаемся, так как смена этого божественна и непостижима.
\vs Ars 1:161
Итак, тебе показано превосходное учение в отношении к различию и памяти, как мы истолковали раздвоенность копыт и жвачность. Это заповедуется душе не безцельно и случайно, но ради истины и руководства к здравому учению.
\vs Ars 1:162
Дав предписания относительно пищи, питья, а также прикосновений, приказывает ничего не делать и слушать необдуманно и, пользуясь силою разума, не обращать его на неправду.
\vs Ars 1:163
То же можно найти и в отношении животных. Куницы, мыши и подобные им, сколько указано, вредны.
\vs Ars 1:164
Мыши всё оскверняют и портят, не только для собственного питания, но и делают совершенно безполезным для человека всё, чему бы они ни начали вредить.
\vs Ars 1:165
А куницы оригинальны. Кроме указанного выше они имеют постыдное устройство: они зачинают ушами, а детей рождают через рот.
\vs Ars 1:166
Поэтому такой характер нечист для людей. То, что они воспринимают слухом, это воплощают в слове и погружают других во зло, производя не случайную нечистоту, но пятная себя всюду осквернением нечестия. Ваш царь прекрасно делает, убивая, как мы слышим, таковых.
\vs Ars 1:167
А я сказал: я полагаю, ты говоришь о доносчиках, так как их он всегда подвергает побоям и мучительной смерти.
Он: и я говорю о них. Действительно, подкарауливать погибель людей нечестиво.
\vs Ars 1:168
А наш закон предписывает никому не вредить ни словом, ни делом.
Итак, тебе показано, насколько это можно было сделать вкратце, что все определения имеют в виду справедливость и писание не предписывает ничего безцельного или баснословного, но чтобы в течение всей жизни в своих действиях мы упражнялись в справедливости по отношению ко всем людям, помня о господствующем Боге.
\vs Ars 1:169
Поэтому всё разсуждение о пище и о нечистых гадах и животных относится к справедливости и справедливым отношениям к людям.
\vs Ars 1:170
По моему мнению он прекрасно защищал каждое. А относительно приносимых в жертву телят, баранов и козлов он говорил, что, взяв из стада быков и овец ручных, их должно приносить в жертву, но не диких, чтобы приносящие жертву, воспользовавшись указаниями законодателя, ничем не гордились и знали свою природу, ибо приносящий жертву приносит в жертву всё настроение своей души.
\vs Ars 1:171
Итак, я полагаю, что и в этом отношении его беседы заслуживают внимания. Поэтому, вследствие твоей любознательности, у меня, Филократ, были побуждения объяснить тебе святость Закона и его согласие с природой.
\vs Ars 1:172
И Елеазар, совершив жертвоприношение и избрав посланцев и приготовив много даров для
\vs Ars 1:173
царя, отправил нас в путь в великой безопасности. И когда мы достигли Александрии, царю тотчас же доложили о нашем прибытии. Будучи приняты во дворце, Андрей и я тепло приветствовали
\vs Ars 1:174
царя и передали ему письмо, написанное Елеазаром. Царь весьма безпокоился о том, чтобы принять посланных, и повелел, чтобы все прочие чиновники вышли, а посланные
\vs Ars 1:175
тотчас же были приведены к нему. Это вызвало всеобщее изумление, ибо обычно те, кто добивается быть допущенным пред царем по важным делам, ждут пять дней, а послы царя или больших городов с трудом добиваются придворного приема на третий день; но этих людей он счел достойными больших почестей, поскольку он имел столь великое почтение к их наставнику, и так он отослал тех, чье присутствие он счел излишним, и прогуливался, пока они не вошли и он смог приветствовать их.
\vs Ars 1:176
Когда они вошли с дарами, которые были посланы с ними, и драгоценными пергаментами, на которых был золотыми еврейскими письменами записан Закон (ибо пергамент был чудно изготовлен и соединение страниц было сделано так, что было невидимо), царь, как только
\vs Ars 1:177
увидел их, стал спрашивать их о книгах. И когда они вынули свитки из коробов и развернули их, царь долго простоял в безмолвии и поклонившись семь раз, он сказал: Благодарю вас, друзья, и еще более благодарю того, кто послал вас,
\vs Ars 1:178
превыше же всего Бога, вещавшего это. И когда все, посланные и прочие, бывшие там, вместе воскликнули в один голос: Бог да хранит царя!, он пролил слёзы радости. Ибо в душе его восторг и переполняющее ощущение оказанной ему чести
\vs Ars 1:179
побудили его плакать от счастья. Он повелел свернуть свитки обратно и затем, поклонившись этим мужам, сказал: Достойно было, люди Божии, мне уделить прежде всего почтение книгам, ради которых я призвал вас сюда, и теперь, после того, как я это сделал, протянуть вам десницу моей дружбы. Ради этого я
\vs Ars 1:180
сделал это прежде всего. Я дал указ о том, чтобы этот день, в который вы прибыли, стал считаться великим днем и стал ежегодно торжественно справляться в течение всей моей жизни. Вышло так, что это еще и годовщина
\vs Ars 1:181
моей победы на море над Антигоном. Поэтому я буду рад пировать с вами сегодня. Все, что вам может потребоваться, сказал он, будет приготовлено как подобает и вместе с вами и для меня тоже. И они выразили своё восхищение, и он повелел отвести их в лучший квартал, прилежащий к цитадели) и готовить пир.
\vs Ars 1:182
И Никанор вызвал главного дворцового распорядителя Дорофея, чиновника, особо назначенного смотреть за евреями, и приказал ему приготовить всё необходимое для каждого из них. Ибо так было установлено царем, и это установление вы увидите соблюденным сегодня. Ибо поскольку многие города имеют свои обычаи в том, что касается еды, питья и возлежания, есть особые чиновники, назначение которых узнавать, что им требуется. И всякий раз, когда те приходят к царю, для них готовят, соблюдая их собственные обычаи, так чтобы они не испытывали безпокойства, наслаждаясь посещением. Та же предусмотрительность была соблюдена и для еврейских посланцев. Дорофей же, назначенный старшим приставником при еврейских гостях, был
\vs Ars 1:183
человеком весьма тщательным. Ради такого пира он открыл все хранилища, бывшие под его надзором и державшиеся особо для подобных гостей. Он расположил сидения в два ряда согласно царскому повелению. Ибо он повелел ему усадить половину мужей справа от себя, а остальных позади, так чтобы он не лишил их величайшей из возможных чести. Когда они заняли сидения, он повелел Дорофею всё делать,
\vs Ars 1:184
сообразуясь с обычаями, принятыми среди еврейских гостей. Поэтому он прибег к услугам священных глашатаев и священников, приносящих жертвы, и прочих, кто привык возносить молитвы, и призвал одного из нашего числа, по имени Елеазар, старейшего из еврейских священников, вознести молитву вместо [себя]. И тот поднялся и сотворил превосходную молитву: Да обогатит
\vs Ars 1:185
тебя Всемогущий Бог, о царь, всяким благом, созданным Им, и да наградит Он тебя и твою жену, и твоих детей и твоих товарищей) непрерывным владычеством их во всю вашу жизнь. При этих словах поднялось громкое и радостное одобрение, длившееся весьма долго, и потом
\vs Ars 1:186
они обратились к наслаждению приготовленным пиром. Все распоряжения за столом совершались согласно внушениям Дорофея. Среди прислуживающих были юноши из свиты царя и иные, занимавшие почетные должности при царском дворе.
\vs Ars 1:187
Воспользовавшись моментом, когда пир приостановился, царь спросил посланца, занимавшего почетное место (ибо их расположили по старшинству), как ему сохранить царство
\vs Ars 1:188
неослабным до конца. Поразмыслив немного, тот отвечал: Лучше всего ты утвердишь его безопасность, если ты будешь подражать безконечной Божьей благости. Ибо если ты будешь являть милость и налагать кроткие наказания на тех, кто их заслуживает сообразно сделанному ими, ты
\vs Ars 1:189
обратишь их от зла и приведешь их к покаянию.
Царь похвалил ответ и затем спросил у еще одного мужа, как может совершать самое лучшее во всём. И тот отвечал: Если муж держится правильного отношения ко всему, он всегда будет поступать правильно во всяком случае, помня, что всякая мысль известна Богу. Если страх Божий станет для тебя исходною чертою, ты никогда не пройдешь мимо цели.
\vs Ars 1:190
Царь похвалил и этого мужа и спросил у иного, как приобрести друзей, мыслящих одинаково с собою. Тот отвечал: Если они будут видеть, что ты ревнуешь о нуждах множества, которым ты правишь, ты сам увидишь, как Бог одаривает Своими благодеяниями
\vs Ars 1:191
человеческий род, подавая им здоровье и пищу и всё остальное в должное время.
Выразив согласие с ответом, царь спросил следующего, как, принимая просящих и вынося суд, он может стяжать похвалу даже от тех, кто не добился исполнения иска. И тот сказал: Если твоя речь будет прилична для всех равно и ты никогда не будешь поступать надменно или как тиранн с
\vs Ars 1:192
преступниками. И ты будешь делать это, если рассмотришь образ деяний Божьих. Прошения достойных всегда исполняются, тогда как те, кто не получает ответа на свои молитвы, уведоляются снами или событиями о том, что в их просьбах было вредное, и что Бог не поражает их по их грехам или по величию Своей силы, но долготерпит им.
\vs Ars 1:193
Царь горячо похвалил мужа за его ответ и спросил следующего за ним, как он может стать непобедимым в делах войны. И тот ответил, что если он не будет полагаться только на множество своих сил и их воинственность, но будет непрестанно взывать к Богу привести начатое к счастливому исполнению, а сам
\vs Ars 1:194
же будет исполнять все свои обязанности в духе праведности.
Поблагодарив за ответ, он спросил другого, как он может сделаться грозен для своих врагов. И тот ответил, что если сохраняя мощный запас оружия и войска, он будет помнить, что этим нельзя добиться постоянного и окончательного итога. Ибо даже Бог внушает страх в умы людей, откладывая исполнение и лишь являя величие Своего могущества.
\vs Ars 1:195
Этого мужа царь похвалил и спросил следующего, что есть высшее благо в жизни. И тот ответил: Познать, что Бог есть Господь Вселенной, и что в конечном исполнении всех наших дел не мы добиваемся успеха, но Бог, Своею властью приводящий всё к завершению и ведущий нас к цели.
\vs Ars 1:196
Царь воскликнул, что этот муж ответил хорошо и затем спросил следующего, как он может сохранить всё, чем владеет в целости и в конце передать своим наследникам таким же. И тот ответил: Постоянно моля Бога вдохновить тебя высокою целью во всех твоих начинаниях и предупреждая твоих наследников не ослепляться молвою или богатством, ибо все эти дары подает Бог, а люди никогда сами по себе не достигают превосходства.
\vs Ars 1:197
Царь выразил своё согласие с ответом и осведомился у следующего гостя, как ему суметь переносить с душевным спокойствием всё, что выпадет ему. И тот сказал: Если ты будешь иметь твердое понимание того, что всем людям надлежит от Бога иметь часть как в величайшем зле, так и в величайшем благе, поскольку невозможно для человека уйти от этого. Но Бог, Которому мы все обязаны молитвою, вселяет в нас мужество претерпевать.
\vs Ars 1:198
Восхищенный ответами мужей, царь сказал, что все их ответы были хороши. Я задам еще один вопрос одному мужу, прибавил он, и тогда я прервусь на время, чтобы мы могли обратить наше внимание
\vs Ars 1:199
к наслаждению пиром и провести время с удовольствием. И тогда он спросил мужа, какова истинная цель мужества. И тот ответил: Если правильный замысел исполняется в час опасности в согласии с первоначальным намерением. Ибо всё совершается Богом к твоему преимуществу, о царь, когда твой умысел благ.
\vs Ars 1:200
Когда все своим одобрением выразили согласие с ответом, царь сказал философам (ибо их там было немало): По моему мнению, эти мужи сияют добродетелями и владеют необычайным знанием, ибо в мгновение они дали правильные ответы на те вопросы, что я им задавал, и все положили Бога источником своих слов.
\vs Ars 1:201
И Менедем, философ из Эритреи, сказал: Истинно, о царь, ибо вселенная управляется провидением, и поскольку мы верно постигаем то, что человек есть создание Бога, отсюда следует,
\vs Ars 1:202
что всякая сила и красота слова исходит от Бога. Когда царь показал, что он согласен с этим чувством, разговор прекратился, и они предались удовольствию. Когда наступил вечер, пир закончился.
\vs Ars 1:203
На следующий день они вновь сели за стол и продолжили пир в прежнем распорядке. Когда царь счел, что настал подходящий момент, чтобы предложить изследование, он стал задавать вопросы тем мужам, которые
\vs Ars 1:204
сидели вслед за отвечавшими накануне. Он приступил к началу беседы с одиннадцатым мужем, ибо десять уже отвечали на прежние вопросы. Когда установилось
\vs Ars 1:205
молчание, он спросил как он сможет и далее оставаться богатым. После краткого раздумья, муж, которому был задан вопрос, отвечал, что если он никогда не делал ничего недостойного своего звания, никогда не вел себя распутно, никогда не расточительствовал ради пустого и тщетного, но своею благотворительностью располагал подданных к себе. Ибо Бог Творец всяческого блага и
\vs Ars 1:206
Ему человек обязан послушанием.
Царь воздал этому хвалу и затем спросил у другого, как ему соблюсти истину. В ответ на вопрос тот сказал: Признав, что ложь наводит великий позор на всех людей, и еще более на царей. Ибо поскольку они имеют власть делать, что хотят, зачем им прибегать ко лжи? В прибавление к этому ты должен помнить, о царь, что Бог любит истину.
\vs Ars 1:207
Царь принял ответ с великим удовольствием и, глядя на другого, сказал: Что есть научение истине. И тот отвечал: Если ты не хочешь, чтобы зло случилось с тобою, но хочешь быть причастником всего благого, тогда ты должен соблюдать одно и то же в отношении к твоим подданным и к преступникам, и ты должен кротко увещевать благородного и доброго. Ибо Бог привлекает к Себе всех людей Своей благостью.
\vs Ars 1:208
Царь похвалил его и спросил у следующего по порядку, как ему быть другом людей. И тот ответил: Наблюдая то, что человеческий род возрастает и рождается во многом смятении и великом страдании. Посему ты не должен легкомысленно наказывать или подвергать их пыткам, поскольку ты знаешь, что человеческая жизнь состоит из боли и наказаний. Ибо если ты поймешь всё, ты преисполнишься жалости, ибо Бог также преисполнен жалости.
\vs Ars 1:209
Царь принял ответ с одобрением и спросил у следующего: Что есть основное отличие правления? Блюсти себя, отвечал тот, свободным от пьянства и соблюдать трезвость в течение большей части жизни, почитать праведность превыше всего и делать своими друзьями людей подобного рода. Ибо Бог любит также праведность.
\vs Ars 1:210
Выказав своё одобрение, царь спросил у другого: Что есть истинный признак благочестия? И тот ответил: Постигать то, что Бог непрестанно творит во Вселенной и знает всё, и ни один человек, делающий несправедливое и творящий развращенное, не может избежать Его взора. Поскольку Бог благотворит всему миру, также и ты должен подражать Ему и быть свободным от преступлений.
\vs Ars 1:211
Царь выказал своё согласие и сказал другому: Что есть сущность царствования? И тот ответил: Хорошо управлять собою и не уклоняться ради славы или богатства к неумеренным или непристойным желаниям, вот истинный путь правления, если ты хорошо уразумеваешь дело. Ибо всё, что тебе нужно, у тебя есть, и Бог свободен от нужд и добросердечен. Пусть твои мысли будут такими, какие достойны мужа, и желай немногого, но только того, что необходимо для правления.
\vs Ars 1:212
Царь похвалил его и спросил у другого мужа, как его разсуждения могут привести к наилучшему. И тот ответил, что если он будет постоянно полагать пред собою праведность во всём и думать, что неправедность равносильна лишению жизни. Ибо Бог всяческих обещает высочайшее благословение праведному.
\vs Ars 1:213
Похвалив его, царь спросил следующего, как он может быть свободен от безпокойных мыслей во время сна. И тот отвечал: Ты задал мне вопрос, на который очень трудно ответить, ибо мы не можем руководить собою в часы сна, но твердо удерживаемся в них
\vs Ars 1:214
воображением, которое не может управляться разумом. Ибо наши души обладают чувствами, которые на самом деле видят то, что входит в наше сознание во время сна. Но мы ошибаемся, если мы полагаем, что мы на самом деле плывем на корабле по морю или летаем по воздуху или путешествуем по другим странам или что-нибудь иное в этом роде. И всё-таки мы на самом деле воображаем, что
\vs Ars 1:215
эти вещи происходят. Насколько я могу решить, я достиг следующего решения. Ты должен, о царь, управлять своими словами и делами в законе благочестия всяким возможным образом, так чтобы ты мог сознавать, что ты соблюдаешь добродетель и никогда не решал вознаграждать себя за расточение разума и никогда не превышать своей власти до того, чтобы
\vs Ars 1:216
презирать праведность. Ибо ум по большей части занимается во сне тем же, чем он занят, бодрствуя. И тот, кто направил все свои мысли и дела к самым благородным целям, утверждает себя в праведности и во время бодрствования и во время сна. Благодаря этому ты можешь пребывать неуклонно в постоянном самоблюдении.
\vs Ars 1:217
Царь воздал хвалу мужу и сказал другому: Поскольку ты отвечаешь десятым, то когда ты выскажешься, мы предадимся пиру. И затем он задал вопрос:
\vs Ars 1:218
Как я могу избегать недостойных дел сам по себе. И тот ответил: Всегда обращай внимание на твою славу и твоё верховное звание, так чтобы ты мог говорить и думать только то,
\vs Ars 1:219
что совместимо с ними, зная что все твои подданные думают и говорят о тебе. Ибо ты не должен казаться хуже актеров, которые тщательно изучают свои роли, которые им надо сыграть, и сообразовывают с ними все свои дела. Ты не играешь роль, но ты истинный царь, поскольку Бог наделил тебя царскою властью, чтобы ты соблюдал её вместе с твоей славой.
\vs Ars 1:220
Когда царь похвалил громко и долго и весьма любезно, гостей стали побуждать отдохнуть. И так, когда прекратился разговор, они предались течению пира.
\vs Ars 1:221
На следующий день всё было устроено по-прежнему, и когда царь уловил подходящий момент, чтобы задавать вопросы мужам, он спросил первого из тех, кто оставался
\vs Ars 1:222
неспрошенным: Что есть высший образ правления? И тот ответил: Управлять собою и не поддаваться порывам. Ибо каждй человек от природы обладает особым умственным увлечением.
\vs Ars 1:223
Возможно, большинство людей склонны к еде, питью и наслаждениям, а царь увлекается приобретением земель и великою славой. Но хорошо, когда во всём этом соблюдается умеренность. Что Бог дает, то мы должны принимать и хранить, но никогда не стремиться приобретать то, чего мы не в силах достичь.
\vs Ars 1:224
Довольный этими словами царь спросил у следующего, как ему быть свободным от зависти. И после краткого молчания тот ответил: Если ты прежде всего будешь смотреть на то, что славою и великим богатством всех царей наделяет Бог, а не сам царь своею властью. Все люди хотят иметь такую славу, но не могут, потому что это дар Божий.
\vs Ars 1:225
Царь похвалил мужа в длинной речи и затем спросил у другого, как ему научиться презирать врагов. И тот ответил: Если ты будешь выказывать добродушие ко всем и добиваться их дружбы, тебе не надо будеть никого бояться. Пользоваться всеобщею любовью это наилучший из даров, которые можно получить от Бога.
\vs Ars 1:226
Похвалив этот ответ, царь велел следующему мужу ответить на вопрос, как он может сохранить свою высокую славу. И тот отвечал так: Если ты будешь щедр и открыт сердцем, оделяя других добродушием и благодеяниями, ты никогда не утратишь твою славу, но если ты хочешь, чтобы и впредь милость пребывала с тобою, ты должен постоянно призывать Бога.
\vs Ars 1:227
Царь изъявил своё согласие и спросил у следующего, кому надлежит мужу являть щедрость. И тот ответил: Всеми признано, что нам надлежит являть щедрость по отношению к тем, кто благорасположен к нам, но я думаю, что нам должно являть то же самое щедрое расположение духа и к тем, кто враждебен нам, чтобы мы могли привлечь их к правде и выгоде для них самих. Но мы должны молить Бога о том, чтобы это совершилось, ибо Он управляет умами всех.
\vs Ars 1:228
Выразив своё согласие с ответом, царь просил шестого мужа ответить на вопрос, кому мы должны выказывать благодарность. И тот ответил: Нашим родителям постоянно, ибо Бог дал нам важнейшую из заповедей относительно должного почитания родителей. Затем Он поместил отношение к друзьям, ибо Он говорит: друг словно твоя душа. Хорошо тебе постараться сделать всех твоими друзьями.
\vs Ars 1:229
Царь сказал ему ласковое слово и затем спросил следующего, что по цене подобно красоте. И тот сказал: Благочестие, ибо оно есть преимущественный образ красоты, и его сила в любви, она же Божий дар. Ты уже приобрел это, и вместе все благословения жизни.
\vs Ars 1:230
Царь с великою любезностью одобрил ответ и спросил у другого, как ему, совершив ошибку, вновь вернуть своё имя на прежнюю степень. И тот сказал: Невозможно тебе совершить ошибку, ибо во всех людях ты взыскал семена благодарности, приносящие урожай благожелательности,
\vs Ars 1:231
который могущественней самого сильного оружия и обезпечивает величайшую безопасность. Но если кто-либо совершает ошибку, он никогда не должен повторять того, что привело к ней, но ему надлежит сообразовываться с дружбою и творить справедливость. Ибо дар от Бога быть способным творить дела добра, а не противные ему.
\vs Ars 1:232
Восхищенный этими словами царь спросил у другого, как ему быть свободным от печали. И тот ответил: Если ты никогда никому не причинял вреда, но творил добро и следовал стезею
\vs Ars 1:233
праведности, ибо её плоды дают свободу от печали. Но нам надлежит молить Бога о том, чтобы нежданное зло вроде смерти, болезни, скорби или чего-либо в этом роде не нашло на нас и не принесло вреда. Но поскольку ты предался благочестию, никакое из этих несчастий никогда не постигнет тебя.
\vs Ars 1:234
Царь одарил его великою похвалой и спросил десятого, что есть высший образ славы. И тот ответил: Почитать Бога, и делать это не [только] дарами и жертвоприношениями, но в чистоте души и святом убеждении, поскольку всё образовано и управляется Богом по Его воле. Этой цели ты добиваешься неотменно, как это видно для всех в твоих свершениях в прошлом и настоящем.
\vs Ars 1:235
Громким голосом царь поблагодарил их всех и говорил к ним милостиво и выразил своё одобрение всем, кто был тут, особенно же философам. Ибо те превосходили их и поведением и доводами, поскольку они положили Бога источником себе. Затем царь, чтобы показать свои добрые чувства, начал пить за здоровье гостей.
\vs Ars 1:236
На следующий день пир был приготовлен так же, и царь, как только представился момент, стал задавать вопросы мужам, которые сидели вслед за теми, кто уже отвечал; и у первого он спросил: Можно ли научить мудрости? И тот сказал: Душа устроена так, что может божественною силою принять всё доброе и отвергнуть противное.
\vs Ars 1:237
Царь выразил одобрение и спросил следующего мужа: Что самое полезное для здоровья? И тот сказал: Умеренность, а её невозможно обрести, доколе Бог не внушит расположение к ней.
\vs Ars 1:238
Царь сказал ему ласковое слово и спросил другого: Как человек может достойно заплатить долг благодарности родителям? И тот сказал: Никогда не причиняя им скорби, а это невозможно, доколе Бог не расположит ум к поиску самых благородных целей.
\vs Ars 1:239
Царь выразил согласие и спросил следующего, как ему сделаться жаждущим слушателем. И тот сказал: Помня, что всякое знание полезно, потому что с Божьей помощью оно делает тебя способным во время опасности выбрать что-либо из того, что ты изучил и применить против нашедшей на тебя напасти. И усилия человеческие в этом исполняются через присутствие Божие.
\vs Ars 1:240
Царь похвалил его и спросил другого, как ему избежать делать что-либо противное закону. И тот сказал: Если ты признаешь, что в сердца законодателей помышления охранять человеческие жизни вложил Бог, ты последуешь им.
\vs Ars 1:241
Царь подтвердил ответ этого мужа и сказал другому: В чем преимущество родства? И тот ответил: Если мы обратим внимание на то, что мы сами поражаемся невзгодами, выпадающими нашим ближним, и если их страдания делаются нашими, тогда сразу
\vs Ars 1:242
же становится явною сила родства, ибо лишь явив такие чувства, мы приобретем честь и достоинство в их глазах. Ибо помощь, когда она связана с добротою, есть сама в себе связь, разорвать которую никак нельзя. И в день их процветания мы не должны вожделеть того, чем они владеют, но молить Бога даровать им блага всякого рода.
\vs Ars 1:243
И вознаградив его тою же похвалою, что и прочих, царь спросил другого, как ему достичь свободы от страха. И тот отвечал: Когда ум сознает, что он не творил зла, и когда Бог направляет его на всякий благородный замысел.
\vs Ars 1:244
Царь выразил одобрение и спросил другого, как ему всегда приходить к правильному суждению. И тот отвечал, что если он будет постоянно держать перед глазами выпадающие людям невзгоды и признавать, что Бог иных лишает благоденствия, а других приводит к великим почестям и славе.
\vs Ars 1:245
Царь оказал мужу ласковый прием и попросил другого ответить на вопрос: Как ему избегать жизни в праздности и наслаждениях? И тот отвечал, что если он будет постоянно помнить о том, что он правитель великого царства и господин над великим множеством, и что его ум не должен заниматься иными вещами, но должен всегда изследовать, как ему наилучшим образом обезпечить их благоденствие. Он также должен молиться Богу о том, чтобы ничто из должного не было в пренебрежении.
\vs Ars 1:246
Воздав ему хвалу, царь спросил десятого, как ему распознать тех, кто поступает с ним лукаво. И тот ответил на вопрос: Если он будет наблюдать, насколько естественно их отношение к нему, и придерживаются ли они правил старшинства во время приемов и совета, а в ежедневном общении выходят ли когда-нибудь за пределы
\vs Ars 1:247
приличий в поздравлениях и в прочих манерах. Но Бог направит твой ум, о царь, ко всему благородному. Когда царь выразил своё громкое одобрение и похвалил всех по-одному (и вместе их похвалили все, кто был там), они обратились к радостям пира.
\vs Ars 1:248
И на следующий день, когда подошло время, царь спросил следующего мужа: Что есть грубейший вид пренебрежения? И тот ответил: Если человек не заботится о своих детях и не прилагает всяческих усилий к их воспитанию. Ибо мы всегда молим Бога не столько о самих себе, сколько о наших детях, чтобы им стяжать всякое благословение. Наши пожелания о том, чтобы наши дети сумели владеть собою, может быть исполнено только Божьей властью.
\vs Ars 1:249
Царь сказал, что он говорил хорошо, и потом спросил другого, как ему любить отечество. Сохраняя в уме, отвечал тот, мысль о том, что хорошо жить и умереть для своей страны. Пребывание на чужбине наводит презрение на бедняка и позор на богача, как если бы они были изгнаны за преступление. Если ты даруешь благодеяния всем, как ты это делаешь постоянно, Бог даст тебе милость во всём, и ты будешь признан любящим отечество.
\vs Ars 1:250
Выслушав этого мужа, царь спросил у следующего по порядку, как ему жить в дружбе с женою. И тот ответил: Признавая, что женщины по природе упорны и деятельны, когда добиваются исполнения своих желаний, и склонны резко менять свои суждения от ложных разсуждений, и их природа слаба по сути своей. Необходимо быть мудрым в обращении с ними
\vs Ars 1:251
и не порождать споров. Для того, чтобы пройти жизнь успешно, кормчий должен знать цель, к которой ему надлежит направиться. Лишь призыванием Божьей помощи люди будут соблюдать истинное направление жизни во всякое время.
\vs Ars 1:252
Царь выразил своё согласие и спросил следующего, как ему быть свободным от заблуждений. И тот отвечал: Если ты всегда будешь действовать с разсуждением и никогда не давать веры клевете, но сам испытывать то, что тебе говорят, и решать своим собственным судом просьбы, которые тебе приносят, и приводить всё на свет твоего суждения, ты будешь свободен от заблуждения, о царь! Но познание и исполнение этого есть дело Божественной силы.
\vs Ars 1:253
Восхищенный этими словами царь спросил другого, как ему быть свободным от гнева. И тот сказал в ответ на вопрос, что если он будет признавать, что он имеет власть надо всеми вплоть до предания их смерти, если он даст место гневу и будет безполезно и жалко, если он, потому что он повелитель,
\vs Ars 1:254
лишит жизни многих. Что за нужда гневаться, когда все покорны и никто не противится ему? Надлежит признавать, что Бог правит всем миром в духе добросердечия и без гнева во всём, и ты, о царь, сказал он, необходимо должен следовать Его примеру.
\vs Ars 1:255
Царь сказал, что он отвечал хорошо и потом спросил у следующего мужа: Что есть добрый совет? Делать добро во всякое время и с должным разсуждением, объяснил тот, сравнивая то, что выгодно для твоих дел с уроном, который может быть следствием принятия противного решения, и таким образом взвешивая каждый шаг, мы можем умудряться, и наши намерения могут быть исполнены. И самый важный из всех твой замысел Божьей силою найдет своё завершение потому, что ты соблюдаешь благочестие.
\vs Ars 1:256
Царь сказал, что этот муж отвечал хорошо, и спросил другого: Что такое философия? И тот объяснил: Правильное обсуждение всякого возникшего вопроса с тем, чтобы никогда не увлекаться порывами, но взвешивать всякий ущерб, причиняемый страстями, и действовать в правильной сообразности с тем, как того требуют обстоятельства, соблюдая умеренность. Но мы должны молиться Богу вселить в наш ум почтительное отношение к этому.
\vs Ars 1:257
Царь выразил своё согласие и спросил другого, как ему встречаться с признательностью, путешествуя в чужих странах. Будучи любезным ко всем, отвечал тот, казаться ниже, нежели выше тех, с кем ты путешествуешь. Ибо признано правило, что Бог по Своей истинной природе приемлет смиренного. А человеческий род любит тех, кто охотно подчиняется им.
\vs Ars 1:258
Выразив своё одобрение с этим ответом, царь спросил другого, как ему строить так, чтобы построенное сохранилось после него. И тот ответил на вопрос, что если сделанное им будет принадлежать к числу прекрасного и благородного, так что обладатели сохранят это ради его красоты, и он никогда не лишит себя тех, кто делает подобные вещи, и никогда не будет вынуждать других служить его
\vs Ars 1:259
нуждам без вознаграждения. Ибо наблюдая, как Бог печется о человеческом роде, наделяя его здоровьем и умственными способностями и иными дарами, он сам должен следовать Его примеру, воздавая людям вознаграждение за их тяжкий труд. Ибо дела, творимые в праведности, пребывают всегда.
\vs Ars 1:260
Царь сказал, что этот муж также отвечал правильно и спросил у десятого: Что есть плод мудрости? И тот ответил: То, что муж должен сознавать в себе, что он не делал зла
\vs Ars 1:261
и что ему надлежит прожить жизнь в истине, ибо от этого, о могучий царь, величайшая радость и стойкость души и крепкая вера в Бога умножатся в тебе, если ты будешь править твоим царством в благочестии. И когда они выслушали ответ, они все приветствовали его громкими восклицаниями, и после этого царь, преисполнившись радости, стал пить за их здоровье.
\vs Ars 1:262
И на следующий день пир продолжился также, как и в предыдущие, и когда подошел момент, царь стал задавать вопросы оставшимся гостям, и
\vs Ars 1:263
сказал первому: Как человеку уберечься от гордыни? И тот ответил: Если он остается уравновешенным и во всех обстоятельствах помнит, что он муж, правящий людьми; и: Бог низводит гордых в ничто, и возносит слабых и смиренных.
\vs Ars 1:264
Царь сказал ему милостивое слово и спросил следующего: Кого мужу следует избирать себе в советники? И тот ответил: Тех, кто был испытан во многих делах и сохранил непревратной благожелательность по отношению к нему и разделял с ним его намерения. И Бог Сам является тем, кто достоин исполнения этих целей.
\vs Ars 1:265
Царь похвалил его и спросил другого: Чем прежде всего надлежит владеть царю? Дружбою и любовью своих подданных, отвечал тот, ибо через них узы доброжелательности делаются нерасторжимыми. И Бог делает прочным это настолько, чтобы оно происходило согласно твоему желанию.
\vs Ars 1:266
Царь похвалил его и спросил у другого: Что есть цель слова? И тот ответил: Убедить твоего противника, показав ему его ошибки хорошо и правильно выстроенными доводами. Ибо так ты приобретешь слушателя, не споря с ним, но хваля его с тем, чтобы убедить его. А убеждение совершается силою Божьей.
\vs Ars 1:267
Царь сказал, что он дал хороший ответ, и спросил другого, как ему жить в дружбе со множеством различных племен, составляющих население его царства. Делая то, что надлежит в отношении каждого из них, отвечал тот, и беря себе праведность вождем, как ты это делаешь ныне с помощью проницательности, которою тебя наделил Бог.
\vs Ars 1:268
Царь был восхищен этим ответом и спросил у другого: При каких обстоятельствах человеку надлежит выносить скорбь? В невзгодах, которые выпадают нашим друзьям, ответил тот, когда мы видим, что они длительны и неизгонимы. Разум не дает нам печалиться о тех, кто умер и освободился от зла, но все люди печалятся о них, потому что думают, лишь о себе и своей собственной выгоде. Одною лишь силою Божьей мы можем избежать всякого зла.
\vs Ars 1:269
Царь сказал, что тот дал приличный ответ, и спросил другого: Как утрачивают доброе имя? И тот отвечал: Когда гордыня и необузданная самоуверенность держат власть, рождаются позор и потеря доброго имени. Ибо Бог есть Господь доброй славы и наделяет ею, когда хочет.
\vs Ars 1:270
Царь подтвердил ответ и спросил следующего мужа, кому люди должны доверяться. Тем, отвечал тот, кто служит тебе по доброй воле, а не из страха или своего интереса, помышляя о собственной наживе. Ибо знак любви это одно, а другое признак дурной воли и временного услужения. Ибо человек, который всегда ищет собственной наживы, в сердце своём предатель. Но ты владеешь привязанностью всех твоих подданных с помощью разсудительности, которою тебя одарил Бог.
\vs Ars 1:271
Царь сказал, что тот отвечал мудро, и спросил у другого, чем сохраняется царство. И тот ответил на этот вопрос: Заботой и предусмотрительностью о том, чтобы никакое зло не было сделано теми, кто стоит у власти над народом, и тем, что ты всегда делаешь с Божьей помощью, внушающей тебе важные суждения.
\vs Ars 1:272
Царь сказал ему одобряющее слово и спросил следующего: Чем соблюдаются благодарность и честь? И тот ответил: Добродетелью, ибо она есть творец добрых дел, и она уничтожается злом, даже если ты выказываешь благородство характера ко всем благодаря Божьему дару, которым Он наделил тебя.
\vs Ars 1:273
Царь милостиво принял ответ и спросил одиннадцатого (поскольку их было семьдесят и еще двое), как во время войны ему сохранять душевное спокойствие. И тот отвечал: Помня, что он не сделал зла никому из подданных, и что все они в ответ будут сражаться за него ради благодеяний, которые они получили, зная, что даже если они потеряют жизнь, ты позаботишься о тех,
\vs Ars 1:274
кто находится на их иждивении. Ибо ты никогда не забудешь о таком воздаянии, таково добросердечие, которым тебя одарил Бог.
Царь громко похвалил их всех и говорил с ними милостиво и потом много пил за здоровье каждого, сам предаваясь радости и осыпая своих гостей самыми щедрыми и радостными выражениями дружбы.
\vs Ars 1:275
На седьмой день были сделаны самые обильные приготовления и собраны и другие мужи от различных городов (и среди них большое число посланников). Когда подошел момент, царь спросил у первого из тех, кто еще не был спрошен, как ему избегать
\vs Ars 1:276
обмана неверным разсуждением. И тот ответил: Обращая тщательное внимание на говорящего, на то, что говорится и на обсуждаемый предмет, и задавая те же самые вопросы через некоторое время по-другому. Но обладать зорким умом и уметь выносить здравое суждение в каждом случае есть один из благих даров от Бога, и ты имеешь его, о царь!
\vs Ars 1:277
Царь громко одобрил ответ и спросил у другого: Почему большинство людей никогда не бывает добродетельно? Потому что, отвечал тот, все люди неумеренны по природе и склонны
\vs Ars 1:278
к наслаждениям. Отсюда возникает неправедность и потоком любостяжание. Обычай добродетели препятствие для тех, кто отдает жизнь на наслаждения, потому что он обязывается их предпочитать умеренности и праведности. Ибо хозяин сего Бог.
\vs Ars 1:279
Царь сказал, что он отвечал хорошо, и спросил [следующего], чему должны повиноваться цари. И тот сказал: Законам, так чтобы праведными поступками они могли возвращать людям жизнь. Когда ты делашь это в повиновении Божественным заповедям, ты откладываешь для себя запас в хранилищах вечной памяти.
\vs Ars 1:280
Царь сказал, что этот муж также говорил хорошо, и спросил у следующего: Кого мы должны назначать наместниками? И тот ответил: Всех тех, кто ненавидит порочность и, подражая твоему поведению, творит праведность с тем, чтобы ему поддерживать постоянно своё доброе имя. Ибо это то, что делаешь ты, о могучий царь, сказал он, и Бог возложил на тебя венец праведности.
\vs Ars 1:281
Царь громко приветствовал этот ответ и затем, глядя на следующего мужа, сказал: Кого мы должны назначать начальниками над войсками? И тот объяснил: Тех, кто блистает храбростью и праведностью и тех, кто заботится более о том, чтобы беречь своих людей, чем о том, чтобы добиться победы, опрометчиво подвергая опасности их жизнь. Ибо так же, как Бог действует ко благу всех людей, так и ты, последуя Ему, творишь благо для всех твоих подданных.
\vs Ars 1:282
Царь сказал, что он дал хороший ответ и спросил у другого: Какой человек достоин восхищения? И тот ответил: Человек, наделенный доброю славой и богатством и силой, и душа которого относится одинаково ко всему. Ты сам своими деяниями являешь себя наидостойнейшим восхищения благодаря Божьей помощи, направляющей твоё внимание к этому.
\vs Ars 1:283
Царь выразил своё одобрение и сказал следующему: Чему царь должен уделять наибольшее время? И тот отвечал: Чтению и изучению записей путешествий твоих чиновников, описывающих различные царства для того, чтобы изменять и охранять твоих подданных. И благодаря таким деяниям ты достиг славы, которой не приближался никто другой с Божьей помощью, исполняющей все твои желания.
\vs Ars 1:284
Царь сказал воодушевленное слово этому мужу и спросил у следующего, чем заниматься человеку в часы отдыха и развлечения. И тот отвечал: Смотреть те игры), в которых можно соблюсти уместность и представлять перед глазами сцены, взятые из жизни и показанные
\vs Ars 1:285
достойно и прилично, и полезно и благопристойно. Ибо даже в таких развлечениях можно найти некое наставление, потому что часто какой-нибудь нужный урок извлекается из самого незначительного житейского обстоятельства. Но соблюдая наивысшую пристойность во всех твоих деяниях, ты явил себя философом и был почтен у Бога ради твоей добродетели.
\vs Ars 1:286
Царь, обрадованный весьма хорошо сказанными словами, сказал девятому мужу: Как надлежит вести себя человеку на пирах? И тот ответил: Тебе надлежит собирать у себя ученых мужей и тех, кто способен указать тебе нечто полезное касательно дел твоего царства и жизни твоих подданных (ибо тебе не найти предмета, более подходящего или более
\vs Ars 1:287
наставительного, чем это), поскольку эти люди угодны Богу, потому что они обучили свой ум созерцать самые благородные предметы, как и ты сам, конечно, делаешь это, поскольку все твои дела направляются Богом.
\vs Ars 1:288
Восхищенный ответом, царь спросил у следующего мужа: Что для народа лучше всего? То ли, что царем над ним может стать частный гражданин или же член царской семьи? И тот
\vs Ars 1:289
ответил: Лучший по природе. Ибо цари, происходящие от царской крови, часто грубы и суровы к своим подданным. И всё же часто бывает с некоторыми из тех, кто вознесся из рядов частных граждан, что испытав злое и пожив некое время
\vs Ars 1:290
в бедности, они, управляя многими, становятся свирепее безбожных тираннов. Но, как я сказал, добрая природа, получив пристойное обучение, способна к управлению, и ты великий царь, не столько потому, что блистаешь славою твоего правления и твоими богатствами, сколько оттого, что ты превзошел всех людей в милости и человеколюбии, благодаря Богу, одарившему тебя этими достоинствами.
\vs Ars 1:291
Царь некоторое время восхвалял этого мужа, а после спросил самого последнего: Что есть величайшее свершение в правлении царством? И тот отвечал: То, когда все подданные могут непрерывно жить в мире, и правосудие быстро разрешает споры.
\vs Ars 1:292
Это может быть достигнуто благодаря силе правителя, когда это муж, ненавидящий зло и любящий благо и отдающий свои устремления спасению человеческой жизни, точно также, как ты считаешь несправедливость худшею разновидностью зла и своим управлением создал себе безсмертное доброе имя, поскольку Бог даровал тебе ум чистый и незапятнанный злом.
\vs Ars 1:293
И когда он закончил, раздались длительные, громкие и радостные голоса одобрения. Когда они замолкли, царь взял чашу и сказал слово в честь всех своих гостей и тех слов, что они произнесли. И в конце он сказал: Я извлек величайшую выгоду из вашего прихода,
\vs Ars 1:294
я многое получил от мудрого учения, преподанного вами мне об искусстве правления. Потом он повелел дать каждому по три таланта серебра и указал одному из своих рабов раздать деньги. Тут же все восклицанием выразили своё согласие, и пир превратился в зрелище радости, в то время как царь предался непрерывной череде застолья.
\vs Ars 1:295
Я писал долго и должен просить тебя о прощении, Филострат. Я был безмерно изумлен этими мужами и тем, как они мгновенно давали ответы,
\vs Ars 1:296
требующие поистине длительного размышления. Ибо несмотря на то, что спрашивающий предлагал каждому сложную мысль в каждом вопросе, отвечавшие один за другим давали свои ответы уже готовыми тотчас же и так, что мне и всем, кто был там, а особенно философам, они казались достойными восхищения. И я полагаю, что это покажется невероятным тем, кто
\vs Ars 1:297
прочтет мой рассказ в будущем. Но не подобает искажать то, что записано в государственных архивах. И неправедным с моей стороны было бы извращать подобный предмет. Я рассказываю эту историю так, как она происходила, тщательно избегая любой ошибки. Сила их высказываний запечатлелась во мне настолько, что я особенно переговорил с теми, чья обязанность состоит в том, чтобы
\vs Ars 1:298
записывать всё, что происходит на царских приемах и пирах. Ибо существует, как ты знаешь, обычай с того момента, как царь затевает какое-либо дело и вплоть до того, как он удаляется на покой, вести запись всего, что он говорит или делает, устроение превосходное и полезное.
\vs Ars 1:299
Ибо на следующий день всё по времени, что говорилось и делалось накануне, читается прежде начала дела, и если находится в этом какая-нибудь неправильность, её тотчас же исправляют.
\vs Ars 1:300
И благодаря этому, как уже сказано, я добился точных сведений из государственных архивов и расположил события в правильном порядке, поскольку я знаю, сколь жадно ты стремишься получать полезные сведения.
\vs Ars 1:301
Через три дня Димитрий собрал этих мужей, проведя их вдоль дамбы за семь стадий к острову, перешел мост и направился к северным кварталам Фароса. Там он собрал их в доме, построенном на дамбе, весьма красивом и уединенном, и пригласил их приступить к переводу, поскольку всё, что было им нужно для этого
\vs Ars 1:302
находилось в их распоряжении. И так они принялись за работу, сравнивая сделанное каждым и согласовываясь между собою, и всё согласованное надлежащим образом копировалось под руководством Димитрия.
\vs Ars 1:303
И работа длилась до девятого часа, после чего они могли свободно исполнять свои
\vs Ars 1:304
телесные нужды. Всё, в чем они нуждались, им доставлялось весьма щедро. В дополнение к этому Дорофей приготовлял для них ежедневно то же, что и для самого царя, ибо таково было повеление царя. Каждое утро они рано появлялись во дворце, и,
\vs Ars 1:305
поклонившись царю, возвращались на своё место. И по еврейскому обычаю они мыли руки в море и молились Богу и затем предавались чтению и
\vs Ars 1:306
переводу того или иного начатого места; и я спросил у них, почему они моют руки перед тем, как молятся. И они объяснили, что это делается в знак того, что они не делают зла (ибо всё, что ни делается, делается руками), поскольку благородно и свято видят во всем символ праведности и истины.
\vs Ars 1:307
Как я уже сказал, они собирались ежедневно на месте, восхитительном из-за его спокойствия и веселости, и принимались за работу. И так случилось, что весь перевод был исполнен за семьдесят два дня, будто бы преднамерено заранее.
\vs Ars 1:308
И когда работа была закончена, Димитрий собрал всё еврейское население туда, где работали переводчики, и прочитал им перевод вслух в присутствии переводчиков, которых также, как и народ, принял с почестями из-за великих благодеяний, которые они привлекли
\vs Ars 1:309
на него. Они горячо похвалили Димитрия и побуждали его переписать весь Закон целиком и передать список их начальникам.
\vs Ars 1:310
После того, как книга была прочитана, священники и старейшие из переводчиков, и еврейская община, и начальники народа поднялись и сказали, что теперь, когда совершен столь превосходный и столь тщательный перевод, будет только справедливо, если он останется таким, каков он есть и никакое
\vs Ars 1:311
изменение не будет внесено в него. И когда все выразили согласие с этим, они просили произнести проклятие согласно их обычаю на тех, кто внесет туда какое-нибудь изменение, или прибавив что-нибудь или изменив как-нибудь хоть одно слово из написанного, или опустив что-нибудь. Это было весьма мудрою предосторожностью для того, чтобы наверняка сохранить эту книгу неизменной на всё время в будущем.
\vs Ars 1:312
Когда об этом доложили царю, он весьма обрадовался, ибо он чувствовал, что его желание было исполнено в целости. Вся книга была прочитана ему вслух, и он весьма изумился духу законодателя. И он сказал Димитрию: Как это никто из историков или поэтов не подумал когда-либо посвятить хотя бы слово столь чудесному
\vs Ars 1:313
деянию? И Димитрий ответил: Потому что Закон свят и исходит от Бога. И некоторые из имевших намерение прикоснуться к нему были поражены Богом и отступили от
\vs Ars 1:314
своих замыслов. Он сказал, что слышал от Феопомпа, как тот на тридцать дней потерял разсудок, потому что попытался вставить в свою историю какие-то события из раннего и малодостоверного перевода Закона. Когда он немного пришел
\vs Ars 1:315
в себя, он просил Бога открыть ему, за что ему выпала эта напасть. И ему было открыто во сне, что из пустого любопытства он хотел передать священную истину обычным людям, и что если он отступится, он обретет вновь здоровье. Я также слышал из уст
\vs Ars 1:316
Феодекта, одного из тех поэтов, что пишут трагедии, что когда он попытался переделать кое-что из происшествий, записанных в этой книге для своей трагедии, оба его глаза покрылись бельмами. Когда он понял причину своего несчастья, он много дней молил Бога и затем обрел здоровье.
\vs Ars 1:317
И после того, как царь, как я уже сказал, получил от Димитрия объяснение этому вопросу, он поклонился и повелел, чтобы с книгами обращались с великою тщательностью, и чтобы их
\vs Ars 1:318
хранили как святыню. И он горячо приглашал переводчиков часто навещать его после того, как они возвратятся в Иудею, ибо, говорил он, лишь так будет справедливо, если теперь он отпустит их домой. Но когда они придут снова, он
\vs Ars 1:319
примет их как своих друзей, как подобает, и они получат богатые подарки от него. Он повелел сделать приготовления к их возвращению домой и явил к ним великую щедрость. Он подарил каждому три превосходнейших одежды, два золотых таланта, сундук весом в талант, всё, что необходимо для трех лож.)
\vs Ars 1:320
А в сопровождение он послал Елеазару десять лож на серебряных ножках и всё необходимое к ним, сундук ценою в тридцать талантов, десять одежд, багряницу и великолепный венец, и сто штук тончайшей шерсти, а также кубки и блюда, и две золотых чаши для посвящения Богу.
\vs Ars 1:321
Он просил его также в письме о том, что если кто из этих мужей захочет вернуться к нему, не препятствовать ему. Ибо он считал за честь наслаждаться обществом столь ученых мужей, предпочитая расточать свои богатства на них, нежели на суетное.
\vs Ars 1:322
И теперь, Филострат, ты получил полный рассказ, согласно моему обещанию. Я полагаю, ты испытаешь большее удовольствие от этого, чем от сочинений баснописцев. Ибо ты предан тому, что может принести пользу душе, и уделяешь этому много времени. Я постараюсь рассказать о каких-нибудь еще событиях, стоящих записывания, так что внимательно читая их, ты сможешь удостоверить высочайшую награду за твоё усердие.

\bibbookdescr{Tjb}{
  inline={Завещание Иова,\\непорочного, жертвы, завоевателя во многих соревнованиях},
  toc={Завещание Иова},
  bookmark={Завещание Иова},
  header={Завещание Иова},
  abbr={Зав~Иов}
}
\vs Tjb 1:1
Однажды он стал больным, и, зная, что он должен будет оставить свою телесную обитель, он призвал своих семерых сыновей (их имена: Терси, Хор, Гион, Никэ, Фор, Фиф, Фруон) и своих трех дочерей вместе и сказал им так:
\vs Tjb 1:2
Составьте круг около меня, дети, и послушайте, и я разскажу вам, что Господь сделал для меня, и всё, что происходило со мною.
\vs Tjb 1:3
Ибо я~--- Иов, ваш отец.
\vs Tjb 1:4
К тому же знайте, мои дети, что вы~--- род избранный, и примите во внимание ваше благородное рождение.
\vs Tjb 1:5
Ибо я~--- из сынов Исава. Мой брат Нерос, и ваша мать Дина. Чрез неё я стал отцом вашим.
\vs Tjb 1:6
Ибо моя первая жена умерла с моими другими десятью детьми горькой смертью.
\vs Tjb 1:7
Послушайте теперь, дети, и я открою вам, что происходило со мною.
\vs Tjb 1:8
Я был очень богатым человеком, живущим на Востоке в земле Уц, и прежде, чем Господь назвал меня Иовом, я был назван Иовав.
\vs Tjb 1:9
Начало моего испытания было таким. Возле моего дома был идол одного поклоняющегося [ему] народа; и я постоянно видел жертвоприношения ему как богу.
\vs Tjb 1:10
Тогда я подумал и сказал в себе: Тот ли это, который сотворил небо и землю, море и нас всех? Как я узнаю истину?
\vs Tjb 1:11
И в ту ночь, когда я лег спать, пришел глас и позвал: Иовав! Иовав! Встань, и я поведаю тебе, Кто~--- Тот, Кого ты пожелал узнать.
\vs Tjb 1:12
Тот, впрочем, кому люди приносят всесожжения и возлияния,~--- не Бог, но это~--- сила и действо Соблазнителя, которыми он обманывает людей.
\vs Tjb 1:13
И когда я слушал это, я пал на землю и я простерся, говоря:
\vs Tjb 1:14
O господин мой, который говорит для спасения души моей! Я прошу тебя, если это~--- идол Сатаны, я прошу тебя, позволь мне пойти отсюда и уничтожить его и очистить это место.
\vs Tjb 1:15
Ибо вот нет ни одного, кто может запретить мне сделать это, потому что я~--- царь этой земли; так чтобы те, что живут в ней, больше не вводились в заблуждение.
\vs Tjb 1:16
И глас, который говорил из пламени, ответил мне: Ты можешь очистить это место.
\vs Tjb 1:17
Но вот, я возвещаю тебе то, что Господь повелел мне, чтобы я сообщил тебе, ибо я~--- архангел Божий.
\vs Tjb 1:18
И я сказал: То, что будет сказано Его рабу, я буду слушать.
\vs Tjb 1:19
И архангел сказал мне: Так говорит Господь: если ты решишься уничтожить и убрать образ Сатаны, он примется с гневом вести войну против тебя, и он покажет на тебе всю свою злобу.
\vs Tjb 1:20
Он принесет тебе много жестоких бедствий и отнимет у тебя всё, что ты имел.
\vs Tjb 1:21
Он отнимет твоих детей и причинит много зла тебе.
\vs Tjb 1:22
Тогда ты должен бороться подобно борцу и сопротивляться боли, уверенный в своей награде, преодолевать испытания и бедствия.
\vs Tjb 1:23
Но когда ты претерпишь, Я сделаю твое имя известным среди всех поколений земли до кончины мира.
\vs Tjb 1:24
И Я возвращу тебе всё, что ты имел; и вдвойне от того, что ты потерял, дам тебе, чтобы ты мог знать, что Бог нелицеприятен, но дает каждому, кто заслужил, благо.
\vs Tjb 1:25
И тебе также будет дано оно, и ты наденешь диадему амарантовую,
\vs Tjb 1:26
и в воскресение ты пробудишься для жизни вечной. Тогда ты узнаешь, что Он~--- Господь праведный и истинный и Всемогущий.
\vs Tjb 1:27
После чего, дети мои, я ответил: Я буду из любви Божьей терпеть до смерти всё, что снизойдет на меня, и я не уклонюсь вспять.
\vs Tjb 1:28
Тогда ангел положил свою печать на мне и покинул меня.

\vs Tjb 2:1
После этого я встал ночью и взял пятьдесят рабов, и пошел к храму идола и разрушил его до основания.
\vs Tjb 2:2
И так я возвратился в мой дом и дал наказы, чтобы дверь его крепко затворили, говоря моим привратницам:
\vs Tjb 2:3
Если кто-нибудь будет спрашивать обо мне, не доносите никакого сообщения ко мне, но скажите ему: Он занят срочными делами, он~--- внутри.
\vs Tjb 2:4
Тогда Сатана притворился нищим и сильно стучал в двери, говоря привратнице:
\vs Tjb 2:5
Сообщите Иову и скажите, что я желаю встретиться с ним.
\vs Tjb 2:6
И привратница пошла и сказала мне это, но услышала от меня, что я занят.
\vs Tjb 2:7
Велиал, потерпев неудачу в этом, ушел и взял на своё плечо старую порванную корзину, и пришел и сказал привратнице, говоря: Скажите Иову: дай мне хлеба от рук твоих, чтобы я мог поесть.
\vs Tjb 2:8
И когда я услышал это, я дал ей подгорелого хлеба, чтобы дать его ему, и я известил его: Не жди есть моего хлеба, ибо это запрещено тебе.
\vs Tjb 2:9
Но привратница, устыдившись вручить ему подгорелый и сожженный хлеб (ибо она не знала, что это был Сатана), взяла своего хорошего хлеба и дала его ему.
\vs Tjb 2:10
Но он [не] взял его и, зная, что произошло, сказал деве: Иди прочь, плохая служанка, и принеси мне хлеб, который дали тебе, чтобы вручить мне.
\vs Tjb 2:11
И служанка воскликнула и сказала в печали: Ты говоришь истину, говоря, что я являюсь плохой служанкой, ибо я не сделала так, как была научена моим владыкой.
\vs Tjb 2:12
И она возвратилась и принесла ему горелого хлеба и сказала ему: Так говорит мой господин: тебе не есть от моего хлеба больше, ибо это запрещено тебе.
\vs Tjb 2:13
И это он дал мне, [говоря: это я дал] с тем, чтобы не могло быть навлечено против меня обвинение, что я не подал врагу, который просил.
\vs Tjb 2:14
И когда Сатана услышал это, он отослал служанку обратно ко мне, говоря: Как ты видишь этот хлеб весь сожженным, так буду я скоро жечь твоё тело, чтобы сделать его подобным тому.
\vs Tjb 2:15
И я ответил: Делай, что ты желаешь делать и исполни всё, что ты замыслил. Ибо я готов претерпеть всё, что бы ты ни навел на меня.
\vs Tjb 2:16
И когда диавол услышал это, он оставил меня, и, взойдя на [самое высокое] поднебесье, он взял от Господа клятву, что он сможет иметь власть над всем моим имением.
\vs Tjb 2:17
И после получения власти он пошел и тотчас взял всё мое богатство.

\vs Tjb 3:1
И я имел сто и тридцать тысяч овец, и из них я отделял семь тысяч на одежду сиротам и вдовам, и нуждающимся и больным.
\vs Tjb 3:2
Я имел загон из восьмисот псов, которые стерегли моих овец, и помимо них~--- двести, чтобы охраняли мой дом.
\vs Tjb 3:3
И я имел девять мельниц, работающих для всего города, и корабли, чтобы перевозить товары, и я доставлял их во всякий город и в селения немощному и больному и тем, которые были несчастны.
\vs Tjb 3:4
И я имел триста и сорок тысяч вьючных ослов, и из них я отбирал пятьсот, и потомство их я определял на продажу, а доходы отдавал нищим и нуждающимся.
\vs Tjb 3:5
Ибо со всех стран нищие приходили, чтобы встретиться со мною.
\vs Tjb 3:6
Ибо четыре двери моего дома были отверсты, каждая, будучи под ответственностью сторожа, который наблюдал, каждый ли из приходящих людей получал милостыню, и видел ли меня сидящим у одной двери, так чтобы они могли выходить через другую и получать всё, в чем они нуждались.
\vs Tjb 3:7
Также я имел тридцать столовых наборов, предназначенных на всякое время для одиноких странников, и я также имел двенадцать просторных столов для вдов.
\vs Tjb 3:8
И каждый, кто бы ни приходил, прося о милостыне, он находил пищу на моем столе, получая всё, в чем он нуждался, и я не позволял никому покинуть мою дверь с пустым животом.
\vs Tjb 3:9
Я также имел три тысячи пятьсот пар волов, и я отбирал из них пятьсот и отводил их пахарям.
\vs Tjb 3:10
И с ними я делал всю работу на всяком поле, у тех, кто хотел бы взять его на попечение, и доход их урожаев я откладывал для нищих на их столе.
\vs Tjb 3:11
Я также имел пятьдесят пекарен, от которых я посылал [хлеб] к столу для бедняков.
\vs Tjb 3:12
И я имел рабов, избранных для служения им.
\vs Tjb 3:13
Имелись также некоторые странники, которые видели мою добрую волю; они желали послужить как служители сами.
\vs Tjb 3:14
Другие, будучи в бедствии и неспособные приобрести средства к существованию, приходили с просьбой, говоря:
\vs Tjb 3:15
Мы просим тебя: поскольку мы тоже можем выполнять эту обязанность служителей и не имеем никакого стяжания, сжалься над нами и ссуди денег нам, с тем чтобы мы могли пойти в большие города и продавать товары.
\vs Tjb 3:16
И излишек от нашей прибыли мы можем отдавать в помощь нищим, и затем мы возвратим тебе твою собственность.
\vs Tjb 3:17
И когда я слышал это, я был доволен тем, что они возьмут это совместно со мною ради бережливости и милосердия к нищим.
\vs Tjb 3:18
И по желанию сердца я давал им что они хотели, и я принимал их расписки, но не брал никакого иного залога от них, кроме письменного свидетельства.
\vs Tjb 3:19
И они ходили повсюду и подавали вовремя бедным, насколько они преуспевали.
\vs Tjb 3:20
Часто, однако, некоторые из их товаров были теряемы на пути или на море, или их он отнимал у них.
\vs Tjb 3:21
Тогда они придут и скажут: Мы просим тебя: будь великодушен к нам, с тем чтобы мы могли подумать, как мы можем возвратить тебе твою собственность.
\vs Tjb 3:22
И когда я слышал это, я сочувствовал им и вручал им их расписку, и часто читавший её перед ними разрывал её, и прощал им из их долга, говоря им:
\vs Tjb 3:23
Что я посвятил для помощи бедным, я не буду брать от тебя.
\vs Tjb 3:24
И так я ничего не брал от моего должника.
\vs Tjb 3:25
И тогда муж с весёлым сердцем приходил ко мне, говоря: Мне не нужен трудовой заработок бедняка, необходимый ему;
\vs Tjb 3:26
но я желаю служить нуждающемуся за вашим столом. И он соглашался работать, и он ел свою долю.
\vs Tjb 3:27
Однако же я давал ему его плату, и я\fnote{я}{он(?)} уходил домой, радуясь.
\vs Tjb 3:28
А когда он не хотел брать это, я вынуждал его, чтобы делать так, говоря: Я знаю, что ты~--- муж труда, который надеется и ждет своей платы, и ты должен брать это.
\vs Tjb 3:29
Никогда я не отсрочиваю плату мзды наемнику или любому другому, ни задерживаю в моём доме в течение одного вечера из его найма, который был должен ему.
\vs Tjb 3:30
Те, которые доили коров и овец, сообщали проходящим, что они должны взять свою долю.
\vs Tjb 3:31
Ибо молоко текло в таком множестве, что оно свертывалось в масло на склонах и на краю дороги; и у камней и холмов скот ложился, чтобы родить своё потомство.
\vs Tjb 3:32
Ибо мои слуги утомлялись хранением мяса вдов и нищих и делили его [себе] на маленькие кусочки.
\vs Tjb 3:33
Ибо они, бывало, ругались и говорили: О, что мы имеем от его мяса, чем мы могли бы быть удовлетворены!, хотя я был очень любезен к ним.
\vs Tjb 3:34
Я также имел шесть арф [и шесть рабов, играющих на арфах], и также лиру десятиструнную, и я ударял по ним в течение дня.
\vs Tjb 3:35
И я брал лиру, и вдовы подпевали [мне] после их еды.
\vs Tjb 3:36
И с музыкальным орудием я напоминал им о Боге, что они должны воздавать хвалу Господу.
\vs Tjb 3:37
И когда мои рабыни, бывало, роптали, тогда я брал музыкальные орудия и играл столько, сколько они сделали бы за их плату, и давал им облегчение от их труда и воздыханий.

\vs Tjb 4:1
И мои дети, взяв ответственность за служение, брали своё ежедневное пропитание вместе с их тремя сестрами, начиная со старшего брата, и делали праздник.
\vs Tjb 4:2
И я вставал утром и предлагал, как искупительную жертву за них, пятьдесят овнов и девятнадцать овец, и что оставалось, как остаток было посвящаемо бедным.
\vs Tjb 4:3
И я говорил им: Берите это как остаток, и молитесь за моих детей.
\vs Tjb 4:4
Возможно, мои сыновья грешили перед Господом, говоря в надменности духа: Мы~--- дети этого богатого человека. Наше~--- всё это добро; почему мы должны быть слугами нищих?
\vs Tjb 4:5
И говоря так в надменности духа, они, возможно, вызывали гнев Бога, ибо гордое превозношение~--- мерзость пред Господом.
\vs Tjb 4:6
И так я приносил волов как жертву священнику на алтарь, говоря: Не хулили ли мои дети когда-либо Бога в своих сердцах?
\vs Tjb 4:7
Пока я жил таким образом, Соблазнитель не мог спокойно смотреть на добро [творимое мною], и он испросил у Бога войну против меня.
\vs Tjb 4:8
И он напал на меня безжалостно.
\vs Tjb 4:9
Сначала он сжег большое количество овец, потом верблюдов, затем он сжег волов и всё моё стадо; или же они были захвачены не только врагами, но также и теми, кто был облагодетельствован мною.
\vs Tjb 4:10
И пастухи пришли и возвестили это мне.
\vs Tjb 4:11
Но когда я услышал это, я воздал хвалу Богу и не богохульствовал.
\vs Tjb 4:12
И когда Соблазнитель познал моё терпение, он замыслил новое дело против меня.
\vs Tjb 4:13
Он вошел в царя Фираса и осадил мой город, и после того, как он увел всё, что было там, он сказал им в злобе, говоря хвастливой речью:
\vs Tjb 4:14
Этот муж Иов~--- тот, который получил все блага земли и не оставил ничего для других, он разрушил и низверг храм бога.
\vs Tjb 4:15
Поэтому я воздам ему тем же, что он сделал дому великого бога.
\vs Tjb 4:16
Ныне идите со мною, и мы будем грабить всё, что осталось в его доме.
\vs Tjb 4:17
И они ответили и сказали ему: Он имеет семь сыновей и трех дочерей.
\vs Tjb 4:18
Позаботься, чтобы они не убежали в другие страны, и не стали бы нашими мучителями, и тогда они превозмогут нас силою и убьют нас.
\vs Tjb 4:19
И он сказал: Ничего не бойтесь. Его стада и его богатство я уничтожил огнем, и остальное я расхитил, и вот, его детей я убью.
\vs Tjb 4:20
И говоря так, он пошел и обрушил дом на моих детей и убил их.
\vs Tjb 4:21
И мои сограждане, видя то, что сказанное им действительно совершилось, пришли и преследовали меня, и отняли у меня всё, что было в моем доме.
\vs Tjb 4:22
И я увидел моими глазами грабеж моего дома, и люди невоспитанные и безчестные сидели за моим столом и на моих ложах, и я не мог возражать против них.
\vs Tjb 4:23
Ибо я был истощен подобно женщине с её ложеснами, освободившимися от множества болей, помня главное~--- что эта война была предсказана мне Господом через Его ангела.
\vs Tjb 4:24
И я стал подобным тому, кто, видя бурное море и противные ветры, в то время как груз судна посреди океана слишком тяжел, сбрасывает тяжесть в море, говоря:
\vs Tjb 4:25
Я хочу уничтожить всё это для того, чтобы благополучно прибыть в город, так чтобы я мог взять как прибыль спасенное судно и лучшее из моих вещей.
\vs Tjb 4:26
Так я управлял моими делами.
\vs Tjb 4:27
Но вот пришел другой вестник и возвестил мне о гибели моих детей, и я был потрясен ужасом.
\vs Tjb 4:28
И я разодрал мою одежду и сказал: Господь дал, Господь и взял. Как это было угодно Господу, так это и сделалось. Да будет имя Господне благословенно.

\vs Tjb 5:1
И когда Сатана увидел, что он не смог возбудить во мне отчаяние, он пошел и выпросил моё тело у Господа, дабы причинить язву мне, ибо Велиал не мог вынести моего терпения.
\vs Tjb 5:2
Тогда Господь предал меня в его руки, чтобы использовать моё тело, как он хотел, но Он не дал ему власти над моей душой.
\vs Tjb 5:3
И он пришел ко мне, я же был сидящим на моём троне, всё еще печалясь по моим детям.
\vs Tjb 5:4
И он был подобен великому урагану и перевернул мой трон и бросил меня оземь.
\vs Tjb 5:5
И я долго лежал на полу в течение трех часов. И он поразил меня тяжкой проказой от темени головы моей до кончиков ног моих.
\vs Tjb 5:6
И я покинул город в великом ужасе и горе и сел на навозную кучу моим телом червоточивым.
\vs Tjb 5:7
И я орошал землю мокротою моего воспаленного тела, ибо гной стекал с моего тела, и множество червей покрывало его.
\vs Tjb 5:8
И когда один [какой-нибудь] червь сползал с моего тела, я клал его назад, говоря: Останься на том месте, где ты находился, пока Тот, Кто послал тебя, не направит тебя куда-нибудь еще.
\vs Tjb 5:9
Так я претерпевал в течение семи лет, сидя на навозной куче вне города, будучи поражен проказой.
\vs Tjb 5:10
И я увидел своими глазами моих томящихся детей
\vs Tjb 5:11
и мою унижающуюся жену, которая [некогда] была приведена в её свадебный чертог в такой великой роскоши и с копьеносцами как телохранителями. Я видел её выполняющею работу носильщика воды, подобно рабу, в доме простого человека, для того чтобы заработать немного хлеба и принести его мне.
\vs Tjb 5:12
И в моем лютом бедствии я сказал: О, что [значат] эти хвастливые правители города, которые будут теперь нанимать мою жену как служанку, которых душу я не подумаю сравнить [даже] с моими сторожевыми псами!
\vs Tjb 5:13
И после этого я обрел храбрость вновь.
\vs Tjb 5:14
Однако, впоследствии они отказывали [ей] даже в хлебе [для меня], чтобы она имела только её собственное пропитание.
\vs Tjb 5:15
Но она брала это и разделяла это между собою и мною, говоря скорбно: Горе мне! Отныне он больше не сможет питаться хлебом, и он не может пойти на торжище попросить хлеба у хлеботорговцев для того, чтобы принести его мне [и] чтобы он мог есть.
\vs Tjb 5:16
И когда Сатана узнал это, он принял облик хлеботорговца; и это было как будто случайным, что моя жена встретила его и спросила его о хлебе, думая, что это был его человеческий вид.
\vs Tjb 5:17
Но Сатана сказал ей: Дай мне цену, и потом бери, что ты пожелаешь.
\vs Tjb 5:18
Тогда она ответила, говоря: Где я возьму денег? Разве ты не знаешь, какая беда произошла со мною? Если ты имеешь жалость, яви её мне; если нет, ты смотри.
\vs Tjb 5:19
И он ответил, говоря: Если бы ты не заслуживала этой беды, ты бы не испытала всё это.
\vs Tjb 5:20
Ныне, если нет сребренника в руке твоей, дай мне волосы головы твоей и возьми три буханки хлеба за это, так что ты сможешь прожить на них три дня.
\vs Tjb 5:21
Тогда она сказала в себе: Что есть волосы головы моей по сравнению с моим голодающим мужем?
\vs Tjb 5:22
И так, подумав над вопросом, она сказала ему: Встань и отрежь мои волосы.
\vs Tjb 5:23
Тогда он взял ножницы и отнял волосы её головы в присутствии всех и дал ей три буханки хлеба.
\vs Tjb 5:24
Тогда она взяла их и принесла их мне. И Сатана последовал за ней по дороге, притаившись когда он шел и весьма безпокоя её сердце.

\vs Tjb 6:1
И тотчас же моя жена пришла ко мне и, вопия громко и плача, она сказала: Иов, Иов! Как долго ты сидишь на навозной куче вне города, размышляя уже на протяжении [столького] времени и ожидая получить твоё желанное спасение!
\vs Tjb 6:2
И я должна была блуждать с места на место, скитаясь повсюду как наемная служанка, [и слышать:] вот их память уже исчезла от земли.
\vs Tjb 6:3
И мои сыновья и дочери, которых я носила на моей груди, и труды и муки, которые я выдержала, были напрасны?
\vs Tjb 6:4
И ты сидишь в смраде болезни и червях, проводя ночи на холодном воздухе.
\vs Tjb 6:5
И я подвергалась всяким испытаниям и скорбям и мукам, днем и ночью, пока я не преуспевала в снабжении тебя хлебом.
\vs Tjb 6:6
Поскольку твоего излишка хлеба больше не позволили мне [брать]; и поскольку я едва могу брать мою собственную пищу и делить её между нами, я размышляла в моем сердце, что это несправедливо, что ты должен находиться в болезни и голоде из-за [отсутствия] хлеба.
\vs Tjb 6:7
И тогда я решилась идти на торжище без робости. И когда хлеботорговец сказал мне: Дай мне деньги и ты получишь хлеб, я открыла ему наше бедственное положение.
\vs Tjb 6:8
Тогда я услышала, как он сказал: Если ты не имеешь никаких денег, вручи мне волосы твоей головы, и возьми три буханки хлеба для того, чтобы ты могла жить на них три дня.
\vs Tjb 6:9
И я уступила несправедливости и сказала ему: Встань и отрежь мои волосы! И он встал, и публично отрезал ножницами волосы моей головы на рыночной площади, в то время как толпа стояла рядом и удивлялась.
\vs Tjb 6:10
Кто тогда не удивлялся, говоря: Это ли Сифь, жена Иова, которая имела четырнадцать занавесов, чтобы закрывать её сокровенный чертог, и двери за дверями, так что тот был весьма польщен, кто был приведен подле него; и ныне, вот, она обменивает свои волосы на хлеб!
\vs Tjb 6:11
\ldots кто имел верблюдов, нагруженных товарами, и они отводились в отдаленные страны к нищим; и ныне она продает свои волосы за хлеб!
\vs Tjb 6:12
Смотрите на неё, кто имела семь неподвижных столовых наборов в её доме, за которыми всякий бедный человек и всякий странник ел; и ныне она продает свои волосы за хлеб!
\vs Tjb 6:13
Смотрите на неё, кто имела купальню, чтобы омывать свои ноги, сделанную из золота и серебра; и ныне она ходит по земле [босая], и [продает свои волосы за хлеб!]
\vs Tjb 6:14
Смотрите на неё, кто имела одеяние, сделанное из виссона, вышитое золотом; и ныне она обменивает свои волосы на хлеб!
\vs Tjb 6:15
Смотрите на неё, кто имела ложа из золота и серебра; и ныне она продает свои волосы за хлеб!
\vs Tjb 6:16
Затем вкратце, Иов, после стольких вещей, которые были сказаны мне, я ныне скажу тебе одним словом:
\vs Tjb 6:17
Так как слабость моего сердца сокрушает мои кости, встань и возьми эти буханки хлеба и насладись ими, и потом прокляни Господа и умри!
\vs Tjb 6:18
Ибо я тоже заменила бы оковы смерти за хлеб насущный моему телу.
\vs Tjb 6:19
Но я ответил ей: Вот, я был в течение этих семи лет пораженным проказой, и я терпел червей в моем теле, и я не был отягощен в моей душе всеми этими мучениями.
\vs Tjb 6:20
И как [за] слово, которое ты говоришь: Прокляни Бога и умри, вместе с тобою я выдержу зло, которое ты видишь? И позволь нам перенести разорение всего, что мы имеем.
\vs Tjb 6:21
Всё же ты хочешь, чтобы мы прокляли Бога и чтобы Он был заменен на великого Плутона.
\vs Tjb 6:22
Почему ты не помнишь тех великих благ, которыми мы обладали? Если эти блага исходят из уделов Господних, не должны ли мы также претерпевать [от Него и] зло и быть премудрыми во всем, пока Господь не помилует [нас] снова и окажет жалость к нам?
\vs Tjb 6:23
Ты не видишь Соблазнителя, ставшего позади тебя и спутавшего твои мысли, чтобы ты обманывала меня.
\vs Tjb 6:24
И он обратился к Сатане и сказал: Почему же ты не приходишь ко мне явно, не перестанешь скрывать себя? Ты~--- жалкий
\vs Tjb 6:25
лев, показывающий свою силу в удобной клетке, или птица, летающая в корзине. Ныне я говорю тебе: выходи и веди твою войну против меня.
\vs Tjb 6:26
Тогда он вышел из-за спины моей жены и поставил себя предо мною, вопия; и он сказал: Вот, Иов, я сдаюсь и уступаю дорогу тебе, который искусен, но~--- плоть, тогда как я~--- дух.
\vs Tjb 6:27
Ты поражен проказой, но я в великой печали.
\vs Tjb 6:28
Ибо я подобен соревнующемуся с борцом борцу, который в бою одной рукою низверг своего соперника и скрыл его в прахе и сокрушил каждый член его, тогда как тот, который лежит внизу, являя свою храбрость, издает звуки торжества, свидетельствующие о его великом превосходстве.
\vs Tjb 6:29
Так и ты, о Иов, унижен и поражен проказой и мукою, и все же ты вынес победу в соревновании со мною, и вот, я уступаю тебе.
\vs Tjb 6:30
Тогда он покинул меня смущенный.
\vs Tjb 6:31
Ныне, мои дети, вы делайте также, являя твердость сердца во всяком зле, которое происходит с вами, ибо твердость сердца~--- больше всех дел.

\vs Tjb 7:1
В это время цари услышали о том, что произошло со мною, и они встали и пришли ко мне, каждый от его земли, чтобы посетить меня и утешить меня.
\vs Tjb 7:2
И когда они подошли ко мне, они возопили громким голосом, и каждый разодрал свою одежду.
\vs Tjb 7:3
И после того, как они поклонились, касаясь земли своими головами, они сидели рядом со мною семь дней и семь ночей, и ни один не сказал ни слова.
\vs Tjb 7:4
Их было числом четверо: Елифаз, царь Фемана, и Вилдад, и Софар, и Елиуй.
\vs Tjb 7:5
И когда они заняли своё место, они беседовали о том, что произошло со мною.
\vs Tjb 7:6
В то время, когда они в первый раз приходили ко мне и я показывал им мои драгоценные камни, они были удивлены и сказали:
\vs Tjb 7:7
Если бы от нас, трех царей, всё наше имущество было бы соединено в одно, оно не сравнилось бы с драгоценными камнями царства Иовава. Ибо твое превосходство больше, чем всех людей Востока.
\vs Tjb 7:8
И поэтому, когда они ныне пришли в землю Уц, чтобы посетить меня, они спросили в городе: Где~--- Иовав, правитель этой всей земли?
\vs Tjb 7:9
И они сказали им обо мне: Он сидит на навозной куче вне города, ибо он не входит в город в течение семи лет.
\vs Tjb 7:10
И тогда они снова спросили о моём имуществе, и вот было показано им всё, что произошло со мною.
\vs Tjb 7:11
И когда они узнали это, они вышли из города с жителями, и мой согражданин показал меня им.
\vs Tjb 7:12
Но они возражали и говорили: Конечно, это~--- не Иовав.
\vs Tjb 7:13
И пока они колебались, вот Елифаз, царь Фемана, говорит: Давайте, подойдем ближе и посмотрим.
\vs Tjb 7:14
И когда они подошли ближе, я вспомнил их, и я сильно плакал, когда я узнал о цели их путешествия.
\vs Tjb 7:15
И я посыпал прах на мою голову, и пока отряхивал свою голову, я открыл им, кто я был.
\vs Tjb 7:16
И когда они увидели меня, трясущего своей головою, они поверглись ниц до земли, все охваченные волнением.
\vs Tjb 7:17
И пока толпа стояла вокруг, я видел этих трех царей лежащими на земле в течение трех часов подобно мертвым.
\vs Tjb 7:18
Тогда они встали и сказали друг другу: Мы не можем поверить, что это~--- Иовав.
\vs Tjb 7:19
И, наконец, после того, как они на седьмой день узнали всё обо мне и искали [и не нашли] мои стада и другое имущество, они сказали:
\vs Tjb 7:20
Разве мы не знаем, сколько товаров посылал он городам и селениям, повсюду подавая нищим, кроме всего, что было отдано им внутри его собственного дома? Как же мог он впасть в таковое состояние погибели и горя!
\vs Tjb 7:21
И после семи дней Елиуй сказал царям: Давайте подойдем ближе и рассмотрим его тщательно, истинно ли он Иовав или нет?
\vs Tjb 7:22
И они, будучи на расстоянии стадии от его зловонного тела, встали и шагнули ближе, неся благовония в их руках, а их воины пошли с ними и бросали ароматные шарики ладана к ним так, чтобы они могли приблизиться ко мне.
\vs Tjb 7:23
И после того, как они так прошли три часа, покрывая путь ароматом, они почти достигли.
\vs Tjb 7:24
И Елифаз начал и сказал: Ты ли, воистину, Иов, соцарствующий нам? Ты ли тот, кто имел великую славу?
\vs Tjb 7:25
Ты ли тот, кто когда-то сиял подобно дневному солнцу на всю землю? Ты ли тот, кто когда-то походил на луну и звезды, сияющие всю ночь?
\vs Tjb 7:26
И я ответил ему и сказал: Это я. И затем все плакали и стенали, и они воспели царскую плачевную песнь, [и] всё их войско соединилось с ними в хоре.
\vs Tjb 7:27
И опять Елифаз сказал мне: Ты ли тот, кто приказал раздать семь тысяч овец для одежды нищим? Поблекла, значит, преходящая слава твоего престола!
\vs Tjb 7:28
Ты ли тот, кто повелел трем тысячам волов пахать поле для бедных? Поблекла, значит, твоя преходящая слава!
\vs Tjb 7:29
Ты ли тот, кто имел золотые ложа, и ныне ты сидишь на навозной куче? [Поблекла, значит, твоя преходящая слава!]
\vs Tjb 7:30
Ты ли тот, кто имел шестьдесят столовых набора для нищих? Ты ли тот, кто имел кадило для прекрасных благовоний, отделанное драгоценными камнями, и ныне ты в зловонии? Поблекла, значит, твоя преходящая слава!
\vs Tjb 7:31
Ты ли тот, кто имел золотой набор подсвечников на серебряных подставках; и ныне должен ты тосковать из-за естественного отражения луны? [Поблекла, значит, твоя преходящая слава!]
\vs Tjb 7:32
Ты ли тот самый, кто делал притирание из смеси ладана, и ныне ты в мерзости! [Поблекла, значит, твоя преходящая слава!]
\vs Tjb 7:33
Ты ли тот, кто высмеивал неправедных делателей и презирал грешников, и ныне ты стал посмешищем у всех! [Поблекла, значит, твоя преходящая слава!]
\vs Tjb 7:34
И пока Eлифаз много времени вопиял и стенал, а все остальные соединились с ним, так что смятение было весьма великим, я сказал им:
\vs Tjb 7:35
Умолкните и я покажу вам мой престол и славу его великолепия: моя слава будет вечной.
\vs Tjb 7:36
Весь мир погибнет и его слава исчезнет, и все те, кто прилепляются к нему, будут в преисподней, но мой престол пребывает в вышнем мире, и его слава и великолепие будут одесную от Искупителя [моего] на небесах.
\vs Tjb 7:37
Мой престол существует в обществе святых и их славы в нетленном мире.
\vs Tjb 7:38
Ибо реки высохнут, и их надменность будет унижена до глубины бездны, но потоки моей земли, в которой мой престол воздвигнут, не высохнут, но останутся нерушимыми в силе.
\vs Tjb 7:39
Цари погибают и князья исчезают, и их слава и гордость~--- как отражение в зеркале; но моё царство продлится всегда и вечно, и его слава и красота пребывают в колеснице моего Отца.

\vs Tjb 8:1
Когда я говорил им так, Елифаз разгневался и сказал другим друзьям: Для этой ли цели мы пришли сюда с нашим войском утешать его? Вот, он поносит нас. Поэтому давайте мы возвратимся в наши страны.
\vs Tjb 8:2
Этот человек сидит здесь в червоточивом страдании среди невыносимого гниения, и все же он испытывает своё спасение: Погибнут царства и их правители, но моё царство, говорит он, продлится вовек.
\vs Tjb 8:3
Eлифаз затем встал в большом смятении, и, отвернувшись от них в великой ярости, сказал: Я ухожу отсюда. Воистину мы пришли утешить его, но он объявляет войну нам ввиду наших войск.
\vs Tjb 8:4
Но тогда Вилдад схватил его за руку и сказал: Не так должно говорить со страдающим человеком, и особенно с пораженным таковыми многими бедствиями.
\vs Tjb 8:5
Вот, мы, будучи в добром здравии, не осмеливались приблизиться к нему по причине зловония, кроме как с помощью множества ароматных благовоний. Но ты, Елифаз, забываешь обо всем этом.
\vs Tjb 8:6
Скажи мне прямо: позволишь ли нам быть великодушными и узнать, какова причина [его бедствия]? Не мог же он при воспоминании его прежних дней счастья стать безумным в его разуме?
\vs Tjb 8:7
Кто не был бы в совершенном недоумении, видя себя впадшим в таковое несчастье и беду? Но позвольте мне приступить к нему, чтобы я смог выяснить, в чем суть его дела.
\vs Tjb 8:8
И Вилдад встал и приблизился ко мне, говоря: Ты ли Иов? И [еще] он сказал: Находится ли твоё сердце в добром расположении?
\vs Tjb 8:9
И я сказал: Я не прилепляюсь к земным делам, тогда как земля со всем, что обитает на ней~--- непостоянна. Но моё сердце прилепляется к небу, ибо там, в небесах, нет печали.
\vs Tjb 8:10
Тогда Вилдад возразил и сказал: Мы знаем, что земля непостоянна, ибо она изменяется по сезонам. По временам она в состоянии мира, и по временам она в состоянии войны. Но о небе мы слышим, что оно совершенно неизменно.
\vs Tjb 8:11
Но истинно ли ты в покое? Поэтому позволь мне спрашивать и говорить; и когда ты ответишь мне на моё первое слово, я задам второй вопрос, и если вновь ты ответишь словами хорошо подобранными, станет очевидно, что твоё сердце не пребывает неуравновешенным.
\vs Tjb 8:12
И я сказал\fnote{я сказал}{он сказал(?)}: На что ты полагаешь твою надежду? И я ответил: На Бога живого.
\vs Tjb 8:13
И он сказал мне: Кто лишил тебя всего, чем ты обладал? И кто причинил тебе эти несчастья? И я сказал: Бог.
\vs Tjb 8:14
И он сказал: Если ты всё еще полагаешь свою надежду на Бога, то как Он может творить неправедный суд, нанося тебе эти несчастья и беды, и отняв у тебя все твои владения?
\vs Tjb 8:15
И так как Он забирает их, то ясно, что Он не дает тебе ничего. Царь станет ли безчестить своего воина, который хорошо служит ему как телохранитель?
\vs Tjb 8:16
[И я ответил на притчу]: Кто уразумеет глубины Господни и Его мудрость, чтобы быть способным обвинить Бога в несправедливости?
\vs Tjb 8:17
[И Вилдад сказал]: Ответь мне, о Иов, на это. Снова я скажу тебе: если ты~--- в здравом разсудке, вразуми меня, если ты имеешь мудрость:
\vs Tjb 8:18
почему мы видим восход солнца на Востоке, а закат на Западе? И опять, когда встаем утром, [почему] мы находим его восходящим на Востоке? Сообщи мне, что ты думаешь об этом?
\vs Tjb 8:19
Тогда сказал я: Зачем я буду выдавать величайшие тайны Божии и мои уста должны преткнуться при раскрытии дел, принадлежащих Владыке? Никогда!
\vs Tjb 8:20
Кто мы, что мы будем вникать в дела вышнего мира, тогда как мы~--- только из плоти; нет~--- земля и прах!
\vs Tjb 8:21
Чтобы ты знал, что моё сердце здраво, послушай, что я спрошу тебя:
\vs Tjb 8:22
Через утробу проходит пища, и воду ты пьешь через уста, и затем это течет через одно и то же горло, и когда оба опускаются, становясь выделением, они снова разделяются; кто производит это разделение?
\vs Tjb 8:23
И Вилдад сказал: Я не знаю. И я возразил и сказал ему: Если ты не понимаешь даже испражнений тела, как можешь ты понимать небесные круговращения?
\vs Tjb 8:24
Тогда Софар возразил и сказал: Мы не спрашиваем о своих делах, но мы желаем знать,~--- в здравом ли ты состоянии, и вот, мы видим, что твой разсудок не поколеблен.
\vs Tjb 8:25
Что ныне ты хочешь, чтобы мы сделали для тебя? Вот, мы пришли сюда и привели лекарей трех царей, и если ты желаешь, ты можешь вылечиться у них.
\vs Tjb 8:26
Но я ответил и сказал: Моё лекарство и моё возстановление происходят от Бога, Творца врачей.

\vs Tjb 9:1
И когда я говорил им так, вот, туда прибежала моя жена Сифь, одетая в лохмотья, от служения тому хозяину, которым она была нанята как рабыня, хотя ей запретили покидать [его], чтобы цари, видя её, не могли взять её как пленницу.
\vs Tjb 9:2
И когда она пришла, она простерлась обезсиленная у их ног, рыдая и говоря: Вспомнили, Елифаз и вы, остальные друзья, сколь я одинока с вами, и как я изменилась, как ныне я одета, чтобы встретить вас!
\vs Tjb 9:3
Тогда цари упали навзничь в великом плаче и, будучи в крайнем недоумении, они хранили молчание. Но Елифаз взял свою пурпурную мантию и бросил её ей, чтобы оделась в это.
\vs Tjb 9:4
Но она попросила его, говоря: Я прошу как милость у вас, моих господ, чтобы вы приказали вашим воинам копать среди развалин нашего дома, который упал на моих детей, так чтобы их кости могли быть принесены целыми к могилам.
\vs Tjb 9:5
Это первое, в чем мы имеем нужду в нашем несчастье, обезсилев совсем, и таким образом мы сможем по крайней мере увидеть их кости.
\vs Tjb 9:6
Ибо я имела безотчетное материнское чувство как у диких зверей, что мои десять детей должны погибнуть в один день; и ни одному из них я не могла бы дать достойное погребение?
\vs Tjb 9:7
И цари дали повеление, чтобы руины моего дома были выкопаны. Но я запретил это, предостерегая:
\vs Tjb 9:8
Не ходите в напрасной скорби; ибо мои дети не будут найдены, потому что они соблюдаются их Творцом и Владыкой.
\vs Tjb 9:9
И цари ответили и сказали: Кто станет сему противоречить? Он [вышел] из своего ума и бредит.
\vs Tjb 9:10
Ибо в то время как мы хотим принести кости его детей обратно, он запрещает нам делать [это], говоря так: Они были взяты и помещены на хранение их Творцом. Поэтому докажи нам [твою] истину.
\vs Tjb 9:11
Но я сказал им: Поднимите меня, чтобы я мог встать; и они подняли меня, поддерживая мои руки с обоих сторон.
\vs Tjb 9:12
И я стал прямо и объявил сначала похвалу Богу, а после молитвы я сказал им: Посмотрите глазами вашими на Восток.
\vs Tjb 9:13
И они посмотрели и увидели моих детей с венцами подле Царя славы, Владыки небес.
\vs Tjb 9:14
И когда моя жена Сифь увидела это, она пала на землю и простерла[сь] пред Богом, говоря: Теперь я знаю, что моя память останется у Господа.
\vs Tjb 9:15
И после того, как она сказала [это] и настал вечер, она пошла в город, обратно к хозяину, которому она служила как рабыня, и легла в воловьих яслях и умерла там от истощения.
\vs Tjb 9:16
И когда её жестокий хозяин искал её и не находил её, он пришел к загону своего стада, и там он увидел её простершейся в яслях мертвой, в то время как все животные вокруг плакали о ней.
\vs Tjb 9:17
И все, кто видели её, плакали и стенали, и вопль распространился повсюду во всем городе.
\vs Tjb 9:18
И люди унесли её и обернули её и похоронили её в доме, который упал на её детей.
\vs Tjb 9:19
И городские нищие сотворили великий плачь по ней и говорили: Вот, это Сифь, подобной кому в благородстве и в славе не найдется среди жен. Увы,~--- она не обрела достойной могилы!
\vs Tjb 9:20
Погребальную песнь по ней вы найдете в записи.

\vs Tjb 10:1
Но Eлифаз и те, что были с ним, были удивлены этим вещам, и они сидели со мною и отвечали мне, говоря в хвастливых словах обо мне в течение двадцати семи дней.
\vs Tjb 10:2
Они повторяли это снова и снова: что я страдал так по заслугам, потому что совершил много грехов, и что, вот, надежды не осталось мне; но я опровергал этих мужей в пылу спора.
\vs Tjb 10:3
И они встали в раздражении, готовые разстаться в гневном духе. Но Елиуй заклинал их остаться еще немного до тех пор, пока он не покажет им, как это было.
\vs Tjb 10:4
Ибо,~--- сказал он,~--- так много дней вы проводите, позволяя Иову хвалиться, что он праведен. Но я больше не буду терпеть этого.
\vs Tjb 10:5
Ибо изначально я продолжаю плакать по нему, помня его прежнее счастье. Но теперь он говорит хвастливо, и в заносчивой гордыне он говорит, что он имеет свой престол на небесах.
\vs Tjb 10:6
Поэтому, послушайте меня, и я поведаю вам, какова причина [такой] его судьбы.
\vs Tjb 10:7
Тогда вдохновляемый духом Сатаны Елиуй сказал жестокие слова, которые записаны в записях, оставленных Елиуем.
\vs Tjb 10:8
И когда он закончил, Бог явился мне в буре и мраке, и говорил, осуждая Елиуя и показывая мне, что тот, кто говорил, был не человек, а бешеное животное.
\vs Tjb 10:9
И когда Бог закончил говорить со мною, Господь сказал Eлифазу: Ты и твои друзья согрешили в том, что вы не говорили истины о Моем рабе Иове.
\vs Tjb 10:10
Поэтому поднимитесь и побудите его принести искупительную жертву за вас, чтобы ваши грехи могли быть прощены; ибо если бы не он, Я уничтожил бы вас.
\vs Tjb 10:11
И так они принесли мне всё, что необходимо для жертвы, и я взял это и принес за них искупительную жертву, и Господь принял её благосклонно и простил им их неправду.
\vs Tjb 10:12
После того, как Елифаз, Вилдад и Софар увидели, что Бог милостиво простил их грех через Его раба Иова, но что Он не соизволил простить Елиуя, тогда Eлифаз начал петь гимн, в то время как остальные подпевали, [и] их воины также присоединились при водруженном алтаре.
\vs Tjb 10:13
И Eлифаз говорил так: Отпущен грех и наша несправедливость омыта;
\vs Tjb 10:14
но Елиуй, оный Велиал, не будет иметь памяти среди живущих; его светило гаснет и теряет свой свет.
\vs Tjb 10:15
Слава его светильника явится ему, ибо он~--- сын тьмы, а не света.
\vs Tjb 10:16
Привратники обиталища тьмы дадут ему их славу и красоту в удел; его царство исчезло, его престол разсыпался, и честь его стати~--- в Шеоле.
\vs Tjb 10:17
Ибо он возлюбил лесть змеи и кожу дракона, его желчь и его яд~--- аспида.
\vs Tjb 10:18
Ибо он не стремился к Господу, ни боялся он Его, но он ненавидел тех, кого Он избрал.
\vs Tjb 10:19
Оттого Бог забыл его, и святые оставили его, его гнев и раздражение будут ему запустением, и он не обретет ни милосердия в его сердце, ни мира, ибо он имел яд аспида на его языке.
\vs Tjb 10:20
Праведен Господь, и Его суды~--- истинны. У Него нет лицеприятия, ибо Он судит всех одинаково.
\vs Tjb 10:21
Вот, Господь грядет! Вот, святые приготовились: венцы и награды победителей предшествуют им!
\vs Tjb 10:22
Да возрадуются святые, и да возликуют сердца их в веселии; ибо они получат славу, которая соблюдается для них.
\vs Tjb 10:23
Хор: Наши грехи прощены, наша несправедливость очищена, но Елиую нет памяти среди живущих.
\vs Tjb 10:24
После того, как Елифаз окончил гимн, мы встали и возвратились в город, каждый к дому, где он жил.
\vs Tjb 10:25
И народ сотворил пир для меня в благодарность и восхищение Богу, и все мои друзья возвратились ко мне.
\vs Tjb 10:26
И все те, кто видели меня в моем прежнем счастье, спросили меня, говоря: Что это за три вещи здесь среди нас? \ldots

\vs Tjb 11:1
Но я, желая взяться снова за мой труд благотворительности для бедных, просил их, говоря:
\vs Tjb 11:2
Дайте мне каждый агнца для одежды нищим в их наготе, и четыре драхмы серебра или золота.
\vs Tjb 11:3
Тогда Господь благословил всё, что было отложено мне, и после немногих дней я снова стал богат имением, в стадах и всех делах, которые я потерял, и я вновь получил всё вдвойне.
\vs Tjb 11:4
Потом я также взял в жену вашу мать и стал отцом вам десятерым вместо десяти детей, которые умерли.
\vs Tjb 11:5
И ныне, дети мои, позвольте мне предупредить вас: Вот, я умираю [и] вы получите моё жилище,
\vs Tjb 11:6
только не оставляйте Господа. Будьте милосердны к нищим; не презирайте немощных; не берите себе жен из иноплеменников.
\vs Tjb 11:7
Вот, дети мои, я разделю среди вас, чем я обладаю, так чтобы каждый мог иметь власть над его собственностью и полную силу делать благое с его долей.
\vs Tjb 11:8
И после того, как он сказал так, он принес всё своё стяжание и разделил его между его семью сыновьями, но он не дал ничего из своего имущества его дочерям.
\vs Tjb 11:9
Тогда они сказали своему отцу: Наш господин и отец! Не являемся ли мы тоже твоими детьми? Почему тогда ты не даешь также и нам часть твоего имения?
\vs Tjb 11:10
Тогда сказал Иов своим дочерям: Не гневайтесь, мои дочери. Я не забыл вас. Вот, я приготовил для вас имение, лучшее чем то, которое взяли ваши братья.
\vs Tjb 11:11
И он призвал свою дочь, имя которой Йемима, и сказал ей: Возьми это витое кольцо, используемое как ключ, и иди к сокровищнице, и принеси мне золотой ковчежец, чтобы я дал вам ваше имение.
\vs Tjb 11:12
И она пошла и принесла его ему, и он открыл его и взял трехслойные опоясания, вид которых человек не может изъяснить.
\vs Tjb 11:13
Ибо они были не земной работы, но небесные искры света сверкали через них подобно лучам солнца.
\vs Tjb 11:14
И он дал по одному поясу каждой из его дочерей и сказал: Оденьте их как опоясания ваши, чтобы все дни жизни вашей они могли окутывать вас и наделять каждую из вас добродетелью.
\vs Tjb 11:15
И другая дочь, имя которой было Касия, сказала: Это достояние, о котором ты говоришь, разве оно лучше, чем таковое же у наших братьев? Что теперь? Сможем ли мы жить на это?
\vs Tjb 11:16
И их отец сказал им: Не только здесь вам хватит на жизнь, но они приведут вас в лучший мир жительства~--- на небеса.
\vs Tjb 11:17
Или вы не знаете, мои дети, значение этих вещей здесь? Тогда послушайте! Когда Господь посчитал меня достойным иметь сострадание ко мне и удалить от моего тела проказу и червей, Он призвал меня и вручил мне эти три пояса.
\vs Tjb 11:18
И Он сказал мне: Встань и препояшь чресла твои, как подобает мужу: Я взыщу тебя и провозглашу тебя Моим.
\vs Tjb 11:19
И я взял их и обвязал их вокруг моих чресл, и тотчас черви оставили моё тело, также и проказа, и всё мое тело обрело новую силу от Господа. И так я пошел, как если бы я никогда не страдал,
\vs Tjb 11:20
но также и в моем сердце я забыл муки. Тогда говорил Господь со мною в Его великом могуществе и показал мне всё, что было и будет.
\vs Tjb 11:21
Теперь, когда вы, дети мои, под охраной их,~--- не будете иметь замышляющего против вас врага, ни [злых] намерений в вашем разуме, потому что этот филактерий от Господа.
\vs Tjb 11:22
Так встаньте же и опояшьте их вокруг себя, прежде чем я умру, дабы вы могли увидеть ангелов, грядущих на моё разлучение, так чтобы вы могли видеть с удивлением силы Божии.
\vs Tjb 11:23
Тогда встала та, чье имя было Йемима, и опоясалась; и тотчас она отлучилась от своего тела, как сказал её отец, и она обрела иное сердце, как будто она никогда не заботилась о земных делах.
\vs Tjb 11:24
И она пела ангельские песни голосом ангелов, и она воспевала ангельскую похвалу Богу в танце.
\vs Tjb 11:25
Тогда другая дочь, по имени Касия, надела пояс, и её сердце преобразилось так, что она больше не желала мирских дел.
\vs Tjb 11:26
И её уста претворились в речь небесных Начал, и она пела благодарственные славословия творения вышней обители, и если кто-либо желал познать творение небес, он мог найти постижение в гимнах Касии.
\vs Tjb 11:27
Тогда другая дочь, по имени Керенгаппух, опоясалась и её уста заговорили на языке том высоком; ибо её сердце преобразилось, восхищаясь превыше мирских дел.
\vs Tjb 11:28
Она говорила речами Керубов, воспевающих похвалу Владыке вселенских сил и превознося их славу.
\vs Tjb 11:29
И тот, кто желает следовать остаткам славы отца, найдет их записанными в молитвах Керенгаппух.

\vs Tjb 12:1
После того, как сии три окончили петь гимны, я, Нерос, брат Иова, сел рядом с ним, когда он возлег.
\vs Tjb 12:2
И я слышал изумительные дела о трех дочерях моего брата: одно всегда сопутствовало другому среди благоговейного безмолвия.
\vs Tjb 12:3
И я написал эту книгу, содержащую гимны, помимо гимнов и знамений [святого] слова, ибо они были великими делами Бога.
\vs Tjb 12:4
И Иов возлег от болезни на его ложе, однако без боли и страдания, потому что его боль не одолела силу разума его по причине чудесного действия пояса, который он обернул вокруг себя.
\vs Tjb 12:5
Но по прошествии трех дней Иов увидел, что святые ангелы пришли за его душой, и тотчас он встал и взял лиру и дал её своей дочери Йемиме.
\vs Tjb 12:6
И Касии он дал касию, а Керенгаппух он дал тимпан, чтобы они могли благословлять святых ангелов, которые пришли за его душою.
\vs Tjb 12:7
И они взяли их, и пели, и играли на арфе и восхваляли и прославляли Бога в святой речи.
\vs Tjb 12:8
И после этого Он пришел~--- Тот, Который возседает на большой колеснице, и облобызал Иова, в то время как его три дочери смотрели, но другие не видели этого.
\vs Tjb 12:9
И Он взял душу Иова, и Он вознесся ввысь, взяв её рукою и перенеся её на колесницу, и Он пошел на Восток.
\vs Tjb 12:10
Его тело, однако, было принесено к могиле, в то время как три дочери шествовали впереди, надев свои пояса и воспевая гимны в похвалу Богу.
\vs Tjb 12:11
Тогда провели Нерос, его брат, и его семь сыновей с остальным народом и нищими, вдовами и немощными, великую скорбь по нему, говоря:
\vs Tjb 12:12
Горе нам, ибо сегодня была взята от нас сила немощных, свет слепых, отец сирот,
\vs Tjb 12:13
приют странников; уведен руководитель заблудших, покров нагих, защита вдов. Кто не будет сетовать по мужу Божию!
\vs Tjb 12:14
И поскольку они скорбели таким и таковым образом, они не хотели перенести его, чтобы положить в могилу.
\vs Tjb 12:15
После трех дней, однако, он был, наконец, положен в могилу, как бы в приятном сне, и он наследовал доброе имя, которое станет прославляться повсюду всеми поколениями мира.
\vs Tjb 12:16
Он оставил семь сыновей и трех дочерей, и вот не нашлось дочерей на земле, столь же прекрасных как дочери Иова.
\vs Tjb 12:17
Имя Иова было прежде Иовав, и он был назван Иовом у Господа.
\vs Tjb 12:18
Он жил прежде его бедствия восемьдесят пять лет, а после бедствия он взял двойную долю всего; следовательно его годы также удвоились, то есть~--- сто семьдесят лет. Таким образом он жил всего двести пятьдесят пять лет.
\vs Tjb 12:19
И он увидел сыновей его сыновей до четвертого поколения.
\vs Tjb 12:20
Так написано: он опять возстанет с теми, кого Господь пробудит.
\vs Tjb 12:21
Нашему Господу слава. Аминь.

\bibbookdescr{Ahh}{
  inline={\LARGE Книга\\\Huge Ахиахара премудрого},
  toc={Книга Ахиахара},
  bookmark={Книга Ахиахара},
  header={Книга Ахиахара премудрого},
  abbr={Ахх}
}
\vs Ahh 1:1
Сказал Ахиахар: Когда жил я во дни Сеннахериба, царя Ниневийского, и когда был я, Ахиахар, хранителем сокровищ его и писцом, и когда был я молод,
\vs Ahh 1:2
прорицатели, волхвы и мудрецы сказали мне: Не будет у тебя дитяти.
\vs Ahh 1:3
И стяжал я богатство великое, и блага имел я в избытке, и взял себе шестьдесят жен,
\vs Ahh 1:4
и построил им шестьдесят дворцов, пространных, чудных и удивительных, и дома многие;
\vs Ahh 1:5
и достиг я шестидесяти лет, и не родилось дитя у меня.
\vs Ahh 1:6
Тогда я, Ахиахар, стал приносить богам жертвы и приношения, возжигал пред ними курения и ароматы
\vs Ahh 1:7
и говорил к ним: о, боги, дайте мне сына, в котором будет благоволение мое до того дня, когда я умру, и он наследует мне, и закроет очи мои, и похоронит меня.
\vs Ahh 1:8
И от дня смерти моей до смерти его, если будет он брать на всякий день от золота моего вволю и расточать его непрестанно, богатство мое не кончится.
\vs Ahh 1:9
Идолы не отвечали ему, и он оставил их и преисполнился мукою и тоскою великою.
\vs Ahh 1:10
И изменил он речь свою, и помолился Богу, и уверовал, и призвал Его в горении сердца своего,
\vs Ahh 1:11
и сказал: Боже небес и земли, Создатель всех тварей, я молю Тебя даровать мне сына, в котором будет благоволение мое, который утешит меня в час мой смертный, и закроет очи мои, и предаст меня погребению.
\vs Ahh 1:12
И пришел голос, и сказал ему:
\vs Ahh 1:13
За то, что ты надеялся на богов, и возложил на них упование твое, и приносил им дары, ты умрешь, ни сынов не имея, ни дочерей;
\vs Ahh 1:14
однако же говорю тебе: вот, у тебя есть Надав, сын сестры твоей; возьми его, научи его всей науке твоей, и он примет наследие твое.

\vs Ahh 2:1
И взял я Надава, сына сестры моей, и пестовал его, и взращивал его, и приставил к нему восемь кормилиц, чтобы питать его.
\vs Ahh 2:2
Я давал ему елея и меда, облачал его в пурпур и багрянец, покоил его на ложах мягких и на коврах.
\vs Ahh 2:3
И преуспевал Надав, сын сестры моей, и возрастал, подобно благородному кедру.
\vs Ahh 2:4
И учил я его письму, и мудрости, и философии.
\vs Ahh 2:5
Когда же вернулся царь Ассур-Аддин от празднеств своих и от странствий своих, он однажды призвал меня, Ахиахара, писца своего и хилиарха своего,
\vs Ahh 2:6
и сказал мне: о, друг мой достославный, дорогой, почитаемый, мудрый и разумный, хранитель печати моей и поверенный тайн моих! ты состарился и одряхлел, и смерть твоя приблизилась; скажи, кто будет мне служить после смерти твоей и погребения твоего?
\vs Ahh 2:7
И сказал я ему: о, владыка мой и царь, вечно живи в роды родов! у меня есть сын сестры моей, который мне как сын.
\vs Ahh 2:8
Вот, я наставил его во всей мудрости моей, и он мудр и рассудителен.
\vs Ahh 2:9
И повелел мне владыка мой: ступай, приведи его, чтобы мне видеть его, и если мне будет угодно, он будет служить мне и ходить предо мною.
\vs Ahh 2:10
Что до тебя, продолжай путь твой; он упокоит тебя от трудов твоих и окружит старость твою почетом и славою.
\vs Ahh 2:11
И я, Ахиахар, взял Надава, сына сестры моей, и представил его пред лице царя Ассур-Аддина, и отдал его в руку цареву;
\vs Ahh 2:12
и когда увидел его царь, он имел в нем свое благоволение и возрадовался ему,
\vs Ahh 2:13
и сказал: да сохранит Господь сына твоего!
\vs Ahh 2:14
Как ты служил мне и отцу моему Сеннахерибу и как ты вел дела наши со тщанием, так будет делать и Надав, сын сестры твоей;
\vs Ahh 2:15
он послужит мне и устроит дела мои, а я воздам ему честь, и возвышу его ради тебя, и позабочусь о нем.
\vs Ahh 2:16
И склонился я пред царем, и сказал ему: владыка мой царь, вовеки живи!
\vs Ahh 2:17
Прошу тебя позаботиться о нем и помогать ему; пусть обитает он в доме твоем, как и я служил тебе и служил отцу твоему.
\vs Ahh 2:18
И подал царь ему руку, и поклялся держать его при себе в почете и чести.
\vs Ahh 2:19
И поднялся я, и сказал: да будет так, о царь!
\vs Ahh 2:20
И наставлял я сына моего Надава, и передавал ему премудрость мою, и обильно уделял ему поучение, пока он не стал писцом, как я.
\vs Ahh 2:21
Вот как наставлял я его, и вот как говорил я, Ахиахар Премудрый.

\vs Ahh 3:1
О, Надав, сын мой, послушай слова мои, последуй советам моим и помни о речах моих.
\vs Ahh 3:2
Ей, Надав, сын мой! Если будешь ты внимать словам моим, и замкнешь их в сердце твоем, и никому не откроешь их,
\vs Ahh 3:3
из страха, чтобы пещь огненная не попалила языка твоего, и чтобы ты не причинил муки телу твоему и урона разуму твоему, и чтобы не посрамиться тебе пред Богом и перед людьми.
\vs Ahh 3:4
О, сын мой, если услышишь слово, никому не открывай его и не говори ничего из того, что увидишь.
\vs Ahh 3:5
Сын мой, не развязывай узла сокровенного и не запечатывай узла развязанного.
\vs Ahh 3:6
Сын мой, направляй стопу свою и слово свое, слушай и не спеши давать ответ.
\vs Ahh 3:7
Сын мой, не желай красоты внешней, ибо красота проходит и минует, но добрая память и доброе имя пребывают вовеки.
\vs Ahh 3:8
Сын мой, не бери жену с речью сварливою, ибо в речах ее горечь, и в нити ее яд, и ты попадешь в западню ее.
\vs Ahh 3:9
Сын мой, если увидишь, что женщина украшена нарядами и умащена благовониями, но нрав ее дурной, сварливый и бесстыжий, пусть сердце твое не желает ее;
\vs Ahh 3:10
если отдашь ей все, что имеешь ты, найдешь, что это не обратится к славе твоей, но ты прогневишь Бога, и ярость Его постигнет тебя.
\vs Ahh 3:11
Сын мой, не спеши говорить и не влагай в ответы и речи твои похвальбы, словно миндальное дерево, пускающее листья свои и зелень свою прежде всех деревьев и дающее плоды свои после всех;
\vs Ahh 3:12
будь, как древо приятное, хвалимое, сладкое, полное утехи, как смоковница, которая склоняет ветви, зеленеет и пускает листья последней, но плод ее бывает вкушаем первым.
\vs Ahh 3:13
Сын мой, склони главу твою, устреми взор твой долу и приготовься быть внимателен.
\vs Ahh 3:14
Будь разумен, покорен, сдержан, невозмутим.
\vs Ahh 3:15
Не будь бесстыден и сварлив.
\vs Ahh 3:16
Не возвышай голоса твоего с похвальбою и буйством,
\vs Ahh 3:17
ибо если бы громкого голоса было довольно, чтобы воздвигнуть дом, осел строил бы по два дома в день;
\vs Ahh 3:18
и если бы плуг направлялся силою, верблюд направлял бы его лучше всех.
\vs Ahh 3:19
Сын мой, лучше таскать камни с мудрым, нежели пить вино с глупцом.
\vs Ahh 3:20
Сын мой, пролей вино твое и окропи им могилы праведных.
\vs Ahh 3:21
Сын мой, иди босыми ногами по терниям и по колючкам, чтобы проторить тропу к детям твоим и детям детей твоих.
\vs Ahh 3:22
Сын мой, когда дует ветер, а море еще не возмутилось, веди ладью твою и корабль твой к гавани, пока море не возмутилось, и не пришло в движение, и не умножило валов своих, и не потопило корабля.
\vs Ahh 3:23
Сын мой, не забывайся с глупцом и не имей общения с нецеломудренным.
\vs Ahh 3:24
Сын мой, не приближайся к женщине сварливой и говорящей заносчиво, не желай красоты женщины словоохотливой и нечистой,
\vs Ahh 3:25
ибо красота женщины есть позор ее, и блеском одежды своей и красотой внешней она пленит тебя и обманет тебя.
\vs Ahh 3:26
Сын мой, как нет пользы от колец в ушах дикого осла, так нет пользы от женщины с пышною осанкою, если она лукава в словах своих и делах своих, лишена мудрости, словоохотлива и многоречива.
\vs Ahh 3:27
Сын мой, если мудрый недужен, врач сможет уврачевать и вылечить его, но нет врачевания для недугов и ран неразумного.
\vs Ahh 3:28
Сын мой, прими того, кто ниже тебя и беднее тебя; если он не воздаст тебе, Бог воздаст тебе.
\vs Ahh 3:29
Сын мой, не уставай наказывать дитя твое; наказание дитяти как удобрение сада, как завязывание кошеля, как обуздание скотины и как затвор на воротах.
\vs Ahh 3:30
Сын мой, оторви сына твоего от зла, чтобы уготовать себе покой в старости твоей;
\vs Ahh 3:31
поучай его и наказывай его, пока он юн, понудь его слушать повелений твоих, чтобы немного после он не стал вопить и восставать на тебя,
\vs Ahh 3:32
чтобы он не навлек на тебя бесчестия пред товарищами твоими,
\vs Ahh 3:33
чтобы не пришлось тебе опустить голову в людных местах и на площадях,
\vs Ahh 3:34
чтобы не краснел ты по причине лукавства дел его и не был ты уничижен по причине порочного бесстыдства его.
\vs Ahh 3:35
Сын мой, не доводи детей твоих до крайности, чтобы они не прокляли тебя и Бог не прогневался на них,
\vs Ahh 3:36
ибо написано: кто злословит отца своего и матерь свою, смертию умрет; это грех, прогневляющий Бога;
\vs Ahh 3:37
и еще: кто чтит отца своего и матерь свою, будет долголетен и будут ему блага обильные.
\vs Ahh 3:38
Сын мой, не пускайся в путь без меча и не переставай помнить о Боге в сердце твоем,
\vs Ahh 3:39
ибо ты не знаешь, когда враги лютые встретятся тебе; будь готов на пути твоем, ибо враги твои многочисленны.
\vs Ahh 3:40
Сын мой, каково древо, изобилующее плодами, листами и ветвями, таков муж с женою доброю, и плоды их дети их и родственники.
\vs Ahh 3:41
У кого нет ни жены, ни детей, ни родственников, презрен и пренебрегаем от врагов своих, как древо, стоящее вдали от дороги, которое прохожие пинают ногами и едят от плодов его, и дикий зверь стряхивает листы его и ест их.
\vs Ahh 3:42
Сын мой, если есть у тебя слуги, не предпочитай одного и не отвергай другого, ибо ты не знаешь, какого выберешь в конце.
\vs Ahh 3:43
Сын мой, коза блуждающая и умножающая шаги свои станет добычею волка.
\vs Ahh 3:44
Сын мой, услади язык твой словами Божьими и усовершенствуй слова уст твоих;
\vs Ahh 3:45
говори к любому с добротою и изяществом, ибо пасть пса промышляет ему хлеб и гортань его навлекает на него удары и камни.
\vs Ahh 3:46
Сын мой, не давай ближнему твоему наступать на ногу твою, чтобы он не наступил на грудь твою.
\vs Ahh 3:47
Сын мой, если зовешь мудрого делать работу твою, не говори ему долгих поучений и вразумлений, ибо он сделает работу твою, как желает сердце твое;
\vs Ahh 3:48
но если зовешь неразумного, не говори с ним перед другим, но лучше ступай и не зови его, ибо не сделает он работу по сердцу твоему, сколь бы долгие советы ни давал ты ему.
\vs Ahh 3:49
Сын мой, поспешно уходи со свадеб и с празднеств, не дожидаясь, чтобы главу твою умастили елеем и благовониями, дабы не навлечь на главу твою ударов и рубцов.
\vs Ahh 3:50
Сын мой, того, чья рука полна, именуют мудрым и досточтимым, а того, чья рука пуста, именуют злым, убогим, бедным и неимущим, и никто не воздает ему чести.
\vs Ahh 3:51
Сын мой, я вкушал полынь и пробовал мирру, но не нашел ничего горше бедности и нужды.
\vs Ahh 3:52
Сын мой, я поднимал железо и свинец, но не нашел ничего тяжелее хулы и клеветы.
\vs Ahh 3:53
Сын мой, я ворочал камни, но не нашел ничего столь тяжкого, как зять, живущий в доме тестя своего.
\vs Ahh 3:54
Сын мой, если нога твоя оступится и ты упадешь, это лучше, чем если ты оступишься языком твоим.
\vs Ahh 3:55
Сын мой, друг близкий лучше, чем брат далекий,
\vs Ahh 3:56
и доброе имя лучше, чем богатства мира, ибо богатства прейдут и развеются, но доброе имя пребудет вечно.
\vs Ahh 3:57
Сын мой, красота гибнет, разрушается и пропадает, и мир преходит, и все престает и прекращается, но доброе имя не преходит, не престает и не разрушается.
\vs Ahh 3:58
Сын мой, шум плача и рыдания лучше, нежели шум веселия и празднества,
\vs Ahh 3:59
ибо внимать шуму плача учит человека постигнуть грех свой и дать за него удовлетворение.
\vs Ahh 3:60
Сын мой, не восставай в суждении твоем на мужей славных и превосходных величием и властью, ибо от шуток и слов глумливых происходят гнев и раздор.
\vs Ahh 3:61
Слово гневное пробуждает и возбуждает ярость, и от ярости этой происходит раздор, а после раздора приходит и убийство.
\vs Ahh 3:62
\ldots если ты окажешься в месте таком, тебя могут убить или тебя могут позвать в свидетели;
\vs Ahh 3:63
и когда от тебя будут требовать и вымогать свидетельство твое, ты претерпишь страдание и от стыда или страха дашь свидетельство ложное и будешь посрамлен.
\vs Ahh 3:64
И я повелеваю тебе: спеши бежать из того места, где спорят, и душа твоя будет умиротворена.
\vs Ahh 3:65
Сын мой, стяжи сердце чистое и неоскверненное, разумение ясное и непомраченное,
\vs Ahh 3:66
доставь себе дух смиренный и найди себе стезю прямую, и не будет на свете человека достойнее тебя, и жизнь твоя будет блаженна.
\vs Ahh 3:67
Сын мой, не входи в сад судей, страшись судилища и не бери в жены дочь судьи.
\vs Ahh 3:68
Сын мой, защищай друга твоего перед начальником словами добрыми и исторгай немощь его из пасти льва.
\vs Ahh 3:69
Сын мой, не радуйся смерти врага твоего.
\vs Ahh 3:70
Сын мой, когда увидишь, что вошел человек старше тебя, встань перед ним.
\vs Ahh 3:71
Сын мой, око человеческое подобно источнику: оно не насытится, пока не наполнится прахом.
\vs Ahh 3:72
Сын мой, если хочешь быть мудр, воспрети устам твоим ложь и руке твоей хищение, и будешь мудр.
\vs Ahh 3:73
Сын мой, не входи в устройство брака женщины, ибо, если она будет недовольна, она проклянет тебя, и, если она будет счастлива, она не вспомнит о тебе.
\vs Ahh 3:74
Сын мой, если ты украл, извести власть имущего и предложи ему долю, и тогда ты можешь получить прощение, в противном же случае приключится тебе зло.
\vs Ahh 3:75
Сын мой, пусть лучше мудрый побьет тебя многими ударами жезла, чем неразумный помажет тебя елеем благовонным.
\vs Ahh 3:76
Сын мой, пусть нога твоя не бежит к другу твоему, чтобы он не пресытился тобою и не возненавидел тебя.
\vs Ahh 3:77
Сын мой, не возлагай кольца золотого на руку твою, если ты небогат, чтобы неразумные не глумились над тобою.

\vs Ahh 4:1
И когда я, Ахиахар, преподал мудрость эту Надаву, сыну сестры моей, я полагал, что он сохранит ее в сердце своем, пребывая при дворе, и не ведал того, что он не слушал слов моих, но бросал их словно на ветер.
\vs Ahh 4:2
Он усвоил обыкновение говорить: Ахиахар, отец мой, стар и утратил дух свой.
\vs Ahh 4:3
И Надав, сын мой, присвоил стада мои, и расточил добро мое, и не пощадил лучших слуг моих, и бил их пред лицем моим, и не пожалел скотов моих и мулов моих, и умерщвлял их.
\vs Ahh 4:4
Когда увидел я, что творил он, я сказал ему:
\vs Ahh 4:5
Сын мой, не тронь добра моего, ибо сказано в изречениях: чего рука твоя не стяжала око твое не видело.
\vs Ahh 4:6
И я известил обо всем этом владыку моего царя,
\vs Ahh 4:7
и повелел царь: пусть никто не дерзает приближаться к добру Ахиахара, писца; пока живет Ахиахар, да не приближается никто ни к достоянию его, ни к дому его.

\vs Ahh 5:1
Когда увидел Надав, что я взял брата его младшего и стал его воспитывать, это было ему неприятно; и позавидовал он, и возымел в уме своем помыслы злые по этой причине,
\vs Ahh 5:2
и сказал: Ахиахар, отец мой, стар, и мудрость его пропала, и слова его достойны презрения; ужели он отдаст добро свое брату моему, а меня изгонит из дома своего?
\vs Ahh 5:3
И когда я, Ахиахар, услышал слова Надавовы, я сказал:
\vs Ahh 5:4
Увы тебе, премудрость моя! Надав, сын мой, лишил тебя вкуса твоего и презрел мудрые слова мои.
\vs Ahh 5:5
Когда сказал я это, сын мой весьма раздражился и приготовил в сердце своем зло мне.
\vs Ahh 5:6
И пошел он ко двору царскому, чтобы сотворить зло, которое было в сердце его, как будто написал Ахиахар от лица своего письма лукавые, а он отправился ко двору объявить о них.
\vs Ahh 5:7
И вот письма от имени моего к царям, враждебным царю Сеннахерибу.
\vs Ahh 5:8
Одно было к царю Персидскому и Еламитскому, и он написал его так:
\vs Ahh 5:9
От Ахиахара, писца и хранителя печати царя Сеннахериба, мир тебе! Когда получишь ты это письмо, выступай немедля, и приходи в Ассирию, и возьмешь ты всю землю сию без войны и без боя.
\vs Ahh 5:10
Другое было от имени моего к фараону, царю Мицрейскому, и он составил его так:
\vs Ahh 5:11
Когда придет к тебе письмо это, выходи ко мне на долину южную двадцать пятого числа месяца Ава; и приведу тебя к Ниневии, и овладеешь ты царством без боя.
\vs Ahh 5:12
Он переписал письма эти по подобию руки моей и запечатал их печатью моей, а после подбросил их в один из покоев царских.

\vs Ahh 6:1
И написал он еще другое письмо от лица владыки моего царя:
\vs Ahh 6:2
От Ассур-Аддина Ахиахару, писцу моему и хранителю печати моей, мир!
\vs Ahh 6:3
Когда получишь ты это письмо, собери все воинство у горы и выступай к долине Нешрин двадцать пятого числа месяца Ава;
\vs Ahh 6:4
и когда увидишь ты, что я приближаюсь к тебе, построй воинства твои пред лицем моим, как бы ты приготовлялся к сражению,
\vs Ahh 6:5
ибо посланцы фараона, царя Мицрейского, придут со мною, и они увидят, каковы мои силы.
\vs Ahh 6:6
И сын мой Надав передал мне письмо через двух письмоносцев.
\vs Ahh 6:7
И взял сын мой Надав одно из писем, так, словно бы нашел его, и прочитал его пред царем.
\vs Ahh 6:8
И тогда царь уязвился весьма, и прогневался на Ахиахара, и говорил:
\vs Ahh 6:9
Боже! В чем погрешил я против Тебя и против Ахиахара, что он решился так поступить со мною?

\vs Ahh 7:1
И тогда ответил Надав и сказал царю:
\vs Ahh 7:2
Не печалься, о, владыка мой царь, но пойдем на долину Нешрин, как написано в этом письме; мы узнаем истину, и все будет так, как ты повелишь.
\vs Ahh 7:3
И повелел царь приготовляться выступать на долину, чтобы узнать, какова истина в этом деле,
\vs Ahh 7:4
и Надав, сын мой, сопровождал царя, и пришли они, и обрели меня с воинством, сопутствовавшим мне, в долине Нешрин.
\vs Ahh 7:5
И когда увидел я, что царь приближается ко мне, я построил воинство в боевой порядок пред лицем его, словно для сражения, доверясь письму, которое послал ко мне сын мой.
\vs Ahh 7:6
И сказал сын мой царю: ступай к себе со всяким спокойствием, о, владыка мой!
\vs Ahh 7:7
Я же приведу пред очи твои Ахиахара, отца моего.
\vs Ahh 7:8
И царь отошел в место свое.

\vs Ahh 8:1
И пришел ко мне Надав, сын мой, и взял слово, и сказал:
\vs Ahh 8:2
Владыка царь послал меня к тебе, чтобы сказать тебе: все, что ты сделал, хорошо, царь весьма хвалит тебя.
\vs Ahh 8:3
Ныне же отпусти воинства; пусть все расходятся к себе, ты же иди к царю один.
\vs Ahh 8:4
И пошел я тогда к царю, и когда увидел он меня, то сказал мне:
\vs Ahh 8:5
Ты пришел, Ахиахар, писец мой, отец и кормилец Ассура и Ниневии!
\vs Ahh 8:6
Я всегда почитал тебя и покоил тебя, ты же отпал от меня и стал один из врагов моих.
\vs Ahh 8:7
После он дал мне письмо, написанное от имени моего и запечатленное печатью моей.
\vs Ahh 8:8
И сказал мне царь: прочти письмо это.
\vs Ahh 8:9
И когда я прочел его, сотряслись члены мои, и язык мой перестал повиноваться мне; искал я слово мудрое и не находил его.
\vs Ahh 8:10
И взял слово Надав, сын мой, и сказал:
\vs Ahh 8:11
Убирайся с глаз царя, старик неразумный, и простри руки твои в узы и ноги твои в железа!
\vs Ahh 8:12
Тогда царь Ассур-Аддин отвратил лице свое от меня и сказал Навусемаку, палачу, который был со мною в дружбе:
\vs Ahh 8:13
Пойди, убей Ахиахара и удали голову его на сто локтей от тела его.
\vs Ahh 8:14
Тогда пал я лицем на землю, поклонился царю и сказал ему:
\vs Ahh 8:15
Владыка царь, вовеки живи! Ты хочешь умертвить меня; да будет по воле твоей.
\vs Ahh 8:16
Я знаю, что не погрешил пред тобою; но повели, владыка царь, чтобы меня умертвили перед дверью дома моего и чтобы тело мое отдали для погребения.
\vs Ahh 8:17
И повелел царь, чтобы было так.

\vs Ahh 9:1
И я, Ахиахар, послал сказать жене моей:
\vs Ahh 9:2
Приходи ко мне и приведи с собою тысячу девиц, одетых в виссон, в пурпур и в шафран, которые будут плясать предо мною и оплакивать меня до самой смерти моей.
\vs Ahh 9:3
И приготовь еды палачу Навусемаку, другу моему, и Парфянам, которые придут с ним;
\vs Ahh 9:4
выйди к ним навстречу и пригласи их войти ко мне, дабы и я смог войти в дом мой, как гость и чужак.
\vs Ahh 9:5
И когда жена моя приняла вестников, исполнилась она премудрости великой и выполнила все, что я велел ей.
\vs Ahh 9:6
И вышла она навстречу Навусемаку и Парфянам, и пригласила их войти в дом ее.
\vs Ahh 9:7
И принесла Эшфагни еды Навусемаку и Парфянам хлеба; и достала она им также вина, и разлила для них;
\vs Ahh 9:8
и служила им Эшфагни на пире их, пока они не опьянели и не уснули.
\vs Ahh 9:9
Когда опьянели Парфяне от вина, уснули они сном глубоким, и каждый из них уснул на месте своем.
\vs Ahh 9:10
И восхвалил я Господа небес и земли за все, что произошло, и сказал я:
\vs Ahh 9:11
Боже, Спаситель мира, ведающий все, что было, и все, что будет, призри на меня оком милостивым пред Навусемаком.

\vs Ahh 10:1
И когда я, Ахиахар, увидел все это, я заговорил и сказал Навусемаку:
\vs Ahh 10:2
Подними глаза твои к небу, Навусемак, и помысли о Боге; вспомни о хлебе и соли, которые мы съели с тобою, и не замышляй моей смерти.
\vs Ahh 10:3
Вспомни, что отец владыки моего царя также предал мне тебя, чтобы я умертвил тебя, а я не умертвил тебя, ибо знал, что ты не согрешил, и я оставил тебе жизнь до того дня, когда царь пожелал видеть тебя и дал мне дары многие. И ты спаси меня ныне.
\vs Ahh 10:4
Чтобы не распространилась молва и не сказали, что он не предан смерти, вот, у меня есть в темнице моей человек, заслуживший смерть; возьми одежды мои, надень на него и после повели Парфянам умертвить его.
\vs Ahh 10:5
Когда я сказал это, Навусемак, палач, друг мой, исполнился печали обо мне;
\vs Ahh 10:6
и взял он одежды мои, и облачил в них раба, который был в темнице, а после разбудил Парфян, которые поднялись под действием вина, и умертвили раба и отдалили голову его на сто локтей от туловища его, и отдали тело его для погребения.
\vs Ahh 10:7
И распространилась молва по Ассирии и по Ниневии, что Ахиахар умерщвлен.

\vs Ahh 11:1
И тогда Навусемак совместно с женою моею Эшфагни устроил мне в земле укром в три локтя ширины, и четыре локтя длины, и пять локтей высоты;
\vs Ahh 11:2
они дали мне есть и пить и послали сказать владыке моему царю, что Ахиахар предан смерти.
\vs Ahh 11:3
И сказал царь: страдание Ахиахарово пало на главу мою; писец мой и мудрец, защищавший пролом в стене града, я отправил тебя на гибель по слову отрока!
\vs Ahh 11:4
И призвал царь Надава, сына моего и сказал ему: ступай, оплакивай отца твоего!
\vs Ahh 11:5
Надав, сын мой, пошел в дом мой, и он не оплакивал меня и не творил память обо мне, но собрал женщин блудных и посадил их есть и пить среди песен и веселия.
\vs Ahh 11:6
Он убивал, и обнажал, и избивал слуг моих и служанок моих, он не постыдился даже женщины, воспитавшей его, но велел ей совершить с ним блуд и распутство.
\vs Ahh 11:7
Я же в недрах рва темного слышал вопль поваров моих, и пирожников моих, и хлебников моих, которые творили плач и стенание.
\vs Ahh 11:8
И тотчас обратил я молитву мою ко Всевидящему.
\vs Ahh 11:9
Прошли дни, и пришел Навусемак, и отворил затвор мой, и дал мне есть и пить,
\vs Ahh 11:10
и я сказал ему: поминай меня пред лицем Бога, и после всего, что ты увидел, скажи Ему:
\vs Ahh 11:11
Яхве, Праведный и Благий на небесах и на земле, доныне Ахиахар был ограждаем Тобою, и он приносил Тебе в жертву тельцов тучных; и вот, лежит он во рву мрачном, и свет не доходит к нему.
\vs Ahh 11:12
Услышь, Яхве, вопль раба Твоего и смилуйся над ним!

\vs Ahh 12:1
И когда узнал царь Мицрейский, что я, Ахиахар, умерщвлен, он пришел в радость великую и послал Ассур-Аддину письмо:
\vs Ahh 12:2
Царь Мицрейский Ассур-Аддину, царю Ассирийскому и Ниневийскому, мир!
\vs Ahh 12:3
Мне нужно построить крепость между небом и землею; пошли мне зодчего мудрого, которому я мог бы поручить все дело, чтобы я его вопрошал, а он мне отвечал.
\vs Ahh 12:4
Если человек, которого ты пошлешь ко мне, сделает все, что я скажу, я соберу и пошлю к тебе через него подать с Мицры за три года.
\vs Ahh 12:5
Если же ты не пошлешь мне человека, который смог бы сделать то, о чем я говорю, собери и пошли ко мне через моего посланника подать с Ассирии и Ниневии за три года.
\vs Ahh 12:6
Когда письмо это было прочитано пред лицем царя, он повелел собрать всех своих вельмож, и мудрецов, и волшебников, и книжников царства своего и сказал им:
\vs Ahh 12:7
Кто из вас пойдет в Мицру и даст ответ фараону?
\vs Ahh 12:8
И ответили вельможи царю, и сказали ему все:
\vs Ahh 12:9
Ты знаешь, владыка царь, что во дни твои и во дни отца твоего все вопросы такого рода разрешал Ахиахар, писец,
\vs Ahh 12:10
ныне же Надав, сын его, наследовавший ремесло писца и наученный им мудрости его, должен заняться делом этим.

\vs Ahh 13:1
И когда услышал Надав слова эти, возопил он пред лицем царя воплем великим и сказал царю:
\vs Ahh 13:2
И боги не могут сделать таких дел, как же смогут это люди?
\vs Ahh 13:3
При словах этих царь опечалился и удручился, и сошел с престола своего, и облачился во вретище, и сел на землю, и возрыдал, и говорил с плачем:
\vs Ahh 13:4
Увы тебе, Ахиахар, писец мой, что я велел погубить тебя по слову отрока, и не осталось никого, кто был бы тебе подобен и равен.
\vs Ahh 13:5
А ныне кто вернет тебя мне? Я заплатил бы за тебя, оценив тебя на вес золота.
\vs Ahh 13:6
И когда услышал Навусемак от царя таковые слова, он простерся ниц, и поклонился царю, и сказал:
\vs Ahh 13:7
Царь, вовеки живи! Презирающий слово владыки своего повинен смерти; повели же распять меня на древе, ибо ослушался я слова твоего;
\vs Ahh 13:8
ведь Ахиахар, которого повелел ты мне умертвить, жив доселе.
\vs Ahh 13:9
И ответил царь Набусемаку: говори, о Навусемак, ибо ты человек добрый и справедливый и не можешь совершить зла.
\vs Ahh 13:10
Если дело и вправду так, как ты сказал, и если ты мне представишь Ахиахара живым, я дам тебе дары великие: серебра мириаду талантов и пурпура сотню риз.
\vs Ahh 13:11
Когда Навусемак услышал, что царь говорит это, он начал говорить:
\vs Ahh 13:12
Я молю владыку моего царя сказать мне одно лишь, что он забывает за мною эту вину и не держит гнева на меня.
\vs Ahh 13:13
И царь поклялся ему в этом с радостию.

\vs Ahh 14:1
И взошел тогда Навусемак на колесницу, и примчался так быстро, как ветер могучий.
\vs Ahh 14:2
И отворил он мне, и я вышел на свет.
\vs Ahh 14:3
И не была посрамлена надежда моя на Бога.
\vs Ahh 14:4
И отвел меня Навусемак к царю, и повергся я на землю.
\vs Ahh 14:5
Волосы мои спадали на плечи мои, и борода моя доходила до груди моей, и тело мое было засыпано землею, и ногти мои отросли, как когти орла.
\vs Ahh 14:6
Когда царь увидел меня, он много плакал и сказал мне:
\vs Ahh 14:7
О, Ахиахар, я не погрешил против тебя, но это сын твой, воспитанный тобою, погрешил против тебя.
\vs Ahh 14:8
И отвечал я, и сказал царю: владыка мой, ныне вижу я лице твое, и скорбь моя отнята у меня.
\vs Ahh 14:10
И отвечал царь, и сказал мне: иди в дом твой, и обрежь власы твои, и омой тело твое в водах, и давай себе покой сорок дней, а после приходи пред очи мои.
\vs Ahh 14:11
И пошел я в дом мой, и делал все, как повелел мне владыка мой царь;
\vs Ahh 14:12
но лишь двадцать дней оставался я в доме моем, а когда возвратились ко мне силы мои, пошел я пред очи царя.
\vs Ahh 14:13
И показал мне царь письмо, пришедшее от царя Мицрейского.
\vs Ahh 14:14
И заговорил царь, и сказал: ты только посмотри, Ахиахар, на Мицрейцев! Что написали они мне, и какую подать наложили они на Ассур и на Ниневию!
\vs Ahh 14:15
И отвечал я ему, и сказал: владыка мой царь, вовеки живи!
\vs Ahh 14:16
Об этом деле не пекись и не печалуйся; я пойду в Мицру, и я дам ответ, и я представлю всем недругам твоим загадку и ее решение, и я принесу подать с Мицры за три года.
\vs Ahh 14:17
И при словах таковых царь возрадовался весьма и устроил день веселия, и оставила печаль лице его, и принес он в жертву тельцов и овнов, и дал мне дары великие.
\vs Ahh 14:18
И поставил он Навусемака надо всеми, и дал ему чин высокий.

\vs Ahh 15:1
И написал я письмо Эшфагни, жене моей:
\vs Ahh 15:2
О, супруга моя, когда письмо это придет к тебе, прикажи, чтобы ловчие изловили для меня двух орлов юных;
\vs Ahh 15:3
и повели; чтобы слуги мои принесли для меня нити льняной и сделали из нее для меня две веревки в палец толщиною и в тысячу локтей длиною; и скажи, чтобы кузнецы сковали для меня две клетки.
\vs Ahh 15:4
И отдай Навухаила и Тевшалома, слуг моих, семи кормилицам первородившим, чтобы те питали их млеком своим, пока они не вырастут;
\vs Ahh 15:5
и помести с ними орлов юных, чтобы они возрастали вместе, и давай орлам в корм по два овна на каждый день.
\vs Ahh 15:6
И пусть отроки выучатся говорить: принесите глины и кирпичей! Зодчие, гости царя, не имеют себе дела.
\vs Ahh 15:7
Жена моя была весьма смышлена и сделала все, что я велел ей;
\vs Ahh 15:8
и получил я приказ от царя отправляться в Мицру.
\vs Ahh 15:9
При вести этой ассирияне и ниневитяне обрадовались весьма и удалились к себе.
\vs Ahh 15:10
И ответил я царю: владыка мой царь, дозволь мне отправляться в Мицру.
\vs Ahh 15:11
И когда повелел он мне отправляться, я взял с собою отряд многочисленный и выступил в путь.
\vs Ahh 15:12
И когда прибыл я к вечернему отдыху, я прежде всего отпустил войско,
\vs Ahh 15:13
после взял из клеток орлов юных, привязал веревки к ногам их и велел отрокам моим влезть на веревки, а после отпустил орлов,
\vs Ahh 15:14
и поднялись они в воздух; а отроки кричали, как были научены:
\vs Ahh 15:15
Принесите кирпичей, глины и строительного раствора! это нужно для гостей и зодчих царя!
\vs Ahh 15:16
И после этого я вернул их к себе на землю.

\vs Ahh 16:1
И когда прибыл я в Мицру, слуги царя доложили обо мне, и царь повелел, чтобы я пришел к нему.
\vs Ahh 16:2
Я вошел к царю и приветствовал его, и после он спросил меня:
\vs Ahh 16:3
Каково твое имя? И ответил я:
\vs Ahh 16:4
Абикам, один из муравьев царя Ниневийского.
\vs Ahh 16:5
И когда фараон услыхал это, он был недоволен и сказал:
\vs Ahh 16:6
Ужели владыка твой настолько презирает меня, что он отрядил ко мне муравья, чтобы отвечать мне?
\vs Ahh 16:7
И ответил я, и сказал ему: владыка, пчела есть малейшая среди птиц и насекомых, и посмотри, какое дивное дело творит она.
\vs Ahh 16:8
С почетом допускают ее к столу государей великих;
\vs Ahh 16:9
а пред Сеннахерибом и малые как великие, и он судит их по величию и по назначению, им определенному.
\vs Ahh 16:10
И после он сказал мне: ступай, Абикам, в отведенное тебе место, а поутру вставай и приходи ко мне.
\vs Ahh 16:11
И повелел царь, чтобы вельможи его заутра облачились в одежды цвета красного, а сам он облачился с утра в одежды из виссона и пурпура;
\vs Ahh 16:12
и воссел он на престоле своем, а вельможи его заняли места вокруг него и перед ним. И я был введен в присутствие царя, и затем он спросил меня:
\vs Ahh 16:13
С кем, можно сравнить меня, о Абикам, и с кем можно сравнить вельмож моих?
\vs Ahh 16:14
И ответил я ему: тебя можно сравнить с Вилом, о владыка мой царь, а вельмож твоих со жрецами его.
\vs Ahh 16:15
И сказал мне царь: ступай, Абикам, и поутру приходи.
\vs Ahh 16:16
И повелел царь вельможам своим сменить одежды свои и облечься в одежды из льна белого,
\vs Ahh 16:17
и сам он также облекся в белое, а после воссел на престоле своем, и вельможи его заняли места перед ним и вокруг него.
\vs Ahh 16:18
И ввели меня пред очи его, и спросил он меня:
\vs Ahh 16:19
С кем можно сравнить меня, о Абикам, и с кем можно сравнить вельмож моих?
\vs Ahh 16:20
И ответил я ему, и сказал ему: тебя можно сравнить с Солнцем, а вельмож твоих с лучами его.
\vs Ahh 16:21
И снова сказал мне царь: ступай, Абикам, а поутру возвращайся ко мне.
\vs Ahh 16:22
И повелел он вельможам своим заутра переоблачиться в одежды черные;
\vs Ahh 16:23
врата дворца были покрыты черным и алым; царь же облекся в одежды алые.
\vs Ahh 16:24
После фараон велел меня ввести; я вошел, и он спросил меня:
\vs Ahh 16:25
С кем можно сравнить меня, о Абикам, и с кем можно сравнить вельмож моих?
\vs Ahh 16:26
И сказал я ему: тебя можно сравнить с Месяцем, о царь, а вельмож твоих со звездами.
\vs Ahh 16:27
И сказал он мне: ступай, Абикам, а поутру приходи ко мне.
\vs Ahh 16:28
И повелел фараон вельможам своим облечься заутра в другие одежды разных цветов;
\vs Ahh 16:29
и врата дворца были затянуты красным разных оттенков; и царь облачился в одежды разноцветные.
\vs Ahh 16:30
После фараон велел меня ввести: я вошел, и он спросил меня:
\vs Ahh 16:31
С кем можно сравнить меня и с кем можно сравнить вельмож моих?
\vs Ahh 16:32
И ответил я ему: тебя можно сравнить с Нисаном, а вельмож твоих с цветами его.
\vs Ahh 16:33
Когда царь услыхал это, он возрадовался весьма и был исполнен веселия. Он сказал мне:
\vs Ahh 16:34
Абикам, ты сравнил меня один раз с Вилом, а вельмож моих со жрецами его,
\vs Ahh 16:35
другой раз ты сравнил меня с Солнцем, а вельмож моих с лучами его,
\vs Ahh 16:36
в третий раз ты сравнил меня с Месяцем, а вельмож моих со звездами,
\vs Ahh 16:37
в четвертый раз ты сравнил меня с Нисаном, а вельмож моих с цветами его;
\vs Ahh 16:38
с кем же ты сравнишь Ассур-Аддина, владыку твоего?
\vs Ahh 16:39
И ответил я, и сказал ему: сохрани меня Бог, о царь, говорить о владыке моем Ассур-Аддине, когда ты сидишь,
\vs Ahh 16:40
ибо владыка мой царь Ассур-Аддин подобен Вилсамину, а вельможи его подобны молниям;
\vs Ahh 16:41
стоит ему пожелать, и он обращает росу и дождь в твердый град, он заставляет дыму восходить к небесам владычества своего, он издает гром и рыкание и возбраняет Солнцу вставать и лучам его показываться;
\vs Ahh 16:42
он возбраняет Вилу и жрецам его уходить и приходить местами людными;
\vs Ahh 16:43
он возбраняет Месяцу восходить и звездам блистать.
\vs Ahh 16:44
И если он захочет повелеть ветру северному, ветер соделает дождь и град, и побьет Нисан, и погубит цветы его.
\vs Ahh 16:45
И возмутился царь, слыша это.
\vs Ahh 16:46
И спросил фараон: заклинаю тебя жизнью владыки твоего Ассур-Аддина, каково имя твое?
\vs Ahh 16:47
И ответил я: Ахиахар, писец! И печать царя Ассур-Аддина вручена мне.
\vs Ahh 16:48
И спросил Фараон: так ты жив?
\vs Ahh 16:49
И ответил я: так, я жив, о владыка мой царь, и я видел Ассур-Аддина, и он продлил дни мои, и Бог избавил меня от смерти, и от казни лютой, и от греха, которого не творили руки мои.
\vs Ahh 16:50
И сказал мне царь: ступай, писец, а поутру приходи к мне и скажи мне слово, которого никто не слыхал и которого не слыхали вельможи мои ни в едином из городов Мицрейских.

\vs Ahh 17:1
И тогда я, Ахиахар, отошел в уединение и написал письмо такое:
\vs Ahh 17:2
От фараона, царя Мицрейского, Ассур-Аддину, царю Ассирийскому, мир!
\vs Ahh 17:3
Цари имеют нужду в царях, и судьи имеют нужду в судьях, а в наше время они имеют нужду в дарах, ибо средства их умалены.
\vs Ahh 17:4
Сокровищнице моей недостает денег, но позволь мне занять в твоей сокровищнице девятьсот талантов серебра, и я вскоре верну их тебе.
\vs Ahh 17:5
Я свернул письмо это и взял его с собою, и я сказал царю:
\vs Ahh 17:6
Слова, написанного в письме этом, не слышал ни ты, ни другой человек.
\vs Ahh 17:7
Все возопили: мы слышали его, и нет в том никакого сомнения!
\vs Ahh 17:8
И тогда ответил я им: итак, вы слышали, что Мицра должна Ассирии девятьсот талантов! И все были исполнены изумления.
\vs Ahh 17:9
И сказал мне тогда царь: Ахиахар!
\vs Ahh 17:10
И я ответил: вот я!
\vs Ahh 17:11
И сказал он мне: построй мне дворец между небом и землею, превыше земли локтей на тысячу.
\vs Ahh 17:12
И тотчас взял я из клеток орлов моих юных, и привязал к ногам их веревку должной длины, и велел посадить на нее мальчиков, которые принялись кричать:
\vs Ahh 17:13
Глины сюда, кирпичей сюда! Вот идут зодчие! Дайте им, с чем работать, что нужно зодчим царевым, и смешайте для зодчих вина.
\vs Ahh 17:14
Вельможи увидели, услышали и были в изумлении.
\vs Ahh 17:15
Тогда я, Ахиахар, взял жезл и бил зодчих, пока они не побежали доставать потребное для стройки.
\vs Ahh 17:16
Тогда сказал царь: ты обезумел, Ахиахар! кто сможет подавать им наверх то, что они требуют?
\vs Ahh 17:17
Я сказал им: зачем же вы поминаете понапрасну имя Ассур-Аддина? Если бы он был здесь и если бы он пожелал построить два дворца в один день, он бы их построил.
\vs Ahh 17:18
Царь сказал мне: оставь ты этот дворец. Приходи ко мне поутру.
\vs Ahh 17:19
И когда поутру я вошел к нему, он посмотрел на меня, и увидел меня, и сказал:
\vs Ahh 17:20
Ахиахар, изъясни мне, что это у нас приключилось? Жеребец твоего владыки заржал в Ассуре, в Ниневии, а наши кобылицы услышали его и выкинули плод.
\vs Ahh 17:21
И тогда я, Ахиахар, вышел от царя; и я велел слугам взять кота, бога Мицрейцев, и бить его до тех пор, покуда Мицрейцы не услышали воплей его.
\vs Ahh 17:22
Мицрейцы пошли и донесли царю: этот Ахиахар взял кота, который есть бог, и бил его.
\vs Ahh 17:23
Царь внял им и спросил меня: о, Ахиахар, зачем ты учиняешь богам нашим бесчестие?
\vs Ahh 17:24
И я сказал ему: царь, вовеки живи! Этот кот учинил мне урон великий и отнюдь не малый;
\vs Ahh 17:25
ибо царь подарил мне петуха, имевшего голос весьма прекрасный, который пел тогда, когда мне надо было идти ко двору и когда царь меня требовал, и будил меня от сна моего.
\vs Ahh 17:26
И вот урон, учиненный мне котом этим: он побывал ночью этой в Ассуре, в Ниневии, и откусил голову петуху тому, и вернулся сюда.
\vs Ahh 17:27
Тогда царь сказал мне: будучи стар, ты заблуждаешься. Между Ассуром и Мицрой триста парасангов: как же кот мог за эту ночь дойти туда, откусить голову этому петуху и вернуться обратно?
\vs Ahh 17:28
Я сказал ему: пусть между Ассуром и Мицрой триста парасангов, разве не слышали мы, что кобылицы ваши услышали ржание жеребца нашего и выкинули плод? Так и с котом.
\vs Ahh 17:29
При этих словах царь смутился и в изумлении сказал мне: о, Ахиахар, изъясни мне то, что я скажу тебе:
\vs Ahh 17:30
есть у меня столп великий, сложенный из восьми тысяч семисот шестидесяти трех кирпичей, на верху которого насаждено двенадцать кедров;
\vs Ahh 17:31
на верху каждого из этих кедров по тридцати колес, и по каждому колесу бегут две нити, одна белая, а другая черная.
\vs Ahh 17:32
Я ответил царю о предмете, о котором он спрашивал меня: умы баранов и быков знают то, что ты спрашиваешь у меня, царь.
\vs Ahh 17:33
Столп, о котором говорил владыка мой царь, это год;
\vs Ahh 17:34
столп этот сложен из восьми тысяч семисот шестидесяти трех кирпичей, каковы суть восемь тысяч семьсот шестьдесят три часа;
\vs Ahh 17:35
двенадцать кедров суть двенадцать месяцев года;
\vs Ahh 17:36
тридцать колес суть тридцать дней месяца;
\vs Ahh 17:37
две нити, одна черная и другая белая, это ночь и день.
\vs Ahh 17:38
Царь сказал мне еще: перестань.
\vs Ahh 17:39
Однако я требую от тебя, о Ахиахар, чтобы ты свил две длинные веревки из песка, по пятидесяти локтей в длину и по пальцу в ширину.
\vs Ahh 17:40
Я отвечал ему: прикажи, владыка мой царь, чтобы мне принесли такую веревку из твоей сокровищницы, чтобы мне свить подобную ей.
\vs Ahh 17:41
Он сказал мне: ты не понял слов моих: если не совьешь ты мне веревки, как я сказал тебе, не получишь ты подати Мицрейской.
\vs Ahh 17:42
И тогда я, Ахиахар, покинул царя и провел ночь ту в размышлении великом, и поутру пришел мне помысл некий.
\vs Ahh 17:43
И стал я позади дворца, в котором обитал царь, и сделал в стене напротив солнца дыру, и прошло солнце сквозь стену дворца.
\vs Ahh 17:44
И сделал я другую дыру в той же стене; после взял я пригоршню пыли и вложил в дыры, и пыль явилась в луче и была увлечена.
\vs Ahh 17:45
И заговорил я, и сказал я царю: повели, владыка мой царь, чтобы эти лучи связали в пучок, и я сделаю подобный пучок, если ты пожелаешь.
\vs Ahh 17:46
Увидев это, царь и вельможи его были объяты изумлением и недоумением и были весьма унижены.
\vs Ahh 17:47
Тогда царь велел принести мне верхний камень от разбитого жернова, а после заговорил и сказал:
\vs Ahh 17:48
Прошей мне этот камень, Ахиахар!
\vs Ahh 17:49
И я тотчас взял пест из того же камня, что и жернов, бросил его и сказал царю:
\vs Ahh 17:50
Владыка мой царь, у меня нет с собой шильев сапожника, и я не нахожу того, что мне потребно;
\vs Ahh 17:51
вели, однако, сапожникам твоим продеть нить в этот пест, который одного естества с жерновом, и я тотчас прошью его.
\vs Ahh 17:52
На эти слова царь засмеялся и сказал: добро же, Ахиахар! День, в который ты рожден, да будет благословен пред лицем богов Мицрейских!
\vs Ahh 17:53
Поелику я вижу тебя живым и здравствующим, я сотворю этот день великим празднеством и временем веселия.
\vs Ahh 17:54
Когда царь фараон был побит во всем, когда я оказал отпор его хитростям, когда я разрешил и упразднил все измышления его и все загадки его, он отдал мне подать с Мицры за три года,
\vs Ahh 17:55
и сверх того я получил те девятьсот талантов, о которых шла речь в изготовленном мною письме, как о ссуженных моим государем, и о которых все будто бы слышали, по собственному их признанию.
\vs Ahh 17:56
Я был осыпан дарами от царя и почестями от вельмож его.

\vs Ahh 18:1
И тотчас царь Ассур-Аддин поспешил мне навстречу. И начал царь говорить мне слова мудрые:
\vs Ahh 18:2
Проси и требуй от меня, чего хочешь.
\vs Ahh 18:3
И сказал я: о, владыка мой царь, вовеки живи!
\vs Ahh 18:4
И царь сошел ко мне навстречу и радовался радостию великою.
\vs Ahh 18:5
Он почтил меня, и посадил подле себя на престоле своем и на твердыне своей, и сказал мне:
\vs Ahh 18:6
Проси у меня, Ахиахар, всего, чего пожелаешь. Если пожелаешь, отдам тебе все царство мое. И сказал ему:
\vs Ahh 18:7
О, владыка мой царь, вовеки живи, в роды и роды! Все, чего прошу я у величия твоего, если нашел я благоволение в очах твоих, это дать хорошее место Навусемаку, копьеносцу, ибо ему обязан я тем, что доселе живу.
\vs Ahh 18:8
И выказал мне тогда царь приязнь свою милостями многими, особенно же дарами и подарками, которые принял я от руки его.
\vs Ahh 18:9
И осыпал меня царь дарами многими, и дарил подарки Навусемаку.
\vs Ahh 18:10
И стал царь расспрашивать меня обо всем, что было со мною пред лицом фараоновым, и о загадках фараоновых;
\vs Ahh 18:11
и рассказывал я ему все от начала и до конца, по порядку и по отдельности; он же, слушая, дивился.
\vs Ahh 18:12
И затем вынул я сокровища, и сребро, и золото, и дары, и подарки, что дал мне царь Мицрейский, чтобы доставил я их ему из Мицры; и радовался он радостию несказанною.
\vs Ahh 18:13
И сказал он мне: сколько желаешь ты получить от меня?
\vs Ahh 18:14
Я же сказал ему: я не прошу ничего, кроме как видеть тебя счастливым и благоденственным.
\vs Ahh 18:15
Что бы делал я с этими богатствами и с прочим? Однако прошу у блаженства твоего, чтобы ты дал мне власть делать все, что я пожелаю, Надаву, дабы отомстить ему, и чтобы не взыскивал ты с меня кровь его.
\vs Ahh 18:16
И тотчас дозволил мне царь делать все, что я пожелаю.
\vs Ahh 18:17
Я взял Надава и пошел в дом мой; и связал я его узами и цепями железными, и возложил я оковы железные на руки его и на ноги его и железо на плечи его,
\vs Ahh 18:18
а после стал я бичевать его розгами и бить его ударами лютыми и припоминал ему поучение, которое преподал ему в премудрости, и в знании, и в философии.

\vs Ahh 19:1
И сказал я: сын мой, того, кто не слушал ушами своими, понуждают слушать спиною его.
\vs Ahh 19:2
Надав, сын мой, заговорил и сказал:
\vs Ahh 19:3
Зачем гневаешься ты на сына твоего? Я отвечал ему:
\vs Ahh 19:4
Я, сын мой, посадил тебя на престоле славы, ты же сбросил меня с престола моего; и правда моя спасла меня.
\vs Ahh 19:5
Ты был для меня, сын мой, как скорпион, который ужалил скалу, и та сказала ему: ужалил ты сердце неуязвимое.
\vs Ahh 19:6
Он ужалил иглу, и та сказал ему: ужалил ты жало, которое сильнее, чем твое.
\vs Ahh 19:7
Ты был для меня, сын мой, как тот, кто бросает камень в небо; до неба он не дометнет, однако, бросив, согрешит.
\vs Ahh 19:8
Ты был для меня, сын мой, как тот, кто увидел ближнего своего дрожащим от стужи, и взял сосуд с водою, и метнул в него.
\vs Ahh 19:9
Сын мой, отвечай мне! Ты напал на меня, как голодный лев на осла, блуждавшего поутру.
\vs Ahh 19:10
Сказал лев ослу: подойди в мире, брат мой и друг мой!
\vs Ahh 19:11
Осел ответил: такого мира пожелаю тому, кто не привязал меня и не помешал мне выйти навстречу тебе.
\vs Ahh 19:12
Сын мой, ты был для меня как западня, укрытая под навозом. И пришел воробей, и увидела его западня, и сказала: брат мой, что делаешь ты здесь?
\vs Ahh 19:13
И ответил воробей: смотрю на тебя.
\vs Ahh 19:14
И сказала западня: помолись Богу, слава Ему!
\vs Ahh 19:15
И спросил воробей: что у тебя за палка?
\vs Ahh 19:16
Ответила западня: это посох мой и опора моя, я подпираюсь им, когда стою на молитве.
\vs Ahh 19:17
Спросил воробей: что за зерна во устах твоих?
\vs Ahh 19:18
Ответила западня: это пища, и это хлеб, восстанавливающий силы тех, кто мучим голодом.
\vs Ahh 19:19
Я поместила его во рту моем, чтобы он служил для пропитания голодных, ищущих у меня прибежища своего.
\vs Ahh 19:20
Воробей сказал: вот, я весьма изнурен голодом, и я прихожу, чтобы есть зерна.
\vs Ahh 19:21
Западня ответила ему и сказала: приблизься, о брат мой, и не страшись!
\vs Ahh 19:22
Когда же приблизился воробей, чтобы взять зерен, она тотчас схватила его за голову; и сказал воробей западне:
\vs Ahh 19:23
Если таков пост твой, и такова молитва твоя, и для такой цели зерна те, Бог не примет ни поста твоего, ни молитвы твоей и не подаст тебе никакого блага.
\vs Ahh 19:24
Сын мой, ты был для меня как жук-долгоносик, обретающийся в хлебах, что не годен ни на что доброе, но губит хлеба.
\vs Ahh 19:25
Сын мой, ты был для меня как котел, к которому приладили золотые ручки, не отчистив его дна от черноты.
\vs Ahh 19:26
Сын мой, ты был для меня как птица, которая замкнута в западне и не может убежать от ловца;
\vs Ahh 19:27
и тогда она поднимает голос приятный и сладостный и собирает вокруг себя многих птиц, малых или больших, дабы они также были уловлены.
\vs Ahh 19:28
Сын мой, ты был для меня как козел, который ведет товарищей своих на живодерню и не может спасти себя же самого.
\vs Ahh 19:29
Сын мой, ты был для меня как пес, которого проняла стужа и который пошел греться к гончарам, а когда согрелся, норовил облаять их и искусать.
\vs Ahh 19:30
Они пытались ударить его, он же залаял, а они, страшась быть искусанными, убили его.
\vs Ahh 19:31
Сын мой, ты был для меня как та свинья, которая пошла в баню вместе с вельможами;
\vs Ahh 19:32
и прошла она в баню, и омылась, а как вышла, увидала грязь и принялась в ней валяться.
\vs Ahh 19:33
Сын мой, рука, которая не трудится, и не утомляется, и не совершает работ, будет отсечена по причине лености своей.
\vs Ahh 19:34
Сын мой, это я показал тебе лице царево, и привел тебя к милостям великим, и научил тебя, и воспитал тебя, и доставил тебе всякое благо; и чем ты воздал мне, и чем отплатил мне?
\vs Ahh 19:35
Увы, и ах, и горе! Если бы ты ничего не получил от меня и ничего не принял от меня, ты не имел бы никакой власти надо мною во все дни жизни твоей.
\vs Ahh 19:36
Сын мой, сказало дерево дровосекам: если бы в руках ваших не было части меня, вы не напали бы на меня.
\vs Ahh 19:37
Ты был для меня как кот, которому сказали: перестань воровать, а тогда входи и выходи, как захочешь.
\vs Ahh 19:38
Он же ответил им: это естество мое, и если бы имел я глаза из серебра, руки из золота и ноги из берилла, я отнюдь не отстал бы от воровских дел моих.
\vs Ahh 19:39
Ты был для меня, сын мой, как змея, что забралась на ветку терновника и плыла по реке;
\vs Ahh 19:40
волк увидел ее и сказал: зло забралось на зло, и зло злейшее несет их.
\vs Ahh 19:41
Ответила змея волку тому: а ты отводишь ли коз к хозяину их?
\vs Ahh 19:42
Сын мой, я видел козу, которую пригнали на живодерню, но еще не пришло время ее, и потому она вернулась к себе, и увидала она детей своих и отпрысков детей своих.
\vs Ahh 19:43
Сын мой, я давал тебе вкушать от всякой снеди доброй, а ты не насытил меня и хлебом, смешанным с прахом;
\vs Ahh 19:44
я помазывал тебя благовониями усладительными, а ты осквернил тело мое прахом;
\vs Ahh 19:45
я упоевал тебя вином старым, а ты не напоил меня даже водою.
\vs Ahh 19:46
Сын мой, я возрастил тебя высоким, как кедр, а ты согнул меня при жизни моей и упоил меня лукавством.
\vs Ahh 19:47
Сын мой, я возвысил тебя как башню, и я говорил:
\vs Ahh 19:48
Когда враг мой придет на меня, я поднимусь и найду прибежище.
\vs Ahh 19:49
Ты же увидел врага моего и склонился к нему.
\vs Ahh 19:50
Ты был для меня, сын мой, как крот, который выходит на лице земли, чтобы обвинять Бога, не давшего ему зрения; и прилетает орел, и уносит его.

\vs Ahh 20:1
И ответил Надав, сын мой, и сказал мне:
\vs Ahh 20:2
Далече да будет от тебя, владыка мой, обычай немилосердных, но поступи со мною по милости твоей!
\vs Ahh 20:3
Даже когда человек погрешает против Бога, Бог отпускает ему грехи; так и ты ныне прости меня, и я буду печься обо всех твоих скотах или буду пасти овец твоих и свиней твоих, и меня будут называть злым, а тебя добрым.
\vs Ahh 20:4
Я ответил ему и сказал ему: сын мой, ты был для меня как пальма, которая обреталась вдали от дороги и не давала плодов; и пришел хозяин ее, и хотел удалить ее.
\vs Ahh 20:5
И сказала ему пальма эта: дай мне год, и я принесу плод сафлора.
\vs Ahh 20:6
И сказал ей хозяин ее: злосчастная, тебя недостало на то, чтобы принести твой собственный плод, как достанет тебя на то, чтобы принести чужой?
\vs Ahh 20:7
Сын мой, старость орла лучше, чем юность грифа.
\vs Ahh 20:8
Сын мой, если бы волку велели держаться вдали от овец, он ответит, что ему мил прах, ими подымаемый.
\vs Ahh 20:9
Сказали волку: учись: буква элэп, буква бит.
\vs Ahh 20:10
Он ответил: баранина, козлятина.
\vs Ahh 20:11
Сын мой, ты оправдал пословицу, которая гласит: кого ты породил, называй сыном твоим, а кого ты воспитал, называй рабом твоим.
\vs Ahh 20:13
Сын мой, больше всякого иного слова оправдал ты это: возьми сына сестры твоей на руки твои и разбей его о камень.
\vs Ahh 20:14
Бог, сохранивший мне жизнь, знающий все и воздающий каждому по делам его, сын мой, да судит и да рассудит между мною и тобою.
\vs Ahh 20:15
Ничего больше не скажу тебе. Бог да воздаст тебе по делам твоим.

\vs Ahh 21:1
И когда юный Надав услышал слово это, тело его тотчас раздулось и стало как мех и бурдюк полный, и внутренности его вышли из чресл его.
\vs Ahh 21:2
Злое его деяние воспламенило его, палило его; иссушало его, обессиливало его, губило его, умертвило его.
\vs Ahh 21:3
Конец его привел его к погибели, и ниспал он в геенну вместе с завистливыми и горделивыми,
\vs Ahh 21:4
как сказано: Сын выроет ров, и согрешит, и падет в яму, которую сам выкопал;
\vs Ahh 21:5
и еще: Кто творит лукавое, впадет в погибель;
\vs Ahh 21:6
и еще: Кто строит кову брату своему, сам падет в нее.
\vs Ahh 21:7
Здесь кончается повесть об Ахиахаре, мудреце и философе достославном, который разумел тайны и толковал загадки.

\bibbookdescr{Tad}{
  inline={Завещание Адама},
  toc={Завещание Адама},
  bookmark={Завещание Адама},
  header={Завещание Адама},
  abbr={Адам}
}

\chhdr{Дневные часы}

\vs Tad 1:1
И более, уразумей о часах дневных и ночных,
и как подобает тебе умолять Бога
и молиться ему в каждое из его времён года.
\vs Tad 1:2
Ибо мой Творец научил меня всему этому,
и он сказал мне имена всех диких животных и зверей,
и птиц небесных;
и потом Бог дал мне уразуметь число дневных и ночных часов,
и он сказал мне, как ангелы славят Бога.
\vs Tad 1:3
Пойми же, о сын мой, что в 1-ый час дня
молитва
моих сыновей\fnote{моих сыновей}{\vsep\ небожителей.}
восходит к Богу.
\vs Tad 1:4
И во 2-ой час происходит молитва и прошение ангелов.
\vs Tad 1:5
В 3-ий час птицы небесные восхваляют его.
\vs Tad 1:6
И в 4-ый час
духи\fnote{духи}{\vsep\ животные.}
поклоняются Ему.
\vs Tad 1:7
И в 5-ый час все
дикие звери и животные\fnote{дикие звери и животные}{\vsep\
живущие выше небес.}
приветствуют его.
\vs Tad 1:8
В 6-ой час происходит прошение Керубов,
которые свидетельствуют против
беззаконий нашей человеческой природы\fnote{которые \ldots\ природы}{\vsep\
--- .}.
\vs Tad 1:9
И в 7-ой час все ангелы предстают Богу,
и отходят от него, ибо в этот час молитва
всякого живого существа восходит к Богу.

\vs Tad 1:10
В 8-ой час сияние
небожителей\fnote{небожителей}{\vsep\ огня и вод.}
восхваляет его.
\vs Tad 1:11
И в 9-ый час ангелы Божии, стоящие пред престолом Всевышнего,
воздают ему почести.
\vs Tad 1:12
И в 10-ый час Святой Дух осеняет воды,
и демоны бегут и удаляются от вод.
И если бы Святой Дух не осенял воды в этот час ежедневно,
никто бы не мог пить воду,
ибо плоть разрушалась бы злыми демонами. 
И все, кто повстречаются демонам в этот час будут ранены.
\vs Tad 1:13
И в 11-ый час происходит прославление праведных.
\vs Tad 1:14
И в 12-ый час Бог Всевышний принимает молитвы
и прошения сынов человеческих.
 
\chhdr{Ночные часы}

\vs Tad 2:1
И в 1-ый час ночи демоны воздают благодарение и хвалу Богу Всевышнему,
и в это время не причиняют они зла и вреда никому из сынов человеческих,
доколе они не кончат своё воздаяние почести.
\vs Tad 2:2
И во 2-ой час ночи
рыбы\fnote{рыбы}{\vsep\ голуби.}
и всякая тварь, существующая в воде,
восхваляет Бога, и дикие звери и чудовища морские.
\vs Tad 2:3
И в 3-ий час огонь восхваляет его
(в этот час он находится в глубочайшей бездне),
и в этот час никто не может обратиться к Нему.
\vs Tad 2:4
И в 4-ый час Серафы восхваляют его так: Святый, Святый, Святый.
\vs Tad 2:5
И в 5-ый час воды, которые превыше небес,
восхваляют его.
Ныне издавна я садился и слушал, вместе с ангелами, и восхищался тому,
как в\acc{о}ды восклицают;
подобно шуму мощных
колёс\fnote{колёс}{\vsep\ крыльев.},
и они восклицали подобно морским водам голосом,
восхваляющим Бога.
\vs Tad 2:6
И в 6-ой час облака восхваляют Бога в страхе и трепете.
\vs Tad 2:7
И в 7-ой час земля утихала в безмолвии,
и всякая тварь на ней, и воды засыпали. 

\vs Tad 2:8
В 8-ой час земля порождает траву и зелень,
и даёт деревьям производить листья и плоды.
\vs Tad 2:9
И в 9-ый час ангелы исполняют своё
служение почитания Бога,
и молитва сынов человеческих приходит перед Бога Всевышнего.
\vs Tad 2:10
И в 10-ый час врата небесные открываются,
и Бог слышит молитвы сынов верующих и прошения,
которые они испрашивают у Бога, исполняются для них;
и при звуке крыльев Серафов в это время
петухи поют и восхваляют Бога.
\vs Tad 2:11
И в 11-ый час радость и ликование стоит по всей земле,
ибо солнце входит в рай,
и его свет восходит по всем концам мира
и освещает всякую тварь.
\vs Tad 2:12
И в 12-ый час прилично моим детям вставать
перед Богом и воздавать ему почести,
ибо в этот час в великом безмолвии
покоятся все небесные духи.
 

\chhdr{Адам предсказывает пришествие Христа}

\vs Tad 3:1
И сказал Адам Сифу, сыну своему:
\vs Tad 3:2
Ныне же познай всё это, и внемли моему слову,
и уразумей, что слово Бога Всевышнего снизойдет на землю,
именно так, как он сказал мне в то время,
когда он изгнал меня из рая.
\vs Tad 3:3
Ибо он сказал мне, что его слово в последние дни
станет мужем от девы, имя которой Мариамь,
и сокроется в ней, и облечётся плотью,
и родится как муж великой силы
и деятельного искусства и познания.
\vs Tad 3:4
Никто не познает его, кроме него самог\acc{о} и того,
кому он откроется явно.
\vs Tad 3:5
И Бог сказал, что он будет ходить среди людей на земле,
и возрастать по дням и годам,
и явно сотворит знамения и чудеса,
и будет ходить по воде как по суше,
и явно запретит морю и ветрам, и они подчинятся ему,
и когда он воскликнет морским волнам,
они скоро поспешат ответить ему.
\vs Tad 3:6
И он даст слепому зрение, и очистит прокажённых,
и откроет слух глухим, и немым~--- язык,
и поднимет расслабленных, и сделает хромых ходящими,
и обратит многих от заблуждения к познанию Бога,
и изгонит демонов из людей.
\vs Tad 3:7
И кроме того Бог сказал мне, говоря:
Не печалься, Адам, ибо ты пожелал стать богом
и нарушил мою заповедь.
\vs Tad 3:8
Вот, я утвержу тебя, но не теперь,
но спустя много дней.
\vs Tad 3:9
И опять он сказал мне, говоря:
я Бог, изгнавший тебя из рая удовольствий на землю,
которая будет порождать тернии и шипы,
и ты будешь жить на ней.
\vs Tad 3:10
Согни спину, и пусть твои колени дрожат в старости,
и я дам твою плоть в пищу червям.
\vs Tad 3:11
И через 5 дней и половину дня я сжалюсь над тобою
и явлю тебе милость в изобилии моего сострадания и милости.
\vs Tad 3:12
И я приду в твой дом, и я буду жить в твоей плоти,
и ради тебя мне угодно будет родиться как младенцу.
\vs Tad 3:13
И ради тебя мне угодно будет ходить по торжищам.
\vs Tad 3:14
И ради тебя мне угодно будет поститься 40 дней.
\vs Tad 3:15
И ради тебя мне угодно будет принять омовение.
\vs Tad 3:16
И ради тебя мне угодно будет претерпеть страдание.
\vs Tad 3:17
И ради тебя мне угодно будет быть повешенным на крёстном древе.
\vs Tad 3:18
Всё это~--- ради тебя, о Адам.

\vs Tad 3:19
И по прошествии 3-ёх дней в могиле я воскрешу тело, которое я принял
от тебя.
\vs Tad 3:20
И поставлю тебя по правую руку мою и сделаю тебя богом, чего ты и хотел.
\vs Tad 3:21
И праведность небесная восстановится.

\vs Tad 3:22
И тогда, я, Сиф, спросил у отца: Как называется плод, который ты съел?
\vs Tad 3:23
И он ответил мне: Инжир, сын мой, стал вратами через которые зашла смерть
ко мне и к моему роду.
\vs Tad 3:24
И также жизнь войдёт в меня и моих сыновей когда Господь наш станет
человеком через деву и наденет тело святое в конце века.

\vs Tad 3:25
Кроме того, ты должен знать, о Сиф, сын мой,
вот придёт Потоп и омоет всю землю,
из-за сынов убийцы Каина, убившего своего брата из зависти,
из-за его сестры Луд.
\vs Tad 3:26
И после Потопа 6000 лет будет предоставлено миру,
а потом придут последние дни,
и всё будет исполнено, и придёт его время,
и огонь потребит всё, что найдено пред Богом,
и земля освятится,
и Господь господствующих будет ходить по ней.

\vs Tad 3:27
И Сиф записал эту заповедь,
и запечатал её своею печатью
и печатью своего отца Адама,
которую он унёс с собою из рая,
и печатью своей матери Евы.

\bibbookdescr{Tmo}{
  inline={Завещание Моисея},
  toc={Завещание Моисея},
  bookmark={Завещание Моисея},
  header={Завещание Моисея},
  abbr={Зав~Мо}
}
\vs Tmo 1:1
Завещание Моисея, данное им в сто двадцатый год жизни его, который есть четырехсотый по отправлении из Ханаана, когда вышел народ с Моисеем и дошел до Бен-Амми за Иорданом по пророчеству Моисееву.
\vs Tmo 1:2
Призвал Моисей к себе Иесуа, сына Нуна, человека, угодного Яхве, дабы стал он преемником народа и ковчега Завета со всеми святынями его и дабы ввел народ в землю, данную коленам его, дать им ее по завету и по клятве, которую произнес он в скинии, что даст ее через Иесуа;
\vs Tmo 1:3
И сказал Моисей к Иесуа такое слово: "Обещай, что все сотворишь, что поручено тебе, сотворишь со старанием, в точности и без ропота, ибо так говорит Владыка мира.
\vs Tmo 1:4
Создал Он мир ради народа Своего и не сделал начала творения ясно видимым от начала мира, дабы обличились тем народы и низкими речами своими обличили себя.
\vs Tmo 1:5
Так Он измыслил и изобрел меня, от начала мира готового стать судьею завета Его.
\vs Tmo 1:6
И ныне открою тебе, что совершилось время лет жизни моей и отхожу я в успение отцов моих.
\vs Tmo 1:7
Предо всем народом прими писание сие, дабы не забывал ты хранить книги, кои передам тебе,
\vs Tmo 1:8
Ты же их расположишь в порядке и запечатаешь и положишь в сосудах глиняных в месте, созданном от начала мира, дабы призывалось имя Его вплоть до дня покаяния с почитанием, коим почтил их Яхве на исходе дней.

\vs Tmo 2:1
Войдут они с тобою в землю, которую назначил и обещал Он дать отцам их.
\vs Tmo 2:2
В ней благословишь ты и дашь каждому и установишь им жребий мой и утвердишь им царство и управление на местах определишь им по тому, как угодно будет Яхве, Богу их, по суду и справедливости.
\vs Tmo 2:3
После того как войдут в землю свою, пройдет пять лет, и будет власть вождей и тираннов восемнадцать лет, и на девятнадцать лет отделятся десять колен, ибо отойдут два колена и перенесут ковчег Завета.
\vs Tmo 2:4
Тогда Бог небесный явит ковчег Свой и башню святилища Своего. И утвердятся два колена святости, десять же колен установят себе по законам своим царства.
\vs Tmo 2:5
И будут приносить жертвы двадцать лет: за семь лет соорудят стены, и ограждать их буду девять лет, и нападать будут на завет Яхве четыре года, и, наконец, осквернят договор, который сотворил с ними Яхве.
\vs Tmo 2:6
И принесут детей своих в жертву чужеземным богам, и установят в скинии идолов, служа им, и в доме Яхве будут вершить преступления, и многих идолов всех животных сделают.

\vs Tmo 3:1
В те времена придет к ним с востока царь, и покроет конница землю их, и сожжет он огнем поселения их со святым храмом Яхве, и все святые сосуды он истребит, и весь народ изгонит, и уведет их в землю отчизны своей, и два колена уведет с собой.
\vs Tmo 3:2
Тогда воззовут два колена к десяти коленам, и лягут словно львица, покрытые пылью в полях, алчущие и жаждущие с детьми нашими, и возопиют: "Праведен и свят Яхве! Отчего вы грешили, а мы так же уведены с вами?"
\vs Tmo 3:3
Тогда восплачут десять колен, слыша слова упрека от двух колен и скажут: "Что сделали мы вам, братья? Не во весь ли дом Израилев вошло горе это?"
\vs Tmo 3:4
И все колена восплачут, вопия к небу и говоря: "Бог Авраама, Бог Исаака, Бог Иакова, воспомни завет Твой, который заключил Ты с ними, и клятву, которою клялся Ты им, что никогда не упразднится семя их от земли, которую Ты дал им".
\vs Tmo 3:5
И в тот день воспомнят имя мое, говоря колено к колену, и всякий человек к ближнему своему: "Не то ли это, в чем удостоверял нас Моисей в пророчествах своих, он, претерпевший многое в Египте и в море Суф, и в пустыне в продолжение сорока лет, свидетельствуя и призывая в свидетели небо и землю, да не преступим мы заповедей Его, в коих был он нам судьею;
\vs Tmo 3:6
И так случилось с нами по словам Его, и по уверению Его, как свидетельствовал он нам в те времена, и вышло, что ведут нас, плененных, в Восточную землю, где и будем мы рабами около семидесяти семи лет".

\vs Tmo 4:1
Тогда войдет один, стоящий над ними, и прострет руки, и преклонит колени свои, и станет молиться за них, говоря:
\vs Tmo 4:2
"Яхве, Царь всех, на высоком престоле властвующий над миром, возжелавший, дабы сей народ был народом Твоим избранным, тогда хотел Ты называться их Богом по завету, который заключил Ты с отцами их.
\vs Tmo 4:3
И пошли они, плененные, в землю чуждую с женами и детьми своими, и пребывают у врат иноплеменных и там. Где же величие великое? Призри и смилуйся над ними, Господь небесный!"
\vs Tmo 4:4
Тогда воспомнит о них Бог по завету, что сотворил Он с отцами их, и явит Он милосердие Свое, и вложит в те времена в душу царя, дабы смиловался над ними, и отпустит их царь в землю и область их.
\vs Tmo 4:5
Тогда поднимутся некоторые части колен и пойдут в свое место установленное и обновят укрепления его.
\vs Tmo 4:6
Два же колена пребудут в вере своей, печальные и плачущие, ибо не смогут принести жертв Яхве, Богу отцов своих, десять же колен увеличатся и умножатся среди племен во времена пленения их.

\vs Tmo 5:1
Когда же приблизятся времена обличения, мщение наступит от царей, соучастников преступлений, кои разделятся воистину.
\vs Tmo 5:2
Потому и было сказано: "Уклонятся от праведности, и перейдут к неправедности, и осквернят нечестиями дом служения своего, и осквернятся служением чужим богам.
\vs Tmo 5:3
И не последуют истине Божией, но осквернят алтарь дарами, кои воздадут Яхве не жрецы, но рожденные рабами от рабов.
\vs Tmo 5:4
Ибо те, которые суть ученые учителя их, будут взирать в те времена на лица страстей, и, принимая дары, продадут праведность в наказаниях.
\vs Tmo 5:5
И настолько наполнится население и предел обитания их преступлениями и обидами Бога, что те, кто творил беззаконие пред лицем Яхве, судьями станут и судить будут, кто как пожелает".

\vs Tmo 6:1
Тогда возстанут у них цари властные. Назовутся они священниками великого Бога и удалят творящих нечестие от святая святых.
\vs Tmo 6:2
И придет вслед за ними царь дерзновенный, который не будет из рода священнического. Сей человек безрассудный и злой, и будет судить он их, как они того достойны.
\vs Tmo 6:3
Истребит он вождей их мечом, и в неизвестные места порознь положит тела их, дабы не ведал никто, где тела их. Погубит он старших возрастом и юношей не пощадит.
\vs Tmo 6:4
Тогда страх пред ним будет великий в земле их, и станет он вершить суд над ними, как вершили его Египтяне, тридцать четыре года, и покарает их.
\vs Tmo 6:5
И породит он сыновей, кои будут царствовать не столь долго, и придут в землю их когорты мощного царя Западного, и одолеет он их и уведет в плен и часть храма их огнем сожжет, некоторых же распнет вокруг поселения их.

\vs Tmo 7:1
После того совершатся времена и будут править ими люди погибельные и нечестивые, называющие себя праведными,
\vs Tmo 7:2
И возбудят они гнев душ своих, будучи людьми коварными, себе угождающими, лживыми во всех делах своих и во всякий час дня, любящими пиры, чревоугодниками, пожиратели имущества бедных,
\vs Tmo 7:3
Скажут, что творили это по милосердию и истребляя стяжателей.
\vs Tmo 7:4
Будут обманывать, скрываясь, дабы не уличили их, нечестивцев, в преступлении, исполненные неправедности, от восхода до заката говоря: "Будут у нас роскошные ложа, и станем есть и пить на них. И помыслили мы, что будем, словно князья".
\vs Tmo 7:5
И руки их, и умы творят нечистое, и уста их полны слов надутых. И скажут они: "Не касайся, да не осквернишь места моего"

\vs Tmo 8:1
Придет к ним мщение и гнев, какого не бывало у них от века до того времени.
\vs Tmo 8:2
Тогда возставит им Яхве царя из царей земли и мощь из мощи великой, что распнет на кресте исповедующих обрезание.
\vs Tmo 8:3
И предаст он пыткам тех, кто откажется, и повелит в оковах отвести в темницу. А жен их отдадут богам языческим; сыновья же их, мальчики, обрезанные врачами, принуждены будут принять необрезание.
\vs Tmo 8:4
Будут карать их пытками, огнем и железом, заставят их носить идолов своих оскверненных, как и те, кто им служит, и заставят их мучители войти в тайное место свое, и понудят их стрекалами произнести слова хульные, а потом и законы похулить, положенные на алтаре их.

\vs Tmo 9:1
Тогда возстанет муж из колена Левиева, имя коему будет Таксо. Имея семерых сыновей, обратится к ним с просьбою:
\vs Tmo 9:2
"Смотрите, сыны, вот, свершилось второе отмщение народу жестокое и нечестивое, и пленение безжалостное, и превосходят они бывшие доселе. Какое племя, какая земля, какой народ, нечестивых, творивших преступление в доме своем, столько бед претерпел, сколько нас обступило?
\vs Tmo 9:3
Ныне, послушайте меня, дети, ибо видите и знаете, что никогда не испытывали Бога ни родители наши, ни праотцы, преступая заповеди Его.
\vs Tmo 9:4
Ибо знаете: в этом сила наша. Сделаем так: будем поститься три дня, а на четвертый день войдем в пещеру, которая на поле, и лучше умрем, чем преступим заповеди Бога богов, Господа отцов наших.
\vs Tmo 9:5
Если так сотворим и умрем, кровь наша отомщена будет пред Яхве".

\vs Tmo 10:1
И тогда явится царствие Его во всяком творении Его. И тогда диавол обретет конец, и скорбь с ним отойдет.
\vs Tmo 10:2
Тогда наполнится рука ангела, утвержденного на небесах, и тотчас избавит он их от врагов их.
\vs Tmo 10:3
Ибо поднимется Небесный с престола царствия Своего и выйдет из святого жилища Своего с негодованием и гневом на сынов Своих.
\vs Tmo 10:4
И задрожит земля и до пределов своих сотрясется, и высокие горы понизятся и сотрясутся, и долины падут, солнце не даст света, и во мрак обратятся рога луны и сокрушатся, и все обратится в кровь, и круг звезд смешается, и море отступит до бездны, и источники вод изсякнут, и реки высохнут.
\vs Tmo 10:5
Ибо возстанет Великий Бог, Единый и Вечный, и явится всем и отомстит народам и уничтожит всех идолов их.
\vs Tmo 10:6
Тогда блажен будешь ты, Израиль, и поднимешься ты на головы и на крылья орлиные, и наполнятся они воздухом, и возвысит тебя Бог и утвердит тебя в небе звездном в месте пребывания звезд.
\vs Tmo 10:7
И воззришь с высоты, и увидишь врагов своих на земле, и узнаешь их, и возрадуешься, и возблагодаришь, и хвалу вознесешь Создателю твоему.
\vs Tmo 10:8
Ты же, Иесуа Нун, сбереги слова сии и книгу сию. Ибо от того дня, когда приму я смерть, пройдет до пришествия Его двести пятьдесят времен. Столько времени пройдет, пока не прейдут времена.
\vs Tmo 10:9
Я же отхожу к успению отцов моих. Итак, ты, Иесуа Нун, мужайся, тебя избрал Бог быть мне преемником в завете сем.

\vs Tmo 11:1
Когда услышал Иесуа слова Моисея, записанные в писании его, и все, что предрек он, разодрал одежды свои, пал к ногам его, и утешал его Моисей и плакал с ним.
\vs Tmo 11:2
И отвечал ему Иесуа и сказал: Утешишь ты меня, господин мой Моисей, и как утешить меня в том, что сказано голосом горьким, что вышел из уст твоих, и полон слез и рыданий, ибо уходишь ты от народа Израилева.
\vs Tmo 11:3
Какое место примет тебя, каков будет памятник могильный, кто осмелится перенести тело твое из одного места в другое?
\vs Tmo 11:4
Ибо у всех, кто умирает в свое время, есть могилы свои на земле, твоя же могила от восхода солнца до заката, и от юга до севера весь мир есть могила твоя, господин мой.
\vs Tmo 11:5
Уходишь ты, и кто будет питать народ сей, и кто сжалится над ними, и кто вождем будет им в пути, и кто молиться станет за них? Не смогу я и одного дня вести их в земле предков.
\vs Tmo 11:6
Как же буду я народу сему словно отец для единого сына или мать для дочери-девицы, что готовит ее для славного мужа, оберегает, боясь, тело ее от солнца и старается, дабы не поранила та ног своих, бегая по земле?
\vs Tmo 11:7
Как дам им пищу по желанию их и насыщу их? Ведь их шестьсот тысяч возросло их число молитвами твоими, господин мой Моисей.
\vs Tmo 11:8
Какая же у меня мудрость и какое разумение в доме Божием словами судить и давать ответы?
\vs Tmo 11:9
Но и цари Аморрейские, когда услышат об этом, помыслят, что одолеют нас,
\vs Tmo 11:10
Ибо нет больше с нами Духа Святого, достойного Яхве, слову многоликого и непонятного Яхве верного во всем, божественного пророка всего мира, ведь умер он и нет более в веке сем учителя, и скажут они тогда:
\vs Tmo 11:11
"Пойдем на них, если нечестивое совершили они единожды Господу Своему, нет у них заступника, который бы вознес молитвы Яхве, каков был Моисей, великий ангел.
\vs Tmo 11:12
Он по целым часам стоял днем и ночью коленами своими на земле, молясь и взирая на мир и всех людей с милосердием и праведностью, памятуя о завете предков своих и клятвами умилостивляя Яхве".
\vs Tmo 11:13
Итак, скажут они: "Не с ними Бог. Пойдем же и сотрем их с лица земли". Вот как будет с народом сим, господин мой Моисей".

\vs Tmo 12:1
И, закончив слова свои, вновь пал Иесуа к стопам Моисеевым. И взял Моисей его за руку и посадил пред собою на седалище.
\vs Tmo 12:2
И отвечал и сказал ему Моисей: Иесуа, не бойся за себя, но будь уверен и внемли словам моим.
\vs Tmo 12:3
Все народы, какие есть в мире, создал Бог, и предусмотрел Он о них и о нас от начала творения всего мира.
\vs Tmo 12:4
И до скончания века нет ничего, чего бы не усмотрел Он до самой малой вещи, но все Он предусмотрел и устроил.
\vs Tmo 12:5
Все в этом мире предусмотрел Он, и вот \ldots
\vs Tmo 12:6
Меня поставил Он молиться за них и за грехи их и заступником быть им не по добродетели моей и не по немощи, но в меру милосердия Его и терпения Его.
\vs Tmo 12:7
Говорю же тебе, Иесуа: не по благочестию народа сего истребишь ты язычников; все тверди небесные Богом созданы и одобрены, и лишь под Его десницею они.
\vs Tmo 12:8
Итак, творящие и совершающие заповеди Божии возрастают и по доброму пути продвигаются, а согрешающим и пренебрегающим, нет им добрых заповедей в том, что предсказано.
\vs Tmo 12:9
И будут они покараны язычниками и преданы многим пыткам. Но не может быть того, чтобы совершенно уничтожил их Он и оставил.
\vs Tmo 12:10
Ибо выйдет Бог, провидящий все вовеки, и тверд Завет Его и клятва в том, что \ldots

\bibbookdescr{Trb}{
  inline={Завещание Рувима,\\первородного сына Иакова и Лии\fns{В греч. тексте $+$ ``о мыслях''.}},
  toc={Завещание Рувима},
  bookmark={Завещание Рувима},
  header={Завещание Рувима},
  abbr={Рув}
}
\vs Trb 1:1
Список завета Рувима, которое он завещал сыновьям своим,
прежде чем умереть, в 125-ый год жизни своей.
\vs Trb 1:2
Два года спустя после кончины Иосифа, брата его, занемог Рувим,
и собрались проведать его дети и дети детей его.
\vs Trb 1:3
И сказал он им: дети мои, я умираю и отправляюсь дорогою отцов моих.
\vs Trb 1:4
Увидев же там Иуду, Гада и Асира,
братьев своих, сказал им:
поднимите меня, братья, дабы говорить мне к братьям моим
и детям моим о том, что сокрыто в сердце у меня.
Ибо я отхожу от вас отныне.
\vs Trb 1:5
И поднявшись, поцеловал их и, рыдая, сказал им:
слушайте, братья мои и дети мои, внемлите Рувиму,
отцу вашему, что завещаю вам.

\vs Trb 1:6
Вот, я заклинаю вас Богом небесным, да не совершите проступка
по незнанию юности, как я предался пороку и осквернил
ложе отца моего Иакова.
\vs Trb 1:7
Говорю же вам, что заполнила болезнь великая
поясницу мою на 7 месяцев, и, если бы не просил отец
наш Иаков обо мне Господа, желал убить меня Господь.
\vs Trb 1:8
Было мне 30 лет, когда совершил я злое пред лицом Господа,
и слабость смертная охватила меня на 7 месяцев.
\vs Trb 1:9
После же этого каялся я пред лицом Господа 7 лет,
ибо возжелала того душа моя.
\vs Trb 1:10
И не пил я вина и сикера, и мясо не входило в уста мои,
и никакого хлеба вожделенного не пробовал я,
но пребывал в печали о согрешении моём, ибо оно было велико,
и не было подобного ему в Израиле.

\vs Trb 2:1
А теперь, выслушайте меня, дети мои, что увидел я в раскаянии моём о
7-и духах соблазна.
\vs Trb 2:2
Ибо 7 духов поставлены против человека Велиаром, и они суть источники дел юношеских.
\vs Trb 2:3
И \bibemph{иные} 7 духов даны ему по сотворении его,
дабы в них было всякое дело человеческое.
\vs Trb 2:4
Первый~--- дух жизни, на коем зиждется его существование.
Второй~--- дух зрения, из коего происходит желание.
\vs Trb 2:5
Третий~--- дух слуха, из коего происходит научение.
Четвёртый~--- дух обоняния, из коего происходит вкус
от втягивания воздуха и вдыхания.
\vs Trb 2:6
Пятый~--- дух речи, из коей происходит знание.
\vs Trb 2:7
Шестой~--- дух вкуса, из коего происходит вкушение пищи и питья,
и сила на нём зиждется, ибо в пище~---  основание силы.
\vs Trb 2:8
Седьмой~--- дух деторождения и плотского сообщения,
из коего от любви к наслаждениям происходит собрание грехов.
\vs Trb 2:9
Оттого этот дух~--- последний в сотворении и первый в юности,
ибо исполнен неразумия, он же ведет юношу, словно слепого в яму
и словно стадо к пропасти.

\vs Trb 3:1
Ко всем же этим есть ещё восьмой дух~--- дух сна,
на коем основан экстаз естества и образ смерти.
\vs Trb 3:2
С этими духами соединяется дух обмана.]
\vs Trb 3:3
Первый~--- дух блуда, содержится он в естестве и чувствах.
Второй~--- дух ненасытности желудка.
\vs Trb 3:4
Третий~--- дух борьбы, что в печени и в желчи.
Четвёртый~--- дух угождения и магии, дабы казаться прекрасным,
в чём нет никакой пользы.
\vs Trb 3:5
Пятый~--- дух гордыни, дабы похваляться и кичиться.
Шестой~--- дух лжи губительной и пристрастной,
дабы измышлять речи и скрывать дела даже от родичей и ближних.
\vs Trb 3:6
Седьмой~--- дух несправедливости, от коего воровство и грабежи
для услаждения сердца своего.
Ибо несправедливость содействует прочим духам,
когда отнимается нечто у других людей.
\vs Trb 3:7
[Со всеми же этими соседствует дух сна, восьмой дух,
он же~--- дух обмана и фантазии.]
\vs Trb 3:8
И так гибнет всякий юноша, затемняющий ум свой от истины,
не входящий в закон Бога, не слушающий наставлений отцов своих,
каков и я был в юности моей.
\vs Trb 3:9
Ныне, дети мои, возлюбите истину, и она убережет вас.
Я учу вас, внемлите словам Рувима, отца вашего.
\vs Trb 3:10
Не взирайте на женщин, не сходитесь с женщиной иного мужа,
не имейте дел ненужных с женщинами.
\vs Trb 3:11
Ибо, если бы не увидел я Баллу, когда купалась она в скрытом месте,
не впал бы я в беззаконие великое.
\vs Trb 3:12
Захватила меня мысль о наготе женской и не давала мне уснуть,
пока не совершил я мерзость.
\vs Trb 3:13
Когда Иаков, отец мой, ушёл к Исааку~--- а были мы в Гадере
близ Ефрафы в Вифлееме опьянела Балла и лежала непокрытая
в спальне.
\vs Trb 3:14
И я, вошедши и увидевши наготу её, сотворил нечестивое,
[а она не чувствовала,] и я отошёл, оставив её спящей.
\vs Trb 3:15
И тотчас ангел Божий открыл нечестивое дело моё отцу моему.
Придя, сетовал он на меня, более не прикасаясь к ней.

\vs Trb 4:1
Так не смотрите же, дети мои, на красоту женскую
и не помышляйте о делах женщин,
но живите в простоте сердечной, в страхе Господнем,
трудитесь, творя добрые дела, и отвлекаясь грамматикой,
и на пастбищах ваших дотоле, пока не даст вам Господь супругу,
какую он пожелает, дабы не претерпеть вам того же, что мне.
\vs Trb 4:2
Ибо вплоть до кончины отца моего не хватало смелости мне посмотреть
в глаза ему или говорить с кем-либо из братьев моих, из-за укоризны.
\vs Trb 4:3
И доныне мучит меня совесть из-за греха моего.
\vs Trb 4:4
И много утешал меня отец мой, и просил за меня Господа,
да отойдёт от меня гнев его, и так поступил со мной Господь.
С тех пор и поныне остерегался я и не грешил.
\vs Trb 4:5
Потому говорю вам, дети мои, сохраните всё,
что внушаю вам, и не грешите.
\vs Trb 4:6
Ибо грех блуда есть пропасть душевная, отделяющая от Бога
и приближающая к идолам, ведь он помрачает ум и помыслы
и сводит юношей в Ад прежде времени.
\vs Trb 4:7
Блуд сгубил многих, ибо стар ли кто, знатен ли, богат или беден,
одинаково порицание обретает он у сынов человеческих и даёт повод
Велиару создать преткновение ему.

\vs Trb 4:8
Слышали же вы об Иосифе, как остерегался он всякой женщины
и хранил помыслы в чистоте ото всякого блуда и обрёл
благодать у Господа и у человеков.
\vs Trb 4:9
А ведь многое творила ему Египтянка, и колдунов призывая
и снадобье ему поднося, но не впал помысел души его
в вожделение злое.
\vs Trb 4:10
За это избавил его Бог отцов наших от всякого зла видимой
и таящейся смерти.
\vs Trb 4:11
Если же не овладеет блуд помыслами вашими, не сможет одолеть вас и Велиар.

\vs Trb 5:1
Злы женщины, дети мои, и, не имея власти и силы над мужами,
коварно действуют своими чарами, дабы привлечь их к себе.
\vs Trb 5:2
Кого же такими чарами не могут приворожить, того обманом покоряют.
\vs Trb 5:3
Говорил же мне ангел Божий и учил меня, что уступают женщины тому духу
блуда больше, нежели мужи.
И замышляют они в сердце своём против мужей,
и украшениями соблазняют помыслы их,
и через очи подсыпают им яд, и так порабощают их.
\vs Trb 5:4
Ибо не может женщина прямо принудить мужа,
но совершает это злодейство своими чарами блудными.
\vs Trb 5:5
Итак, убегайте блуда, дети мои,
и приказывайте женам вашим и дочерям вашим,
да не украшают голов и лиц своих для обмана здравых помыслов мужчин.
Ибо всякая женщина, прибегающая к этим ухищрениям,
обречена на муку вечную.
\vs Trb 5:6
Ибо так обольстили они Стражей, бывших до
Катаклизма\fnote{Катаклизма}{Потопа или расстворения предыдущего мира.}.
Те постоянно смотрели на них, и возжелали их,
и замыслили дело: приняв человеческое обличье, сошлись с
женщинами в образе мужей их.
\vs Trb 5:7
А те, вообразив в вожделении своем, породили Гигантов,
ведь показались им Стражи достигающими небес.

\vs Trb 6:1
Так остерегайтесь же блуда.
И если желаете очистить разум, то сдерживайте чувства
свои от женщин.
\vs Trb 6:2
А женщинам внушайте не иметь общения с мужами,
дабы и они очищали \bibemph{свой} разум.
\vs Trb 6:3
Ибо постоянное общение,
если и не совершится нечестивое,
для них есть болезнь неисцелимая,
для нас же погибель Велиарова и позор вечный.
\vs Trb 6:4
Нет ни совести, ни благочестия в блуде,
и всякая ревность живет в вожделении его.
\vs Trb 6:5
Потому и говорю вам:
будете вы ревновать и стремиться превзойти сыновей Левия,
но не сможете.
\vs Trb 6:6
Потому что Бог отомстит за них, вы же умрёте смертью злою.
\vs Trb 6:7
Ибо Левию дал Бог власть
[и с ним Иуде, и мне, и Дану, и Иосифу, дабы мы были вождями].
\vs Trb 6:8
Посему завещаю вам слушать Левия, ибо он познает закон Господа
и установит суд и будет приносить жертвы за Израиля вплоть
до конца времен~--- первосвященник помазанный,
коего призвал Господь.
\vs Trb 6:9
Хочу, чтобы поклялись вы Богом небесным, что будете творить правду,
каждый ближнему своему, и иметь любовь, каждый к брату своему.
\vs Trb 6:10
А к Левию подойдите в смирении сердца вашего,
да примете благословение из уст его.
\vs Trb 6:11
Ибо он благословит Израиля и Иуду,
ибо в нём избрал Господь царствовать над всеми народами.
\vs Trb 6:12
И поклонитесь семени его, ибо за вас будет умирать оно в
войнах зримых и незримых.
И будет он над вaми царём вечным.

\vs Trb 7:1
И умер Рувим, завещав это сыновьям своим.
\vs Trb 7:2
И положили его во гроб, а после вынесли его из Египта
и погребли в Хевроне в пещере, где погребён был и отец его.

\bibbookdescr{Tsm}{
  inline={Завещание Симеона,\\второго сына Иакова и Лии\fns{В греч. тексте $+$ ``о зависти''.}},
  toc={Завещание Симеона},
  bookmark={Завещание Симеона},
  header={Завещание Симеона},
  abbr={Сим}
}
\vs Tsm 1:1
Список слов Симеона, речённых им к сыновьям его перед тем,
как умер он в 120-ый год жизни своей,
в тот же год, что и брат его Иосиф.
\vs Tsm 1:2
Когда занемог Симеон, пришли проведать его дети его, и, сделав
усилие, сел он, поцеловал их и сказал:
\vs Tsm 2:1
послушайте, дети мои, Симеона, отца вашего; возвещу вам то,
что имею я в сердце моём.

\vs Tsm 2:2
Родился я от Иакова и был вторым сыном отца моего, и Лия, мать моя,
нарекла меня Симеоном, ибо услышал Господь мольбу ее.
\vs Tsm 2:3
Сделался я весьма сильным, не боялся труда и не страшился никакого дела.
\vs Tsm 2:4
Ибо сердце моё было сухим, печень моя недвижимой,
а внутренности мои нечувствительными.
\vs Tsm 2:5
Ведь и мужество даётся от Всевышнего людям в душах и телах.
\vs Tsm 2:6
Во время юности моей завидовал я сильно Иосифу,
ибо возлюбил его отец мой более всех.
\vs Tsm 2:7
И утвердился я против него в сердце моём, возжелав убить его,
так как Князь обмана и дух зависти ослепили мне ум, и забыл я,
что это брат мой, и не пощадил отца моего Иакова.
\vs Tsm 2:8
Но Бог его и Бог отцов наших послал ангела своего и избавил
Иосифа от рук моих.

\vs Tsm 2:9
Ибо, когда я отправился в Сиким, чтобы принести притирание для стада,
а Рувим  в Дофаим, где было необходимое нам и все хранилища наши,
Иуда, брат мой, продал Иосифа Измаильтянам.
\vs Tsm 2:10
Рувим, услышав об этом, опечалился, ибо он хотел отвести его к отцу.
\vs Tsm 2:11
Я же, услышав это, сильно разгневался на Иуду,
ибо он отпустил Иосифа живым,
и 5 месяцев пребывал я в гневе на него.
\vs Tsm 2:12
И сковал меня Господь и удалил от меня дело рук моих,
ибо правая рука моя стала наполовину сухой на 7 дней.
\vs Tsm 2:13
И познал я, дети, что из-за Иосифа случилось это со мною.
И, раскаявшись, заплакал я и молил Господа Бога,
чтобы восстановилась рука моя и удержался я от всякой скверны
и зависти и ото всякого безрассудства.
\vs Tsm 2:14
Ибо понял я, что злое дело замыслил перед лицом Господа и Иакова,
отца моего, против Иосифа, брата моего, позавидовав ему.

\vs Tsm 3:1
Ныне, дети мои, послушайте меня и остерегитесь духа обмана и зависти.
\vs Tsm 3:2
Ведь зависть властвует надо всем помыслом человека
и не дает ему ни есть, ни пить, ни делать ничего доброго.
\vs Tsm 3:3
Но всечасно подстрекает она убить того, кому человек завидует,
но тот всечасно процветает, а завистник чахнет.
\vs Tsm 3:4
И вот, 2 года сокрушал я в страхе Господнем душу мою постом.
И узнал я, что избавление от зависти происходит через страх Божий.
\vs Tsm 3:5
Если кто прибегает к Господу, оставляет его злой дух
и становится разум лёгким.
\vs Tsm 3:6
И наконец, начинает он сочувствовать тому, кому завидовал, и
примиряется с любящими его, и так избавляется от зависти.

\vs Tsm 4:1
Спросил отец мой, что со мною, ибо заметил меня скорбящим, и
сказал я ему, что переполняется печень моя.
\vs Tsm 4:2
Ибо печалился я чрезвычайно, что виновен в продаже Иосифа.
\vs Tsm 4:3
И когда пошли мы в Египет и связали меня как соглядатая,
познал я, что справедливо страдаю и не опечалился.
\vs Tsm 4:4
Иосиф же был добрый муж, дух Божий в себе имевший,
милостивый и сострадательный; не вспомнил мне зла, но
возлюбил меня с братьями моими.

\vs Tsm 4:5
Так остерегайтесь же, дети мои, всякой ревности и зависти и живите в
простоте сердечной, чтобы дал и вам Бог милость и славу и благословение на
головы ваши, как вы видите то на Иосифе.
\vs Tsm 4:6
Ни в какой день не стыдил он нас за дело это,
но возлюбил нас как душу свою, и более сыновей своих почтил нас,
и богатство, и скот, и плоды даровал нам.

\vs Tsm 4:7
И вы, дети мои, возлюбите каждый брата своего в доброте сердечной,
и отойдёт от вас дух зависти.
\vs Tsm 4:8
Ибо озлобляет он душу и губит тело, гнев и вражду вводит в
помышление и побуждает к крови и вводит разум в экстаз,
и смятение создает в душе и дрожь в теле.
\vs Tsm 4:9
Даже во сне злая зависть, соблазняя человека,
пожирает его и духами злыми возмущает душу его,
и заставляет тело его содрогаться,
и смятением лишает сна ум его,
и как дух злой и губительный является людям.
\vs Tsm 5:1
Оттого Иосиф был прекрасен лицом и приятен видом своим,
что не поселялось в нем ничто злое;
ибо смущение духа проступает явно на лице человека.

\vs Tsm 5:2
Ныне, дети мои, смягчите сердце ваше пред Господом
и выпрямите пути ваши пред людьми,
и стяжаете благодать пред лицом Господа и людей.
\vs Tsm 5:3
И остерегайтесь блуда, ибо блуд порождает всякое зло,
отдаляя от Бога и приближая к Велиару.
\vs Tsm 5:4
Видел я в книге Еноха, что сыновья ваши совратятся
от блуда и обиду нанесут мечом своим сыновьям Левия.
\vs Tsm 5:5
Но не смогут они противостоять Левию,
ибо поведёт он брань Господню и одолеет всякое войско ваше.
\vs Tsm 5:6
И будут они малочисленны, разделенные в Левин и в Иуде, и
не будет из вас никого, кто властвовал бы,
как и пророчествовал отец наш в благословениях своих.

\vs Tsm 6:1
И вот, сказал я вам всё, дабы оправдать себя от греха вашего.
\vs Tsm 6:2
И если удалите от себя зависть и всякое жестокосердие,
словно роза расцветут кости мои в Израиле,
и словно лилия плоть моя в Иакове,
и будет благоухание моё словно аромат Ливана,
и умножатся святые от меня во веки веков,
и взрастут отрасли их.
\vs Tsm 6:3
Тогда погибнет семя Ханаана,
и не будет остатка у Амалика,
и сгинут все Каппадокийцы,
и все Хетты истребятся.
\vs Tsm 6:4
Тогда угаснет земля Хама, и погибнет весь народ.
Тогда почиет вся земля от смуты, и всё, что под небесами, от войны.
\vs Tsm 6:5
Тогда прославится Сим,
ибо Господь Бог Израиля придет на землю [как человек] и тем
спасёт Адама.
\vs Tsm 6:6
Тогда предан будет всякий дух соблазна на поругание,
и люди обретут власть над злыми духами.
\vs Tsm 6:7
Тогда воскресну и я в радости и благословлю Всевышнего ради чудес его,
[ибо Господь, приняв тело и вкусив пищу с людьми, спас людей.]

\vs Tsm 7:1
Ныне, дети мои, слушайте Левия и Иуду,
и не восставайте на два эти колена,
ибо от них исполнится нам спасение Божие.
\vs Tsm 7:2
Ибо восстанет Господь из Левия как Первосвященник,
а из Иуды как Царь [Бог и человек].
Он спасёт [все народы и] род Израиля.
\vs Tsm 7:3
Для того внушаю вам это, дабы и вы внушили детям вашим,
да сохранят всё в поколениях своих.

\vs Tsm 8:1
Завершил Симеон наставление сыновей своих и почил с отцами
своими, будучи 120-и лет.
\vs Tsm 8:2
И положили его во гроб деревянный,
чтобы отнести кости его в Хеврон.
И отнесли их втайне, пока Египтяне вели войну.
\vs Tsm 8:3
Ибо кости Иосифа сохранили Египтяне в гробнице царей.
\vs Tsm 8:4
Сказали им прорицатели, что, если вынесут кости Иосифа,
тьма и мрак будут по всей земле и несчастье великое Египтянам,
так что и со светильником не узнает никто брата своего.

\vs Tsm 9:1
И оплакали сыновья Симеона, отца своего.
И пребывали в Египте вплоть до дней, когда Моисей вывел их рукою своею.

\bibbookdescr{Tlv}{
  inline={Завещание Левия,\\третьего сына Иакова и Лии\fns{В греч. тексте $+$ ``о священстве и о гордыне''.}},
  toc={Завещание Левия},
  bookmark={Завещание Левия},
  header={Завещание Левия},
  abbr={Лви}
}
\vs Tlv 1:1
Список слов Левия, которые говорил он сыновьям своим обо всём,
что они совершат и что случится с ними вплоть до дней Суда.
\vs Tlv 1:2
Он был здоров, когда призвал их к себе;
открылось же ему, что должен он вскоре умереть.
И когда собрались они, сказал к ним:

\vs Tlv 2:1
Я, Левий, родился в Хевроне и пришёл с отцом моим в Сиким.
\vs Tlv 2:2
Не было мне ещё двадцати лет,
когда сотворил я возмездие Еммору вместе с Симеоном за сестру нашу Дину.
\vs Tlv 2:3
И когда я пас стадо в Авелмехоле,
дух познания Господа сошел на меня,
и узрел я всех людей, уклонившихся от пути своего,
и грех выстроил себе дом на стенах, а неправедность восседала на башнях.
\vs Tlv 2:4
И был я в скорби о роде сынов человеческих и молил Господа,
чтобы спас меня.
\vs Tlv 2:5
Тогда снизошел на меня сон, и увидел я гору высокую и сам был на ней.

\vs Tlv 2:6
И вот, разверзлись небеса, и ангел Господень сказал мне:
Левий, Левий, войди!
\vs Tlv 2:7
И взошёл я на первое небо и узрел там великую воду висящую.
\vs Tlv 2:8
И ещё увидел я второе небо, много более светлое и сияющее,
высота же его была бесконечной.
\vs Tlv 2:9
И сказал я ангелу: что это такое?
И отвечал мне ангел: не удивляйся тому, ибо иное небо узришь,
более светлое и несравненное.
\vs Tlv 2:10
И поднявшись туда, встанешь ты рядом с Господом, и слугой ему будешь,
и тайны его возвестишь людям, и о грядущем избавлении Израиля возгласишь.
\vs Tlv 2:11
И через тебя и через Иуду явится Господь людям,
чтобы спасти собою весь род человеческий.
\vs Tlv 2:12
И жизнь твоя~--- удел Господа,
и будет он тебе полем и виноградником, и плодом, и золотом, и серебром.

\vs Tlv 3:1
Так услышь о показанных тебе небесах.
Нижнее оттого мрачно на вид, что зрит оно нечестия людские.
\vs Tlv 3:2
И имеет оно. огонь, снег и лед,
уготовленные на день Суда Божией справедливостью.
В нём~---  все духи воздаяний для возмездия людям.
\vs Tlv 3:3
На втором же небе~--- силы войск, построенных на день Суда,
дабы воздать духам соблазна и Велиара, а на них~--- святые.
\vs Tlv 3:4
В высшем же из всех пребывает великая слава,
превосходящая всякую святость.
\vs Tlv 3:5
В следующем же небе~--- архангелы,
служащие Господу и умилостивляющие его
ко всякому неведению праведных.
\vs Tlv 3:6
Подносят они Господу ароматы благоуханные,
жертву мысленную и незапятнанную кровью.
\vs Tlv 3:7
В том же небе, что за ним,~--- ангелы,
несущие молитвы ангелам о лице Божием.
\vs Tlv 3:8
В следующем за ним~--- престолы и власти,
коими хвалебная песнь Богу воспевается.

\vs Tlv 3:9
Когда же смотрит на нас Господь,
все мы дрожим, а небо и земля
и бездна от лица величия его сотрясаются.
\vs Tlv 3:10
Сыны же человеческие, не чувствующие того,
согрешают и гневят Всевышнего.
\vs Tlv 4:1
Познай же ныне, что сотворит Господь Суд над сынами человеческими.
Когда скалы рухнут, и солнце погаснет, и воды высохнут,
и огонь затаится, и всякое творение смутится,
и незримые духи истощатся, и Ад лишится защиты своей
[от страдания Всевышнего],
тогда люди утратят веру и будут упорствовать в неправедности своей,
и за то будут судимы и примут кару.

\vs Tlv 4:2
И услышал Всевышний молитву твою, да избавит тебя от неправедности
и сделает сыном своим, и рабом, и слугою пред лицом его.
\vs Tlv 4:3
Светом знания просияешь ты в Иакове,
и будешь как солнце всему семени Израиля.
\vs Tlv 4:4
И дастся тебе благословение и всему семени твоему дотоле,
пока не посетит Господь все народы по милосердию своему,
во веки веков.
[Но только сыновья твои возложат руки на него, дабы распять его.]

\vs Tlv 4:5
И для того даны тебе совет и знание,
чтобы наставил ты сыновей своих в этом.
\vs Tlv 4:6
Ибо благословляющие тебя благословенны будут,
а проклинающие тебя погибнут.

\vs Tlv 5:1
И вслед за тем открыл мне ангел врата небесные,
и увидел я Святого Всевышнего, восседающего на престоле.
\vs Tlv 5:2
И сказал он мне: Левий, тебе дал я благословение на священство,
доколе не приду и не поселюсь среди Израиля.
\vs Tlv 5:3
И тогда свёл меня ангел на землю и дал мне оружие и меч и сказал мне:
сотвори месть Сихему за Дину, сестру твою, и я буду с тобой,
ибо Господь послал меня.
\vs Tlv 5:4
И погубил я в то время сынов Еммора, как написано на скрижалях отцов.
\vs Tlv 5:5
И сказал ему: прошу тебя, господи, научи меня имени твоему,
дабы призывать мне его в день скорби.
\vs Tlv 5:6
И отвечал он: я ангел, просящий за народ Израилев,
да знает, что не сокрушится он.
\vs Tlv 5:7
И я, проснувшись, благословил Всевышнего.

\vs Tlv 6:1
И тогда пошел я к отцу моему, обрел бронзовый щит, отчего и имя
горы~--- Щит, что близ Гевала одесную Авимы.
\vs Tlv 6:2
И сохранил я слова те в сердце моём.
\vs Tlv 6:3
И совещался я с отцом моим и Рувимом, дабы сказать сынам Еммора,
чтобы приняли они обрезание, ибо пылал я рвением из-за мерзости,
которую сотворили они над сестрою моей.
\vs Tlv 6:4
И я убил первым Сихема, а Симеон~--- Еммора.
\vs Tlv 6:5
А вслед за тем пришли братья мои и перебили город тот остриём меча.
\vs Tlv 6:6
И услышал о том отец мой и, разгневавшись, огорчился,
что приняли они обрезание и умерли,
и в благословениях своих обошел нас.

\vs Tlv 6:7
В том согрешили мы, что сотворили это против воли его, а он болен
был в тот день.
\vs Tlv 6:8
Но я видел, что воля Божия была во зло Сикимам,
так как они хотели и Сарре и Ревекке сделать то,
что сделали Дине, сестре нашей,
но воспрепятствовал им Господь.
\vs Tlv 6:9
И преследовали они Авраама, отца нашего, бывшего чужеземцем,
и изнуряли скот беременный,
и Евлаю, родившуюся в доме Авраама, жестоко оскорбляли.
\vs Tlv 6:10
И так делали они всем чужеземцам,
силою похищая жен их и принуждая их.
\vs Tlv 6:11
И настиг их, наконец, гнев Божий.

\vs Tlv 7:1
И сказал я отцу моему Иакову:
тобою уничтожит Господь Хананеев
и даст землю их тебе и семени твоему после тебя.
\vs Tlv 7:2
Отныне назовутся Сикимы городом глупцов.
Ибо как смеются над глупцами, так посмеёмся и мы над ними.
\vs Tlv 7:3
Безумие сотворили они в Израиле, осквернив сестру мою.
И, встав, пошли мы в Вефиль.

\vs Tlv 8:1
И снова узрел я видение, подобное прежнему,
после того как были мы здесь 70 дней.
\vs Tlv 8:2
И узрел я семерых мужей в белых одеждах, говорящих мне:
восстав, облачись в одеяния священства,
и венец праведности,
и наперсник знания,
и подир правды,
и дощечку веры,
и митру главы,
и ефод пророчества.
\vs Tlv 8:3
И каждый из них нёс нечто, вручал мне и говорил мне:
отныне стань священником, и ты, и всё семя твое.
\vs Tlv 8:4
И первый помазал меня елеем святым и дал мне жезл.
\vs Tlv 8:5
А второй омыл меня водою чистой, и дал мне вкусить хлеба и вина,
и облачил меня в одеяние святое и славное.
\vs Tlv 8:6
Третий же облачил меня в виссон, подобный ефоду.
\vs Tlv 8:7
Четвёртый же надел на меня пояс, подобный порфире.
\vs Tlv 8:8
Пятый же дал мне ветвь тучной оливы.
\vs Tlv 8:9
Шестой надел мне на голову венец.
\vs Tlv 8:10
Седьмой надел мне диадему священства и наполнил руки мои фимиамом,
дабы служил я священником Господу Богу.
\vs Tlv 8:11
И говорят мне: Левий, разделится семя твоё на три чина в знак славы
Господа грядущего.
\vs Tlv 8:12
И будет первый жребий велик, и над ним не явится другого.
\vs Tlv 8:13
Второй будет жребий священства.
\vs Tlv 8:14
Третьему наречено будет новое имя, ибо восстанет
царь от Иуды и сотворит новое священство
по образу народов для всех народов.
\vs Tlv 8:15
И обретёт любовь явление его, ибо он будет
пророком Всевышнего от семени Авраама, отца вашего.
\vs Tlv 8:16
И всё желанное в Израиле твоё будет и семени твоего, и семя твоё
вкушать будет всё прекрасное видом, и трапезу Господа разделит.
\vs Tlv 8:17
И будут из них священники и судьи, и книжники, и на устах у них святое будет.
\vs Tlv 8:18
И очнувшись от сна, понял я, что этот сон подобен первому.
\vs Tlv 8:19
И скрыл это в сердце моём, и не возвестил о том ни одному человеку на земле.

\vs Tlv 9:1
Спустя же два дня пришли я, Иуда и отец наш Иаков к Исааку, праотцу нашему.
\vs Tlv 9:2
И благословил меня отец отца моего по видениям, которые видел я.
И не пожелал он отправиться с нами в Вефиль.
\vs Tlv 9:3
Когда же пришли мы в Вефиль, увидел отец мой Иаков видение обо мне,
что буду я у них священником.
\vs Tlv 9:4
И восстав наутро, принёс через меня Господу десятину от всего.
\vs Tlv 9:5

И так пришли мы в Хеврон, чтобы пребывать там.
\vs Tlv 9:6
И постоянно призывал меня Исаак, дабы наставлять меня в законе Господа, как и
явил мне ангел.
\vs Tlv 9:7
И учил он меня закону священства, жертвоприношений, всесожжении,
первенцев от плодов, жертв доброхотных и искупительных.
\vs Tlv 9:8
И каждый день наставлял он меня, и занят был со мною, и говорил мне:
\vs Tlv 9:9
удерживай себя от духа блуда, ибо он продолжителен и осквернит
святое через семя твое.
\vs Tlv 9:10
Потому возьми себе жену, ещё будучи молод,
чтобы не было на ней позора и скверны,
и не из рода иноплеменных народов.
\vs Tlv 9:11
И прежде чем войти в святое место, соверши омовение;
и когда приносишь жертву, омойся;
и закончив жертвоприношение, также омойся.
\vs Tlv 9:12
И 12 деревьев, имеющих листья, принеси Господу,
как учил и меня Авраам.
\vs Tlv 9:13
И от всякого животного чистого и пернатого принеси жертву Господу.
\vs Tlv 9:14
И от всех первых плодов и вина принеси первины в жертву Господу Богу.
И осоли всякую жертву солью.

\vs Tlv 10:1
Ныне, сохраните то, что завещаю вам, дети,
ибо услышанное мною от отцов наших возвестил вам.
\vs Tlv 10:2
И вот, неповинен я в нечестии вашем и в преступлениях,
которые совершите вы в конце веков [против Спасителя мира Христа],
соблазняя Израиль и навлекая на него беды всякие от Бога.
\vs Tlv 10:3
И сотворите вы беззакония в Израиле, так что не вынесет
Иерусалим злых дел ваших,
но порвётся завеса в Храме и не скроет непристойности вашей.
\vs Tlv 10:4
И рассеетесь вы пленниками среди народов и будет
там позор и проклятие на вас.
\vs Tlv 10:5
Ибо дом, который изберёт Господь,
Иерусалимом наречётся, как сказано в книге Еноха праведного.

\vs Tlv 11:1
Когда же я взял себе жену, было мне 28 лет;
ей было имя Мелха. 
\vs Tlv 11:2
И зачала она, и родила сына, и нарекли ему имя Гирсон,
ведь были мы в чужой земле.
\vs Tlv 11:3
И увидел я, что не быть ему среди первых.
\vs Tlv 11:4
Кааф же родился в 35-ый год жизни моей,
и было то при восходе солнца.
\vs Tlv 11:5
И узрел я в видении:
стоял он в вышних посреди собрания.
\vs Tlv 11:6
Оттого нарек я ему имя Кааф,
[что значит начало великих дел и наставление].
\vs Tlv 11:7
И 3-го сына родила мне на 40-ом году жизни моей, и оттого,
что страдала в родах его, нарек я его Мерари, что значит огорчение.
\vs Tlv 11:8
Иохаведа же родилась в Египте на 64-ом году моем:
ибо был я во славе между братьев моих.

\vs Tlv 12:1
И взял Гирсон жену и родил от нее Ливни и Шимеи.
\vs Tlv 12:2
Сыновья же Каафовы суть Амрам, Ицгар, Хеврон и Узиил.
\vs Tlv 12:3
А сыновья Мерари суть Махли и Муши.
\vs Tlv 12:4
На 94-ом же году моём взял Амрам Иохаведу,
дочь мою, себе в жёны, ибо в один день родились он и дочь моя.
\vs Tlv 12:5
8-и лет был я, когда вошли мы в землю Ханаанскую,
18-и лет, когда убил я Сихема;
с 19-и лет был я священником,
в 28 лет взял я жену,
и 48-и вошёл я в Египет.
\vs Tlv 12:6
И вот, дети мои, вы~--- третье поколение.
\vs Tlv 12:7
Иосиф умер, когда было мне 118 лет.

\vs Tlv 13:1
Ныне, дети мои, завещаю вам:
бойтесь Господа Бога вашего всем сердцем вашим,
и живите в простоте по всем законам его.
\vs Tlv 13:2
И учите детей ваших грамоте,
дабы имели они знание во всю жизнь свою,
читая постоянно закон Божий.
\vs Tlv 13:3
Ибо всякий, кто познает закон Господа,
почитаем будет, и не примут его как чужого,
куда бы ни пришел он.
\vs Tlv 13:4
И многих друзей, б\acc{о}льших, нежели родители, обретет он,
и возжелают многие из людей служить ему и слушать закон из уст его.

\vs Tlv 13:5
Творите же справедливость, дети мои, на земле, да обретёте её на небесах.
\vs Tlv 13:6
И сейте в душах ваших доброе, и обретёте его в жизни вашей;
если же посеете злое, всякую смуту и скорбь пожнёте.
\vs Tlv 13:7
Мудрость обретёте вы в страхе Божием, ибо если придёт пленение
и уничтожатся города, и земли, и золото, и серебро,
и всякое имущество погибнут,
то мудрости у мудрого никто не сможет отнять,
разве только ослепление нечестия и ожесточение греха.

\vs Tlv 13:8
Если кто убережет себя от злых этих дел,
то будет у него мудрость, и для неприятелей~--- сияющая,
и в чужой земле родина,
и среди врагов друга даст ему.
\vs Tlv 13:9
Всякий, кто учит добру и творит добро,
воссядет на престоле рядом с царями, как Иосиф, брат мой.

\vs Tlv 14:1
Познал я, дети мои, из писаний Еноха,
что в конце веков согрешите вы против Господа,
наложив руки [на Него] и у всех народов будете посмешищем.
\vs Tlv 14:2
А ведь отец наш Израиль чист от нечестия первосвященников
[которые возложат руки свои на Спасителя мира].
\vs Tlv 14:3
Как чисто солнце над землёй пред лицом Господа, так и вы
будьте светочами Израиля надо всеми народами.
\vs Tlv 14:4
И если вы помрачитесь нечестием, что тогда делать народам,
в слепоте пребывающим?
И навлечёте вы проклятие на род ваш за то,
что свет закона, данный вам для просвещения всякого человека,
его захотите вы убить, уча заповедям, которые противны законам Божиим.

\vs Tlv 14:5
Приношения Господу расхитите, и от частей его украдёте отборные,
и пожрёте их дерзко с блудницами.
\vs Tlv 14:6
И заповедям Господним учить станете из алчности,
и замужних женщин оскверните,
и с блудницами и с прелюбодейками осквернитесь,
дочерей же язычников возьмёте в жёны,
и будет смешение ваше подобно Содому и Гоморре.
\vs Tlv 14:7
И возгордитесь вы в священстве вашем, вознесясь над людьми,
и не только над ними, но и над заповедями Божиими.
\vs Tlv 14:8
Ибо презрите вы святое, ругаясь и насмехаясь.
 
\vs Tlv 15:1
Оттого Храм, избранный Господом, запустеет в нечистоте вашей,
а вы пленниками будете у всех народов.
\vs Tlv 15:2
И мерзостью будете для них, и срам стяжаете и позор вечный
от правосудия Божия.
\vs Tlv 15:3
И все ненавидящие вас возрадуются погибели вашей.
\vs Tlv 15:4
И если не обретёте милости через Авраама, Исаака и Иакова,
отцов ваших, ни единого из семени вашего не останется на земле.

\vs Tlv 16:1
И ныне познал я, что 70 седмин пребудете вы в заблуждении
и станете осквернять священство и жертвенники пятнать.
\vs Tlv 16:2
И закон отвергнете, и речи пророков уничтожите в совращении злом.
Преследовать будете вы мужей справедливых, и благочестивых возненавидите,
а словами правдивыми гнушаться станете.
\vs Tlv 16:3
А человека, обновляющего закон силою Всевышнего,
в обмане обвините, и затем и подниметесь, чтобы убить его, не зная,
что восстанет он, и во злобе вашей примете кровь его невинную на головы ваши.
\vs Tlv 16:4
Говорю же вам, что из-за того запустеют святыни ваши до основания.
\vs Tlv 16:5
И не будет чисто место ваше, но будете прокляты
и рассеяны среди народов дотоле, пока не явится он вновь,
и не смилуется, и не примет вас к себе.

\vs Tlv 17:1
И как услышали вы о семидесяти седминах, услышьте и о
священстве.
\vs Tlv 17:2
Ибо каждый юбилей будет священство.
И в первый юбилей первый помазанный
на священство велик будет и станет говорить с Богом как с отцом,
и священство его наполнится Господом, и во дни радости его для спасения мира
он воскреснет.
\vs Tlv 17:3
Во второй же юбилей помазанный взят будет в печали возлюбленного,
и будет священство его почтено и превыше всего прославится.
\vs Tlv 17:4
Третий же священник скорбью объят будет.
\vs Tlv 17:5
Четвёртый же в страданиях будет,
ибо множество несправедливости поднимется против него,
и во всём Израиле возненавидит каждый ближнего своего.
\vs Tlv 17:6
Пятый тьмою будет объят.
\vs Tlv 17:7
Так же~--- и шестой, и седьмой.
\vs Tlv 17:8
В седьмой же юбилей будет мерзость,
которой не могу высказать пред лицом людей,
ибо тогда узнают, как творить её.
\vs Tlv 17:9
Оттого пленены будут и ограблены, и исчезнет земля их и само бытие их.
\vs Tlv 17:10
В пятую же седмину вернутся они в землю опустошения
их и возобновят Дом Господень.
\vs Tlv 17:11
В седьмую же седмину обретут они священников,
которые будут идолопоклонники, любостяжатели, гордецы,
беззаконники, нечестивцы, растлители детей и скотоложцы.

\vs Tlv 18:1
И когда придёт отмщение им от Господа,
исчезнет священство.
\vs Tlv 18:2
Тогда восставит Господь священника нового,
которому все слова Господа откроются,
и сам будет вершить он суд правды на земле множество дней.
\vs Tlv 18:3
И взойдёт на небесах звезда его, словно царская,
свет знания несущая, словно свет солнца, и возвеличится во вселенной.
\vs Tlv 18:4
Озарит она землю, словно солнце, и истребит всякий мрак из
поднебесной, и настанет мир на всей земле.
\vs Tlv 18:5
Небеса возвеселятся во дни его,
и земля возрадуется, и облака возликуют,
[и знание Господне прольется на землю, как вода морская,]
и ангелы славы лика Господня возрадуются ему.
\vs Tlv 18:6
Небеса разверзнутся, и из Храма Славы сойдёт
на него святость с голосом Отцовым, словно голос Авраама к Исааку.
\vs Tlv 18:7
И прольётся на него слава Всевышнего,
и дух знания и святости почиет на нём [в воде].
\vs Tlv 18:8
Ибо он даст величие Господа сынам своим воистину навеки;
и не унаследует ему никто в поколениях и поколениях до века.
\vs Tlv 18:9
И в священство его народы исполнятся знанием на земле и освящены
будут благодатью Господней.
[Израиль же умалится в незнании и помрачится в скорби.]
В священство его исчезнет грех, и беззаконники перестанут творить зло.
\vs Tlv 18:10
И отверзнет он врата Рая и отвратит меч, угрожающий Адаму.
\vs Tlv 18:11
И даст он святым вкусить от Древа Жизни, и дух святости пребудет на них.
\vs Tlv 18:12
И Велиара он свяжет и даст власть детям своим попрать злых духов.
\vs Tlv 18:13
И возрадуется Господь детям своим, и благоволить будет возлюбленным
его до века.
\vs Tlv 18:14
Тогда возвеселятся Авраам, Исаак и Иаков, и я возрадуюсь,
и все святые облекутся радостью.

\vs Tlv 19:1
Ныне же, дети мои, всё вы слышали.
Изберите себе либо свет, либо тьму;
либо закон Господа, либо дела Велиара.
\vs Tlv 19:2
И отвечали ему сыновья его, говоря:
пред лицом Господа будем жить мы, и по закону его.
\vs Tlv 19:3
И сказал им отец их: свидетель Господь, и свидетели ангелы его,
и свидетели вы, и свидетель я речам уст ваших.
И сказали ему сыновья его: свидетели.

\vs Tlv 19:4
Так окончил Левий завещание сыновьям своим, и вытянул ноги свои на
ложе, и приложился к отцам своим, прожив 137 лет.
\vs Tlv 19:5
И положили его во гроб и после погребли его в Хевроне с
Авраамом, Исааком и Иаковом.

\bibbookdescr{Tju}{
  inline={Завещание Иуды,\\четвёртого сына Иакова и Лии},
  toc={Завещание Иуды},
  bookmark={Завещание Иуды},
  header={Завещание Иуды},
  abbr={Ида}
}
\vs Tju 1:1
Список слов Иуды, кои сказал он сыновьям своим, прежде чем умереть.
\vs Tju 1:2
Собравшись, пришли они к нему, и сказал он им:
\vs Tju 1:3
внемлите, дети мои, Иуде, отцу вашему.
Четвёртым сыном был я у отца моего Иакова, и Лия,
мать моя, нарекла меня Иудой, говоря: благодарю Господа за то,
что дал он мне и четвёртого сына.

\vs Tju 1:4
Смышлён был я в юности моей и слушался каждого слова отца моего.
\vs Tju 1:5
И чтил я мать мою и сестру матери моей.
\vs Tju 1:6
И когда настала пора зрелости моей,
благословил меня отец мой, говоря:
царем будешь ты, благим путем идущим во всем.

\vs Tju 2:1
И дал мне Господь милость во всех делах моих: в поле и в доме.
\vs Tju 2:2
Помню, что гнался я за оленем, и взял его,
и приготовил его в пищу отцу моему, и ел он.
\vs Tju 2:3
И серну одолел я в беге, и всё, что было на равнине, ловил я.
\vs Tju 2:4
Льва убил я и спас козленка из пасти его.
Медведицу поймал я за лапы, и бросил её в пропасть, и разбилась она.
\vs Tju 2:5
Дикую свинью нагнал я, и схватил на бегу, и растерзал её.
\vs Tju 2:6
Барс в Хевроне напал на пса моего, и я схватил барса за хвост,
и бросил его о скалу, и разбился он надвое.
\vs Tju 2:7
Быка дикого нашёл я, пасшегося в поле, и за рога поймал его,
и по кругу прогнав его, и помрачив зрение его, бросил и убил его.

\vs Tju 3:1
Когда же пришли два царя Ханаанских вооруженных к пастбищам нашим,
и народ многочисленный с ними, подбежал я один к царю одному,
и, ударив его по голеням, убил его.
\vs Tju 3:2
Другого же царя, Таппуаха, сидящего на коне
[убил я и тем весь народ его рассеял.
\vs Tju 3:3
И царя Ахора,] мужа огромного роста нашёл я,
стрелявшего из лука вперед и назад,
и поднял я камень в 60 фунтов, бросил его в коня и убил его.
\vs Tju 3:4
[И бился я с Ахором 2 часа, и убил его, и рассёк щит его на две части,
и отсёк ноги его.]
\vs Tju 3:5
Когда же снимал я панцирь его, вот, 8 мужей, бывших с ним,
сразились со мною.
\vs Tju 3:6
Намотал я одежду на руку мою, и метал в них камни, как из пращи,
и 4-х убил, а остальные бежали.

\vs Tju 3:7
Отец же мой Иаков убил Велисафа, царя всех царей,
мужа огромного роста, в 12 локтей.
\vs Tju 3:8
И трепет напал на них, и перестали воевать с нами.
\vs Tju 3:9
Оттого не знал беды отец мой в войнах, что с ним был я и братья мои.
\vs Tju 3:10
Ибо узрел он в видении обо мне, что ангел силы следует за мной повсюду,
да не буду побежден.

\vs Tju 4:1
После того произошла у нас война на юге, б\acc{о}льшая бывшей в Сикиме.
И встал я рядом с братьями моими, и преследовали мы 1000-у,
и убили из них 200.
\vs Tju 4:2
И взошёл я на стены и убил царя их.
\vs Tju 4:3
Так освободили мы Хеврон, и взяли всех врагов в плен.

\vs Tju 5:1
На другой день пошли мы в Арету, город могучий и сильный,
грозящий нам смертью.
\vs Tju 5:2
Я и Гад подошли к городу с востока, а Рувим и Левий~--- с запада.
\vs Tju 5:3
И помыслили те, что были на стенах, что мы одни, и пошли на нас.
\vs Tju 5:4
И тут тайно вошли братья наши с других сторон в город.
\vs Tju 5:5
И взяли мы его острием меча, а тех, кто бежал в башню,
огнём сожгли, и так захватили всех и всё имущество их.
\vs Tju 5:6
Когда же уходили мы, мужи из Таппуаха отняли у нас добычу нашу,
и, увидев то, вступили мы в битву с ними.
\vs Tju 5:7
И перебили всех, и обратно взяли добычу.

\vs Tju 6:1
И когда были мы у вод Хозевы, пошли на нас войной люди из Иовеля.
\vs Tju 6:2
И восстав на них, обратили мы их в бегство,
и союзников их из Силома убили,
и не дали им прохода, чтобы идти на нас.
\vs Tju 6:3
И вновь пошли на нас люди из Махира на 5-ый день,
и, восстав на них с мощным ножом,
победили мы их и убили также и их прежде,
нежели выступили они в поход.
\vs Tju 6:4
Когда же подошли мы к городу их,
покатили на нас камни женщины их с высоты горы, где был город.
\vs Tju 6:5
И, укрывшись, я и Симеон сзади взошли на гору и уничтожили и этот город.

\vs Tju 7:1
А на другой день сказали нам,
что царь города Гааш с народом многочисленным идёт на нас.
\vs Tju 7:2
Тогда я и Дан, сделав вид, что мы Амореяне,
как союзники вошли в город их.
\vs Tju 7:3
И глубокой ночью пришли братья наши, мы же открыли им ворота,
и всех жителей перебили и ограбили и 3 стены их разрушили.
\vs Tju 7:4
И подошли к Фамне, где было всё хранилище их.
\vs Tju 7:5
Тут разгневали меня насмешки их, и двинулся я к ним на вершину,
а они метали в меня камни и стреляли из лука.
\vs Tju 7:6
И если бы Дан, брат мой, не вступил в бой вместе со мною, убили бы они меня.
\vs Tju 7:7
И отважно наступили мы на них, и бежали они все, и,
отойдя иным путем к отцу нашему, они умолили его, и он заключил мир с ними.
\vs Tju 7:8
И не сделали мы им никакого зла,
а сделали их данниками нашими и отдали им добытое от них.
\vs Tju 7:9
И восстановили мы города их:
я пострил Фамну, а отец мой построил Рабаэл.
\vs Tju 7:10
Было же мне 20 лет, когда совершилась война эта.
\vs Tju 7:11
И страшились Хананеи меня и братьев моих.

\vs Tju 8:1
Было у меня много скота, и имел я начальником над пастухами
Хиру Одолламитянина.
\vs Tju 8:2
Придя к нему, увидел я Варсаву, царя Одоллама.
И говорил он с нами, и устроил нам пир.
И предложил он, и дал мне в жёны дочь свою, именем Вирсавию.
\vs Tju 8:3
Она родила мне Ира, Онана и Шелу.
И двоих погубил Господь, а Шела остался жить.

\vs Tju 9:1
18 лет был мой отец в мире с братом своим Исавом,
и дети Исава с нами, после того как пришли мы из Месопотамии, от Лавана.
\vs Tju 9:2
Когда же исполнились 18 лет, пришел к нам Исав, брат отца моего,
с народом сильно вооруженным и могучим.
\vs Tju 9:3
И поразил стрелою Иаков Исава,
и тот был унесен раненым на гору Сеир и умер.
\vs Tju 9:4
И мы преследовали сыновей Исава,
а был у них город с железными стенами и медными воротами,
и не могли мы войти в него.
Окружили мы и осадили его.
\vs Tju 9:5
И когда не отворяли они нам 20 дней,
приставил я лестницу на виду у всех и,
держа щит над головой моей и сдерживая удары камней,
взошёл наверх и убил четверых могучих мужей их.
\vs Tju 9:6
Рувим и Гад убили еще шестерых.
\vs Tju 9:7
Тогда просили они нас о мире, и, посоветовавшись с отцом нашим,
приняли мы их в данники.
\vs Tju 9:8
И давали они нам 50 гомеров пшеницы, и масла 50 батов,
и вина 50 мер вплоть до голода, когда пошли мы в Египет.

\vs Tju 10:1
После того взял в жены Ир, сын мой, Фамарь из Месопотамии,
бывшую дочерью Арама.
\vs Tju 10:2
Ир же был недобрым и смущался пред Фамарью,
ибо не была она из Ханаана, и умертвил его ангел Господень.
\vs Tju 10:3
И дал я её Онану, 2-му сыну моему, и его убил Господь.
\vs Tju 10:4
Ибо он не познавал её, хотя прожил с нею год, не желая иметь детей от неё.
\vs Tju 10:5
Когда же пригрозил я ему, сошёлся он с нею,
но излил семя на землю по совету матери своей.
И от греха этого умер и он.
\vs Tju 10:6
Я же хотел дать Фамари и Шелу, но мать его не дозволила.
Злые помыслы имела она,
ибо не была Фамарь из дочерей Хананеев, как она сама.

\vs Tju 11:1
Я же знал, что злой род Хананеи, но мысли юности ослепили разум мой.
\vs Tju 11:2
И, увидев, как она разливает вино, прельстился я
и взял её без воли на то отца моего.
\vs Tju 11:3
Она же в мое отсутствие пошла и взяла Шелу жену из Ханаана.
\vs Tju 11:4
А я, узнав, что сотворила она, проклял ее в скорби души моей.
\vs Tju 11:5
И умерла она от грехов своих вслед за детьми своими.

\vs Tju 12:1
Когда овдовела Фамарь и прошло 2 года, услышала она,
что иду я стричь овец и, нарядившись в наряд свадебный,
села в городе Енаиме у ворот.
\vs Tju 12:2
Был же закон у Амореев, чтобы вдова 7 дней сидела блудницей у ворот.
\vs Tju 12:3
И я, опьянённый вином, не узнал её,
и прельстила меня красота её из-за прекрасного наряда.
\vs Tju 12:4
И свернув с дороги к ней, сказал я: войду к тебе.
А она спросила: а что ты дашь мне?
И дал я ей посох мой, и пояс, и диадему царства моего в залог.
И когда вошёл к ней, зачала она.
\vs Tju 12:5
И не зная, что сам сотворил, хотел я убить Фамарь.
Она же, послав мне тайно всё данное ей, устыдила меня.
\vs Tju 12:6
И призвав её, услышал я те слова тайные, что говорил ей,
когда возлежал с нею в опьянении моём.
И не мог убить её, ибо то было дано Господом.
\vs Tju 12:7
И сказал я: не лукавила она, взяв у другой женщины этот знак.
\vs Tju 12:8
Но не сходился я с ней более до конца жизни моей,
ибо мерзость сотворил я во всём Израиле.
\vs Tju 12:9
А жители города того говорили, что не было блудницы у ворот,
ибо она пришла из другого места и недолго сидела там.
\vs Tju 12:10
И помыслил я, что не видел никто, как вошёл я к ней.

\vs Tju 12:11
После того пошли мы в Египет к Иосифу, так как был голод.
\vs Tju 12:12
Было мне 46 лет, и 73 года провел я в Египте.

\vs Tju 13:1
Ныне завещаю вам, дети мои, послушайте Иуду, отца вашего,
и сохраните слова мои, чтобы делать всё по велениям Господа
и подчиняться заповедям его.
\vs Tju 13:2
Не идите за вожделениями вашими в гордыне сердца своего,
и не похваляйтесь делами и силой молодости вашей,
ибо это злое дело пред лицом Господа.
\vs Tju 13:3
Когда я возгордился, что в войнах не прельстило меня лицо
женщины благообразной, и позорил брата моего Рувима из-за Баллы,
женщины отца моего, тогда стал приступать ко мне дух зависти и блуда,
пока не согрешил я с Вирсавией Хананеянкой и с Фамарью,
невесткой моей.
\vs Tju 13:4
Ибо сказал я тестю моему:
посоветуюсь с отцом моим и тогда возьму дочь твою.
Он же не захотел, но показал мне золота несметное количество,
что было за дочерью его, ибо он был царь.
\vs Tju 13:5
И нарядил он её в золото и жемчуги, велел ей разливать вино на пиру.
\vs Tju 13:6
И совратило вино очи мои, и помрачило мне сердце наслаждением.
\vs Tju 13:7
И возлюбив её, возлёг с нею, и пренебрег заповедью Господа
и заповедью отца моего, и взял её в жены.
\vs Tju 13:8
И воздал мне Господь за помысел души моей, ибо не был я счастлив в детях её.

\vs Tju 14:1
И ныне говорю, дети мои: не опьяняйтесь вином,
ибо вино отвращает разум от истины,
и производит страсть вожделения,
и вводит очи в соблазн.
\vs Tju 14:2
Ведь дух блуда словно слугою имеет вино, дабы услаждать ум,
так что совращают эти два помысла человека.
\vs Tju 14:3
Ибо, если некто пьёт вино до опьянения,
мыслями нечистыми возмущает он ум свой,
и для блуда разгорячает тело свое,
дабы насладиться, и грех совершает, и не стыдится.
\vs Tju 14:4
Таково вино, дети мои, ибо не стыдится опьяневший никого.
\vs Tju 14:5
Вот, и меня оно соблазнило,
так что не устыдился я множества жителей города,
ибо на глазах у всех возлёг с Фамарью,
и совершил грех великий, и раскрыл тайну своей нечистоты сыновьям моим.
\vs Tju 14:6
Пил я вино, и не устыдился заповеди Божией, и взял в жёны Хананеянку.
\vs Tju 14:7
Ибо великое умение нужно пьющему вино, дети мои; это умение винопития,
дабы пить до того времени, пока имеет человек стыд.
\vs Tju 14:8
Когда же перейдёт он предел, входит в ум его дух соблазна
и заставляет пьяного сквернословить,
и творить беззакония, и не стыдиться бесчестия своего,
но кичиться им и мнить себя прекрасным.

\vs Tju 15:1
Блудящий наказания не чувствует и бесчестия не стыдится.
\vs Tju 15:2
Если же царь блудит, лишается он царства,
порабощённый блудом, как и я то претерпел.
\vs Tju 15:3
Отдал я посох мой, который есть опора племени моего,
и пояс мой, который есть сила,
и диадему, которая есть слава царства моего.
\vs Tju 15:4
И раскаявшись в том, не пил я вина и не вкушал мяса до старости моей,
и никакого веселья не видел.
\vs Tju 15:5
И показал мне ангел Божий, что и царем, и нищим правят женщины.
Но не в них преуспеяние жизни.
\vs Tju 15:6
У царя отнимают они славу,
у мужественного~--- силу,
а у нищего~--- самую малую опору в его нищете.

\vs Tju 16:1
Остерегайтесь же, дети мои, преступить предел, положенный вину,
ибо в нём~--- 4 злые духа:
вожделения, жаркой страсти, распутства и алчности.
\vs Tju 16:2
Когда пьёте вино в радости, будьте умеренны, боясь Бога.
Ибо если в радости исчезнет страх Божий, наступит опьянение,
и придет бесстыдство.
\vs Tju 16:3
Если же хотите жить разумно, вовсе не прикасайтесь к вину,
дабы не согрешить в словах надменных, и побоищах, и клевете,
и нарушении заповедей Божиих, и не погибнете не в свой час.
\vs Tju 16:4
Также раскрывает вино тайны Божии и людские,
как и я раскрыл заповеди Божий и тайны Иакова, отца моего,
Хананеянке Вирсавии, чего не велел мне Бог раскрывать.

\vs Tju 17:1
И ныне завещаю вам, дети мои,
не любить серебра и не смотреть на красоту женщин,
ибо и я от серебра и золота, и от красоты соблазнился Вирсавией Хананеянкой.
\vs Tju 17:2
[И знаю, что из-за этих двух предан будет род мой на погибель блуда.
\vs Tju 17:3
Ибо и мудрых мужей из сынов моих собьют они с пути,
и умалят царство Иуды, данное мне Господом за послушание отцу моему.
\vs Tju 17:4
Ведь я никогда не огорчал отца моего Иакова, ибо делал всё,
что говорил он мне.
\vs Tju 17:5
И Авраам, отец деда моего, благословил меня царствовать в Израиле,
и так же благословил меня Иаков.
\vs Tju 17:6
И знаю я, что от меня восстановится царство.

\vs Tju 18:1
И познал я, и читал в книгах Еноха праведного,
какое зло сотворите вы в последние дни.]
\vs Tju 18:2
Остерегайтесь же, дети мои, блуда и сребролюбия, и послушайте Иуду,
отца вашего.
\vs Tju 18:3
Ибо они уводят от закона
Божия и помрачают помысел душевный,
и гордыне научают, и не дают мужу иметь сострадание к ближнему своему.
\vs Tju 18:4
Лишают они душу его всякой доброты
и утесняют его болями и страданием,
сон прогоняют от него и плоть его истребляют.
\vs Tju 18:5
Жертвам Богу он препятствует, о благословении Божием не помнит,
и когда пророк говорит, не слушает, и от слов благочестия отвращается.
\vs Tju 18:6
Ибо двум страстям, противным заповедям Божиим,
рабски служит он и Богу повиноваться не может.
Помрачили они душу его, и днем ходит он словно ночью.

\vs Tju 19:1
Дети мои, сребролюбие ведёт к идолопоклонству,
ибо в соблазне серебра называют богами тех,
кто не есть Бог, а тот, кто имеет серебро, в безумие впадает.
\vs Tju 19:2
От серебра погиб я, дети мои, и если бы не раскаяние моё,
и смирение, и мольбы отца моего, бездетным умер бы я.
\vs Tju 19:3
Но Бог отцов наших смиловался надо мною,
ибо по неведению сотворил я это.
\vs Tju 19:4
Ибо ослепил меня отец обмана и пребывал я в заблуждении
как человек и плоть, грехами сокрушенная, и познал я немощь мою,
когда думал, что непобедим я.

\vs Tju 20:1
Знайте же, дети мои, что 2 духа смотрят
за человеком~--- дух правды и дух лжи.
\vs Tju 20:2
Посредине же~--- дух познания, склоняющего ум туда, куда пожелает.
\vs Tju 20:3
А правдивое и лживое написаны на сердце человека, и всё это известно Господу.
\vs Tju 20:4
И нет часа, в который могли бы укрыться дела людские,
ибо на самом сердце написано пред лицом Господа.
\vs Tju 20:5
А дух правды обличает всё, и жжет грешника огнём в его же сердце,
и не может он поднять лица к Судье.

\vs Tju 21:1
Ныне, дети мои, возвещаю вам:
любите Левия, и пребывайте с ним, и не возноситесь над ним,
да не уничтожитесь вы.
\vs Tju 21:2
Ибо мне дал Бог царство, ему же~--- священство,
и подчинил царство священству.
\vs Tju 21:3
Мне дал он то, что на земле, ему~--- то, что на небесах.
\vs Tju 21:4
Как небеса выше земли,
так священство Божие выше стоит, нежели царство земное,
если согрешив, не отпадёт оно от Господа
и не станет править священством царство земное.
\vs Tju 21:5
Ибо сказал мне ангел Господень: избрал его Господь и поставил выше тебя,
чтобы приблизился ты к нему, и вкушал от трапезы его,
и первенцев от богатств сынов Израиля приносил ему.
Ты же будешь царем над Иаковом.
\vs Tju 21:6
И будешь ты подобен морю.
Ибо, как на море праведные и неправедные попадают в бурю,
и одни попадают в плен, другие же обогащаются,
так и в тебе со всяким родом людей так будет:
одни будут терпеть опасности и пленение,
другие же обогащаться, похищая чужое.
\vs Tju 21:7
Ибо цари китам уподобятся: пожирая людей, словно рыб,
станут они порабощать сыновей и дочерей свободных и грабить дома,
поля, пастбища и всякое добро.
\vs Tju 21:8
И неправедно тела многих отдадут в пищу воронам и цаплям,
и преуспеют во зле, и возвысятся в алчности.

\vs Tju 21:9
И будут лжепророки, словно вихри, и многих праведных будут преследовать.
\vs Tju 22:1
И наведёт на них Господь раздоры друг с другом,
и войны будут в Израиле непрерывные.
\vs Tju 22:2
И к иноплеменным перейдёт царство моё до прихода спасения к Израилю,
до явления Бога справедливого, когда почиет Иаков в мире и все народы.
\vs Tju 22:3
И сам Господь сохранит навеки силу царства моего,
ибо клялся он мне клятвою, что не угаснет царство семени моего до века.

\vs Tju 23:1
Великое горе для меня, дети мои, от нечестия и обмана,
которые сотворите вы в царстве моём,
когда последуете за чревовещателями, прорицателями и бесами соблазна.
\vs Tju 23:2
Дочерей ваших певицами и блудницами сделаете,
и смешаетесь с мерзостью языческой.
\vs Tju 23:3
За то наведёт на вас Господь голод и мор, смерть и меч,
осаду от врагов и позор от друзей, и воспаление очей,
и убийство детей, и похищение имущества, и сожжение Храма Божьего,
и порабощение вас самих язычниками.
\vs Tju 23:4
И оскопят сыновей ваших, чтобы стали они евнухами у жен их.
\vs Tju 23:5
И будет так дотоле, пока не посетит вас Господь,
когда раскаетесь вы и станете жить по всем заповедям его,
и выведет он вас из плена языческого.

\vs Tju 24:1
После того взойдет вам звезда из Иакова в знак мира,
и восстанет человек [от семени моего],
как солнце праведности, и будет жить с людьми в кротости и справедливости,
и не будет на нём никакого греха.
\vs Tju 24:2
И разверзнутся над ним небеса, дабы излить дух благословения
Отца Святого, и сам он изольёт дух милости на вас.
\vs Tju 24:3
И будете ему сыновьями истинными, и жить будете по заветам
его первым и последним.
\vs Tju 24:4
[Он есть отрасль Бога Всевышнего и источник, дающий всем жизнь.]
\vs Tju 24:5
Тогда воссияет скипетр царства моего, и из корня вашего выйдет ствол.
\vs Tju 24:6
А на нём взрастёт жезл праведности для народов,
дабы судить и спасти всех призывающих имя Господа.

\vs Tju 25:1
После того восстанут к жизни Авраам, Исаак и Иаков,
а я и братья мои станем вождями колен Израиля:
1-ый~--- Левий,
2-ой~--- я,
3-ий~--- Иосиф,
4-ый~--- Вениамин,
5-ый~--- Симеон,
6-ой~--- Иссахар,
и так все по порядку.
\vs Tju 25:2
И благословил Господь Левия;
ангел лика Господня~--- меня;
силы славы~--- Симеона;
небо~--- Рувима;
земля~--- Иссахара;
море~--- Завулона;
горы~--- Иосифа;
скиния~--- Вениамина;
светильники~--- Дана;
сад Едемский~--- Неффалима;
солнце~--- Гада;
луна~--- Асира.
\vs Tju 25:3
И будете вы один народ Господень и один язык,
и не будет там духа соблазна Велиарова,
ибо он будет ввержен в огонь навечно.
\vs Tju 25:4
И в скорби скончавшиеся восстанут в радости, а нищие Господом
обогащены будут, а умирающие Господом вдохновлены к жизни будут.
\vs Tju 25:5
И в веселии побегут олени Иакова,
и орлы Израиля полетят в радости, [а нечестивые
восскорбят, и грешники зарыдают], и все народы прославят Господа навеки.

\vs Tju 26:1
Храните же, дети мои, все законы Господни, ибо он есть надежда для всех,
соблюдающих пути его.

\vs Tju 26:2
И сказал Иуда: вот, 118-ти лет умираю я сегодня.
\vs Tju 26:3
Да не погребает меня никто в пышной одежде,
и да не разрезают мне чрево,
что угодно творить царствующим,
но отнесите меня в Хеврон, где и отцы мои.
\vs Tju 26:4
И сказав это, почил он, и сделали сыновья его во всём так, как
завещал он им, и погребли его с отцами его в Хевроне.

\bibbookdescr{Tis}{
  inline={Завещание Иссахара,\\пятого сына Иакова и Лии},
  toc={Завещание Иссахара},
  bookmark={Завещание Иссахара},
  header={Завещание Иссахара},
  abbr={Исс}
}
\vs Tis 1:1
Список слов Иссахара.
Ибо он призвал сыновей своих и сказал им:
выслушайте, дети, Иссахара, отца вашего;
внемлите словам возлюбленного Господом.

\vs Tis 1:2
Родился я пятым сыном Иакова, платою за мандрагоры.
\vs Tis 1:3
Ибо Рувим, брат мой, принёс с поля мандрагоры,
и Рахиль, встретив его, взяла их.
\vs Tis 1:4
И плакал Рувим, и на голос его вышла Лия, мать моя.
\vs Tis 1:5
А были то яблоки благовонные, которые рождаются в земле Харана
на дне ложбин водных.
\vs Tis 1:6
И сказала Рахиль: не дам я тебе их, но мне самой нужны они,
дабы иметь детей.
Ведь обошёл меня Господь, и не рождала я сыновей Иакову.
\vs Tis 1:7
А яблок было 2.
И сказала Лия Рахили: да будет довольно тебе,
что взяла ты мужа моего, так возьмёшь ещё и это у меня?
\vs Tis 1:8
Отвечала ей Рахиль:
да будет Иаков с тобою в ночь эту за мандрагоры сына твоего.
\vs Tis 1:9
Сказала же ей Лия: мой Иаков, ибо я~--- жена юности его.
\vs Tis 1:10
И сказала Рахиль:
не возносись и не похваляйся,
ибо ко мне первой прилепился он и ради меня служил отцу моему 14 лет.
\vs Tis 1:11
И если бы не возросла хитрость на земле,
и злоба человеческая не преуспевала бы,
не была бы ты тою, что узрела лицо Иакова.
\vs Tis 1:12
Ибо ты не жена его, но хитростью опередила меня.
\vs Tis 1:13
И обманул меня отец мой, и удалил в ту ночь,
и не позволил мне видеть Иакова, ибо, если бы я там была,
не случилось бы того.
\vs Tis 1:14
Но за мандрагоры уступлю тебе на одну ночь Иакова.
\vs Tis 1:15
И познал Иаков Лию, и, зачав, родила она меня,
и от этой платы наречен я был Иссахаром.

\vs Tis 2:1
Тогда явился Иакову ангел Господень, говоря:
родит двоих детей Рахиль,
ибо пренебрегла она сообщением с мужем своим и воздержание избрала.
\vs Tis 2:2
И если бы мать моя Лия за сообщение с Иаковом не отдала 2 яблока,
то родила бы 8-ых сыновей, но из-за того родила 6-ых,
а Рахиль двоих, ибо в мандрагорах призрел на неё Господь.
\vs Tis 2:3
Ибо видел он, что ради детей желала она сойтись с Иаковом,
а не ради любострастия.
\vs Tis 2:4
И на другой день отдала она Иакова, чтобы взять и другую мандрагору.
Ибо в мандрагорах услышал Господь Рахиль.
\vs Tis 2:5
А она, возжелав их, не вкусила, но отнесла их в дом Господень
и отдала священнику, бывшему в то время.

\vs Tis 3:1
Когда же возмужал я, дети мои, жил я в прямоте сердечной,
и стал земледельцем отцу моему и братьям моим,
и приносил плоды с полей.
\vs Tis 3:2
И благословил меня отец мой, видя, что в простоте живу я.
\vs Tis 3:3
И не был я суетным в делах моих, ни завистником, ни клеветником
ближнему моему.
\vs Tis 3:4
Не наговаривал я никогда ни на кого, и не хулил жизнь никакого человека.
\vs Tis 3:5
45-и лет взял я себе жену, ибо труд снедал силы мои,
и не помышлял я о наслаждении от женщины, но от усталости засыпал я.
\vs Tis 3:6
И радовался простоте моей отец мой, ибо всякое первородное через
священника приносил я Господу, а после и отцу моему.
\vs Tis 3:7
И Господь утысячерял добро моё в руках моих,
и знал Иаков, отец мой, что Бог помогает простоте моей.
\vs Tis 3:8
Ибо всем бедным и страждущим уделял я от благ земли в простоте сердца моего.

\vs Tis 4:1
И ныне, внемлите мне, дети мои, и живите в простоте сердец
ваших, ибо узрел я, что в этом всякое благоугождение Господу.
\vs Tis 4:2
Простосердечный не стремится к золоту, и не хочет превзойти ближнего
богатством, и не домогается многообразных яств, и не желает разных одежд.
\vs Tis 4:3
Не хочет он приписать многих лет к своей жизни, но приемлет одну лишь
волю Божию.
\vs Tis 4:4
И духи соблазна ничего не могут против него,
ибо не воззрел он на красоту женскую, дабы не осквернить порчею ума своего.
\vs Tis 4:5
Не завидует он в помыслах своих, и клевета не изнуряет души его,
ни желание ненасытное ума его.
\vs Tis 4:6
Живет он в простоте души, всё зрит в прямоте сердца,
но оберегает очи свои от соблазна мирского,
дабы не видеть уклонений от заповедей Господних.

\vs Tis 5:1
Храните же, дети мои, закон Божий, и простоту обретайте, и в
беззлобии живите, не заботясь излишне о делах ближнего.
\vs Tis 5:2
Но возлюбите Господа и ближнего, а бедного и слабого жалейте.
\vs Tis 5:3
Склоните спины ваши к земледелию и утруждайте себя всяким земледелием,
принося Господу плоды с благодарностью.
\vs Tis 5:4
Ибо в первенцах плодов земных благословит вас Господь,
как благословил он всех святых от Авеля и доныне.
\vs Tis 5:5
Ибо не дастся вам иной удел, кроме тучности земли в трудах плодородия.
\vs Tis 5:6
Так и отец мой Иаков благословениями земли
и первенцев плодов благословил меня.
\vs Tis 5:7
А Левий и Иуда прославлены у Господа и в сынах Иакова.
И дал им наследие Господь:
Левию дал он священство, а Иуде~--- царство.
\vs Tis 5:8
Вы же слушайтесь их и пребывайте в прямодушии отца вашего.

\vs Tis 6:1
Знайте же, дети мои, что в последние времена
оставят сыновья ваши простоту, и погрязнут в алчности,
и отринут беззлобие, и совершат злодеяния,
и оставят заповеди Господа,
и прилепятся к Велиару.
\vs Tis 6:2
И оставят они земледелие, и последуют злым помыслам своим,
и рассеются среди народов, и рабами будут врагам своим.
\vs Tis 6:3
И вы скажите это детям вашим, дабы, если согрешат, тотчас обращались
вновь к Господу.
\vs Tis 6:4
Ибо он милостив, и пожалеет их, и вернёт в землю их.

\vs Tis 7:1
Вот, как видите вы, живу я 126 лет и не знал греха смертного.
\vs Tis 7:2
И кроме жены моей, не познавал я другой и не совершал блуда,
взирая очами моими.
\vs Tis 7:3
Вина соблазняющего не пил, не желал ничего из имущества ближнего моего.
\vs Tis 7:4
Хитрость не рождалась в сердце моём, ложь не входила на уста мои.
\vs Tis 7:5
Сострадал я всякому человеку скорбящему, и с нищим делил хлеб мой.
Благочестие творил я во все дни мои и правду хранил.
\vs Tis 7:6
Господа любил я и всякого человека всем сердцем моим.
\vs Tis 7:7
Так и вы делайте, дети мои, и всякий дух Велиаров бежит от вас,
и никакое дело злых людей не возобладает над вами,
и всякого зверя дикого усмирите,
если с вами будет Бог небес и земли, помогающий людям простосердечным.

\vs Tis 7:8
И сказав это сыновьям своим, завещал им, дабы отнесли его в Хеврон и
погребли там с отцами его.
\vs Tis 7:9
И вытянув ноги свои, почил он в старости прекрасной сном вечным.

\bibbookdescr{Tzb}{
  inline={Завещание Завулона,\\шестого сына Иакова и Лии},
  toc={Завещание Завулона},
  bookmark={Завещание Завулона},
  header={Завещание Завулона},
  abbr={Зав}
}
\vs Tzb 1:1
Список слов Завулона, речённых им к сыновьям своим,
прежде чем умер он в 114-ый год жизни своей,
спустя 2 года после смерти Иосифа.
\vs Tzb 1:2
Сказал он им: слушайте меня, сыновья Завулона, внемлите речам отца вашего.

\vs Tzb 1:3
Я, Завулон, даром прекрасным родился у отца и матери моих.
Ибо когда родился я, премного возрос отец мой мелким и крупным
скотом, когда пеструю скотину получил он в удел.
\vs Tzb 1:4
Не знал я греха за собою во все дни мои, кроме только мысленного.
\vs Tzb 1:5
Не вспомню, чтобы совершил я несправедливость,
кроме греха неведения, сотворенного мною против Иосифа,
когда сговорился я с братьями моими не возвещать отцу моему о случившемся.
\vs Tzb 1:6
Но плакал я много дней из-за Иосифа, втайне, ибо страшился я братьев моих,
ибо положили они, что выдавший тайну будет убит.
\vs Tzb 1:7
Но когда желали убить Иосифа, молил я их со слезами не делать греха этого.

\vs Tzb 2:1
Ибо Симеон, Дан и Гад приступили к Иосифу, чтобы убить его,
и говорил он им со слезами, пав на лицо своё:
\vs Tzb 2:2
пощадите меня, братья мои; пожалейте сердце Иакова, отца нашего.
Не поднимайте рук ваших, дабы пролить кровь невинную,
ибо не согрешил я против вас.
\vs Tzb 2:3
Если же и согрешил, наказанием накажите меня,
но не поднимайте руки,
чтобы убить брата вашего ради отца нашего Иакова.
\vs Tzb 2:4
Когда же говорил он, скорбя, эти речи,
не вынес я стонов его и начал плакать,
и сотряслись внутренности мои, и всё во мне ослабло.
\vs Tzb 2:5
И заплакал я с Иосифом и забилось громко сердце моё,
и задрожали суставы тела моего, и не был я в силах стоять.
\vs Tzb 2:6
Когда же увидел Иосиф, что плачу вместе с ним,
а они подступили и хотят убить его, спрятался за спину мою,
умоляя помочь ему.
\vs Tzb 2:7
Тогда встал Рувим посредине и сказал:
братья мои, не будем убивать его,
но бросим его в один из сухих колодцев,
которые рыли отцы наши и не находили там воды.
\vs Tzb 2:8
Ибо для того не дал Господь подняться туда воде,
чтобы спасся Иосиф.
\vs Tzb 2:9
И сделали они так до той поры,
когда продали его Измаильтянам.

\vs Tzb 3:1
От платы за Иосифа не взял я своей части, дети мои.
\vs Tzb 3:2
Но Симеон, Дан, Гад и другие братья наши,
взяв плату за него, купили обувь себе и жёнам и детям своим,
говоря:
\vs Tzb 3:3
пропитания не купим, ибо это цена крови брата нашего,
но ногами потопчем её за то, что говорил он,
будто станет властвовать над нами;
и увидим, что будет из его снов.
\vs Tzb 3:4
Оттого записано в законе Моисеевом:
с нежелающего возместить семя брату своему
да снимут обувь его и плюнут в лицо ему.
\vs Tzb 3:5
А братья Иосифа не хотели, чтобы жил он,
и Господь снял с них обувь,
которую носили они за брата своего Иосифа.
\vs Tzb 3:6
И когда пришли они в Египет, за воротами сняли с них
обувь дети Иосифа,
и так преклонились они пред Иосифом, словно пред фараоном.
\vs Tzb 3:7
И не только преклонились пред ним,
но и оплёваны были в тот же час,
пав пред ним, и опозорены были пред Египтянами.
\vs Tzb 3:8
Ибо Египтяне слышали обо всех злых делах,
которые совершили они против Иосифа.

\vs Tzb 4:1
И сделав это, сели братья мои есть и пить.
\vs Tzb 4:2
Я же, терзаясь из-за Иосифа, не ел,
но смотрел на колодец, ибо опасался,
как бы Симеон, Дан и Гад не пошли и не убили Иосифа.
\vs Tzb 4:3
Увидев, что я не ем, они оставили меня стеречь его,
\vs Tzb 4:4
пока не продали Измаильтянам.

\vs Tzb 4:5
Затем пришел Рувим и, услышав,
что продали Иосифа в его отсутствие,
разодрал хитон свой и сказал плача:
как посмотрю я в лицо отцу моему Иакову?
\vs Tzb 4:6
И взяв серебро,
побежал вслед за купцами и, не найдя их,
вернулся опечаленный.
Купцы же, сойдя с широкой дороги,
пошли кратчайшим путём через землю Троглодитов.
\vs Tzb 4:7
И скорбел Рувим, и не ел хлеба в тот день.
И подойдя к нему, сказал Дан:
\vs Tzb 4:8
не плачь и не скорби; мы найдем, что сказать отцу нашему.
\vs Tzb 4:9
Зарежем козла, и вымараем хитон Иосифа,
и пошлём его Иакову, говоря:
узнай, сына ли твоего этот хитон?
Так они и сделали.
\vs Tzb 4:10
Ибо, продавая Иосифа, сняли с него хитон и одели на него плащ рабский.
\vs Tzb 4:11
Симеон же взял хитон и не хотел отдать его, ибо он желал убить Иосифа
и гневался, что не убили его.
\vs Tzb 4:12
И встав, сказали все мы ему:
если не отдашь хитон, скажем отцу нашему,
что ты один сотворил это зло в Израиле.
\vs Tzb 4:13
И отдал он им хитон. И сделали так, как сказал Дан.

\vs Tzb 5:1
Ныне, дети мои, завещаю вам хранить заповеди Господа,
и творить милость ближнему,
и добросердечными быть не только к людям,
но и к бессловесным животным.
\vs Tzb 5:2
За это и благословил меня Господь,
и когда занемогли все братья мои,
оставался я здоров; ибо знал Господь помыслы каждого.
\vs Tzb 5:3
Имейте же милость в сердцах ваших, ибо что сделает человек ближнему
своему, то сделает Господь с ним самим.
\vs Tzb 5:4
И болели, и умирали сыновья братьев моих из-за Иосифа,
ибо не имели милости в сердцах своих.
А мои сыновья сохранялись в здравии,
как вам то известно.
\vs Tzb 5:5
И когда были мы в земле Ханаанской,
ловил я рыб для отца моего Иакова,
и многие утонули в море,
а я невредим остался.

\vs Tzb 6:1
Первым я был, кто сделал лодку, чтобы плавать по морю,
ибо дал мне Господь для того знание и мудрость.
\vs Tzb 6:2
И приладил я весло деревянное сзади у неё,
а на другом прямом куске дерева натянул парус посредине.
\vs Tzb 6:3
И плавал я в лодке той по морским водам,
и ловил рыбу для дома отца моего,
пока не пошли мы в Египет.
\vs Tzb 6:4
[И из добычи моей всякому человеку чужому уделял я от доброты сердца.
\vs Tzb 6:5
Был ли кто чужестранец, или больной, или старый, готовил я рыбу,
делал её хорошо и давал всякому по надобности его,
соболезнуя и сострадая.
\vs Tzb 6:6
За это много рыб давал мне Господь, когда ловил я.
Ибо тот, кто делится с ближним,
получает многократно от Господа.]
\vs Tzb 6:7
5 лет ловил я рыбу [давая всякому человеку,
какого видел, и вдоволь отдавая дому отца моего].
\vs Tzb 6:8
Летом ловил я, а зимою пас стада вместе с братьями моими.

\vs Tzb 7:1
[Ныне возвещу вам, что сделал я.
Увидел я зимою страждущего от наготы,
и сжалился над ним, и, украв плащ из дома отца моего,
тайно дал страждущему.
\vs Tzb 7:2
Так и вы, дети мои, милостиво уделяйте из того, что даёт вам Господь,
всем без различия, и давайте всякому человеку в доброте сердечной.
\vs Tzb 7:3
Если же не имеете, что подать нуждающемуся,
сострадайте ему сердцем своим.
\vs Tzb 7:4
Помню, как не нашла рука моя, что подать нуждающемуся,
и прошёл я с ним семь стадиев и плакал с ним вместе,
и сердце моё сотрясалось от сострадания к нему.
\vs Tzb 8:1
Так и вы, дети мои, добросердечны будьте со всяким человеком в милости,
дабы и Господь, сжалившись, помиловал вас.
\vs Tzb 8:2
Ибо в последние дни пошлёт Бог сердце своё на землю,
и где найдет сердце милостивое, поселится в нём.
\vs Tzb 8:3
Ибо как человек жалеет ближнего своего, так сжалится и Господь над ним.]

\vs Tzb 8:4
Когда же пришли мы в Египет, не вспомнил нам зла Иосиф.
\vs Tzb 8:5
Воззрев на него, дети мои, возлюбите и вы друг друга, и не замышляйте
зла каждый на брата своего.
\vs Tzb 8:6
Ибо это разделяет единое, и всякое родство уничтожает,
и душу возмущает, и лицо искажает.

\vs Tzb 9:1
Посмотрите на воды и узрите, что когда текут они вместе,
то камни, деревья, землю и иное сметают они.
\vs Tzb 9:2
Если же на много частей разделятся, земля поглотит их,
и станут они ничтожными.
\vs Tzb 9:3
И вы, если разделитесь, будете таковыми.
\vs Tzb 9:4
Не разделяйтесь же на две головы, ибо всё, что сотворил Господь,
одну голову имеет, а два плеча, две руки, две ноги
и все другие члены слушаются одной головы.
\vs Tzb 9:5
Узнал же я из Писаний отцов наших,
что разделитесь вы в Израиле,
и за двумя царями последуете, и всякую мерзость сотворите.

\vs Tzb 9:6
И возьмут вас в плен враги ваши,
и зло будет вам от язычников среди многой скорби и бессилия.
\vs Tzb 9:7
После того, вспомнив о Господе, обратитесь вы, и помилует он вас,
ибо он милостив и добр сердцем,
и не мыслит зла против сынов человеческих,
ведь они~--- плоть и блуждают во злых делах своих.
\vs Tzb 9:8
И после того взойдёт для вас сам Господь,
свет справедливости, и возвратитесь вы в землю вашу,
и узрите его в Иерусалиме, избранном ради имени его святого.
\vs Tzb 9:9
И вновь злобою дел ваших прогневаете его,
и отвержены будете им вплоть до конца времен.

\vs Tzb 10:1
И ныне, дети мои, не скорбите, что умираю я,
и не унывайте, когда отойду я.
\vs Tzb 10:2
Ибо снова восстану я среди вас,
как предводитель среди сынов своих,
и возрадуюсь среди тех из рода моего,
кто сохранит закон Господень и заповеди Завулона, отца своего.
\vs Tzb 10:3
На нечестивых же наведёт Господь огонь вечный,
и погубит их до потомства потомков их.
\vs Tzb 10:4
Я же ныне отхожу к покою, как и отцы мои отошли.
\vs Tzb 10:5
Вы же бойтесь Господа Бога нашего всеми силами вашими во все дни жизни вашей.

\vs Tzb 10:6
И сказав это, почил он сном прекрасным,
и положили его сыновья во гроб деревянный.
\vs Tzb 10:7
После же отнесли его и погребли в Хевроне с отцами его.

\bibbookdescr{Tdn}{
  inline={Завещание Дана,\\седьмого сына Иакова и Баллы},
  toc={Завещание Дана},
  bookmark={Завещание Дана},
  header={Завещание Дана},
  abbr={Дна}
}
\vs Tdn 1:1
Список слов Дана, речённых им к сыновьям своим
в последние дни его в 125-ый год жизни его.
\vs Tdn 1:2
Ибо, призвав семейство свое, сказал он:
услышьте, сыновья Дановы, слова мои и внемлите речам отца вашего.

\vs Tdn 1:3
Испытал я сердцем моим и всею жизнью моей,
что прекрасны и угодны Богу правда и совершение справедливых дел,
и что злы ложь и гнев, научающий человека злому.
\vs Tdn 1:4
Исповедуюсь вам сегодня, дети мои,
что положил я в сердце моём о смерти Иосифа, брата моего,
мужа доброго и правдивого.
\vs Tdn 1:5
Радовался я тому, что продали его, ибо возлюбил его отец более нас.
\vs Tdn 1:6
Сказал же мне дух зависти и гордыни: ты тоже сын его.
\vs Tdn 1:7
И один из духов Велиаровых возбуждал меня: возьми меч и убей им Иосифа,
и, когда умрёт он, возлюбит тебя отец.
\vs Tdn 1:8
Дух гнева убеждал меня задушить Иосифа, как барс душит козла.
\vs Tdn 1:9
Но Бог отцов моих не предал его в руки мои, дабы убил я его,
найдя одного, и уничтожил тем два скипетра в Израиле.

\vs Tdn 2:1
И ныне, дети мои, вот, я умираю и воистину говорю вам,
что если не сохраните себя от духа лжи и гнева
и не возлюбите правду и долготерпение, погибнете вы.
\vs Tdn 2:2
Ибо гнев есть ослепление, и не дозволяет он видеть ничьего лица в правде.
\vs Tdn 2:3
Ибо, если и отец то или мать, как на врагов взирает на них;
если и брат, не ведает; если и пророк Господа, не слушает;
если и праведник, не видит; если и друг, не узнаёт.
\vs Tdn 2:4
Ибо одолевает его дух гнева сетью соблазна,
и ослепляет очи его, и ложью помрачает помысел его,
и своё зрение даёт ему.
\vs Tdn 2:5
Чем же ослепляет он очи его?
Ненавистью сердечной, дабы завидовал брату своему.
\vs Tdn 3:1
Ибо зол гнев, дети мои, который душу возмущает.
\vs Tdn 3:2
И овладевает он телом того, кто гневается,
и властвует над душою его, и даёт силу телу,
да творит оно всякие беззакония.
\vs Tdn 3:3
Когда же сотворит всё это тело, оправдывает содеянное и душа,
ибо неправильно видит она.
\vs Tdn 3:4
Из-за того гневный, если он силён телом,
во гневе тройную силу обретает:
одну~--- от помощи помогающих ему,
вторую~--- от богатства, которым убеждает и побеждает он неправедно,
третью же~--- телесную, которой и творит он зло.
\vs Tdn 3:5
Если же слаб гневный, двойная сила от ярости у него возникает,
ибо помогает ему гнев постоянно в беззаконии.
\vs Tdn 3:6
Этот дух всегда с ложью от десницы Сатаны исходит,
дабы в жестокости и лжи творились дела его.
\vs Tdn 4:1
Так познайте же, что тщетна сила гнева.
\vs Tdn 4:2
Ибо в речах обостряется он сперва, потом в делах силу дает тому,
кто гневается, и вредом горьким возмущает рассудок его,
и так великим гневом возбуждает душу его.
\vs Tdn 4:3
И потому, если кто говорит против вас, не предавайтесь гневу,
и если кто станет восхвалять вас как святых, не превозноситесь.
И не переменяйтесь ни от наслаждения, ни от неприязни.
\vs Tdn 4:4
Ибо вначале услаждается слух, а оттого обостряется ум
и внимает возбуждению, и прогневавшись, думает человек,
что справедлива ярость его.
\vs Tdn 4:5
Если же вред какой или погибель приступают к вам, дети, не тревожьтесь,
ибо тот дух хочет волновать гибнущего, дабы пал он от возбуждения ярости.
\vs Tdn 4:6
И если претерпеваете страдания вольно или невольно, не огорчайтесь, ибо
от горя пробуждаются и гнев с ложью.
\vs Tdn 4:7
Ведь это двуликое зло~--- гнев с ложью,
и помогают они друг другу, возмущая сердце.
А когда возмущается душа постоянно,
отступает Господь от неё, и властвует над нею Велиар.

\vs Tdn 5:1
Храните же, дети мои, заповеди Господа, и закон его блюдите.
А от гнева отступите, и ложь возненавидьте, дабы поселился в вас Господь
и бежал от вас Велиар.
\vs Tdn 5:2
Правду говорите каждый ближнему своему и не впадайте в ярость и смятение,
но пребывайте в мире, имея Бога мира,
и не будет иметь война власти над вами.

\vs Tdn 5:3
Возлюбите Господа во всю жизнь вашу,
так же и друг друга сердцем правдивым.

\vs Tdn 5:4
Знаю я, что в последние дни отступите вы от Господа,
и вознегодуете на Левия, и выступите против Иуды,
но не сможете ничего против них.
Поведёт их обоих ангел Господень,
ибо на них будет стоять Израиль.
\vs Tdn 5:5
И когда отступите вы от Господа, живя во всяком зле,
сотворите мерзости языческие,
блуду предаваясь с жёнами беззаконников,
и всякое зло будут творить через вас духи злые.
\vs Tdn 5:6
[Ибо читал я книгу Еноха праведного и узнал,
что владыка ваш~--- Сатана,
и что духи злобы и гордыни присоветуют вам стать друзьями сынов Левия,
дабы заставить их согрешить перед Господом.
\vs Tdn 5:7
И приблизились сыны мои к сынам Левия,
и грешили с ними во всем, а сыны Иуды пребудут в алчности,
похищая чужое, словно львы.]
\vs Tdn 5:8
Оттого уведены будете [ими] в плен,
и там примете все казни Египетские и всё зло языческое.
\vs Tdn 5:9
Тогда, обратившись к Господу, помилованы будете,
и приведёт вас к святыне своей и даст вам мир.
\vs Tdn 5:10
И приведёт вам Господь спасение от [Иуды и] Левия,
и поведёт он войну против Велиара и отмщение воздаст за отцов ваших.
\vs Tdn 5:11
И заберёт пленных у Велиара [--- души святых],
и обратит сердца непокорные к Господу,
и даст призывающим его мир вечный.
\vs Tdn 5:12
И почиют в Едеме святые, и возрадуются праведные новому Иерусалиму,
славе Бога вечного.
\vs Tdn 5:13
И не будет более Иерусалим в запустении,
и не будет пленён Израиль, ибо Господь будет посреди него
[живя между людей], и Святой Израилев, царствующий в нём
[в унижении и в нищете, и верующий в него царствовать
будет над людьми воистину].

\vs Tdn 6:1
И ныне, бойтесь Господа, дети мои, и берегите себя от Сатаны и духов его.
\vs Tdn 6:2
Придите к Богу и ангелу, утешающему вас,
ибо он~--- посредник между Богом и людьми,
и за мир у Израиля восстанет он против царства Врага.
\vs Tdn 6:3
Оттого строит козни Враг всем, призывающим Господа.
\vs Tdn 6:4
Ибо знает Враг, что в тот день, когда обратится Израиль,
кончится царство его.
\vs Tdn 6:5
Сам ангел мира даст силу Израилю не впасть в погибель зла.
\vs Tdn 6:6
И будет во времена беззакония Израиля:
не отступится от них Господь, но придёт и к народам,
ищущим воли его, ибо не равен ему ни один из ангелов.
\vs Tdn 6:7
Имя же его во всяком месте Израиля и в народах.
\vs Tdn 6:8
Оберегайте же себя, дети мои, от всякого злого дела,
и отриньте от себя гнев и ложь, и возлюбите правду и долготерпение.
\vs Tdn 6:9
И что услышали от отца вашего, передайте и вы детям вашим,
[да примет вас Спаситель народов, ибо он правдив и долготерпелив,
кроток и смирен, и научает делами своими закону Господа.]
\vs Tdn 6:10
Так отступите от всякой неправедности, и прилепитесь к справедливости
Божией, и будет род ваш спасён навеки.
\vs Tdn 6:11
А меня похороните рядом с отцами моими.

\vs Tdn 7:1
И сказав это, поцеловал он их и почил сном прекрасным.
\vs Tdn 7:2
И погребли его сыновья его.
А после того отнесли кости его туда,
где погребены Авраам, Исаак и Иаков.
\vs Tdn 7:3
[Также пророчествовал им Дан,
что забудут они Бога своего и лишатся земли удела своего,
и рода семени своего.]

\bibbookdescr{Tnf}{
  inline={Завещание Неффалима,\\восьмого сына Иакова и Баллы},
  toc={Завещание Неффалима},
  bookmark={Завещание Неффалима},
  header={Завещание Неффалима},
  abbr={Неф}
}
\vs Tnf 1:1
Список завещания Неффалима,
данного им в час кончины его в 130-ый год жизни его.
\vs Tnf 1:2
Когда собрались сыновья его в 7-ой месяц 1-го числа, устроил им пир.
\vs Tnf 1:3
И пробудившись наутро, сказал он им:
я умираю.
И они не поверили ему.
\vs Tnf 1:4
И восславив Господа, собрал он силы и сказал:
после пира, бывшего вчера, умерла плоть моя.

\vs Tnf 1:5
И начал говорить:
слушайте, дети мои, сыновья Неффалима, слушайте слова отца вашего.
\vs Tnf 1:6
Я родился от Баллы, ибо хитрость сотворила Рахиль,
и вместо себя дала Баллу Иакову,
и та зачала и родила меня на колени Рахили,
и оттого наречено мне было имя Неффалим.
\vs Tnf 1:7
Премного возлюбила меня Рахиль, ибо на колени её родился я,
и когда был я ещё мал, целовала меня, говоря:
да будет мне дан брат твой от чрева моего, такой, как ты.
\vs Tnf 1:8
Оттого сходен был со мною во всем Иосиф по мольбам Рахили.
\vs Tnf 1:9
Мать же моя Балла была дочерью Руфея, брата Деворы,
кормилицы Ревекки, и родилась в один день с Рахилью.
\vs Tnf 1:10
Руфей же был из рода Авраама, Халдей,
чтущий Бога, свободный и знатный.
\vs Tnf 1:11
И попав в плен, был он куплен Лаваном,
и тот дал ему в жёны Енан, служанку свою,
которая родила дочь и нарекла ей имя Зелфа по имени того города,
где Руфей был взят в плен.
\vs Tnf 1:12
После же родила она Баллу и сказала:
к новому торопится дочь моя,
ибо родилась быстро и, взяв грудь,
сразу принялась сосать.

\vs Tnf 2:1
Был я лёгок ногами, словно серна, и поручал мне всякую весть
отец мой Иаков, и как серну благословил меня.
\vs Tnf 2:2
Как знает гончар сосуд, сколько вмещает он,
и сообразно с этим берёт глину для него,
так же и Господь в согласии с духом творит тело,
а по силе телесной влагает дух.
\vs Tnf 2:3
И нет расхождения ни на треть волоса,
ибо всё творение весами, и мерою, и правилом совершается.
\vs Tnf 2:4
И как знает гончар пользу всякого сосуда, для чего он пригоден,
так же и Господь знает тело,
до какого предела пребывает оно в добре,
а когда ко злу переходит.
\vs Tnf 2:5
Ибо нет творения и никакого помысла нет,
которых не ведал бы Господь.
Ибо всякого человека сотворил он по образу своему.
\vs Tnf 2:6
И какова сила его, таково и дело его;
каково око его, таков и сон его;
какова душа его, таково и слово его~--- либо по закону Господа,
либо по закону Велиара.
\vs Tnf 2:7
И как различают между светом и тьмою,
между зрением и слухом, так и между мужем и мужем различают,
и между женщиной и женщиной, и нельзя сказать,
что один подобен другому лицом или помыслом.
\vs Tnf 2:8
Ибо всё сделал Бог прекрасно в порядке своём:
5 чувств поместил в голове,
и горло приладил к голове,
и волосы на ней взрастил для красоты и славы;
после сердце сотворил для рассуждения,
желудок~--- для пищеварения,
чрево~--- для очищения тела,
горло~--- для дыхания,
печень~--- для гнева,
желчь~--- для огорчения,
селезёнку~--- для веселья,
почки~--- для разумения,
бока~--- для сна,
бёдра~--- для мощи,
и так далее.
\vs Tnf 2:9
Так да будут, дети мои, все дела ваши в порядке своём,
и в добром помышлении,
и в страхе Божием,
и ничего безрассудного не делайте в небрежении,
или же не в свой час.
\vs Tnf 2:10
Ибо если скажешь оку: слушай,~--- не сможет оно.
Так и вы, пребывая во тьме, не сможете творить дела света.

\vs Tnf 3:1
Так не стремитесь в любостяжании погубить дела ваши,
и словами пустыми не обманывайте душ ваших,
ибо молча, в чистоте сердца узрите,
как поддержать волю Божию, а волю Велиара отвергнуть.
\vs Tnf 3:2
Солнце, луна и звёзды не меняют порядка своего;
так и вы не меняйте закона Божьего,
лишая порядка дела ваши.
\vs Tnf 3:3
Язычники, заблуждаясь и отвергая Господа,
изменили порядок свой,
стали слушаться они дерева и камня,
духов соблазна.
\vs Tnf 3:4
Вы же не делайте так, дети мои;
узнавайте Господа на небосводе,
на земле, на море и во всех творениях его,
создавшего всё,
дабы не уподобиться вам Содому,
изменившему строй естества своего.
\vs Tnf 3:5
Так же и Стражи изменили строй естества своего,
и низверг их Господь потопом,
из-за них сделав землю лишённой поселений и плодов.

\vs Tnf 4:1
То говорю я вам, дети мои, что узнал я из писаний Еноха,
что и вы отступитесь от Господа,
и станете жить во всём беззаконии языческом,
и сотворите всё зло Содомское.
\vs Tnf 4:2
И наведёт на вас Господь пленение,
и рабами будете врагам вашим,
и всякой беде и горю подвергнетесь,
пока не избавит Господь всех вас.
\vs Tnf 4:3
Когда уменьшитесь вы и умалитесь,
обратитесь вы и познаете Бога вашего,
и он возвратит вас в землю вашу по великому милосердию своему.
\vs Tnf 4:4
И будет: пришедшие в землю отцов своих вновь забудут Господа
и совершат нечестия.
\vs Tnf 4:5
И рассеет их Господь по лицу всей земли,
пока не придет милосердие Господне,~--- Человек,
справедливость творящий и милость всем дальним и ближним.
\vs Tnf 5:1
Ибо в 40-ой год жизни моей узрел я видение
на горе Елеонской к востоку от Иерусалима,
что солнце и луна остановились.
\vs Tnf 5:2
И вот, Исаак, отец отца моего, сказал нам:
бегите и возьмите каждый по силе своей,
и получит овладевший солнце и луну.
\vs Tnf 5:3
И побежали все разом, и Левий овладел солнцем,
а Иуда успел взять луну,
и возвысились оба с тем, что взяли они.
\vs Tnf 5:4
И когда Левий стал как солнце, вот,
некий юноша дал ему 12 ветвей пальмовых,
а Иуда стал светел как луна,
и было под ногами их 12 лучей.
\vs Tnf 5:5
[И побежали оба, Левий и Иуда, и взяли их себе.]
\vs Tnf 5:6
И вот, явился на земле бык, имеющий 2 рога великих
и крылья орла на спине своей;
и когда хотели мы взять его, не смогли.
\vs Tnf 5:7
Иосиф же, придя, схватил его и взошел с ним на высоту.
\vs Tnf 5:8
И увидел я, что был я там,
и вот, святое писание увидели мы,
говорящее:
Ассирийцы, Мидяне, Персы, Халдеи, Сирияне
унаследуют пленение 12-ти скипетров Израиля.

\vs Tnf 6:1
И опять, через 5 дней, узрел я,
что отец мой Иаков стоит на море Ямнийском, и мы с ним.
\vs Tnf 6:2
И вот, корабль подошёл, плывущий без моряков и рулевых,
и написано было на нём, что это корабль Иакова.
\vs Tnf 6:3
И сказал нам отец наш: взойдем на корабль наш.
\vs Tnf 6:4
Когда же взошли мы, сделалась сильная буря и вихрь великий,
и ушёл от нас отец наш, державший руль.
\vs Tnf 6:5
А мы, гонимые бурей, носились по морю,
и наполнился корабль водою, и заливали его валы огромные,
и от них рассыпался он.
\vs Tnf 6:6
И поплыл Иосиф в лодке, а мы разделились на 10 досок.
Левий же и Иуда были на одной доске.
\vs Tnf 6:7
И разметало нас всех по разным концам земли.
\vs Tnf 6:8
И Левий, облачившись во вретище, молился Господу.
\vs Tnf 6:9
Когда же утихла буря, прибыло судно к земле в мире.
\vs Tnf 6:10
И вот, пришёл отец наш, и все мы вместе возвеселились.

\vs Tnf 7:1
2 этих сна поведал я отцу моему, и сказал он мне:
должно тому исполниться в свои времена,
когда многое вынесет Израиль.
\vs Tnf 7:2
Потом сказал отец мой:
верю я Богу, что жив Иосиф, ибо всечасно вижу я,
что числит его с живыми Господь.
\vs Tnf 7:3
И сказал плача: увы, дитя мое Иосиф, ты жив,
a я не вижу тебя, и ты не видишь Иакова,
породившего тебя.
\vs Tnf 7:4
От этих слов его заплакал и я;
и воспылал я сердцем моим возвестить,
что продан был Иосиф, но побоялся я братьев моих.

\vs Tnf 8:1
И вот, дети мои, показал я вам последние времена,
когда всё совершится в Израиле.
\vs Tnf 8:2
А вы поведайте о том детям вашим,
дабы они едины были с Левием и с Иудою.
Ибо через них придёт спасение Израилю,
и в них благословен будет Иаков.
\vs Tnf 8:3
От скипетра их явится Бог [живущий в людях] на земле,
дабы спасти род Израиля и привести к нему праведных из язычников.
\vs Tnf 8:4
И если вы также будете делать добро,
благословят вас люди и ангелы,
и через вас прославлен будет Бог среди народов,
а дьявол бежит от вас, и звери убоятся вас,
и Господь возлюбит вас [и ангелы обнимут вас].
\vs Tnf 8:5
Вырастивший доброго сына стяжает память добрую,
так же и память добрая о благом деле у Бога пребывает.
\vs Tnf 8:6
Того же, кто сотворит недоброе, проклянут его и ангелы,
и люди, а Бога хулить станут язычники из-за него,
а дьявол поселится в нём, как в орудии своём,
а всякий зверь власть будет иметь над ним,
и возненавидит его Господь.
\vs Tnf 8:7
Заповеди же закона двояки, и нужно искусство для их исполнения.
\vs Tnf 8:8
Ибо время сообщению с женой, и время воздержанию для молитвы.
\vs Tnf 8:9
И обе заповеди~--- от Бога,
и если бы не исполнялись они в порядке своём,
грех великий учинялся бы людьми.
Так же и с остальными заповедями.
\vs Tnf 8:10
Будьте же мудры в Боге, дети мои, и благоразумны,
видя порядок заповедей его и законы всех дел,
дабы возлюбил вас Господь.

\vs Tnf 9:1
И много подобного завещав им, просил,
чтобы отнесли кости его в Хеврон и погребли там с отцами его.
\vs Tnf 9:2
И вкушал он, и пил в веселии души, после же закрыл лицо своё и умер.
\vs Tnf 9:3
И сделали сыновья его всё так, как завещал им Неффалим, отец их.

\bibbookdescr{Tgd}{
  inline={Завещание Гада,\\девятого сына Иакова и Зелфы},
  toc={Завещание Гада},
  bookmark={Завещание Гада},
  header={Завещание Гада},
  abbr={Гад}
}
\vs Tgd 1:1
Список завещания Гада,
речённого им к сыновьям своим в 125-ый год жизни его.
Сказал он им:
\vs Tgd 1:2
послушайте, дети мои:
родился я 9-ым сыном Иакова и смелым был на пастбищах.
\vs Tgd 1:3
Стерёг я по ночам стадо, и когда приходил лев,
или волк, или другой зверь на пастбище,
преследовал я его, и настигал, и ловил за ногу его рукою моей,
и бросал его словно камень, и убивал его.
\vs Tgd 1:4
Иосиф же, брат мой, пас стадо вместе с нами около 30-ти дней и,
будучи молод, занемог от зноя.
\vs Tgd 1:5
И возвратился он в Хеврон, к отцу нашему;
и положил его тот рядом с собой,
ибо весьма любил его.

\vs Tgd 1:6
И сказал Иосиф отцу нашему:
сыновья Зелфы и Баллы приносят жертвы
из добрых животных и поедают их, а Рувим и Иуда того не ведают.
\vs Tgd 1:7
Видел же он, как вытащил я барана из пасти медведицы,
и её убил, а барана принес в жертву:
горевали мы, что не может он жить, и съели его.
\vs Tgd 1:8
И оттого гневался я на Иосифа вплоть до дня, когда был он продан.
\vs Tgd 1:9
И был во мне дух ненависти, и не желал я ни слышать об Иосифе,
ни видеть его, ибо в лицо укорял он нас,
говоря, что без Иуды едим мы животных.
Ибо всему, что говорил он, верил отец.

\vs Tgd 2:1
Исповедуюсь ныне в грехе моём, дети,
ибо многократно желал я убить Иосифа, возненавидев его душою.
\vs Tgd 2:2
И за сон его обратил я на него ненависть,
и хотел его извести с земли (живого),
как телец изводит траву на поле.
\vs Tgd 2:3
Но Иуда и я продали его Измаильтянам за 30 золотых, спрятав 10 и показав
только 20 своим братьям.
\vs Tgd 2:4
И так из-за жадности наш замысел был приведён в исполнение.
\vs Tgd 2:5
Так Бог отцов наших избавил его от рук моих,
дабы не сотворил я беззакония великого в Израиле.

\vs Tgd 3:1
А ныне услышьте слово правды:
творите справедливость,
и делайте всё по закону Всевышнего, 
и не соблазняйтесь духом ненависти,
ибо зло это во всех делах человеческих.
\vs Tgd 3:2
Всё, что творит человек, мерзостно для ненавидящего:
если делает по закону Господа, не хвалит его,
если боится Господа и желает справедливости, не любит его.
\vs Tgd 3:3
Правду порицает, счастливому завидует,
злословию радуется, гордыню любит,
ибо ненависть ослепляет душу его, как и меня,
когда смотрел я на Иосифа.

\vs Tgd 4:1
Берегитесь же ненависти, дети мои,
ибо ненавидящий и против Господа беззаконие творит.
\vs Tgd 4:2
Ибо не хочет он слышать заповедей его о любви к ближнему,
и тем грешит против Бога.
\vs Tgd 4:3
Если падёт брат его,
стремится сразу возвестить всем и желает,
чтобы осуждённый и покаранный умер он.
\vs Tgd 4:4
Если же раб какой, клевещет на него перед господином его,
и радуется, если в мучениях умрёт он.
\vs Tgd 4:5
Ибо зависти содействует ненависть также и против счастливых:
видящий успех чей-то или слышащий о нём всегда изнемогает.
\vs Tgd 4:6
Как любовь мёртвых желает оживить и на смерть обречённых
воззвать к жизни хочет, так ненависть живых желает убить
и не хочет, чтобы лишь немного согрешившие живы были.
\vs Tgd 4:7
Ибо дух ненависти через малодушие содействует Сатане
во всём на погибель людей, а дух любви через
долготерпение закону Божию содействует во спасение людей.

\vs Tgd 5:1
И потому ненависть~--- зло,
что постоянно содействует она лжи, говоря против правды,
и малое великим делает, и свет тьмою представляет,
и о сладком говорит, что оно горько,
и клевете научает, и гнев возбуждает,
и войну поднимает, и гордыню, и всякую алчность,
а сердц\acc{а} злом и ядом дьявольским наполняет.
\vs Tgd 5:2
По опыту своему говорю вам это, дети мои, дабы изгнали вы ненависть
дьявольскую и Бога возлюбили.
\vs Tgd 5:3
Праведность изгоняет ненависть,
смирение убивает зависть,
ибо праведный и смиренный стыдится творить неправедное,
и не от того, что другой осудит его, а своё же сердце,
ибо видит Господь душу его.
\vs Tgd 5:4
Не станет говорить он против человека благочестивого,
ибо страх Божий живёт в нём.
\vs Tgd 5:5
Ибо страшась оскорбить Господа, вовек не пожелает
он обидеть и человека, даже и в мыслях своих.
\vs Tgd 5:6
Узнал в конце о том и я, когда раскаялся об Иосифе.
\vs Tgd 5:7
Ибо истинное обращение к Богу [убивает незнание и]
прогоняет тьму, и освещает очи, и знание дает душе,
и помыслы ведёт ко спасению.
\vs Tgd 5:8
И не от людей научился я этому, а в покаянии познал.
\vs Tgd 5:9
Навёл же на меня Бог болезнь печени,
и если бы не помогли мольбы отца моего,
испустил бы я, верно, дух мой.
\vs Tgd 5:10
Ибо чем человек грешит, тем он и карается.
\vs Tgd 5:11
Оттого, что была печень моя безжалостна к Иосифу,
был я осужден на страдание печени немилосердное
в течение 11 месяцев, по времени, что гневался я на Иосифа.

\vs Tgd 6:1
И ныне, дети мои, даю вам совет:
любите каждый ближнего своего, прогоняйте ненависть из сердец ваших.
Возлюбите друг друга делом, словом и помыслом душевным.
\vs Tgd 6:2
Ибо я пред лицом отца моего мирно говорил с Иосифом;
когда же вышел от него, дух ненависти помрачил мой разум
и смутил рассуждение мое, так что захотел я убить Иосифа.
\vs Tgd 6:3
Возлюбите друг друга от сердца, и если кто согрешит против тебя,
говори ему: мир тебе; и не затаи коварства в душе своей.
Если же, раскаявшись, признает он вину свою, отпусти ему.
\vs Tgd 6:4
Если же станет отрицать он, не вступай в спор с ним,
дабы не согрешить дважды, когда он начнёт ругаться.
\vs Tgd 6:5
Да не услышит во время тяжбы чужой человек тайн твоих,
дабы не возненавидел он тебя и не сделался врагом тебе,
и великий грех не сотворил тебе, ибо часто будет
он замышлять коварство и зло творить, вникая в дела твои.
\vs Tgd 6:6
Если же будет он отрицать и, уличённый во грехе,
устыдится, успокойся и не обличай его;
ибо раскается он, что согрешил против тебя,
и, устрашившись, пожелает жить в мире с тобой.
\vs Tgd 6:7
Если же нет в нём стыда и упорствует он во зле,
и тогда отпусти ему от сердца, а возмездие оставь Богу.

\vs Tgd 7:1
И если кто-либо счастливее вас, не огорчайтесь,
но молитесь за него, дабы и в конце был он счастлив,
ибо это полезно вам будет.
\vs Tgd 7:2
И если и далее он возвышается, не завидуйте ему,
помня, что всякая плоть умрет.
Хвалы же возносите Господу,
добро и счастье дающему людям.
\vs Tgd 7:3
Исследуйте суды Господа, и просветит он,
и успокоит помыслы ваши.
\vs Tgd 7:4
Если же кто злом богатеет, как Исав, брат отца моего,
не ревнуйте: ожидайте, что Господь положит предел.
\vs Tgd 7:5
Если отнимется злое богатство, и раскается человек,
простит Господь, а не раскается~--- предан будет на вечные муки.
\vs Tgd 7:6
А бедный, если без зависти радуется он всему,
что дает Господь, превыше всех богатеет,
ибо не ведает он суеты праздных людей.
\vs Tgd 7:7
Удалите же зависть от душ ваших и возлюбите друг друга в прямоте сердца.

\vs Tgd 8:1
Скажите это и вы детям вашим, дабы чтили они Левия и Иуду,
ибо от них восставит Господь спасение Израилю.
\vs Tgd 8:2
Ибо познал я, что отступятся дети ваши от них,
и во всяком зле, вреде и порче будут пред Богом.

\vs Tgd 8:3
И отдохнув немного, сказал ещё:
дети мои, послушайте отца вашего,
и похороните меня рядом с отцами моими.
\vs Tgd 8:4
И вытянув ноги, почил в мире.
\vs Tgd 8:5
И спустя 5 лет отнесли его в Хеврон и положили рядом с отцами его.

\bibbookdescr{Tas}{
  inline={Завещание Асира,\\десятого сына Иакова и Зелфы},
  toc={Завещание Асира},
  bookmark={Завещание Асира},
  header={Завещание Асира},
  abbr={Аср}
}
\vs Tas 1:1
Список завещания Асира, данного им сыновьям его в 125-ый год жизни его.
\vs Tas 1:2
Будучи здоров, говорил он им:
послушайте, дети Асира, отца вашего,
и всё прямое пред лицом Господа покажу вам.

\vs Tas 1:3
2 пути дал Бог сынам человеческим,
и 2 помысла, и 2 дела, и 2 способа, и 2 исхода.
\vs Tas 1:4
Оттого все по 2 одно против другого.
\vs Tas 1:5
Ибо есть 2 пути доброго и злого, и 2 помышления о них в груди нашей,
различающие их.
\vs Tas 1:6
Если желает душа быть доброй,
все дела свои творит она в справедливости,
а если и согрешит, тотчас же кается.
\vs Tas 1:7
Помышляя праведное и отвергая худое,
тотчас же истребляет она зло и с корнем вырывает грех.
\vs Tas 1:8
Если же к худому клонится помышление души,
всякое дело её во зле, и отвергает она добро,
и прилепляется ко злу, и властвует над нею Велиар;
а если и доброе творит, во зло его обращает.
\vs Tas 1:9
Когда начинает творить добро,
исход дела того злым бывает,
ибо сокровище помышления злым духом наполняется.

\vs Tas 2:1
Бывает, что душа на словах доброе выше злого ставит,
но исход дела ее злой.
\vs Tas 2:2
Бывает, что человек не щадит тех,
кто в недобром ему помогает,
и это двулико, но всё в целом~--- зло.
\vs Tas 2:3
Бывает, что человек возлюбит делающего зло,
так что и умереть во зле согласится ради него, и ясно,
что это двулико, но всё в целом~--- злое дело.
\vs Tas 2:4
И если и есть любовь, во зле тот,
кто скрывает злое под именем доброго;
исход же дела недобрый.

\vs Tas 2:5
Иной крадёт, обижает, грабит, корыстолюбив,
но бедных жалеет; и это двулико, но всё в целом~--- зло.
\vs Tas 2:6
Отнимающий у ближнего своего гневит Бога,
ложно клянётся Всевышним,
а нищего жалеет.
Наставляющего в законе Господнем гонит и хулит,
а бедняку подаёт помощь.
\vs Tas 2:7
Душу пятнает он, а тело украшает, многих убивает,
а немногих жалеет, и это двулико, а всё в целом~--- зло.

\vs Tas 2:8
Иной предаётся блуду и разврату, а от пищи воздерживается;
и в посте злые дела творит, и силою богатства многих притесняет,
а наставления даёт несмотря на великое зло своё;
и это двулико, всё же вместе~--- зло.
\vs Tas 2:9
Такие люди~--- как зайцы, ибо наполовину чисты они,
но по правде нечисты.
\vs Tas 2:10
Ибо так сказал Бог на скрижалях заповедей.

\vs Tas 3:1
Вы же, дети мои, сами не будьте двуликими~--- и добрыми,
и злыми вместе, но к одной доброте прилепитесь,
ибо в ней обитает Господь Бог,
и люди её желают.
\vs Tas 3:2
А зла убегайте, убивая помышление злое делами добрыми,
ибо двуликие служат не Богу, но страстям своим,
дабы угодить Велиару и людям, подобным себе.

\vs Tas 4:1
А люди добрые и одноликие праведны пред Богом,
если и говорят двуликие, что согрешают они.
\vs Tas 4:2
Многие убивающие злых 2 дела совершают~--- доброе и злое,
но всё в целом~--- добро, ибо гибнет вырванное с корнем зло.
\vs Tas 4:3
Ненавидящий того, кто и милостив и неправеден вместе,
и блудит и постится вместе, также двуликое совершает,
но всё дело его~--- доброе;
ибо он уподобляется Господу, не принимая за истинное добро то,
что добрым только кажется.
\vs Tas 4:4
Иной же не хочет видеть дня праздничного с распутными,
дабы не осрамить тела своего и не запятнать души своей,
и это двулико, но в целом~--- добро.
\vs Tas 4:5
Такие люди оленям и ланям подобны, ибо они,
имея обличье диких зверей, кажутся нечистыми,
но в целом~--- чисты.
Ведь в ревности Господней живут они, удаляясь от того,
что и Бог возненавидел и запретил заповедями своими,
отделяя доброе от злого.

\vs Tas 5:1
Смотрите, дети, что во всём есть
2 стороны~--- одна противоположна другой,
и одна за другой сокрыта:
в приобретении~--- любостяжательство,
в радости~--- опьянение,
в веселии~--- скорбь,
в браке~--- распутство.
\vs Tas 5:2
Жизни следует смерть,
славе~--- бесчестие,
дню~--- ночь,
свету~--- тьма
и всё под днём, под жизнью~--- праведное, а под смертью~--- неправедное.
Оттого и за смертью грядёт жизнь вечная.
\vs Tas 5:3
И нельзя назвать правду ложью,
или праведное~--- неправедным,
ибо всякая правда~--- в свете, как всё~--- под Богом.
\vs Tas 5:4
Всё это испытал я в жизни моей,
и не уклонялся от правды Господней,
и заповеди Всевышнего изучал,
и был одноликим, всею силою души моей стремясь к добру.
\vs Tas 6:1
Следуйте и вы, дети мои, заповедям Господа,
и будьте одноликими, следуя правде.
\vs Tas 6:2
Ибо двуликие двоякий грех совершают,
ибо и делают злое, и одобряют делающих,
подражая духам соблазна и борясь против людей.
\vs Tas 6:3
Вы же, дети мои, храните закон Господа,
и не внимайте злу, схожему с добром,
а взирайте на то, что сутью своей благо,
и его блюдите по всем заповедям Господним,
в нём пребывая и почивая.
\vs Tas 6:4
Ибо конец жизни человека являет праведность его,
и встречает он либо ангелов Господних, либо Велиаровых.
\vs Tas 6:5
Когда смятенная душа отходит,
обличается она злым духом,
ибо человек тот был рабом страстей и дурных дел.
\vs Tas 6:6
Если же спокойна душа,
в радости узнаёт она ангела мира,
и ведёт он её в жизнь вечную.

\vs Tas 7:1
Не уподобляйтесь Содому,
не узнавшему ангелов Господних
и погибшему навечно.
\vs Tas 7:2
Ибо знаю я, что согрешите вы и преданы
будете в руки врагов ваших,
и земля ваша запустеет, и святыни ваши разрушатся,
вы же рассеяны будете по 4-ём углам земли,
и будете в рассеянии презираемы как вода бесполезная.
\vs Tas 7:3
До той поры будет это, когда посмотрит Всевышний на землю,
и сам придёт как человек, с людьми вкушающий и пьющий,
и снесёт голову дракона в воде, и избавит он Израиля и все народы
Бог, в человека облёкшийся.
\vs Tas 7:4
Скажите же, дети мои, и вы детям вашим об этом,
дабы не ослушались его.
\vs Tas 7:5
Ибо узнал я, что ослушаетесь вы и пребудете в нечестии,
внимая не закону Божию, но советам людским,
совращаясь во зле.
\vs Tas 7:6
И за то разделены будете вы, подобно Гаду и Дану, братьям моим,
чьей земли, рода и языка не узн\acc{а}ете.
\vs Tas 7:7
Но вновь восставит он вас в вере милосердием своим
и ради Авраама, Исаака и Иакова.

\vs Tas 8:1
И сказав это, завещал им: похороните меня в Хевроне.
И умер, почив сном прекрасным.
\vs Tas 8:2
И сделали сыновья его, как завещал он им,
и отнесли его в Хеврон, и погребли там с отцами его.

\bibbookdescr{Tjs}{
  inline={Завещание Иосифа,\\одиннадцатого сына Иакова и Рахили},
  toc={Завещание Иосифа},
  bookmark={Завещание Иосифа},
  header={Завещание Иосифа},
  abbr={Исф}
}
\vs Tjs 1:1
Список завещания Иосифа.
Когда собрался он умирать, то,
призвав сыновей и братьев своих,
сказал им:
\vs Tjs 1:2
братья мои и дети мои,
послушайте Иосифа, возлюбленного Израиля, внемлите речам уст моих.
\vs Tjs 1:3
Видел я в жизни моей зависть и смерть.
И не соблазнился, но пребывал в правде Господней.
\vs Tjs 1:4
Братья мои возненавидели меня, Господь же возлюбил меня.
Они желали меня убить, но Бог отцов моих сохранил меня.
В колодец меня бросили, но Всевышний вывел меня оттуда.
\vs Tjs 1:5
Продан я был в рабство, но Владыка над всеми освободил меня.
Был я в плену, но могучая рука его помогла мне.
Голод мучил меня, но сам Господь накормил меня.
\vs Tjs 1:6
Одинок я был, и Бог утешил меня;
занемог, и Господь посетил меня,
в темнице был я, и Бог мой смиловался надо мною;
горькие слова слышал от Египтян, и избавил меня;
рабом был и возвысил меня.

\vs Tjs 2:1
И главный повар фараона доверил мне дом свой.
\vs Tjs 2:2
И боролся я с женщиной бесстыдной, склонявшей меня согрешить с нею,
но Бог отцов моих избавил меня от огня пылающего.
\vs Tjs 2:3
Заключили меня в темницу, били и насмехались надо мною,
но дал мне Господь благорасположение тюремщика.
\vs Tjs 2:4
Ибо не оставляет Господь боящихся его ни во тьме, ни в оковах,
ни в скорби, ни в нужде.
\vs Tjs 2:5
Ведь не стыдится Бог подобно человеку, и не робеет подобно сыну
человеческому, и не бежит в страхе подобно землеродному.
\vs Tjs 2:6
Но во всех этих печалях помогает он и различными способами утешает,
и лишь ненадолго отступает от человека,
дабы испытать помышление души его.
\vs Tjs 2:7
10-ти испытаниям подверг он меня,
и все их выдержал я терпеливо.
Ибо великое средство~--- долготерпение,
и много благого даёт стойкость.

\vs Tjs 3:1
Сколько раз угрожала мне смертью Египтянка!
Сколь часто, предав меня пыткам, звала к себе,
а когда не хотел я сойтись с нею, говорила мне:
\vs Tjs 3:2
будешь владыкою надо мною и надо всем, что есть в доме моём,
если предашь себя мне, и будешь ты как хозяин наш.
\vs Tjs 3:3
Я же памятовал о словах отца моего и,
войдя в комнату, плача молил Господа.
\vs Tjs 3:4
И постился я тогда 7 лет,
а Египтянам казалось, что живу я в роскоши.
Ибо постящиеся ради Господа радость на лице являют.
\vs Tjs 3:5
Когда же отсутствовал господин мой, не пил я вина
и раз в 3 дня принимал пищу,
а остальное отдавал бедным и слабым.
\vs Tjs 3:6
И на рассвете обращался я к Господу
и плакал о Египтянке из Мемфиса,
ибо непрестанно и премного беспокоила она меня.
Ибо и ночью подходила она ко мне под тем предлогом,
что желает проведать меня.
\vs Tjs 3:7
И поскольку не было у неё ребёнка мужского пола,
делала она так, будто я~--- сын её.
\vs Tjs 3:8
И до времени как сына меня обнимала, а я не знал того.
После же захотела она во блуд вовлечь меня.
\vs Tjs 3:9
И когда понял, опечалился я до смерти.
И когда удалилась она, пошёл я к себе
и горевал о ней многие дни,
ибо познал я хитрость её и соблазн.
\vs Tjs 3:10
И говорил я ей слова Всевышнего,
дабы отвратилась она от страсти злой.

\vs Tjs 4:1
И часто льстила она мне речами своими как святому мужу,
и хитрыми словами хвалила чистоту мою пред лицом мужа своего,
желая ввести меня в искушение,
когда будем мы одни.
\vs Tjs 4:2
Явно прославляла она целомудрие мое,
а втайне говорила мне: не бойся мужа моего, ибо он убеждён
в целомудрии твоём;
и если кто скажет ему о нас, не поверит он.
\vs Tjs 4:3
Тогда я, пав на землю, молил Бога,
чтобы избавил он меня от коварства её.
\vs Tjs 4:4
Когда же ничего не достигла она,
снова приходила ко мне как бы наставления ради,
дабы слушать слово Божие.
\vs Tjs 4:5
И говорила мне:
если хочешь, чтобы оставила я идолов,
сойдись со мною, а я сумею убедить мужа моего отречься от них,
и будем жить пред лицом Господа твоего.
\vs Tjs 4:6
Я же отвечал ей, что не хочет Господь,
чтобы в нечистоте почитали его,
и не развратникам благоволит он,
но только тем, кто с чистым сердцем 
и устами незапятнанными приходит к нему.
\vs Tjs 4:7
Она же была рассержена и желала исполнить желание своё.
\vs Tjs 4:8
А я предался посту и молитве, дабы избавил меня Господь от неё.

\vs Tjs 5:1
И вновь, в иное время сказала она мне:
если блудить не желаешь,
тогда убью я мужа моего ядом и возьму тебя в мужья.
\vs Tjs 5:2
Я же, услышав это, разодрал одежды мои и сказал ей:
женщина, постыдись Бога и не сотвори дела этого злого,
дабы не погибнуть тебе.
Ибо знай, что я разглашу всем этот твой умысел.
\vs Tjs 5:3
Она же, убоявшись, молила меня,
чтобы не разглашал я замысла того.
\vs Tjs 5:4
И удалилась она, ублажив меня дарами и услаждениями всяческими.

\vs Tjs 6:1
А после того послала мне кушанья, намешав в них колдовское зелье.
\vs Tjs 6:2
Но когда пришел евнух и принес кушанья, взглянул я и увидел
испуганного мужа, подающего мне блюдо и нож;
и понял я, что делается это, дабы соблазнить меня.
\vs Tjs 6:3
И когда вышел он, плакал я и не испробовал ни этого,
ни другого какого-либо из кушаний её.
\vs Tjs 6:4
Через день же пришла она ко мне и, увидев, сказала мне:
отчего не отведал ты кушанья?
\vs Tjs 6:5
И отвечал я ей:
оттого, что наполнила ты его зельем смертельным;
и как говорила ты, что, мол, не приближусь я к идолам,
а к одному только Господу?
\vs Tjs 6:6
Ныне же знай, что Бог отца моего открыл мне
через ангела своего зло твоё, и сохранил я кушанье это,
дабы обличить тебя, и, увидев то, быть может, покаешься ты.
\vs Tjs 6:7
Но дабы узнала ты,
что против чтящих Бога в целомудрии не имеет
силы зло нечестивцев,
вот, возьму я от кушанья и съем пред тобою.
И сказав это, помолился я так:
да будет со мною Бог отцов моих и ангел Авраама.
И вкусил я.
\vs Tjs 6:8
Она же, узрев это, пала с плачем на лицо своё к ногам моим,
и поднял я её и вразумлял.
\vs Tjs 6:9
Она же обещала мне не творить никогда нечестия такого.

\vs Tjs 7:1
Но сердце её лежало ещё во зле, и смотрела она,
каким бы способом поймать меня в западню.
И стеная непрестанно, чахла она,
хоть и не была больна.
\vs Tjs 7:2
Увидев же это, сказал ей муж её:
отчего исхудало лицо твоё?
Она же отвечала ему:
страдаю я болью сердечной, и стенание духа мучает меня.
И утешал он её словами своими.
\vs Tjs 7:3
Она же, улучив удобное время, вбежала ко мне,
когда уже ушёл муж её, и сказала мне:
терзаюсь я, и если не возляжешь со мною, брошусь я со скалы.
\vs Tjs 7:4
Я же, поняв, что дух Велиаров мучит её,
обратился с мольбою к Господу и сказал ей:
\vs Tjs 7:5
Что ты, несчастная женщина, терзаешься и мятёшься,
ослеплённая грехом?
помни, что если убьёшь ты себя,
то Астифо, наложница мужа твоего и соперница твоя,
перебъёт всех детей твоих,
и исчезнет память о тебе на земле.
\vs Tjs 7:6
И сказала она мне: вот, всё же ты любишь меня.
Да будет мне довольно этого.
Только вступись за жизнь мою и детей моих,
а я буду ожидать, пока не услажу страсти моей.
\vs Tjs 7:7
Ибо не знала она, что ради Господа моего сказал я так,
а не ради неё.
\vs Tjs 7:8
Но кто одержим страстью желания и рабски служит ей,
как эта женщина, тот, если и доброе что услышит
ко страданию своему, относит это к страсти злой.

\vs Tjs 8:1
И вот, говорю, дети мои, что было около 6-го часа,
когда вышла она от меня.
И преклонив колени к Господу,
стоял я так весь день и всю ночь,
а на рассвете восстал,
плача и моля избавить меня от Египтянки.
\vs Tjs 8:2
И тогда, наконец, схватила она меня за одежды,
силою желая принудить меня сойтись с нею.
\vs Tjs 8:3
И увидев, что в безумии схватила меня за хитон,
оставил его ей и убежал нагим.
\vs Tjs 8:4
Она же, взяв хитон, ложно донесла на меня.
И муж её, придя, заключил меня под стражу
в доме своём и, побив бичами, отослал в темницу фараонову.
\vs Tjs 8:5
И когда был я в оковах, терзалась Египтянка от горя.
И, приходя, внимала она тому, как благодарил
я Господа и пел хвалы ему в доме тьмы и ликовал,
радостным голосом славя Бога моего, ибо избавил
он меня от Египтянки.

\vs Tjs 9:1
Она же часто посылала ко мне, говоря:
благоволи исполнить желание моё,
и я освобожу тебя из оков и от тьмы избавлю.
\vs Tjs 9:2
А я даже мыслию не склонился к ней.
Ведь больше любит Бог целомудренного,
который терпит тьму во рву,
нежели распутника, который роскошествует в царских палатах.
\vs Tjs 9:3
Если же тот, кто живёт в целомудрии, желает и славы,
и знает Всевышний, что это полезно ему,
подаст он, как подал и мне.
\vs Tjs 9:4
Сколько раз она, и будучи больной,
сходила ко мне по вечерам и слушала голос мой,
когда молился я;
я же, слыша стенания её, молчал.
\vs Tjs 9:5
И когда был я в доме её, обнажала она руки и бёдра свои,
дабы возлёг я с нею;
ибо она была прекрасна весьма и украшалась премного,
чтобы соблазнить меня.
И уберёг меня Господь от злых умыслов её.

\vs Tjs 10:1
Зрите же, дети мои, что творит терпение и молитва с постом.
\vs Tjs 10:2
Так и вы, если к целомудрию и чистоте стремиться будете
в терпении и молитве с постом,
в смирении сердечном, поселится в вас Господь,
ибо он любит целомудрие.
\vs Tjs 10:3
А там, где живёт Всевышний, если и зависть приступит,
или рабство, или клевета, Господь, живущий в том человеке,
за целомудрие не только избавит его от бед,
но и возвысит, и прославит, как и меня.
\vs Tjs 10:4
Ибо всякий человек прельщается или в делах, или в словах,
или в помыслах своих.

\vs Tjs 10:5
Знают братья мои, как возлюбил меня отец мой,
но нисколько не возносился я в мыслях моих,
и хоть был ещё ребёнком, страх Божий имел
в сердце моём, ибо знал, что всё прейдет.
\vs Tjs 10:6
И не восстал я в злобе против них,
но почтил их; и уважая их,
даже когда продали меня, умолчал пред Измаильтянами,
что я сын Иакова, мужа великого и праведного.

\vs Tjs 11:1
Так и вы, дети мои, имейте во всяком деле вашем
страх Божий перед очами и чтите братьев ваших,
ибо всякий творящий закон Божий возлюблен им будет.
\vs Tjs 11:2
И когда шёл я с Измаильтянами,
вопрошали они меня: раб ли ты?
И говорил я, что раб домашний,
дабы не опозорить братьев моих.
\vs Tjs 11:3
Говорил же мне старший из них:
ты не раб, ибо видно это по тебе.
Я же сказал им: я ваш раб.
\vs Tjs 11:4
Когда же пришли в Египет, спорили из-за меня,
кто из них даст золото и возьмёт меня.
\vs Tjs 11:5
И решили все, что должен я остаться в Египте
с перекупщиком товаров их, пока не вернутся они с товарами своими.
\vs Tjs 11:6
Господь же даровал мне милость в очах перекупщика того,
и доверил он мне дом свой.
\vs Tjs 11:7
И благословил его Бог рукою моей и обогатил его
золотом, серебром и имуществом.
\vs Tjs 11:8
И был я у него 3 месяца.

\vs Tjs 12:1
А в то время прибыла в пышности великой Мемфиянка,
жена Пентефриса, ибо услышала она обо мне от евнухов своих.
\vs Tjs 12:2
И сказала она мужу своему, что разбогател тот купец
руками некоего юного Еврея, и говорят, что украли его
из земли Ханаанской.
\vs Tjs 12:3
Сотвори же ныне суд и забери юношу в дом наш,
и благословит тебя Бог Еврейский,
ибо благодать небесная на юноше том.

\vs Tjs 13:1
Послушался Пентефрис слов её, и призвал к себе купца,
и сказал ему: что это слышу я о тебе, что крадёшь
ты души из земли Ханаанской,
и в рабы перепродаёшь их?
\vs Tjs 13:2
А купец пал к ногам его и стал умолять:
прошу тебя, господин, не знаю я, что ты говоришь.
\vs Tjs 13:3
И сказал ему Пентефрис: откуда же этот Еврейский юноша?
И отвечал тот: Измаильтяне отдали мне его до той поры,
когда возвратятся они.
\vs Tjs 13:4
И не поверил ему Пентефрис, но приказал раздеть его донага и бить.
Когда же оставался тот при словах своих, сказал Пентефрис:
да будет приведён юноша.
\vs Tjs 13:5
И войдя, поклонился я Пентефрису,
ибо он был 3-им в ряду владык после фараона.
\vs Tjs 13:6
И отведя меня в сторону, вопросил он:
раб ты или свободный?
Я же ответил: раб.
\vs Tjs 13:7
И вопросил он: чей?
И сказал я: Измаильтян.
\vs Tjs 13:8
Он же вопросил: как сделался ты рабом их?
И отвечал я: в земле Ханаанской купили они меня.
\vs Tjs 13:9
И сказал он мне: ты лжёшь.
И тотчас приказал бить и меня нагого.

\vs Tjs 14:1
А Мемфиянка видела через окно,
как били меня, ибо рядом был дом её,
и послала к Пентефрису, говоря:
неправеден суд твой, ибо свободного
и украденного наказываешь ты как преступника.
\vs Tjs 14:2
А я не отказывался от слов моих, хотя и били меня,
и приказал он охранять меня, пока, сказал он,
не придут хозяева юноши.
\vs Tjs 14:3
И сказала ему жена его: за что мучаешь ты и держишь
в оковах юношу, попавшего в плен, коего лучше
бы было освободить, дабы служил он тебе?
\vs Tjs 14:4
Ибо желала она видеть меня, чтобы совершить грех,
а я не знал ничего об этом.
\vs Tjs 14:5
И сказал ей муж её: не отнимают чужого Египтяне,
пока не совершится разбирательство.
\vs Tjs 14:6
А затем сказал купцу: юноша должен быть заключен в тюрьму.

\vs Tjs 15:1
Спустя же 24 дня пришли Измаильтяне;
ибо услышали они, что Иаков, отец мой, премного печалится обо мне.
И придя, сказали они мне: 
\vs Tjs 15:2
что же это ты назвал себя рабом?
И вот, узнали мы, что ты сын человека великого в земле Ханаанской,
и печалится о тебе отец твой во вретище и пепле.
\vs Tjs 15:3
Когда услышал я это, размягчилось и растаяло сердце моё,
и хотел я заплакать громко, но сдержал себя,
дабы не опозорить братьев моих, и сказал им:
ничего не знаю, раб я.
\vs Tjs 15:4
Тогда решили они продать меня, дабы не был я найден
в руках у них.
\vs Tjs 15:5
Ибо они страшились отца моего, как бы не пришёл он,
дабы отомстить им ужасно.
Слышали они, что велик он пред Богом и людьми.
\vs Tjs 15:6
Тут сказал им купец: избавьте меня от суда Пентефриса.
\vs Tjs 15:7
И пошли они и просили меня:
скажи, что за серебро был продан ты нам,
и он освободит нас от ответственности.

\vs Tjs 16:1
А Мемфиянка сказала мужу своему:
купи этого юношу, ибо я слышу, говорят,
что продают его.
\vs Tjs 16:2
И послала она евнуха к Измаильтянам с просьбой купить меня.
\vs Tjs 16:3
А евнух не купил меня, но возвратился и сказал госпоже своей,
что большую цену просят они за юношу.
\vs Tjs 16:4
И послала она евнуха обратно, говоря:
если и 2 мины просят они, дай им,
не жалей золота;
только купи юношу и приведи его ко мне.
\vs Tjs 16:5
И пошёл евнух и, отдав им 80 золотых, взял меня;
Египтянке же сказал он, что отдал 100.
\vs Tjs 16:6
А я знал о том, но промолчал, дабы не опозорить евнуха.

\vs Tjs 17:1
Смотрите же, дети мои, сколько пришлось перенести мне,
дабы не опозорить братьев моих.
\vs Tjs 17:2
И вы любите друг друга, и в долготерпении скрывайте
прегрешения друг друга.
\vs Tjs 17:3
Ибо радуется Бог единомыслию братьев и помыслу сердца благого,
стремящегося к добру.
\vs Tjs 17:4
Когда же пришли братья мои в Египет, знают они,
что возвратил я им серебро, и не укорял их,
и утешил их.
\vs Tjs 17:5
А после смерти Иакова, отца моего, ещё более возлюбил их и всё,
чего желали они, в изобилии делал им.
\vs Tjs 17:6
И не допускал я, чтобы горевали они хотя бы из-за самого малого,
и всё, что было в руке моей, давал им.
\vs Tjs 17:7
И сыновья их~--- мои сыновья, а мои сыновья~--- как рабы их,
и душа их~--- моя душа, и всякая боль их~--- моя боль,
и всякая истома их~--- моя болезнь,
и воля их~--- моя воля.
\vs Tjs 17:8
И не превозносился я среди них, хвалясь славой моей в мире,
но был среди них как один из малейших.

\vs Tjs 18:1
Если и вы, дети мои, жить будете по заповедям Господним,
возвысит вас Бог вовеки.
\vs Tjs 18:2
И если кто-либо пожелает зло сделать вам,
сотворите доброе дело и помолитесь за него,
и ото всякого зла избавлены будете вы Господом.
\vs Tjs 18:3
Ибо вот, видите вы, что за смирение и долготерпение
моё взял я в жёны себе дочь жреца Гелиопольского,
и 100 талантов золота дали мне с нею,
и сделал их Господь мой рабами моими.
\vs Tjs 18:4
И обличье прекрасное дал он мне превыше прекрасных в Израиле,
и до старости во здравии и в красоте хранил меня.
Ибо во всём был я подобен Иакову.

\vs Tjs 19:1
Услышьте же, дети мои, также и о сне, который видел я.
\vs Tjs 19:2
Видел я 12 оленей, которые паслись,
и 9 из них были рассеяны по всей земле,
3 же спаслись, но на следующий день и они были рассеяны.
\vs Tjs 19:3
И узрел я, что 3 оленя сделались 3-мя агнцами и возопили к Господу,
и привёл их Господь на место цветущее и водою обильное
и вывел из тьмы на свет.
\vs Tjs 19:4
И тут возопили к Господу 9 оленей,
потом собрались они и стали как 12 овец
и в недолгом времени увеличились
и стали многими стадами.
\vs Tjs 19:5
После того взглянул я, и вот, явилось 12 быков,
сосущих одну телицу, которая море молока давала,
и пили от неё 12 стад и бесчисленные стада.
\vs Tjs 19:6
И у 4-го быка выросли рога до неба и стали
как стена для стад, а между двух рогов вырос иной рог.
\vs Tjs 19:7
И узрел я тельца, который двенадцатикратно окружил их,
и подал он помощь всем быкам.
\vs Tjs 19:8
И увидел я среди рогов некую деву,
имеющую пёструю одежду,
и от неё произошёл агнец,
и слева от него~--- лев,
и пошли против него все звери и все гады,
и победил их агнец, и погубил их.
\vs Tjs 19:9
И радовались ему быки, и телица, и ангелы, и вся земля.
\vs Tjs 19:10
И должно тому быть в последние дни.

\vs Tjs 19:11
Вы же, дети мои, храните заповеди Господа и чтите Левия и Иуду,
ибо от семени их придёт агнец Божий, дабы принять на себя грех мира,
Спаситель всех народов и Израиля.
\vs Tjs 19:12
Ибо царствие его будет вечным, и не прейдёт оно.
Моего же царства, которое в вас, не станет,
словно сторожки в саду, что уничтожается по прошествии лета.

\vs Tjs 20:1
Знаю я, что после кончины моей притеснять будут вас Египтяне,
и Бог отомстит за вас и приведёт вас к обещанному отцам моим.
\vs Tjs 20:2
Вы же возьмите с собою кости мои, ибо,
когда понесёте вы туда эти кости,
будет с вами Господь в свете, а Велиар во тьме будет с Египтянами.
\vs Tjs 20:3
А мать свою Асинефу отведите к Ипподрому и похороните её рядом с Рахилью,
матерью моей.

\vs Tjs 20:4
И сказав это, вытянул он ноги свои и почил сном прекрасным.
\vs Tjs 20:5
И оплакал его весь Израиль и весь Египет в скорби великой.
Ибо и для Египтян был он как соплеменник их,
и добро им творил, помогая во всем и советом, и делом своим.
\vs Tjs 20:6
А когда вышли сыны Израиля из Египта,
взяли они с собою кости Иосифа и погребли их
в Хевроне с отцами его.
И было лет жизни его 110.

\bibbookdescr{Tbn}{
  inline={Завещание Вениамина,\\двенадцатого сына Иакова и Рахили},
  toc={Завещание Вениамина},
  bookmark={Завещание Вениамина},
  header={Завещание Вениамина},
  abbr={Внм}
}
\vs Tbn 1:1
Список слов Вениамина, которые сказал он сыновьям своим,
прожив 125 лет.
\vs Tbn 1:2
Поцеловав их, молвил он:
как Исаак родился у Авраама в старости его,
так же и я родился у Иакова.
\vs Tbn 1:3
А Рахиль, мать моя, родив меня, умерла, и я не имел молока.
Потому кормила меня Балла, служанка её.
\vs Tbn 1:4
Рахиль, родив Иосифа, 12 лет была неплодна,
и молила Господа, и постилась, и зачав, родила меня.
\vs Tbn 1:5
Ибо премного любил отец мой Рахиль
и желал видеть двоих сыновей от неё.
\vs Tbn 1:6
Оттого назван был я Вениамин, то есть сын дней.

\vs Tbn 2:1
Когда же пришёл я в Египет, узнал меня брат мой Иосиф,
и спросил он меня: что сказали братья мои отцу,
когда продали меня?
\vs Tbn 2:2
И сказал я ему:
вымазали они хитон твой кровью и отослали его отцу, говоря:
узнай, сына ли твоего этот хитон.
\vs Tbn 2:3
И сказал он мне: да, брат, ибо взяли меня Измаильтяне,
и один из них снял с меня хитон, дал мне какую-то одежду,
ударил бичом и велел бежать.
\vs Tbn 2:4
И пошёл он спрятать одежду мою,
и встретился ему лев и убил того Измаильтянина.
\vs Tbn 2:5
И те, кто был с ним, устрашились и продали меня другим людям.
\vs Tbn 2:6
И не солгали братья мои в словах своих.
Ибо Иосиф желал скрыть от меня дела братьев наших,
и позвав их к себе, сказал им:
\vs Tbn 2:7
не говорите отцу моему, что сделали вы мне,
но так скажите, как рассказал я Вениамину.
\vs Tbn 2:8
И да будут мысли ваши такими же,
и да не дойдут слова эти до сердца отца моего.

\vs Tbn 3:1
И ныне, дети мои, возлюбите вы Господа Бога небес и земли,
и храните заповеди его, уподобляясь доброму и благочестивому
мужу Иосифу.
\vs Tbn 3:2
И да будут помыслы ваши добрыми,
как вы знаете то обо мне.
Ибо имеющий правильные помыслы всё правильно видит.
\vs Tbn 3:3
Бойтесь Господа и любите ближнего;
и если духи Велиаровы во всякую злую печаль ввергнут вас,
да не обретут власти над вами, как не смогли того над Иосифом,
братом моим.
\vs Tbn 3:4
Сколь многие люди желали убить его, и Бог защитил его.
Ибо тот, кто боится Бога и любит ближнего,
не будет сражен духом Велиаровым,
но защитит его страх Божий.
\vs Tbn 3:5
И кознями людей или зверей не может он быть порабощён,
но поможет ему любовь, которую имеет он к ближнему.
И до смерти Иакова не хотел Иосиф говорить о том,
но Иаков, узнав от Господа, сказал ему.
Но и тогда отрицал Иосиф, и едва убедился клятвами Израиля.
\vs Tbn 3:6
И просил Иосиф отца нашего помолиться за братьев его,
дабы не зачёл им Господь грех тот злой,
что совершили против него.
\vs Tbn 3:7
И воскликнул Иаков: О достойное дитя,
победил ты сердце Иакова, отца своего;
и обняв его, целовал 2 часа, говоря:
\vs Tbn 3:8
Исполнится на тебе пророчество небесное об агнце Божием
и Спасителе мира, что безупречный предан будет за беззаконников,
а безгрешный умрёт за нечестивцев в крови Завета во спасение народов
и Израиля, и уничтожит Велиара и слуг его.

\vs Tbn 4:1
Зрите же, дети мои, каков исход доброго мужа.
В доброте уподобляйтесь милосердию его, дабы и вам носить венцы славы.
\vs Tbn 4:2
Ибо у доброго человека око не омрачится, он ведь жалеет всех, если и
грешники это.
\vs Tbn 4:3
Если и недоброго желают ему, всё же творящий добро побеждает зло,
обороняемый Богом.
Праведных же любит он как душу свою.
\vs Tbn 4:4
Если кто славен, не завидует ему; если кто богат, не ревнует;
если мужествен кто, хвалит его; мудрого любит он, бедного жалеет;
слабому сострадает, Бога славит.
\vs Tbn 4:5
Имеющего страх Божий защищает он, любящему Господа помогает;
отвергающего Всевышнего наставляет он и обращает,
а имеющего благодать доброго духа любит как душу свою.

\vs Tbn 5:1
Если и вы будете иметь добрые помыслы,
то даже злые люди примирятся с вами,
и распутные устыдятся вас и обратятся ко благу,
и любостяжатели не только отступят от страсти своей,
но и то, что нажили они алчностью, отдадут страждущим.
\vs Tbn 5:2
Если будете творить добро, то и нечистые духи побегут от вас,
и звери устрашатся вас.
\vs Tbn 5:3
Ибо где свет добрых дел, там и тьма бежит от него.
\vs Tbn 5:4
И тот, кто надменно хулит благочестивого мужа, раскается,
ибо жалеет благочестивый хулителя и молчит.
\vs Tbn 5:5
И если кто предаст праведника, будет молиться тот.
И пусть ненадолго унижен будет, вскоре ещё светлее засияет,
как было то с Иосифом, братом моим.

\vs Tbn 6:1
Помышление доброго мужа~--- не в соблазняющей руке духа Велиарова.
Ибо ангел мира ведёт душу его.
\vs Tbn 6:2
И не взирает он с вожделением на тленное
и не собирает золота из любви к наслаждениям.
\vs Tbn 6:3
Не радуется он наслаждениям, не обижает ближнего,
не наполняется роскошью, не соблазняется взорами очей.
Ибо Господь~--- удел его.
\vs Tbn 6:4
Доброе помышление не внимает ни славе, ни хуле человеческой,
и ни лжи, ни спора, ни хулы не ведает.
Ибо Господь обитает в нём, и освещает душу его,
и радуется он за всех во всякий час.
\vs Tbn 6:5
Благой помысел не имеет двух языков~--- благословения и проклятия,
чести и поругания, покоя и смятения, лицемерия и правды,
бедности и богатства, но обо всех у него чистое и незамутненное суждение.
\vs Tbn 6:6
Нет у такого человека ни зрения двойного, ни слуха, ибо во всём,
что делает и что говорит, знает, что видит Господь душу его.
\vs Tbn 6:7
И очищает он помыслы свои, дабы не осудили его Бог и люди.
У Велиара же всякое дело двойное, и нет в нём простоты.
\vs Tbn 7:1
Потому, дети мои, говорю вам:
убегайте зла Велиарова, ибо нож дает он повинующимся ему.
\vs Tbn 7:2
А нож этот 7 зол порождает, сначала же зачинает мысль от Велиара.
И первое зло~--- убийство,
второе~--- разрушение,
третье~--- угнетение,
четвёртое~--- изгнание,
пятое~--- нужда,
шестое~--- смятение,
седьмое~--- опустошение.
\vs Tbn 7:3
Оттого и Каин 7-ми возмездиям подвергся от Господа,
ибо каждые 100 лет по одному удару наносил ему Господь.
\vs Tbn 7:4
Когда было Каину 200 лет, начал он получать их,
а в 900 был повержен за Авеля, праведного брата его.
7 зол было суждено Каину, а Ламеху~--- 70 раз 7.
\vs Tbn 7:5
Ибо до века будут караться таким судом подражающие Каину
в зависти и ненависти к братьям.
 
\vs Tbn 8:1
Вы же, дети мои, убегайте злобы, зависти и ненависти к братьям,
а прилепитесь к доброте и любви.
\vs Tbn 8:2
Ибо имеющий чистый помысел не взирает на женщину для блуда,
и незапятнано сердце его, ибо почиет на нем дух Божий.
\vs Tbn 8:3
Ибо как солнце не оскверняется, если и видит грязь и нечистоты,
но напротив, оно высушивает их и дурной запах изгоняет,
так же и чистый ум, в мерзостях земных пребывающий,
скорее очищает их, сам же не оскверняется.

\vs Tbn 9:1
Скажу вам, по словам Еноха праведного, и о недобрых делах ваших,
ибо блудить станете вы блудом Содомским, и не останется вас,
кроме немногих.
И вновь с женщинами предадитесь распутству,
и не будет в вас царства Божия,
ибо Господь тотчас заберёт его.
\vs Tbn 9:2
Только в одном уделе вашем возникнет храм Божий,
и будет последний славнее первого, и соберутся туда 12 колен
и все народы до той поры, когда пошлёт Всевышний спасение своё
посещением единородного Пророка.
\vs Tbn 9:3
И войдёт он в 1-ый храм, и там будет поруган Господь и поднят на древо.
\vs Tbn 9:4
И раздерётся завеса в храме, и перейдёт дух Божий к народам,
словно огонь прольётся.
\vs Tbn 9:5
И поднявшись из ада, взойдет он с земли на небо.
Познал я, сколь смирен будет он на земле и сколь прославлен на небе.

\vs Tbn 10:1
Когда же был Иосиф в Египте, желал я видеть лицо его и обличье его,
и по молитвам Иакова, отца моего, узрел я его, бодрствуя днём,
таким, каким был весь вид его.
\vs Tbn 10:2
И сказал им затем: Знайте, дети мои, что я умираю.
\vs Tbn 10:3
Творите же правду каждый ближнему своему, и закон Господа,
и заповеди его храните.
\vs Tbn 10:4
Ибо оставляю вам это вместо всякого наследства,
а вы передайте детям вашим на владение вечное,
ибо делали так Авраам, Исаак и Иаков.
\vs Tbn 10:5
И всё это оставили они нам в наследство, сказав:
Храните заповеди Бога до той поры, когда откроет Господь
спасение своё всем народам.
\vs Tbn 10:6
И тогда узрите вы Еноха, и Ноя, и Сима, и Авраама,
и Исаака, и Иакова восставшими одесную его в радости.
\vs Tbn 10:7
Тогда и мы воскреснем, каждый в уделе власти своей
и преклонимся пред Царём Небесным на землю явившимся
в обличье человеческом смиренно, и те, кто уверует
в него на земле, возрадуются с ним.
\vs Tbn 10:8
И все воскреснут: одни~--- для славы, другие~--- для бесславия,
и будет судить Господь первых Израиля за неправедность
их ибо не уверовали они в Бога, явившегося во плоти.
\vs Tbn 10:9
После же будет судить он все народы ибо не уверовали в него,
явившегося на землю.
\vs Tbn 10:10
И обличит он Израиля через народы избранные,
как обличил он Исава через Мадианитян,
возлюбивших братьев их.
Будьте же, дети мои, в уделе боящихся Господа.
\vs Tbn 10:11
Если пребудете вы в святости, дети мои,
и по заповедям Господа, то в твёрдой надежде будете вновь жить со мною,
и соберётся пред Господом весь Израиль.

\vs Tbn 11:1
И не назовусь я более волком хищным за хищность вашу,
но работником Господним, пищу раздающим тем, кто творит добро.
\vs Tbn 11:2
И восстанет в последние времена возлюбленный Господа от семени Иуды и Левия,
творящий благоволение уст его знанием новым освещая все народы.
Свет знания, придёт он к Израилю во спасение его,
и похитит у них как волк и отдаст собранию народов.
\vs Tbn 11:3
До скончания века пребудет он в собраниях народов и во властителях их,
словно песня сладкозвучная на устах всех.
\vs Tbn 11:4
И записан будет он в книги святые, и дело, и слово его,
и будет он избранником Божиим до века.
\vs Tbn 11:5
И будет ходить он среди них, подобно Иакову, отцу моему, говоря:
Сам восполнит он недостаток племени твоего.

\vs Tbn 12:1
И закончив речи свои, сказал он: Завещаю вам, дети мои,
вынесите кости мои из Египта и похороните меня в Хевроне рядом с отцами моими.
\vs Tbn 12:2
И умер Вениамин, будучи 125-и лет, в старости прекрасной,
и положили его во гроб.
\vs Tbn 12:3
И в 91-ый год прихода сынов Израиля в Египет,
отнесли кости отца своего тайно,
во время войны Ханаанской, в Хеврон и погребли там у ног отцов его.
\vs Tbn 12:4
А сами возвратились они из земли Ханаанской
и пребывали в Египте вплоть до дней исхода их из земли Египетской.

\bibbookdescr{Pss}{
  inline={Псалмы Соломона},
  toc={Псалмы Соломона},
  bookmark={Псалмы Соломона},
  header={Псалмы Соломона},
  abbr={Псс}
}
\vs Pss 1:1
Воззвал я к Господу в смертной тоске моей, к Богу воззвал в засильи
грешников.
\vs Pss 1:2
Раздался внезапно клич военный предо мною: я буду услышан, ибо
исполнился праведности.
\vs Pss 1:3
Помыслил я в сердце моем, что исполнился праведности, ибо
достиг процветания и умножился в потомках моих.
\vs Pss 1:4
Пусть богатство их разойдется по всей земле, а слава~--- до
края земли.
\vs Pss 1:5
Возвысились они до звезд, и сказал я: Нет, они не упадут!
\vs Pss 1:6
Но возгордились в богатстве своем и не удержались.
\vs Pss 1:7
Грехи их~--- в тайне, и я не ведал.
\vs Pss 1:8
Беззакониями своими превзошли они племена, бывшие до них,
осквернили святыни Господни скверною.

\vs Pss 2:1
Возгордился грешник, стенобитным орудием сокрушил стены
крепкие, и Ты не воспрепятствовал.
\vs Pss 2:2
Взошли к жертвеннику Твоему народы чуждые, попирали сандалиями
своими в надменности,
\vs Pss 2:3
за то, что сыны Иерусалимские осквернили святыни Господни,
принесли дары Богу в нечестии.
\vs Pss 2:4
Из-за таких их дел сказал Он: Отбросьте это далеко от Меня, нет
на том Моего благоволения.
\vs Pss 2:5
Блеск славы их обратился в ничто, пред лицем Божиим унижен
окончательно.
\vs Pss 2:6
Сыны и дочери - в плену позорном, печать на шеях их,
заклеймлены они между народами.
\vs Pss 2:7
По прегрешениям их сделал им: предал в руки превосходящих
силою,
\vs Pss 2:8
отвратил лице Свое, отказался миловать их~--- юношу и
старика, детей вместе с ними.
\vs Pss 2:9
Ибо недостойное совершили все вместе~--- не слушали.
\vs Pss 2:10
И небосвод омрачился, и земля отвратилась от них,
\vs Pss 2:11
ибо не сделал всякий человек на ней, сколько они сотворили.
\vs Pss 2:12
И узнает земля праведность судов Твоих, Боже.
\vs Pss 2:13
И вот, выставил Бог сынов Иерусалимских на глумление: вместо
притонов разврата прохожий всякий сюда
сворачивал, при свете дня смеялись беззаконию своему,
\vs Pss 2:14
тем же, что совершали при свете дня, выставили напоказ нечестие
свое.
И дочери Иерусалимские доступны стали все по суду Твоему,
\vs Pss 2:15
за то, что осквернили себя, сходясь с кем попало. Чревом и утробой
моей скорблю о них.
\vs Pss 2:16
Я оправдаю Тебя, Боже, в правоте сердца, ибо в судах
Твоих~--- правда Твоя, Боже,
\vs Pss 2:17
ибо воздал Ты грешникам по делам их, по грехам их, тяжким
весьма.
\vs Pss 2:18
Открыл Ты грехи их, дабы ясен стал приговор Твой,
\vs Pss 2:19
стер память о них с лица земли.
Бог~--- Судия праведный и не смотрит на лица.
\vs Pss 2:20
Совлек Он красу Иерусалима с престола славы, предали поруганию его
язычники, топча ногами.
\vs Pss 2:21
Препоясал он вретище вместо красивых одежд, бечеву вокруг головы
своей вместо венца,
\vs Pss 2:22
снял митру славы, которую возложил на него Бог:
\vs Pss 2:23
в бесчестии пала краса его на землю.
\vs Pss 2:24
И увидел я, и вознес мольбу к лицу Господа, и сказав: Довольно,
Господи, тяготела рука Твоя над Иерусалимом~--- положи конец нашествию
язычников!
\vs Pss 2:25
Ибо насмеялись и не знали пощады в гневе и ярости злобной.
\vs Pss 2:26
И довершат они дело свое, если Ты, Господи, не накажешь их в гневе
Твоем,
\vs Pss 2:27
ибо не из любви к Тебе сделали так, но из похоти душевной
\vs Pss 2:28
излили гнев свой на нас в расхищении.
Так не помедли, Боже, воздать им на головы их,
\vs Pss 2:29
дабы обречь высокомерие дракона на поругание!
\vs Pss 2:30
И не помедлил Бог, явив мне надменность его пронзенной на горах
Египетских, хуже самого последнего униженной на земле и на море,
\vs Pss 2:31
тело его~--- носимым на волнах в полном нечестии, и не было
никого, кто бы похоронил:
\vs Pss 2:32
ибо унизил его Господь и обрек на поругание. И не помыслил он, что
человек,
и о последнем не помыслил.
\vs Pss 2:33
Сказал: Господином земли и моря сделаюсь!~--- и не знал, что Бог
велик, могуч в крепости Своей великой.
\vs Pss 2:34
Он~--- Царь на небесах, судящий царей и владык,
\vs Pss 2:35
пробуждающий меня к славе и усыпляющий надменных к погибели вечной в
бесчестии, ибо не узнали Его.
\vs Pss 2:36
Теперь же узрите, владыки земные, суд Господень, ибо велик Царь и
праведен, судящий поднебесную.
\vs Pss 2:37
Благословляйте Бога, боящиеся Господа в мудрости, ибо милость
Господня в день суда~--- на боящихся Его,
\vs Pss 2:38
когда разведет Он праведного и грешника, воздаст грешникам навеки по
делам их
\vs Pss 2:39
и помилует праведного, освободит от унижения перед грешником,
воздаст грешнику за то, что он сделал праведному.
\vs Pss 2:40
Ибо благ Господь к уповающим на Него с твердостью. Да сделает Он по
милости Своей тем, кто с Ним, да пребудут непрестанно пред лицем Его в
силе.
\vs Pss 2:41
Славен Господь вовек пред лицем слуг Его!

\vs Pss 3:1
Что ты спишь, душа, и не славишь Господа?
\vs Pss 3:2
Пойте новую песнь Богу хвалимому! Пой и бодрствуй, дабы служить
Ему, ибо псалом Богу, идущий от всего сердца, благ.
\vs Pss 3:3
Праведные всегда помнят Господа, признают и хвалят суды
Господни.
\vs Pss 3:4
Не пренебрежет праведный вразумлением Господним, неизменно
благоволение его к Господу.
\vs Pss 3:5
Оступился праведный~--- и прославил Господа, упал~--- и
взирает, что сделает ему Бог,
\vs Pss 3:6
смотрит, откуда придет спасение к нему.
\vs Pss 3:7
Честны праведные пред Богом, Спасителем их, нет в доме
праведного прегрешения на прегрешении.
\vs Pss 3:8
Праведный всегда блюдет дом свой, и в падении своем изгоняя
неправедность.
\vs Pss 3:9
Постясь, молит он простить неведение его и смиряет душу
свою.
\vs Pss 3:10
И Господь очищает всякого, кто благочестив, и дом его.
\vs Pss 3:11
Оступился грешник~--- и проклинает жизнь свою, день, в который
родился, и материнское чрево.
\vs Pss 3:12
Громоздил он прегрешения на прегрешения в жизни своей.
\vs Pss 3:13
Он пал, ибо мерзостно тело его, и уже не восстанет: гибель
грешника~--- навеки,
\vs Pss 3:14
И когда воззрит на праведных, не опомнится он.
\vs Pss 3:15
Таков удел грешников навеки.
\vs Pss 3:16
Те же, кто боится Господа, восстанут для жизни вечной, и жизнь
их~--- в свете Господнем и не угаснет никогда.

\vs Pss 4:1
Что ты сидишь, нечестивый, в синедрионе? Сердце твое далеко
отступило от Господа, беззакониями гневишь ты Бога Израиля!
\vs Pss 4:2
Чрезмерен он речами, чрезмерен и подозрительностью более всех,
жесток в словах, когда осуждает грешников на суде.
\vs Pss 4:3
И рука его среди первых ляжет на грешника, словно бы в
ревности, сам же он отягощен пестротою грехов и неумеренностью.
\vs Pss 4:4
Глаза его~--- на всякой женщине без различия, язык его лжив
в договоре клятвенном.
\vs Pss 4:5
В ночи и в тайне согрешает он, словно никем не зрим, глазами
говорит он со всякою женщиной, замышляя недоброе,
\vs Pss 4:6
скоро входит он во всякое жилище, веселясь духом, словно
безгрешен.
\vs Pss 4:7
Истреби, Боже, живущих в лицемерии рядом с благочестивыми,
истреби разрушением плоти и бедностью жизнь их.
\vs Pss 4:8
Открой, Боже, дела людей-человекоугодников, да будут осмеяны и
поруганы дела их.
\vs Pss 4:9
И да признают благочестивые праведность кары Бога их, когда
истребляет Он грешников от лица праведника,
\vs Pss 4:10
человекоугодников, говорящих закон с хитростью.
\vs Pss 4:11
И словно змея глаза их в жилище человека твердого, дабы разрушить
мудрость невинных речами преступными.
\vs Pss 4:12
Речи его~--- обман, дабы потворствовать вожделениям
неправедного.
\vs Pss 4:13
Не отступал он, пока не рассеет как сирот и не опустошит дом ради
вожделения преступного.
\vs Pss 4:14
Лгал в речах своих, будто нет Видящего и Судящего.
\vs Pss 4:15
Исполнился он беззакония, и глаза его на доме чужом, дабы разрушить
его речами
соблазна. И не насыщается всем этим душа его, подобно преисподней.
\vs Pss 4:16
Да будет, Господи, удел его~--- бесчестие пред лицом Твоим,
выхождение его~--- в стенаниях, а вхождение его~--- в несчастии.
\vs Pss 4:17
В скорбях и в нищете и в смятении жизнь его, Господи: сон его~--- в
скорбях, а пробуждение его~--- в смятении.
\vs Pss 4:18
Да отнимется сон от висков его в ночи, да отпадет он бесславно от
всякого дела рук своих.
\vs Pss 4:19
С пустыми руками да войдет он в дом свой, и пусть не станет в доме
том ничего, чем насыщал он душу свою.
\vs Pss 4:20
В одиночестве бездетном старость его до самой смерти.
\vs Pss 4:21
Да будет разбросана плоть человекоугодников зверями, а кости
преступников~--- под солнцем в бесславии.
\vs Pss 4:22
Да выклюют вороны глаза людей льстивых,
\vs Pss 4:23
ибо опустошили они многие жилища людские в бесславии и разметали в
вожделении.
\vs Pss 4:24
И не вспомнили они о Боге, и не убоялись Бога во всем том,
\vs Pss 4:25
но прогневили они Бога и рассердили. Да истребит Он их от земли, ибо
души кротких совратили они обманом.
\vs Pss 4:26
Блаженны боящиеся Господа в кротости своей.
\vs Pss 4:27
Избавит их Господь от людей коварных и грешников и избавит нас от
всякого соблазна преступного.
\vs Pss 4:28
Да истребит Бог творящих в гордыне всякую неправедность, ибо
Он~--- великий Судия и могучий Господь Бог наш праведный.
\vs Pss 4:29
Да будет, Господи, милость Твоя на всех любящих Тебя.

\vs Pss 5:1
Господи Боже, прославлю имя Твое в ликовании средь видящих
праведность судов Твоих.
\vs Pss 5:2
Ибо Ты благ и милостив, Ты~--- прибежище бедного.
\vs Pss 5:3
Когда я взываю к Тебе, не отвратись в молчании от
меня;
\vs Pss 5:4
ибо не возьмет добычи человек от мужа могучего.
\vs Pss 5:5
Да и кто возьмет из всего, что сотворил Ты, если Ты не
дашь?
\vs Pss 5:6
Ибо человек и жребий его у Тебя на весах, и не прибавит он,
чтобы наполнить сверх суда Твоего, Боже!
\vs Pss 5:7
В печали нашей призовем на помощь Тебя, и Ты не отвергнешь
мольбу нашу, ибо Ты Единый Бог наш.
\vs Pss 5:8
Да не опустишь тяжко руки Твоей на нас, чтобы не пришлось
согрешить нам.
\vs Pss 5:9
И если не обратишься к нам, не покинем Тебя, но последуем за
Тобою.
\vs Pss 5:10
Ибо если взалчу, воззову к Тебе, Боже, и Ты дашь мне.
\vs Pss 5:11
Пернатых и рыб кормишь Ты, давая дождь в пустынях для взращения
зелени, дабы уготовить пропитание в пустыне всякому живущему.
\vs Pss 5:12
И если взалчут они, к Тебе поднимут лица свои.
\vs Pss 5:13
Царей и начальников и народы Ты, Боже, питаешь; нищего и убогого кто
надежда, если не Ты, Господи?
\vs Pss 5:14
И Ты услышишь~--- ибо кто благ и праведен, кроме Тебя?~--- услышишь,
как ликует душа смиренного, когда раскрываешь Ты милостиво руку
Твою.
\vs Pss 5:15
Доброта человеческая к ближнему скупа и холодна, и если повторит
даяние без ропота, и тому подивишься.
\vs Pss 5:16
Но Твое даяние великое благостно и обильно, и тот, чья надежда на
Тебя, Господи, не поскупится на даяние сам.
\vs Pss 5:17
На всей земле милость Твоя благостная, Господи!
\vs Pss 5:18
Блажен тот, о ком помнит Бог, давая ему достаток умеренный:
\vs Pss 5:19
если получает чрезмерно человек, согрешает.
\vs Pss 5:20
Довольно для него меры праведной, и потому благодарение Господу за
насыщение праведное.
\vs Pss 5:21
Да возвеселятся в благах боящиеся Господа, и благость Твоя на
Израиле в Царствии Твоем.
\vs Pss 5:22
Благословенна слава Господня, ибо Он Царь наш!

\vs Pss 6:1
Блажен муж, в ком сердце его готово призвать имя Господне:
\vs Pss 6:2
помня имя Господне, спасется он.
\vs Pss 6:3
Пути его направляются Господом, и хранимы дела рук его.
\vs Pss 6:4
Лукавыми ночными видениями его не возмутится душа его,
\vs Pss 6:5
на переправе рек, среди волнения вод морских не устрашится.
\vs Pss 6:6
Восстал он от сна своего и благословил имя Господне.
\vs Pss 6:7
В постоянстве сердца своего воспел имя Бога своего, молился, да
не отвратится лице Господне от всего дома его.
\vs Pss 6:8
И Господь услышал мольбу каждого, пребывающего в страхе Божием,
и всякую просьбу души, уповающей на Него, исполнит Господь.
\vs Pss 6:9
Благословен Господь, дающий милость воистину любящим Его!

\vs Pss 7:1
Не отступись от нас, Боже, дабы не потеснили нас ненавидящие
нас без причины.
\vs Pss 7:2
Ибо Ты отразил их, Боже, и да не потопчет нога их наследие
Святыни Твоей.
\vs Pss 7:3
Вразуми нас по воле Твоей и не предай язычникам.
\vs Pss 7:4
Если же смерть пошлешь, прикажи ей о нас, ибо Ты милостив и не
прогневаешься на дела наши.
\vs Pss 7:5
Имя Твое живет среди нас, и Ты пощадишь нас.
\vs Pss 7:6
И не осилит нас язычник, ибо Ты защитник наш.
\vs Pss 7:7
И мы призовем Тебя, и Ты услышишь нас.
\vs Pss 7:8
Ибо Ты сжалишься над родом Израиля вовеки и не отвергнешь.
И мы под игом Твоим вовеки и под бичом вразумления Твоего.
\vs Pss 7:9
Ты направишь нас на путь в час заступничества Твоего, помилуешь
дом Иакова в день, который Ты обещал им.

\vs Pss 8:1
Скорбь и голос войны услышало ухо мое, голос трубы, возвещающей
убийство и гибель,
\vs Pss 8:2
голос народа великого, словно ветра великого, словно вихрь огня
великого, несущегося по пустыне.
\vs Pss 8:3
И сказал я в сердце моем: где положит остановить его Бог?
\vs Pss 8:4
Голос услышал я: в Иерусалиме, граде Святыни.
\vs Pss 8:5
Сокрушились чресла мои от этих слов, ослабели колени мои.
\vs Pss 8:6
Убоялось сердце мое, сотряслись кости мои, как лен.
\vs Pss 8:7
Сказал я: направляют они пути свои в праведности. Помыслил тут
о судах Божиих от сотворения неба и земли, оправдал Бога в судах Его от
века.
\vs Pss 8:8
Открыл Бог прегрешения их при свете дня, узнала вся земля суды
Божий праведные.
\vs Pss 8:9
В местах потаенных беззакония их,
\vs Pss 8:10
гневя Господа, сын с матерью, отец с дочерью смешивались,
\vs Pss 8:11
прелюбодействовали каждый с женой ближнего своего, сговаривались
друг с другом о подобном, скрепляя сговор клятвами.
\vs Pss 8:12
Святыни Божий расхищали~--- и не было наследника-избавителя,
\vs Pss 8:13
топтали жертвенник Господень~--- только что от всякой
скверны, во время кровей поганили жертвы, словно мясо нечистое.
\vs Pss 8:14
Не оставили греха, какого не сотворили бы хуже язычников.
\vs Pss 8:15
Потому приготовил для них Бог дух заблуждения~--- напоил чашей
вина неразбавленного допьяна.
\vs Pss 8:16
Привел мужа от края земли~--- мужа, бьющего крепко,
\vs Pss 8:17
положил войну на Иерусалим и землю его.
\vs Pss 8:18
Вышли к мужу тому старейшины земли с радостью, сказали ему: Желанен
приход твой! сюда, войдите с миром!
\vs Pss 8:19
Сгладили пути неровные при входе их, открыли врата в Иерусалим,
увенчали стены его.
\vs Pss 8:20
Вошел он как отец в дом сыновей своих, с миром, утвердил стопы свои
прочно,
\vs Pss 8:21
занял башни города и стену Иерусалимскую:
\vs Pss 8:22
ибо Бог привел его, неколебимого, в заблуждении их.
\vs Pss 8:23
Погубил он старейшин их и всякого мудрого в совете, проливал кровь
жителей Иерусалимских, как воду нечистую,
\vs Pss 8:24
увел сынов и дочерей их, которых родили в скверне.
\vs Pss 8:25
И сделали они в меру нечистоты своей по примеру отцов своих,
\vs Pss 8:26
осквернили Иерусалим и посвященное имени Божию.
\vs Pss 8:27
Оправдан Бог в судах Его над народами земли,
\vs Pss 8:28
и чтящие Бога словно агнцы безгрешные среди них.
\vs Pss 8:29
Хвалим Господь, судящий всякую землю в праведности Его!
\vs Pss 8:30
Ведь вот, Боже, явил Ты нам суд Свой в праведности Твоей,
\vs Pss 8:31
увидели глаза их суды Твои, Боже, оправдали мы имя чтимое Твое
вовек.
\vs Pss 8:32
Ибо ты~--- Бог праведности, судящий Израиля во вразумлении.
\vs Pss 8:33
Обрати, Боже, милость Свою на нас и пожалей нас,
\vs Pss 8:34
собери рассеяние Израилево, будь благ и милостив,
\vs Pss 8:35
ибо вера Твоя~--- с нами, и мы ожесточим шею нашу, и Ты наставник
наш.
\vs Pss 8:36
Не оставь нас, Боже наш, чтобы не поглотили нас язычники в
отсутствие избавителя.
\vs Pss 8:37
И Ты~--- Бог наш от начала, и на Тебя понадеялись мы,
Господи.
\vs Pss 8:38
И мы не отступимся от Тебя, ибо благи суды Твои над нами.
\vs Pss 8:39
На нас и детях наших благоволение Твое вовек, Господи, Спаситель
наш, и не будем мы поколеблены впредь на вечное время.
\vs Pss 8:40
Хвалим Господь в судах Его устами праведников,
\vs Pss 8:41
ты же благословен, Израиль, Господом вовек.

\vs Pss 9:1
Уведен Израиль от дома своего в чужую землю, отступили от
Господа, избавившего их~---
\vs Pss 9:2
отказано им в наследстве, что даровал им Господь
пред всеми народами, рассеян Израиль по слову Божию.
\vs Pss 9:3
И да оправдаешься Ты, Боже, в праведности Твоей беззакониями
их,
\vs Pss 9:4
ибо Ты Судия праведный над всеми народами земли.
\vs Pss 9:5
И не скроется от ведения Твоего ни один, творящий неправое,
\vs Pss 9:6
так же, как праведность чтящих Тебя~--- пред лицем Твоим,
Господи, ибо где скроется человек от ведения Твоего?
\vs Pss 9:7
Боже, дела наши~--- выбор и власть души нашей, совершить
правое или неправое делами рук наших.
\vs Pss 9:8
Ты же в правде Своей будь милостив к сынам человеческим!
\vs Pss 9:9
Делающий правое готовит себе жизнь подле Господа, делающий же
неправое сам повинен в гибели души своей.
\vs Pss 9:10
Ибо праведен суд Господень над мужем и домом.
\vs Pss 9:11
Кому явишь доброту, Боже, если не призывающим Господа?
\vs Pss 9:12
Очистит Господь грешную душу, дав исповедаться, дав излить ей грехи
ее,
\vs Pss 9:13
ибо стыдно нам и лицам нашим всего содеянного.
\vs Pss 9:14
Да и кому отпустит Он грехи, как не тем, что согрешили?
\vs Pss 9:15
Праведных, Боже, благословишь, не накажешь, если где оступились, и
благость Твоя~--- на грешащих и кающихся.
\vs Pss 9:16
И ныне Ты Бог, и мы~--- народ, который Ты возлюбил. Взгляни и
сжалься, Бог Израилев: ибо мы~--- Твои, и не отврати милости Твоей от нас,
дабы не подступились к нам.
\vs Pss 9:17
Ибо избрал Ты семя Авраамово среди всех народов
\vs Pss 9:18
и положил имя Свое на нас, Господи, и не оставишь вовек.
\vs Pss 9:19
В Завете свидетельствовал Ты отцам нашим о нас, и мы станем уповать
на Тебя, обратив к Тебе душу нашу.
\vs Pss 9:20
Милость Господня на доме Израилевом во веки веков и впредь!

\vs Pss 10:1
Блажен муж, кого воспомнил Господь в обличении, кто отвращен
был бичом от пути злого и очистился от грехов, дабы не умножать числа их.
\vs Pss 10:2
Готовящий спину свою для бича очистится, ибо Господь благ к
терпящим наказание Его.
\vs Pss 10:3
Ибо Он прямыми сделает пути праведных и не искусит вразумлением
Своим.
\vs Pss 10:4
И милость Господня на воистину любящих Его, и воспомнит Господь
о рабах Своих в милости.
\vs Pss 10:5
Свидетельство~--- в законе Завета вечного, свидетельство
Господа~--- в посещении путей человеческих.
\vs Pss 10:6
Праведен и благ Господь наш в судах Его вовеки, и прославит
Израиль имя Господне в радости.
\vs Pss 10:7
И возблагодарят Его благочестивые в собрании народа, и над
нищими смилуется Господь в радости Израиля.
\vs Pss 10:8
Ибо Он милостив и благ вовеки, и собрания Израиля прославят имя
Господне.
\vs Pss 10:9
От Господа спасение на дом Израилев в радость вечную!

\vs Pss 11:1
Вострубите на Сионе трубою знамения священные!
\vs Pss 11:2
Возгласите в Иерусалиме глас благовествующего! Ибо
смилостивился Бог над Израилем и посетил их.
\vs Pss 11:3
Встань, Иерусалим, на возвышенности и узри детей твоих, от
востока и запада собранных воедино Господом.
\vs Pss 11:4
От севера идут, радуясь Богу своему, с островов отдаленных
собрал их Бог.
\vs Pss 11:5
Горы высокие принизились до равнины для них,
\vs Pss 11:6
холмы бежали прочь при входе их, дубравы дали тень им на пути
их:
\vs Pss 11:7
всякое древо благоуханное взрастил для них Бог, дабы прошел
Израиль в присутствии славы Бога их.
\vs Pss 11:8
Облекись, Иерусалим, в одеяние славы твоей, приготовь одежды
святости твоей, ибо возвестил Бог благо Израиля во веки веков и впредь.
\vs Pss 11:9
Да исполнит Господь, что возвестил над Израилем и Иерусалимом,
да поднимет Господь Израиля именем славы Своей! Милость Господня да пребудет
над Израилем во веки веков и впредь!

\vs Pss 12:1
Господи, избавь душу мою от мужа преступного и порочного, от
языка преступного и злоречивого, говорящего ложь и обман!
\vs Pss 12:2
Чиня разврат, речи с языка мужа порочного подобны огню,
поджигающему на току солому.
\vs Pss 12:3
Так пусть же приход его наполнит жилища словом лживым, пусть
вырубит деревья воспаляющей преступной радости,
\vs Pss 12:4
пусть посеет вражду среди преступных жилищ злоречивыми устами!
Да отдалит Бог от непорочных уста совершающих беззаконие в неведении, и да
будут разметаны кости злоречивцев знающими страх Божий!
\vs Pss 12:5
В пламени огненном язык лукавый да будет умерщвлен руками
праведных!
\vs Pss 12:6
Да охранит Господь покой души, ненавидящей делающих беззаконие,
и да направит Господь мужа, несущего мир в дом.
\vs Pss 12:7
Да пребудет спасение Господне на Израиле, чаде Его, вовек,
\vs Pss 12:8
и да сгинут грешники от лица Господа безвозвратно;
праведники же Господни да унаследуют обетование Господне!

\vs Pss 13:1
Десница Господня оборонила меня, десница Господня оберегла
нас.
\vs Pss 13:2
Рука Господня избавила нас от меча нависшего, от голода и
смерти грешников.
\vs Pss 13:3
Звери злобные набежали на них, зубами своими разорвали плоть
их
и челюстями своими раздробили кости их, и от всего этого избавил нас
Господь.
\vs Pss 13:4
Смутился благочестивый от согрешений своих, и да не будет он
взят вместе с грешниками.
\vs Pss 13:5
Ибо ужасно низвержение грешника, но ничто из этого не коснется
праведного:
\vs Pss 13:6
неравно вразумление праведных, согрешающих в неведении, и
низвержение грешников.
\vs Pss 13:7
В одеждах наказывается праведник, да не восторжествует грешник
над праведным.
\vs Pss 13:8
Ибо наставляет Он праведного как сына любимого и вразумляет его
как первенца.
\vs Pss 13:9
И милостив Господь к почитающим Его, и согрешения их загладит
вразумлением, ибо жизнь праведных навеки;
\vs Pss 13:10
но грешники будут изъяты и погублены, и не сыщется более памяти о
них.
\vs Pss 13:11
На благочестивых же милость Господня, и на боящихся Его милость
Его.

\vs Pss 14:1
Верен Господь воистину любящим Его, терпящим наказание Его,
совершающим путь в праведности предписаний Его, по закону, какой дал Он нам для
жизни нашей.
\vs Pss 14:2
Праведники Господни будут жить по нему вовек, рай Господень,
дерева жизни~--- праведники Господни.
\vs Pss 14:3
Росток их пустил корни вовек,
и не будут вырваны из земли во все дни, ибо жребий и наследие
Божие~--- Израиль.
\vs Pss 14:4
И не так грешники и беззаконники,
что возлюбили день в участи греха своего, в горечи гнили, в похоти своей,
\vs Pss 14:5
и не вспомнили о Боге,
ибо пути людские открыты пред лицем Его вполне, и хранилища сердец ведомы
ранее, нежели исполнится.
\vs Pss 14:6
За то удел их~--- ад, мрак и гибель, и не найти их будет в
день, когда милость коснется
праведных: чтящие же Господа унаследуют жизнь в радости.

\vs Pss 15:1
В беде моей призвал я имя Господне,
на помощь понадеялся Бога Иаковлева и был спасен,
\vs Pss 15:2
ибо надежда и прибежище несчастных Ты, Боже.
\vs Pss 15:3
Да и что имеет силу, Боже, как не воздать хвалу Тебе по
истине?
\vs Pss 15:4
И что может человек, как не воздать хвалу имени
Твоему~---
\vs Pss 15:5
псалмом и хвалебной песнью в радости сердца, плодом уст на
слаженном инструменте языка, первенцем уст от сердца благочестивого и
праведного?
\vs Pss 15:6
Делающий так не поколеблется вовек силами зла, пламя огненное и
гнев нечестивцев не коснутся его,
\vs Pss 15:7
когда выйдет он к грешникам от лица Божьего, чтобы сокрушить
всю твердость грешников,
\vs Pss 15:8
ибо знак Божий на праведных во спасение: голод, меч и смерть
далеко от праведных,
\vs Pss 15:9
побегут, как от войны, от благочестивых, кинутся в погоню за
грешниками и настигнут их. И не избегнут делающие беззаконие суда Божьего, но
словно врагами хитроумными будут захвачены,
\vs Pss 15:10
ибо знак гибели на челе их,
\vs Pss 15:11
и наследство грешников~--- гибель и мрак, и беззакония их
устремятся за ними до глубин ада.
\vs Pss 15:12
Наследство их не отойдет к детям их,
\vs Pss 15:13
но беззакония обратят в пустыню дома грешников, и погибнут грешники
в день суда Господня вовек,
\vs Pss 15:14
когда посетит Бог землю в день суда Своего, дабы воздать грешникам
во веки веков.
\vs Pss 15:15
Боящиеся же Господа узнают милость Его в тот день и продолжат жизнь
свою милостынью Бога их.

\vs Pss 16:1
Погрузилась в дрему душа моя, забыв о Господе, и едва не впал я
в забытье сонное,
\vs Pss 16:2
подобно тем, кто удалился от Бога.
Немного~--- и была бы предана смерти душа моя вблизи врат преисподней вместе
с грешником,
\vs Pss 16:3
унесена была бы душа моя от Господа Бога Израиля, если бы не
помог мне Господь милостью Своею вечной.
\vs Pss 16:4
Уколол Он меня, как шипом коня колют, чтобы служил я Ему,
Спаситель и Заступник мой во всякий час~--- Он спас меня.
\vs Pss 16:5
Славлю Тебя, Боже, ибо Ты помог мне во спасение мое и не
сопричислил меня к грешникам в погибели.
\vs Pss 16:6
Не удали от меня милость Твою, Боже, ниже память о Тебе от
сердца моего, доколе я не умру.
\vs Pss 16:7
Отврати меня, Боже, от согрешения злого и ото всякой злой жены,
соблазняющей неразумного.
\vs Pss 16:8
И да не совратит меня красота жены беззаконной и никто
одержимый согрешением негодным.
\vs Pss 16:9
Дела рук моих направь в страхе Твоем, и пути мои сохрани в
памяти о Тебе.
\vs Pss 16:10
Язык мой и уста мои одень словами правды, гнев и ярость безрассудную
далекими сделай от меня,
\vs Pss 16:11
ропот и малодушие в скорби удали от меня; если согрешу, вразуми
меня, дабы я обратился к Тебе.
\vs Pss 16:12
Благоволением и радостью утверди душу мою; укрепишь душу мою~--- и
довольно будет мне даяния Твоего,
\vs Pss 16:13
ибо, если не укрепишь Ты, кто вынесет наказание бедностью,
\vs Pss 16:14
когда изобличает душу человека рука мерзости его? Испытание
Твое~--- в плоти человеческой и в скорби бедности.
\vs Pss 16:15
Хранящий праведность во всех бедах помилован будет Господом.

\vs Pss 17:1
Господи, Ты Сам Царь наш во веки вечные, ибо Тобою, Боже,
похвалится душа наша.
\vs Pss 17:2
Каково время жизни человека на земле? По времени его~--- и
надежда на него.
\vs Pss 17:3
Но мы понадеемся на Бога, Спасителя нашего, ибо сила Бога
нашего вовеки с милостью,
\vs Pss 17:4
и Царствие Бога нашего вовеки в суде над народами.
\vs Pss 17:5
Ты, Господи, избрал Давида царем над Израилем, и Ты клялся ему
о семени его навеки, да не угаснет пред Тобою царство его.
\vs Pss 17:6
И в согрешениях наших восстали на нас грешники, и напали на нас
и притеснили нас.
Чего не обещал Ты им, взяли они силою,
\vs Pss 17:7
и не прославили честного имени Твоего. В славе воздвигли они
царство в знак величия своего.
\vs Pss 17:8
Пустым сделали они трон Давидов в многошумном высокомерии.
Но Ты, Боже, низвергнешь их и возьмешь семя их от земли,
\vs Pss 17:9
восставив на них человека, чуждого роду нашему,
\vs Pss 17:10
по грехам их воздашь Ты им, Боже, да случится с ними по делам
их.
\vs Pss 17:11
По делам их не помилует их Бог, испытал Он семя их и не отпустил
им.
\vs Pss 17:12
Верен Господь во всех судах Его, кои вершит Он на земле.
\vs Pss 17:13
Лишил беззаконный землю нашу живущих на ней, похитил юного и старого
и детей их вместе с ними.
\vs Pss 17:14
Во гневе своем отослал он их на Запад и насмеялся над правителями
земли и не пощадил их.
\vs Pss 17:15
Чуждый нам, возгордился враг, и сердце его чуждо Бога нашего.
\vs Pss 17:16
И все, что соделал в Иерусалиме,- как язычники в городах своих богам
своим.
\vs Pss 17:17
И превосходили их сыны Завета среди народов смешанных, и не было
среди них никого, кто творил бы в Иерусалиме милость и правду.
\vs Pss 17:18
Бежали от них любящие собрания благочестивых, словно воробьи
разлетелись от гнезда своего.
\vs Pss 17:19
Скитались в пустынях, дабы спасти души свои от зла, и драгоценна в
очах переселенца душа, спасшаяся из них.
\vs Pss 17:20
По всей земле соделалось рассеяние их беззаконными, ибо перестало
небо изливать дождь на землю,
\vs Pss 17:21
сомкнулись источники вечные в безднах средь гор высоких, ибо не было
среди людей тех никого, кто творил бы справедливость и суд. От правителя их до
малейшего из народа~--- во грехе всяческом.
\vs Pss 17:22
Царь в беззаконии, а судья в неправде, а народ во грехе.
\vs Pss 17:23
Призри на них, Господи, и восставь им царя их, сына Давидова, в тот
час, который Ты знаешь, Боже, да царит он над Израилем, отроком Твоим.
\vs Pss 17:24
И препояшь его силою поражать правителей неправедных.
\vs Pss 17:25
Да очистит он Иерусалим от язычников, топчущих город на погибель. В
премудрости и праведливости
\vs Pss 17:26
да изгонит он грешников от наследия Твоего, да искоренит гордыню
грешников, подобно сосудам глиняным сокрушит жезлом железным всякое упорство
их.
\vs Pss 17:27
Да погубит он язычников беззаконных словами уст своих, угрозою его
побегут язычники от лица его, и обличит он грешников словом сердца их.
\vs Pss 17:28
И соберет он народ святой, и возглавит его в справедливости, и будет
судить колена народа, освященного Господом Богом его.
\vs Pss 17:29
И не позволит он поселиться среди них неправедности, и не будет с
ними никакой человек, ведающий зло.
\vs Pss 17:30
Ибо он будет знать, что все они~--- сыны Бога их, и разделит он
их по коленам их на земле.
\vs Pss 17:31
Ни переселенец, ни чужеродный не поселятся с ними более. Будет
судить он народы и племена в премудрости и справедливости его.
\vs Pss 17:32
И возьмет он народы язычников служить ему под игом его, и прославит
он Господа в очах всей земли,
\vs Pss 17:33
и очистит он Иерусалим, освятив его, как был он в начале.
\vs Pss 17:34
Придут племена от края земли, дабы видеть славу его, неся в дар
истомленных сынов Иерусалима,
\vs Pss 17:35
и дабы видеть славу Господа, коею прославил Он эту землю; и сам
справедливый царь научен будет Богом о них.
\vs Pss 17:36
И нет неправедности во дни его среди них, ибо все святы, а царь
их~--- помазанник Господень.
\vs Pss 17:37
Не понадеется он на коня, всадника и лук, не станет собирать себе в
изобилии золота и серебра для войны, не станет оружием стяжать надежд на день
войны.
\vs Pss 17:38
Сам Господь~--- Царь его, надежда сильного надеждою на Бога, и
поставит он все племена пред собою в страхе.
\vs Pss 17:39
Ударит он по земле словом уст своих навеки,
\vs Pss 17:40
благословит он народ Господа в премудрости с радостию.
\vs Pss 17:41
И сам он чист от согрешения, дабы править народом великим;
обличит он правителей и уничтожит грешников влаетию слова своего.
\vs Pss 17:42
И не ослабеет во дни те, уповая на Бога своего, ибо сделал его Бог
сильным духом святым и премудрым в рассуждении, с мощью и справедливостью.
\vs Pss 17:43
И благословение Господа с ним в силе его, и не ослабеет он.
Упование его на Господа,
\vs Pss 17:44
и кто может против него?
Мощный в делах своих и сильный в страхе Божием,
\vs Pss 17:45
пасущий стадо Господне в вере и справедливости, не даст он ослабеть
никому на пастбище их.
\vs Pss 17:46
В благочестии поведет он их всех, и не будет среди них гордыни для
угнетения среди них.
\vs Pss 17:47
Такова краса царя Израиля, которую познал Бог, восставив его над
домом Израиля для вразумления его.
\vs Pss 17:48
Речи его, пламенем очищенные, драгоценнее золота, в собраниях будет
судить он людей~--- колена освященных.
\vs Pss 17:49
Слова его~--- как слова святых среди людей освященных.
\vs Pss 17:50
Блаженны, кто будет жить в те дни, ибо узрят они сотворенное Богом
счастие Израиля в собрании колен его.
\vs Pss 17:51
Да ускорит Бог милость Свою над Израилем, да избавит нас от
нечистоты врагов нечестивых. Сам Господь~--- Царь наш во веки вечные.
с того дня, как утвердил их Бог, и от века.
И не отклонились с того дня, как утвердил Он их; с давних времен не отступили
они от пути своего, если Сам Бог не приказал им через слуг Своих.

\vs Pss 18:1
Господи! милость Твоя~--- на творениях рук Твоих вовек,
\vs Pss 18:2
благодать Твоя, дар изобилия - на Израиле.
Глаза Твои взирают на них, и не узнает нужды ни одно из них,
\vs Pss 18:3
уши Твои услышат мольбу несчастного, молящегося с упованием.
Суды Твои надо всею землей исполнены милосердия,
\vs Pss 18:4
любовь же Твоя~--- на семени Авраамовом, сынах
Израилевых.
Наказание Твое на нас~--- как на сыне первородном и единственном,
\vs Pss 18:5
дабы отвратить душу послушную от греха по неведению.
\vs Pss 18:6
Да очистит Бог Израиля в день милости благословением, в день
избрания~--- возвращением помазанника Его.
\vs Pss 18:7
Блаженны, кому случится во дни те увидеть блага Господни, какие
явит Он роду, грядущему
\vs Pss 18:8
под жезл вразумления помазанника Господня~--- в страхе пред
Богом своим, в мудрости духа, праведности и силы,
\vs Pss 18:9
дабы направил Он человека в делах праведности страхом Божиим,
восстановил их всех в страхе Господнем~---
\vs Pss 18:10
благой род, боящийся Бога, во дни милости.
\vs Pss 18:11
Велик Бог наш и славен, живущий в вышних,
\vs Pss 18:12
определивший ход светил по времени от дней ко дням, и не сошли с
пути, какой Он заповедал им.
\vs Pss 18:13
В страхе Божием путь их каждый день~--- с того дня, как утвердил
их Бог, и от века.
\vs Pss 18:14
И не отклонились с того дня, как утвердил Он их; с давних времен не
отступили они от пути своего, если Сам Бог не приказал им через слуг Своих.

\bibbookdescr{Ode}{
  inline={Оды Соломона},
  toc={Оды Соломона},
  bookmark={Оды Соломона},
  header={Оды Соломона},
  abbr={Оды}
}
\vs Ode 1:1
Яхве на
главе моей подобен венцу, и да не пребуду никогда без Него.
\vs Ode 1:2
Сплетенный
для меня~--- истинный венец, и он побудил ветви Твои прорасти во мне.
\vs Ode 1:3
Ибо не похож
он на увядший венец, который не цветет;
\vs Ode 1:4
ибо Ты
обитаешь над моею главой и расцвел на мне.
\vs Ode 1:5
Полны и
спелы плоды Твои; они исполнены спасения Твоего \ldots

\vs Ode 2:1
\bibemph{не сохранилась.}

\vs Ode 3:1
\ldots\ полагаюсь я на любовь Яхве.
\vs Ode 3:2
И чресла Его
пребывают с Ним, и зависим я от них; и Он любит меня.
\vs Ode 3:3
Ибо мне бы
не стоило узнавать о том, как любит Яхве, если бы постоянно не любил Он меня.
\vs Ode 3:4
Кто же
способен различать любовь, как не тот, кто любим?
\vs Ode 3:5
Люблю я
Возлюбленного и сам я люблю Его, ибо, где бы ни был покой Его, также и я там.
\vs Ode 3:6
И не
сделаюсь чужим я, ибо нет ревности между Яхве Всевышним и Милостивым.
\vs Ode 3:7
Я соединился
с Ним, ибо любящий отыскал Возлюбленного; поскольку же люблю я того, кто есть
Сын, я сделаюсь Сыном.
\vs Ode 3:8
Истинно,
тот, кто соединится с бессмертным, воистину станет бессмертен.
\vs Ode 3:9
И тот, кто
восторгается Жизни, оживет.
\vs Ode 3:10
Вот Дух Яхве, не обманчивый, учащий сынов человеческих познавать пути Его.
\vs Ode 3:11
Будь же
мудрым, и понимающим, и пробужденным.
Аллилуйя.

\vs Ode 4:1
Ни один
человек не может осквернить святое место Твоё, о, Боже мой, как не сможет он и
изменить его и поместить его в иное место,
\vs Ode 4:2
ибо нет у
него власти над ним; ибо Святилище Свое создал Ты прежде, чем создал Ты особые
места.
\vs Ode 4:3
Древний же
не извратится тем, кто ниже него. Дал ты сердце Своё, о, Яхве, верующим в
Тебя.
\vs Ode 4:4
Не будешь ты
ни праздным, ни бесплодным;
\vs Ode 4:5
ибо один
день веры Твоей дивнее всех дней и лет.
\vs Ode 4:6
Ибо кто
положится на милость Твою и отвергнут будет?
\vs Ode 4:7
Ибо известна
печать Твоя; и творения Твои известны ей,
\vs Ode 4:8
и воинство
Твоё одержимо ею, и архангелы избранные облачены ею.
\vs Ode 4:9
Ты воздал
нам сопричастностью Твоею, не оттого, что Ты нуждался в нас, но чтобы мы
всегда нуждались в Тебе.
\vs Ode 4:10
Излей же на
нас живительный дождь Свой, и раскрой обильные источники Свои, щедро дающие
нам молоко и мёд.
\vs Ode 4:11
Ибо не
пребывает с Тобою печаль; чтобы не сожалел Ты ни о чем обещанном Тобою,
\vs Ode 4:12
ибо
результат был явлен Тебе.
\vs Ode 4:13
Ибо
отданное Тобой Ты отдал свободно, так что не передумаешь и снова не заберешь,
\vs Ode 4:14
ибо всё
было явлено Тебе как Богу и восставлено пред Тобою от начала.
\vs Ode 4:15
И Ты, о,
Яхве, создал всё.
Аллилуйя.

\vs Ode 5:1
Я молю Тебя,
о Яхве, ибо я люблю Тебя.
\vs Ode 5:2
О,
Всевышний, не отрекись от меня, ибо Ты~--- надежда моя.
\vs Ode 5:3
Да получу я
свободно милость Твою, и да буду жить ею.
4.
Преследователи мои придут, но да не увидят они меня.
\vs Ode 5:5
Да ниспадет
на очи их облако тьмы; да окутает их воздух тьмы густой.
\vs Ode 5:6
И да не
будет у них света, чтобы видеть, дабы им было не схватить меня.
\vs Ode 5:7
Да замрут
все их поползновения, дабы, где бы ни спрятались они, пасть на их собственные
головы.
\vs Ode 5:8
Ибо
замыслили они нечто, что не для них.
\vs Ode 5:9
Приуготовили
себя они злонамеренно, но найдут их бессильными.
\vs Ode 5:10
Истинно
надеюсь на Яхве, да не убоюсь.
\vs Ode 5:11
А раз Яхве спасение моё, да не убоюсь.
\vs Ode 5:12
И подобен
Ты сотканному венцу на главе моей, и да не дрогну я.
\vs Ode 5:13
Даже если
всё дрогнет, я устою неколебимо.
\vs Ode 5:14
И даже если
погибнет всё видимое, я не умру;
\vs Ode 5:15
Ибо со мною
Яхве, а я~--- с Ним.
Аллилуйя.

\vs Ode 6:1
Как ветер
проскальзывает сквозь арфу и струны говорят,
\vs Ode 6:2
так и Дух Яхве
говорит через чресла мои, а я глаголю через любовь Его,
\vs Ode 6:3
ибо сокрушает
Он всё, что чуждо, и всё сущее~--- от Яхве,
\vs Ode 6:4
ибо так было
от начала и пребудет до самого конца,
\vs Ode 6:5
чтобы не было
ничего вопреки и ничто не восстало бы на Него.
\vs Ode 6:6
Умножил Яхве
знание Своё, и был Он усерден, дабы должное стать известным по милости Его,
воздалось бы нам.
\vs Ode 6:7
И похвалу свою
воздал Он нам именем Своим, духи же наши восхваляли Его Святой Дух.
\vs Ode 6:8
И изошел
Поток, и стал рекой~--- великой и широкой; истинно, смыла она всё, и разрушила
всё, и вынесла к Храму.
\vs Ode 6:9
И преграды,
возведенные людьми, не смогли сдержать её, как не смогли даже умения тех, кто
обыкновенно сдерживает воду.
\vs Ode 6:10
Ибо разлилась
она по поверхности всей земли и заполонила всё.
\vs Ode 6:11
Когда пьют
все жаждущие на земле, и жажда облегчается и утоляется;
\vs Ode 6:12
ибо питие
дано от Всевышнего.
\vs Ode 6:13
Потому
блаженны служащие этого пития, которым вверили воду Его.
\vs Ode 6:14
Освежили они
уста пересохшие и воспрянули увядшей было волей.
\vs Ode 6:15
Даже живые,
близкие к угасанию, восстали из смерти.
\vs Ode 6:16
И омертвевшие
члены воспрянули и восстановились.
\vs Ode 6:17
Они дали силу
идти и свет глазам их.
\vs Ode 6:18
Ибо всякий
узнал их как (принадлежащих) Яхве и ожил живой водою вечности.
Аллилуйя.

\vs Ode 7:1
Как гневаются
над нечестивостью, так же и радуются Возлюбленному, и вкушают свободно от плодов
этих.
\vs Ode 7:2
Радость же моя
Яхве, и путь мой~--- к Нему, и этот путь мой превосходен.
\vs Ode 7:3
Ибо есть у
меня Помощник~--- Яхве. Он щедро явил Себя мне в простоте Своей, ибо благость Его
умалила суровость Его.
\vs Ode 7:4
Сделался Он
подобным мне, чтобы я заполучил Его. Он решил пребывать в облике, подобном
моему, чтобы я положился на Него.
\vs Ode 7:5
И не трепетал
я, когда увидел Его, ибо Он был милостив ко мне.
\vs Ode 7:6
Подобным
природе моей сделался Он, чтобы мог я понять Его. И подобным облику моему, дабы
не отвратился я от Него.
\vs Ode 7:7
Отец же знания
Слово знания.
\vs Ode 7:8
Он,
сотворивший мудрость, мудрее трудов Своих.
\vs Ode 7:9
И Он,
сотворивший меня, когда я еще не знал, что мне следовало делать, когда я начал
быть.
\vs Ode 7:10
Оттого был Он
милостив ко мне Своей щедрой милостью и позволил мне просить у Него и извлечь
пользу из жертвы Его.
\vs Ode 7:11
Ибо именно Он
нетленный, совершенство миров и их Отца.
\vs Ode 7:12
Он позволил
им явиться тем, которые Его; для того, чтобы они опознали Его, сотворившего их и
не думали, что они родились сами собой.
\vs Ode 7:13
Ибо к знанию
направил Он путь Свой, Он расширил его, и удлинил его, и привел его к полному
совершенству.
\vs Ode 7:14
И наставил
над ним следы Света Своего, и продолжалось это от начала до конца.
\vs Ode 7:15
Ибо Сам по
Себе служил Он, и Сыном наслаждался Он.
\vs Ode 7:16
И в силу
спасения Своего всем овладеет Он. И узнают Всевышнего святые Его:
\vs Ode 7:17
Возгласит
тем, кто поет песни явления Яхве, чтобы вышли они встречать Его и пели Ему,
радостно и с арфой, берущей многие ноты.
\vs Ode 7:18
Грядут
пророки прежде Него, и узрят их пред Ним.
\vs Ode 7:19
И восхвалят
они Яхве в любви Его, ибо близок Он и видит.
\vs Ode 7:20
И ненависть
исчезнет с земли, и вместе с ревностью утонет она.
\vs Ode 7:21
Ибо
невежество рушилось на ней, ибо знание Яхве пребывало на ней.
\vs Ode 7:22
Да воспоют
певцы милость Всевышнего Яхве, и да привнесут песнопения свои.
\vs Ode 7:23
И да пребудет
сердце их подобным дню, а бархатные голоса их подобными волшебной красе Яхве.
\vs Ode 7:24
Да не
пребудет там никто дышащий, в ком нет знания или гласа.
\vs Ode 7:25
Ибо дал Он
уста тварям Своим, чтобы открыли голос уст навстречу Ему и чтобы воспеть Его.
\vs Ode 7:26
Веруйте же в
силу Его и возгласите милость Его.
Аллилуйя.

\vs Ode 8:1
Откройте же,
откройте сердца ваши ликованию Яхве, и да пребудет ваша любовь обильной от
сердца к устам,
\vs Ode 8:2
чтобы принести
плоды Яхве, святую жизнь, и чтобы говорить со смирением в свете Его.
\vs Ode 8:3
Восстаньте же
и стойте неподвижно, вы, кого порой ниспосылают вниз.
\vs Ode 8:4
Вы,
пребывавшие в безмолвии, говорите же, ибо отверзлись уста ваши.
\vs Ode 8:5
Вы, бывшие
презираемыми, отныне возвыситесь, ибо возвеличена была Правда ваша.
\vs Ode 8:6
Ибо с вами
десница Яхве, и будет Он помощником вам.
\vs Ode 8:7
И уготован вам
мир прежде того, что может быть войной вашей.
\vs Ode 8:8
Слушайте же
слово истины, и получайте знание Всевышнего.
\vs Ode 8:9
Ни плоть вашу
не следует понимать так, как возвещу я вам, ни одежду вашу, что явлю я вам.
\vs Ode 8:10
Храните же
тайну мою, вы, хранимые ею, храните веру мою, вы, хранимые ею.
\vs Ode 8:11
И понимайте
знание моё, вы, понимающие меня в истине, любите меня с нежностью, вы, любящие;
\vs Ode 8:12
ибо не
отвращу лица моего от тех, кто мои, ибо я знаю их.
\vs Ode 8:13
И прежде, чем
появились они, распознал я их и наложил печать на лица их.
\vs Ode 8:14
Я создал
чресла их, и Мои собственные груди приуготовил Я для них, чтобы могли они испить
Моё святое молоко и жить им.
\vs Ode 8:15
Я любезен им,
и не пристыжён Я ими.
\vs Ode 8:16
Ибо
мастерство Моё~--- они, и сила мыслей Моих.
\vs Ode 8:17
Затем кто
восстанет против труда Моего? И кто не подвержен им?
\vs Ode 8:18
Я возжелал и
создал разум и сердце, и они~--- Мои. И одесную посадил Я избранных Моих.
\vs Ode 8:19
И правда Моя
шествует перед ними, и не лишатся они имени Моего, ибо с ними оно.
\vs Ode 8:20
Молитесь же и
возрастайте, и блюдите себя в любви Яхве.
\vs Ode 8:21
И вы, бывшие
возлюбленными в Возлюбленном, и вы, хранимые в Том, Кто жив, и вы, спасенные в
Том, Кто спасен.
\vs Ode 8:22
И найдут вас
непорочными в любые времена, во имя Отца вашего.
Аллилуйя.

\vs Ode 9:1
Навострите же
уши ваши, и скажу я вам.
\vs Ode 9:2
Отдайтесь же
мне, чтобы я также отдался вам.
\vs Ode 9:3
(Чтобы отдал я
вам) Слово Яхве и страсти Его, святую мысль, которую думал Он о Помазаннике
Своем.
\vs Ode 9:4
Ибо в воле
Яхве~--- жизнь ваша, и цель Его~--- вечная жизнь, и совершенство ваше~--- непорочно.
\vs Ode 9:5
Обогатитесь же
в Боге-Отце, и стяжайте цель Всевышнего. Станьте же сильными и искупленными
милостью Его.
\vs Ode 9:6
Ибо возвещаю я
мир вам, святым Его, дабы никто из слышащих не впал в войну.
\vs Ode 9:7
А также дабы
познавшие Его не погибли, и дабы стяжавшие Его не устыдились.
\vs Ode 9:8
Вечный же
венец суть Истина; блаженны носящие ее на главах своих.
\vs Ode 9:9
Это~--- камень
драгоценный, ибо из-за венца этого велись войны.
\vs Ode 9:10
Но взяла его
Правда и отдала вам.
\vs Ode 9:11
Возложите же
венец этот в истинном согласии с Яхве, и всех побежденных впишут в книгу Его.
\vs Ode 9:12
Ибо книга их
награда победы вашей, и видит она вас пред собою и желает, чтобы вы спасены
были.
Аллилуйя.

\vs Ode 10:1
Яхве наставил
уста мои Словом Своим и открыл сердце моё Светом Своим.
\vs Ode 10:2
И велел Он мне
остаться в бессмертной жизни Его и позволил мне возвестить о плоде покоя Его,
\vs Ode 10:3
преобразить
жизни жаждущих прийти к Нему и вести пленных к свободе.
\vs Ode 10:4
Я же осмелел и
стал сильным, и захватил мир этот, и стала неволя Моей во славу Всевышнего и
Бога, Отца моего.
\vs Ode 10:5
И кротких,
бывших рассеянными, собрали вместе, но не осквернился я любовью своей к ним, ибо
они благодарили меня в вышних местах.
\vs Ode 10:6
И следы света
легли на сердце их, и шли они как по жизни моей, и спасены были, и сделались они
народом моим вовеки веков.
Аллилуйя.

\vs Ode 11:1
Моё сердце
было обрезано и появился цветок у него, затем же милость проросла в нем, и дало
плоды сердце моё ради Яхве.
\vs Ode 11:2
Ибо Всевышний
обрезал его своим Духом Святым, и он открыл мою внутреннюю жизнь навстречу Ему и
наполнил меня любовью Своей.
\vs Ode 11:3
И обрезание
Его сделалось спасением моим, и взошел я на путь, на покой Его, на путь истины.
\vs Ode 11:4
От начала до
конца стяжал я знание Его.
\vs Ode 11:5
И встал я на
скале истины, где Он оставил меня.
\vs Ode 11:6
И говорящие
воды щедро коснулись уст моих из фонтана Яхве.
\vs Ode 11:7
И так пил я и
пьянел от живой воды бессмертной.
\vs Ode 11:8
И опьянение
моё не привело к невежеству, но отрекся я от спеси,
\vs Ode 11:9
и обратился ко
Всевышнему, Богу моему, и обогатился пользой Его.
\vs Ode 11:10
И отверг я
глупость, павшую на землю, и разоблачил её и отбросил прочь от себя.
\vs Ode 11:11
И Яхве
обновил меня одеянием Своим и овладел мною светом Своим.
\vs Ode 11:12
И воздал Он
мне свыше бессмертным покоем, и сделался я подобным земле цветущей и радостной
плодами своими.
\vs Ode 11:13
И Яхве
подобен солнцу над лицом земли.
\vs Ode 11:14
Мне
просветлили очи, и лицо моё окропилось росой;
\vs Ode 11:15
и освежилось
дыхание моё благоуханным ароматом Яхве.
\vs Ode 11:16
И взял Он
меня в Рай Свой, где богатство удовольствия Яхве. Я узрел цветущие и
плодоносящие деревья, и самовозросшей была крона их. Прорастали ветви их и сияли
плоды их. Из бессмертной земли были корни их. И река радости орошала их и
окружала их в земле вечной жизни.
\vs Ode 11:17
Затем
поклонился я Яхве за великолепие Его.
\vs Ode 11:18
И сказал я:
Блаженны, о Яхве, возросшие в земле Твоей, и те, кому есть место в Раю Твоем,
\vs Ode 11:19
и кто растет
ростом деревьев Твоих и перебрался из тьмы в свет.
\vs Ode 11:20
Узри же: все
работники Твои чисты, они, делающие добрые дела, и обращающиеся из дикости в
приязнь Твою.
\vs Ode 11:21
Ибо резкий
запах деревьев сих изменился в земле Твоей,
\vs Ode 11:22
и всё
делается частичкой Тебя. Блаженны же труженики вод Твоих и вечные памятники
набожных слуг Твоих.
\vs Ode 11:23
Истинно, в
Раю Твоем обителей много. И нет там ничего пустого, но всё исполнено плодами.
\vs Ode 11:24
Славься же
Ты, Боже, (и да пребудет) райское ликование вовеки.
Аллилуйя.

\vs Ode 12:1
Он наполнил
меня словами истины, дабы я проповедовал Его.
\vs Ode 12:2
И подобно
течению вод, истина вытекает из уст моих, и уста мои возвещают плоды Его.
\vs Ode 12:3
И побудил Он
знание Своё изобиловать во мне, ибо уста Яхве суть Слово истинное и врата Света
Его.
\vs Ode 12:4
И Всевышний
воздал Его коленам Его, которые суть толковники красы Его, и толковники славы
Его, и исповедники цели Его, и проповедники разума Его, и учителя дел Его.
\vs Ode 12:5
Ибо невыразима
тонкость Слова этого; и каково изречение Его, таковы и быстрота Его, и острота
Его, ибо беспредельность Его суть развитие Его.
\vs Ode 12:6
Никогда не
падает Он, но выстаивает, и никто не может понять нисхождение Его или же путь
Его.
\vs Ode 12:7
Ибо каково
дело Его, такова же надежда Его, ибо Он суть свет и заря мысли.
\vs Ode 12:8
И через Него
колена говорят друг с другом, и безмолвные речь обрели.
\vs Ode 12:9
И из Него
изошли любовь и равенство, и друг другу говорили они, что это принадлежало им.
\vs Ode 12:10
И подтолкнуло
их Слово, и познали Того, Кто создал их, ибо они пребывали в гармонии.
\vs Ode 12:11
Ибо уста
Всевышнего говорили им, и объяснение Его расцвело через Него.
\vs Ode 12:12
Ибо жилище
Слова~--- человек, а истина Его~--- любовь.
\vs Ode 12:13
Блаженны же
воспринявшие всё через Него и познавшие Яхве в истине Его.
Аллилуйя.

\vs Ode 13:1
Узрите же,
Яхве~--- зеркало наше. Откройте очи ваши и узрите их в Нем.
\vs Ode 13:2
И изучите вид
лица вашего, вознося хвалы Духу Святому,
\vs Ode 13:3
и сотрите
краску с лица вашего, и возлюбите святость Его и положитесь на неё.
\vs Ode 13:4
Тогда будете
вы безупречными с Ним во все времена.
Аллилуйя.

\vs Ode 14:1
Как очи сына
на отца его, также и мои очи, о, Яхве, к Тебе во все времена.
\vs Ode 14:2
Ибо сердце моё
и радость моя~--- с Тобой.
\vs Ode 14:3
Не отврати же
милостей Своих от меня, о Яхве, и не отними доброты Своей у меня.
\vs Ode 14:4
Протяни же ко
мне, мой Господь, на все времена, десницу Свою, и веди меня до самого конца,
согласно воле Твоей.
\vs Ode 14:5
Позволь же мне
быть угодным Тебе, о Яхве, во славу Твою и во имя Твоё позволь мне спастись от
Нечистого.
\vs Ode 14:6
И снисхождение
Твое, о, Яхве, да снизойдет на меня, и плоды любви Твоей.
\vs Ode 14:7
Обучи же меня
одам истины Твоей, дабы плодоносил я в Тебе.
\vs Ode 14:8
И открой мне
арфу Твоего Святого Духа, чтобы с каждой нотой восхвалял я Тебя, о Яхве.
\vs Ode 14:9
И по многим
милостям Твоим дари мне и спеши дарить по просьбам нашим.
\vs Ode 14:10
Ибо хватит
Тебя на все нужды наши.
Аллилуйя.

\vs Ode 15:1
Как солнце~---
радость алчущих восхода его, так же моя радость~--- Яхве.
\vs Ode 15:2
Ибо Он~---
Солнце моё, и лучи Его вознесли меня, и свет Его рассеял всю тьму с лица моего.
\vs Ode 15:3
Глаза обрел я
в Нем и увидел святой день Его.
\vs Ode 15:4
Уши обрел я и
услышал истину Его.
\vs Ode 15:5
Мысль знания
обрел я и преисполнился восторгом всецело через Него.
\vs Ode 15:6
Отрекся я от
пути ложного и пришел к Нему и премного стяжал спасения у Него.
\vs Ode 15:7
И по щедрости
Своей воздал Он мне, и по образу превосходной красоты Своей создал Он меня.
\vs Ode 15:8
Облекся я
бессмертием именем Его и очистился от тлена милостью Его.
\vs Ode 15:9
Повержена
смерть перед лицом моим, и повержен Шеол словом моим.
\vs Ode 15:10
И взошла в
земле Яхве жизнь вечная, и возвестили её верным Его и беспредельно дана была
всем верующим в Него.
Аллилуйя.

\vs Ode 16:1
Как дело
пахаря~--- пахота, а дело рулевого~--- править кораблем, так и моё дело~--- псалом
Яхве в гимнах Его.
\vs Ode 16:2
Искусство моё
и служение моё~--- в гимнах Его, ибо любовь Его питала сердце мо, а плоды Свои
излил Он на уста мои.
\vs Ode 16:3
Ибо любовь моя
Сам Яхве, затем же стану я петь о Нем.
\vs Ode 16:4
Ибо силен я
похвалами Его и веру имею в Нем.
\vs Ode 16:5
Открою я уста
мои, а Дух Его возвестит через меня славу Яхве и красу Его,
\vs Ode 16:6
работу рук Его
и труд перст Его
\vs Ode 16:7
ради многих
милостей Его и силы Слова Его.
\vs Ode 16:8
Ибо Слово Яхве
изучает невидимое и открывает мысль Его.
\vs Ode 16:9
Ибо видит око
труды Его, а ухо слышит мысль Его.
\vs Ode 16:10
Ибо именно Он
создал землю широкой и налил воды в море,
\vs Ode 16:11
Он расширил
небо и зажег звезды,
\vs Ode 16:12
и Он создал
творение и наставил его, а затем почил Он от трудов Своих.
\vs Ode 16:13
И сотворил
все вещи бегущими путями своими и делающими дела свои, ибо никогда не могут они
ни перестать быть, ни потерпеть неудачу.
\vs Ode 16:14
И светила~---
подданные Слова Его.
\vs Ode 16:15
Сосуд же
света~--- солнце, а сосуд тьмы~--- ночь.
\vs Ode 16:16
Ибо создал Он
солнце во имя дня, дабы был свет, ночь же приносит тьму на лицо земли,
\vs Ode 16:17
и по частичке
друг от друга составляют они красоту Божью.
\vs Ode 16:18
И нет ничего
вне Яхве, ибо Он был прежде, чем что-либо начало быть.
\vs Ode 16:19
И эти миры~---
по слову Его и по мысли сердца Его.
\vs Ode 16:20
Восхваляйте
же и чтите имя Его.
Аллилуйя.

\vs Ode 17:1
Затем
увенчался я Богом моим, и венец мой был живым.
\vs Ode 17:2
И оправдался я
Господом моим, ибо спасение моё нетленно.
\vs Ode 17:3
Освободился я
от гордыни, и не осужден.
\vs Ode 17:4
Узы мои были
разрублены руками Его, стяжал я образ и подобие новой личности, и я ходил под
Ним и был спасен.
\vs Ode 17:5
И водила мной
мысль истины, и я следовал за ней и не блуждал я.
\vs Ode 17:6
И все видевшие
меня были изумлены, и незнакомцем казался я им.
\vs Ode 17:7
А Тот, Кто
знал и возвысил меня, суть Всевышний во всем совершенстве Своем.
\vs Ode 17:8
И прославил Он
меня добротой Своей и вознес понимание моё до вершин истины.
\vs Ode 17:9
И оттуда
указал мне путь шагов Своих, и отверз я двери закрытые.
\vs Ode 17:10
И сокрушил я
засовы железные, ибо собственные кандалы мои сделались горячи и расплавились
предо мною.
\vs Ode 17:11
И ничто не
являлось мне закрытым, ибо всё открывал я.
\vs Ode 17:12
И шел я ко
всем узам моим, чтобы избавиться от них, дабы не оставить никого скованным или
же связанным.
\vs Ode 17:13
И щедро
раздавал я знание своё и восстание своё через любовь свою.
\vs Ode 17:14
И посеял я
плоды свои в сердцах и преобразил их собою.
\vs Ode 17:15
Затем стяжали
они благодать мою и жили, и собирались подле меня и спасались.
\vs Ode 17:16
Ибо сделались
они чреслами моими, а я был главой их.
\vs Ode 17:17
Слава Тебе,
Глава наша, о Яхве, Помазанник.
Аллилуйя.

\vs Ode 18:1
Сердце моё
воспрянуло и обогатилось в любви Всевышнего, дабы под именем моим восхвалял я
Его.
\vs Ode 18:2
Чресла же мои
усилены были, дабы не выпасть из-под власти Его.
\vs Ode 18:3
Немощи вышли
из тела моего, и стояло оно твердо, ради Яхве, по воле Его, ибо твердо Царство
Его.
\vs Ode 18:4
О, Яхве, ради
нуждающихся, не отпускай Слово Своё от меня.
\vs Ode 18:5
И ради трудов
их, удержи при мне совершенство Своё.
\vs Ode 18:6
Да не
низвергнется свет тьмою, а истина да отделится от лжи.
\vs Ode 18:7
Да приведет к
победе десница Твоя спасение наше, и да придет она из всякой области и да
утвердится на берегу всякого, снедаемого горестями.
\vs Ode 18:8
Ты~--- Бог мой,
не в Твоих устах ложь и смерть, только совершенство~--- воля Твоя.
\vs Ode 18:9
И не ведаешь
Ты гордыни, ибо никто из творящих её не ведает Тебя.
\vs Ode 18:10
И не ведаешь
Ты ошибки, ибо никто из творящих её не ведает Тебя.
\vs Ode 18:11
И явилось
невежество словно пыль и словно пена морская.
\vs Ode 18:12
И пустые люди
думали, что величественно оно, и сделались они подобными типу его и были они
истощены.
\vs Ode 18:13
Но те, кто
ведали, поняли и осмыслили и не осквернились мыслями своими,
\vs Ode 18:14
ибо пребывали
они в разуме Всевышнего и высмеяли ходивших ложными путями.
\vs Ode 18:15
Затем же
изрекали они истину от дыхания, которое вдохнул в них Всевышний.
\vs Ode 18:16
Хвала и честь
великая имени Его.
Аллилуйя.

\vs Ode 19:1
Чашку молока
предложили мне, и пил я его во сладости доброты Яхве.
\vs Ode 19:2
Сын~--- чаша
эта, а Отец~--- Тот, кто доил, а Дух Святой~--- Та, которая доила Его;
\vs Ode 19:3
ибо груди Его
были полны, и не хотелось бы, чтобы млеко Его пропало без толку.
\vs Ode 19:4
Дух Святой
обнажил грудь Её, и смешал молоко из двух грудей Отца.
\vs Ode 19:5
Затем дала Она
смесь колену без ведома их, и получившие её пребывают в совершенстве одесную.
\vs Ode 19:6
Чрево девы
поглотило её, и обрела она идею и порождала.
\vs Ode 19:7
Так дева стала
матерью с милосердием превеликим.
\vs Ode 19:8
И трудилась
она и родила Сына, но без боли, ибо не было это бесцельно.
\vs Ode 19:9
И не надо было
ей повитухи, ибо Он побудил её к порождению жизни.
\vs Ode 19:10
Родила же
она, подобно сильному человеку, со страстью, и родила она согласно проявлению, и
приобрела она согласно Великой Силе.
\vs Ode 19:11
И любила она
искупительной (любовью), и оберегала с добротой, и вещала великолепно.
Аллилуйя.

\vs Ode 20:1
Я~--- священник
Яхве, и ему служу я священником;
\vs Ode 20:2
и Ему
предлагаю я подношение мысли Его.
\vs Ode 20:3
Ибо мысль Его
не подобна ни миру сему, ни плоти, ни поклоняющимся по законам плоти.
\vs Ode 20:4
Приношение же
Яхве~--- правда и чистота сердца и уст.
\vs Ode 20:5
Жертвуйте же
безупречно внутренней жизнью вашей, и да не погасится сострадание состраданием
вашим, и да не станете вы угнетать себя.
\vs Ode 20:6
Не подкупайте
иноземца, ибо он не похож на вас, и не следует также пытаться обмануть ближнего
вашего или же лишить его одежды, дабы обнажить его.
\vs Ode 20:7
Но щедро
положитесь на милость Яхве, и придите в Рай Его, и сделайте себе гирлянду из
древа Его.
\vs Ode 20:8
Затем положите
её на главу вашу, и будьте радостны, и положитесь на покой Его.
\vs Ode 20:9
ибо слава Его
проследует пред вами, и стяжаете вы от доброты Его и от милости Его, и помажут
вас в истине, с хвалою святости Его.
\vs Ode 20:10
Хвала и честь
имени Его.
Аллилуйя.

\vs Ode 21:1
Воздел я руки
ввысь во имя сострадания Яхве.
\vs Ode 21:2
Ибо совлек Он
с меня узы мои, а Помощник мой вознес меня соответственно состраданию Его и
спасению Его.
\vs Ode 21:3
И совлек я
тьму и облекся светом
\vs Ode 21:4
и даже сам
обрел чресла. В них не было болезни, или несчастья, или страдания.
\vs Ode 21:5
И щедро
помогала мне мысль Яхве, и Его вечное братство.
\vs Ode 21:6
И был я поднят
в свет, и я проследовал перед Ним.
\vs Ode 21:7
И постоянно
пребывал я подле Него, хваля и исповедуя Его.
\vs Ode 21:8
Подвиг Он
сердце моё переполниться, и нашли его в устах моих; и вросло оно в уста мои.
\vs Ode 21:9
Затем же
сделалось чертой лица моего ликование Яхве и похвала Его.
Аллилуйя.

\vs Ode 22:1
Он, подвигший
меня снизойти свыше и вознестись из мест нижних,
\vs Ode 22:2
и Он,
собирающий тех, что в середине, и сбрасывающий их ко мне,
\vs Ode 22:3
Он,
раскидавший врагов моих и соперников моих,
\vs Ode 22:4
Он, давший мне
власть над узами, дабы мог я развязать их,
\vs Ode 22:5
Он, моими
руками свергнувший дракона семиглавого и поставивший меня на корни его, дабы
сокрушил я семя его,
\vs Ode 22:6
Ты был там и
помогал мне, и в каждом месте имя Твоё окружало меня.
\vs Ode 22:7
Десница Твоя
сокрушила едкий яд его, и рука Твоя указала путь верующим в Тебя.
\vs Ode 22:8
И вызволила
она их из могил и отделила от мертвых.
\vs Ode 22:9
Она взяла
мертвые кости и покрыла их плотью.
\vs Ode 22:10
Но были они
неподвижны, поэтому дала она им жизненную силу.
\vs Ode 22:11
Непорочен был
путь Твой и лицо Твоё; привел Ты мир Свой к тлену, чтобы всё распалось и
обновилось.
\vs Ode 22:12
И основание
всего~--- скала Твоя. И на ней воздвиг Ты Царство Своё, и сделалось оно местом
обитания святых.
Аллилуйя.

\vs Ode 23:1
Радость~---
святым. И кто же облечется в неё, как не сами они?
\vs Ode 23:2
Милость~---
избранным. И кому же стяжать её, как не верующим в нее от начала?
\vs Ode 23:3
Любовь~---
избранным. И кто же облечется в неё, как не одержимые ею от начала?
\vs Ode 23:4
Ходите в
знании Яхве, и щедро познаете вы милость Яхве; как ради ликования Его, так и во
имя совершенства знания Его.
\vs Ode 23:5
И мысль Его
уподобилась письменам, и воля Его снизошла свыше
\vs Ode 23:6
и послана была
она подобно стреле из лука, выстрелившей с усилием.
\vs Ode 23:7
И много рук
поспешили к письму, дабы похитить его, а затем взять и прочесть его.
\vs Ode 23:8
Но ускользнуло
оно из пальцев их, и испугались они его, и печати, бывшей на нем.
\vs Ode 23:9
Ибо не
дозволялось им терять печать его, ибо власть печати этой была величественнее их.
\vs Ode 23:10
Но видевшие
письмо пришли за ним, чтобы изучить, где его должно бросить, и кто должен
прочесть его, и кто должен услышать его.
\vs Ode 23:11
Но колесу
досталось оно, и вошло оно чрез него.
\vs Ode 23:12
И знак был с
ним~--- Царства и Провидения.
\vs Ode 23:13
И всё,
мешавшее колесу, скосило оно и обрубило.
\vs Ode 23:14
И сдержало
оно множество недругов, и вымостило реки.
\vs Ode 23:15
И испещрило и
выкорчевало оно многие леса и расчистило путь.
\vs Ode 23:16
Пала глава к
ногам, ибо к ногам прикатилось колесо, и всё пришедшее с ним.
\vs Ode 23:17
Письмо же
было одним из повелений, и поэтому все области собрались воедино.
\vs Ode 23:18
И виден был
на главе его, на открывшейся главе, даже Сын Истины от Всевышнего Отца.
\vs Ode 23:19
И Он
наследовал всё и обладал всем, и прекратились затем происки многих.
\vs Ode 23:20
Затем же все
соблазнители заупрямились и сбежали, а преследователи увяли и были уничтожены.
\vs Ode 23:21
А письмо
сделалось большим томом, полностью начертанным перстом Божьим.
\vs Ode 23:22
И имя Отца
было на нем, а также Сына и Святого Духа, дабы властвовать вовеки веков.
Аллилуйя.

\vs Ode 24:1
Парила голубка
над главой нашего Яхве Помазанника, ибо был Он главой её,
\vs Ode 24:2
и пела она над
Ним, и услышан был глас её.
\vs Ode 24:3
Затем убоялись
живущие, и обеспокоились чужестранцы.
\vs Ode 24:4
Птица же стала
взлетать, и всякая тварь ползучая подохла в норе своей.
\vs Ode 24:5
И отверзались
и захлопывались бездны, и искали Яхве они, подобно готовым родить.
\vs Ode 24:6
Но не был Он
отдан им на съедение, ведь Он не принадлежал им.
\vs Ode 24:7
Но пропасти
погрузились в печать Яхве, и погибли они в той мысли, с которой оставались от
начала.
\vs Ode 24:8
Ибо от начала
пребывали в трудах они, и концом их тяжкого труда была жизнь.
\vs Ode 24:9
И все они,
пребывавшие в лишениях, погибли, ибо неспособны были они сказать такое слово,
чтобы суметь остаться.
\vs Ode 24:10
И сокрушил
Яхве символы всех не нашедших правды в них.
\vs Ode 24:11
Ибо их
обделили мудростью, их, занимавшихся самовосхвалением в разуме своем.
\vs Ode 24:12
Так их
отвергли, ибо правда не была с ними.
\vs Ode 24:13
Ибо Яхве
открыл путь Свой и широко распространил милость Свою.
\vs Ode 24:14
И те, кто
поняли это, познали святость Его.
Аллилуйя.

\vs Ode 25:1
Спасен я был
от уз моих и бежал к Тебе, о мой Господь.
\vs Ode 25:2
Ибо Ты~---
десница спасения и Спаситель мой.
\vs Ode 25:3
Ты сдержал
восставших против меня, и больше их не видели.
\vs Ode 25:4
Ибо лицо Твоё
пребывало со мной, спасая меня милостью Твоей.
\vs Ode 25:5
Я же был
презираемым и отверженным в глазах многих, и был я в их глазах свинцу подобен.
\vs Ode 25:6
И обрел я силу
от Тебя, и помощь.
\vs Ode 25:7
Светильник
водрузил ты ради меня и справа, и слева, чтобы не было во мне ничего
неосвещенного.
\vs Ode 25:8
И облекся я
покровом Духа Твоего и отбросил прочь от себя одежды кожаные.
\vs Ode 25:9
Ибо десница
Твоя вознесла меня и принудила болезнь оставить меня.
\vs Ode 25:10
И стал я
могучим истиной Твоей и святым правдой Твоей.
\vs Ode 25:11
И все недруги
мои убоялись меня, и сделался я (человеком) Яхве во имя Яхве.
\vs Ode 25:12
И оправдался
я добротой Его и покоем Его во веки веков.
Аллилуйя.

\vs Ode 26:1
Излил я хвалу
Яхве, ибо я~--- Он Сам.
\vs Ode 26:2
И зачитаю я
святую оду Его, ибо сердце моё~--- с Ним.
\vs Ode 26:3
Ибо арфа его в
руке моей, и не утихнут оды покоя Его.
\vs Ode 26:4
Желаю я
воззвать к Нему всем сердцем моим, желаю я восхвалять и возвышать Его всеми
чреслами моими.
\vs Ode 26:5
Ибо от востока
до запада пребывает хвала Его,
\vs Ode 26:6
а также с юга
на север простирается благодарение Его,
\vs Ode 26:7
даже с пиков
вершин и до края их в совершенстве Его.
\vs Ode 26:8
Кто же сумеет
записать оды Яхве и кто сумеет прочесть их?
\vs Ode 26:9
И кто сумеет
приуготовить себя к жизни, чтобы спастись самому?
\vs Ode 26:10
И кто сумеет
вынудить Всевышнего, чтобы Собственными устами зачитал Он?
\vs Ode 26:11
Кто же сумеет
истолковать чудеса Яхве? Убьют толкующего~--- истолкованное еще останется.
\vs Ode 26:12
Ибо
достаточно воспринять и удовольствоваться, ведь сочинители од стоят спокойные;
\vs Ode 26:13
подобно реке
с усиленно бьющим истоком и текущей на смену им, дабы отыскать это.
Аллилуйя.

\vs Ode 27:1
Простер я руки
свои и освятил Господа Моего,
\vs Ode 27:2
ибо
простирание рук моих~--- знак Его,
\vs Ode 27:3
а простирание
моё~--- вертикальный крест.
Аллилуйя.

\vs Ode 28:1
Как крылья
голубей распростерты над птенцами их, а клювики птенцов смотрят в клювы их, так
же и крылья Духа распростерты над сердцем моим.
\vs Ode 28:2
Сердце моё
непрерывно освежается и прыгает от радости, как малыш, прыгающий от радости во
чреве матери своей.
\vs Ode 28:3
Я веровал, а
значит пребывал я в покое, ибо верующий~--- Тот, в кого я уверовал.
\vs Ode 28:4
Он с чувством
благословил меня, и глава моя пребывает с Ним.
\vs Ode 28:5
Ни кинжал не
разделит меня с Ним, ни меч,
\vs Ode 28:6
ибо готов я
прежде, чем настанет погибель, и восстал в Его бессмертном уделе.
\vs Ode 28:7
И объяла меня
жизнь бессмертная, и облобызала меня.
\vs Ode 28:8
И от жизни
этой исходит Дух, Который во мне. И не может Он умереть, ибо Он~--- Жизнь.
\vs Ode 28:9
Видевшие же
меня изумились; ибо притесняли меня.
\vs Ode 28:10
И думали они,
что поглощали меня, ибо казался я им одним из пропавших.
\vs Ode 28:11
Но
несправедливость моя сделалась спасением моим.
\vs Ode 28:12
И сделался я
отвращением их, ибо не было во мне ревности.
\vs Ode 28:13
Ибо долго
делал я добро всякому, ненавидящему меня.
\vs Ode 28:14
И окружили
они меня словно собаки, что по глупости нападают на хозяев своих.
\vs Ode 28:15
Ибо мысль их
развращена и разум их извращен.
\vs Ode 28:16
Но несу воду
я в деснице моей, а горечь их продлил я приязнью своей.
\vs Ode 28:17
И не погиб я,
ибо ни братом их не был я, ни рождение моё не было подобным их.
\vs Ode 28:18
И искали они
смерти моей, но не нашли ее возможной, ибо был я старше, чем память их; и во
тщете массами набросились они на меня.
\vs Ode 28:19
И бывшие
после меня тщетно пытались сокрушить памятник Тому, Кто был пред ними.
\vs Ode 28:20
Ибо мысль
Всевышнего не могла быть предрассудком, а сердце Его превыше всякой мудрости.
Аллилуйя.

\vs Ode 29:1
Яхве~--- надежда
моя, и да не устыжусь Его.
\vs Ode 29:2
Ибо во хвале
Своей создал Он меня, и милостью Своей даже её воздал Он мне.
\vs Ode 29:3
И милостями
Своими возвеличил Он меня, и великой честью Своей возвысил Он меня.
\vs Ode 29:4
И побудил Он
меня восстать из глубин Шеола, и из уст смерти вызволил Он меня.
\vs Ode 29:5
И умалил я
врагов своих, и оправдал меня Он милостью Своей.
\vs Ode 29:6
Ибо уверовал я
в Помазанника Яхве и счел, что Он и есть Бог.
\vs Ode 29:7
И явил Он мне
знак, и водил меня светом Своим.
\vs Ode 29:8
И дал Он мне
скипетр власти Своей, чтобы подчинил я машины людские и умалил силу могущества,
\vs Ode 29:9
воевал словом
Его и одержал победу силой Его.
\vs Ode 29:10
И низверг
Яхве врага моего словом Своим, и тот уподобился пыли, сдуваемой ветром.
\vs Ode 29:11
И воздал я
хвалу Всевышнему, ибо возвеличил Он слугу Своего и Сына служанки Своей.
Аллилуйя.

\vs Ode 30:1
Наполнитесь же
водою из живого источника Яхве, ибо он открылся вам,
\vs Ode 30:2
и придите все
алчущие и напейтесь, и отдохните подле источника Яхве,
\vs Ode 30:3
ибо приятен и
сверкающ он и вечно освежает.
\vs Ode 30:4
Ибо много
слаще вода Его, нежели мёд, и соты пчелиные не сравнятся с ним;
\vs Ode 30:5
ибо исходил он
из уст Яхве, а вызван был из сердца Яхве.
\vs Ode 30:6
И пришла она
бесконечной и невидимой, и пока не налилась в середине, они не узнали её.
\vs Ode 30:7
Блаженны же
пившие оттуда и освежившиеся ею.
Аллилуйя.

\vs Ode 31:1
Пред Яхве
исчезали пропасти, а Тьма рассеивалась пред появлением Его.
\vs Ode 31:2
Ошибка же
ошиблась и погибла из-за Него, а неуважению не дали тропу, ибо была она
затоплена правдой Яхве.
\vs Ode 31:3
Отверз Он уста
Свои и изрек милость и ликование, и зачитал новый гимн имени Своему.
\vs Ode 31:4
Затем же
возвысил Он глас Свой ко Всевышнему и вверил ему ставших Сыновьями из-за Него.
\vs Ode 31:5
И лицо Его
признано было, ибо так воздал Ему Святой Отец Его.
\vs Ode 31:6
Придите,
обиженные, и возрадуйтесь.
\vs Ode 31:7
И владейте
собою милостиво и вберите в себя жизнь бессмертную.
\vs Ode 31:8
И осудили меня
они, когда восстал я,~--- меня, не осужденного.
\vs Ode 31:9
Затем же
разделили они добро моё, хотя ничего не причиталось им.
\vs Ode 31:10
Но вытерпел
я, и держал себя в руках, и безмолвен был, дабы не быть умерщвленным ими.
\vs Ode 31:11
Но
непоколебимо стоял я, словно твердая скала, долгое время побиваемая накатами
волн, и терпел.
\vs Ode 31:12
И желчность
их смиренно вынес я, дабы искупить народ мой и наставить его,
\vs Ode 31:13
и дабы не
отречься от обетов патриархам, которым был обещан я во спасение потомков их.
Аллилуйя.

\vs Ode 32:1
Блаженным~---
радость сердец их и свет Того, Кто пребывает в них,
\vs Ode 32:2
и Слово правды
само родившееся,
\vs Ode 32:3
ибо усилен был
Он Святою Силою Всевышнего, и не поколеблен Он вовеки веков.
Аллилуйя.

\vs Ode 33:1
Но вновь
поспешила милость и отвергла Искусителя и снизошла в него, дабы низвергнуть его.
\vs Ode 33:2
И вызвал он
сплошное разрушение перед собой и испортил весь труд свой.
\vs Ode 33:3
И стоял он на
пике горной вершины и громко вопил из конца в конец земли.
\vs Ode 33:4
Затем приволок
он к нему всех подчиненных ему, ибо не как грешник явился он.
\vs Ode 33:5
Однако именно
совершенная дева стояла, проповедуя, и взывая, и говоря:
\vs Ode 33:6
О, вы, сыновья
человеческие, вернитесь, и вы, дщери их, придите,
\vs Ode 33:7
и оставьте
пути Искусителя этого, и приблизьтесь ко мне.
\vs Ode 33:8
И войду я в
вас, и выведу из разрушения вас, и сделаю мудрыми вас на путях правды.
\vs Ode 33:9
Не будьте же
ни грешными, ни гибнущими.
\vs Ode 33:10
Повинуйтесь
мне, и спасены будете, ибо возглашу я вам милость Божью.
\vs Ode 33:11
И чрез меня
спасетесь вы и сделаетесь блаженными. Я~--- суд ваш;
\vs Ode 33:12
и
положившихся на меня не обвинят неправедно, но нетленность стяжают они в новом
мире.
\vs Ode 33:13
Избранные мои
пошли за мною, и пути мои сделаю я известными тем, кто взыскивает меня; и
завещаю я им имя своё.
Аллилуйя.

\vs Ode 34:1
Нет ни
трудного пути там, где есть простодушие, ни преграды для прямодушия,
\vs Ode 34:2
ни урагана в
глубине просветленной мысли.
\vs Ode 34:3
Там, где
окружен некто со всех сторон приятной страной, там ничто не разделено в нем.
\vs Ode 34:4
То, что внизу,
подобно тому, что вверху.
\vs Ode 34:5
Ибо всё~---
свыше, а снизу~--- ничего, но спорят с этим те, в ком нет понимания.
\vs Ode 34:6
Милость же
явлена во спасение ваше.
Аллилуйя.

\vs Ode 35:1
Живительный
ливень Яхве преспокойно накрыл меня и облако покоя: побудили они взойти над
головой моей,
\vs Ode 35:2
чтобы могло
оно оберегать меня во все времена. И стало оно спасением мне.
\vs Ode 35:3
Всякий
обеспокоился и убоялся, и изошли из них дым и судилище.
\vs Ode 35:4
Я же спокоен
был в воинстве Яхве; больше, нежели тенью был он для меня, и больше, нежели
основанием.
\vs Ode 35:5
И носили меня,
как мать дитя своё, Он же дал мне молоко, росу Яхве.
\vs Ode 35:6
И обогатился я
пользою Его, и покоился в совершенстве Его.
\vs Ode 35:7
И распростер я
в вознесении руки свои, и направился ко Всевышнему, и искуплен был пред Ним.
Аллилуйя.

\vs Ode 36:1
Покоился я в
Духе Яхве, и Он вознес меня к небесам;
\vs Ode 36:2
и велел мне
стать на ноги в вышнем месте Яхве, перед совершенством Его и славой Его, где и
продолжил я славить Его сочинением од Его.
\vs Ode 36:3
Дух же принес
меня к лицу Яхве, а поскольку был я Сыном человека, меня назвали Светом, Сыном
Божьим;
\vs Ode 36:4
ибо был я
достославным среди славных и величайшим среди великих.
\vs Ode 36:5
Ибо к величию
Всевышнего сделал меня Он, и по новизне Своей обновил Он меня.
\vs Ode 36:6
И помазал Он
меня совершенством Своим, и сделался я одним из тех, кто подле Него.
\vs Ode 36:7
И открылись
уста мои, словно облако росы, и хлынуло сердце моё как скважина праведности.
\vs Ode 36:8
И приближение
моё было мирным, и поставили меня в Духе Провидения.
Аллилуйя.

\vs Ode 37:1
Воздел я руки
свои к Яхве, и ко Всевышнему возвысил свой глас я.
\vs Ode 37:2
И говорил я
устами сердца своего, и слышал Он меня, когда глас мой достигал Его.
\vs Ode 37:3
Его же Слово
изошло ко мне, дабы воздались мне плоды трудов моих,
\vs Ode 37:4
и воздался бы
покой мне милостью Яхве.
Аллилуйя.

\vs Ode 38:1
Взошел я во
Свет Истины словно в колесницу, и Истина вела меня и велела прийти,
\vs Ode 38:2
и велела мне
пройти через пропасти и заливы и спасала меня от скал и долин,
\vs Ode 38:3
и стала для
меня небесами спасения, и водрузила меня в месте бессмертной жизни.
\vs Ode 38:4
И шел Он со
мною и велел мне отдыхать и не позволял мне оступаться, ибо был Он и является
Истиной.
\vs Ode 38:5
Мне было
безопасно, ибо я всегда шел с Ним, и не оступался ни в чем, ибо я слушался Его;
\vs Ode 38:6
ибо ошибка
сбежало от Него и никогда не сталкивалась с Ним.
\vs Ode 38:7
Но Истина шла
прямым путем, и всё, чего не смыслил я, являл Он мне:
\vs Ode 38:8
все яды Ошибки
и смертельные болезни, считавшиеся сладостными.
\vs Ode 38:9
И поражая
Искусителя, я видел, как украшена была порочная невеста, и жениха, совращающего
и совращаемого.
\vs Ode 38:10
И спросил я
Истину: Кто они? И сказала Она мне: Это обманщик и ошибка,
\vs Ode 38:11
и подражают
они Возлюбленному и Невесте Его, и понуждают мир сей оступиться и совращают его.
\vs Ode 38:12
И зовут они
многих на пир брачный, и позволяют им пить вино отравы своей,
\vs Ode 38:13
дабы вынудить
их вытошнить мудростью и знанием их, и готовят для них бессмыслицу.
\vs Ode 38:14
Затем же они
покидают их, и так они спотыкаются, словно безумные и растленные люди,
\vs Ode 38:15
ведь в них
нет ведения, да и не ищут они его.
\vs Ode 38:16
Но я сделался
мудрым, дабы не пасть в руки обманщиков, и возрадовался внутри себя, ибо истина
шла со мной.
\vs Ode 38:17
Ибо я был
создан, и жил, и был искуплен, и начала мои легли из-за руки Яхве, ибо посадил
Он меня.
\vs Ode 38:18
Ибо посадил
Он корень, и полил его, и ухаживал за ним, и благословлял его, и плоды его
пребудут вовеки.
\vs Ode 38:19
Он глубоко
врос, и пророс, и развился, и полнился, и ширился,
\vs Ode 38:20
и Яхве одним
славился, посадкой Его и выращиванием Его,
\vs Ode 38:21
и заботой
Его, и благословением уст Его, в чудесном саду одесную Его,
\vs Ode 38:22
и в знаниях
сада Его, и в понимании разума Его.
Аллилуйя.

\vs Ode 39:1
Свирепые реки
сила Яхве, они бросают головой вниз презирающих Его,
\vs Ode 39:2
и путают тропы
их, и разрушают переправы их,
\vs Ode 39:3
и хватают тела
их, и растлевают естества их.
\vs Ode 39:4
Ибо они~---
быстрее молний, даже скорее.
\vs Ode 39:5
Но не помешают
тем, кто переправляется через них с верою
\vs Ode 39:6
и не выбросят
тех, кто безупречно сплавляется по ним.
\vs Ode 39:7
Ибо знак на
них~--- Сам Яхве, и знак этот~--- путь для переправляющихся во имя Яхве.
\vs Ode 39:8
Затем же
положитесь на имя Всевышнего и познайте Его, и вы переправитесь безопасно, ибо
реки станут послушны вам.
\vs Ode 39:9
Яхве же
вымостил их Словом Своим, и он ходил и пересекал их как посуху.
\vs Ode 39:10
И твердо
отпечатывались на водах следы Его, и не стирались, но подобны были бруску древа,
на Истине выстроенного.
\vs Ode 39:11
С обеих
сторон вздымались волны, но следы нашего Яхве Помазанника оставались тверды,
\vs Ode 39:12
и ни намокали
они, ни разрушались.
\vs Ode 39:13
И путь этот
был предначертан переправляющимся следом за Ним, и строго следующим стезею веры
Его, и чтящим имя Его.
Аллилуйя.

\vs Ode 40:1
Как истекает
мёд из сот пчелиных, а молоко~--- из жены, любящей детей своих, так и надежда на
Тебя, Боже мой.
\vs Ode 40:2
Как хлынет
вода из фонтана, так и сердце моё хлынет восхвалением Яхве, и вознесут хвалу Ему
уста мои.
\vs Ode 40:3
И усладится
язык мой гимнами Его, и чресла мои помажутся одами Его.
\vs Ode 40:4
Лицо же моё
веселится ликованием Его, и дух мой ликует в любви Его, и естество моё светится
в Нем.
\vs Ode 40:5
И уверует в
Него убоявшийся, и искупления достигнет в Нем.
\vs Ode 40:6
И владения Его
жизнь бессмертная, и стяжавшие её непорочны.
Аллилуйя.

\vs Ode 41:1
Да восхвалят
Яхве все чада Яхве, да стяжаем мы истину веры Его.
\vs Ode 41:2
И дети его да
утвердятся в Нем, а поэтому воспоем же любви Его.
\vs Ode 41:3
Живем мы в
Яхве милостью Его, а жизнь стяжали мы у Помазанника Его.
\vs Ode 41:4
Ибо великий
день засветил нам, и чудесен Тот, кто воздал нам славою Своей.
\vs Ode 41:5
Пребудем же
поэтому все мы в согласии во имя Яхве и воздадим почести в доброте Его.
\vs Ode 41:6
И да воссияют
лица наши во свете Его, и да сосредоточатся сердца наши на любви Его и днем, и
ночью.
\vs Ode 41:7
Так возликуем
же ликованием Яхве!
\vs Ode 41:8
Все видящие
меня да изумятся, ибо я~--- из иного рода.
\vs Ode 41:9
Ибо Отец
Истины вспомнил обо мне, Он, от начала владеющий мною.
\vs Ode 41:10
Ибо богатства
Его породили меня, и мысль сердца Его.
\vs Ode 41:12
И пребывает с
нами Слово Его на всем пути нашем~--- Спаситель, дающий жизнь и не отвергающий
нас.
\vs Ode 41:13
Сын же
Всевышнего явился в совершенстве Отца Своего.
\vs Ode 41:14
И забрезжил
из Слова свет, пребывавший в Нем прежде времени.
\vs Ode 41:15
Помазанник же
один в истине. И знали Его прежде основ мира сего, чтобы истиной имени Своего
вовеки наделял он жизнью людей.
\vs Ode 41:16
Новая же
песнь~--- Яхве, от любящих Его.
Аллилуйя.

\vs Ode 42:1
Распростер я
руки и приблизился к Яхве, ибо распростертые руки мои~--- знак Его.
\vs Ode 42:2
И
распростертые мои~--- вертикальный крест, поднявшийся на стезе Праведного.
\vs Ode 42:3
И сделался я
бесполезным для не знавших меня, ибо скроюсь я от тех, кто не одержим мною,
\vs Ode 42:4
а пребуду с
любящими меня.
\vs Ode 42:5
Все
преследовавшие меня мертвы, искавшие меня, злословившие против меня,~--- ибо жив
я.
\vs Ode 42:6
К тому же
воскрес я, и я с ними, и буду говорят устами их.
\vs Ode 42:7
Ибо отвергли
они притеснявших их, и запечатлел их клеймом любви моей.
\vs Ode 42:8
Подобно клейму
жениха на невесте клеймо моё на знающих меня.
\vs Ode 42:9
И как чертог
брачный выстраивается родными брачной пары, так и любовь моя (живет) верующими в
меня.
\vs Ode 42:10
Меня не
отвергли, хотя и хотели, и не погиб я, хотя они и считали меня (погибшим).
\vs Ode 42:11
Шеол же
увидел меня и поколебался, и Смерть низвергла меня и многих вместе со мною.
\vs Ode 42:12
Я был уксусом
и горечью его, и спустился я с нею вниз на глубину его.
\vs Ode 42:13
Затем же
отпустил он ноги и голову, ибо был он неспособен выносить лицо моё.
\vs Ode 42:14
И создал я
собрание живых среди его мертвых и говорил с ними устами живыми, дабы слово моё
не было бесполезным.
\vs Ode 42:15
И ринулись ко
мне умершие, и возопили, и заголосили:
<<Сын Божий, смилуйся над нами,
\vs Ode 42:16
и поступи с
нами по доброте Твоей, и вызволи нас из оков тьмы,
\vs Ode 42:17
и открой нам
дверь, в которую мы сможем выйти к Тебе, ибо осознали мы, что смерть наша не
коснулась Тебя.
\vs Ode 42:18
А еще да
спасемся с Тобою, ибо Ты~--- Спаситель наш.>>
\vs Ode 42:19
Затем услышал
я голос их, и вложил веру их в сердце моё.
\vs Ode 42:20
И вложил я
имя своё в головы их, ибо свободны они и принадлежат мне.
Аллилуйя.

\bibbookdescr{Tsm}{
  inline={Завещание Симеона,\\второго сына Иакова и Лии\fns{В греч. тексте $+$ ``о зависти''.}},
  toc={Завещание Симеона},
  bookmark={Завещание Симеона},
  header={Завещание Симеона},
  abbr={Сим}
}
\vs Tsm 1:1
Список слов Симеона, речённых им к сыновьям его перед тем,
как умер он в 120-ый год жизни своей,
в тот же год, что и брат его Иосиф.
\vs Tsm 1:2
Когда занемог Симеон, пришли проведать его дети его, и, сделав
усилие, сел он, поцеловал их и сказал:
\vs Tsm 2:1
послушайте, дети мои, Симеона, отца вашего; возвещу вам то,
что имею я в сердце моём.

\vs Tsm 2:2
Родился я от Иакова и был вторым сыном отца моего, и Лия, мать моя,
нарекла меня Симеоном, ибо услышал Господь мольбу ее.
\vs Tsm 2:3
Сделался я весьма сильным, не боялся труда и не страшился никакого дела.
\vs Tsm 2:4
Ибо сердце моё было сухим, печень моя недвижимой,
а внутренности мои нечувствительными.
\vs Tsm 2:5
Ведь и мужество даётся от Всевышнего людям в душах и телах.
\vs Tsm 2:6
Во время юности моей завидовал я сильно Иосифу,
ибо возлюбил его отец мой более всех.
\vs Tsm 2:7
И утвердился я против него в сердце моём, возжелав убить его,
так как Князь обмана и дух зависти ослепили мне ум, и забыл я,
что это брат мой, и не пощадил отца моего Иакова.
\vs Tsm 2:8
Но Бог его и Бог отцов наших послал ангела своего и избавил
Иосифа от рук моих.

\vs Tsm 2:9
Ибо, когда я отправился в Сиким, чтобы принести притирание для стада,
а Рувим  в Дофаим, где было необходимое нам и все хранилища наши,
Иуда, брат мой, продал Иосифа Измаильтянам.
\vs Tsm 2:10
Рувим, услышав об этом, опечалился, ибо он хотел отвести его к отцу.
\vs Tsm 2:11
Я же, услышав это, сильно разгневался на Иуду,
ибо он отпустил Иосифа живым,
и 5 месяцев пребывал я в гневе на него.
\vs Tsm 2:12
И сковал меня Господь и удалил от меня дело рук моих,
ибо правая рука моя стала наполовину сухой на 7 дней.
\vs Tsm 2:13
И познал я, дети, что из-за Иосифа случилось это со мною.
И, раскаявшись, заплакал я и молил Господа Бога,
чтобы восстановилась рука моя и удержался я от всякой скверны
и зависти и ото всякого безрассудства.
\vs Tsm 2:14
Ибо понял я, что злое дело замыслил перед лицом Господа и Иакова,
отца моего, против Иосифа, брата моего, позавидовав ему.

\vs Tsm 3:1
Ныне, дети мои, послушайте меня и остерегитесь духа обмана и зависти.
\vs Tsm 3:2
Ведь зависть властвует надо всем помыслом человека
и не дает ему ни есть, ни пить, ни делать ничего доброго.
\vs Tsm 3:3
Но всечасно подстрекает она убить того, кому человек завидует,
но тот всечасно процветает, а завистник чахнет.
\vs Tsm 3:4
И вот, 2 года сокрушал я в страхе Господнем душу мою постом.
И узнал я, что избавление от зависти происходит через страх Божий.
\vs Tsm 3:5
Если кто прибегает к Господу, оставляет его злой дух
и становится разум лёгким.
\vs Tsm 3:6
И наконец, начинает он сочувствовать тому, кому завидовал, и
примиряется с любящими его, и так избавляется от зависти.

\vs Tsm 4:1
Спросил отец мой, что со мною, ибо заметил меня скорбящим, и
сказал я ему, что переполняется печень моя.
\vs Tsm 4:2
Ибо печалился я чрезвычайно, что виновен в продаже Иосифа.
\vs Tsm 4:3
И когда пошли мы в Египет и связали меня как соглядатая,
познал я, что справедливо страдаю и не опечалился.
\vs Tsm 4:4
Иосиф же был добрый муж, дух Божий в себе имевший,
милостивый и сострадательный; не вспомнил мне зла, но
возлюбил меня с братьями моими.

\vs Tsm 4:5
Так остерегайтесь же, дети мои, всякой ревности и зависти и живите в
простоте сердечной, чтобы дал и вам Бог милость и славу и благословение на
головы ваши, как вы видите то на Иосифе.
\vs Tsm 4:6
Ни в какой день не стыдил он нас за дело это,
но возлюбил нас как душу свою, и более сыновей своих почтил нас,
и богатство, и скот, и плоды даровал нам.

\vs Tsm 4:7
И вы, дети мои, возлюбите каждый брата своего в доброте сердечной,
и отойдёт от вас дух зависти.
\vs Tsm 4:8
Ибо озлобляет он душу и губит тело, гнев и вражду вводит в
помышление и побуждает к крови и вводит разум в экстаз,
и смятение создает в душе и дрожь в теле.
\vs Tsm 4:9
Даже во сне злая зависть, соблазняя человека,
пожирает его и духами злыми возмущает душу его,
и заставляет тело его содрогаться,
и смятением лишает сна ум его,
и как дух злой и губительный является людям.
\vs Tsm 5:1
Оттого Иосиф был прекрасен лицом и приятен видом своим,
что не поселялось в нем ничто злое;
ибо смущение духа проступает явно на лице человека.

\vs Tsm 5:2
Ныне, дети мои, смягчите сердце ваше пред Господом
и выпрямите пути ваши пред людьми,
и стяжаете благодать пред лицом Господа и людей.
\vs Tsm 5:3
И остерегайтесь блуда, ибо блуд порождает всякое зло,
отдаляя от Бога и приближая к Велиару.
\vs Tsm 5:4
Видел я в книге Еноха, что сыновья ваши совратятся
от блуда и обиду нанесут мечом своим сыновьям Левия.
\vs Tsm 5:5
Но не смогут они противостоять Левию,
ибо поведёт он брань Господню и одолеет всякое войско ваше.
\vs Tsm 5:6
И будут они малочисленны, разделенные в Левин и в Иуде, и
не будет из вас никого, кто властвовал бы,
как и пророчествовал отец наш в благословениях своих.

\vs Tsm 6:1
И вот, сказал я вам всё, дабы оправдать себя от греха вашего.
\vs Tsm 6:2
И если удалите от себя зависть и всякое жестокосердие,
словно роза расцветут кости мои в Израиле,
и словно лилия плоть моя в Иакове,
и будет благоухание моё словно аромат Ливана,
и умножатся святые от меня во веки веков,
и взрастут отрасли их.
\vs Tsm 6:3
Тогда погибнет семя Ханаана,
и не будет остатка у Амалика,
и сгинут все Каппадокийцы,
и все Хетты истребятся.
\vs Tsm 6:4
Тогда угаснет земля Хама, и погибнет весь народ.
Тогда почиет вся земля от смуты, и всё, что под небесами, от войны.
\vs Tsm 6:5
Тогда прославится Сим,
ибо Господь Бог Израиля придет на землю [как человек] и тем
спасёт Адама.
\vs Tsm 6:6
Тогда предан будет всякий дух соблазна на поругание,
и люди обретут власть над злыми духами.
\vs Tsm 6:7
Тогда воскресну и я в радости и благословлю Всевышнего ради чудес его,
[ибо Господь, приняв тело и вкусив пищу с людьми, спас людей.]

\vs Tsm 7:1
Ныне, дети мои, слушайте Левия и Иуду,
и не восставайте на два эти колена,
ибо от них исполнится нам спасение Божие.
\vs Tsm 7:2
Ибо восстанет Господь из Левия как Первосвященник,
а из Иуды как Царь [Бог и человек].
Он спасёт [все народы и] род Израиля.
\vs Tsm 7:3
Для того внушаю вам это, дабы и вы внушили детям вашим,
да сохранят всё в поколениях своих.

\vs Tsm 8:1
Завершил Симеон наставление сыновей своих и почил с отцами
своими, будучи 120-и лет.
\vs Tsm 8:2
И положили его во гроб деревянный,
чтобы отнести кости его в Хеврон.
И отнесли их втайне, пока Египтяне вели войну.
\vs Tsm 8:3
Ибо кости Иосифа сохранили Египтяне в гробнице царей.
\vs Tsm 8:4
Сказали им прорицатели, что, если вынесут кости Иосифа,
тьма и мрак будут по всей земле и несчастье великое Египтянам,
так что и со светильником не узнает никто брата своего.

\vs Tsm 9:1
И оплакали сыновья Симеона, отца своего.
И пребывали в Египте вплоть до дней, когда Моисей вывел их рукою своею.

\bibbookdescr{1Sb}{
  inline={Первая книга Сивилл},
  toc={1-я Сивилл},
  bookmark={1-я Сивилл},
  header={1-я Сивилл},
  abbr={1~Сив}
}
\vs 1Sb 1:1 С самых истоков начав, возвещу я судьбу поколений,

\vs 1Sb 1:2 Все по порядку скажу от первого века и дальше,

\vs 1Sb 1:3 То, что случилось уже, что есть и что впредь ожидает

\vs 1Sb 1:4 Смертных людей, преступивших священный закон благочестья.

\vs 1Sb 1:5 Первым мне Бог повелел рассказать правдиво о том, как

\vs 1Sb 1:6 Мир порожден был,  а ты внимай моим песням прилежно,

\vs 1Sb 1:7 Смертный, чтобы из них ни слова зря не пропало.

\vs 1Sb 1:8 Царь, всех превыше стоящий, создал и небо и землю,

\vs 1Sb 1:9 Да зародится,  сказав, и тут же все зародилось.

\vs 1Sb 1:10 Тартаром твердь окружив, Он свет дал миру сладчайший,

\vs 1Sb 1:11 Сверху воздвиг небосвод, простер воды светлого моря,

\vs 1Sb 1:12 Множество ярких созвездий обвил вкруг полюса, землю

\vs 1Sb 1:13 Всю цветами украсил, смешал с потоками море,

\vs 1Sb 1:14 Воздух ветрами смутил и влажные дал ему тучи.

\vs 1Sb 1:15 К тварям живым перейдя, Он рыб глубинам доверил,

\vs 1Sb 1:16 Птиц  воздушным потокам, чащобам  зверей густошерстых,

\vs 1Sb 1:17 Гадов пустил по земле, и все, что ныне мы видим,

\vs 1Sb 1:18 Словом единым создал, и по слову все появилось

\vs 1Sb 1:19 Быстро и точно: сие созерцает теперь Нерожденный,

\vs 1Sb 1:20 С неба на землю смотря,  на том Его труд завершен был.

\vs 1Sb 1:21 И уже после того слепил Он живое творенье 

\vs 1Sb 1:22 Образ Свой запечатлев, человека, прекрасного видом,

\vs 1Sb 1:23 Сходного с Богом. Ему повелел в раю поселиться,

\vs 1Sb 1:24 Чтобы благие дела предметом забот его были.

\vs 1Sb 1:25 Тот, оказавшись один средь цветущего райского сада,

\vs 1Sb 1:26 Стал по беседе скучать и вседневно желаньем томился

\vs 1Sb 1:27 Облик увидеть такой же, как свой. Тут, кость его вынув,

\vs 1Sb 1:28 Бог из нее сотворил супругу законную  Еву,

\vs 1Sb 1:29 Женщину дивной красы, и велел ей в раю с человеком

\vs 1Sb 1:30 Жить совместно. Адам, ее оглядев, удивлен был.

\vs 1Sb 1:31 Радуясь сердцем, смотрел на себе подобную. С речью

\vs 1Sb 1:32 К ней обратился разумной, и сами собой получались

\vs 1Sb 1:33 Те слова у него  предусмотрено все было Богом.

\vs 1Sb 1:34 Похоть не застила ум их, стыда они не знавали,

\vs 1Sb 1:35 Были сердца далеки от всякого зла. Словно звери,

\vs 1Sb 1:36 Тело свое напоказ безпечно они выставляли.

\vs 1Sb 1:37 Сразу же после того, как создал, им указанье

\vs 1Sb 1:38 Дал Господь, чтоб они не трогали древа: на это

\vs 1Sb 1:39 Змей их ужасный подбил, на горе обманом заставив

\vs 1Sb 1:40 Смертную долю принять, а с нею вместе  познанье

\vs 1Sb 1:41 Зла и Добра. Между тем, предательство первой свершила

\vs 1Sb 1:42 Женщина: мужу дала, убедив неразумного словом.

\vs 1Sb 1:43 Он же, речами жены увлечен, позабыл о безсмертном

\vs 1Sb 1:44 Мира Творце и совет без внимания мудрый оставил.

\vs 1Sb 1:45 Так получили они по заслугам, когда им досталось

\vs 1Sb 1:46 Зло вместо блага в удел. Проткнув смоковницы листья,

\vs 1Sb 1:47 Сшили одежды себе и друг на друга надели,

\vs 1Sb 1:48 Чресла листвою прикрыв, ибо стыд у них появился.

\vs 1Sb 1:49 Бог же безсмертный обрушил Свой гнев, их выгнал из Рая

\vs 1Sb 1:50 В смертной юдоли свой век коротать, когда не хранили,

\vs 1Sb 1:51 Раз услыхав, они в памяти слово великого Бога.

\vs 1Sb 1:52 Те, очутившись внезапно среди плодородной равнины,

\vs 1Sb 1:53 Стали слезами ее поливать, непрерывно стеная.

\vs 1Sb 1:54 К ним смягчился Безсмертный тогда и сказал в утешенье:

\vs 1Sb 1:55 Род продолжайте, плодитесь, с землей обращайтесь умело,

\vs 1Sb 1:56 Так, чтобы в поте лица добывали, чем голод насытить.

\vs 1Sb 1:57 Слово такое изрек. В обмане виновного змея

\vs 1Sb 1:58 Землю заставил тереть животом и хвостом, без пощады

\vs 1Sb 1:59 Выгнав из Рая. Вражду тогда между ним поселил Он

\vs 1Sb 1:60 И человеком. Один уберечь свою голову тщится,

\vs 1Sb 1:61 Пятку спасает другой: близка ведь стала отныне

\vs 1Sb 1:62 Смерть и к людям, и к тем, кто злом отравляет советы.

\vs 1Sb 1:63 Начал тут род пополняться людской, как велено было

\vs 1Sb 1:64 Им, всемогущим Владыкой. Одно за другим приходили,

\vs 1Sb 1:65 Множа число, поколенья. Дома они начали строить,

\vs 1Sb 1:66 Стены и города возводить с немалым искусством.

\vs 1Sb 1:67 Долгий и радостный день им сопутствовал в жизни. Не зная

\vs 1Sb 1:68 Горя, смерть принимали они, погружаясь как будто

\vs 1Sb 1:69 В сон. Счастливыми были те люди; могучих героев,

\vs 1Sb 1:70 Бог возлюбил их, безсмертный Спаситель и Царь. Но однако

\vs 1Sb 1:71 Стали и эти в безумье грешить, безстыдно принявшись

\vs 1Sb 1:72 На смех отцов выставлять и над матерями глумиться.

\vs 1Sb 1:73 С близкими начали тут обращаться они, как с чужими,

\vs 1Sb 1:74 Брат поднял руку на брата. Пресытились кровью убитых,

\vs 1Sb 1:75 Ею себя запятнав, вели безразсудные войны.

\vs 1Sb 1:76 Пала за это на них с небес наивысшая кара:

\vs 1Sb 1:77 Люди из жизни теснимы быть начали. Всех их, преступных,

\vs 1Sb 1:78 Принял Аид. Называют его Аидом с тех пор, как

\vs 1Sb 1:79 Первым в нем очутился Адам, чашу смерти пригубив.

\vs 1Sb 1:80 Всюду его обступила земля. И это причиной

\vs 1Sb 1:81 Стало того, что о тех, кто живет на земле, говорится:

\vs 1Sb 1:82 В царство Аида уходят. Однако, и сгинув в Аиде,

\vs 1Sb 1:83 Первые люди почет заслужили, что первым дается.

\vs 1Sb 1:84 Сразу же после того, как их земля поглотила,

\vs 1Sb 1:85 Род сотворил Он другой  из тех, кто еще оставался

\vs 1Sb 1:86 Праведной жизни. Они трудились усердно, прекрасны

\vs 1Sb 1:87 Были дела их, стыдом превзошли остальных и имели

\vs 1Sb 1:88 Разум надежный. Искусства им были знакомы. Искали

\vs 1Sb 1:89 Выход в любом затрудненье и быстро его находили.

\vs 1Sb 1:90 Способ один изобрел, как надо вспахивать плугом

\vs 1Sb 1:91 Землю. Другой размышлял над тем, как строить прочнее,

\vs 1Sb 1:92 Третий  как по морю плавать, по птицам гадать, и на небе

\vs 1Sb 1:93 Звезды четвертый умел наблюдать. Про яды знал пятый.

\vs 1Sb 1:94 Магия делом была еще одного. Все ремесла

\vs 1Sb 1:95 Разным поручены были умельцам. Безсонными звали

\vs 1Sb 1:96 Хлебоедами их, поскольку они отличались

\vs 1Sb 1:97 Вечно ясным умом и ненаполнимым желудком.

\vs 1Sb 1:98 Телом могучие, все отошли, однако, под своды

\vs 1Sb 1:99 Страшного царства Аида. Там, скованны прочно цепями,

\vs 1Sb 1:100 Грех свой должны искупать, пребывая в геенне, где пламя

\vs 1Sb 1:101 Неугасимое жжет и огонь жестокий пылает.

\vs 1Sb 1:102 Вслед за ушедшими племя явилось, мощное духом.

\vs 1Sb 1:103 Третьим было по счету оно. Надменных и дерзких

\vs 1Sb 1:104 Объединяло людей, которые многие беды

\vs 1Sb 1:105 В мир принесли. Очень скоро сражения, войны, убийства

\vs 1Sb 1:106 Их истребили, носивших в груди жестокое сердце.

\vs 1Sb 1:107 Та же причина была, что еще один род прекратился.

\vs 1Sb 1:108 Младшее из четырех людских поколений, и это

\vs 1Sb 1:109 Кровью себя осквернило, повсюду ее проливая.

\vs 1Sb 1:110 Был им страх перед Богом неведом, как друг перед другом 

\vs 1Sb 1:111 Чувство стыда. Наконец, против них же самих обратились

\vs 1Sb 1:112 Гнев, сводящий с ума, совместно с буйным нечестьем.

\vs 1Sb 1:113 Так повергли несчастных убийства, сражения, войны

\vs 1Sb 1:114 В мрак преисподней  мужей, преступивших закон. Их небесный,

\vs 1Sb 1:115 Гневаясь, Бог перенес потом за пределы вселенной,

\vs 1Sb 1:116 Тартаром отгородив под самой земли сердцевиной.

\vs 1Sb 1:117 Был за этим еще один род человеческий создан,

\vs 1Sb 1:118 Много хуже других. Ему злую участь безсмертный

\vs 1Sb 1:119 Бог уготовил, когда творить беззаконие стали.

\vs 1Sb 1:120 Нравом надменнее были они, чем прежние люди, 

\vs 1Sb 1:121 Племя Гигантов, в речах нечестиво хулившее Бога.

\vs 1Sb 1:122 Только один среди всех человек был правдивый и верный 

\vs 1Sb 1:123 Ной, закон почитавший и думавший лишь о хорошем.

\vs 1Sb 1:124 Вот с такими словами с небес к нему Бог обратился:

\vs 1Sb 1:125 Мужество, Ной, собери, тотчас призови к покаянью

\vs 1Sb 1:126 Всех людей на земле, чтоб свои они жизни спасали.

\vs 1Sb 1:127 Дела безстыжим коль нет до того, что повсюду творится,

\vs 1Sb 1:128 Род Я их весь погублю невиданным прежде потопом.

\vs 1Sb 1:129 Ты же на прочной основе, воде не дающей прохода,

\vs 1Sb 1:130 Дом себе быстро построй деревянный, надежно стоящий.

\vs 1Sb 1:131 Знание дам Я тебе для того и умение строить,

\vs 1Sb 1:132 Дам укромное место, размер  обо всем позабочусь,

\vs 1Sb 1:133 Так что спасешься ты сам и все, кто живут с тобой вместе.

\vs 1Sb 1:134 Я же есть Сущий, и ты в своем сердце обдумай такое:

\vs 1Sb 1:135 Небо навлек на Себя, вокруг Себя море раскинул,

\vs 1Sb 1:136 Мне опора для ног  земля, вкруг тела разлился

\vs 1Sb 1:137 Воздух, и звезд хоровод Меня кругом обегает.

\vs 1Sb 1:138 Девять имею Я букв, Меня составляют четыре

\vs 1Sb 1:139 Слога, кто Я  ты пойми: три первых слога содержат

\vs 1Sb 1:140 Каждый две буквы, последний же слог  остальные. Согласных

\vs 1Sb 1:141 Пять. Всего же числа  девятнадцать сотен, десятков

\vs 1Sb 1:142 Три и вдобавок семерка. Узнай, кто Я есть, и ты станешь

\vs 1Sb 1:143 Мудрости высшей Моей чуждым уже не совсем.

\vs 1Sb 1:144 Так сказал. И того, кто все это слышал, великий

\vs 1Sb 1:145 Страх охватил. В уме остальное предвидя, он начал

\vs 1Sb 1:146 Тут людей умолять и такие слова говорил им:

\vs 1Sb 1:147 Веры в вас нет, безумья гонимые жалом! Не спустит

\vs 1Sb 1:148 Бог ничего из того, что вы сделали. Знает Безсмертный

\vs 1Sb 1:149 Все, Спаситель всезрящий, и вам об этом поведать

\vs 1Sb 1:150 Он направил меня, чтоб вы души свои не сгубили.

\vs 1Sb 1:151 Трезво на мир посмотрите, от зла отрекитесь и войны

\vs 1Sb 1:152 Между собой перестаньте вести в исступленье жестоком,

\vs 1Sb 1:153 Щедро землю кругом человеческой кровью питая.

\vs 1Sb 1:154 Люди, побойтесь Того, Кто Сам нерушим и огромен,

\vs 1Sb 1:155 На небе сущего Бога, создавшего все во вселенной.

\vs 1Sb 1:156 Все к Нему обратитесь с мольбами  Он милосердный! 

\vs 1Sb 1:157 Жизнь сохранить городов и всего великого мира,

\vs 1Sb 1:158 Четвероногих и птиц  пусть милостив будет ко всем Он.

\vs 1Sb 1:159 Время наступит, когда безкрайний, людьми населенный

\vs 1Sb 1:160 Мир, от вод погибая, провоет жуткую песню.

\vs 1Sb 1:161 Время наступит, и воздух над вами вдруг всколыхнется,

\vs 1Sb 1:162 Бога великого гнев устремится с неба на землю.

\vs 1Sb 1:163 Истинно время придет, когда на людей опрокинет

\vs 1Sb 1:165 Вечно живущий Спаситель, снискать если вам не удастся

\vs 1Sb 1:166 Милость Его и отныне совсем жить иначе, чем прежде,

\vs 1Sb 1:167 Так, чтоб ни зла, ни обид друг другу преступно не строя,

\vs 1Sb 1:168 Каждый праведной жизнью прикрыт был от Божьего гнева.

\vs 1Sb 1:169 Слыша такие слова, его на смех все поднимали,

\vs 1Sb 1:170 Звали несчастным безумцем, которого разум покинул.

\vs 1Sb 1:171 Ной же к ним вновь и опять обращался с докучливой речью:

\vs 1Sb 1:172 Жалость внушаете вы, постоянства лишенные, сердцем

\vs 1Sb 1:173 Злобные, стыд кто отринул, кого влечет лишь безстыдство,

\vs 1Sb 1:174 Жадные мира владыки, насильники и нечестивцы,

\vs 1Sb 1:175 Те, что неверья полны, злодеи, лжецы, кто ни слова

\vs 1Sb 1:176 Правды вовек не сказал, богохульники, прелюбодеи,

\vs 1Sb 1:177 Бога Всевышнего гнев кому не страшен,  расплата

\vs 1Sb 1:178 Всех вас теперь ожидает до родичей в пятом колене.

\vs 1Sb 1:179 С криком не мечетесь вы, жестокие, только смеетесь:

\vs 1Sb 1:180 Будет язвительный смех на губах, когда вдруг наступит

\vs 1Sb 1:181 То, о чем говорю: невиданный прежде, ужасный

\vs 1Sb 1:182 Хлынет на землю потоп, самим низпосланный Богом.

\vs 1Sb 1:183 Новый род на земле, священный, тут создан водою

\vs 1Sb 1:185 Будет  продолжится он, на корне сухом произросший.

\vs 1Sb 1:186 Сам собою поток в одну ночь исчезнет. Тогда же

\vs 1Sb 1:187 Вместе с людьми города разметает земли Колебатель,

\vs 1Sb 1:188 Их в укромных ущельях достав, и стены разрушит.

\vs 1Sb 1:189 Так погибнет весь мир, и люди исчезнут без счета,

\vs 1Sb 1:190 Те, что его населяют. А мне еще сколько придется

\vs 1Sb 1:191 Горя изведать и скольких еще погибших оплакать

\vs 1Sb 1:192 В доме своем деревянном? С волнами сколько смешаю

\vs 1Sb 1:193 Слез? Ведь только нахлынут по слову Божьему воды,

\vs 1Sb 1:194 Все поплывет  и земля, и горы, и небо над ними.

\vs 1Sb 1:195 Мир весь станет водой и водами будет погублен.

\vs 1Sb 1:195 Ветры дуть прекратят, наступит другая эпоха.

\vs 1Sb 1:196 Фригия! первою ты из воды приподнимешь вершину,

\vs 1Sb 1:197 Первая будешь кормить ты новое племя людское,

\vs 1Sb 1:199 Кончил когда он впустую слова расточать нечестивцам,

\vs 1Sb 1:200 Сам Всевышний явился, и вновь прозвучал Его голос:

\vs 1Sb 1:201 Время настало, о Ной, объявить обо всем по порядку,

\vs 1Sb 1:202 Что Я в тот день обещал тебе привести в исполненье:

\vs 1Sb 1:203 Неисчислимое зло, которое люди свершили,

\vs 1Sb 1:204 Миру без края вернуть за непослушание смертных.

\vs 1Sb 1:205 Ты же прийти поспеши с женой своей и с сыновьями,

\vs 1Sb 1:206 Также их жен позови и тех, кому повелел Я

\vs 1Sb 1:207 Волю Мою объявить: животных, змей и пернатых.

\vs 1Sb 1:208 Этим Сам зароню Я в сердце желанье явиться 

\vs 1Sb 1:209 Всем, кому Я предназначил продолжить дни свои дальше.

\vs 1Sb 1:210 Так было сказано. Ной пошел и громко об этом

\vs 1Sb 1:211 Им возвестил. Тогда жена, сыновья и невестки

\vs 1Sb 1:212 В дом деревянный взошли, и сразу за ними туда же

\vs 1Sb 1:213 Прочие твари, кому Господь повелел это сделать.

\vs 1Sb 1:214 Тут же засов закрепили, надежно дверь закрывавший.

\vs 1Sb 1:215 Косо он приходился в борту, что был гладко оструган.

\vs 1Sb 1:216 Воля небесного Бога тем самым вполне совершилась.

\vs 1Sb 1:217 Тучи собрал Он и скрыл сверкавший ярко диск солнца,

\vs 1Sb 1:218 Звезды вместе с луной и корону, венчавшую небо.

\vs 1Sb 1:219 Тьмою тут все окружив, загремел, людей повергая

\vs 1Sb 1:220 В ужас, наслал ураган  и ветры разом проснулись,

\vs 1Sb 1:221 Вздулись водные жилы и русла покинули, с неба

\vs 1Sb 1:222 Хлынули, вдруг открывшись, огромные водопады,

\vs 1Sb 1:223 Массы воды из трещин, глубоких провалов внезапно

\vs 1Sb 1:224 Вышли на свет, и под ними земля вся безкрайняя скрылась.

\vs 1Sb 1:225 Плавал тогда под дождем ковчег, что по слову был создан

\vs 1Sb 1:226 Бога: удары терпя от волн, подчиняясь порывам

\vs 1Sb 1:227 Ветра, вдруг поднимался он вверх, и множество пены

\vs 1Sb 1:228 Киль разсекал под журчанье воды, что двигалась всюду.

\vs 1Sb 1:229 Тут, когда весь уже мир затопил дождями Всевышний,

\vs 1Sb 1:230 В голову Ною пришло посмотреть, как исполнилась воля

\vs 1Sb 1:231 Господня, и заглянуть самому в морскую пучину.

\vs 1Sb 1:232 Быстро он дверь распахнул в борту, что был гладко оструган,

\vs 1Sb 1:233 Плотно створки которой одна к другой прилегали.

\vs 1Sb 1:234 Только ее он открыл  представилось взору пространство,

\vs 1Sb 1:235 Сплошь покрыто водой, везде, без конца и без края.

\vs 1Sb 1:236 Страх тут и трепет его охватили. В это мгновенье

\vs 1Sb 1:237 Стал редеть понемногу туман, в течение многих

\vs 1Sb 1:238 Дней уставший окутывать мир. Он бледно-кровавый

\vs 1Sb 1:239 Неба вечернего свод показал и усталого солнца

\vs 1Sb 1:240 Огненный диск. Насилу вернулось мужество к Ною.

\vs 1Sb 1:241 Вдаль направив полет, он сизую выпустил птицу,

\vs 1Sb 1:242 Чтобы узнала она, вдруг где-то еще сохранилась

\vs 1Sb 1:243 Твердая почва. Устав бить крыльями воздух, вернулась

\vs 1Sb 1:244 Птица, кругом облетев: вода нигде не спадала,

\vs 1Sb 1:245 Все было ею полно. Через несколько дней он отправил

\vs 1Sb 1:246 Снова голубку узнать, отступили ли воды. Она же,

\vs 1Sb 1:247 Легкая, в дальний опять отправилась путь и достигла

\vs 1Sb 1:248 Влажной земли. Проведя там какое-то время, обратно

\vs 1Sb 1:249 К Ною вернулась, неся засохшую ветку оливы 

\vs 1Sb 1:250 Знак удачи посольства. В сердцах пробудилась отвага,

\vs 1Sb 1:251 Землю увидеть надежда вселила великую радость.

\vs 1Sb 1:252 Сразу же после того еще чернокрылую птицу

\vs 1Sb 1:253 Ной поспешил отпустить. Она, доверившись крыльям,

\vs 1Sb 1:254 Вдаль устремилась охотно  достигнув земли, там осталась.

\vs 1Sb 1:255 Стало тогда очевидно, что ближе придвинулась суша.

\vs 1Sb 1:256 Скоро, плывя среди волн наугад но шумящему понту,

\vs 1Sb 1:257 Горы встречая воды повсюду, нетленное судно

\vs 1Sb 1:258 Дном увязнув, на узкой полоске земли утвердилось.

\vs 1Sb 1:259 Есть во Фригии черной, что без конца и без края,

\vs 1Sb 1:260 Горный обрывистый кряж, называется он Араратом:

\vs 1Sb 1:261 Здесь предстояло спастись всем тем, кто был с Ноем,  и жажду

\vs 1Sb 1:262 В душу вложил им Господь, едва в это место попали,

\vs 1Sb 1:263 Было тут много ключей, от которых питается Марсий.

\vs 1Sb 1:264 После, как спала вода, ковчег на высокой вершине

\vs 1Sb 1:265 Так и остался лежать, и вновь прозвучал тогда с неба

\vs 1Sb 1:266 Голос Великого Бога нетленный. Он слово такое

\vs 1Sb 1:267 Молвил: Ной, избранник судьбы, справедливый и верный!

\vs 1Sb 1:268 Смело покинь свой ковчег с сыновьями вместе, с женою,

\vs 1Sb 1:269 Три пусть выходят невестки: собой наполнить отныне

\vs 1Sb 1:270 Землю должны вы, плодясь и множа свой род, по закону

\vs 1Sb 1:271 Каждому часть уделив, из колена в колено, доколе

\vs 1Sb 1:272 Время суда не придет, который вас всех ожидает.

\vs 1Sb 1:273 Так произнес вечный голос, и Ной, осмелев, из ковчега

\vs 1Sb 1:274 Спрыгнул на землю, а следом  жена, сыновья и невестки,

\vs 1Sb 1:275 Племя пернатых, ползучие гады, и четвероногих

\vs 1Sb 1:276 Разные виды. Все вместе оставили дом деревянный,

\vs 1Sb 1:277 Вместе на землю сошли, и стала она общим домом.

\vs 1Sb 1:278 Ной тогда, всех людей превзошедший праведной жизнью,

\vs 1Sb 1:279 После Адама восьмой, спустился на твердую землю,

\vs 1Sb 1:280 Сорок дней и один проплавав по воле Господней.

\vs 1Sb 1:281 Так поднялся тогда новый род и жизнь свою начал,

\vs 1Sb 1:282 Первый и золотой, шестым был он и наилучшим

\vs 1Sb 1:283 С тех самых пор, как Господь впервые создал человека.

\vs 1Sb 1:284 Буду его называть я небесным, поскольку заботу

\vs 1Sb 1:285 Бог возложил на Себя обо всем, в чем нужда возникала.

\vs 1Sb 1:286 О поколение первое рода шестого! О радость,

\vs 1Sb 1:287 Что ты доставило мне, когда неминуемой смерти

\vs 1Sb 1:288 Я избежала, устав на волнах качаться и страха

\vs 1Sb 1:289 Много перетерпев вместе с мужем и деверьями,

\vs 1Sb 1:290 С женами их, со свекровью и свекром! Достойную славу

\vs 1Sb 1:291 Я тебе пропою: цветок на смоковнице будет

\vs 1Sb 1:292 Пестрый, до середины дойдут века и положат

\vs 1Sb 1:293 Царской власти начало, что носит скипетр, и трое

\vs 1Sb 1:294 Духом могучих царей, справедливейших, земли поделят.

\vs 1Sb 1:295 Многие годы продлится их власть. Делить по закону

\vs 1Sb 1:296 Между людьми они станут заботы и радость. Земля же

\vs 1Sb 1:297 Будет гордиться плодами, что сами собой вызревают,

\vs 1Sb 1:298 Вся расцветет и зерном осыпет счастливое племя.

\vs 1Sb 1:299 Старость с годами к отцам не придет, не зная болезней,

\vs 1Sb 1:300 Смерть сразу многим несущих, и даже озноба, как будто

\vs 1Sb 1:301 В сон погружаясь, умрут, отойдут к берегам Ахеронта,

\vs 1Sb 1:302 В царство Аида, где им будут возданы почести. Ибо

\vs 1Sb 1:303 Род их был родом блаженных и те изведали счастья,

\vs 1Sb 1:304 В головы чьи заложил Саваоф глубокую мудрость 

\vs 1Sb 1:305 С ними всегда обсуждал Он Свою безсмертную волю.

\vs 1Sb 1:306 Но даже этих счастливцев Аид впереди ожидает.

\vs 1Sb 1:307 После на смену придет тяжелое, крепкое племя

\vs 1Sb 1:308 Земнородных людей и будет по счету второе.

\vs 1Sb 1:309 Имя тем людям Титаны, один на другого похожи,

\vs 1Sb 1:310 Каждый ростом, лицом остальных напомнит. Осанка,

\vs 1Sb 1:311 Голос будет один, какой был Богом заложен

\vs 1Sb 1:312 Некогда предкам их в грудь. Однако и эти, имея

\vs 1Sb 1:313 Дерзкий нрав, замахнутся на то, что им не по силам;

\vs 1Sb 1:314 Смерть приближая свою, захотят сразиться со звездным

\vs 1Sb 1:315 Небом. За это на них океана великого воды

\vs 1Sb 1:316 Хлынут бурным потоком  и сам Саваоф, разсердившись,

\vs 1Sb 1:317 Их удерживать будет, мешая тому, чтобы снова

\vs 1Sb 1:318 Из-за злонравия смертных весь мир под водой оказался.

\vs 1Sb 1:319 Но когда Он заставит всех вод безпредельных волненье

\vs 1Sb 1:320 Гнев усмирить свой, сшибая валы и лишая их силы,

\vs 1Sb 1:321 На неглубоких местах, напротив, волну уменьшая

\vs 1Sb 1:322 Тем, что море землей окружит и о берег неровный

\vs 1Sb 1:323 Биться принудит его великий Бог-громовержец

\vs 1Sb 1:324 Сын Его к людям придет, уподобившись обликом смертным,

\vs 1Sb 1:325 В плоть облечен, как и все на земле. Он гласных четыре

\vs 1Sb 1:326 Будет иметь и двойной согласный. Тебе назову я

\vs 1Sb 1:327 Все число целиком: единиц в нем содержится восемь,

\vs 1Sb 1:328 Столько же, сколько десятков; вдобавок к этому сотен

\vs 1Sb 1:329 Тоже восемь предъявит неверящим людям то имя.

\vs 1Sb 1:330 Должен умом ты постичь, что Сын Безсмертного Бога,

\vs 1Sb 1:331 Выше Которого нет,  Христос, Помазанник Божий.

\vs 1Sb 1:332 Он исполнит закон Отца своего, не разрушит;

\vs 1Sb 1:333 Образ Его воплотив, передаст в полноте и ученье.

\vs 1Sb 1:334 Золото в дар принесут волхвы ему, ладан и смирну,

\vs 1Sb 1:335 Ибо он все совершит, что рожденье его предвещало.

\vs 1Sb 1:336 Голос тогда донесется неслыханный через пустыню,

\vs 1Sb 1:337 Чтобы людей известить, и всем повелит приготовить

\vs 1Sb 1:338 Тропы прямые, изгнать пороки с корнем из сердца.

\vs 1Sb 1:339 Также водою омыть велит он каждому тело,

\vs 1Sb 1:340 Свет чтоб оно обрело и чтобы, рожденные свыше,

\vs 1Sb 1:341 Люди больше нигде с благого пути не свернули.

\vs 1Sb 1:342 Этот Божественный голос опутанный пляскою варвар

\vs 1Sb 1:343 Разом отделит от тела, за что понесет наказанье.

\vs 1Sb 1:344 Будет тут знаменье смертным, когда из Египта нежданно

\vs 1Sb 1:345 Камень придет драгоценный, хранимый Богом. Споткнется

\vs 1Sb 1:346 Племя Евреев на Нем, Другие народы, напротив,

\vs 1Sb 1:347 Вместе Его руководству доверятся, ибо познают

\vs 1Sb 1:348 Бога Всевышнего так и дорогу увидят при свете,

\vs 1Sb 1:349 Что возсияет для всех. Ведь вечную жизнь Он укажет

\vs 1Sb 1:350 Избранным и принесет огонь на века нечестивым.

\vs 1Sb 1:351 Станет тогда же лечить больных Он и немощных телом 

\vs 1Sb 1:352 Всех, кто поверил в Него и свои возложил упованья.

\vs 1Sb 1:353 Видеть слепые начнут, хромые пойдут без поддержки,

\vs 1Sb 1:354 Те, кто не слышал, услышат, и вновь залепечут немые.

\vs 1Sb 1:355 Демонов выгонит Он, возстанут из гроба, кто умер.

\vs 1Sb 1:356 Будет ходить по волнам, пять тысяч в пустыне накормит

\vs 1Sb 1:357 Он от пяти хлебов и единой рыбы. Двенадцать

\vs 1Sb 1:358 Трапезы этой остатки корзин собою наполнят.

\vs 1Sb 1:360 Пьяный Израиль тогда ни во что не сможет проникнуть,

\vs 1Sb 1:361 На ухо туг, он никак не ответит, от хмеля тяжелый.

\vs 1Sb 1:362 Но когда на Евреев Всевышний гнев свой обрушит

\vs 1Sb 1:363 Меткоразящий и веру у их народа отнимет,

\vs 1Sb 1:364 Из-за того, что они распяли Божьего Сына,

\vs 1Sb 1:365 Будет Израиль плевать в Него из уст нечестивых

\vs 1Sb 1:366 Яда полной слюной и бить по щекам Его станет.

\vs 1Sb 1:367 Желчь Ему вместо еды и уксус вместо напитка

\vs 1Sb 1:368 Тут нечестиво дадут, побуждаемы тяжким безумьем,

\vs 1Sb 1:369 Ум поразившим и сердце, глазами смотря и не видя 

\vs 1Sb 1:370 Слепы хуже кротов, ужаснее змей ядовитых,

\vs 1Sb 1:371 Ползают что по земле, опутаны сонным дурманом.

\vs 1Sb 1:372 Он же как руки раскинет и все до конца перетерпит,

\vs 1Sb 1:373 На голове понесет венец терновый, и в ребра

\vs 1Sb 1:374 Ткнут Ему острый тростник  среди белого дня воцарится

\vs 1Sb 1:375 Ночь тогда на три часа и тьмою кругом все покроет.

\vs 1Sb 1:376 Знак тут храм Соломонов народам подаст величайший,

\vs 1Sb 1:377 В домы Аида когда отправится Он, возвещая

\vs 1Sb 1:378 Тем, кто умер, что день придет  и из гроба возстанут.

\vs 1Sb 1:379 Через три дня же обратно на свет из Аида вернется,

\vs 1Sb 1:380 Смертным дабы явить Свой образ и научить их.

\vs 1Sb 1:381 После по облакам пройдет Он к жилищу на небе,

\vs 1Sb 1:382 Миру вместо Себя завет Благовестья оставив.

\vs 1Sb 1:383 Здесь во имя Его росток появится новый

\vs 1Sb 1:384 Из народов, что чтут Закон великого Бога.

\vs 1Sb 1:385 Будут тогда на земле мудрецы, что дорогу покажут,

\vs 1Sb 1:386 Всяким пророкам конец после этого в мире настанет.

\vs 1Sb 1:387 С той поры, как Евреи пожнут недобрую жатву,

\vs 1Sb 1:388 Много у них серебра и золота много отнимет

\vs 1Sb 1:389 Римский кесарь. А после другие царства сменяться

\vs 1Sb 1:390 Станут одно за другим со смертью владык и обиды

\vs 1Sb 1:391 Людям чинить. Тогда великие беды придется

\vs 1Sb 1:392 Вынести смертным за то, что гордыми будут не в меру.

\vs 1Sb 1:393 Храм же когда Соломонов в Священной Земле под ударом

\vs 1Sb 1:394 Варварских полчищ падет, одетых в доспехи из меди,

\vs 1Sb 1:395 Изгнаны будут Евреи с Земли, и по миру скитаться

\vs 1Sb 1:396 Им предстоит, претерпев разорение полное, плевел

\vs 1Sb 1:397 В хлеб добавлять. Незавидный удел их всех ожидает.

\vs 1Sb 1:398 Что же до городов, то одних оплачут другие,

\vs 1Sb 1:399 Сами изведав позор  ведь некогда все согрешили,

\vs 1Sb 1:400 Гнев Великого Бога за это приняв в наказанье.

\bibbookdescr{2Sb}{
  inline={Вторая книга Сивилл},
  toc={2-я Сивилл},
  bookmark={2-я Сивилл},
  header={2-я Сивилл},
  abbr={2~Сив}
}
\vs 2Sb 1:1 Только дал смолкнуть Господь, мольбам моим частым внимая,

\vs 2Sb 1:2 Мудрой песне, как снова вложил Он мне радостный голос 

\vs 2Sb 1:3 В сердце, дабы могла я реченное Богом поведать. 

\vs 2Sb 1:4 Телом всем содрогаясь, начну говорить  ведь не знаю, 

\vs 2Sb 1:5 Что говорю, но от Бога исходят мои прорицанья.

\vs 2Sb 1:6 Время настанет, и в мир придут сотрясенья земные, 

\vs 2Sb 1:7 Молнии жгучие, громы и ржавый налет на растеньях, 

\vs 2Sb 1:8 Бешенство быстрых волков, убийства кровавые, гибель 

\vs 2Sb 1:9 Жизней людских, за людьми же быки мычащие сгинут, 

\vs 2Sb 1:10 Множество коз и овец, ослов терпеливых и прочий 

\vs 2Sb 1:11 Четвероногий скот, а пашни обширные будут 

\vs 2Sb 1:12 Брошены и в запустенье придут, а плоды не родятся, 

\vs 2Sb 1:13 Всюду в обычай войдет продажа и купля свободных, 

\vs 2Sb 1:14 Словно рабов, и везде разграбят священные храмы.

\vs 2Sb 1:15 Явится после того поколенье десятое смертных, 

\vs 2Sb 1:16 И вот тогда сокрушит Колебатель и Молниевержец 

\vs 2Sb 1:17 Идолов, бывших в почете, и мощь семихолмного Рима 

\vs 2Sb 1:18 Он потрясет, уничтожив богатства несметные разом: 

\vs 2Sb 1:19 Мощный пожрет их огонь, великое пламя Гефеста.

\vs 2Sb 1:20 Наземь с высоких небес поток польется кровавый

\vs 2Sb 1:21 Люди в ту пору начнут по всему безконечному миру 

\vs 2Sb 1:22 Гибель друг другу нести, и к этой смуте ужасной 

\vs 2Sb 1:23 Бог пошлет им еще чуму, перуны и голод, 

\vs 2Sb 1:24 Так за неправедный суд карая людей нечестивых. 

\vs 2Sb 1:25 В мире число людей тогда сократится настолько,

\vs 2Sb 1:26 Что если кто-то увидит ноги только след человечьей, 

\vs 2Sb 1:27 То подивится немало. Но Бог, в эфире живущий, 

\vs 2Sb 1:28 Всем справедливым мужам опять избавителем станет. 

\vs 2Sb 1:29 И на земле воцарятся надежный мир и согласье, 

\vs 2Sb 1:30 Вновь будет почва рождать и плод принесет изобильный, 

\vs 2Sb 1:31 Ибо делить перестанут и мучить ее как рабыню. 

\vs 2Sb 1:32 Всякая пристань и порт откроются людям свободно, 

\vs 2Sb 1:33 Как это было и прежде, безстыдство же вовсе исчезнет.

\vs 2Sb 1:34 После на небе Господь великое знаменье явит:

\vs 2Sb 1:35 Люди созвездие узрят  венку оно будет подобно, 

\vs 2Sb 1:36 Ярким сияньем своим небеса озарит и надолго 

\vs 2Sb 1:37 Так сохранится. И люди поймут, что грядет состязанье, 

\vs 2Sb 1:38 И за вот этот венок зовет их бороться Безсмертный. 

\vs 2Sb 1:39 Ибо наступит затем триумфа великого время

\vs 2Sb 1:40 В граде небесном: сюда весь мир сойдется обширный, 

\vs 2Sb 1:41 Стать можно каждому будет причастным славе нетленной. 

\vs 2Sb 1:42 Все народы тогда в безсмертных ристаньях к победе, 

\vs 2Sb 1:43 Коей прекраснее нет, устремятся; и грешник не сможет 

\vs 2Sb 1:44 Там победный венок купить за деньги безстыдно.

\vs 2Sb 1:45 И справедливо раздаст награжденья Спаситель блаженный. 

\vs 2Sb 1:46 Верных Он увенчает, а тем, кто мучения принял, 

\vs 2Sb 1:47 Смертью окончив борьбу, безсмертная будет награда. 

\vs 2Sb 1:48 Тем, кто девство храня, к победе нетленной стремился, 

\vs 2Sb 1:49 Он по заслугам воздаст, и тем, кто берег справедливость,

\vs 2Sb 1:50 И никого не забудет Он даже из дальних народов,

\vs 2Sb 1:51 Если праведно жили и знали единого Бога.

\vs 2Sb 1:52 Те, кто брак почитал, позорный блуд отвергая,

\vs 2Sb 1:53 Дар получат богатый, вовек не умрет в них надежда. 

\vs 2Sb 1:54 Ибо любая душа человечья  даяние Бога.

\vs 2Sb 1:55 Смертный не вправе пятнать ее никакими грехами.

\vs 2Sb 1:56 Следует честным трудом пропитанье стяжать, не пытаясь,

\vs 2Sb 1:57 Делая зло, богатеть, не нужно трогать чужого, 

\vs 2Sb 1:58 Хватит тебе своего; не лги, будь истине верен, 

\vs 2Sb 1:59 Идолам не поклоняйся, но Вечносущего Бога 

\vs 2Sb 1:60 В первую очередь чти, уважай и родителей также. 

\vs 2Sb 1:61 Праведен будь, чтобы суд над тобою неправедным не был. 

\vs 2Sb 1:62 Не обижай бедняка, чуждайся лицеприятья, 

\vs 2Sb 1:63 Коль будешь плохо судить, то Бог тебя же осудит.

\vs 2Sb 1:64 Ложных свидетельств беги, говори только чистую правду; 

\vs 2Sb 1:65 Чист оставайся и сам, подходи ко всем людям с любовью;

\vs 2Sb 1:66 Верную меру блюди и лучше дай больше, чем меньше.

\vs 2Sb 1:67 Ровными чаши весов должны быть, не наклоняй их.

\vs 2Sb 1:68 В клятвах своих не лги  случайно, иль с умыслом вредным 

\vs 2Sb 1:69 Страшен Бог для того, кто хоть в чем-либо клятву нарушил. 

\vs 2Sb 1:70 Дара не принимай, если он добыт преступленьем.

\vs 2Sb 1:71 И семена не кради: кто отнимет их, будет навеки

\vs 2Sb 1:72 Проклят, ибо украл он то, что дало бы пищу.

\vs 2Sb 1:73 Пусть клеветы и разврата ты будешь чужд, и убийства;

\vs 2Sb 1:74 Бедного не обижай, плати за работу исправно. 

\vs 2Sb 1:75 Речи разумно веди, а тайны храни в своем сердце.

\vs 2Sb 1:76 Помощь вдовам подай, сиротам и всем, кто несчастен.

\vs 2Sb 1:77 Сам не твори беззаконий и злу не позволь совершиться;

\vs 2Sb 1:78 Нищему сразу давай, не откладывай это на завтра;

\vs 2Sb 1:79 Щедрой рукой удели неимущему часть урожая  

\vs 2Sb 1:80 Кто помогает другому, ссужает даримое Богу.

\vs 2Sb 1:81 Милость во дни Суда от смерти даст избавленье,

\vs 2Sb 1:82 Милости хочет Господь от людей, а вовсе не жертвы.

\vs 2Sb 1:83 Дай одежду нагому, тому, кто голоден, хлеба,

\vs 2Sb 1:84 В дом свой бездомных прими, слепых проводи на дорогу. 

\vs 2Sb 1:85 Тех пожалей, с кем беда в коварном море случилась.

\vs 2Sb 1:86 Падает кто  поддержи, спаси, коль нависла угроза;

\vs 2Sb 1:87 Все страдают, а жизнь  колесо, и счастье неверно.

\vs 2Sb 1:88 Если богат, протяни несчастному помощи руку

\vs 2Sb 1:89 И удели из того, что сам получил ты от Бога. 

\vs 2Sb 1:90 Жизни схожи людские, и только жребий неравен.

\vs 2Sb 1:91 Над бедняком никогда не должен ты насмехаться.

\vs 2Sb 1:92 Злобно ты не ругай и того, кто упрека достоин.

\vs 2Sb 1:93 Ясным делает смерть, как жизнь прожита человеком 

\vs 2Sb 1:94 Был справедлив он, иль нет, на Суде великом решится. 

\vs 2Sb 1:95 Пей умеренно, разум вином повреждаться не должен;

\vs 2Sb 1:96 Крови не ешь и того, что идолам в жертву приносят.

\vs 2Sb 1:97 Меч можешь взять для защиты  мечом против друга не действуй,

\vs 2Sb 1:98 Лучше, впрочем, совсем никогда не брать его в руки:

\vs 2Sb 1:99 Если убьешь и врага, рука все ж запятнана будет. 

\vs 2Sb 1:100 Землю соседа не тронь, не ступай на нее ни ногою:

\vs 2Sb 1:101 Должно блюсти рубежи, неправедно их нарушенье.

\vs 2Sb 1:102 Польза и приобретенье, коль честно, и вред, коль нечестно. 

\vs 2Sb 1:103 Злак, на поле растущий, не смей губить никогда ты, 

\vs 2Sb 1:104 Пусть уваженье пришельцам не меньше, чем гражданам, будет.

\vs 2Sb 1:105 Люди за тягостный труд почитают гостеприимство, 

\vs 2Sb 1:106 Словно все чужды они друг другу, но так не должно быть: 

\vs 2Sb 1:107 Ибо смертные все от крови одной происходят, 

\vs 2Sb 1:108 А на земле для людей не назначено мест постоянных. 

\vs 2Sb 1:109 В мыслях своих не стремись к богатству, желай одного лишь:

\vs 2Sb 1:110 Малым довольствуясь, жить, ничего не стяжав не по праву. 

\vs 2Sb 1:111 Алчность, пристрастье к деньгам все пороки ведут за собою. 

\vs 2Sb 1:112 К золоту и серебру опасно влеченье  сокрыто 

\vs 2Sb 1:113 В этих металлах железо, несущее верную гибель; 

\vs 2Sb 1:114 В золоте и серебре обман для смертных таится,

\vs 2Sb 1:115 Золото, зол предводитель, ты смертью всему угрожаешь, 

\vs 2Sb 1:116 Не становись никогда для людей несчастьем желанным, 

\vs 2Sb 1:117 Из-за тебя и война, и все грабежи, и убийства, 

\vs 2Sb 1:118 Ты причина вражды детей с отцами и братьев.

\vs 2Sb 1:119 Козней не замышляй, против друга не вооружайся; 

\vs 2Sb 1:120 Не говори одного, коли в сердце держишь иное;

\vs 2Sb 1:121 Если же место меняешь, то сам как полип не меняйся.

\vs 2Sb 1:122 Честен будь, говори только то, что чувствуешь сердцем.

\vs 2Sb 1:123 Грех добровольный  зло, но если по принужденью 

\vs 2Sb 1:124 Точно судить не могу: вина в человеческой воле. 

\vs 2Sb 1:125 Ты не гордись ни умом, ни силой своей, ни богатством:

\vs 2Sb 1:126 Мудрость только у Бога, и мощь, и полное счастье.

\vs 2Sb 1:127 Прошлые злые дела твой дух пускай не смущают,

\vs 2Sb 1:128 Ведь невозможно никак небывшим бывшее сделать.

\vs 2Sb 1:129 Силу не применяй опрометчиво, сдерживай чувства: 

\vs 2Sb 1:130 Часто нанесший удар совершает убийство невольно.

\vs 2Sb 1:131 Жить без страданий нельзя, но боль пусть не будит гордыню;

\vs 2Sb 1:132 И к изобилью во всем не нужно людям стремиться,

\vs 2Sb 1:133 Роскошь большая влечет любовь к наслажденьям чрезмерным,

\vs 2Sb 1:134 Тот, кто богат, легко впадает в безстыдную дерзость. 

\vs 2Sb 1:135 Гнев, закипевший в душе, губительным сделаться может,

\vs 2Sb 1:136 Легок гнев небольшой, но, выросши, станет безумьем.

\vs 2Sb 1:137 Рвение в добром похвально, но пагубна ревность дурная,

\vs 2Sb 1:138 Злая дерзость  позор, благому дерзанию  слава,

\vs 2Sb 1:139 Слава любви к добру  позор влеченью Киприды. 

\vs 2Sb 1:140 Мил согражданам муж, приветливый и дружелюбный.

\vs 2Sb 1:141 Мера важна в еде, питье, но также и в слове.

\vs 2Sb 1:142 Мера  лучше всего, и вред в ее нарушеньи.

\vs 2Sb 1:143 Бранных речей не веди, не завидуй, не будь вероломен,

\vs 2Sb 1:144 Мыслей дурных избегай и не смей обманывать злостно. 

\vs 2Sb 1:145 Благоразумью учись и от постыдных дел воздержанью.

\vs 2Sb 1:146 Нравам не следуй дурным, за зло воздавай справедливо;

\vs 2Sb 1:147 Пользу несут уговоры, а гнев только гнев порождает.

\vs 2Sb 1:148 Слишком быстро не верь, убедись сначала надежно.

\vs 2Sb 1:149 Вот каково состязанье, и вот какие награды! 

\vs 2Sb 1:150 Это к безсмертию путь и жизни вечной ворота  

\vs 2Sb 1:151 Бог небесный открыл их самым праведным людям, 

\vs 2Sb 1:152 Здесь одержавшим победу; они увенчаны будут 

\vs 2Sb 1:153 И сквозь безсмертья врата пройдут с великою славой.

\vs 2Sb 1:154 Но когда миру всему вдруг знаменье будет такое:

\vs 2Sb 1:155 Дети с седыми висками начнут на свет появляться, 

\vs 2Sb 1:156 Беды к людям придут, и мор, и голод, и войны,

\vs 2Sb 1:157 Всем изменит удача, и горькие слезы польются.

\vs 2Sb 1:158 О, скольким детям придется тут справить пир поминальный,

\vs 2Sb 1:159 Жалко оплакав своих матерей и отцов; в покрывала 

\vs 2Sb 1:160 Трупы их завернут и зароют в землю сырую,

\vs 2Sb 1:161 Сами в пыли и крови. Увы, несчастные люди

\vs 2Sb 1:162 Рода последнего в мире, злодеи ужасные, как же

\vs 2Sb 1:163 Не понимают, глупцы, что, если жены не станут

\vs 2Sb 1:164 Больше рождать детей, людское племя угаснет? 

\vs 2Sb 1:165 Время жатвы приспело, коль некие, словно пророки,

\vs 2Sb 1:166 Будут вещать по земле и много обмана измыслят.

\vs 2Sb 1:167 Тут придет Велиал и немало знамений явит.

\vs 2Sb 1:168 Избранных, праведных самых в то время великие беды

\vs 2Sb 1:169 Ждут и смятенье, они подвергнутся все ограбленьям 

\vs 2Sb 1:170 Также, как и Евреи,  грозит им гневом ужасным

\vs 2Sb 1:171 Некий с Востока народ, из колен десяти состоящий.

\vs 2Sb 1:172 Станут искать они тех, кто погиб от руки Ассирийца;

\vs 2Sb 1:173 Кровью Евреям близки, язычникам смерть уготовят.

\vs 2Sb 1:174 После же власть обретут они и над людом могучим

\vs 2Sb 1:175 Избранных верных Евреев, в рабов их всех обращая, 

\vs 2Sb 1:176 Так же, как было и в прошлом; и сила еще не покинет 

\vs 2Sb 1:177 Этих мужей. А Всевышний, Всевидящий, в небе Живущий

\vs 2Sb 1:178 Сон нашлет на людей, глаза их тьмой покрывая. 

\vs 2Sb 1:179 Счастливы Божии слуги, которые бодрствовать будут

\vs 2Sb 1:180 В час, как придет Господь, их сонными Он не застанет, 

\vs 2Sb 1:181 Ибо все время глаза у них открытыми были. 

\vs 2Sb 1:182 То на рассвете случится, иль вечером, или же в полдень, 

\vs 2Sb 1:183 Но обязательно Он грядет  пророчество верно. 

\vs 2Sb 1:184 Сном будут люди объяты  и тут все звезды на небе

\vs 2Sb 1:185 Вдруг среди дня засияют и с ними оба светила;

\vs 2Sb 1:186 Быстрое время свой круг пройдет  и все совершится. 

\vs 2Sb 1:187 И в колеснице небесной тогда сойдет Фесвитянин, 

\vs 2Sb 1:188 Чтобы явить на земле тройное знаменье скорой 

\vs 2Sb 1:189 Мира кончины, и всем должны быть ясны эти знаки.

\vs 2Sb 1:190 Горе женам, в те дни имеющим плод в своем чреве,

\vs 2Sb 1:191 Иль неразумных детей молоком кормящих, а также 

\vs 2Sb 1:192 Тем из людей, кто тогда окажется в море плывущим. 

\vs 2Sb 1:193 Горе тому человеку, что день этот страшный увидит: 

\vs 2Sb 1:194 Ночь безпросветная мир до края окутает мглою,

\vs 2Sb 1:195 Сразу и Север, и Юг, и Восход, и Заход затмевая. 

\vs 2Sb 1:196 Тут величайший поток огня и пламени хлынет 

\vs 2Sb 1:197 С неба на землю, и все, что есть на свете, погубит: 

\vs 2Sb 1:198 Сушу, и Океан огромный, и синее море, 

\vs 2Sb 1:199 Реки, озера, ручьи и даже безжалостный Тартар,

\vs 2Sb 1:200 Даже небесную ось. А горящие в небе светила 

\vs 2Sb 1:201 Все воедино сольются и полностью форму утратят. 

\vs 2Sb 1:202 Звезды тогда упадут с небосвода в пучину морскую, 

\vs 2Sb 1:203 Души людей, умирая, зубами тогда заскрежещут, 

\vs 2Sb 1:204 Пламя будет их жечь и ливень серы ужасный

\vs 2Sb 1:205 Вплоть до часа, когда всю землю пепел покроет. 

\vs 2Sb 1:206 Все элементы вселенной тогда одинокими станут  

\vs 2Sb 1:207 Воздух, свет, небеса, земля и моря, дни и ночи,  

\vs 2Sb 1:208 Стаи безчисленных птиц не будут в небе метаться,

\vs 2Sb 1:209 По морю не проплывут водяные животные больше,

\vs 2Sb 1:210 Судно груженое также волны уже не прорежет. 

\vs 2Sb 1:211 И не оставят волы борозды ни единой на пашнях, 

\vs 2Sb 1:212 Не зашумят уж деревья от ветра. [Но все воедино 

\vs 2Sb 1:213 Сплавит Господь, а потом разнимет для очищенья.]

\vs 2Sb 1:214 После же явятся в мир посланники вечные Бога:

\vs 2Sb 1:215 Он Михаила пошлет, Гавриила с ним, Уриила

\vs 2Sb 1:216 И Рафаила  известно им зло, совершенное всяким. 

\vs 2Sb 1:217 Выведут души людские на свет из тумана и мрака, 

\vs 2Sb 1:218 Чтобы судил их Господь, на троне сидящий небесном, 

\vs 2Sb 1:219 Ибо только лишь Он великий Владыка нетленный,

\vs 2Sb 1:220 Он Вседержтггелъ, Который Судьею станет для смертных. 

\vs 2Sb 1:221 Тем, кто землей погребен, отдаст их жизни небесный 

\vs 2Sb 1:222 Бог, и дыхание вложит, и голос вернет им, а кости 

\vs 2Sb 1:223 Вместе соединит и плотью затем их оденет, 

\vs 2Sb 1:224 Жилы приладит Он к жилам и вены кровью наполнит, 

\vs 2Sb 1:225 Кожею тело покроет и вырастит волосы снова 

\vs 2Sb 1:225 Части сложит Он все, придаст им дух и движенье; 

\vs 2Sb 1:226 Так людские тела за один лишь день воскресит Он. 

\vs 2Sb 1:227 Тут стальные засовы Аида, что чужд милосердья 

\vs 2Sb 1:228 И нерушим в своей силе всегда был прежде, сломает 

\vs 2Sb 1:229 Ангел тот, Уриил, что послан Богом, великий.

\vs 2Sb 1:230 Тартара мощь низложив, на суд печальные тени 

\vs 2Sb 1:231 Он поведет, и всех раньше Титанов, в давнее время 

\vs 2Sb 1:232 Живших, а с ними Гигантов, и сгинувших в водах Потопа, 

\vs 2Sb 1:233 Также и в бурной волне морской свою смерть повстречавших, 

\vs 2Sb 1:234 Тех, кого дикие звери, и змеи, и птицы пожрали, 

\vs 2Sb 1:235 Всех приведет Уриил к подножью Господнего трона; 

\vs 2Sb 1:236 С ними и тех, кто в огне, что плоть пожирает, сгорели, 

\vs 2Sb 1:237 Он соберет для Суда и пред Божьим престолом поставит.

\vs 2Sb 1:238 Мертвых когда воскресит он и судьбы земные разрушит, 

\vs 2Sb 1:239 В небе высоко гремящий Господь Саваоф Адонаи,

\vs 2Sb 1:240 Сев на престоле Своем, могучий столп установит, 

\vs 2Sb 1:241 На облаках придет к Безсмертному также Безсмертный 

\vs 2Sb 1:242 В славе Христос и с ним безупречные ангелы вместе. 

\vs 2Sb 1:243 Он одесную возсядет Великого Бога и станет 

\vs 2Sb 1:244 Праведно живших судить и людей, прозябавших в нечестье.

\vs 2Sb 1:245 Дружный с Самим Всевышним, придет Моисей, облеченный

\vs 2Sb 1:246 Плотью, как в жизни, а с ним Авраам предстанет великий, 

\vs 2Sb 1:247 Тут Исаак и Иаков, затем Иисус с Илиею,

\vs 2Sb 1:248 Аввакум, Даниил, Иона и кто от Евреев 

\vs 2Sb 1:249 Приняли смерть, будут здесь. И все Евреи погибнут,

\vs 2Sb 1:250 Чтобы за зло отплатить по слову Иеремии

\vs 2Sb 1:251 И получить по заслугам за все, что содеяли в жизни. 

\vs 2Sb 1:252 Всем тут придется пройти сквозь пламени жгучую реку 

\vs 2Sb 1:253 И негасимый огонь, в котором праведник всякий 

\vs 2Sb 1:254 Жизнь свою сохранит, но сгинут все нечестивцы;

\vs 2Sb 1:255 Тем исчезнуть навек, кто раньше зло сотворили: 

\vs 2Sb 1:256 Кто или сам убивал, или рядом стоял, не мешая, 

\vs 2Sb 1:257 Воры, обманщики все, домов разорители злые, 

\vs 2Sb 1:258 Клеветники, попрошайки и гадкие прелюбодеи, 

\vs 2Sb 1:259 Дерзко закон преступавшие, чтящие идолов разных,

\vs 2Sb 1:260 Те, кто отрекся от веры в Великого, Вечного Бога, 

\vs 2Sb 1:261 Кто притеснял и бранил людей справедливых и честных, 

\vs 2Sb 1:262 Кто убивал святых, гонитель истинной веры; 

\vs 2Sb 1:263 Также кто хитростью злой полны и безстыдством двуличным, 

\vs 2Sb 1:264 [Будучи старцами даже почтенными, правду боялись

\vs 2Sb 1:265 Ясно сказать на Суде и обиду другим учиняли, 

\vs 2Sb 1:266 Веря обманчивым слухам \ldots

\vs 2Sb 1:267 Люди, несущие гибель, страшнее, чем волки и барсы, 

\vs 2Sb 1:268 Те, что гордятся безмерно, и те, что деньги ссужают, 

\vs 2Sb 1:269 Дабы потом по домам лихву собирать за лихвою,

\vs 2Sb 1:270 Вдов несчастных, сирот обирая позорно до нитки; 

\vs 2Sb 1:271 С ними и все, кто сиротам и вдовам то уделяет, 

\vs 2Sb 1:272 Что нечестно нажили, и все, кто, делясь с неимущим, 

\vs 2Sb 1:273 Станут его попрекать; и дети, что бросили старых 

\vs 2Sb 1:274 Мать и отца, не отдав им доли сыновнего долга,

\vs 2Sb 1:275 Также и дети такие, которые не подчинялись

\vs 2Sb 1:276 Воле родителей, им отвечая лишь руганью злобной; 

\vs 2Sb 1:277 Те, кто поклялся, а после держать не хотел свое слово; 

\vs 2Sb 1:278 Слуги, что против власти господ мятеж затевали; 

\vs 2Sb 1:279 Все запятнавшие тело свое развратом безстыдным

\vs 2Sb 1:280 И вступавшие в связь нечестивую, пояс девичий 

\vs 2Sb 1:281 Развязавшие тайно; и женщины, что из утробы 

\vs 2Sb 1:282 Силой изгнали свой плод, и те, что рожденных сгубили; 

\vs 2Sb 1:283 И колдуны, и колдуньи. Всех этих грешников вместе 

\vs 2Sb 1:284 Гнев Живущего в небе Безсмертного Бога погонит

\vs 2Sb 1:285 Вплоть до того столпа, который кругом обегает 

\vs 2Sb 1:286 Пламени неугасимый поток. И ангелы Божьи, 

\vs 2Sb 1:287 Вестники Сущего вечно, их всех наказаньям подвергнут: 

\vs 2Sb 1:288 Ждет их пламени бич и жуткая цепь огневая,

\vs 2Sb 1:289 Не разорвать им оков, что тесно опутывать будут

\vs 2Sb 1:290 Их тела; а потом, во мраке ночи кромешной

\vs 2Sb 1:291 Адским зверям на съеденье в геенну их всех побросают  

\vs 2Sb 1:292 Множество страшных чудовищ таится в той тьме безграничной.

\vs 2Sb 1:293 Но когда казней различных от ангелов вдоволь претерпят, 

\vs 2Sb 1:294 Все, кто сердцем был зол, грядет им последняя кара:

\vs 2Sb 1:295 Огненный круг колеса из потока великого выйдет, 

\vs 2Sb 1:296 Тяжко давить оно будет вершителей дел беззаконных; 

\vs 2Sb 1:297 И раздадутся тогда отовсюду плач и стенанья, 

\vs 2Sb 1:298 Горький удел ужаснет и отцов, и детей неразумных, 

\vs 2Sb 1:299 И матерей, и младенцев, еще кормящихся грудью.

\vs 2Sb 1:300 Слез не выплакать им никогда, и жалкие крики 

\vs 2Sb 1:301 Уж не услышит никто, хоть будут звучать отовсюду: 

\vs 2Sb 1:302 Так что, терзаясь во тьме глубокого Тартара, станут 

\vs 2Sb 1:303 Вопли они испускать напрасно, и в скорбных угодьях 

\vs 2Sb 1:304 Трижды заплатят за все совершенные ими злодейства,

\vs 2Sb 1:305 Пламенем жарким палимы, зубами они заскрежещут, 

\vs 2Sb 1:306 Жажда сильнейшая им причинит мучения злые 

\vs 2Sb 1:307 И пожелают тогда умереть, но больше не смогут: 

\vs 2Sb 1:308 Не успокоит их смерть, и ночь не даст передышки. 

\vs 2Sb 1:309 Долго Всевышнего Бога молить они будут напрасно 

\vs 2Sb 1:310 И отвратит Господь Свой лик, чтоб их больше не видеть: 

\vs 2Sb 1:311 Ибо ведь людям заблудшим Он семь веков предоставил 

\vs 2Sb 1:312 Для покаянья  за них просила Дева святая. 

\vs 2Sb 1:313 Тех же, кто делал добро и был всегда справедливым, 

\vs 2Sb 1:314 Славился кто благочестьем и верным ума разсужденьем 

\vs 2Sb 1:315 Ангелы этих людей поднимут над страшным потоком 

\vs 2Sb 1:316 Пламени и поведут их к свету и к жизни безпечной 

\vs 2Sb 1:317 Тем нетленным путем, что Богом проложен Великим, 

\vs 2Sb 1:318 Где три источника бьют  медовый, винный и млечный. 

\vs 2Sb 1:319 Общею станет земля; перестав уже быть разделенной

\vs 2Sb 1:320 Стенами и рубежами, сама даст плод изобильный; 

\vs 2Sb 1:321 Вместе все заживут, нужды не имея в богатстве. 

\vs 2Sb 1:322 Тут не будет уже никто ни богатым, ни бедным, 

\vs 2Sb 1:323 Ни рабом, ни тираном, ни малым и ни великим; 

\vs 2Sb 1:324 Нет ни царей, ни вождей  все люди равны меж собою.

\vs 2Sb 1:325 Больше не скажет никто: наступила ночь, или завтра, 

\vs 2Sb 1:326 Или вчера это было, и дней, заботами полных, 

\vs 2Sb 1:327 Также не станет; исчезнут четыре времени года,

\vs 2Sb 1:328 Смерть и брачный союз; покупка вещей и продажа; 

\vs 2Sb 1:329 Даже Запад с Востоком  все в долгий день превратится,

\vs 2Sb 1:330 Тут Вседержитель Нетленный еще одно людям дарует: 

\vs 2Sb 1:331 Те, кто был праведной жизни, к Нему с мольбой обратятся, 

\vs 2Sb 1:332 Чтобы грешных Он спас от огня и от мук непрерывных,  

\vs 2Sb 1:333 Просьбы услышит Господь, и все это так совершится: 

\vs 2Sb 1:334 Он невредимыми всех из огня неусыпного вынет

\vs 2Sb 1:335 И через Свой народ в иные пошлет их угодья, 

\vs 2Sb 1:336 И для жизни иной, нетленной в полях Елисейских, 

\vs 2Sb 1:337 Там, где широко простер свои воды поток Ахеронта, 

\vs 2Sb 1:338 Озером став глубочайшим, которое вечно пребудет. 

\vs 2Sb 1:339 Горе мне, горе, несчастной! В тот день что будет со мною?!

\vs 2Sb 1:340 Я ведь стремилась в грехе превзойти всех людей безрассудно.

\vs 2Sb 1:341 И о супруге своем, и о здравом уме позабывши. 

\vs 2Sb 1:342 Но во дворце я жила моего богатого мужа, 

\vs 2Sb 1:343 Бедных туда не пускала; и зная, что зло совершаю, 

\vs 2Sb 1:344 На беззакония шла. Спаситель, меня от мучений,

\vs 2Sb 1:345 Наглую псицу, избавь, и все безстыдства прости мне. 

\vs 2Sb 1:346 Также молю Тебя: дай этой песне немного покоя, 

\vs 2Sb 1:347 Манны Податель благой, Владыка великого Царства!

\bibbookdescr{3Sb}{
  inline={Третья книга Сивилл},
  toc={3-я Сивилл},
  bookmark={3-я Сивилл},
  header={3-я Сивилл},
  abbr={3~Сив}
}
\vs 3Sb 1:1 В небе на троне Cидящий превыше самих херувимов, 

\vs 3Sb 1:2 О Громовержец Блаженный, молю Тебя  дай мне покоя! 

\vs 3Sb 1:3 Вестница истины всей, я устала вещать непрестанно. 

\vs 3Sb 1:4 Но отчего мое сердце трепещет все снова и снова? 

\vs 3Sb 1:5 Бич меня нудит какой устами правдивое пенье

\vs 3Sb 1:6 Смертным открыто излить? Опять обо всем расскажу я, 

\vs 3Sb 1:7 Что бы Господь ни велел мне людям ясно поведать.

\vs 3Sb 1:8 Люди, в облике вашем творение Божие зримо, 

\vs 3Sb 1:9 Что ж вы блуждаете зря, отнюдь не желая тропою

\vs 3Sb 1:10 В жизни прямою идти, о Безсмертном Создателе помня? 

\vs 3Sb 1:11 Только Единый есть Бог  в небесах, никем не рожденный,

\vs 3Sb 1:12 Неизречен и невидим, Он видит все, что есть в мире. 

\vs 3Sb 1:13 Бог не был создан ничьею рукой никогда, ни из камня 

\vs 3Sb 1:14 И ни из золота и ни из кости слоновьей твореньем

\vs 3Sb 1:15 Не был. Извечность Свою Он сам доказал непреложно  

\vs 3Sb 1:16 Сущий ныне, был раньше и впредь всегда Он пребудет. 

\vs 3Sb 1:17 Смертным дано ли очам Всевышнего Бога увидеть? 

\vs 3Sb 1:18 Разве вместит кто-нибудь одно только имя услышать 

\vs 3Sb 1:19 Бога Великого, в небе Живущего, мира Владыки?

\vs 3Sb 1:20 Сущее все Он создал Своим словом  и небо, и море, 

\vs 3Sb 1:21 Неутомимое солнце, луну, что растет постепенно, 

\vs 3Sb 1:22 Множество звезд светоносных и матерь могучую Тефис, 

\vs 3Sb 1:23 Дни с ночами, источники рек, огонь негасимый. 

\vs 3Sb 1:24 Бог сотворил человека, который был первым из смертных,

\vs 3Sb 1:25 Имя ему  Адам, и эти буквы четыре

\vs 3Sb 1:26 Север, и Юг, и Восток, и Запад собой заполняют. 

\vs 3Sb 1:27 Сам Господь утвердил людские вид и обличье, 

\vs 3Sb 1:28 Сделал зверей Он, и гадов, и птиц, летающих в небе.

\vs 3Sb 1:29 Бога не чтите вы и не боитесь в своем заблужденье,

\vs 3Sb 1:30 Вы поклоняетесь змеям и жертвы приносите кошкам, 

\vs 3Sb 1:31 Также и всяким кумирам, людей изваяньям из камня, 

\vs 3Sb 1:32 И перед входами в храмы безбожные вечно сидите. 

\vs 3Sb 1:33 Сущего Бога побойтесь, Который все наблюдает, 

\vs 3Sb 1:34 О почитатели мерзости каменной, как позабыли

\vs 3Sb 1:35 Вы о Мессии, что создал, Безсмертный, и небо и землю? 

\vs 3Sb 1:36 Род людей кровожадных, лукавых, дурных, нечестивых, 

\vs 3Sb 1:37 Племя злонравное с полными лжи языками двойными, 

\vs 3Sb 1:38 Идолов чтите, прелюбы творите и злое коварство 

\vs 3Sb 1:39 В сердце своем замышляете вы, друг у друга крадете \ldots

\vs 3Sb 1:40 В мыслях безстыдство у вас, в груди  свирепое жало! 

\vs 3Sb 1:41 Тот, кто владеет богатством, ничем не поделится с бедным; 

\vs 3Sb 1:42 Злобы ужасной полны, все люди про верность забудут; 

\vs 3Sb 1:43 Многие вдовы и с ними замужние женщины даже 

\vs 3Sb 1:44 Тайно станут любить других, желая наживы,

\vs 3Sb 1:45 После ж и вовсе в открытую будут греху предаваться.

\vs 3Sb 1:46 Рим пока еще медлит, но время настанет  Египтом 

\vs 3Sb 1:47 Он овладеет. Тогда величайшее царство на землю 

\vs 3Sb 1:48 Скоро к людям сойдет, им Царь будет править Безсмертный, 

\vs 3Sb 1:49 Вождь священный придет, держащий скиптры земные,

\vs 3Sb 1:50 Сколько б времен ни прошло, все ж нет конца Его Царству. 

\vs 3Sb 1:51 Вспыхнет тогда у латинских мужей великая ярость; 

\vs 3Sb 1:52 Трое разрушат Рим, когда бросят жребий несчастный. 

\vs 3Sb 1:53 Люди погибнут все под гнетом собственных кровель, 

\vs 3Sb 1:54 Ибо огненный ливень с небес на землю прольется.

\vs 3Sb 1:55 О, я несчастная, страшный тот день  когда ж он настанет, 

\vs 3Sb 1:56 День, когда призовет на суд Властитель Небесный? 

\vs 3Sb 1:57 Стройтесь до времени, о города, и еще украшайтесь 

\vs 3Sb 1:58 Пышностью храмов, рынков, ристалищ, кумиров из камня, 

\vs 3Sb 1:59 Золота и серебра, и все это так сохранится

\vs 3Sb 1:60 Вплоть до горького дня  в парах удушливой серы 

\vs 3Sb 1:61 Люди тогда задохнутся \ldots\ Но лучше все по порядку 

\vs 3Sb 1:62 Я о несчастьях скажу, в каких городах они будут \ldots

\vs 3Sb 1:63 Явится вслед за тем Велиал, он придет из Себасты, 

\vs 3Sb 1:64 Станет горы сдвигать, усмирит и бурное море, 

\vs 3Sb 1:65 Солнце с луной светоносные он в небесах остановит,

\vs 3Sb 1:66 Тех, кто усоп, воскресит и много знамений чудных 

\vs 3Sb 1:67 Людям он явит, но мира конец еще не наступит  

\vs 3Sb 1:68 Будет все только соблазн, хоть, конечно, немало обманет 

\vs 3Sb 1:69 Верных сей Велиал Евреев и множество прочих

\vs 3Sb 1:70 Смертных мужей, что Закона и Божьего Слова не знают. 

\vs 3Sb 1:71 Но лишь начнут исполняться угрозы великого Бога, 

\vs 3Sb 1:72 Пламень, сжигающий все, потоками хлынет на землю; 

\vs 3Sb 1:73 Сгинут в пламени том Велиал и надменные люди  

\vs 3Sb 1:74 Все, кто веру речам и делам его даровали.

\vs 3Sb 1:75 Женщине миром всецело тогда завладеет, и станет 

\vs 3Sb 1:76 Он ей во всем подчиняться и слушаться безпрекословно. 

\vs 3Sb 1:77 После того вдова окажется мира царицей; 

\vs 3Sb 1:78 Бросит в море она серебро и злато людское, 

\vs 3Sb 1:79 Также всю медь и железо утопит о соленой пучине;

\vs 3Sb 1:80 Все элементы тогда с лица земного исчезнут. 

\vs 3Sb 1:81 Руки могучие Бог из чертогов эфирных протянет, 

\vs 3Sb 1:82 Свод небесный свернет, как будто свиток прочтенный; 

\vs 3Sb 1:83 Весь небосвод многовидный обрушится наземь и в море, 

\vs 3Sb 1:84 Огненный дождь будет лить и все сжигать непрестанно 

\vs 3Sb 1:85 Землю и воду спалит и небесную ось уничтожит. 

\vs 3Sb 1:86 Так творение Божье окажется сплавом единым, 

\vs 3Sb 1:87 После же снова на части разнимется для очищенья. 

\vs 3Sb 1:88 Больше не будут с небес никогда смеяться светила, 

\vs 3Sb 1:89 Ночь и заря упразднятся, и дней, заботами полных,

\vs 3Sb 1:90 Также не станет, исчезнут четыре времени года. 

\vs 3Sb 1:91 Век начнется великий, и Суд Всемощного Бога 

\vs 3Sb 1:92 Будет над миром, когда реченное все совершится.

\vs 3Sb 1:93 О судоходные воды, о суша вся от Востока 

\vs 3Sb 1:94 И до Заката  хоть больше уже не закатится солнце  

\vs 3Sb 1:95 Все подчинится Ему, в этот мир пришедшему снова, 

\vs 3Sb 1:96 Ибо сам Он познал Свою силу могучую первым.

\vs 3Sb 1:97 Все угрозы привел Безсмертный Бог в исполненье, 

\vs 3Sb 1:98 Коими людям грозил  они в земле Ассирийской 

\vs 3Sb 1:99 Башню построили  все меж собою согласными были  

\vs 3Sb 1:100 Страстно желали до звезд по этой башне добраться. 

\vs 3Sb 1:101 Тут Безсмертный ветрам повелел лететь что есть силы 

\vs 3Sb 1:102 К месту тому  ветра повергли огромное зданье,

\vs 3Sb 1:103 Ссору тогда меж собой учинили строители башни; 

\vs 3Sb 1:104 Вот почему с тех пор это место зовут Вавилоном.

\vs 3Sb 1:105 После крушения башни язык людской разделился 

\vs 3Sb 1:106 И превратился в обилье наречий разных, а дальше 

\vs 3Sb 1:107 Смертные, землю заполнив, ее поделили на царства. 

\vs 3Sb 1:108 То поколение было десятым с тех пор, как всемирный 

\vs 3Sb 1:109 Залил землю Потоп и первых людей уничтожил.

\vs 3Sb 1:110 Кронос, Титан и Япет над миром стали царями, 

\vs 3Sb 1:111 Люди их называли сынами Урана и Геи, 

\vs 3Sb 1:112 Имя земли и небес потому к царям прилагая, 

\vs 3Sb 1:113 Что наилучшими были они средь того поколенья. 

\vs 3Sb 1:114 Землю натрое всю разделили и бросили жребий,

\vs 3Sb 1:115 Каждый стал управлять в удел полученной частью; 

\vs 3Sb 1:116 Все отцу дали клятвы, и правильным было деленье, 

\vs 3Sb 1:117 Так что меж ними вражды не возникло. Но время настало 

\vs 3Sb 1:118 Умер старый отец. И, клятвы нарушив преступно, 

\vs 3Sb 1:119 Дети тогда меж собой учинили раздор величайший:

\vs 3Sb 1:120 Стали спорить, кому быть царем над всею землею. 

\vs 3Sb 1:121 Тут вражда началась у Титана и Кроноса злая. 

\vs 3Sb 1:122 Рея  сестра, мать Гея, Деметра и Афродита, 

\vs 3Sb 1:123 Та, что венки сплетать мастерица, и Гестия с ними, 

\vs 3Sb 1:124 Также Диона прекрасноволосая их помирили;

\vs 3Sb 1:125 Вместе собрали царей, их братьев и родичей разных, 

\vs 3Sb 1:126 Всех отцов и потомков, кто кровью был близок, созвали. 

\vs 3Sb 1:127 Те же держали совет и решили, что Кронос над всеми 

\vs 3Sb 1:128 Царствовать должен, поскольку он старше, благообразней. 

\vs 3Sb 1:129 Должен был Кронос Титану поклясться страшною клятвой

\vs 3Sb 1:130 В том, что мужского потомства иметь никогда он не будет, 

\vs 3Sb 1:131 Чтоб после смерти отца не мог его сын воцариться. 

\vs 3Sb 1:132 И когда срок наступал разрешиться от бремени Рее, 

\vs 3Sb 1:133 Подле Титаны садились и мальчиков всех разрывали 

\vs 3Sb 1:134 В клочья, а девочек всех у сосцов оставляли кормиться.

\vs 3Sb 1:135 Третьи роды настали у Реи, и первая Гера

\vs 3Sb 1:136 Вышла на свет, и, увидев младенца своими глазами, 

\vs 3Sb 1:137 Злые встали Титаны и все ушли восвояси. 

\vs 3Sb 1:138 Только затем появился ребенок пола мужского, 

\vs 3Sb 1:139 Рея тайком отослала его, чтобы спасся и вырос,

\vs 3Sb 1:140 Через трех жителей Крита во Фригию, клятвой связав их; 

\vs 3Sb 1:141 То, что сына вот так переслала, дало ему имя.

\vs 3Sb 1:142 Позже Рея спасла и другое дитя  Посейдона. 

\vs 3Sb 1:143 Третьим сыном Плутон был у этой женщины чудной  

\vs 3Sb 1:144 Недалеко от Додоны от бремени им разрешилась,

\vs 3Sb 1:145 Там, где несет свои воды Эвроп, который, с Пенеем 

\vs 3Sb 1:146 Слившись, в море течет и зовется Стигийской рекою. 

\vs 3Sb 1:147 Стало известно Титанам, что тайно смерти избегли 

\vs 3Sb 1:148 Дети, рожденные Реей от Кроноса. Тут возмутился 

\vs 3Sb 1:149 Сам Титан. Шестьдесят сыновей призвавши на помощь,

\vs 3Sb 1:150 Брата цепями сковал, а с ним и жену его Рею, 

\vs 3Sb 1:151 В землю упрятал обоих и там в оковах держал их. 

\vs 3Sb 1:152 Только узнали о том могучего Кроноса дети, 

\vs 3Sb 1:153 Подняли шум боевой и затеяли жаркую битву, 

\vs 3Sb 1:154 Битва великая та положила всем войнам начало,

\vs 3Sb 1:155 Первоначало всех войн среди смертных собою явила. 

\vs 3Sb 1:156 Вот за это наслал Господь на Титанов несчастье, 

\vs 3Sb 1:157 Сгинули все их потомки, но племя Кроноса  тоже. 

\vs 3Sb 1:158 Время затем совершило свой круг, и царство Египта , 

\vs 3Sb 1:159 Было воздвигнуто, вслед появились новые царства 

\vs 3Sb 1:160 Персов, Мидян, Эфиопов, в Ассирии вкруг Вавилона,

\vs 3Sb 1:161 У Македонцев и снова в Египте, в конце же  у Римлян. 

\vs 3Sb 1:162 Бог Всемогущий тогда вложил мне пророчество в душу, 

\vs 3Sb 1:163 И возвестить повелел по всей земле это слово, 

\vs 3Sb 1:164 Дабы властителям стало известно, что будет в грядущем.

\vs 3Sb 1:165 Первое то мне открыл Господь Единый, какие

\vs 3Sb 1:166 Царства людские возникнут и сколько их будет на свете. 

\vs 3Sb 1:167 Первым дом Соломонов над Азией всей воцарится, 

\vs 3Sb 1:168 Персии, Фригии станет владыкою и Финикии, 

\vs 3Sb 1:169 Островитяне, Карийцы, Мизийцы ему подчинятся,

\vs 3Sb 1:170 Он покорит и Лидийцев, богатое золотом племя... 

\vs 3Sb 1:171 Эллинов род, злодеев надменных, господствовать будет 

\vs 3Sb 1:172 Дальше, а после него  великое пестрое племя 

\vs 3Sb 1:173 Тех Македонцев, что тучи войны надвинут на смертных; 

\vs 3Sb 1:174 Бог небесный, однако, их всех уничтожит под корень.

\vs 3Sb 1:175 Но вслед за ними грядет другого царства начало: 

\vs 3Sb 1:176 Белый, могучий народ с берегом Гесперийского моря 

\vs 3Sb 1:177 Выйдет, разные страны захватит и в ужас повергнет 

\vs 3Sb 1:178 Многих, а в душах царей он страх надолго поселит. 

\vs 3Sb 1:179 Золота и серебра в городах награблено будет

\vs 3Sb 1:180 Тут немало, но вновь появится золото в мире

\vs 3Sb 1:181 И серебро, а потом и других украшений в достатке. 

\vs 3Sb 1:182 Смертные много тогда претерпят, но в наказанье

\vs 3Sb 1:183 Низко падут нечестивцы надменные, мерзостью жуткой 

\vs 3Sb 1:184 Жизнь их наполнится вся, мужчина с мужчиною станут

\vs 3Sb 1:185 Здесь предаваться разврату, а малых детей на продажу 

\vs 3Sb 1:186 Будут в позорных домах выставлять. Великое горе 

\vs 3Sb 1:187 К людям придет в те дни и посеет страшную смуту, 

\vs 3Sb 1:188 Все устои разрушит и злом это царство наполнит; 

\vs 3Sb 1:189 Страсть к наживе лихой, позорная алчность охватят

\vs 3Sb 1:190 Многие страны, а больше других  Македонскую землю. 

\vs 3Sb 1:191 Долго у них в чести коварство и ненависть будут, 

\vs 3Sb 1:192 Это продлится до царства седьмого по счету в Египте  

\vs 3Sb 1:193 Родом должен быть Эллин в то время Египта владыка  

\vs 3Sb 1:194 Сила появится вновь у народа великого Бога:

\vs 3Sb 1:195 Праведной жизни пути он смертным указывать станет...

\vs 3Sb 1:196 Но отчего же Господь вещать меня заставляет,

\vs 3Sb 1:197 Что будет первым несчастьем для всех человеков, что дальше

\vs 3Sb 1:198 С ними случится, где бедствий конец и где их источник?

\vs 3Sb 1:199 Первыми примут Титаны от Бога жестокую кару: 

\vs 3Sb 1:200 Мощного Кроноса им сыновья отомстят по заслугам, 

\vs 3Sb 1:201 Ибо сковали Титаны отца их и мать вероломно. 

\vs 3Sb 1:202 Позже у Эллинов власть захватят злые тираны, 

\vs 3Sb 1:203 И воцарятся у них надменные прелюбодеи 

\vs 3Sb 1:204 И нечестивцы, которым все доброе чуждо, а войнам 

\vs 3Sb 1:205 Впредь не будет конца. Фригийцы грозные сгинут, 

\vs 3Sb 1:206 Бедствий черные дни для Троянского града настанут. 

\vs 3Sb 1:207 Горе придет не замедлив и к Персам и к Ассирийцам, 

\vs 3Sb 1:208 В Ливию и к Эфиопам прошествует через Египет, 

\vs 3Sb 1:209 Быть ему и у Карийцев, в Памфилии также... да что я 

\vs 3Sb 1:210 Перечисляю народы?  у всех людей будет горе. 

\vs 3Sb 1:211 Только конец одному, как вскоре второе нагрянет 

\vs 3Sb 1:212 К людям несчастье, но я вначале скажу о первейшем.

\vs 3Sb 1:213 Горе постигнет и тех, кто возле великого храма, 

\vs 3Sb 1:214 Что Соломон воздвиг, живут в благочестье  и предки 

\vs 3Sb 1:215 Праведны были у них, и вот теперь поведу я 

\vs 3Sb 1:216 Речь об этом народе, земле его, предках и ясно 

\vs 3Sb 1:217 Все опишу для тебя, коварный и суетный смертный!

\vs 3Sb 1:218 Город есть на Востоке, зовется он Уром Халдейским, 

\vs 3Sb 1:219 Праведной жизни народ происходит оттуда, те люди

\vs 3Sb 1:220 Мыслили здраво всегда и много благого творили. 

\vs 3Sb 1:221 Их не заботит ничуть светил небесных вращенье, 

\vs 3Sb 1:222 Не помышляют они о земных чудовищах жутких. 

\vs 3Sb 1:223 Ни о манящих глубинах соленых вод Океана. 

\vs 3Sb 1:224 То, как птицы клюют, иль то, как люди чихают,

\vs 3Sb 1:225 Их не волнует, они чародеям и магам не верят,

\vs 3Sb 1:226 Чревовещатели ложью своей соблазнить их не в силах. 

\vs 3Sb 1:227 Не признают они там ни халдейских гаданий по звездам, 

\vs 3Sb 1:228 Ни астрономии. Нет, все то почитают обманом, 

\vs 3Sb 1:229 Чем занимаются изо дня в день неразумные люди,

\vs 3Sb 1:230 Души свои упражняя в вещах совершенно ненужных. 

\vs 3Sb 1:231 Этим своим заблужденьям они еще обучают 

\vs 3Sb 1:232 Разных глупцов, оттого много бед бывает, ведь люди, 

\vs 3Sb 1:233 Сбившись с благого пути, забывают о праведной жизни. 

\vs 3Sb 1:234 Те же, о ком говорю, почитают все справедливость

\vs 3Sb 1:235 И добродетель, не думают, как бы им стать побогаче 

\vs 3Sb 1:236 (Смертным нажива несет лишь зло, и голод, и войны), 

\vs 3Sb 1:237 Верная мера у них во всем в городах и в селеньях. 

\vs 3Sb 1:238 Здесь никто по ночам ничего у других не ворует, 

\vs 3Sb 1:239 Коз, овец и волов не бывает, чтоб тут угоняли,

\vs 3Sb 1:240 В поле земли никогда не отнимет сосед у соседа, 

\vs 3Sb 1:241 Самый богатый у них не обидит того, кто беднее, 

\vs 3Sb 1:242 Горя не причинит вдове, а напротив, поможет 

\vs 3Sb 1:243 Хлебом в нужде, вином и оливками  не поскупится. 

\vs 3Sb 1:244 Есть тут немало счастливцев, но, если кто-то несчастен,

\vs 3Sb 1:245 С бедным своим урожаем поделится летом имущий. 

\vs 3Sb 1:246 Ибо послушны они реченью великого Бога  

\vs 3Sb 1:247 Общей создал для всех небесный Царь эту землю.

\vs 3Sb 1:248 В дни, как Египет покинут и двинутся в путь по пустыне 

\vs 3Sb 1:249 Эти двенадцать колен, от Господа сопровождены;

\vs 3Sb 1:250 Будет дано им: в ночи столп огня озарит их дорогу, 

\vs 3Sb 1:251 Скрытые обликом, днем пойдут они безопасно. 

\vs 3Sb 1:252 Посланный Богом народу, его предводителем станет 

\vs 3Sb 1:253 Славный муж Моисей, который ребенком в болоте 

\vs 3Sb 1:254 Был царицею найден  она его воспитала,

\vs 3Sb 1:255 Сыном назван; и вот, с ним вышел народ из Египта. 

\vs 3Sb 1:256 Бог к Синайской горе привел их и с неба народу 

\vs 3Sb 1:257 Дал закон благочестья, на двух записав его досках. 

\vs 3Sb 1:258 И повелел: того, кто не станет блюсти предписаний,

\vs 3Sb 1:259 Или закон покарает, иль руки накажут людские,

\vs 3Sb 1:260 Если ж и скрыться сумеет, расплата его не минует. 

\vs 3Sb 1:261 [Общей создал для всех небесный Царь эту землю, 

\vs 3Sb 1:262 В сердце им всем Господь благое вложил помышленье.] 

\vs 3Sb 1:263 Только добрым сторицей воздаст хлебодарная пашня, 

\vs 3Sb 1:264 Так отмерил сам Бог. Но и добрых людей ожидают

\vs 3Sb 1:265 Беды, им не избегнуть никак ужасного мора.

\vs 3Sb 1:266 И побежишь ты тогда, покинув храм свой чудесный, 

\vs 3Sb 1:267 Ибо священную землю оставить велят тебе судьбы. 

\vs 3Sb 1:268 Жить придется тебе в земле Ассирийской, увидишь 

\vs 3Sb 1:269 Жен и малых детей рабами, людям враждебным.

\vs 3Sb 1:270 Все тут богатство погибнет, не сможешь добыть пропитанья; 

\vs 3Sb 1:271 Будут тобою полны все земли и воды морские, 

\vs 3Sb 1:272 Но не полюбит никто обычай твой и законы. 

\vs 3Sb 1:273 Вся же твоя страна опустеет: холм укрепленный, 

\vs 3Sb 1:274 Храм великого Бога и длинные мощные стены

\vs 3Sb 1:275 Рухнут тогда во прах, а причиною  то, что не чтил ты 

\vs 3Sb 1:276 Господом данный закон священный, но в заблужденье 

\vs 3Sb 1:277 Идолам мерзким служил и ничуть Того не боялся, 

\vs 3Sb 1:278 Кто породил всех богов и людей  Безсмертного Бога; 

\vs 3Sb 1:279 Чтить ты Его не желал, почитал изваяния смертных.

\vs 3Sb 1:280 Вот за это земля плодородная будет пустыней 

\vs 3Sb 1:281 Семь десятков времен, и во храме чудес не увидят. 

\vs 3Sb 1:282 Но в конце тебя ждут великая радость и слава: 

\vs 3Sb 1:283 Все исполнят Господь и смертный, когда не предашь ты 

\vs 3Sb 1:284 Веры в священный закон, полученный некогда свыше, 

\vs 3Sb 1:285 Ноги устанут твои, но светлого дня ты достигнешь.

\vs 3Sb 1:286 Царь будет послан от Бога с высокого неба на землю,

\vs 3Sb 1:287 Каждого станет судить в крови и в пламенном свете. 

\vs 3Sb 1:288 Только один парод, одно лишь царское племя 

\vs 3Sb 1:289 Не поколеблется тут. Ему предназначено править 

\vs 3Sb 1:290 В смене времен и начать строительство нового храма. 

\vs 3Sb 1:291 Всякий владыка Персидский тут помощь оказывать станет 

\vs 3Sb 1:292 Бронзою, кованым прочным железом и золотом даже, 

\vs 3Sb 1:293 Ибо Господь сновиденья священные ночью пошлет им. 

\vs 3Sb 1:294 Так, воздвигнувшись вновь, святыня пребудет, как прежде.

\vs 3Sb 1:295 В сердце утихло моем звучанье божественной песни, 

\vs 3Sb 1:296 И обратила мольбы я к Творцу, чтоб Он дал мне покоя.

\vs 3Sb 1:297 Но Всемогущий опять вложил пророчество в душу

\vs 3Sb 1:298 И возвестить повелел по всей земле это слово,

\vs 3Sb 1:299 Дабы властителям стало известно, что будет в грядущем.

\vs 3Sb 1:300 Первое, что наказал мне Господь Единый поведать,  

\vs 3Sb 1:301 Сколько горьких напастей отмерил Он Вавилону 

\vs 3Sb 1:302 Карою за разграбленье великого Божьего храма. 

\vs 3Sb 1:303 Горе тебе, Вавилон и племя мужей Ассирийских! 

\vs 3Sb 1:304 Шум ужасный услышат родившие грешников земли,

\vs 3Sb 1:305 Клич боевой принесет погибель внезапную людям, 

\vs 3Sb 1:306 Бог, моих песен владыка, сразит их могучим ударом. 

\vs 3Sb 1:307 Бог к тебе, Вавилон, сойдет из высей воздушных, 

\vs 3Sb 1:308 Спустится Он со святых небес на грешную землю.

\vs 3Sb 1:309 Гнев Господень сулит сынам твоим вечную гибель.

\vs 3Sb 1:310 Станешь тогда ты, как если б и не было вовсе на свете 

\vs 3Sb 1:311 Города никогда такого, наполнишься кровью, 

\vs 3Sb 1:312 Вспомнишь, как сам проливал кровь добрых и справедливых, 

\vs 3Sb 1:313 Что и доселе еще вопиет к высокому небу.

\vs 3Sb 1:314 Страшный удар потрясет твои жилища, Египет, 

\vs 3Sb 1:315 Ты никогда и представить не мог, что случится такое! 

\vs 3Sb 1:316 Меч тяжелый тебя пронзит посредине, а следом 

\vs 3Sb 1:317 Голод и мор и рассеянье будут, идя чередою, 

\vs 3Sb 1:318 В царство седьмое губить страну, и так ты исчезнешь.

\vs 3Sb 1:319 Гог и Магог, увы, увы тебе, край Эфиопский! 

\vs 3Sb 1:320 Реки текут по тебе сейчас, но в будущем хлынет 

\vs 3Sb 1:321 Кровь потоками здесь, и, затоплена черною кровью, 

\vs 3Sb 1:322 Судным местом тогда среди смертных ты прозовешься.

\vs 3Sb 1:323 Ливия, горе тебе, и землям горе и водам!

\vs 3Sb 1:324 Дочери Запада, вас настигнет день несчастливый, 

\vs 3Sb 1:325 Вам не уйти от борьбы тяжелой, она неотступно

\vs 3Sb 1:326 Будет преследовать вас, и суд наступит ужасный.

\vs 3Sb 1:327 И поневоле придется погибнуть вам всем в это время.

\vs 3Sb 1:328 Ибо бессмертного Бога жилища вы источили

\vs 3Sb 1:329 И растерзали его вконец зубами стальными, 

\vs 3Sb 1:330 Край свой увидишь тогда в страну мертвецов превращенным:

\vs 3Sb 1:331 Сгинут одни от войны и по воле несчастного рока, 

\vs 3Sb 1:332 Голод иных уничтожит, чума и бешенство вражье, 

\vs 3Sb 1:333 Все твои города, все земли пустынею станут. 

\vs 3Sb 1:334 Вспыхнет звезда на Заходе  она наречется кометой  

\vs 3Sb 1:335 Вестницей станет она сражений, голода, смерти, 

\vs 3Sb 1:336 Гибели славных вождей и прочих людей знаменитых.

\vs 3Sb 1:337 Знаменья будут тогда даны величайшие смертным: 

\vs 3Sb 1:338 Течь прекратит Танаис в Меотиду струей многоводной,

\vs 3Sb 1:339 Высохнув, русло его плодородною пашнею станет,

\vs 3Sb 1:340 В озеро воды польются по множеству малых протоков. 

\vs 3Sb 1:341 Много в почве возникнет провалов и пропастей, рухнут 

\vs 3Sb 1:342 В них города со всеми людьми. Эта страшная участь 

\vs 3Sb 1:343 В Азии Смирну постигнет, Иас, Кебрен, Пандонию, 

\vs 3Sb 1:344 Антиохию, Эфес, Колофон, Никею, Скиагру,

\vs 3Sb 1:345 Астипалею, Синоп, Иераполь, счастливую Газу. 

\vs 3Sb 1:346 Так же погибнут в Европе Танагра и Меропея, 

\vs 3Sb 1:347 Так пропадут Микены, Магнезия и Антигона. 

\vs 3Sb 1:348 Знайте тогда, что вскоре конец настанет Египту, 

\vs 3Sb 1:349 Прежнее лето всегда будет лучшим для Александрийцев.

\vs 3Sb 1:350 Сколько бы Рим ни взял с покоренной Азии дани, 

\vs 3Sb 1:351 Втрое больше ему возвратить сокровищ придется 

\vs 3Sb 1:352 Азии, ибо надменным она победителем станет. 

\vs 3Sb 1:353 Много богатств возьмет с Азиатов народ Италийский,

\vs 3Sb 1:354 Двадцатикратно, однако, он собственной рабскою службой

\vs 3Sb 1:355 Должен будет вернуть, в нищете пребывая великой. 

\vs 3Sb 1:356 В золоте, в роскоши ты, о дочь Латинского Рима, 

\vs 3Sb 1:357 С множеством женихов сколь часто вином упивалась!  

\vs 3Sb 1:358 В жены тебя отдадут не в пышном наряде  служанкой, 

\vs 3Sb 1:359 Срежет тебе госпожа копну волос твоих пышных.

\vs 3Sb 1:360 Восторжествует тогда справедливость, и с неба на землю

\vs 3Sb 1:361 Сброшено будет одно, из праха восстанет другое  

\vs 3Sb 1:362 Слишком уж люди погрязли в пороке и жизни нечестной.

\vs 3Sb 1:363 Делос невидимым станет, а Самос в песок превратится, 

\vs 3Sb 1:364 Рим руинами будет  исполнятся все предсказанья. 

\vs 3Sb 1:365 Больше ни слова о Смирне  пускай себе погибает 

\vs 3Sb 1:366 От преступлений вождей, неразумных и несправедливых.

\vs 3Sb 1:367 В Азии тихий покой воцарится, счастливою станет 

\vs 3Sb 1:368 В те времена и Европа: блаженную жизнь и здоровье 

\vs 3Sb 1:369 Небо людям пошлет вместо злого снега и града,

\vs 3Sb 1:370 Даст оно много зверей и птиц и ползучих в достатке.

\vs 3Sb 1:371 О, сколь счастливы те мужи и жены, которым 

\vs 3Sb 1:372 Жить доведется в тот век, похожий на дивную сказку.

\vs 3Sb 1:373 Благозаконие и справедливость со звездного неба 

\vs 3Sb 1:374 К людям придут, и тогда воцарится всем смертным на пользу

\vs 3Sb 1:375 Мудрое мыслей единство, а с ним  любовь и доверье,

\vs 3Sb 1:376 Гостеприимства законы блюсти станут люди; при этом 

\vs 3Sb 1:377 Вовсе исчезнут нужда и насилие, больше не будет 

\vs 3Sb 1:378 Зависти, гнева, насмешек, безумства и преступлений; 

\vs 3Sb 1:379 Ссоры, жестокая брань, грабеж по ночам и убийства 

\vs 3Sb 1:380 В общем, всякое зло в те дни на земле прекратится. 

\vs 3Sb 1:381 Но Македонцы сулят всей Азии тяжкие беды; 

\vs 3Sb 1:382 Вырастет и для Европы еще великое горе  

\vs 3Sb 1:383 Горе от племени мнимых Кронидов и рабского рода; 

\vs 3Sb 1:384 И Вавилон, хорошо укрепленный, захватит их войско.

\vs 3Sb 1:385 Этих людей назовут владыками целого света,

\vs 3Sb 1:386 Но погибнет их царство от страшных бед, не оставив 

\vs 3Sb 1:387 Даже законов потомкам, по разным разбредшимся странам.

\vs 3Sb 1:388 Муж коварный в то время в счастливую Азию вступит, 

\vs 3Sb 1:389 Будет носить на плечах порфирное он одеянье.

\vs 3Sb 1:390 Молния в мир его принесет. Потому-то, свирепый, 

\vs 3Sb 1:391 Дикий и непостоянный, ярмо для Азии злое 

\vs 3Sb 1:392 Этот муж приготовит, и кровью убийства сырая 

\vs 3Sb 1:393 Здесь упьется земля, но Аид усмирит кровопийцу. 

\vs 3Sb 1:394 Род, который под корень хотелось ему уничтожить,

\vs 3Sb 1:395 Сам погубит потом его, а единственный корень  

\vs 3Sb 1:396 Срубит после один из десятка рогов кровожадный, 

\vs 3Sb 1:397 После же новый побег он с прежними рядом насадит. 

\vs 3Sb 1:398 Но, погубив отца порфироносного рода, 

\vs 3Sb 1:399 Тоже погибнет от рук детей, заговорщиков дерзких,

\vs 3Sb 1:400 И воцарится затем тот рог, что посажен был рядом.

\vs 3Sb 1:401 Знаменье явится вскоре Фригийской земле плодоносной: 

\vs 3Sb 1:402 Род огромный и злой  потомки матери Реи  

\vs 3Sb 1:403 Вечным мнивший себя, ибо вырос из корня сухого, 

\vs 3Sb 1:404 В ночь исчезнет одну. В эту ночь земли Колебатель

\vs 3Sb 1:405 Почву разверзнет в том граде, которому люди позднее 

\vs 3Sb 1:406 Имя дадут Дорилейон. Все это в древней случится 

\vs 3Sb 1:407 Черной Фригийской земле, не раз слезами политой. 

\vs 3Sb 1:408 Времени этому люди дадут Колебателя имя, 

\vs 3Sb 1:409 Ибо Он щели разверзнет земные и стены разрушит.

\vs 3Sb 1:410 Знаки все это дурные  за ними беды начнутся.

\vs 3Sb 1:411 Множество разных народов придет со своими вождями, 

\vs 3Sb 1:412 Чтобы в земле воевать, где предки живут Энеадов. 

\vs 3Sb 1:413 Станут добычей они снедаемым жадностью людям.

\vs 3Sb 1:414 Горе тебе, Илион! Эриния вырастит в Спарте

\vs 3Sb 1:415 Ветвь чудесную, чья красота всем известною станет. 

\vs 3Sb 1:416 Но породит она бурю над Азией и над Европой, 

\vs 3Sb 1:417 Ты, Илион, больше всех услышишь тут плача и стонов, 

\vs 3Sb 1:418 Вечно будут, однако, виновницу помнить потомки.

\vs 3Sb 1:419 Явится старец затем и напишет много неправды, 

\vs 3Sb 1:420 Ложно и город родной назовет. Хоть света не узрят 

\vs 3Sb 1:421 Очи его, но с великим умом и, мысль облекая 

\vs 3Sb 1:422 Ясно в слова, он напишет  но, Хиос своим называя, 

\vs 3Sb 1:423 Речь о том поведет, что у стен Илиона свершилось. 

\vs 3Sb 1:424 Ложь его будет правдивой казаться: слова и размеры 

\vs 3Sb 1:425 Он ведь из книг моих почерпнет, сперва прочитав их.

\vs 3Sb 1:426 Очень красиво опишет воителей подвиги старец,

\vs 3Sb 1:427 Гектора, сына Приама, и сына Пелея, Ахилла,

\vs 3Sb 1:428 Также и прочих, кто в этой войне подвизался отважно.

\vs 3Sb 1:429 Изобразит он, что боги сражавшимся там помогали, 

\vs 3Sb 1:430 Самую разную ложь сочинит для людей скудоумных.

\vs 3Sb 1:431 Павшим у стен Илиона причтется великая слава:

\vs 3Sb 1:432 Поочередно певец про оба войска расскажет.

\vs 3Sb 1:433 Много зла причинит Ликийцам Локра потомство;

\vs 3Sb 1:434 Ты, Халкидон, у морской теснины лежащий, погибнешь 

\vs 3Sb 1:435 В час, как придет к тебе дитя земли Этолийской.

\vs 3Sb 1:436 Кизик, море отнимет твое богатство и счастье;

\vs 3Sb 1:437 Плохо придется тебе, Византий, стоящий напротив

\vs 3Sb 1:438 Азии; стоном и кровью до края ты будешь наполнен.

\vs 3Sb 1:439 С той вершины горы, что над Ликией высится, воды 

\vs 3Sb 1:440 Хлынут потоками вниз, из твердого выбившись камня.

\vs 3Sb 1:441 Чтобы утихли они, надо сбыться отцов предсказаньям.

\vs 3Sb 1:442 Город обильной вином Пропонтиды, увы тебе, Кизик! 

\vs 3Sb 1:443 Бурной Риндакской волною ты будешь залит и потоплен.

\vs 3Sb 1:444 Родос, и твой век недолог, хотя и немалое время 

\vs 3Sb 1:445 Рабства ты не познаешь и славиться будешь богатством, 

\vs 3Sb 1:446 И не оспорит никто твоего господства над морем. 

\vs 3Sb 1:447 Все же станешь добычей снедаемым жадностью людям, 

\vs 3Sb 1:448 За красоту и богатство  ужасное иго претерпишь.

\vs 3Sb 1:449 В Лидии вздрогнет земля, и Персия вся сокрушится;

\vs 3Sb 1:450 Сколько несчастий Европу и Азию тут ожидают!

\vs 3Sb 1:451 Царь Сидонский и много других владык кровожадных 

\vs 3Sb 1:452 За море смерть понесут с собою  на Самос и дальше. 

\vs 3Sb 1:453 В море много земли потоки кровавые смоют, 

\vs 3Sb 1:454 Жены и девы в красивых нарядах горько заплачут;

\vs 3Sb 1:455 Жалкую долю свою проклянут они, ибо навеки 

\vs 3Sb 1:456 Эти любимых отцов, а те  сыновей потеряют.

\vs 3Sb 1:457 Будет для Кипра знак  ужасное землетрясенье, 

\vs 3Sb 1:458 Множество душ оно Аиду отдаст в одночасье!

\vs 3Sb 1:459 Рухнут и мощные стены с Эфесом соседнего Тралла 

\vs 3Sb 1:460 За преступления их обитателей, злых и жестоких. 

\vs 3Sb 1:461 Воды горячие с неба на землю польются, и станет 

\vs 3Sb 1:462 Впитывать почва ту влагу и запах удушливой серы.

\vs 3Sb 1:463 Самос царский дворец построит, как сроки настанут.

\vs 3Sb 1:464 Не чужеземный Арей твоим, Италия, бедам 

\vs 3Sb 1:465 Будет причиной, но кровь, с которою тяжко бороться, 

\vs 3Sb 1:466 Кровь родных сыновей разорит тебя без пощады! 

\vs 3Sb 1:467 Все об этом позоре узнают, и, в пепле простершись, 

\vs 3Sb 1:468 Ты погибнешь  и раньше могла все это предвидеть: 

\vs 3Sb 1:469 Матерь добрых людей, зверенышей диких вскормила.

\vs 3Sb 1:470 Муж-paзоритель когда придет из Италии новый, 

\vs 3Sb 1:471 Лаодикия, во прах преклонишь тогда ты колени. 

\vs 3Sb 1:472 Славный город Карийский, у струй прекрасного Лика, 

\vs 3Sb 1:473 Горько оплакав надменного предка, навеки умолкнешь.

\vs 3Sb 1:474 Племя Фракийцев взойдет на вершины высокого Гема.

\vs 3Sb 1:475 Лихо придет и к Кампанцам  опустошительный голод; 

\vs 3Sb 1:476 Древний день своего основания также оплачут 

\vs 3Sb 1:477 Кирн и Сардиния, их удары холодного ветра, 

\vs 3Sb 1:478 Посланы Богом святым, в пучину соленую сбросят, 

\vs 3Sb 1:479 Станут они в волнах морским обитальцам добычей.

\vs 3Sb 1:480 Горе! сколько Аид невест прекрасных добудет,

\vs 3Sb 1:481 Сколько непогребенных юнцов не отпустят глубины! 

\vs 3Sb 1:482 О, невинные дети! О, тяжкое злато в пучине!

\vs 3Sb 1:483 Царский возникнет род в земле Мизийцев счастливой.

\vs 3Sb 1:484 Жизнь Кархедона, однако, не долго вовсе продлится. 

\vs 3Sb 1:485 Жалобный стон разнесется среди Галатов, и будет 

\vs 3Sb 1:486 На Тенедосе несчастье последнее самым ужасным. 

\vs 3Sb 1:487 И Сикион, и Коринф возгордятся лаем доспехов, 

\vs 3Sb 1:488 Но не минует их участь вести бесславные войны.

\vs 3Sb 1:489 В сердце утихло моем звучанье божественной песни,

\vs 3Sb 1:490 Но Всемогущий опять вложил мне пророчество в душу 

\vs 3Sb 1:491 И возвестить повелел по всей земле это слово.

\vs 3Sb 1:492 Горе вам всем, Финикийцы, мужчинам и женщинам горе! 

\vs 3Sb 1:493 Также и всем городам Побережья морского  не сможет 

\vs 3Sb 1:494 Светлой дорогой никто из вас до солнца достигнуть.

\vs 3Sb 1:495 Больше семей не возникнет и новых детей не родится  

\vs 3Sb 1:496 За дерзновенный язык и жизнь порочную смертных, 

\vs 3Sb 1:497 Наглых и беззаконных хулителей и святотатцев. 

\vs 3Sb 1:498 Страшные лживые речи вели преступники эти, 

\vs 3Sb 1:499 И против Господа мерзкий мятеж они учиняли;

\vs 3Sb 1:500 Карой за все злодеянья Господь бичевые удары

\vs 3Sb 1:501 Страшно обрушит на смертных от края земли и до края. 

\vs 3Sb 1:502 Горькая ждет их судьба, когда города и жилища 

\vs 3Sb 1:503 До основанья огнем небесным выжжены будут.

\vs 3Sb 1:504 Горе, о горе тебе, печали и скорби обитель, 

\vs 3Sb 1:505 Крит, от удара ты рухнешь ужасного, сгинешь навеки. 

\vs 3Sb 1:506 Ты задымишься тогда на глазах у целого света, 

\vs 3Sb 1:507 И не оставит уж пламя тебя, пока не исчезнешь.

\vs 3Sb 1:508 Фракия, горе тебе  рабынею жалкою станешь: 

\vs 3Sb 1:509 Время настанет, Галаты набег совершат на Элладу 

\vs 3Sb 1:510 Вместе с Дарданцами, тут-то несчастье тебя ожидает  

\vs 3Sb 1:511 Беды несла ты другим, теперь же возмездие примешь.

\vs 3Sb 1:512 Горе вам, Гог и Магог и все племена по соседству  

\vs 3Sb 1:513 Марсы, Анги, иные  вас ждет ужасная участь.

\vs 3Sb 1:514 Много крушений грядут к Ликийцам, Мизийцам, Фригийцам,

\vs 3Sb 1:515 К жителям Лидии и Памфилийцам, а также и к людям 

\vs 3Sb 1:516 Варварской речи  ко всем Эфиопам, Маврам, Арабам, 

\vs 3Sb 1:517 Каппадокийцам. Но что я пророчить каждому стану 

\vs 3Sb 1:518 Жребий его? Ибо всем племенам, населяющим землю, 

\vs 3Sb 1:519 Страшный удар ниспошлет и тяжкую кару Всевышний.

\vs 3Sb 1:520 Варварский, чуждый народ появится в Эллинском крае  

\vs 3Sb 1:521 Многим голов не сносить в то время мужам знаменитым, 

\vs 3Sb 1:522 Много жирных овец у смертных угнано будет, 

\vs 3Sb 1:523 Зычно ревущих быков, коней и мулов без счета. 

\vs 3Sb 1:524 Крепкой постройки дома предав огню беззаконно,

\vs 3Sb 1:525 Жителей в рабство насильно угонят враги на чужбину. 

\vs 3Sb 1:526 Эллин, увидишь картины ужасные: жен беззащитных 

\vs 3Sb 1:527 Вместе с детьми из покоев выбрасывать станут на землю 

\vs 3Sb 1:528 Нежным коленом, и жены в оковах вражьих претерпят 

\vs 3Sb 1:529 Весь позор униженья у варваров. Нет им защиты

\vs 3Sb 1:530 Здесь от расправы, никто им в страшной беде не поможет! 

\vs 3Sb 1:531 Враг все богатство твое и все достоянье присвоит, 

\vs 3Sb 1:532 И затрясутся колени твои, если это увидишь. 

\vs 3Sb 1:533 Сто человек побегут, чтоб спастись, но один всех погубит. 

\vs 3Sb 1:534 Пятеро гневом ужасным тогда вскипят, но позорно

\vs 3Sb 1:535 Между собой препираться начнут и оружье подымут 

\vs 3Sb 1:536 Друг против друга на радость врагам, а Элладе  на горе. 

\vs 3Sb 1:537 В рабстве придется Элладе сносить тяжелое иго; 

\vs 3Sb 1:538 Всех будут мучить война и с нею губительный голод. 

\vs 3Sb 1:539 Небо высокое медью Господь покроет, а землю

\vs 3Sb 1:540 Высушит всю, и в железо поверхность ее превратится. 

\vs 3Sb 1:541 И, убедившись, что больше нельзя ни вспахать, ни посеять, 

\vs 3Sb 1:542 Горько люди заплачут. Но Бог великий, что создал 

\vs 3Sb 1:543 Все на свете, теперь обрушит жестокое пламя 

\vs 3Sb 1:544 Вниз, и тогда лишь треть людей на земле уцелеет.

\vs 3Sb 1:545 О, для чего ты, Эллада, на смертных вождей полагалась? 

\vs 3Sb 1:546 Им не дано ведь никак избегнуть конца рокового; 

\vs 3Sb 1:547 Что ж ублажаешь дарами никчемными тех, кто погибнет, 

\vs 3Sb 1:548 А изваяниям жертвы приносишь? Отколь научилась 

\vs 3Sb 1:549 Делать такое, презрев Лицо всемогущего Бога?

\vs 3Sb 1:550 Имя Родителя общего чти, не оставь в небреженье! 

\vs 3Sb 1:551 Правили тысячу лет и еще пять сотен вдобавок 

\vs 3Sb 1:552 Много надменных царей в Элладе, и вот от них-то 

\vs 3Sb 1:553 Первых учиться злу неразумные смертные стали: 

\vs 3Sb 1:554 Идолов мертвых воздвигли для тех, кто сами невечны;

\vs 3Sb 1:555 Этим в умы вам вложили пустые ложные мысли. 

\vs 3Sb 1:556 Но когда Божий гнев разразится над вами внезапно, 

\vs 3Sb 1:557 Сразу узнаете тут Лицо всемогущего Бога. 

\vs 3Sb 1:558 Тут же все души людские, наполнив воздух стенаньем, 

\vs 3Sb 1:559 Руки к широкому небу с мольбою протягивать станут,

\vs 3Sb 1:560 И о защите молить Царя великого в небе,

\vs 3Sb 1:561 И вопрошать: кто же их от страшного гнева избавит?

\vs 3Sb 1:562 Должен еще ты узнать, и в уме твоем пусть сохранится, 

\vs 3Sb 1:563 Сколько несчастий несут с собою бегущие годы

\vs 3Sb 1:564 Зычно ревущих быков и коров соберет в изобилье 

\vs 3Sb 1:565 К храму великого Бога Эллада, и больше не станет 

\vs 3Sb 1:566 Злобных побоищ на землях ее и гнетущего страха, 

\vs 3Sb 1:567 Голод и рабское иго тогда же вскоре исчезнут. 

\vs 3Sb 1:568 Род нечестивцев, однако, дотоле продлится, покуда 

\vs 3Sb 1:569 Срок судеб истечет и день настанет реченный. 

\vs 3Sb 1:570 Жертвуйте Богу тогда лишь, когда все исполнится, ибо 

\vs 3Sb 1:571 То, чего Он желает, небывшим остаться не может, 

\vs 3Sb 1:572 Все заставит Господь по воле Его совершиться.

\vs 3Sb 1:573 Явится племя святое людей, благочестия полных, 

\vs 3Sb 1:574 Господу истинно верных и волей и помыслом всяким.

\vs 3Sb 1:575 Храм великого Бога почтят кроплением влаги 

\vs 3Sb 1:576 И возжиганием дыма от тучных жертв, гекатомбой 

\vs 3Sb 1:577 Тою священной, когда закалают быков превосходных, 

\vs 3Sb 1:578 Жирных баранов, овец и едва родившихся агнцев; 

\vs 3Sb 1:579 Многое с мыслью благой на алтарь великий возложат.

\vs 3Sb 1:580 И по законам великого Бога, храня справедливость,

\vs 3Sb 1:581 Будут счастливую жизнь проводить в городах и на пашнях. 

\vs 3Sb 1:582 Их Бессмертный возвысит, пророками сделав, и станут 

\vs 3Sb 1:583 Радость большую нести всем смертным, на свете живущим. 

\vs 3Sb 1:584 Только им даровал разуменье благое Всевышний,

\vs 3Sb 1:585 Веру и лучшие чувства людские вложил Он в их души. 

\vs 3Sb 1:586 Делу рук человечьих они молиться не станут, 

\vs 3Sb 1:587 Золото, медь, серебро, слоновья кость не прельстят их, 

\vs 3Sb 1:588 Крашеным идолам шатким из дерева, камня и глины, 

\vs 3Sb 1:589 Изображеньям зверей и всему, что смертных бездумных

\vs 3Sb 1:590 Вводит легко в соблазн, не будут они поклоняться. 

\vs 3Sb 1:591 Но поутру ото сна пробудясь, омывают водою 

\vs 3Sb 1:592 Руки и чистыми их всегда к небесам воздевают. 

\vs 3Sb 1:593 Чтут одного лишь Владыку  Безсмертного, Вечного Бога, 

\vs 3Sb 1:594 Мать и отца вслед за Ним; и больше, чем прочие люди,

\vs 3Sb 1:595 В мысли имеют они сохранять целомудрие ложа, 

\vs 3Sb 1:596 С детями пола мужского не водят позорную дружбу, 

\vs 3Sb 1:597 Как Египтяне и как Финикийцы, а также Латины, 

\vs 3Sb 1:598 Эллины разных племен и множество прочих народов: 

\vs 3Sb 1:599 Персы, Галаты и целая Азия  все, кто забыли

\vs 3Sb 1:600 И преступили священный закон Безсмертного Бога. 

\vs 3Sb 1:601 Людям пошлет Господь за эти грехи наказанье: 

\vs 3Sb 1:602 Пагубу разную, голод, страдания, жалкие стоны, 

\vs 3Sb 1:603 Войны жестокие, мор и слезы от боли ужасной. 

\vs 3Sb 1:604 К вечному ибо Отцу всех тех, кто мир населяет,

\vs 3Sb 1:605 Не пожелали почтенья иметь, а идолов чтили; 

\vs 3Sb 1:606 Будет, однако, пора, и дело рук своих сами 

\vs 3Sb 1:607 Сбросят в расселины гор, дабы скрыть позор величайший. 

\vs 3Sb 1:608 Новый владыка тогда воцарится в Египте, и станет 

\vs 3Sb 1:609 Он седьмым от начала правления Эллинов, то есть

\vs 3Sb 1:610 С той поры, как начнется здесь власть мужей Македонских. 

\vs 3Sb 1:611 Тут горящим орлом великий царь Азиатский 

\vs 3Sb 1:612 Явится, землю покрыв и пешим войском и конным; 

\vs 3Sb 1:613 Все на пути своем уничтожит и злом переполнит. 

\vs 3Sb 1:614 Царская власть сокрушится и Египте тогда, а захватчик,

\vs 3Sb 1:615 Всю добычу забрав, уплывет за широкое море.

\vs 3Sb 1:616 Люди пред Господом Богом, великим и вечным, колена 

\vs 3Sb 1:617 Белые тут преклонят, опустясь на кормилицу-землю. 

\vs 3Sb 1:618 Рухнут и сгинут в пожаре творения рук человечьих; 

\vs 3Sb 1:619 Но получат взамен от Бога великую радость

\vs 3Sb 1:620 Смертные, ибо земля, деревья и пастбища будут

\vs 3Sb 1:621 Истинный плод приносить, и тогда появится вдоволь 

\vs 3Sb 1:622 Сладкого меда, вина, молока белоснежного, хлеба  

\vs 3Sb 1:623 Главное, хлеба, ведь он  наивысшее благо для смертных.

\vs 3Sb 1:624 Только медлить не смей, злонравный и суетный смертный,

\vs 3Sb 1:625 Но обратись покаянно, моли прощенья у Бога! 

\vs 3Sb 1:626 В жертву Ему приноси козлят и ягнят первородных, 

\vs 3Sb 1:627 Сотни быков приноси, пока сменяются годы. 

\vs 3Sb 1:628 Господа ты умоляй низойти и явить Свою милость, 

\vs 3Sb 1:629 Ибо единый Он Бог, и быть другого не может.

\vs 3Sb 1:630 Чти справедливость всегда, никому не делай обиды  

\vs 3Sb 1:631 Это  Безсмертного Бога веление людям несчастным. 

\vs 3Sb 1:632 Но берегись пробужденья всевышнего Божьего гнева 

\vs 3Sb 1:633 В час, как на смертных чума нагрянет, несущая гибель, 

\vs 3Sb 1:634 И не уйдет человек тогда от расплаты ужасной.

\vs 3Sb 1:635 Встанет царь на царя, победит и землю отнимет, 

\vs 3Sb 1:636 Племя на племя пойдет, правители многих погубят, 

\vs 3Sb 1:637 Все вожди разбегутся по разным странам в то время. 

\vs 3Sb 1:638 Облик изменит земля, засилье варваров диких 

\vs 3Sb 1:639 Опустошит всю Элладу, ее плодородная почва

\vs 3Sb 1:640 Всяких лишится богатств; но вражды причиною будут 

\vs 3Sb 1:641 Золото и серебро  готовит многие беды 

\vs 3Sb 1:642 Любостяжательство людям, нет пастыря хуже, чем алчность.

\vs 3Sb 1:643 \ldots\ и на чужбине они останутся без погребенья, 

\vs 3Sb 1:644 Дикие звери и злые стервятники здесь растерзают

\vs 3Sb 1:645 Их тела, а когда реченное все совершится, 

\vs 3Sb 1:646 Этих усопших останки земля широкая скроет. 

\vs 3Sb 1:647 Но уж не вспашет ту землю никто, и никто не засеет; 

\vs 3Sb 1:648 Скажет несчастьем своим о позоре множества смертных \ldots

\vs 3Sb 1:649 В долгой смене времен и годов обращенья не станет 

\vs 3Sb 1:650 Копий, щитов и другого оружья, привычного людям; 

\vs 3Sb 1:651 Чтобы разжечь костры, железом древес не коснутся.

\vs 3Sb 1:652 Бог на землю пошлет царя, что придет от Восхода, 

\vs 3Sb 1:653 Злую войну прекратит этот царь по всей поднебесной, 

\vs 3Sb 1:654 Жизни лишая одних и клятвы другим выполняя. 

\vs 3Sb 1:655 Но не по воле своей совершит он деяния эти, 

\vs 3Sb 1:656 А подчиняясь благим веленьям великого Бога \ldots\

\vs 3Sb 1:657 Свой же народ Господь одарит чудесным богатством  

\vs 3Sb 1:658 Золотом и серебром и прекрасной одеждой пурпурной; 

\vs 3Sb 1:659 И плодородная почва и даже соленое море

\vs 3Sb 1:660 Много благ принесут. Но снова цари друг на друга 

\vs 3Sb 1:661 Примутся зло замышлять и творить его в гневе великом; 

\vs 3Sb 1:662 Будет недобрая зависть в обычае смертных несчастных. 

\vs 3Sb 1:663 Против все той же земли цари поднимут народы, 

\vs 3Sb 1:664 Только в поход соберутся они себе на погибель.

\vs 3Sb 1:665 Храм великого Бога святой и мужей наилучших 

\vs 3Sb 1:666 Всех истребить захотят; и, в землю эту явившись, 

\vs 3Sb 1:667 Город цари окружат и средь войск своих непокорных 

\vs 3Sb 1:668 Сядут на трон и начнут приносить нечестивые жертвы. 

\vs 3Sb 1:669 В этот-то час и раздастся с небес голос Бога могучий

\vs 3Sb 1:670 К диким и глупым народам, и суд начнется над ними, 

\vs 3Sb 1:671 Суд великого Бога, Который бессмертной рукою 

\vs 3Sb 1:672 Их умертвит. С высоты мечи огневые на землю 

\vs 3Sb 1:673 Он обрушит тогда; и огромные факелы будут 

\vs 3Sb 1:674 Всех людей освещать, внезапно средь них появившись.

\vs 3Sb 1:675 И от Господней руки земля, что все порождает, 

\vs 3Sb 1:676 Тут сотрясется, и все затрепещут  рыбы морские, 

\vs 3Sb 1:677 Звери земные и птиц несчетные стаи и воды, 

\vs 3Sb 1:678 Также и души людей содрогнутся, как только увидят 

\vs 3Sb 1:679 Лик Безсмертного Бога  и ужас будет великий.

\vs 3Sb 1:680 Гор высоких вершины, холмы и крутые обрывы 

\vs 3Sb 1:681 Он сокрушит, и черный всем взорам явится Тартар. 

\vs 3Sb 1:682 Полными трупов предстанут ущелья туманные в скалах, 

\vs 3Sb 1:683 Брызнет кровь из камней и вниз потоками хлынет 

\vs 3Sb 1:684 С гор по теснинам и быстро долины собою затопит.

\vs 3Sb 1:685 Рухнут крепкие стены, ведь их неразумные люди 

\vs 3Sb 1:686 Строили, вовсе не зная закона великого Бога, 

\vs 3Sb 1:687 Ни суда, что их ждет. Потому-то резню учинили 

\vs 3Sb 1:688 И в безумии вы на Святыню подняли копья. 

\vs 3Sb 1:689 Будет судить Господь вас всех мечом и войною,

\vs 3Sb 1:690 Пламенем и дождем, затопляющим землю, и серой, 

\vs 3Sb 1:691 С неба летящей, камнями огромными, градом ужасным 

\vs 3Sb 1:692 И умерщвленьем повсюду животных четвероногих. 

\vs 3Sb 1:693 Ясно люди поймут, что суд Безсмертного Бога 

\vs 3Sb 1:694 К ним пришел, и тогда умирающих стоны и вопли

\vs 3Sb 1:695 Землю всю огласят, и, лишаясь речи от страха,

\vs 3Sb 1:696 Кровью умывшись своей, погибнут. И почва впитает 

\vs 3Sb 1:697 Кровь, а тела мертвецов разорвут ненасытные звери.

\vs 3Sb 1:698 Все это мне повелел Господь великий и вечный 

\vs 3Sb 1:699 Так предсказать. И не может реченное мною не сбыться 

\vs 3Sb 1:700 Иль не исполниться в чем-то, ведь все задумано Богом  

\vs 3Sb 1:701 Чуждый обмана, витает Господний Дух в поднебесной.

\vs 3Sb 1:702 Дети великого Бога вокруг Святыни в покое 

\vs 3Sb 1:703 Будут жизнь проводить, Господним деяниям рады, 

\vs 3Sb 1:704 Ибо Творец, справедливый Судья и Владыка, дарует

\vs 3Sb 1:705 Многое: Он на защиту народа встанет, могучий, 

\vs 3Sb 1:706 Словно высокую стену, огонь кругом воздвигая. 

\vs 3Sb 1:707 Ни в городах, ни в селеньях война грозить им не будет; 

\vs 3Sb 1:708 Злую руку вражды отразит святая десница  

\vs 3Sb 1:709 Ведь Безсмертный Боец обороной им станет надежной.

\vs 3Sb 1:710 Скажут все города и все острова, что Всевышний 

\vs 3Sb 1:711 Этих мужей возлюбил великой любовью, и будут 

\vs 3Sb 1:712 И небосвод, и луна, и солнце, водимое Богом, 

\vs 3Sb 1:713 Им помогать во всем и печься о них неустанно. 

\vs 3Sb 1:714 И сотрясется в те дни земля, что все порождает.

\vs 3Sb 1:715 И людские уста воспоют сладкогласные гимны:

\vs 3Sb 1:716 Все к земле припадем, обратимся с молитвою к Богу! 

\vs 3Sb 1:717 Он  Безсмертный Царь, Владыка великий и вечный, 

\vs 3Sb 1:718 Он  наш единый Господь, пошлем же к Господнему храму, 

\vs 3Sb 1:719 Будем все вместе внимать законам Всевышнего Бога,

\vs 3Sb 1:720 Ибо нет ничего справедливей, чем эти законы. 

\vs 3Sb 1:721 Мы заблудились, свернув с дороги, указанной Богом, 

\vs 3Sb 1:722 Стали творения рук человеческих чтить неразумно, 

\vs 3Sb 1:723 Статуям смертных людей деревянным кланяться стали. 

\vs 3Sb 1:724 Веру обретшие души такие крики исторгнут.

\vs 3Sb 1:725 Люди Господа, все падем устами на землю,

\vs 3Sb 1:726 В каждом доме Творцу воспоем прекрасные гимны! 

\vs 3Sb 1:727 Все оружие вражье по миру всему соберем мы. 

\vs 3Sb 1:728 Смогут семь лет совершить свое обращенье по кругу 

\vs 3Sb 1:729 Копья, шлемы, щиты и множество разных доспехов,

\vs 3Sb 1:730 Луки и стрелы. Все это уйдет из рук нечестивцев, 

\vs 3Sb 1:731 Чтобы разжечь костры, железом древес не коснутся.

\vs 3Sb 1:732 Ты же в гордыне своей перестань возноситься, Эллада! 

\vs 3Sb 1:733 Жалкая, остерегись, моли милосердного Бога,

\vs 3Sb 1:734 Свой неразумный народ войной не веди в этот город  

\vs 3Sb 1:735 Пусть спокойно живут и земле великого Бога.

\vs 3Sb 1:736 А Камарину не трогай  ей лучше быть неподвижной, 

\vs 3Sb 1:737 И не буди леопарда, не то может зло приключиться. 

\vs 3Sb 1:738 Будь же воздержна, и пусть в груди твоей не проснется 

\vs 3Sb 1:739 Гордый дух и надменный, стремящийся в жаркую битву. 

\vs 3Sb 1:740 Господу верно служи и радости станешь причастна: 

\vs 3Sb 1:741 Ибо, как срок совершится, отмеренный точно судьбою,

\vs 3Sb 1:742 Суд Безсмертного Бога в тот день настанет для смертных.

\vs 3Sb 1:743 Власть Господня в тот день на добрых людей обратится. 

\vs 3Sb 1:744 Плод наилучший земля, которая все порождает, 

\vs 3Sb 1:745 Смертным даст  изобилье пшеницы, вина и оливок. 

\vs 3Sb 1:746 Множество сладкого меда пошлют небеса человеку, 

\vs 3Sb 1:747 Будет древесных плодов в достатке и тучной скотины: 

\vs 3Sb 1:748 Коз, и коров, и овец с ягнятами малыми вместе. 

\vs 3Sb 1:749 Вырвутся из-под земли молока белоснежного струи.

\vs 3Sb 1:750 Будут полны богатств города, а поля плодоносны;

\vs 3Sb 1:751 Шум боевой и резня ужасная вовсе исчезнут,

\vs 3Sb 1:752 С тяжким стоном земля уж больше не содрогнется,

\vs 3Sb 1:753 Войны и засуха миру угрозою быть перестанут,

\vs 3Sb 1:754 С ними же голод и град, что бьет урожай, упразднятся.

\vs 3Sb 1:755 Мир на землю сойдет великий, неведомый прежде, 

\vs 3Sb 1:756 Станут друзьями теперь цари до скончания века, 

\vs 3Sb 1:757 Люди по всей земле одним жить будут законом, 

\vs 3Sb 1:758 Что установит Господь, на небе правящий звездном; 

\vs 3Sb 1:759 Этим законом Безсмертный дела людские измерит,

\vs 3Sb 1:760 Ибо единый Он Бог, и быть другого не может, 

\vs 3Sb 1:761 И сожжет Он огнем человеческий род нечестивый.

\vs 3Sb 1:762 Так поспешите же, люди, слова мои сердцем усвоить: 

\vs 3Sb 1:763 Идолов мерзких оставьте, живому Богу служите; 

\vs 3Sb 1:764 Остерегайтеся блуда и грязного ложа мужского, 

\vs 3Sb 1:765 Если дети родятся, растите их, не убивайте  

\vs 3Sb 1:766 Все прегрешения эти влекут Божий гнев за собою.

\vs 3Sb 1:767 Бог ниспошлет наконец всем людям вечное царство: 

\vs 3Sb 1:768 Дав священный закон Его почитавшим как должно, 

\vs 3Sb 1:769 Пообещал Он, что мир и землю для них Он откроет 

\vs 3Sb 1:770 И распахнет им врата блаженства  великая радость, 

\vs 3Sb 1:771 Вечно здравый рассудок и мысли светлые станут 

\vs 3Sb 1:772 Их достояньем. Тогда к жилищу великого Бога

\vs 3Sb 1:773 С целого света дары принесут и ладан воскурят. 

\vs 3Sb 1:774 Люди не спросят уже к другому дому дороги,

\vs 3Sb 1:775 Кроме того, что велел Господь почитать Своим верным. 

\vs 3Sb 1:776 Сыном великого Бога те люди его называют.

\vs 3Sb 1:777 Все пути по равнинам и все обрывы крутые, 

\vs 3Sb 1:778 Горные выси и волны, что на море дико бушуют,  

\vs 3Sb 1:779 Станет все в эти дни легко человеку доступным.

\vs 3Sb 1:780 Ибо у добрых людей покой и мир воцарятся, 

\vs 3Sb 1:781 Меч упразднят пророки великого Бога, и сами 

\vs 3Sb 1:782 Смертных станут они судить и царить справедливо. 

\vs 3Sb 1:783 Праведным станет тогда и все богатство людское. 

\vs 3Sb 1:784 Вот каковы будут суд и власть Всевышнего Бога.

\vs 3Sb 1:785 Возвеселись и ликуй, о дева! Вечную радость 

\vs 3Sb 1:786 Тот даровал тебе, Кто создал небо и землю. 

\vs 3Sb 1:787 Он, в тебе поселившись, твоим станет светом безсмертным. 

\vs 3Sb 1:788 Овцы вместе с волками в горах травою питаться 

\vs 3Sb 1:789 Будут, а дикие барсы  пастись с козлятами вместе.

\vs 3Sb 1:790 Сможет теленок с медведем в загоне быть безопасно, 

\vs 3Sb 1:791 Лев плотоядный мякиной, как вол, насытится в яслях; 

\vs 3Sb 1:792 Малые дети его, связав, поведут за собою, 

\vs 3Sb 1:793 Зверь этот станет ручным по воле великого Бога. 

\vs 3Sb 1:794 Змей с младенцем уснет в одной постели спокойно

\vs 3Sb 1:795 И не сделает зла  Господня рука не позволит.

\vs 3Sb 1:796 Ясное знаменье я укажу тебе, чтобы узнал ты

\vs 3Sb 1:797 Время, когда конец всему земному настанет.

\vs 3Sb 1:798 Звездный свод озарять мечи огромные будут 

\vs 3Sb 1:799 В небе встанут они с Востока и с Запада ночью.

\vs 3Sb 1:800 Пепел внезапно и пыль посыплются сверху на землю 

\vs 3Sb 1:801 В целом мире, и днем угаснет солнца сиянье, 

\vs 3Sb 1:802 Вместо того луна в небесах появится тотчас, 

\vs 3Sb 1:803 Бледным лучом своим осветив земную поверхность. 

\vs 3Sb 1:804 Кровь, просочившись из камня, вам тоже даст несомненный

\vs 3Sb 1:805 Знак, а в тумане предстанет сражение пеших и конных, 

\vs 3Sb 1:806 Схожее с травлей зверей и во мгле виденью подобно. 

\vs 3Sb 1:807 Значит, вскоре Господь, живущий в небе, положит 

\vs 3Sb 1:808 Войнам конец. Но каждый пусть жертвы Богу приносит.

\vs 3Sb 1:809 Из Ассирийской земли, от мощных стен Вавилона 

\vs 3Sb 1:810 Я в Элладу явилась, как некий огонь, в исступленье,

\vs 3Sb 1:811 Чтобы всем смертным изречь, чем Бог угрожает во гневе \ldots\

\vs 3Sb 1:812 Смертным все предскажу в загадках, внушенных мне Богом.

\vs 3Sb 1:813 Станут тогда говорить в Элладе, что я  чужеземка, 

\vs 3Sb 1:814 Что родилась я в Эритрах, безстыдная; и нарекут мне

\vs 3Sb 1:815 Имя Сивиллы и скажут, как будто Гностом и Киркой 

\vs 3Sb 1:816 Мать и отца моих звали, а я  безумная лгунья. 

\vs 3Sb 1:817 Но когда все свершится, слова мои вспомните сразу 

\vs 3Sb 1:818 И не безумной сочтете  великой пророчицей Бога. 

\vs 3Sb 1:819 Мне Господь не открыл того, что родителям прежде

\vs 3Sb 1:820 Он поведал моим, но то, что было вначале, 

\vs 3Sb 1:821 И грядущее все вложил мне в душу Всевышний, 

\vs 3Sb 1:822 Чтобы пророчить могла я о бывшем и будущем людям. 

\vs 3Sb 1:823 Ибо, покрыли когда всю землю воды Потопа, 

\vs 3Sb 1:824 Славный муж лишь один в то время спасся от смерти:

\vs 3Sb 1:825 Дом деревянный построив, по водам проплыл и с собою 

\vs 3Sb 1:826 Взял он птиц и зверей, чтобы мир наполнился снова. 

\vs 3Sb 1:827 Мне же стать довелось невесткой этого мужа; 

\vs 3Sb 1:828 Первое с ним совершилось, ему же последнее ясным 

\vs 3Sb 1:829 Сделал Господь  и во всем уста мои будут правдивы.

\bibbookdescr{4Sb}{
  inline={Четвёртая книга Сивилл},
  toc={4-я Сивилл},
  bookmark={4-я Сивилл},
  header={4-я Сивилл},
  abbr={4~Сив}
}
\vs 4Sb 1:1 Слушай, Азийский народ надменный и Европейцы, 

\vs 4Sb 1:2 Все, что намерена я правдиво вам напророчить, 

\vs 4Sb 1:3 Мощные звуки издав из широкоотверстого горла! 

\vs 4Sb 1:4 И не от лживого Феба, которого глупые люди

\vs 4Sb 1:5 Богом назвали, ему приписав, что будто пророк он, 

\vs 4Sb 1:6 Стану вещать, но послушна желанию вечного Бога, 

\vs 4Sb 1:7 Руки кого не слепили людские, подобно тому как 

\vs 4Sb 1:8 Идолов лепят немых и из камня их высекают. 

\vs 4Sb 1:9 В камень не заключен, безмолвно в храме стоящий  

\vs 4Sb 1:10 Глух ко всему, позор жалчайший для рода людского, 

\vs 4Sb 1:11 Бог не видим с земли, глазами Его не измерить,

\vs 4Sb 1:12 Теми, что смертным даны, рукою смертной не создан, 

\vs 4Sb 1:13 Разом всех видя с небес, никем быть увиден не может. 

\vs 4Sb 1:14 Ночь, несущую мрак, светлый день и яркое солнце, 

\vs 4Sb 1:15 Звезды вместе с луной, кишащее рыбами море,

\vs 4Sb 1:16 Землю, реки и устья источников вечнотекущих  

\vs 4Sb 1:17 Все сотворил Он для жизни; дожди, что прольются над пашней,

\vs 4Sb 1:18 Ей обещав урожай, дав деревья, лозу и оливу. 

\vs 4Sb 1:19 Он меня в грудь поразил, бичом полоснул мне по сердцу, 

\vs 4Sb 1:20 Чтобы я людям про то, что есть теперь и что будет

\vs 4Sb 1:21 С ними, от первого рода начав и окончив десятым, 

\vs 4Sb 1:22 Все достоверно сказала. Ведь Тот мне это доверил, 

\vs 4Sb 1:23 Сам Кто причина всему. Народ, послушай Сивиллу, 

\vs 4Sb 1:24 Льющую голос правдивый из уст, благочестия полных!

\vs 4Sb 1:25 Те из людей на земле изведают счастье, что Бога 

\vs 4Sb 1:26 Будут великого чтить и Его прославлять непрестанно, 

\vs 4Sb 1:27 Прежде еды и питья стремясь к благочестию в жизни. 

\vs 4Sb 1:28 Храмы отвергнут они все сразу, лишь только увидят;

\vs 4Sb 1:29 То же и алтари  постройки из мертвого камня 

\vs 4Sb 1:30 Лживых во славу богов, оскверненные кровью животных, 

\vs 4Sb 1:31 Жертвенным дымом. Их люди во имя единого Бога 

\vs 4Sb 1:32 Все забросают камнями, запрет положив на убийство, 

\vs 4Sb 1:33 Тайную мзду не приняв, что стало бы худшим началом. 

\vs 4Sb 1:34 Также не будут искать утех на чужом они ложе, 

\vs 4Sb 1:35 Дерзость мужей им чужда и всегда ненавистна пребудет.

\vs 4Sb 1:36 Не переймут никогда тот характер, нрав и обычай 

\vs 4Sb 1:37 Прочие люди. Они, во всем тяготея к безстыдству, 

\vs 4Sb 1:38 Горло в насмешках сорвав, над этими станут смеяться  

\vs 4Sb 1:39 Глупые дети!  и первым наивно приписывать станут 

\vs 4Sb 1:40 То беззаконье и зло, которое сами свершили.

\vs 4Sb 1:41 Полон неверия род людской. Когда же наступит

\vs 4Sb 1:42 Суд над людьми и всем миром, что Сам вселенной Создатель 

\vs 4Sb 1:43 Будет вершить, на него нечестивых и праведных вместе 

\vs 4Sb 1:44 Вызвав,  Он их разведет: порочных в пламень отправит, 

\vs 4Sb 1:45 В мрак преисподней, чтоб там осознали они, что творили;

\vs 4Sb 1:46 Праведным выпадет жить на равнине, обильной плодами, 

\vs 4Sb 1:47 Вместе с дыханьем Господь им жизнь и радость дарует. 

\vs 4Sb 1:48 Это случиться должно при людях в десятом колене, 

\vs 4Sb 1:49 Что же в первом их ждет и дальше  о том расскажу я.

\vs 4Sb 1:50 Править всеми людьми Ассирийцы будут вначале, 

\vs 4Sb 1:51 Власть над миром держа в пределах шести поколений,  

\vs 4Sb 1:52 После чего Божий гнев на них обрушится с неба, 

\vs 4Sb 1:53 На города и людей, живущих под тем небосводом. 

\vs 4Sb 1:54 Морем тут станет земля из-за вод, что внезапно нахлынут.

\vs 4Sb 1:55 Свергнут Мидийцы их власть и сами возсядут на троны  

\vs 4Sb 1:56 Два поколенья всего будут править. При них совершится 

\vs 4Sb 1:57 Вот что: наступит вдруг ночь среди дня и землю накроет, 

\vs 4Sb 1:58 Звезды с небес пропадут, и Луны круг тоже исчезнет; 

\vs 4Sb 1:59 Почва, вся сотрясаясь от мощных подземных ударов, 

\vs 4Sb 1:60 Много сметет городов и того, что построили люди,  

\vs 4Sb 1:61 Из глубины же морской острова всплывут на поверхность.

\vs 4Sb 1:62 Но когда разольется Евфрат великий от крови, 

\vs 4Sb 1:63 Страшная битва случится тут между Мидийцев и Персов 

\vs 4Sb 1:64 В их друг с другом войне. Под копьями Персов Мидийцы,

\vs 4Sb 1:65 Падая, прочь побегут через воды великого Тигра. 

\vs 4Sb 1:66 Сила же Персов пускай величайшею в мире пребудет, 

\vs 4Sb 1:67 Им предстоит лишь одно поколение счастливо править.

\vs 4Sb 1:68 Многие беды ждут мир, их осыплют проклятьями люди: 

\vs 4Sb 1:69 Кровопролитные войны, убийства, изгнания, распри, 

\vs 4Sb 1:70 Гибель больших городов, падение башен высоких  

\vs 4Sb 1:71 Эллин надменный когда по соленой волне Геллеспонта, 

\vs 4Sb 1:72 Смерть неся Финикийцам и Азии, путь свой направит.

\vs 4Sb 1:73 В хлебном Египте, где вся земля распахана плугом, 

\vs 4Sb 1:74 Голод и неурожай на двадцать лет воцарятся. 

\vs 4Sb 1:75 Нил тому будет причиной, колосьям жизнь приносящий,  

\vs 4Sb 1:76 Темные воды свои упрячет он где-то под землю.

\vs 4Sb 1:77 Будет из Азии царь, что копье большое поднимет, 

\vs 4Sb 1:78 На несметных судах. По влажным дорогам пучины 

\vs 4Sb 1:79 Шагом пройдет, проплывет, разсекши высокую гору. 

\vs 4Sb 1:80 После же бегства с войны его грозная Азия примет.

\vs 4Sb 1:81 Остров Сицилия весь сожжен будет мощным потоком 

\vs 4Sb 1:82 Лавы кипящей, из недр что извергнет с пламенем Этна. 

\vs 4Sb 1:83 Город же славный Кротон погрузится в глубокое море.

\vs 4Sb 1:84 Вспыхнет в Элладе вражда. Совсем обезумев от гнева, 

\vs 4Sb 1:85 Много с землей городов сровняют и многих погубят 

\vs 4Sb 1:86 В битве жестокой. Война принесет всем поровну горя.

\vs 4Sb 1:87 В роде когда же людском поколений сменится десять, 

\vs 4Sb 1:88 Персов рабский ярем тогда ожидает и ужас.

\vs 4Sb 1:89 Слава правителей мира когда отойдет к Македонцам,

\vs 4Sb 1:90 Фивам не избежать позорного будет захвата, 

\vs 4Sb 1:91 Тир населят Карийцы, а жители Тира погибнут.

\vs 4Sb 1:92 Самос засыплет песком, сровняет его с берегами.

\vs 4Sb 1:93 Делос исчезнет из глаз, и все, что на Делосе, тоже.

\vs 4Sb 1:94 Грозный на вид Вавилон, однако слабый в сраженье,

\vs 4Sb 1:95 Будет стоять, возведен на надеждах, не могущих сбыться. 

\vs 4Sb 1:96 Бактры займут Македонцы; их жители, город оставив,

\vs 4Sb 1:97 Так же, как жители Суз, все ринутся в землю Эллады.

\vs 4Sb 1:98 Все это в будущем ждет, когда Пирам среброструйный, 

\vs 4Sb 1:99 Воду в залив вынося, священный остров омоет. 

\vs 4Sb 1:100 В море сползут Сибарис и Кизик, от колебаний 

\vs 4Sb 1:101 Почвы; оба падут под напором подземных ударов. 

\vs 4Sb 1:102 Родос последним постигнет несчастье, но будет сильнейшим.

\vs 4Sb 1:103 Власть Македонцев продлится недолго: там, где заходит 

\vs 4Sb 1:104 Солнце, начнется война Италийская, ей подчинятся 

\vs 4Sb 1:105 Все и под рабским ярмом Италийцам прислуживать станут. 

\vs 4Sb 1:106 Ты же, несчастный Коринф, свое разоренье увидишь. 

\vs 4Sb 1:107 Башни твои, Кархедон, к земле преклонят колено.

\vs 4Sb 1:108 Стойкая Лаодикия, тебя опрокинет однажды 

\vs 4Sb 1:109 Землетрясенье, но ты поднимешься вновь из развалин. 

\vs 4Sb 1:110 О Ликийские Миры, краса городов! Никогда вас 

\vs 4Sb 1:111 Прочно земля не удержит, сама сотрясаясь. В паденье 

\vs 4Sb 1:112 Ниже и ниже клонясь, вы другую страну изберете, 

\vs 4Sb 1:113 Чтобы в ней жизнь продолжать  настоящий метек, а не город.

\vs 4Sb 1:114 Из-за нечестья тогда же затихнет и город Патары, 

\vs 4Sb 1:115 Море его поглотит при землетрясенье и буре.

\vs 4Sb 1:116 Также тебе предстоит, Армения, рабская участь.

\vs 4Sb 1:117 Грозной войны ураган домчит до Иерусалима,

\vs 4Sb 1:118 Путь свой начав с Апеннин, и Храм великий разрушит. 

\vs 4Sb 1:119 Тут, безрассудству отдавшись, когда благочестье отринут 

\vs 4Sb 1:120 И в преддверии храма творить будут жуткую бойню,  

\vs 4Sb 1:121 Царь великий тогда из Италии, словно разбойник,

\vs 4Sb 1:122 Пустится в бегство, невидим, неслышим, за воды Евфрата. 

\vs 4Sb 1:123 После того, как он грех величайший  матери гибель  

\vs 4Sb 1:124 Не побоится принять, и другие свершит преступленья. 

\vs 4Sb 1:125 Многие кровью зальют подножие Римского трона 

\vs 4Sb 1:126 Сразу, как тот убежит через земли Парфянского царства.

\vs 4Sb 1:127 В Сирию воин из Рима придет. Он, Иерусалимский 

\vs 4Sb 1:128 Храм предоставив огню и многих убив Иудеев, 

\vs 4Sb 1:129 Их великую землю, дорогами славную, сгубит.

\vs 4Sb 1:130 Пафос и Саламин уничтожит землетрясенье,

\vs 4Sb 1:131 Кипр, омываем полной, когда черная скроет пучина.

\vs 4Sb 1:132 В час, когда, из глубин разверстой земли Италийской 

\vs 4Sb 1:133 Вырвавшись, огненный столб до широкого неба достанет, 

\vs 4Sb 1:134 Много тут городов он сожжет и многих погубит. 

\vs 4Sb 1:135 Тучи горящего пепла весь воздух собою заполнят, 

\vs 4Sb 1:136 С неба частички его будут падать как красная краска.

\vs 4Sb 1:137 В этом увидеть должны явление Божьего гнева, 

\vs 4Sb 1:138 Благочестивое племя поскольку безвинно страдает. 

\vs 4Sb 1:139 Повод для новой войны появится скоро: на Запад 

\vs 4Sb 1:140 Явится тот, кто бежал из Рима; копье он поднимет, 

\vs 4Sb 1:141 Снова Евфрат перейдя, приведет несметное войско.

\vs 4Sb 1:142 Бедная Антиохия! Ты городом зваться не станешь 

\vs 4Sb 1:143 После того, как падешь под копьями по безразсудству. 

\vs 4Sb 1:144 Голод погубит тогда Киприотов и страшная битва.

\vs 4Sb 1:145 Остров несчастный, о Кипр! Увы тебе! Волны морские 

\vs 4Sb 1:146 Скроют тебя под собой  добычу неистовой бури.

\vs 4Sb 1:147 В Азию груз драгоценный прибудет, что некогда Римом 

\vs 4Sb 1:148 Был добыт на войне и в городе этом хранился. 

\vs 4Sb 1:149 Дважды по столько затем придется еще им отправить 

\vs 4Sb 1:150 Азии в качестве платы за все неудачи в сраженьях.

\vs 4Sb 1:151 Карии все города, что лежат по теченью Меандра  

\vs 4Sb 1:152 Стенами окружены, прекрасны,  свирепый погубит 

\vs 4Sb 1:153 Голод, сокроет когда Меандр свою черную воду.

\vs 4Sb 1:154 Стоит в сердцах человечьих изсякнуть почтению к Богу,

\vs 4Sb 1:155 Вере и праву навеки из мира стоит исчезнуть,

\vs 4Sb 1:156 Как, нетвердые духом в дерзаньях своих нечестивых,

\vs 4Sb 1:157 Люди станут вершить произвол и творить злодеянья. 

\vs 4Sb 1:158 С благочестивым никто к беседе стремиться не будет, 

\vs 4Sb 1:159 Их же, напротив, самих истребят глупцы и безумцы, 

\vs 4Sb 1:160 Наглости собственной рады, с руками, покрытыми кровью. 

\vs 4Sb 1:161 Тут им придется узнать, что милостив дольше не будет

\vs 4Sb 1:162 Бог, но, безудержный в гневе, намерен род погубить их  

\vs 4Sb 1:163 Так, чтобы весь он сгорел во время большого пожара.

\vs 4Sb 1:164 Образ мыслей смените, пустые люди, и кару

\vs 4Sb 1:165 Не вынуждайте Его для вас выбирать. Отказавшись

\vs 4Sb 1:166 От мечей и убийств, от стонов и беззаконья, 

\vs 4Sb 1:167 В реках вечнотекущих омойте все свое тело.

\vs 4Sb 1:168 Руки воздев к небесам, к тому, что прежде свершили, 

\vs 4Sb 1:169 О снисхожденье просите и, Богу хвалу воздавая, 

\vs 4Sb 1:170 Милость Его призывайте к себе, нечестивым. Дарует 

\vs 4Sb 1:171 Он прощение всем, не погубит  снова утихнет 

\vs 4Sb 1:172 Гнев, если в душах своих воспитаете вы благочестье. 

\vs 4Sb 1:173 Если ж не верите мне и нечестие вашему сердцу, 

\vs 4Sb 1:174 Глупые люди, дороже всего, а речи  впустую, 

\vs 4Sb 1:175 Пламя охватит тогда весь мир и знак величайший 

\vs 4Sb 1:176 Меч подаст и труба на восходе дневного светила. 

\vs 4Sb 1:177 Глас тот мощный и рев услышат во всей поднебесной. 

\vs 4Sb 1:178 Выжжена будет земля, человеческий род уничтожен, 

\vs 4Sb 1:179 Вместе же с ним города, пресноводные реки и море. 

\vs 4Sb 1:180 Пеплом все станет, и прах раскаленный ляжет повсюду.

\vs 4Sb 1:181 Но когда, кроме золы, ничего уже в мире не будет, 

\vs 4Sb 1:182 Бог успокоит огонь несказанный, как некогда вызвал.

\vs 4Sb 1:183 Пепел и кости людские вновь Сам соберет и придаст им

\vs 4Sb 1:184 Прежнюю форму. Так род Он смертных людей возстановит.

\vs 4Sb 1:185 После того будет Суд, и Сам Он вершить его станет,

\vs 4Sb 1:186 Мир к ответу призвав: тут тех, кто, живя нечестиво, 

\vs 4Sb 1:187 Истинной веры не знал, земляная толща накроет

\vs 4Sb 1:188 Душного Тартара, пропасть поглотит ужасной геенны.

\vs 4Sb 1:189 Людям же праведным вновь разрешит на земле поселиться,

\vs 4Sb 1:190 Вместе с дыханием жизнь Господь им и радость дарует. 

\vs 4Sb 1:191 Все они тотчас себя увидят при благостном свете

\vs 4Sb 1:192 Солнца, которое впредь уходить с небосвода не будет.

\vs 4Sb 1:193 Счастлив тот человек, кому жить в это время придется.

\bibbookdescr{5Sb}{
  inline={Пятая книга Сивилл},
  toc={5-я Сивилл},
  bookmark={5-я Сивилл},
  header={5-я Сивилл},
  abbr={5~Сив}
}
\vs 5Sb 1:1 Слушай, что я расскажу о горестном веке Латинян: 

\vs 5Sb 1:2 Прежде всего, как умрут владыки Египта, которых 

\vs 5Sb 1:3 Всех, одного за другим, земля забрала равнодушно; 

\vs 5Sb 1:4 После рожденного в Пелле, которому под ноги пали

\vs 5Sb 1:5 Все Восточные страны и Запад, безмерно богатый, 

\vs 5Sb 1:6 Кто, посрамлен Вавилоном, был мертвым отправлен к Филиппу,

\vs 5Sb 1:7 Сыном Аммона и Зевса напрасно кого называли; 

\vs 5Sb 1:8 Также и после того, кто плоть и кровь Ассарака, 

\vs 5Sb 1:9 Кто из-под Трои бежал, пройдя сквозь пламени стены;

\vs 5Sb 1:10 После ряда царей  мужей, возлюбивших сраженья, 

\vs 5Sb 1:11 После младенцев, рожденных от зверя, губителя стада, 

\vs 5Sb 1:12 Будет первый властитель, который буквой начальной 

\vs 5Sb 1:13 Увенчает двадцатку. Не знать ему равного в битвах, 

\vs 5Sb 1:14 Имя его начинаться с десятки будет. За этим

\vs 5Sb 1:15 Тот станет править, чей знак  начало всего алфавита. 

\vs 5Sb 1:16 Робость пред ним ощутят Сицилия, Фракия, Мемфис  

\vs 5Sb 1:17 Мемфис, поверженный в прах виною своих полководцев, 

\vs 5Sb 1:18 Из-за упрямства жены, что бросится в волны морские,  

\vs 5Sb 1:19 Он установит народам законы и всех подчинит их. 

\vs 5Sb 1:20 Времени много пройдет  другому власть он оставит, 

\vs 5Sb 1:21 Будет который иметь знак триста на первую букву. 

\vs 5Sb 1:22 Даст ему имя река. Владыкой Персов считаться 

\vs 5Sb 1:23 Станет он и Вавилона, на копья насадит Мидийцев. 

\vs 5Sb 1:24 После же тот примет власть, чьим знаком выпала тройка.

\vs 5Sb 1:25 Царь будет править за ним, которому знак дважды десять. 

\vs 5Sb 1:26 Он доберется до самых окраинных вод Океана 

\vs 5Sb 1:27 И к берегам Авзонийским едва до отлива поспеет. 

\vs 5Sb 1:28 Тот, кому знак пятьдесят назначен судьбой, государем

\vs 5Sb 1:29 Станет, чудовищный змей, войною дышащий тяжкой, 

\vs 5Sb 1:30 Руку который па мать поднимет и смуту посеет,

\vs 5Sb 1:31 Сам нападая, гоня, убивая, творя беззаконье.

\vs 5Sb 1:32 Гору, что между двух волн, разсечет и забрызгает грязью.

\vs 5Sb 1:33 После же смерти исчезнет. Затем вернется обратно,

\vs 5Sb 1:34 С Богом равняясь. Но скоро покажет, что вовсе не Бог он. 

\vs 5Sb 1:35 Трое царей вслед за ним один другого погубят.

\vs 5Sb 1:36 Благочестивых убийца тогда станет править, могучий.

\vs 5Sb 1:37 Семь раз по десять  свой знак  он явит миру. Отнимет

\vs 5Sb 1:38 Власть у него его сын, чьим знаком будут три сотни.

\vs 5Sb 1:39 После судьбой решено быть царю  по знаку четверке. 

\vs 5Sb 1:40 Вслед за этим придет старик, чье число пять десятков.

\vs 5Sb 1:41 Тот же, кто после него воцарится на троне, от века

\vs 5Sb 1:42 Имени первую букву значением триста имеет.

\vs 5Sb 1:43 Горы ногою поправ, спеша на Восточную битву,

\vs 5Sb 1:44 Кельт, он безславную смерть найдет в пути от болезни. 

\vs 5Sb 1:45 Примет мертвого пыль чужеземная; имя Немейский

\vs 5Sb 1:46 Дал ей цветок. А затем  другой, в серебряном шлеме,

\vs 5Sb 1:47 Станет у власти. Ему свое имя море подарит.

\vs 5Sb 1:48 Будет он доблестный муж, с умом, проникающим всюду.

\vs 5Sb 1:49 Так, при тебе, наилучший, что всех превзошел, темнокудрый, 

\vs 5Sb 1:50 И при потомках твоих придет, наконец, это время.

\vs 5Sb 1:51 После него  три царя, из которых третий  не скоро.

\vs 5Sb 1:52 Трижды несчастна, терзаюсь: в груди недоброе слово 

\vs 5Sb 1:53 Давит, Изиды сестра томится пророческой песнью.

\vs 5Sb 1:54 Первым вокруг твоего многослезного храма, Египет,

\vs 5Sb 1:55 Вихрем помчатся менады, и ты попадешь в злые руки

\vs 5Sb 1:56 В тот самый день, когда Нил понесет свои воды однажды

\vs 5Sb 1:57 Через страну Египтян на шестнадцать локтей полноводней,

\vs 5Sb 1:58 Так что омоет всю землю, пустив по ней литься потоки.

\vs 5Sb 1:59 Смолкнет тут радостный смех, чело земли омрачится.

\vs 5Sb 1:60 Мемфис! Ты горше других Египта беды оплачешь,

\vs 5Sb 1:61 Ибо, над всею землей до сих пор величаво царивший,

\vs 5Sb 1:62 Станешь чертогом печали. Тогда призовет тебя с неба

\vs 5Sb 1:63 Зычно сам Повелитель перунов: О, Мемфис могучий,

\vs 5Sb 1:64 Раньше пред жалким народом кичился своею ты славой,

\vs 5Sb 1:65 Ныне в несчастье и скорби заплачешь: придет постиженье

\vs 5Sb 1:66 Вечного Бога к тебе, Безсмертного, сущего в небе. 

\vs 5Sb 1:67 Где теперь воля твоя, что судьбы людские вершила? 

\vs 5Sb 1:68 В диком безумстве детей ты моих, помазанных Богом, 

\vs 5Sb 1:69 Тяжким гоненьям подверг и праведным зло уготовил.

\vs 5Sb 1:70 Будешь за эти дела наказан. Мачеха злая

\vs 5Sb 1:71 Станет уделом твоим, и вовек тебе счастья не видеть: 

\vs 5Sb 1:72 С неба скатившись звездой, обратно не сможешь подняться.

\vs 5Sb 1:73 Вот что Господь мне внушил правдиво Египту поведать 

\vs 5Sb 1:74 В самом исходе времен, когда люди погрязнут в пороках.

\vs 5Sb 1:75 Но продолжают страдать, нечестивцы, в преддверии кары  

\vs 5Sb 1:76 Гнева безсмертного Бога, что тяжко гремит, Небожитель. 

\vs 5Sb 1:77 Вместо Него почитают чудовищ и камни, повсюду 

\vs 5Sb 1:78 Видят священного страха предметы, в которых ни смысла 

\vs 5Sb 1:79 Нет, ни рассудка, ни слуха  о них говорить не пристало

\vs 5Sb 1:80 Мне, называя божков  творения рук человека. 

\vs 5Sb 1:81 Взявшись сами за труд воплотить нечестивые мысли, 

\vs 5Sb 1:82 Люди богов сотворили и каменных, и деревянных, 

\vs 5Sb 1:83 Медных и золотых, серебряных  в коих ни пользы 

\vs 5Sb 1:84 Нет, ни души, глухих, на огне из металла отлитых.

\vs 5Sb 1:85 Сделав же их для себя, впустую на них уповали: 

\vs 5Sb 1:86 Тмуис и Ксуис в беде, конец приходит засилью 

\vs 5Sb 1:87 Зевса, Геракла, Гермеса \ldots

\vs 5Sb 1:88 Славная меть городов, ты тоже, Александрия, 

\vs 5Sb 1:89 Жертвой войны упадешь и то, чем прежде владела,

\vs 5Sb 1:90 Все до остатка отдашь в наказанье за дерзкий характер. 

\vs 5Sb 1:91 Долгим молчание будет, но радостный день возвращенья 

\vs 5Sb 1:92 Больше тебе не нальет напиток нежный \ldots

\vs 5Sb 1:93 Перс наводнит твою землю, подобен жестокому граду, 

\vs 5Sb 1:94 Смерть и разруху неся, людей злонравных погубит.

\vs 5Sb 1:95 Кровью зальет алтари, завалит телами убитых 

\vs 5Sb 1:96 Варвар могучий, свершит он другие безумства, как эти, 

\vs 5Sb 1:97 Словно песчаная буря, замыслив конец твой ускорить. 

\vs 5Sb 1:98 Город счастливый, тогда претерпишь ты многие беды! 

\vs 5Sb 1:99 Вся будет Азия плакать, дары вспоминая, какими

\vs 5Sb 1:100 Голову ты ей венчала  теперь она тоже погибнет. 

\vs 5Sb 1:101 Новый Персидский владыка подвергнет страну разоренью, 

\vs 5Sb 1:102 Всякий им будет убит, и жизнь в тех местах прекратится. 

\vs 5Sb 1:103 Третья лишь часть уцелеет от жалкого племени смертных.

\vs 5Sb 1:104 Он же тут легким прыжком помчится на крыльях к Востоку, 

\vs 5Sb 1:105 Мучая землю войной, в пустыню ее превращая. 

\vs 5Sb 1:106 Власти на гребне своей, хотя и терзаемый страхом, 

\vs 5Sb 1:107 К городу праведных он подойдет, желая разрушить. 

\vs 5Sb 1:108 Посланный Богом тогда некий царь на него ополчится, 

\vs 5Sb 1:109 Что всех великих царей погубит и воинов лучших. 

\vs 5Sb 1:110 Так над людьми приговор исполнит Безсмертный Владыка.

\vs 5Sb 1:111 Гадкое сердце! Зачем ты меня подстрекаешь на это  

\vs 5Sb 1:112 Многих царей предсказать Египту ужасное царство? 

\vs 5Sb 1:113 Лучше вернись на Восток, к потомкам несмысленных

\vs 5Sb 1:114 Персов, Им покажи все как есть, и то, что еще ожидает.

\vs 5Sb 1:115 Воды Евфрата, разлившись, затопят окрестные земли, 

\vs 5Sb 1:116 Персов погубят они, Иберов и Вавилонян, 

\vs 5Sb 1:117 И Массагетов, войну ведущих при помощи луков. 

\vs 5Sb 1:118 Азия до Островов все блеском пожаров осветит. 

\vs 5Sb 1:119 Некогда великолепный, Пергам совсем опустеет.

\vs 5Sb 1:120 Так же, как он, и Питана предстанет безлюдной пустыней. 

\vs 5Sb 1:121 Лесбос опустится весь в пучину бездонную моря. 

\vs 5Sb 1:122 Смирна, с крутых берегов скользнув, заплачет однажды  

\vs 5Sb 1:123 Та, что была столь горда и известна, безславно погибнет. 

\vs 5Sb 1:124 Землю, что стала золой, слезами Вифинцы омоют,

\vs 5Sb 1:125 Сирию всю целиком, многолюдную с ней Финикию. 

\vs 5Sb 1:126 Ликия, горе тебе  столько бед для тебя замышляет 

\vs 5Sb 1:127 Море: однажды само на несчастную землю нахлынув, 

\vs 5Sb 1:128 Скроет в соленых волнах, при страшных подземных ударах 

\vs 5Sb 1:129 Берег Ликийский, где миро растет и где нет его вовсе.

\vs 5Sb 1:130 Гнев на Фригийцев падет ужасный из-за печали, 

\vs 5Sb 1:131 Ради которой пришла сюда Рея и здесь поселилась. 

\vs 5Sb 1:132 Море Таврский народ уничтожит и варваров племя, 

\vs 5Sb 1:133 А Эпидана поток по земле разметает Лапифов, 

\vs 5Sb 1:134 Водовороты крутя, Фессалийскую область погубит.

\vs 5Sb 1:135 Глубоководный Пеней увлечет за собою животных, 

\vs 5Sb 1:136 Тех, что когда-то родил Эпидан, как люди считают.

\vs 5Sb 1:137 Трижды несчастной Эллады поэты участь оплачут, 

\vs 5Sb 1:138 Царь Италийский когда перебьет сухожилие Истма, 

\vs 5Sb 1:139 Богу подобный, могучий, великого Рима властитель.

\vs 5Sb 1:140 Сам его Зевс, говорят, породил и владычица Гера. 

\vs 5Sb 1:141 Кто при стеченье народа поет сладкозвучные гимны

\vs 5Sb 1:142 Голосом нежным, убьет свою мать и многих несчастных.

\vs 5Sb 1:143 Вождь трусливый и наглый, бежит от стен Вавилона 

\vs 5Sb 1:144 Тот, кого среди смертных сильнейшие даже боятся; 

\vs 5Sb 1:145 Многих он жизни лишил, не щадил и матери чрева,

\vs 5Sb 1:146 Грязной любви предавался, вместилище всяких пороков.

\vs 5Sb 1:147 Путь свой к Мидийцам направит и грозным правителям Персов 

\vs 5Sb 1:148 Их он всех раньше призвал и славу им уготовил,

\vs 5Sb 1:149 На неугодный народ замышляя с толпой нечестивцев. 

\vs 5Sb 1:150 Богом поставленный храм захватил он, сжег безпощадно

\vs 5Sb 1:151 Тех, что входили в него, кого я по заслугам воспела.

\vs 5Sb 1:152 В храм он лишь только вступил, как здание все содрогнулось,

\vs 5Sb 1:153 Гибли повсюду цари, а те, кто остался у власти,

\vs 5Sb 1:154 Город великий сгубили с народом праведным вместе.

\vs 5Sb 1:155 В год же четвертый, когда звезда засияет большая, 

\vs 5Sb 1:156 Землю которая всю уничтожит одна ради мести,

\vs 5Sb 1:157 \ldots

\vs 5Sb 1:158 С неба большая звезда упадет в соленые воды, 

\vs 5Sb 1:159 Море она подожжет и с ним Вавилона твердыни, 

\vs 5Sb 1:160 Землю Италии, много виною которой погибло 

\vs 5Sb 1:161 Благочестивых Евреев, угодного Богу народа.

\vs 5Sb 1:162 Между порочных мужей ты себя запятнаешь пороком, 

\vs 5Sb 1:163 Целую вечность потом простоишь совсем опустелым,

\vs 5Sb 1:164 \ldots

\vs 5Sb 1:165 День основанья прокляв, за то, что требовал яда: 

\vs 5Sb 1:166 Ложу измены в тебе, малолетних детей совращенье, 

\vs 5Sb 1:167 Женственный город, дурной, нечестивый и самый несчастный,

\vs 5Sb 1:168 Самый порочный из всех городов земли Италийской, 

\vs 5Sb 1:169 Помесь менады с ехидной, вдовой на холмах ты возляжешь,

\vs 5Sb 1:170 Тибра поток по тебе будет плакать, по милой подруге, 

\vs 5Sb 1:171 В сердце чьем мерзость убийства, а дух отягчен преступленьем,

\vs 5Sb 1:172 Разве не знал ты, что может Господь и что замышляет? 

\vs 5Sb 1:173 Ты говорил: Я один, и никто меня не разрушит!

\vs 5Sb 1:174 Ныне же граждан твоих и тебя вечный Бог уничтожит, 

\vs 5Sb 1:175 Впредь никакое жилье не укажет на то, что здесь было, 

\vs 5Sb 1:176 Как в то время, когда твою славу Господь лишь задумал. 

\vs 5Sb 1:177 Будь же один, безрассудный, и, пламенем жарким охвачен, 

\vs 5Sb 1:178 Рухни в безжалостный мрак забытого Богом Аида.

\vs 5Sb 1:179 Снова теперь о твоем я горюю несчастье, Египет! 

\vs 5Sb 1:180 Мемфис, под гнетом страданий ты первым падешь на колени,

\vs 5Sb 1:181 Даже твои пирамиды ужасные вопли исторгнут.

\vs 5Sb 1:182 Пифон, что некогда прежде Диполисом звался по праву,

\vs 5Sb 1:183 Ты замолчишь навсегда, чтобы впредь не творить злодеяний,

\vs 5Sb 1:184 Город надменный, ларец всевозможных пороков, менадой 

\vs 5Sb 1:185 Жалкой, несчастной вдовой навеки отныне пребудешь 

\vs 5Sb 1:186 Ты, что была рождена править миром долгие годы.

\vs 5Sb 1:187 Но когда на себя кипассий Барка набросит

\vs 5Sb 1:188 Грязного белый поверх, то лучше бы ей не родиться.

\vs 5Sb 1:189 Фивы, великая сила куда ваша делась? Разбойник 

\vs 5Sb 1:190 Сгубит народ ваш, а вы, надев одежды печали,

\vs 5Sb 1:191 В плаче зайдетесь, одни, вину искупая несчастьем 

\vs 5Sb 1:192 Те прегрешенья, что прежде свершили, о, город надменный,

\vs 5Sb 1:193 Мир будет видеть ваш плач  за то, что не чтили закона.

\vs 5Sb 1:194 Царь Эфиопов могучий разрушит город Сиену, 

\vs 5Sb 1:195 Силой Тевхиру населит народ темнокожий Индийцев.

\vs 5Sb 1:196 Слезы, Пентаполь, прольешь: тебя муж многомощный погубит.

\vs 5Sb 1:197 Скорбная Ливия, кто твои беды возьмется исчислить?

\vs 5Sb 1:198 Кто, Кирена, тебя среди смертных достойно оплачет?

\vs 5Sb 1:199 Смолкнут стенанья твои только в час ненавистной кончины.

\vs 5Sb 1:200 В земли Британцев и Галлов, богатых золотом, хлынет 

\vs 5Sb 1:201 Вод Океанских поток, от крови все полноводней. 

\vs 5Sb 1:202 Много ведь горя они доставили детям Господним, 

\vs 5Sb 1:203 В год, когда царь Финикийский в Сидон огромное войско 

\vs 5Sb 1:204 Галлов из Сирии вел. Саму тебя тоже погубит

\vs 5Sb 1:205 Он, Равенна, с собой твоих граждан ведя на убийство.

\vs 5Sb 1:206 Не заноситесь, Индийцы и храбрый народ Эфиопов! 

\vs 5Sb 1:207 Ибо когда колесо небесной оси, Козерога

\vs 5Sb 1:208 Звезды, Телец побегут вкруг центра в созвездии Братьев  

\vs 5Sb 1:209 Дева, на небо взойдя, и Солнце, крутясь непрерывно, 

\vs 5Sb 1:210 Их хоровод поведут по всему небесному своду 

\vs 5Sb 1:211 Будет тут страшный пожар, который охватит всю землю, 

\vs 5Sb 1:212 В битве небесных светил обновится природа, погибнет, 

\vs 5Sb 1:213 Плачем мир огласив, в огне страна Эфиопов!

\vs 5Sb 1:214 Плачь ты тоже, Коринф, над своею судьбою несчастной!

\vs 5Sb 1:215 Мойры когда, три сестры, прядущие нити витые, 

\vs 5Sb 1:216 Вспять беглеца поведут, который тайком с перешейка, 

\vs 5Sb 1:217 Горы минуя, бежал, чтобы снова явить его людям. 

\vs 5Sb 1:218 Кто однажды скалу разсек безудержной медью, 

\vs 5Sb 1:219 Тот сгубит землю твою, разорит, как назначено было,

\vs 5Sb 1:220 Ибо от Бога дана ему сила дерзнуть на такое,

\vs 5Sb 1:221 Что ни один из царей до него не отважился сделать. 

\vs 5Sb 1:222 Прежде всего, отделив от трех голов основанья, 

\vs 5Sb 1:223 Щедро позволит другим голов этих мяса отведать, 

\vs 5Sb 1:224 Так что пожрут они плоть родную царя-нечестивца.

\vs 5Sb 1:225 Прочих людей на земле ожидают убийство и ужас 

\vs 5Sb 1:226 Из-за великого Града и верного Богу народа, 

\vs 5Sb 1:227 Что был спасаем всегда, кого Провиденье избрало.

\vs 5Sb 1:228 Ветреный и безрассудный, в себя все несчастья вобравший, 

\vs 5Sb 1:229 Тяжких страданий исток и их наивысшая степень,

\vs 5Sb 1:230 Город, задержанный в росте, но Мойрами все же спасенный, 

\vs 5Sb 1:231 Дерзкий, зачинщик всех бед, великое горе народам  

\vs 5Sb 1:232 Кто пожелал в тебе жить? Живя в тебе, кто не страдал бы? 

\vs 5Sb 1:233 Кто из царей твоих пал, достойную жизнь завершая? 

\vs 5Sb 1:234 Все ты испортил, что мог, залив всякой мерзостью землю, 

\vs 5Sb 1:235 Мира прекрасные складки тобой изменили свой облик. 

\vs 5Sb 1:236 Может быть,  думаешь ты,  она ищет ссоры со мною? 

\vs 5Sb 1:237 Что за нелепость! Хочу вразумить и вот  упрекаю: 

\vs 5Sb 1:238 Некогда свет возсиял средь людей благодатного Солнца, 

\vs 5Sb 1:239 Лившего те же лучи, что и солнце древних пророков. 

\vs 5Sb 1:240 Мед стекал с языка  напиток сладчайший для смертных; 

\vs 5Sb 1:241 Он обвинял, объяснял  и день на земле продолжался. 

\vs 5Sb 1:242 Из-за того, что был Он  о источник тягчайших пороков!  

\vs 5Sb 1:243 Горе и войны с земли однажды навеки исчезнут. 

\vs 5Sb 1:244 Ты же, начало всех зол и их наивысшая степень, 

\vs 5Sb 1:245 Город, задержанный в росте, но Мойрами все же спасенный, 

\vs 5Sb 1:246 Горькому слову внемли, неприятному, смертных несчастье!

\vs 5Sb 1:247 Люди когда воевать на Персидской земле перестанут,

\vs 5Sb 1:248 Стоны покинут ее и голод, тогда появиться

\vs 5Sb 1:249 Должен в ней будет народ Иудеев блаженных, небесный.

\vs 5Sb 1:250 Он ее среднюю часть вкруг Божьего града заселит, 

\vs 5Sb 1:251 Стену соорудив великую вплоть до Иоппы, 

\vs 5Sb 1:252 Ту, что поднимется ввысь под самые темные тучи. 

\vs 5Sb 1:253 Больше труба никогда не издаст воинственный голос, 

\vs 5Sb 1:254 Люди от вражьей руки перестанут гибнуть в сраженьях

\vs 5Sb 1:255 И установят трофей победе над злом в этом мире.

\vs 5Sb 1:256 Муж на землю с небес сойдет, Кому равных не будет, 

\vs 5Sb 1:257 Руки раскинет Свои на древе, обильном плодами. 

\vs 5Sb 1:258 Лучший среди Иудеев, Он солнца бег остановит 

\vs 5Sb 1:259 Речью прекрасной, что с губ Его безупречных польется.

\vs 5Sb 1:260 Больше не нужно тебе скорбеть душою, блаженный, 

\vs 5Sb 1:261 Богом рожденный цветок, желанный для всех и богатый, 

\vs 5Sb 1:262 Свет благодатный, достойный исход вожделенный, святыня, 

\vs 5Sb 1:263 Город земли Иудейской прекрасный, возвышенный в гимнах! 

\vs 5Sb 1:264 В пляске безумной тебя попирать нечестивой стопою

\vs 5Sb 1:265 Эллины больше не будут, душой исзступленью отдавшись  

\vs 5Sb 1:266 Вместо того окружат почитанием дети Господни  

\vs 5Sb 1:267 Те, что воздвигнут алтарь при звуках священных напевов, 

\vs 5Sb 1:268 Богу многие жертвы неся и молясь непрерывно. 

\vs 5Sb 1:269 Все, кто прежде терпел мучения из-за гонений,

\vs 5Sb 1:270 Радостных дней череду теперь в утешенье получат  

\vs 5Sb 1:271 Те же, кто в небеса нечестиво хулу возносили, 

\vs 5Sb 1:272 Вдруг умолкнут, осыпав друг друга безсмысленной бранью. 

\vs 5Sb 1:273 Скроет в себе их земля, до тех пор пока мир существует. 

\vs 5Sb 1:274 Тут прольется из туч пылающий огненный ливень,

\vs 5Sb 1:275 С пашен отныне собрать не придется блестящих колосьев  

\vs 5Sb 1:276 Все незасеянным впредь и невспаханным будет, доколе 

\vs 5Sb 1:277 Власть не признают над миром Безсмертного, вечного Бога 

\vs 5Sb 1:278 Смертные люди и чтить не забудут земли порожденья  

\vs 5Sb 1:279 Коршунов, также собак, которых дал миру Египет,

\vs 5Sb 1:280 Суетно превозносить, утруждая глупые губы. 

\vs 5Sb 1:281 Родина благочестивых, святая земля принесет им 

\vs 5Sb 1:282 Струи медовые, что из скал и источников льются.

\vs 5Sb 1:283 К чистым душою тогда притечет молоко неземное  

\vs 5Sb 1:284 К тем, что надежды свои на Творца одного возложили, 

\vs 5Sb 1:285 Вышнего Бога, Ему принеся почитанье и веру.

\vs 5Sb 1:286 Ясный мне ум для чего велит поведать такое? 

\vs 5Sb 1:287 Бедная Азия, жалость к тебе мою душу терзает, 

\vs 5Sb 1:288 Скорбь о народе Карийцев, богатых Лидийцев, Ионян. 

\vs 5Sb 1:289 Сарды, увы вам! И вам увы, сердцу милые Траллы! 

\vs 5Sb 1:290 Лаодикия, увы! прекраснейший город  погубит 

\vs 5Sb 1:291 Землетрясение вас и в прах обратит ваши стены.

\vs 5Sb 1:292 В скорбной Азийской земле, в стране богатых Лидийцев 

\vs 5Sb 1:293 Храм Артемиды Эфесской падет под ударами бури; 

\vs 5Sb 1:294 Трещины в почве, толчки  и с берега в море сползет он. 

\vs 5Sb 1:295 Так заливают корабль в непогоду свирепые волны. 

\vs 5Sb 1:296 Навзничь упав, тут Эфес испустит вопль, орошая 

\vs 5Sb 1:297 Берег слезами. Искать будет храм он, что высился прежде.

\vs 5Sb 1:298 Гневом тогда распален, нерушимый небесный Владыка 

\vs 5Sb 1:299 Молнию с силой метнет в преступника из поднебесья 

\vs 5Sb 1:300 Вместо зимы в этот день наступит пора урожая. 

\vs 5Sb 1:301 После того на земле людей ожидают несчастья: 

\vs 5Sb 1:302 В высях Гремящий убьет до единого всех нечестивцев, 

\vs 5Sb 1:303 Громы и молнии в ход пустив, горящие стрелы, 

\vs 5Sb 1:304 Целые тучи врагов  и род истребит их настолько,

\vs 5Sb 1:305 Что мертвых тел на земле будет больше, чем мелких песчинок.

\vs 5Sb 1:306 Смирна тогда же придет своего Ликурга оплакать 

\vs 5Sb 1:307 Под стенами Эфеса и здесь сама же погибнет.

\vs 5Sb 1:308 Глупая Кима с ее священной божественной влагой

\vs 5Sb 1:309 Брошена в руки людей безбожных, неправедных, диких, 

\vs 5Sb 1:310 Впредь возносить в небеса не будет радостных песен,

\vs 5Sb 1:311 Но безжизненным телом в волнах прибрежных качаться.

\vs 5Sb 1:312 Те, кто останутся жить, заплачут в голос от горя.

\vs 5Sb 1:313 Будет им знак  по нему поймут, за что претерпели 

\vs 5Sb 1:314 Кимский злосчастный народ, стыда лишенное племя. 

\vs 5Sb 1:315 После, лишь только они сожженную землю оплачут,

\vs 5Sb 1:316 Лесбос навеки уйдет под воды реки Эридана.

\vs 5Sb 1:317 Горе тебе, Керкира прекрасная! Пляски прервешь ты, 

\vs 5Sb 1:318 И Иераполь, живущий в позорном союзе с богатством! 

\vs 5Sb 1:319 Что пожелал, обретешь, оплаканный многими город  

\vs 5Sb 1:320 Там, где течет Термодонт, засыпан землею ты будешь. 

\vs 5Sb 1:321 Триполь, возросший на скалах близ вод Меандра, который 

\vs 5Sb 1:322 Волнами ночью под берег быть смытым судьбою назначен! 

\vs 5Sb 1:323 До основанья тебя разрушит промысел Божий.

\vs 5Sb 1:324 Пусть не желаю я зла земле, что соседняя Фебу: 

\vs 5Sb 1:325 Пущенный с неба перун роскошный Милет уничтожит 

\vs 5Sb 1:326 Из-за того, что коварным он Феба песням поверил \ldots

\vs 5Sb 1:327 Благоразумный совет и о смертных людях забота.

\vs 5Sb 1:328 Смилуйся, мира Создатель, над тучной землей Иудейской, 

\vs 5Sb 1:329 Щедро несущей плоды, чтоб мы Твои помыслы знали! 

\vs 5Sb 1:330 Ибо Ты первой ее сотворил в Своей милости, Боже, 

\vs 5Sb 1:331 С тем, чтобы даром Твоим она для смертных явилась 

\vs 5Sb 1:332 И могла бы внимать всему, что ей Бог доверяет.

\vs 5Sb 1:333 Трижды несчастная, жажду я видеть творенья Фракийцев,

\vs 5Sb 1:334 Стену промеж двух морей, что вихрем несущейся пыли

\vs 5Sb 1:335 Совлечена, как поток в глубину устремится, где рыбы.

\vs 5Sb 1:336 О Геллеспонт разнесчастный! Тебя запряжет Ассириец, 

\vs 5Sb 1:337 Битва Фракийцев великую силу разрушит. 

\vs 5Sb 1:338 С войском Египетский царь Македонии земли захватит, 

\vs 5Sb 1:339 Варваров область низложит могущество власть предержащих. 

\vs 5Sb 1:340 Там Памфилийцы, Галаты, Лидийцы и Писидийцы 

\vs 5Sb 1:341 Вместе одержат победу, на грозную битву собравшись.

\vs 5Sb 1:342 Трижды несчастная, ляжешь, Италия, мертвой пустыней, 

\vs 5Sb 1:343 Змей доколе в твоей цветущей земле не издохнет.

\vs 5Sb 1:344 В высях заоблачных, в небе широком однажды раздастся 

\vs 5Sb 1:345 Грома раскат, призывая прислушаться к голосу Бога. 

\vs 5Sb 1:346 Больше не явятся миру лучи нетленные солнца, 

\vs 5Sb 1:347 Также сияющий свет луны навеки угаснет  

\vs 5Sb 1:348 В самом исходе времен, когда Божья исполнится воля.

\vs 5Sb 1:349 Тьма тут окутает мир, и мрак по земле расползется, 

\vs 5Sb 1:350 Страшные звери на ней, ослепшие люди и горе.

\vs 5Sb 1:351 Долго продлится тот день, и смертные Бога узнают 

\vs 5Sb 1:352 Сущего на небесах Владыку, чье око всезряще.

\vs 5Sb 1:353 Не пожалеет тогда Он врагов Своих, но уничтожит 

\vs 5Sb 1:354 Тех, что баранов, овец, быков стада и мычащих 

\vs 5Sb 1:355 Телок золоторогих и тучных в жертву приносят

\vs 5Sb 1:356 Гермам бездушным, камням, из которых сделаны боги.

\vs 5Sb 1:357 Мудрый пусть торжествует закон и праведных слава!

\vs 5Sb 1:358 Чтобы нетленный Господь, разгневавшись, смерти не предал

\vs 5Sb 1:359 Весь человеческий род нечестивый, безстыдное племя, 

\vs 5Sb 1:360 Нужно Создателя чтить  Безсмертного Вечного Бога.

\vs 5Sb 1:361 В самом исходе времен, лунный свет когда потускнеет, 

\vs 5Sb 1:362 Мир безумство войны охватит, коварной и подлой.

\vs 5Sb 1:363 С края земли человек придет, что на мать покусился, 

\vs 5Sb 1:364 Бегством спасаясь и в сердце своем замышляя дурное. 

\vs 5Sb 1:365 Он всю землю захватит, и все ему станет подвластно,

\vs 5Sb 1:366 В самые тайные мысли людей он свободно проникнет,

\vs 5Sb 1:367 Из-за которой умрет, саму он, вернувшись, погубит.

\vs 5Sb 1:368 Многих мужей истребит, в том числе и великих тиранов, 

\vs 5Sb 1:369 Всех он огнем будет жечь, что никто доселе не делал, 

\vs 5Sb 1:370 Павших снова подняться, ревнуя к Богу, заставит.

\vs 5Sb 1:371 С Запада будет война грозить великая людям, 

\vs 5Sb 1:372 Крови потоки стекут с берегов в полноводные реки, 

\vs 5Sb 1:373 Желчь будет капать по капле в долинах земли Македонской \ldots

\vs 5Sb 1:374 Помощь с Заката придет, придет и смерть властелину. 

\vs 5Sb 1:375 И вот тогда по земле подует ветер холодный,

\vs 5Sb 1:376 Снова жестокой войной наполнится поле сражений.

\vs 5Sb 1:377 С неба на смертных людей прольется огненный ливень,

\vs 5Sb 1:378 Пламя, кровь и вода, блеск молний, тьма и мрак ночи.

\vs 5Sb 1:379 В битве настигшая смерть, резня под покровом тумана 

\vs 5Sb 1:380 Всех уничтожат царей  а с ними воинов лучших.

\vs 5Sb 1:381 Так прекратится война, и стихнет жуткая бойня.

\vs 5Sb 1:382 Больше никто за мечи и железо рукой не возьмется,

\vs 5Sb 1:383 Копий не тронет никто, что будут теперь под запретом.

\vs 5Sb 1:384 Мир тут получит народ разумный, в живых кто остался, 

\vs 5Sb 1:385 Выдержав пробу войной, чтоб радость вкушать беззаботно.

\vs 5Sb 1:386 Мать кто убил, откажитесь от дерзкой преступной отваги! 

\vs 5Sb 1:387 Те, кто на ложе свое нечестиво детей возводили, 

\vs 5Sb 1:388 И превращали в блудниц под кровом своим непорочных 

\vs 5Sb 1:389 Силой и страхом расправы, разнузданным, наглым безстыдством \ldots

\vs 5Sb 1:390 Мать с порожденьем своим смешалась в тебе беззаконно, 

\vs 5Sb 1:391 Дочь с породившим ее позорный союз заключала, 

\vs 5Sb 1:392 Пачкали в стенах твоих цари покорные губы, 

\vs 5Sb 1:393 Ложе делить со скотом искали в тебе нечестивцы. 

\vs 5Sb 1:394 Смолкни же, мерзостный город, жалчайший, средь праздников шумных,

\vs 5Sb 1:395 Ибо уже никогда горящей легко древесины 

\vs 5Sb 1:396 Чистые девы огонь священный в тебе не увидят. 

\vs 5Sb 1:397 Дом, извечно желанный, с тобою погас, когда снова 

\vs 5Sb 1:398 Видеть мне довелось, как падает он под ударом, 

\vs 5Sb 1:399 Весь охвачен огнем, сраженный рукой нечестивой 

\vs 5Sb 1:400 Вечно цветущий предел, хранящее Бога жилище. 

\vs 5Sb 1:401 Храм, что святыми построен и будет стоять нерушимо, 

\vs 5Sb 1:402 Тот, кому телом и духом поверили смертные люди.

\vs 5Sb 1:403 Он не начал бездумно заморскому богу молиться 

\vs 5Sb 1:404 И его из камней высекать, премудрый строитель.

\vs 5Sb 1:405 Также и золота блеск не чтил он  для душ обольщенье: 

\vs 5Sb 1:406 Богу, вдохнувшему жизнь в тела, Создателю мира 

\vs 5Sb 1:407 Издавна в жертву они овец и быков приносили. 

\vs 5Sb 1:408 Ныне же царь, что пришел невидимым, страшный преступник, 

\vs 5Sb 1:409 Всю их страну разорил и лежать в запустенье оставил,

\vs 5Sb 1:410 С войском явившись большим, с мужами, отважными духом.

\vs 5Sb 1:411 Сам он, на землю вступив безсмертную, жизни лишился. 

\vs 5Sb 1:412 Больше явлено людям такого не было знака, 

\vs 5Sb 1:413 Что и другие придут великий город разрушить.

\vs 5Sb 1:414 Муж с высоких небес сошел блаженный на землю, 

\vs 5Sb 1:415 Руки скиптр держали, что Бог ему вечный доверил. 

\vs 5Sb 1:416 Мощью он всех превзошел и тем справедливо богатство 

\vs 5Sb 1:417 Роздал, кто праведно жил  а прежние лишь отбирали. 

\vs 5Sb 1:418 Все он сжигал города и до основания рушил,

\vs 5Sb 1:419 Жег жилища людей, творивших когда-то злодейства.

\vs 5Sb 1:420 Город же, избранный Богом, блестеть заставил он ярко  

\vs 5Sb 1:421 Ярче сияющих звезд на небе и солнца с луною  

\vs 5Sb 1:422 Пышно украсил, и храм в нем Богу священный поставил, 

\vs 5Sb 1:423 В камень одетый, прекрасный, каких не бывало доселе. 

\vs 5Sb 1:424 Стену построил вокруг на много стадиев, в небо

\vs 5Sb 1:425 Что уходила и туч касалась, видна отовсюду  

\vs 5Sb 1:426 Так что могли созерцать все люди праведной веры 

\vs 5Sb 1:427 Славу Безсмертного Бога, давно желанное чудо. 

\vs 5Sb 1:428 Солнца восход и закат пропели Ему свои гимны. 

\vs 5Sb 1:429 Злу среди рода людского отныне места не будет:

\vs 5Sb 1:430 В браке изменам, с детьми не дозволенным Богом сношеньям,

\vs 5Sb 1:431 Смертоубийству, вражде  лишь законному единоборству. 

\vs 5Sb 1:432 Праведных время придет в конце, тогда и исполнит 

\vs 5Sb 1:433 Все это Бог-Громовержец, Строитель великого храма.

\vs 5Sb 1:434 Горе тебе, Вавилон, златотронный и златообутый, 

\vs 5Sb 1:435 Древний царский чертог, один управляющий миром!

\vs 5Sb 1:436 Тот, что некогда был великим и властным  ты больше,

\vs 5Sb 1:437 Город, в горах золотых у вод Евфрата не ляжешь.

\vs 5Sb 1:438 Но по земле распростершись в смятенье подземных ударов,

\vs 5Sb 1:439 Лишь под властью Парфян всем миром потом овладеешь. 

\vs 5Sb 1:440 Попридержи свой язык, нечестивый потомок Халдеев!

\vs 5Sb 1:441 Слов понапрасну не трать на то, как Персами править

\vs 5Sb 1:442 Станешь, Мидийцами как: ведь и власть, что имел, получил ты,

\vs 5Sb 1:443 Риму заложника дав и Азийских наемников выслав.

\vs 5Sb 1:444 Вот потому-то пойдешь, расчетливый царь, ты в Афины 

\vs 5Sb 1:445 Для выяснения цели: зачем посылал, дескать, выкуп.

\vs 5Sb 1:446 Вместо неискренних слов врагам свой гнев ты покажешь.

\vs 5Sb 1:447 В самом исходе времен однажды высохнет море, 

\vs 5Sb 1:448 Так что не смогут приплыть корабли к берегам Италийским. 

\vs 5Sb 1:449 Азия станет тогда, напротив, водой животворной, 

\vs 5Sb 1:450 То же и Крит. Много бед испытать тут придется и Кипру: 

\vs 5Sb 1:451 Пафос оплакивать будет печальный свой жребий, узнают 

\vs 5Sb 1:452 Все о судьбе Саламина, который постигло несчастье. 

\vs 5Sb 1:453 Больше плодов приносить не станет земля побережья, 

\vs 5Sb 1:454 Мощный набег саранчи погубит страну Киприотов.

\vs 5Sb 1:455 Будете плакать вы, глядя ни Тир, злополучные люди! 

\vs 5Sb 1:456 Гнев тебя ждет, Финикия, ужасный, доколе не рухнешь 

\vs 5Sb 1:457 Тяжко на землю  могли чтобы искренне плакать Сирены.

\vs 5Sb 1:458 В пятом колене людском, когда беды Египта отступят, 

\vs 5Sb 1:459 И цари Египтян друг с другом безстыдно мешаться 

\vs 5Sb 1:460 Станут, в Египте взойдут на трон Памфилийцев потомки. 

\vs 5Sb 1:461 У Македонцев тогда, и в Азии, и у Ликийцев 

\vs 5Sb 1:462 Ужас кровавой войны покроет все пылью и прахом. 

\vs 5Sb 1:463 Римский прервет его царь с владыками Запада вместе.

\vs 5Sb 1:464 Только лишь ветер холодный и снег приносящий подует,

\vs 5Sb 1:465 Только покроются льдом большая река и озера  

\vs 5Sb 1:466 Варварский тотчас народ устремится в Азийскую землю, 

\vs 5Sb 1:467 Словно безсильных, погубит он грозное племя Фракийцев. 

\vs 5Sb 1:468 Будут отцов тут своих поедать несчастные люди, 

\vs 5Sb 1:469 Мучимы голодом, в пищу себе их мясо готовить. 

\vs 5Sb 1:470 Звери же будут кормиться, беря из людского жилища  

\vs 5Sb 1:471 С птицами вместе, они всех смертных людей уничтожат. 

\vs 5Sb 1:472 В ходе жестокой войны прибудет воды в Океане, 

\vs 5Sb 1:473 Примет кровавый он цвет от тел и крови безумцев. 

\vs 5Sb 1:474 В это же время земля настолько уже истощится, 

\vs 5Sb 1:475 Что можно будет в уме мужчин перечислить и женщин.

\vs 5Sb 1:476 Жалкое племя в конце испустит страшные вопли, 

\vs 5Sb 1:477 В час, когда солнце зайдет, чтобы больше уже не подняться, 

\vs 5Sb 1:478 Но, в океанской воде оставаясь, очиститься ею  

\vs 5Sb 1:479 Ибо многих людей пришлось ему видеть нечестье. 

\vs 5Sb 1:480 Темная ночь без луны по небу тогда разольется,

\vs 5Sb 1:481 Мгла, какой прежде не знали, окутает складки земные. 

\vs 5Sb 1:482 Снова, однако, дорогу потом свет Божий укажет 

\vs 5Sb 1:483 Праведным людям, что в гимнах поспели Великого Бога.

\vs 5Sb 1:484 Ты, несчастная трижды Изида! У Нильских потоков 

\vs 5Sb 1:485 Сядешь одна, как менада немая у вод Ахеронта, 

\vs 5Sb 1:486 Память сама о тебе скоро жить на земле перестанет. 

\vs 5Sb 1:487 Ты же, Серапис, мученья претерпишь на каменном ложе, 

\vs 5Sb 1:488 В трижды несчастном Египте руиной падешь величайшей. 

\vs 5Sb 1:489 Все, что тянулись к тебе в стране Египтян, будут скоро

\vs 5Sb 1:490 Плакать, а те, кто вложил в свое сердце разум нетленный, 

\vs 5Sb 1:491 Бога кто в гимнах воспел, поймут, что ты вовсе ничтожен.

\vs 5Sb 1:492 Скажет один из жрецов, одетый в льняные одежды: 

\vs 5Sb 1:493 Люди, построим святыню в честь истинно Сущего Бога! 

\vs 5Sb 1:494 Люди, ужасный обычай, от предков идущий, изменим 

\vs 5Sb 1:495 Тот, по которому деды богам из глины и камня, 

\vs 5Sb 1:496 Шествия, жертвы, обряды творя, потеряли разсудок. 

\vs 5Sb 1:497 Несокрушимого Бога возславив, душой обратимся, 

\vs 5Sb 1:498 Люди, к Нему Самому  Создателю, Сущему вечно, 

\vs 5Sb 1:499 Кто всеми правит, Царю, справедливому мира Владыке,

\vs 5Sb 1:500 Душ Кормильцу, Отцу, Великому, Вечно Живому! 

\vs 5Sb 1:501 Так возведен будет храм в Египте, великий, священный; 

\vs 5Sb 1:502 Жертвы к нему понесет народ, наставленный Богом,  

\vs 5Sb 1:503 Те, кому вечную жизнь Господь на земле уготовил.

\vs 5Sb 1:504 Но лишь только уйдут Эфиопы от дерзких Трибаллов 

\vs 5Sb 1:505 И вознамерятся сами в Египте распахивать земли, 

\vs 5Sb 1:506 Зло они станут творить, чтобы гибель вселенной ускорить, 

\vs 5Sb 1:507 Наземь повергнут и храм великий в Египетском царстве. 

\vs 5Sb 1:508 Бог же за это на них Свой гнев ужасный обрушит, 

\vs 5Sb 1:509 Гибель тем самым неся преступникам и нечестивцам. 

\vs 5Sb 1:510 Больше никто в той земле уже не получит пощады, 

\vs 5Sb 1:511 Ибо сберечь не смогли того, что Господь им доверил.

\vs 5Sb 1:512 Видела я среди звезд сверкавшего Солнца угрозу,

\vs 5Sb 1:513 Гнев Луны величайший при свете блещущих молний.

\vs 5Sb 1:514 Звезды родили войну  Господь повелел им сражаться. 

\vs 5Sb 1:515 Вместо Солнца вовсю бушевало огромное пламя,

\vs 5Sb 1:516 Лунный двурогий изгиб потерял свою прежнюю форму.

\vs 5Sb 1:517 В битву вступила Венера, ко Льву на спину взобравшись;

\vs 5Sb 1:518 Прямо в загривок Тельца Козерог молодого ударил,

\vs 5Sb 1:519 Тот же за это лишил Козерога надежд на спасенье; 

\vs 5Sb 1:520 Дольше на небе сиять Орион Весам не позволил;

\vs 5Sb 1:521 Дева судьбу Близнецов в созвездье Овна изменила;

\vs 5Sb 1:522 Звезды Плеяд не взошли  их пояс Дракон уничтожил;

\vs 5Sb 1:523 В панцирь созвездия Льва наносить стали Рыбы удары;

\vs 5Sb 1:524 Рак не сумел устоять, боясь больше всех Ориона; 

\vs 5Sb 1:525 Встал на свой хвост Скорпион, перед Львом робея ужасным;

\vs 5Sb 1:526 Пес помчался стремглав от огня палящего Солнца;

\vs 5Sb 1:527 Гнев большого Светила заставил пылать Водолея. 

\vs 5Sb 1:528 Начал трястись Небосвод, пока не стряхнул воевавших. 

\vs 5Sb 1:529 Сильно разгневавшись, он с высоты на землю их бросил, 

\vs 5Sb 1:530 Так что, стремительно вниз в океанские воды сорвавшись, 

\vs 5Sb 1:531 Землю спалили огнем, а небо лишилось созвездий.

\bibbookdescr{6Sb}{
  inline={Шестая книга Сивилл},
  toc={6-я Сивилл},
  bookmark={6-я Сивилл},
  header={6-я Сивилл},
  abbr={6~Сив}
}
\vs 6Sb 1:1 Сына Безсмертного Бога пою я, Великого в славе, 

\vs 6Sb 1:2 Кто еще не был рожден, когда Всевышний Родитель 

\vs 6Sb 1:3 Трон Ему уготовил, и Кто родился вторично, 

\vs 6Sb 1:4 В плоть и кровь облечен, омытый водой Иордана,

\vs 6Sb 1:5 Что блестящей стопой, катя свои воды, несется. 

\vs 6Sb 1:6 Я воспеваю Того, Кто огня избежит и увидит 

\vs 6Sb 1:7 Первым Духа Господня, слетевшего в белой голубке. 

\vs 6Sb 1:8 Чистый цветок расцветет, источники вод заструятся  

\vs 6Sb 1:9 Людям укажет пути, укажет и торные тропы,

\vs 6Sb 1:10 К небу ведущие: всех Он умным словом научит.

\vs 6Sb 1:11 Будет к суду призывать, убеждать народ непослушный, 

\vs 6Sb 1:12 Смело славный Свой род от Отца Небесного выдав: 

\vs 6Sb 1:13 Будет ходить по волнам, людей избавлять от болезней, 

\vs 6Sb 1:14 Мертвых поднимет и прочь отгонит тяжкие муки,

\vs 6Sb 1:15 Всех из котомки одной Он досыта хлебом накормит. 

\vs 6Sb 1:16 Семя Давыдово даст росток  в руке Его будет 

\vs 6Sb 1:17 Целый мир и земля и огромное небо и море. 

\vs 6Sb 1:18 Молнией землю осветит, подобно тому как явился 

\vs 6Sb 1:19 Он впервые двоим, от общей плоти рожденным.

\vs 6Sb 1:20 Будет все так, когда миру Ребенок надежду подарит.

\vs 6Sb 1:21 Только тебе одному злые беды, Содом, угрожают:

\vs 6Sb 1:22 Ибо в безумии ты своего не узнал Господина,

\vs 6Sb 1:23 К смертным пришедшего людям, Которого тернием колким

\vs 6Sb 1:24 Ты увенчал, примешав к дерзновению черную злобу 

\vs 6Sb 1:25 В сердце своем, и сулит тебе это тяжкие муки. 

\vs 6Sb 1:26 О блаженное древо, на коем Бога распяли! 

\vs 6Sb 1:27 Не на земле пребывать тебе предстоит, а на небе, 

\vs 6Sb 1:28 Бог когда огненный взгляд обновленный как молнию кинет.

\bibbookdescr{7Sb}{
  inline={Седьмая книга Сивилл},
  toc={7-я Сивилл},
  bookmark={7-я Сивилл},
  header={7-я Сивилл},
  abbr={7~Сив}
}
\vs 7Sb 1:1 Родос злосчастный, тебя, тебя я первым оплачу! 

\vs 7Sb 1:2 Первым среди городов ты будешь и первым погибнешь, 

\vs 7Sb 1:3 Жизни лишен, без людей, одинокий и очень несчастный.

\vs 7Sb 1:4 Делос, ты поплывешь и на волнах будешь качаться. 

\vs 7Sb 1:5 Кипр, однажды тебя затопят воды морские.

\vs 7Sb 1:6 Остров Сицилия, ты погибнешь, охвачен пожаром.

\vs 7Sb 1:7 То, о чем говорю: ужасный, невиданный прежде 

\vs 7Sb 1:8 Хлынет на землю потоп, Самим низпосланный Богом.

\vs 7Sb 1:9 Ной лишь один уцелел, от всех людей убежавший.

\vs 7Sb 1:10 Все поплывет  и земля, и горы, и небо над ними; 

\vs 7Sb 1:11 Мир весь станет водой и водами будет погублен. 

\vs 7Sb 1:12 Ветры дуть прекратят, наступит другая эпоха.

\vs 7Sb 1:13 Фригия! Первой на свет суждено тебе снова подняться, 

\vs 7Sb 1:14 Первой, в нечестие впав, сама отречешься от Бога 

\vs 7Sb 1:15 И, предпочтенье отдав немым изваяньям, за это, 

\vs 7Sb 1:16 Жалкая, годы спустя ужасною смертью погибнешь.

\vs 7Sb 1:17 Много бед претерпев, Эфиопы несчастные, в страхе 

\vs 7Sb 1:18 Телом дрожа, под мечи себя безрассудно поставят.

\vs 7Sb 1:19 Трудолюбивый Египет, от века растящий колосья, 

\vs 7Sb 1:20 Тот, которого Нил питает семью рукавами, 

\vs 7Sb 1:21 Междуусобная рознь погубит. Тогда же, нежданно, 

\vs 7Sb 1:22 Аписа люди изгонят за то, что вовсе не бог он.

\vs 7Sb 1:23 Лаодикия, увы! Ты Бога впредь не увидишь,

\vs 7Sb 1:24 Но, погрязнув во лжи, будешь смыта Ликской волною.

\vs 7Sb 1:25 Сам рожденный Господь, великий, Который без счета 

\vs 7Sb 1:26 Звезд сотворит и ось проденет сквозь неба средину, 

\vs 7Sb 1:27 Людям на страх, в вышине, чтобы видели все, установит 

\vs 7Sb 1:28 Столп, измерив его огнем великим, чьи капли 

\vs 7Sb 1:29 Жизнь отнимут у тех, кто себя запятнал преступленьем. 

\vs 7Sb 1:30 Время такое наступит однажды, и смертные люди 

\vs 7Sb 1:31 Бога тут станут молить, но не будет предела страданьям 

\vs 7Sb 1:32 Их безконечным. Тогда чрез дом все свершится Давидов, 

\vs 7Sb 1:33 Ибо сам Бог удостоил его небесного трона. 

\vs 7Sb 1:34 Ангелы лягут у ног, которые именем Божьим 

\vs 7Sb 1:35 Свет огням алтарей дают и воды  потокам: 

\vs 7Sb 1:36 Те хранят города, другие  ветра посылают.

\vs 7Sb 1:37 Многих людей ожидают невзгоды, что, в души несчастных 

\vs 7Sb 1:38 Путь пролагая, сердца их всех измениться заставят.

\vs 7Sb 1:39 В пору, как юный росток, на корне возросший, прозреет, 

\vs 7Sb 1:40 Власть он, что некогда всех в избытке пищей снабжала.

\vs 7Sb 1:41 Это случиться должно с исполнением срока. Но стоит 

\vs 7Sb 1:42 Править воинственным Персам начать, как тогда же покои 

\vs 7Sb 1:43 Брачные чистых невест омрачатся всеобщим нечестьем. 

\vs 7Sb 1:44 Сына мать своего как супруга на ложе допустит, 

\vs 7Sb 1:45 Мать соблазнит ее сын. Уснет, к отцу прижимаясь,

\vs 7Sb 1:46 Дочь, исполняя обычай их варварский. Позже над ними 

\vs 7Sb 1:47 Римский Арей заблестит оружьем несметного войска. 

\vs 7Sb 1:48 Много смешают земли тут с кровью убитых в сраженье, 

\vs 7Sb 1:49 Но от твердости копий бежит Италийский воитель. 

\vs 7Sb 1:50 Бросят они в той стране из золота сделанный символ 

\vs 7Sb 1:51 Тот, что вперед выходя, всегда означал неизбежность.

\vs 7Sb 1:52 Время настанет, и весь погрязший в пороке, несчастный 

\vs 7Sb 1:53 Смерть обретет Илион вместо свадеб, когда зарыдают 

\vs 7Sb 1:54 Горько юные жены о том, что не ведали Бога, 

\vs 7Sb 1:55 Но ударяли в тимпаны и били ногами о землю.

\vs 7Sb 1:56 Бога спроси, Колофон: тебя страшный пожар ожидает.

\vs 7Sb 1:57 Ты, несчастная в браке, Фессалия! Снова увидеть 

\vs 7Sb 1:58 Лик твой земле не дано, как и пепел. Отсюда по морю, 

\vs 7Sb 1:59 Храбрая, вдаль отплывешь и войны испражнением станешь, 

\vs 7Sb 1:60 Пав под ударом мечей и сгинув в стремительных реках. 

\vs 7Sb 1:61 Стойкий Коринф! Ты у стен Арея грозного примешь: 

\vs 7Sb 1:62 Горе тебе, ибо вы падете сраженные оба.

\vs 7Sb 1:63 Тир, тебе одному пережить уготовано столько: 

\vs 7Sb 1:64 Набожных граждан твоих безсилье тебя же разрушит.

\vs 7Sb 1:65 Горная Сирия, ты поднялась над землей Финикийской, 

\vs 7Sb 1:66 Где к берегам приливают валы Беритского моря. 

\vs 7Sb 1:67 Бога узнать своего не смогла, несчастная  влагой 

\vs 7Sb 1:68 Кто Иорданской омыт, на Кого Божий Дух опустился. 

\vs 7Sb 1:69 Кто, до того, как земля и звездное небо возникли, 

\vs 7Sb 1:70 Словом Отца был рожден, Властелин, и, плотью облекшись 

\vs 7Sb 1:71 Через Духа Святого, к Отцу вскоре в домы вознесся. 

\vs 7Sb 1:72 Три Ему башни Уран великий поставил, в которых 

\vs 7Sb 1:73 Матери Бога теперь живут благородные. Имя 

\vs 7Sb 1:74 Первой  Надежда, второй  Благочестье и Набожность  третьей.

\vs 7Sb 1:75 Ни серебра не хотят, ни золота  радость приносят 

\vs 7Sb 1:76 Им поклоненье людей, их жертвы и чистые мысли.

\vs 7Sb 1:77 Жертвовать вечному Богу, великому, славному станешь, 

\vs 7Sb 1:78 Не растопив на огне крупицу ладана, нож свой 

\vs 7Sb 1:79 Не занеся над бараном с волнистым руном, но со всеми, 

\vs 7Sb 1:80 В ком течет твоя кровь, взяв птицу дикую в руки, 

\vs 7Sb 1:81 Вверх направишь ее, с молитвою глядя на небо.

\vs 7Sb 1:82 Воду на чистый огонь прольешь и скажешь при этом: 

\vs 7Sb 1:83 Твой Отец Тебя создал как Слово, Отче. Я птицу 

\vs 7Sb 1:84 Быструю выпустил с вестью  о Слове Слово, крещенье 

\vs 7Sb 1:85 Влагой Твое окропив  огонь, из какого Ты вышел.

\vs 7Sb 1:86 Ты не закроешь дверей, когда чужестранец безвестный 

\vs 7Sb 1:87 К дому придет твоему, нуждаясь в пище и крове. 

\vs 7Sb 1:88 Но, его голову взяв в ладони, обрызгав водою, 

\vs 7Sb 1:89 Трижды мольбу вознеси, обратись к своему Господину: 

\vs 7Sb 1:90 Я не жажду богатства, простой  простого я принял.

\vs 7Sb 1:91 Вдвое подай нам, Отец, склони Свой слух, Покровитель! 

\vs 7Sb 1:92 Даст Он, мольбе твоей вняв, когда же уйдет чужеземец: 

\vs 7Sb 1:93 Мукам меня не предай, о праведной веры Святыня, 

\vs 7Sb 1:94 Чистый, свободный, прошедший сквозь пламя \ldots

\vs 7Sb 1:95 Слабый мой дух укрепи. Отец. На Тебя я взираю, 

\vs 7Sb 1:96 Кто всякой скверны далек, руками не создан людскими \ldots

\vs 7Sb 1:97 Жребий, Сардиния, твой несчастен  в золу превратишься, 

\vs 7Sb 1:98 Сменится десять эпох  и островом быть перестанешь. 

\vs 7Sb 1:99 Тщетно тебя среди волн искать мореходам придется, 

\vs 7Sb 1:100 Птицы свой жалобный плач по тебе над морем поднимут.

\vs 7Sb 1:101 Камнем покрыта сплошным, Мигдония, крепость на море, 

\vs 7Sb 1:102 Славиться будешь века, чтобы после навеки погибнуть

\vs 7Sb 1:103 Всей под горячим дыханьем, от боли придя в изступленье.

\vs 7Sb 1:104 Кельтов земля! По горам, у подножия Альп недоступных 

\vs 7Sb 1:105 Скроет глубокий песок тебя; не выплатишь дани, 

\vs 7Sb 1:106 Колос не дашь и траву  безлюдною ляжешь пустыней, 

\vs 7Sb 1:107 Вечно покрытая льдом, под слоем холодных кристаллов, 

\vs 7Sb 1:108 Будешь страдать за вину, которой преступно не помнишь.

\vs 7Sb 1:109 Рим, чей дух непреклонен! Вослед Македонскому царству 

\vs 7Sb 1:110 Дротик метнешь ты в Олимп  за это немым и печальным 

\vs 7Sb 1:111 Бог тебя сделает, пусть казаться ты будешь в то время 

\vs 7Sb 1:112 Сильным как никогда  тут я обращусь к тебе с речью. 

\vs 7Sb 1:113 Чувствуя гибель, оплачешь ты блеск и славу былую  

\vs 7Sb 1:114 Я во второй раз, о Рим, возьмусь тебе это напомнить.

\vs 7Sb 1:115 Ныне же я по тебе, несчастная Сирия, плачу.

\vs 7Sb 1:116 Разум оставил вас, Фивы; нависли ужасные звуки 

\vs 7Sb 1:117 Громко взывающих флейт, труба им грозная вторит  

\vs 7Sb 1:118 Так что увидите вы поверженной в прах вашу землю.

\vs 7Sb 1:119 Горе, о горе тебе, несчастное злобное море?

\vs 7Sb 1:120 Все тебя пламя пожрет, людей ты погубишь волнами 

\vs 7Sb 1:121 Ибо такой на земле пожар забушует, что воды 

\vs 7Sb 1:122 Станут огнем, потекут и землю безкрайнюю сгубят, 

\vs 7Sb 1:123 Горы заставят они пылать, ключи, и потоки. 

\vs 7Sb 1:124 Мир же со смертью людей прекрасный свой облик утратит, 

\vs 7Sb 1:125 В муках сгорая, тогда не увидят несчастные неба,

\vs 7Sb 1:126 Полного звезд, но огнем оно все выжжено будет. 

\vs 7Sb 1:127 Быстро они не умрут: под гибнущей в пламени плотью 

\vs 7Sb 1:128 Души их будут пылать в продолжение многих столетий. 

\vs 7Sb 1:129 Так, злые муки терпя, Закон познают Господень  

\vs 7Sb 1:130 Тот, что всегда справедлив. Земля же под гнетом несчастья

\vs 7Sb 1:131 Всяких богов приняла на своих алтарях без разбора 

\vs 7Sb 1:132 И обманулась, понять не сумев зловещего дыма. 

\vs 7Sb 1:133 Тем страдать суждено сверх меры, кто ради корысти 

\vs 7Sb 1:134 Станет предсказывать зло, продляя тяжелое время. 

\vs 7Sb 1:135 Эти, надев на себя овец густорунные шкуры,

\vs 7Sb 1:136 Будут себя выдавать за Евреев, хоть рода иного, 

\vs 7Sb 1:137 Хитрые речи плести, наживаясь на общем несчастье. 

\vs 7Sb 1:138 Жизнь поменяют свою, но праведных не убедить им  

\vs 7Sb 1:139 Тех, что, от чистого сердца уверовав, молятся Богу.

\vs 7Sb 1:140 В третьем жребии лет, что пройдут, друг друга сменяя,

\vs 7Sb 1:141 В первой восьмерке опять грядет обновление мира. 

\vs 7Sb 1:142 Долгая ночь на земле тогда неподвижная ляжет, 

\vs 7Sb 1:143 Запах серы зловещий начнет повсюду носиться  

\vs 7Sb 1:144 Вестник насильственной смерти, другие же люди в то время 

\vs 7Sb 1:145 Будут под пологом ночи от голода гибнуть. Тут явит

\vs 7Sb 1:146 Бог чистый ум средь людей и род возстановит, который 

\vs 7Sb 1:147 Некогда жил на земле. Никто больше пашню не взрежет 

\vs 7Sb 1:148 Выгнутым плугом, быки железо вглубь не опустят, 

\vs 7Sb 1:149 Больше ростки не взойдут, не будет колосьев. Все вместе 

\vs 7Sb 1:150 Манну росистую есть белоснежными станут зубами.

\vs 7Sb 1:151 С ними пребудет тогда Господь, и Он их научит 

\vs 7Sb 1:152 Так же, как и меня, несчастную  столько свершила 

\vs 7Sb 1:153 Прежде недобрых я дел и знала об этом. Другое 

\vs 7Sb 1:154 Было содеяно мною невольно. Со многими ложе 

\vs 7Sb 1:155 Я разделила без мысли о браке; ужасную клятву

\vs 7Sb 1:156 Всем вероломно дала. На порог не пустила я бедных. 

\vs 7Sb 1:157 И, среди первых спускаясь в долину смерти, Господних

\vs 7Sb 1:158 Слов не сумела понять  за то и пожрет меня пламень. 

\vs 7Sb 1:159 Вечно мне жить не дано  погубит жестокое время, 

\vs 7Sb 1:160 Люди меня погребут, проплывая по морю мимо. 

\vs 7Sb 1:161 Буду побита камнями вещав, что велел мне родитель, 

\vs 7Sb 1:162 Сына я предала! Все киньте по камню, кидайте  

\vs 7Sb 1:163 Так сохраню себе жизнь и к небу очи воздену.

\bibbookdescr{8Sb}{
  inline={Восьмая книга Сивилл},
  toc={8-я Сивилл},
  bookmark={8-я Сивилл},
  header={8-я Сивилл},
  abbr={8~Сив}
}
\vs 8Sb 1:1 Гнев Господень ужасный грядет непокорному свету! 

\vs 8Sb 1:2 Все, чем Бог угрожает последнему веку, скажу я 

\vs 8Sb 1:3 Жителям каждого града, всем будет пророчество ясным. 

\vs 8Sb 1:4 С той поры, когда башня упала, а вслед человечий 

\vs 8Sb 1:5 Множеством говоров стал язык, внезапно распавшись, 

\vs 8Sb 1:6 Царство Египта вначале возникло, потом государства 

\vs 8Sb 1:7 Персов, Мидян, Эфиопов, в Ассирии вкруг Вавилона, 

\vs 8Sb 1:8 И в Македонии гордой  про всех уже сказано мною; 

\vs 8Sb 1:9 Ныне же я обращусь к пресловутой земле Италийской.

\vs 8Sb 1:10 Множество зол в конце времен причинит она смертным: 

\vs 8Sb 1:11 Всюду сведет на нет старания разных народов, 

\vs 8Sb 1:12 Многих отважных мужей она в плен угонит на Запад, 

\vs 8Sb 1:13 Все подчинит и народам свои предпишет законы. 

\vs 8Sb 1:14 Долго пусть мелют зерно жернова Господни, но мелко.

\vs 8Sb 1:15 Сгинет все от огня, и в тонкий пух превратятся 

\vs 8Sb 1:16 Гор высоких вершины, а всякая плоть станет пылью. 

\vs 8Sb 1:17 Алчность и безразсудство  всех бед и несчастий начало. 

\vs 8Sb 1:18 К золоту и серебру, к обманчивым, люди стремятся, 

\vs 8Sb 1:19 Лучшим, что есть на земле, металлы эти считая:

\vs 8Sb 1:20 Лучше, чем солнца сиянье, и лучше, чем небо иль море, 

\vs 8Sb 1:21 Или земля, что, простершись широко, все порождает, 

\vs 8Sb 1:22 Иль даже Бог, сотворивший весь мир и все подающий; 

\vs 8Sb 1:23 Веру и благочестье поставили ниже металлов. 

\vs 8Sb 1:24 Это безумье  источник неправды и смуты зачинщик,

\vs 8Sb 1:25 Мирной жизни оно враждебно, а войнам  причина, 

\vs 8Sb 1:26 Ведь от него и отцы с сыновьями своими враждуют, 

\vs 8Sb 1:27 Также и брак не в чести у тех, кто золото любит.

\vs 8Sb 1:28 Всюду на землях  границы, и всюду стражи на море, 

\vs 8Sb 1:29 Делится все хитроумно меж теми, кто златом владеет,

\vs 8Sb 1:30 Будто навечно хотят забрать плодоносную землю. 

\vs 8Sb 1:31 Грабят они бедняков, лишь бы только именье расширить, 

\vs 8Sb 1:32 Тех, кто им отдал свое, в рабов обращая хвастливо. 

\vs 8Sb 1:33 Если б земля не была далека от звездного неба, 

\vs 8Sb 1:34 Не был бы также и свет одинаково людям доступным,

\vs 8Sb 1:35 Но только тот, кто богат, покупать его мог бы за деньги, 

\vs 8Sb 1:36 Новый же мир сотворить для бедных Богу пришлось бы. 

\vs 8Sb 1:37 Рим надменный, тебе испытать когда-то придется 

\vs 8Sb 1:38 Неба удар справедливый, ты первый шею преклонишь, 

\vs 8Sb 1:39 Рухнешь наземь, огнем истребишься до основанья,

\vs 8Sb 1:40 Лежа на собственных землях, и все богатство погибнет, 

\vs 8Sb 1:41 А во дворцах будут жить лишь дикие волки и лисы; 

\vs 8Sb 1:42 Станешь пустынею, словно и не было города вовсе. 

\vs 8Sb 1:43 Где твой палладий? И где тот бог, что пришел бы на помощь, 

\vs 8Sb 1:44 Медный, иль золотой, иль каменный? Где же сената

\vs 8Sb 1:45 Постановления? Где потомки Кроноса, Реи

\vs 8Sb 1:46 Или же Зевса и всех, кто так были чтимы тобою? 

\vs 8Sb 1:47 То  божества без души, подобия трупов безсильных, 

\vs 8Sb 1:48 Скроет которых земля несчастного Крита, и станут 

\vs 8Sb 1:49 С гордостью там почитать мертвецов, что не чувствуют больше.

\vs 8Sb 1:50 После трижды пяти царей, о изнеженный город, 

\vs 8Sb 1:51 Что покорят весь мир от Восхода и вплоть до Заката, 

\vs 8Sb 1:52 Вождь воцарится седой, соименник ближнего моря. 

\vs 8Sb 1:53 Грозной ногою пройдет он по миру, дары добывая, 

\vs 8Sb 1:54 Многое множество злата, а с ним серебра еще больше

\vs 8Sb 1:55 Он у врагов заберет и, награбив, домой возвратится. 

\vs 8Sb 1:56 Царь тот в святилищах магов участником станет мистерий, 

\vs 8Sb 1:57 Мальчика сделает богом, но все богов почитанье 

\vs 8Sb 1:58 Сам низпровергнет, открыв всю лживость мистерий для смертных. 

\vs 8Sb 1:59 Будет ужасное время, когда сам Ужасный погибнет.

\vs 8Sb 1:60 Скажет однажды народ: О город, падет твоя сила!  

\vs 8Sb 1:61 Ибо почувствует вдруг дурного дня приближенье. 

\vs 8Sb 1:62 Горько заплачут тогда, предвидя удел твой несчастный, 

\vs 8Sb 1:63 Вместе родители все и все неразумные дети, 

\vs 8Sb 1:64 Скорбным рыданием их огласятся два берега Тибра.

\vs 8Sb 1:65 Трое за тем царем в последние дни будут править, 

\vs 8Sb 1:66 Имя собою исполнив Небесного Бога, чья сила 

\vs 8Sb 1:67 Вплоть до скончания всех времен пребудет, как ныне. 

\vs 8Sb 1:68 Скипетр удержит надолго один из них, муж престарелый, 

\vs 8Sb 1:69 Царь, сожаленья достойный, который сокровища мира

\vs 8Sb 1:70 Все в чертогах своих укроет, чтобы раздать их

\vs 8Sb 1:71 Людям, когда от границ земных беглец возвратится, 

\vs 8Sb 1:72 Мать погубивший; тогда богатой Азия станет. 

\vs 8Sb 1:73 Снимешь в те дни ты наряд, с широкою красной каймою 

\vs 8Sb 1:74 И облечешься, печалясь, в одежду скорби глубокой,

\vs 8Sb 1:75 О надменное царство, о дочь Латинского Рима! 

\vs 8Sb 1:76 Славною гордость твоя надменная быть перестанет, 

\vs 8Sb 1:77 Будешь лежать распростершись и больше тебе не подняться. 

\vs 8Sb 1:78 Честь легионов твоих падет со всеми орлами, 

\vs 8Sb 1:79 Где ж твоя сила? И кто союзником быть согласится,

\vs 8Sb 1:80 Пред неразумьем твоим безбожным главу преклоняя? 

\vs 8Sb 1:81 Тут среди смертных по всей земле начнется смятенье, 

\vs 8Sb 1:82 В день, как придет Вседержитель, возсядет на троне и будет 

\vs 8Sb 1:83 Души судить живых и мертвых  суд над вселенной. 

\vs 8Sb 1:84 Станут тогда немилы родители детям, а дети

\vs 8Sb 1:85 Тем, кто родил их, от горя нежданного и от нечестья. 

\vs 8Sb 1:86 Скрежет зубов тебя ждет, раскаянье и покоренье, 

\vs 8Sb 1:87 Грады рухнут когда и земля провалы разверзнет. 

\vs 8Sb 1:88 Тут пурпурный дракон по морским приплывет к тебе волнам, 

\vs 8Sb 1:89 Чревом наполненным он детей твоих вскармливать будет

\vs 8Sb 1:90 В дни, когда голод придет и войны гражданские грянут; 

\vs 8Sb 1:91 Все это значит, что день последний этого мира 

\vs 8Sb 1:92 Близок, и вскоре все люди на Суд будут призваны Богом. 

\vs 8Sb 1:93 Римлянам первым придется познать Его гнев безпощадный, 

\vs 8Sb 1:94 Ждет их кровавое время и в жизни сплошные несчастья.

\vs 8Sb 1:95 Горе тебе, о земля Италийская  варваров племя!  

\vs 8Sb 1:96 Ты позабыла о том, что, на свет появившись нагою, 

\vs 8Sb 1:97 И недостойной, назад уйдешь ты опять без одежды 

\vs 8Sb 1:98 В то же самое место, где суд над тобою свершится, 

\vs 8Sb 1:99 Ибо неправедно ты сама осуждала \ldots

\vs 8Sb 1:100 Руки твои словно руки Гигантов были, когда ты

\vs 8Sb 1:101 Шла в этот мир с высоты  теперь под землей твое место. 

\vs 8Sb 1:102 Нефтью, серой, асфальтом, великим огнем истребишься 

\vs 8Sb 1:103 И превратишься ты в груду горячего пепла навеки. 

\vs 8Sb 1:104 Каждый, кто ни посмотрит, услышит, из Тартара вопли,

\vs 8Sb 1:105 Громкие, полные скорби, и скрежет зубов, и удары 

\vs 8Sb 1:106 Жалких ладоней твоих, что в грудь безбожную бьются.

\vs 8Sb 1:107 Ночь одинаково всех ожидает  богатых и бедных, 

\vs 8Sb 1:108 Мы из земли нагими выходим, нагими же в землю 

\vs 8Sb 1:109 Снова ложимся, как только исполнится срок нашей жизни.

\vs 8Sb 1:110 Там уже нет никаких рабов, ни господ, ни тиранов, 

\vs 8Sb 1:111 Нет ни царей, ни вождей надменных и полных гордыни, 

\vs 8Sb 1:112 Ни хитроумных витий, ни начальников нет лихоимцев, 

\vs 8Sb 1:113 Больше уж на алтарях не прольется жертвенной крови, 

\vs 8Sb 1:114 Не зазвучат ни тимпан, ни кимвал \ldots

\vs 8Sb 1:115 Флейты многоотверстной безумных звуков не станет, 

\vs 8Sb 1:116 Нет и песни свирели, подобной извивам дракона, 

\vs 8Sb 1:117 Нет и варварских труб, что людям войну возвещают, 

\vs 8Sb 1:118 И не упьется вином уж никто на пирах нечестивых, ъ

\vs 8Sb 1:119 Нет ни плясок, ни пенья кифары; исчезнет коварство,

\vs 8Sb 1:120 Ссоры и гнев многовидный, и острых ножей не имеют 

\vs 8Sb 1:121 Те, кто жизнь завершил; единый лишь век остается. 

\vs 8Sb 1:122 Ключник великой ограды, привратник Божьего трона \ldots

\vs 8Sb 1:123 Идолы пусть вас украсят из золота, дерева, камня, 

\vs 8Sb 1:124 Пусть простоят до тех пор, когда день самый горький настанет,

\vs 8Sb 1:125 Чтоб твою кару узнать, о Рим, и вопли услышать. 

\vs 8Sb 1:126 Шею больше как раб под ярмо твое не преклонят 

\vs 8Sb 1:127 Ни Сириец, ни Эллин, ни варвар, ни племя иное. 

\vs 8Sb 1:128 Ждет разграбленье тебя, воздаcтся за то, что творил ты, 

\vs 8Sb 1:129 Будешь от страха стенать, пока за все не отплатишь.

\vs 8Sb 1:130 Целый мир над тобой, опозоренным, справит победу \ldots

\vs 8Sb 1:131 \ldots\ И в поколенье шестом царей Латинских угаснут 

\vs 8Sb 1:132 Жизни остатки, и руки удерживать скиптры не смогут. 

\vs 8Sb 1:133 Будет властвовать царь другой из того поколенья, 

\vs 8Sb 1:134 Скиптры все подчинит и землю всю покорит он, 

\vs 8Sb 1:135 Править будет один, но по воле Всевышнего Бога; 

\vs 8Sb 1:136 Дети и внуки его составят род нерушимый. 

\vs 8Sb 1:137 Так назначено Богом, когда круг времен совершится 

\vs 8Sb 1:138 И трижды пять царей над Египтом правленье закончат.

\vs 8Sb 1:139 Как подойдет к концу век птицы Феникса пятый \ldots

\vs 8Sb 1:140 \ldots\ Явится тут погубитель родов и племен без разбора, 

\vs 8Sb 1:141 И между ними  Евреев, Арес уничтожит Ареса, 

\vs 8Sb 1:142 Сам ведь на смерть обречет он угрозы надменные Рима. 

\vs 8Sb 1:143 Прежде цветущая, пала могучая Римлян держава, 

\vs 8Sb 1:144 Издавна всех городов, окрест лежащих, царица.

\vs 8Sb 1:145 Больше не будет победы для тучной Римской равнины, 

\vs 8Sb 1:146 После того как придет из Азии войск предводитель. 

\vs 8Sb 1:147 Все совершив, он захватит великий град; и в то время, 

\vs 8Sb 1:148 Как трижды триста исполнишь годов и еще сорок восемь, 

\vs 8Sb 1:149 Участь несчастная ждет тебя, от насилья погибнешь 

\vs 8Sb 1:150 И даже имя твое навеки будет забыто.

\vs 8Sb 1:151 Трижды несчастной, увы мне! когда же я день тот увижу, 

\vs 8Sb 1:152 Гибель несущий тебе, о Рим, и роду Латинян? 

\vs 8Sb 1:153 Радуйся, если желаешь, тому, кто, тайно рожденный 

\vs 8Sb 1:154 В Азии где-то, взошел на Троянскую колесницу, 

\vs 8Sb 1:155 Гневом кипя. Но когда перешеек Истмийский пробьет он, 

\vs 8Sb 1:156 Все озирая вокруг, враждебный ко всем, через море 

\vs 8Sb 1:157 Переплывет  тогда вслед за зверем великим польется 

\vs 8Sb 1:158 Черная кровь, но собака догонит губителя стада, 

\vs 8Sb 1:159 Льва; и, скиптра лишенный, пойдет он в царство Аида.

\vs 8Sb 1:160 Будет Родосцев несчастье последнее самым ужасным, 

\vs 8Sb 1:161 Фивы позорный захват ожидает затем, а Египет 

\vs 8Sb 1:162 От преступлений вождей своих негодных погибнет. 

\vs 8Sb 1:163 Тех же из смертных, кто смог избежать погибели страшной, 

\vs 8Sb 1:164 Трижды счастливыми нужно считать, и четырежды даже.

\vs 8Sb 1:165 Рим будет улочкой жалкой, а Делос невидимым станет, 

\vs 8Sb 1:166 Самос в песок превратится \ldots

\vs 8Sb 1:167 После придут, наконец, и к Персам великие беды 

\vs 8Sb 1:168 Карой за нрав их надменный  и вся гордыня исчезнет.

\vs 8Sb 1:169 Вождь святой подчинит себе все скиптры земные 

\vs 8Sb 1:170 С этой поры навсегда, и умерших от сна он пробудит.

\vs 8Sb 1:171 Волей Всевышнего жребий несчастный выпадет Риму:

\vs 8Sb 1:172 Всех, кто в городе этом живет, обрек Он на гибель.

\vs 8Sb 1:173 Но не хотят покориться, хоть лучше бы им это было.

\vs 8Sb 1:174 В пору, как день тот созреет, великой бедою чреватый  

\vs 8Sb 1:175 Голодом, мором и шумом войны, для людей нестерпимым,

\vs 8Sb 1:176 Снова тогда созовет несчастный, что раньше владыкой

\vs 8Sb 1:177 Был над Римом, совет, чтобы гибель тому уготовить \ldots

\vs 8Sb 1:178 \ldots\ Листья, едва распустившись, тотчас же станут сухими,

\vs 8Sb 1:179 Только на твердые скалы дожди с небосвода польются, 

\vs 8Sb 1:180 Почве же только ветра и огонь достанется жаркий, 

\vs 8Sb 1:181 Много семян оттого зазря пропадет в целом мире \ldots

\vs 8Sb 1:182 \ldots\ Зло продолжают творить, ибо всякий стыд потеряли 

\vs 8Sb 1:183 И не боятся уже ни людского, ни Божьего гнева, 

\vs 8Sb 1:184 Срам утратили все, променяли его на безстыдство; 

\vs 8Sb 1:185 Много насилий творят, закон попирают, тираны, 

\vs 8Sb 1:186 Лгут, обещаний не держат, ни слова правды не молвят, 

\vs 8Sb 1:187 Любят надутые речи, а веру поносят и гонят; 

\vs 8Sb 1:188 Нет для них насыщенья в богатстве, стремятся безстыдно 

\vs 8Sb 1:189 Больше и больше собрать  и сгинут под гнетом тиранов.

\vs 8Sb 1:190 Звезды все упадут прямо с неба в пучину морскую, 

\vs 8Sb 1:191 Но вместо них взойдут другие, одну из которых, 

\vs 8Sb 1:192 Что ярче всех заблестит, назовут кометою люди, 

\vs 8Sb 1:193 Знаменьем станет она войны и множества бедствий.

\vs 8Sb 1:194 Нет, не хотела б я жить во дни правленья нечистой, 

\vs 8Sb 1:195 Страстно желала б, напротив, когда небесная милость 

\vs 8Sb 1:196 В мире царицею станет, а всех коварных злодеев 

\vs 8Sb 1:197 Сын святой закует в оковы и в страшную бездну 

\vs 8Sb 1:198 Бросит  и смертных внезапно обнимет дом деревянный.

\vs 8Sb 1:199 После того как сойдет поколенье десятое в Тартар, 

\vs 8Sb 1:200 Женщина властью великой тогда завладеет, и много 

\vs 8Sb 1:201 Бед Господь низпошлет, когда она увенчает 

\vs 8Sb 1:202 Царским венцом главу  все в мире изменится сразу. 

\vs 8Sb 1:203 Жаркое солнце свой бег являть будет людям и ночью, 

\vs 8Sb 1:204 Звезды с неба исчезнут, и бешеный вихрь пронесется, 

\vs 8Sb 1:205 Опустошая весь мир, и мертвые всюду воскреснут, 

\vs 8Sb 1:206 Быстро хромые пойдут, и слышать смогут глухие, 

\vs 8Sb 1:207 Зренье получат слепцы, обретут дар речи немые. 

\vs 8Sb 1:208 Жизнь и богатство тогда для смертных общими будут, 

\vs 8Sb 1:209 Общей и вся земля; и, быть перестав разделенной 

\vs 8Sb 1:210 Стенами и рубежами, сама даст плод изобильный 

\vs 8Sb 1:211 И родники молока белоснежного, сладкого меда

\vs 8Sb 1:212 Даст и вино источит \ldots

\vs 8Sb 1:213 Суд бессмертного Бога \ldots

\vs 8Sb 1:214 Бог переменит все времена \ldots

\vs 8Sb 1:215 Зиму сделает летом; да сбудутся все предсказанья. 

\vs 8Sb 1:216 Но после гибели мира \ldots

\vs 8Sb 1:217 Из земли источит близость Судного дня капли пота,

\vs 8Sb 1:218 И снизойдет с небес тот Царь, что вечно пребудет.

\vs 8Sb 1:219 Суд учинит Он великий над миром и всякою плотью, 

\vs 8Sb 1:220 Узрят и верные Бога, и все неверные тоже,

\vs 8Sb 1:221 С высей как спустится Он в конце времен со святыми.

\vs 8Sb 1:222 Души плотских людей придут для суда к Его трону;

\vs 8Sb 1:223 Худо придется земле от засухи страшной и терний;

\vs 8Sb 1:224 Разных кумиров и все богатство смертные бросят, 

\vs 8Sb 1:225 И проникающий всюду огонь и сушу, и море

\vs 8Sb 1:226 Сгубит, и небосвод, и крепкие двери Аида.

\vs 8Sb 1:227 Тем, кто был жизни святой, свободы свет засияет;

\vs 8Sb 1:228 Огненной каре навек предаст нечестивцев Всевышний,

\vs 8Sb 1:229 Скажет всякий из них о зле, что тайно творил он, 

\vs 8Sb 1:230 Ибо сердечная тьма озарится светом Господним.

\vs 8Sb 1:231 Скрежет зубовный тогда и плач всеобщий раздастся,

\vs 8Sb 1:232 Солнца померкнет сиянье, и скроются звезд хороводы,

\vs 8Sb 1:233 Небо свернется как свиток, луны мерцанье угаснет,

\vs 8Sb 1:234 Высями станут долины, в низины холмы обратятся. 

\vs 8Sb 1:235 Больше высот на земле губительных вовсе не будет,

\vs 8Sb 1:236 Общее примут обличье равнины и горные кряжи;

\vs 8Sb 1:237 Глади морской не коснутся суда; земля запылает,

\vs 8Sb 1:238 А источники рек и бурные воды изсякнут.

\vs 8Sb 1:239 С неба труба пропоет печальным голосом песню, 

\vs 8Sb 1:240 Страшный позор несчастных оплачет и мира мученья,

\vs 8Sb 1:241 Пропасти, в почве разверзшись, покажут Тартара хаос,

\vs 8Sb 1:242 А пред небесным престолом Господним все люди сойдутся.

\vs 8Sb 1:243 С неба потоки огня и серы хлынут на землю.

\vs 8Sb 1:244 Будет для смертных тогда непреложным знаком, печатью, 

\vs 8Sb 1:245 Древом для верных тот рог, о котором так долго мечтали,

\vs 8Sb 1:246 Камень соблазна для мира, но жизнь для мужей справедливых,

\vs 8Sb 1:247 Радость света для званых двенадцатью водами давший,

\vs 8Sb 1:248 Есть у него и жезл железный, чтоб смертными править.

\vs 8Sb 1:249 Сам Господь наш Небесный записан тут акростихами  

\vs 8Sb 1:250 То Спаситель безсмертный и Царь, за нас пострадавший.

\vs 8Sb 1:251 Запечатлел же Его еще Моисей, распростерши 

\vs 8Sb 1:252 Руки святые, когда победил Амалика он верой; 

\vs 8Sb 1:253 Понял тогда народ, что избран Богом и славен 

\vs 8Sb 1:254 Жезл Давыда и Камень, что был заранее предсказан: 

\vs 8Sb 1:255 Тот, кто поверит в Него, сподобится жизни безсмертной.

\vs 8Sb 1:256 Он не во славе придет на суд, но как смертный несчастный, 

\vs 8Sb 1:257 Без красоты, без почета, чтоб дать несчастным надежду; 

\vs 8Sb 1:258 Тленному телу придаст Он форму, неверным дарует 

\vs 8Sb 1:259 Веру небесную Он, и вылепит вновь человека,

\vs 8Sb 1:260 Коего Сам Господь творил Своими руками. 

\vs 8Sb 1:261 Но обманул человека коварный змей и заставил 

\vs 8Sb 1:262 Смертную участь принять и познать благое и злое; 

\vs 8Sb 1:263 Так люди бросили Бога и тленному кланяться стали. 

\vs 8Sb 1:264 Сына в советники взяв изначально, рек Вседержитель:

\vs 8Sb 1:265 Сделаем вместе с Тобою, о Чадо, смертное племя, 

\vs 8Sb 1:266 Слепим его, отразив в нем Наше с Тобою обличье. 

\vs 8Sb 1:267 Я руками теперь, Ты позже словом послужишь 

\vs 8Sb 1:268 Делу Нашему, чтобы оно стало общим твореньем. 

\vs 8Sb 1:269 Помня об этом решенье, сойдет для суда Он на землю,

\vs 8Sb 1:270 В Деву святую вселившись подобным отображеньем, 

\vs 8Sb 1:271 Свет водой даровав через руки того, кто был старше, 

\vs 8Sb 1:272 Все Своим словом творя и любую болезнь исцеляя. 

\vs 8Sb 1:273 Словом же Он успокоит ветра и море разгладит, 

\vs 8Sb 1:274 Всюду покой принесет и веру, по свету скитаясь.

\vs 8Sb 1:275 Он хлебами пятью и рыбой из моря одною

\vs 8Sb 1:276 Целых пять тысяч мужей легко насытит в пустыне; 

\vs 8Sb 1:277 После, собрав те куски, что остались от пищи, наполнит 

\vs 8Sb 1:278 Ими двенадцать корзин, да придет надежда к народам. 

\vs 8Sb 1:279 Вызовет души блаженных и тех несчастных возлюбит,

\vs 8Sb 1:280 Кто, подвергаясь насмешкам, отплатит за зло только благом, 

\vs 8Sb 1:281 Кто нищету возлюбил, кто гоним и бичами терзаем. 

\vs 8Sb 1:282 Все доступно Его уму, Он все видит и слышит, 

\vs 8Sb 1:283 Узрит и то, что таится внутри, обнажит и очистит, 

\vs 8Sb 1:284 Ибо Он сам разуменье, и слух, и зренье всех сущих,

\vs 8Sb 1:285 Он же и Слово-Творец всех форм, и Ему все покорно. 

\vs 8Sb 1:286 Мертвых Он воскресит, исцелит любые болезни. 

\vs 8Sb 1:287 Он попадет, наконец, в безбожные руки неверных, 

\vs 8Sb 1:288 Станут грешной рукой наносить пощечины Богу, 

\vs 8Sb 1:289 Полную яда слюну из грязных уст извергая.

\vs 8Sb 1:290 Спину святую Свою ударам кнута Он подставит, 

\vs 8Sb 1:291 Ибо Он миру придет отдать непорочную Деву. 

\vs 8Sb 1:292 Будет молчанье хранить под ударами, чтоб не узнали, 

\vs 8Sb 1:293 Кто Он, чей и откуда; но к падшим речет Свое слово. 

\vs 8Sb 1:294 И увенчают Его венцом терновым, и станет

\vs 8Sb 1:295 Шип колючий наградой святым избранникам вечной. 

\vs 8Sb 1:296 Во исполненье закона пронзят тростником Его ребра, 

\vs 8Sb 1:297 Ведь подготовил тростник, не простым колеблемый ветром, 

\vs 8Sb 1:298 Душу Его к осужденью, обидам и наказанью. 

\vs 8Sb 1:299 Как совершится все то, что ныне предсказано мною,

\vs 8Sb 1:300 В Нем растворится всецело закон, что прежде народу 

\vs 8Sb 1:301 Дан непокорному был, в словах человечьих записан. 

\vs 8Sb 1:302 Руки раскинет и все, что есть в мире, Он ими обнимет; 

\vs 8Sb 1:303 Желчью кормили Его и уксусом горьким поили, 

\vs 8Sb 1:304 Кару получат они за враждебную трапезу эту.

\vs 8Sb 1:305 В храме порвется завеса, и день превратится внезапно 

\vs 8Sb 1:306 В ночь, что на три часа всю землю мраком покроет. 

\vs 8Sb 1:307 Скрытое бреднями мира, теперь станет ясно: не нужно 

\vs 8Sb 1:308 Больше храм почитать и закон, для людей непонятный  

\vs 8Sb 1:309 Сам ведь на землю сошел Безсмертный Владыка Небесный.

\vs 8Sb 1:310 Спустится после в Аид и для праведных вестником будет 

\vs 8Sb 1:311 Доброй надежды и дня, которым века прекратятся. 

\vs 8Sb 1:312 Смертный удел победит Он, на третий день пробудившись;

\vs 8Sb 1:313 Мертвых покинет тогда и вновь появится в мире, 

\vs 8Sb 1:314 Тем лишь, кто избран, сперва открыв воскресенья начало;

\vs 8Sb 1:315 Смоет водой родника, что дает безсмертия влагу, 

\vs 8Sb 1:316 Всякое прежнее зло, и, заново свыше родившись, 

\vs 8Sb 1:317 Быть перестанут рабами обычаев мира безбожных. 

\vs 8Sb 1:318 Явится прежде Господь своим, чтоб увидели ясно: 

\vs 8Sb 1:319 Вновь Он пришел во плоти, как прежде был, и покажет

\vs 8Sb 1:320 Им на руках и ногах четыре следа кровавых, 

\vs 8Sb 1:321 То будут Севера, Юга, Востока и Запада знаки, 

\vs 8Sb 1:322 Или число тех царств, что в мире свершат нечестиво 

\vs 8Sb 1:323 Дело позорное, кое нам всем в осужденье послужит.

\vs 8Sb 1:324 Дочь святая Сиона, ты много бед претерпела, 

\vs 8Sb 1:325 Радуйся ныне! Твой царь въезжает на ослике в Город, 

\vs 8Sb 1:326 Кротко, смиренно грядет, чтобы наше рабское иго, 

\vs 8Sb 1:327 Тяжко давившее спины, теперь упразднилось навеки, 

\vs 8Sb 1:328 Чтобы неправый закон исчез и гнетущие цепи.

\vs 8Sb 1:329 Так почти же Его как Бога и Божьего Сына, 

\vs 8Sb 1:330 В сердце свое прими и славь Его в радостных гимнах,

\vs 8Sb 1:331 Всею душой возлюби и себе возьми Его имя.

\vs 8Sb 1:332 Прежних всех удали и кровь, Им пролитую, смой ты;

\vs 8Sb 1:333 Жалких воплей твоих, и тленных жертв, и молений

\vs 8Sb 1:334 Вовсе не нужно Тому, Кто Сам безсмертен и вечен, 

\vs 8Sb 1:335 Но песнопений Он хочет из уст и сердца святого.

\vs 8Sb 1:336 Знай же, Кто Он таков  тогда и Отца Его узришь.

\vs 8Sb 1:337 Мира все элементы в то время придут в запустенье: 

\vs 8Sb 1:338 Воздух, море, земля и свет, от огня исходящий, 

\vs 8Sb 1:339 Ось небесная, ночь и все дни воедино сольются,

\vs 8Sb 1:340 В пламени формы свои они совершенно утратят. 

\vs 8Sb 1:341 Все светоносные звезды с небес упадут и исчезнут; 

\vs 8Sb 1:342 В воздухе больше не станут летать крылатые птицы, 

\vs 8Sb 1:343 Землю не тронет нога, ибо твари живые погибнут; 

\vs 8Sb 1:344 Смолкнут все голоса  людей, зверей и пернатых,

\vs 8Sb 1:345 Миру в его неустройстве не будет звука на пользу, 

\vs 8Sb 1:346 Но угрожающий шум издаст глубокое море, 

\vs 8Sb 1:347 Жители вод содрогнутся, и тут же конец им настанет; 

\vs 8Sb 1:348 Не поплывет по волнам и судно, везущее грузы. 

\vs 8Sb 1:349 Тяжко застонет земля, в сраженьях политая кровью;

\vs 8Sb 1:350 И заскрежещут зубами тогда все души людские,

\vs 8Sb 1:351 Грешные души погибель в стенаньях и ужасе встретят. 

\vs 8Sb 1:352 Жажда будет их жечь, убийства, болезни и голод, 

\vs 8Sb 1:353 И пожелают они умереть, но больше не смогут: 

\vs 8Sb 1:354 Не успокоит их смерть, и ночь не даст передышки.

\vs 8Sb 1:355 Долго Всевышнего Бога молить они будут напрасно  

\vs 8Sb 1:356 И отвратит Господь Свой лик, чтоб их больше не видеть: 

\vs 8Sb 1:357 Ибо ведь людям заблудшим Он семь веков предоставил 

\vs 8Sb 1:358 Для покаянья  за них просила Дева святая.

\vs 8Sb 1:359 Все эти речи Сам Бог вложил мне в сердце, и нужно, 

\vs 8Sb 1:360 Чтобы сбылось непременно все то, что уста мои молвят. 

\vs 8Sb 1:361 Мне известно число песчинок и вод в Океане, 

\vs 8Sb 1:362 Знаю земли тайники и Тартара мрачное царство, 

\vs 8Sb 1:363 Знаю все звезды на небе, и все деревья, и сколько 

\vs 8Sb 1:364 В мире четвероногих, плавучих и птиц оперенных, 

\vs 8Sb 1:365 Сколько людей живет, жило раньше и позже родится.

\vs 8Sb 1:366 В смертных запечатлел Я Сам и облик, и разум, 

\vs 8Sb 1:367 Дал им здравую мысль, научил их знаниям всяким,

\vs 8Sb 1:368 Создал Я ухо и глаз, и Сам все вижу и слышу, 

\vs 8Sb 1:369 Мысли чувствую все и всех событий Свидетель,

\vs 8Sb 1:370 Тот, который в начале молчит, а потом обличает, 

\vs 8Sb 1:371 Тяжко за тайное зло карая любого из смертных, 

\vs 8Sb 1:372 И у Господнего трона людей собеседником будет. 

\vs 8Sb 1:373 Я и глухого пойму и даже немого услышу. 

\vs 8Sb 1:374 Знаю и то, каково от земли до небес расстоянье,

\vs 8Sb 1:375 Знаю конец и начало, ведь Я создал небо и землю, 

\vs 8Sb 1:376 Создал Он все  от истока Он ведает все до предела. 

\vs 8Sb 1:377 Я  единственный Бог, иного не узрите Бога. 

\vs 8Sb 1:378 Люди подобье Мое деревянное обожествили, 

\vs 8Sb 1:379 Сделав своей же рукой изваянья безгласные, стали

\vs 8Sb 1:380 Кланяться им и молиться, служа нечестивые службы. 

\vs 8Sb 1:381 Чтили то, что несвято, Творца же совсем позабыли, 

\vs 8Sb 1:382 Все, рожденные Мной, приносят ненужные жертвы, 

\vs 8Sb 1:383 Точно во славу Мою, и достойным это считают, 

\vs 8Sb 1:384 Дым воскуряют, как будто заботясь о собственных мертвых.

\vs 8Sb 1:385 Мясо спаляют они и кости, полные мозга, 

\vs 8Sb 1:386 Жертвуя на алтарях и демонам кровь возливая. 

\vs 8Sb 1:387 Свечи Мне возжигают, хоть Я же свет даровал им, 

\vs 8Sb 1:388 Смертные Богу вино возливают, словно Он жаждет, 

\vs 8Sb 1:389 И напиваются сами у ног изваяний никчемных.

\vs 8Sb 1:390 Ваших не нужно Мне жертв, не нужно и возлияний; 

\vs 8Sb 1:391 Мне грязный дым неугоден и мерзостной крови потоки. 

\vs 8Sb 1:392 В память царей и тиранов свершают службы такие, 

\vs 8Sb 1:393 Кланяясь демонам мертвым, как будто в небе живущим,  

\vs 8Sb 1:394 Так почитаньем безбожным они себе гибель готовят.

\vs 8Sb 1:395 Бога лишенные люди зовут изваянья богами, 

\vs 8Sb 1:396 Но позабыли Творца и видят жизнь и надежду 

\vs 8Sb 1:397 В идолах, что не лишены и слуха, и голоса вовсе. 

\vs 8Sb 1:398 В злых лишь надежны делах, а добра не имеют и в мыслях.

\vs 8Sb 1:399 Мною даны два пути: есть к жизни дорога и к смерти,

\vs 8Sb 1:400 Я же благой дал совет  держаться праведной жизни, 

\vs 8Sb 1:401 Но предпочли эти люди погибель и вечное пламя. 

\vs 8Sb 1:402 Образ Мой  человек, наделенный здравою мыслью: 

\vs 8Sb 1:403 Вот ему и подай не кровавую  чистую пищу, 

\vs 8Sb 1:404 Сделай добро: удели тому, кто голоден, хлеба,

\vs 8Sb 1:405 Если жаждет  воды, если наг  одежды для тела;

\vs 8Sb 1:406 Милость твори от своих же трудов и ладонью святою. 

\vs 8Sb 1:407 Кто-то в унынье  утешь, устал кто  приди на подмогу: 

\vs 8Sb 1:408 Эту жертву живую воздай ты Богу Живому. 

\vs 8Sb 1:409 Сей зерно благочестья, тогда в награду получишь

\vs 8Sb 1:410 Плод безсмертный и свет негасимый, и жизни нетленной 

\vs 8Sb 1:411 Будешь достоин, когда всех людей огонь испытает. 

\vs 8Sb 1:412 Сплавлю Я все воедино и вновь разниму и очищу, 

\vs 8Sb 1:413 Небо сверну Я, как свиток, открою недра земные, 

\vs 8Sb 1:414 Мертвых в тот день воскрешу и судьбы людские разрушу,

\vs 8Sb 1:415 Жало смерти Я вырву; и суд последний устрою: 

\vs 8Sb 1:416 Стану жизни судить и праведных, и нечестивых, 

\vs 8Sb 1:417 Рядом встанет баран с бараном, и с пастырем пастырь, 

\vs 8Sb 1:418 Рядом с тельцом телец, чтоб друг друга они обличили. 

\vs 8Sb 1:419 Здесь обличатся все те, кто в жизни своей возвышался,

\vs 8Sb 1:420 Рты затыкая другим, и к зависти их побуждали, 

\vs 8Sb 1:421 Дабы и те справедливых людей в рабов обращали, 

\vs 8Sb 1:422 Всех заставляли молчать, влекомые только наживой. 

\vs 8Sb 1:423 Те же, кто праведно жил, все встанут рядом со Мною. 

\vs 8Sb 1:424 Больше в печали никто не скажет: То будет завтра,

\vs 8Sb 1:425 Или: То было вчера; и дней, заботами полных, 

\vs 8Sb 1:426 Также не станет, исчезнут четыре времени года, 

\vs 8Sb 1:427 С ними Восход и Закат, и все в долгий день обратится. 

\vs 8Sb 1:428 Свет пребудет вовеки, желанный и долгожданный  

\vs 8Sb 1:429 Навсегда Иисус Христос \ldots

\vs 8Sb 1:430 Боже, никем не рожденный, Чистейший, Вечный, Безсмертный!

\vs 8Sb 1:431 В небе живешь Ты и властью Своей усмиряешь дыханье 

\vs 8Sb 1:432 Пламени и поражаешь огнем могущество скиптров, 

\vs 8Sb 1:433 Гулких грома раскатов легко укрощаешь Ты силу, 

\vs 8Sb 1:434 Движешь Ты землю и держишь в узде волненье морское, 

\vs 8Sb 1:435 И ослабляешь бичи сверкающих огненных молний,

\vs 8Sb 1:436 Мощные ливней потоки, падение града и снега, 

\vs 8Sb 1:437 Что из тучи морозной летит, и бури порывы \ldots

\vs 8Sb 1:438 Ангелы, слуги Твои, заботливо распределяют

\vs 8Sb 1:439 Все, что Тобой решено и чему повелел Ты свершиться.

\vs 8Sb 1:440 Прежде творенья всего на совет призвав Свое Чадо, 

\vs 8Sb 1:441 Создал Ты смертных людей и жизни дал зародиться.

\vs 8Sb 1:442 Первым к Нему обратился Ты сладостной речью Твоею: 

\vs 8Sb 1:443 Сделаем ныне с Тобою подобного Нам человека 

\vs 8Sb 1:444 И дающее жизнь дыхание в грудь его вложим. 

\vs 8Sb 1:445 Смертен он будет, но все пускай на земле ему служит,

\vs 8Sb 1:446 Все ему подчиним, хоть его мы слепим из глины. 

\vs 8Sb 1:447 Это Ты Слову сказал, и стало так, как решил Ты. 

\vs 8Sb 1:448 Тотчас же все элементы велениям повиновались, 

\vs 8Sb 1:449 И со смертным твореньем все вечное соединилось: 

\vs 8Sb 1:450 Небо, воздух, огонь, земля и волны морские,

\vs 8Sb 1:451 Солнце, луна и созвездья \ldots

\vs 8Sb 1:452 Дни и ночи, и сон, пробуждение, дух и движенье, 

\vs 8Sb 1:453 И душа, и разсудок, искусство, сила и голос, 

\vs 8Sb 1:454 Стаи диких зверей, пернатые птицы и рыбы, 

\vs 8Sb 1:455 Те, кто ходит, и те, кто ползут, и амфибии также 

\vs 8Sb 1:456 Все человеку Он дал, Твоим решеньем наставлен.

\vs 8Sb 1:457 А у конца времен изменил Он землю: 

\vs 8Sb 1:458 Младенец Тело девы Марии покинул, и новый зажегся 

\vs 8Sb 1:459 Свет: пришел Он с небес и смертное принял обличье. 

\vs 8Sb 1:460 Мощный телом своим, сперва Гавриил появился,

\vs 8Sb 1:461 После же голос возвысил архангел и деве сказал он: 

\vs 8Sb 1:462 В чистое лоно свое прими ныне Бога, о дева! 

\vs 8Sb 1:463 Рек, и вдохнул благодать ей Господь; она же, услышав, 

\vs 8Sb 1:464 В страхе была и в смущенье, поскольку мужа не знала. 

\vs 8Sb 1:465 С места сойти не могла, в испуге дрожала, и сердце

\vs 8Sb 1:466 Быстро билось в груди от вести, еще непонятной. 

\vs 8Sb 1:467 Вскоре, однако, слова проникли в сердце, и тут же 

\vs 8Sb 1:468 Смех ее охватил, разрумянились юные щеки; 

\vs 8Sb 1:469 Перемешались в душе у девицы смущенье и радость, 

\vs 8Sb 1:470 И осмелела она. А слово, влетевшее в чрево,

\vs 8Sb 1:471 Плотью со временем стало и, в теле ожив материнском, 

\vs 8Sb 1:472 Облик людской обрело; и Мальчик на свет появился, 

\vs 8Sb 1:473 Девой рожден. Без сомненья  для смертных великое чудо, 

\vs 8Sb 1:474 Но не бывает великих чудес для Отца и для Сына. 

\vs 8Sb 1:475 Чада рожденье земле принесло великую радость,

\vs 8Sb 1:476 Возвеселился и в небе Престол, и мир  в ликованье. 

\vs 8Sb 1:477 Маги воздали честь звезде, невиданной прежде, 

\vs 8Sb 1:478 И, уверовав в Бога, лежащего в яслях узрели; 

\vs 8Sb 1:479 Пасшие коз и овец Дитя увидели также. 

\vs 8Sb 1:480 И наречен Вифлеем богоизбранный родиной Слова.

\vs 8Sb 1:481 \ldots\ Нужно в сердце иметь смиренье и зло ненавидеть, 

\vs 8Sb 1:482 К ближним питать любовь такую, как к собственной жизни, 

\vs 8Sb 1:483 Бога душою любить и Ему воздавать почитанье. 

\vs 8Sb 1:484 Ибо ведь мы от Него и святого Рожденья Христова 

\vs 8Sb 1:485 Небом произведены и единую кровь получили.

\vs 8Sb 1:486 Богу служа, мы всегда о блаженстве будущем помним, 

\vs 8Sb 1:487 Правды дорогой идя, прямым путем благочестья. 

\vs 8Sb 1:488 В храмы язычников мы никогда входить не дерзаем, 

\vs 8Sb 1:489 Нет для кумиров у нас ни молитвы, ни возлияний, 

\vs 8Sb 1:490 Ни аромата цветов, ни огня светильников ярких,

\vs 8Sb 1:491 Ни приношеньем даров богатых мы не почтим их 

\vs 8Sb 1:492 И благовонья на их алтарях никогда не воскурим; 

\vs 8Sb 1:493 В жертву быков и овец приносить им не станем, пытаясь 

\vs 8Sb 1:494 От наказания их таким путем откупиться; 

\vs 8Sb 1:495 Жирным дымом костра, который плоть пожирает,

\vs 8Sb 1:496 Мы осквернять не хотим сияния ясного неба.

\vs 8Sb 1:497 Но в помышлениях чистых, ликуя радостным сердцем, 

\vs 8Sb 1:498 Щедро дающей рукой и богатством любви безконечным, 

\vs 8Sb 1:499 Сладостной песней и в гимнах, достойных великого Бога, 

\vs 8Sb 1:500 Должно Тебя воспевать нам, неложно и непрестанно 

\vs 8Sb 1:501 Мудрый мира Создатель \ldots

\bibbookdescr{9Sb}{
  inline={Девятая книга Сивилл},
  toc={9-я Сивилл},
  bookmark={9-я Сивилл},
  header={9-я Сивилл},
  abbr={9~Сив}
}
\vs 9Sb 1:1 Смертные плотские люди, ведь вы совершенно ничтожны.

\vs 9Sb 1:2 Как же столь быстро в гордыне возноситесь вы, что забыли

\vs 9Sb 1:3 Вовсе о смерти и страхе пред Богом, Который все видит?

\vs 9Sb 1:4 Он, Высочайший, всеведущ, всех дел Он зоркий свидетель,

\vs 9Sb 1:5 Он, питающий все,  Создатель, вложивший дыханье

\vs 9Sb 1:6 Сладкое всем, и Вождя для смертных всех сотворивший.

\vs 9Sb 1:7 Бог  единый Владыка, безмерный и нерожденный,

\vs 9Sb 1:8 Правит всем и невидим, а Сам все сущее видит.

\vs 9Sb 1:9 Тленна плоть, для нее Господь остается незримым,

\vs 9Sb 1:10 Ибо ведь истинный Бог, Безсмертный, в небе живущий,

\vs 9Sb 1:11 Разве же плотским очам доступным сделаться может?

\vs 9Sb 1:12 Яркого солнца лучей не могут вынести люди,

\vs 9Sb 1:13 Смертны они, нелегко им стоять против света такого,

\vs 9Sb 1:14 Кости и жилы одни их плоть связуют собою.

\vs 9Sb 1:15 Чтите только Его, единого мира Владыку,

\vs 9Sb 1:16 Он лишь сущий от века и впредь вовеки пребудет,

\vs 9Sb 1:17 Сам он возник, нерожденный, и правит всем во вселенной,

\vs 9Sb 1:18 В каждом смертном живет и к свету путь указует.

\vs 9Sb 1:19 За неразумье свое обретете достойную плату 

\vs 9Sb 1:20 Истинно сущего Бога Безсмертного чтить перестали,

\vs 9Sb 1:21 Нет у вас для Него теперь гекатомбы священной,

\vs 9Sb 1:22 Ныне приносите жертвы Аида мрачного духам.

\vs 9Sb 1:23 Вы в слепоте и безумье бредете. Дорогу прямую,

\vs 9Sb 1:24 Ту что, верно ведет, покинули и заблудились

\vs 9Sb 1:25 Все средь шипов и колючек. О смертные, остановитесь!

\vs 9Sb 1:26 Бросьте скитаться во тьме и в черной ночи безпросветной 

\vs 9Sb 1:27 Мрак оставьте ночной, примите свет благодатный!

\vs 9Sb 1:28 Вот, он  видим для всех, и в нем нельзя ошибиться.

\vs 9Sb 1:29 Так подойдите сюда, не гонитесь за мраком и мглою  

\vs 9Sb 1:30 Солнца сладостный свет, смотрите, как ярко сияет!

\vs 9Sb 1:31 Знайте это и ныне в сердца свои мудрость вложите:

\vs 9Sb 1:32 Бог един  и дожди, и ветра, и мор, и голод, и беды,

\vs 9Sb 1:33 И снегопады, и лед. Смогу ли все перечислить?

\vs 9Sb 1:34 Он  и Небесный вождь, и Земной Владыка, и Сущий \ldots

\vs 9Sb 1:35 Если б рождалися боги и смерти при этом не знали,

\vs 9Sb 1:36 Смертных людей числом они превзошли бы намного,

\vs 9Sb 1:37 И не осталось бы места, где люди селиться могли бы \ldots

\vs 9Sb 1:38 Если рожденное все должно и погибнуть, то разве

\vs 9Sb 1:39 Могут богов создавать мужские и женские чресла?

\vs 9Sb 1:40 Только один есть Бог, Высочайший и все сотворивший:

\vs 9Sb 1:41 Небо, яркое солнце, луну и горящие звезды,

\vs 9Sb 1:42 И плодородную землю, и воды глубокого моря,

\vs 9Sb 1:43 Гор могучие выси, источники, бьющие вечно.

\vs 9Sb 1:44 Создал Он без числа и тех, кто в воде обитает,

\vs 9Sb 1:45 Пищу и тем подает, кому предназначено ползать,

\vs 9Sb 1:46 Разных питает он птиц, певучих и самых крикливых

\vs 9Sb 1:47 И рассекающих громко своими крыльями воздух;

\vs 9Sb 1:48 Дикое племя зверей породил Он в горных ущельях,

\vs 9Sb 1:49 Всякий скот Он заставил покорно служить человеку.

\vs 9Sb 1:50 Он надо всеми вождями поставил творение Божье,

\vs 9Sb 1:51 Сделав слугами людям обилье вещей непонятных  

\vs 9Sb 1:52 Как же смертная плоть постичь столь многое может?

\vs 9Sb 1:53 Мог только Тот познать, Кто все сотворил изначально,

\vs 9Sb 1:54 В небе живущий Создатель, Господь Безсмертный и вечный.

\vs 9Sb 1:55 Он возвращает сторицей тому, чьи деяния благи,

\vs 9Sb 1:56 Но против злых нечестивцев ужасный гнев возбуждает,

\vs 9Sb 1:57 Войны и голод несет им, страданья и многие слезы.

\vs 9Sb 1:58 Что же вы тщетной гордыней себя вырываете с корнем?

\vs 9Sb 1:59 Чтить как богов постыдитесь куниц и разных чудовищ!

\vs 9Sb 1:60 Это безумье и бред отнимают всякое чувство,

\vs 9Sb 1:61 Если боги у вас горшки похищают и чаши.

\vs 9Sb 1:62 Вместо того, чтобы жить счастливо в небе чудесном,

\vs 9Sb 1:63 Здесь, на земле пауков боятся, и черви их точат.

\vs 9Sb 1:64 Вы поклоняетесь змеям, собакам и кошкам, невежды,

\vs 9Sb 1:65 Тех, кто по небу летает, и тех, кто ползает, чтите,

\vs 9Sb 1:66 Также кумиров из камня и рук человечьих творенье,

\vs 9Sb 1:67 Статуи и даже кучи камней, что лежат при дорогах, 

\vs 9Sb 1:68 Много нелепых вещей, о которых и молвить позорно.

\vs 9Sb 1:69 Боги также ведут неразумных смертных коварно,

\vs 9Sb 1:70 Лживые их уста смертоносный яд источают.

\vs 9Sb 1:71 Но перед Тем, Кто есть жизнь и вечный свет негасимый,

\vs 9Sb 1:72 Радость роду людскому, сладчайшего сладостней меда,

\vs 9Sb 1:73 Щедро дает  перед Ним лишь одним главы преклоняйте,

\vs 9Sb 1:74 Следуйте той тропой, что в праведный век проторили.

\vs 9Sb 1:75 Все это бросили вы, испили отмщения кубок 

\vs 9Sb 1:76 Крепок напиток и пьян, водою ничуть не разбавлен 

\vs 9Sb 1:77 И помутился тотчас ваш дух в безумии тяжком.

\vs 9Sb 1:78 Вы протрезветь не хотите, вернуться к мыли разумной,

\vs 9Sb 1:79 Сущего Бога узнать, Царя, Который все видит.

\vs 9Sb 1:80 Вот за такие дела придет к вам жгучее пламя,

\vs 9Sb 1:81 Будет вас вечно палить огонь, который не гаснет,

\vs 9Sb 1:82 И постыдитесь тогда обмана негодных кумиров.

\vs 9Sb 1:83 Те же, кто Господа чтут, воистину сущего Бога,

\vs 9Sb 1:84 Вечную жизнь обретут, и в жизни той бесконечной

\vs 9Sb 1:85 Все, поселившись в Раю, в саду, прекрасно цветущем,

\vs 9Sb 1:86 Сладостный хлеб вкусят, со звездного посланный неба \ldots

\bibbookdescr{10Sb}{
  inline={\LARGE Десятая книга\\\Huge Сивилл},
  toc={10-я Сивилл},
  bookmark={10 Sybille},
  header={10-я Сивилл},
  abbr={10~Сив}
}
\vs 10Sb 1:1 Тогда, как прахом всё уляжется в земле,

\vs 10Sb 1:2 И Бог, оставив мир без света в мрачной мгле,

\vs 10Sb 1:3 Из праха воззовет и кости человека,

\vs 10Sb 1:4 И смертных оживит для будущего века,

\vs 10Sb 1:5 Тогда последует правдивый Божий суд,

\vs 10Sb 1:6 Где по заслугам все достойное найдут:

\vs 10Sb 1:7 Нечестие, порок закроются землею,

\vs 10Sb 1:8 Но святость и любовь возвысятся над нею,

\vs 10Sb 1:9 Когда Превечный Судия изволит дать

\vs 10Sb 1:10 Благочестивым дух и жизнь и благодать.

\vs 10Sb 1:11 Тогда узнает верно всякий сам себя,

\vs 10Sb 1:12 И узрит же тогда преясно всяк себя \ldots

\bibbookdescr{Tho}{
  inline={Евангелие от Фомы},
  toc={От Фомы},
  bookmark={От Фомы},
  header={От Фомы},
  abbr={Фм}
}

\vs Tho 1:0
Это тайные слова,
которые сказал Иисус живой
и которые записал Дидим Иуда Фома.
И он сказал:
тот, кто обретает истолкование этих слов, не вкусит смерти.

\vs Tho 1:1
Иисус сказал: пусть тот, кто ищет,
не перестаёт искать до тех пор, пока не найдёт,
и, когда он найдёт, он будет потрясён,
и, если он потрясён, он будет удивлён,
и он будет царствовать над всем.

\vs Tho 1:2
Иисус сказал: если те, которые ведут вас, говорят вам:
смотрите, царствие в небе!~--- тогда птицы небесные
опередят вас.
Если они говорят вам, что оно~--- в море,
тогда рыбы опередят вас.
Но царствие внутри вас и вне вас.

\vs Tho 1:3
Когда вы познаете себя,
тогда вы будете познаны и вы узнаете,
что вы~--- дети Отца живого.
Если же вы не познаете себя,
тогда вы в бедности и вы~--- бедность.

\vs Tho 1:4
Иисус сказал: старый человек в его дни
не замедлит спросить малого ребёнка 7-ми дней о месте жизни,
и он будет жить.
Ибо много первых будут последними, и они станут одним.

\vs Tho 1:5
Иисус сказал: познай того, кто перед лицом твоим,
и тот, кто скрыт от тебя,~--- откроется тебе.
Ибо нет ничего тайного, что не будет явным.

\vs Tho 1:6
Ученики его спросили его;
они сказали ему: хочешь ли ты, чтобы мы постились,
и как нам молиться, давать милостыню и воздерживаться в пище?
Иисус сказал: не лгите, и то, что вы ненавидите, не делайте этого.
Ибо всё открыто перед небом.
Ибо нет ничего тайного, что не будет явным,
и нет ничего сокровенного, что осталось бы нераскрытым.

\vs Tho 1:7
Иисус сказал: блажен тот лев, которого съест человек,
и лев станет человеком.
И проклят тот человек, которого съест лев, и лев станет человеком.

\vs Tho 1:8
И он сказал: человек подобен мудрому рыбаку,
который бросил свою сеть в море.
Он вытащил её из моря, полную малых рыб;
среди них этот мудрый рыбак нашёл большую хорошую рыбу.
Он выбросил всех малых рыб в море, он без труда выбрал большую рыбу.
Тот, кто имеет уши слышать, да слышит!

\vs Tho 1:9
Иисус сказал:
вот, сеятель вышел, он наполнил свою руку, он бросил семена.
Но иные упали на дорогу, прилетели птицы, поклевали их.
Иные упали на камень, и не пустили корня в землю,
и не послали колоса в небо.
И иные упали в терния, они заглушили семя, и червь съел их.
И иные упали на добрую землю и дали добрый плод в небо.
Это принесло 60 мер на одну и 120 мер на одну.

\vs Tho 1:10
Иисус сказал:
я бросил огонь в мир, и вот я охраняю его, пока он не запылает.

\vs Tho 1:11
Иисус сказал:
это небо прейдёт, и то, что над ним, прейдёт,
и те, которые мертвы, не живы, и те, которые живы, не умрут.

\vs Tho 1:12
В те дни вы ели мёртвое, вы делали его живым.
Когда вы окажетесь в свете, что вы будете делать?
В этот день вы~--- одно, вы стали двое.
Когда же вы станете двое, что вы будете делать?

\vs Tho 1:13
Ученики сказали Иисусу:
мы знаем, что ты уйдешь от нас.
Кто тот, который будет б\acc{о}льшим над нами?
Иисус сказал им:
в том месте, куда вы пришли, вы пойдёте к Иакову праведному,
из-за которого возникли небо и земля.

\vs Tho 1:14
Иисус сказал ученикам своим:
уподобьте меня, скажите мне, на кого я похож.
Симон Пётр сказал ему:
ты похож на ангела праведного.
Матфей сказал ему: ты похож на философа мудрого.
Фома сказал ему:
Господи, мои уста никак не примут сказать, на кого ты похож.
Иисус сказал:
я не твой господин, ибо ты выпил,
ты напился из источника кипящего, который я измерил.
И он взял его, отвёл его и сказал ему 3 слова.
Когда же Фома пришёл к своим товарищам, они спросили его:
что сказал тебе Иисус?
Фома сказал им: если я скажу вам одно из слов,
которые он сказал мне, вы возьмёте камни, бросите в меня,
огонь выйдет из камней и сожжёт вас.

\vs Tho 1:15
Иисус сказал:
если вы поститесь, вы зародите в себе грех,
и, если вы молитесь, вы будете осуждены,
и, если вы подаете милостыню, вы причините зло вашему духу.
И если вы приходите в какую-то землю и идёте в селения,
если вас примут, ешьте то, что вам поставят.
Тех, которые среди них больны, лечите.
Ибо то, что войдёт в ваши уста, не осквернит вас,
но то, что выходит из ваших уст, это вас осквернит.

\vs Tho 1:16
Иисус сказал:
когда вы увидите того, который не рождён женщиной,
падите ниц и почитайте его; он~--- ваш Отец.

\vs Tho 1:17
Иисус сказал:
может быть, люди думают, что я пришёл бросить мир в космос,
и они не знают, что я пришёл бросить на землю разделения,
огонь, меч, войну.
Ибо 5-ро будут в доме: 3-ое будут против 2-их и 2-ое против 3-их.
Отец против сына и сын против отца; и они будут стоять как единственные.

\vs Tho 1:18
Иисус сказал: я дам вам то, чего не видел глаз,
и то, чего не слышало ухо,
и то, чего не коснулась рука,
и то, что не вошло в сердце человека.

\vs Tho 1:19
Ученики сказали Иисусу:
скажи нам, каким будет наш конец.
Иисус сказал: открыли ли вы начало, чтобы искать конец?
Ибо в месте, где начало, там будет конец.
Блажен тот, кто будет стоять в начале:
и он познает конец, и он не вкусит смерти.

\vs Tho 1:20
Иисус сказал: блажен тот, кто был до того, как возник.

\vs Tho 1:21
Если вы у меня ученики и если слушаете мои слова,
эти камни будут служить вам.

\vs Tho 1:22
Ибо есть у вас 5 деревьев в раю,
которые неподвижны и летом и зимой,
и их листья не опадают.
Тот, кто познает их, не вкусит смерти.

\vs Tho 1:23
Ученики сказали Иисусу:
скажи нам, чему подобно царствие небесное.
Он сказал им: оно подобно зерну горчичному,
самому малому среди всех семян.
Когда же оно падает на возделанную землю,
оно даёт большую ветвь и становится укрытием для птиц небесных.

\vs Tho 1:24
Мария сказала Иисусу:
на кого похожи твои ученики?
Он сказал: они похожи на детей малых,
которые расположились на поле, им не принадлежащем.
Когда придут хозяева поля, они скажут: оставьте нам наше поле.
Они обнажаются перед ними, чтобы оставить это им и дать им их поле.

\vs Tho 1:25
Поэтому я говорю:
если хозяин дома знает, что приходит вор,
он будет бодрствовать до тех пор, пока он не придёт,
и он не позволит ему проникнуть в его дом царствия его,
чтобы унести его вещи.
Вы же бодрствуйте перед миром, препояшьте ваши чресла
с большой силой, чтобы разбойники не нашли пути пройти к вам.
Ибо нужное, что вы ожидаете, будет найдено.

\vs Tho 1:26
Да был бы среди вас знающий человек!
Когда плод созрел, он пришёл поспешно,~--- его
серп в руке его,~--- и он убрал его.
Тот, кто имеет уши слышать, да слышит!

\vs Tho 1:27
Иисус увидел младенцев, которые сосали молоко.
Он сказал ученикам своим:
эти младенцы, которые сосут молоко, подобны тем,
которые входят в царствие.
Они сказали ему:
что же, если мы~--- младенцы, мы войдём в царствие?
Иисус сказал им: когда вы сделаете воих одним,
и когда вы сделаете внутреннюю сторону как внешнюю сторону,
и внешнюю сторону как внутреннюю сторону,
и верхнюю сторону как нижнюю сторону,
и когда вы сделаете мужчину и женщину одним,
чтобы мужчина не был мужчиной и женщина не была женщиной,
когда вы сделаете глаз\acc{а} вместо гл\acc{а}за,
и руку вместо руки,
и ногу вместо ноги,
образ вместо образа,~--- тогда вы войдёте.

\vs Tho 1:28
Иисус сказал:
я выберу вас 1-го на 1000-чу и 2-их на 10000, и они будут стоять как 1.

\vs Tho 1:29
Ученики его сказали:
покажи нам место, где ты, ибо нам необходимо найти его.
Он сказал им: тот, кто имеет уши, да слышит!
Есть свет внутри человека света, и он освещает весь мир.
Если он не освещает, то~--- тьма.

\vs Tho 1:30
Иисус сказал:
люби брата твоего, как душу твою.
Охраняй его как зеницу ока твоего.

\vs Tho 1:31
Иисус сказал:
сучок в глазе брата твоего ты видишь,
бревна же в твоём глазе ты не видишь.
Когда ты вынешь бревно из твоего глаза,
тогда ты увидишь, как вынуть сучок из глаза брата твоего.

\vs Tho 1:32
Если вы не поститесь от мира, вы не найдёте царствия.
Если не делаете субботу субботой, вы не увидите Отца.

\vs Tho 1:33
Иисус сказал:
я встал посреди мира, и я явился им во плоти.
Я нашёл всех их пьяными, я не нашёл никого из них жаждущим,
и душа моя опечалилась за детей человеческих.
Ибо они слепы в сердце своём и они не видят,
что они приходят в мир пустыми;
они ищут снова уйти из мира пустыми.
Но теперь они пьяны.
Когда они отвергнут своё вино, тогда они покаются.

\vs Tho 1:34
Иисус сказал:
если плоть произошла ради духа, это~--- чудо.
Если же дух ради тела, это~--- чудо из чудес.
Но я, я удивляюсь тому,
как такое большое богатство заключено в такой бедности.

\vs Tho 1:35
Иисус сказал:
там, где 3 бога, там боги.
Там, где 2 или 1, я с ним.

\vs Tho 1:36
Иисус сказал:
нет пророка, принятого в своём селении.
Не лечит врач тех, которые знают его.

\vs Tho 1:37
Иисус сказал:
город, построенный на высокой горе, укреплённый,
не может пасть, и он не может быть тайным.

\vs Tho 1:38
Иисус сказал:
то, что ты услышишь твоим ухом,
возвещай это другому уху с ваших кровель.
Ибо никто не зажигает светильника
и не ставит его под сосуд
и никто не ставит его в тайное место,
но ставит его на подставку для светильника,
чтобы все, кто входит и выходит, видели его свет.

\vs Tho 1:39
Иисус сказал: если слепой ведет слепого, оба падают в яму.

\vs Tho 1:40
Иисус сказал: невозможно, чтобы кто-то вошёл в дом сильного и
взял его силой, если он не свяжет его руки.
Тогда он разграбит дом его.

\vs Tho 1:41
Иисус сказал:
не заботьтесь с утра до вечера и с вечера до утра о том,
что вы наденете на себя.

\vs Tho 1:42
Ученики его сказали:
в какой день ты явишься нам и в какой день мы увидим тебя?
Иисус сказал:
когда вы обнажитесь и не застыдитесь и возьмёте ваши одежды,
пол\acc{о}жите их у ваших ног, подобно малым детям,
растопчете их, тогда вы увидите сына того, кто жив,
и вы не будете бояться.

\vs Tho 1:43
Иисус сказал:
много раз вы желали слышать эти слова, которые я вам говорю,
и у вас нет другого, от кого слышать их.
Наступят дни~---- вы будете искать меня, но не найдёте меня.

\vs Tho 1:44
Иисус сказал:
фарисеи и книжники взяли ключи от знания.
Они спрятали их и не вошли и не позволили тем,
которые хотят войти.
Вы же будьте мудры, как змии, и чисты, как голуби.

\vs Tho 1:45
Иисус сказал:
виноградная лоза была посажена без Отца, и она не укрепилась.
Её выкорчуют, она погибнет.

\vs Tho 1:46
Иисус сказал:
тот, кто имеет в своей руке,~--- ему дадут;
и тот, у кого нет, то малое, что имеет,~--- у него возьмут.

\vs Tho 1:47
Иисус сказал: будьте странниками.

\vs Tho 1:48
Ученики его сказали ему:
кто ты, который говоришь нам это?
Иисус сказал им:
из того, что я вам говорю, вы не узнаёте, кто я?
Но вы стали как иудеи,
ибо они любят дерево и ненавидят его плод,
они любят плод и ненавидят дерево.

\vs Tho 1:49
Иисус сказал:
тот, кто высказал хулу на Отца,~--- ему простится,
и тот, кто высказал хулу на Сына,~--- ему простится.
Но тот, кто высказал хулу на Духа святого,~--- ему
не простится ни на земле, ни на небе.

\vs Tho 1:50
Иисус сказал:
не собирают винограда с терновника
и не пожинают смокв с верблюжьих колючек.
Они не дают плода.
Добрый человек выносит доброе из своего сокровища.
Злой человек выносит злое из своего злого сокровища,
которое в его сердце, и он говорит злое,
ибо из избытка сердца он выносит злое.

\vs Tho 1:51
Иисус сказал:
от Адама до Иоанна Крестителя из рождённых жёнами
нет выше Иоанна Крестителя \ldots\ Но я сказал:
тот из вас, кто станет малым, познает царствие и будет выше Иоанна.

\vs Tho 1:52
Иисус сказал:
невозможно человеку сесть на двух коней,
натянуть два лука,
и невозможно рабу служить двум господам:
или он будет почитать одного и другому он будет грубить.
Ни один человек, который пьёт старое вино,
тотчас не стремится выпить вино молодое.
И не наливают молодое вино в старые мехи,
дабы они не разорвались,
и не наливают старое вино в новые мехи,
дабы они не испортили его.
Не накладывают старую заплату на новую одежду,
ибо произойдёт разрыв.

\vs Tho 1:53
Иисус сказал:
если двое в мире друг с другом в одном и том же доме,
они скажут горе: переместись!~--- и она переместится.

\vs Tho 1:54
Иисус сказал:
блаженны единственные и избранные, ибо вы найдёте царствие,
ибо вы от него, вы снова туда возвратитесь.

\vs Tho 1:55
Иисус сказал:
если вам говорят: откуда вы произошли?~--- скажите им:
мы пришли от света, от места,
где свет произошёл от самого себя.
Он \ldots\ в их образ.
Если вам говорят: кто вы?~--- скажите: мы его дети,
и мы избранные Отца живого.
Если вас спрашивают:
каков знак вашего Отца, который в вас?~--- скажите им:
это движение и покой.

\vs Tho 1:56
Ученики его сказали ему:
в какой день наступит покой тех, которые мертвы?
И в какой день новый мир приходит?
Он сказал им:
тот покой, который вы ожидаете, пришёл,
но вы не познали его.

\vs Tho 1:57
Ученики его сказали ему:
24 пророка высказались в Израиле, и все они сказали о тебе.
Он сказал им:
вы оставили того, кто жив перед вами, и вы сказали о тех, кто мёртв.

\vs Tho 1:58
Ученики его сказали ему:
обрезание полезно или нет?
Он сказал им:
если бы оно было полезно, их отец зачал бы их в их матери обрезанными.
Но истинное обрезание в духе обнаружило полную пользу.

\vs Tho 1:59
Иисус сказал: блаженны бедные, ибо ваше~--- царствие небесное.

\vs Tho 1:60
Иисус сказал:
тот, кто не возненавидел своего отца и свою мать,
не сможет быть моим учеником,
и тот, кто не возненавидел своих братьев и своих сестёр
и не понес свой крест, как я, не станет достойным меня.

\vs Tho 1:61
Иисус сказал:
тот, кто познал мир, нашёл труп,
и тот, кто нашёл труп~--- мир недостоин его.

\vs Tho 1:62
Иисус сказал:
царствие Отца подобно человеку, у которого хорошие семена.
Его враг пришёл ночью, высеял плевел вместе с хорошими семенами.
Человек не позволил им вырвать плевел.
Он сказал им:
не приходите, чтобы, вырывая плевел,
вы не вырвали пшеницу вместе с ним!
Ибо в день жатвы плевелы появятся, их вырвут и их сожгут.

\vs Tho 1:63
Иисус сказал: блажен человек, который потрудился: он нашёл жизнь.

\vs Tho 1:64
Иисус сказал:
посмотрите на того, кто жив, пока вы живёте,
дабы вы не умерли,~--- ищите увидеть его!
И вы не сможете увидеть самаритянина,
который несёт ягнёнка и входит в Иудею.
Он сказал ученикам своим: почему он с ягнёнком?
Они сказали ему: чтобы убить его и съесть его.
Он сказал им: пока он жив, он его не съест,
но если он убивает его, и ягнёнок становится трупом.
Они сказали: иначе он не сможет ударить.
Он сказал им: вы также ищите себе место в покое,
дабы вы не стали трупом и вас не съели.

\vs Tho 1:65
Иисус сказал:
двое будут отдыхать на ложе: один умрёт, другой будет жить.
Саломея сказала: кто ты, человек, и чей ты сын?
Ты взошёл на моё ложе, и ты поел за моим столом.
Иисус сказал ей:
я тот, который произошёл от того, который равен;
мне дано принадлежащее моему Отцу.
Саломея сказала: я твоя ученица.
Иисус сказал ей: поэтому я говорю следующее:
когда он станет пустым, он наполнится светом,
но, когда он станет разделённым, он наполнится тьмой.

\vs Tho 1:66
Иисус сказал:
я говорю мои тайны \ldots\ тайна.
То, что твоя правая рука будет делать,~--- пусть
твоя левая рука не знает того, что она делает.

\vs Tho 1:67
Иисус сказал:
был человек богатый, у которого было много добра.
Он сказал: я использую моё добро,
чтобы засеять, собрать, насадить,
наполнить мои амбары плодами,
дабы мне не нуждаться ни в чём.
Вот о чём он думал в сердце своём.
И в ту же ночь он умер.
Тот, кто имеет уши, да слышит!

\vs Tho 1:68
Иисус сказал:
у человека были гости, и, когда он приготовил ужин,
он послал своего раба, чтобы он пригласил гостей.
Он пошёл к первому, он сказал ему:
мой господин приглашает тебя.
Он сказал: у меня деньги для торговцев, они придут ко мне вечером,
я пойду и дам им распоряжение: я отказываюсь от ужина.
Он пошёл к другому, он сказал ему: мой господин пригласил тебя.
Он сказал ему: я купил дом, и меня просят днём.
У меня не будет времени.
Он пошёл к другому, он сказал ему: мой господин приглашает тебя.
Он сказал ему: мой друг будет праздновать свадьбу,
и я буду устраивать ужин. Я не смогу прийти. Я отказываюсь от ужина.
Он пошёл к другому, он сказал ему: мой господин приглашает тебя.
Он сказал ему: я купил деревню, я пойду собирать доход.
Я не смогу прийти. Я отказываюсь.
Раб пришёл, он сказал своему господину:
те, кого ты пригласил на ужин, отказались.
Господин сказал своему рабу:
пойди на дороги, кого найдёшь, приведи их, чтобы они поужинали.
Покупатели и торговцы не войдут в места моего отца.

\vs Tho 1:69
Он сказал: У доброго человека был виноградник; он отдал его
работникам, чтобы они обработали его и чтобы он получил его плод от них. Он
послал своего раба, чтобы работники дали ему плод виноградника. Те схватили его
раба, они избили его, еще немного~--- и они убили бы его. Раб пришёл, он
рассказал своему господину. Его господин сказал: Может быть, они его не узнали
(в оригинале: Может быть, он их не узнал). Он послал другого раба. Работники
побили этого. Тогда хозяин послал своего сына. Он сказал: Может быть, они
постыдятся моего сына. Эти работники, когда узнали, что он наследник
виноградника, схватили его, они убили его. Тот, кто имеет уши, да слышит!

\vs Tho 1:70
Иисус сказал:
покажи мне камень, который строители отбросили!
Он~--- краеугольный камень.

\vs Tho 1:71
Иисус сказал:
тот, кто знает всё, нуждаясь в самом себе, нуждается во всём.

\vs Tho 1:72
Иисус сказал:
блаженны вы, когда вас ненавидят и вас преследуют.
И не найдут места там, где вас преследовали.

\vs Tho 1:73
Иисус сказал:
блаженны те, которых преследовали в их сердце;
это те, которые познали Отца в истине.
Блаженны голодные, потому что чрево того, кто желает, будет насыщено.

\vs Tho 1:74
Иисус сказал:
когда вы рождаете это в себе, то, что вы имеете, спасёт вас.
Если вы не имеете этого в себе, то, чего вы не имеете в себе,
умертвит вас.

\vs Tho 1:75
Иисус сказал:
я разрушу этот дом, и нет никого, кто сможет построить его ещё раз.

\vs Tho 1:76
Некий человек сказал ему:
скажи моим братьям, чтобы они разделили вещи моего отца со мной.
Он сказал ему: о человек, кто сделал меня тем, кто делит?
Он повернулся к своим ученикам, сказал им:
да не стану я тем, кто делит!

\vs Tho 1:77
Иисус сказал:
жатва обильна, работников же мало.
Просите же господина, чтобы он послал работников на жатву.

\vs Tho 1:78
Он сказал:
Господи, много вокруг источника, но никого нет в источнике.

\vs Tho 1:79
Иисус сказал:
многие стоят перед дверью, но единственные те,
которые войдут в брачный чертог.

\vs Tho 1:80
Иисус сказал:
царствие Отца подобно торговцу, имеющему товары,
который нашёл жемчужину.
Этот торговец~--- мудрый:
он продал товары и купил себе одну жемчужину.
Вы также~--- ищите его сокровище, которое не гибнет,
которое остаётся там, куда не проникает моль, чтобы съесть,
и не губит червь.

\vs Tho 1:81
Иисус сказал:
я~--- свет, который на всех.
Я~--- всё: всё вышло из меня и всё вернулось ко мне.
Разруби дерево, я~--- там;
подними камень, и ты найдёшь меня там.

\vs Tho 1:82
Иисус сказал:
почему вы пошли в поле?
чтобы видеть тростник, колеблемый ветром, и видеть человека,
носящего на себе мягкие одежды?
Смотрите, ваши цари и ваши знатные люди~--- это они носят на себе мягкие
одежды и они не смогут познать истину!

\vs Tho 1:83
Женщина в толпе сказала ему:
блаженно чрево, которое выносило тебя, и груди, которые вскормили тебя.
Он сказал ей:
блаженны те, которые услышали слово Отца и сохранили его в истине.
Ибо придут дни, вы скажете:
блаженно чрево, которое не зачало, и груди, которые не дали молока.

\vs Tho 1:84
Иисус сказал:
тот, кто познал мир, нашёл тело,
но тот, кто нашёл тело,~--- мир недостоин его.

\vs Tho 1:85
Иисус сказал:
тот, кто сделался богатым, пусть царствует,
и тот, у кого сила, пусть откажется.

\vs Tho 1:86
Иисус сказал:
тот, кто вблизи меня, вблизи огня,
и кто вдали от меня, вдали от царствия.

\vs Tho 1:87
Иисус сказал:
образы являются человеку, и свет, который в них, скрыт.
В образе света Отца он откроется,
и его образ скрыт из-за его света.

\vs Tho 1:88
Иисус сказал:
когда вы видите ваше подобие, вы радуетесь.
Но когда вы видите ваши образы,
которые произошли до вас,~--- они не умирают
и не являются~--- сколь великое вы перенесёте?

\vs Tho 1:89
Иисус сказал:
Адам произошёл от большой силы и большого богатства,
и он недостоин вас.
Ибо \ldots\ смерти.

\vs Tho 1:90
Иисус сказал:
лисицы имеют свои норы, и птицы имеют свои гнезда,
а Сын человеческий не имеет места,
чтобы преклонить свою голову и отдохнуть.

\vs Tho 1:91
Иисус сказал:
несчастно тело, которое зависит от тела,
и несчастна душа, которая зависит от них обоих.

\vs Tho 1:92
Иисус сказал:
ангелы приходят к вам и пророки, и они дадут вам то,
что ваше, и вы также дайте им то, что в ваших руках,
и скажите себе:
в какой день они приходят и берут то, что принадлежит им?

\vs Tho 1:93
Иисус сказал:
почему вы моете внутри чаши и не понимаете того,
что тот, кто сделал внутреннюю часть,
сделал также внешнюю часть?

\vs Tho 1:94
Иисус сказал:
придите ко мне, ибо иго моё~--- благо и власть моя кротка,
и вы найдёте покой себе.

\vs Tho 1:95
Они сказали ему:
скажи нам, кто ты, чтобы мы поверили в тебя.
Он сказал им: вы испытываете лицо неба и земли;
и того, кто перед вами,~--- вы не познали его;
и это время~--- вы не знаете, как испытать его.

\vs Tho 1:96
Иисус сказал:
ищите и вы найдете, но те вещи,
о которых вы спросили меня в те дни,~--- я не сказал вам тогда.
Теперь я хочу сказать их, но вы не ищете их.

\vs Tho 1:97
Не давайте того, что свято, собакам,
чтобы они не бросили это в навоз.
Не бросайте жемчуга свиньям, чтобы они не сделали это \ldots

\vs Tho 1:98
Иисус сказал:
тот, кто ищет, найдёт, и тот, кто стучит, ему откроют.

\vs Tho 1:99
Иисус сказал:
если у вас есть деньги, не давайте в рост,
но дайте \ldots\ от кого вы не возьмёте их.

\vs Tho 1:100
Иисус сказал:
царствие Отца подобно женщине, которая взяла немного закваски,
положила это в тесто и разделила это в большие хлебы.
Кто имеет уши, да слышит!

\vs Tho 1:101
Иисус сказал:
царствие подобно женщине, которая несёт сосуд,
полный муки, идёт удаляющейся дорогой.
Ручка сосуда разбилась, мука рассыпалась позади неё на дороге.
Она не знала, она не поняла, как действовать.
Когда она достигла своего дома,
она поставила сосуд на землю и нашла его пустым.

\vs Tho 1:102
Иисус сказал:
царствие Отца подобно человеку,
который хочет убить сильного человека.
Он извлёк меч в своём доме, он вонзил его в стену,
дабы узнать, будет ли рука его крепка.
Тогда он убил сильного.

\vs Tho 1:103
Ученики сказали ему:
твои братья и твоя мать стоят снаружи.
Он сказал им:
те, которые здесь,
которые исполняют волю моего Отца,~--- мои братья и моя мать.
Они те, которые войдут в царствие моего Отца.

\vs Tho 1:104
Иисусу показали золотой и сказали ему:
те, кто принадлежит Цезарю, требуют от нас подати.
Он сказал им: Дайте Цезарю то, что принадлежит Цезарю,
дайте Богу то, что принадлежит Богу,
и то, что моё, дайте это мне!

\vs Tho 1:105
Тот, кто не возненавидел своего отца и свою мать,
как я, не может быть моим учеником,
и тот, кто не возлюбил своего отца и свою мать,
как я, не может быть моим учеником.
Ибо моя мать \ldots\ но поистине она дала мне жизнь.

\vs Tho 1:106
Иисус сказал: Горе им, фарисеям! Ибо они похожи на собаку,
которая спит на кормушке быков. Ибо она и не ест и не дает есть быкам.

\vs Tho 1:107
Иисус сказал:
блажен человек, который знает, в какую пору приходят разбойники,
так что он встанет, соберёт \ldots\ и препояшет свои чресла,
прежде чем они придут.

\vs Tho 1:108
Они сказали:
пойдём помолимся сегодня и попостимся.
Иисус сказал:
каков же грех, который я совершил или которому я поддался?
Но когда жених выйдет из чертога брачного,
тогда пусть они постятся и пусть молятся!

\vs Tho 1:109
Иисус сказал:
тот, кто познает отца и мать,~--- его назовут сыном блудницы.

\vs Tho 1:110
Иисус сказал:
когда вы сделаете 2-ух 1-им, вы станете Сыном человеческим,
и, если вы скажете гор\acc{е}: сдвинься, она переместится.

\vs Tho 1:111
Иисус сказал: царствие подобно пастуху, у которого 100 овец.
Одна из них, самая большая, заблудилась.
Он оставил 99 и стал искать одну, пока не нашёл её.
После того как он потрудился, он сказал овце:
я люблю тебя больше, чем 99.

\vs Tho 1:112
Иисус сказал:
тот, кто напился из моих уст, станет как я.
Я также, я стану им, и тайное откроется ему.

\vs Tho 1:113
Иисус сказал:
царствие подобно человеку,
который имеет на своём поле тайное сокровище,
не зная о нём.
И он не нашёл до того, как умер, он оставил его своему сыну.
Сын не знал; он получил это поле и продал его.
И тот, кто купил его, пришёл, раскопал и нашёл сокровище.
Он начал давать деньги под проценты тем, кому он хотел.

\vs Tho 1:114
Иисус сказал:
тот, кто нашёл мир и стал богатым, пусть откажется от мира!

\vs Tho 1:115
Иисус сказал:
небеса, как и земля, свернутся перед вами,
и тот, кто живой от живого, не увидит смерти.
Ибо Иисус сказал: тот, кто нашёл самого себя,~--- мир недостоин его.

\vs Tho 1:116
Иисус сказал:
горе той плоти, которая зависит от души;
горе той душе, которая зависит от плоти.

\vs Tho 1:117
Ученики его сказали ему:
в какой день царствие приходит?
Оно не приходит, когда ожидают.
Не скажут: Вот, здесь!~--- или:
Вот, там!~--- Но царствие Отца распространяется по земле,
и люди не видят его.

\vs Tho 1:118
Симон Пётр сказал им:
пусть Мария уйдёт от нас, ибо женщины недостойны жизни.
Иисус сказал:
смотрите, я направлю её, дабы сделать её мужчиной,
чтобы она также стала духом живым, подобным вам, мужчинам.
Ибо всякая женщина, которая станет мужчиной, войдёт в царствие небесное.

\bibbookdescr{Did}{
  inline={Учение двенадцати апостолов},
  toc={Учение 12-и апостолов},
  bookmark={Учение 12-и апостолов},
  header={Учение 12-и апостолов},
  abbr={Дид}
}
\chhdr{Учение Господа народам через двенадцать апостолов}
\vs Did 1:1
Есть два пути: один~--- жизни и один~--- смерти,
но между обоими путями большое различие.
\vs Did 1:2
Путь жизни таков.
Во-первых, ты должен любить Бога, создавшего тебя;
во-вторых~--- ближнего своего, как себя самого;
и всего того, чего не хочешь, чтобы было с тобою,
и ты не делай другому.

\vs Did 1:3
Слов же сих учение таково: благословляйте проклинающих вас и
молитесь за врагов ваших, поститесь за гонящих вас, ибо какая
вам за то благодарность, если вы любите любящих вас?
Не то же ли делают и язычники?
Вы же любите ненавидящих вас и не будете иметь врага.
\vs Did 1:4
Удаляйся от плотских и мирских похотей.
Если кто ударит тебя в правую щёку, обрати к нему и другую и будешь совершен.
Если кто наймёт тебя на одну милю, иди с ним две.
Если кто отнимет у тебя верхнюю одежду, отдай и хитон.
Если кто возьмет у тебя твоё, не требуй назад, да и не сможешь.
\vs Did 1:5
Всякому, просящему у тебя, давай и не требуй назад, ибо Отец
хочет чтобы всё подаваемо было из его даров.
Блажен дающий по заповеди, ибо он неповинен.
Горе принимающему, ибо если кто, имея нужду,
принимает, тот будет неповинен, если же кто принимает,
не имея нужды, тот даст отчёт, почему принял и на что:
подвергшись же заключению, испытан будет относительно того,
что сделал, и не выйдет оттуда, пока не отдаст последнего кодранта.
\vs Did 1:6
Но и о сём также сказано: пусть милостыня твоя запотеет
в руках твоих, пока ты не узнаешь, кому дать.

\vs Did 2:1
Вторая же заповедь учения:
\vs Did 2:2
Не убивай,
не прелюбодействуй,
не совершай деторастления,
не будь блудником,
не кради,
не занимайся магией,
не изготавливай волшебных снадобий,
не умерщвляй дитя в зародыше и рожденного не убивай,
не пожелай достояния ближнего твоего.
\vs Did 2:3
Не клянись,
не лжесвидетельствуй,
не злословь,
не злопамятствуй.
\vs Did 2:4
Не будь двоедушным и двуязычным,
ибо двуязычие есть сеть смерти.
\vs Did 2:5
Да не будет слово твоё лживым и пустым, но преисполненным дела.
\vs Did 2:6
Не будь
ни корыстолюбивым,
ни хищником,
ни лицемером,
ни злобным,
ни надменным,
не принимай лукавого умысла на ближнего своего.
\vs Did 2:7
Не имей ненависти ни к одному человеку, но одних обличай, за
других молись, а иных люби более души своей.

\vs Did 3:1
Чадо моё!
Беги от всякого зла и от
всего\fnote{всего}{\vsep\ дела.}
подобного ему.
\vs Did 3:2
Не будь
ни гневливым, ибо гнев ведет к убийству,
ни ревнивым,
ни сварливым,
ни запальчивым, ибо от всего этого рождаются убийства.

\vs Did 3:3
Чадо моё!
Не будь ни похотником, ибо похоть ведет к блуду,
ни срамословом,
ни бесстыжеглазым, ибо от всего этого рождаются прелюбодеяния.

\vs Did 3:4
Чадо моё!
Не гадай по полёту птиц, ибо птицегадание ведет к идолослужению,
не заговаривай,
не занимайся математикой,
ни очищениями,
не желай смотреть на это, ибо от всего этого рождается идолослужение.

\vs Did 3:5
Чадо моё!
Не будь ни лживым, поелику ложь доводит до воровства,
ни сребролюбцем,
ни тщеславным, ибо от всего этого рождается воровство.

\vs Did 3:6
Чадо моё!
Не будь ни ропотником, ибо ропот доводит до богохульства,
ни своенравным,
ни лукавомыслящим, ибо от всего этого рождаются богохульства.
\vs Did 3:7
Но будь кротким, ибо кроткие наследуют землю.
\vs Did 3:8
Будь долготерпеливым, и милостивым, и незлобивым, и смиренным,
и благим, и всегда трепещущим от слов, которые услышал.
\vs Did 3:9
Не превозносись и не допускай в душе своей дерзости.
Да не прилепится душа твоя к гордым,
но обращайся с праведными и смиренными.
\vs Did 3:10
То, что случается с тобой, принимай как благо,
зная, что без Бога ничего не пргоисходит.

\vs Did 4:1
Чадо моё!
Возвещающего тебе Слово Божие помни день и ночь,
почитай же его, как Господа, ибо где возвещается господство,
там Господь есть.
\vs Did 4:2
Даже ищи каждый день иметь личное общение со святыми,
чтобы ты почивал на словах учения их.
\vs Did 4:3
Не производи разделения, а примиряй спорящих; суди праведно, не
будь лицеприятен при обличении преступлений.
\vs Did 4:4
Не думай двоедушно, так или нет.

\vs Did 4:5
Не будь протягивающим руки для принятия подаяний,
но сжимающим её для подаяния.
\vs Did 4:6
Если ты имеешь, что подать от труда рук твоих,
то дай выкуп за грехи твои.
\vs Did 4:7
Не колеблись подать и, подавая, не ропщи,
ибо ты должен знать, кто добрый Мздовоздаятель.
\vs Did 4:8
Не отвращайся от нуждающегося (ср. Сир.4:5), но во всем будь
общником с братом твоим и \bibemph{ничего} не называй своим,
ибо если вы соучастники в нетленном, то насколько более в тленном!
\vs Did 4:9
Не отнимай руки твоей от сына твоего или от дочери твоей,
но от юности учи их страху Божию.

\vs Did 4:10
В гневе твоём не отдавай приказаний рабу твоему или служанке твоей,
надеющимся на того же Бога, дабы они никогда не перестали бояться Бога,
сущего над обоими вами, ибо он пришел призвать ко спасению,
не по лицу судя, а тех коих уготовал дух.
\vs Did 4:11
Вы же, рабы, подчиняйтесь господам своим, как образу Божию,
по совести и со страхом.

\vs Did 4:12
Ненавидь всякое лицемерие и всё, что неугодно Господу.
\vs Did 4:13
Не оставляй заповедей Господа, но храни то, что принял,
не прибавляя и не убавляя.
\vs Did 4:14
В церкви исповедуй преступления свои и не приступай к молитве
своей в лукавой совести.
Этот путь есть путь жизни.

\vs Did 5:1
Путь же смерти таков.
Прежде всего он лукав и исполнен проклятия.
Убийства, прелюбодеяния, вожделения, блуд, кражи, идолослужение,
магия, изготовления снадобий, ограбления, лжесвидетельства, лицемерия,
двоедушие, хитрость, гордыня, злоба, самодовольство, любостяжание,
сквернословие, ревнование, дерзость, высокомерие, бахвальство, бесстрашие.
\vs Did 5:2
гонители благих, ненавистники истины, любители лжи,
не признающие воздаяния за праведность,
не привязывающиеся к благому, ни к суду праведному,
бдящие не во благо, но в зло; от которых далеки кротость и
терпение, любящие суетное, гоняющиеся за мздовоздаянием,
не милующие нищего, не болеззнующие об удрученном,
не вещающие Создавшего их, убийцы детей, губители создания Божия,
отвращающиеся от нуждающегося, обременяющие угнетенного,
заступники богатых, беззаконные судьи бедных, всегрешные.
О если бы вы, чада, избавились от всех таких!

\vs Did 6:1
Смотри, чтобы кто не совратил тебя с этого пути учения, ибо
таковой учит тебя вне Бога.
\vs Did 6:2
Ибо если ты сможешь понести всё иго Господне, то будешь
совершен, а если не можешь, то делай то, что можешь.
\vs Did 6:3
Относительно пищи понеси то, что можешь, но крепко
воздерживайся от идоложертвенного, ибо это есть служение
мёртвым богам.

\vs Did 7:1
Что же \bibemph{касается} крещения,
крестите так: преподав наперед всё это,
крестите во имя Отца и Сына и Святого Духа
в проточной воде.
\vs Did 7:2
Если же нет проточной воды, окрести в иной воде,
а если не можешь в холодной~--- в теплой.
\vs Did 7:3
Если же нет ни той, ни другой, то возлей воду на голову трижды
во имя Отца и Сына и Святого Духа.
\vs Did 7:4
А пред крещением пусть постятся крещающий и крещаемый и, если
могут, некоторые другие; вели же \bibemph{обязательно} поститься
крещаемому день или два до \bibemph{крещения}.

\vs Did 8:1
Посты же ваши да не будут с лицемерами:
они постятся во второй и пятый день недели,
вы же поститесь в четвертый и шестой.
\vs Did 8:2
И не молитесь, как лицемеры, но как повелел Господь в Евангелии своём,
так молитесь:
Отче наш, Cущий на Небе!
Да святится Имя твоё;
да приидет Царствие Твоё;
да будет Воля твоя и на земле, как на Небе;
хлеб наш насущный дай нам на сей день,
и оставь нам долг наш, как и мы оставляем должникам нашим,
и не введи нас в искушение,
но избавь нас от лукавого,
потому что твоя есть сила и слава во веки.
\vs Did 8:3
Трижды в день молитесь так.

\vs Did 9:1
Что же касается евхаристии, совершайте ее так.
\vs Did 9:2
Сперва о чаше:
Благодарим тебя, Отче наш, за святой виноград Давида,
отрока твоего, который виноград ты открыл нам чрез Иисуса, отрока твоего.
Тебе слава во веки!
\vs Did 9:3
О хлебе же ломимом: Благодарим Тебя, Отче наш, за жизнь и знание,
которые ты открыл нам чрез Иисуса, отрока твоего.
Тебе слава во веки.
\vs Did 9:4
Как сей преломляемый хлеб был рассеян по холмам и собранный
вместе стал единым, так и экклесия твоя от концов земли
да соберется в Царствие твоё, ибо твоя есть слава и сила
чрез Иисуса Христа во веки.
\vs Did 9:5
И от евхаристии вашей никто да не вкушает и не пьет, кроме
крещенных во имя Господне, ибо и о сем сказал Господь:
не давайте святыни псам.

\vs Did 10:1
По исполнении же вкушения так благодарите:
\vs Did 10:2
Благодарим тебя, Отче святый, за имя твоё святое,
которое ты вселил в сердцах наших,
и за знание, и веру, и бессмертие,
которые ты открыл нам чрез Иисуса, отрока твоего.
Тебе слава во веки!
\vs Did 10:3
Ты, Владыко Вседержитель, сотворил всё ради имени твоего,
пищу же и питие дал людям в наслаждение,
чтобы они благодарили тебя, а нам даровал духовную пищу и питие,
и жизнь вечную чрез отрока твоего.
\vs Did 10:4
Более всего благодарим тебя потому, что ты всемогущ.
Тебе слава во веки!
\vs Did 10:5
Помяни, Господи, экклесию твою, да избавишь её от всякого зла
и усовершишь её в любви твоей, и от четырёх ветров собери её,
освящённую, в царство твоё, которое ты уготовал ей,
потому что Твоя есть сила и слава во веки.
\vs Did 10:6
Да приидет благодать и да прейдёт мир сей.
Осанна Богу Давидову!
Кто свят, да приступает, кто нет, пусть покается.
Маранат. Аминь.
\vs Did 10:7
Пророкам же позволяйте совершать евхаристию когда захотят.

\vs Did 11:1
Кто, пришедши, будет учить вас всему этому,
пред сим сказанному, примите его.
\vs Did 11:2
Если же сам учащий, совратившись, будет преподавать иное
учение, к ниспровержению, не слушайте его; но если для
преумножения правды и познания Господа, примите его, как Господа.
\vs Did 11:3
Относительно же апостолов и пророков поступайте
согласно учению евангельскому.
\vs Did 11:4
Всякий апостол, приходящий к вам, пусть будет принят, как Господь.
\vs Did 11:5
Но пусть он не остаётся более одного дня, а если будет нужда,
то и другой \bibemph{день}, но если он пробудет три дня, то лжепророк.
\vs Did 11:6
Уходя же, апостол пусть ничего не принимает, кроме хлеба,
до \bibemph{следуюшего} места ночлега;
а если он будет требовать серебра, то он лжепророк.
\vs Did 11:7
И всякого пророка, говорящего в духе, не испытывайте и не
судите, ибо всякий грех отпустится,
а этот грех не отпустится.
\vs Did 11:8
Но не всякий, говорящий в духе,~--- пророк,
но \bibemph{только} тот, кто хранит пути Господни;
так что по путям распознаётся и лжепророк и пророк.
\vs Did 11:9
И никакой пророк, в духе определяющий быть трапезе,
не вкушает от неё, а если не так, то он лжепророк.
\vs Did 11:10
Всякий пророк, учащий истине, если он сам не делает того, чему
учит, есть лжепророк.
\vs Did 11:11
Но всякий пророк, признанный истинным, вступающий в мирское
таинство экклесии, но не учащий делать то, что сам делает,
не должен быть судим вами, ибо он суд имеет у Бога,
ибо так поступали и древние пророки.
\vs Did 11:12
Если же кто в духе скажет: дай мне серебра или чего другого,
вы не должны слушать того; но если он назначит подаяние для других, неимущих,
то никто да не осуждает его.

\vs Did 12:1
Всякий, приходящий во имя Господне, да будет принят, а потом,
уже испытав его,
вы узнаете,~--- ибо вы будете иметь разумение,~--- правого и ложного.
\vs Did 12:2
Если приходящий~--- странник, помогите ему, сколько можете,
но он не должен оставаться у вас более двух или трёх дней,
и то если бы нужда оказалась.
\vs Did 12:3
Если же он желает поселиться у вас, то, если он ремесленник,
пусть трудится и ест.
\vs Did 12:4
А если он не знает ремесла, то вы по своему усмотрению
позаботьтесь о нём, но так, чтобы христианин не жил среди вас праздным.
\vs Did 12:5
Если же он не желает так поступать, то он христоторговец.
Остерегайтесь таковых!

\vs Did 13:1
А всякий истинный пророк, желающий поселиться у вас,
достоин своего пропитания.
\vs Did 13:2
Точно так же и истинный учитель, и он достоин, как трудящийся,
своего пропитания.
\vs Did 13:3
Поэтому всякий начаток от произведений точила и гумна, от волов
и овец возьми и отдай пророкам, ибо они ваши архиереи.
\vs Did 13:4
Если же вы не имеете пророка, то отдайте начаток нищим.
\vs Did 13:5
Если ты приготовишь пищу, то, взявши начаток, отдай его по
заповеди.
\vs Did 13:6
Точно так же если ты открыл сосуд вина или елея, то возьми
начаток и отдай пророкам.
\vs Did 13:7
И от серебра, и от одежды, и от всякого имения возьми начаток,
сколько тебе угодно, и отдай его по заповеди.

\vs Did 14:1
Каждый день\fnote{Каждый день}{\vsep\
В день Господень, \bibemph{т.е. в субботу}.},
собравшись, преломите хлеб и благодарите,
исповедав прежде согрешения ваши дабы жертва ваша была чиста.
\vs Did 14:2
Всякий же, имеющий распрю с другом своим, да не приходит вместе
с вами, пока они не примирятся, чтобы не осквернилась жертва ваша.
\vs Did 14:3
Ибо о ней сказал Господь: на всяком месте и во всякое время
должно приносить мне жертву чистую, потому что я царь великий,
говорит Господь, и имя мое чудно в народах.

\vs Did 15:1
Рукополагайте себе старейшин и прислужников, достойных Господа,
мужей кротких и несребролюбивых, и истинных, и испытанных,
ибо и они исполняют для вас служение пророков и учителей.
\vs Did 15:2
Поэтому не презирайте их, ибо они~--- почтенные ваши
наравне с пророками и учителями.

\vs Did 15:3
Обличайте друг друга, но не во гневе, а в мире, как имеете в
евангелии, и со всяким, поступающим оскорбительно по отношению к другому,
пусть никто не говорит и никто у вас не слушает его, пока не покается.
\vs Did 15:4
Молитва же ваша и милостыня, и все вообще добрые дела творите
так, как имеете в евангелии Господа нашего.

\vs Did 16:1
Бодрствуйте относительно жизни вашей;
светильники ваши да не будут погашены,
и чресла ваши не препоясаны,
но будьте готовыми, ибо вы не знаете часа,
в который Господь ваш приидет.
\vs Did 16:2
Вы должны часто собираться вместе, исследуя, что потребно душам
вашим, ибо не принесёт вам пользы всё время вашей веры,
если не сделаетесь совершенными в последний час.
\vs Did 16:3
Ибо в последние дни умножатся лжепророки и губители, и овцы
превратятся в волков, и любовь превратится в ненависть.
\vs Did 16:4
Ибо, когда возрастёт беззаконие, люди будут ненавидеть друг
друга и преследовать, и тогда явится мирообольститель,
как бы Сын Божий, и совершит знамения и чудеса, и земля
предана будет в руки его, и сотворит беззакония,
каких никогда не было от века.
\vs Did 16:5
Тогда тварь человеческая пойдет в огонь испытания и многие
соблазнятся и погибнут, а устоявшие в вере своей спасутся
от проклятия его\fnote{от проклятия его}{\vsep\ этим самим проклятием.}.
\vs Did 16:6
И тогда явится знамение истины:
во-первых, знамение отверстия на небе,
потом знамение звука трубного
и третье~--- воскресение мертвых.
\vs Did 16:7
Но не всех вместе, а как сказано: приидет Господь и все святые с ним.
\vs Did 16:8
Тогда увидит мир Господа, грядущего на облаках небесных.

\bibbookdescr{1Cl}{
  inline={Послание Климента к Коринфянам},
  toc={1-е Климента},
  bookmark={1-е Климента},
  header={1-е Климента},
  abbr={1~Кли}
}
\vs 1Cl 1:1
Церковь Божья,
находящаяся в Риме, Церкви Божьей, находящейся в Коринфе, званным, освященным
по воле Божьей через Господа нашего Иисуса Христа.
\vs 1Cl 1:2
Благодать вам и мир от
Всемогущего Бога чрез Иисуса Христа да умножится.
\vs 1Cl 1:3
Внезапные и одно за другим случившиеся с нами
несчастия и бедствия были причиною того, братья, что поздно, как думается нам,
обратили мы внимание на спорные у вас дела, возлюбленные, и на неприличный и
чуждый избранникам Божьим, преступный и нечестивый мятеж,
\vs 1Cl 1:4
который немногие дерзкие и высокомерные люди
разожгли до такого безумия, что почтенное, славное и для всех достолюбезное
имя ваше подверглось великому поруганию.
\vs 1Cl 1:5
Ибо кто, побывавший у вас, не хвалил вашей,
всеми добродетелями исполненной и твердой, веры, не удивлялся вашему
трезвенному и кроткому во Христе благочестию, не превозносил вашей великой
щедрости в гостеприимстве, не прославлял вашего совершенного и верного знания?
\vs 1Cl 1:6
Во всем вы поступали нелицеприятно, ходили в
заповедях Божьих, повинуясь предводителям вашим и воздавая должную честь
старшим между вами.
\vs 1Cl 1:7
Юношам внушили скромность и благопристойность;
жен наставляли, чтобы они все делали с неукоризненной, честной и чистой
совестью, любя, как должно, своих мужей,
\vs 1Cl 1:8
и учили их, чтобы они, не выступая из правила
повиновения, пристойно распоряжались домашними делами, и вели себя вполне
целомудренно.

\vs 1Cl 2:1
Все вы были смиренны и
чужды тщеславия, любили более подчиняться, нежели повелевать, и давать, нежели
принимать.
\vs 1Cl 2:2
Довольствуясь тем, что Бог
дал вам на путь, и тщательно внимая словам Его, вы хранили их в глубине
сердца, и страдания Его были пред очами вашими.
\vs 1Cl 2:3
Таким образом всем был
дарован глубокий и прекрасный мир и ненасытное стремление делать добро: и на
всех было полное излияние Святого Духа.
\vs 1Cl 2:4
Полные святых желаний, с
искренним усердием и благочестивым упованием, вы простирали руки свои ко
Всемогущему Богу и умоляли Его быть милосердым, если вы в чем невольно
погрешили.
\vs 1Cl 2:5
День и ночь подвигом вашим
было попечение о всем братстве, чтобы число избранных Его спасалось в
добродушии и единомыслии.
\vs 1Cl 2:6
Вы были искренни,
чистосердечны, и не помнили зла друг на друге.
\vs 1Cl 2:7
Всякий мятеж и всякое
разделение было вам противно.
\vs 1Cl 2:8
Вы плакали о проступках
ближних; их недостатки считали собственными.
\vs 1Cl 2:9
Не скучали делать добро,
готовые на всякое дело доброе.
\vs 1Cl 2:10
Будучи украшены такою
добродетельною и почтенною жизнью, вы все совершали в страхе Господа: Его
повеления и заповеди были написаны на скрижалях сердца вашего.

\vs 1Cl 3:1
Вся слава и широта дана
была вам, и исполнилось, что написано: он ел и пил, разжирел и растолстел, и
сделался непокорен возлюбленный.
\vs 1Cl 3:2
А отсюда ревность и
зависть, вражда и раздор, гонение и возмущение, война и плен.
\vs 1Cl 3:3
Таким образом, люди
бесчестные восстали против почтенных, бесславные против славных, глупые против
разумных, молодые против старших.
\vs 1Cl 3:4
Поэтому удалились правда и
мир,~--- так как всякий оставил страх Божий, сделался туп в вере Его, не ходит
по правилам заповедей Его, и не ведет жизни, достойной Христа,
\vs 1Cl 3:5
но каждый последовал злым
своим пожеланиям, допустив снова беззаконную и нечестивую зависть, чрез
которую и смерть вошла в мир.

\vs 1Cl 4:1
Ибо так написано: и было
спустя несколько дней, приносил Каин от плодов земли жертву Богу: и также
Авель приносил от первородных овец и от туков их;
\vs 1Cl 4:2
и призрел Бог на Авеля и
на дары Его; на Каина же и на жертвы его не посмотрел.
\vs 1Cl 4:3
И весьма опечалился Каин,
и поникло лицо его.
\vs 1Cl 4:4
И сказал Бог Каину: что ты
стал печален, и от чего поникло лицо твое? Не согрешил ли ты, если ты
правильно принес, но неправильно разделил?
\vs 1Cl 4:5
Успокойся. К тебе
обращение его и ты будешь обладать тем.
\vs 1Cl 4:6
И сказал Каин Авелю, брату
своему: пойдем в поле;
\vs 1Cl 4:7
и было в то время, как они
находились в поле, возстал Каин на Авеля, брата своего, и убил его.
\vs 1Cl 4:8
Видите, братья, ревность и
зависть произвели братоубийство.
\vs 1Cl 4:9
По причине зависти отец
наш Иаков убежал от лица Исава, брата своего.
\vs 1Cl 4:10
Зависть была причиною,
что Иосиф гоним был на смерть и подвергся рабству.
\vs 1Cl 4:11
Зависть принудила Моисея
бежать от лица фараона, царя Египетского, когда услышал он от единоплеменника
своего:
\vs 1Cl 4:12
кто поставил тебя
решителем или судьею над нами? Не хочешь ли убить меня, как убил вчера
египтянина?
\vs 1Cl 4:13
За зависть Аарон и
Мариамь жили вне стана.
\vs 1Cl 4:14
Зависть Дафана и Авирона
живых низвела в ад за то, что они возмутились против Моисея, служителя
Божьего.
\vs 1Cl 4:15
По причине зависти Давид
не только подвергся ненависти иноплеменных, но был гоним и от Саула, царя
Израильского.

\vs 1Cl 5:1
Но оставив древние
примеры, перейдем к ближайшим подвижникам: возьмем достойные примеры нашего
поколения.
\vs 1Cl 5:2
По ревности и зависти
величайшие и праведные столпы подверглись гонению и смерти. Представим пред
глазами нашими блаженных апостолов.
\vs 1Cl 5:3
Петр от беззаконной
зависти понес не одно, не два, но многие страдания, и таким образом
претерпевши мученичество, отошел в подобающее место славы.
\vs 1Cl 5:4
Павел, по причине зависти,
получил награду за терпение: он был в узах семь раз, был изгоняем, побиваем
камнями.
\vs 1Cl 5:5
Будучи проповедником на
Востоке и Западе, он приобрел благородную славу за свою веру, так как научил
весь мир правде,
\vs 1Cl 5:6
и доходил до границы
Запада, и мученически засвидетельствовал истину перед правителями.
\vs 1Cl 5:7
Так он переселился из
мира, и перешел в место святое, сделавшись величайшим образцом терпения.

\vs 1Cl 6:1
К этим мужам, свято
провождавшим жизнь, присовокупилось великое множество избранных,
\vs 1Cl 6:2
которые по причине зависти
претерпели многие поругания и мучения, и оставили среди нас прекрасный пример.
\vs 1Cl 6:3
Завистью гонимы были
женщины Данаида и Дирка;
\vs 1Cl 6:4
претерпевши тяжкие и
ужасные мучения, они прошли твердым путем веры, и, немощные телом, получили
славную награду.
\vs 1Cl 6:5
Зависть отлучала жен от
мужей и извращала слова праотца нашего Адама: вот ныне кость от костей моих,
и плоть от плоти моей.
\vs 1Cl 6:6
Зависть и раздор
ниспровергли великие города и совершенно истребили великие народы.

\vs 1Cl 7:1
Это, возлюбленные, пишем
мы не только для вашего наставления, но и для собственного напоминания;
\vs 1Cl 7:2
потому что мы находимся на
том же поприще, и тот же подвиг предлежит нам.
\vs 1Cl 7:3
Итак, оставим пустые и
суетные заботы, и обратимся к славному и досточтимому правилу святого звания
нашего.
\vs 1Cl 7:4
Будем смотреть на то, что
добро, что угодно и приятно Создателю нашему.
\vs 1Cl 7:5
Обратим внимание на кровь
Христа,~--- и увидим, как драгоценна пред Богом кровь Его, которая была пролита
для нашего спасения, и всему миру принесла благодать покаяния.
\vs 1Cl 7:6
Пройдем все поколения и
узнаем, что Господь в каждом поколении милостиво принимал покаяние желавших
обратиться к Нему.
\vs 1Cl 7:7
Ной проповедовал покаяние,
и послушавшиеся его спаслись.
\vs 1Cl 7:8
Иона возвестил ниневитянам
погибель, но они, раскаявшись в своих грехах, умилостивили Бога своими
молитвами и получили спасение, хотя были далеки от Бога.

\vs 1Cl 8:1
Служители благодати
Божьей по вдохновению Духа Святого говорили о покаянии;
\vs 1Cl 8:2
и Сам Владыка всего
говорил о покаянии с клятвою: жив Я, говорит ЯХВЕ, не хочу смерти грешника,
но покаяния;
\vs 1Cl 8:3
и присовокупил еще
следующую прекрасную мысль: дом Израилев, обратитесь от нечестия вашего.
\vs 1Cl 8:4
Скажи сынам народа Моего:
хотя грехи ваши будут простираться от земли до неба, и хотя будут краснее
червленицы и чернее власяницы,
\vs 1Cl 8:5
но если вы обратитесь ко
Мне от всего сердца, и скажете: Отец! то Я услышу вас как народ святой.
\vs 1Cl 8:6
И в другом месте так
говорит: омойтесь, и очиститесь, удалите лукавство из душ ваших пред очами
Моими, отстаньте от злодейств ваших;
\vs 1Cl 8:7
научитесь делать добро,
ищите правды, избавьте обиженного, рассудите о сироте, оправдайте вдовицу, и
придите и будем судиться, говорит ЯХВЕ:
\vs 1Cl 8:8
и если будут грехи ваши,
как пурпур, то убелю их как снег; и если будут как червленица, то убелю их как
волну; и если хотите и послушаете Меня, то будете наслаждаться благами земли;
\vs 1Cl 8:9
если же не хотите и не
послушаете Меня, то меч истребит вас: ибо уста ЯХВЕ сказали это.
\vs 1Cl 8:10
Итак, Он всех Своих
возлюбленных хочет сделать участниками покаяния, и утвердил это всемогущею
Своею волею!

\vs 1Cl 9:1
Поэтому покоримся
величественной и славной воле Его, и, оставив суетные дела, раздор и зависть,
ведущую к смерти, припадем и обратимся к Его милосердию, умоляя Его милость и
благость.
\vs 1Cl 9:2
Будем постоянно взирать на
тех, которые совершенно послужили величественной Его славе.
\vs 1Cl 9:3
Возьмем Еноха, который по
своему послушанию был найден праведным, и преставился, и не видели его смерти.
\vs 1Cl 9:4
Ной был найден верным, и
по своему служению проповедал миру обновление, и через него спас Господь
животных, согласно вошедших в ковчег.

\vs 1Cl 10:1
Авраам, названный другом,
найден верным по своему послушанию словам Божьим.
\vs 1Cl 10:2
Он из послушания вышел из
земли своей, и от родства своего, и из дома отца своего, чтобы оставить землю
малую, родство малосильное и небольшой дом, наследовал обетованию Божьему.
\vs 1Cl 10:3
Ибо так сказал ему:
удались из земли твоей, и от родства твоего, и из дома отца твоего в землю,
которую покажут тебе.
\vs 1Cl 10:4
И сделаю тебя народом
великим; и благословлю тебя, и возвеличу имя твое, и будешь благословен.
\vs 1Cl 10:5
И благословлю
благословляющих тебя, и проклинающих тебя прокляну, и благословятся в тебе все
племена земные.
\vs 1Cl 10:6
И опять по разделении его
с Лотом сказал ему Бог: подними глаза твои, и взгляни с места, где ты теперь,
к северу и югу, и к востоку и к морю: ибо всю землю, которую ты видишь, отдам
тебе и семени твоему на век.
\vs 1Cl 10:7
И сделаю семя твое как
песок земной: если кто может сосчитать песок земной, то и семя твое сочтется.
\vs 1Cl 10:8
И еще сказано: вывел Бог
Авраама и сказал ему: взгляни на небо и сосчитай звезды, если можешь счесть
их: так будет семя твое.
\vs 1Cl 10:9
И поверил Авраам Богу, и
это вменилось ему в праведность.
\vs 1Cl 10:10
За веру и гостеприимство
был дан ему в старости сын, но он из послушания принес его в жертву Богу на
одной из показанных от Него гор.

\vs 1Cl 11:1
За гостеприимство и
благочестие Лот вышел невредимым из Содома, тогда как вся окрестная страна
была наказана огнем и серою:
\vs 1Cl 11:2
и тем ясно показал
Господь, что Он не оставляет уповающих на Него; а уклоняющихся от Него
подвергает мучениям и казни.
\vs 1Cl 11:3
Ибо вышедшая с ним жена
его, так как была других мыслей и не согласна с ним, поставлена в знамение:
\vs 1Cl 11:4
она сделалась соляным
столбом, и даже до сего дня, чтобы все знали, что двоедушные и сомневающиеся о
могуществе Божьем служат примером суда и знамением для всех родов.

\vs 1Cl 12:1
За веру и гостеприимство
была спасена Раав блудница.
\vs 1Cl 12:2
Когда Иисусом Нуном были
посланы соглядатаи в Иерихон, и царь земли той узнал, что они пришли
осматривать его землю, то послал людей схватить их, чтобы схватив, предать их
смерти.
\vs 1Cl 12:3
Но гостеприимная Раав,
приняв их к себе, скрыла на верху своего дома в снопах льна.
\vs 1Cl 12:4
И когда от царя явились к
ней и говорили: люди пришли к тебе, соглядатаи земли нашей, выведи их, так
повелевает царь,
\vs 1Cl 12:5
то она отвечала: приходили
ко мне два человека, которых вы ищете, но они скоро ушли, и теперь в пути;
таким образом, она не показала их посланным.
\vs 1Cl 12:6
А мужам тем сказала: знаю
верно, что ЯХВЕ, Бог ваш, предаст вам этот город; потому что страх и трепет от
вас напал на живущих в нем. Итак, когда удастся вам взять его, сохраните меня
и дом отца моего.
\vs 1Cl 12:7
А они сказали ей: будет
так, как ты сказала нам. Как скоро узнаешь о приближении нашем, собери всех
своих под кровлю твою и будут целы, а кто будет найден вне дома, погибнет.
\vs 1Cl 12:8
Притом дали ей знак, чтобы
она свесила из дома своего красную вервь,~--- и тем показали, что всем верующим
и уповающим на Бога будет искупление кровью Господа.
\vs 1Cl 12:9
Видите, возлюбленные, в
этой жене была не только вера, но и пророчество.

\vs 1Cl 13:1
Итак, будем смиренны,
братья, отложив всякое надмение, гордость, неразумие и гнев, и будем
поступать, как написано.
\vs 1Cl 13:2
Ибо говорит Дух Святой:
да не похвалится мудрый мудростью своей, ни сильный силой своей, ни богатый
богатством своим,
\vs 1Cl 13:3
но хвалящийся пусть
хвалится ЯХВЕ, ища Его, и творя суд и правду.
\vs 1Cl 13:4
Особенно будем помнить
слова Господа Иисуса, которые изрек Он, научая кротости и великодушию.
\vs 1Cl 13:5
Он так сказал: милуйте,
чтобы быть помилованными, отпускайте, дабы вам было отпущено;
\vs 1Cl 13:6
как вы делаете, так вам
будут делать; как даете, так вам дано будет;
\vs 1Cl 13:7
как судите, так сами
судимы будете; как будете снисходить, так к вам будут снисходить; какою мерою
мерите, такою отмерится вам.
\vs 1Cl 13:8
Этой заповедью и этими
внушениями утвердим себя, чтобы ходить со смирением, повинуясь святым
повелениям Его.
\vs 1Cl 13:9
Ибо святое слово говорит:
на кого воззрю,~--- только на кроткого и тихого, и трепещущего слов Моих.

\vs 1Cl 14:1
Итак, праведное и святое
дело, братья, более повиноваться Богу, нежели последовать тем, которые в
надменности и кичливости стали предводителями презренной зависти.
\vs 1Cl 14:2
Ибо не малому вреду, а
напротив, подвергнемся великой опасности, если опрометчиво отдадим себя на
волю тех людей, которые подстрекают нас к раздору и мятежам, чтобы отвести нас
от добродетели.
\vs 1Cl 14:3
Будем снисходительны друг
к другу, как милосерд и благ Сотворивший нас;
\vs 1Cl 14:4
ибо написано: добрые
будут обитателями земли, и невинные останутся на ней; а беззаконные истребятся
с нее.
\vs 1Cl 14:5
И опять говорит Писание:
я видел нечестивого превозносящегося и возвышающегося, как кедры ливанские; и
прошел я мимо, и вот его уже не стало, и искал я места его, и не нашел.
\vs 1Cl 14:6
Храни невинность и
соблюдай правоту, потому что мирного человека ожидают добрые последствия.

\vs 1Cl 15:1
Итак, присоединимся к
тем, которые с благочестием хранят мир, а не к тем, которые с лицемерием
желают мира;
\vs 1Cl 15:2
ибо сказано где-то: эти
люди почитают Меня устами, сердце же их далеко отстоит от Меня.
\vs 1Cl 15:3
И в другом месте: устами
своими они благословляли, а сердцем своим проклинали.
\vs 1Cl 15:4
И еще сказано: возлюбили
Его устами своими, и языком своим солгали Ему; сердце же их не было право с
Ним; и они не были верны в завете Его.
\vs 1Cl 15:5
Да будут немы уста
льстивые, и да истребит Господь уста льстивых и язык велеречивый,~--- тех,
которые говорят: язык наш возвеличим, уста наши при нас: кто нам Господь?
\vs 1Cl 15:6
Ради бедствий нищих, и
воздыхания убогих, ныне Я восстану, говорит ЯХВЕ: послужу им спасением, и буду
поступать с ними честно.

\vs 1Cl 16:1
Ибо Христос принадлежит
смиренным, а не тем, которые возносятся над стадом Его.
\vs 1Cl 16:2
Жезл величия Божьего,
Господь наш Иисус Христос, не пришел в блеске великолепия и надменности, хотя
и мог бы, но смиренно, как сказал о Нем Дух Святой.
\vs 1Cl 16:3
Ибо говорит Он: ЯХВЕ, кто
верил слуху нашему? и рука ЯХВЕ кому открылась?
\vs 1Cl 16:4
Мы возвестили пред Ним; Он
как малый отрок, как корень в земле жаждущей,~--- не имеет ни вида, ни славы.
\vs 1Cl 16:5
И мы видели Его, и не имел
Он ни вида, ни красоты; но вид Его бесчестен, унижен более вида человеков: Он
человек в язве и страдании, умеющий переносить болезнь;
\vs 1Cl 16:6
потому что отвратилось
лицо Его,~--- было поругано и презрено. Он грехи наши носит и за нас страдает;
\vs 1Cl 16:7
а мы думали, что Он
праведно подвержен страданию, и язве, и мучению;
\vs 1Cl 16:8
но Он уязвлен был за грехи
наши и мучен был за беззакония наши; наказание мира нашего на Нем, чрез рану
Его мы исцелились.
\vs 1Cl 16:9
Все мы, как овцы,
заблудились; человек блуждал на пути своем, и ЯХВЕ предал Его за грехи наши.
\vs 1Cl 16:10
И Он, будучи мучим, не
отверзает уст: как овца был веден на заклание, и как агнец безгласный пред
стригущим его, так Он не отверзает уст Своих. За смирение Его с Него снят был
суд.
\vs 1Cl 16:11
Кто расскажет Его род,
когда жизнь Его берется от земли? За беззакония людей Моих Он идет на смерть.
\vs 1Cl 16:12
И потому помилую злых за
гроб Его, и богатых за смерть Его; ибо Он не сделал беззакония и обмана не
нашлось в устах Его.
\vs 1Cl 16:13
И ЯХВЕ угодно очистить
Его от язвы; если дадите жертву о грехе, то душа ваша узрит семя долговечное.
\vs 1Cl 16:14
И Господь хочет спасти
Его от страдания души Его, показать Ему свет и образовать разумом, и оправдать
праведного, который благодетельно послужил многим; и грехи их Он понесет.
\vs 1Cl 16:15
Поэтому Он будет обладать
многими и разделит добычи сильных,~--- за то, что предана была на смерть душа
Его и был причтен к злодеям; и Он уничтожил грехи многих и за беззакония их
был предан.
\vs 1Cl 16:16
И опять Он же говорит: Я
червь, а не человек, поношение человеков и уничижение людей.
\vs 1Cl 16:17
Все видящие Меня
издевались надо Мною, говорили устами и кивали головою, говоря: Он уповал на
Господа, пусть избавит Его и сохранит Его, так как благоволит к Нему.
\vs 1Cl 16:18
Видите возлюбленные,
какой дан нам образец: ибо если Господь так смирил Себя, то что должны делать
мы, которые чрез Него пришли под иго благодати Его?

\vs 1Cl 17:1
Будем подражать и тем,
которые скитались в козьих и овечьих кожах, проповедуя о пришествии Христовом:
\vs 1Cl 17:2
разумеем пророков Илию,
Елисея и Иезекииля, также и тех, которые получили прекрасное свидетельство.
\vs 1Cl 17:3
Авраам получил великое
свидетельство, и назван другом Божьим: но, взирая на славу Божью, со смирением
говорит: я земля и пепел.
\vs 1Cl 17:4
Далее и об Иове так
написано: Иов был праведен и непорочен, истинен и благочестив, и удалялся от
всякого зла.
\vs 1Cl 17:5
Но он, сам себя осуждая,
сказал: никто не чист от скверны, хотя бы и один день была жизнь Его.
\vs 1Cl 17:6
Моисей назван верным во
всем доме его, и Бог через его служение совершил суд над Египтом посредством
мучений и казней:
\vs 1Cl 17:7
но и он, столько
прославленный, не величался, но, когда из купины было к нему Божественное
слово, сказал: кто я, что Ты меня посылаешь? Я заика и косноязычен. И опять
говорит: я пар из котла.

\vs 1Cl 18:1
Что же скажем о
прославленном Давиде, о котором сказал Бог: Я нашел человека по сердцу Моему,
Давида сына Иессеева, милостью вечной Я помазал его?
\vs 1Cl 18:2
Но и он говорит Богу:
помилуй меня, Боже, по великой милости Твоей, и по множеству щедрот Твоих
очисти беззаконие мое.
\vs 1Cl 18:3
Еще более~--- омой меня от
неправды моей, и очисти меня от греха моего, ибо я знаю неправду свою и грех
мой всегда предо мною.
\vs 1Cl 18:4
Тебе одному согрешил я и
пред Тобою сделал зло, чтобы Ты оправдался в словах Твоих и победил, когда
станут судить Тебя.
\vs 1Cl 18:5
Ибо в беззакониях зачат я
и в грехах родила меня мать моя.
\vs 1Cl 18:6
Ты возлюбил истину:
сокровенные тайны премудрости Твоей Ты открыл мне.
\vs 1Cl 18:7
Окропи меня иссопом, и
буду чист, омой меня, и буду белее снега.
\vs 1Cl 18:8
Слуху моему дай радость и
веселье: и сокрушенные кости мои возрадуются.
\vs 1Cl 18:9
Отврати лицо от грехов
моих, и изгладь все беззакония мои.
\vs 1Cl 18:10
Создай во мне сердце
чистое, Боже, и дух правый обнови в утробе моей.
\vs 1Cl 18:11
Не отвергни меня от лица
Твоего, и Духа Твоего Святого не отними от меня.
\vs 1Cl 18:12
Воздай мне радость
спасения Твоего, и укрепи меня Духом Адонаи.
\vs 1Cl 18:13
Научу грешников путям
Твоим и нечестивые обратятся к Тебе.
\vs 1Cl 18:14
Избавь меня от пролития
крови, Боже, Бог спасения моего. Язык мой воспоет правду Твою.
\vs 1Cl 18:15
Адонай, открой уста мои,
и уста мои возвестят хвалу Твою.
\vs 1Cl 18:16
Если бы Ты восхотел иной
жертвы, я принес бы; но всесожжения Тебе неугодны.
\vs 1Cl 18:17
Жертва Богу~--- дух
сокрушенный; сердце сокрушенное и смиренное Бог не презрит.

\vs 1Cl 19:1
Смирение и послушливая
покорность этих мужей, получивших столь славное свидетельство от Самого Бога,
сделали лучшими не только нас, но и прежде бывшие поколения,
\vs 1Cl 19:2
именно тех, которые со
страхом и искренностью принимали глаголы Его.
\vs 1Cl 19:3
Итак, имея пред собою
столь многие великие и славные деяния, обратимся к цели мира, указанной нам
изначала,
\vs 1Cl 19:4
и взирая к Отцу и
Создателю всего мира, вникнем в Его величественные и превосходные дары мира и
в Его благодеяния.
\vs 1Cl 19:5
Воззрим на Него умом и
душевными очами, посмотрим на долготерпение Его воли, и помыслим, как Он
кроток ко всему творению Своему.

\vs 1Cl 20:1
Небеса, по Его
распоряжению движущиеся, в мире повинуются Ему: и день и ночь совершают
определенное им течение, не препятствуя друг другу.
\vs 1Cl 20:2
Солнце и лики звезд, по
Его велению, согласно, без малейшего уклонения проникают на назначенные им
пути.
\vs 1Cl 20:3
Плодоносящая земля, по Его
воле, в определенные времена производит изобильную пищу человекам, зверям и
всем находящимся на ней животным, не замедляя и не изменяя ничего из
предписанного им.
\vs 1Cl 20:4
Неисследуемые и
непостижимые области бездны и преисподней держатся теми же велениями.
\vs 1Cl 20:5
Беспредельное море, по Его
устроению совокупленное в большие водные массы, не выступает за положенные ему
преграды, но делает так, как Он повелел.
\vs 1Cl 20:6
Ибо Он сказал: доселе
дойдешь, и волны твои в тебе сокрушатся.
\vs 1Cl 20:7
Непроходимый для людей
океан, и миры за ним находящиеся, управляются теми же повелениями Господа.
\vs 1Cl 20:8
Времена года~--- весна,
лето, осень и зима мирно сменяются одни другими.
\vs 1Cl 20:9
Определенные ветры, каждый
в свое время, беспрепятственно совершают свое служение.
\vs 1Cl 20:10
Неиссякающие источники,
созданные для наслаждения и здравия, непрестанно доставляют людям свою влагу,
необходимую для их жизни.
\vs 1Cl 20:11
Наконец, малейшие
животные мирно и согласно составляют сожительства между собою.
\vs 1Cl 20:12
Всему этому повелел быть
в согласии и мире великий Создатель и Владыка всего,
\vs 1Cl 20:13
Который благотворит всем,
а преимущественно нам, которые прибегли к милосердию Его чрез Господа нашего
Иисуса Христа, Которому слава и величие во веки веков. Аминь.

\vs 1Cl 21:1
Смотрите, возлюбленные,
чтобы столь многие благодеяния Его не обратились всем нам в осуждение, если
мы, живя достойно Его, не будем единодушно совершать благое и угодное Ему.
\vs 1Cl 21:2
Ибо сказано где-то: Дух
Господа есть светильник, испытующий тайны утробы.
\vs 1Cl 21:3
Помыслим, как Он близок к
нам, и что ни одна из наших мыслей или совещаний, какие мы делаем, не закрыты
от Него.
\vs 1Cl 21:4
Итак, надлежит нам не
отступать от воли Его: лучше воспротивимся глупым и несмысленным,
превозносящимся и хвалящимся пышностью слова своего людям, нежели Богу.
\vs 1Cl 21:5
Будем благоговеть перед
Господом Иисусом Христом, кровь Которого предана за нас, будем почитать
предстоятелей наших, уважать пресвитеров, юношей воспитывать в страхе Божьем,
\vs 1Cl 21:6
жен своих направлять к
добру, чтобы они отличались достолюбезным нравом целомудрия, показывали чистое
свое расположение к кротости, скромность языка своего обнаруживали молчанием,
любовь свою оказывали не по склонностям, но равную ко всем, свято боящимся
Бога.
\vs 1Cl 21:7
Дети ваши пусть получают
воспитание христианина; пусть научаются, как сильно пред Богом смирение, что
значит пред Богом чистая любовь, как прекрасен и велик страх Божий и
спасителен для всех, свято ходящих в нем с чистым умом.
\vs 1Cl 21:8
Ибо Он есть испытатель
мыслей и желаний наших: Его дыхание в нас, и когда захочет, возьмет его.

\vs 1Cl 22:1
Все сие подтверждает вера
христианская. Ибо Сам Христос чрез Духа Святого так взывает к нам:
\vs 1Cl 22:2
приходите, дети,
послушайте Меня; страху ЯХВЕ научу вас.
\vs 1Cl 22:3
Кто есть человек, хотящий
жизни, любящий видеть дни благие?
\vs 1Cl 22:4
Удержи язык твой от зла, и
уста твои, чтобы не говорить коварства.
\vs 1Cl 22:5
Уклонись от зла и сотвори
доброе; ищи мира, и гонись за ним.
\vs 1Cl 22:6
Очи ЯХВЕ~--- на праведных, и
уши Его~--- на молитву их:
\vs 1Cl 22:7
а на делающих злое лице
ЯХВЕ для того, чтобы истребить с земли память их.
\vs 1Cl 22:8
Воззвал праведник, и ЯХВЕ
услышал его, и избавил его от всех скорбей его.
\vs 1Cl 22:9
Много бичей грешному:
уповающих же на ЯХВЕ будет окружать милость.

\vs 1Cl 23:1
Милосердый во всем и
благодетельный Отец милостив к боящимся Его, и дары Свои охотно и ласково
раздает приступающим к Нему с чистым расположением.
\vs 1Cl 23:2
Посему не будем
сомневаться, и душа наша да не отчаивается о превосходных и славных дарах Его:
\vs 1Cl 23:3
да будет далеко от нас
сказанное в Писании, где оно говорит: несчастны двоедушные, колеблющиеся
душою и говорящие:
\vs 1Cl 23:4
это мы слышали и во время
отцов наших, и вот мы состарились, но ничего такого с нами не случилось.
\vs 1Cl 23:5
Неразумные! Сравните себя
с деревом, возьмите виноградную лозу:
\vs 1Cl 23:6
сперва она теряет лист,
потом образуется отпрыск, потом лист, потом цвет, и после этого незрелый,
наконец, спелый виноград.
\vs 1Cl 23:7
Видите, как в короткое
время древесный плод достигает зрелости.
\vs 1Cl 23:8
Скоро поистине и внезапно
совершится воля Господа по свидетельству самого Писания: скоро придет, и не
замедлит, и внезапно придет в храм Свой ЯХВЕ и Святой, Которого вы ожидаете.

\vs 1Cl 24:1
Рассмотрим, возлюбленные,
как Господь постоянно показывает нам будущее воскресение, которого начатком
сделал Господа Иисуса Христа, воскресив Его из мертвых.
\vs 1Cl 24:2
Посмотрим, возлюбленные,
на воскресение, совершающееся во всякое время.
\vs 1Cl 24:3
День и ночь представляют
нам воскресение: ночь отходит ко сну,~--- встает день; проходит день,~--- настает
ночь.
\vs 1Cl 24:4
Посмотрим на плоды, каким
образом происходит сеяние зерен.
\vs 1Cl 24:5
Вышел сеятель, бросил их в
землю, и брошенные семена, которые упали на землю сухие и голые, сгнивают;
\vs 1Cl 24:6
но после этого разрушения
великая сила Промысла Господня воскрешает их, и из одного возвращает многие и
производит плод.

\vs 1Cl 25:1
Взглянем на необычайное
знамение, бывающее в восточных странах, то есть около Аравии.
\vs 1Cl 25:2
Есть там птица, которая
называется Феникс. Она рождается только одна и живет по пяти сот лет.
\vs 1Cl 25:3
Приближаясь к своему
разрушению смертному, она из ливана, смирны и других ароматов делает себе
гнездо, в которое, по исполнении своего времени, входит и умирает.
\vs 1Cl 25:4
Из гниющего же тела
рождается червь, который, питаясь влагою умершего животного, оперяется;
\vs 1Cl 25:5
потом, пришедши в
крепость, берет то гнездо, в котором лежат кости его предка, и с этою ношею
совершает путь из Аравии в Египет, в город, называемый Илиополь,
\vs 1Cl 25:6
и прилетая днем, в виду
всех кладет это на жертвенник солнца, и таким образом назад удаляется.
\vs 1Cl 25:7
Жрецы рассматривают
летописи, и находят, что она являлась по исполнении пятисот лет.

\vs 1Cl 26:1
Итак, почтем ли мы
великим и удивительным, если Творец всего воскресит тех, которые в уповании
благой веры свято служили Ему,
\vs 1Cl 26:2
когда Он и посредством
птицы открывает нам Свое великое обещания Своего?
\vs 1Cl 26:3
Ибо говорится где-то: и
Ты воскресишь меня и восхвалю Тебя.
\vs 1Cl 26:4
И еще: я уснул, и спал,
восстал, потому что Ты со мной.
\vs 1Cl 26:5
Так же Иов говорит: и Ты
воскресишь эту плоть мою, которая терпит все это.

\vs 1Cl 27:1
В этой надежде да
прилепятся души наши к Тому, Кто верен в обещаниях и праведен в судах.
\vs 1Cl 27:2
Заповедавший не лгать, тем
более Сам не солжет; ибо для Бога ничего нет невозможного: невозможно только
солгать.
\vs 1Cl 27:3
Итак, да воспламенится в
нас вера Его, и будем помышлять, что все близко к Нему.
\vs 1Cl 27:4
Словом величества Своего
Он все создал, словом же может и разрушить это.
\vs 1Cl 27:5
Кто скажет Ему: зачем
сделал? или кто воспротивится могуществу силы Его.
\vs 1Cl 27:6
Когда Ему угодно, Он все
сделает, и ничего из определенного Им не останется без исполнения.
\vs 1Cl 27:7
Все пред Ним, и ничто не
скрыто от совета Его.
\vs 1Cl 27:8
Если небеса поведают
славу Божью, то твердь возвещает о творении рук Его;
\vs 1Cl 27:9
день дню отрыгает слово, и
ночь ночи возвращает ведение.
\vs 1Cl 27:10
И нет слов, ни речей,
звуки которых не были бы слышимы.

\vs 1Cl 28:1
Итак, если Бог все видит
и слышит, то убоимся Его, и оставим нечистые стремления к худым делам, чтобы
милосердием Его покрыться от будущих судов.
\vs 1Cl 28:2
Ибо куда может кто-либо из
нас убежать от крепкой руки Его? Какой мир примет убежавшего от Него?
\vs 1Cl 28:3
Ибо говорит негде Писание:
куда пойду и где скроюсь от лица Твоего?
\vs 1Cl 28:4
Если взойду на небо, Ты
там; если пойду на конец земли, и там десница Твоя; если расположусь в
безднах, и там Дух Твой.
\vs 1Cl 28:5
Итак, куда мог бы кто
удалиться, или куда убежать от Того, Кто все объемлет?

\vs 1Cl 29:1
Итак, приступим к Нему в
святости души, поднимая к Нему чистые и нескверные руки,
\vs 1Cl 29:2
и любя кроткого
милосердого Отца нашего, Который избрал нас в достояние Себе;
\vs 1Cl 29:3
ибо так написано: когда
Вышний разделял народы, когда расселял сынов Адамовых, то Он поставил пределы
народов по числу ангелов Божьих:
\vs 1Cl 29:4
и уделом ЯХВЕ стал народ
Его Иаков, межею наследия Его~--- Израиль.
\vs 1Cl 29:5
И в другом месте говорится
вот ЯХВЕ избирает Себе народ из среды народов, как человек берет начатки с
гумна своего и произойдет из того народа святое святых.

\vs 1Cl 30:1
Итак, будучи уделом
Святого, будем делать все относящееся к святости,
\vs 1Cl 30:2
убегая злословия, нечистых
и порочных связей, пьянства, страсти к нововведениям,
\vs 1Cl 30:3
низких пожеланий,
скверного расового смешения и гнусной гордости.
\vs 1Cl 30:4
Ибо говорится: Бог гордым
противится, смиренным же дает благодать.
\vs 1Cl 30:5
Итак, присоединимся к тем,
которым дана от Бога благодать.
\vs 1Cl 30:6
Облечемся в единомыслие,
будем смиренны, воздержны, далеки от всякой клеветы и злоречия, оправдывая
себя делами, а не словами.
\vs 1Cl 30:7
Ибо сказано: кто говорит
много, тот должен и слушать в свою очередь; или многоречивый будет праведен?
Благословен рожденный от жены, малолетний. Не будь многоречив.
\vs 1Cl 30:8
Хвала наша да будет у
Бога, а не от нас самих; Бог ненавидит тех, которые сами хвалят себя.
\vs 1Cl 30:9
Пусть свидетельство о
добром поведении нашем будет дано от других, так как дано было оно отцам нашим
праведным.
\vs 1Cl 30:10
Наглость, надменность и
дерзость свойственны проклятым от Бога; умеренность, смиренномудрие и кротость
у благословенных от Бога.

\vs 1Cl 31:1
Итак, взыщем
благословения Его, и посмотрим, какие пути приводят к благословению. Вспомним,
что было от начала.
\vs 1Cl 31:2
За что был благословен
отец наш Авраам? Не за то ли, что по вере своей творил правду и истину?
\vs 1Cl 31:3
Исаак, с уверенностью зная
будущее, охотно стал жертвою.
\vs 1Cl 31:4
Иаков со смирением оставил
из-за брата землю свою, пошел к Лавану и служил; и даны ему двенадцать колен
Израилевых.

\vs 1Cl 32:1
Если кто рассмотрит все в
подробности, то познает величие даров, данных от Бога.
\vs 1Cl 32:2
От Иакова все священники и
левиты, служащие при жертвеннике Божьем.
\vs 1Cl 32:3
От него Господь Иисус по
плоти: от него цари, начальники, вожди чрез Иуду;
\vs 1Cl 32:4
и прочие его колена в
немалой славе, так как обещал Бог: будет семя твое, как звезды небесные.
\vs 1Cl 32:5
И все они прославились и
возвеличились не сами собой, и не делами своими, и не правотой действий,
совершенных ими, но волей Божьей.
\vs 1Cl 32:6
Так и мы, будучи призваны
по воле Его во Христе Иисусе, оправдываемся не сами собою, и не своею
мудростью, или разумом, или благочестием, или делами, в святости сердца нами
совершаемыми,
\vs 1Cl 32:7
но посредством веры,
которую Вседержитель Бог от века всех оправдывал. Ему да будет слава во веки
веков. Аминь.

\vs 1Cl 33:1
Итак, что нам делать,
братья? Отстать ли от добродетели и любви?~--- Отнюдь нет, не дай Господь, чтоб
это сталось с нами;
\vs 1Cl 33:2
напротив, со всем усилием
и готовностью поспешим совершать доброе дело.
\vs 1Cl 33:3
Ибо Сам Творец и Владыка
всего веселится о делах Своих.
\vs 1Cl 33:4
Он высочайшею Своею силою
утвердил небеса и непостижимою Своею мудростью украсил их;
\vs 1Cl 33:5
Он отделил землю от
окружающей ее воды, и утвердил на прочном основании Своего хотения,
\vs 1Cl 33:6
и Своею властью повелел
быть ходящим на ней животным.
\vs 1Cl 33:7
Он также сотворил море и в
нем животных, и оградил Своим могуществом.
\vs 1Cl 33:8
Сверх всего этого Он
святыми и чистыми руками образовал отличнейшее и по разуму превосходнейшее
существо, человека, начертание Своего образа.
\vs 1Cl 33:9
Ибо так говорит Бог:
сотворим человека по образу и подобию Нашему. И сотворил Бог человека, мужа и
жену сотворил их.
\vs 1Cl 33:10
Совершив все это, Он
одобрил и благословил и сказал: раститесь и умножайтесь.
\vs 1Cl 33:11
Познаем также, что все
праведные украсились добрыми делами; и Сам Господь радовался, украсив Себя
делами.
\vs 1Cl 33:12
Имея такой пример,
неленостно последуем воле Его, и всею силою будем творить дело правды.

\vs 1Cl 34:1
Добрый работник смело
получает хлеб за труд свой; ленивый же и беспечный не смеет и взглянуть на
того, кто дал ему работу.
\vs 1Cl 34:2
И нам надлежит быть
ревностными в делании добра, ибо все от Него.
\vs 1Cl 34:3
Ибо предсказывает нам:
вот ЯХВЕ, и награда Его перед лицом Его, чтобы воздать каждому по делу его.
\vs 1Cl 34:4
Так увещевает Он нас всем
сердцем обратиться к Нему, и ни в каком добром деле не быть беспечными и
нерадивыми;
\vs 1Cl 34:5
в Нем да будет похвала и
надежда наша; покоримся воле Его.
\vs 1Cl 34:6
Помыслим о всем множестве
ангелов Его, как они, предстоя, исполняют волю Его.
\vs 1Cl 34:7
Ибо говорит Писание: тьмы
тем предстояли пред Ним и тысячи тысяч служили Ему, и взывали: свят, свят,
свят ЯХВЕ Цебаот; полно все творение славы Его.
\vs 1Cl 34:8
Так и мы, в единомысленном
собрании, единым духом, как бы из одних уст, будем взывать к Нему непрестанно,
чтобы сделаться нам участниками великих и славных обетований Его.
\vs 1Cl 34:9
Ибо говорит: око не
видело, и ухо не слышало, и на сердце человеку не приходило то, что Он
уготовал уповающим на Него.

\vs 1Cl 35:1
Как блаженны и чудны дары
Божьи, возлюбленные~--- жизнь в бессмертии, сияние в правде, истина в свободе,
вера в уповании, воздержание в святости: все это доступно нашему разумению.
\vs 1Cl 35:2
Какие же еще уготовляются
ждущим? Творец и Отец веков, Всесвятой, Он Сам знает их величие и красоту.
\vs 1Cl 35:3
Итак, употребим все усилия
быть в числе уповающих на Него, чтобы участвовать в обетованных дарах.
\vs 1Cl 35:4
Каким же образом это
будет, возлюбленные? Если ум наш будет утвержден в вере в Бога; если будем
искать того, что Ему угодно и приятно;
\vs 1Cl 35:5
если будем исполнять то,
что согласно с Его святою волею, и ходить путем истины, отвергнув от себя
всякую неправду и беззаконие,
\vs 1Cl 35:6
любостяжание, распри,
злонравие и коварство, клеветы и злословие, нечестие, гордость и величавость,
тщеславие и негостеприимность.
\vs 1Cl 35:7
Ибо делающие это
ненавистны Богу, и не только делающие, но и одобряющие это.
\vs 1Cl 35:8
Писание говорит: грешнику
сказал Бог: зачем ты познаешь заповеди Мои и принимаешь завет Мой устами
твоими, а возненавидел вразумление и отверг слова Мои?
\vs 1Cl 35:9
Если ты видел вора, то
бежал с ним, и с прелюбодеем принимал участие.
\vs 1Cl 35:10
Уста твои были исполнены
злобы, и язык твой сплетал обманы.
\vs 1Cl 35:11
Сидя на суде, ты клеветал
на брата твоего и сыну матери твоей строил ковы.
\vs 1Cl 35:12
Ты это делал, и Я молчал;
ты, беззаконный, подумал, что буду тебе подобен.
\vs 1Cl 35:13
Но обличу тебя и
представлю тебя перед лицом твоим.
\vs 1Cl 35:14
Разумейте же это вы,
забывающие Бога, чтобы вам не быть похищенными как бы львом, и некому будет
избавить вас.
\vs 1Cl 35:15
Жертва хвалы прославит
Меня, и там путь, на котором явлю ему спасение Божье.

\vs 1Cl 36:1
Таков путь, возлюбленные,
которым мы обретаем наше спасение, Иисуса Христа, Первосвященника наших
приношений, заступника и помощника в немощи нашей.
\vs 1Cl 36:2
Посредством Него взираем
мы на высоту небес; чрез Него, как бы в зеркале видим чистое и пресветлое лицо
Бога;
\vs 1Cl 36:3
чрез Него отверзлись очи
сердца нашего; чрез Него несмысленный и омраченный ум наш возникает в чудный
Его свет;
\vs 1Cl 36:4
чрез Него восхотел
Господь, чтобы мы вкусили бессмертного знания.
\vs 1Cl 36:5
Он, будучи сиянием величия
Его, столько превосходнее ангелов, сколько славнейшее пред ними наследовал
имя.
\vs 1Cl 36:6
Ибо так написано: Он
творит ангелов Своих духами и служителей Своих пламенем огненным.
\vs 1Cl 36:7
О Сыне же Своем так сказал
Господь: Сын Мой Ты, Я ныне родил Тебя, проси от Меня и дам Тебе народы в
достояние Твое, и пределы земли~--- в обладание Твое.
\vs 1Cl 36:8
И еще говорит к Нему:
сиди одесную Меня, доколе положу врагов Твоих в подножие ног Твоих.
\vs 1Cl 36:9
Кто же враги?~--- Порочные,
противящиеся воле Божьей.

\vs 1Cl 37:1
Итак, братья! будем всеми
силами воинствовать под святыми Его повелениями.
\vs 1Cl 37:2
Представим себе
воинствующих под начальством вождей наших; как стройно, как усердно, как
покорно исполняют они приказания.
\vs 1Cl 37:3
Не все епархи, не все
тысяченачальники, или стоначальники или пятидесятиначальники и так далее, но
каждый в своем чине исполняет приказания царя и полководцев.
\vs 1Cl 37:4
Ни великие без малых, ни
малые без великих не могут существовать.
\vs 1Cl 37:5
Все они как бы связаны
вместе, и это доставляет пользу.
\vs 1Cl 37:6
Возьмем тело наше: голова
без ног ничего не значит, равно и ноги без головы, и малейшие члены в теле
нашем нужны и полезны для целого тела;
\vs 1Cl 37:7
все они согласны и
стройным подчинением служат для целого тела.

\vs 1Cl 38:1
Так пусть будет здраво и
все тело наше в Иисусе Христе, и каждый повинуется ближнему своему сообразно
со степенью, на которой он поставлен дарованием Его.
\vs 1Cl 38:2
Сильный не пренебрегай
слабого, и слабый почитай сильного; богатый подавай бедному, и бедный
благодари Бога, что Он даровал ему, чрез кого может быть восполнена его
скудость.
\vs 1Cl 38:3
Мудрый показывай мудрость
свою не в словах, но в добрых делах.
\vs 1Cl 38:4
Смиренный не сам о себе
свидетельствуй, но предоставляй другому дать о тебе свидетельство.
\vs 1Cl 38:5
Чистый по плоти молчи и не
превозносись, зная, что есть другой, дарующий ему воздержание.
\vs 1Cl 38:6
Помыслим, братья, из
какого вещества мы произошли, и какими вошли в мир, как бы из гроба и мрака.
\vs 1Cl 38:7
Творец и Создатель наш
ввел нас в мир Свой, наперед приготовил нам Свои благодеяния прежде рождения
нашего.
\vs 1Cl 38:8
Итак, все имея от Него, мы
должны за все благодарить Его. Ему слава во веки веков. Аминь.

\vs 1Cl 39:1
Безумные, несмысленные,
глупые и невежды смеются и ругаются над нами, желая самих себя возвысить в
собственных мыслях своих.
\vs 1Cl 39:2
Но что может смертный, или
какая крепость в земнородном?
\vs 1Cl 39:3
Ибо написано: не было
образа пред глазами моими; но я слышал тихое веяние и голос:
\vs 1Cl 39:4
что же? будет ли человек
чист пред ЯХВЕ, или в делах своих непорочен, если Он на служителей Своих не
полагается и в ангелах Своих усматривает недостатки?
\vs 1Cl 39:5
Небо не чисто пред Ним;
тем менее живущие в бренных храминах, из числа которых и мы сами из того же
брения.
\vs 1Cl 39:6
Как бы моль поела их, и от
утра до вечера их уже нет: от того, что не могут помочь самим себе, они
погибли.
\vs 1Cl 39:7
Дунул на них и погибли,
потому что не имеют мудрости.
\vs 1Cl 39:8
Призови же, услышит ли
тебя кто-нибудь, или увидишь ли кого из святых ангелов?
\vs 1Cl 39:9
Безумного губит гнев и
глупого умерщвляет рвение.
\vs 1Cl 39:10
Я видел безумных
укореняющихся, но тотчас истреблено было их жилище.
\vs 1Cl 39:11
Да будут сыны их далеко
от спасения и да будут презрены при дверях меньших, и некому будет спасти их.
\vs 1Cl 39:12
Ибо что они собрали,
поедят праведные, сами же от зол не будут изъяты.

\vs 1Cl 40:1
Будучи убеждены в этом и
проникая в глубины божественного ведения, мы должны в порядке совершать все,
что Господь повелел совершать в определенные времена.
\vs 1Cl 40:2
Он повелел, чтобы жертвы и
священные действия совершались не случайно и не без порядка, но в определенные
времена и часы.
\vs 1Cl 40:3
Также где и через кого
должно быть это совершаемо, Сам Он определил высочайшим Своим изволением,
чтобы все совершалось свято и богоугодно, и было приятно воле Его.
\vs 1Cl 40:4
Итак, приятны Ему и
блаженны те, которые в установленные времена приносят жертвы свои;
\vs 1Cl 40:5
ибо, следуя заповедям
Господним, они не погрешают.
\vs 1Cl 40:6
Первосвященнику дано свое
служение, священникам назначено свое дело, и на левитов возложены свои
должности;
\vs 1Cl 40:7
из народа человек связан
постановлениями для народа.

\vs 1Cl 41:1
Каждый из вас, братья,
благодари Бога за свое собственное положение, храня добрую совесть и с
благоговением не преступая определенного правила служения своего.
\vs 1Cl 41:2
Не повсюду, братья,
приносятся жертвы непрерывные, или обетные или жертвы за грех, и жертвы
повинности, но только в Иерусалиме,
\vs 1Cl 41:3
и там не на всяком месте
совершается приношение, а пред храмом на жертвеннике, после того как жертва
будет осмотрена первосвященником и вышеназванными служителями.
\vs 1Cl 41:4
Те же, которые делают
что-либо вопреки Его воле, наказываются смертью.
\vs 1Cl 41:5
Видите, братья, чем
большего сподобились мы видения, тем большей подлежим опасности.

\vs 1Cl 42:1
Апостолы были посланы
проповедовать благовестие нам от Господа Иисуса Христа, Иисус Христос~--- от
Бога.
\vs 1Cl 42:2
Христос был послан от
Бога, а апостолы~--- от Христа; то и другое было в порядке по воле Божьей.
\vs 1Cl 42:3
Итак принявши повеление,
совершенно убежденные чрез воскресение Господа нашего Иисуса Христа и
утвержденные в вере словом Божьим, с полнотой Духа Святого пошли
благовествовать наступающее царство Божье.
\vs 1Cl 42:4
Проповедуя по странам и
городам, они первенцев, по духовном испытании поставляли в епископы и диаконы
для будущих верующих.
\vs 1Cl 42:5
И это не новое
установление; ибо много веков прежде было писано о епископах и диаконах.
\vs 1Cl 42:6
Так говорит Писание:
поставлю епископов их в правде и диаконов в вере.

\vs 1Cl 43:1
И чему дивиться, если те,
которым во Христе вверено было бы от Бога это дело, поставляли вышеупомянутых?
\vs 1Cl 43:2
Блаженный Моисей,
вверенный служитель во всем доме Божьем, все заповеданное Ему изобразил в
священных книгах;
\vs 1Cl 43:3
ему последовали и прочие
пророки, утверждая своим свидетельством его узаконения.
\vs 1Cl 43:4
Когда возникла распря о
священстве, и колена разногласили о том, какое из них должно быть украшено
этим славным именем,
\vs 1Cl 43:5
то повелел двенадцати
начальникам колен принести к нему жезлы, на которых было написано имя каждого
колена;
\vs 1Cl 43:6
и взявши их, связал,
запечатал перстнями начальников колен, положил их в скинии свидетельства на
трапезе Господней.
\vs 1Cl 43:7
И, заключив скинию,
запечатал замки также, как и жезлы, и сказал им: братья, которого колена жезл
расцветет, то избрал Бог к священству и служению Себе.
\vs 1Cl 43:8
На другой день утром
созвал он всего Израиля, шесть сот тысяч человек, и показал начальникам колен
печати их, и отворил скинию свидетельства и вынес жезлы:
\vs 1Cl 43:9
и оказалось, что жезл
Аарона не только расцвел но даже имел на себе плод.
\vs 1Cl 43:10
Как вы думаете,
возлюбленные, не знал ли Моисей прежде, что это будет?
\vs 1Cl 43:11
Конечно знал, но так
поступил он для того, чтобы не было возмущения в Израиле, для прославления
имени истинного и единого Бога. Ему слава во веки веков. Аминь.

\vs 1Cl 44:1
И апостолы наши знали
чрез Господа нашего Иисуса Христа, что будет раздор о епископском достоинстве.
\vs 1Cl 44:2
По этой самой причине они,
получивши совершенное предведение, поставили вышеозначенных, и потом
присовокупили закон, чтобы когда они почиют, другие испытанные мужи принимали
на себя их служение.
\vs 1Cl 44:3
Итак, почитаем
несправедливым лишить служения тех, которые поставлены самими апостолами или
после них другими достоуважаемыми мужами, с согласия всей Церкви, и служили
стаду Христову неукоризненно, со смирением, кротко и беспорочно, и притом в
течение долгого времени от всех получили одобрение.
\vs 1Cl 44:4
И не малый будет на нас
грех, если неукоризненно и свято приносящих дары будем лишать епископства.
\vs 1Cl 44:5
Блаженны предшествовавшие
нам пресвитеры, которые разрешились от тела после многоплодной и совершенной
жизни:
\vs 1Cl 44:6
им нечего опасаться, чтобы
кто мог свергнуть их с занимаемого ими места.
\vs 1Cl 44:7
Ибо мы видим, что вы
некоторых, похвально провождающих жизнь, лишили служения безукоризненно ими
проходимого.

\vs 1Cl 45:1
Вы, братья, спорливы и
ревностны в том, что ни мало не относится к спасению.
\vs 1Cl 45:2
Загляните в Писания, эти
истинные глаголы Духа Святого. Заметьте, что в них ничего несправедливого и
превратного не написано.
\vs 1Cl 45:3
Вы не найдете чтобы люди
праведные были низвергаемы людьми святыми.
\vs 1Cl 45:4
Были гонимы праведные, но
от беззаконных; были заключаемы в темницу, но от нечестивых; были побиваемы
камнями от злодеев; были убиваемы от порочных, увлекавшихся преступною
завистью. Все эти страдания они перенесли со славою.
\vs 1Cl 45:5
Ибо что скажем, братья?
Даниил от богобоязненных ли людей был брошен в ров львиный? Анания, Азария и
Мисаил от чтителей ли благолепного и славного служения Всевышнему были
ввержены в пещь огненную?~--- Отнюдь нет.
\vs 1Cl 45:6
Кто же сделал это?~--- Люди
порочные, полные всякого зла, дошли до такого неистовства, что святою и
непорочною волею служащих Богу подвергли мучениям:
\vs 1Cl 45:7
они не знали того, что
Вышний есть заступник и защитник тех, которые с чистой совестью чтут
всесовершенное имя Его. Ему слава во веки веков. Аминь.
\vs 1Cl 45:8
А они, терпя в уповании, и
были превознесены Богом, и сделались достолюбезными в памяти их во веки веков.
Аминь.

\vs 1Cl 46:1
Таким примерам и мы
должны подражать, братья.
\vs 1Cl 46:2
Ибо написано: прилепитесь
к святым; ибо прилепляющиеся к ним освятятся.
\vs 1Cl 46:3
И опять в другом месте
сказано: с мужем невинным будешь невинен, и с избранным будешь избран, а с
развращенным развратишься.
\vs 1Cl 46:4
Итак, присоединимся к
невинным и праведным, они-то суть избранные Божьи.
\vs 1Cl 46:5
К чему у вас распри; гнев
несогласия, разделения, война?
\vs 1Cl 46:6
Не одного ли Бога и одного
Христа имеем мы? Не один ли Дух благодати излит на нас, не одно ли призвание
во Христе?
\vs 1Cl 46:7
Для чего раздираем и
расторгаем члены Христовы, восстаем против собственного тела, и до такого
доходим безумия, что забываем, что мы друг другу члены?
\vs 1Cl 46:8
Вспомните слова Иисуса,
Господа нашего. Он сказал: горе тому человеку; хорошо было бы ему не
родиться, нежели соблазнить одного из избранных Моих;
\vs 1Cl 46:9
было бы лучше для него,
если бы он повесил камень жерновный и ввергнулся в море, нежели соблазнить
одного из малых Моих.
\vs 1Cl 46:10
Ваше разделение многих
развратило, многих повергло в уныние, многих в сомнение, и всех нас в печаль,
а смятение ваше все еще продолжается.

\vs 1Cl 47:1
Возьмите послание
блаженного апостола Павла. О чем он прежде всего писал вам в начале ангельской
проповеди?
\vs 1Cl 47:2
Истинно он по вдохновению
написал вам как о себе самом, так и о Кифе и Аполлосе, потому что и тогда
произошло у вас разделение на различные стороны.
\vs 1Cl 47:3
Но тогдашнее разделение
подвергло вас меньшему греху; ибо вы преклонялись на стороны прославленных
апостолов, и на сторону мужа, им одобренного.
\vs 1Cl 47:4
А теперь подумайте, какие
люди развратили вас и уменьшили красоту знаменитой братской любви вашей.
\vs 1Cl 47:5
Постыдное, возлюбленные, и
чрезвычайно постыдное и христианской жизни недостойное слышится дело:
твердейшая и древняя церковь Коринфская из-за одного или двух человек
возмутилась против пресвитеров.
\vs 1Cl 47:6
И этот слух дошел не
только до нас, но и до самых врагов наших, так что чрез ваше безумие имя
Господне подвергается поруганию, и вам самим готовится опасность.

\vs 1Cl 48:1
Итак, прекратим это как
можно скорее, и припадем к Господу, и слезно будем умолять Его, чтобы Он,
умилосердившись, примирился с нами, и восстановил в нас прежнюю прекрасную и
чистую жизнь братской любви.
\vs 1Cl 48:2
Это~--- врата правды,
отверстые к жизни как написано: откройте Мне врата правды; Я войду с вами и
восхвалю ЯХВЕ. Это~--- врата ЯХВЕ, праведные войдут ими.
\vs 1Cl 48:3
Из многих открытых врат
врата правды суть врата Христовы, и блаженны те, которые входят ими и
направляют шествие свое в святости и правде, все совершая без возмущения.
\vs 1Cl 48:4
Если кто тверд в вере, или
способен предлагать ведение, или мудр в обсуждении речей, или чист по своим
делам; тем более он должен смиряться, чем более кажется великим, и должен
искать общей пользы, а не своей.

\vs 1Cl 49:1
Кто имеет любовь во
Христе, тот должен соблюдать заповеди Христовы.
\vs 1Cl 49:2
Кто может изъяснить союз
любви Божьей? Кто способен, как должно, высказать величие благости Его?
\vs 1Cl 49:3
Несказанна высота, на
которую возводит любовь. Любовь соединяет нас с Богом; любовь покрывает
множество грехов, любовь все принимает, все терпит великодушно.
\vs 1Cl 49:4
В любви нет ничего
низкого, ничего надменного, любовь не допускает разделения, любовь не заводит
возмущения,
\vs 1Cl 49:5
любовь все делает в
согласии, любовью все избранные Божьи достигли совершенства, без любви нет
ничего благоугодного Богу.
\vs 1Cl 49:6
По любви воспринял нас
Господь; по любви, которую имел к нам Иисус Христос, Господь наш, по воле
Божьей дал кровь за нас, и плоть за плоть нашу, и душу за души наши.

\vs 1Cl 50:1
Видите ли, возлюбленные,
как велика и дивна любовь, и невыразимо ее совершенство.
\vs 1Cl 50:2
Кто может иметь ее, если
кого Сам Бог не удостоит?
\vs 1Cl 50:3
Итак будем просить и
умолять Его милосердие, чтобы жить нам в любви непорочно, без человеческого
разделения.
\vs 1Cl 50:4
Все роды от Адама до сего
дня миновали; но усовершившиеся в любви по благодати Божьей находятся на месте
благочестивых: они откроются с пришествием царства Христова.
\vs 1Cl 50:5
Ибо написано: войди на
некоторое время в храмины, пока пройдет гнев и негодование Мое, и вспомню о
дне добром, и воскрешу вас от гробов ваших.
\vs 1Cl 50:6
Блаженны мы, возлюбленные,
если исполняем заповеди Божьи в единомыслии любви, дабы чрез любовь были
прощены нам грехи наши.
\vs 1Cl 50:7
Ибо написано: блаженны
те, которых беззакония отпущены и которых покрылись грехи. Блажен человек,
которому ЯХВЕ не вменит греха, и в устах его нет обмана.
\vs 1Cl 50:8
Это обещание блаженства
относится к тем, которые избраны Богом чрез Иисуса Христа, Господа нашего. Ему
слава во веки веков. Аминь.

\vs 1Cl 51:1
Итак, в чем мы согрешили
по каким-либо наветам врага, должны мы просить прощения.
\vs 1Cl 51:2
И те, которые были
предводителями возмущения и раздора, должны иметь в виду общую надежду.
\vs 1Cl 51:3
Ибо провождающие жизнь со
страхом и любовью лучше хотят сами подвергнуться неприятностям, нежели ближних
своих,
\vs 1Cl 51:4
и охотнее на себя примут
осуждение, нежели на преданное нам доброе и святое согласие.
\vs 1Cl 51:5
И лучше человеку
признаться в своих грехах, нежели ожесточать сердце свое,
\vs 1Cl 51:6
как ожесточилось сердце
возмутившихся против раба Божьего Моисея:
\vs 1Cl 51:7
суд над ними совершился
явно, ибо они живые снизошли во ад и поглотила их смерть.
\vs 1Cl 51:8
Фараон, войско его, все
начальники Египетские, и колесницы и всадники их не по другой какой причине
потонули в море Суф и погибли, но потому, что ожесточились их несмысленные
сердца, после стольких знамений и чудес, совершенных в земле Египетской через
раба Божьего Моисея.

\vs 1Cl 52:1
Братья! Господь ни в чем
не имел нужды, и ничего ни от кого не требует, кроме исповедания Ему.
\vs 1Cl 52:2
Ибо говорит избранный
Давид: исповедуюсь ЯХВЕ, и это будет Ему приятнее, нежели молодой телец, у
которого растут рога и копыта. Пусть видят это бедные и возрадуются.
\vs 1Cl 52:3
И опять говорит: принеси
Богу жертву хвалы и воздай Вышнему молитвы твои.
\vs 1Cl 52:4
И призови Меня в день
скорби твоей, и избавлю тебя, и ты прославишь Меня.
\vs 1Cl 52:5
Ибо жертва Богу~--- дух
сокрушенный.

\vs 1Cl 53:1
Вы знаете, возлюбленные,
и хорошо знаете священные Писания, и разумеете слова Божьи.
\vs 1Cl 53:2
Итак, приведите себе на
память: когда Моисей взошел на гору и провел сорок дней и сорок ночей в посте
и смирении,
\vs 1Cl 53:3
тогда сказал ему ЯХВЕ:
Моисей, Моисей, сойди поскорей отсюда, потому что совершил преступление народ
твой, который ты вывел из земли Египетской;
\vs 1Cl 53:4
скоро они совратились с
пути, который ты заповедал им,~--- сделали себе слияния.
\vs 1Cl 53:5
И сказал ему ЯХВЕ: говорил
Я тебе раз и два, говоря: видел Я народ этот, и вот он~--- народ жестоковыйный.
\vs 1Cl 53:6
Дай Мне истребить его, и
погублю имя его под небом, а тебя сделаю народом великим и дивным и
многочисленнее этого.
\vs 1Cl 53:7
Моисей же сказал: нет,
ЯХВЕ, прости грех народу этому, или и меня истреби из книги живых.
\vs 1Cl 53:8
О, великая любовь! О,
несравненное совершенство! Раб смело говорит Господу, просит прощения народу;
в противном случае хочет и сам быть истребленным вместе с ними.

\vs 1Cl 54:1
Итак, кто из вас
благороден, кто добродушен, кто исполнен любви, тот пусть скажет:
\vs 1Cl 54:2
если из-за меня мятеж
раздор и разделение, я отхожу, иду, куда вам угодно, и исполню все, что велит
народ, только бы стадо Христово было в мире с поставленными пресвитерами.
\vs 1Cl 54:3
Кто поступит таким
образом, тот приобретет себе великую славу в Господе, и всякое место примет
его:
\vs 1Cl 54:4
ибо ЯХВЕ земля и
исполнение ее.
\vs 1Cl 54:5
Так поступали и будут
поступать все, провождающие похвальную божественную жизнь.

\vs 1Cl 55:1
Но представим примеры
народов. Многие цари и вожди во время моровой язвы, по внушениям прорицалища,
предавали себя на смерть, чтобы своею кровью спасти граждан.
\vs 1Cl 55:2
Многие удалялись из своих
городов, чтобы прекратилось возмущение в них.
\vs 1Cl 55:3
И из своих мы знаем
многих, которые предали себя в узы, дабы других освободить.
\vs 1Cl 55:4
Многие предали себя в
рабство, и, взявши за себя цену, питали других.
\vs 1Cl 55:5
Многие женщины,
укрепленные благодатью Божьей, совершили много дел мужественных.
\vs 1Cl 55:6
Блаженная Иудифь во время
осады города испросила позволения у старейшин пойти в стан иноплеменников.
\vs 1Cl 55:7
И пошла она, подвергая
себя опасности из любви к своему отечеству и народу осажденному, и Господь
предал Олоферна в руки женщины.
\vs 1Cl 55:8
Не меньшей опасности
подвергла себя совершенная по вере Есфирь, дабы избавить от предстоявшей
погибели двенадцать колен Израилевых.
\vs 1Cl 55:9
В посте и смирении она
умоляла всевидящего ЯХВЕ, Бога веков, Который, видя смирение души ее, избавил
народ, для блага которого она подвергла себя опасности.

\vs 1Cl 56:1
Будем и мы молиться о
тех, которые находятся во грехе, чтобы дарована им была кротость и смирение,
чтобы они послушались не нас, но воли Божьей.
\vs 1Cl 56:2
Ибо таким образом будет
для них плодотворно и совершенно милосердное воспоминание их пред Богом и
святыми.
\vs 1Cl 56:3
Примем наказание, на
которое никто не должен досадовать, возлюбленные!
\vs 1Cl 56:4
Взаимно делаемое нами друг
другу вразумление хорошо и весьма полезно, ибо оно не прилепляет нас к воле
Божьей.
\vs 1Cl 56:5
Ибо так говорит Святое
Слово: тяжко наказал меня ЯХВЕ, но смерти не предал меня;
\vs 1Cl 56:6
ибо кого любит ЯХВЕ, того
наказывает, и бьет всякого сына, которого принимает.
\vs 1Cl 56:7
Праведник накажет меня
милостиво и обличит меня; елей же грешного да не намастит головы моей.
\vs 1Cl 56:8
И еще говорит: блажен
человек, которого обличил ЯХВЕ; и вразумления Вседержителя не отвращайся,
\vs 1Cl 56:9
ибо Он производит скорбь и
опять восстановляет, поражает и руки Его исцеляют.
\vs 1Cl 56:10
Шесть раз избавит тебя от
бед, в седьмой же не коснется тебя зло.
\vs 1Cl 56:11
Во время голода избавит
тебя от смерти, во время войны спасет тебя от руки железа;
\vs 1Cl 56:12
от бича языка защитит
тебя, и не убоишься пред наступающими бедствиями.
\vs 1Cl 56:13
Ты посмеешься над
неправедными и беззаконными, и диких зверей не устрашишься; ибо звери дикие
будут мирны с тобою.
\vs 1Cl 56:14
Потом ты узнаешь, что дом
твой будет наслаждаться миром и не будет недостатка в помещении твоего шатра.
\vs 1Cl 56:15
Узнаешь также, что велико
семя твое и дети твои будут, как различные злаки полевые.
\vs 1Cl 56:16
Во гроб же сойдешь, как
пшеница созрелая, вовремя пожатая, или как стог гумна, вовремя свезенный.
\vs 1Cl 56:17
Видите, возлюбленные, что
наказуемые Господом~--- под Его защитою,
\vs 1Cl 56:18
ибо, как благой, Бог
наказывает для того, чтобы мы вразумились святым Его наказанием.

\vs 1Cl 57:1
Итак, вы, положившие
начало возмущению, покоритесь пресвитерам, и примите вразумление к покаянию,
преклонив колена сердца своего.
\vs 1Cl 57:2
Научитесь покорности,
отложивши тщеславную и надменную дерзость языка.
\vs 1Cl 57:3
Ибо лучше вам быть в стаде
Христа малыми и уважаемыми, нежели казаться чрезмерно высокими и лишиться
упования Его.
\vs 1Cl 57:4
Ибо так говорит
всесовершенная Премудрость: вот предложу вам слово Моего дыхания и научу вас
Моему разуму.
\vs 1Cl 57:5
Поскольку Я звала, и вы не
послушали, Я простирала слова, и вы не внимали, но отвергали Мои советы, и не
покорялись Моим обличениям:
\vs 1Cl 57:6
то Я посмеюсь вашей
погибели, и порадуюсь, когда придет вам пагуба, и когда внезапно настигнет вас
смятение, явится переворот подобно буре, или когда придет вам скорбь и
бедствие.
\vs 1Cl 57:7
Будет тогда, что призовете
Меня, а Я не послушаю вас; будут искать Меня злые и не найдут.
\vs 1Cl 57:8
Ибо они возненавидели
премудрость, страха ЯХВЕ не приняли, и не хотели внимать Моим советам, но
смеялись Моим обличениям.
\vs 1Cl 57:9
И потому они вкусят плоды
своих путей и насытятся своего нечестия.
\vs 1Cl 57:10
Ибо за то, что обидели
младенцев, они убиты будут, и суд нечестивых погубит.
\vs 1Cl 57:11
Меня же слушающий будет
обитать уверенно в надежде и упокоится без страха от всякого зла.

\vs 1Cl 58:1
Итак, будем повиноваться
всесвятому и славному имени Его, избегая прореченных Премудростью угроз
непокорным, дабы обитать уверенно в пресвятом имени величия Его.
\vs 1Cl 58:2
Примите совет наш, и не
раскаетесь. Ибо жив Бог и жив Господь Иисус Христос и Дух Святой, вера и
надежда избранных,
\vs 1Cl 58:3
так что выполнивший в
смиренномудрии, с непрестанной кротостью, Богом данные заповеди и повеления,
не раскаиваясь,~---
\vs 1Cl 58:4
сей поставится и изберется
в число спасающихся чрез Иисуса Христа, чрез Которого Ему слава во веки веков.
Аминь.

\vs 1Cl 59:1
Если же некоторые не
покорятся сказанному Им через нас,~--- пусть знают, что свяжут себя падением и
немалою опасностью.
\vs 1Cl 59:2
Мы же неповинны будем во
грехе сем и будем непрестанно молиться, прося и умоляя:
\vs 1Cl 59:3
да сохранит Творец всех
нерушимо исчисленное число избранных Своих во всем мире чрез возлюбленного
Отрока Своего, Иисуса Христа,~---
\vs 1Cl 59:4
чрез Которого Ты призвал
нас из тьмы в свет, из неведения~--- в познание славы Имени Его,
\vs 1Cl 59:5
надеяться на прежде всего
творения Имя Твое~--- ЯХВЕ.
\vs 1Cl 59:6
Отверзший очи сердца
нашего, чтобы познать Тебя, Единого Вышнего в вышних,
\vs 1Cl 59:7
Святого во святых
почивающего, смиряющего надмение гордых, разрушающего замыслы народов,
\vs 1Cl 59:8
смиренных возносящего и
смиряющего вознесенных,
\vs 1Cl 59:9
обогащающего и
разоряющего, убивающего и животворящего, Единого Благодетеля духов и Бога
всякой плоти,
\vs 1Cl 59:10
видящего бездны, Всевидца
человеческих дел, в опасности пребывающих Помощника,
\vs 1Cl 59:11
отчаявшихся Спасителя,
всякого духа Творца и Надзирателя, умножающего на земле народы и из всех
избравшего любящих Тебя чрез возлюбленного Отрока Твоего Иисуса Христа, чрез
Которого Ты нас научил, освятил, почтил.
\vs 1Cl 59:12
Просим, Владыка,
Помощником и Заступником нашим быть, пребывающих из нас в скорби спаси,
смиренных помилуй,
\vs 1Cl 59:13
падших воздвигни,
просящим явись, немощных исцели,
\vs 1Cl 59:14
заблуждающихся от народа
Твоего обрати, напитай алчущих, плененных из нас освободи, восставь немощных,
утешь малодушных:
\vs 1Cl 59:15
да познают Тебя все
народы, ибо Ты~--- Един Бог, и Иисус Христос~--- Отрок Твой, и мы люди Твои и
овцы пажити Твоей.

\vs 1Cl 60:1
Ибо Ты через совершаемое
Тобой сделал зримым вечный состав мира;
\vs 1Cl 60:2
Ты, ЯХВЕ, сотворил
вселенную, верный во всех родах и праведный в судах, чудный в силе и
великолепии,
\vs 1Cl 60:3
мудрый в творении и
разумный в основании сотворенного, благой в видимом и верный в надеющихся на
Тебя, милостивый и щедрый, оставь нам беззакония наши и неправды и грехи и
прегрешения.
\vs 1Cl 60:4
Не вмени всякого греха
рабов Твоих и рабынь, но очисти нас очищением истины Твоей и исправь стопы
наши, чтобы ходить в святости, правде и простоте сердца и творить благое и
угодное пред Тобою и пред князьями нашими.
\vs 1Cl 60:5
О, ЯХВЕ, яви лице Твое нам
во благо в мире, чтобы осениться нам рукою Твоею сильною и избавиться от
всякого зла Твоею мышцею высокою, и избавь нас от ненавидящих нас неправедно.
\vs 1Cl 60:6
Подай единомыслие и мир
нам и всем населяющим землю также, как Ты дал отцам нашим, призывающим им Тебя
свято в вере и истине,
\vs 1Cl 60:7
чтобы покорными быть
всемогущему и всесовершенному Имени Твоему, и князьям и вождям нашим на земле.

\vs 1Cl 61:1
Ты, Владыка, дал власть
царства им ради великолепия и неизреченной Твоей державы, чтобы познать нам
данную Тобою им славу и честь покоряться им, ни в чем не противиться воле
Твоей;
\vs 1Cl 61:2
подай им, ЯХВЕ, здравие,
мир, единомыслие, благостояние, дабы исполнять им Тобою данное им водительство
без соблазна.
\vs 1Cl 61:3
Ибо Ты, Владыка
пренебесный, Царь веков, дающий сынам человеческим славу и честь и власть над
сущими на земле;
\vs 1Cl 61:4
Ты, ЯХВЕ, исправь совет их
ко благу и угодному пред Тобою, да совершая в мире и кротости благочестно
данную им Тобою власть, обретут Тебя милостива.
\vs 1Cl 61:5
Единый могущий творить сие
и великое благо с нами, Тебе исповедуемся чрез Первосвященника и Ходатая душ
наших Иисуса Христа, чрез Которого Тебе слава и величие и ныне и в род родов
и во веки веков. Аминь.

\vs 1Cl 62:1
Итак, о делах приличных
богопочтению нашему и полезнейших для жизни добродетельной, желающим вести ее
благочестно и праведно, мы достаточно написали вам, мужи братия.
\vs 1Cl 62:2
Ибо мы всюду касались
того, что относится к вере, покаянию, искренней любви, воздержанию,
целомудрию и терпению,
\vs 1Cl 62:3
напоминая, что должно вам
в справедливости, истине и великодушии свято благоугождать Вседержителю Богу,
в единомыслии, незлопамятно, в любви и мире, с непрестанною кротостью,
\vs 1Cl 62:4
как и названные выше отцы
наши благоугождали, смиренномудрствуя по отношению к Отцу, Богу и Творцу, и ко
всем людям.
\vs 1Cl 62:5
И тем приятнее нам было
напомнить об этом, что мы пишем~--- как мы ясно знаем~--- мужам верным и славным,
вникающим в изречения учения Божьего.
1
Итак, справедливо,~--- следуя столь великим и многим примерам,~--- склонить выю и
занять место послушания,
\vs 1Cl 63:2
дабы успокоившись от
суетного волнения, достигли мы в истине предлежащей нам цели без всякого
позора.
\vs 1Cl 63:3
Ибо вы доставите нам
радость и веселье, если послушаетесь написанного нами чрез Святого Духа и
пресечете несправедливый гнев ревности вашей, сообразно увещанию к миру и
согласию, нами обращенному к вам в этом послании.
\vs 1Cl 63:4
Послали же мы мужей верных
и мудрых, от юности до старости обращавшихся непорочно среди нас, которые и
будут свидетелями между нами и вами.
\vs 1Cl 63:5
А поступили мы так, дабы
знали вы, что вся забота наша и была и есть~--- чтобы в скорости достигли вы
мира.

\vs 1Cl 64:1
Всевидящий Бог и Владыка
духов и Господь всякой плоти, избравший Господа Иисуса Христа и чрез Него~---
нас в народ избранный,
\vs 1Cl 64:2
да даст всякой душе,
призывающей великое и святое имя Его, веру, страх, мир, терпение, великодушие,
воздержание, чистоту и целомудрие
\vs 1Cl 64:3
в благоугождение имени его чрез первосвященника и ходатая
нашего Иисуса Христа, чрез которого ему слава,
величие, держава и честь ныне и во веки веков.
Аминь.

\vs 1Cl 65:1
Посланных от нас, Клавдия Эфеба и Валерия Витона с Фортунатом,
немедленно отпустите к нам в мире с радостью,
\vs 1Cl 65:2
чтобы они скорее известили нас о желаемом
и вожделенном для нас мире и согласии вашем,
\vs 1Cl 65:3
дабы и мы скорее могли порадоваться о вашем благоустройстве.
\vs 1Cl 65:4
Благодать Господа нашего Иисуса Христа да будет
с вами и со всеми, которые повсюду призваны Богом и чрез него:
\vs 1Cl 65:5
чрез которого ему слава,
честь, держава и величие, престол вечный,
от веков во веки веков. Аминь.

\bibbookdescr{3Co}{
  inline={Третье Послание к Коринфянам Святого Апостола Павла},
  toc={3-е Коринфянам},
  bookmark={3-е Коринфянам},
  header={3-е Коринфянам},
  abbr={3~Кор}
}
\vs 3Co 1:1
Павел, узник Иисуса Христа, братьям в Коринфе~--- радоваться!
\vs 3Co 1:2
Так как я пребываю во многих бедах,
не удивляюсь тому, что учение лукавого столь быстро множится.
\vs 3Co 1:3
Потому и придёт вскоре Господь Иисус Христос,
что отвергнут он теми, кто извращает слова его.
\vs 3Co 1:4
Передавал же я изначально вам то,
что получил от апостолов,
которые прежде меня были и всё время с
Господом Иисусом Христом пребывали:
\vs 3Co 1:5
что был Господь наш Иисус Христос Марией рождён
от семени Давидова, когда ниспослан был в неё Отцом
с небес Дух Святой,
\vs 3Co 1:6
чтобы мог он прийти в сей мир и всякую плоть
искупить плотью своею и во плоти нас из мертвых
поднять, и явил он собой пример нам в том.
\vs 3Co 1:7
И поскольку человек был сотворён его отцом,
\vs 3Co 1:8
будучи пропавшим, он был найден,
дабы воскреснуть через усыновление.
\vs 3Co 1:9
И потому послал сперва Всемогущий Бог,
сотворивший небо и землю,
пророков евреям, чтобы избавились те от грехов своих;
\vs 3Co 1:10
и потому положил он спасти дом Израилев,
что посылал он частицу Духа Христова пророкам,
которые во многие времена возвещали
безупречное почитание Бога.
\vs 3Co 1:11
Но поскольку возжелал князь неправедный сам
Богом быть, налагал он руки на них и истреблял пророков,
и потому страстями опутана всякая плоть человеческая.
\vs 3Co 1:12
Но справедлив Бог Всемогущий,
не отрекается он от творений своих,
\vs 3Co 1:13
и он послал в огне духа в Марию Галилеянку,
\vs 3Co 1:14
веровавшую всем сердцем своим, и приняла
она Духа Святого во чреве своём,
дабы Иисусу в сей мир явиться
\vs 3Co 1:15
с тем, чтобы сокрушён был лукавый тою же плотью,
через которую он приобрёл власть,
и убедился, что не Бог он вовсе.
\vs 3Co 1:16
И потому собственным телом своим спас
Иисус Христос всякую плоть и привёл
её через веру в жизнь вечную,
\vs 3Co 1:17
чтобы храм праведности мог явить он телом своим,
\vs 3Co 1:18
которым мы искуплены.

\vs 3Co 1:19
Так что они дети не праведности, но сыны зла,
отрицающие истину вопреки промыслу Божьему,
говорящие, будто земля и небо и всё,
что в них, не отцом созданы.
\vs 3Co 1:20
Самые что ни есть они сыны зла,
ибо исповедуют они проклятую веру змея.
\vs 3Co 1:21
Отвернитесь от них и учения их бегите!
\vs 3Co 1:22
Ибо вы сыны не строптивости,
но сыны церкви возлюбленной.
\vs 3Co 1:23
И сего-то ради возвещаются воскресения сроки.
\vs 3Co 1:24
А что до тех, кто говорит вам,
будто нет воскресения во плоти,
\vs 3Co 1:25
то для них-то и нет воскресения,
ибо в того не верят, кто воскрес уже.
\vs 3Co 1:26
И воистину неведомо им,
о мужи коринфские, как пшеницу сеют или семена иные,
что бросают их голыми в землю, и когда уже истлеют в ней,
поднимаются вновь они волей Божьей, обретая тело и одежду.
\vs 3Co 1:27
И не только в том поднимаются теле,
которое в землю брошено было,
но приумноженные изобильно.
\vs 3Co 1:28
А если нас не должно сравнивать с одним лишь зерном, то
\vs 3Co 1:29
ведомо вам, что Иона,
сын Амафии, не пожелавший учить в Ниневии,
был китом проглочен,
\vs 3Co 1:30
а через 3 дня и 3 ночи Бог услышал из глубин
преисподней молитву Ионы, и не повредился ни единый
член его, даже волос или ресница.
\vs 3Co 1:31
Тем более воскресит он вас,
уверовавших во Христа Иисуса,
подобно тому, как и сам он воскрес.
\vs 3Co 1:32
И если труп ожил, сброшенный сынами Израилевыми
на кости пророка Елисея, то вы тем более воскреснете,
ибо брошены вы на тело и кости и дух Господни,
и восстанете в сей же день целыми во плоти своей.
\vs 3Co 1:33
Подобно и об Илии пророке известно,
что он воскресил из мёртвых сына вдовицы;
тем более вас Господь Иисус во гласе трубы,
во мгновение ока, воскресит,
ибо он показал нам образ в своём теле.

\vs 3Co 1:34
А ежели принимаете вы и иное что-то,
то уж не отягощайте меня.
\vs 3Co 1:35
Ведь для того на руках моих оковы,
чтобы мог я Христа приобрести,
и для того язвы его на теле моем,
дабы я мог достигнуть воскресения из мёртвых.

\vs 3Co 1:36
И всякий, кто живет по заповедям,
которые он получил от блаженных пророков
и святого благовествования, получит награду и,
возстав из мёртвых, обретёт жизнь вечную.
\vs 3Co 1:37
Тот же, кто отступает от них,
пусть горит в огне, с теми вместе,
кто ведёт его такой дорогой,
\vs 3Co 1:38
потому что они безбожные люди и ехиднино порождение.
\vs 3Co 1:39
Отвергнитесь от них силой Господней,
\vs 3Co 1:40
и да пребудут с вами мир и любовь, и милость. Аминь.

\bibbookdescr{Lao}{
  inline={Послание к Лаодикийцам Святого Апостола Павла},
  toc={к Лаодикийцам},
  bookmark={к Лаодикийцам},
  header={к Лаодикийцам},
  abbr={Лао}
}
\vs Lao 1:1
Павел, апостол не от человеков и не через человека,
но Иисусом Христом,~--- братьям, которые в Лаодикии.
\vs Lao 1:2
Благодать вам и мир от Бога Отца и Господа Иисуса Христа.

\vs Lao 1:3
Благодарю Христа всякою молитвою моею,
что пребываете в нём и твёрдо стоите в делах его,
ожидая обетования в день суда.
\vs Lao 1:4
И да не погубят вас пустословия некоторых
вкравшихся в доверие, чтобы отвратить вас
от истины Евангелия, которое мною проповедуется.
\vs Lao 1:5
И ныне устраивает Бог, что те, кто без меня~--- служат
к успеху истины благовествования и творят благость
дел спасения, жизни вечной;
\vs Lao 1:6
и ныне явны узы мои, которые терплю во Христе,
которыми веселюсь и радуюсь.
\vs Lao 1:7
И сие мне есть ко спасению вечному,
которое соделывается и вашими молитвами,
и служению Святого Духа, жизнью ли, или смертью.
\vs Lao 1:8
Ибо мне воистину жизнь во Христе и умереть~--- радость.
\vs Lao 1:9
И сам он в вас сотворит милость свою,
чтобы сию любовь имели вы, и пребывали единодушны.
\vs Lao 1:10
Посему, возлюбленные, что слышали в моём присутствии,
то сохраняйте и творите в страхе Божьем,
и да будет вам жизнь вовеки;
\vs Lao 1:11
ибо действующий в вас Бог~--- Господь.
\vs Lao 1:12
И что бы вы ни делали~--- делайте неотступно.
\vs Lao 1:13
То есть, возлюбленные, радуйтесь о Христе.
И остерегайтесь нечистых в корысти.
\vs Lao 1:14
Да будут все прошения ваши явны пред Богом.
И будьте непоколебимы в духе Христовом.
\vs Lao 1:15
И что непорочно, и истинно, и честно, и праведно,
и любезно~--- творите.
\vs Lao 1:16
И что слышали и приняли,
в сердце сохраните, да будет вам мир.

\vs Lao 1:17
Приветствуйте всех братьев целованием святым.
\vs Lao 1:18
Приветствуют вас святые.
\vs Lao 1:19
Благодать Господа Иисуса с духом вашим.
\vs Lao 1:20
Распорядитесь, чтобы послание Колоссянам было прочитано вам.

\include{tex/Brn}
\bibbookdescr{1Er}{
  inline={Пастырь Ермы. Книга 1. Видения},
  toc={1-я Ермы},
  bookmark={1-я Ермы},
  header={1-я Ермы},
  abbr={1~Ермы}
}
\chhdr{Видение 1-е.}
\vs 1Er 1:1
Воспитатель мой продал в Риме одну отроковицу.
По прошествии многих лет я увидел её, узнал и полюбил как сестру.
\vs 1Er 1:2
Через некоторое время, увидев, что она купается в реке Тибр,
я подал ей руку и вывел из реки.
\vs 1Er 1:3
Глядя на ее красоту, я думал:
<<Счастлив бы я был, если бы имел жену такую же и лицом и нравом.>>
Только это, и ничего более я не подумал.
\vs 1Er 1:4
Позже шёл я с такими мыслями и прославлял творение Божье,
раздумывая, сколь величественно оно и прекрасно.
\vs 1Er 1:5
Во время прогулки я заснул, и дух подхватил меня
и понёс куда-то, через местность,
по которой человек не мог пройти.
Была она скалиста, крута и непроходима из-за вод.
\vs 1Er 1:6
Миновав её, я достиг равнины и, преклонив колена,
начал молиться Господу и исповедовать грехи свои.
\vs 1Er 1:7
И во время моей молитвы отверзлось небо
и увидел я ту женщину, которую пожелал себе.
\vs 1Er 1:8
Она приветствовала меня с неба:
<<Здравствуй, Ерма.>>
\vs 1Er 1:9
Взглянув на неё, я спросил:
<<Госпожа, что ты здесь делаешь?>>
\vs 1Er 1:10
Я взята сюда, чтобы обличить пред Господом грехи твои,~--- она ответила.
\vs 1Er 1:11
Госпожа, ужели ты меня будешь обвинять?
\vs 1Er 1:12
Нет, но выслушай слова, которые хочу сказать тебе.
Бог, живущий на небесах, сотворивший из ничего всё
сущее и умноживший ради святой Церкви своей,
гневается на тебя за то, что ты согрешил против меня.
\vs 1Er 1:13
Госпожа, если я согрешил против тебя,
то каким образом?~--- спросил я.~--- Где или когда я сказал тебе
какое-нибудь дурное слово?
\vs 1Er 1:14
Не всегда ли я уважал тебя как госпожу;
не всегда ли я почитал тебя как сестру?
Что же наговариваешь на меня столь дурное?
\vs 1Er 1:15
Тогда она, улыбаясь, ответила мне:
<<В сердце твоём возникло нечистое пожелание.
Ужели не думаешь, что для человека праведного и то порочно,
если в сердце его возникает худое пожелание?
Это~--- грех для него, и притом тяжкий.
\vs 1Er 1:16
Ибо человек праведный и помышляет праведное.
И когда он помышляет праведное и неуклонно к тому стремится,
то имеет на небесах благоволение Господа во всяком деле.
\vs 1Er 1:17
Те же, которые затаили нечистое в сердцах своих,
навлекают на себя смерть и тлен; особенно те,
которые любят настоящий век, роскошествуют
в богатстве своём и не ожидают благ будущих,~--- гибнут души их.
\vs 1Er 1:18
А это делают двоедушные, которые не имеют надежды
в Господе, не радеют о своей жизни.
\vs 1Er 1:19
Но ты молись Господу, и исцелит он грехи твои,
и всего дома твоего, и всех святых.>>

\vs 1Er 2:1
После того как произнесла она эти слова, небеса заключились.
\vs 1Er 2:2
И я, весь в скорби и страхе, сказал себе:
<<Если это вменяется мне в грех, то как могу спастись или
каким образом умолю Господа о бесчисленных грехах моих?
Какими словами упрошу Господа быть ко мне милостивым?>>
\vs 1Er 2:3
Размышляя так, увидел я вдруг перед собой большую кафедру,
словно сотворённую из в\acc{о}лны, белой как снег.
\vs 1Er 2:4
И пришла старая женщина в блестящей одежде с книгою в руке,
села одна и приветствовала меня:
<<Здравствуй, Ерма.>>
\vs 1Er 2:5
И я, в печали и слезах, ответил:
<<Здравствуй, госпожа.>>
\vs 1Er 2:6
Она спросила:
<<Что печален, Ерма, ты, который был терпелив, умерен и всегда весел?>>
\vs 1Er 2:7
Госпожа, одна прекрасная женщина, укорила меня,
будто я согрешил против неё,~--- ответил я.
\vs 1Er 2:8
И она сказала мне:
<<В сердце твоё м возникло вожделение к ней.
Это должно быть чуждо рабу Господню,
ведь для рабов Божьих даже и такой помысел составляет грех.
\vs 1Er 2:9
И сердце чистое не должно желать дурного~--- особенно твоё, Ерма;
ты избегаешь всякого преступного пожелания
и исполнен простоты и великого незлобия.

\vs 1Er 3:1
Впрочем, не ради тебя гневается на тебя Господь,
но за дом твой, который впал в нечестие перед
Господом и своими родителями.
\vs 1Er 3:2
И ты, любя детей, не вразумлял своего семейства,
но позволил им сильно развратиться.
\vs 1Er 3:3
За это и гневается на тебя Господь,
но он исправит всё, что прежде сделано худого в доме твоём.
\vs 1Er 3:4
За их грехи и беззакония ты подавлен мирскими делами.
\vs 1Er 3:5
Но милосердие Божье сжалилось над тобою и семейством твоим
и сохранило тебя в славе.
\vs 1Er 3:6
Ты только не колеблись, но будь благодушен и укрепляй свое семейство.
\vs 1Er 3:7
Как кузнец, усердно работая молотом,
совершает свой труд, так и праведное слово
ежедневное победит всякое зло.
\vs 1Er 3:8
Поэтому не переставай вразумлять детей своих,
ибо Господь знает, что они покаются от всего сердца
своего и будут написаны в Книге жизни.>>
\vs 1Er 3:9
Сказав это, она спросила меня:
<<Хочешь послушать, что я буду читать?>>
\vs 1Er 3:10
Хочу, госпожа,~--- ответил я.
\vs 1Er 3:11
Итак, слушай.
И, раскрыв книгу, она читала величественные и дивные слова,
которых не мог я удержать в памяти, ибо были они страшны,
человек не мог вынести их.
\vs 1Er 3:12
Впрочем, самые последние слова я запомнил,
так как были они краткими и отрадными для нас:
\vs 1Er 3:13
<<Вот Бог Саваоф, который невидимою силою
и великим своим разумом сотворил мир,
и славным светом своим благоукрасил тварь,
\vs 1Er 3:14
и всесильным словом своим утвердил небо,
и землю основал на водах, и всемощной силой своею создал свою
святую Церковь, которую и благословил.
\vs 1Er 3:15
Вот, он изменит небеса и горы, холмы и моря,
и всё уравняется для избранных его,
\vs 1Er 3:16
чтобы исполнить обещание, которое он дал,
с великою славою и торжеством, если они соблюдут заповеди
Божьи, полученные ими с великою верою.>>

\vs 1Er 4:1
Окончив чтение, она встала с кафедры;
и пришли четверо юношей и понесли кафедру на восток.
\vs 1Er 4:2
А она подозвала меня к себе и, коснувшись груди моей, спросила:
<<Понравилось ли тебе мое чтение?>>
\vs 1Er 4:3
Госпожа, самое последнее мне нравится,
но предыдущее страшно и жестоко.
\vs 1Er 4:4
И она сказала:
<<Эти последние слова относятся к праведным,
а первые~--- к отступникам и народам.>>
\vs 1Er 4:5
В это время явились 2 каких-то мужа,
подняли её на плечи и отправились вслед за кафедрой, на восток.
\vs 1Er 4:6
Она удалилась весёлая и на прощание произнесла:
<<Мужайся, Ерма!>>

\chhdr{Видение 2-е.}
\vs 1Er 5:1
Гуляя в окрестностях Кумских в то же примерно время,
что и в прошлом году, вспомнил я о прежнем видении,
и снова вознёс меня дух туда же, где прежде.
\vs 1Er 5:2
Достигнув того места,
я преклонил колена и начал молиться Господу
и прославлять имя его за то, что он удостоил меня
и открыл мне прежние грехи мои.
\vs 1Er 5:3
И когда восстал я от молитвы,
увидел пред собою ту старицу,
которую видел прежде: она гуляла и читала какую-то книгу.
\vs 1Er 5:4
Можешь ли возвестить это избранникам Божьим?~--- спросила она меня.
\vs 1Er 5:5
Я ответил:
<<Госпожа, так много я не могу запомнить, но дай мне книгу; я перепишу.>>
\vs 1Er 5:6
Возьми,~--- сказала она,~--- а потом возврати её мне.
\vs 1Er 5:7
Взяв книгу, я удалился в поле и списал всё буква в букву,
не понимая смысла.
\vs 1Er 5:8
И когда окончил я списывание книги,
вдруг забрали её из рук моих, но кто это был~--- не увидел я.

\vs 1Er 6:1
Спустя 15 дней, в которые я постился
и много молился Господу открылся мне смысл написанного.
\vs 1Er 6:2
Написано было следующее:
<<Дети твои, Ерма, отступили от Господа, хулили его и в великом нечестии
предали своих родителей; и прослыли они предателями родителей;
\vs 1Er 6:3
предавши их, они не исправились,
но присоединили к грехам своим распутство и нечестие скверны и
таким образом исполнили неправды свои.
\vs 1Er 6:4
Объяви эти слова всем детям своим и жене своей,
так как и она не воздержана в речах своих и тем согрешает.
\vs 1Er 6:5
Услышав же эти слова, она обуздает свой язык
и заслужит помилование.
\vs 1Er 6:6
Она образумится после того, как передашь ей слова,
которые Господь повелел открыть тебе.
\vs 1Er 6:7
Тогда отпустятся грехи, совершённые прежде,
как им, так и всем святым, если от всего сердца покаются
они и удалят сомнения из сердец своих.
\vs 1Er 6:8
Ибо славою своею поклялся Господь,
что тот из избранных его, кто и в этот предопределённый день будет
продолжать грешить, не получит спасения.
\vs 1Er 6:9
Ибо покаянию праведных положены сроки,
и определены дни покаяния для всех святых,
но народам позволено каяться до самого последнего дня.
\vs 1Er 6:10
Поэтому скажи настоятелям Церкви,
чтобы они совершали пути свои в истине,
дабы могли получить обетования со многою славою.
\vs 1Er 6:11
И вы, праведники, стойте твердо и не будьте двоедушны,
чтобы переселение ваше было со святыми ангелами.
\vs 1Er 6:12
Блаженны те, кто претерпит наступающее великое гонение
и не отречётся от своей жизни,
\vs 1Er 6:13
ибо сыном своим поклялся Господь,
что отрекающиеся от Господа губят свою жизнь.
\vs 1Er 6:14
Это относится к тем, которые отрекутся в предстоящие дни;
\vs 1Er 6:15
к тем же, которые прежде отрекались,
по великому милосердию он сделался милостивым.

\vs 1Er 7:1
А ты, Ерма, не помни неправды детей своих
и не оставляй жены своей, но позаботься о том, чтобы они
освободились от прежних грехов.
\vs 1Er 7:2
Они образумятся правым
учением, если ты не будешь держать зла на них.
\vs 1Er 7:3
Ибо злопамятство приводит
к смерти, забвение зла~--- к жизни вечной.
\vs 1Er 7:4
А ты, Ерма, потерпел большие мирские бедствия
за преступления дома твоего, поскольку не обращал на
них внимания как на не касающиеся тебя нисколько
и предался неправедным своим занятиям.
\vs 1Er 7:5
Но то, что не отступил ты от живого Бога, спасёт тебя;
простота твоя и великое воздержание спасут тебя,
если ты пребудешь в них; и всех спасут они,
кто поступает так же.
\vs 1Er 7:6
Пребывающие в невинности и простоте будут
сильны против всякого зла и обретут жизнь вечную.
\vs 1Er 7:7
Блаженны все делающие правду: они не погибнут вовек.
\vs 1Er 7:8
Но скажешь: вот приходит великое гонение.
Если тебе кажется, то опять отрекись.
\vs 1Er 7:9
Господь близок к обращающимся,
как написали в книгах Елдада и Модада,
которые в пустыне пророчествовали народу.>>

\vs 1Er 8:1
Во время сна моего,
братия, один красивый юноша явился мне и спросил:
<<Кто, ты думаешь, та старица, от которой получил ты книгу?>>
\vs 1Er 8:2
Сивилла,~--- ответил я.
\vs 1Er 8:3
Ошибаешься,~--- сказал он,~--- она не сивилла.
\vs 1Er 8:4
Кто же она, господин?
\vs 1Er 8:5
Она есть Церковь Божья.
\vs 1Er 8:6
Я спросил его, почему же она стара.
\vs 1Er 8:7
Так как,~--- объяснил он,~--- сотворена она прежде всего,
и для неё сотворён мир.
\vs 1Er 8:8
После того было мне видение в доме моём,
и пришла та старица и спросила меня,
отдал ли я уже книгу предстоятелям Церкви.
\vs 1Er 8:9
Я отвечал, что нет ещё, и она сказала:
<<Хорошо, потому что я добавлю ещё несколько слов.
\vs 1Er 8:10
Когда же исчерпаю все слова,
тогда пусть через тебя они дойдут до избранных.
\vs 1Er 8:11
Для этого ты напишешь 2 книги
и одну отдашь Клименту; а другую~--- Гранте.>>
\vs 1Er 8:12
Климент отошлёт во внешние города, ибо ему это предоставлено;
Гранта же будет назидать вдов и сирот.
\vs 1Er 8:13
А ты прочтёшь её в этом городе вместе с пресвитерами,
предстоятелями Церкви.

\chhdr{Видение 3-е.}
\vs 1Er 9:1
Было мне, братья, следующее видение.
После того как я много раз постился и молил
Господа об откровении, которое было обещано мне чрез ту старицу,
\vs 1Er 9:2
ночью явилась старица и сказала:
<<Так как ты очень просишь и желаешь знать всё,
то приходи в поле и около 6-и часов я явлюсь тебе и покажу то,
что нужно тебе видеть.>>
\vs 1Er 9:3
Я спросил её:
<<На каком месте поля?>>
\vs 1Er 9:4
Она говорит:
<<Где хочешь; место же выбери сам.>>
\vs 1Er 9:5
И я избрал место прекрасное, уединенное.
Но прежде, нежели начал я говорить и сказал ей о месте,
она говорит мне:
<<Приду, куда пожелаешь.>>
\vs 1Er 9:6
Итак, братья, заметил я часы и явился на поле,
к месту куда назначил ей прийти.
\vs 1Er 9:7
И вижу я поставленную скамью, на ней льняная подушка,
а над скамьей простёрта парусина.
\vs 1Er 9:8
Видя такие приготовления,
между тем как никого нет на месте,
я изумился, волосы у меня поднялись,
и ужас объял меня оттого, что я был один.
\vs 1Er 9:9
Но придя в себя и вспомнив славу Божью,
я ободрился и, преклонив колена,
исповедал Богу свои грехи, как всегда.
\vs 1Er 9:10
Вот, пришла старица с 6-ю юношами,
которых я прежде видел, и, ставши позади меня,
слушала, как я молился и исповедовался перед Богом.
\vs 1Er 9:11
Коснувшись меня, она сказала:
<<Перестань молиться только о грехах своих,
молись и о правде, чтобы часть из неё получил ты для дома своего.>>
\vs 1Er 9:12
Взяв меня за руку, она привела меня к скамейке
и велела тем юношам:
<<Идите и стройте.>>
\vs 1Er 9:13
Когда мы остались одни, она сказала мне:
<<Садись здесь.>>
\vs 1Er 9:14
Госпожа, пусть прежде сядут пресвитеры.
\vs 1Er 9:15
Я тебе говорю, настаивала она,~--- садись.
\vs 1Er 9:16
Я хотел было сесть по правую сторону,
но она рукою показала,
чтобы садился я по левую сторону.
\vs 1Er 9:17
Когда опечалился я, что не позволила сесть мне
по правую сторону, она проговорила:
<<Не печалься, Ерма.
Место по правую сторону принадлежит тому
кто уже угодил Богу и пострадал за имя его.>>
\vs 1Er 9:18
У тебя много недостает для того, чтобы сидеть с ними.
Но оставайся в простоте своей, как прежде, и будешь сидеть с ними,
\vs 1Er 9:19
равно как и все, кто будет творить дела их и претерпит то,
что они претерпели.>>

\vs 1Er 10:1
Я сказал ей:
<<Госпожа, я желал бы узнать, что они претерпели.>>
\vs 1Er 10:2
Слушай:
<<Лютых зверей, бичевание, темницы, кресты ради имени его.
За это принадлежит правая сторона святыни им и всякому,
кто пострадает за имя Божье, а остальным~--- левая сторона.
\vs 1Er 10:3
Но для тех и других, и для сидящих по правую сторону
и для сидящих по левую,~--- одни и те же дары обетования;
только сидящие по правую сторону имеют некоторую честь.
\vs 1Er 10:4
Ты желаешь сидеть по правую сторону с ними,
но у тебя много слабостей.
Очисти себя от своих слабостей,
и все недвоедушные должны очиститься к тому дню от своих слабостей.>>
\vs 1Er 10:5
Сказав это, она хотела удалиться, но я бросился к ногам её
и умолял её Господом, чтобы явила мне обещанное видение.
\vs 1Er 10:6
И она опять взяла меня за руку,
подняла и посадила на скамейку по левую сторону и,
поднимая какой-то блестящий жезл, спросила:
<<Видишь ли большую работу?>>
\vs 1Er 10:7
Госпожа, ничего не вижу.
\vs 1Er 10:8
Неужели не видишь против себя великой башни,
которая на водах строится из блестящих квадратных камней?
\vs 1Er 10:9
Действительно, строилась квадратная башня теми 6-ю юношами,
которые пришли с нею.
Многие тысячи других мужей приносили камни.
\vs 1Er 10:10
Некоторые доставали камни со дна,
другие из земли и подавали тем 6-и юношам,
они же принимали их и строили.
\vs 1Er 10:11
Камни, извлечённые со дна, сразу клали в здание,
потому что они были гладкие и ровные и так примыкали один к другому,
что соединения их нельзя было заметить,
и башня казалась возведенной из одного камня.
\vs 1Er 10:12
Камни же, принесённые из земли,
не все использовались для строительства.
Некоторые из них строители откладывали,
потому что были они шероховаты,
или с трещинами, или светлы и круглы
и не годились для здания башни.
\vs 1Er 10:13
А некоторые камни они раскалывали и отбрасывали далеко в сторону.
И отброшенные камни, видел я, падали на дорогу и,
не оставаясь на ней, скатывались:
\vs 1Er 10:14
одни~--- в место пустынное,
другие попадали в огонь и горели,
иные падали близ воды и не могли скатиться в воду;
хотя и стремились попасть в неё.

\vs 1Er 11:1
Показав мне это, старица хотела удалиться, но я сказал:
<<Госпожа, какая польза мне видеть, но не понимать,
что значит это строение?>>
\vs 1Er 11:2
Она отвечала мне:
<<Любопытный ты человек, если желаешь понять значение башни.>>
\vs 1Er 11:3
Действительно, госпожа,
говорю я,~--- желаю узнать и возвестить братьям, чтобы и они возрадовались,
услышав это, и прославили Господа.
\vs 1Er 11:4
Услышат многие.
И, услышавши, некоторые возрадуются, а другие восплачут;
\vs 1Er 11:5
впрочем, и последние, если, услышавши, принесут покаяние,
также будут радоваться.
\vs 1Er 11:6
Выслушай теперь объяснение башни, я открою всё,
и не докучай мне более об откровении.
\vs 1Er 11:7
Откровения эти закончились, ибо имеют предел.
А ты не перестаёшь требовать откровений, потому что настойчив.
\vs 1Er 11:8
Итак, башня, которую видишь строящейся,~--- это я, Церковь,
которая явилась тебе теперь и прежде.
\vs 1Er 11:9
Спрашивай же что хочешь о башне, и я открою тебе,
чтобы возрадовался ты со святыми.
\vs 1Er 11:10
Госпожа, если однажды сочла ты меня достойным того,
чтобы всё открыть мне, то открой,~--- просил я старицу.
\vs 1Er 11:11
Всё, что следует открыть тебе, откроется,
только бы сердце твоё было с Господом
и ты не сомневался, что бы ни увидел.
\vs 1Er 11:12
Госпожа,~--- спросил я её, почему башня построена на водах?
\vs 1Er 11:13
И прежде я говорила тебе, отвечала она,~--- что ты любопытен
и усердно изыскиваешь; ища~--- найдёшь истину:
\vs 1Er 11:14
слушай же, почему башня строится на водах:
жизнь ваша через воду спасена и спасётся.
А башня основана словом всемогущего и преславного имени
и держится невидимою силою Господа.

\vs 1Er 12:1
Я на это сказал ей:
<<Величественное и дивное дело!
А кто, госпожа, те 6 юношей, которые строят?>>
\vs 1Er 12:2
Это~--- первозданные ангелы Божии,
которым Господь вверил всё своё творение для того,
чтобы они умножали, благоустраивали и управляли его творением:
их силами и будет окончено строительство башни.>>
\vs 1Er 12:3
А кто те остальные, которые приносят камни?
\vs 1Er 12:4
И это~--- святые ангелы Господа, но первые выше.
Когда окончится строительство башни, они все вместе
будут ликовать около башни и прославлять Господа за то,
что совершилось строительство башни.
\vs 1Er 12:5
Желал бы я знать,~--- сказал я,~--- какое значение
и в чём различие камней.
\vs 1Er 12:6
И она отвечала мне:
<<Разве ты лучше всех, чтобы тебе это было открыто?
Есть более достойные, которым следовало бы открыть эти видения.
\vs 1Er 12:7
Но, чтобы прославлялось имя Божье,
тебе это открыто и ещё откроется ради тех, кто имеет сомнение в
сердце своем, будет ли это или нет.
\vs 1Er 12:8
Скажи им, что всё это истинно и что ничего нет ложного,
но всё твердо и крепко основано.

\vs 1Er 13:1
Выслушай теперь и о камнях, на которых возведено здание.
\vs 1Er 13:2
Камни квадратные и белые,
хорошо приходящиеся один к другому своими соединениями,
это суть апостолы, епископы, учителя и дьяконы,
\vs 1Er 13:3
которые ходили в святом учении Божьем,
надзирали и свято и непорочно служили
избранникам Божьим как почившие, так и живущие еще доселе,
\vs 1Er 13:4
которые всегда пребывали в мире и согласии
и слушали взаимно друг друга: потому-то они и в здании башни
хорошо примыкают один к другому.
\vs 1Er 13:5
А камни, извлекаемые из
глубины и закладываемые в здание и соприкасающиеся с прочими камнями,
вошедшими в здание, это суть те,
которые уже умерли и пострадали за имя Господа.>>
\vs 1Er 13:6
Госпожа, я желаю знать, кого означают другие камни,
которые достали из земли.
\vs 1Er 13:7
Те, которые неотделанными
кладутся в основание, означают людей, которых Бог одобрил за то, что они жили
праведно пред Господом и исполняли его заповеди.
\vs 1Er 13:8
А которые приносятся и
кладутся в само здание башни, это суть новообращенные к вере и верные.
\vs 1Er 13:9
Ангелами призываются они к
совершению добра, и потому не нашлось в них зла.
\vs 1Er 13:10
А те камни, которые откладываются в сторону возле башни?
\vs 1Er 13:11
Она ответила:
<<Это те, которые согрешили и желают покаяться;
потому они брошены невдалеке от башни,
что будут пригодны, если покаются.
\vs 1Er 13:12
Посему желающие покаяться
будут тверды в вере, если только принесут покаяние теперь,
пока строится башня.
\vs 1Er 13:13
Ибо когда строительство окончится,
то им уже не найдётся места, и они, отверженные,
только останутся лежать при башне.

\vs 1Er 14:1
Желаешь знать, кто те,
которые раскалывают и отбрасывают далеко от башни?
\vs 1Er 14:2
Желаю, госпожа.
\vs 1Er 14:3
Это суть сыны беззакония,
которые уверовали притворно и от которых не отступила неправда всякого рода;
\vs 1Er 14:4
потому они не имеют спасения,
что не годны в здание по неправедности своей,~--- они расколоты и
отброшены далеко по гневу Господа за то, что оскорбили его.
\vs 1Er 14:5
А значение прошлых камней,
которые во множестве видел ты сложенными
и не использованными в строительстве, таково.
\vs 1Er 14:6
Шероховатые суть те,
которые познали истину; но не остались в ней и не находятся в общении со
святыми, потому они и не годны.
\vs 1Er 14:7
Камни с трещинами~--- это
суть те, которые держат в сердцах вражду друг к другу; будучи вместе, они
миролюбивы, но, разойдясь, обретают в сердцах злобу.
И эта злоба~--- трещины в камнях.
\vs 1Er 14:8
Камни меньшего размера
это те люди, которые, хоть и уверовали, но имеют еще много неправды, поэтому
они коротки.
\vs 1Er 14:9
Кто же, госпожа, белые и
круглые, что тоже не идут в здание башни?
\vs 1Er 14:10
Она отвечала мне:
<<Доколе ты будешь глуп и неразумен?
Ты обо всём спрашиваешь и ничего не понимаешь.>>
\vs 1Er 14:11
Белые и круглые камни
это те, которые имеют веру, но имеют и богатства века сего; и когда придёт
гонение, то ради богатств своих и попечений они отрекутся от Господа.
\vs 1Er 14:12
Когда же будут они угодны Господу?
\vs 1Er 14:13
Когда отсечены будут богатства их, которые их утешают,
тогда они будут полезны Господу для здания.
\vs 1Er 14:14
Ибо как круглый камень,
пока не будет обсечен и не лишится некоторых своих частей, не сможет стать
квадратным, так и богатые в нынешнем веке, если не лишатся своих богатств, не
смогут быть угодными Господу.
\vs 1Er 14:15
Прежде всего ты должен знать это по себе самому:
когда ты был богат, был бесполезен; а теперь ты
полезен и годен для жизни; ты и сам был из тех камней.

\vs 1Er 15:1
Прочие же камни, которые
ты видел, были отброшены далеко от башни, катились по дороге и с дороги
скатывались в места пустынные, означают тех, которые, хотя уверовали, но, по
сомнению своему, оставили истинный путь, думая, что они могут найти лучший.
\vs 1Er 15:2
Но они обольщаются и бедствуют, ходя по путям пустынным.
\vs 1Er 15:3
Камни, упавшие в огонь и
горевшие, означают тех, которые навсегда отказались от живого Бога и которым,
по причине преступных похотей, ими творимых, уже не приходит мысль покаяться.
\vs 1Er 15:4
Кого же означают камни,
которые падали близ воды и не могли скатиться в неё?
\vs 1Er 15:5
Тех, которые слышали Слово
и желают креститься во имя Господа, когда приходит им на память святость
истины, но потом они уклоняются и опять предаются своим порочным пожеланиям.
\vs 1Er 15:6
Так она окончила объяснение башни.
\vs 1Er 15:7
Но я, будучи настойчив, спросил её:
<<Есть ли покаяние для тех камней, которые отброшены,
и будет ли им место в этой башне?>>
\vs 1Er 15:8
Она сказала:
<<Есть для них покаяние; но в этой башне не найдут они места,
а попадут в иное, низшее место,
причем когда они пострадают и исполнятся дни грехов их.
\vs 1Er 15:9
И за то они будут
переведены, что приняли Слово истинное.
\vs 1Er 15:10
И тогда избавятся они от
наказаний своих, когда содрогнутся сердцем от порочных дел,
ими сотворенных, и они покаются.
\vs 1Er 15:11
Если же они не опомнятся,
то не спасутся из-за упорства своего сердца.>>

\vs 1Er 16:1
Когда я перестал спрашивать старицу обо всём этом,
она предложила:
<<Хочешь увидеть ещё что-то?>>
\vs 1Er 16:2
И так как я очень желал
увидеть, то радость отразилась на лице моём.
\vs 1Er 16:3
Взглянув на меня, она улыбнулась
и спросила:
<<Видишь 7 женщин вокруг башни?>>
\vs 1Er 16:4
Вижу, госпожа.
\vs 1Er 16:5
Башня эта по распоряжению Господа ими поддерживается.
\vs 1Er 16:6
Слушай теперь об их действиях.
Первая из них, которая держит башню руками, называется Верою;
посредством неё спасаются избранники Божьи.
\vs 1Er 16:7
Другая же, которая препоясана и ведет себя мужественно,
называется Воздержанием, она~--- дочь Веры.
\vs 1Er 16:8
Кто последует за нею, будет блажен в своей жизни,
ибо удержится от всех худых дел и всякой злой
похоти и станет наследником вечной жизни.
\vs 1Er 16:9
Кто же другие 5, госпожа?
\vs 1Er 16:10
Дочери одна другой.
Одна называется Простотою,
другая~--- Невинностью,
3-я~--- Скромностью,
4-я~--- Знанием,
5-я~--- Любовью.
\vs 1Er 16:11
Поэтому, когда исполнишь дела матери их,
тогда сможешь и всё соблюсти.
\vs 1Er 16:12
Хотел бы я знать, госпожа, какую каждая из них имеет силу?
\vs 1Er 16:13
Слушай,~--- отвечала она, силы их одинаковы:
они связаны между собою и следуют одна за другою,
как и рождены.
\vs 1Er 16:14
От Веры рождается Воздержание,
от Воздержания Простота,
от Простоты Невинность,
от Невинности Скромность,
от Скромности Знание,
от Знания Любовь.
\vs 1Er 16:15
Действия их чисты, целомудренны и святы,
и кто послужит им и будет в силе исполнять дела их, тот
будет иметь обитель в башне со святыми Божьими.
\vs 1Er 16:16
Я спросил её о времени, не конец ли уж теперь.
\vs 1Er 16:17
Но она громко воскликнула:
<<Неразумный человек!
Неужели не видишь ты, что башня всё ещё строится?
Когда башня будет построена, тогда и будет конец.
\vs 1Er 16:18
Не спрашивай у меня ничего более.
И этого напоминания и обновления душ ваших
достаточно для тебя и для всех святых.
\vs 1Er 16:19
Не для тебя одного это открыто, но чтобы ты возвестил всем.
\vs 1Er 16:20
Итак, по прошествии 3-х дней ты, Ерма,
должен уразуметь следующие слова, которые имею сказать тебе,
чтобы ты довел их до ушей святых, чтобы, слушая и исполняя их,
очистились от своих неправд~--- и ты вместе с ними.>>

\vs 1Er 17:1
Послушайте меня, дети.
Я воспитала вас в великой простоте,
невинности и целомудрии, по милосердию Господа,
\vs 1Er 17:2
Который излил на вас
правду, чтобы вы очистились от всякого беззакония и лжи, а вы не хотите
отступиться от неправд ваших. Итак, теперь послушайте меня.
\vs 1Er 17:3
Живите в мире, заботьтесь
друг о друге, поддерживайте себя взаимно и не пользуйтесь одни творениями
Божьими, но щедро раздавайте нуждающимся.
\vs 1Er 17:4
Некоторые от многих яств
наносят вред своей плоти и истощают её.
А у других, не имеющих пропитания,
также истощается плоть оттого,
что нет в достатке пищи и гибнут тела их.
\vs 1Er 17:5
Такое невоздержание пагубно для тех,
кто имеет и не делится с нуждающимися.
Подумайте о грядущем суде.
\vs 1Er 17:6
Вы, кто превосходит других, отыскивайте алчущих,
пока ещё не окончена башня.
\vs 1Er 17:7
Ибо после, когда завершится строительство,
пожелаете благотворить, но не будет вам места.
\vs 1Er 17:8
Итак, смотрите вы, гордящиеся своими богатствами,
чтобы не восстенали терпящие нужду,
\vs 1Er 17:9
стон их взойдет к Господу
и удалены вы будете со своими сокровищами за двери башни.
\vs 1Er 17:10
Тем теперь говорю, кто
начальствует в Церкви и главенствует: не будьте подобны злодеям.
\vs 1Er 17:11
Злодеи, по крайней мере, яд свой носят в сосудах,
а вы отраву свою и яд держите в сердце;
\vs 1Er 17:12
не хотите очистить сердец
ваших и чистым сердцем сойтись в единомыслие,
чтобы иметь милость от Великого Царя.
\vs 1Er 17:13
Смотрите, дети, чтобы такие разделения ваши не лишили вас жизни.
\vs 1Er 17:14
Как хотите вы воспитывать избранников Божьих,
когда сами не имеете научения?
\vs 1Er 17:15
Поэтому вразумляйте себя взаимно и будьте в мире между собою,
чтобы и я, радостно представ пред Отцом вашим,
могла дать отчёт за вас Господу.

\vs 1Er 18:1
Когда она перестала говорить со мною,
пришли те 6 юношей, которые строили,
и понесли её к башне,
а другие 4 взяли скамью и также отнесли её в башню.
\vs 1Er 18:2
Лица сих последних я не видел,
потому что они были обращены в другую сторону.
\vs 1Er 18:3
Когда она удалялась, я просил её объяснить различные облики,
в которых являлась она мне.
\vs 1Er 18:4
Но она сказала в ответ:
<<Это пусть другой объяснит тебе.>>
\vs 1Er 18:5
А явилась она мне, братья, в 1-м видении, в прошлом году,
очень старою, сидящею на кафедре.
\vs 1Er 18:6
Во 2-м видении она имела лицо юное,
но тело и волосы старческие, и беседовала со мною стоя;
впрочем, была веселее, нежели прежде.
\vs 1Er 18:7
В 3-м же видении она вся была гораздо моложе,
с прекрасным лицом, но со старческими волосами;
она была вполне весела и сидела на скамье.
\vs 1Er 18:8
И очень я печалился, что не понятны мне такие различия,
пока не увидел во сне ночном ту старицу,
\vs 1Er 18:9
и она сказала мне:
<<Всякая молитва нуждается в смирении,
поэтому постись и получишь от Господа, чего просишь.>>
\vs 1Er 18:10
Итак, я постился 1 день,
и в ту же ночь явился мне юноша и сказал:
<<Почему ты так часто в молитве просишь откровений?
\vs 1Er 18:11
Смотри, чтобы, прося многого, не повредить тебе своей плоти.
Достаточно для тебя и этих откровений.
\vs 1Er 18:12
Сможешь ли видеть откровения ещё больше тех, которые видел?>>
\vs 1Er 18:13
Господин, я об одном только прошу,
чтобы мне было дано полное объяснение насчёт 3-х обликов той старицы.
\vs 1Er 18:14
Доколе будете вы неразумны?~--- укорил он.~--- Сомнения ваши
делают вас неразумными, потому что не
имеете в сердцах ваших устремления к Господу.
\vs 1Er 18:15
Я отвечал ему:
<<От тебя мы узнаем об этом вернее.>>

\vs 1Er 19:1
Слушай,~--- сказал он,~--- об обликах, которые тебя интересуют.
Почему в 1-м видении явилась тебе старица, сидящая на кафедре?
\vs 1Er 19:2
Потому что дух ваш обветшал и ослабел
и не имеет силы от грехов ваших и сомнений сердца.
\vs 1Er 19:3
Ибо как старцы, не имеющие
надежды на обновление, ничего другого не желают,
кроме успокоения на ложе,
\vs 1Er 19:4
так и вы, обременённые житейскими делами,
впали в беспечность и не возложили попечений своих на
Господа; одряхлел ваш разум и состарились вы в печалях ваших.
\vs 1Er 19:5
Я желаю узнать, господин, почему она сидела на кафедре?
\vs 1Er 19:6
Потому,~--- отвечал он мне,
что всякий немощный сидит на седалище по причине своей слабости,
чтобы имело поддержку немощное тело его.
Вот тебе смысл 1-го явления.

\vs 1Er 20:1
Во 2-м видении ты видел её стоящей,
с помолодевшим лицом и более веселою, нежели прежде;
а тело и волосы были у неё старческие.
\vs 1Er 20:2
Выслушай и эту притчу.
Когда кто сильно состарится и отчается в самом себе
из-за своей слабости и бедности,
то ничего другого не ожидает,
только последнего дня своей жизни.
\vs 1Er 20:3
Но вдруг получает он наследство.
Узнав об этом, он вскакивает повеселевший, к нему возвращаются
силы, обновляется дух его, который одряхлел от прежних дел;
он уже не лежит, но, восставши, мужественно действует.
\vs 1Er 20:4
То же произошло и с вами, когда услышали вы об откровении,
которое Бог сообщил вам.
\vs 1Er 20:5
Господь сжалился над вами и обновил дух ваш~--- и вы отложили
свои немощи, пришло к вам мужество, вы укрепились в вере,
и Господь, видя вашу верность, возрадовался.
\vs 1Er 20:6
Поэтому показал он вам строение башни~--- и иное покажет,
если будет между вами чистосердечный мир.

\vs 1Er 21:1
В 3-м видении ты видел, что она ещё моложе, прекрасна,
весела и лицо её светло.
\vs 1Er 21:2
Сравнить это с тем можно, как если бы к печалящемуся
человеку пришёл добрый вестник
\vs 1Er 21:3
тотчас он забыл бы прежнюю скорбь,
ни о чём другом не думал, как об услышанной им вести;
ободряется и обновляется дух его от радости.
\vs 1Er 21:4
Так точно и вы получили обновление душ ваших,
узнав такие блага.
\vs 1Er 21:5
А что ты видел её сидящею на скамье~--- это означает
твёрдое положение, так как скамейка имеет 4 ножки и стоит твёрдо,
да и мир поддерживается 4-мя стихиями.
\vs 1Er 21:6
Поэтому и те, которые
всецело, от всего сердца покаются, помолодеют и окрепнут.
\vs 1Er 21:7
Теперь имеешь ты полное объяснение.
Не проси более никаких откровений.
Если же нужно будет, то откроется тебе.

\chhdr{Видение 4-е.}
\vs 1Er 22:1
Спустя 20 дней было мне, братья,
видение гонения, которое должно случиться.
\vs 1Er 22:2
Шел я по полю при дороге Шампанской,
от большой дороги до поля почти 10 стадиев:
через это место путь бывает редко.
\vs 1Er 22:3
Гуляя один, я молил
Господа, чтобы он подтвердил те откровения,
которые явил мне чрез святую свою Церковь,
укрепил меня и дал покаяние всем рабам своим,
которые соблазнились;
\vs 1Er 22:4
дабы прославлялось великое и досточтимое имя его за то,
что удостоил показать мне чудеса свои.
\vs 1Er 22:5
И в то время когда я прославлял и благодарил его, голос был мне:
<<Не сомневайся, Ерма!>>
\vs 1Er 22:6
Стал я думать:
<<Что мне сомневаться, когда я так укреплён Господом и видел дивные дела?>>
\vs 1Er 22:7
Пройдя немного, братья, вдруг увидел я пыль,
поднимающуюся до неба, и подумал, что это идёт скот,
пыль поднимая.
\vs 1Er 22:8
Расстояние между тучей пыли и мною было около стадия.
\vs 1Er 22:9
Между тем пыль поднималась гуще и гуще,
так что мне стало это казаться чем-то сверхъестественным.
\vs 1Er 22:10
Несколько проглянуло солнце, и увидел я огромного зверя
наподобие дракона, из уст которого выходила огненная саранча.
\vs 1Er 22:11
В длину это животное имело около 100 футов,
а голова была подобна глиняному сосуду.
\vs 1Er 22:12
И начал я плакать и молить Господа, чтобы спас меня от него.
\vs 1Er 22:13
Потом вспомнил я слова, которые слышал:
<<Не сомневайся, Ерма.>>
\vs 1Er 22:14
Итак, братья, облёкшись верою в Бога и вспомнив явленные
мне им великие дела, я смело пошёл к зверю.
\vs 1Er 22:15
Зверь же метался с такою яростью и был так свиреп и силён,
что, казалось, при нападении мог бы разрушить город.
\vs 1Er 22:16
Я приблизился к нему,
и это огромное устрашающее животное мирно
растянулось на земле, высунув язык.
\vs 1Er 22:17
Я прошёл мимо него, и оно не пошевелилось.
\vs 1Er 22:18
Голова этого зверя была 4-х цветов:
чёрного, потом красного, или кровавого,
далее золотистого и, наконец, белого.

\vs 1Er 23:1
После того как я прошёл мимо зверя и удалился почти на 30 футов,
встречается мне разукрашенная дева,
словно выходящая из брачного чертога,
\vs 1Er 23:2
в белых башмаках, покрытая белыми одеждами до самого чела;
митра была ее покрывалом, волосы у ней были белые.
\vs 1Er 23:3
По прежним видениям я догадался,
что это Церковь, и обрадовался.
\vs 1Er 23:4
Она приветствовала меня:
<<Здравствуй, человек.>>
И я ответил ей также приветствием.
\vs 1Er 23:5
Она спросила:
<<Ничто не встретилось тебе, человек?>>
\vs 1Er 23:6
Госпожа, мне встретилось такое животное,
которое могло бы истребить народы, но силою Бога
и по великому его милосердию я спасся от него.
\vs 1Er 23:7
Счастливо спасся ты, сказала она,~--- потому,
что заботу свою ты возложил на Господа и ему открыл своё сердце,
\vs 1Er 23:8
веруя, что никем другим не можешь быть спасён,
кроме его великого и преславного имени.
\vs 1Er 23:9
За это Господь послал ангела своего,
поставленного над зверями, которому имя Егрин,
и он заградил пасть его, чтобы не пожрал тебя.
\vs 1Er 23:10
Ты избежал великого бедствия по вере твоей,
так как ты не усомнился при виде такого зверя.
\vs 1Er 23:11
Итак, пойди и возвести избранникам Бога
о великих делах его и скажи им,
что зверь этот есть образ грядущей великой напасти.
\vs 1Er 23:12
Поэтому, если приготовите себя и от всего сердца
покаетесь перед Господом, то можете избежать её,
\vs 1Er 23:13
если сердце ваше будет чисто и непорочно
и в остальные дни жизни вашей
будете безукоризненно служить Богу.
\vs 1Er 23:14
Возложите на Господа печали ваши, и он сам уврачует их.
\vs 1Er 23:15
Вы, двоедушные, веруйте в Бога, что он всё может~--- и
отвратить от вас гнев свой, и послать бичи на двоедушных.
\vs 1Er 23:16
Горе тем, которые услышат эти слова и презрят их,
лучше было им не родиться.

\vs 1Er 24:1
Я спросил её о 4-х цветах, которые были у зверя на голове.
\vs 1Er 24:2
Она сказала на это:
<<Опять ты любопытствуешь, спрашивая о вещах такого рода.>>
\vs 1Er 24:3
Да, госпожа, объясни мне, что они означают.
\vs 1Er 24:4
Слушай же,~--- отвечала она.
Чёрный цвет означает мир, в котором вы живете.
\vs 1Er 24:5
Огненный и кровавый~--- то,
что этому миру должно погибнуть посредством крови и огня.
\vs 1Er 24:6
А золотистая часть~--- это все вы, избегающие этого мира.
\vs 1Er 24:7
Ибо как золото испытывается посредством огня
и становится годным, так испытываетесь и вы,
живущие среди них.
\vs 1Er 24:8
Те, которые сохранят твёрдость и будут искушены ими, очистятся.
\vs 1Er 24:9
И как золото оставляет в огне примеси свои,
так и вы оставите всякую скорбь и печаль, очиститесь и
будете годны для здания башни.
\vs 1Er 24:10
Белая же часть означает будущий век,
в котором станут жить избранники Божьи,
\vs 1Er 24:11
потому что непорочны и чисты будут те,
которые избраны Богом в жизнь вечную.
\vs 1Er 24:12
Итак, не переставай доносить это до слуха святых.
\vs 1Er 24:13
Имеете вы и образ грядущего великого бедствия.
Если захотите вы, оно будет не страшно для вас:
помните заповеданное вам.
\vs 1Er 24:14
Сказав это, она удалилась, и не видел я, куда она ушла.
\vs 1Er 24:15
Раздался шум, и я в страхе бросился назад,
думая, что приближается тот зверь \ldots

\bibbookdescr{2Er}{
  inline={Пастырь Ермы. Книга 2. Заповеди},
  toc={2-я Ермы},
  bookmark={2-я Ермы},
  header={2-я Ермы},
  abbr={2~Ермы}
}
\chhdr{Видение 5-е.}
\vs 2Er 1:1
Когда я
помолился дома и сидел на ложе, вошел ко мне человек почтенного вида, в
пастушеской одежде: на нем был белый плащ, сума за плечами и посох в руке.
\vs 2Er 1:2
Он приветствовал меня, и я
ответил ему также приветствием. Тотчас же он сел подле меня и говорит:
\vs 2Er 1:3
Я послан от
достопоклоняемого ангела, чтобы жить с тобою остальные дни твоей жизни.
\vs 2Er 1:4
Мне показалось, что он
искушает меня, и сказал я ему: кто же ты? Я знаю того, кому препоручен я.
\vs 2Er 1:5
Не узнаешь меня? Нет.
\vs 2Er 1:6
Я тот самый пастырь,
которому препоручен ты.
\vs 2Er 1:7
Пока он говорил, вид его
изменился, и я узнал, что это тот, которому я препоручен.
\vs 2Er 1:8
Тотчас я смутился, объял
меня страх, и весь я разрывался от скорби, что отвечал ему так лукаво и
неразумно.
\vs 2Er 1:9
Он же сказал мне: не
смущайся, но укрепись заповедями, которые услышишь от меня.
\vs 2Er 1:10
Ибо я послан для того,
чтобы снова показать тебе все, что видел ты прежде, и особенно то, что полезно
для вас.
\vs 2Er 1:11
Итак, я приказываю тебе
сперва записать заповеди мои и притчи, чтобы перечитывать их время от времени,
так удобнее будет тебе выполнять их.
\vs 2Er 1:12
Поэтому я записал
заповеди и притчи так, как повелел он мне.
\vs 2Er 1:13
Если, услышав их, вы
будете поступать по ним и исполните их с чистым сердцем, то получите от
Господа то, что обещал Он вам.
\vs 2Er 1:14
Если же, услышав их, не
покаетесь, но обратитесь к грехам вашим, то воспримите от Господа наказание.
\vs 2Er 1:15
Все это повелел мне
записать этот пастырь, ангел покаяния.
 
\chhdr{Заповедь 1-я.}
\vs 2Er 2:1
Прежде
всего веруй, что един есть Бог, все сотворивший и совершивший, приведший все
не из чего в бытие.
\vs 2Er 2:2
Он все объемлет, Сам
будучи необъятен, и не может быть ни словом определен, ни умом постигнут.
\vs 2Er 2:3
Итак, веруй в Него, бойся
Его и, боясь, соблюдай воздержание.
\vs 2Er 2:4
Храни это и отринешь от
себя всякую похоть и беззаконие, и облечешься во всякую добродетель и правду и
будешь жить с Богом, если сохранишь эту заповедь.

\chhdr{Заповедь 2-я.}
\vs 2Er 3:1
Пастырь
сказал мне: имей простоту и будь незлобив, будь как дитя, которое не знает
лукавства, губящего жизнь людей.
\vs 2Er 3:2
Ни о ком не говори худо и
не стремись слушать того, кто говорит худо.
\vs 2Er 3:3
Если же будешь слушать, то
будешь причастен греху злословящего; веря ему, ты будешь подобен ему, потому
что поверил злословящему на брата твоего.
\vs 2Er 3:4
Гибельно злословие: это~--- дух беспокойный,
который никогда не находится в мире, но всегда живёт в несогласии.
\vs 2Er 3:5
Удерживайся от него и
всегда имей мир с братом твоим.
\vs 2Er 3:6
Облекись
благопристойностью, в которой нет ничего оскорбительного, но все ровно и
приятно.
\vs 2Er 3:7
Делай добро и от плода
трудов твоих, который дарует тебе Бог, давай всем бедным просто, нимало не
сомневаясь, кому даешь.
\vs 2Er 3:8
Давай всем, ибо Бог хочет,
чтобы всем досталось от Его даров.
\vs 2Er 3:9
Берущие дадут отчет Богу
почему и на что брали. Берущие по нужде не будут осуждены, а берущие притворно
подвергнутся суду.
\vs 2Er 3:10
Дающий же не будет
виноват: ибо он исполнил служение, назначенное Богом, не разбирая, кому дать и
кому не давать, и исполнил с похвалою пред Богом.
\vs 2Er 3:11
Итак, соблюдай эту
заповедь, как я сказал тебе, чтобы покаяние твоё и семейства твоего было в
простоте и сердце твое явилось чистым и непорочным пред Богом.

\chhdr{Заповедь 3-я.}
\vs 2Er 4:1
Также
сказал он мне: люби истину, и пусть исходит из уст твоих всякая истина,
\vs 2Er 4:2
чтобы дух, который Господь
поселил в этом теле, предстал истинным пред всеми людьми, и чтобы прославлялся
Господь, который дал тебе дух,
\vs 2Er 4:3
потому что Бог истинен во
всяком слове и никакой лжи нет в Нем.
\vs 2Er 4:4
И те, которые лгут,
отвергают Бога и не возвращают Ему залога, который получили; а они получили от
Него дух нелживый.
\vs 2Er 4:5
Если они возвращают его
лживым, то бесчестят заповедь Господа и становятся похитителями.
\vs 2Er 4:6
Услышав это, я горько
заплакал. Видя скорбь мою, он спросил: о чем ты плачешь?
\vs 2Er 4:7
Не знаю, господин, могу ли
спастись я. Почему?
\vs 2Er 4:8
Потому, что никогда в
жизни, господин, не произнес я слова правдивого, но всегда говорил коварно и
выдавал пред всеми ложь за истину; и никто не прекословил мне, потому что
доверяли моему слову.
\vs 2Er 4:9
Как же я могу жить, когда
поступал таким образом?
\vs 2Er 4:10
Он отвечал: ты
рассуждаешь хорошо и верно, ибо следовало тебе, как рабу Божьему, ходить в
истине, и не соединять лукавой совести с духом истины, и не оскорблять Духа
Божьего Святого и истинного.
\vs 2Er 4:11
И я сказал ему: никогда,
господин, я не слышал таких слов.
\vs 2Er 4:12
Услышав сейчас, впредь
соблюдай их и старайся, чтобы и те лживые слова, которые прежде говорил ты,
стали верными от последующих речей твоих, если они окажутся правдивыми.
\vs 2Er 4:13
Ибо и те могут сделаться
верными, если отныне будешь говорить правду; и если будешь соблюдать истину,
можешь получить себе жизнь.
\vs 2Er 4:14
И всякий, кто только
послушается этой заповеди, будет исполнять ее и удаляться от лжи, будет жить
с Богом.

\chhdr{Заповедь 4-я.}
\vs 2Er 5:1
Заповедую тебе, говорил пастырь, соблюдать целомудрие.
\vs 2Er 5:2
И да не войдет тебе в
сердце помысел о чужой жене, или о любодеянии, или о каком-либо подобном
дурном деле, ибо все это великий грех.
\vs 2Er 5:3
А ты помни о Господе во
все часы и никогда не согрешишь.
\vs 2Er 5:4
Если какой низкий помысел
взойдет на твое сердце, то совершишь великий грех; и кто творит такое
преступное дело, обрекает себя на смерть.
\vs 2Er 5:5
Итак, смотри ты,
воздерживайся от таких помыслов.
\vs 2Er 5:6
Ибо где обитает
целомудрие, там никогда не должен возникать худой помысел в сердце человека
праведного.
\vs 2Er 5:7
И попросил я: позволь мне,
господин, спросить тебя немного. Спрашивай.
\vs 2Er 5:8
Если, господин, сказал
я, муж имеет жену верную в Господе и заметит ее в прелюбодеянии, то будет ли
он грешен, живя с нею?
\vs 2Er 5:9
И ответил он мне: доколе
муж не знает греха своей жены, он не грешит, если живет с нею.
\vs 2Er 5:10
Если же узнает муж о
грехе ее, а она не покается в своем прелюбодеянии, то муж согрешит, живя с
нею, и сделается участником в ее прелюбодеянии.
\vs 2Er 5:11
Что же делать, спросил
я, если жена будет оставаться в своем пороке?
\vs 2Er 5:12
Пусть муж отпустит ее и
останется один. Если же, отпустивши свою жену, возьмет другую, то и сам примет
грех прелюбодеяния.
\vs 2Er 5:13
Что же, господин, если
жена отпущенная покается и пожелает возвратиться к мужу своему, то должна ли
она быть принята мужем?
\vs 2Er 5:14
Если не примет ее муж, он
совершит грех великий, он ответил мне.
\vs 2Er 5:15
Должно принимать
грешницу, которая раскаивается, но не много раз. Ибо для рабов Божьих покаяние
положено одно.
\vs 2Er 5:16
Поэтому ради раскаяния не
должен муж, отпустив жену свою, брать себе другую. Так же делать надлежит и
жене.
\vs 2Er 5:17
Но прелюбодейство не
только в осквернении плоти своей: прелюбодействует и тот, кто поступает
подобно народам.
\vs 2Er 5:18
Избегай общения с тем,
кто совершает такие дела и не кается, иначе и ты будешь причастен греху его.
\vs 2Er 5:19
Итак, заповедуется вам,
чтобы вы оставались одинокими как муж, так и жена, ибо в этом случае может
иметь место покаяние.
\vs 2Er 5:20
Но я не даю повода к тому
чтобы так делалось: пусть не грешит более тот, кто уже согрешил.
\vs 2Er 5:21
Что касается прежних
грехов его, то есть Бог, который может дать исцеление, ибо Он имеет власть над
всем.

\vs 2Er 6:1
И я опять просил его:
поскольку Господь удостоил меня того, чтобы ты всегда жил со мною, то дозволь
сказать мне еще несколько слов,
\vs 2Er 6:2
потому что я ничего не
понимаю и сердце мое омрачено прежними делами моими. Вразуми меня, так как я
совершенно ничего не смыслю.
\vs 2Er 6:3
И он в ответ сказал мне: я
приставник покаяния и всем кающимся даю разум. Или самое покаяние, ты думаешь,
не есть великий разум?
\vs 2Er 6:4
Грешник кающийся уразумел,
что он согрешил пред Господом, он осудил всем сердцем содеянные им дела и,
раскаявшись, более уже не делает зла, но совершает добро, и смиряет душу и
мучит ее за то, что согрешила. Итак, понимаешь, что покаяние есть великий
разум?
\vs 2Er 6:5
Потому-то, господин, я и
расспрашиваю тебя подробно обо всем, что я грешник и желаю узнать, что мне
делать, чтобы жить, ибо грехи мои многочисленны и разнообразны.
\vs 2Er 6:6
Ты будешь жив, сказал
он, если сохранишь мои заповеди и будешь поступать по ним; и всякий, кто
только услышит и исполнит эти заповеди, будет жить с Богом.

\vs 2Er 7:1
Я сказал ему: господин, я
слышал от некоторых учителей, что нет иного покаяния, кроме того, когда сходим
в воду и получаем отпущение прежних грехов наших.
\vs 2Er 7:2
Справедливо ты слышал. Ибо
получившему отпущение грехов не должно более грешить, но жить в чистоте.
\vs 2Er 7:3
И так как ты обо всем
расспрашиваешь, объясню тебе это, не давая повода к заблуждению тем, которые
собираются уверовать или только что уверовали в Господа.
\vs 2Er 7:4
Они не имеют покаяния во
грехах, но имеют отпущение прежних грехов своих.
\vs 2Er 7:5
Тем же, которые призваны
прежде, положил Господь покаяние, ибо Он сердцеведец, провидящий все, знал
слабость людей и великое коварство дьявола, который будет сеять вред и злобу
среди рабов Божьих.
\vs 2Er 7:6
Поэтому милосердный
Господь сжалился над своим созданием и положил покаяние, над которым и дана
мне власть.
\vs 2Er 7:7
Итак, я говорю тебе, после
этого великого и святого призвания, если кто, будучи искушен дьяволом,
согрешил, пусть покается.
\vs 2Er 7:8
Если же часто он будет
грешить и творить покаяние, не принесет ему покаяние пользы, ибо с трудом он
будет жить с Богом.
\vs 2Er 7:9
И я сказал: господин, я
обновился, когда услышал об этом так обстоятельно. Ибо я знаю, что спасусь,
если еще не присовокуплю ничего к грехам своим.
\vs 2Er 7:10
Спасешься, ответил он,
ты и все, которые сделают то же.

\vs 2Er 8:1
И опять я попросил его:
господин, так как ты терпеливо меня выслушиваешь, объясни мне еще вот что.
\vs 2Er 8:2
Если муж или жена умрет и
один из них вступит в брак согрешает ли вступающий в брак?
\vs 2Er 8:3
Не согрешает, но если
останется сам по себе, то приобретет себе большую славу у Господа.
\vs 2Er 8:4
Поэтому храни чистоту и
целомудрие и будешь жить с Богом.
\vs 2Er 8:5
То, что я говорю и
собираюсь сказать тебе после, соблюдай с этого самого дня, ибо ты поручен мне
и живу в твоем доме.
\vs 2Er 8:6
И прежним грехам твоим
будет отпущение, если сохранишь мои заповеди; и все, кто сохранит их и будет
ходить в чистоте, получит отпущение.

\chhdr{Заповедь 5-я.}
\vs 2Er 9:1
Будь
великодушен и терпелив, сказал пастырь, и будешь господствовать над всеми
злыми делами и сотворишь всякую правду.
\vs 2Er 9:2
Если будешь великодушен,
то Дух Святой, в тебе обитающий, останется чист и не омрачится от какого-либо
злого духа, но, ликуя, расширится, и вместе с сосудом, в котором обитает,
будет радостно служить Господу.
\vs 2Er 9:3
Если же найдет какой-либо
гнев, то Дух Святой, сущий в тебе, тотчас же будет стеснен и постарается
удалиться,
\vs 2Er 9:4
ибо подавляется злым духом
и, оскорбляемый гневом, не имеет возможности служить Господу, как желает.
\vs 2Er 9:5
Поэтому, когда оба духа
обитают вместе, плохо бывает человеку.
\vs 2Er 9:6
Так, если взять немножко
полыни и положить в сосуд с медом, не весь ли мед испортится?
\vs 2Er 9:7
И столько меда пропадает
от незначительного количества полыни, теряет сладость и уже не имеет
приятности для своего владельца, потому что делается горьким и негодным к
употреблению. Но если в мед не класть полынь, он останется сладок.
\vs 2Er 9:8
Сам видишь, великодушие
слаще меда, и оно угодно Богу и Господь обитает в нем, а гнев горек и
неугоден.
\vs 2Er 9:9
Итак, если к великодушию
примешивается гнев, то дух возмущается, и неприятна Богу молитва его.
\vs 2Er 9:10
И я сказал ему: желал бы
я узнать, господин, действие гнева, чтобы уберечь себя от него.
\vs 2Er 9:11
Если ты и твои домочадцы
не будете удерживаться от него, то потеряете всякую надежду спасения.
\vs 2Er 9:12
Но воздерживайся от
гнева, ибо я с тобою; и от него воздержатся все, которые покаются от всего
сердца своего, ибо я буду с ними и сохраню их.
\vs 2Er 9:13
Все такие принимаются
святейшим ангелом в число праведных.

\vs 2Er 10:1
Послушай теперь и о
действии гнева, как он вреден и как губит рабов Божьих и отвращает их от
правды.
\vs 2Er 10:2
Он не может вредить людям,
исполненным веры, потому что с ними пребывает сила Божья; совращает же
сомневающихся и не имеющих ее.
\vs 2Er 10:3
Как скоро он увидит таких
людей спокойными проникает в сердце их, и муж или жена сердятся друг на
друга по каким-нибудь житейским делам:
\vs 2Er 10:4
или из-за пищи, или
пустого слова, или какого приятеля, или долга, или из-за подобных мелочных
вещей. Все это глупо, пусто и неприлично рабам Божьим.
\vs 2Er 10:5
Но великодушие твердо и
мужественно, имеет крепкую силу и пребывает в великой широте, весело и
беззаботно радуясь, и прославляет Господа во всякое время чуждое всякой
горечи, всегда мирное и кроткое.
\vs 2Er 10:6
Это великодушие живет с
имеющими полную веру. А гнев безрассуден, пуст и легкомыслен.
\vs 2Er 10:7
От безрассудства рождается
огорчение, от огорчения раздражение, от раздражения гнев, от гнева же
неистовство.
\vs 2Er 10:8
Неистовство, происшедшее
от стольких зол, есть великий и неискупимый грех.
\vs 2Er 10:9
И когда все это находится
в одном сосуде, где обитает и Дух Святой, то сосуд не вмещает их в себе, но
переполняется:
\vs 2Er 10:10
добрый дух не может жить
вместе со злым духом, а удаляется от такого человека и ищет себе пристанища в
кротости и тишине.
\vs 2Er 10:11
Когда он отступит от
человека, в котором обитал, человек, исполненный духами злыми, делается чужд
Святого Духа и закрыт для благой мысли. Так бывает со всеми гневливыми.
\vs 2Er 10:12
Итак, ты удаляйся гнева,
но облекись в великодушие и противься всякому огорчению и будешь в чистоте и
святости, любезной Богу.
\vs 2Er 10:13
Смотри поэтому, чтобы
как-нибудь не пренебречь тебе этой заповедью, ибо если соблюдешь эту заповедь,
то можешь исполнить и прочие мои заповеди, которые хочу тебе преподать.
\vs 2Er 10:14
Итак, теперь утверждайся
в этих заповедях, чтобы тебе жить с Богом, равно и все, кто соблюдет их,
будут жить с Богом.

\chhdr{Заповедь 6-я.}
\vs 2Er 11:1
Я повелел тебе, сказал пастырь, в первой заповеди, чтобы хранил ты веру,
страх и воздержание.
\vs 2Er 11:2
Да, господин, подтвердил
я.
\vs 2Er 11:3
А теперь я хочу объяснить
тебе силу этих добродетелей, чтобы знал ты, как каждая из них действует и
какую имеет власть.
\vs 2Er 11:4
Двоякого рода их действия
и состоят в праведном и неправедном.
\vs 2Er 11:5
Ты веруй праведному,
неправедному нисколько не веруй.
\vs 2Er 11:6
Ибо правда имеет путь
прямой, а неправда кривой.
\vs 2Er 11:7
Но ты иди путем прямым, а
кривой оставь.
\vs 2Er 11:8
Кривой путь неровен, имеет
множество преткновений, скалист и тернист и ведет к погибели идущих по нему.
\vs 2Er 11:9
А те, которые следуют
прямому пути, идут ровно и без препятствий, потому что он не скалист и не
тернист. Итак, видишь, что лучше идти этим путем.
\vs 2Er 11:10
Я сказал: мне нравится
идти этим путем.
\vs 2Er 11:11
И пойдешь ты, равно как
пойдут по нему и все, которые от всего сердца обратятся к Господу.

\vs 2Er 12:1
Послушай теперь,
продолжал он, о вере. Два ангела с человеком: один добрый, а другой злой.
\vs 2Er 12:2
Я спросил его: каким
образом, господин, я могу распознать их, если оба ангела живут со мною?
\vs 2Er 12:3
Слушай и разумей. Добрый
ангел тих и скромен, кроток и мирен.
\vs 2Er 12:4
Поэтому войдя в твое
сердце, постоянно будет внушать он тебе справедливость, целомудрие, чистоту
ласковость, снисходительность, любовь и благочестие.
\vs 2Er 12:5
Когда все это вселится в
твое сердце, знай, что добрый ангел с тобою: верь этому ангелу и следуй делам
его.
\vs 2Er 12:6
Послушай и о действиях
ангела злого. Прежде всего он злобен, гневлив и безрассуден, и действия его
злы и развращают рабов Божьих.
\vs 2Er 12:7
Поэтому когда войдет он в
твое сердце, из действий его разумей, что это ангел злой.
\vs 2Er 12:8
Каким образом, спросил
я, узнаю его, господин?
\vs 2Er 12:9
Слушай. Когда овладеют
тобой гнев или досада, знай, что он в тебе;
\vs 2Er 12:10
также когда возникнет в
сердце твоем пожелание разных и роскошных яств, и напитков, и чужих жен, то
вселяются в него гордость, хвастовство, надменность и тому подобное тогда
знай, что с тобою злой ангел.
\vs 2Er 12:11
Поэтому ты, зная его
дела, избегай и не верь ему: дела его злы и не свойственны рабам Божьим.
\vs 2Er 12:12
Таковы действия того и
другого ангела. Разумей их, верь ангелу доброму и удаляйся от ангела злого,
потому что внушение его во всяком деле злое.
\vs 2Er 12:13
Даже если в сердце
человека верующего войдет помысел злого ангела, то он непременно согрешит.
\vs 2Er 12:14
Если же злые люди откроют
сердце свое делам ангела доброго, то обязательно он сделает что-нибудь доброе.
\vs 2Er 12:15
Итак, видишь, что хорошо
следовать ангелу доброму. Если станешь повиноваться ему и творить его дела, то
будешь жить с Богом;
\vs 2Er 12:16
равно как и все, которые
будут следовать его делам, будут жить с Богом.

\chhdr{Заповедь 7-я.}
\vs 2Er 13:1
Бойся, говорил пастырь, Господа и соблюдай заповеди его,
\vs 2Er 13:2
ибо, соблюдая заповеди
Божьи, будешь тверд в любом деле и преуспеешь в нем.
\vs 2Er 13:3
Боясь Господа, будешь все
делать хорошо. Вот страх, которым должно страшиться, чтобы спастись.
\vs 2Er 13:4
Дьявола же не бойся: боясь
Господа, ты будешь господствовать над дьяволом, потому что в нем нет никакой
силы.
\vs 2Er 13:5
А в ком нет силы, того не
должно бояться.
\vs 2Er 13:6
В ком есть превосходная
сила, того и должно бояться.
\vs 2Er 13:7
Ибо всякий, имеющий силу,
внушает страх; а кто не имеет силы, всеми презирается.
\vs 2Er 13:8
Бойся, впрочем, дел
дьявола, потому что они злы; боясь Господа, ты не совершишь дел дьявола, но
удержишься от них.
\vs 2Er 13:9
Двоякий есть страх. Если
ты захотел сделать злое, то бойся Бога и не сделаешь этого.
\vs 2Er 13:10
Равно если бы захотел ты
сделать доброе, то опять бойся Бога и сделаешь его.
\vs 2Er 13:11
Подлинно, страх Божий
велик, силен и славен.
\vs 2Er 13:12
Итак, бойся Бога, и
будешь жить. И все те, которые будут бояться Его, соблюдая Его заповеди, будут
жить с Богом;
\vs 2Er 13:13
а которые не соблюдают
Его заповедей, в тех нет жизни.

\chhdr{Заповедь 8-я.}
\vs 2Er 14:1
Я сказал тебе, продолжал поучения пастырь, что творения Божьи двояки, двояко
и воздержание.
\vs 2Er 14:2
Поэтому от некоторых
следует воздерживаться, а от иных не следует.
\vs 2Er 14:3
Открой мне, господин,
попросил я, от чего следует воздерживаться и от чего не следует.
\vs 2Er 14:4
Воздерживайся, отвечал
он, от зла и не делай его, а от доброго не воздерживайся, но делай его.
\vs 2Er 14:5
Ибо если будешь
удерживаться от доброго и не будешь его делать, согрешишь.
\vs 2Er 14:6
Итак, удерживайся от
всякого зла и делай всякое добро.
\vs 2Er 14:7
От какого зла, спросил
я, должно удерживаться?
\vs 2Er 14:8
От прелюбодеяния, пьянства
и чрезмерных пиршеств, от излишеств в яствах, от роскоши и тщеславия,
\vs 2Er 14:9
от гордости, от лжи и
клеветы, от лицемерия, злопамятства и всякого оскорбления чести другого.
\vs 2Er 14:10
Таковы дела злые, от
которых должно воздерживаться рабу Божьему. Кто не воздерживается от них, тот
не может жить с Богом.
\vs 2Er 14:11
Послушай теперь и о
делах, следующих за ними.
\vs 2Er 14:12
Разве и еще есть,
господин, дела злые?
\vs 2Er 14:13
И подлинно есть еще много
такого, от чего должен воздерживаться раб Божий. Это воровство,
лжесвидетельство, пожелание чужого, надменность и тому подобное.
\vs 2Er 14:14
Не почитаешь ли всего
этого злым? Подлинно, это есть зло рабов Божьих и от всего этого должен
воздерживаться раб Божий, чтобы жить с Богом и быть вместе с теми, которые
воздерживаются от злых дел.
\vs 2Er 14:15
А теперь слушай о тех
добрых делах, которые положено творить, чтобы спастись.
\vs 2Er 14:16
Прежде всего это вера,
страх Божий, любовь, согласие, справедливость, истина, терпение лучше их
ничего нет в жизни человеческой:
\vs 2Er 14:17
кто соблюдает их и во все
дни не станет избегать, тот блажен в своей жизни.
\vs 2Er 14:18
Затем следуют добрые
дела, состоящие в том, чтобы служить вдовам, печься о сиротах и бедных,
избавлять от нужды рабов Божьих,
\vs 2Er 14:19
быть гостеприимным, не
прекословить, быть уравновешенным, считать себя ниже всех людей,
\vs 2Er 14:20
почитать старших
возрастом, соблюдать правду, хранить братство, переносить обиды, быть
великодушным,
\vs 2Er 14:21
не отвергать отпадших от
веры, но обращать и успокаивать их, вразумлять согрешающих, не притеснять
должников и тому подобное.
\vs 2Er 14:22
Не почитаешь ли это
добром?
\vs 2Er 14:23
Нет ничего лучше и
достойнее этого! воскликнул я.
\vs 2Er 14:24
Вот и твори эти дела и не
воздерживайся жить с Богом, равно как и все, которые соблюдут эту заповедь,
будут жить с Богом.

\chhdr{Заповедь 9-я.}
\vs 2Er 15:1
Далее говорил мне пастырь:
отринь от себя сомнения и нисколько не колеблись просить
чего-либо у Господа,
\vs 2Er 15:2
говоря себе: каким образом
могу я просить у Господа и получить, столько согрешив пред Ним?
\vs 2Er 15:3
Не помышляй этого, но от
всего сердца обращайся к Господу и проси без сомнения и познаешь великую
благость Его,
\vs 2Er 15:4
потому что Он не презрит
тебя, но исполнит прошение души твоей.
\vs 2Er 15:5
Ибо Бог не как люди,
которые помнят обиды, Он не помнит зла и милосерден к своему созданию.
\vs 2Er 15:6
Итак, очисти сердце свое
от всех сует настоящего века и прежде всего выполняй данные тебе от Бога
наказы
\vs 2Er 15:7
и получишь все блага,
которых просишь, и все прошения твои не будут оставлены, если будешь просить у
Господа без сомнения.
\vs 2Er 15:8
Те же, которые
сомневаются, совсем ничего не получают из того, о чем просят.
\vs 2Er 15:9
Исполненные веры всего
просят с упованием и получают от Господа, ибо просят без сомнения.
\vs 2Er 15:10
Всякий колеблющийся
человек с трудом спасется, если только не покается.
\vs 2Er 15:11
Поэтому очисти сердце
свое от сомнения, облекись в веру и, веруя Господу, получишь все, о чем
просишь.
\vs 2Er 15:12
Но если иногда, прося о
чем-либо Господа, долго не получаешь, не колеблись оттого, что сразу не
выполняются прошения души твоей.
\vs 2Er 15:13
Ибо, может быть, для
испытания или за грех твой, которого не знаешь, позднее получишь то, что
просишь.
\vs 2Er 15:14
Но ты не переставай
высказывать желание души своей и будешь вознагражден.
\vs 2Er 15:15
Если же придешь в уныние
и перестанешь просить, то жалуйся на себя, а не на Бога, что Он не дает тебе.
\vs 2Er 15:16
Итак, видишь, как
гибельно и ужасно сомнение, и многих даже твердых в вере совсем отторгает от
веры.
\vs 2Er 15:17
Ибо сомнение это дочь
дьявола и сильно злоумышляет на рабов Божьих.
\vs 2Er 15:18
Итак, отвергни сомнение и
одолей его во всяком деле, вооружившись сильной и могущественной верой.
\vs 2Er 15:19
Ибо вера все обещает и
все совершает, сомнение же ни в чем не доверяет себе и оттого не имеет успеха
в делах своих.
\vs 2Er 15:20
Итак, видишь, что вера
исходит свыше от Бога и имеет великую силу.
\vs 2Er 15:21
Сомнение же есть земной
дух, от дьявола, и силы не имеет.
\vs 2Er 15:22
Поэтому служи вере,
имеющей силу, и удаляйся от сомнения, которое бессильно,
\vs 2Er 15:23
и будете жить с Богом
ты и все люди, поступающие так же.

\chhdr{Заповедь 10-я.}
\vs 2Er 16:1
Удаляй от себя всякую печаль, потому что она сестра сомнения и гнева.
\vs 2Er 16:2
Каким образом, господин,
удивился я, она сестра их? Мне кажется, печаль это одно, другое гнев, и
сомнение само по себе.
\vs 2Er 16:3
И он ответил: неразумен
ты. Неужели не понимаешь, что печаль самый злой из всех духов и самый
вредный для рабов Божьих?
\vs 2Er 16:4
Она губит человека как
ничто другое и изгоняет из него Святого Духа и опять спасает.
\vs 2Er 16:5
Господин, не могу я
постичь смысла этих притчей и не понимаю, каким образом печаль может погубить
и опять спасти.
\vs 2Er 16:6
Слушай, сказал он, и
разумей. Кто никогда не изыскивал истины и не исследовал Божество, но только
уверовал и потом предался разным языческим занятиям и другим делам сего мира,
\vs 2Er 16:7
тот не понимает притчей
божественных, потому что помрачается от таких дел, повреждается и загрубевает
разумом.
\vs 2Er 16:8
Как хорошие виноградные
лозы, оставленные без ухода, подавляются и заглушаются разными сорняками и
терниями,
\vs 2Er 16:9
так и люди, которые только
уверовали и вдались в дела этого мира, лишаются своего смысла и, думая о
богатствах, совершенно ничего не понимают, и разум их, занятый мирской суетой,
глух к Господу.
\vs 2Er 16:10
Но те, которые живут в
страхе Божьем, тщательно исследуют истину и божественное и сердцем обращены к
Господу, они легко принимают и разумеют все, что говорится им.
\vs 2Er 16:11
Ибо, где обитает Господь,
там много разума.
\vs 2Er 16:12
Поэтому прилепись к
Господу и все поймешь и уразумеешь.

\vs 2Er 17:1
Послушай теперь,
неразумный, каким образом печаль изгоняет Духа Святого и как опять спасает.
\vs 2Er 17:2
Когда сомневающийся не
обретает успеха в каком-либо деле из-за своего сомнения, то печаль входит в
сердце такого человека, омрачает Духа Святого и изгоняет его.
\vs 2Er 17:3
И когда охватывают
человека гнев и сильное раздражение по какому-нибудь поводу, то опять печаль
входит в сердце, он скорбит о своем поступке, раскаивается, что разгневался.
\vs 2Er 17:4
Эта печаль кажется
спасительною, потому что влечет раскаянье.
\vs 2Er 17:5
Но и в том и в другом
случае печаль оскорбляет Святого Духа.
\vs 2Er 17:6
Печаль, вызванная
сомнением или тем, что не удалось человеку его дело, печаль неправедная.
\vs 2Er 17:7
Печаль же от досады на
дурной поступок не плохая печаль, но и она оскорбляет Святого Духа.
\vs 2Er 17:8
Посему удаляй от себя
печаль и не оскорбляй Святого Духа, в тебе живущего, чтобы он не возроптал на
тебя к Господу и не удалился от тебя.
\vs 2Er 17:9
Ибо Дух Божий, обитающий в
этом теле, не терпит печали.
\vs 2Er 17:10
Итак, облекись ты в
радость, которая всегда имеет благодать пред Господом и угодна Ему и утешайся
ею.
\vs 2Er 17:11
Всякий радующийся человек
совершает добро и помышляет о добре, презирая печаль.
\vs 2Er 17:12
А человек печальный
всегда зол, во-первых, потому, что оскорбляет Святого Духа, который дан
человеку радостным;
\vs 2Er 17:13
и, во-вторых, потому, что
он творит беззаконие, не обращаясь к Господу и не исповедуясь перед Ним.
\vs 2Er 17:14
Молитва печального
человека никогда не достигает престола Божьего.
\vs 2Er 17:15
И я спросил его: почему
же, господин, молитва печального человека не восходит к престолу Господню?
\vs 2Er 17:16
Потому, ответил он,
что печаль пребывает в его сердце.
\vs 2Er 17:17
Печаль, смешанная с
молитвою, не допускает молитву чистою взойти к престолу Божьему.
\vs 2Er 17:18
Как вино с добавлением
уксуса уже не имеет прежней приятности, так и печаль, примешанная к Святому
Духу, не имеет той же чистой молитвы.
\vs 2Er 17:19
Посему очищайся от злой
печали и будешь жить с Богом,
\vs 2Er 17:20
и все будут жить с Богом,
если только отбросят от себя печаль и облекутся в радость.

\chhdr{Заповедь 11-я.}
\vs 2Er 18:1
Пастырь показал мне людей, сидящих на скамьях, и одного стоящего на кафедре,
\vs 2Er 18:2
и сказал: посмотри на них.
Те, которые сидят на скамьях, верующие, а стоящий на кафедре лжепророк,
погубляющий смысл рабов Божьих тех, которые двоедушествуют, а не истинно
верующих.
\vs 2Er 18:3
Эти двоедушные приходят к
нему как к пророку и спрашивают его о том, что станет с ними,
\vs 2Er 18:4
и он, не имея в себе силы
Духа Божественного, отвечает им, говоря то, что хотят они услышать, и
наполняет души их лживыми обещаниями.
\vs 2Er 18:5
Будучи суетен, он суетно и
отвечает суетным людям.
\vs 2Er 18:6
Впрочем, он говорит и
кое-что справедливое, потому что дьявол вселяет в него свой дух, дабы привлечь
кого-либо из праведных.
\vs 2Er 18:7
Но сильные в вере,
облеченные в истину не присоединяются к таким духам, но удаляются от них.
\vs 2Er 18:8
Двоедушные же и часто
кающиеся обращаются за прорицаниями, как и народы, и навлекают на себя великий
грех своим идолопоклонством,
\vs 2Er 18:9
потому что спрашивающий
лжепророка является идолопоклонником, он чужд истины и неразумен.
\vs 2Er 18:10
А всякий дух, Богом
данный, не дожидается расспросов, но, имея силу Божественную, говорит все сам,
потому что он свыше, от силы Духа Божьего.
\vs 2Er 18:11
Дух, который отвечает на
вопросы согласно желаниям человеческим, есть дух земной, легкомысленный, не
имеющий силы: он совсем не говорит, если его не спрашивают.
\vs 2Er 18:12
И я сказал: как же можно
распознать, кто истинный пророк и кто лжепророк?
\vs 2Er 18:13
Выслушай, говорит, об
обоих пророках; и по тому, что я скажу тебе, отличишь пророка Божьего от
ложного пророка.
\vs 2Er 18:14
По делам узнавай
человека, который имеет Дух Божий.
\vs 2Er 18:15
Во-первых, он спокоен,
кроток и смирен, удаляется от всякого зла и суетного желания этого века,
\vs 2Er 18:16
ставит себя ниже всех
людей и никому не отвечает на вопросы, не говорит наедине;
\vs 2Er 18:16
Дух Божий говорит не
тогда, когда человек того желает, но когда угодно Богу.
\vs 2Er 18:17
Поэтому когда человек,
имеющий Дух Божий, придет в церковь праведных, имеющих веру, там совершается
молитва к Господу;
\vs 2Er 18:18
тогда ангел пророческого
духа, приставленный к нему, исполняет этого человека Духом Святым, и он
говорит к собранию, как угодно Богу. Так проявляется Дух Божественный и сила
его.
\vs 2Er 18:19
Слушай теперь и о духе
земном, суетном, неразумном и не имеющем силы.
\vs 2Er 18:20
Прежде всего человек,
кажущийся исполненным духа, возвышает себя, стремится к власти, нагл и
многословен,
\vs 2Er 18:21
живет среди роскоши и
многих удовольствий, берет мзду за свое прорицание, без вознаграждения не
пророчествует.
\vs 2Er 18:22
Может ли Дух Божий брать
мзду и пророчествовать?
\vs 2Er 18:23
Это не свойственно
пророку Божьему, и в поступающих таким образом обитает дух земной.
\vs 2Er 18:24
Далее, он не входит в
собрание мужей праведных, но избегает их
\vs 2Er 18:25
и, наоборот, общается с
людьми двоедушными и пустыми, пророчествует в местах потаенных и обманывает
речами, которые хотят услышать, и говорит суетное людям суетным:
\vs 2Er 18:26
так пустая посуда, когда
складывается с другими пустыми же, не разбивается, но они хорошо приходятся
одна к другой.
\vs 2Er 18:27
А когда он оказывается
среди людей праведных, исполненных Духа Божественного, возносящих молитву,
тогда и обнаруживается его пустота:
\vs 2Er 18:28
земной дух от страха
покидает его, и он, совершенно поверженный, ничего не может говорить.
\vs 2Er 18:29
Если в кладовую поместить
вино или масло и туда же поставить пустой сосуд, а после брать запасы из
кладовой, то сосуд, который поставил пустым, пустым и найдешь.
\vs 2Er 18:30
И пустые пророки, какими
приходят к людям, имеющим Святого Духа, такими и остаются. Вот образ пророка
истинного и ложного.
\vs 2Er 18:31
Итак, испытывай по делам
и по жизни того человека, который говорит, что он имеет Святого Духа.
\vs 2Er 18:32
Верь Духу, приходящему от
Бога и имеющему силу; духу же земному и пустому, в котором нет силы, не верь,
ибо он приходит от дьявола.
\vs 2Er 18:33
Задумайся над примером,
который приведу я тебе. Если взять камень и бросить в небо, то сможешь ли
докинуть до него?
\vs 2Er 18:34
Или же если взять трубу с
водою, направить струю в небо, то сможешь ли ты пробить небо?
\vs 2Er 18:35
Что ты, господин,
воскликнул я, все это невозможно!
\vs 2Er 18:36
Вот, сказал он, как
этого не может быть, так точно дух земной бессилен и недейственен.
\vs 2Er 18:37
Осознай теперь силу,
свыше приходящую. Град крупинка очень малая, но, попадая в голову человека,
какую причиняет боль?
\vs 2Er 18:38
Или еще пример: дождевая
капля, которая, с крыши скатываясь вниз, источает камень.
\vs 2Er 18:39
Видишь, и самое малое,
что сверху падает на землю, имеет великую силу: так силен и Дух Божественный,
приходящий свыше.
\vs 2Er 18:40
Этому Духу ты верь, а от
другого удаляйся.

\chhdr{Заповедь 12-я.}
\vs 2Er 19:1
Пастырь сказал мне: удали от себя всякую похоть злую и облекись в желание
доброе и святое.
\vs 2Er 19:2
Ибо, облекшись в желание
доброе, ты возненавидишь зло и будешь управлять им, как захочешь.
\vs 2Er 19:3
Похоть злая люта и с
трудом усмиряется: она страшна и своею лютостью сокрушает людей.
\vs 2Er 19:4
Но сокрушает тех людей,
которые не имеют стремления доброго и погрузились в дела этого века: их-то она
предает смерти.
\vs 2Er 19:5
Какие действия, господин,
спросил я, злой похоти обрекают людей на смерть? Объясни мне, чтобы я мог
избегать их.
\vs 2Er 19:6
Послушай, посредством
каких действий злая похоть умерщвляет рабов Божьих.

\vs 2Er 20:1
Злая похоть состоит в том,
чтобы желать чужой жены, или жене желать чужого мужа, желать великого
богатства, множества роскошных яств и питий и других наслаждений:
\vs 2Er 20:2
ибо всякое наслаждение
бессмысленно и суетно для рабов Божьих.
\vs 2Er 20:3
Таковы пожелания злые,
умерщвляющие рабов Божьих.
\vs 2Er 20:4
Злая похоть есть дочь
дьявола. Поэтому должно удаляться злой похоти, чтобы жить с Богом.
\vs 2Er 20:5
А те, которые поддадутся
злой похоти и не воспротивятся ей, погибнут, потому что она смертоносна.
\vs 2Er 20:6
Итак, ты стремись к правде
и, вооружившись страхом Господним, противостой злой похоти. Ибо страх Божий
обитает в добрых пожеланиях.
\vs 2Er 20:7
И злая похоть, видя тебя
вооруженным страхом Господним и противящимся ей, убежит от тебя далеко и не
явится к тебе, боясь твоего оружия;
\vs 2Er 20:8
и одержавши победу и
увенчанный за нее, предайся стремлению к правде и, воздавши Ему за полученную
тобою победу, служи Ему по Его воле.
\vs 2Er 20:9
И если послужишь доброму
началу и покоришься Ему, то можешь владычествовать над злою похотью и
управлять ею, как тебе угодною.
\vs 2Er 20:10
Желал бы я услышать,
господин, сказал я, как должно служить доброму желанию?
\vs 2Er 20:11
Слушай. Имей страх Божий
и веру в Бога, люби истину, твори правду и подобные добрые дела.
\vs 2Er 20:12
Делая это, ты будешь
угодным рабом Божьим и будешь жить с Богом; и все, которые будут служить
стремлению доброму, будут жить с Богом.

\vs 2Er 21:1
И так окончил он
двенадцать заповедей и сказал мне:
\vs 2Er 21:2
вот тебе заповеди,
поступай по ним и к тому же убеждай людей слушать тебя, чтобы покаяние их было
чисто в остальные дни их жизни.
\vs 2Er 21:3
И это служение, которое
поручаю тебе, исполняй тщательно и получишь великий плод,
\vs 2Er 21:4
ибо найдешь любовь у всех,
которые покаются и послушаются слов твоих.
\vs 2Er 21:5
Я буду с тобою и буду
побуждать их слушаться тебя.
\vs 2Er 21:6
И я сказал ему: господин,
эти заповеди величественны, прекрасны и могут возвеселить сердце человека,
который исполнит их.
\vs 2Er 21:7
Но не знаю, господин,
способен ли человек соблюдать эти заповеди, потому что они очень трудны.
\vs 2Er 21:8
Он отвечал мне: эти
заповеди легко соблюсти, и не покажутся они трудными, если будешь убежден, что
их можно соблюсти;
\vs 2Er 21:9
но если закралось в сердце
твое сомнение, что не по силам человеку, то не соблюдешь их.
\vs 2Er 21:10
Теперь же говорю тебе:
если не соблюдешь этих заповедей и пренебрежешь ими, то не спасешься ты и дети
твои, и весь дом твой,
\vs 2Er 21:11
потому что ты сам себе
присудил, что этих заповедей нельзя соблюсти человеку.

\vs 2Er 22:1
Произносил он это с
большим гневом, и я очень смутился и испугался.
\vs 2Er 22:2
Лицо его изменилось так,
что вид его стал невыносим для человека.
\vs 2Er 22:3
Но, видя, что я весь в
смущении и страхе, начал он говорить умереннее и ласковее:
\vs 2Er 22:4
неразумный и непостоянный,
не видишь ли славу Божью, не понимаешь, как велик и дивен Тот, который
сотворил мир для человека,
\vs 2Er 22:5
и все творение покорил
человеку, и дал ему всю власть господствовать над всем поднебесным?
\vs 2Er 22:6
Если человек есть владыка
тварей Божьих и над всем господствует, то ужели он не может господствовать и
над этими заповедями?
\vs 2Er 22:7
Это по силам человеку,
имеющему Господа в сердце своем.
\vs 2Er 22:8
Кто же имеет Господа
только в устах своих, огрубел сердцем и далек от Господа, для того эти
заповеди тяжки и неисполнимы.
\vs 2Er 22:9
Итак вы, слабые и
нетвердые в вере, положите себе Господа вашего в сердце и узнаете, что ничего
нет легче этих заповедей, ничего приятнее и доступнее их.
\vs 2Er 22:10
Обратитесь к Господу,
оставьте дьяволу его удовольствия, которые злы и горьки, и не бойтесь дьявола,
потому что над вами он не имеет силы.
\vs 2Er 22:11
Ибо я с вами, ангел
покаяния, и я господствую над ним.
\vs 2Er 22:12
Дьявол наводит страх, но
страх его не имеет силы.
\vs 2Er 22:13
Посему не бойтесь его, и
он покинет вас.

\vs 2Er 23:1
И я попросил его:
господин, выслушай несколько слов моих.
\vs 2Er 23:2
Говори, разрешил он.
\vs 2Er 23:3
Всякий человек желает
исполнять Божьи заповеди, и нет такого, который бы не просил у Бога силы
соблюдать Его заповеди;
\vs 2Er 23:4
но дьявол упорен и своею
силою противодействует рабам Божьим.
\vs 2Er 23:5
Не может дьявол,
возразил он, пересилить рабов Божьих, которые веруют в Господа от всего
сердца.
\vs 2Er 23:6
Дьявол может
противоборствовать, но победить не может.
\vs 2Er 23:7
Если воспротивитесь ему,
то, побежденный, он с позором покинет вас.
\vs 2Er 23:8
Боятся дьявола, как будто
имеющего власть, те, которые не тверды в вере.
\vs 2Er 23:9
Дьявол искушает рабов
Божьих и, если найдет слабых, губит их.
\vs 2Er 23:10
Когда человек наполняет
сосуды хорошим вином и между ними ставит несколько сосудов неполных,
\vs 2Er 23:11
то, приходя попробовать
вино, не думает о полных, ибо знает, что они хороши, а отведывает из неполных,
не скисло ли в них вино,
\vs 2Er 23:12
потому что в неполных
сосудах вино скоро скисает и теряет вкус.
\vs 2Er 23:13
Так и дьявол приходит к
рабам Божьим, чтобы искусить их.
\vs 2Er 23:14
И все те, которые полны
веры, мужественно противятся ему; и он удаляется от них, потому что негде
войти ему.
\vs 2Er 23:15
Тогда он подступает к
тем, которые не полны веры, и, имея возможность, вселяется в них, делает с
ними что хочет, и они становятся его рабами.

\vs 2Er 24:1
Но, говорю вам я, ангел
покаяния: не бойтесь дьявола, ибо я послан для того, чтобы быть с вами,
кающимися от всего сердца, и утвердить вас в вере.
\vs 2Er 24:2
Посему верьте вы, которые
по грехам своим отчаялись в спасении, и, прилагая грехи к грехам, отягощаете
жизнь свою:
\vs 2Er 24:3
если обратитесь к Господу
от всего сердца вашего и будете творить правду в остальные дни своей жизни и
служить Ему по воле Его,
\vs 2Er 24:4
то Он простит прежние
грехи ваши, и обретете власть над делами дьявола.
\vs 2Er 24:5
Угроз же дьявола вовсе не
бойтесь, потому что они бессильны, как нервы человека мертвого.
\vs 2Er 24:6
Итак, слушайте меня и
бойтесь Господа, Который может спасти и погубить: соблюдайте заповеди Его и
будете жить с Богом.
\vs 2Er 24:7
И я сказал ему: господин,
теперь я проникся всеми заповедями Господа, потому что ты со мною;
\vs 2Er 24:8
знаю, что сокрушишь всю
силу дьявола, и мы восторжествуем над ним;
\vs 2Er 24:9
и надеюсь, что могу
соблюсти при помощи Божьей заповеди, которые ты передал.
\vs 2Er 24:10
Соблюдешь, сказал он,
если сердце твое будет чисто пред Господом, и все соблюдут, которые очистят
сердца свои от суетных похотей этого века, и будут жить с Богом.

\bibbookdescr{3Er}{
  inline={Пастырь Ермы. Книга 3. Подобия},
  toc={3-я Ермы},
  bookmark={3-я Ермы},
  header={3-я Ермы},
  abbr={3~Ермы}
}
\chhdr{Подобие 1-е.}
\vs 3Er 1:1
Мы, не имея в этом мире постоянного города, должны искать будущего.
\vs 3Er 1:2
Пастырь сказал мне: знаете
ли, что вы, рабы Божьи, находитесь в странствии? Ваш город далеко отсюда. Если
знаете ваше отечество, в котором надлежит вам жить, то зачем здесь покупаете
поместья, строите великолепные здания и ненужные жилища?
\vs 3Er 1:3
Ибо кто занимается подобными приготовлениями в этом городе, тот не помышляет о
возвращении в свое отечество. Несмысленный, двоедушный и жалкий человек, разве
не понимаешь, что всё это чужое и под властью другого?
\vs 3Er 1:4
Ибо господин этого города
говорит: или следуй моим законам, или убирайся вон из моих пределов. Что же
поэтому сделаешь ты, имея собственный закон в твоем отечестве? Ужели ради
полей или других стяжаний своих откажешься от отечественного закона?
\vs 3Er 1:5
Если же ты откажешься, а
потом пожелаешь возвратиться в свое отечество, то не будешь принят, но изгнан
оттуда.
\vs 3Er 1:6
Итак, смотри, подобно
страннику на чужой стороне, не приготовляй себе ничего более того, сколько
тебе необходимо для жизни;
\vs 3Er 1:7
и будь готов к тому,
чтобы, когда господин этого города захочет изгнать тебя за то, что не
повинуешься закону его,~--- идти тебе в своё отечество и жить по своему закону
беспечально и радостно.
\vs 3Er 1:8
Итак, вы, служащие Богу и
имеющие Его в сердцах своих, смотрите: делайте дела Божьи, помня о заповедях
Его и обетованиях, Им данных, и веруйте Ему, что Он исполнит их, если будут
соблюдены Его заповеди.
\vs 3Er 1:9
Вместо полей искупайте
души от нужд, сколько кто может, помогайте вдовам и сиротам; богатство и все
стяжания ваши употребляйте на такого рода дела, ради которых вы и получили их
от Бога.
\vs 3Er 1:10
Ибо Господь обогатил вас
для того, чтобы вы исполняли такое служение Ему.
\vs 3Er 1:11
Гораздо лучше делать это,
нежели покупать дома и поместья, ибо имущество тленно, тогда как то, что
сделаешь во имя Божье, обретешь в своём городе и будешь иметь радость без
печали и страха.
\vs 3Er 1:12
Итак, не желайте богатств
народов, ибо несвойственны они рабам Божьим; избытком же своим распоряжайтесь
так, чтобы могли вы получить радость.
\vs 3Er 1:13
И не делайте фальшивой
монеты, не касайтесь и не желайте чужого. Делай своё дело~--- и спасешься.

\chhdr{Подобие 2-е.}
\vs 3Er 2:1
Однажды, когда я, прогуливаясь по полю, увидел вяз и
виноградное дерево и размышлял о плодах их~--- пастырь явился мне и спросил: что
ты думаешь об этом виноградном дереве и вязе?
\vs 3Er 2:2
Думаю, что они пригодны
друг для друга.
\vs 3Er 2:3
И сказал он мне: эти два
дерева являют рабам Божьим глубокий смысл.
\vs 3Er 2:4
Желал бы я познать,
господин, этот смысл.
\vs 3Er 2:5
Смотрите же,~--- сказал он.
Это виноградное дерево имеет плод, а вяз~--- дерево бесплодное; но виноградное
дерево не может приносить обильных плодов, если не будет опираться на вяз.
\vs 3Er 2:6
Ибо, лёжа на земле, оно
дает гнилой плод; но если виноградная лоза будет висеть на вязе, то дает плод
и за себя, и за вяз.
\vs 3Er 2:7
Итак, видишь, что вяз дает
плод не меньший, а гораздо больший, нежели виноградная лоза, потому что
виноградная лоза, поддерживаемая вязом, дает плод и обильный и хороший, но,
лёжа на земле, дает плод плохой и малый.
\vs 3Er 2:8
Это служит притчею для
рабов Божьих, для бедного и богатого.
\vs 3Er 2:9
Каким образом, объясни
мне?
\vs 3Er 2:10
Слушай,~--- говорит он.
Богатый имеет много сокровищ, но беден перед Господом. Занятый своими
богатствами, он очень мало молится Господу и если имеет какую молитву, то
скудную и не имеющую силы.
\vs 3Er 2:11
Но когда богатый подает
бедному то, в чем он нуждается, тогда бедный молит Господа за богатого, и Бог
подает богатому все блага, потому что бедный богат в молитве и молитва его
имеет великую силу пред Господом.
\vs 3Er 2:12
Богатый подает бедному,
веруя, что ему внимает Господь, и охотно и без сомнения подает ему всё,
заботясь, чтобы у него не было в чем-нибудь недостатка.
\vs 3Er 2:13
Бедный благодарит Бога за
богатого, дающего ему.
\vs 3Er 2:14
Так люди, думая, что вяз
не дает плода, не понимают того, что во время засухи вяз, имея в себе влагу
питает виноградную лозу, и виноградная лоза благодаря этому дает двойной плод
и за себя, и за вяз.
\vs 3Er 2:15
Так и бедные, моля
Господа за богатых, бывают услышаны и умножают богатства их, а богатые,
помогая бедным, ободряют их души. Те и другие участвуют в добром деле.
\vs 3Er 2:16
Итак, кто поступает таким
образом, не будет оставлен Господом, но будет вписан в книгу жизни.
\vs 3Er 2:17
Блаженны те, которые,
имея богатство, сознают, что они обогащаются от Господа, ибо кто почувствует
это, тот может совершать добро.

\chhdr{Подобие 3-е.}
\vs 3Er 3:1
Пастырь показал мне много деревьев без листьев, казавшихся
иссохшими: все они были похожи.
\vs 3Er 3:2
Видишь эти деревья?
\vs 3Er 3:3
Вижу,~--- говорю я.~--- Они
похожи друг на друга и сухи.
\vs 3Er 3:4
Эти деревья служат образом
людей, живущих в этом мире.
\vs 3Er 3:5
Почему же, господин,~--- спросил я,~--- они как бы засохли и похожи друг на друга?
\vs 3Er 3:6
Потому,~--- отвечал он,~--- что в этом веке не различимы
ни праведные, ни нечестивые люди: одни походят на других.
\vs 3Er 3:7
Ибо настоящий век есть
зима для праведных, которые, живя с грешниками, по виду не отличаются от них.
\vs 3Er 3:8
Как во время зимы все
деревья с облетевшими листьями сходны между собою, и не видно, которые из них
действительно засохли, а которые живы, так точно в настоящем веке нельзя
распознать праведников и грешников, но все похожи одни на других.

\chhdr{Подобие 4-е.}
\vs 3Er 4:1
Снова показал мне пастырь многие деревья, из которых одни
расцвели, а другие были иссохшие.
\vs 3Er 4:2
Видишь ли эти деревья?
\vs 3Er 4:3
Вижу, господин,~--- отвечал
я,~--- одни засохли, а другие покрыты листьями.
\vs 3Er 4:4
Эти зеленеющие деревья,~--- сказал он,~--- означают праведных, которые будут жить в грядущем веке.
\vs 3Er 4:5
Ибо будущий век есть лето
для праведных и зима для грешников.
\vs 3Er 4:6
Итак, когда воссияет
благость Господа, тогда явятся служащие Богу и все будут видимы.
\vs 3Er 4:7
Ибо как летом созревает
плод всякого дерева, и становится понятно, каково оно, так точно обнаружится и
будет видим и плод праведных, и все они явятся радостными в том веке.
\vs 3Er 4:8
Народы же и грешники суть
сухие деревья, которые ты видел. Они обретутся в будущем веке сухими и
бесплодными, и будут преданы огню, как дрова, и обнаружится, что во время их
жизни дела их были злы.
\vs 3Er 4:9
Грешники будут преданы
огню, потому что согрешили и не раскаялись в грехах своих, народы же потому,
что не познали Бога~--- Творца своего.
\vs 3Er 4:10
Посему ты приноси плод
добрый, чтобы он явился во время того лета. Воздерживайся от многих попечений
и никогда не согрешишь.
\vs 3Er 4:11
Ибо имеющие многие заботы
согрешают во многом, потому что озабочены своими делами и не служат Богу.
\vs 3Er 4:12
Каким же образом человек,
не служащий Богу, может просить и получить что-либо от Бога?
\vs 3Er 4:13
Те, которые служат Богу,
просят и получат свои прошения, а не служащие Богу~--- не получат.
\vs 3Er 4:14
Кто занимается одним
делом, тот может и служить Богу; потому что дух его не отчуждается от Господа,
но чистою мыслию служит Богу.
\vs 3Er 4:15
Итак, если исполнишь это,
будешь иметь плод в грядущем веке; равно как и все, которые исполнят это,
будут иметь плод.

\chhdr{Подобие 5-е.}
\vs 3Er 5:1
Однажды во время поста сидел я на горе, благодарил Господа за
то, что сделал Он со мною, и увидел вдруг пастыря рядом с собою.
\vs 3Er 5:2
И спрашивает он у меня: что так рано пришел ты сюда?
\vs 3Er 5:3
Потому, господин, что нахожусь на стоянии.
\vs 3Er 5:4
А что такое стояние?
\vs 3Er 5:5
То есть пощусь, господин,~--- объяснил я.
\vs 3Er 5:6
Каким же образом,~--- спросил он,~--- постишься ты?
\vs 3Er 5:7
Как постился по обыкновению, так и пощусь.
\vs 3Er 5:8
Не умеете вы,~--- сказал он,
поститься Богу; и пост, который совершаете, бесполезен.
\vs 3Er 5:9
Почему, господин, говоришь так?
\vs 3Er 5:10
То, как вы думаете
поститься, не есть пост, но я научу тебя, какой пост есть совершенный и
угодный Богу.
\vs 3Er 5:11
Слушай: Бог не хочет
такого суетного поста, ибо, постясь таким образом, ты не совершаешь правды.
\vs 3Er 5:12
Постись же Богу следующим
постом: не лукавствуй в жизни, но служи Богу чистым сердцем; соблюдай Его
заповеди, ходи в Его повелениях и не допускай никакой злой похоти в сердце
своем.
\vs 3Er 5:13
Веруй в Бога, и если
исполнишь это и будешь иметь страх Божий и удержишься от всякого злого дела,
то будешь жить с Богом.
\vs 3Er 5:14
И таким образом ты
совершишь великий и угодный Богу пост.

\vs 3Er 6:1
Послушай притчу
относительно поста, которую я намерен поведать тебе.
\vs 3Er 6:2
Некто имел поместье и
много рабов. В одной части земли своей он насадил виноградник, и потом,
отправляясь в дальнее путешествие, избрал раба, самого верного и честного, и
поручил ему виноградник с тем, чтобы он к виноградным лозам приставил
подпорки, обещая за исполнение этого приказания дать ему свободу.
\vs 3Er 6:3
Только это хозяин приказал
рабу сделать в винограднике и с тем отправился.
\vs 3Er 6:4
Раб тщательно сделал, что
господин повелел: он расставил подпорки в винограднике, но, приметив в нём
много сорных трав, стал рассуждать сам с собою: я исполнил приказание
господина, вскопаю теперь виноградник, и он будет красивее; а если выполоть
сорную траву, он, не заглушаемый сорняками, даст больше плода.
\vs 3Er 6:5
И принялся за работу;
вскопал виноградник и выполол в нём все сорняки, и стал виноградник красивым и
цветущим, не засоренным травами.
\vs 3Er 6:6
Через некоторое время
возвратился господин его и пришел в виноградник. Когда он увидел, что
виноградник хорошо обставлен и сверх того вскопан, прополот, и лозы обильны
плодами, то был весьма доволен поступком раба своего.
\vs 3Er 6:7
Итак, пригласил он
любимого сына, своего наследника, и друзей, своих советников, и рассказал им,
что приказал он сделать рабу своему и что тот сверх этого сделал.
\vs 3Er 6:8
Они тотчас приветствовали
раба с тем, что он получил столь высокую похвалу от своего господина.
\vs 3Er 6:9
Господин же говорит им:
<<Я обещал свободу этому рабу, если он исполнит данное приказание, он исполнил его
и сверх того приложил к винограднику добрый труд, который мне весьма
понравился.
\vs 3Er 6:10
Поэтому за его усердие я
хочу сделать его сонаследником моего сына, потому что, помысливши доброе, он
не оставил его, но исполнил.>>
\vs 3Er 6:11
Это намерение господина,
то есть чтобы раб был сонаследником сыну, одобрили и сын, и друзья его.
\vs 3Er 6:12
Потом, спустя несколько
дней, когда созваны были гости, господин со своего пира посылал тому рабу
много яств.
\vs 3Er 6:13
Получая их, раб брал из
них то, что было для него достаточно, остальное же делил между товарищами
своими.
\vs 3Er 6:14
Они, обрадованные, начали
желать ему, чтобы он еще большую любовь нашел у хозяина за свою доброту и
щедрость.
\vs 3Er 6:15
Когда обо всем этом узнал
господин его, он опять весьма обрадовался и снова рассказал друзьям и сыну о
поступке своего раба, и они еще более одобрили мысль господина, чтобы раб этот
был сонаследником сына.

\vs 3Er 7:1
Я сказал: господин, не
знаю этих притчей и не смогу понять, если ты не объяснишь мне их.
\vs 3Er 7:2
Всё,~--- обещал он,~--- объясню, что только скажу и покажу тебе. Соблюдай заповеди Господа, и будешь
угоден Богу и включен в число тех, которые соблюли Его заповеди.
\vs 3Er 7:3
Если же сделаешь что-либо
доброе сверх заповеданного Господом, то приобретешь себе еще большее
достоинство и будешь пред Господом славнее, нежели мог быть прежде.
\vs 3Er 7:4
Итак, если соблюдешь
заповеди Господа и к ним присоединишь эти стояния, то получишь великую
радость, особенно если будешь исполнять их согласно с моим внушением.
\vs 3Er 7:5
Господин,~--- говорю,~--- я
исполню все, что ни повелишь мне, ибо я знаю, что ты будешь со мною.
\vs 3Er 7:6
Буду,~--- сказал он,~--- с
тобою, потому что имеешь такое доброе намерение; буду также и со всеми
имеющими такое намерение.
\vs 3Er 7:7
Этот пост,~--- продолжал он,
при исполнении заповедей Господа очень хорош, и соблюдай его таким образом:
прежде всего воздерживайся от всякого дурного слова и злой похоти и очисти
сердце своё от всех сует века сего.
\vs 3Er 7:8
Если соблюдать это, пост у
тебя будет праведный.
\vs 3Er 7:9
Поступай же так: исполнив
вышесказанное, в тот день, в который постишься, ничего не вкушай, кроме хлеба
и воды; а то из пищи, что ты в этот день сбережешь таким образом, отложи и
отдай вдове, сироте или бедному;
\vs 3Er 7:10
таким образом ты смиришь
свою душу; а получивший от тебя насытит свою душу и будет за тебя молиться
Господу.
\vs 3Er 7:11
Если будешь совершать
пост так, как я повелел тебе, то жертва твоя будет приятна Господу, и этот
пост будет написан, и дело, таким образом совершаемое, прекрасно, радостно и
угодно Господу.
\vs 3Er 7:12
Если ты соблюдешь это с
детьми своими и со всеми домашними твоими, то будешь блажен; и все, кто только
соблюдут это, будут блаженны и что ни попросят у Господа, всё получат.

\vs 3Er 8:1
И упрашивал я его, чтобы
объяснил мне эту притчу о поместье и господине, о винограднике и рабе,
поставившем подпорки в нем, о травах, выполотых в винограднике, о сыне и
друзьях, призванных для совета: ибо я понял, что все это~--- притча.
\vs 3Er 8:2
Он сказал мне: очень смел
ты на вопросы. Ты ни о чем не должен спрашивать; что должно быть объяснено, то
объяснится тебе.
\vs 3Er 8:3
Господин, я напрасно буду
видеть то, что ты покажешь мне, не истолковав, что это значит; напрасно буду
слушать и притчи, если ты будешь предлагать их мне без объяснения.
\vs 3Er 8:4
Он сказал мне снова: кто
раб Божий и в сердце своем имеет Господа, тот просит у Него разума и получает,
и постигает всякую притчу, и понимает слова Господа, сказанные приточно.
\vs 3Er 8:5
А беспечные и ленивые к
молитве колеблются просить Господа, тогда как Господь многомилостив и
непрестанно дает всем просящим у Него.
\vs 3Er 8:6
Ты же утвержден тем
достопоклоняемым ангелом и получил от Него столь могущественную молитву.
Почему, если не ленив ты, не просишь разума и не получаешь от Господа?
\vs 3Er 8:7
Если ты при мне,~--- сказал
я ему~--- надлежит мне тебя обо всем просить и спрашивать, ибо ты всё мне
показываешь и говоришь со мною. Если бы без тебя я видел это или слышал, тогда
бы Господа просил, чтобы было мне объяснено.

\vs 3Er 9:1
И он отвечал: я и прежде
говорил тебе, что ты искусен и смел на то, чтобы спрашивать смысл притчей.
\vs 3Er 9:2
Так как ты настойчив, то
объясню тебе притчу о поместье и о прочем, чтобы ты рассказал всем.
\vs 3Er 9:3
Слушай же и разумей.
Поместье, о котором говорится в притче, означает мир. Владелец поместья есть
Творец, который всё создал и утвердил. Сын есть Дух Святой. Раб~--- Сын Божий.
\vs 3Er 9:4
Виноградник означает
народ, который насадил Господь. Подпорки суть ангелы, приставленные Господом
для сохранения Его народа.
\vs 3Er 9:5
Травы, уничтоженные в
винограднике, суть преступления рабов Божьих. Яства, которые с пира посылал
господин рабу, суть заповеди, которые через Сына своего дал Господь своему
народу.
\vs 3Er 9:6
Друзья, призванные на
совет, суть святые ангелы первозданные.
\vs 3Er 9:7
Отсутствие же господина
означает время, остающееся до Его Пришествия.

\vs 3Er 10:1
Я сказал тогда: господин,
величественно, дивно и славно всё, что ты поведал, но мог ли я, господин,
понять это?
\vs 3Er 10:2
Да ни один человек, хотя
бы и очень разумный, не может постичь этого. Теперь же спрошу тебя вот о чем.
\vs 3Er 10:3
Спрашивай, что хочешь.
\vs 3Er 10:4
Почему Сын Божий в этой
притче представляется рабом?
\vs 3Er 10:5
Слушай,~--- сказал он. Сын
Божий предстает в рабском положении, но имеет великое могущество и власть.
\vs 3Er 10:6
Каким образом, господин,
не понимаю?
\vs 3Er 10:7
Бог насадил виноградник,
то есть создал народ и поручил Сыну своему; Сын же приставил ангелов для
сохранения каждого из людей и сам усердно трудился и изрядно пострадал, чтобы
искупить грехи их.
\vs 3Er 10:8
Ибо никакой виноградник не
может быть очищен без труда и подвига.
\vs 3Er 10:9
Итак, очистив грехи народа
Своего, Он показал им путь жизни и дал им закон, принятый Им от Отца.
\vs 3Er 10:10
Видишь, что Он есть
Господь народа со всею властью, полученною от Отца.
\vs 3Er 10:11
А вот почему Господь
держал совет о наследстве с Сыном Своим и славными ангелами. Дух Святой,
прежде Сущий, создавший всю тварь, Бог поселил в плоть, какую Он пожелал.
\vs 3Er 10:12
И эта плоть, в которую
вселился Дух Святой, хорошо послужила Духу, ходя в чистоте и святости и ничем
не осквернив Духа.
\vs 3Er 10:13
И так как жила она
непорочно, и подвизалась вместе с Духом, и мужественно содействовала Ему во
всяком деле, то Бог принял её в общение, ибо Ему угодно было житие плоти,
которая не осквернилась на земле, имея в себе Дух Святой.
\vs 3Er 10:14
И призвал Он в совет Сына
и добрых ангелов, чтобы и эта плоть, непорочно послужившая Духу, обрела место
успокоения, чтобы не оказалась без награды непорочная и чистая, в которой
поселился Святой Дух. Вот тебе объяснение этой притчи.

\vs 3Er 11:1
Возрадовался я, господин,
сказал я, услышав такое объяснение.
\vs 3Er 11:2
Слушай далее. Эту плоть
храни неоскверненною и чистою, чтобы дух, живущий в ней, был доволен ею и
спаслась твоя плоть.
\vs 3Er 11:3
Смотри также, никогда не
допускай мысли, что эта плоть погибнет, и не злоупотребляй ею в какой-либо
похоти.
\vs 3Er 11:4
Ибо если осквернишь плоть
свою, то осквернишь и Духа Святого, если же осквернишь Духа Святого, не будешь
жить.
\vs 3Er 11:5
И спросил я: что же, если
кто по неведению, до того, как услышать эти слова, осквернил свою плоть, каким
образом получит он спасение?
\vs 3Er 11:6
Прежние грехи неведения,~--- сказал он,~--- исцелить может один Бог, ибо Ему принадлежит всякая власть.
\vs 3Er 11:7
Но теперь храни себя; и
Господь Всемогущий и милостивый даст искупление для прежних грехов, если
впредь не осквернишь плоти своей и духа. Ибо они взаимопричастны, и одна без
другого не оскверняется.
\vs 3Er 11:8
Итак, и то и другое
сохраняй чистым и будешь жить с Богом.

\chhdr{Подобие 6-е.}
\vs 3Er 12:1
Когда я, сидя дома, прославлял Господа за всё то, что видел,
и размышлял о заповедях, как они прекрасны, тверды, почтенны и сладостны и
могут спасти душу человека,
\vs 3Er 12:2
то я говорил сам себе:
<<Блажен буду, если стану поступать по этим заповедям; и всякий поступающий по
ним, будет блажен!>>
\vs 3Er 12:3
Когда рассуждал таким
образом, вдруг пастырь появился возле меня
\vs 3Er 12:4
и сказал: что раздумываешь
о заповедях моих, которые я тебе преподал? Они прекрасны, нисколько не
сомневайся; но облекись верою в Господа и будешь исполнять их, ибо наделю тебя
для этого силой.
\vs 3Er 12:5
Заповеди эти полезны для
тех, которые хотят покаяться; если не будут исполнять их, то тщетным будет их
покаяние.
\vs 3Er 12:6
Итак, вы, кающиеся,
отриньте от себя лукавства этого века, губящие вас. Облекитесь же во всякую
добродетель, чтобы вы могли соблюсти эти заповеди, и ничего не прибавляйте к
грехам вашим.
\vs 3Er 12:7
Ибо если снова не будете
грешить, то загладите прежние грехи. Поступайте по заповедям моим и будете
жить с Богом. Все это мною наказано вам.
\vs 3Er 12:8
После этих слов он
продолжал: пойдем в поле, и я покажу тебе пастухов овец.
\vs 3Er 12:9
Пойдем, господин,~--- согласился я.
\vs 3Er 12:10
Пошли мы и в поле увидели
молодого пастуха, одетого в богатые одежды багряного цвета; стадо его было
многочисленно, и ухоженные овцы весело резвились в травах. И сам пастух
радовался на свое стадо и с довольным лицом ходил около овец.

\vs 3Er 13:1
Ангел указал мне на
пастуха и сказал: это~--- ангел наслаждения и лжи, он изводит души рабов Божьих,
отвращая их от истины, обольщая злыми пожеланиями;
\vs 3Er 13:2
и они забывают заповеди
живого Бога и живут в роскоши и суетных удовольствиях, и этот злой ангел губит
их~--- некоторых до смерти, а некоторых до растления.
\vs 3Er 13:3
Господин,~--- спросил я,~--- как понять <<до смерти>>
и что значит <<до растления>>?
\vs 3Er 13:4
Слушай. Овцы, которых ты
видел резвящимися, это те, которые навсегда отреклись от Бога и предались
удовольствиям этого века.
\vs 3Er 13:5
Поэтому им нет возврата к
жизни через покаяние, ибо они к другим своим преступлениям прибавили еще
больше~--- нечестиво хулили имя Господа. Жизнь таких людей подобна смерти.
\vs 3Er 13:6
А овцы, которые не скакали
по полю, а скучно паслись, означают тех, которые хоть и предавались
наслаждениям и удовольствиям, но не возводили хулы против Господа: они не
отошли от истины, и для них есть еще покаяние, посредством которого они спасут
жизнь.
\vs 3Er 13:7
В растлении есть некоторая
надежда на восстановление; а смерть имеет окончательную погибель.
\vs 3Er 13:8
Еще прошли мы немного, и
он показал мне большого пастуха, дикого на вид, одетого в белую козью шкуру, с
сумой на плечах, сучковатой и крепкой палкой и большим бичом в руках; лицо его
было суровое и грозное, так что становилось страшно.
\vs 3Er 13:9
Он принимал от юного
пастуха овец, которые жили в неге и наслаждении, но не скакали; он отгонял их
в местность скалистую и тернистую, и овцы, запутавшись в колючках, сильно
страдали, а пастух осыпал их ударами, гонял туда и сюда, не давая им покоя и
не позволяя где-либо остановиться.

\vs 3Er 14:1
Видя, что овцы
подвергаются побоям, терпят такие мучения и не находят покоя, я пожалел их и
спросил пастыря, кто этот безжалостный и жестокий пастух, не имеющий ни
малейшего сострадания к овцам.
\vs 3Er 14:2
Это,~--- ответил пастырь,~--- ангел наказания; он из праведных ангелов, но приставлен для наказания. Ему
вверяются те, которые уклонились от Бога и предались похотям и удовольствиям
этого века; и он наказывает их, как они того заслуживают, различными жестокими
мучениями.
\vs 3Er 14:3
Расскажи мне, господин,~---
попросил я,~--- что это за мучения, какого рода они?
\vs 3Er 14:4
Слушай: эти различные
наказания и мучения~--- те, которые люди терпят в своей ежедневной жизни. Одни
терпят убытки, другие~--- бедность, иные~--- различные болезни,
некоторые~--- непостоянство в жизни, другие подвергаются обидам от людей недостойных и
многим иным неприятностям.
\vs 3Er 14:5
Очень многие с
непостоянными намерениями принимаются за различные дела, но ничто им не
удается, и жалуются они, что не имеют успеха в своих начинаниях; не приходит
им мысль, что они творят худые дела, но жалуются на Господа.
\vs 3Er 14:6
После того, как натерпятся
они всякой скорби, они предаются мне для доброго увещевания, укрепляются в
вере в Господа и в остальные дни жизни своей служат Господу чистым сердцем.
\vs 3Er 14:7
И когда начнут они каяться
в преступлениях, тогда на сердце их приходят беззаконные дела их и они воздают
славу Господу, говоря, что Он~--- Судия праведный и что они всё претерпели
достойно по делам своим.
\vs 3Er 14:8
И в остальное время служат
Богу чистым сердцем и имеют успех во всех делах своих, получая от Бога всё,
чего ни попросят; и тогда благодарят Бога, что вручены мне, и уже не
подвергаются более никакой жестокости.

\vs 3Er 15:1
И захотел я узнать,
столько ли времени мучаются оставившие страх Божий, сколько наслаждались
удовольствиями, и спросил пастыря об этом.
\vs 3Er 15:2
Столько же времени и
мучаются,~--- ответил он.
\vs 3Er 15:3
Мало они мучаются, надобно
бы предавшимся удовольствиям и забывшим Бога терпеть наказания в семь раз
более.
\vs 3Er 15:4
Неразумен ты,~--- сказал он,
и не понимаешь силы наказания.
\vs 3Er 15:5
Господин, если бы я
понимал, то и не просил бы тебя объяснить мне.
\vs 3Er 15:6
Слушай,~--- сказал он,~--- какова сила того и другого~--- наслаждения и наказания. Один час наслаждения
ограничивается своим протяжением, а один час наказания имеет силу тридцати
дней.
\vs 3Er 15:7
Кто один день предавался
наслаждению и удовольствию, тот будет мучиться один день, но день мучения
будет стоить целого года.
\vs 3Er 15:8
Следовательно, сколько
дней кто наслаждается, столько лет мучится. Видишь,~--- заключил он,~--- что время
мирского наслаждения и обольщения очень кратко, а время наказания и мучения
велико.

\vs 3Er 16:1
Я сказал ему: не совсем
понимаю относительно времени наслаждения и наказания, объясни мне лучше.
\vs 3Er 16:2
Он ответил: неразумие твоё
упорно остается с тобою, и ты не хочешь очистить сердце своё и служить Богу.
Смотри, чтобы не оказаться тебе неразумным, когда исполнится время.
\vs 3Er 16:3
А теперь слушай, если
желаешь понять. Кто один день предавался удовольствиям и делал, что было
угодно душе его, тот исполняется великим неразумием и наутро не понимает своих
действий и не помнит, что делал накануне, ибо наслаждение и обольщение не
имеют никакой памяти по причине неразумия, которым человек исполняется.
\vs 3Er 16:4
Но когда на один день
придет человеку наказание и мучение, то он страдает целый год, потому что
наказание и мучение имеют великую память.
\vs 3Er 16:5
Страдающий в течение
целого года вспоминает и о суетном наслаждении и сознаёт, что за него он
терпит зло.
\vs 3Er 16:6
Таким-то образом
наказываются те, которые предались наслаждению и обольщению; потому что,
наделенные жизнью, сами себя предали смерти.
\vs 3Er 16:7
Я спросил: господин, какие
удовольствия вредны?
\vs 3Er 16:8
Любое дело,~--- ответил он,
доставляет удовольствие человеку, если он выполняет его с приятностью.
\vs 3Er 16:9
Ибо и гневливый, исполняя
свое дело, получает удовольствие, и расово смешивающийся, и пьяница, и
клеветник, и лжец, и любостяжательный человек, и хищник, и всякий совершающий
что-либо подобное удовлетворяет свою страсть и наслаждается своим делом. Все
эти наслаждения вредны рабам Божьим, и за них-то они страдают и терпят
наказания.
\vs 3Er 16:11
Но есть также
удовольствия, спасительные для людей: многие, совершая добрые дела, получают
удовольствие, находя в них для себя сладость. Это удовольствие полезно рабам
Божьим и приготовляет жизнь таким людям.
\vs 3Er 16:12
А те, о которых сказано
прежде, заслуживают наказания и мучения, и те, которые будут нести их и не
покаются в своих преступлениях, обрекут себя на смерть.

\chhdr{Подобие 7-е.}
\vs 3Er 17:1
Спустя несколько дней я встретил пастыря на том поле, на
котором прежде видел пастухов,
\vs 3Er 17:2
и спросил он меня: чего ты
ищешь?
\vs 3Er 17:3
Я пришел, господин,
просить тебя, чтобы ты приказал удалиться из моего дома пастырю,
приставленному для наказания, потому что он сильно поражает меня.
\vs 3Er 17:4
Он сказал мне в ответ:
необходимо пережить тебе бедствия и скорби, потому что так заповедал тебе тот
славный ангел, который хочет испытать тебя.
\vs 3Er 17:5
Какое же зло, господин,
сделал я, что предан этому ангелу?
\vs 3Er 17:6
Слушай,~--- сказал он. Ты
имеешь очень много грехов, но не столь много, чтобы следовало тебя предать
этому ангелу;
\vs 3Er 17:7
но домочадцы твои
совершили великие грехи и преступления, и тот славный ангел прогневался на их
дела и повелел понести тебе наказание некоторое время, чтобы и они покаялись в
своих прегрешениях и очистились от всякой скверны этого века. И когда они
покаются и очистятся, тогда удалится от тебя ангел наказания.
\vs 3Er 17:8
Я сказал ему: господин,
если они так вели себя, что рассердили славного ангела, в чем же моя вина?
\vs 3Er 17:9
Он отвечал: они не могут
быть наказаны, если ты, глава всего дома, не подвергнешься наказанию. Ибо всё,
что претерпишь ты, неизбежно претерпят и они, а при твоем благополучии они не
могут испытать никакого мучения.
\vs 3Er 17:10
Но теперь, господин,~--- сказал я,~--- они уже покаялись от всего сердца своего.
\vs 3Er 17:11
Знаю, что они покаялись
от всего сердца. Но не думаешь ли ты, что тотчас отпускаются грехи кающихся?
Нет, кающийся должен помучить свою душу, смириться во всяком деле своем и
перенести многие и различные скорби.
\vs 3Er 17:12
И когда перенесет всё,
что ему назначено, тогда, конечно, Тот, Который всё сотворил и утвердил,
подвигнется к нему Своею милостью и даст ему спасительное врачевание, и лишь
тогда, когда увидит, что сердце кающегося чисто от всякого злого дела.
\vs 3Er 17:13
А тебе и семейству твоему
пострадать теперь полезно. Нужно пострадать так, как повелел тот ангел
Господа, который мне предал тебя.
\vs 3Er 17:14
А ты лучше благодари
Господа, что Он удостоил предварительно открыть тебе наказание, чтобы, наперед
зная о нём, ты стойко перенёс его.
\vs 3Er 17:15
И я просил его: господин,
будь со мною, и я легко перенесу всякое бедствие.
\vs 3Er 17:16
Я буду с тобою и даже
попрошу ангела наказания, чтобы он легче поражал тебя; впрочем, не долго ты
потерпишь бедствие и снова возвратишься в свое благосостояние, только пребывай
в смиренномудрии и повинуйся Господу от чистого сердца.
\vs 3Er 17:17
Пусть и дети твои, и весь
дом твой живут по заповедям, которые я тебе преподал,~--- и покаяние ваше может
сделаться твердым и чистым.
\vs 3Er 17:18
И если ты с семьей своей
соблюдешь мои заповеди, то удалится от тебя всякое бедствие; и от всех тех,
которые будут придерживаться этих заповедей, удалится всякое бедствие.

\chhdr{Подобие 8-е.}
\vs 3Er 18:1
Пастырь показал мне заросли ивы, покрывшие поля и горы, в
тень которых пришли все призванные в имени Господа.
\vs 3Er 18:2
И подле этой ивы стоял славный, весьма высокий ангел, он большим серпом срезал
с ивы ветки и раздавал их народу.
\vs 3Er 18:3
После того, как все получили ветки, ангел положил серп, но дерево осталось
таким же целым, каким я видел его прежде. Очень я удивился этому,
\vs 3Er 18:4
а пастырь сказал: не удивляйся, что дерево осталось невредимо после того, как
срезано было с него столько веток. Подожди, что будет дальше, и станет
понятным тебе, что всё это означает.
\vs 3Er 18:5
Ангел, раздававший ветки,
потребовал их назад. В том же порядке, в каком получали, он подзывал людей:
они подходили и возвращали ветки.
\vs 3Er 18:6
Ангел Господень принимал
их и рассматривал. От некоторых он получал сухие, как бы изъеденные молью
ветки, и тем он повелел стать отдельно; те, которые вернули ветки сухие, но не
тронутые молью, тоже стали отдельно.
\vs 3Er 18:7
Особо стали и те, кто
принес ветки полусухие и с трещинами, и те, чьи ветки были наполовину сухие,
наполовину зеленые.
\vs 3Er 18:8
Некоторые возвращали ветки
на две трети сухими, а на треть~--- зелеными; а некоторые~--- наоборот: на две
трети зелеными и на треть~--- сухими. Ангел их также поставил отдельно.
\vs 3Er 18:9
Иные подавали ветки
полностью зеленые, и только малая часть их, самая верхушка, была сухая, и они
были потрескавшиеся.
\vs 3Er 18:10
А в других ветках было
совсем мало зеленого.
\vs 3Er 18:11
А у большинства людей
были такие же зеленые ветки, какими они их и получили; ангел весьма радовался
им.
\vs 3Er 18:12
Иные отдавали ветки
зелеными и с молодыми побегами, ангел принимал их также с большим
удовольствием.
\vs 3Er 18:13
У некоторых зеленые ветки
были и с новыми отростками, и с плодами на них. Мужи, возвращающие такие
ветки, приходили с очень довольным видом, и сам ангел был весьма весел, и
пастырь тоже радовался.

\vs 3Er 19:1
Потом ангел Господа велел
принести венцы.
\vs 3Er 19:2
Принесены были венцы,
словно сплетенные из пальмовых листьев, и ангел надел их на тех мужей, ветки
которых были с отростками и плодами, и велел им идти в башню;
\vs 3Er 19:3
и других мужей, ветки
которых были зелены и с побегами, но без плодов, послал туда же, дав им
печать.
\vs 3Er 19:4
На всех входивших в башню
была одежда, белая как снег.
\vs 3Er 19:5
В ту же башню послал он и
тех, которые возвратили свои ветки такими же зелеными, как приняли, дав им
печать и белую одежду.
\vs 3Er 19:6
По окончании этого он
обратился к пастырю: я пойду, а ты впусти их внутрь стен, на то место, какое
каждый заслужил, но прежде рассмотри внимательно их ветки; следи, чтобы
кто-нибудь не миновал тебя; если же кто пройдет мимо, я обличу их перед
алтарем.
\vs 3Er 19:7
Он удалился, после чего
пастырь сказал мне: возьмем у них ветки и посадим их в землю, может быть,
некоторые из них зазеленеют снова?
\vs 3Er 19:8
Я удивился: господин,
каким образом могут снова зазеленеть ветки, которые уже засохли?
\vs 3Er 19:9
Он ответил мне: это дерево
ива, и оно всегда любит жизнь: поэтому, если эти ветки будут посажены и
получат чуть-чуть влаги, очень многие из них опять зазеленеют.
\vs 3Er 19:10
Попробую полью их водой,
и если какая из них сможет ожить, порадуюсь за неё; если же нет, по крайней
мере, видно будет, что я не был небрежен.
\vs 3Er 19:11
Потом пастырь приказал
мне позвать их в том порядке, в каком они стояли; подошли они и передали свои
ветки. Получив их, пастырь каждую посадил по порядку.
\vs 3Er 19:12
И, рассадив, так обильно
поливал их водою, что вода полностью покрыла их.
\vs 3Er 19:13
Полив, он сказал: пойдем,
а через несколько дней возвратимся и осмотрим все ветки. Ибо Сотворивший это
дерево хочет, чтобы были живы все происшедшие от него ветки.
\vs 3Er 19:14
А я надеюсь, что после
того, как эти ветки политы водою, очень многие из них оживут, напоенные
влагою.

\vs 3Er 20:1
Я попросил: господин,
объясни мне, что означает это дерево; я недоумеваю, почему оно остается целым:
ведь срезано с него столько веток, но не видно, чтобы от него что-нибудь
убавилось?
\vs 3Er 20:2
Слушай,~--- сказал он. Это
большое дерево, покрывающее поля и горы и всю землю, означает Закон Божий,
данный всему миру.
\vs 3Er 20:3
Закон этот есть Сын Божий,
проповеданный во всех концах земли. Люди, стоящие под сенью его, означают тех,
которые услышали проповедь и уверовали в Него.
\vs 3Er 20:4
Величественный и сильный
ангел есть Михаил, который имеет власть над этим народом и управляет им: он
насаждает Закон в сердцах верующих и наблюдает за теми, которым дал Закон,
соблюдают ли они его.
\vs 3Er 20:5
У каждого есть ветки:
ветки означают также Закон Господа.
\vs 3Er 20:6
Видишь, многие из них
сделались негодными, и ты узнаешь всех тех, которые не соблюли Закона, и
увидишь место каждого из них.
\vs 3Er 20:7
Почему же, господин, одних
Он отослал в башню, а других здесь оставил, при тебе?
\vs 3Er 20:8
Те, которые преступили
Закон, от Него принятый, оставлены в моей власти, чтобы покаялись в своих
преступлениях; а которые удовлетворили Закону и его соблюли, находятся под
собственною Его властью.
\vs 3Er 20:9
Кто же, господин, те,
которые увенчаны и вошли в башню?
\vs 3Er 20:10
Он ответил: это те,
которые вели борьбу с дьяволом и победили его; те, которые, соблюдая Закон,
пострадали за него;
\vs 3Er 20:11
другие, которые
возвратили ветки зелеными и с отростками, но без плодов,~--- это те, которые,
хотя и потерпели мучение за тот Закон, но не вкусили смерти и не отреклись от
своего Закона;
\vs 3Er 20:12
те же, которые возвратили
зелеными, какими и взяли, суть кроткие и праведные, которые жили с чистым
сердцем и соблюли заповеди Божии.
\vs 3Er 20:13
Остальное ты узнаешь
тогда, когда пересмотрю ветки, которые я посадил в землю и полил.

\vs 3Er 21:1
Через несколько дней мы
возвратились туда, и пастырь сел на месте того ангела, а я стал подле него, и
он велел мне подпоясаться полотенцем и помогать ему.
\vs 3Er 21:2
Я подпоясался чистым
платом, сделанным из мешка. Видя, что я готов служить ему,
\vs 3Er 21:3
он сказал: зови тех мужей,
ветки которых посажены в землю, в том порядке, в каком каждый их подавал.
\vs 3Er 21:4
И отправился я в поле,
созвал всех, и они стали на свои места. Пусть каждый вынет свою ветку и подаст
мне,~--- указал он.
\vs 3Er 21:5
Прежде всего подали те, у
которых были ветки сухие и гнилые. И так как они опять оказались загнившими и
сухими, то он повелел им стать отдельно.
\vs 3Er 21:6
После подали те, у которых
ранее они были сухие, но не гнилые. Одни из них подали ветки зеленые, а другие
сухие и загнившие, как бы тронутые молью.
\vs 3Er 21:7
Тем, которые подали
зеленые, велел он стать отдельно; а тем, которые подали сухие и загнившие,
велел стать вместе с первыми.
\vs 3Er 21:8
Потом подали те, чьи были
полузасохшие и с трещинами; многие из них принесли ветки зеленые и без трещин;
а некоторые~--- зеленые, имеющие побеги и даже плоды~--- как те, которые
увенчанные вошли в башню;
\vs 3Er 21:9
другие подали сухие и
поврежденные, иные сухие, но не гнилые, а некоторые полусухие и с трещинами,
какими и прежде были.
\vs 3Er 21:10
И всех их пастырь
разделил на группы, повелел каждой стать отдельно.

\vs 3Er 22:1
Потом принесли ветки те, у
которых они были хотя зеленые, но с трещинами: все они подали их теперь
зелеными и стали на своем месте, и пастырь радовался за них, что все они
оправились и заживили свои трещины.
\vs 3Er 22:2
Подали и те, которые
прежде имели ветки наполовину сухие; ветки некоторых из них оказались все
зелеными, других~--- полусухими, иных~--- сухими и поврежденными,
а иных~--- зелёными и с отростками.
\vs 3Er 22:3
Потом подали те, у которых
ветки на две трети были зеленые и на треть сухие; многие из них подали ветки
зеленые, многие полусухие, прочие же сухие и гнилые.
\vs 3Er 22:4
Далее подали те, у которых
до того ветки на две трети были сухие, а на треть зеленые; из них многие
подали полусухие, некоторые сухие и гнилые, другие полусухие и с трещинами, а
иные зеленые.
\vs 3Er 22:5
Потом подали те, у которых
ветки были зелены, но немного и сухи и с трещинами; из них некоторые
возвратили ветки зеленые, другие же зеленые и с побегами; и они отошли на свое
место.
\vs 3Er 22:6
Наконец, у тех, у которых
в ветках было немного зелени, а остальное засохло, ветки большею частью
оказались зелеными, с отростками и даже с плодом на них, а остальные были
зеленые. Этими ветками пастырь весьма был доволен.
\vs 3Er 22:7
И каждого он отправлял на
своё место.

\vs 3Er 23:1
Пересмотрев все ветки,
сказал мне пастырь: я говорил тебе, что дерево это любит жизнь. Видишь, многие
покаялись и получили спасение.
\vs 3Er 23:2
Вижу, господин.
\vs 3Er 23:3
Знай же,~--- продолжал он,~--- велики и славны благость и милость Господа, Который дал дух, способный
покаяться.
\vs 3Er 23:4
Почему же, господин,~--- спросил я,~--- не все покаялись?
\vs 3Er 23:5
Он ответил: Господь дал
покаяние тем, чьи сердца, он видел, будут чисты и кто будет служить Ему
усердно и праведно.
\vs 3Er 23:6
А тем, у которых
чувствовал лукавство, и неправду, и притворное к Нему обращение, не дал
покаяния, чтобы они снова не осквернили имени Его.
\vs 3Er 23:7
Теперь, господин, объясни
мне, что означает каждый из тех, кто возвратил ветки, и где его место, чтобы
узнали об этом уверовавшие, которые получили печать, но сокрушили её и не
сохранили в целости
\vs 3Er 23:8
и, дабы, познав дела свои,
покаялись и, приняв от тебя печать, воздали славу Господу, что подвигся Он к
ним Своею милостью, и послал тебя для обновления душ их.
\vs 3Er 23:9
Слушай,~--- сказал он. У
кого ветки найдены сухими и гнилыми, как бы поврежденными тлёю,~--- это суть
отступники и предатели Церкви, которые во грехах своих хулили Господа и
постыдились имени Его, на них призванного: все они умерли для Бога.
\vs 3Er 23:10
И ты видишь, что никто из
них не покаялся, и они презрели слова Божьи, которые я заповедал тебе; от этих
людей отступила жизнь.
\vs 3Er 23:11
Равным образом недалеко
от них те, которые возвратили ветки сухими, хотя не гнилыми, ибо они были
лицемеры, вводили чуждые учения и совращали рабов Божьих, особенно тех,
которые согрешили, не дозволяя им возвращаться к покаянию, но внушая им
вредные мысли.
\vs 3Er 23:12
Они имеют надежду
покаяния; и ты видишь, что многие из них уже покаялись после того, как я
возвестил им мои заповеди, и еще покаются.
\vs 3Er 23:13
Те, которые не покаются,
потеряли жизнь свою; те же, которые покаялись, сделались добрыми и
местопребыванием их стали первые стены, а некоторые вошли даже внутрь башни.
\vs 3Er 23:14
Итак, видишь, покаяние
грешников несет в себе жизнь, а нераскаянность~--- смерть.

\vs 3Er 24:1
Послушай и о тех, которые
вернули ветки полусухие и с трещинами,~--- говорил пастырь далее.
\vs 3Er 24:2
Те, у которых ветки были
только полусухие,~--- это сомневающиеся: они ни живы, ни мертвы; а те, которые
подали ветки полусухие и с трещинами,~--- это сомневающиеся и вместе с тем
злоязычные, которые поносят отсутствующих, никогда не живут в мире, но
постоянно находятся в раздоре.
\vs 3Er 24:3
Впрочем, и им есть
покаяние. Видишь, и из них некоторые покаялись.
\vs 3Er 24:4
Из них немедленно
покаявшиеся найдут себе место в башне, а те, которые позднее покаялись, будут
обитать на стенах.
\vs 3Er 24:5
Те же, которые не
покаялись, но остались при своих делах, обретут погибель.
\vs 3Er 24:6
Те, которые подали ветки
зеленые, но с трещинами, всегда были верными и добрыми, хотя имеют между собою
зависть и соперничество о первенстве и достоинстве: только глупы люди,
спорящие между собою о первенстве.
\vs 3Er 24:7
Впрочем, они были добры,
послушались моих заповедей, исправились и скоро покаялись, потому и место их в
башне.
\vs 3Er 24:8
Если же кто-нибудь из них
возвратится к раздору, будет изгнан из башни и погубит жизнь свою.
\vs 3Er 24:9
Ибо жизнь званных Богом
состоит в соблюдении заповедей Господа: в этом жизнь, а не в первенстве или
каком-либо достоинстве.
\vs 3Er 24:10
Чрез терпение и смирение
духа люди получат жизнь от Господа, а пренебрегающие Законом приобретут себе
смерть.

\vs 3Er 25:1
Те, у которых ветки
наполовину сухи, наполовину зелены,~--- это привязанные к мирским занятиям и
отчуждавшиеся от общения со святыми, и потому половина их жива, половина
мертва.
\vs 3Er 25:2
И из них многие,
послушавшись заповедей моих, покаялись и получили место в башне; некоторые же
вовсе отпали.
\vs 3Er 25:3
Для них нет покаяния,
потому что они хулили Господа и наконец отвергли Его, и за это нечестие они
потеряли жизнь свою.
\vs 3Er 25:4
Но многие из них
двоедушествовали: этим еще есть покаяние, и если вскоре покаются, будут иметь
жилище в башне; если позднее~--- будут обитать на стенах; если же совсем не
покаются~--- потеряют жизнь свою.
\vs 3Er 25:5
Те, у которых ветки на две
трети были зеленые, а на треть сухие, означают тех, которые, будучи различным
образом совращены, отреклись от Господа:
\vs 3Er 25:6
из них многие покаялись и
уже получили место в башне; а иные навсегда отпали от Бога и совсем потеряли
жизнь.
\vs 3Er 25:7
А некоторые из них
двоедушествовали и возбуждали раздоры: им еще есть покаяние, если вскоре
покаются и откажутся от своих удовольствий; если же останутся при своих делах,
то приготовят себе смерть.

\vs 3Er 26:1
Подавшие свои ветки на две
трети сухими, а на треть зелеными суть верные, но, обогатившись и обретя славу
среди народов, они впали в большую гордость, стали высокомерными, оставили
истину и не имели общения с праведными, но жили вместе с народами, и эта жизнь
казалась им приятнее; от Бога, впрочем, они не отпали и сохраняли веру; только
не творили дела веры.
\vs 3Er 26:2
Многие из них уже
покаялись и стали обитать в башне.
\vs 3Er 26:3
Другие, живя с народами и
набравшись надменного тщеславия у них, совершенно отошли от Бога, предавшись
делам народов: такие люди причислились к народам.
\vs 3Er 26:4
Некоторые же из них начали
колебаться, не надеясь спастись по делам, ими совершаемым; другие пришли в
сомнение и стали возбуждать несогласия.
\vs 3Er 26:5
И тем и другим еще есть
покаяние, но покаяние их должно быть немедленным, чтобы осталось для них место
в башне.
\vs 3Er 26:6
А тем, которые не
раскаются, пребывая в своих удовольствиях, скоро предстоит смерть.

\vs 3Er 27:1
Те, которые подали ветки
зеленые, за исключением их сухих верхушек, и с трещинами, те всегда были
добрыми, верными и славными у Бога, но согрешили несколько раз по причине
небольших удовольствий и мелких несогласий, которые имели между собою.
\vs 3Er 27:2
Услышав слова мои, очень
многие тотчас покаялись, и место их стало в башне.
\vs 3Er 27:3
Некоторые из них пришли в
сомнение, а некоторые, сверх того, произвели большой раздор. Для таких есть
надежда покаяния, потому что всегда были добрыми и едва ли кто из них умрет.
\vs 3Er 27:4
Те же, которые подали
сухие ветки с зелеными верхушками, они только уверовали в Бога, но творили
беззаконие; впрочем, они никогда не отступали от Бога, но всегда охотно носили
Его имя и с любовью принимали рабов Божьих в дома свои.
\vs 3Er 27:5
Услышав о покаянии, они
немедленно покаялись и делают всякую добродетель и правду.
\vs 3Er 27:6
Некоторые из них
претерпели смерть, а другие охотно перенесли несчастия,
помня о делах своих,~--- всем таковым место будет в башне.

\vs 3Er 28:1
Окончив объяснение всех
веток, он повелел мне: пойди и скажи всем, чтобы покаялись и жили для Бога,
потому что Господь по Своему милосердию послал меня дать всем покаяние, даже и
тем, которые по делам своим не заслуживают спасения. Но терпелив Господь и
хочет, чтобы спаслись призванные Его Сыном.
\vs 3Er 28:2
Я надеюсь, господин,~--- ответил я,~--- что все услышавшие это покаются. Ибо я убежден, что всякий
обратится к покаянию, познав дела свои и убоявшись Бога.
\vs 3Er 28:3
Все те, которые от всего
сердца покаются и очистятся от всех неправедных дел, о которых говорилось
прежде, и не приумножат еще чем-либо свои преступления, получат от Господа
прощение прежних грехов своих, если не усомнятся в этих заповедях моих и будут
жить с Богом.
\vs 3Er 28:4
И ты ходи в этих заповедях
и будешь жить с Богом; и все, кто только будет верно исполнять их, будут жить
с Богом.
\vs 3Er 28:5
Показав мне всё это, он
пообещал: остальное я покажу тебе спустя несколько дней.

\chhdr{Подобие 9-е.}
\vs 3Er 29:1
После того как я написал заповеди и притчи пастыря, ангела
покаяния, он пришел ко мне и сказал: я хочу показать тебе всё, что показал
тебе Дух Святой, Который беседовал с тобою в образе Церкви: Дух тот есть Сын
Божий.
\vs 3Er 29:2
И так как ты был слаб телом, то не было открываемо тебе через ангела, доколе
ты не утвердился духом и не укрепился силами, чтобы мог видеть ангела.
\vs 3Er 29:3
Тогда Церковью показано было тебе строение башни хорошо и величественно; но ты
видел, как было показано тебе всё девою.
\vs 3Er 29:4
А теперь ты получишь откровение через ангела, но от того же Духа. Ты должен
тщательно всё узнать от меня; ибо для того и послан я тем досточтимым ангелом
обитать в доме твоём, чтобы ты рассмотрел всё хорошо, ничего не страшась, как
прежде.
\vs 3Er 29:5
И повел он меня в Аркадию,
на гору, имеющую форму груди, и сели мы на её вершине. И показал он мне
большое поле, которое окружали двенадцать гор, не похожих одна на другую.
\vs 3Er 29:6
Первая из них была черная
как сажа. Вторая была голая, без растений. Третья заросла сорняками и
терниями. На четвертой были растения полузасохшие, с зеленой верхушкой и
мёртвым стеблем, а некоторые растения совсем засохли от солнечного жара.
\vs 3Er 29:7
Пятая гора была скалистая,
но на ней зеленели растения. Шестая гора была с расселинами, в иных местах
малыми, в других большими; в этих расселинах были растения, но не цветущие, а
слегка увядшие.
\vs 3Er 29:8
На седьмой горе цвели
растения, и была она плодородна: всякий скот и птицы небесные собирали там
корм, и чем более питались они на ней, тем обильнее росли растения.
\vs 3Er 29:9
Восьмую гору сплошь
покрывали источники, и из этих источников утоляли жажду твари Божьи. Девятая
гора вовсе не имела никакой воды и вся была обнажена: на ней обитали ядовитые
змеи, гибельные для людей.
\vs 3Er 29:10
Десятая гора вся была
затенена огромными деревьями, на ней растущими, и в тени лежал скот, отдыхая и
пережевывая жвачку.
\vs 3Er 29:11
На одиннадцатой горе тоже
во множестве росли деревья, и они изобиловали разными плодами, и видевший их
желал вкусить этих плодов. Двенадцатая гора, вся белая, имела вид самый
приятный, всё было на ней прекрасно.

\vs 3Er 30:1
В середине поля он показал
мне огромный белый камень; камень этот, квадратный по форме, был выше тех гор,
так что мог бы держать всю землю.
\vs 3Er 30:2
Он был древний, но имел
высеченную дверь, которая казалась недавно сделанною. Дверь эта сияла ярче
солнца, так что я поразился ее блеску
\vs 3Er 30:3
Двенадцать дев стояли
возле двери, по четырем сторонам её, в середине попарно.
\vs 3Er 30:4
Четверо из них, стоявшие
по углам двери, показались мне самыми великолепными, но и остальные были
прекрасны.
\vs 3Er 30:5
Веселые и радостные, эти
девы одеты были в полотняные туники, красиво подпоясанные; их правые плечи
были обнажены, словно девы намеревались нести какую-то ношу.
\vs 3Er 30:6
Я залюбовался этим
величественным и дивным зрелищем, но в то же время недоумевая, что девы,
будучи столь нежны, стояли мужественно, будто готовясь понести на себе целое
небо.
\vs 3Er 30:7
И когда размышлял я так,
пастырь сказал мне: что размышляешь ты и недоумеваешь и сам на себя навлекаешь
заботу? Чего не можешь понять, за то не берись, но проси Господа, чтобы
вразумил понять это.
\vs 3Er 30:8
Что за тобою, того не
можешь видеть; а видишь, что перед тобою. Чего не можешь видеть, то оставь и
не мучь себя.
\vs 3Er 30:9
Владей тем, что видишь, о
прочем же не беспокойся. Я объясню тебе всё, что покажу; а теперь смотри, что
будет дальше.

\vs 3Er 31:1
И вот увидел я, что пришли
шесть высоких и почтенных мужей, и все были похожи один на другого; они
призвали множество других мужей, которые также были высоки, красивы и сильны.
\vs 3Er 31:2
И те шесть мужей приказали
строить башню над дверью.
\vs 3Er 31:3
Тогда мужи, которые пришли
для строительства башни, подняли великий шум и беготню около двери.
\vs 3Er 31:4
Девы, стоявшие при двери,
сказали им поспешить со строительством и сами протянули свои руки, как бы
готовясь что-нибудь брать у них.
\vs 3Er 31:5
Те шестеро приказали
доставать камни со дна и подносить их к башне. И подняты были десять камней
белых, квадратных, обтесанных.
\vs 3Er 31:6
Те шесть мужей подозвали
дев и приказали им носить все камни, которые должны были идти на
строительство, проходить через дверь и передавать камни строителям башни.
\vs 3Er 31:7
И тотчас же девы начали
возлагать друг на друга первые камни, извлеченные со дна, и носить их вместе
по одному камню.

\vs 3Er 32:1
Как стояли девы около
двери, так они и носили: те, которые казались сильнее, брались за углы камня,
а другие держали по бокам.
\vs 3Er 32:2
И таким образом носили они
все камни, проходили через дверь, как было велено, и передавали строителям
башни; а те, принимая их, строили.
\vs 3Er 32:3
Башня строилась на большом
камне, над дверью. Те десять камней были положены в основание башни: камень же
и дверь держали на себе всю башню.
\vs 3Er 32:4
После извлекли со дна
другие двадцать пять камней, и они были принесены девами и использованы для
строительства башни.
\vs 3Er 32:5
После них подняли другие
тридцать пять, которые подобным же образом уложили в башню.
\vs 3Er 32:6
Затем подняли еще сорок
камней, и они все пошли на строительство этой башни.
\vs 3Er 32:7
Таким образом в основание
башни легло четыре ряда камней.
\vs 3Er 32:8
Когда закончились все
камни, которые брали со дна, немного отдохнули строители.
\vs 3Er 32:9
Потом те шесть мужей
приказали народу приносить для башни камни с двенадцати гор.
\vs 3Er 32:10
И стали мужи приносить со
всех гор камни обсеченные, различных цветов, и подавали их девам, а те
проносили их через дверь и подавали строителям.
\vs 3Er 32:11
И когда эти разнообразные
камни были положены в здание, то изменили свои прежние цвета и сделались
белыми и одинаковыми.
\vs 3Er 32:12
Но некоторые камни не
были передаваемы девами и не проносились через дверь, а подавались самими
мужами прямо в строение и не делались светлыми, а оставались такими, какими
клались.
\vs 3Er 32:13
Эти камни безобразно
смотрелись в здании башни. Увидев их, те шесть мужей приказали вынуть и
положить на то место, откуда их взяли.
\vs 3Er 32:14
И сказали они тем,
которые приносили эти камни: вы совсем не подавайте камней для строения, но
кладите их возле башни, чтобы девы проносили через дверь и подавали их, иначе
камни не смогут изменить цветов своих, так что не трудитесь понапрасну.

\vs 3Er 33:1
И кончились в тот день
работы, но башня не была завершена; строительство её должно было опять
возобновиться, и только на время сделана некоторая остановка.
\vs 3Er 33:2
Те шесть мужей приказали
строившим удалиться и отдохнуть немного; девам же повелели не отходить от
башни, чтобы охранять её.
\vs 3Er 33:3
После того как ушли все, я
спросил пастыря, почему не окончено здание башни.
\vs 3Er 33:4
Не может оно быть
завершено прежде, нежели придет господин башни и испытает это строение, чтобы,
если окажутся некоторые камни негодными, заменить их, ибо по его воле строится
эта башня,~--- отвечал он.
\vs 3Er 33:5
Господин,~--- попросил я,~---
я желал бы знать, что означает строение башни, а также узнать и об этом камне,
и о двери, и о горах, и о девах, и о камнях, извлеченных со дна и не
отёсанных, но сразу положенных в здание;
\vs 3Er 33:6
и почему сперва положены в
основание десять камней, потом двадцать пять, затем тридцать пять и, наконец,
сорок;
\vs 3Er 33:7
равно и о тех камнях,
которые положены были в строение, но потом вынуты и отнесены на свое место;
всё это, господин, объясни и успокой душу мою.
\vs 3Er 33:8
И сказал он мне: если не
будешь попусту любопытен, то всё узнаешь и увидишь, что дальше будет с этой
башней, и все притчи обстоятельно узнаешь.
\vs 3Er 33:9
Через несколько дней
пришли мы на то же самое место, где сидели прежде, и позвал он меня: пойдем к
башне, ибо господин её придет, чтобы испытать её.
\vs 3Er 33:10
И пришли мы к башне и
никого другого не нашли, кроме дев.
\vs 3Er 33:11
Пастырь спросил их, не
прибыл ли господин башни. И они ответили, что он скоро придет осмотреть это
здание.

\vs 3Er 34:1
И вот, спустя немного
времени, увидел я, что идет великое множество мужей, и в середине муж такого
величайшего роста, что он превышал саму башню;
\vs 3Er 34:2
окружали его шесть мужей,
которые распоряжались строительством, и все те, которые строили эту башню, и
сверх того еще очень многие славные мужи.
\vs 3Er 34:3
Девы, охранявшие башню,
поспешили к нему навстречу; облобызали его, и стали они вместе ходить вокруг
башни.
\vs 3Er 34:4
И он так внимательно
осматривал строение, что испытал каждый камень: по каждому камню он ударил
трижды тростью, которую держал в руке.
\vs 3Er 34:5
Некоторые камни после его
ударов сделались черны как сажа, некоторые шероховаты, другие потрескались,
иные стали коротки, некоторые ни черны, ни белы, другие неровны и не подходили
к прочим камням, иные покрылись множеством пятен. Так разнообразны были камни,
найденные негодными для здания.
\vs 3Er 34:6
Господин повелел убрать
все их из башни и оставить подле неё, а на место их принести другие камни.
\vs 3Er 34:7
И спросили его строившие:
с какой горы прикажешь принести камни и положить на место выброшенных?
\vs 3Er 34:8
Он запретил приносить с
гор, но велел носить с ближайшего поля.
\vs 3Er 34:9
Взрыли поле и нашли камни
блестящие, квадратные, а некоторые и круглые.
\vs 3Er 34:10
И все камни, сколько их
было на этом поле, были принесены и девами пронесены через дверь;
\vs 3Er 34:11
из них квадратные были
обтёсаны и положены на место выброшенных, а круглые не употреблены в здание,
ибо трудно и долго было их обсекать.
\vs 3Er 34:12
Их оставили около башни,
чтобы после обсечь и употребить в здание, потому как они были очень блестящи.

\vs 3Er 35:1
Окончив это,
величественный муж, господин этой башни, призвал пастыря и поручил ему камни,
не одобренные для здания и положенные около башни.
\vs 3Er 35:2
Тщательно очисти эти
камни,~--- велел он,~--- и положи в здание башни те, которые могут приладиться к
прочим, а неподходящие отбрасывай далеко в сторону.
\vs 3Er 35:3
Приказав это, он удалился
со всеми, с кем пришел к башне. Девы же остались около башни охранять её.
\vs 3Er 35:4
И спросил я пастыря: каким
образом эти камни могут снова пойти в здание башни, когда они уже найдены
негодными?
\vs 3Er 35:5
Он отвечал: я из этих
камней большую часть обсеку и использую для строения, и они придутся к прочим.
\vs 3Er 35:6
Господин,~--- сказал я,~--- каким образом, обсечённые, они могут занять то же самое место?
\vs 3Er 35:7
Те, которые кажутся
малыми, пойдут в середину здания; а большие лягут снаружи и будут их
удерживать.
\vs 3Er 35:8
Потом он сказал: пойдем и
через два дня возвратимся и, очистив эти камни, положим в здание. Ибо всё, что
находится около башни, должно быть очищено, а то вдруг случайно явится
господин, увидит, что нечисто около башни, и прогневается; тогда эти камни не
пойдут на строительство башни, и сочтет он меня нерадивым.
\vs 3Er 35:9
Спустя два дня, когда
пришли мы к башне, он сказал мне: рассмотрим все эти камни и узнаем, которые
из них могут идти в здание.
\vs 3Er 35:10
Рассмотрим, господин,~--- ответил я.

\vs 3Er 36:1
Сначала мы рассмотрели
черные камни. Они оказались такими же, какими были отложены от здания.
\vs 3Er 36:2
Он приказал отнести их от
башни и положить отдельно.
\vs 3Er 36:3
Потом он рассмотрел камни
шероховатые и многие из них велел обсечь и девам взять их и положить в здание;
\vs 3Er 36:4
и они, взяв их, положили в
середину башни.
\vs 3Er 36:5
Остальные же он велел
положить с черными камнями, потому что и они оказались черными.
\vs 3Er 36:6
Затем он рассмотрел камни
с трещинами и из них многие обсек и велел чрез дев отнести в здание: они были
положены снаружи, как более крепкие;
\vs 3Er 36:7
остальные же, по множеству
трещин, не могли быть обработанными и потому были удалены от здания башни.
\vs 3Er 36:8
Далее он рассмотрел камни,
которые были коротки: многие из них оказались черными, а некоторые с большими
трещинами, и он велел положить их с теми, которые были отброшены;
\vs 3Er 36:9
остальные же, очищенные и
обработанные, он велел использовать, и девы, взяв их, положили в середину
здания башни, потому что они были не так крепки.
\vs 3Er 36:10
Потом он рассмотрел камни
наполовину белые и наполовину черные: многие из них оказались черными, и он
велел их перенести к отброшенным.
\vs 3Er 36:11
Остальные же все были
найдены белыми и взяты девами и положены снаружи, будучи крепкими, так что
могли удерживать камни, помещенные в середине, ибо в них ничего не было
отсечено.
\vs 3Er 36:12
Затем он рассмотрел камни
неровные и крепкие: некоторые из них отбросил, потому что по причине твердости
нельзя было обработать их;
\vs 3Er 36:13
остальные же были
обсечены и положены девами в середину здания башни, как более слабые.
\vs 3Er 36:14
Далее он рассмотрел камни
с пятнами, и из них немногие оказались черными и были отброшены к прочим;
остальные же оказались белыми~--- они в целости были использованы девами для
строительства и уложены снаружи по причине их твердости.

\vs 3Er 37:1
Потом стал он
рассматривать камни белые и круглые и спросил меня, что делать с ними.
\vs 3Er 37:2
Не знаю, господин,~--- ответил я.
\vs 3Er 37:3
Значит, ты ничего не
можешь придумать насчет них?
\vs 3Er 37:4
Господин,~--- сказал я,~--- не
владею этим искусством, я не каменщик и ничего не могу придумать.
\vs 3Er 37:5
И сказал он: разве не
видишь, что они круглы? Если я захочу сделать их квадратными, то нужно очень
много от них отсекать, но необходимо, чтобы некоторые из них вошли в здание
башни.
\vs 3Er 37:6
Если необходимо,~--- сказал
я,~--- что же ты затрудняешься, не выбираешь, что хочешь, и не подгоняешь в это
здание?
\vs 3Er 37:7
И он выбрал камни большие
и блестящие и обсек их; а девы, взяв их, положили во внешних частях здания.
\vs 3Er 37:8
Остальные же были отнесены
на то же поле, откуда взяты, но не отброшены. Потому что,~--- объяснил пастырь,
несколько еще недостает башне для окончания; господину угодно, чтобы эти
камни пошли в здание башни, так как они очень белы.
\vs 3Er 37:9
Потом призваны были
двенадцать очень красивых женщин, одетых в черное, с обнаженными плечами и
распущенными волосами. Эти женщины казались деревенскими.
\vs 3Er 37:10
Пастырь приказал им взять
отброшенные от здания камни и отнести их на горы, откуда они были принесены.
\vs 3Er 37:11
И они с радостью подняли,
отнесли все камни и положили туда, откуда они взяты.
\vs 3Er 37:12
Когда же не осталось
возле башни ни одного камня, он сказал: обойдем башню и посмотрим, нет ли в
ней какого изъяна.
\vs 3Er 37:13
Обойдя башню, пастырь
увидел, что она прекрасна и построена безукоризненно, и очень развеселился.
\vs 3Er 37:14
И всякий залюбовался бы
постройкою, потому что не было видно ни одного соединения и башня казалась
высеченною из единого камня.

\vs 3Er 38:1
И я, ходя вместе с
пастырем, весьма был доволен таким прекрасным зрелищем.
\vs 3Er 38:2
И повелел он мне: принеси
известь и мелкие черепицы, чтобы мне исправить вид тех камней, которые опять
вынули из здания, ибо всё вокруг башни должно быть ровно и гладко.
\vs 3Er 38:3
И я всё принес, как
приказал он мне, и он добавил: послужи мне: это дело скоро окончится.
\vs 3Er 38:4
Он исправил вид тех камней
и приказал навести порядок около башни.
\vs 3Er 38:5
Тогда девы, взяв веники,
убрали всю грязь и полили водою~--- и место около башни стало красивым и
веселым.
\vs 3Er 38:6
Пастырь сказал мне: всё
очищено; если Господь придет посмотреть эту башню, не найдет ничего, за что бы
укорить нас,
\vs 3Er 38:7
и он хотел удалиться, но я
схватил его за суму и начал умолять его Господом, чтобы объяснил мне
показанное.
\vs 3Er 38:8
Мне нужно отдохнуть
немного, потом я всё объясню тебе,~--- пообещал он.~--- Дожидайся меня здесь.
\vs 3Er 38:9
Господин, что я здесь буду
один делать?
\vs 3Er 38:10
Ты не один,~--- отвечал он,
все девы с тобою.
\vs 3Er 38:11
Господин,~--- попросил я,~--- передай им меня. И он позвал их и сказал: поручаю вам его, пока не вернусь.
\vs 3Er 38:12
И так я остался один с
теми девами. И они были веселы и ласковы со мною, особенно же четыре из них,
превосходнейшие.

\vs 3Er 39:1
Девы сказали: сегодня
пастырь сюда не придет.
\vs 3Er 39:2
Что же я буду делать?
\vs 3Er 39:3
Подожди до вечера, может
быть, придет и будет говорить с тобою, если же не придет, пробудешь с нами,
доколе придет.
\vs 3Er 39:4
Буду дожидаться его до
вечера,~--- решил я,~--- если же не придет, пойду домой и возвращусь поутру.
\vs 3Er 39:5
Но они воспротивились: ты
нам перепоручен и не можешь уйти от нас.
\vs 3Er 39:6
Я спросил тогда: где я
останусь?
\vs 3Er 39:7
С нами,~--- ответили они,~--- ты уснешь, как брат, а не как муж, ибо ты~--- брат наш и после мы будем обитать
с тобою, потому что очень тебя полюбили.
\vs 3Er 39:8
Мне же стыдно было
оставаться с ними. Но та, которая из них казалась главною, обняла меня и
начала лобызать. И прочие, увидев это, тоже начали лобызать меня, как брата,
водить около башни и играть со мною.
\vs 3Er 39:9
Некоторые из них пели
псалмы, а иные водили хороводы. А я в молчании ходил с ними около башни, и
казалось мне, что я помолодел.
\vs 3Er 39:10
С наступлением вечера я
хотел уйти домой, но они удержали меня и не позволили уйти.
\vs 3Er 39:11
И так я провел с ними эту
ночь около башни. Они постлали на землю свои полотняные туники и уложили меня
на них, сами же ничего другого не делали, только молились.
\vs 3Er 39:12
И я с ними молился
непрерывно и столь же усердно, и девы радовались моему усердию. Так оставался
я с девами до следующего дня.
\vs 3Er 39:13
Потом пришел пастырь и
спросил их: вы не причинили ему никакой обиды?
\vs 3Er 39:14
И отвечали они: спроси
его самого.
\vs 3Er 39:15
Господин,~--- сказал я,~--- я
получил великое удовольствие оттого, что остался с ними.
\vs 3Er 39:16
Что ты ужинал?~--- спросил
он.
\vs 3Er 39:17
Я ответил: всю ночь,
господин, я питался словами Господа.
\vs 3Er 39:18
Хорошо ли они тебя
приняли?
\vs 3Er 39:19
Хорошо, господин.
\vs 3Er 39:20
Теперь что прежде всего
желаешь услышать?
\vs 3Er 39:21
Чтобы ты, господин,
объяснил мне, всё, что до этого показал.
\vs 3Er 39:22
Как желаешь,~--- сказал он,
так и буду объяснять тебе и ничего от тебя не скрою.

\vs 3Er 40:1
Прежде всего, господин,~--- попросил я,~--- объясни мне, что означают камень и дверь.
\vs 3Er 40:2
Камень и дверь,~--- сказал
он,~--- это Сын Божий.
\vs 3Er 40:3
Как же так, господин,~--- удивился я,~--- ведь камень древний, а дверь новая?
\vs 3Er 40:4
Слушай, неразумный, и
понимай. Сын Божий древнее всякой твари, так что присутствовал на совете Отца
Своего о создании твари.
\vs 3Er 40:5
А дверь новая потому, что
Он явился в последние дни, сделался новою дверью для того, чтобы желающие
спастись через неё вошли в царство Божье.
\vs 3Er 40:6
Ты видел, что камни через
дверь были пронесены в здание башни, а те, которые не пронесены через неё,
были возвращены на своё место.
\vs 3Er 40:7
Так,~--- продолжал он,~--- никто не войдет в царство Божье, если не примет имени Сына Божьего.
\vs 3Er 40:8
Ибо если бы ты захотел
войти в какой-либо город, окруженный стеною с одними только воротами, не мог
бы ты проникнуть в этот город иначе как только через эти ворота.
\vs 3Er 40:9
По-другому и быть не
может, господин,~--- согласился я.

\vs 3Er 41:1
Итак, как в этот город
можно войти только через ворота его, так и в царство Божье не попадет человек
иначе как только через имя Сына Божьего возлюбленного.
\vs 3Er 41:2
Видел ли ты множество
строящих этими духовными силами? Будут один дух и одно тело, и будет один цвет
одежд их; тот именно заслужит место в башне, кто будет носить имена этих дев.
\vs 3Er 41:3
Почему же, господин,~--- спросил я,~--- отброшены и забракованы были некоторые камни, тогда как и их
пронесли через дверь и передали через руки дев в здание башни?
\vs 3Er 41:4
Так как у тебя есть
обыкновение всё тщательно исследовать, то слушай и об отброшенных камнях.
\vs 3Er 41:5
Все они приняли имя Сына
Божьего и силу этих дев. Приняв эти дары Духа, они укрепились и были в числе
рабов Божьих, и стали у них один дух, одно тело и одна одежда, потому что они
были единомысленны и делали правду.
\vs 3Er 41:6
Но спустя некоторое время
они увлеклись теми красивыми женщинами, которых ты видел одетыми в черную
одежду с обнаженными плечами и распущенными волосами;
\vs 3Er 41:7
увидев их, они возжелали
их и облеклись их силою, а силу дев свергли с себя.
\vs 3Er 41:8
Поэтому они изгнаны из
дома Божьего и преданы тем женщинам. А не соблазнившиеся красотою их остались
в доме Божьем.
\vs 3Er 41:9
Вот тебе,~--- заключил он,~--- значение камней отброшенных.

\vs 3Er 42:1
Что если, господин,~--- продолжал я расспросы,~--- такие люди покаются, отринут пожелания тех женщин и,
вновь обратившись к девам, облекутся их силою,~--- то войдут ли они в дом Божий?
\vs 3Er 42:2
Войдут, если отвергнут
дела тех женщин и снова приобретут силу дев и будут ходить в делах их.
\vs 3Er 42:3
Для того и остановлено
строительство, чтобы они покаялись и вошли в здание башни; если же не
покаются, то другие займут их место, а они будут отвержены навсегда.
\vs 3Er 42:4
За всё это я возблагодарил
Господа, что Он, подвигнутый милостью ко всем призывающим Его имя, послал
ангела покаяния к ним, согрешившим против Него, и обновил души наши, уже
ослабевшие и не имеющие надежды на спасение, восстановив нас к жизни.
\vs 3Er 42:5
Теперь, господин,~--- сказал
я,~--- объясни мне, почему башня строится не на земле, но на камне и двери?
\vs 3Er 42:6
Ты спрашиваешь, потому что
неразумен.
\vs 3Er 42:7
Господин, я вынужден обо
всем тебя спрашивать, потому что совершенно не могу ничего понять, ведь всё
это так величественно и дивно, что людям трудно постичь.
\vs 3Er 42:8
Слушай,~--- сказал мне
пастырь.~--- Имя Сына Божьего велико и неизмеримо, и оно держит весь мир.
\vs 3Er 42:9
Если всё творение держится
Сыном Божьим,~--- спросил я,~--- то как думаешь, поддерживает ли Он тех, которые
призваны Им, носят имя Его и живут по Его заповедям?
\vs 3Er 42:10
Видишь, Он поддерживает
тех, которые от всего сердца носят Его имя. Он Сам служит для них основанием и
с любовью держит их, потому что они не стыдятся носить Его имя.

\vs 3Er 43:1
Открой мне, господин,~--- попросил я,~--- имена дев и тех женщин, облеченных в черную одежду
\vs 3Er 43:2
Слушай. Из тех, которые
могущественнее и стоят по углам двери, первая зовется Верою, вторая~--- Воздержанием, третья~--- Мощью, четвертая~--- Терпением.
\vs 3Er 43:3
Прочие же, которые в
середине, имеют следующие имена: Простота, Невинность, Целомудрие, Радость,
Правдивость, Разумение, Согласие и Любовь.
\vs 3Er 43:4
Носящие эти имена и имя
Сына Божьего могут войти в царство Божье.
\vs 3Er 43:5
Слушай теперь имена
женщин, одетых в черную одежду. Четыре самые могущественные: первую зовут
Вероломством, вторую~--- Неумеренностью, третью~--- Неверием,
четвертую~--- Сластолюбием.
\vs 3Er 43:6
Имена следующих за ними:
Печаль, Лукавство, Похоть, Гнев, Ложь, Неразумие, Злословие, Ненависть.
\vs 3Er 43:7
Раб Божий, носящий такие
имена, хоть и увидит царство Божье, но не войдет в него!
\vs 3Er 43:8
Тогда решил узнать я у
пастыря, что означают камни, которые со дна подняты для здания.
\vs 3Er 43:9
Первые 10,~--- ответил он,~--- положенные в основание,
означают первый век,
следующие 25~--- второй век мужей праведных;
\vs 3Er 43:35
означают пророков и служителей
Господа; 40 же означают апостолов и учителей Евангелия Сына Божьего.
\vs 3Er 43:10
Почему же, господин, девы
подавали и эти камни в здание башни, пронеся их через дверь?
\vs 3Er 43:11
Потому, что они первые
имели силы этих дев, и те и другие не отступали~--- ни духовные силы от людей,
ни люди от сил; но эти силы пребывали с ними до дня упокоения;
\vs 3Er 43:12
если бы они не имели этих
сил духовных, то не годились бы для здания башни.

\vs 3Er 44:1
И снова я попросил: еще,
господин, объясни мне, почему эти камни были извлечены со дна и положены в
здание башни, тогда как они уже имели этих духов?
\vs 3Er 44:2
Им было необходимо пройти
через воду, чтобы оживотвориться; не могли они иначе войти в царство Божие,
как отринув мертвость прежней жизни.
\vs 3Er 44:3
Посему эти почившие
получили печать Сына Божьего и вошли в царство Божье.
\vs 3Er 44:4
Ибо человек до принятия
имени Сына Божьего мертв; но как скоро примет эту печать, он отлагает
мертвость и воспринимает жизнь.
\vs 3Er 44:5
Печать же эта есть вода, в
неё сходят люди мертвыми, а восходят из неё живыми; посему и им проповедана
была эта печать, и они воспользовались ею, чтобы войти в царство Божье.
\vs 3Er 44:6
Почему же,~--- спросил я,~--- вместе с ними взяты со дна и те сорок камней, уже имеющие эту печать?
\vs 3Er 44:7
Потому, что эти апостолы и
учители, проповедовавшие имя Сына Божьего, скончавшись с верою в Него и с
силою, проповедовали Его и прежде почившим, и сами дали им эту печать; они
вместе с ними нисходили в воду и с ними опять восходили.
\vs 3Er 44:8
Но они нисходили живыми, а
те, которые почили прежде, нисходили мертвыми, а вышли живыми; через апостолов
они восприняли жизнь и познали имя Сына Божьего и потому взяты вместе с ними и
положены в здание башни;
\vs 3Er 44:9
они употреблены в строение
не обсеченные, потому что они скончались в праведности и чистоте, только не
имели этой печати. Вот тебе объяснение этих камней.

\vs 3Er 45:1
Теперь, господин,~--- сказал
я,~--- объясни мне значение тех гор: почему они такие разные?
\vs 3Er 45:2
Слушай. Эти двенадцать
гор, которые ты видишь, означают двенадцать племен, населяющих весь мир; среди
них был проповедан Сын Божий через апостолов.
\vs 3Er 45:3
Почему же они различны и
вид имеют неодинаковый?
\vs 3Er 45:4
Эти двенадцать племен,
населяющие весь мир, суть двенадцать народов; и как различны, ты видел, горы,
так различны мысль и внутреннее настроение этих народов. Я поясню тебе смысл
каждого из них.
\vs 3Er 45:5
Прежде всего, господин,
скажи мне вот что: если эти горы так различны, то каким образом камни с них,
будучи положены в здание башни, сделались одноцветными и блестящими, как и
камни, поднятые со дна?
\vs 3Er 45:6
Потому, что все народы под
небом, услышав проповедь, уверовали и нареклись одним именем Сына Божьего и,
приняв печать Его, все получили один дух и один разум, и стала у них одна вера
и одна любовь, и вместе с именем Его они облеклись духовными силами дев.
\vs 3Er 45:7
Потому-то здание башни
сделалось одноцветным и сияющим, подобно солнцу.
\vs 3Er 45:8
Но после того как они
сошлись воедино и стали одним телом, некоторые из них осквернили себя и были
извергнуты из рода праведных; опять возвратились к прежнему состоянию и даже
сделались хуже.

\vs 3Er 46:1
Каким образом, господин,~--- говорю я,~--- они, познав Господа, сделались худшими?
\vs 3Er 46:2
Если не познавший Господа,
сказал он,~--- сделает зло, он подлежит наказанию за свою неправду. Но кто
познал Господа, тот уже должен удерживаться от зла и делать добро.
\vs 3Er 46:3
И если тот, который должен
совершать добро, вместо этого причиняет зло, то не более ли он преступен,
нежели не ведающий Бога?
\vs 3Er 46:4
Посему хотя и не познавшие
Бога и делающие зло обречены на смерть; но те, которые познали Господа и
видели дивные дела Его, делая зло, будут вдвойне наказаны и умрут навеки.
\vs 3Er 46:5
Так очистится Церковь
Божья.
\vs 3Er 46:6
Ты видел: забракованные
камни были выброшены из башни и преданы злым духам, и башня так очистилась,
что казалась вся высеченною из одного камня;
\vs 3Er 46:7
такою будет и Церковь
Божья, когда она очистится и будут изринуты из неё злые, лицемеры,
богохульники, двоедушные и все виновные в различной неправде;
\vs 3Er 46:8
она будет единое тело,
единый дух, единый разум, единая вера и единая любовь, и тогда Сын Божий будет
торжествовать между ними и радоваться, приняв Свой народ чистым.
\vs 3Er 46:9
Господин,~--- сказал я,~--- всё это величественно и славно. Теперь объясни мне значение каждой из гор,
чтобы всякая душа, уповающая на Господа, услышав это, прославляла великое,
дивное и славное имя Его.
\vs 3Er 46:10
Слушай и об этих
различных горах, то есть двенадцати народах.

\vs 3Er 47:1
Первая гора черная
означает верующих отступников, хулителей Господа и предателей рабов Божиих: им
назначена смерть и нет покаяния, и они черны потому, что род их беззаконен.
\vs 3Er 47:2
Вторая гора голая~--- это
верующие лицемеры и учители неправды; они весьма близки к первым и не имеют
плода правды.
\vs 3Er 47:3
Ибо как гора их пуста и
бесплодна, так и эти люди, хотя и имеют имя, но не имеют веры и нет в них
никакого плода истины.
\vs 3Er 47:4
Им, впрочем, есть
покаяние, если только немедленно покаются; а если замедлят, то и им будет
смерть вместе с первыми.
\vs 3Er 47:5
Почему же, господин,
последним есть доступ к покаянию, а первым нет? Ведь дела их почти те же
самые.
\vs 3Er 47:6
Потому для них есть
покаяние, что они не хулили Господа своего и не были предателями рабов Божьих;
но, стремясь к корысти, они обольщали людей, и каждый потворствовал похотям
грешных;
\vs 3Er 47:7
за это дело они понесут
наказание, но так как не были хулителями и предателями Господа, есть у них
возможность покаяния.

\vs 3Er 48:1
Третья гора,~--- продолжал
пастырь,~--- покрытая терниями и сорняками, знаменует верующих, из которых одни
богаты, а другие занялись множеством дел,
\vs 3Er 48:2
ибо сорняки означают
богатых, а тернии~--- тех, которые предались многим попечениям.
\vs 3Er 48:3
Таковые не имеют общения с
рабами Божьими, но удаляются от них, увлекаемые делами своими.
\vs 3Er 48:4
А богатые с трудом
вступают в общение с рабами Божьими, опасаясь, чтобы у них не попросили
чего-либо.
\vs 3Er 48:5
И как разутыми ногами
трудно ходить по колючкам, так и людям такого рода трудно попасть в царство
Божье.
\vs 3Er 48:6
Но и им есть покаяние,
только они должны немедленно обратиться к нему, чтобы упущенное ими прежде
вознаградить в остающиеся дни и делать добро.
\vs 3Er 48:7
Покаявшись и творя добрые
дела, они будут жить с Богом; если же пребудут в своих делах, то будут преданы
тем женщинам, которые лишат их жизни.

\vs 3Er 49:1
И далее повествовал
пастырь: четвертая гора, на которой очень много растений, в верхней части
зеленых, а к корням сухих и даже увядших от солнечного зноя, означает
верующих, которые колеблются или же имеют Господа только на устах, но не в
сердце.
\vs 3Er 49:2
Потому они в основании
сухи и лишены силы, и только слова их живы, а дела мертвы~--- и сами они ни
мертвы, ни живы.
\vs 3Er 49:3
Подобным образом и
колеблющиеся~--- ни зелены, ни сухи, то есть ни живы и ни мертвы.
\vs 3Er 49:4
Как те растения засохли,
едва лишь показалось солнце, так точно и двоедушные, услышав о гонении, по
малодушию поклоняются идолам и стыдятся имени своего Господа;
\vs 3Er 49:5
такие люди ни живы и ни
мертвы; но и они могут жить, если скоро покаются; если же не покаются, то
будут преданы тем женщинам, которые лишат их жизни.

\vs 3Er 50:1
Пятая гора скалистая, но
поросшая зелеными травами, означает верующих таких, которые хоть и веруют, но
мало учатся, дерзки и самодовольны, желают казаться всезнающими, но ничего не
знают.
\vs 3Er 50:2
За эту дерзость разум
отступил от них и вошло в них тщеславное безрассудство.
\vs 3Er 50:3
Они выдают себя за умных
и, будучи глупы, желают быть учителями.
\vs 3Er 50:4
За это высокомудрие многие
из них уничижены, ибо великое беснование~--- дерзость и суетная самонадеянность.
\vs 3Er 50:5
Из них многие отвержены,
другие же, осознав свое заблуждение, покаялись и покорились имеющим разум.
\vs 3Er 50:6
Но и прочим, подобным им,
есть покаяние, потому что они не столько были злы, сколько неразумны и глупы.
\vs 3Er 50:7
Посему, если покаются, они
будут жить с Богом; если же не покаются, будут обитать с женщинами,
коварствующими над ними.

\vs 3Er 51:1
Шестая гора, с большими и
малыми расселинами и с сухими растениями в них, означает верующих.
\vs 3Er 51:2
Малые расселины~--- тех,
которые имели между собою распри и от взаимных пререканий притупилась их вера;
\vs 3Er 51:3
многие из них покаялись,
то же сделают и прочие, услышав мои заповеди, потому что незначительны их
распри и легко они обратятся к покаянию.
\vs 3Er 51:4
Большие~--- это те, что
упорствуют в распрях, злопамятны и гневливы;
\vs 3Er 51:5
они отброшены от башни и
не годятся для здания, трудно им жить с Богом.
\vs 3Er 51:6
Если Бог и Господь наш,
владычествующий над всем Своим творением, не помнит зла на исповедующих грехи
свои, но умилостивляется, то пристало ли человеку, смертному и исполненному
грехов, упорно гневаться на другого, словно он может спасти или погубить его?
\vs 3Er 51:7
Я, ангел покаяния, убеждаю
вас, склонных к этому: одумайтесь и обратитесь к покаянию~--- и Господь уврачует
прежние ваши прегрешения, если очиститесь от этого бесовского зла, если же нет
будете преданы смерти.

\vs 3Er 52:1
Седьмая гора,~--- продолжал
пастырь свои объяснения,~--- на которой растительность зеленая, цветущая и
обильная, так что всякий скот и птицы небесные питаются ею, и она, будучи
срываема, растет еще лучше,
\vs 3Er 52:2
знаменует верующих,
которые просты и добры, не враждуют между собою, но всегда радуются за рабов
Божьих, исполнены духом дев, милосердны к любому человеку и плодами от трудов
своих делятся со всяким немедленно и без колебания.
\vs 3Er 52:3
Посему Господь, видя
простоту и доброту их, благопоспешал трудам рук их и даровал успех во всяком
деле.
\vs 3Er 52:4
Я, ангел покаяния,
убеждаю вас пребывать в таком расположении, и семя ваше не искоренится вовек.
\vs 3Er 52:5
Господь одобрил вас и
вписал в наше число, и всё семя ваше будет обитать с Сыном Божьим, потому что
вы~--- от Духа Его.

\vs 3Er 53:1
Восьмая гора, со многими
источниками, которые поили всякую тварь Божью, означает апостолов и учителей,
\vs 3Er 53:2
которые проповедовали по
всему миру, и свято и чисто учили слову Господню, и не склонялись к дурным
желаниям, но постоянно пребывали в правде и истине, приняв Святого Духа.
\vs 3Er 53:3
Посему они обитают с
ангелами.

\vs 3Er 54:1
Камни с пятнами на девятой
горе, пустынной и населенной вредоносными змеями, означают дьяконов,
\vs 3Er 54:2
которые плохо проходили
служение, расхищая блага вдов и сирот и обогащаясь от своего служения.
\vs 3Er 54:3
Если останутся в своем
пороке, то они мертвы и нет в них никакой надежды жизни; если же обратятся и
будут непорочно исполнять свое служение, то смогут жить.
\vs 3Er 54:4
А камни шероховатые
означают тех, которые отреклись и не обратились к Господу, одичали и
уподобились пустыне, не общаются с рабами Божьими, но, живя одиноко, губят
свои души.
\vs 3Er 54:5
Как виноградная лоза,
оставленная без всякого ухода, пропадает, заглушается травами, со временем
делается дикой и бесполезной для хозяина, так и эти люди, отчаявшись в себе
самих, одичали и стали бесполезны для своего Господа.
\vs 3Er 54:6
Для них возможно покаяние,
если отреклись они не от сердца; если же кто сделал это от сердца, не знаю,
сможет ли он возродиться.
\vs 3Er 54:7
Я не о настоящих днях
говорю, чтобы отрекшийся мог покаяться;
\vs 3Er 54:8
невозможно обрести
спасение тем, кто намерен отречься от своего Господа; но покаяние дается
тем, кто отрекся в прошлом.
\vs 3Er 54:9
Итак, кто намерен
покаяться, пусть сделает это немедленно, прежде чем закончится строительство
башни.
\vs 3Er 54:10
Если же кто не поспешит,
то будет предан смерти теми женщинами.
\vs 3Er 54:11
Камни короткие означают
людей коварных и клеветников: они подобны змеям, которых ты видел на девятой
горе.
\vs 3Er 54:12
Ибо как яд змеи
смертоносен для человека, так и слова таких людей губительны для других.
Несовершенны они в своей вере по причине их образа действий.
\vs 3Er 54:13
Впрочем, некоторые из
них покаялись и спаслись.
\vs 3Er 54:14
Равным образом прочие
таковые получат спасение, если покаются; если же не покаются, то погибнут от
тех женщин, силою и властью которых они обладают.

\vs 3Er 55:1
Деревья на десятой горе,
которые служат кровом для скота, означают епископов и верующих страннолюбцев,
\vs 3Er 55:2
которые всегда непритворно
и радушно принимали в домах своих рабов Божьих;
\vs 3Er 55:3
епископов, которые
беспрестанно покровительствовали бедным и вдовствующим и жили всегда
непорочно.
\vs 3Er 55:4
Таким людям
покровительствует сам Господь: они почтенны у Бога, и им место среди ангелов,
если пребудут до конца в служении Господу.

\vs 3Er 56:1
Одиннадцатая гора, деревья
на которой обильны разными плодами, означает верующих, пострадавших за имя
Сына Божьего, пострадавших с любовью и от всего сердца своего.
\vs 3Er 56:2
Я спросил: почему же,
господин, все деревья имеют плоды, но на некоторых плоды менее приятны?
\vs 3Er 56:3
И это объясню тебе.
Пострадавшие за имя Господне почтенны у Бога, и всем им отпущены грехи, потому
что пострадали за имя Сына Божьего.
\vs 3Er 56:4
Но некоторые, будучи
допущены ко властям и спрошены, не отреклись от Господа, но с готовностью
пострадали,~--- они почтенны у Бога, и плод их превосходнее.
\vs 3Er 56:5
А некоторые, охваченные
страхами и смущением, колебание имели в своем сердце, проповедать ли Бога или
отречься, и пострадали~--- их плоды хуже, потому что в сердце их был лукавый
помысел раба отречься от своего господина.
\vs 3Er 56:6
Смотрите вы, помышляющие
так, чтобы эта мысль не утвердилась в ваших сердцах и чтобы не умереть вам для
Бога.
\vs 3Er 56:7
А вы, страдающие за имя
Божье, должны прославлять Господа, что удостоил вас носить Его имя, ибо
исцелятся все грехи ваши.
\vs 3Er 56:8
Ужели вы не почитаете себя
более других блаженными? Вы думаете, что совершили великое дело, если кто из
вас пострадал?
\vs 3Er 56:9
Но Господь дарует вам
жизнь, и вы об этом не помышляете. Вас отягощали грехи ваши, и если бы не
пострадали вы за имя Господне, то вы умерли бы для Бога за грехи свои.
\vs 3Er 56:10
Это я говорю вам,
сомневающимся, исповедать ли Бога или отречься.
\vs 3Er 56:11
Исповедуйте, что вы
имеете Господа, и, не отрекаясь, отдавайте себя в оковы.
\vs 3Er 56:12
Если все народы
наказывают рабов за отречение от своего хозяина, то что, думаете вы, сделает с
вами Господь, имеющий власть над всеми?
\vs 3Er 56:13
Итак, удалите из сердец
своих такие помыслы, чтобы вовеки жить вам с Богом.

\vs 3Er 57:1
Двенадцатая гора, белая,
означает верующих, подобных младенцам, коим не всходила на сердце никакая
злоба, которые не знают, что такое лукавство, но всегда пребывают в простоте.
\vs 3Er 57:2
Такие люди, без сомнения,
будут обитать в царстве Божьем, потому что они ни в одном деле не преступили
заповедей Божьих, но с простотою пребывали в том же расположении все дни своей
жизни.
\vs 3Er 57:3
Те, которые останутся как
младенцы, не имеющие злобы, будут почетнее всех, о которых сказано выше: все
младенцы славны у Господа и почитаются у Него первыми.
\vs 3Er 57:4
Итак, блаженны вы, которые
удалили от себя лукавство и облеклись в невинность, потому что вы первые
будете жить с Богом.
\vs 3Er 57:5
После того как пастырь
истолковал мне все горы, я сказал ему:
\vs 3Er 57:6
<<Господин, теперь поведай о тех камнях, которые принесены с поля и заложены в
башню вместо вынутых, а также о тех круглых камнях, которые вошли в здание
башни, и о тех, которые доселе остаются круглыми.>>

\vs 3Er 58:1
Слушай и об этом. Камни,
которые были принесены с поля и заложены в здание башни
вместо отвергнутых,~--- это суть отроги белой горы.
\vs 3Er 58:2
Поскольку верующие с этой
горы оказались невинными, то господин башни поместил их в здание башни, ибо
знал, что, войдя в здание, они останутся белыми и ни один из них не почернеет.
\vs 3Er 58:3
А если бы он приказал
положить в здание башни камни и с прочих гор, то нужно было бы ему снова
осматривать эту башню и очищать.
\vs 3Er 58:4
Эти белые камни суть
новообращенные, которые уверовали и уверуют, ибо они веруют от сердца. Блажен
этот род, потому что невинен.
\vs 3Er 58:5
Слушай теперь и о круглых
блестящих камнях. И они все от белой горы.
\vs 3Er 58:6
Круглыми же они оказались
потому, что богатство немного омрачило их, но они не отступили от Бога и ни
единое слово хулы не сошло с языка их~--- только правда, добродетель и истина.
\vs 3Er 58:7
Посему Господь, зная душу
их и то, что они родились и остаются добрыми, повелел отсечь их богатства, но
не совсем отнять их, чтобы из оставшегося они могли делать добро и жить с
Богом, ибо и они из доброго рода.
\vs 3Er 58:8
Посему их несколько
отесали и положили в здание башни.

\vs 3Er 59:1
А прочие камни, которые
остались круглыми и были негодны для здания, еще не получили печати и
возвращены на свое место, ибо оказались слишком круглыми.
\vs 3Er 59:2
Должно лишить их благ
настоящего века и суетного богатства~--- и тогда они будут годны в царстве
Божьем.
\vs 3Er 59:3
Они должны войти в царство
Божье, ибо Господь благословил этот род, и из него никто не погибнет;
\vs 3Er 59:4
может быть, кто из них,
искушенный злым дьяволом, и согрешит в чем-либо, но скоро вновь обратится к
Господу своему.
\vs 3Er 59:5
Я, ангел покаяния, почитаю
счастливыми вас, которые невинны, как дети, потому что ваша участь благая и
почтенная перед Богом.
\vs 3Er 59:6
И всем, которые приняли
печать Сына Божьего, говорю: имейте простоту, не помните обид, не пребывайте в
злобе, да не будет в душе кого-либо из вас горечи злопамятства;
\vs 3Er 59:7
врачуйте и удаляйте от
себя злые раздоры, чтобы господин стада пришел и возрадовался, найдя целыми
овец своих.
\vs 3Er 59:8
Если же какая овца будет
потеряна пастырями или самих пастырей господин найдет дурными, что ответят
ему? Ужели скажут, что они измучены стадом?
\vs 3Er 59:9
Не поверят им, ибо не
может пастырь потерпеть что от овец и еще более будет наказан за ложь свою.
\vs 3Er 59:10
И я~--- пастырь и должен
дать Всевышнему отчет за вас.

\vs 3Er 60:1
Итак, позаботьтесь о себе,
пока еще строится башня.
\vs 3Er 60:2
Господь обитает в людях,
любящих мир, ибо Он Сам любит мир и далек от сварливых и развращенных злобою.
\vs 3Er 60:3
Возвратите Ему дух целым,
какой приняли от Него.
\vs 3Er 60:4
Ибо если ты отдашь
валяльщику одежду целую, то желаешь и получить ее обратно целою, а если
валяльщик возвратит тебе её изодранною, возьмешь ли ты ее?
\vs 3Er 60:5
Не прогневаешься ли и не
будешь ли бранить его, говоря: я дал тебе одежду целою, а ты изодрал её, и
теперь она из-за дыр, которые ты на ней сделал, стала непригодна. Разве не так
будешь пенять ты валяльщику и скорбеть о своей одежде?
\vs 3Er 60:6
Так что же, думаешь,
сделает тебе Господь, который вручил тебе дух чистый, а ты повредил его и
привел в негодность, так что он никак не может служить Господу?
\vs 3Er 60:7
И за это Господь предаст
тебя смерти.
\vs 3Er 60:8
Так накажет Он всех тех,
которых найдет упорно помнящими обиды.
\vs 3Er 60:9
Не пренебрегайте Его
милосердием, но лучше прославляйте Его за то, что Он, не в пример вам, столь
терпим к вашим преступлениям.
\vs 3Er 60:10
Покайтесь, ибо это
полезно для вас.

\vs 3Er 61:1
Всё, что описано выше,
показал я, пастырь, ангел покаяния, ради покаяния.
\vs 3Er 61:2
Я всегда говорил и теперь
говорю рабам Божьим: если поверите и послушаетесь слов моих, будете поступать
по ним и исправите пути ваши, то сможете спастись.
\vs 3Er 61:3
Если же будете
упорствовать в лукавстве и злопамятстве, ни один из таких грешников не будет
жить с Богом: ибо всё это мною наперед сказано вам.
\vs 3Er 61:4
И после этих слов пастырь
спросил меня: всё ли ты проведал у меня?
\vs 3Er 61:5
Я ответил, что всё.
\vs 3Er 61:6
Почему же ты не спросил
меня,~--- сказал тогда он,~--- о камнях, положенных в здание, вид которых мы
исправили?
\vs 3Er 61:7
Забыл, господин.
\vs 3Er 61:8
Выслушай и о них. Это те,
до которых дошли теперь мои заповеди, и они от всего сердца покаялись, и
Господь, видя, что покаяние их доброе и чистое и что пребудут они в нем,
повелел загладить прежние грехи их.
\vs 3Er 61:9
Так грехи их изглажены,
чтобы после они не были видны.

\chhdr{Подобие 10-е.}
\vs 3Er 62:1
После того как я написал эту книгу, тот ангел, который
вручил меня пастырю, пришел в дом мой и сел на ложе, а справа от него стал
пастырь.
\vs 3Er 62:2
Позвал ангел меня и сказал: я поручил тебя и дом твой этому пастырю под его
покровительство.
\vs 3Er 62:3
Так, господин,~--- подтвердил я.
\vs 3Er 62:4
Итак, если хочешь быть
защищен от всякого бедствия и злополучия, иметь успех во всяком благом деле и
слове и во всякой истинной добродетели, то поступай по тем заповедям, которые
он дал тебе, и будешь господствовать над всякою неправдою.
\vs 3Er 62:5
Ибо, если будешь соблюдать
эти заповеди, покорятся тебе всякое пожелание и сладость этого века и будет
сопровождать тебя удача во всяком добром деле.
\vs 3Er 62:6
Почитай его достоинство и
святость и скажи всем, что он в великой чести и славе у Бога и имеет великую
власть и силу.
\vs 3Er 62:7
Ему одному во всей
вселенной вручена власть покаяния. Разве он не кажется тебе могущественным?
\vs 3Er 62:8
Но вы пренебрегаете его
достоинством и властью, которую он имеет над вами.

\vs 3Er 63:1
Я сказал: спроси,
господин, самого его, сделал ли я что дурное или оскорбил его чем-нибудь за то
время, что он находится в доме моем.
\vs 3Er 63:2
И я знаю, что ты не сделал
и не сделаешь ничего дурного, потому я и говорю это тебе, чтобы ты всегда был
таков. Ибо он предо мною хорошо засвидетельствовал о тебе.
\vs 3Er 63:3
Скажи это и прочим, чтобы
и они, если покаялись или намерены покаяться, чувствовали то же, что и ты,~--- и
он засвидетельствует доброе о них предо мною, а я пред Господом.
\vs 3Er 63:4
Господин,~--- ответил я,~--- я
всякому человеку возвещу великие дела Божьи и надеюсь, что все прежде
согрешившие, услышав это, покаются, чтобы получить жизнь.
\vs 3Er 63:5
Итак, совершай неуклонно
это служение и впредь.
\vs 3Er 63:6
Кто исполнит заповеди Его,
будет иметь жизнь и великую честь у Господа.
\vs 3Er 63:7
А кто не соблюдет Его
заповедей, бежит от своей жизни; кто не чтит Его, теряет свою честь у Господа.
\vs 3Er 63:8
Презирающие Его и не
соблюдающие Его заповедей обрекают себя на смерть, и любой из них виновен в
крови своей.
\vs 3Er 63:9
Тебе же наказываю
соблюдать эти заповеди~--- и получишь искупление всех грехов своих.

\vs 3Er 64:1
Я послал к тебе также и
этих дев, чтобы они жили с тобою, ибо я видел, что они очень ласковы к тебе.
\vs 3Er 64:2
Они станут тебе
помощниками, чтобы усерднее ты мог соблюдать заповеди, ибо без этих дев
невозможно соблюсти заповеди.
\vs 3Er 64:3
Я вижу, что им приятно
быть с тобою, и я прикажу, чтобы они вовсе не выходили из твоего дома.
\vs 3Er 64:4
Ты только очисти дом свой:
в чистом доме они живут охотно.
\vs 3Er 64:5
Они сами чисты, непорочны
и рачительны и весьма угодны Господу.
\vs 3Er 64:6
Итак, если будет чист дом
твой, они останутся с тобою. Если же чем осквернится дом твой, они совсем
удалятся из него, ибо не любят никакой нечистоты.
\vs 3Er 64:7
Я надеюсь угодить им, так
что они охотно и безотлучно будут жить в доме моем. И как тот, которому ты
передал меня, ни в чем на меня не жалуется, так и они не будут жаловаться.
\vs 3Er 64:8
Ангел сказал пастырю: я
вижу, что раб Божий хочет соблюдать эти заповеди и поместить дев в чистом
жилище.
\vs 3Er 64:9
Произнеся это, он опять
поручил меня пастырю и обратился к девам: так как я вижу, что вам приятно жить
в этом доме, то вручаю вам Ерму и семью его с тем, чтобы вы не покидали этого
дома.
\vs 3Er 64:10
И они с удовольствием
вняли этим словам.

\vs 3Er 65:1
Потом он сказал мне:
мужественно проходи это служение и поведай всякому человеку величие Божье~--- и
будешь иметь благодать в своем служении.
\vs 3Er 65:2
Всякий, кто исполнит эти
заповеди, будет жить и будет блажен; а кто пренебрежет ими, не будет жить и
будет несчастлив в своей жизни.
\vs 3Er 65:3
Скажи всем, чтобы не
переставали, кто может, благотворить, ибо благотворение полезно им.
\vs 3Er 65:4
Говорю о том, что должно
всякого человека вызволять из бедствия. Неимущий в ежедневной жизни терпит
великое мучение и скорбь.
\vs 3Er 65:5
Кто вырвет из нужды душу
такого человека, тот обретет великую радость, ибо терпящий подобное бедствие
испытывает страдания сродни заключенному в узах.
\vs 3Er 65:6
Многие, не вынеся
бедственного положения, причиняют себе смерть. Посему кто знает о бедствии
такого человека и не избавляет его, тот совершает великий грех и принимает
вину за кровь его.
\vs 3Er 65:7
Итак, благотворите,
сколько кто получил от Господа. Не медлите, пока не окончилось строительство
башни, ибо ради вас приостановлено оно.
\vs 3Er 65:8
Если не поспешите
исправиться, будет достроена башня и вы не попадете в неё.
\vs 3Er 65:9
После этих слов он встал с
ложа и, взяв пастыря и дев, удалился, но обещал мне, что пастыря и дев
отпустит обратно в дом мой.

\end{document}
